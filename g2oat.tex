\newif\ifaspectratiobasic
\aspectratiobasictrue % uncomment for 4:3

\ifaspectratiobasic
% 4:3
\newcommand\mytikztextsize{}
\newcommand\mytikzscale{1.50}
\newcommand\myaspectratio{43}
\else
% 16:9
\newcommand\mytikztextsize{\small}
\newcommand\mytikzscale{1.40}
\newcommand\myaspectratio{169}
\fi

\documentclass[aspectratio=\myaspectratio]{beamer}
\usetheme{moloch}

\usepackage[czech,english]{babel}
\usepackage{booktabs}
\usepackage{graphicx}
\usepackage{tikz}

\title{NAC-COLORINGS SEARCH\@:\newline COMPLEXITY AND ALGORITHMS}
\author{Petr Laštovička, Jan Legerský}
\institute{Czech Technical University, Faculty of Information Technology}
\date{September 22, 2025}


%%%%%%%%%%%%%%%
%% Tikz
%%%%%%%%%%%%%%%
\colorlet{ecol}{black!50!white}
\definecolor{colR}{rgb}{.932,.172,.172} %x11 Firebrick2
\definecolor{colB}{rgb}{.255,.41,.884} %svgnames RoyalBlue
\definecolor{colOrange}{RGB}{255,191,0} %Amber
%% Tikz styles
\tikzstyle{vertex}=[circle, draw, fill=black, inner sep=0pt, minimum size=4pt]
\tikzstyle{fvertex}=[circle, draw, fill=white, inner sep=0pt, minimum size=4pt]
\tikzstyle{vertexSig}=[circle, draw, fill=colOrange, inner sep=0pt, minimum size=4pt]
\tikzstyle{edge}=[line width=1.5pt,ecol]
% dotted, densely dotted, loosely dotted, dashed
\tikzstyle{dots}=[dotted dash=1.5pt]
\tikzstyle{redge}=[edge,colR]
\tikzstyle{bedge}=[edge,colB]
\tikzstyle{yedge}=[edge,colOrange]
%%%%%%%%%%%%%%%


\begin{document}
\maketitle

% \section{Goals}
\begin{frame}
	\frametitle{Contents}
	\begin{itemize}
		\item
		      Introduction to Rigidity theory, flexible realizations and NAC-colorings
		\item
		      NAC-coloring existence is NP-complete also on graphs with maximum degree five.
		\item
		      FPT algorithm for NAC-coloring counting parametrized by treewidth.
		\item
		      Design, implementation and evaluation of an algorithm and heuristics for NAC-coloring search.
	\end{itemize}
\end{frame}

\section{Rigidity theory}

\begin{frame}
	\frametitle{Rigidity theory}
	\begin{definition}[\( d \)-realization]
		A~\emph{\( d \)-realization} of a~graph \( G \) is a~mapping \( p: V(G) \to \R^d \).
	\end{definition}
	%
	\begin{definition}[Framework]
		A~\emph{framework} is a~pair of a~graph \( G \) and its realization.
	\end{definition}
\end{frame}

\begin{frame}
	\frametitle{Rigidity theory}
	\begin{definition}[Nontrivial flex]
		A~\emph{nontrivial flex} of framework \( (G, p) \) is a~continuous curve of realizations \( p_t \)
		for \( 0 \le t \le 1\) such that
		\( p_0 = p \) and for~all \( 0 < t \le 1 \)
		we have that
		\( \|p_t(u) - p_t(v)\| = \|p(u) - p(v)\|\) for~every \( \{u, v\} \in E(G) \),
		but \( \|p_t(u) - p_t(v)\| \ne \|p(u) - p(v)\| \) for~some \( u, v \in V (G) \).
	\end{definition}
	\begin{definition}[\( d \)-flexible, \( d \)-rigid]
		A~framework is \emph{\( d \)-flexible}, if~it has a~nontrivial flex.
		Otherwise, it~is \emph{\( d \)-rigid}.
	\end{definition}
\end{frame}
\begin{frame}
	\frametitle{Paradoxical flexibility}
	\begin{definition}[Flexible \& rigid graphs]
		A~graph is \emph{(generically) \( d \)-flexible} if~almost all of
		its \( d \)-realizations are \( d \)-flexible.
		A~graph is \emph{(generically) \( d \)-rigid} if~almost all of
		its \( d \)-realizations are \( d \)-rigid.
	\end{definition}
	\begin{figure}[ht]
		\centering
		\begin{tikzpicture}[rotate=90,scale=1.5]
			\node[vertex] (a) at (0,0) {};
			\node[vertex] (b) at (1,0) {};
			\node[vertex] (c) at (0.5,0.5) {};
			\node[vertex] (d) at (0,1.5) {};
			\node[vertex] (e) at (1,1.5) {};
			\node[vertex] (f) at (0.5,1) {};
			\draw[edge] (a)edge(b) (b)edge(c) (c)edge(a) (d)edge(e) (e)edge(f) (f)edge(d) ;
			\draw[edge] (a)edge(d) (b)edge(e) (c)edge(f);
		\end{tikzpicture}
		\qquad
		\qquad
		\begin{tikzpicture}[rotate=90,scale=1.5]
			\node[vertex] (a) at (0.00,0) {};
			\node[vertex] (b) at (1.00,0) {};
			\node[vertex] (c) at (0.50,0.5) {};
			\node[vertex] (d) at (0.25,1) {};
			\node[vertex] (e) at (1.25,1) {};
			\node[vertex] (f) at (0.75,1.5) {};
			\draw[edge] (a)edge(b) (b)edge(c) (c)edge(a) (d)edge(e) (e)edge(f) (f)edge(d) ;
			\draw[edge] (a)edge(d) (b)edge(e) (c)edge(f);
		\end{tikzpicture}
	\end{figure}
\end{frame}

\section{NAC-coloring}

\begin{frame}
	\frametitle{NAC-coloring}
	\begin{definition}[NAC-coloring]
		Let~\( G \) be a~graph and \( \delta: E(G) \to \{ \red, \blue \} \)
		be a~coloring of its edges:
		%
		\begin{itemize}
			\item A~cycle in~\( G \) is a~\( \red \) cycle, if~all its edges are \( \red \),
			      analogously for~\( \blue \) cycles.
			\item A~cycle in~\( G \) is an~\emph{almost \( \red \) cycle},
			      if~exactly one of its edges is \( \blue \).
			      \emph{Almost \( \blue \) cycle} is defined analogously.
		\end{itemize}
		%
		The~coloring~\( \delta \) is called a~\emph{NAC-coloring}, if~it is surjective
		and there are no red nor blue almost cycles.
	\end{definition}
\end{frame}
\begin{frame}
	\frametitle{NAC-coloring}
	\begin{lemma}%
		Let~\( G \) be a~graph. If~\( \delta: E(G) \to \{ \red, \blue \} \) is a~coloring of edges,
		then there are no almost cycles in~\( G \) if~and only if~the~connected components
		of \( G[E_\red] \) and \( G[E_\blue] \)%
		are induced subgraphs of \( G \).
	\end{lemma}
\end{frame}

\begin{frame}
	\frametitle{NAC-coloring}

	\begin{theorem}
		A~connected graph \( G \) with at least one edge has a~flexible
		quasi-injective \( 2 \)-dimensional realization if~and only if~it has a~NAC-coloring.
	\end{theorem}

	\begin{figure}[ht]
		% original scale * 4/5
		\centering
		\begin{tikzpicture}[scale=1.0] % 1.5
			\node[fvertex] (a) at (0,0) {};
			\node[fvertex] (b) at (1,0) {};
			\node[fvertex] (c) at (0.5,0.5) {};
			\node[fvertex] (d) at (0,1) {};
			\node[fvertex] (e) at (1,1) {};
			\node[fvertex] (f) at (0.5,1.5) {};
			\draw[edge] (a)edge(b) (b)edge(c) (c)edge(a) (d)edge(e) (e)edge(f) (f)edge(d) (a)edge(d) (b)edge(e) (c)edge(f);
		\end{tikzpicture}
		\quad
		\begin{tikzpicture}[scale=1.0] % 1.5
			\node[vertex] (a) at (0,0) {};
			\node[vertex] (b) at (1,0) {};
			\node[vertex] (c) at (0.5,0.5) {};
			\node[vertex] (d) at (0,1.5) {};
			\node[vertex] (e) at (1,1.5) {};
			\node[vertex] (f) at (0.5,1) {};
			\draw[bedge] (a)edge(b) (b)edge(c) (c)edge(a) (d)edge(e) (e)edge(f) (f)edge(d) ;
			\draw[redge] (a)edge(d) (b)edge(e) (c)edge(f);
		\end{tikzpicture}
		\qquad
		\begin{tikzpicture}[scale=0.50] % 0.75
			\draw[black!50!white, dashed] (-1.5,0)edge(2.3,0);
			\draw[black!50!white, dashed] (0,1.55)edge(0,-2.1);
			\node[fvertex] (2) at (1.8, 0) {};
			\node[fvertex] (5) at (-1., 0) {};
			\node[fvertex] (7) at (0.7, 0) {};
			\node[fvertex] (1) at (0, -1.75) {};
			\node[fvertex] (6) at (0,  1.2) {};
			\node[fvertex] (4) at (0, -0.8) {};
			\draw[edge]  (6)edge(5) (5)edge(4) (7)edge(4) (7)edge(6);
			\draw[edge] (2)edge(4) (2)edge(6);
			\draw[edge] (1)edge(5) (7)edge(1) ;
			\draw[edge] (2)edge(1);
		\end{tikzpicture}
		\quad
		\begin{tikzpicture}[scale=0.80] % 1.2
			\node[vertex] (a1) at (-0.5, -0.866025) {};
			\node[vertex] (a2) at (0.5, -0.866025) {};
			\node[vertex] (a3) at (1., 0.) {};
			\node[vertex] (a4) at (0.5, 0.866025) {};
			\node[vertex] (a5) at (-0.5, 0.866025) {};
			\node[vertex] (a6) at (-1.,  0.) {};
			\draw[bedge] (a1)edge(a2) (a1)edge(a4) (a1)edge(a6);
			\draw[redge] (a2)edge(a3) (a2)edge(a5) (a3)edge(a4) (a3)edge(a6) (a5)edge(a4) (a5)edge(a6);
		\end{tikzpicture}
		\quad
		\begin{tikzpicture}[scale=0.80] % 1.2
			\node[vertex] (a1) at (-0.5, -0.866025) {};
			\node[vertex] (a2) at (0.5, -0.866025) {};
			\node[vertex] (a3) at (1., 0.) {};
			\node[vertex] (a4) at (0.5, 0.866025) {};
			\node[vertex] (a5) at (-0.5, 0.866025) {};
			\node[vertex] (a6) at (-1.,  0.) {};
			\draw[bedge] (a1)edge(a2) (a1)edge(a4) (a2)edge(a3) (a3)edge(a4) (a5)edge(a6);
			\draw[redge] (a1)edge(a6) (a2)edge(a5) (a3)edge(a6) (a5)edge(a4) ;
		\end{tikzpicture}
	\end{figure}
\end{frame}

\section{NAC-coloring existence complexity}

\begin{frame}
	\frametitle{Previous research}

	\begin{itemize}
		\item
		      It is NP-complete to decide whether a graph has a NAC-coloring.
		\item
		      Correspondence with stable cuts (independent vertex cuts).
		\item
		      Graphs where \( |E| \leq 2|V| - 3 \) can be checked polynomially.
		      \begin{itemize}
			      \item Hence maximum degree three graphs are polynomially checkable.
		      \end{itemize}
		\item
		      And other polynomial checks\ldots
	\end{itemize}
\end{frame}

\begin{frame}
	\frametitle{Original reduction}
	\begin{itemize}
		\item
		      Reduction from 3-SAT\@.
		\item
		      No bound on the degree of vertices.
		\item
		      Our reduction idea is based upon this original construction.
	\end{itemize}
\end{frame}

\begin{frame}
	\frametitle{Graphs with maximum degree five}

	\begin{itemize}
		\item
		      Auxiliary graph is constructed for a 3-SAT formula.
		\item
		      For each variable \( x \),
		      we ensure that \( x \) and \( \bar{x} \)
		      they correspond to different color and different truth value.
		\item
		      Then for each clause we ensure that at least one of the literals is true.
		\item
		      We use leathers to connect individual gadgets.
		\item
		      Maximum degree is kept at five.
	\end{itemize}
\end{frame}

\begin{frame}
	\frametitle{Reduction from 3-SAT}

	\begin{figure}[h]
		\centering
		% \begin{tikzpicture}[scale=1.50] % 4:3
		\begin{tikzpicture}[scale=1.40] % 16:9
			% for~i in~range(1,8):
			%       for~j in~range(1,5):
			%           print(f"\\node[vertex] ({i}{j}) at ({i/2}, {j/2}) {{}};")
			% \node[vertex] (11) at (0.5, 0.5) {};
			\node[vertex]      (22) at (1.25, 1.00) {};
			\node[]           (d22) at (1.00, 1.00) {};
			\node[vertex]      (23) at (1.25, 1.50) {};
			\node[]           (d23) at (1.00, 1.50) {};
			\node[vertex]      (32) at (1.50, 1.00) {};
			\node[vertex]      (33) at (1.50, 1.50) {};
			\node[vertex]      (42) at (2.00, 1.00) {};
			\node[vertex]      (43) at (2.00, 1.50) {};
			\node[vertexSig]   (44) at (2.00, 1.75) {};
			\node[vertex]      (45) at (2.00, 2.25) {};
			\node[vertex]      (46) at (2.00, 2.50) {};
			\node[]           (d46) at (2.00, 2.75) {};
			\node[vertex]      (52) at (2.50, 1.00) {};
			\node[vertex]      (53) at (2.50, 1.50) {};
			\node[vertexSig]   (54) at (2.50, 1.75) {};
			\node[vertex]      (55) at (2.50, 2.25) {};
			\node[vertex]      (56) at (2.50, 2.50) {};
			\node[]           (d56) at (2.50, 2.75) {};
			\node[vertex]      (62) at (3.00, 1.00) {};
			\node[vertex]      (63) at (3.00, 1.50) {};
			\node[vertex]      (72) at (3.25, 1.00) {};
			\node[]           (d72) at (3.50, 1.00) {};
			\node[vertex]      (73) at (3.25, 1.50) {};
			\node[]           (d73) at (3.50, 1.50) {};
			\node[vertex] (special) at (2.25, 0.75) {};

			\node[] at (2.25, 1.5) {\mytikztextsize $A_i$};

			%%% Left part
			% Bridge to center
			\draw[edge] (32)edge(42) (33)edge(43) (32)edge(33) (32)edge(43);
			\draw[edge] (22)edge(32) (23)edge(33) (22)edge(23) (22)edge(33);
			% Center
			\draw[edge] (32)edge(special) (42)edge(special) (33)edge(44) (43)edge(44) (42)edge(43);
			%%% Decoration
			\draw[edge] (22)edge(d22) (23)edge(d23);
			\node[] at (1.0, 1.25) {\mytikztextsize $x_i$};
			\node[] at (2.125, 1.25) {\mytikztextsize $x_i$};

			%%% Right part
			% Bridge to center
			\draw[edge] (52)edge(62) (53)edge(63) (62)edge(63) (53)edge(62);
			\draw[edge] (62)edge(72) (63)edge(73) (72)edge(73) (63)edge(72);
			% Center
			\draw[edge] (62)edge(special) (52)edge(special) (63)edge(54) (53)edge(54) (52)edge(53);
			% Decoration
			\draw[edge] (72)edge(d72) (73)edge(d73);
			\node[] at (3.50,  1.25) {\mytikztextsize $\bar{x}_i$};
			\node[] at (2.375, 1.25) {\mytikztextsize $\bar{x}_i$};

			%%% Center piece
			% Center peace and the~one above
			\draw[bedge] (44)edge(45) (54)edge(55) (44)edge(55);
			\draw[bedge] (45)edge(46) (55)edge(56) (45)edge(56);
			\draw[bedge] (44)edge(54) (45)edge(55) (46)edge(56);
			%%% Decoration
			\draw[bedge] (46)edge(d46) (56)edge(d56);
			\node[] at (2.25, 2.75)  {\mytikztextsize $t$};
			\node[] at (2.25, 1.875) {\mytikztextsize $t$};

			\begin{scope}[xshift=4cm]
				\node[vertex]    (13) at (0.75, 1.50) {};
				\node[]         (d13) at (0.50, 1.50) {};
				\node[vertex]    (14) at (0.75, 2.00) {};
				\node[]         (d14) at (0.50, 2.00) {};
				\node[vertex]    (23) at (1.00, 1.50) {};
				\node[vertex]    (24) at (1.00, 2.00) {};
				\node[vertex]    (31) at (1.50, 0.75) {};
				\node[]         (d31) at (1.50, 0.50) {};
				\node[vertex]    (32) at (1.50, 1.00) {};
				\node[vertexSig] (33) at (1.50, 1.50) {};
				\node[vertexSig] (34) at (1.50, 2.00) {};
				\node[vertex]    (35) at (1.50, 2.50) {};
				\node[vertex]    (36) at (1.50, 2.75) {};
				\node[]         (d36) at (1.50, 3.00) {};
				\node[vertex]    (41) at (2.00, 0.75) {};
				\node[]         (d41) at (2.00, 0.50) {};
				\node[vertex]    (42) at (2.00, 1.00) {};
				\node[vertexSig] (43) at (2.00, 1.50) {};
				\node[vertexSig] (44) at (2.00, 2.00) {};
				\node[vertex]    (45) at (2.00, 2.50) {};
				\node[vertex]    (46) at (2.00, 2.75) {};
				\node[]         (d46) at (2.00, 3.00) {};
				\node[vertex]    (53) at (2.50, 1.50) {};
				\node[vertex]    (54) at (2.50, 2.00) {};
				\node[vertex]    (63) at (2.75, 1.50) {};
				\node[]         (d63) at (3.00, 1.50) {};
				\node[vertex]    (64) at (2.75, 2.00) {};
				\node[]         (d64) at (3.00, 2.00) {};

				%%% Center
				\draw[edge] (33)edge(34);
				\draw[edge] (43)edge(44);
				\draw[bedge] (34)edge(44);
				\draw[redge] (33)edge(43);
				%%% Labels
				\node[] at (1.750, 1.750) {\mytikztextsize $B_i$};
				\node[] at (1.375, 1.700) {\mytikztextsize $x_i$};
				\node[] at (2.125, 1.800) {\mytikztextsize $\bar{x}_i$};
				\node[] at (1.700, 2.125) {\mytikztextsize $t$};
				\node[] at (1.800, 1.375) {\mytikztextsize $f$};

				%%% Left
				\draw[edge] (23)edge(33) (24)edge(34) (23)edge(24) (23)edge(34);
				\draw[edge] (13)edge(23) (14)edge(24) (13)edge(14) (13)edge(24);
				\draw[edge] (13)edge(d13) (14)edge(d14);
				\node[] at (0.50, 1.75) {\mytikztextsize $x_i$};

				%%% Right
				\draw[edge] (43)edge(53) (44)edge(54) (53)edge(54) (43)edge(54);
				\draw[edge] (53)edge(63) (54)edge(64) (63)edge(64) (53)edge(64);
				\draw[edge] (63)edge(d63) (64)edge(d64);
				\node[] at (3.00, 1.75) {\mytikztextsize $\bar{x}_i$};

				%%% Top
				\draw[bedge] (34)edge(35) (44)edge(45) (35)edge(45) (35)edge(44);
				\draw[bedge] (35)edge(36) (45)edge(46) (36)edge(46) (36)edge(45);
				\draw[bedge] (36)edge(d36) (46)edge(d46);
				\node[] at (1.75, 3.00) {\mytikztextsize $t$};

				%%% Bottom
				\draw[redge] (32)edge(33) (42)edge(43) (32)edge(42) (33)edge(42);
				\draw[redge] (31)edge(32) (41)edge(42) (31)edge(41) (32)edge(41);
				\draw[redge] (31)edge(d31) (41)edge(d41);
				\node[] at (1.75, 0.50) {\mytikztextsize $f$};

			\end{scope}
		\end{tikzpicture}
	\end{figure}

	\vfill

	\begin{figure}[h]
		\centering
		% \begin{tikzpicture}[scale=1.50] % 4:3
		\begin{tikzpicture}[scale=1.40] % 16:9
			%%% Center
			\node[vertexSig] (35) at (1.5, 2.5) {};
			\node[vertexSig] (53) at (2.5, 1.5) {};
			\node[vertex]    (57) at (2.5, 3.5) {};
			\node[vertex]    (75) at (3.5, 2.5) {};
			\node[] at (2.5, 2.5) {\mytikztextsize $C_i$};

			%%%%%%%%%%%%%%%%%%%%%%%%%%%%%%%%%%%%%%%%%%%%%%%%%%%%%%%%%%%%%%%%%%%%%%%%%%%%
			%%% t
			%%%%%%%%%%%%%%%%%%%%%%%%%%%%%%%%%%%%%%%%%%%%%%%%%%%%%%%%%%%%%%%%%%%%%%%%%%%%
			\node[vertex] (13) at (0.5, 1.5) {};
			\node[vertex] (14) at (0.5, 2.0) {};
			\node[vertex] (23) at (1.0, 1.5) {};
			\node[vertex] (24) at (1.0, 2.0) {};
			\draw[bedge] (13)edge(23) (14)edge(24) (13)edge(14) (23)edge(24) (13)edge(24);
			\draw[bedge] (53)edge(35) (53)edge(23) (35)edge(24) (53)edge(24);
			%%%% Extensions
			\node[] (13d) at (0.25, 1.5 ) {};
			\node[] (14d) at (0.25, 2.0 ) {};
			\draw[bedge] (13)edge(13d) (14)edge(14d);
			%%%% Labels
			\node[] at (0.25, 1.75) {\mytikztextsize $t$};
			\node[] at (2.125, 2.125) {\mytikztextsize $t$};

			%%%%%%%%%%%%%%%%%%%%%%%%%%%%%%%%%%%%%%%%%%%%%%%%%%%%%%%%%%%%%%%%%%%%%%%%%%%%
			%%% x_1
			%%%%%%%%%%%%%%%%%%%%%%%%%%%%%%%%%%%%%%%%%%%%%%%%%%%%%%%%%%%%%%%%%%%%%%%%%%%%
			\node[vertex]    (06)   at (0.25, 3.0 ) {};
			\node[vertex]    (07)   at (0.25, 3.5 ) {};
			\node[vertexSig] (16)   at (0.5 , 3.0 ) {};
			\node[vertexSig] (17)   at (0.5 , 3.5 ) {};
			\node[vertexSig] (26)   at (1.25, 3.0 ) {};
			\node[vertexSig] (27)   at (1.25, 3.5 ) {};
			\node[vertex]    (36)   at (1.5 , 3.0 ) {};
			\node[vertex]    (37)   at (1.5 , 3.5 ) {};
			\node[vertex]    (46)   at (2.0 , 3.0 ) {};
			\node[vertexSig] (p1m1) at (1.0,  3.25) {}; % prism x_1 middle
			\node[vertexSig] (p1m2) at (0.75, 3.25) {};
			\node[vertex]    (p1t1) at (1.0,  2.75) {}; % prism x_1 true
			\node[vertex]    (p1t2) at (0.75, 2.75) {};
			%%%% Construction
			\draw[edge] (35)edge(46) (46)edge(57) (57)edge(37) (36)edge(37) (46)edge(37);
			\draw[edge] (37)edge(27) (36)edge(26) (37)edge(26) (35)edge(36) (46)edge(36) (27)edge(26);
			%%%% Prism
			\draw[bedge] (26)edge(16) (27)edge(17); % linking horizontal edges
			\draw[edge] (26)edge(p1m1) (27)edge(p1m1); % left triangle
			\draw[edge] (16)edge(p1m2) (17)edge(p1m2); % right triangle
			\draw[bedge] (p1m1)edge(p1m2) (p1t1)edge(p1t2);
			\draw[bedge] (p1m1)edge(p1t1) (p1m2)edge(p1t2) (p1m1)edge(p1t2);
			%%%% Train
			\draw[edge] (16)edge(17) (06)edge(07); % vertical
			\draw[edge] (16)edge(06) (17)edge(07); % horizontal
			\draw[edge] (16)edge(07); % diagonal
			%%%% Extensions
			\node[] (06d) at (0.0 , 3.0) {};
			\node[] (07d) at (0.0 , 3.5) {};
			\draw[edge] (07)edge(07d) (06)edge(06d);
			\node[] (p1t1d) at (1.0 , 2.5) {};
			\node[] (p1t2d) at (0.75, 2.5) {};
			\draw[bedge] (p1t1)edge(p1t1d) (p1t2)edge(p1t2d);
			%%%% Labels
			\node[] at (1.875, 2.625) {\mytikztextsize $\hat{x}_1$};
			\node[] at (2.375, 3.125) {\mytikztextsize $\hat{x}_1$};
			\node[] at (0.0  , 3.25 ) {\mytikztextsize $\hat{x}_1$};
			\node[] at (0.875, 2.625) {\mytikztextsize $t$};


			%%%%%%%%%%%%%%%%%%%%%%%%%%%%%%%%%%%%%%%%%%%%%%%%%%%%%%%%%%%%%%%%%%%%%%%%%%%%
			%%% x_2
			%%%%%%%%%%%%%%%%%%%%%%%%%%%%%%%%%%%%%%%%%%%%%%%%%%%%%%%%%%%%%%%%%%%%%%%%%%%%
			\node[vertex]    (66)   at (3.0 , 3.0 ) {};
			\node[vertex]    (76)   at (3.5 , 3.0 ) {};
			\node[vertex]    (77)   at (3.5 , 3.5 ) {};
			\node[vertexSig] (86)   at (3.75, 3.0 ) {};
			\node[vertexSig] (87)   at (3.75, 3.5 ) {};
			\node[vertexSig] (96)   at (4.5 , 3.0 ) {};
			\node[vertexSig] (97)   at (4.5 , 3.5 ) {};
			\node[vertex]    (A6)   at (4.75, 3.0 ) {};
			\node[vertex]    (A7)   at (4.75, 3.5 ) {};
			\node[vertexSig] (p2m1) at (4.0,  3.25) {}; % prism 2 middle
			\node[vertexSig] (p2m2) at (4.25, 3.25) {};
			\node[vertex]    (p2t1) at (4.0,  2.75) {}; % prism 2 true
			\node[vertex]    (p2t2) at (4.25, 2.75) {};
			%%%% Construction
			\draw[edge] (75)edge(66) (66)edge(57) (57)edge(77) (76)edge(77) (66)edge(77);
			\draw[edge] (77)edge(87) (76)edge(86) (77)edge(86) (75)edge(76) (66)edge(76) (87)edge(86);
			%%%% Prism
			\draw[bedge] (86)edge(96) (87)edge(97); % linking horizontal edges
			\draw[edge] (86)edge(p2m1) (87)edge(p2m1); % left triangle
			\draw[edge] (96)edge(p2m2) (97)edge(p2m2); % right triangle
			\draw[bedge] (p2m1)edge(p2m2) (p2t1)edge(p2t2);
			\draw[bedge] (p2m1)edge(p2t1) (p2m2)edge(p2t2) (p2m1)edge(p2t2);
			%%%% Train
			\draw[edge] (96)edge(97) (A6)edge(A7); % vertical
			\draw[edge] (96)edge(A6) (97)edge(A7); % horizontal
			\draw[edge] (96)edge(A7); % diagonal
			%%%% Extensions
			\node[] (A6d) at (5.0 , 3.0) {};
			\node[] (A7d) at (5.0 , 3.5) {};
			\draw[edge] (A7)edge(A7d) (A6)edge(A6d);
			\node[] (p2t1d) at (4.0 , 2.5) {};
			\node[] (p2t2d) at (4.25, 2.5) {};
			\draw[bedge] (p2t1)edge(p2t1d) (p2t2)edge(p2t2d);
			%%%% Labels
			\node[] at (2.625, 3.125) {\mytikztextsize $\hat{x}_2$};
			\node[] at (3.125, 2.625) {\mytikztextsize $\hat{x}_2$};
			\node[] at (5.00 , 3.25 ) {\mytikztextsize $\hat{x}_2$};
			\node[] at (4.125, 2.625) {\mytikztextsize $t$};

			%%%%%%%%%%%%%%%%%%%%%%%%%%%%%%%%%%%%%%%%%%%%%%%%%%%%%%%%%%%%%%%%%%%%%%%%%%%%
			%%% x_3
			%%%%%%%%%%%%%%%%%%%%%%%%%%%%%%%%%%%%%%%%%%%%%%%%%%%%%%%%%%%%%%%%%%%%%%%%%%%%
			\node[vertex]    (64)   at (3.0 , 2.0 ) {};
			\node[vertex]    (74)   at (3.5 , 2.0 ) {};
			\node[vertex]    (73)   at (3.5 , 1.5 ) {};
			\node[vertexSig] (84)   at (3.75, 2.0 ) {};
			\node[vertexSig] (83)   at (3.75, 1.5 ) {};
			\node[vertexSig] (94)   at (4.5 , 2.0 ) {};
			\node[vertexSig] (93)   at (4.5 , 1.5 ) {};
			\node[vertex]    (A4)   at (4.75, 2.0 ) {};
			\node[vertex]    (A3)   at (4.75, 1.5 ) {};
			\node[vertexSig] (p3m1) at (4.0,  1.75) {}; % prism 3 middle
			\node[vertexSig] (p3m2) at (4.25, 1.75) {};
			\node[vertex]    (p3t1) at (4.0,  2.25) {}; % prism 3 true
			\node[vertex]    (p3t2) at (4.25, 2.25) {};
			%%%% Construction
			\draw[edge] (75)edge(64) (64)edge(53) (53)edge(73) (74)edge(73) (64)edge(73);
			\draw[edge] (73)edge(83) (74)edge(84) (73)edge(84) (75)edge(74) (64)edge(74) (83)edge(84);
			%%%% Prism
			\draw[bedge] (84)edge(94) (83)edge(93); % linking horizontal edges
			\draw[edge] (84)edge(p3m1) (83)edge(p3m1); % left triangle
			\draw[edge] (94)edge(p3m2) (93)edge(p3m2); % right triangle
			\draw[bedge] (p3m1)edge(p3m2) (p3t1)edge(p3t2);
			\draw[bedge] (p3m1)edge(p3t1) (p3m2)edge(p3t2) (p3m1)edge(p3t2);
			%%%% Train
			\draw[edge] (94)edge(93) (A4)edge(A3); % vertical
			\draw[edge] (94)edge(A4) (93)edge(A3); % horizontal
			\draw[edge] (94)edge(A3); % diagonal
			%%%% Extensions
			\node[] (A4d) at (5.0 , 2.0) {};
			\node[] (A3d) at (5.0 , 1.5) {};
			\draw[edge] (A3)edge(A3d) (A4)edge(A4d);
			\node[] (p3t1d) at (4.0 , 2.5) {};
			\node[] (p3t2d) at (4.25, 2.5) {};
			\draw[bedge] (p3t1)edge(p3t1d) (p3t2)edge(p3t2d);
			%%%% Labels
			\node[] at (3.125, 2.375) {\mytikztextsize $\hat{x}_3$};
			\node[] at (2.625, 1.875) {\mytikztextsize $\hat{x}_3$};
			\node[] at (5.00 , 1.75 ) {\mytikztextsize $\hat{x}_3$};
			\node[] at (4.125, 2.375) {\mytikztextsize $t$};

		\end{tikzpicture}
	\end{figure}

\end{frame}

\section{FPT algorithm for NAC-coloring counting}

\begin{frame}
	\frametitle{\MSO{} logic}
	\begin{itemize}
		\item
		      We defined the NAC-coloring existence problem in \MSO{}.
		\item
		      Hence, as of Courcelle's theorem, there exists an FPT algorithm
		      for NAC-coloring existence parametrized by treewidth.
	\end{itemize}
\end{frame}

\begin{frame}
	\frametitle{State space}

	For a bag
\end{frame}

\section{Algorithms for NAC-coloring search}

\begin{frame}
	\frametitle{NAC-coloring search}
	\begin{itemize}
		\item
		      Naive approach tries all the red-blue-colorings.
		\item
		      Basic optimizations (also in FlexRiLoG):
		      \begin{itemize}
			      \item
			            Ignore symmetric NAC-colorings.
			      \item
			            Color \( \triangle \)-connected components as a whole.
		      \end{itemize}
	\end{itemize}
\end{frame}

\begin{frame}
	\frametitle{Monochromatic classes}

	\begin{itemize}
		\item
		      \( \triangle \)-connected component is a monochromatic class.
		\item
		      An edge incident to vertices in the same monochromatic component belongs
		      to the same monochromatic component.
		\item
		      If there are two edges over a monochromatic component,
		      then they belong to the same monochromatic component.
	\end{itemize}

	\begin{figure}[h]
		\centering
		\begin{tikzpicture}[scale=2]
			\node[vertex] (0) at (0, 0) {};
			\node[vertex] (1) at (1, 0.5) {};
			\node[vertex] (2) at (2, 0) {};
			\node[vertex] (3) at (0.5, 0.866) {};
			\node[vertex] (4) at (1.5, 0.866) {};
			\draw[redge] (0)edge(1) (1)edge(2) (0)edge(3) (1)edge(3) (1)edge(4) (2)edge(4) (3)edge(4)  ;
			\draw[edge]  (0)edge(2)  ;
		\end{tikzpicture}
		\qquad
		\begin{tikzpicture}[scale=2]
			\node[vertex] (0) at (0, 0) {};
			\node[vertex] (1) at (1, 0.5) {};
			\node[vertex] (2) at (2, 0) {};
			\node[vertex] (3) at (0.5, 0.866) {};
			\node[vertex] (4) at (1.5, 0.866) {};
			\node[vertex,label={north:$v$}] (5) at (1, 0) {};
			\draw[redge] (0)edge(1) (1)edge(2) (0)edge(3) (1)edge(3) (1)edge(4) (2)edge(4) (3)edge(4)  ;
			\draw[edge]  (0)edge(5) (2)edge(5)  ;
		\end{tikzpicture}
	\end{figure}
\end{frame}

\begin{frame}
	\frametitle{NAC-coloring search}
	\begin{itemize}
		\item
		      NAC-coloring restricted to a subgraph
		      is monochromatic or a NAC-coloring.
		\item
		      Strategy:
		      \begin{itemize}
			      \item
			            Decompose into smaller subgraphs.
			      \item
			            Find all the NAC-colorings of the subgraphs.
			      \item
			            Choose colorings merge order for the subgraphs.
		      \end{itemize}
		\item
		      Multiple heuristics for each stage.
	\end{itemize}
\end{frame}

\begin{frame}
	\frametitle{}

\end{frame}
\begin{frame}
	\frametitle{}

\end{frame}
\begin{frame}
	\frametitle{}

\end{frame}
\begin{frame}
	\frametitle{}

\end{frame}
\begin{frame}
	\frametitle{}

\end{frame}
%%%%%%%%%%%%%%%%%%%%%%%%%%%%%%%%%%%%%%%%%%%%%%%%%%%%%%%%%%%%%%%%%%%%%%%%%%%%%%%%

%%%%%%%%%%%%%%%%%%%%%%%%%%%%%%%%%%%%%%%%%%%%%%%%%%%%%%%%%%%%%%%%%%%%%%%%%%%%%%%%
\begin{frame}
	\frametitle{FPT algorithm}
	\begin{itemize}
		\item
		      Algorithms polynomial in graph size with a factor \( f(k) \), where~\( k \) is a graph's parameter.
		\item
		      Algorithm for NAC-coloring counting parametrized by treewidth \( k \).
		\item
		      Treewidth represents similarity of a graph with trees.
		\item
		      Decomposition tree where nodes are bags of vertices\newline{}(vertex cuts) of the original graph.
		\item
		      Dynamic programming on the decomposition tree.
	\end{itemize}
\end{frame}

\begin{frame}
	\frametitle{FPT algorithm}
	\begin{itemize}
		\item
		      Information about connectivity in bags needs to be preserved.
		\item
		      Super-exponential complexity in \( k \), linear in the graph size.
		\item
		      Recursive definition of the cache function.
		\item
		      Additional optimizations proposed and variations discussed.
		\item
		      Not implemented, proved only.
	\end{itemize}
\end{frame}

% \begin{frame}
% 	\frametitle{Stable cuts}
% 	\begin{itemize}
% 		\item
% 		      A \emph{stable cut} is a vertex cut that is also an independent set.
% 		\item
% 		      If a stable cut exists, a NAC-coloring also trivially exists.
% 		\item
% 		      Algorithm for stable cut search in flexible graphs implemented.
% 	\end{itemize}
% \end{frame}


\begin{frame}
	\frametitle{Benchmarks}
	\begin{itemize}
		\item
		      Many NAC-colorings/few NAC-colorings/no NAC-coloring.
		\item
		      Any NAC-coloring/all NAC-colorings/the number of NAC-colorings.
	\end{itemize}
\end{frame}

\begin{frame}
	\frametitle{Graphs with many NAC-colorings}
	\begin{itemize}
		\item
		      Improved naive approach is the fastest for find some NAC-coloring.
		\item
		      Fast for finding any NAC-coloring on over 100 vertex graphs.
		\item
		      Listing all NAC-colorings runs fast enough for 30 vertex graphs,
		      naive approach is slow for small graphs.
	\end{itemize}
\end{frame}

\begin{frame}
	\frametitle{Graphs with no/few NAC-colorings}
	\begin{itemize}
		\item
		      Naive search in not feasible.
		\item
		      Monochromatic classes reduce search space significantly.
		      \begin{itemize}
			      \item Hard to randomly find hard graphs.
		      \end{itemize}
		\item
		      Tens of vertices/monochromatic classes run in few seconds.
	\end{itemize}
\end{frame}

\begin{frame}
	\frametitle{NAC-colorings search}
	\begin{itemize}
		\item
		      Extension of our paper from VýLeT.
		\item
		      Code contributed to PyRigi.
		\item
		      I am ready for your questions and discussion.
	\end{itemize}
\end{frame}
\end{document}
