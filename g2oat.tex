\newif\ifaspectratiobasic
\aspectratiobasictrue % uncomment for 4:3

\ifaspectratiobasic
% 4:3
\newcommand\mytikztextsize{}
\newcommand\mytikzscale{1.50}
\newcommand\myaspectratio{43}
\else
% 16:9
\newcommand\mytikztextsize{\small}
\newcommand\mytikzscale{1.40}
\newcommand\myaspectratio{169}
\fi

\documentclass[aspectratio=\myaspectratio]{beamer}
\usetheme{moloch}

\usepackage[czech,english]{babel}
\usepackage{booktabs}
\usepackage{graphicx}
\usepackage{tikz}
\usepackage{amsthm}
\usepackage{mathtools}
\usepackage{amssymb}

\title{NAC-COLORINGS SEARCH\@:\newline COMPLEXITY AND ALGORITHMS}
\author{Petr Laštovička, Jan Legerský}
\institute{Czech Technical University, Faculty of Information Technology}
\date{September 22, 2025}


%%%%%%%%%%%%%%%
%% Tikz
%%%%%%%%%%%%%%%
\colorlet{ecol}{black!50!white}
\definecolor{colR}{rgb}{.932,.172,.172} %x11 Firebrick2
\definecolor{colB}{rgb}{.255,.41,.884} %svgnames RoyalBlue
\definecolor{colOrange}{RGB}{255,191,0} %Amber
%% Tikz styles
\tikzstyle{vertex}=[circle, draw, fill=black, inner sep=0pt, minimum size=4pt]
\tikzstyle{fvertex}=[circle, draw, fill=white, inner sep=0pt, minimum size=4pt]
\tikzstyle{vertexSig}=[circle, draw, fill=colOrange, inner sep=0pt, minimum size=4pt]
\tikzstyle{edge}=[line width=1.5pt,ecol]
% dotted, densely dotted, loosely dotted, dashed
\tikzstyle{dots}=[dotted dash=1.5pt]
\tikzstyle{redge}=[edge,colR]
\tikzstyle{bedge}=[edge,colB]
\tikzstyle{yedge}=[edge,colOrange]
%%%%%%%%%%%%%%%

\newcommand{\BenchFigureScale}{0.42}

\newcommand{\includegraphicsmax}[2][]{%
    \includegraphics[width=0.80\textwidth,height=0.69\textheight,keepaspectratio,#1]{#2}%
}

\begin{document}
\maketitle

% \section{Goals}
\begin{frame}
	\frametitle{Contents}
	\begin{itemize}
		\item
		      Introduction to Rigidity theory, flexible realizations and NAC-colorings.
		\item
		      NAC-coloring existence is NP-complete also on graphs with maximum degree five.
		\item
		      FPT algorithm for NAC-coloring counting parametrized by treewidth.
		\item
		      Design, implementation and evaluation of an algorithm and heuristics for NAC-coloring search.
	\end{itemize}
\end{frame}

\section{Rigidity theory}

\begin{frame}
	\frametitle{Rigidity theory}

	Properties of objects formed from joints and rigid bars.

	\begin{definition}[\( d \)-realization]
		A~\emph{\( d \)-realization} of a~graph \( G \) is a~mapping \( p: V(G) \to \R^d \).
	\end{definition}
	%
	\begin{definition}[Framework]
		A~\emph{framework} is a~pair of a~graph \( G \) and its realization.
	\end{definition}
\end{frame}

\begin{frame}
	\frametitle{Rigidity theory}
	\begin{definition}[Nontrivial flex]
		A~\emph{nontrivial flex} of framework \( (G, p) \) is a~continuous curve of realizations \( p_t \)
		for \( 0 \le t \le 1\) such that
		\( p_0 = p \) and for~all \( 0 < t \le 1 \)
		we have that
		\( \|p_t(u) - p_t(v)\| = \|p(u) - p(v)\|\) for~every \( \{u, v\} \in E(G) \),
		but \( \|p_t(u) - p_t(v)\| \ne \|p(u) - p(v)\| \) for~some \( u, v \in V (G) \).
	\end{definition}
	\begin{definition}[\( d \)-flexible, \( d \)-rigid]
		A~framework is \emph{\( d \)-flexible}, if~it has a~nontrivial flex.
		Otherwise, it~is \emph{\( d \)-rigid}.
	\end{definition}
\end{frame}
\begin{frame}
	\frametitle{Paradoxical flexibility}
	\begin{definition}[Flexible \& rigid graphs]
		A~graph is \emph{(generically) \( d \)-flexible} if~almost all of
		its \( d \)-realizations are \( d \)-flexible.
		A~graph is \emph{(generically) \( d \)-rigid} if~almost all of
		its \( d \)-realizations are \( d \)-rigid.
	\end{definition}
	\begin{figure}[ht]
		\centering
		\begin{tikzpicture}[rotate=90,scale=1.5]
			\node[vertex] (a) at (0,0) {};
			\node[vertex] (b) at (1,0) {};
			\node[vertex] (c) at (0.5,0.5) {};
			\node[vertex] (d) at (0,1.5) {};
			\node[vertex] (e) at (1,1.5) {};
			\node[vertex] (f) at (0.5,1) {};
			\draw[edge] (a)edge(b) (b)edge(c) (c)edge(a) (d)edge(e) (e)edge(f) (f)edge(d) ;
			\draw[edge] (a)edge(d) (b)edge(e) (c)edge(f);
		\end{tikzpicture}
		\qquad
		\qquad
		\begin{tikzpicture}[rotate=90,scale=1.5]
			\node[vertex] (a) at (0.00,0) {};
			\node[vertex] (b) at (1.00,0) {};
			\node[vertex] (c) at (0.50,0.5) {};
			\node[vertex] (d) at (0.25,1) {};
			\node[vertex] (e) at (1.25,1) {};
			\node[vertex] (f) at (0.75,1.5) {};
			\draw[edge] (a)edge(b) (b)edge(c) (c)edge(a) (d)edge(e) (e)edge(f) (f)edge(d) ;
			\draw[edge] (a)edge(d) (b)edge(e) (c)edge(f);
		\end{tikzpicture}
	\end{figure}
\end{frame}

\section{NAC-coloring}

\begin{frame}
	\frametitle{NAC-coloring}
	\begin{definition}[NAC-coloring]
		Let~\( G \) be a~graph and \( \delta: E(G) \to \{ \red, \blue \} \)
		be a~coloring of its edges:
		%
		\begin{itemize}
			\item A~cycle in~\( G \) is a~\( \red \) cycle, if~all its edges are \( \red \),
			      analogously for~\( \blue \) cycles.
			\item A~cycle in~\( G \) is an~\emph{almost \( \red \) cycle},
			      if~exactly one of its edges is \( \blue \).
			      \emph{Almost \( \blue \) cycle} is defined analogously.
		\end{itemize}
		%
		The~coloring~\( \delta \) is called a~\emph{NAC-coloring}, if~it is surjective
		and there are no red nor blue almost cycles.
	\end{definition}
\end{frame}
\begin{frame}
	\frametitle{NAC-coloring}
	\begin{lemma}%
		Let~\( G \) be a~graph. If~\( \delta: E(G) \to \{ \red, \blue \} \) is a~coloring of edges,
		then there are no almost cycles in~\( G \) if~and only if~the~connected components
		of \( G[E_\red] \) and \( G[E_\blue] \)%
		are induced subgraphs of \( G \).
	\end{lemma}
\end{frame}

\begin{frame}
	\frametitle{NAC-coloring}

	\begin{theorem}
		A~connected graph \( G \) with at least one edge has a~flexible
		quasi-injective \( 2 \)-dimensional realization if~and only if~it has a~NAC-coloring.
	\end{theorem}

	\begin{figure}[ht]
		% original scale * 4/5
		\centering
		\begin{tikzpicture}[scale=1.0] % 1.5
			\node[fvertex] (a) at (0,0) {};
			\node[fvertex] (b) at (1,0) {};
			\node[fvertex] (c) at (0.5,0.5) {};
			\node[fvertex] (d) at (0,1) {};
			\node[fvertex] (e) at (1,1) {};
			\node[fvertex] (f) at (0.5,1.5) {};
			\draw[edge] (a)edge(b) (b)edge(c) (c)edge(a) (d)edge(e) (e)edge(f) (f)edge(d) (a)edge(d) (b)edge(e) (c)edge(f);
		\end{tikzpicture}
		\quad
		\begin{tikzpicture}[scale=1.0] % 1.5
			\node[vertex] (a) at (0,0) {};
			\node[vertex] (b) at (1,0) {};
			\node[vertex] (c) at (0.5,0.5) {};
			\node[vertex] (d) at (0,1.5) {};
			\node[vertex] (e) at (1,1.5) {};
			\node[vertex] (f) at (0.5,1) {};
			\draw[bedge] (a)edge(b) (b)edge(c) (c)edge(a) (d)edge(e) (e)edge(f) (f)edge(d) ;
			\draw[redge] (a)edge(d) (b)edge(e) (c)edge(f);
		\end{tikzpicture}
		\qquad
		\begin{tikzpicture}[scale=0.50] % 0.75
			\draw[black!50!white, dashed] (-1.5,0)edge(2.3,0);
			\draw[black!50!white, dashed] (0,1.55)edge(0,-2.1);
			\node[fvertex] (2) at (1.8, 0) {};
			\node[fvertex] (5) at (-1., 0) {};
			\node[fvertex] (7) at (0.7, 0) {};
			\node[fvertex] (1) at (0, -1.75) {};
			\node[fvertex] (6) at (0,  1.2) {};
			\node[fvertex] (4) at (0, -0.8) {};
			\draw[edge]  (6)edge(5) (5)edge(4) (7)edge(4) (7)edge(6);
			\draw[edge] (2)edge(4) (2)edge(6);
			\draw[edge] (1)edge(5) (7)edge(1) ;
			\draw[edge] (2)edge(1);
		\end{tikzpicture}
		\quad
		\begin{tikzpicture}[scale=0.80] % 1.2
			\node[vertex] (a1) at (-0.5, -0.866025) {};
			\node[vertex] (a2) at (0.5, -0.866025) {};
			\node[vertex] (a3) at (1., 0.) {};
			\node[vertex] (a4) at (0.5, 0.866025) {};
			\node[vertex] (a5) at (-0.5, 0.866025) {};
			\node[vertex] (a6) at (-1.,  0.) {};
			\draw[bedge] (a1)edge(a2) (a1)edge(a4) (a1)edge(a6);
			\draw[redge] (a2)edge(a3) (a2)edge(a5) (a3)edge(a4) (a3)edge(a6) (a5)edge(a4) (a5)edge(a6);
		\end{tikzpicture}
		\quad
		\begin{tikzpicture}[scale=0.80] % 1.2
			\node[vertex] (a1) at (-0.5, -0.866025) {};
			\node[vertex] (a2) at (0.5, -0.866025) {};
			\node[vertex] (a3) at (1., 0.) {};
			\node[vertex] (a4) at (0.5, 0.866025) {};
			\node[vertex] (a5) at (-0.5, 0.866025) {};
			\node[vertex] (a6) at (-1.,  0.) {};
			\draw[bedge] (a1)edge(a2) (a1)edge(a4) (a2)edge(a3) (a3)edge(a4) (a5)edge(a6);
			\draw[redge] (a1)edge(a6) (a2)edge(a5) (a3)edge(a6) (a5)edge(a4) ;
		\end{tikzpicture}
	\end{figure}
\end{frame}

\section{NAC-coloring existence complexity}

\begin{frame}
	\frametitle{Problem's complexity}

	\begin{itemize}
		\item
		      It is NP-complete to decide whether a graph has a~NAC-coloring.
		\item
		      Correspondence with stable cuts (independent vertex cuts).
		\item
		      Graphs where \( |E| \leq 2|V| - 3 \) can be checked polynomially.
		      \begin{itemize}
			      \item Hence maximum degree three graphs are polynomially checkable.
		      \end{itemize}
		\item
		      And other polynomial checks\ldots
		\item
		      We show that it is NP-complete to decide whether
		      a graph with maximum degree five has a NAC-coloring.
	\end{itemize}
\end{frame}

\begin{frame}
	\frametitle{Original reduction}
	\begin{itemize}
		\item
		      Reduction from 3-SAT, a graph gadget.
		\item
		      No bound on the degree of vertices.
		\item
		      Our reduction idea is based on this original construction.
	\end{itemize}
\end{frame}

\begin{frame}
	\frametitle{Graphs with maximum degree five}

	\begin{itemize}
		\item
		      Auxiliary graph is constructed for a 3-SAT formula.
		\item
		      For each variable \( x \),
		      we ensure that \( x \) and \( \bar{x} \)
		      they correspond to different color and different truth value.
		\item
		      For each clause we ensure that at least one of the literals is true.
		\item
		      Color correspondence to true/false.
		\item
		      We use leathers to connect individual gadgets.
		\item
		      Maximum degree is kept at five.
	\end{itemize}
\end{frame}

\begin{frame}
	\frametitle{Reduction from 3-SAT}

	\begin{figure}[h]
		\centering
		% \begin{tikzpicture}[scale=1.50] % 4:3
		\begin{tikzpicture}[scale=1.40] % 16:9
			% for~i in~range(1,8):
			%       for~j in~range(1,5):
			%           print(f"\\node[vertex] ({i}{j}) at ({i/2}, {j/2}) {{}};")
			% \node[vertex] (11) at (0.5, 0.5) {};
			\node[vertex]      (22) at (1.25, 1.00) {};
			\node[]           (d22) at (1.00, 1.00) {};
			\node[vertex]      (23) at (1.25, 1.50) {};
			\node[]           (d23) at (1.00, 1.50) {};
			\node[vertex]      (32) at (1.50, 1.00) {};
			\node[vertex]      (33) at (1.50, 1.50) {};
			\node[vertex]      (42) at (2.00, 1.00) {};
			\node[vertex]      (43) at (2.00, 1.50) {};
			\node[vertexSig]   (44) at (2.00, 1.75) {};
			\node[vertex]      (45) at (2.00, 2.25) {};
			\node[vertex]      (46) at (2.00, 2.50) {};
			\node[]           (d46) at (2.00, 2.75) {};
			\node[vertex]      (52) at (2.50, 1.00) {};
			\node[vertex]      (53) at (2.50, 1.50) {};
			\node[vertexSig]   (54) at (2.50, 1.75) {};
			\node[vertex]      (55) at (2.50, 2.25) {};
			\node[vertex]      (56) at (2.50, 2.50) {};
			\node[]           (d56) at (2.50, 2.75) {};
			\node[vertex]      (62) at (3.00, 1.00) {};
			\node[vertex]      (63) at (3.00, 1.50) {};
			\node[vertex]      (72) at (3.25, 1.00) {};
			\node[]           (d72) at (3.50, 1.00) {};
			\node[vertex]      (73) at (3.25, 1.50) {};
			\node[]           (d73) at (3.50, 1.50) {};
			\node[vertex] (special) at (2.25, 0.75) {};

			\node[] at (2.25, 1.5) {\mytikztextsize $A_i$};

			%%% Left part
			% Bridge to center
			\draw[edge] (32)edge(42) (33)edge(43) (32)edge(33) (32)edge(43);
			\draw[edge] (22)edge(32) (23)edge(33) (22)edge(23) (22)edge(33);
			% Center
			\draw[edge] (32)edge(special) (42)edge(special) (33)edge(44) (43)edge(44) (42)edge(43);
			%%% Decoration
			\draw[edge] (22)edge(d22) (23)edge(d23);
			\node[] at (1.0, 1.25) {\mytikztextsize $x_i$};
			\node[] at (2.125, 1.25) {\mytikztextsize $x_i$};

			%%% Right part
			% Bridge to center
			\draw[edge] (52)edge(62) (53)edge(63) (62)edge(63) (53)edge(62);
			\draw[edge] (62)edge(72) (63)edge(73) (72)edge(73) (63)edge(72);
			% Center
			\draw[edge] (62)edge(special) (52)edge(special) (63)edge(54) (53)edge(54) (52)edge(53);
			% Decoration
			\draw[edge] (72)edge(d72) (73)edge(d73);
			\node[] at (3.50,  1.25) {\mytikztextsize $\bar{x}_i$};
			\node[] at (2.375, 1.25) {\mytikztextsize $\bar{x}_i$};

			%%% Center piece
			% Center peace and the~one above
			\draw[bedge] (44)edge(45) (54)edge(55) (44)edge(55);
			\draw[bedge] (45)edge(46) (55)edge(56) (45)edge(56);
			\draw[bedge] (44)edge(54) (45)edge(55) (46)edge(56);
			%%% Decoration
			\draw[bedge] (46)edge(d46) (56)edge(d56);
			\node[] at (2.25, 2.75)  {\mytikztextsize $t$};
			\node[] at (2.25, 1.875) {\mytikztextsize $t$};

			\begin{scope}[xshift=4cm]
				\node[vertex]    (13) at (0.75, 1.50) {};
				\node[]         (d13) at (0.50, 1.50) {};
				\node[vertex]    (14) at (0.75, 2.00) {};
				\node[]         (d14) at (0.50, 2.00) {};
				\node[vertex]    (23) at (1.00, 1.50) {};
				\node[vertex]    (24) at (1.00, 2.00) {};
				\node[vertex]    (31) at (1.50, 0.75) {};
				\node[]         (d31) at (1.50, 0.50) {};
				\node[vertex]    (32) at (1.50, 1.00) {};
				\node[vertexSig] (33) at (1.50, 1.50) {};
				\node[vertexSig] (34) at (1.50, 2.00) {};
				\node[vertex]    (35) at (1.50, 2.50) {};
				\node[vertex]    (36) at (1.50, 2.75) {};
				\node[]         (d36) at (1.50, 3.00) {};
				\node[vertex]    (41) at (2.00, 0.75) {};
				\node[]         (d41) at (2.00, 0.50) {};
				\node[vertex]    (42) at (2.00, 1.00) {};
				\node[vertexSig] (43) at (2.00, 1.50) {};
				\node[vertexSig] (44) at (2.00, 2.00) {};
				\node[vertex]    (45) at (2.00, 2.50) {};
				\node[vertex]    (46) at (2.00, 2.75) {};
				\node[]         (d46) at (2.00, 3.00) {};
				\node[vertex]    (53) at (2.50, 1.50) {};
				\node[vertex]    (54) at (2.50, 2.00) {};
				\node[vertex]    (63) at (2.75, 1.50) {};
				\node[]         (d63) at (3.00, 1.50) {};
				\node[vertex]    (64) at (2.75, 2.00) {};
				\node[]         (d64) at (3.00, 2.00) {};

				%%% Center
				\draw[edge] (33)edge(34);
				\draw[edge] (43)edge(44);
				\draw[bedge] (34)edge(44);
				\draw[redge] (33)edge(43);
				%%% Labels
				\node[] at (1.750, 1.750) {\mytikztextsize $B_i$};
				\node[] at (1.375, 1.700) {\mytikztextsize $x_i$};
				\node[] at (2.125, 1.800) {\mytikztextsize $\bar{x}_i$};
				\node[] at (1.700, 2.125) {\mytikztextsize $t$};
				\node[] at (1.800, 1.375) {\mytikztextsize $f$};

				%%% Left
				\draw[edge] (23)edge(33) (24)edge(34) (23)edge(24) (23)edge(34);
				\draw[edge] (13)edge(23) (14)edge(24) (13)edge(14) (13)edge(24);
				\draw[edge] (13)edge(d13) (14)edge(d14);
				\node[] at (0.50, 1.75) {\mytikztextsize $x_i$};

				%%% Right
				\draw[edge] (43)edge(53) (44)edge(54) (53)edge(54) (43)edge(54);
				\draw[edge] (53)edge(63) (54)edge(64) (63)edge(64) (53)edge(64);
				\draw[edge] (63)edge(d63) (64)edge(d64);
				\node[] at (3.00, 1.75) {\mytikztextsize $\bar{x}_i$};

				%%% Top
				\draw[bedge] (34)edge(35) (44)edge(45) (35)edge(45) (35)edge(44);
				\draw[bedge] (35)edge(36) (45)edge(46) (36)edge(46) (36)edge(45);
				\draw[bedge] (36)edge(d36) (46)edge(d46);
				\node[] at (1.75, 3.00) {\mytikztextsize $t$};

				%%% Bottom
				\draw[redge] (32)edge(33) (42)edge(43) (32)edge(42) (33)edge(42);
				\draw[redge] (31)edge(32) (41)edge(42) (31)edge(41) (32)edge(41);
				\draw[redge] (31)edge(d31) (41)edge(d41);
				\node[] at (1.75, 0.50) {\mytikztextsize $f$};

			\end{scope}
		\end{tikzpicture}
	\end{figure}

	\vfill

	\begin{figure}[h]
		\centering
		% \begin{tikzpicture}[scale=1.50] % 4:3
		\begin{tikzpicture}[scale=1.40] % 16:9
			%%% Center
			\node[vertexSig] (35) at (1.5, 2.5) {};
			\node[vertexSig] (53) at (2.5, 1.5) {};
			\node[vertex]    (57) at (2.5, 3.5) {};
			\node[vertex]    (75) at (3.5, 2.5) {};
			\node[] at (2.5, 2.5) {\mytikztextsize $C_i$};

			%%%%%%%%%%%%%%%%%%%%%%%%%%%%%%%%%%%%%%%%%%%%%%%%%%%%%%%%%%%%%%%%%%%%%%%%%%%%
			%%% t
			%%%%%%%%%%%%%%%%%%%%%%%%%%%%%%%%%%%%%%%%%%%%%%%%%%%%%%%%%%%%%%%%%%%%%%%%%%%%
			\node[vertex] (13) at (0.5, 1.5) {};
			\node[vertex] (14) at (0.5, 2.0) {};
			\node[vertex] (23) at (1.0, 1.5) {};
			\node[vertex] (24) at (1.0, 2.0) {};
			\draw[bedge] (13)edge(23) (14)edge(24) (13)edge(14) (23)edge(24) (13)edge(24);
			\draw[bedge] (53)edge(35) (53)edge(23) (35)edge(24) (53)edge(24);
			%%%% Extensions
			\node[] (13d) at (0.25, 1.5 ) {};
			\node[] (14d) at (0.25, 2.0 ) {};
			\draw[bedge] (13)edge(13d) (14)edge(14d);
			%%%% Labels
			\node[] at (0.25, 1.75) {\mytikztextsize $t$};
			\node[] at (2.125, 2.125) {\mytikztextsize $t$};

			%%%%%%%%%%%%%%%%%%%%%%%%%%%%%%%%%%%%%%%%%%%%%%%%%%%%%%%%%%%%%%%%%%%%%%%%%%%%
			%%% x_1
			%%%%%%%%%%%%%%%%%%%%%%%%%%%%%%%%%%%%%%%%%%%%%%%%%%%%%%%%%%%%%%%%%%%%%%%%%%%%
			\node[vertex]    (06)   at (0.25, 3.0 ) {};
			\node[vertex]    (07)   at (0.25, 3.5 ) {};
			\node[vertexSig] (16)   at (0.5 , 3.0 ) {};
			\node[vertexSig] (17)   at (0.5 , 3.5 ) {};
			\node[vertexSig] (26)   at (1.25, 3.0 ) {};
			\node[vertexSig] (27)   at (1.25, 3.5 ) {};
			\node[vertex]    (36)   at (1.5 , 3.0 ) {};
			\node[vertex]    (37)   at (1.5 , 3.5 ) {};
			\node[vertex]    (46)   at (2.0 , 3.0 ) {};
			\node[vertexSig] (p1m1) at (1.0,  3.25) {}; % prism x_1 middle
			\node[vertexSig] (p1m2) at (0.75, 3.25) {};
			\node[vertex]    (p1t1) at (1.0,  2.75) {}; % prism x_1 true
			\node[vertex]    (p1t2) at (0.75, 2.75) {};
			%%%% Construction
			\draw[edge] (35)edge(46) (46)edge(57) (57)edge(37) (36)edge(37) (46)edge(37);
			\draw[edge] (37)edge(27) (36)edge(26) (37)edge(26) (35)edge(36) (46)edge(36) (27)edge(26);
			%%%% Prism
			\draw[bedge] (26)edge(16) (27)edge(17); % linking horizontal edges
			\draw[edge] (26)edge(p1m1) (27)edge(p1m1); % left triangle
			\draw[edge] (16)edge(p1m2) (17)edge(p1m2); % right triangle
			\draw[bedge] (p1m1)edge(p1m2) (p1t1)edge(p1t2);
			\draw[bedge] (p1m1)edge(p1t1) (p1m2)edge(p1t2) (p1m1)edge(p1t2);
			%%%% Train
			\draw[edge] (16)edge(17) (06)edge(07); % vertical
			\draw[edge] (16)edge(06) (17)edge(07); % horizontal
			\draw[edge] (16)edge(07); % diagonal
			%%%% Extensions
			\node[] (06d) at (0.0 , 3.0) {};
			\node[] (07d) at (0.0 , 3.5) {};
			\draw[edge] (07)edge(07d) (06)edge(06d);
			\node[] (p1t1d) at (1.0 , 2.5) {};
			\node[] (p1t2d) at (0.75, 2.5) {};
			\draw[bedge] (p1t1)edge(p1t1d) (p1t2)edge(p1t2d);
			%%%% Labels
			\node[] at (1.875, 2.625) {\mytikztextsize $\hat{x}_1$};
			\node[] at (2.375, 3.125) {\mytikztextsize $\hat{x}_1$};
			\node[] at (0.0  , 3.25 ) {\mytikztextsize $\hat{x}_1$};
			\node[] at (0.875, 2.625) {\mytikztextsize $t$};


			%%%%%%%%%%%%%%%%%%%%%%%%%%%%%%%%%%%%%%%%%%%%%%%%%%%%%%%%%%%%%%%%%%%%%%%%%%%%
			%%% x_2
			%%%%%%%%%%%%%%%%%%%%%%%%%%%%%%%%%%%%%%%%%%%%%%%%%%%%%%%%%%%%%%%%%%%%%%%%%%%%
			\node[vertex]    (66)   at (3.0 , 3.0 ) {};
			\node[vertex]    (76)   at (3.5 , 3.0 ) {};
			\node[vertex]    (77)   at (3.5 , 3.5 ) {};
			\node[vertexSig] (86)   at (3.75, 3.0 ) {};
			\node[vertexSig] (87)   at (3.75, 3.5 ) {};
			\node[vertexSig] (96)   at (4.5 , 3.0 ) {};
			\node[vertexSig] (97)   at (4.5 , 3.5 ) {};
			\node[vertex]    (A6)   at (4.75, 3.0 ) {};
			\node[vertex]    (A7)   at (4.75, 3.5 ) {};
			\node[vertexSig] (p2m1) at (4.0,  3.25) {}; % prism 2 middle
			\node[vertexSig] (p2m2) at (4.25, 3.25) {};
			\node[vertex]    (p2t1) at (4.0,  2.75) {}; % prism 2 true
			\node[vertex]    (p2t2) at (4.25, 2.75) {};
			%%%% Construction
			\draw[edge] (75)edge(66) (66)edge(57) (57)edge(77) (76)edge(77) (66)edge(77);
			\draw[edge] (77)edge(87) (76)edge(86) (77)edge(86) (75)edge(76) (66)edge(76) (87)edge(86);
			%%%% Prism
			\draw[bedge] (86)edge(96) (87)edge(97); % linking horizontal edges
			\draw[edge] (86)edge(p2m1) (87)edge(p2m1); % left triangle
			\draw[edge] (96)edge(p2m2) (97)edge(p2m2); % right triangle
			\draw[bedge] (p2m1)edge(p2m2) (p2t1)edge(p2t2);
			\draw[bedge] (p2m1)edge(p2t1) (p2m2)edge(p2t2) (p2m1)edge(p2t2);
			%%%% Train
			\draw[edge] (96)edge(97) (A6)edge(A7); % vertical
			\draw[edge] (96)edge(A6) (97)edge(A7); % horizontal
			\draw[edge] (96)edge(A7); % diagonal
			%%%% Extensions
			\node[] (A6d) at (5.0 , 3.0) {};
			\node[] (A7d) at (5.0 , 3.5) {};
			\draw[edge] (A7)edge(A7d) (A6)edge(A6d);
			\node[] (p2t1d) at (4.0 , 2.5) {};
			\node[] (p2t2d) at (4.25, 2.5) {};
			\draw[bedge] (p2t1)edge(p2t1d) (p2t2)edge(p2t2d);
			%%%% Labels
			\node[] at (2.625, 3.125) {\mytikztextsize $\hat{x}_2$};
			\node[] at (3.125, 2.625) {\mytikztextsize $\hat{x}_2$};
			\node[] at (5.00 , 3.25 ) {\mytikztextsize $\hat{x}_2$};
			\node[] at (4.125, 2.625) {\mytikztextsize $t$};

			%%%%%%%%%%%%%%%%%%%%%%%%%%%%%%%%%%%%%%%%%%%%%%%%%%%%%%%%%%%%%%%%%%%%%%%%%%%%
			%%% x_3
			%%%%%%%%%%%%%%%%%%%%%%%%%%%%%%%%%%%%%%%%%%%%%%%%%%%%%%%%%%%%%%%%%%%%%%%%%%%%
			\node[vertex]    (64)   at (3.0 , 2.0 ) {};
			\node[vertex]    (74)   at (3.5 , 2.0 ) {};
			\node[vertex]    (73)   at (3.5 , 1.5 ) {};
			\node[vertexSig] (84)   at (3.75, 2.0 ) {};
			\node[vertexSig] (83)   at (3.75, 1.5 ) {};
			\node[vertexSig] (94)   at (4.5 , 2.0 ) {};
			\node[vertexSig] (93)   at (4.5 , 1.5 ) {};
			\node[vertex]    (A4)   at (4.75, 2.0 ) {};
			\node[vertex]    (A3)   at (4.75, 1.5 ) {};
			\node[vertexSig] (p3m1) at (4.0,  1.75) {}; % prism 3 middle
			\node[vertexSig] (p3m2) at (4.25, 1.75) {};
			\node[vertex]    (p3t1) at (4.0,  2.25) {}; % prism 3 true
			\node[vertex]    (p3t2) at (4.25, 2.25) {};
			%%%% Construction
			\draw[edge] (75)edge(64) (64)edge(53) (53)edge(73) (74)edge(73) (64)edge(73);
			\draw[edge] (73)edge(83) (74)edge(84) (73)edge(84) (75)edge(74) (64)edge(74) (83)edge(84);
			%%%% Prism
			\draw[bedge] (84)edge(94) (83)edge(93); % linking horizontal edges
			\draw[edge] (84)edge(p3m1) (83)edge(p3m1); % left triangle
			\draw[edge] (94)edge(p3m2) (93)edge(p3m2); % right triangle
			\draw[bedge] (p3m1)edge(p3m2) (p3t1)edge(p3t2);
			\draw[bedge] (p3m1)edge(p3t1) (p3m2)edge(p3t2) (p3m1)edge(p3t2);
			%%%% Train
			\draw[edge] (94)edge(93) (A4)edge(A3); % vertical
			\draw[edge] (94)edge(A4) (93)edge(A3); % horizontal
			\draw[edge] (94)edge(A3); % diagonal
			%%%% Extensions
			\node[] (A4d) at (5.0 , 2.0) {};
			\node[] (A3d) at (5.0 , 1.5) {};
			\draw[edge] (A3)edge(A3d) (A4)edge(A4d);
			\node[] (p3t1d) at (4.0 , 2.5) {};
			\node[] (p3t2d) at (4.25, 2.5) {};
			\draw[bedge] (p3t1)edge(p3t1d) (p3t2)edge(p3t2d);
			%%%% Labels
			\node[] at (3.125, 2.375) {\mytikztextsize $\hat{x}_3$};
			\node[] at (2.625, 1.875) {\mytikztextsize $\hat{x}_3$};
			\node[] at (5.00 , 1.75 ) {\mytikztextsize $\hat{x}_3$};
			\node[] at (4.125, 2.375) {\mytikztextsize $t$};

		\end{tikzpicture}
	\end{figure}

\end{frame}

\begin{frame}
	\frametitle{NP-completeness for maximum degree four}

	\begin{itemize}
		\item
		      We show that the problem is NP-complete for graphs with average degree
		      \( 4 + \varepsilon, \varepsilon > 0 \).
		\item
		      Maximum degree four remains an open question.
	\end{itemize}
\end{frame}

\section{FPT algorithm for NAC-coloring counting}

\begin{frame}
	\frametitle{\MSO{} logic}
	\begin{itemize}
		\item
		      We defined the NAC-coloring existence problem in \MSO{}.
		\item
		      Hence, as of Courcelle's theorem, there exists an FPT algorithm
		      for NAC-coloring existence parametrized by treewidth.
	\end{itemize}
\end{frame}

\begin{frame}
	\frametitle{Our FPT algorithm}

	\begin{itemize}
		\item
		      Parametrized by treewidth.
		\item
		      NAC-coloring counting, not only existence.
		\item
		      Uses a nice-tree decomposition \( T \) of width \( k \)
		      with \IntroduceEdgeNode{}s.
		\item
		      Dynamic programming algorithm.
	\end{itemize}
\end{frame}

\begin{frame}
	\frametitle{State space}

	\begin{itemize}
		\item
		      Let us have a node \( t \) in a nice tree decomposition \( T \),
		      a~vertex bag \( X_t \) and
		      a~graph \( G_t \) already formed by the child nodes.
		\item
		      Let us have a part \( P \in \mathcal{P}(X_t) \)
		      and a binary relation \( R \) on \( P \).
		\item
		      We define a \emph{state} \( s \) as a tuple \( P_\red, R_\red, P_\blue, R_\blue \).
		      We define \emph{state space} \( S_t \) of \( t \) as the set of all such states \( s \).
		\item
		      By red, resp.\ blue, halve of a state \( s \)
		      we mean \( P_\red, R_\red \), resp. \( P_\blue, R_\blue \).
	\end{itemize}
\end{frame}

\begin{frame}
	\frametitle{State space}
	\begin{itemize}
		\item
		      An \emph{almost NAC-coloring} is a red-blue-coloring
		      such that there are not almost cycles.
		\item
		      A state \( s \in S_t \) is consistent with an almost NAC-coloring~\( \delta \)
		      on~\( G_t \) if for both halves, w.l.o.g.\ for red halve, it holds that:
		      \begin{itemize}
			      \item
			            If two vertices \( u, v \in X_t \) are in the same red connected component,
			            then they are also in the same part in \( P_\red \).
			      \item
			            If there is a \( \blue \) edge connecting two red components
			            spanning vertices \( u, v \in X_t \),
			            it holds that \( (p_u, p_v) \in R_\red \).
		      \end{itemize}
	\end{itemize}
\end{frame}

\begin{frame}
	\frametitle{State space}

	\only<1>{
		\begin{figure}
			\centering
			\includegraphicsmax{./assets/state_example_01}
			\caption{Basic state example.}
		\end{figure}
	}
	\only<2>{
		\begin{figure}
			\centering
			\includegraphicsmax{./assets/state_example_02}
			\caption{State with relations between parts.}
		\end{figure}
	}
	\only<3>{
		\begin{figure}
			\centering
			\includegraphicsmax{./assets/state_example_03}
			\caption{State with a single part for both colors.}
		\end{figure}
	}
\end{frame}

\begin{frame}
	\frametitle{Cache function}

	\begin{itemize}
		\item
		      Let \( \mathcal{S} = \{ (t, s) | s \in S_t, t \in T \} \).
		      Let us have the cache function \( c: \mathcal{S} \to \N_0 \)
		      such that \( c(t, s) \) is the number of almost NAC-colorings
		      in \( G_t \) consistent with \( s \).
		\item
		      We show recursive relations of the cache function.
	\end{itemize}
\end{frame}

% \begin{frame}
% 	\frametitle{\LeafNode{}}
% 	Let~us have a~\LeafNode{} \( t \in T \).
% 	It~holds that \( c[t, s] = 0 \) for~all \( s \in S_t \).
% \end{frame}


\begin{frame}
	\frametitle{\IntroduceVertexNode{}}

	\only<1>{
		\begin{figure}
			\centering
			\includegraphicsmax{./assets/introduce_vertex_cases_happy}
			\caption{Isolated vertex is introduced correctly.}
		\end{figure}
	}
	\only<2>{
		\begin{figure}
			\centering
			\includegraphicsmax{./assets/introduce_vertex_cases_sad}
			\caption{Such states cannot represent any almost NAC-coloring.}
		\end{figure}
	}
\end{frame}

\begin{frame}
	\frametitle{\IntroduceVertexNode{}}

	Let \( t' \) be the only child of \( t \) in \( T \).

	\begin{align*}
		c[t, s] & =
		\begin{cases}
			0,         & \text{if } \exists a \in \{\red, \blue\} : \{v\} \not\in P_a,                    \\
			0,         & \text{if } \exists a \in \{\red, \blue\} \exists p \in P_a : (\{v\}, p) \in R_a, \\
			c[t', s'], & \text{otherwise},
		\end{cases}
	\end{align*}
	where
	\begin{align*}
		s' & \coloneqq (P_\red \setminus \{\{v\}\}, R_\red, P_\blue \setminus \{\{v\}\}, R_\blue).
	\end{align*}
\end{frame}

\begin{frame}
	\frametitle{\ForgetVertexNode{}}

	\only<1>{
		\begin{figure}
			\centering
			\includegraphicsmax{./assets/forget_node_part}
		\end{figure}
	}
	\only<2>{
		\begin{figure}
			\centering
			\includegraphicsmax{./assets/forget_node_relations}
		\end{figure}
	}

	Let \( S' \subset S_{t'} \) be states yielding \( s \) after removing \( v \):
	%
	\begin{align*}
		c[t, s] & = \sum_{s' \in S'} c[t', s'].
	\end{align*}
\end{frame}

\begin{frame}
	\frametitle{\IntroduceEdgeNode{}}

	\begin{figure}
		\centering
		\includegraphicsmax{./assets/introduce_edge_cases}
	\end{figure}

	Each possibly consistent state \( s \in S_t \)
	is an extension of almost NAC-colorings on \( G_{t'} \)
	of the following four cases. This is evaluated for both colors.
\end{frame}

\begin{frame}
	\frametitle{\JoinNode{}}

	\begin{figure}
		\centering
		\includegraphicsmax{./assets/join_node_happy}
	\end{figure}

	\begin{itemize}
		\item
		      There are no edges in \( G_t \) among vertices in \( X_t \).
		\item
		      Merge partitions using reflex transitive closure
		      while avoiding almost cycles.
		\item
		      Let \( S' \subset S_{t_1'} \times S_{t_2'} \) be tuples of states yielding \( s \) after merging:
		      %
		      \begin{align*}
			      c[t, s] & = \sum_{(s_1', s_2') \in S'} c[t_1', s_1'] \cdot c[t_{2}', s_{2}'].
		      \end{align*}
	\end{itemize}
\end{frame}

\begin{frame}
	\frametitle{\JoinNode{} --- bad cases}

	\only<1>{
		\begin{figure}
			\centering
			\includegraphicsmax{./assets/join_node_sad_01}
			\caption{Left blue relation causes an almost cycle in a red part on right.}
		\end{figure}
	}
	\only<2>{
		\begin{figure}
			\centering
			\includegraphicsmax{./assets/join_node_sad_02}
			\caption{Multiple parts merged, problems occur ``transitively''.}
		\end{figure}
	}
\end{frame}

\begin{frame}
	\frametitle{Complexity}

	\begin{itemize}
		\item
		      Proved by induction over tree nodes.%
		      \footnote{The algorithm will never give you up.}
		\item
		      State space size: up to \( {(k+1)}^{(k+1)} \) partitions per node,
		      up to \( 2^{\binom{k+1}{2}} \) relations per part.
		\item
		      The algorithm yields \( 2^{O(k^2)} \cdot O(n) \) time complexity.
	\end{itemize}
\end{frame}

\begin{frame}
	\frametitle{Omitted details}

	\begin{itemize}
		\item
		      Special case for the first introduced edge and operations before that.
		\item
		      Definition of \IntroduceVertexWithEdgesNode{}.
		\item
		      Optimizations --- additional detection of states not corresponding
		      to any almost NAC-coloring.
		\item
		      Only proved, not implemented.
	\end{itemize}
\end{frame}

\section{Algorithms for NAC-coloring search}

\begin{frame}
	\frametitle{NAC-coloring search}
	\begin{itemize}
		\item
		      Polynomial check whether a red-blue-coloring is a NAC-coloring.
		\item
		      Naive approach tries all the red-blue-colorings.
		\item
		      Basic optimizations (also in FlexRiLoG):
		      \begin{itemize}
			      \item
			            Ignore symmetric NAC-colorings.
			      \item
			            Color \( \triangle \)-connected components as a whole.
		      \end{itemize}
	\end{itemize}
\end{frame}

\begin{frame}
	\frametitle{Monochromatic classes}

	\begin{itemize}
		\item
		      \( \triangle \)-connected component is a monochromatic class.
		\item
		      An edge incident to vertices in the same monochromatic component belongs
		      to the same monochromatic component.
		\item
		      If there are two edges over a monochromatic component,
		      then they belong to the same monochromatic component.
	\end{itemize}

	\begin{figure}[h]
		\centering
		\begin{tikzpicture}[scale=2]
			\node[vertex] (0) at (0, 0) {};
			\node[vertex] (1) at (1, 0.5) {};
			\node[vertex] (2) at (2, 0) {};
			\node[vertex] (3) at (0.5, 0.866) {};
			\node[vertex] (4) at (1.5, 0.866) {};
			\draw[redge] (0)edge(1) (1)edge(2) (0)edge(3) (1)edge(3) (1)edge(4) (2)edge(4) (3)edge(4)  ;
			\draw[edge]  (0)edge(2)  ;
		\end{tikzpicture}
		\qquad
		\begin{tikzpicture}[scale=2]
			\node[vertex] (0) at (0, 0) {};
			\node[vertex] (1) at (1, 0.5) {};
			\node[vertex] (2) at (2, 0) {};
			\node[vertex] (3) at (0.5, 0.866) {};
			\node[vertex] (4) at (1.5, 0.866) {};
			\node[vertex,label={north:$v$}] (5) at (1, 0) {};
			\draw[redge] (0)edge(1) (1)edge(2) (0)edge(3) (1)edge(3) (1)edge(4) (2)edge(4) (3)edge(4)  ;
			\draw[yedge]  (0)edge(5) (2)edge(5)  ;
		\end{tikzpicture}
	\end{figure}
\end{frame}

\begin{frame}
	\frametitle{Cycles detection}

	\begin{itemize}
		\item
		      Polynomial check runs in \( O(n+m) \) time.
		\item
		      For each short cycle, we create a bit-mask
		      as in the internal representation.
		\item
		      Each coloring first checked against the mask,
		      the~original check used as fallback.
		\item
		      Balance between the number of cycles and time spent checking cycles.
	\end{itemize}
\end{frame}

\begin{frame}
	\frametitle{NAC-coloring search}
	\begin{itemize}
		\item
		      NAC-coloring restricted to a subgraph
		      is monochromatic or a~NAC-coloring.
		\item
		      Strategy:
		      \begin{itemize}
			      \item
			            Decompose into smaller subgraphs.
			      \item
			            Find all the NAC-colorings of the subgraphs.
			      \item
			            Choose colorings merge order for the subgraphs.
		      \end{itemize}
		\item
		      Multiple heuristics for each stage.
	\end{itemize}
\end{frame}

\begin{frame}
	\frametitle{Graph decomposition strategies}

	\begin{itemize}
		\item
		      Specified number of monochromatic classes per subgraph.
		\item
		      Best performance for few monochromatic classes per subgraph.
		\item
		      Well performing strategies:
		      \begin{itemize}
			      \item
			            \None{} -- take vertices in the input order.
			      \item
			            \Neighbors{} -- finding subgraphs where monochromatic classes share cycles.
		      \end{itemize}
	\end{itemize}
\end{frame}

\begin{frame}
	\frametitle{Merging strategies}

	\begin{itemize}
		\item
		      Each time, a pair of subgraphs and corresponding NAC-colorings is merged.
		\item
		      Lazy merging -- heavy use of iterators, improves the existence question case.
		\item
		      Well performing strategies:
		      \begin{itemize}
			      \item
			            \MergeLinear{} -- reduce called on the list in the order provided by decomposition.
			      \item
			            \SharedVertices{} -- group classes with many common vertices.
		      \end{itemize}
		\item
		      Tree-like merging strategy unreliable.
	\end{itemize}
\end{frame}

\begin{frame}
	\frametitle{Benchmarks}
	\begin{itemize}
		\item
		      Many NAC-colorings/few NAC-colorings/no NAC-coloring.
		\item
		      Any NAC-coloring/all NAC-colorings/the number of NAC-colorings.
	\end{itemize}
\end{frame}

\begin{frame}
	\frametitle{Graphs with many NAC-colorings}
	\begin{itemize}
		\item
		      Naive approach with cycles is the fastest to check for NAC-coloring existence.
		\item
		      Moustly because of additional overhead as the problem is simple.
		\item
		      \Subgraphs{} still fast for finding any NAC-coloring on over 100 vertex graphs.
		\item
		      Listing all NAC-colorings runs fast enough for 30 vertex graphs,
		      naive approach is slow for small graphs.
	\end{itemize}
\end{frame}

\begin{frame}

	\begin{figure}[ht]
		\centering
		\scalebox{\BenchFigureScale}{%% Creator: Matplotlib, PGF backend
%%
%% To include the figure in your LaTeX document, write
%%   \input{<filename>.pgf}
%%
%% Make sure the required packages are loaded in your preamble
%%   \usepackage{pgf}
%%
%% Also ensure that all the required font packages are loaded; for instance,
%% the lmodern package is sometimes necessary when using math font.
%%   \usepackage{lmodern}
%%
%% Figures using additional raster images can only be included by \input if
%% they are in the same directory as the main LaTeX file. For loading figures
%% from other directories you can use the `import` package
%%   \usepackage{import}
%%
%% and then include the figures with
%%   \import{<path to file>}{<filename>.pgf}
%%
%% Matplotlib used the following preamble
%%   \def\mathdefault#1{#1}
%%   \everymath=\expandafter{\the\everymath\displaystyle}
%%   \IfFileExists{scrextend.sty}{
%%     \usepackage[fontsize=10.000000pt]{scrextend}
%%   }{
%%     \renewcommand{\normalsize}{\fontsize{10.000000}{12.000000}\selectfont}
%%     \normalsize
%%   }
%%   
%%   \ifdefined\pdftexversion\else  % non-pdftex case.
%%     \usepackage{fontspec}
%%     \setmainfont{DejaVuSans.ttf}[Path=\detokenize{/home/petr/Projects/PyRigi/.venv/lib/python3.12/site-packages/matplotlib/mpl-data/fonts/ttf/}]
%%     \setsansfont{DejaVuSans.ttf}[Path=\detokenize{/home/petr/Projects/PyRigi/.venv/lib/python3.12/site-packages/matplotlib/mpl-data/fonts/ttf/}]
%%     \setmonofont{DejaVuSansMono.ttf}[Path=\detokenize{/home/petr/Projects/PyRigi/.venv/lib/python3.12/site-packages/matplotlib/mpl-data/fonts/ttf/}]
%%   \fi
%%   \makeatletter\@ifpackageloaded{under\Score{}}{}{\usepackage[strings]{under\Score{}}}\makeatother
%%
\begingroup%
\makeatletter%
\begin{pgfpicture}%
\pgfpathrectangle{\pgfpointorigin}{\pgfqpoint{8.384376in}{2.841849in}}%
\pgfusepath{use as bounding box, clip}%
\begin{pgfscope}%
\pgfsetbuttcap%
\pgfsetmiterjoin%
\definecolor{currentfill}{rgb}{1.000000,1.000000,1.000000}%
\pgfsetfillcolor{currentfill}%
\pgfsetlinewidth{0.000000pt}%
\definecolor{currentstroke}{rgb}{1.000000,1.000000,1.000000}%
\pgfsetstrokecolor{currentstroke}%
\pgfsetdash{}{0pt}%
\pgfpathmoveto{\pgfqpoint{0.000000in}{0.000000in}}%
\pgfpathlineto{\pgfqpoint{8.384376in}{0.000000in}}%
\pgfpathlineto{\pgfqpoint{8.384376in}{2.841849in}}%
\pgfpathlineto{\pgfqpoint{0.000000in}{2.841849in}}%
\pgfpathlineto{\pgfqpoint{0.000000in}{0.000000in}}%
\pgfpathclose%
\pgfusepath{fill}%
\end{pgfscope}%
\begin{pgfscope}%
\pgfsetbuttcap%
\pgfsetmiterjoin%
\definecolor{currentfill}{rgb}{1.000000,1.000000,1.000000}%
\pgfsetfillcolor{currentfill}%
\pgfsetlinewidth{0.000000pt}%
\definecolor{currentstroke}{rgb}{0.000000,0.000000,0.000000}%
\pgfsetstrokecolor{currentstroke}%
\pgfsetstrokeopacity{0.000000}%
\pgfsetdash{}{0pt}%
\pgfpathmoveto{\pgfqpoint{0.588387in}{0.521603in}}%
\pgfpathlineto{\pgfqpoint{5.487514in}{0.521603in}}%
\pgfpathlineto{\pgfqpoint{5.487514in}{2.741849in}}%
\pgfpathlineto{\pgfqpoint{0.588387in}{2.741849in}}%
\pgfpathlineto{\pgfqpoint{0.588387in}{0.521603in}}%
\pgfpathclose%
\pgfusepath{fill}%
\end{pgfscope}%
\begin{pgfscope}%
\pgfsetbuttcap%
\pgfsetroundjoin%
\definecolor{currentfill}{rgb}{0.000000,0.000000,0.000000}%
\pgfsetfillcolor{currentfill}%
\pgfsetlinewidth{0.803000pt}%
\definecolor{currentstroke}{rgb}{0.000000,0.000000,0.000000}%
\pgfsetstrokecolor{currentstroke}%
\pgfsetdash{}{0pt}%
\pgfsys@defobject{currentmarker}{\pgfqpoint{0.000000in}{-0.048611in}}{\pgfqpoint{0.000000in}{0.000000in}}{%
\pgfpathmoveto{\pgfqpoint{0.000000in}{0.000000in}}%
\pgfpathlineto{\pgfqpoint{0.000000in}{-0.048611in}}%
\pgfusepath{stroke,fill}%
}%
\begin{pgfscope}%
\pgfsys@transformshift{1.098414in}{0.521603in}%
\pgfsys@useobject{currentmarker}{}%
\end{pgfscope}%
\end{pgfscope}%
\begin{pgfscope}%
\definecolor{textcolor}{rgb}{0.000000,0.000000,0.000000}%
\pgfsetstrokecolor{textcolor}%
\pgfsetfillcolor{textcolor}%
\pgftext[x=1.098414in,y=0.424381in,,top]{\color{textcolor}{\rmfamily\fontsize{10.000000}{12.000000}\selectfont\catcode`\^=\active\def^{\ifmmode\sp\else\^{}\fi}\catcode`\%=\active\def%{\%}$\mathdefault{4}$}}%
\end{pgfscope}%
\begin{pgfscope}%
\pgfsetbuttcap%
\pgfsetroundjoin%
\definecolor{currentfill}{rgb}{0.000000,0.000000,0.000000}%
\pgfsetfillcolor{currentfill}%
\pgfsetlinewidth{0.803000pt}%
\definecolor{currentstroke}{rgb}{0.000000,0.000000,0.000000}%
\pgfsetstrokecolor{currentstroke}%
\pgfsetdash{}{0pt}%
\pgfsys@defobject{currentmarker}{\pgfqpoint{0.000000in}{-0.048611in}}{\pgfqpoint{0.000000in}{0.000000in}}{%
\pgfpathmoveto{\pgfqpoint{0.000000in}{0.000000in}}%
\pgfpathlineto{\pgfqpoint{0.000000in}{-0.048611in}}%
\pgfusepath{stroke,fill}%
}%
\begin{pgfscope}%
\pgfsys@transformshift{1.673091in}{0.521603in}%
\pgfsys@useobject{currentmarker}{}%
\end{pgfscope}%
\end{pgfscope}%
\begin{pgfscope}%
\definecolor{textcolor}{rgb}{0.000000,0.000000,0.000000}%
\pgfsetstrokecolor{textcolor}%
\pgfsetfillcolor{textcolor}%
\pgftext[x=1.673091in,y=0.424381in,,top]{\color{textcolor}{\rmfamily\fontsize{10.000000}{12.000000}\selectfont\catcode`\^=\active\def^{\ifmmode\sp\else\^{}\fi}\catcode`\%=\active\def%{\%}$\mathdefault{8}$}}%
\end{pgfscope}%
\begin{pgfscope}%
\pgfsetbuttcap%
\pgfsetroundjoin%
\definecolor{currentfill}{rgb}{0.000000,0.000000,0.000000}%
\pgfsetfillcolor{currentfill}%
\pgfsetlinewidth{0.803000pt}%
\definecolor{currentstroke}{rgb}{0.000000,0.000000,0.000000}%
\pgfsetstrokecolor{currentstroke}%
\pgfsetdash{}{0pt}%
\pgfsys@defobject{currentmarker}{\pgfqpoint{0.000000in}{-0.048611in}}{\pgfqpoint{0.000000in}{0.000000in}}{%
\pgfpathmoveto{\pgfqpoint{0.000000in}{0.000000in}}%
\pgfpathlineto{\pgfqpoint{0.000000in}{-0.048611in}}%
\pgfusepath{stroke,fill}%
}%
\begin{pgfscope}%
\pgfsys@transformshift{2.247769in}{0.521603in}%
\pgfsys@useobject{currentmarker}{}%
\end{pgfscope}%
\end{pgfscope}%
\begin{pgfscope}%
\definecolor{textcolor}{rgb}{0.000000,0.000000,0.000000}%
\pgfsetstrokecolor{textcolor}%
\pgfsetfillcolor{textcolor}%
\pgftext[x=2.247769in,y=0.424381in,,top]{\color{textcolor}{\rmfamily\fontsize{10.000000}{12.000000}\selectfont\catcode`\^=\active\def^{\ifmmode\sp\else\^{}\fi}\catcode`\%=\active\def%{\%}$\mathdefault{12}$}}%
\end{pgfscope}%
\begin{pgfscope}%
\pgfsetbuttcap%
\pgfsetroundjoin%
\definecolor{currentfill}{rgb}{0.000000,0.000000,0.000000}%
\pgfsetfillcolor{currentfill}%
\pgfsetlinewidth{0.803000pt}%
\definecolor{currentstroke}{rgb}{0.000000,0.000000,0.000000}%
\pgfsetstrokecolor{currentstroke}%
\pgfsetdash{}{0pt}%
\pgfsys@defobject{currentmarker}{\pgfqpoint{0.000000in}{-0.048611in}}{\pgfqpoint{0.000000in}{0.000000in}}{%
\pgfpathmoveto{\pgfqpoint{0.000000in}{0.000000in}}%
\pgfpathlineto{\pgfqpoint{0.000000in}{-0.048611in}}%
\pgfusepath{stroke,fill}%
}%
\begin{pgfscope}%
\pgfsys@transformshift{2.822446in}{0.521603in}%
\pgfsys@useobject{currentmarker}{}%
\end{pgfscope}%
\end{pgfscope}%
\begin{pgfscope}%
\definecolor{textcolor}{rgb}{0.000000,0.000000,0.000000}%
\pgfsetstrokecolor{textcolor}%
\pgfsetfillcolor{textcolor}%
\pgftext[x=2.822446in,y=0.424381in,,top]{\color{textcolor}{\rmfamily\fontsize{10.000000}{12.000000}\selectfont\catcode`\^=\active\def^{\ifmmode\sp\else\^{}\fi}\catcode`\%=\active\def%{\%}$\mathdefault{16}$}}%
\end{pgfscope}%
\begin{pgfscope}%
\pgfsetbuttcap%
\pgfsetroundjoin%
\definecolor{currentfill}{rgb}{0.000000,0.000000,0.000000}%
\pgfsetfillcolor{currentfill}%
\pgfsetlinewidth{0.803000pt}%
\definecolor{currentstroke}{rgb}{0.000000,0.000000,0.000000}%
\pgfsetstrokecolor{currentstroke}%
\pgfsetdash{}{0pt}%
\pgfsys@defobject{currentmarker}{\pgfqpoint{0.000000in}{-0.048611in}}{\pgfqpoint{0.000000in}{0.000000in}}{%
\pgfpathmoveto{\pgfqpoint{0.000000in}{0.000000in}}%
\pgfpathlineto{\pgfqpoint{0.000000in}{-0.048611in}}%
\pgfusepath{stroke,fill}%
}%
\begin{pgfscope}%
\pgfsys@transformshift{3.397124in}{0.521603in}%
\pgfsys@useobject{currentmarker}{}%
\end{pgfscope}%
\end{pgfscope}%
\begin{pgfscope}%
\definecolor{textcolor}{rgb}{0.000000,0.000000,0.000000}%
\pgfsetstrokecolor{textcolor}%
\pgfsetfillcolor{textcolor}%
\pgftext[x=3.397124in,y=0.424381in,,top]{\color{textcolor}{\rmfamily\fontsize{10.000000}{12.000000}\selectfont\catcode`\^=\active\def^{\ifmmode\sp\else\^{}\fi}\catcode`\%=\active\def%{\%}$\mathdefault{20}$}}%
\end{pgfscope}%
\begin{pgfscope}%
\pgfsetbuttcap%
\pgfsetroundjoin%
\definecolor{currentfill}{rgb}{0.000000,0.000000,0.000000}%
\pgfsetfillcolor{currentfill}%
\pgfsetlinewidth{0.803000pt}%
\definecolor{currentstroke}{rgb}{0.000000,0.000000,0.000000}%
\pgfsetstrokecolor{currentstroke}%
\pgfsetdash{}{0pt}%
\pgfsys@defobject{currentmarker}{\pgfqpoint{0.000000in}{-0.048611in}}{\pgfqpoint{0.000000in}{0.000000in}}{%
\pgfpathmoveto{\pgfqpoint{0.000000in}{0.000000in}}%
\pgfpathlineto{\pgfqpoint{0.000000in}{-0.048611in}}%
\pgfusepath{stroke,fill}%
}%
\begin{pgfscope}%
\pgfsys@transformshift{3.971802in}{0.521603in}%
\pgfsys@useobject{currentmarker}{}%
\end{pgfscope}%
\end{pgfscope}%
\begin{pgfscope}%
\definecolor{textcolor}{rgb}{0.000000,0.000000,0.000000}%
\pgfsetstrokecolor{textcolor}%
\pgfsetfillcolor{textcolor}%
\pgftext[x=3.971802in,y=0.424381in,,top]{\color{textcolor}{\rmfamily\fontsize{10.000000}{12.000000}\selectfont\catcode`\^=\active\def^{\ifmmode\sp\else\^{}\fi}\catcode`\%=\active\def%{\%}$\mathdefault{24}$}}%
\end{pgfscope}%
\begin{pgfscope}%
\pgfsetbuttcap%
\pgfsetroundjoin%
\definecolor{currentfill}{rgb}{0.000000,0.000000,0.000000}%
\pgfsetfillcolor{currentfill}%
\pgfsetlinewidth{0.803000pt}%
\definecolor{currentstroke}{rgb}{0.000000,0.000000,0.000000}%
\pgfsetstrokecolor{currentstroke}%
\pgfsetdash{}{0pt}%
\pgfsys@defobject{currentmarker}{\pgfqpoint{0.000000in}{-0.048611in}}{\pgfqpoint{0.000000in}{0.000000in}}{%
\pgfpathmoveto{\pgfqpoint{0.000000in}{0.000000in}}%
\pgfpathlineto{\pgfqpoint{0.000000in}{-0.048611in}}%
\pgfusepath{stroke,fill}%
}%
\begin{pgfscope}%
\pgfsys@transformshift{4.546479in}{0.521603in}%
\pgfsys@useobject{currentmarker}{}%
\end{pgfscope}%
\end{pgfscope}%
\begin{pgfscope}%
\definecolor{textcolor}{rgb}{0.000000,0.000000,0.000000}%
\pgfsetstrokecolor{textcolor}%
\pgfsetfillcolor{textcolor}%
\pgftext[x=4.546479in,y=0.424381in,,top]{\color{textcolor}{\rmfamily\fontsize{10.000000}{12.000000}\selectfont\catcode`\^=\active\def^{\ifmmode\sp\else\^{}\fi}\catcode`\%=\active\def%{\%}$\mathdefault{28}$}}%
\end{pgfscope}%
\begin{pgfscope}%
\pgfsetbuttcap%
\pgfsetroundjoin%
\definecolor{currentfill}{rgb}{0.000000,0.000000,0.000000}%
\pgfsetfillcolor{currentfill}%
\pgfsetlinewidth{0.803000pt}%
\definecolor{currentstroke}{rgb}{0.000000,0.000000,0.000000}%
\pgfsetstrokecolor{currentstroke}%
\pgfsetdash{}{0pt}%
\pgfsys@defobject{currentmarker}{\pgfqpoint{0.000000in}{-0.048611in}}{\pgfqpoint{0.000000in}{0.000000in}}{%
\pgfpathmoveto{\pgfqpoint{0.000000in}{0.000000in}}%
\pgfpathlineto{\pgfqpoint{0.000000in}{-0.048611in}}%
\pgfusepath{stroke,fill}%
}%
\begin{pgfscope}%
\pgfsys@transformshift{5.121157in}{0.521603in}%
\pgfsys@useobject{currentmarker}{}%
\end{pgfscope}%
\end{pgfscope}%
\begin{pgfscope}%
\definecolor{textcolor}{rgb}{0.000000,0.000000,0.000000}%
\pgfsetstrokecolor{textcolor}%
\pgfsetfillcolor{textcolor}%
\pgftext[x=5.121157in,y=0.424381in,,top]{\color{textcolor}{\rmfamily\fontsize{10.000000}{12.000000}\selectfont\catcode`\^=\active\def^{\ifmmode\sp\else\^{}\fi}\catcode`\%=\active\def%{\%}$\mathdefault{32}$}}%
\end{pgfscope}%
\begin{pgfscope}%
\definecolor{textcolor}{rgb}{0.000000,0.000000,0.000000}%
\pgfsetstrokecolor{textcolor}%
\pgfsetfillcolor{textcolor}%
\pgftext[x=3.037950in,y=0.234413in,,top]{\color{textcolor}{\rmfamily\fontsize{10.000000}{12.000000}\selectfont\catcode`\^=\active\def^{\ifmmode\sp\else\^{}\fi}\catcode`\%=\active\def%{\%}Monochromatic classes}}%
\end{pgfscope}%
\begin{pgfscope}%
\pgfsetbuttcap%
\pgfsetroundjoin%
\definecolor{currentfill}{rgb}{0.000000,0.000000,0.000000}%
\pgfsetfillcolor{currentfill}%
\pgfsetlinewidth{0.803000pt}%
\definecolor{currentstroke}{rgb}{0.000000,0.000000,0.000000}%
\pgfsetstrokecolor{currentstroke}%
\pgfsetdash{}{0pt}%
\pgfsys@defobject{currentmarker}{\pgfqpoint{-0.048611in}{0.000000in}}{\pgfqpoint{-0.000000in}{0.000000in}}{%
\pgfpathmoveto{\pgfqpoint{-0.000000in}{0.000000in}}%
\pgfpathlineto{\pgfqpoint{-0.048611in}{0.000000in}}%
\pgfusepath{stroke,fill}%
}%
\begin{pgfscope}%
\pgfsys@transformshift{0.588387in}{0.924341in}%
\pgfsys@useobject{currentmarker}{}%
\end{pgfscope}%
\end{pgfscope}%
\begin{pgfscope}%
\definecolor{textcolor}{rgb}{0.000000,0.000000,0.000000}%
\pgfsetstrokecolor{textcolor}%
\pgfsetfillcolor{textcolor}%
\pgftext[x=0.289968in, y=0.871579in, left, base]{\color{textcolor}{\rmfamily\fontsize{10.000000}{12.000000}\selectfont\catcode`\^=\active\def^{\ifmmode\sp\else\^{}\fi}\catcode`\%=\active\def%{\%}$\mathdefault{10^{1}}$}}%
\end{pgfscope}%
\begin{pgfscope}%
\pgfsetbuttcap%
\pgfsetroundjoin%
\definecolor{currentfill}{rgb}{0.000000,0.000000,0.000000}%
\pgfsetfillcolor{currentfill}%
\pgfsetlinewidth{0.803000pt}%
\definecolor{currentstroke}{rgb}{0.000000,0.000000,0.000000}%
\pgfsetstrokecolor{currentstroke}%
\pgfsetdash{}{0pt}%
\pgfsys@defobject{currentmarker}{\pgfqpoint{-0.048611in}{0.000000in}}{\pgfqpoint{-0.000000in}{0.000000in}}{%
\pgfpathmoveto{\pgfqpoint{-0.000000in}{0.000000in}}%
\pgfpathlineto{\pgfqpoint{-0.048611in}{0.000000in}}%
\pgfusepath{stroke,fill}%
}%
\begin{pgfscope}%
\pgfsys@transformshift{0.588387in}{1.548212in}%
\pgfsys@useobject{currentmarker}{}%
\end{pgfscope}%
\end{pgfscope}%
\begin{pgfscope}%
\definecolor{textcolor}{rgb}{0.000000,0.000000,0.000000}%
\pgfsetstrokecolor{textcolor}%
\pgfsetfillcolor{textcolor}%
\pgftext[x=0.289968in, y=1.495450in, left, base]{\color{textcolor}{\rmfamily\fontsize{10.000000}{12.000000}\selectfont\catcode`\^=\active\def^{\ifmmode\sp\else\^{}\fi}\catcode`\%=\active\def%{\%}$\mathdefault{10^{2}}$}}%
\end{pgfscope}%
\begin{pgfscope}%
\pgfsetbuttcap%
\pgfsetroundjoin%
\definecolor{currentfill}{rgb}{0.000000,0.000000,0.000000}%
\pgfsetfillcolor{currentfill}%
\pgfsetlinewidth{0.803000pt}%
\definecolor{currentstroke}{rgb}{0.000000,0.000000,0.000000}%
\pgfsetstrokecolor{currentstroke}%
\pgfsetdash{}{0pt}%
\pgfsys@defobject{currentmarker}{\pgfqpoint{-0.048611in}{0.000000in}}{\pgfqpoint{-0.000000in}{0.000000in}}{%
\pgfpathmoveto{\pgfqpoint{-0.000000in}{0.000000in}}%
\pgfpathlineto{\pgfqpoint{-0.048611in}{0.000000in}}%
\pgfusepath{stroke,fill}%
}%
\begin{pgfscope}%
\pgfsys@transformshift{0.588387in}{2.172083in}%
\pgfsys@useobject{currentmarker}{}%
\end{pgfscope}%
\end{pgfscope}%
\begin{pgfscope}%
\definecolor{textcolor}{rgb}{0.000000,0.000000,0.000000}%
\pgfsetstrokecolor{textcolor}%
\pgfsetfillcolor{textcolor}%
\pgftext[x=0.289968in, y=2.119322in, left, base]{\color{textcolor}{\rmfamily\fontsize{10.000000}{12.000000}\selectfont\catcode`\^=\active\def^{\ifmmode\sp\else\^{}\fi}\catcode`\%=\active\def%{\%}$\mathdefault{10^{3}}$}}%
\end{pgfscope}%
\begin{pgfscope}%
\pgfsetbuttcap%
\pgfsetroundjoin%
\definecolor{currentfill}{rgb}{0.000000,0.000000,0.000000}%
\pgfsetfillcolor{currentfill}%
\pgfsetlinewidth{0.602250pt}%
\definecolor{currentstroke}{rgb}{0.000000,0.000000,0.000000}%
\pgfsetstrokecolor{currentstroke}%
\pgfsetdash{}{0pt}%
\pgfsys@defobject{currentmarker}{\pgfqpoint{-0.027778in}{0.000000in}}{\pgfqpoint{-0.000000in}{0.000000in}}{%
\pgfpathmoveto{\pgfqpoint{-0.000000in}{0.000000in}}%
\pgfpathlineto{\pgfqpoint{-0.027778in}{0.000000in}}%
\pgfusepath{stroke,fill}%
}%
\begin{pgfscope}%
\pgfsys@transformshift{0.588387in}{0.598132in}%
\pgfsys@useobject{currentmarker}{}%
\end{pgfscope}%
\end{pgfscope}%
\begin{pgfscope}%
\pgfsetbuttcap%
\pgfsetroundjoin%
\definecolor{currentfill}{rgb}{0.000000,0.000000,0.000000}%
\pgfsetfillcolor{currentfill}%
\pgfsetlinewidth{0.602250pt}%
\definecolor{currentstroke}{rgb}{0.000000,0.000000,0.000000}%
\pgfsetstrokecolor{currentstroke}%
\pgfsetdash{}{0pt}%
\pgfsys@defobject{currentmarker}{\pgfqpoint{-0.027778in}{0.000000in}}{\pgfqpoint{-0.000000in}{0.000000in}}{%
\pgfpathmoveto{\pgfqpoint{-0.000000in}{0.000000in}}%
\pgfpathlineto{\pgfqpoint{-0.027778in}{0.000000in}}%
\pgfusepath{stroke,fill}%
}%
\begin{pgfscope}%
\pgfsys@transformshift{0.588387in}{0.676077in}%
\pgfsys@useobject{currentmarker}{}%
\end{pgfscope}%
\end{pgfscope}%
\begin{pgfscope}%
\pgfsetbuttcap%
\pgfsetroundjoin%
\definecolor{currentfill}{rgb}{0.000000,0.000000,0.000000}%
\pgfsetfillcolor{currentfill}%
\pgfsetlinewidth{0.602250pt}%
\definecolor{currentstroke}{rgb}{0.000000,0.000000,0.000000}%
\pgfsetstrokecolor{currentstroke}%
\pgfsetdash{}{0pt}%
\pgfsys@defobject{currentmarker}{\pgfqpoint{-0.027778in}{0.000000in}}{\pgfqpoint{-0.000000in}{0.000000in}}{%
\pgfpathmoveto{\pgfqpoint{-0.000000in}{0.000000in}}%
\pgfpathlineto{\pgfqpoint{-0.027778in}{0.000000in}}%
\pgfusepath{stroke,fill}%
}%
\begin{pgfscope}%
\pgfsys@transformshift{0.588387in}{0.736537in}%
\pgfsys@useobject{currentmarker}{}%
\end{pgfscope}%
\end{pgfscope}%
\begin{pgfscope}%
\pgfsetbuttcap%
\pgfsetroundjoin%
\definecolor{currentfill}{rgb}{0.000000,0.000000,0.000000}%
\pgfsetfillcolor{currentfill}%
\pgfsetlinewidth{0.602250pt}%
\definecolor{currentstroke}{rgb}{0.000000,0.000000,0.000000}%
\pgfsetstrokecolor{currentstroke}%
\pgfsetdash{}{0pt}%
\pgfsys@defobject{currentmarker}{\pgfqpoint{-0.027778in}{0.000000in}}{\pgfqpoint{-0.000000in}{0.000000in}}{%
\pgfpathmoveto{\pgfqpoint{-0.000000in}{0.000000in}}%
\pgfpathlineto{\pgfqpoint{-0.027778in}{0.000000in}}%
\pgfusepath{stroke,fill}%
}%
\begin{pgfscope}%
\pgfsys@transformshift{0.588387in}{0.785936in}%
\pgfsys@useobject{currentmarker}{}%
\end{pgfscope}%
\end{pgfscope}%
\begin{pgfscope}%
\pgfsetbuttcap%
\pgfsetroundjoin%
\definecolor{currentfill}{rgb}{0.000000,0.000000,0.000000}%
\pgfsetfillcolor{currentfill}%
\pgfsetlinewidth{0.602250pt}%
\definecolor{currentstroke}{rgb}{0.000000,0.000000,0.000000}%
\pgfsetstrokecolor{currentstroke}%
\pgfsetdash{}{0pt}%
\pgfsys@defobject{currentmarker}{\pgfqpoint{-0.027778in}{0.000000in}}{\pgfqpoint{-0.000000in}{0.000000in}}{%
\pgfpathmoveto{\pgfqpoint{-0.000000in}{0.000000in}}%
\pgfpathlineto{\pgfqpoint{-0.027778in}{0.000000in}}%
\pgfusepath{stroke,fill}%
}%
\begin{pgfscope}%
\pgfsys@transformshift{0.588387in}{0.827702in}%
\pgfsys@useobject{currentmarker}{}%
\end{pgfscope}%
\end{pgfscope}%
\begin{pgfscope}%
\pgfsetbuttcap%
\pgfsetroundjoin%
\definecolor{currentfill}{rgb}{0.000000,0.000000,0.000000}%
\pgfsetfillcolor{currentfill}%
\pgfsetlinewidth{0.602250pt}%
\definecolor{currentstroke}{rgb}{0.000000,0.000000,0.000000}%
\pgfsetstrokecolor{currentstroke}%
\pgfsetdash{}{0pt}%
\pgfsys@defobject{currentmarker}{\pgfqpoint{-0.027778in}{0.000000in}}{\pgfqpoint{-0.000000in}{0.000000in}}{%
\pgfpathmoveto{\pgfqpoint{-0.000000in}{0.000000in}}%
\pgfpathlineto{\pgfqpoint{-0.027778in}{0.000000in}}%
\pgfusepath{stroke,fill}%
}%
\begin{pgfscope}%
\pgfsys@transformshift{0.588387in}{0.863881in}%
\pgfsys@useobject{currentmarker}{}%
\end{pgfscope}%
\end{pgfscope}%
\begin{pgfscope}%
\pgfsetbuttcap%
\pgfsetroundjoin%
\definecolor{currentfill}{rgb}{0.000000,0.000000,0.000000}%
\pgfsetfillcolor{currentfill}%
\pgfsetlinewidth{0.602250pt}%
\definecolor{currentstroke}{rgb}{0.000000,0.000000,0.000000}%
\pgfsetstrokecolor{currentstroke}%
\pgfsetdash{}{0pt}%
\pgfsys@defobject{currentmarker}{\pgfqpoint{-0.027778in}{0.000000in}}{\pgfqpoint{-0.000000in}{0.000000in}}{%
\pgfpathmoveto{\pgfqpoint{-0.000000in}{0.000000in}}%
\pgfpathlineto{\pgfqpoint{-0.027778in}{0.000000in}}%
\pgfusepath{stroke,fill}%
}%
\begin{pgfscope}%
\pgfsys@transformshift{0.588387in}{0.895794in}%
\pgfsys@useobject{currentmarker}{}%
\end{pgfscope}%
\end{pgfscope}%
\begin{pgfscope}%
\pgfsetbuttcap%
\pgfsetroundjoin%
\definecolor{currentfill}{rgb}{0.000000,0.000000,0.000000}%
\pgfsetfillcolor{currentfill}%
\pgfsetlinewidth{0.602250pt}%
\definecolor{currentstroke}{rgb}{0.000000,0.000000,0.000000}%
\pgfsetstrokecolor{currentstroke}%
\pgfsetdash{}{0pt}%
\pgfsys@defobject{currentmarker}{\pgfqpoint{-0.027778in}{0.000000in}}{\pgfqpoint{-0.000000in}{0.000000in}}{%
\pgfpathmoveto{\pgfqpoint{-0.000000in}{0.000000in}}%
\pgfpathlineto{\pgfqpoint{-0.027778in}{0.000000in}}%
\pgfusepath{stroke,fill}%
}%
\begin{pgfscope}%
\pgfsys@transformshift{0.588387in}{1.112145in}%
\pgfsys@useobject{currentmarker}{}%
\end{pgfscope}%
\end{pgfscope}%
\begin{pgfscope}%
\pgfsetbuttcap%
\pgfsetroundjoin%
\definecolor{currentfill}{rgb}{0.000000,0.000000,0.000000}%
\pgfsetfillcolor{currentfill}%
\pgfsetlinewidth{0.602250pt}%
\definecolor{currentstroke}{rgb}{0.000000,0.000000,0.000000}%
\pgfsetstrokecolor{currentstroke}%
\pgfsetdash{}{0pt}%
\pgfsys@defobject{currentmarker}{\pgfqpoint{-0.027778in}{0.000000in}}{\pgfqpoint{-0.000000in}{0.000000in}}{%
\pgfpathmoveto{\pgfqpoint{-0.000000in}{0.000000in}}%
\pgfpathlineto{\pgfqpoint{-0.027778in}{0.000000in}}%
\pgfusepath{stroke,fill}%
}%
\begin{pgfscope}%
\pgfsys@transformshift{0.588387in}{1.222003in}%
\pgfsys@useobject{currentmarker}{}%
\end{pgfscope}%
\end{pgfscope}%
\begin{pgfscope}%
\pgfsetbuttcap%
\pgfsetroundjoin%
\definecolor{currentfill}{rgb}{0.000000,0.000000,0.000000}%
\pgfsetfillcolor{currentfill}%
\pgfsetlinewidth{0.602250pt}%
\definecolor{currentstroke}{rgb}{0.000000,0.000000,0.000000}%
\pgfsetstrokecolor{currentstroke}%
\pgfsetdash{}{0pt}%
\pgfsys@defobject{currentmarker}{\pgfqpoint{-0.027778in}{0.000000in}}{\pgfqpoint{-0.000000in}{0.000000in}}{%
\pgfpathmoveto{\pgfqpoint{-0.000000in}{0.000000in}}%
\pgfpathlineto{\pgfqpoint{-0.027778in}{0.000000in}}%
\pgfusepath{stroke,fill}%
}%
\begin{pgfscope}%
\pgfsys@transformshift{0.588387in}{1.299949in}%
\pgfsys@useobject{currentmarker}{}%
\end{pgfscope}%
\end{pgfscope}%
\begin{pgfscope}%
\pgfsetbuttcap%
\pgfsetroundjoin%
\definecolor{currentfill}{rgb}{0.000000,0.000000,0.000000}%
\pgfsetfillcolor{currentfill}%
\pgfsetlinewidth{0.602250pt}%
\definecolor{currentstroke}{rgb}{0.000000,0.000000,0.000000}%
\pgfsetstrokecolor{currentstroke}%
\pgfsetdash{}{0pt}%
\pgfsys@defobject{currentmarker}{\pgfqpoint{-0.027778in}{0.000000in}}{\pgfqpoint{-0.000000in}{0.000000in}}{%
\pgfpathmoveto{\pgfqpoint{-0.000000in}{0.000000in}}%
\pgfpathlineto{\pgfqpoint{-0.027778in}{0.000000in}}%
\pgfusepath{stroke,fill}%
}%
\begin{pgfscope}%
\pgfsys@transformshift{0.588387in}{1.360408in}%
\pgfsys@useobject{currentmarker}{}%
\end{pgfscope}%
\end{pgfscope}%
\begin{pgfscope}%
\pgfsetbuttcap%
\pgfsetroundjoin%
\definecolor{currentfill}{rgb}{0.000000,0.000000,0.000000}%
\pgfsetfillcolor{currentfill}%
\pgfsetlinewidth{0.602250pt}%
\definecolor{currentstroke}{rgb}{0.000000,0.000000,0.000000}%
\pgfsetstrokecolor{currentstroke}%
\pgfsetdash{}{0pt}%
\pgfsys@defobject{currentmarker}{\pgfqpoint{-0.027778in}{0.000000in}}{\pgfqpoint{-0.000000in}{0.000000in}}{%
\pgfpathmoveto{\pgfqpoint{-0.000000in}{0.000000in}}%
\pgfpathlineto{\pgfqpoint{-0.027778in}{0.000000in}}%
\pgfusepath{stroke,fill}%
}%
\begin{pgfscope}%
\pgfsys@transformshift{0.588387in}{1.409807in}%
\pgfsys@useobject{currentmarker}{}%
\end{pgfscope}%
\end{pgfscope}%
\begin{pgfscope}%
\pgfsetbuttcap%
\pgfsetroundjoin%
\definecolor{currentfill}{rgb}{0.000000,0.000000,0.000000}%
\pgfsetfillcolor{currentfill}%
\pgfsetlinewidth{0.602250pt}%
\definecolor{currentstroke}{rgb}{0.000000,0.000000,0.000000}%
\pgfsetstrokecolor{currentstroke}%
\pgfsetdash{}{0pt}%
\pgfsys@defobject{currentmarker}{\pgfqpoint{-0.027778in}{0.000000in}}{\pgfqpoint{-0.000000in}{0.000000in}}{%
\pgfpathmoveto{\pgfqpoint{-0.000000in}{0.000000in}}%
\pgfpathlineto{\pgfqpoint{-0.027778in}{0.000000in}}%
\pgfusepath{stroke,fill}%
}%
\begin{pgfscope}%
\pgfsys@transformshift{0.588387in}{1.451573in}%
\pgfsys@useobject{currentmarker}{}%
\end{pgfscope}%
\end{pgfscope}%
\begin{pgfscope}%
\pgfsetbuttcap%
\pgfsetroundjoin%
\definecolor{currentfill}{rgb}{0.000000,0.000000,0.000000}%
\pgfsetfillcolor{currentfill}%
\pgfsetlinewidth{0.602250pt}%
\definecolor{currentstroke}{rgb}{0.000000,0.000000,0.000000}%
\pgfsetstrokecolor{currentstroke}%
\pgfsetdash{}{0pt}%
\pgfsys@defobject{currentmarker}{\pgfqpoint{-0.027778in}{0.000000in}}{\pgfqpoint{-0.000000in}{0.000000in}}{%
\pgfpathmoveto{\pgfqpoint{-0.000000in}{0.000000in}}%
\pgfpathlineto{\pgfqpoint{-0.027778in}{0.000000in}}%
\pgfusepath{stroke,fill}%
}%
\begin{pgfscope}%
\pgfsys@transformshift{0.588387in}{1.487753in}%
\pgfsys@useobject{currentmarker}{}%
\end{pgfscope}%
\end{pgfscope}%
\begin{pgfscope}%
\pgfsetbuttcap%
\pgfsetroundjoin%
\definecolor{currentfill}{rgb}{0.000000,0.000000,0.000000}%
\pgfsetfillcolor{currentfill}%
\pgfsetlinewidth{0.602250pt}%
\definecolor{currentstroke}{rgb}{0.000000,0.000000,0.000000}%
\pgfsetstrokecolor{currentstroke}%
\pgfsetdash{}{0pt}%
\pgfsys@defobject{currentmarker}{\pgfqpoint{-0.027778in}{0.000000in}}{\pgfqpoint{-0.000000in}{0.000000in}}{%
\pgfpathmoveto{\pgfqpoint{-0.000000in}{0.000000in}}%
\pgfpathlineto{\pgfqpoint{-0.027778in}{0.000000in}}%
\pgfusepath{stroke,fill}%
}%
\begin{pgfscope}%
\pgfsys@transformshift{0.588387in}{1.519665in}%
\pgfsys@useobject{currentmarker}{}%
\end{pgfscope}%
\end{pgfscope}%
\begin{pgfscope}%
\pgfsetbuttcap%
\pgfsetroundjoin%
\definecolor{currentfill}{rgb}{0.000000,0.000000,0.000000}%
\pgfsetfillcolor{currentfill}%
\pgfsetlinewidth{0.602250pt}%
\definecolor{currentstroke}{rgb}{0.000000,0.000000,0.000000}%
\pgfsetstrokecolor{currentstroke}%
\pgfsetdash{}{0pt}%
\pgfsys@defobject{currentmarker}{\pgfqpoint{-0.027778in}{0.000000in}}{\pgfqpoint{-0.000000in}{0.000000in}}{%
\pgfpathmoveto{\pgfqpoint{-0.000000in}{0.000000in}}%
\pgfpathlineto{\pgfqpoint{-0.027778in}{0.000000in}}%
\pgfusepath{stroke,fill}%
}%
\begin{pgfscope}%
\pgfsys@transformshift{0.588387in}{1.736016in}%
\pgfsys@useobject{currentmarker}{}%
\end{pgfscope}%
\end{pgfscope}%
\begin{pgfscope}%
\pgfsetbuttcap%
\pgfsetroundjoin%
\definecolor{currentfill}{rgb}{0.000000,0.000000,0.000000}%
\pgfsetfillcolor{currentfill}%
\pgfsetlinewidth{0.602250pt}%
\definecolor{currentstroke}{rgb}{0.000000,0.000000,0.000000}%
\pgfsetstrokecolor{currentstroke}%
\pgfsetdash{}{0pt}%
\pgfsys@defobject{currentmarker}{\pgfqpoint{-0.027778in}{0.000000in}}{\pgfqpoint{-0.000000in}{0.000000in}}{%
\pgfpathmoveto{\pgfqpoint{-0.000000in}{0.000000in}}%
\pgfpathlineto{\pgfqpoint{-0.027778in}{0.000000in}}%
\pgfusepath{stroke,fill}%
}%
\begin{pgfscope}%
\pgfsys@transformshift{0.588387in}{1.845874in}%
\pgfsys@useobject{currentmarker}{}%
\end{pgfscope}%
\end{pgfscope}%
\begin{pgfscope}%
\pgfsetbuttcap%
\pgfsetroundjoin%
\definecolor{currentfill}{rgb}{0.000000,0.000000,0.000000}%
\pgfsetfillcolor{currentfill}%
\pgfsetlinewidth{0.602250pt}%
\definecolor{currentstroke}{rgb}{0.000000,0.000000,0.000000}%
\pgfsetstrokecolor{currentstroke}%
\pgfsetdash{}{0pt}%
\pgfsys@defobject{currentmarker}{\pgfqpoint{-0.027778in}{0.000000in}}{\pgfqpoint{-0.000000in}{0.000000in}}{%
\pgfpathmoveto{\pgfqpoint{-0.000000in}{0.000000in}}%
\pgfpathlineto{\pgfqpoint{-0.027778in}{0.000000in}}%
\pgfusepath{stroke,fill}%
}%
\begin{pgfscope}%
\pgfsys@transformshift{0.588387in}{1.923820in}%
\pgfsys@useobject{currentmarker}{}%
\end{pgfscope}%
\end{pgfscope}%
\begin{pgfscope}%
\pgfsetbuttcap%
\pgfsetroundjoin%
\definecolor{currentfill}{rgb}{0.000000,0.000000,0.000000}%
\pgfsetfillcolor{currentfill}%
\pgfsetlinewidth{0.602250pt}%
\definecolor{currentstroke}{rgb}{0.000000,0.000000,0.000000}%
\pgfsetstrokecolor{currentstroke}%
\pgfsetdash{}{0pt}%
\pgfsys@defobject{currentmarker}{\pgfqpoint{-0.027778in}{0.000000in}}{\pgfqpoint{-0.000000in}{0.000000in}}{%
\pgfpathmoveto{\pgfqpoint{-0.000000in}{0.000000in}}%
\pgfpathlineto{\pgfqpoint{-0.027778in}{0.000000in}}%
\pgfusepath{stroke,fill}%
}%
\begin{pgfscope}%
\pgfsys@transformshift{0.588387in}{1.984279in}%
\pgfsys@useobject{currentmarker}{}%
\end{pgfscope}%
\end{pgfscope}%
\begin{pgfscope}%
\pgfsetbuttcap%
\pgfsetroundjoin%
\definecolor{currentfill}{rgb}{0.000000,0.000000,0.000000}%
\pgfsetfillcolor{currentfill}%
\pgfsetlinewidth{0.602250pt}%
\definecolor{currentstroke}{rgb}{0.000000,0.000000,0.000000}%
\pgfsetstrokecolor{currentstroke}%
\pgfsetdash{}{0pt}%
\pgfsys@defobject{currentmarker}{\pgfqpoint{-0.027778in}{0.000000in}}{\pgfqpoint{-0.000000in}{0.000000in}}{%
\pgfpathmoveto{\pgfqpoint{-0.000000in}{0.000000in}}%
\pgfpathlineto{\pgfqpoint{-0.027778in}{0.000000in}}%
\pgfusepath{stroke,fill}%
}%
\begin{pgfscope}%
\pgfsys@transformshift{0.588387in}{2.033678in}%
\pgfsys@useobject{currentmarker}{}%
\end{pgfscope}%
\end{pgfscope}%
\begin{pgfscope}%
\pgfsetbuttcap%
\pgfsetroundjoin%
\definecolor{currentfill}{rgb}{0.000000,0.000000,0.000000}%
\pgfsetfillcolor{currentfill}%
\pgfsetlinewidth{0.602250pt}%
\definecolor{currentstroke}{rgb}{0.000000,0.000000,0.000000}%
\pgfsetstrokecolor{currentstroke}%
\pgfsetdash{}{0pt}%
\pgfsys@defobject{currentmarker}{\pgfqpoint{-0.027778in}{0.000000in}}{\pgfqpoint{-0.000000in}{0.000000in}}{%
\pgfpathmoveto{\pgfqpoint{-0.000000in}{0.000000in}}%
\pgfpathlineto{\pgfqpoint{-0.027778in}{0.000000in}}%
\pgfusepath{stroke,fill}%
}%
\begin{pgfscope}%
\pgfsys@transformshift{0.588387in}{2.075444in}%
\pgfsys@useobject{currentmarker}{}%
\end{pgfscope}%
\end{pgfscope}%
\begin{pgfscope}%
\pgfsetbuttcap%
\pgfsetroundjoin%
\definecolor{currentfill}{rgb}{0.000000,0.000000,0.000000}%
\pgfsetfillcolor{currentfill}%
\pgfsetlinewidth{0.602250pt}%
\definecolor{currentstroke}{rgb}{0.000000,0.000000,0.000000}%
\pgfsetstrokecolor{currentstroke}%
\pgfsetdash{}{0pt}%
\pgfsys@defobject{currentmarker}{\pgfqpoint{-0.027778in}{0.000000in}}{\pgfqpoint{-0.000000in}{0.000000in}}{%
\pgfpathmoveto{\pgfqpoint{-0.000000in}{0.000000in}}%
\pgfpathlineto{\pgfqpoint{-0.027778in}{0.000000in}}%
\pgfusepath{stroke,fill}%
}%
\begin{pgfscope}%
\pgfsys@transformshift{0.588387in}{2.111624in}%
\pgfsys@useobject{currentmarker}{}%
\end{pgfscope}%
\end{pgfscope}%
\begin{pgfscope}%
\pgfsetbuttcap%
\pgfsetroundjoin%
\definecolor{currentfill}{rgb}{0.000000,0.000000,0.000000}%
\pgfsetfillcolor{currentfill}%
\pgfsetlinewidth{0.602250pt}%
\definecolor{currentstroke}{rgb}{0.000000,0.000000,0.000000}%
\pgfsetstrokecolor{currentstroke}%
\pgfsetdash{}{0pt}%
\pgfsys@defobject{currentmarker}{\pgfqpoint{-0.027778in}{0.000000in}}{\pgfqpoint{-0.000000in}{0.000000in}}{%
\pgfpathmoveto{\pgfqpoint{-0.000000in}{0.000000in}}%
\pgfpathlineto{\pgfqpoint{-0.027778in}{0.000000in}}%
\pgfusepath{stroke,fill}%
}%
\begin{pgfscope}%
\pgfsys@transformshift{0.588387in}{2.143537in}%
\pgfsys@useobject{currentmarker}{}%
\end{pgfscope}%
\end{pgfscope}%
\begin{pgfscope}%
\pgfsetbuttcap%
\pgfsetroundjoin%
\definecolor{currentfill}{rgb}{0.000000,0.000000,0.000000}%
\pgfsetfillcolor{currentfill}%
\pgfsetlinewidth{0.602250pt}%
\definecolor{currentstroke}{rgb}{0.000000,0.000000,0.000000}%
\pgfsetstrokecolor{currentstroke}%
\pgfsetdash{}{0pt}%
\pgfsys@defobject{currentmarker}{\pgfqpoint{-0.027778in}{0.000000in}}{\pgfqpoint{-0.000000in}{0.000000in}}{%
\pgfpathmoveto{\pgfqpoint{-0.000000in}{0.000000in}}%
\pgfpathlineto{\pgfqpoint{-0.027778in}{0.000000in}}%
\pgfusepath{stroke,fill}%
}%
\begin{pgfscope}%
\pgfsys@transformshift{0.588387in}{2.359887in}%
\pgfsys@useobject{currentmarker}{}%
\end{pgfscope}%
\end{pgfscope}%
\begin{pgfscope}%
\pgfsetbuttcap%
\pgfsetroundjoin%
\definecolor{currentfill}{rgb}{0.000000,0.000000,0.000000}%
\pgfsetfillcolor{currentfill}%
\pgfsetlinewidth{0.602250pt}%
\definecolor{currentstroke}{rgb}{0.000000,0.000000,0.000000}%
\pgfsetstrokecolor{currentstroke}%
\pgfsetdash{}{0pt}%
\pgfsys@defobject{currentmarker}{\pgfqpoint{-0.027778in}{0.000000in}}{\pgfqpoint{-0.000000in}{0.000000in}}{%
\pgfpathmoveto{\pgfqpoint{-0.000000in}{0.000000in}}%
\pgfpathlineto{\pgfqpoint{-0.027778in}{0.000000in}}%
\pgfusepath{stroke,fill}%
}%
\begin{pgfscope}%
\pgfsys@transformshift{0.588387in}{2.469746in}%
\pgfsys@useobject{currentmarker}{}%
\end{pgfscope}%
\end{pgfscope}%
\begin{pgfscope}%
\pgfsetbuttcap%
\pgfsetroundjoin%
\definecolor{currentfill}{rgb}{0.000000,0.000000,0.000000}%
\pgfsetfillcolor{currentfill}%
\pgfsetlinewidth{0.602250pt}%
\definecolor{currentstroke}{rgb}{0.000000,0.000000,0.000000}%
\pgfsetstrokecolor{currentstroke}%
\pgfsetdash{}{0pt}%
\pgfsys@defobject{currentmarker}{\pgfqpoint{-0.027778in}{0.000000in}}{\pgfqpoint{-0.000000in}{0.000000in}}{%
\pgfpathmoveto{\pgfqpoint{-0.000000in}{0.000000in}}%
\pgfpathlineto{\pgfqpoint{-0.027778in}{0.000000in}}%
\pgfusepath{stroke,fill}%
}%
\begin{pgfscope}%
\pgfsys@transformshift{0.588387in}{2.547691in}%
\pgfsys@useobject{currentmarker}{}%
\end{pgfscope}%
\end{pgfscope}%
\begin{pgfscope}%
\pgfsetbuttcap%
\pgfsetroundjoin%
\definecolor{currentfill}{rgb}{0.000000,0.000000,0.000000}%
\pgfsetfillcolor{currentfill}%
\pgfsetlinewidth{0.602250pt}%
\definecolor{currentstroke}{rgb}{0.000000,0.000000,0.000000}%
\pgfsetstrokecolor{currentstroke}%
\pgfsetdash{}{0pt}%
\pgfsys@defobject{currentmarker}{\pgfqpoint{-0.027778in}{0.000000in}}{\pgfqpoint{-0.000000in}{0.000000in}}{%
\pgfpathmoveto{\pgfqpoint{-0.000000in}{0.000000in}}%
\pgfpathlineto{\pgfqpoint{-0.027778in}{0.000000in}}%
\pgfusepath{stroke,fill}%
}%
\begin{pgfscope}%
\pgfsys@transformshift{0.588387in}{2.608151in}%
\pgfsys@useobject{currentmarker}{}%
\end{pgfscope}%
\end{pgfscope}%
\begin{pgfscope}%
\pgfsetbuttcap%
\pgfsetroundjoin%
\definecolor{currentfill}{rgb}{0.000000,0.000000,0.000000}%
\pgfsetfillcolor{currentfill}%
\pgfsetlinewidth{0.602250pt}%
\definecolor{currentstroke}{rgb}{0.000000,0.000000,0.000000}%
\pgfsetstrokecolor{currentstroke}%
\pgfsetdash{}{0pt}%
\pgfsys@defobject{currentmarker}{\pgfqpoint{-0.027778in}{0.000000in}}{\pgfqpoint{-0.000000in}{0.000000in}}{%
\pgfpathmoveto{\pgfqpoint{-0.000000in}{0.000000in}}%
\pgfpathlineto{\pgfqpoint{-0.027778in}{0.000000in}}%
\pgfusepath{stroke,fill}%
}%
\begin{pgfscope}%
\pgfsys@transformshift{0.588387in}{2.657550in}%
\pgfsys@useobject{currentmarker}{}%
\end{pgfscope}%
\end{pgfscope}%
\begin{pgfscope}%
\pgfsetbuttcap%
\pgfsetroundjoin%
\definecolor{currentfill}{rgb}{0.000000,0.000000,0.000000}%
\pgfsetfillcolor{currentfill}%
\pgfsetlinewidth{0.602250pt}%
\definecolor{currentstroke}{rgb}{0.000000,0.000000,0.000000}%
\pgfsetstrokecolor{currentstroke}%
\pgfsetdash{}{0pt}%
\pgfsys@defobject{currentmarker}{\pgfqpoint{-0.027778in}{0.000000in}}{\pgfqpoint{-0.000000in}{0.000000in}}{%
\pgfpathmoveto{\pgfqpoint{-0.000000in}{0.000000in}}%
\pgfpathlineto{\pgfqpoint{-0.027778in}{0.000000in}}%
\pgfusepath{stroke,fill}%
}%
\begin{pgfscope}%
\pgfsys@transformshift{0.588387in}{2.699316in}%
\pgfsys@useobject{currentmarker}{}%
\end{pgfscope}%
\end{pgfscope}%
\begin{pgfscope}%
\pgfsetbuttcap%
\pgfsetroundjoin%
\definecolor{currentfill}{rgb}{0.000000,0.000000,0.000000}%
\pgfsetfillcolor{currentfill}%
\pgfsetlinewidth{0.602250pt}%
\definecolor{currentstroke}{rgb}{0.000000,0.000000,0.000000}%
\pgfsetstrokecolor{currentstroke}%
\pgfsetdash{}{0pt}%
\pgfsys@defobject{currentmarker}{\pgfqpoint{-0.027778in}{0.000000in}}{\pgfqpoint{-0.000000in}{0.000000in}}{%
\pgfpathmoveto{\pgfqpoint{-0.000000in}{0.000000in}}%
\pgfpathlineto{\pgfqpoint{-0.027778in}{0.000000in}}%
\pgfusepath{stroke,fill}%
}%
\begin{pgfscope}%
\pgfsys@transformshift{0.588387in}{2.735495in}%
\pgfsys@useobject{currentmarker}{}%
\end{pgfscope}%
\end{pgfscope}%
\begin{pgfscope}%
\definecolor{textcolor}{rgb}{0.000000,0.000000,0.000000}%
\pgfsetstrokecolor{textcolor}%
\pgfsetfillcolor{textcolor}%
\pgftext[x=0.234413in,y=1.631726in,,bottom,rotate=90.000000]{\color{textcolor}{\rmfamily\fontsize{10.000000}{12.000000}\selectfont\catcode`\^=\active\def^{\ifmmode\sp\else\^{}\fi}\catcode`\%=\active\def%{\%}Time [ms]}}%
\end{pgfscope}%
\begin{pgfscope}%
\pgfpathrectangle{\pgfqpoint{0.588387in}{0.521603in}}{\pgfqpoint{4.899126in}{2.220246in}}%
\pgfusepath{clip}%
\pgfsetrectcap%
\pgfsetroundjoin%
\pgfsetlinewidth{1.505625pt}%
\pgfsetstrokecolor{currentstroke1}%
\pgfsetdash{}{0pt}%
\pgfpathmoveto{\pgfqpoint{0.811075in}{0.690726in}}%
\pgfpathlineto{\pgfqpoint{0.954744in}{0.714850in}}%
\pgfpathlineto{\pgfqpoint{1.098414in}{0.730772in}}%
\pgfpathlineto{\pgfqpoint{1.242083in}{0.676077in}}%
\pgfpathlineto{\pgfqpoint{1.385752in}{0.692503in}}%
\pgfpathlineto{\pgfqpoint{1.529422in}{0.639898in}}%
\pgfpathlineto{\pgfqpoint{1.673091in}{0.627220in}}%
\pgfpathlineto{\pgfqpoint{1.816761in}{0.714946in}}%
\pgfpathlineto{\pgfqpoint{1.960430in}{0.755063in}}%
\pgfpathlineto{\pgfqpoint{2.104099in}{0.821281in}}%
\pgfpathlineto{\pgfqpoint{2.247769in}{0.944635in}}%
\pgfpathlineto{\pgfqpoint{2.391438in}{1.031572in}}%
\pgfpathlineto{\pgfqpoint{2.535108in}{1.198937in}}%
\pgfpathlineto{\pgfqpoint{2.678777in}{1.279435in}}%
\pgfpathlineto{\pgfqpoint{2.822446in}{1.502609in}}%
\pgfpathlineto{\pgfqpoint{2.966116in}{1.574900in}}%
\pgfpathlineto{\pgfqpoint{3.109785in}{1.746184in}}%
\pgfpathlineto{\pgfqpoint{3.253455in}{1.868961in}}%
\pgfpathlineto{\pgfqpoint{3.397124in}{2.073765in}}%
\pgfpathlineto{\pgfqpoint{3.540793in}{2.215391in}}%
\pgfpathlineto{\pgfqpoint{3.684463in}{2.408094in}}%
\pgfpathlineto{\pgfqpoint{3.828132in}{2.556582in}}%
\pgfusepath{stroke}%
\end{pgfscope}%
\begin{pgfscope}%
\pgfpathrectangle{\pgfqpoint{0.588387in}{0.521603in}}{\pgfqpoint{4.899126in}{2.220246in}}%
\pgfusepath{clip}%
\pgfsetrectcap%
\pgfsetroundjoin%
\pgfsetlinewidth{1.505625pt}%
\pgfsetstrokecolor{currentstroke2}%
\pgfsetdash{}{0pt}%
\pgfpathmoveto{\pgfqpoint{0.811075in}{0.704624in}}%
\pgfpathlineto{\pgfqpoint{0.954744in}{0.719328in}}%
\pgfpathlineto{\pgfqpoint{1.098414in}{0.720594in}}%
\pgfpathlineto{\pgfqpoint{1.242083in}{0.645372in}}%
\pgfpathlineto{\pgfqpoint{1.385752in}{0.622524in}}%
\pgfpathlineto{\pgfqpoint{1.529422in}{0.647172in}}%
\pgfpathlineto{\pgfqpoint{1.673091in}{0.771672in}}%
\pgfpathlineto{\pgfqpoint{1.816761in}{0.827066in}}%
\pgfpathlineto{\pgfqpoint{1.960430in}{0.900386in}}%
\pgfpathlineto{\pgfqpoint{2.104099in}{0.940714in}}%
\pgfpathlineto{\pgfqpoint{2.247769in}{1.066928in}}%
\pgfpathlineto{\pgfqpoint{2.391438in}{1.141190in}}%
\pgfpathlineto{\pgfqpoint{2.535108in}{1.253569in}}%
\pgfpathlineto{\pgfqpoint{2.678777in}{1.291541in}}%
\pgfpathlineto{\pgfqpoint{2.822446in}{1.500679in}}%
\pgfpathlineto{\pgfqpoint{2.966116in}{1.488940in}}%
\pgfpathlineto{\pgfqpoint{3.109785in}{1.615763in}}%
\pgfpathlineto{\pgfqpoint{3.253455in}{1.622641in}}%
\pgfpathlineto{\pgfqpoint{3.397124in}{1.879442in}}%
\pgfpathlineto{\pgfqpoint{3.540793in}{1.878760in}}%
\pgfpathlineto{\pgfqpoint{3.684463in}{2.030727in}}%
\pgfpathlineto{\pgfqpoint{3.828132in}{1.994660in}}%
\pgfpathlineto{\pgfqpoint{3.971802in}{2.236454in}}%
\pgfpathlineto{\pgfqpoint{4.115471in}{2.216924in}}%
\pgfpathlineto{\pgfqpoint{4.259140in}{2.314960in}}%
\pgfpathlineto{\pgfqpoint{4.402810in}{2.422334in}}%
\pgfpathlineto{\pgfqpoint{4.546479in}{2.375334in}}%
\pgfpathlineto{\pgfqpoint{4.690149in}{2.496143in}}%
\pgfpathlineto{\pgfqpoint{4.977487in}{2.620181in}}%
\pgfusepath{stroke}%
\end{pgfscope}%
\begin{pgfscope}%
\pgfpathrectangle{\pgfqpoint{0.588387in}{0.521603in}}{\pgfqpoint{4.899126in}{2.220246in}}%
\pgfusepath{clip}%
\pgfsetrectcap%
\pgfsetroundjoin%
\pgfsetlinewidth{1.505625pt}%
\pgfsetstrokecolor{currentstroke3}%
\pgfsetdash{}{0pt}%
\pgfpathmoveto{\pgfqpoint{0.811075in}{0.697764in}}%
\pgfpathlineto{\pgfqpoint{0.954744in}{0.723734in}}%
\pgfpathlineto{\pgfqpoint{1.098414in}{0.720594in}}%
\pgfpathlineto{\pgfqpoint{1.242083in}{0.651049in}}%
\pgfpathlineto{\pgfqpoint{1.385752in}{0.626678in}}%
\pgfpathlineto{\pgfqpoint{1.529422in}{0.652502in}}%
\pgfpathlineto{\pgfqpoint{1.673091in}{0.773133in}}%
\pgfpathlineto{\pgfqpoint{1.816761in}{0.829599in}}%
\pgfpathlineto{\pgfqpoint{1.960430in}{0.899474in}}%
\pgfpathlineto{\pgfqpoint{2.104099in}{0.939877in}}%
\pgfpathlineto{\pgfqpoint{2.247769in}{1.085265in}}%
\pgfpathlineto{\pgfqpoint{2.391438in}{1.139745in}}%
\pgfpathlineto{\pgfqpoint{2.535108in}{1.252458in}}%
\pgfpathlineto{\pgfqpoint{2.678777in}{1.287237in}}%
\pgfpathlineto{\pgfqpoint{2.822446in}{1.489165in}}%
\pgfpathlineto{\pgfqpoint{2.966116in}{1.473662in}}%
\pgfpathlineto{\pgfqpoint{3.109785in}{1.611308in}}%
\pgfpathlineto{\pgfqpoint{3.253455in}{1.601649in}}%
\pgfpathlineto{\pgfqpoint{3.397124in}{1.807805in}}%
\pgfpathlineto{\pgfqpoint{3.540793in}{1.835485in}}%
\pgfpathlineto{\pgfqpoint{3.684463in}{1.991356in}}%
\pgfpathlineto{\pgfqpoint{3.828132in}{1.941114in}}%
\pgfpathlineto{\pgfqpoint{3.971802in}{2.209076in}}%
\pgfpathlineto{\pgfqpoint{4.115471in}{2.127180in}}%
\pgfpathlineto{\pgfqpoint{4.259140in}{2.251178in}}%
\pgfpathlineto{\pgfqpoint{4.402810in}{2.325477in}}%
\pgfpathlineto{\pgfqpoint{4.546479in}{2.411982in}}%
\pgfpathlineto{\pgfqpoint{4.690149in}{2.394642in}}%
\pgfpathlineto{\pgfqpoint{4.977487in}{2.629270in}}%
\pgfusepath{stroke}%
\end{pgfscope}%
\begin{pgfscope}%
\pgfpathrectangle{\pgfqpoint{0.588387in}{0.521603in}}{\pgfqpoint{4.899126in}{2.220246in}}%
\pgfusepath{clip}%
\pgfsetrectcap%
\pgfsetroundjoin%
\pgfsetlinewidth{1.505625pt}%
\pgfsetstrokecolor{currentstroke4}%
\pgfsetdash{}{0pt}%
\pgfpathmoveto{\pgfqpoint{0.811075in}{0.711314in}}%
\pgfpathlineto{\pgfqpoint{0.954744in}{0.728069in}}%
\pgfpathlineto{\pgfqpoint{1.098414in}{0.723317in}}%
\pgfpathlineto{\pgfqpoint{1.242083in}{0.649672in}}%
\pgfpathlineto{\pgfqpoint{1.385752in}{0.629608in}}%
\pgfpathlineto{\pgfqpoint{1.529422in}{0.645372in}}%
\pgfpathlineto{\pgfqpoint{1.673091in}{0.770204in}}%
\pgfpathlineto{\pgfqpoint{1.816761in}{0.833353in}}%
\pgfpathlineto{\pgfqpoint{1.960430in}{0.901295in}}%
\pgfpathlineto{\pgfqpoint{2.104099in}{0.945686in}}%
\pgfpathlineto{\pgfqpoint{2.247769in}{1.079384in}}%
\pgfpathlineto{\pgfqpoint{2.391438in}{1.151558in}}%
\pgfpathlineto{\pgfqpoint{2.535108in}{1.271661in}}%
\pgfpathlineto{\pgfqpoint{2.678777in}{1.301917in}}%
\pgfpathlineto{\pgfqpoint{2.822446in}{1.508434in}}%
\pgfpathlineto{\pgfqpoint{2.966116in}{1.517459in}}%
\pgfpathlineto{\pgfqpoint{3.109785in}{1.652323in}}%
\pgfpathlineto{\pgfqpoint{3.253455in}{1.663786in}}%
\pgfpathlineto{\pgfqpoint{3.397124in}{1.929239in}}%
\pgfpathlineto{\pgfqpoint{3.540793in}{1.921408in}}%
\pgfpathlineto{\pgfqpoint{3.684463in}{2.079038in}}%
\pgfpathlineto{\pgfqpoint{3.828132in}{2.054013in}}%
\pgfpathlineto{\pgfqpoint{3.971802in}{2.267630in}}%
\pgfpathlineto{\pgfqpoint{4.115471in}{2.264386in}}%
\pgfpathlineto{\pgfqpoint{4.259140in}{2.451023in}}%
\pgfpathlineto{\pgfqpoint{4.402810in}{2.423274in}}%
\pgfpathlineto{\pgfqpoint{4.546479in}{2.535050in}}%
\pgfpathlineto{\pgfqpoint{4.690149in}{2.517597in}}%
\pgfpathlineto{\pgfqpoint{4.977487in}{2.401382in}}%
\pgfusepath{stroke}%
\end{pgfscope}%
\begin{pgfscope}%
\pgfpathrectangle{\pgfqpoint{0.588387in}{0.521603in}}{\pgfqpoint{4.899126in}{2.220246in}}%
\pgfusepath{clip}%
\pgfsetrectcap%
\pgfsetroundjoin%
\pgfsetlinewidth{1.505625pt}%
\pgfsetstrokecolor{currentstroke5}%
\pgfsetdash{}{0pt}%
\pgfpathmoveto{\pgfqpoint{0.811075in}{0.697764in}}%
\pgfpathlineto{\pgfqpoint{0.954744in}{0.728069in}}%
\pgfpathlineto{\pgfqpoint{1.098414in}{0.731063in}}%
\pgfpathlineto{\pgfqpoint{1.242083in}{0.655384in}}%
\pgfpathlineto{\pgfqpoint{1.385752in}{0.624312in}}%
\pgfpathlineto{\pgfqpoint{1.529422in}{0.643559in}}%
\pgfpathlineto{\pgfqpoint{1.673091in}{0.769466in}}%
\pgfpathlineto{\pgfqpoint{1.816761in}{0.832107in}}%
\pgfpathlineto{\pgfqpoint{1.960430in}{0.896257in}}%
\pgfpathlineto{\pgfqpoint{2.104099in}{0.947731in}}%
\pgfpathlineto{\pgfqpoint{2.247769in}{1.085819in}}%
\pgfpathlineto{\pgfqpoint{2.391438in}{1.144532in}}%
\pgfpathlineto{\pgfqpoint{2.535108in}{1.256464in}}%
\pgfpathlineto{\pgfqpoint{2.678777in}{1.296350in}}%
\pgfpathlineto{\pgfqpoint{2.822446in}{1.490631in}}%
\pgfpathlineto{\pgfqpoint{2.966116in}{1.494532in}}%
\pgfpathlineto{\pgfqpoint{3.109785in}{1.621221in}}%
\pgfpathlineto{\pgfqpoint{3.253455in}{1.619772in}}%
\pgfpathlineto{\pgfqpoint{3.397124in}{1.815395in}}%
\pgfpathlineto{\pgfqpoint{3.540793in}{1.829509in}}%
\pgfpathlineto{\pgfqpoint{3.684463in}{2.035358in}}%
\pgfpathlineto{\pgfqpoint{3.828132in}{1.988541in}}%
\pgfpathlineto{\pgfqpoint{3.971802in}{2.225202in}}%
\pgfpathlineto{\pgfqpoint{4.115471in}{2.167549in}}%
\pgfpathlineto{\pgfqpoint{4.259140in}{2.329195in}}%
\pgfpathlineto{\pgfqpoint{4.402810in}{2.413779in}}%
\pgfpathlineto{\pgfqpoint{4.546479in}{2.421860in}}%
\pgfpathlineto{\pgfqpoint{4.690149in}{2.459194in}}%
\pgfpathlineto{\pgfqpoint{4.977487in}{2.452644in}}%
\pgfpathlineto{\pgfqpoint{5.264826in}{2.640929in}}%
\pgfusepath{stroke}%
\end{pgfscope}%
\begin{pgfscope}%
\pgfpathrectangle{\pgfqpoint{0.588387in}{0.521603in}}{\pgfqpoint{4.899126in}{2.220246in}}%
\pgfusepath{clip}%
\pgfsetrectcap%
\pgfsetroundjoin%
\pgfsetlinewidth{1.505625pt}%
\pgfsetstrokecolor{currentstroke6}%
\pgfsetdash{}{0pt}%
\pgfpathmoveto{\pgfqpoint{0.811075in}{0.697764in}}%
\pgfpathlineto{\pgfqpoint{0.954744in}{0.732336in}}%
\pgfpathlineto{\pgfqpoint{1.098414in}{0.744545in}}%
\pgfpathlineto{\pgfqpoint{1.242083in}{0.678122in}}%
\pgfpathlineto{\pgfqpoint{1.385752in}{0.635792in}}%
\pgfpathlineto{\pgfqpoint{1.529422in}{0.645642in}}%
\pgfpathlineto{\pgfqpoint{1.673091in}{0.778184in}}%
\pgfpathlineto{\pgfqpoint{1.816761in}{0.844314in}}%
\pgfpathlineto{\pgfqpoint{1.960430in}{0.908461in}}%
\pgfpathlineto{\pgfqpoint{2.104099in}{0.960465in}}%
\pgfpathlineto{\pgfqpoint{2.247769in}{1.122723in}}%
\pgfpathlineto{\pgfqpoint{2.391438in}{1.188929in}}%
\pgfpathlineto{\pgfqpoint{2.535108in}{1.317909in}}%
\pgfpathlineto{\pgfqpoint{2.678777in}{1.349269in}}%
\pgfpathlineto{\pgfqpoint{2.822446in}{1.574170in}}%
\pgfpathlineto{\pgfqpoint{2.966116in}{1.610743in}}%
\pgfpathlineto{\pgfqpoint{3.109785in}{1.753173in}}%
\pgfpathlineto{\pgfqpoint{3.253455in}{1.788703in}}%
\pgfpathlineto{\pgfqpoint{3.397124in}{1.990433in}}%
\pgfpathlineto{\pgfqpoint{3.540793in}{2.013896in}}%
\pgfpathlineto{\pgfqpoint{3.684463in}{2.196520in}}%
\pgfpathlineto{\pgfqpoint{3.828132in}{2.199252in}}%
\pgfpathlineto{\pgfqpoint{3.971802in}{2.451939in}}%
\pgfpathlineto{\pgfqpoint{4.115471in}{2.400124in}}%
\pgfpathlineto{\pgfqpoint{4.259140in}{2.493325in}}%
\pgfpathlineto{\pgfqpoint{4.402810in}{2.441910in}}%
\pgfpathlineto{\pgfqpoint{4.690149in}{2.485760in}}%
\pgfusepath{stroke}%
\end{pgfscope}%
\begin{pgfscope}%
\pgfpathrectangle{\pgfqpoint{0.588387in}{0.521603in}}{\pgfqpoint{4.899126in}{2.220246in}}%
\pgfusepath{clip}%
\pgfsetrectcap%
\pgfsetroundjoin%
\pgfsetlinewidth{1.505625pt}%
\pgfsetstrokecolor{currentstroke7}%
\pgfsetdash{}{0pt}%
\pgfpathmoveto{\pgfqpoint{0.811075in}{0.697764in}}%
\pgfpathlineto{\pgfqpoint{0.954744in}{0.728069in}}%
\pgfpathlineto{\pgfqpoint{1.098414in}{0.744960in}}%
\pgfpathlineto{\pgfqpoint{1.242083in}{0.662180in}}%
\pgfpathlineto{\pgfqpoint{1.385752in}{0.633207in}}%
\pgfpathlineto{\pgfqpoint{1.529422in}{0.663303in}}%
\pgfpathlineto{\pgfqpoint{1.673091in}{0.769466in}}%
\pgfpathlineto{\pgfqpoint{1.816761in}{0.837669in}}%
\pgfpathlineto{\pgfqpoint{1.960430in}{0.903105in}}%
\pgfpathlineto{\pgfqpoint{2.104099in}{0.957730in}}%
\pgfpathlineto{\pgfqpoint{2.247769in}{1.096126in}}%
\pgfpathlineto{\pgfqpoint{2.391438in}{1.164498in}}%
\pgfpathlineto{\pgfqpoint{2.535108in}{1.282711in}}%
\pgfpathlineto{\pgfqpoint{2.678777in}{1.332279in}}%
\pgfpathlineto{\pgfqpoint{2.822446in}{1.531500in}}%
\pgfpathlineto{\pgfqpoint{2.966116in}{1.563169in}}%
\pgfpathlineto{\pgfqpoint{3.109785in}{1.704780in}}%
\pgfpathlineto{\pgfqpoint{3.253455in}{1.731198in}}%
\pgfpathlineto{\pgfqpoint{3.397124in}{1.913635in}}%
\pgfpathlineto{\pgfqpoint{3.540793in}{1.919796in}}%
\pgfpathlineto{\pgfqpoint{3.684463in}{2.120315in}}%
\pgfpathlineto{\pgfqpoint{3.828132in}{2.119489in}}%
\pgfpathlineto{\pgfqpoint{3.971802in}{2.327937in}}%
\pgfpathlineto{\pgfqpoint{4.115471in}{2.322175in}}%
\pgfpathlineto{\pgfqpoint{4.259140in}{2.479285in}}%
\pgfpathlineto{\pgfqpoint{4.402810in}{2.427057in}}%
\pgfpathlineto{\pgfqpoint{4.546479in}{2.548199in}}%
\pgfpathlineto{\pgfqpoint{4.690149in}{2.527431in}}%
\pgfusepath{stroke}%
\end{pgfscope}%
\begin{pgfscope}%
\pgfsetrectcap%
\pgfsetmiterjoin%
\pgfsetlinewidth{0.803000pt}%
\definecolor{currentstroke}{rgb}{0.000000,0.000000,0.000000}%
\pgfsetstrokecolor{currentstroke}%
\pgfsetdash{}{0pt}%
\pgfpathmoveto{\pgfqpoint{0.588387in}{0.521603in}}%
\pgfpathlineto{\pgfqpoint{0.588387in}{2.741849in}}%
\pgfusepath{stroke}%
\end{pgfscope}%
\begin{pgfscope}%
\pgfsetrectcap%
\pgfsetmiterjoin%
\pgfsetlinewidth{0.803000pt}%
\definecolor{currentstroke}{rgb}{0.000000,0.000000,0.000000}%
\pgfsetstrokecolor{currentstroke}%
\pgfsetdash{}{0pt}%
\pgfpathmoveto{\pgfqpoint{5.487514in}{0.521603in}}%
\pgfpathlineto{\pgfqpoint{5.487514in}{2.741849in}}%
\pgfusepath{stroke}%
\end{pgfscope}%
\begin{pgfscope}%
\pgfsetrectcap%
\pgfsetmiterjoin%
\pgfsetlinewidth{0.803000pt}%
\definecolor{currentstroke}{rgb}{0.000000,0.000000,0.000000}%
\pgfsetstrokecolor{currentstroke}%
\pgfsetdash{}{0pt}%
\pgfpathmoveto{\pgfqpoint{0.588387in}{0.521603in}}%
\pgfpathlineto{\pgfqpoint{5.487514in}{0.521603in}}%
\pgfusepath{stroke}%
\end{pgfscope}%
\begin{pgfscope}%
\pgfsetrectcap%
\pgfsetmiterjoin%
\pgfsetlinewidth{0.803000pt}%
\definecolor{currentstroke}{rgb}{0.000000,0.000000,0.000000}%
\pgfsetstrokecolor{currentstroke}%
\pgfsetdash{}{0pt}%
\pgfpathmoveto{\pgfqpoint{0.588387in}{2.741849in}}%
\pgfpathlineto{\pgfqpoint{5.487514in}{2.741849in}}%
\pgfusepath{stroke}%
\end{pgfscope}%
\begin{pgfscope}%
\pgfsetbuttcap%
\pgfsetmiterjoin%
\definecolor{currentfill}{rgb}{1.000000,1.000000,1.000000}%
\pgfsetfillcolor{currentfill}%
\pgfsetfillopacity{0.800000}%
\pgfsetlinewidth{1.003750pt}%
\definecolor{currentstroke}{rgb}{0.800000,0.800000,0.800000}%
\pgfsetstrokecolor{currentstroke}%
\pgfsetstrokeopacity{0.800000}%
\pgfsetdash{}{0pt}%
\pgfpathmoveto{\pgfqpoint{5.575014in}{1.343633in}}%
\pgfpathlineto{\pgfqpoint{8.259376in}{1.343633in}}%
\pgfpathquadraticcurveto{\pgfqpoint{8.284376in}{1.343633in}}{\pgfqpoint{8.284376in}{1.368633in}}%
\pgfpathlineto{\pgfqpoint{8.284376in}{2.654349in}}%
\pgfpathquadraticcurveto{\pgfqpoint{8.284376in}{2.679349in}}{\pgfqpoint{8.259376in}{2.679349in}}%
\pgfpathlineto{\pgfqpoint{5.575014in}{2.679349in}}%
\pgfpathquadraticcurveto{\pgfqpoint{5.550014in}{2.679349in}}{\pgfqpoint{5.550014in}{2.654349in}}%
\pgfpathlineto{\pgfqpoint{5.550014in}{1.368633in}}%
\pgfpathquadraticcurveto{\pgfqpoint{5.550014in}{1.343633in}}{\pgfqpoint{5.575014in}{1.343633in}}%
\pgfpathlineto{\pgfqpoint{5.575014in}{1.343633in}}%
\pgfpathclose%
\pgfusepath{stroke,fill}%
\end{pgfscope}%
\begin{pgfscope}%
\pgfsetrectcap%
\pgfsetroundjoin%
\pgfsetlinewidth{1.505625pt}%
\pgfsetstrokecolor{currentstroke1}%
\pgfsetdash{}{0pt}%
\pgfpathmoveto{\pgfqpoint{5.600014in}{2.578129in}}%
\pgfpathlineto{\pgfqpoint{5.725014in}{2.578129in}}%
\pgfpathlineto{\pgfqpoint{5.850014in}{2.578129in}}%
\pgfusepath{stroke}%
\end{pgfscope}%
\begin{pgfscope}%
\definecolor{textcolor}{rgb}{0.000000,0.000000,0.000000}%
\pgfsetstrokecolor{textcolor}%
\pgfsetfillcolor{textcolor}%
\pgftext[x=5.950014in,y=2.534379in,left,base]{\color{textcolor}{\rmfamily\fontsize{9.000000}{10.800000}\selectfont\catcode`\^=\active\def^{\ifmmode\sp\else\^{}\fi}\catcode`\%=\active\def%{\%}\NaiveCycles{}}}%
\end{pgfscope}%
\begin{pgfscope}%
\pgfsetrectcap%
\pgfsetroundjoin%
\pgfsetlinewidth{1.505625pt}%
\pgfsetstrokecolor{currentstroke2}%
\pgfsetdash{}{0pt}%
\pgfpathmoveto{\pgfqpoint{5.600014in}{2.394657in}}%
\pgfpathlineto{\pgfqpoint{5.725014in}{2.394657in}}%
\pgfpathlineto{\pgfqpoint{5.850014in}{2.394657in}}%
\pgfusepath{stroke}%
\end{pgfscope}%
\begin{pgfscope}%
\definecolor{textcolor}{rgb}{0.000000,0.000000,0.000000}%
\pgfsetstrokecolor{textcolor}%
\pgfsetfillcolor{textcolor}%
\pgftext[x=5.950014in,y=2.350907in,left,base]{\color{textcolor}{\rmfamily\fontsize{9.000000}{10.800000}\selectfont\catcode`\^=\active\def^{\ifmmode\sp\else\^{}\fi}\catcode`\%=\active\def%{\%}\Neighbors{} \& \MergeLinear{}}}%
\end{pgfscope}%
\begin{pgfscope}%
\pgfsetrectcap%
\pgfsetroundjoin%
\pgfsetlinewidth{1.505625pt}%
\pgfsetstrokecolor{currentstroke3}%
\pgfsetdash{}{0pt}%
\pgfpathmoveto{\pgfqpoint{5.600014in}{2.211185in}}%
\pgfpathlineto{\pgfqpoint{5.725014in}{2.211185in}}%
\pgfpathlineto{\pgfqpoint{5.850014in}{2.211185in}}%
\pgfusepath{stroke}%
\end{pgfscope}%
\begin{pgfscope}%
\definecolor{textcolor}{rgb}{0.000000,0.000000,0.000000}%
\pgfsetstrokecolor{textcolor}%
\pgfsetfillcolor{textcolor}%
\pgftext[x=5.950014in,y=2.167435in,left,base]{\color{textcolor}{\rmfamily\fontsize{9.000000}{10.800000}\selectfont\catcode`\^=\active\def^{\ifmmode\sp\else\^{}\fi}\catcode`\%=\active\def%{\%}\Neighbors{} \& \SharedVertices{}}}%
\end{pgfscope}%
\begin{pgfscope}%
\pgfsetrectcap%
\pgfsetroundjoin%
\pgfsetlinewidth{1.505625pt}%
\pgfsetstrokecolor{currentstroke4}%
\pgfsetdash{}{0pt}%
\pgfpathmoveto{\pgfqpoint{5.600014in}{2.024235in}}%
\pgfpathlineto{\pgfqpoint{5.725014in}{2.024235in}}%
\pgfpathlineto{\pgfqpoint{5.850014in}{2.024235in}}%
\pgfusepath{stroke}%
\end{pgfscope}%
\begin{pgfscope}%
\definecolor{textcolor}{rgb}{0.000000,0.000000,0.000000}%
\pgfsetstrokecolor{textcolor}%
\pgfsetfillcolor{textcolor}%
\pgftext[x=5.950014in,y=1.980485in,left,base]{\color{textcolor}{\rmfamily\fontsize{9.000000}{10.800000}\selectfont\catcode`\^=\active\def^{\ifmmode\sp\else\^{}\fi}\catcode`\%=\active\def%{\%}\NeighborsDegree{} \& \MergeLinear{}}}%
\end{pgfscope}%
\begin{pgfscope}%
\pgfsetrectcap%
\pgfsetroundjoin%
\pgfsetlinewidth{1.505625pt}%
\pgfsetstrokecolor{currentstroke5}%
\pgfsetdash{}{0pt}%
\pgfpathmoveto{\pgfqpoint{5.600014in}{1.837285in}}%
\pgfpathlineto{\pgfqpoint{5.725014in}{1.837285in}}%
\pgfpathlineto{\pgfqpoint{5.850014in}{1.837285in}}%
\pgfusepath{stroke}%
\end{pgfscope}%
\begin{pgfscope}%
\definecolor{textcolor}{rgb}{0.000000,0.000000,0.000000}%
\pgfsetstrokecolor{textcolor}%
\pgfsetfillcolor{textcolor}%
\pgftext[x=5.950014in,y=1.793535in,left,base]{\color{textcolor}{\rmfamily\fontsize{9.000000}{10.800000}\selectfont\catcode`\^=\active\def^{\ifmmode\sp\else\^{}\fi}\catcode`\%=\active\def%{\%}\NeighborsDegree{} \& \SharedVertices{}}}%
\end{pgfscope}%
\begin{pgfscope}%
\pgfsetrectcap%
\pgfsetroundjoin%
\pgfsetlinewidth{1.505625pt}%
\pgfsetstrokecolor{currentstroke6}%
\pgfsetdash{}{0pt}%
\pgfpathmoveto{\pgfqpoint{5.600014in}{1.650334in}}%
\pgfpathlineto{\pgfqpoint{5.725014in}{1.650334in}}%
\pgfpathlineto{\pgfqpoint{5.850014in}{1.650334in}}%
\pgfusepath{stroke}%
\end{pgfscope}%
\begin{pgfscope}%
\definecolor{textcolor}{rgb}{0.000000,0.000000,0.000000}%
\pgfsetstrokecolor{textcolor}%
\pgfsetfillcolor{textcolor}%
\pgftext[x=5.950014in,y=1.606584in,left,base]{\color{textcolor}{\rmfamily\fontsize{9.000000}{10.800000}\selectfont\catcode`\^=\active\def^{\ifmmode\sp\else\^{}\fi}\catcode`\%=\active\def%{\%}\None{} \& \MergeLinear{}}}%
\end{pgfscope}%
\begin{pgfscope}%
\pgfsetrectcap%
\pgfsetroundjoin%
\pgfsetlinewidth{1.505625pt}%
\pgfsetstrokecolor{currentstroke7}%
\pgfsetdash{}{0pt}%
\pgfpathmoveto{\pgfqpoint{5.600014in}{1.466863in}}%
\pgfpathlineto{\pgfqpoint{5.725014in}{1.466863in}}%
\pgfpathlineto{\pgfqpoint{5.850014in}{1.466863in}}%
\pgfusepath{stroke}%
\end{pgfscope}%
\begin{pgfscope}%
\definecolor{textcolor}{rgb}{0.000000,0.000000,0.000000}%
\pgfsetstrokecolor{textcolor}%
\pgfsetfillcolor{textcolor}%
\pgftext[x=5.950014in,y=1.423113in,left,base]{\color{textcolor}{\rmfamily\fontsize{9.000000}{10.800000}\selectfont\catcode`\^=\active\def^{\ifmmode\sp\else\^{}\fi}\catcode`\%=\active\def%{\%}\None{} \& \SharedVertices{}}}%
\end{pgfscope}%
\end{pgfpicture}%
\makeatother%
\endgroup%
}
		\caption[Running time for~minimally rigid graphs]{
			Mean running time to find all NAC-colorings for~minimally rigid graphs.}%
	\end{figure}%

	\begin{figure}[thbp]
		\centering
		\scalebox{\BenchFigureScale}{%% Creator: Matplotlib, PGF backend
%%
%% To include the figure in your LaTeX document, write
%%   \input{<filename>.pgf}
%%
%% Make sure the required packages are loaded in your preamble
%%   \usepackage{pgf}
%%
%% Also ensure that all the required font packages are loaded; for instance,
%% the lmodern package is sometimes necessary when using math font.
%%   \usepackage{lmodern}
%%
%% Figures using additional raster images can only be included by \input if
%% they are in the same directory as the main LaTeX file. For loading figures
%% from other directories you can use the `import` package
%%   \usepackage{import}
%%
%% and then include the figures with
%%   \import{<path to file>}{<filename>.pgf}
%%
%% Matplotlib used the following preamble
%%   \def\mathdefault#1{#1}
%%   \everymath=\expandafter{\the\everymath\displaystyle}
%%   \IfFileExists{scrextend.sty}{
%%     \usepackage[fontsize=10.000000pt]{scrextend}
%%   }{
%%     \renewcommand{\normalsize}{\fontsize{10.000000}{12.000000}\selectfont}
%%     \normalsize
%%   }
%%   
%%   \ifdefined\pdftexversion\else  % non-pdftex case.
%%     \usepackage{fontspec}
%%     \setmainfont{DejaVuSans.ttf}[Path=\detokenize{/home/petr/Projects/PyRigi/.venv/lib/python3.12/site-packages/matplotlib/mpl-data/fonts/ttf/}]
%%     \setsansfont{DejaVuSans.ttf}[Path=\detokenize{/home/petr/Projects/PyRigi/.venv/lib/python3.12/site-packages/matplotlib/mpl-data/fonts/ttf/}]
%%     \setmonofont{DejaVuSansMono.ttf}[Path=\detokenize{/home/petr/Projects/PyRigi/.venv/lib/python3.12/site-packages/matplotlib/mpl-data/fonts/ttf/}]
%%   \fi
%%   \makeatletter\@ifpackageloaded{under\Score{}}{}{\usepackage[strings]{under\Score{}}}\makeatother
%%
\begingroup%
\makeatletter%
\begin{pgfpicture}%
\pgfpathrectangle{\pgfpointorigin}{\pgfqpoint{8.384376in}{2.841849in}}%
\pgfusepath{use as bounding box, clip}%
\begin{pgfscope}%
\pgfsetbuttcap%
\pgfsetmiterjoin%
\definecolor{currentfill}{rgb}{1.000000,1.000000,1.000000}%
\pgfsetfillcolor{currentfill}%
\pgfsetlinewidth{0.000000pt}%
\definecolor{currentstroke}{rgb}{1.000000,1.000000,1.000000}%
\pgfsetstrokecolor{currentstroke}%
\pgfsetdash{}{0pt}%
\pgfpathmoveto{\pgfqpoint{0.000000in}{0.000000in}}%
\pgfpathlineto{\pgfqpoint{8.384376in}{0.000000in}}%
\pgfpathlineto{\pgfqpoint{8.384376in}{2.841849in}}%
\pgfpathlineto{\pgfqpoint{0.000000in}{2.841849in}}%
\pgfpathlineto{\pgfqpoint{0.000000in}{0.000000in}}%
\pgfpathclose%
\pgfusepath{fill}%
\end{pgfscope}%
\begin{pgfscope}%
\pgfsetbuttcap%
\pgfsetmiterjoin%
\definecolor{currentfill}{rgb}{1.000000,1.000000,1.000000}%
\pgfsetfillcolor{currentfill}%
\pgfsetlinewidth{0.000000pt}%
\definecolor{currentstroke}{rgb}{0.000000,0.000000,0.000000}%
\pgfsetstrokecolor{currentstroke}%
\pgfsetstrokeopacity{0.000000}%
\pgfsetdash{}{0pt}%
\pgfpathmoveto{\pgfqpoint{0.588387in}{0.521603in}}%
\pgfpathlineto{\pgfqpoint{5.257411in}{0.521603in}}%
\pgfpathlineto{\pgfqpoint{5.257411in}{2.741849in}}%
\pgfpathlineto{\pgfqpoint{0.588387in}{2.741849in}}%
\pgfpathlineto{\pgfqpoint{0.588387in}{0.521603in}}%
\pgfpathclose%
\pgfusepath{fill}%
\end{pgfscope}%
\begin{pgfscope}%
\pgfsetbuttcap%
\pgfsetroundjoin%
\definecolor{currentfill}{rgb}{0.000000,0.000000,0.000000}%
\pgfsetfillcolor{currentfill}%
\pgfsetlinewidth{0.803000pt}%
\definecolor{currentstroke}{rgb}{0.000000,0.000000,0.000000}%
\pgfsetstrokecolor{currentstroke}%
\pgfsetdash{}{0pt}%
\pgfsys@defobject{currentmarker}{\pgfqpoint{0.000000in}{-0.048611in}}{\pgfqpoint{0.000000in}{0.000000in}}{%
\pgfpathmoveto{\pgfqpoint{0.000000in}{0.000000in}}%
\pgfpathlineto{\pgfqpoint{0.000000in}{-0.048611in}}%
\pgfusepath{stroke,fill}%
}%
\begin{pgfscope}%
\pgfsys@transformshift{0.718197in}{0.521603in}%
\pgfsys@useobject{currentmarker}{}%
\end{pgfscope}%
\end{pgfscope}%
\begin{pgfscope}%
\definecolor{textcolor}{rgb}{0.000000,0.000000,0.000000}%
\pgfsetstrokecolor{textcolor}%
\pgfsetfillcolor{textcolor}%
\pgftext[x=0.718197in,y=0.424381in,,top]{\color{textcolor}{\rmfamily\fontsize{10.000000}{12.000000}\selectfont\catcode`\^=\active\def^{\ifmmode\sp\else\^{}\fi}\catcode`\%=\active\def%{\%}$\mathdefault{0}$}}%
\end{pgfscope}%
\begin{pgfscope}%
\pgfsetbuttcap%
\pgfsetroundjoin%
\definecolor{currentfill}{rgb}{0.000000,0.000000,0.000000}%
\pgfsetfillcolor{currentfill}%
\pgfsetlinewidth{0.803000pt}%
\definecolor{currentstroke}{rgb}{0.000000,0.000000,0.000000}%
\pgfsetstrokecolor{currentstroke}%
\pgfsetdash{}{0pt}%
\pgfsys@defobject{currentmarker}{\pgfqpoint{0.000000in}{-0.048611in}}{\pgfqpoint{0.000000in}{0.000000in}}{%
\pgfpathmoveto{\pgfqpoint{0.000000in}{0.000000in}}%
\pgfpathlineto{\pgfqpoint{0.000000in}{-0.048611in}}%
\pgfusepath{stroke,fill}%
}%
\begin{pgfscope}%
\pgfsys@transformshift{1.336338in}{0.521603in}%
\pgfsys@useobject{currentmarker}{}%
\end{pgfscope}%
\end{pgfscope}%
\begin{pgfscope}%
\definecolor{textcolor}{rgb}{0.000000,0.000000,0.000000}%
\pgfsetstrokecolor{textcolor}%
\pgfsetfillcolor{textcolor}%
\pgftext[x=1.336338in,y=0.424381in,,top]{\color{textcolor}{\rmfamily\fontsize{10.000000}{12.000000}\selectfont\catcode`\^=\active\def^{\ifmmode\sp\else\^{}\fi}\catcode`\%=\active\def%{\%}$\mathdefault{15}$}}%
\end{pgfscope}%
\begin{pgfscope}%
\pgfsetbuttcap%
\pgfsetroundjoin%
\definecolor{currentfill}{rgb}{0.000000,0.000000,0.000000}%
\pgfsetfillcolor{currentfill}%
\pgfsetlinewidth{0.803000pt}%
\definecolor{currentstroke}{rgb}{0.000000,0.000000,0.000000}%
\pgfsetstrokecolor{currentstroke}%
\pgfsetdash{}{0pt}%
\pgfsys@defobject{currentmarker}{\pgfqpoint{0.000000in}{-0.048611in}}{\pgfqpoint{0.000000in}{0.000000in}}{%
\pgfpathmoveto{\pgfqpoint{0.000000in}{0.000000in}}%
\pgfpathlineto{\pgfqpoint{0.000000in}{-0.048611in}}%
\pgfusepath{stroke,fill}%
}%
\begin{pgfscope}%
\pgfsys@transformshift{1.954479in}{0.521603in}%
\pgfsys@useobject{currentmarker}{}%
\end{pgfscope}%
\end{pgfscope}%
\begin{pgfscope}%
\definecolor{textcolor}{rgb}{0.000000,0.000000,0.000000}%
\pgfsetstrokecolor{textcolor}%
\pgfsetfillcolor{textcolor}%
\pgftext[x=1.954479in,y=0.424381in,,top]{\color{textcolor}{\rmfamily\fontsize{10.000000}{12.000000}\selectfont\catcode`\^=\active\def^{\ifmmode\sp\else\^{}\fi}\catcode`\%=\active\def%{\%}$\mathdefault{30}$}}%
\end{pgfscope}%
\begin{pgfscope}%
\pgfsetbuttcap%
\pgfsetroundjoin%
\definecolor{currentfill}{rgb}{0.000000,0.000000,0.000000}%
\pgfsetfillcolor{currentfill}%
\pgfsetlinewidth{0.803000pt}%
\definecolor{currentstroke}{rgb}{0.000000,0.000000,0.000000}%
\pgfsetstrokecolor{currentstroke}%
\pgfsetdash{}{0pt}%
\pgfsys@defobject{currentmarker}{\pgfqpoint{0.000000in}{-0.048611in}}{\pgfqpoint{0.000000in}{0.000000in}}{%
\pgfpathmoveto{\pgfqpoint{0.000000in}{0.000000in}}%
\pgfpathlineto{\pgfqpoint{0.000000in}{-0.048611in}}%
\pgfusepath{stroke,fill}%
}%
\begin{pgfscope}%
\pgfsys@transformshift{2.572619in}{0.521603in}%
\pgfsys@useobject{currentmarker}{}%
\end{pgfscope}%
\end{pgfscope}%
\begin{pgfscope}%
\definecolor{textcolor}{rgb}{0.000000,0.000000,0.000000}%
\pgfsetstrokecolor{textcolor}%
\pgfsetfillcolor{textcolor}%
\pgftext[x=2.572619in,y=0.424381in,,top]{\color{textcolor}{\rmfamily\fontsize{10.000000}{12.000000}\selectfont\catcode`\^=\active\def^{\ifmmode\sp\else\^{}\fi}\catcode`\%=\active\def%{\%}$\mathdefault{45}$}}%
\end{pgfscope}%
\begin{pgfscope}%
\pgfsetbuttcap%
\pgfsetroundjoin%
\definecolor{currentfill}{rgb}{0.000000,0.000000,0.000000}%
\pgfsetfillcolor{currentfill}%
\pgfsetlinewidth{0.803000pt}%
\definecolor{currentstroke}{rgb}{0.000000,0.000000,0.000000}%
\pgfsetstrokecolor{currentstroke}%
\pgfsetdash{}{0pt}%
\pgfsys@defobject{currentmarker}{\pgfqpoint{0.000000in}{-0.048611in}}{\pgfqpoint{0.000000in}{0.000000in}}{%
\pgfpathmoveto{\pgfqpoint{0.000000in}{0.000000in}}%
\pgfpathlineto{\pgfqpoint{0.000000in}{-0.048611in}}%
\pgfusepath{stroke,fill}%
}%
\begin{pgfscope}%
\pgfsys@transformshift{3.190760in}{0.521603in}%
\pgfsys@useobject{currentmarker}{}%
\end{pgfscope}%
\end{pgfscope}%
\begin{pgfscope}%
\definecolor{textcolor}{rgb}{0.000000,0.000000,0.000000}%
\pgfsetstrokecolor{textcolor}%
\pgfsetfillcolor{textcolor}%
\pgftext[x=3.190760in,y=0.424381in,,top]{\color{textcolor}{\rmfamily\fontsize{10.000000}{12.000000}\selectfont\catcode`\^=\active\def^{\ifmmode\sp\else\^{}\fi}\catcode`\%=\active\def%{\%}$\mathdefault{60}$}}%
\end{pgfscope}%
\begin{pgfscope}%
\pgfsetbuttcap%
\pgfsetroundjoin%
\definecolor{currentfill}{rgb}{0.000000,0.000000,0.000000}%
\pgfsetfillcolor{currentfill}%
\pgfsetlinewidth{0.803000pt}%
\definecolor{currentstroke}{rgb}{0.000000,0.000000,0.000000}%
\pgfsetstrokecolor{currentstroke}%
\pgfsetdash{}{0pt}%
\pgfsys@defobject{currentmarker}{\pgfqpoint{0.000000in}{-0.048611in}}{\pgfqpoint{0.000000in}{0.000000in}}{%
\pgfpathmoveto{\pgfqpoint{0.000000in}{0.000000in}}%
\pgfpathlineto{\pgfqpoint{0.000000in}{-0.048611in}}%
\pgfusepath{stroke,fill}%
}%
\begin{pgfscope}%
\pgfsys@transformshift{3.808901in}{0.521603in}%
\pgfsys@useobject{currentmarker}{}%
\end{pgfscope}%
\end{pgfscope}%
\begin{pgfscope}%
\definecolor{textcolor}{rgb}{0.000000,0.000000,0.000000}%
\pgfsetstrokecolor{textcolor}%
\pgfsetfillcolor{textcolor}%
\pgftext[x=3.808901in,y=0.424381in,,top]{\color{textcolor}{\rmfamily\fontsize{10.000000}{12.000000}\selectfont\catcode`\^=\active\def^{\ifmmode\sp\else\^{}\fi}\catcode`\%=\active\def%{\%}$\mathdefault{75}$}}%
\end{pgfscope}%
\begin{pgfscope}%
\pgfsetbuttcap%
\pgfsetroundjoin%
\definecolor{currentfill}{rgb}{0.000000,0.000000,0.000000}%
\pgfsetfillcolor{currentfill}%
\pgfsetlinewidth{0.803000pt}%
\definecolor{currentstroke}{rgb}{0.000000,0.000000,0.000000}%
\pgfsetstrokecolor{currentstroke}%
\pgfsetdash{}{0pt}%
\pgfsys@defobject{currentmarker}{\pgfqpoint{0.000000in}{-0.048611in}}{\pgfqpoint{0.000000in}{0.000000in}}{%
\pgfpathmoveto{\pgfqpoint{0.000000in}{0.000000in}}%
\pgfpathlineto{\pgfqpoint{0.000000in}{-0.048611in}}%
\pgfusepath{stroke,fill}%
}%
\begin{pgfscope}%
\pgfsys@transformshift{4.427042in}{0.521603in}%
\pgfsys@useobject{currentmarker}{}%
\end{pgfscope}%
\end{pgfscope}%
\begin{pgfscope}%
\definecolor{textcolor}{rgb}{0.000000,0.000000,0.000000}%
\pgfsetstrokecolor{textcolor}%
\pgfsetfillcolor{textcolor}%
\pgftext[x=4.427042in,y=0.424381in,,top]{\color{textcolor}{\rmfamily\fontsize{10.000000}{12.000000}\selectfont\catcode`\^=\active\def^{\ifmmode\sp\else\^{}\fi}\catcode`\%=\active\def%{\%}$\mathdefault{90}$}}%
\end{pgfscope}%
\begin{pgfscope}%
\pgfsetbuttcap%
\pgfsetroundjoin%
\definecolor{currentfill}{rgb}{0.000000,0.000000,0.000000}%
\pgfsetfillcolor{currentfill}%
\pgfsetlinewidth{0.803000pt}%
\definecolor{currentstroke}{rgb}{0.000000,0.000000,0.000000}%
\pgfsetstrokecolor{currentstroke}%
\pgfsetdash{}{0pt}%
\pgfsys@defobject{currentmarker}{\pgfqpoint{0.000000in}{-0.048611in}}{\pgfqpoint{0.000000in}{0.000000in}}{%
\pgfpathmoveto{\pgfqpoint{0.000000in}{0.000000in}}%
\pgfpathlineto{\pgfqpoint{0.000000in}{-0.048611in}}%
\pgfusepath{stroke,fill}%
}%
\begin{pgfscope}%
\pgfsys@transformshift{5.045183in}{0.521603in}%
\pgfsys@useobject{currentmarker}{}%
\end{pgfscope}%
\end{pgfscope}%
\begin{pgfscope}%
\definecolor{textcolor}{rgb}{0.000000,0.000000,0.000000}%
\pgfsetstrokecolor{textcolor}%
\pgfsetfillcolor{textcolor}%
\pgftext[x=5.045183in,y=0.424381in,,top]{\color{textcolor}{\rmfamily\fontsize{10.000000}{12.000000}\selectfont\catcode`\^=\active\def^{\ifmmode\sp\else\^{}\fi}\catcode`\%=\active\def%{\%}$\mathdefault{105}$}}%
\end{pgfscope}%
\begin{pgfscope}%
\definecolor{textcolor}{rgb}{0.000000,0.000000,0.000000}%
\pgfsetstrokecolor{textcolor}%
\pgfsetfillcolor{textcolor}%
\pgftext[x=2.922899in,y=0.234413in,,top]{\color{textcolor}{\rmfamily\fontsize{10.000000}{12.000000}\selectfont\catcode`\^=\active\def^{\ifmmode\sp\else\^{}\fi}\catcode`\%=\active\def%{\%}Monochromatic classes}}%
\end{pgfscope}%
\begin{pgfscope}%
\pgfsetbuttcap%
\pgfsetroundjoin%
\definecolor{currentfill}{rgb}{0.000000,0.000000,0.000000}%
\pgfsetfillcolor{currentfill}%
\pgfsetlinewidth{0.803000pt}%
\definecolor{currentstroke}{rgb}{0.000000,0.000000,0.000000}%
\pgfsetstrokecolor{currentstroke}%
\pgfsetdash{}{0pt}%
\pgfsys@defobject{currentmarker}{\pgfqpoint{-0.048611in}{0.000000in}}{\pgfqpoint{-0.000000in}{0.000000in}}{%
\pgfpathmoveto{\pgfqpoint{-0.000000in}{0.000000in}}%
\pgfpathlineto{\pgfqpoint{-0.048611in}{0.000000in}}%
\pgfusepath{stroke,fill}%
}%
\begin{pgfscope}%
\pgfsys@transformshift{0.588387in}{1.046565in}%
\pgfsys@useobject{currentmarker}{}%
\end{pgfscope}%
\end{pgfscope}%
\begin{pgfscope}%
\definecolor{textcolor}{rgb}{0.000000,0.000000,0.000000}%
\pgfsetstrokecolor{textcolor}%
\pgfsetfillcolor{textcolor}%
\pgftext[x=0.289968in, y=0.993803in, left, base]{\color{textcolor}{\rmfamily\fontsize{10.000000}{12.000000}\selectfont\catcode`\^=\active\def^{\ifmmode\sp\else\^{}\fi}\catcode`\%=\active\def%{\%}$\mathdefault{10^{1}}$}}%
\end{pgfscope}%
\begin{pgfscope}%
\pgfsetbuttcap%
\pgfsetroundjoin%
\definecolor{currentfill}{rgb}{0.000000,0.000000,0.000000}%
\pgfsetfillcolor{currentfill}%
\pgfsetlinewidth{0.803000pt}%
\definecolor{currentstroke}{rgb}{0.000000,0.000000,0.000000}%
\pgfsetstrokecolor{currentstroke}%
\pgfsetdash{}{0pt}%
\pgfsys@defobject{currentmarker}{\pgfqpoint{-0.048611in}{0.000000in}}{\pgfqpoint{-0.000000in}{0.000000in}}{%
\pgfpathmoveto{\pgfqpoint{-0.000000in}{0.000000in}}%
\pgfpathlineto{\pgfqpoint{-0.048611in}{0.000000in}}%
\pgfusepath{stroke,fill}%
}%
\begin{pgfscope}%
\pgfsys@transformshift{0.588387in}{1.657004in}%
\pgfsys@useobject{currentmarker}{}%
\end{pgfscope}%
\end{pgfscope}%
\begin{pgfscope}%
\definecolor{textcolor}{rgb}{0.000000,0.000000,0.000000}%
\pgfsetstrokecolor{textcolor}%
\pgfsetfillcolor{textcolor}%
\pgftext[x=0.289968in, y=1.604243in, left, base]{\color{textcolor}{\rmfamily\fontsize{10.000000}{12.000000}\selectfont\catcode`\^=\active\def^{\ifmmode\sp\else\^{}\fi}\catcode`\%=\active\def%{\%}$\mathdefault{10^{2}}$}}%
\end{pgfscope}%
\begin{pgfscope}%
\pgfsetbuttcap%
\pgfsetroundjoin%
\definecolor{currentfill}{rgb}{0.000000,0.000000,0.000000}%
\pgfsetfillcolor{currentfill}%
\pgfsetlinewidth{0.803000pt}%
\definecolor{currentstroke}{rgb}{0.000000,0.000000,0.000000}%
\pgfsetstrokecolor{currentstroke}%
\pgfsetdash{}{0pt}%
\pgfsys@defobject{currentmarker}{\pgfqpoint{-0.048611in}{0.000000in}}{\pgfqpoint{-0.000000in}{0.000000in}}{%
\pgfpathmoveto{\pgfqpoint{-0.000000in}{0.000000in}}%
\pgfpathlineto{\pgfqpoint{-0.048611in}{0.000000in}}%
\pgfusepath{stroke,fill}%
}%
\begin{pgfscope}%
\pgfsys@transformshift{0.588387in}{2.267444in}%
\pgfsys@useobject{currentmarker}{}%
\end{pgfscope}%
\end{pgfscope}%
\begin{pgfscope}%
\definecolor{textcolor}{rgb}{0.000000,0.000000,0.000000}%
\pgfsetstrokecolor{textcolor}%
\pgfsetfillcolor{textcolor}%
\pgftext[x=0.289968in, y=2.214682in, left, base]{\color{textcolor}{\rmfamily\fontsize{10.000000}{12.000000}\selectfont\catcode`\^=\active\def^{\ifmmode\sp\else\^{}\fi}\catcode`\%=\active\def%{\%}$\mathdefault{10^{3}}$}}%
\end{pgfscope}%
\begin{pgfscope}%
\pgfsetbuttcap%
\pgfsetroundjoin%
\definecolor{currentfill}{rgb}{0.000000,0.000000,0.000000}%
\pgfsetfillcolor{currentfill}%
\pgfsetlinewidth{0.602250pt}%
\definecolor{currentstroke}{rgb}{0.000000,0.000000,0.000000}%
\pgfsetstrokecolor{currentstroke}%
\pgfsetdash{}{0pt}%
\pgfsys@defobject{currentmarker}{\pgfqpoint{-0.027778in}{0.000000in}}{\pgfqpoint{-0.000000in}{0.000000in}}{%
\pgfpathmoveto{\pgfqpoint{-0.000000in}{0.000000in}}%
\pgfpathlineto{\pgfqpoint{-0.027778in}{0.000000in}}%
\pgfusepath{stroke,fill}%
}%
\begin{pgfscope}%
\pgfsys@transformshift{0.588387in}{0.619886in}%
\pgfsys@useobject{currentmarker}{}%
\end{pgfscope}%
\end{pgfscope}%
\begin{pgfscope}%
\pgfsetbuttcap%
\pgfsetroundjoin%
\definecolor{currentfill}{rgb}{0.000000,0.000000,0.000000}%
\pgfsetfillcolor{currentfill}%
\pgfsetlinewidth{0.602250pt}%
\definecolor{currentstroke}{rgb}{0.000000,0.000000,0.000000}%
\pgfsetstrokecolor{currentstroke}%
\pgfsetdash{}{0pt}%
\pgfsys@defobject{currentmarker}{\pgfqpoint{-0.027778in}{0.000000in}}{\pgfqpoint{-0.000000in}{0.000000in}}{%
\pgfpathmoveto{\pgfqpoint{-0.000000in}{0.000000in}}%
\pgfpathlineto{\pgfqpoint{-0.027778in}{0.000000in}}%
\pgfusepath{stroke,fill}%
}%
\begin{pgfscope}%
\pgfsys@transformshift{0.588387in}{0.727379in}%
\pgfsys@useobject{currentmarker}{}%
\end{pgfscope}%
\end{pgfscope}%
\begin{pgfscope}%
\pgfsetbuttcap%
\pgfsetroundjoin%
\definecolor{currentfill}{rgb}{0.000000,0.000000,0.000000}%
\pgfsetfillcolor{currentfill}%
\pgfsetlinewidth{0.602250pt}%
\definecolor{currentstroke}{rgb}{0.000000,0.000000,0.000000}%
\pgfsetstrokecolor{currentstroke}%
\pgfsetdash{}{0pt}%
\pgfsys@defobject{currentmarker}{\pgfqpoint{-0.027778in}{0.000000in}}{\pgfqpoint{-0.000000in}{0.000000in}}{%
\pgfpathmoveto{\pgfqpoint{-0.000000in}{0.000000in}}%
\pgfpathlineto{\pgfqpoint{-0.027778in}{0.000000in}}%
\pgfusepath{stroke,fill}%
}%
\begin{pgfscope}%
\pgfsys@transformshift{0.588387in}{0.803646in}%
\pgfsys@useobject{currentmarker}{}%
\end{pgfscope}%
\end{pgfscope}%
\begin{pgfscope}%
\pgfsetbuttcap%
\pgfsetroundjoin%
\definecolor{currentfill}{rgb}{0.000000,0.000000,0.000000}%
\pgfsetfillcolor{currentfill}%
\pgfsetlinewidth{0.602250pt}%
\definecolor{currentstroke}{rgb}{0.000000,0.000000,0.000000}%
\pgfsetstrokecolor{currentstroke}%
\pgfsetdash{}{0pt}%
\pgfsys@defobject{currentmarker}{\pgfqpoint{-0.027778in}{0.000000in}}{\pgfqpoint{-0.000000in}{0.000000in}}{%
\pgfpathmoveto{\pgfqpoint{-0.000000in}{0.000000in}}%
\pgfpathlineto{\pgfqpoint{-0.027778in}{0.000000in}}%
\pgfusepath{stroke,fill}%
}%
\begin{pgfscope}%
\pgfsys@transformshift{0.588387in}{0.862804in}%
\pgfsys@useobject{currentmarker}{}%
\end{pgfscope}%
\end{pgfscope}%
\begin{pgfscope}%
\pgfsetbuttcap%
\pgfsetroundjoin%
\definecolor{currentfill}{rgb}{0.000000,0.000000,0.000000}%
\pgfsetfillcolor{currentfill}%
\pgfsetlinewidth{0.602250pt}%
\definecolor{currentstroke}{rgb}{0.000000,0.000000,0.000000}%
\pgfsetstrokecolor{currentstroke}%
\pgfsetdash{}{0pt}%
\pgfsys@defobject{currentmarker}{\pgfqpoint{-0.027778in}{0.000000in}}{\pgfqpoint{-0.000000in}{0.000000in}}{%
\pgfpathmoveto{\pgfqpoint{-0.000000in}{0.000000in}}%
\pgfpathlineto{\pgfqpoint{-0.027778in}{0.000000in}}%
\pgfusepath{stroke,fill}%
}%
\begin{pgfscope}%
\pgfsys@transformshift{0.588387in}{0.911139in}%
\pgfsys@useobject{currentmarker}{}%
\end{pgfscope}%
\end{pgfscope}%
\begin{pgfscope}%
\pgfsetbuttcap%
\pgfsetroundjoin%
\definecolor{currentfill}{rgb}{0.000000,0.000000,0.000000}%
\pgfsetfillcolor{currentfill}%
\pgfsetlinewidth{0.602250pt}%
\definecolor{currentstroke}{rgb}{0.000000,0.000000,0.000000}%
\pgfsetstrokecolor{currentstroke}%
\pgfsetdash{}{0pt}%
\pgfsys@defobject{currentmarker}{\pgfqpoint{-0.027778in}{0.000000in}}{\pgfqpoint{-0.000000in}{0.000000in}}{%
\pgfpathmoveto{\pgfqpoint{-0.000000in}{0.000000in}}%
\pgfpathlineto{\pgfqpoint{-0.027778in}{0.000000in}}%
\pgfusepath{stroke,fill}%
}%
\begin{pgfscope}%
\pgfsys@transformshift{0.588387in}{0.952006in}%
\pgfsys@useobject{currentmarker}{}%
\end{pgfscope}%
\end{pgfscope}%
\begin{pgfscope}%
\pgfsetbuttcap%
\pgfsetroundjoin%
\definecolor{currentfill}{rgb}{0.000000,0.000000,0.000000}%
\pgfsetfillcolor{currentfill}%
\pgfsetlinewidth{0.602250pt}%
\definecolor{currentstroke}{rgb}{0.000000,0.000000,0.000000}%
\pgfsetstrokecolor{currentstroke}%
\pgfsetdash{}{0pt}%
\pgfsys@defobject{currentmarker}{\pgfqpoint{-0.027778in}{0.000000in}}{\pgfqpoint{-0.000000in}{0.000000in}}{%
\pgfpathmoveto{\pgfqpoint{-0.000000in}{0.000000in}}%
\pgfpathlineto{\pgfqpoint{-0.027778in}{0.000000in}}%
\pgfusepath{stroke,fill}%
}%
\begin{pgfscope}%
\pgfsys@transformshift{0.588387in}{0.987407in}%
\pgfsys@useobject{currentmarker}{}%
\end{pgfscope}%
\end{pgfscope}%
\begin{pgfscope}%
\pgfsetbuttcap%
\pgfsetroundjoin%
\definecolor{currentfill}{rgb}{0.000000,0.000000,0.000000}%
\pgfsetfillcolor{currentfill}%
\pgfsetlinewidth{0.602250pt}%
\definecolor{currentstroke}{rgb}{0.000000,0.000000,0.000000}%
\pgfsetstrokecolor{currentstroke}%
\pgfsetdash{}{0pt}%
\pgfsys@defobject{currentmarker}{\pgfqpoint{-0.027778in}{0.000000in}}{\pgfqpoint{-0.000000in}{0.000000in}}{%
\pgfpathmoveto{\pgfqpoint{-0.000000in}{0.000000in}}%
\pgfpathlineto{\pgfqpoint{-0.027778in}{0.000000in}}%
\pgfusepath{stroke,fill}%
}%
\begin{pgfscope}%
\pgfsys@transformshift{0.588387in}{1.018632in}%
\pgfsys@useobject{currentmarker}{}%
\end{pgfscope}%
\end{pgfscope}%
\begin{pgfscope}%
\pgfsetbuttcap%
\pgfsetroundjoin%
\definecolor{currentfill}{rgb}{0.000000,0.000000,0.000000}%
\pgfsetfillcolor{currentfill}%
\pgfsetlinewidth{0.602250pt}%
\definecolor{currentstroke}{rgb}{0.000000,0.000000,0.000000}%
\pgfsetstrokecolor{currentstroke}%
\pgfsetdash{}{0pt}%
\pgfsys@defobject{currentmarker}{\pgfqpoint{-0.027778in}{0.000000in}}{\pgfqpoint{-0.000000in}{0.000000in}}{%
\pgfpathmoveto{\pgfqpoint{-0.000000in}{0.000000in}}%
\pgfpathlineto{\pgfqpoint{-0.027778in}{0.000000in}}%
\pgfusepath{stroke,fill}%
}%
\begin{pgfscope}%
\pgfsys@transformshift{0.588387in}{1.230325in}%
\pgfsys@useobject{currentmarker}{}%
\end{pgfscope}%
\end{pgfscope}%
\begin{pgfscope}%
\pgfsetbuttcap%
\pgfsetroundjoin%
\definecolor{currentfill}{rgb}{0.000000,0.000000,0.000000}%
\pgfsetfillcolor{currentfill}%
\pgfsetlinewidth{0.602250pt}%
\definecolor{currentstroke}{rgb}{0.000000,0.000000,0.000000}%
\pgfsetstrokecolor{currentstroke}%
\pgfsetdash{}{0pt}%
\pgfsys@defobject{currentmarker}{\pgfqpoint{-0.027778in}{0.000000in}}{\pgfqpoint{-0.000000in}{0.000000in}}{%
\pgfpathmoveto{\pgfqpoint{-0.000000in}{0.000000in}}%
\pgfpathlineto{\pgfqpoint{-0.027778in}{0.000000in}}%
\pgfusepath{stroke,fill}%
}%
\begin{pgfscope}%
\pgfsys@transformshift{0.588387in}{1.337818in}%
\pgfsys@useobject{currentmarker}{}%
\end{pgfscope}%
\end{pgfscope}%
\begin{pgfscope}%
\pgfsetbuttcap%
\pgfsetroundjoin%
\definecolor{currentfill}{rgb}{0.000000,0.000000,0.000000}%
\pgfsetfillcolor{currentfill}%
\pgfsetlinewidth{0.602250pt}%
\definecolor{currentstroke}{rgb}{0.000000,0.000000,0.000000}%
\pgfsetstrokecolor{currentstroke}%
\pgfsetdash{}{0pt}%
\pgfsys@defobject{currentmarker}{\pgfqpoint{-0.027778in}{0.000000in}}{\pgfqpoint{-0.000000in}{0.000000in}}{%
\pgfpathmoveto{\pgfqpoint{-0.000000in}{0.000000in}}%
\pgfpathlineto{\pgfqpoint{-0.027778in}{0.000000in}}%
\pgfusepath{stroke,fill}%
}%
\begin{pgfscope}%
\pgfsys@transformshift{0.588387in}{1.414086in}%
\pgfsys@useobject{currentmarker}{}%
\end{pgfscope}%
\end{pgfscope}%
\begin{pgfscope}%
\pgfsetbuttcap%
\pgfsetroundjoin%
\definecolor{currentfill}{rgb}{0.000000,0.000000,0.000000}%
\pgfsetfillcolor{currentfill}%
\pgfsetlinewidth{0.602250pt}%
\definecolor{currentstroke}{rgb}{0.000000,0.000000,0.000000}%
\pgfsetstrokecolor{currentstroke}%
\pgfsetdash{}{0pt}%
\pgfsys@defobject{currentmarker}{\pgfqpoint{-0.027778in}{0.000000in}}{\pgfqpoint{-0.000000in}{0.000000in}}{%
\pgfpathmoveto{\pgfqpoint{-0.000000in}{0.000000in}}%
\pgfpathlineto{\pgfqpoint{-0.027778in}{0.000000in}}%
\pgfusepath{stroke,fill}%
}%
\begin{pgfscope}%
\pgfsys@transformshift{0.588387in}{1.473244in}%
\pgfsys@useobject{currentmarker}{}%
\end{pgfscope}%
\end{pgfscope}%
\begin{pgfscope}%
\pgfsetbuttcap%
\pgfsetroundjoin%
\definecolor{currentfill}{rgb}{0.000000,0.000000,0.000000}%
\pgfsetfillcolor{currentfill}%
\pgfsetlinewidth{0.602250pt}%
\definecolor{currentstroke}{rgb}{0.000000,0.000000,0.000000}%
\pgfsetstrokecolor{currentstroke}%
\pgfsetdash{}{0pt}%
\pgfsys@defobject{currentmarker}{\pgfqpoint{-0.027778in}{0.000000in}}{\pgfqpoint{-0.000000in}{0.000000in}}{%
\pgfpathmoveto{\pgfqpoint{-0.000000in}{0.000000in}}%
\pgfpathlineto{\pgfqpoint{-0.027778in}{0.000000in}}%
\pgfusepath{stroke,fill}%
}%
\begin{pgfscope}%
\pgfsys@transformshift{0.588387in}{1.521579in}%
\pgfsys@useobject{currentmarker}{}%
\end{pgfscope}%
\end{pgfscope}%
\begin{pgfscope}%
\pgfsetbuttcap%
\pgfsetroundjoin%
\definecolor{currentfill}{rgb}{0.000000,0.000000,0.000000}%
\pgfsetfillcolor{currentfill}%
\pgfsetlinewidth{0.602250pt}%
\definecolor{currentstroke}{rgb}{0.000000,0.000000,0.000000}%
\pgfsetstrokecolor{currentstroke}%
\pgfsetdash{}{0pt}%
\pgfsys@defobject{currentmarker}{\pgfqpoint{-0.027778in}{0.000000in}}{\pgfqpoint{-0.000000in}{0.000000in}}{%
\pgfpathmoveto{\pgfqpoint{-0.000000in}{0.000000in}}%
\pgfpathlineto{\pgfqpoint{-0.027778in}{0.000000in}}%
\pgfusepath{stroke,fill}%
}%
\begin{pgfscope}%
\pgfsys@transformshift{0.588387in}{1.562446in}%
\pgfsys@useobject{currentmarker}{}%
\end{pgfscope}%
\end{pgfscope}%
\begin{pgfscope}%
\pgfsetbuttcap%
\pgfsetroundjoin%
\definecolor{currentfill}{rgb}{0.000000,0.000000,0.000000}%
\pgfsetfillcolor{currentfill}%
\pgfsetlinewidth{0.602250pt}%
\definecolor{currentstroke}{rgb}{0.000000,0.000000,0.000000}%
\pgfsetstrokecolor{currentstroke}%
\pgfsetdash{}{0pt}%
\pgfsys@defobject{currentmarker}{\pgfqpoint{-0.027778in}{0.000000in}}{\pgfqpoint{-0.000000in}{0.000000in}}{%
\pgfpathmoveto{\pgfqpoint{-0.000000in}{0.000000in}}%
\pgfpathlineto{\pgfqpoint{-0.027778in}{0.000000in}}%
\pgfusepath{stroke,fill}%
}%
\begin{pgfscope}%
\pgfsys@transformshift{0.588387in}{1.597847in}%
\pgfsys@useobject{currentmarker}{}%
\end{pgfscope}%
\end{pgfscope}%
\begin{pgfscope}%
\pgfsetbuttcap%
\pgfsetroundjoin%
\definecolor{currentfill}{rgb}{0.000000,0.000000,0.000000}%
\pgfsetfillcolor{currentfill}%
\pgfsetlinewidth{0.602250pt}%
\definecolor{currentstroke}{rgb}{0.000000,0.000000,0.000000}%
\pgfsetstrokecolor{currentstroke}%
\pgfsetdash{}{0pt}%
\pgfsys@defobject{currentmarker}{\pgfqpoint{-0.027778in}{0.000000in}}{\pgfqpoint{-0.000000in}{0.000000in}}{%
\pgfpathmoveto{\pgfqpoint{-0.000000in}{0.000000in}}%
\pgfpathlineto{\pgfqpoint{-0.027778in}{0.000000in}}%
\pgfusepath{stroke,fill}%
}%
\begin{pgfscope}%
\pgfsys@transformshift{0.588387in}{1.629072in}%
\pgfsys@useobject{currentmarker}{}%
\end{pgfscope}%
\end{pgfscope}%
\begin{pgfscope}%
\pgfsetbuttcap%
\pgfsetroundjoin%
\definecolor{currentfill}{rgb}{0.000000,0.000000,0.000000}%
\pgfsetfillcolor{currentfill}%
\pgfsetlinewidth{0.602250pt}%
\definecolor{currentstroke}{rgb}{0.000000,0.000000,0.000000}%
\pgfsetstrokecolor{currentstroke}%
\pgfsetdash{}{0pt}%
\pgfsys@defobject{currentmarker}{\pgfqpoint{-0.027778in}{0.000000in}}{\pgfqpoint{-0.000000in}{0.000000in}}{%
\pgfpathmoveto{\pgfqpoint{-0.000000in}{0.000000in}}%
\pgfpathlineto{\pgfqpoint{-0.027778in}{0.000000in}}%
\pgfusepath{stroke,fill}%
}%
\begin{pgfscope}%
\pgfsys@transformshift{0.588387in}{1.840765in}%
\pgfsys@useobject{currentmarker}{}%
\end{pgfscope}%
\end{pgfscope}%
\begin{pgfscope}%
\pgfsetbuttcap%
\pgfsetroundjoin%
\definecolor{currentfill}{rgb}{0.000000,0.000000,0.000000}%
\pgfsetfillcolor{currentfill}%
\pgfsetlinewidth{0.602250pt}%
\definecolor{currentstroke}{rgb}{0.000000,0.000000,0.000000}%
\pgfsetstrokecolor{currentstroke}%
\pgfsetdash{}{0pt}%
\pgfsys@defobject{currentmarker}{\pgfqpoint{-0.027778in}{0.000000in}}{\pgfqpoint{-0.000000in}{0.000000in}}{%
\pgfpathmoveto{\pgfqpoint{-0.000000in}{0.000000in}}%
\pgfpathlineto{\pgfqpoint{-0.027778in}{0.000000in}}%
\pgfusepath{stroke,fill}%
}%
\begin{pgfscope}%
\pgfsys@transformshift{0.588387in}{1.948258in}%
\pgfsys@useobject{currentmarker}{}%
\end{pgfscope}%
\end{pgfscope}%
\begin{pgfscope}%
\pgfsetbuttcap%
\pgfsetroundjoin%
\definecolor{currentfill}{rgb}{0.000000,0.000000,0.000000}%
\pgfsetfillcolor{currentfill}%
\pgfsetlinewidth{0.602250pt}%
\definecolor{currentstroke}{rgb}{0.000000,0.000000,0.000000}%
\pgfsetstrokecolor{currentstroke}%
\pgfsetdash{}{0pt}%
\pgfsys@defobject{currentmarker}{\pgfqpoint{-0.027778in}{0.000000in}}{\pgfqpoint{-0.000000in}{0.000000in}}{%
\pgfpathmoveto{\pgfqpoint{-0.000000in}{0.000000in}}%
\pgfpathlineto{\pgfqpoint{-0.027778in}{0.000000in}}%
\pgfusepath{stroke,fill}%
}%
\begin{pgfscope}%
\pgfsys@transformshift{0.588387in}{2.024526in}%
\pgfsys@useobject{currentmarker}{}%
\end{pgfscope}%
\end{pgfscope}%
\begin{pgfscope}%
\pgfsetbuttcap%
\pgfsetroundjoin%
\definecolor{currentfill}{rgb}{0.000000,0.000000,0.000000}%
\pgfsetfillcolor{currentfill}%
\pgfsetlinewidth{0.602250pt}%
\definecolor{currentstroke}{rgb}{0.000000,0.000000,0.000000}%
\pgfsetstrokecolor{currentstroke}%
\pgfsetdash{}{0pt}%
\pgfsys@defobject{currentmarker}{\pgfqpoint{-0.027778in}{0.000000in}}{\pgfqpoint{-0.000000in}{0.000000in}}{%
\pgfpathmoveto{\pgfqpoint{-0.000000in}{0.000000in}}%
\pgfpathlineto{\pgfqpoint{-0.027778in}{0.000000in}}%
\pgfusepath{stroke,fill}%
}%
\begin{pgfscope}%
\pgfsys@transformshift{0.588387in}{2.083683in}%
\pgfsys@useobject{currentmarker}{}%
\end{pgfscope}%
\end{pgfscope}%
\begin{pgfscope}%
\pgfsetbuttcap%
\pgfsetroundjoin%
\definecolor{currentfill}{rgb}{0.000000,0.000000,0.000000}%
\pgfsetfillcolor{currentfill}%
\pgfsetlinewidth{0.602250pt}%
\definecolor{currentstroke}{rgb}{0.000000,0.000000,0.000000}%
\pgfsetstrokecolor{currentstroke}%
\pgfsetdash{}{0pt}%
\pgfsys@defobject{currentmarker}{\pgfqpoint{-0.027778in}{0.000000in}}{\pgfqpoint{-0.000000in}{0.000000in}}{%
\pgfpathmoveto{\pgfqpoint{-0.000000in}{0.000000in}}%
\pgfpathlineto{\pgfqpoint{-0.027778in}{0.000000in}}%
\pgfusepath{stroke,fill}%
}%
\begin{pgfscope}%
\pgfsys@transformshift{0.588387in}{2.132019in}%
\pgfsys@useobject{currentmarker}{}%
\end{pgfscope}%
\end{pgfscope}%
\begin{pgfscope}%
\pgfsetbuttcap%
\pgfsetroundjoin%
\definecolor{currentfill}{rgb}{0.000000,0.000000,0.000000}%
\pgfsetfillcolor{currentfill}%
\pgfsetlinewidth{0.602250pt}%
\definecolor{currentstroke}{rgb}{0.000000,0.000000,0.000000}%
\pgfsetstrokecolor{currentstroke}%
\pgfsetdash{}{0pt}%
\pgfsys@defobject{currentmarker}{\pgfqpoint{-0.027778in}{0.000000in}}{\pgfqpoint{-0.000000in}{0.000000in}}{%
\pgfpathmoveto{\pgfqpoint{-0.000000in}{0.000000in}}%
\pgfpathlineto{\pgfqpoint{-0.027778in}{0.000000in}}%
\pgfusepath{stroke,fill}%
}%
\begin{pgfscope}%
\pgfsys@transformshift{0.588387in}{2.172886in}%
\pgfsys@useobject{currentmarker}{}%
\end{pgfscope}%
\end{pgfscope}%
\begin{pgfscope}%
\pgfsetbuttcap%
\pgfsetroundjoin%
\definecolor{currentfill}{rgb}{0.000000,0.000000,0.000000}%
\pgfsetfillcolor{currentfill}%
\pgfsetlinewidth{0.602250pt}%
\definecolor{currentstroke}{rgb}{0.000000,0.000000,0.000000}%
\pgfsetstrokecolor{currentstroke}%
\pgfsetdash{}{0pt}%
\pgfsys@defobject{currentmarker}{\pgfqpoint{-0.027778in}{0.000000in}}{\pgfqpoint{-0.000000in}{0.000000in}}{%
\pgfpathmoveto{\pgfqpoint{-0.000000in}{0.000000in}}%
\pgfpathlineto{\pgfqpoint{-0.027778in}{0.000000in}}%
\pgfusepath{stroke,fill}%
}%
\begin{pgfscope}%
\pgfsys@transformshift{0.588387in}{2.208286in}%
\pgfsys@useobject{currentmarker}{}%
\end{pgfscope}%
\end{pgfscope}%
\begin{pgfscope}%
\pgfsetbuttcap%
\pgfsetroundjoin%
\definecolor{currentfill}{rgb}{0.000000,0.000000,0.000000}%
\pgfsetfillcolor{currentfill}%
\pgfsetlinewidth{0.602250pt}%
\definecolor{currentstroke}{rgb}{0.000000,0.000000,0.000000}%
\pgfsetstrokecolor{currentstroke}%
\pgfsetdash{}{0pt}%
\pgfsys@defobject{currentmarker}{\pgfqpoint{-0.027778in}{0.000000in}}{\pgfqpoint{-0.000000in}{0.000000in}}{%
\pgfpathmoveto{\pgfqpoint{-0.000000in}{0.000000in}}%
\pgfpathlineto{\pgfqpoint{-0.027778in}{0.000000in}}%
\pgfusepath{stroke,fill}%
}%
\begin{pgfscope}%
\pgfsys@transformshift{0.588387in}{2.239512in}%
\pgfsys@useobject{currentmarker}{}%
\end{pgfscope}%
\end{pgfscope}%
\begin{pgfscope}%
\pgfsetbuttcap%
\pgfsetroundjoin%
\definecolor{currentfill}{rgb}{0.000000,0.000000,0.000000}%
\pgfsetfillcolor{currentfill}%
\pgfsetlinewidth{0.602250pt}%
\definecolor{currentstroke}{rgb}{0.000000,0.000000,0.000000}%
\pgfsetstrokecolor{currentstroke}%
\pgfsetdash{}{0pt}%
\pgfsys@defobject{currentmarker}{\pgfqpoint{-0.027778in}{0.000000in}}{\pgfqpoint{-0.000000in}{0.000000in}}{%
\pgfpathmoveto{\pgfqpoint{-0.000000in}{0.000000in}}%
\pgfpathlineto{\pgfqpoint{-0.027778in}{0.000000in}}%
\pgfusepath{stroke,fill}%
}%
\begin{pgfscope}%
\pgfsys@transformshift{0.588387in}{2.451205in}%
\pgfsys@useobject{currentmarker}{}%
\end{pgfscope}%
\end{pgfscope}%
\begin{pgfscope}%
\pgfsetbuttcap%
\pgfsetroundjoin%
\definecolor{currentfill}{rgb}{0.000000,0.000000,0.000000}%
\pgfsetfillcolor{currentfill}%
\pgfsetlinewidth{0.602250pt}%
\definecolor{currentstroke}{rgb}{0.000000,0.000000,0.000000}%
\pgfsetstrokecolor{currentstroke}%
\pgfsetdash{}{0pt}%
\pgfsys@defobject{currentmarker}{\pgfqpoint{-0.027778in}{0.000000in}}{\pgfqpoint{-0.000000in}{0.000000in}}{%
\pgfpathmoveto{\pgfqpoint{-0.000000in}{0.000000in}}%
\pgfpathlineto{\pgfqpoint{-0.027778in}{0.000000in}}%
\pgfusepath{stroke,fill}%
}%
\begin{pgfscope}%
\pgfsys@transformshift{0.588387in}{2.558698in}%
\pgfsys@useobject{currentmarker}{}%
\end{pgfscope}%
\end{pgfscope}%
\begin{pgfscope}%
\pgfsetbuttcap%
\pgfsetroundjoin%
\definecolor{currentfill}{rgb}{0.000000,0.000000,0.000000}%
\pgfsetfillcolor{currentfill}%
\pgfsetlinewidth{0.602250pt}%
\definecolor{currentstroke}{rgb}{0.000000,0.000000,0.000000}%
\pgfsetstrokecolor{currentstroke}%
\pgfsetdash{}{0pt}%
\pgfsys@defobject{currentmarker}{\pgfqpoint{-0.027778in}{0.000000in}}{\pgfqpoint{-0.000000in}{0.000000in}}{%
\pgfpathmoveto{\pgfqpoint{-0.000000in}{0.000000in}}%
\pgfpathlineto{\pgfqpoint{-0.027778in}{0.000000in}}%
\pgfusepath{stroke,fill}%
}%
\begin{pgfscope}%
\pgfsys@transformshift{0.588387in}{2.634965in}%
\pgfsys@useobject{currentmarker}{}%
\end{pgfscope}%
\end{pgfscope}%
\begin{pgfscope}%
\pgfsetbuttcap%
\pgfsetroundjoin%
\definecolor{currentfill}{rgb}{0.000000,0.000000,0.000000}%
\pgfsetfillcolor{currentfill}%
\pgfsetlinewidth{0.602250pt}%
\definecolor{currentstroke}{rgb}{0.000000,0.000000,0.000000}%
\pgfsetstrokecolor{currentstroke}%
\pgfsetdash{}{0pt}%
\pgfsys@defobject{currentmarker}{\pgfqpoint{-0.027778in}{0.000000in}}{\pgfqpoint{-0.000000in}{0.000000in}}{%
\pgfpathmoveto{\pgfqpoint{-0.000000in}{0.000000in}}%
\pgfpathlineto{\pgfqpoint{-0.027778in}{0.000000in}}%
\pgfusepath{stroke,fill}%
}%
\begin{pgfscope}%
\pgfsys@transformshift{0.588387in}{2.694123in}%
\pgfsys@useobject{currentmarker}{}%
\end{pgfscope}%
\end{pgfscope}%
\begin{pgfscope}%
\definecolor{textcolor}{rgb}{0.000000,0.000000,0.000000}%
\pgfsetstrokecolor{textcolor}%
\pgfsetfillcolor{textcolor}%
\pgftext[x=0.234413in,y=1.631726in,,bottom,rotate=90.000000]{\color{textcolor}{\rmfamily\fontsize{10.000000}{12.000000}\selectfont\catcode`\^=\active\def^{\ifmmode\sp\else\^{}\fi}\catcode`\%=\active\def%{\%}Time [ms]}}%
\end{pgfscope}%
\begin{pgfscope}%
\pgfpathrectangle{\pgfqpoint{0.588387in}{0.521603in}}{\pgfqpoint{4.669024in}{2.220246in}}%
\pgfusepath{clip}%
\pgfsetrectcap%
\pgfsetroundjoin%
\pgfsetlinewidth{1.505625pt}%
\pgfsetstrokecolor{currentstroke1}%
\pgfsetdash{}{0pt}%
\pgfpathmoveto{\pgfqpoint{0.800616in}{0.817980in}}%
\pgfpathlineto{\pgfqpoint{0.841825in}{0.841584in}}%
\pgfpathlineto{\pgfqpoint{0.883034in}{0.844513in}}%
\pgfpathlineto{\pgfqpoint{0.924244in}{0.780579in}}%
\pgfpathlineto{\pgfqpoint{0.965453in}{0.705357in}}%
\pgfpathlineto{\pgfqpoint{1.006663in}{0.698510in}}%
\pgfpathlineto{\pgfqpoint{1.047872in}{0.753264in}}%
\pgfpathlineto{\pgfqpoint{1.089081in}{0.779787in}}%
\pgfpathlineto{\pgfqpoint{1.130291in}{0.850277in}}%
\pgfpathlineto{\pgfqpoint{1.171500in}{0.845752in}}%
\pgfpathlineto{\pgfqpoint{1.212709in}{0.914392in}}%
\pgfpathlineto{\pgfqpoint{1.253919in}{0.931284in}}%
\pgfpathlineto{\pgfqpoint{1.295128in}{0.975929in}}%
\pgfpathlineto{\pgfqpoint{1.336338in}{0.993774in}}%
\pgfpathlineto{\pgfqpoint{1.377547in}{1.027066in}}%
\pgfpathlineto{\pgfqpoint{1.418756in}{1.045857in}}%
\pgfpathlineto{\pgfqpoint{1.459966in}{1.077790in}}%
\pgfpathlineto{\pgfqpoint{1.501175in}{1.097857in}}%
\pgfpathlineto{\pgfqpoint{1.542385in}{1.119419in}}%
\pgfpathlineto{\pgfqpoint{1.583594in}{1.134885in}}%
\pgfpathlineto{\pgfqpoint{1.624803in}{1.170061in}}%
\pgfpathlineto{\pgfqpoint{1.666013in}{1.180603in}}%
\pgfpathlineto{\pgfqpoint{1.707222in}{1.199617in}}%
\pgfpathlineto{\pgfqpoint{1.748432in}{1.220505in}}%
\pgfpathlineto{\pgfqpoint{1.789641in}{1.244441in}}%
\pgfpathlineto{\pgfqpoint{1.830850in}{1.253393in}}%
\pgfpathlineto{\pgfqpoint{1.872060in}{1.280163in}}%
\pgfpathlineto{\pgfqpoint{1.913269in}{1.283477in}}%
\pgfpathlineto{\pgfqpoint{1.954479in}{1.295166in}}%
\pgfpathlineto{\pgfqpoint{1.995688in}{1.324311in}}%
\pgfpathlineto{\pgfqpoint{2.036897in}{1.326535in}}%
\pgfpathlineto{\pgfqpoint{2.078107in}{1.344502in}}%
\pgfpathlineto{\pgfqpoint{2.119316in}{1.359039in}}%
\pgfpathlineto{\pgfqpoint{2.160525in}{1.368288in}}%
\pgfpathlineto{\pgfqpoint{2.201735in}{1.373954in}}%
\pgfpathlineto{\pgfqpoint{2.242944in}{1.404422in}}%
\pgfpathlineto{\pgfqpoint{2.284154in}{1.448823in}}%
\pgfpathlineto{\pgfqpoint{2.325363in}{1.418002in}}%
\pgfpathlineto{\pgfqpoint{2.366572in}{1.434138in}}%
\pgfpathlineto{\pgfqpoint{2.407782in}{1.450857in}}%
\pgfpathlineto{\pgfqpoint{2.448991in}{1.454263in}}%
\pgfpathlineto{\pgfqpoint{2.490201in}{1.471636in}}%
\pgfpathlineto{\pgfqpoint{2.531410in}{1.494260in}}%
\pgfpathlineto{\pgfqpoint{2.572619in}{1.489291in}}%
\pgfpathlineto{\pgfqpoint{2.613829in}{1.492132in}}%
\pgfpathlineto{\pgfqpoint{2.655038in}{1.517123in}}%
\pgfpathlineto{\pgfqpoint{2.696248in}{1.522021in}}%
\pgfpathlineto{\pgfqpoint{2.737457in}{1.535098in}}%
\pgfpathlineto{\pgfqpoint{2.778666in}{1.546272in}}%
\pgfpathlineto{\pgfqpoint{2.819876in}{1.559641in}}%
\pgfpathlineto{\pgfqpoint{2.861085in}{1.561090in}}%
\pgfpathlineto{\pgfqpoint{2.902295in}{1.568210in}}%
\pgfpathlineto{\pgfqpoint{2.943504in}{1.633164in}}%
\pgfpathlineto{\pgfqpoint{2.984713in}{1.588909in}}%
\pgfpathlineto{\pgfqpoint{3.025923in}{1.585987in}}%
\pgfpathlineto{\pgfqpoint{3.067132in}{1.624278in}}%
\pgfpathlineto{\pgfqpoint{3.108341in}{1.610781in}}%
\pgfpathlineto{\pgfqpoint{3.149551in}{1.648334in}}%
\pgfpathlineto{\pgfqpoint{3.190760in}{1.648587in}}%
\pgfpathlineto{\pgfqpoint{3.231970in}{1.631541in}}%
\pgfpathlineto{\pgfqpoint{3.273179in}{1.656474in}}%
\pgfpathlineto{\pgfqpoint{3.314388in}{1.642007in}}%
\pgfpathlineto{\pgfqpoint{3.355598in}{1.689406in}}%
\pgfpathlineto{\pgfqpoint{3.396807in}{1.674114in}}%
\pgfpathlineto{\pgfqpoint{3.438017in}{1.681064in}}%
\pgfpathlineto{\pgfqpoint{3.479226in}{1.689991in}}%
\pgfpathlineto{\pgfqpoint{3.520435in}{1.694057in}}%
\pgfpathlineto{\pgfqpoint{3.561645in}{1.700680in}}%
\pgfpathlineto{\pgfqpoint{3.602854in}{1.701632in}}%
\pgfpathlineto{\pgfqpoint{3.644064in}{1.719534in}}%
\pgfpathlineto{\pgfqpoint{3.685273in}{1.714033in}}%
\pgfpathlineto{\pgfqpoint{3.726482in}{1.735647in}}%
\pgfpathlineto{\pgfqpoint{3.767692in}{1.741430in}}%
\pgfpathlineto{\pgfqpoint{3.808901in}{1.748563in}}%
\pgfpathlineto{\pgfqpoint{3.891320in}{1.768931in}}%
\pgfpathlineto{\pgfqpoint{3.932529in}{1.756240in}}%
\pgfpathlineto{\pgfqpoint{3.973739in}{1.795330in}}%
\pgfpathlineto{\pgfqpoint{4.056157in}{1.772256in}}%
\pgfpathlineto{\pgfqpoint{4.097367in}{1.838101in}}%
\pgfpathlineto{\pgfqpoint{4.138576in}{1.804150in}}%
\pgfpathlineto{\pgfqpoint{4.179786in}{1.787344in}}%
\pgfpathlineto{\pgfqpoint{4.220995in}{1.805830in}}%
\pgfpathlineto{\pgfqpoint{4.262204in}{1.858702in}}%
\pgfpathlineto{\pgfqpoint{4.303414in}{1.824361in}}%
\pgfpathlineto{\pgfqpoint{4.385833in}{1.842464in}}%
\pgfpathlineto{\pgfqpoint{4.427042in}{1.863307in}}%
\pgfpathlineto{\pgfqpoint{4.468251in}{1.843028in}}%
\pgfpathlineto{\pgfqpoint{4.509461in}{1.864219in}}%
\pgfpathlineto{\pgfqpoint{4.550670in}{1.847957in}}%
\pgfpathlineto{\pgfqpoint{4.591880in}{1.845364in}}%
\pgfpathlineto{\pgfqpoint{4.633089in}{1.880493in}}%
\pgfpathlineto{\pgfqpoint{4.715508in}{1.880113in}}%
\pgfpathlineto{\pgfqpoint{4.756717in}{1.949580in}}%
\pgfpathlineto{\pgfqpoint{4.797926in}{1.909810in}}%
\pgfpathlineto{\pgfqpoint{4.880345in}{1.916369in}}%
\pgfpathlineto{\pgfqpoint{4.962764in}{1.930912in}}%
\pgfpathlineto{\pgfqpoint{5.003973in}{1.893482in}}%
\pgfpathlineto{\pgfqpoint{5.045183in}{1.912352in}}%
\pgfusepath{stroke}%
\end{pgfscope}%
\begin{pgfscope}%
\pgfpathrectangle{\pgfqpoint{0.588387in}{0.521603in}}{\pgfqpoint{4.669024in}{2.220246in}}%
\pgfusepath{clip}%
\pgfsetrectcap%
\pgfsetroundjoin%
\pgfsetlinewidth{1.505625pt}%
\pgfsetstrokecolor{currentstroke2}%
\pgfsetdash{}{0pt}%
\pgfpathmoveto{\pgfqpoint{0.800616in}{0.817980in}}%
\pgfpathlineto{\pgfqpoint{0.841825in}{0.832596in}}%
\pgfpathlineto{\pgfqpoint{0.883034in}{0.833465in}}%
\pgfpathlineto{\pgfqpoint{0.924244in}{0.764780in}}%
\pgfpathlineto{\pgfqpoint{0.965453in}{0.709499in}}%
\pgfpathlineto{\pgfqpoint{1.006663in}{0.696153in}}%
\pgfpathlineto{\pgfqpoint{1.047872in}{0.756937in}}%
\pgfpathlineto{\pgfqpoint{1.089081in}{0.783330in}}%
\pgfpathlineto{\pgfqpoint{1.130291in}{0.838025in}}%
\pgfpathlineto{\pgfqpoint{1.171500in}{0.847600in}}%
\pgfpathlineto{\pgfqpoint{1.212709in}{0.990458in}}%
\pgfpathlineto{\pgfqpoint{1.253919in}{0.988278in}}%
\pgfpathlineto{\pgfqpoint{1.295128in}{1.033675in}}%
\pgfpathlineto{\pgfqpoint{1.336338in}{1.083617in}}%
\pgfpathlineto{\pgfqpoint{1.377547in}{1.193959in}}%
\pgfpathlineto{\pgfqpoint{1.418756in}{1.172929in}}%
\pgfpathlineto{\pgfqpoint{1.459966in}{1.280038in}}%
\pgfpathlineto{\pgfqpoint{1.501175in}{1.384734in}}%
\pgfpathlineto{\pgfqpoint{1.542385in}{1.559256in}}%
\pgfpathlineto{\pgfqpoint{1.583594in}{1.454865in}}%
\pgfpathlineto{\pgfqpoint{1.624803in}{1.525671in}}%
\pgfpathlineto{\pgfqpoint{1.666013in}{1.604955in}}%
\pgfpathlineto{\pgfqpoint{1.707222in}{1.412370in}}%
\pgfpathlineto{\pgfqpoint{1.748432in}{1.684046in}}%
\pgfpathlineto{\pgfqpoint{1.789641in}{1.760042in}}%
\pgfpathlineto{\pgfqpoint{1.830850in}{1.977512in}}%
\pgfpathlineto{\pgfqpoint{1.872060in}{2.211966in}}%
\pgfpathlineto{\pgfqpoint{1.913269in}{2.262261in}}%
\pgfpathlineto{\pgfqpoint{1.954479in}{2.188021in}}%
\pgfpathlineto{\pgfqpoint{1.995688in}{2.081864in}}%
\pgfpathlineto{\pgfqpoint{2.036897in}{2.250369in}}%
\pgfpathlineto{\pgfqpoint{2.078107in}{2.111147in}}%
\pgfpathlineto{\pgfqpoint{2.119316in}{2.261671in}}%
\pgfpathlineto{\pgfqpoint{2.160525in}{2.175812in}}%
\pgfpathlineto{\pgfqpoint{2.201735in}{2.329405in}}%
\pgfpathlineto{\pgfqpoint{2.242944in}{2.309862in}}%
\pgfpathlineto{\pgfqpoint{2.284154in}{2.275212in}}%
\pgfpathlineto{\pgfqpoint{2.325363in}{2.255752in}}%
\pgfpathlineto{\pgfqpoint{2.366572in}{2.058431in}}%
\pgfpathlineto{\pgfqpoint{2.407782in}{2.350516in}}%
\pgfpathlineto{\pgfqpoint{2.448991in}{1.508403in}}%
\pgfpathlineto{\pgfqpoint{2.490201in}{2.415728in}}%
\pgfpathlineto{\pgfqpoint{2.531410in}{2.487203in}}%
\pgfpathlineto{\pgfqpoint{2.572619in}{2.465524in}}%
\pgfpathlineto{\pgfqpoint{2.613829in}{2.286313in}}%
\pgfpathlineto{\pgfqpoint{2.655038in}{2.281864in}}%
\pgfpathlineto{\pgfqpoint{2.696248in}{2.383955in}}%
\pgfpathlineto{\pgfqpoint{2.737457in}{2.416724in}}%
\pgfpathlineto{\pgfqpoint{2.778666in}{1.820710in}}%
\pgfpathlineto{\pgfqpoint{2.819876in}{2.257724in}}%
\pgfpathlineto{\pgfqpoint{2.861085in}{2.052576in}}%
\pgfpathlineto{\pgfqpoint{2.902295in}{2.275404in}}%
\pgfpathlineto{\pgfqpoint{2.943504in}{1.868193in}}%
\pgfpathlineto{\pgfqpoint{2.984713in}{2.365233in}}%
\pgfpathlineto{\pgfqpoint{3.025923in}{1.794147in}}%
\pgfpathlineto{\pgfqpoint{3.067132in}{2.461183in}}%
\pgfpathlineto{\pgfqpoint{3.108341in}{2.415417in}}%
\pgfpathlineto{\pgfqpoint{3.149551in}{2.393677in}}%
\pgfpathlineto{\pgfqpoint{3.190760in}{2.194382in}}%
\pgfpathlineto{\pgfqpoint{3.231970in}{2.383468in}}%
\pgfpathlineto{\pgfqpoint{3.273179in}{2.235714in}}%
\pgfpathlineto{\pgfqpoint{3.314388in}{2.336639in}}%
\pgfpathlineto{\pgfqpoint{3.355598in}{1.728591in}}%
\pgfpathlineto{\pgfqpoint{3.396807in}{2.399121in}}%
\pgfpathlineto{\pgfqpoint{3.438017in}{1.748094in}}%
\pgfpathlineto{\pgfqpoint{3.479226in}{2.259076in}}%
\pgfpathlineto{\pgfqpoint{3.520435in}{2.516753in}}%
\pgfpathlineto{\pgfqpoint{3.561645in}{2.468375in}}%
\pgfpathlineto{\pgfqpoint{3.602854in}{2.591412in}}%
\pgfpathlineto{\pgfqpoint{3.644064in}{2.311099in}}%
\pgfpathlineto{\pgfqpoint{3.685273in}{2.590342in}}%
\pgfpathlineto{\pgfqpoint{3.726482in}{2.466452in}}%
\pgfpathlineto{\pgfqpoint{3.767692in}{2.010927in}}%
\pgfpathlineto{\pgfqpoint{3.808901in}{1.859218in}}%
\pgfpathlineto{\pgfqpoint{3.891320in}{2.358931in}}%
\pgfpathlineto{\pgfqpoint{3.932529in}{2.326771in}}%
\pgfpathlineto{\pgfqpoint{3.973739in}{2.483850in}}%
\pgfpathlineto{\pgfqpoint{4.056157in}{2.475055in}}%
\pgfpathlineto{\pgfqpoint{4.097367in}{1.867235in}}%
\pgfpathlineto{\pgfqpoint{4.138576in}{2.482666in}}%
\pgfpathlineto{\pgfqpoint{4.179786in}{2.538507in}}%
\pgfpathlineto{\pgfqpoint{4.220995in}{2.530909in}}%
\pgfpathlineto{\pgfqpoint{4.303414in}{2.395994in}}%
\pgfpathlineto{\pgfqpoint{4.385833in}{2.532641in}}%
\pgfpathlineto{\pgfqpoint{4.427042in}{2.621419in}}%
\pgfpathlineto{\pgfqpoint{4.468251in}{2.554870in}}%
\pgfpathlineto{\pgfqpoint{4.509461in}{2.521473in}}%
\pgfpathlineto{\pgfqpoint{4.550670in}{2.382215in}}%
\pgfpathlineto{\pgfqpoint{4.591880in}{2.522235in}}%
\pgfpathlineto{\pgfqpoint{4.633089in}{2.373263in}}%
\pgfpathlineto{\pgfqpoint{4.715508in}{2.530945in}}%
\pgfpathlineto{\pgfqpoint{4.756717in}{2.524858in}}%
\pgfpathlineto{\pgfqpoint{4.797926in}{1.953074in}}%
\pgfpathlineto{\pgfqpoint{4.880345in}{2.076881in}}%
\pgfpathlineto{\pgfqpoint{4.962764in}{1.959504in}}%
\pgfusepath{stroke}%
\end{pgfscope}%
\begin{pgfscope}%
\pgfpathrectangle{\pgfqpoint{0.588387in}{0.521603in}}{\pgfqpoint{4.669024in}{2.220246in}}%
\pgfusepath{clip}%
\pgfsetrectcap%
\pgfsetroundjoin%
\pgfsetlinewidth{1.505625pt}%
\pgfsetstrokecolor{currentstroke3}%
\pgfsetdash{}{0pt}%
\pgfpathmoveto{\pgfqpoint{0.800616in}{0.817980in}}%
\pgfpathlineto{\pgfqpoint{0.841825in}{0.837128in}}%
\pgfpathlineto{\pgfqpoint{0.883034in}{0.823791in}}%
\pgfpathlineto{\pgfqpoint{0.924244in}{0.773140in}}%
\pgfpathlineto{\pgfqpoint{0.965453in}{0.782978in}}%
\pgfpathlineto{\pgfqpoint{1.006663in}{0.709815in}}%
\pgfpathlineto{\pgfqpoint{1.047872in}{0.622524in}}%
\pgfpathlineto{\pgfqpoint{1.089081in}{0.651737in}}%
\pgfpathlineto{\pgfqpoint{1.130291in}{0.645153in}}%
\pgfpathlineto{\pgfqpoint{1.171500in}{0.644758in}}%
\pgfpathlineto{\pgfqpoint{1.212709in}{0.666991in}}%
\pgfpathlineto{\pgfqpoint{1.253919in}{0.669092in}}%
\pgfpathlineto{\pgfqpoint{1.295128in}{0.736217in}}%
\pgfpathlineto{\pgfqpoint{1.336338in}{0.706976in}}%
\pgfpathlineto{\pgfqpoint{1.377547in}{0.736848in}}%
\pgfpathlineto{\pgfqpoint{1.418756in}{0.745591in}}%
\pgfpathlineto{\pgfqpoint{1.459966in}{0.774924in}}%
\pgfpathlineto{\pgfqpoint{1.501175in}{0.778251in}}%
\pgfpathlineto{\pgfqpoint{1.542385in}{0.812737in}}%
\pgfpathlineto{\pgfqpoint{1.583594in}{0.799765in}}%
\pgfpathlineto{\pgfqpoint{1.624803in}{0.834872in}}%
\pgfpathlineto{\pgfqpoint{1.666013in}{0.818586in}}%
\pgfpathlineto{\pgfqpoint{1.707222in}{0.852671in}}%
\pgfpathlineto{\pgfqpoint{1.748432in}{0.842786in}}%
\pgfpathlineto{\pgfqpoint{1.789641in}{0.866097in}}%
\pgfpathlineto{\pgfqpoint{1.830850in}{0.866395in}}%
\pgfpathlineto{\pgfqpoint{1.872060in}{0.883653in}}%
\pgfpathlineto{\pgfqpoint{1.913269in}{0.934796in}}%
\pgfpathlineto{\pgfqpoint{1.954479in}{0.905558in}}%
\pgfpathlineto{\pgfqpoint{1.995688in}{0.900534in}}%
\pgfpathlineto{\pgfqpoint{2.036897in}{0.961309in}}%
\pgfpathlineto{\pgfqpoint{2.078107in}{0.927543in}}%
\pgfpathlineto{\pgfqpoint{2.119316in}{0.955144in}}%
\pgfpathlineto{\pgfqpoint{2.160525in}{0.953459in}}%
\pgfpathlineto{\pgfqpoint{2.201735in}{0.992099in}}%
\pgfpathlineto{\pgfqpoint{2.242944in}{0.952866in}}%
\pgfpathlineto{\pgfqpoint{2.284154in}{0.957009in}}%
\pgfpathlineto{\pgfqpoint{2.325363in}{0.966909in}}%
\pgfpathlineto{\pgfqpoint{2.366572in}{0.996710in}}%
\pgfpathlineto{\pgfqpoint{2.407782in}{0.976686in}}%
\pgfpathlineto{\pgfqpoint{2.448991in}{0.984377in}}%
\pgfpathlineto{\pgfqpoint{2.490201in}{0.983525in}}%
\pgfpathlineto{\pgfqpoint{2.531410in}{1.018632in}}%
\pgfpathlineto{\pgfqpoint{2.572619in}{0.993953in}}%
\pgfpathlineto{\pgfqpoint{2.613829in}{1.016357in}}%
\pgfpathlineto{\pgfqpoint{2.655038in}{1.017681in}}%
\pgfpathlineto{\pgfqpoint{2.696248in}{1.259184in}}%
\pgfpathlineto{\pgfqpoint{2.737457in}{1.016166in}}%
\pgfpathlineto{\pgfqpoint{2.778666in}{1.034952in}}%
\pgfpathlineto{\pgfqpoint{2.819876in}{1.027775in}}%
\pgfpathlineto{\pgfqpoint{2.861085in}{1.055868in}}%
\pgfpathlineto{\pgfqpoint{2.902295in}{1.031891in}}%
\pgfpathlineto{\pgfqpoint{2.943504in}{1.066968in}}%
\pgfpathlineto{\pgfqpoint{2.984713in}{1.050000in}}%
\pgfpathlineto{\pgfqpoint{3.025923in}{1.071832in}}%
\pgfpathlineto{\pgfqpoint{3.067132in}{1.056633in}}%
\pgfpathlineto{\pgfqpoint{3.108341in}{1.084895in}}%
\pgfpathlineto{\pgfqpoint{3.149551in}{1.075991in}}%
\pgfpathlineto{\pgfqpoint{3.190760in}{1.162751in}}%
\pgfpathlineto{\pgfqpoint{3.231970in}{1.089022in}}%
\pgfpathlineto{\pgfqpoint{3.273179in}{1.103593in}}%
\pgfpathlineto{\pgfqpoint{3.314388in}{1.120614in}}%
\pgfpathlineto{\pgfqpoint{3.355598in}{1.145070in}}%
\pgfpathlineto{\pgfqpoint{3.396807in}{1.107835in}}%
\pgfpathlineto{\pgfqpoint{3.438017in}{1.116120in}}%
\pgfpathlineto{\pgfqpoint{3.479226in}{1.948258in}}%
\pgfpathlineto{\pgfqpoint{3.520435in}{1.126126in}}%
\pgfpathlineto{\pgfqpoint{3.561645in}{1.104757in}}%
\pgfpathlineto{\pgfqpoint{3.602854in}{1.129379in}}%
\pgfpathlineto{\pgfqpoint{3.644064in}{1.220422in}}%
\pgfpathlineto{\pgfqpoint{3.685273in}{1.234707in}}%
\pgfpathlineto{\pgfqpoint{3.726482in}{1.146373in}}%
\pgfpathlineto{\pgfqpoint{3.767692in}{1.158440in}}%
\pgfpathlineto{\pgfqpoint{3.808901in}{1.116120in}}%
\pgfpathlineto{\pgfqpoint{3.891320in}{1.159206in}}%
\pgfpathlineto{\pgfqpoint{3.932529in}{1.157569in}}%
\pgfpathlineto{\pgfqpoint{3.973739in}{1.171168in}}%
\pgfpathlineto{\pgfqpoint{4.056157in}{1.185818in}}%
\pgfpathlineto{\pgfqpoint{4.097367in}{1.187240in}}%
\pgfpathlineto{\pgfqpoint{4.138576in}{1.173368in}}%
\pgfpathlineto{\pgfqpoint{4.179786in}{1.154058in}}%
\pgfpathlineto{\pgfqpoint{4.220995in}{1.164717in}}%
\pgfpathlineto{\pgfqpoint{4.262204in}{1.202393in}}%
\pgfpathlineto{\pgfqpoint{4.303414in}{1.183312in}}%
\pgfpathlineto{\pgfqpoint{4.385833in}{1.220684in}}%
\pgfpathlineto{\pgfqpoint{4.427042in}{1.202393in}}%
\pgfpathlineto{\pgfqpoint{4.468251in}{1.224583in}}%
\pgfpathlineto{\pgfqpoint{4.509461in}{1.243260in}}%
\pgfpathlineto{\pgfqpoint{4.550670in}{1.230325in}}%
\pgfpathlineto{\pgfqpoint{4.591880in}{1.230325in}}%
\pgfpathlineto{\pgfqpoint{4.633089in}{1.357676in}}%
\pgfpathlineto{\pgfqpoint{4.715508in}{1.194925in}}%
\pgfpathlineto{\pgfqpoint{4.756717in}{1.261551in}}%
\pgfpathlineto{\pgfqpoint{4.797926in}{1.230325in}}%
\pgfpathlineto{\pgfqpoint{4.880345in}{1.216727in}}%
\pgfpathlineto{\pgfqpoint{4.962764in}{1.216727in}}%
\pgfpathlineto{\pgfqpoint{5.003973in}{1.216727in}}%
\pgfpathlineto{\pgfqpoint{5.045183in}{1.255593in}}%
\pgfusepath{stroke}%
\end{pgfscope}%
\begin{pgfscope}%
\pgfpathrectangle{\pgfqpoint{0.588387in}{0.521603in}}{\pgfqpoint{4.669024in}{2.220246in}}%
\pgfusepath{clip}%
\pgfsetrectcap%
\pgfsetroundjoin%
\pgfsetlinewidth{1.505625pt}%
\pgfsetstrokecolor{currentstroke4}%
\pgfsetdash{}{0pt}%
\pgfpathmoveto{\pgfqpoint{0.800616in}{0.824866in}}%
\pgfpathlineto{\pgfqpoint{0.841825in}{0.837128in}}%
\pgfpathlineto{\pgfqpoint{0.883034in}{0.833465in}}%
\pgfpathlineto{\pgfqpoint{0.924244in}{0.749181in}}%
\pgfpathlineto{\pgfqpoint{0.965453in}{0.719581in}}%
\pgfpathlineto{\pgfqpoint{1.006663in}{0.696153in}}%
\pgfpathlineto{\pgfqpoint{1.047872in}{0.745760in}}%
\pgfpathlineto{\pgfqpoint{1.089081in}{0.767624in}}%
\pgfpathlineto{\pgfqpoint{1.130291in}{0.830762in}}%
\pgfpathlineto{\pgfqpoint{1.171500in}{0.832446in}}%
\pgfpathlineto{\pgfqpoint{1.212709in}{0.899429in}}%
\pgfpathlineto{\pgfqpoint{1.253919in}{0.914605in}}%
\pgfpathlineto{\pgfqpoint{1.295128in}{0.962700in}}%
\pgfpathlineto{\pgfqpoint{1.336338in}{0.977561in}}%
\pgfpathlineto{\pgfqpoint{1.377547in}{1.015399in}}%
\pgfpathlineto{\pgfqpoint{1.418756in}{1.034635in}}%
\pgfpathlineto{\pgfqpoint{1.459966in}{1.059499in}}%
\pgfpathlineto{\pgfqpoint{1.501175in}{1.081167in}}%
\pgfpathlineto{\pgfqpoint{1.542385in}{1.116594in}}%
\pgfpathlineto{\pgfqpoint{1.583594in}{1.124752in}}%
\pgfpathlineto{\pgfqpoint{1.624803in}{1.152877in}}%
\pgfpathlineto{\pgfqpoint{1.666013in}{1.165831in}}%
\pgfpathlineto{\pgfqpoint{1.707222in}{1.193976in}}%
\pgfpathlineto{\pgfqpoint{1.748432in}{1.204615in}}%
\pgfpathlineto{\pgfqpoint{1.789641in}{1.227410in}}%
\pgfpathlineto{\pgfqpoint{1.830850in}{1.273335in}}%
\pgfpathlineto{\pgfqpoint{1.872060in}{1.281656in}}%
\pgfpathlineto{\pgfqpoint{1.913269in}{1.283260in}}%
\pgfpathlineto{\pgfqpoint{1.954479in}{1.290805in}}%
\pgfpathlineto{\pgfqpoint{1.995688in}{1.304833in}}%
\pgfpathlineto{\pgfqpoint{2.036897in}{1.324994in}}%
\pgfpathlineto{\pgfqpoint{2.078107in}{1.362915in}}%
\pgfpathlineto{\pgfqpoint{2.119316in}{1.352152in}}%
\pgfpathlineto{\pgfqpoint{2.160525in}{1.355247in}}%
\pgfpathlineto{\pgfqpoint{2.201735in}{1.373587in}}%
\pgfpathlineto{\pgfqpoint{2.242944in}{1.390140in}}%
\pgfpathlineto{\pgfqpoint{2.284154in}{1.397212in}}%
\pgfpathlineto{\pgfqpoint{2.325363in}{1.411230in}}%
\pgfpathlineto{\pgfqpoint{2.366572in}{1.425665in}}%
\pgfpathlineto{\pgfqpoint{2.407782in}{1.436992in}}%
\pgfpathlineto{\pgfqpoint{2.448991in}{1.449034in}}%
\pgfpathlineto{\pgfqpoint{2.490201in}{1.459840in}}%
\pgfpathlineto{\pgfqpoint{2.531410in}{1.479143in}}%
\pgfpathlineto{\pgfqpoint{2.572619in}{1.485065in}}%
\pgfpathlineto{\pgfqpoint{2.613829in}{1.490990in}}%
\pgfpathlineto{\pgfqpoint{2.655038in}{1.515232in}}%
\pgfpathlineto{\pgfqpoint{2.696248in}{1.518020in}}%
\pgfpathlineto{\pgfqpoint{2.737457in}{1.535447in}}%
\pgfpathlineto{\pgfqpoint{2.778666in}{1.582247in}}%
\pgfpathlineto{\pgfqpoint{2.819876in}{1.549790in}}%
\pgfpathlineto{\pgfqpoint{2.861085in}{1.555039in}}%
\pgfpathlineto{\pgfqpoint{2.902295in}{1.559222in}}%
\pgfpathlineto{\pgfqpoint{2.943504in}{1.591813in}}%
\pgfpathlineto{\pgfqpoint{2.984713in}{1.581810in}}%
\pgfpathlineto{\pgfqpoint{3.025923in}{1.580737in}}%
\pgfpathlineto{\pgfqpoint{3.067132in}{1.594944in}}%
\pgfpathlineto{\pgfqpoint{3.108341in}{1.598582in}}%
\pgfpathlineto{\pgfqpoint{3.149551in}{1.617555in}}%
\pgfpathlineto{\pgfqpoint{3.190760in}{1.623866in}}%
\pgfpathlineto{\pgfqpoint{3.231970in}{1.640897in}}%
\pgfpathlineto{\pgfqpoint{3.273179in}{1.634899in}}%
\pgfpathlineto{\pgfqpoint{3.314388in}{1.636495in}}%
\pgfpathlineto{\pgfqpoint{3.355598in}{1.671199in}}%
\pgfpathlineto{\pgfqpoint{3.396807in}{1.656474in}}%
\pgfpathlineto{\pgfqpoint{3.438017in}{1.657004in}}%
\pgfpathlineto{\pgfqpoint{3.479226in}{1.678326in}}%
\pgfpathlineto{\pgfqpoint{3.520435in}{1.688230in}}%
\pgfpathlineto{\pgfqpoint{3.561645in}{1.701088in}}%
\pgfpathlineto{\pgfqpoint{3.602854in}{1.696352in}}%
\pgfpathlineto{\pgfqpoint{3.644064in}{1.696961in}}%
\pgfpathlineto{\pgfqpoint{3.685273in}{1.693287in}}%
\pgfpathlineto{\pgfqpoint{3.726482in}{1.720787in}}%
\pgfpathlineto{\pgfqpoint{3.767692in}{1.722967in}}%
\pgfpathlineto{\pgfqpoint{3.808901in}{1.730272in}}%
\pgfpathlineto{\pgfqpoint{3.891320in}{1.753350in}}%
\pgfpathlineto{\pgfqpoint{3.932529in}{1.748094in}}%
\pgfpathlineto{\pgfqpoint{3.973739in}{1.763020in}}%
\pgfpathlineto{\pgfqpoint{4.056157in}{1.763693in}}%
\pgfpathlineto{\pgfqpoint{4.097367in}{1.809871in}}%
\pgfpathlineto{\pgfqpoint{4.138576in}{1.788369in}}%
\pgfpathlineto{\pgfqpoint{4.179786in}{1.778272in}}%
\pgfpathlineto{\pgfqpoint{4.220995in}{1.797559in}}%
\pgfpathlineto{\pgfqpoint{4.262204in}{1.832690in}}%
\pgfpathlineto{\pgfqpoint{4.303414in}{1.815398in}}%
\pgfpathlineto{\pgfqpoint{4.385833in}{1.832885in}}%
\pgfpathlineto{\pgfqpoint{4.427042in}{1.842087in}}%
\pgfpathlineto{\pgfqpoint{4.468251in}{1.835988in}}%
\pgfpathlineto{\pgfqpoint{4.509461in}{1.850525in}}%
\pgfpathlineto{\pgfqpoint{4.550670in}{1.851799in}}%
\pgfpathlineto{\pgfqpoint{4.591880in}{1.848601in}}%
\pgfpathlineto{\pgfqpoint{4.633089in}{1.861032in}}%
\pgfpathlineto{\pgfqpoint{4.715508in}{1.859321in}}%
\pgfpathlineto{\pgfqpoint{4.756717in}{1.925190in}}%
\pgfpathlineto{\pgfqpoint{4.797926in}{1.886045in}}%
\pgfpathlineto{\pgfqpoint{4.880345in}{1.891665in}}%
\pgfpathlineto{\pgfqpoint{4.962764in}{1.907244in}}%
\pgfpathlineto{\pgfqpoint{5.003973in}{1.875502in}}%
\pgfpathlineto{\pgfqpoint{5.045183in}{1.914870in}}%
\pgfusepath{stroke}%
\end{pgfscope}%
\begin{pgfscope}%
\pgfpathrectangle{\pgfqpoint{0.588387in}{0.521603in}}{\pgfqpoint{4.669024in}{2.220246in}}%
\pgfusepath{clip}%
\pgfsetrectcap%
\pgfsetroundjoin%
\pgfsetlinewidth{1.505625pt}%
\pgfsetstrokecolor{currentstroke5}%
\pgfsetdash{}{0pt}%
\pgfpathmoveto{\pgfqpoint{0.800616in}{0.824866in}}%
\pgfpathlineto{\pgfqpoint{0.841825in}{0.841584in}}%
\pgfpathlineto{\pgfqpoint{0.883034in}{0.836271in}}%
\pgfpathlineto{\pgfqpoint{0.924244in}{0.759181in}}%
\pgfpathlineto{\pgfqpoint{0.965453in}{0.713263in}}%
\pgfpathlineto{\pgfqpoint{1.006663in}{0.696153in}}%
\pgfpathlineto{\pgfqpoint{1.047872in}{0.740637in}}%
\pgfpathlineto{\pgfqpoint{1.089081in}{0.774989in}}%
\pgfpathlineto{\pgfqpoint{1.130291in}{0.831680in}}%
\pgfpathlineto{\pgfqpoint{1.171500in}{0.831470in}}%
\pgfpathlineto{\pgfqpoint{1.212709in}{0.944177in}}%
\pgfpathlineto{\pgfqpoint{1.253919in}{0.941848in}}%
\pgfpathlineto{\pgfqpoint{1.295128in}{0.996245in}}%
\pgfpathlineto{\pgfqpoint{1.336338in}{1.036814in}}%
\pgfpathlineto{\pgfqpoint{1.377547in}{1.094900in}}%
\pgfpathlineto{\pgfqpoint{1.418756in}{1.057302in}}%
\pgfpathlineto{\pgfqpoint{1.459966in}{1.090525in}}%
\pgfpathlineto{\pgfqpoint{1.501175in}{1.110772in}}%
\pgfpathlineto{\pgfqpoint{1.542385in}{1.238521in}}%
\pgfpathlineto{\pgfqpoint{1.583594in}{1.158945in}}%
\pgfpathlineto{\pgfqpoint{1.624803in}{1.190854in}}%
\pgfpathlineto{\pgfqpoint{1.666013in}{1.335632in}}%
\pgfpathlineto{\pgfqpoint{1.707222in}{1.471748in}}%
\pgfpathlineto{\pgfqpoint{1.748432in}{1.372297in}}%
\pgfpathlineto{\pgfqpoint{1.789641in}{1.274486in}}%
\pgfpathlineto{\pgfqpoint{1.830850in}{1.393255in}}%
\pgfpathlineto{\pgfqpoint{1.872060in}{1.956679in}}%
\pgfpathlineto{\pgfqpoint{1.913269in}{1.565234in}}%
\pgfpathlineto{\pgfqpoint{1.954479in}{1.533987in}}%
\pgfpathlineto{\pgfqpoint{1.995688in}{1.560733in}}%
\pgfpathlineto{\pgfqpoint{2.036897in}{2.092953in}}%
\pgfpathlineto{\pgfqpoint{2.078107in}{1.745278in}}%
\pgfpathlineto{\pgfqpoint{2.119316in}{1.910533in}}%
\pgfpathlineto{\pgfqpoint{2.160525in}{1.596184in}}%
\pgfpathlineto{\pgfqpoint{2.201735in}{2.093887in}}%
\pgfpathlineto{\pgfqpoint{2.242944in}{1.798670in}}%
\pgfpathlineto{\pgfqpoint{2.284154in}{1.705634in}}%
\pgfpathlineto{\pgfqpoint{2.325363in}{1.815901in}}%
\pgfpathlineto{\pgfqpoint{2.366572in}{1.580989in}}%
\pgfpathlineto{\pgfqpoint{2.407782in}{1.583054in}}%
\pgfpathlineto{\pgfqpoint{2.448991in}{1.538689in}}%
\pgfpathlineto{\pgfqpoint{2.490201in}{1.814370in}}%
\pgfpathlineto{\pgfqpoint{2.531410in}{1.547849in}}%
\pgfpathlineto{\pgfqpoint{2.572619in}{1.763151in}}%
\pgfpathlineto{\pgfqpoint{2.613829in}{1.536127in}}%
\pgfpathlineto{\pgfqpoint{2.655038in}{2.038233in}}%
\pgfpathlineto{\pgfqpoint{2.696248in}{2.206791in}}%
\pgfpathlineto{\pgfqpoint{2.737457in}{2.004811in}}%
\pgfpathlineto{\pgfqpoint{2.778666in}{1.595946in}}%
\pgfpathlineto{\pgfqpoint{2.819876in}{2.137952in}}%
\pgfpathlineto{\pgfqpoint{2.861085in}{2.089897in}}%
\pgfpathlineto{\pgfqpoint{2.902295in}{1.902601in}}%
\pgfpathlineto{\pgfqpoint{2.943504in}{1.676424in}}%
\pgfpathlineto{\pgfqpoint{2.984713in}{1.706490in}}%
\pgfpathlineto{\pgfqpoint{3.025923in}{2.514805in}}%
\pgfpathlineto{\pgfqpoint{3.067132in}{2.011377in}}%
\pgfpathlineto{\pgfqpoint{3.108341in}{2.166799in}}%
\pgfpathlineto{\pgfqpoint{3.149551in}{2.030058in}}%
\pgfpathlineto{\pgfqpoint{3.190760in}{1.987511in}}%
\pgfpathlineto{\pgfqpoint{3.231970in}{1.830451in}}%
\pgfpathlineto{\pgfqpoint{3.273179in}{2.295015in}}%
\pgfpathlineto{\pgfqpoint{3.314388in}{2.111986in}}%
\pgfpathlineto{\pgfqpoint{3.355598in}{1.777432in}}%
\pgfpathlineto{\pgfqpoint{3.396807in}{2.148049in}}%
\pgfpathlineto{\pgfqpoint{3.438017in}{1.720370in}}%
\pgfpathlineto{\pgfqpoint{3.479226in}{1.745139in}}%
\pgfpathlineto{\pgfqpoint{3.520435in}{1.711886in}}%
\pgfpathlineto{\pgfqpoint{3.561645in}{2.214215in}}%
\pgfpathlineto{\pgfqpoint{3.602854in}{1.826701in}}%
\pgfpathlineto{\pgfqpoint{3.644064in}{2.148935in}}%
\pgfpathlineto{\pgfqpoint{3.685273in}{2.230423in}}%
\pgfpathlineto{\pgfqpoint{3.726482in}{1.994177in}}%
\pgfpathlineto{\pgfqpoint{3.767692in}{1.829251in}}%
\pgfpathlineto{\pgfqpoint{3.808901in}{2.130208in}}%
\pgfpathlineto{\pgfqpoint{3.891320in}{2.106930in}}%
\pgfpathlineto{\pgfqpoint{3.932529in}{1.776589in}}%
\pgfpathlineto{\pgfqpoint{3.973739in}{2.289702in}}%
\pgfpathlineto{\pgfqpoint{4.056157in}{1.975273in}}%
\pgfpathlineto{\pgfqpoint{4.097367in}{1.822947in}}%
\pgfpathlineto{\pgfqpoint{4.138576in}{2.157153in}}%
\pgfpathlineto{\pgfqpoint{4.179786in}{1.979484in}}%
\pgfpathlineto{\pgfqpoint{4.220995in}{1.894899in}}%
\pgfpathlineto{\pgfqpoint{4.262204in}{1.880113in}}%
\pgfpathlineto{\pgfqpoint{4.303414in}{1.842087in}}%
\pgfpathlineto{\pgfqpoint{4.385833in}{1.985530in}}%
\pgfpathlineto{\pgfqpoint{4.427042in}{1.925431in}}%
\pgfpathlineto{\pgfqpoint{4.468251in}{2.244927in}}%
\pgfpathlineto{\pgfqpoint{4.509461in}{2.523802in}}%
\pgfpathlineto{\pgfqpoint{4.550670in}{2.360453in}}%
\pgfpathlineto{\pgfqpoint{4.591880in}{2.086059in}}%
\pgfpathlineto{\pgfqpoint{4.633089in}{2.393479in}}%
\pgfpathlineto{\pgfqpoint{4.715508in}{2.134875in}}%
\pgfpathlineto{\pgfqpoint{4.756717in}{2.538650in}}%
\pgfpathlineto{\pgfqpoint{4.797926in}{2.369536in}}%
\pgfpathlineto{\pgfqpoint{4.880345in}{2.292390in}}%
\pgfpathlineto{\pgfqpoint{4.962764in}{2.256898in}}%
\pgfpathlineto{\pgfqpoint{5.003973in}{2.025848in}}%
\pgfpathlineto{\pgfqpoint{5.045183in}{2.525860in}}%
\pgfusepath{stroke}%
\end{pgfscope}%
\begin{pgfscope}%
\pgfpathrectangle{\pgfqpoint{0.588387in}{0.521603in}}{\pgfqpoint{4.669024in}{2.220246in}}%
\pgfusepath{clip}%
\pgfsetrectcap%
\pgfsetroundjoin%
\pgfsetlinewidth{1.505625pt}%
\pgfsetstrokecolor{currentstroke6}%
\pgfsetdash{}{0pt}%
\pgfpathmoveto{\pgfqpoint{0.800616in}{0.838125in}}%
\pgfpathlineto{\pgfqpoint{0.841825in}{0.850277in}}%
\pgfpathlineto{\pgfqpoint{0.883034in}{0.836271in}}%
\pgfpathlineto{\pgfqpoint{0.924244in}{0.753792in}}%
\pgfpathlineto{\pgfqpoint{0.965453in}{0.719581in}}%
\pgfpathlineto{\pgfqpoint{1.006663in}{0.696153in}}%
\pgfpathlineto{\pgfqpoint{1.047872in}{0.738038in}}%
\pgfpathlineto{\pgfqpoint{1.089081in}{0.774989in}}%
\pgfpathlineto{\pgfqpoint{1.130291in}{0.832596in}}%
\pgfpathlineto{\pgfqpoint{1.171500in}{0.833419in}}%
\pgfpathlineto{\pgfqpoint{1.212709in}{0.899429in}}%
\pgfpathlineto{\pgfqpoint{1.253919in}{0.918592in}}%
\pgfpathlineto{\pgfqpoint{1.295128in}{0.972980in}}%
\pgfpathlineto{\pgfqpoint{1.336338in}{0.984165in}}%
\pgfpathlineto{\pgfqpoint{1.377547in}{1.012675in}}%
\pgfpathlineto{\pgfqpoint{1.418756in}{1.031287in}}%
\pgfpathlineto{\pgfqpoint{1.459966in}{1.062191in}}%
\pgfpathlineto{\pgfqpoint{1.501175in}{1.082061in}}%
\pgfpathlineto{\pgfqpoint{1.542385in}{1.111335in}}%
\pgfpathlineto{\pgfqpoint{1.583594in}{1.122447in}}%
\pgfpathlineto{\pgfqpoint{1.624803in}{1.154646in}}%
\pgfpathlineto{\pgfqpoint{1.666013in}{1.165831in}}%
\pgfpathlineto{\pgfqpoint{1.707222in}{1.192069in}}%
\pgfpathlineto{\pgfqpoint{1.748432in}{1.204394in}}%
\pgfpathlineto{\pgfqpoint{1.789641in}{1.226571in}}%
\pgfpathlineto{\pgfqpoint{1.830850in}{1.237752in}}%
\pgfpathlineto{\pgfqpoint{1.872060in}{1.259400in}}%
\pgfpathlineto{\pgfqpoint{1.913269in}{1.277776in}}%
\pgfpathlineto{\pgfqpoint{1.954479in}{1.285924in}}%
\pgfpathlineto{\pgfqpoint{1.995688in}{1.314656in}}%
\pgfpathlineto{\pgfqpoint{2.036897in}{1.318341in}}%
\pgfpathlineto{\pgfqpoint{2.078107in}{1.342108in}}%
\pgfpathlineto{\pgfqpoint{2.119316in}{1.348287in}}%
\pgfpathlineto{\pgfqpoint{2.160525in}{1.358725in}}%
\pgfpathlineto{\pgfqpoint{2.201735in}{1.369137in}}%
\pgfpathlineto{\pgfqpoint{2.242944in}{1.383462in}}%
\pgfpathlineto{\pgfqpoint{2.284154in}{1.397212in}}%
\pgfpathlineto{\pgfqpoint{2.325363in}{1.407859in}}%
\pgfpathlineto{\pgfqpoint{2.366572in}{1.443199in}}%
\pgfpathlineto{\pgfqpoint{2.407782in}{1.431297in}}%
\pgfpathlineto{\pgfqpoint{2.448991in}{1.444238in}}%
\pgfpathlineto{\pgfqpoint{2.490201in}{1.469393in}}%
\pgfpathlineto{\pgfqpoint{2.531410in}{1.482364in}}%
\pgfpathlineto{\pgfqpoint{2.572619in}{1.486279in}}%
\pgfpathlineto{\pgfqpoint{2.613829in}{1.489460in}}%
\pgfpathlineto{\pgfqpoint{2.655038in}{1.500371in}}%
\pgfpathlineto{\pgfqpoint{2.696248in}{1.515772in}}%
\pgfpathlineto{\pgfqpoint{2.737457in}{1.524266in}}%
\pgfpathlineto{\pgfqpoint{2.778666in}{1.563526in}}%
\pgfpathlineto{\pgfqpoint{2.819876in}{1.547902in}}%
\pgfpathlineto{\pgfqpoint{2.861085in}{1.553645in}}%
\pgfpathlineto{\pgfqpoint{2.902295in}{1.549613in}}%
\pgfpathlineto{\pgfqpoint{2.943504in}{1.576461in}}%
\pgfpathlineto{\pgfqpoint{2.984713in}{1.590987in}}%
\pgfpathlineto{\pgfqpoint{3.025923in}{1.575381in}}%
\pgfpathlineto{\pgfqpoint{3.067132in}{1.588711in}}%
\pgfpathlineto{\pgfqpoint{3.108341in}{1.604393in}}%
\pgfpathlineto{\pgfqpoint{3.149551in}{1.615136in}}%
\pgfpathlineto{\pgfqpoint{3.190760in}{1.615861in}}%
\pgfpathlineto{\pgfqpoint{3.231970in}{1.612437in}}%
\pgfpathlineto{\pgfqpoint{3.273179in}{1.628482in}}%
\pgfpathlineto{\pgfqpoint{3.314388in}{1.629726in}}%
\pgfpathlineto{\pgfqpoint{3.355598in}{1.655675in}}%
\pgfpathlineto{\pgfqpoint{3.396807in}{1.650383in}}%
\pgfpathlineto{\pgfqpoint{3.438017in}{1.678632in}}%
\pgfpathlineto{\pgfqpoint{3.479226in}{1.678020in}}%
\pgfpathlineto{\pgfqpoint{3.520435in}{1.672452in}}%
\pgfpathlineto{\pgfqpoint{3.561645in}{1.697596in}}%
\pgfpathlineto{\pgfqpoint{3.602854in}{1.680660in}}%
\pgfpathlineto{\pgfqpoint{3.644064in}{1.694824in}}%
\pgfpathlineto{\pgfqpoint{3.685273in}{1.690186in}}%
\pgfpathlineto{\pgfqpoint{3.726482in}{1.723001in}}%
\pgfpathlineto{\pgfqpoint{3.767692in}{1.713498in}}%
\pgfpathlineto{\pgfqpoint{3.808901in}{1.724855in}}%
\pgfpathlineto{\pgfqpoint{3.891320in}{1.746763in}}%
\pgfpathlineto{\pgfqpoint{3.932529in}{1.741237in}}%
\pgfpathlineto{\pgfqpoint{3.973739in}{1.752599in}}%
\pgfpathlineto{\pgfqpoint{4.056157in}{1.764658in}}%
\pgfpathlineto{\pgfqpoint{4.097367in}{1.789765in}}%
\pgfpathlineto{\pgfqpoint{4.138576in}{1.779166in}}%
\pgfpathlineto{\pgfqpoint{4.179786in}{1.766259in}}%
\pgfpathlineto{\pgfqpoint{4.220995in}{1.789518in}}%
\pgfpathlineto{\pgfqpoint{4.262204in}{1.821526in}}%
\pgfpathlineto{\pgfqpoint{4.303414in}{1.808751in}}%
\pgfpathlineto{\pgfqpoint{4.385833in}{1.823554in}}%
\pgfpathlineto{\pgfqpoint{4.427042in}{1.826818in}}%
\pgfpathlineto{\pgfqpoint{4.468251in}{1.833664in}}%
\pgfpathlineto{\pgfqpoint{4.509461in}{1.878393in}}%
\pgfpathlineto{\pgfqpoint{4.550670in}{1.839103in}}%
\pgfpathlineto{\pgfqpoint{4.591880in}{1.836758in}}%
\pgfpathlineto{\pgfqpoint{4.633089in}{1.854120in}}%
\pgfpathlineto{\pgfqpoint{4.715508in}{1.845581in}}%
\pgfpathlineto{\pgfqpoint{4.756717in}{1.890752in}}%
\pgfpathlineto{\pgfqpoint{4.797926in}{1.869920in}}%
\pgfpathlineto{\pgfqpoint{4.880345in}{1.898505in}}%
\pgfpathlineto{\pgfqpoint{4.962764in}{1.904131in}}%
\pgfpathlineto{\pgfqpoint{5.003973in}{1.887993in}}%
\pgfpathlineto{\pgfqpoint{5.045183in}{2.026177in}}%
\pgfusepath{stroke}%
\end{pgfscope}%
\begin{pgfscope}%
\pgfpathrectangle{\pgfqpoint{0.588387in}{0.521603in}}{\pgfqpoint{4.669024in}{2.220246in}}%
\pgfusepath{clip}%
\pgfsetrectcap%
\pgfsetroundjoin%
\pgfsetlinewidth{1.505625pt}%
\pgfsetstrokecolor{currentstroke7}%
\pgfsetdash{}{0pt}%
\pgfpathmoveto{\pgfqpoint{0.800616in}{0.824866in}}%
\pgfpathlineto{\pgfqpoint{0.841825in}{0.837128in}}%
\pgfpathlineto{\pgfqpoint{0.883034in}{0.843565in}}%
\pgfpathlineto{\pgfqpoint{0.924244in}{0.761477in}}%
\pgfpathlineto{\pgfqpoint{0.965453in}{0.711547in}}%
\pgfpathlineto{\pgfqpoint{1.006663in}{0.696153in}}%
\pgfpathlineto{\pgfqpoint{1.047872in}{0.740637in}}%
\pgfpathlineto{\pgfqpoint{1.089081in}{0.773775in}}%
\pgfpathlineto{\pgfqpoint{1.130291in}{0.831680in}}%
\pgfpathlineto{\pgfqpoint{1.171500in}{0.834389in}}%
\pgfpathlineto{\pgfqpoint{1.212709in}{0.949186in}}%
\pgfpathlineto{\pgfqpoint{1.253919in}{0.950507in}}%
\pgfpathlineto{\pgfqpoint{1.295128in}{1.021611in}}%
\pgfpathlineto{\pgfqpoint{1.336338in}{1.051670in}}%
\pgfpathlineto{\pgfqpoint{1.377547in}{1.143570in}}%
\pgfpathlineto{\pgfqpoint{1.418756in}{1.083156in}}%
\pgfpathlineto{\pgfqpoint{1.459966in}{1.116120in}}%
\pgfpathlineto{\pgfqpoint{1.501175in}{1.138844in}}%
\pgfpathlineto{\pgfqpoint{1.542385in}{1.308053in}}%
\pgfpathlineto{\pgfqpoint{1.583594in}{1.209990in}}%
\pgfpathlineto{\pgfqpoint{1.624803in}{1.167833in}}%
\pgfpathlineto{\pgfqpoint{1.666013in}{1.293903in}}%
\pgfpathlineto{\pgfqpoint{1.707222in}{1.537001in}}%
\pgfpathlineto{\pgfqpoint{1.748432in}{1.365027in}}%
\pgfpathlineto{\pgfqpoint{1.789641in}{1.454894in}}%
\pgfpathlineto{\pgfqpoint{1.830850in}{1.601586in}}%
\pgfpathlineto{\pgfqpoint{1.872060in}{1.560546in}}%
\pgfpathlineto{\pgfqpoint{1.913269in}{1.464073in}}%
\pgfpathlineto{\pgfqpoint{1.954479in}{2.109553in}}%
\pgfpathlineto{\pgfqpoint{1.995688in}{1.816105in}}%
\pgfpathlineto{\pgfqpoint{2.036897in}{2.078192in}}%
\pgfpathlineto{\pgfqpoint{2.078107in}{1.790175in}}%
\pgfpathlineto{\pgfqpoint{2.119316in}{1.721931in}}%
\pgfpathlineto{\pgfqpoint{2.160525in}{1.476386in}}%
\pgfpathlineto{\pgfqpoint{2.201735in}{2.144913in}}%
\pgfpathlineto{\pgfqpoint{2.242944in}{1.837003in}}%
\pgfpathlineto{\pgfqpoint{2.284154in}{1.929145in}}%
\pgfpathlineto{\pgfqpoint{2.325363in}{1.993401in}}%
\pgfpathlineto{\pgfqpoint{2.366572in}{2.199086in}}%
\pgfpathlineto{\pgfqpoint{2.407782in}{1.885711in}}%
\pgfpathlineto{\pgfqpoint{2.448991in}{1.877817in}}%
\pgfpathlineto{\pgfqpoint{2.490201in}{1.954293in}}%
\pgfpathlineto{\pgfqpoint{2.531410in}{2.224163in}}%
\pgfpathlineto{\pgfqpoint{2.572619in}{1.932605in}}%
\pgfpathlineto{\pgfqpoint{2.613829in}{1.561278in}}%
\pgfpathlineto{\pgfqpoint{2.655038in}{1.949793in}}%
\pgfpathlineto{\pgfqpoint{2.696248in}{2.234606in}}%
\pgfpathlineto{\pgfqpoint{2.737457in}{2.033895in}}%
\pgfpathlineto{\pgfqpoint{2.778666in}{1.621820in}}%
\pgfpathlineto{\pgfqpoint{2.819876in}{2.137223in}}%
\pgfpathlineto{\pgfqpoint{2.861085in}{2.115313in}}%
\pgfpathlineto{\pgfqpoint{2.902295in}{2.249430in}}%
\pgfpathlineto{\pgfqpoint{2.943504in}{2.331935in}}%
\pgfpathlineto{\pgfqpoint{2.984713in}{2.012228in}}%
\pgfpathlineto{\pgfqpoint{3.025923in}{1.742872in}}%
\pgfpathlineto{\pgfqpoint{3.067132in}{2.200850in}}%
\pgfpathlineto{\pgfqpoint{3.108341in}{1.998631in}}%
\pgfpathlineto{\pgfqpoint{3.149551in}{2.147648in}}%
\pgfpathlineto{\pgfqpoint{3.190760in}{2.201659in}}%
\pgfpathlineto{\pgfqpoint{3.231970in}{2.156863in}}%
\pgfpathlineto{\pgfqpoint{3.273179in}{2.326814in}}%
\pgfpathlineto{\pgfqpoint{3.314388in}{2.041704in}}%
\pgfpathlineto{\pgfqpoint{3.355598in}{2.516132in}}%
\pgfpathlineto{\pgfqpoint{3.396807in}{2.308190in}}%
\pgfpathlineto{\pgfqpoint{3.438017in}{1.847957in}}%
\pgfpathlineto{\pgfqpoint{3.479226in}{2.197744in}}%
\pgfpathlineto{\pgfqpoint{3.520435in}{1.792959in}}%
\pgfpathlineto{\pgfqpoint{3.561645in}{2.330050in}}%
\pgfpathlineto{\pgfqpoint{3.602854in}{1.923902in}}%
\pgfpathlineto{\pgfqpoint{3.644064in}{2.111318in}}%
\pgfpathlineto{\pgfqpoint{3.685273in}{1.931227in}}%
\pgfpathlineto{\pgfqpoint{3.726482in}{2.306517in}}%
\pgfpathlineto{\pgfqpoint{3.767692in}{2.060712in}}%
\pgfpathlineto{\pgfqpoint{3.808901in}{2.268216in}}%
\pgfpathlineto{\pgfqpoint{3.891320in}{2.131186in}}%
\pgfpathlineto{\pgfqpoint{3.932529in}{2.308023in}}%
\pgfpathlineto{\pgfqpoint{3.973739in}{1.810490in}}%
\pgfpathlineto{\pgfqpoint{4.056157in}{2.373649in}}%
\pgfpathlineto{\pgfqpoint{4.097367in}{1.887993in}}%
\pgfpathlineto{\pgfqpoint{4.138576in}{2.241039in}}%
\pgfpathlineto{\pgfqpoint{4.179786in}{2.362823in}}%
\pgfpathlineto{\pgfqpoint{4.220995in}{2.177134in}}%
\pgfpathlineto{\pgfqpoint{4.262204in}{1.867235in}}%
\pgfpathlineto{\pgfqpoint{4.303414in}{1.855900in}}%
\pgfpathlineto{\pgfqpoint{4.385833in}{1.891300in}}%
\pgfpathlineto{\pgfqpoint{4.427042in}{2.111947in}}%
\pgfpathlineto{\pgfqpoint{4.468251in}{2.070642in}}%
\pgfpathlineto{\pgfqpoint{4.509461in}{1.988746in}}%
\pgfpathlineto{\pgfqpoint{4.550670in}{2.115380in}}%
\pgfpathlineto{\pgfqpoint{4.591880in}{2.521676in}}%
\pgfpathlineto{\pgfqpoint{4.633089in}{2.272260in}}%
\pgfpathlineto{\pgfqpoint{4.715508in}{2.099381in}}%
\pgfpathlineto{\pgfqpoint{4.756717in}{1.982219in}}%
\pgfpathlineto{\pgfqpoint{4.797926in}{1.979091in}}%
\pgfpathlineto{\pgfqpoint{4.880345in}{2.456779in}}%
\pgfpathlineto{\pgfqpoint{4.962764in}{2.007416in}}%
\pgfpathlineto{\pgfqpoint{5.003973in}{1.995116in}}%
\pgfpathlineto{\pgfqpoint{5.045183in}{1.946929in}}%
\pgfusepath{stroke}%
\end{pgfscope}%
\begin{pgfscope}%
\pgfpathrectangle{\pgfqpoint{0.588387in}{0.521603in}}{\pgfqpoint{4.669024in}{2.220246in}}%
\pgfusepath{clip}%
\pgfsetrectcap%
\pgfsetroundjoin%
\pgfsetlinewidth{1.505625pt}%
\definecolor{currentstroke}{rgb}{0.498039,0.498039,0.498039}%
\pgfsetstrokecolor{currentstroke}%
\pgfsetdash{}{0pt}%
\pgfpathmoveto{\pgfqpoint{0.800616in}{0.824866in}}%
\pgfpathlineto{\pgfqpoint{0.841825in}{0.850277in}}%
\pgfpathlineto{\pgfqpoint{0.883034in}{0.851982in}}%
\pgfpathlineto{\pgfqpoint{0.924244in}{0.778379in}}%
\pgfpathlineto{\pgfqpoint{0.965453in}{0.720309in}}%
\pgfpathlineto{\pgfqpoint{1.006663in}{0.699447in}}%
\pgfpathlineto{\pgfqpoint{1.047872in}{0.745760in}}%
\pgfpathlineto{\pgfqpoint{1.089081in}{0.765123in}}%
\pgfpathlineto{\pgfqpoint{1.130291in}{0.820437in}}%
\pgfpathlineto{\pgfqpoint{1.171500in}{0.826533in}}%
\pgfpathlineto{\pgfqpoint{1.212709in}{0.876206in}}%
\pgfpathlineto{\pgfqpoint{1.253919in}{0.893098in}}%
\pgfpathlineto{\pgfqpoint{1.295128in}{0.944850in}}%
\pgfpathlineto{\pgfqpoint{1.336338in}{0.953056in}}%
\pgfpathlineto{\pgfqpoint{1.377547in}{0.980076in}}%
\pgfpathlineto{\pgfqpoint{1.418756in}{0.993953in}}%
\pgfpathlineto{\pgfqpoint{1.459966in}{1.031968in}}%
\pgfpathlineto{\pgfqpoint{1.501175in}{1.051114in}}%
\pgfpathlineto{\pgfqpoint{1.542385in}{1.072392in}}%
\pgfpathlineto{\pgfqpoint{1.583594in}{1.078474in}}%
\pgfpathlineto{\pgfqpoint{1.624803in}{1.111319in}}%
\pgfpathlineto{\pgfqpoint{1.666013in}{1.133285in}}%
\pgfpathlineto{\pgfqpoint{1.707222in}{1.143925in}}%
\pgfpathlineto{\pgfqpoint{1.748432in}{1.154860in}}%
\pgfpathlineto{\pgfqpoint{1.789641in}{1.188698in}}%
\pgfpathlineto{\pgfqpoint{1.830850in}{1.205720in}}%
\pgfpathlineto{\pgfqpoint{1.872060in}{1.219252in}}%
\pgfpathlineto{\pgfqpoint{1.913269in}{1.223341in}}%
\pgfpathlineto{\pgfqpoint{1.954479in}{1.242206in}}%
\pgfpathlineto{\pgfqpoint{1.995688in}{1.263966in}}%
\pgfpathlineto{\pgfqpoint{2.036897in}{1.273549in}}%
\pgfpathlineto{\pgfqpoint{2.078107in}{1.281466in}}%
\pgfpathlineto{\pgfqpoint{2.119316in}{1.301152in}}%
\pgfpathlineto{\pgfqpoint{2.160525in}{1.312517in}}%
\pgfpathlineto{\pgfqpoint{2.201735in}{1.313601in}}%
\pgfpathlineto{\pgfqpoint{2.242944in}{1.327163in}}%
\pgfpathlineto{\pgfqpoint{2.284154in}{1.343068in}}%
\pgfpathlineto{\pgfqpoint{2.325363in}{1.360008in}}%
\pgfpathlineto{\pgfqpoint{2.366572in}{1.372666in}}%
\pgfpathlineto{\pgfqpoint{2.407782in}{1.375339in}}%
\pgfpathlineto{\pgfqpoint{2.448991in}{1.387489in}}%
\pgfpathlineto{\pgfqpoint{2.490201in}{1.398206in}}%
\pgfpathlineto{\pgfqpoint{2.531410in}{1.411589in}}%
\pgfpathlineto{\pgfqpoint{2.572619in}{1.412757in}}%
\pgfpathlineto{\pgfqpoint{2.613829in}{1.466218in}}%
\pgfpathlineto{\pgfqpoint{2.655038in}{1.447771in}}%
\pgfpathlineto{\pgfqpoint{2.696248in}{1.458527in}}%
\pgfpathlineto{\pgfqpoint{2.737457in}{1.474566in}}%
\pgfpathlineto{\pgfqpoint{2.778666in}{1.469429in}}%
\pgfpathlineto{\pgfqpoint{2.819876in}{1.482566in}}%
\pgfpathlineto{\pgfqpoint{2.861085in}{1.491181in}}%
\pgfpathlineto{\pgfqpoint{2.902295in}{1.489460in}}%
\pgfpathlineto{\pgfqpoint{2.943504in}{1.509835in}}%
\pgfpathlineto{\pgfqpoint{2.984713in}{1.510196in}}%
\pgfpathlineto{\pgfqpoint{3.025923in}{1.510296in}}%
\pgfpathlineto{\pgfqpoint{3.067132in}{1.530823in}}%
\pgfpathlineto{\pgfqpoint{3.108341in}{1.542799in}}%
\pgfpathlineto{\pgfqpoint{3.149551in}{1.555607in}}%
\pgfpathlineto{\pgfqpoint{3.190760in}{1.551328in}}%
\pgfpathlineto{\pgfqpoint{3.231970in}{1.551250in}}%
\pgfpathlineto{\pgfqpoint{3.273179in}{1.566207in}}%
\pgfpathlineto{\pgfqpoint{3.314388in}{1.591512in}}%
\pgfpathlineto{\pgfqpoint{3.355598in}{1.627595in}}%
\pgfpathlineto{\pgfqpoint{3.396807in}{1.604608in}}%
\pgfpathlineto{\pgfqpoint{3.438017in}{1.599498in}}%
\pgfpathlineto{\pgfqpoint{3.479226in}{1.612942in}}%
\pgfpathlineto{\pgfqpoint{3.520435in}{1.604393in}}%
\pgfpathlineto{\pgfqpoint{3.561645in}{1.617020in}}%
\pgfpathlineto{\pgfqpoint{3.602854in}{1.623114in}}%
\pgfpathlineto{\pgfqpoint{3.644064in}{1.628679in}}%
\pgfpathlineto{\pgfqpoint{3.685273in}{1.614956in}}%
\pgfpathlineto{\pgfqpoint{3.726482in}{1.653087in}}%
\pgfpathlineto{\pgfqpoint{3.767692in}{1.642707in}}%
\pgfpathlineto{\pgfqpoint{3.808901in}{1.657225in}}%
\pgfpathlineto{\pgfqpoint{3.891320in}{1.686631in}}%
\pgfpathlineto{\pgfqpoint{3.932529in}{1.678388in}}%
\pgfpathlineto{\pgfqpoint{3.973739in}{1.686061in}}%
\pgfpathlineto{\pgfqpoint{4.056157in}{1.690470in}}%
\pgfpathlineto{\pgfqpoint{4.097367in}{1.724513in}}%
\pgfpathlineto{\pgfqpoint{4.138576in}{1.731676in}}%
\pgfpathlineto{\pgfqpoint{4.179786in}{1.703121in}}%
\pgfpathlineto{\pgfqpoint{4.220995in}{1.744746in}}%
\pgfpathlineto{\pgfqpoint{4.262204in}{1.736565in}}%
\pgfpathlineto{\pgfqpoint{4.303414in}{1.787749in}}%
\pgfpathlineto{\pgfqpoint{4.385833in}{1.780896in}}%
\pgfpathlineto{\pgfqpoint{4.427042in}{1.767572in}}%
\pgfpathlineto{\pgfqpoint{4.468251in}{1.792959in}}%
\pgfpathlineto{\pgfqpoint{4.509461in}{1.774895in}}%
\pgfpathlineto{\pgfqpoint{4.550670in}{1.750434in}}%
\pgfpathlineto{\pgfqpoint{4.591880in}{1.789765in}}%
\pgfpathlineto{\pgfqpoint{4.633089in}{1.783259in}}%
\pgfpathlineto{\pgfqpoint{4.715508in}{1.780777in}}%
\pgfpathlineto{\pgfqpoint{4.756717in}{1.820097in}}%
\pgfpathlineto{\pgfqpoint{4.797926in}{1.804985in}}%
\pgfpathlineto{\pgfqpoint{4.880345in}{1.801806in}}%
\pgfpathlineto{\pgfqpoint{4.962764in}{1.835409in}}%
\pgfpathlineto{\pgfqpoint{5.003973in}{1.809871in}}%
\pgfpathlineto{\pgfqpoint{5.045183in}{1.866033in}}%
\pgfusepath{stroke}%
\end{pgfscope}%
\begin{pgfscope}%
\pgfpathrectangle{\pgfqpoint{0.588387in}{0.521603in}}{\pgfqpoint{4.669024in}{2.220246in}}%
\pgfusepath{clip}%
\pgfsetrectcap%
\pgfsetroundjoin%
\pgfsetlinewidth{1.505625pt}%
\definecolor{currentstroke}{rgb}{0.737255,0.741176,0.133333}%
\pgfsetstrokecolor{currentstroke}%
\pgfsetdash{}{0pt}%
\pgfpathmoveto{\pgfqpoint{0.800616in}{0.824866in}}%
\pgfpathlineto{\pgfqpoint{0.841825in}{0.850277in}}%
\pgfpathlineto{\pgfqpoint{0.883034in}{0.854297in}}%
\pgfpathlineto{\pgfqpoint{0.924244in}{0.775714in}}%
\pgfpathlineto{\pgfqpoint{0.965453in}{0.723033in}}%
\pgfpathlineto{\pgfqpoint{1.006663in}{0.703040in}}%
\pgfpathlineto{\pgfqpoint{1.047872in}{0.731426in}}%
\pgfpathlineto{\pgfqpoint{1.089081in}{0.762598in}}%
\pgfpathlineto{\pgfqpoint{1.130291in}{0.814632in}}%
\pgfpathlineto{\pgfqpoint{1.171500in}{0.818437in}}%
\pgfpathlineto{\pgfqpoint{1.212709in}{0.907016in}}%
\pgfpathlineto{\pgfqpoint{1.253919in}{0.892475in}}%
\pgfpathlineto{\pgfqpoint{1.295128in}{1.003743in}}%
\pgfpathlineto{\pgfqpoint{1.336338in}{1.036814in}}%
\pgfpathlineto{\pgfqpoint{1.377547in}{1.188089in}}%
\pgfpathlineto{\pgfqpoint{1.418756in}{1.097829in}}%
\pgfpathlineto{\pgfqpoint{1.459966in}{1.150563in}}%
\pgfpathlineto{\pgfqpoint{1.501175in}{1.263695in}}%
\pgfpathlineto{\pgfqpoint{1.542385in}{1.493533in}}%
\pgfpathlineto{\pgfqpoint{1.583594in}{1.247916in}}%
\pgfpathlineto{\pgfqpoint{1.624803in}{1.580501in}}%
\pgfpathlineto{\pgfqpoint{1.666013in}{1.568470in}}%
\pgfpathlineto{\pgfqpoint{1.707222in}{1.694417in}}%
\pgfpathlineto{\pgfqpoint{1.748432in}{1.744046in}}%
\pgfpathlineto{\pgfqpoint{1.789641in}{1.758464in}}%
\pgfpathlineto{\pgfqpoint{1.830850in}{1.813301in}}%
\pgfpathlineto{\pgfqpoint{1.872060in}{2.240194in}}%
\pgfpathlineto{\pgfqpoint{1.913269in}{2.026375in}}%
\pgfpathlineto{\pgfqpoint{1.954479in}{2.065700in}}%
\pgfpathlineto{\pgfqpoint{1.995688in}{2.089900in}}%
\pgfpathlineto{\pgfqpoint{2.036897in}{2.251979in}}%
\pgfpathlineto{\pgfqpoint{2.078107in}{2.143899in}}%
\pgfpathlineto{\pgfqpoint{2.119316in}{2.368937in}}%
\pgfpathlineto{\pgfqpoint{2.160525in}{2.240175in}}%
\pgfpathlineto{\pgfqpoint{2.201735in}{2.433407in}}%
\pgfpathlineto{\pgfqpoint{2.242944in}{2.189916in}}%
\pgfpathlineto{\pgfqpoint{2.284154in}{2.369437in}}%
\pgfpathlineto{\pgfqpoint{2.325363in}{2.321730in}}%
\pgfpathlineto{\pgfqpoint{2.366572in}{2.325006in}}%
\pgfpathlineto{\pgfqpoint{2.407782in}{2.387531in}}%
\pgfpathlineto{\pgfqpoint{2.448991in}{2.487543in}}%
\pgfpathlineto{\pgfqpoint{2.490201in}{2.321869in}}%
\pgfpathlineto{\pgfqpoint{2.531410in}{2.432890in}}%
\pgfpathlineto{\pgfqpoint{2.572619in}{2.486193in}}%
\pgfpathlineto{\pgfqpoint{2.613829in}{2.414449in}}%
\pgfpathlineto{\pgfqpoint{2.655038in}{2.450961in}}%
\pgfpathlineto{\pgfqpoint{2.696248in}{2.475652in}}%
\pgfpathlineto{\pgfqpoint{2.737457in}{2.412906in}}%
\pgfpathlineto{\pgfqpoint{2.778666in}{2.455487in}}%
\pgfpathlineto{\pgfqpoint{2.819876in}{2.435830in}}%
\pgfpathlineto{\pgfqpoint{2.861085in}{2.592830in}}%
\pgfpathlineto{\pgfqpoint{2.902295in}{2.524962in}}%
\pgfpathlineto{\pgfqpoint{2.943504in}{2.560898in}}%
\pgfpathlineto{\pgfqpoint{2.984713in}{2.392472in}}%
\pgfpathlineto{\pgfqpoint{3.025923in}{2.516650in}}%
\pgfpathlineto{\pgfqpoint{3.067132in}{2.400795in}}%
\pgfpathlineto{\pgfqpoint{3.108341in}{2.212741in}}%
\pgfpathlineto{\pgfqpoint{3.149551in}{2.465453in}}%
\pgfpathlineto{\pgfqpoint{3.190760in}{2.573926in}}%
\pgfpathlineto{\pgfqpoint{3.231970in}{2.446226in}}%
\pgfpathlineto{\pgfqpoint{3.273179in}{2.565657in}}%
\pgfpathlineto{\pgfqpoint{3.314388in}{2.514907in}}%
\pgfpathlineto{\pgfqpoint{3.355598in}{2.640929in}}%
\pgfpathlineto{\pgfqpoint{3.396807in}{2.601472in}}%
\pgfpathlineto{\pgfqpoint{3.438017in}{2.515248in}}%
\pgfpathlineto{\pgfqpoint{3.479226in}{2.547484in}}%
\pgfpathlineto{\pgfqpoint{3.561645in}{2.581176in}}%
\pgfpathlineto{\pgfqpoint{3.644064in}{2.566820in}}%
\pgfpathlineto{\pgfqpoint{3.685273in}{2.589189in}}%
\pgfpathlineto{\pgfqpoint{3.726482in}{2.592290in}}%
\pgfpathlineto{\pgfqpoint{3.767692in}{2.435962in}}%
\pgfpathlineto{\pgfqpoint{3.808901in}{2.474487in}}%
\pgfpathlineto{\pgfqpoint{3.891320in}{2.583185in}}%
\pgfpathlineto{\pgfqpoint{3.932529in}{2.471166in}}%
\pgfpathlineto{\pgfqpoint{3.973739in}{2.542764in}}%
\pgfpathlineto{\pgfqpoint{4.056157in}{2.498019in}}%
\pgfpathlineto{\pgfqpoint{4.097367in}{2.611777in}}%
\pgfpathlineto{\pgfqpoint{4.138576in}{2.572663in}}%
\pgfpathlineto{\pgfqpoint{4.179786in}{2.517425in}}%
\pgfpathlineto{\pgfqpoint{4.220995in}{2.540858in}}%
\pgfpathlineto{\pgfqpoint{4.303414in}{2.621018in}}%
\pgfpathlineto{\pgfqpoint{4.385833in}{2.616438in}}%
\pgfpathlineto{\pgfqpoint{4.427042in}{2.591412in}}%
\pgfpathlineto{\pgfqpoint{4.468251in}{2.555906in}}%
\pgfpathlineto{\pgfqpoint{4.550670in}{2.371736in}}%
\pgfpathlineto{\pgfqpoint{4.591880in}{2.520403in}}%
\pgfpathlineto{\pgfqpoint{4.633089in}{2.434503in}}%
\pgfpathlineto{\pgfqpoint{4.715508in}{2.591477in}}%
\pgfpathlineto{\pgfqpoint{4.797926in}{2.362267in}}%
\pgfpathlineto{\pgfqpoint{4.880345in}{2.430082in}}%
\pgfpathlineto{\pgfqpoint{4.962764in}{1.868432in}}%
\pgfpathlineto{\pgfqpoint{5.003973in}{1.894567in}}%
\pgfusepath{stroke}%
\end{pgfscope}%
\begin{pgfscope}%
\pgfsetrectcap%
\pgfsetmiterjoin%
\pgfsetlinewidth{0.803000pt}%
\definecolor{currentstroke}{rgb}{0.000000,0.000000,0.000000}%
\pgfsetstrokecolor{currentstroke}%
\pgfsetdash{}{0pt}%
\pgfpathmoveto{\pgfqpoint{0.588387in}{0.521603in}}%
\pgfpathlineto{\pgfqpoint{0.588387in}{2.741849in}}%
\pgfusepath{stroke}%
\end{pgfscope}%
\begin{pgfscope}%
\pgfsetrectcap%
\pgfsetmiterjoin%
\pgfsetlinewidth{0.803000pt}%
\definecolor{currentstroke}{rgb}{0.000000,0.000000,0.000000}%
\pgfsetstrokecolor{currentstroke}%
\pgfsetdash{}{0pt}%
\pgfpathmoveto{\pgfqpoint{5.257411in}{0.521603in}}%
\pgfpathlineto{\pgfqpoint{5.257411in}{2.741849in}}%
\pgfusepath{stroke}%
\end{pgfscope}%
\begin{pgfscope}%
\pgfsetrectcap%
\pgfsetmiterjoin%
\pgfsetlinewidth{0.803000pt}%
\definecolor{currentstroke}{rgb}{0.000000,0.000000,0.000000}%
\pgfsetstrokecolor{currentstroke}%
\pgfsetdash{}{0pt}%
\pgfpathmoveto{\pgfqpoint{0.588387in}{0.521603in}}%
\pgfpathlineto{\pgfqpoint{5.257411in}{0.521603in}}%
\pgfusepath{stroke}%
\end{pgfscope}%
\begin{pgfscope}%
\pgfsetrectcap%
\pgfsetmiterjoin%
\pgfsetlinewidth{0.803000pt}%
\definecolor{currentstroke}{rgb}{0.000000,0.000000,0.000000}%
\pgfsetstrokecolor{currentstroke}%
\pgfsetdash{}{0pt}%
\pgfpathmoveto{\pgfqpoint{0.588387in}{2.741849in}}%
\pgfpathlineto{\pgfqpoint{5.257411in}{2.741849in}}%
\pgfusepath{stroke}%
\end{pgfscope}%
\begin{pgfscope}%
\pgfsetbuttcap%
\pgfsetmiterjoin%
\definecolor{currentfill}{rgb}{1.000000,1.000000,1.000000}%
\pgfsetfillcolor{currentfill}%
\pgfsetfillopacity{0.800000}%
\pgfsetlinewidth{1.003750pt}%
\definecolor{currentstroke}{rgb}{0.800000,0.800000,0.800000}%
\pgfsetstrokecolor{currentstroke}%
\pgfsetstrokeopacity{0.800000}%
\pgfsetdash{}{0pt}%
\pgfpathmoveto{\pgfqpoint{5.344911in}{0.969732in}}%
\pgfpathlineto{\pgfqpoint{8.259376in}{0.969732in}}%
\pgfpathquadraticcurveto{\pgfqpoint{8.284376in}{0.969732in}}{\pgfqpoint{8.284376in}{0.994732in}}%
\pgfpathlineto{\pgfqpoint{8.284376in}{2.654349in}}%
\pgfpathquadraticcurveto{\pgfqpoint{8.284376in}{2.679349in}}{\pgfqpoint{8.259376in}{2.679349in}}%
\pgfpathlineto{\pgfqpoint{5.344911in}{2.679349in}}%
\pgfpathquadraticcurveto{\pgfqpoint{5.319911in}{2.679349in}}{\pgfqpoint{5.319911in}{2.654349in}}%
\pgfpathlineto{\pgfqpoint{5.319911in}{0.994732in}}%
\pgfpathquadraticcurveto{\pgfqpoint{5.319911in}{0.969732in}}{\pgfqpoint{5.344911in}{0.969732in}}%
\pgfpathlineto{\pgfqpoint{5.344911in}{0.969732in}}%
\pgfpathclose%
\pgfusepath{stroke,fill}%
\end{pgfscope}%
\begin{pgfscope}%
\pgfsetrectcap%
\pgfsetroundjoin%
\pgfsetlinewidth{1.505625pt}%
\pgfsetstrokecolor{currentstroke3}%
\pgfsetdash{}{0pt}%
\pgfpathmoveto{\pgfqpoint{5.369911in}{2.578129in}}%
\pgfpathlineto{\pgfqpoint{5.494911in}{2.578129in}}%
\pgfpathlineto{\pgfqpoint{5.619911in}{2.578129in}}%
\pgfusepath{stroke}%
\end{pgfscope}%
\begin{pgfscope}%
\definecolor{textcolor}{rgb}{0.000000,0.000000,0.000000}%
\pgfsetstrokecolor{textcolor}%
\pgfsetfillcolor{textcolor}%
\pgftext[x=5.719911in,y=2.534379in,left,base]{\color{textcolor}{\rmfamily\fontsize{9.000000}{10.800000}\selectfont\catcode`\^=\active\def^{\ifmmode\sp\else\^{}\fi}\catcode`\%=\active\def%{\%}\NaiveCycles{}}}%
\end{pgfscope}%
\begin{pgfscope}%
\pgfsetrectcap%
\pgfsetroundjoin%
\pgfsetlinewidth{1.505625pt}%
\pgfsetstrokecolor{currentstroke1}%
\pgfsetdash{}{0pt}%
\pgfpathmoveto{\pgfqpoint{5.369911in}{2.394657in}}%
\pgfpathlineto{\pgfqpoint{5.494911in}{2.394657in}}%
\pgfpathlineto{\pgfqpoint{5.619911in}{2.394657in}}%
\pgfusepath{stroke}%
\end{pgfscope}%
\begin{pgfscope}%
\definecolor{textcolor}{rgb}{0.000000,0.000000,0.000000}%
\pgfsetstrokecolor{textcolor}%
\pgfsetfillcolor{textcolor}%
\pgftext[x=5.719911in,y=2.350907in,left,base]{\color{textcolor}{\rmfamily\fontsize{9.000000}{10.800000}\selectfont\catcode`\^=\active\def^{\ifmmode\sp\else\^{}\fi}\catcode`\%=\active\def%{\%}\CyclesMatchChunks{} \& \MergeLinear{}}}%
\end{pgfscope}%
\begin{pgfscope}%
\pgfsetrectcap%
\pgfsetroundjoin%
\pgfsetlinewidth{1.505625pt}%
\pgfsetstrokecolor{currentstroke2}%
\pgfsetdash{}{0pt}%
\pgfpathmoveto{\pgfqpoint{5.369911in}{2.207707in}}%
\pgfpathlineto{\pgfqpoint{5.494911in}{2.207707in}}%
\pgfpathlineto{\pgfqpoint{5.619911in}{2.207707in}}%
\pgfusepath{stroke}%
\end{pgfscope}%
\begin{pgfscope}%
\definecolor{textcolor}{rgb}{0.000000,0.000000,0.000000}%
\pgfsetstrokecolor{textcolor}%
\pgfsetfillcolor{textcolor}%
\pgftext[x=5.719911in,y=2.163957in,left,base]{\color{textcolor}{\rmfamily\fontsize{9.000000}{10.800000}\selectfont\catcode`\^=\active\def^{\ifmmode\sp\else\^{}\fi}\catcode`\%=\active\def%{\%}\CyclesMatchChunks{} \& \SharedVertices{}}}%
\end{pgfscope}%
\begin{pgfscope}%
\pgfsetrectcap%
\pgfsetroundjoin%
\pgfsetlinewidth{1.505625pt}%
\pgfsetstrokecolor{currentstroke4}%
\pgfsetdash{}{0pt}%
\pgfpathmoveto{\pgfqpoint{5.369911in}{2.020756in}}%
\pgfpathlineto{\pgfqpoint{5.494911in}{2.020756in}}%
\pgfpathlineto{\pgfqpoint{5.619911in}{2.020756in}}%
\pgfusepath{stroke}%
\end{pgfscope}%
\begin{pgfscope}%
\definecolor{textcolor}{rgb}{0.000000,0.000000,0.000000}%
\pgfsetstrokecolor{textcolor}%
\pgfsetfillcolor{textcolor}%
\pgftext[x=5.719911in,y=1.977006in,left,base]{\color{textcolor}{\rmfamily\fontsize{9.000000}{10.800000}\selectfont\catcode`\^=\active\def^{\ifmmode\sp\else\^{}\fi}\catcode`\%=\active\def%{\%}\Neighbors{} \& \MergeLinear{}}}%
\end{pgfscope}%
\begin{pgfscope}%
\pgfsetrectcap%
\pgfsetroundjoin%
\pgfsetlinewidth{1.505625pt}%
\pgfsetstrokecolor{currentstroke5}%
\pgfsetdash{}{0pt}%
\pgfpathmoveto{\pgfqpoint{5.369911in}{1.837285in}}%
\pgfpathlineto{\pgfqpoint{5.494911in}{1.837285in}}%
\pgfpathlineto{\pgfqpoint{5.619911in}{1.837285in}}%
\pgfusepath{stroke}%
\end{pgfscope}%
\begin{pgfscope}%
\definecolor{textcolor}{rgb}{0.000000,0.000000,0.000000}%
\pgfsetstrokecolor{textcolor}%
\pgfsetfillcolor{textcolor}%
\pgftext[x=5.719911in,y=1.793535in,left,base]{\color{textcolor}{\rmfamily\fontsize{9.000000}{10.800000}\selectfont\catcode`\^=\active\def^{\ifmmode\sp\else\^{}\fi}\catcode`\%=\active\def%{\%}\Neighbors{} \& \SharedVertices{}}}%
\end{pgfscope}%
\begin{pgfscope}%
\pgfsetrectcap%
\pgfsetroundjoin%
\pgfsetlinewidth{1.505625pt}%
\pgfsetstrokecolor{currentstroke6}%
\pgfsetdash{}{0pt}%
\pgfpathmoveto{\pgfqpoint{5.369911in}{1.650334in}}%
\pgfpathlineto{\pgfqpoint{5.494911in}{1.650334in}}%
\pgfpathlineto{\pgfqpoint{5.619911in}{1.650334in}}%
\pgfusepath{stroke}%
\end{pgfscope}%
\begin{pgfscope}%
\definecolor{textcolor}{rgb}{0.000000,0.000000,0.000000}%
\pgfsetstrokecolor{textcolor}%
\pgfsetfillcolor{textcolor}%
\pgftext[x=5.719911in,y=1.606584in,left,base]{\color{textcolor}{\rmfamily\fontsize{9.000000}{10.800000}\selectfont\catcode`\^=\active\def^{\ifmmode\sp\else\^{}\fi}\catcode`\%=\active\def%{\%}\NeighborsDegree{} \& \MergeLinear{}}}%
\end{pgfscope}%
\begin{pgfscope}%
\pgfsetrectcap%
\pgfsetroundjoin%
\pgfsetlinewidth{1.505625pt}%
\pgfsetstrokecolor{currentstroke7}%
\pgfsetdash{}{0pt}%
\pgfpathmoveto{\pgfqpoint{5.369911in}{1.463384in}}%
\pgfpathlineto{\pgfqpoint{5.494911in}{1.463384in}}%
\pgfpathlineto{\pgfqpoint{5.619911in}{1.463384in}}%
\pgfusepath{stroke}%
\end{pgfscope}%
\begin{pgfscope}%
\definecolor{textcolor}{rgb}{0.000000,0.000000,0.000000}%
\pgfsetstrokecolor{textcolor}%
\pgfsetfillcolor{textcolor}%
\pgftext[x=5.719911in,y=1.419634in,left,base]{\color{textcolor}{\rmfamily\fontsize{9.000000}{10.800000}\selectfont\catcode`\^=\active\def^{\ifmmode\sp\else\^{}\fi}\catcode`\%=\active\def%{\%}\NeighborsDegree{} \& \SharedVertices{}}}%
\end{pgfscope}%
\begin{pgfscope}%
\pgfsetrectcap%
\pgfsetroundjoin%
\pgfsetlinewidth{1.505625pt}%
\definecolor{currentstroke}{rgb}{0.498039,0.498039,0.498039}%
\pgfsetstrokecolor{currentstroke}%
\pgfsetdash{}{0pt}%
\pgfpathmoveto{\pgfqpoint{5.369911in}{1.276433in}}%
\pgfpathlineto{\pgfqpoint{5.494911in}{1.276433in}}%
\pgfpathlineto{\pgfqpoint{5.619911in}{1.276433in}}%
\pgfusepath{stroke}%
\end{pgfscope}%
\begin{pgfscope}%
\definecolor{textcolor}{rgb}{0.000000,0.000000,0.000000}%
\pgfsetstrokecolor{textcolor}%
\pgfsetfillcolor{textcolor}%
\pgftext[x=5.719911in,y=1.232683in,left,base]{\color{textcolor}{\rmfamily\fontsize{9.000000}{10.800000}\selectfont\catcode`\^=\active\def^{\ifmmode\sp\else\^{}\fi}\catcode`\%=\active\def%{\%}\None{} \& \MergeLinear{}}}%
\end{pgfscope}%
\begin{pgfscope}%
\pgfsetrectcap%
\pgfsetroundjoin%
\pgfsetlinewidth{1.505625pt}%
\definecolor{currentstroke}{rgb}{0.737255,0.741176,0.133333}%
\pgfsetstrokecolor{currentstroke}%
\pgfsetdash{}{0pt}%
\pgfpathmoveto{\pgfqpoint{5.369911in}{1.092962in}}%
\pgfpathlineto{\pgfqpoint{5.494911in}{1.092962in}}%
\pgfpathlineto{\pgfqpoint{5.619911in}{1.092962in}}%
\pgfusepath{stroke}%
\end{pgfscope}%
\begin{pgfscope}%
\definecolor{textcolor}{rgb}{0.000000,0.000000,0.000000}%
\pgfsetstrokecolor{textcolor}%
\pgfsetfillcolor{textcolor}%
\pgftext[x=5.719911in,y=1.049212in,left,base]{\color{textcolor}{\rmfamily\fontsize{9.000000}{10.800000}\selectfont\catcode`\^=\active\def^{\ifmmode\sp\else\^{}\fi}\catcode`\%=\active\def%{\%}\None{} \& \SharedVertices{}}}%
\end{pgfscope}%
\end{pgfpicture}%
\makeatother%
\endgroup%
}
		\caption[Mean runtime for~minimally rigid graphs (some)]{
			Mean running time to find some NAC-coloring for~minimally rigid graphs.}%
	\end{figure}%
\end{frame}

\begin{frame}
	\frametitle{Graphs with no/few NAC-colorings}
	\begin{itemize}
		\item
		      Naive search in not feasible.
		\item
		      Monochromatic classes reduce search space significantly.
		      \begin{itemize}
			      \item Hard to randomly find hard graphs.
		      \end{itemize}
		\item
		      Tens of vertices/monochromatic classes run in few seconds.
	\end{itemize}
\end{frame}

\begin{frame}
	\begin{figure}[thbp]
		\centering
		\scalebox{\BenchFigureScale}{%% Creator: Matplotlib, PGF backend
%%
%% To include the figure in your LaTeX document, write
%%   \input{<filename>.pgf}
%%
%% Make sure the required packages are loaded in your preamble
%%   \usepackage{pgf}
%%
%% Also ensure that all the required font packages are loaded; for instance,
%% the lmodern package is sometimes necessary when using math font.
%%   \usepackage{lmodern}
%%
%% Figures using additional raster images can only be included by \input if
%% they are in the same directory as the main LaTeX file. For loading figures
%% from other directories you can use the `import` package
%%   \usepackage{import}
%%
%% and then include the figures with
%%   \import{<path to file>}{<filename>.pgf}
%%
%% Matplotlib used the following preamble
%%   \def\mathdefault#1{#1}
%%   \everymath=\expandafter{\the\everymath\displaystyle}
%%   \IfFileExists{scrextend.sty}{
%%     \usepackage[fontsize=10.000000pt]{scrextend}
%%   }{
%%     \renewcommand{\normalsize}{\fontsize{10.000000}{12.000000}\selectfont}
%%     \normalsize
%%   }
%%   
%%   \ifdefined\pdftexversion\else  % non-pdftex case.
%%     \usepackage{fontspec}
%%     \setmainfont{DejaVuSans.ttf}[Path=\detokenize{/home/petr/Projects/PyRigi/.venv/lib/python3.12/site-packages/matplotlib/mpl-data/fonts/ttf/}]
%%     \setsansfont{DejaVuSans.ttf}[Path=\detokenize{/home/petr/Projects/PyRigi/.venv/lib/python3.12/site-packages/matplotlib/mpl-data/fonts/ttf/}]
%%     \setmonofont{DejaVuSansMono.ttf}[Path=\detokenize{/home/petr/Projects/PyRigi/.venv/lib/python3.12/site-packages/matplotlib/mpl-data/fonts/ttf/}]
%%   \fi
%%   \makeatletter\@ifpackageloaded{under\Score{}}{}{\usepackage[strings]{under\Score{}}}\makeatother
%%
\begingroup%
\makeatletter%
\begin{pgfpicture}%
\pgfpathrectangle{\pgfpointorigin}{\pgfqpoint{8.384376in}{2.841849in}}%
\pgfusepath{use as bounding box, clip}%
\begin{pgfscope}%
\pgfsetbuttcap%
\pgfsetmiterjoin%
\definecolor{currentfill}{rgb}{1.000000,1.000000,1.000000}%
\pgfsetfillcolor{currentfill}%
\pgfsetlinewidth{0.000000pt}%
\definecolor{currentstroke}{rgb}{1.000000,1.000000,1.000000}%
\pgfsetstrokecolor{currentstroke}%
\pgfsetdash{}{0pt}%
\pgfpathmoveto{\pgfqpoint{0.000000in}{0.000000in}}%
\pgfpathlineto{\pgfqpoint{8.384376in}{0.000000in}}%
\pgfpathlineto{\pgfqpoint{8.384376in}{2.841849in}}%
\pgfpathlineto{\pgfqpoint{0.000000in}{2.841849in}}%
\pgfpathlineto{\pgfqpoint{0.000000in}{0.000000in}}%
\pgfpathclose%
\pgfusepath{fill}%
\end{pgfscope}%
\begin{pgfscope}%
\pgfsetbuttcap%
\pgfsetmiterjoin%
\definecolor{currentfill}{rgb}{1.000000,1.000000,1.000000}%
\pgfsetfillcolor{currentfill}%
\pgfsetlinewidth{0.000000pt}%
\definecolor{currentstroke}{rgb}{0.000000,0.000000,0.000000}%
\pgfsetstrokecolor{currentstroke}%
\pgfsetstrokeopacity{0.000000}%
\pgfsetdash{}{0pt}%
\pgfpathmoveto{\pgfqpoint{0.588387in}{0.521603in}}%
\pgfpathlineto{\pgfqpoint{5.257411in}{0.521603in}}%
\pgfpathlineto{\pgfqpoint{5.257411in}{2.531888in}}%
\pgfpathlineto{\pgfqpoint{0.588387in}{2.531888in}}%
\pgfpathlineto{\pgfqpoint{0.588387in}{0.521603in}}%
\pgfpathclose%
\pgfusepath{fill}%
\end{pgfscope}%
\begin{pgfscope}%
\pgfsetbuttcap%
\pgfsetroundjoin%
\definecolor{currentfill}{rgb}{0.000000,0.000000,0.000000}%
\pgfsetfillcolor{currentfill}%
\pgfsetlinewidth{0.803000pt}%
\definecolor{currentstroke}{rgb}{0.000000,0.000000,0.000000}%
\pgfsetstrokecolor{currentstroke}%
\pgfsetdash{}{0pt}%
\pgfsys@defobject{currentmarker}{\pgfqpoint{0.000000in}{-0.048611in}}{\pgfqpoint{0.000000in}{0.000000in}}{%
\pgfpathmoveto{\pgfqpoint{0.000000in}{0.000000in}}%
\pgfpathlineto{\pgfqpoint{0.000000in}{-0.048611in}}%
\pgfusepath{stroke,fill}%
}%
\begin{pgfscope}%
\pgfsys@transformshift{0.985162in}{0.521603in}%
\pgfsys@useobject{currentmarker}{}%
\end{pgfscope}%
\end{pgfscope}%
\begin{pgfscope}%
\definecolor{textcolor}{rgb}{0.000000,0.000000,0.000000}%
\pgfsetstrokecolor{textcolor}%
\pgfsetfillcolor{textcolor}%
\pgftext[x=0.985162in,y=0.424381in,,top]{\color{textcolor}{\rmfamily\fontsize{10.000000}{12.000000}\selectfont\catcode`\^=\active\def^{\ifmmode\sp\else\^{}\fi}\catcode`\%=\active\def%{\%}$\mathdefault{12}$}}%
\end{pgfscope}%
\begin{pgfscope}%
\pgfsetbuttcap%
\pgfsetroundjoin%
\definecolor{currentfill}{rgb}{0.000000,0.000000,0.000000}%
\pgfsetfillcolor{currentfill}%
\pgfsetlinewidth{0.803000pt}%
\definecolor{currentstroke}{rgb}{0.000000,0.000000,0.000000}%
\pgfsetstrokecolor{currentstroke}%
\pgfsetdash{}{0pt}%
\pgfsys@defobject{currentmarker}{\pgfqpoint{0.000000in}{-0.048611in}}{\pgfqpoint{0.000000in}{0.000000in}}{%
\pgfpathmoveto{\pgfqpoint{0.000000in}{0.000000in}}%
\pgfpathlineto{\pgfqpoint{0.000000in}{-0.048611in}}%
\pgfusepath{stroke,fill}%
}%
\begin{pgfscope}%
\pgfsys@transformshift{1.538801in}{0.521603in}%
\pgfsys@useobject{currentmarker}{}%
\end{pgfscope}%
\end{pgfscope}%
\begin{pgfscope}%
\definecolor{textcolor}{rgb}{0.000000,0.000000,0.000000}%
\pgfsetstrokecolor{textcolor}%
\pgfsetfillcolor{textcolor}%
\pgftext[x=1.538801in,y=0.424381in,,top]{\color{textcolor}{\rmfamily\fontsize{10.000000}{12.000000}\selectfont\catcode`\^=\active\def^{\ifmmode\sp\else\^{}\fi}\catcode`\%=\active\def%{\%}$\mathdefault{18}$}}%
\end{pgfscope}%
\begin{pgfscope}%
\pgfsetbuttcap%
\pgfsetroundjoin%
\definecolor{currentfill}{rgb}{0.000000,0.000000,0.000000}%
\pgfsetfillcolor{currentfill}%
\pgfsetlinewidth{0.803000pt}%
\definecolor{currentstroke}{rgb}{0.000000,0.000000,0.000000}%
\pgfsetstrokecolor{currentstroke}%
\pgfsetdash{}{0pt}%
\pgfsys@defobject{currentmarker}{\pgfqpoint{0.000000in}{-0.048611in}}{\pgfqpoint{0.000000in}{0.000000in}}{%
\pgfpathmoveto{\pgfqpoint{0.000000in}{0.000000in}}%
\pgfpathlineto{\pgfqpoint{0.000000in}{-0.048611in}}%
\pgfusepath{stroke,fill}%
}%
\begin{pgfscope}%
\pgfsys@transformshift{2.092440in}{0.521603in}%
\pgfsys@useobject{currentmarker}{}%
\end{pgfscope}%
\end{pgfscope}%
\begin{pgfscope}%
\definecolor{textcolor}{rgb}{0.000000,0.000000,0.000000}%
\pgfsetstrokecolor{textcolor}%
\pgfsetfillcolor{textcolor}%
\pgftext[x=2.092440in,y=0.424381in,,top]{\color{textcolor}{\rmfamily\fontsize{10.000000}{12.000000}\selectfont\catcode`\^=\active\def^{\ifmmode\sp\else\^{}\fi}\catcode`\%=\active\def%{\%}$\mathdefault{24}$}}%
\end{pgfscope}%
\begin{pgfscope}%
\pgfsetbuttcap%
\pgfsetroundjoin%
\definecolor{currentfill}{rgb}{0.000000,0.000000,0.000000}%
\pgfsetfillcolor{currentfill}%
\pgfsetlinewidth{0.803000pt}%
\definecolor{currentstroke}{rgb}{0.000000,0.000000,0.000000}%
\pgfsetstrokecolor{currentstroke}%
\pgfsetdash{}{0pt}%
\pgfsys@defobject{currentmarker}{\pgfqpoint{0.000000in}{-0.048611in}}{\pgfqpoint{0.000000in}{0.000000in}}{%
\pgfpathmoveto{\pgfqpoint{0.000000in}{0.000000in}}%
\pgfpathlineto{\pgfqpoint{0.000000in}{-0.048611in}}%
\pgfusepath{stroke,fill}%
}%
\begin{pgfscope}%
\pgfsys@transformshift{2.646080in}{0.521603in}%
\pgfsys@useobject{currentmarker}{}%
\end{pgfscope}%
\end{pgfscope}%
\begin{pgfscope}%
\definecolor{textcolor}{rgb}{0.000000,0.000000,0.000000}%
\pgfsetstrokecolor{textcolor}%
\pgfsetfillcolor{textcolor}%
\pgftext[x=2.646080in,y=0.424381in,,top]{\color{textcolor}{\rmfamily\fontsize{10.000000}{12.000000}\selectfont\catcode`\^=\active\def^{\ifmmode\sp\else\^{}\fi}\catcode`\%=\active\def%{\%}$\mathdefault{30}$}}%
\end{pgfscope}%
\begin{pgfscope}%
\pgfsetbuttcap%
\pgfsetroundjoin%
\definecolor{currentfill}{rgb}{0.000000,0.000000,0.000000}%
\pgfsetfillcolor{currentfill}%
\pgfsetlinewidth{0.803000pt}%
\definecolor{currentstroke}{rgb}{0.000000,0.000000,0.000000}%
\pgfsetstrokecolor{currentstroke}%
\pgfsetdash{}{0pt}%
\pgfsys@defobject{currentmarker}{\pgfqpoint{0.000000in}{-0.048611in}}{\pgfqpoint{0.000000in}{0.000000in}}{%
\pgfpathmoveto{\pgfqpoint{0.000000in}{0.000000in}}%
\pgfpathlineto{\pgfqpoint{0.000000in}{-0.048611in}}%
\pgfusepath{stroke,fill}%
}%
\begin{pgfscope}%
\pgfsys@transformshift{3.199719in}{0.521603in}%
\pgfsys@useobject{currentmarker}{}%
\end{pgfscope}%
\end{pgfscope}%
\begin{pgfscope}%
\definecolor{textcolor}{rgb}{0.000000,0.000000,0.000000}%
\pgfsetstrokecolor{textcolor}%
\pgfsetfillcolor{textcolor}%
\pgftext[x=3.199719in,y=0.424381in,,top]{\color{textcolor}{\rmfamily\fontsize{10.000000}{12.000000}\selectfont\catcode`\^=\active\def^{\ifmmode\sp\else\^{}\fi}\catcode`\%=\active\def%{\%}$\mathdefault{36}$}}%
\end{pgfscope}%
\begin{pgfscope}%
\pgfsetbuttcap%
\pgfsetroundjoin%
\definecolor{currentfill}{rgb}{0.000000,0.000000,0.000000}%
\pgfsetfillcolor{currentfill}%
\pgfsetlinewidth{0.803000pt}%
\definecolor{currentstroke}{rgb}{0.000000,0.000000,0.000000}%
\pgfsetstrokecolor{currentstroke}%
\pgfsetdash{}{0pt}%
\pgfsys@defobject{currentmarker}{\pgfqpoint{0.000000in}{-0.048611in}}{\pgfqpoint{0.000000in}{0.000000in}}{%
\pgfpathmoveto{\pgfqpoint{0.000000in}{0.000000in}}%
\pgfpathlineto{\pgfqpoint{0.000000in}{-0.048611in}}%
\pgfusepath{stroke,fill}%
}%
\begin{pgfscope}%
\pgfsys@transformshift{3.753358in}{0.521603in}%
\pgfsys@useobject{currentmarker}{}%
\end{pgfscope}%
\end{pgfscope}%
\begin{pgfscope}%
\definecolor{textcolor}{rgb}{0.000000,0.000000,0.000000}%
\pgfsetstrokecolor{textcolor}%
\pgfsetfillcolor{textcolor}%
\pgftext[x=3.753358in,y=0.424381in,,top]{\color{textcolor}{\rmfamily\fontsize{10.000000}{12.000000}\selectfont\catcode`\^=\active\def^{\ifmmode\sp\else\^{}\fi}\catcode`\%=\active\def%{\%}$\mathdefault{42}$}}%
\end{pgfscope}%
\begin{pgfscope}%
\pgfsetbuttcap%
\pgfsetroundjoin%
\definecolor{currentfill}{rgb}{0.000000,0.000000,0.000000}%
\pgfsetfillcolor{currentfill}%
\pgfsetlinewidth{0.803000pt}%
\definecolor{currentstroke}{rgb}{0.000000,0.000000,0.000000}%
\pgfsetstrokecolor{currentstroke}%
\pgfsetdash{}{0pt}%
\pgfsys@defobject{currentmarker}{\pgfqpoint{0.000000in}{-0.048611in}}{\pgfqpoint{0.000000in}{0.000000in}}{%
\pgfpathmoveto{\pgfqpoint{0.000000in}{0.000000in}}%
\pgfpathlineto{\pgfqpoint{0.000000in}{-0.048611in}}%
\pgfusepath{stroke,fill}%
}%
\begin{pgfscope}%
\pgfsys@transformshift{4.306997in}{0.521603in}%
\pgfsys@useobject{currentmarker}{}%
\end{pgfscope}%
\end{pgfscope}%
\begin{pgfscope}%
\definecolor{textcolor}{rgb}{0.000000,0.000000,0.000000}%
\pgfsetstrokecolor{textcolor}%
\pgfsetfillcolor{textcolor}%
\pgftext[x=4.306997in,y=0.424381in,,top]{\color{textcolor}{\rmfamily\fontsize{10.000000}{12.000000}\selectfont\catcode`\^=\active\def^{\ifmmode\sp\else\^{}\fi}\catcode`\%=\active\def%{\%}$\mathdefault{48}$}}%
\end{pgfscope}%
\begin{pgfscope}%
\pgfsetbuttcap%
\pgfsetroundjoin%
\definecolor{currentfill}{rgb}{0.000000,0.000000,0.000000}%
\pgfsetfillcolor{currentfill}%
\pgfsetlinewidth{0.803000pt}%
\definecolor{currentstroke}{rgb}{0.000000,0.000000,0.000000}%
\pgfsetstrokecolor{currentstroke}%
\pgfsetdash{}{0pt}%
\pgfsys@defobject{currentmarker}{\pgfqpoint{0.000000in}{-0.048611in}}{\pgfqpoint{0.000000in}{0.000000in}}{%
\pgfpathmoveto{\pgfqpoint{0.000000in}{0.000000in}}%
\pgfpathlineto{\pgfqpoint{0.000000in}{-0.048611in}}%
\pgfusepath{stroke,fill}%
}%
\begin{pgfscope}%
\pgfsys@transformshift{4.860636in}{0.521603in}%
\pgfsys@useobject{currentmarker}{}%
\end{pgfscope}%
\end{pgfscope}%
\begin{pgfscope}%
\definecolor{textcolor}{rgb}{0.000000,0.000000,0.000000}%
\pgfsetstrokecolor{textcolor}%
\pgfsetfillcolor{textcolor}%
\pgftext[x=4.860636in,y=0.424381in,,top]{\color{textcolor}{\rmfamily\fontsize{10.000000}{12.000000}\selectfont\catcode`\^=\active\def^{\ifmmode\sp\else\^{}\fi}\catcode`\%=\active\def%{\%}$\mathdefault{54}$}}%
\end{pgfscope}%
\begin{pgfscope}%
\definecolor{textcolor}{rgb}{0.000000,0.000000,0.000000}%
\pgfsetstrokecolor{textcolor}%
\pgfsetfillcolor{textcolor}%
\pgftext[x=2.922899in,y=0.234413in,,top]{\color{textcolor}{\rmfamily\fontsize{10.000000}{12.000000}\selectfont\catcode`\^=\active\def^{\ifmmode\sp\else\^{}\fi}\catcode`\%=\active\def%{\%}Vertices}}%
\end{pgfscope}%
\begin{pgfscope}%
\pgfsetbuttcap%
\pgfsetroundjoin%
\definecolor{currentfill}{rgb}{0.000000,0.000000,0.000000}%
\pgfsetfillcolor{currentfill}%
\pgfsetlinewidth{0.803000pt}%
\definecolor{currentstroke}{rgb}{0.000000,0.000000,0.000000}%
\pgfsetstrokecolor{currentstroke}%
\pgfsetdash{}{0pt}%
\pgfsys@defobject{currentmarker}{\pgfqpoint{-0.048611in}{0.000000in}}{\pgfqpoint{-0.000000in}{0.000000in}}{%
\pgfpathmoveto{\pgfqpoint{-0.000000in}{0.000000in}}%
\pgfpathlineto{\pgfqpoint{-0.048611in}{0.000000in}}%
\pgfusepath{stroke,fill}%
}%
\begin{pgfscope}%
\pgfsys@transformshift{0.588387in}{1.028209in}%
\pgfsys@useobject{currentmarker}{}%
\end{pgfscope}%
\end{pgfscope}%
\begin{pgfscope}%
\definecolor{textcolor}{rgb}{0.000000,0.000000,0.000000}%
\pgfsetstrokecolor{textcolor}%
\pgfsetfillcolor{textcolor}%
\pgftext[x=0.289968in, y=0.975448in, left, base]{\color{textcolor}{\rmfamily\fontsize{10.000000}{12.000000}\selectfont\catcode`\^=\active\def^{\ifmmode\sp\else\^{}\fi}\catcode`\%=\active\def%{\%}$\mathdefault{10^{1}}$}}%
\end{pgfscope}%
\begin{pgfscope}%
\pgfsetbuttcap%
\pgfsetroundjoin%
\definecolor{currentfill}{rgb}{0.000000,0.000000,0.000000}%
\pgfsetfillcolor{currentfill}%
\pgfsetlinewidth{0.803000pt}%
\definecolor{currentstroke}{rgb}{0.000000,0.000000,0.000000}%
\pgfsetstrokecolor{currentstroke}%
\pgfsetdash{}{0pt}%
\pgfsys@defobject{currentmarker}{\pgfqpoint{-0.048611in}{0.000000in}}{\pgfqpoint{-0.000000in}{0.000000in}}{%
\pgfpathmoveto{\pgfqpoint{-0.000000in}{0.000000in}}%
\pgfpathlineto{\pgfqpoint{-0.048611in}{0.000000in}}%
\pgfusepath{stroke,fill}%
}%
\begin{pgfscope}%
\pgfsys@transformshift{0.588387in}{2.433286in}%
\pgfsys@useobject{currentmarker}{}%
\end{pgfscope}%
\end{pgfscope}%
\begin{pgfscope}%
\definecolor{textcolor}{rgb}{0.000000,0.000000,0.000000}%
\pgfsetstrokecolor{textcolor}%
\pgfsetfillcolor{textcolor}%
\pgftext[x=0.289968in, y=2.380524in, left, base]{\color{textcolor}{\rmfamily\fontsize{10.000000}{12.000000}\selectfont\catcode`\^=\active\def^{\ifmmode\sp\else\^{}\fi}\catcode`\%=\active\def%{\%}$\mathdefault{10^{2}}$}}%
\end{pgfscope}%
\begin{pgfscope}%
\pgfsetbuttcap%
\pgfsetroundjoin%
\definecolor{currentfill}{rgb}{0.000000,0.000000,0.000000}%
\pgfsetfillcolor{currentfill}%
\pgfsetlinewidth{0.602250pt}%
\definecolor{currentstroke}{rgb}{0.000000,0.000000,0.000000}%
\pgfsetstrokecolor{currentstroke}%
\pgfsetdash{}{0pt}%
\pgfsys@defobject{currentmarker}{\pgfqpoint{-0.027778in}{0.000000in}}{\pgfqpoint{-0.000000in}{0.000000in}}{%
\pgfpathmoveto{\pgfqpoint{-0.000000in}{0.000000in}}%
\pgfpathlineto{\pgfqpoint{-0.027778in}{0.000000in}}%
\pgfusepath{stroke,fill}%
}%
\begin{pgfscope}%
\pgfsys@transformshift{0.588387in}{0.605239in}%
\pgfsys@useobject{currentmarker}{}%
\end{pgfscope}%
\end{pgfscope}%
\begin{pgfscope}%
\pgfsetbuttcap%
\pgfsetroundjoin%
\definecolor{currentfill}{rgb}{0.000000,0.000000,0.000000}%
\pgfsetfillcolor{currentfill}%
\pgfsetlinewidth{0.602250pt}%
\definecolor{currentstroke}{rgb}{0.000000,0.000000,0.000000}%
\pgfsetstrokecolor{currentstroke}%
\pgfsetdash{}{0pt}%
\pgfsys@defobject{currentmarker}{\pgfqpoint{-0.027778in}{0.000000in}}{\pgfqpoint{-0.000000in}{0.000000in}}{%
\pgfpathmoveto{\pgfqpoint{-0.000000in}{0.000000in}}%
\pgfpathlineto{\pgfqpoint{-0.027778in}{0.000000in}}%
\pgfusepath{stroke,fill}%
}%
\begin{pgfscope}%
\pgfsys@transformshift{0.588387in}{0.716495in}%
\pgfsys@useobject{currentmarker}{}%
\end{pgfscope}%
\end{pgfscope}%
\begin{pgfscope}%
\pgfsetbuttcap%
\pgfsetroundjoin%
\definecolor{currentfill}{rgb}{0.000000,0.000000,0.000000}%
\pgfsetfillcolor{currentfill}%
\pgfsetlinewidth{0.602250pt}%
\definecolor{currentstroke}{rgb}{0.000000,0.000000,0.000000}%
\pgfsetstrokecolor{currentstroke}%
\pgfsetdash{}{0pt}%
\pgfsys@defobject{currentmarker}{\pgfqpoint{-0.027778in}{0.000000in}}{\pgfqpoint{-0.000000in}{0.000000in}}{%
\pgfpathmoveto{\pgfqpoint{-0.000000in}{0.000000in}}%
\pgfpathlineto{\pgfqpoint{-0.027778in}{0.000000in}}%
\pgfusepath{stroke,fill}%
}%
\begin{pgfscope}%
\pgfsys@transformshift{0.588387in}{0.810560in}%
\pgfsys@useobject{currentmarker}{}%
\end{pgfscope}%
\end{pgfscope}%
\begin{pgfscope}%
\pgfsetbuttcap%
\pgfsetroundjoin%
\definecolor{currentfill}{rgb}{0.000000,0.000000,0.000000}%
\pgfsetfillcolor{currentfill}%
\pgfsetlinewidth{0.602250pt}%
\definecolor{currentstroke}{rgb}{0.000000,0.000000,0.000000}%
\pgfsetstrokecolor{currentstroke}%
\pgfsetdash{}{0pt}%
\pgfsys@defobject{currentmarker}{\pgfqpoint{-0.027778in}{0.000000in}}{\pgfqpoint{-0.000000in}{0.000000in}}{%
\pgfpathmoveto{\pgfqpoint{-0.000000in}{0.000000in}}%
\pgfpathlineto{\pgfqpoint{-0.027778in}{0.000000in}}%
\pgfusepath{stroke,fill}%
}%
\begin{pgfscope}%
\pgfsys@transformshift{0.588387in}{0.892043in}%
\pgfsys@useobject{currentmarker}{}%
\end{pgfscope}%
\end{pgfscope}%
\begin{pgfscope}%
\pgfsetbuttcap%
\pgfsetroundjoin%
\definecolor{currentfill}{rgb}{0.000000,0.000000,0.000000}%
\pgfsetfillcolor{currentfill}%
\pgfsetlinewidth{0.602250pt}%
\definecolor{currentstroke}{rgb}{0.000000,0.000000,0.000000}%
\pgfsetstrokecolor{currentstroke}%
\pgfsetdash{}{0pt}%
\pgfsys@defobject{currentmarker}{\pgfqpoint{-0.027778in}{0.000000in}}{\pgfqpoint{-0.000000in}{0.000000in}}{%
\pgfpathmoveto{\pgfqpoint{-0.000000in}{0.000000in}}%
\pgfpathlineto{\pgfqpoint{-0.027778in}{0.000000in}}%
\pgfusepath{stroke,fill}%
}%
\begin{pgfscope}%
\pgfsys@transformshift{0.588387in}{0.963917in}%
\pgfsys@useobject{currentmarker}{}%
\end{pgfscope}%
\end{pgfscope}%
\begin{pgfscope}%
\pgfsetbuttcap%
\pgfsetroundjoin%
\definecolor{currentfill}{rgb}{0.000000,0.000000,0.000000}%
\pgfsetfillcolor{currentfill}%
\pgfsetlinewidth{0.602250pt}%
\definecolor{currentstroke}{rgb}{0.000000,0.000000,0.000000}%
\pgfsetstrokecolor{currentstroke}%
\pgfsetdash{}{0pt}%
\pgfsys@defobject{currentmarker}{\pgfqpoint{-0.027778in}{0.000000in}}{\pgfqpoint{-0.000000in}{0.000000in}}{%
\pgfpathmoveto{\pgfqpoint{-0.000000in}{0.000000in}}%
\pgfpathlineto{\pgfqpoint{-0.027778in}{0.000000in}}%
\pgfusepath{stroke,fill}%
}%
\begin{pgfscope}%
\pgfsys@transformshift{0.588387in}{1.451180in}%
\pgfsys@useobject{currentmarker}{}%
\end{pgfscope}%
\end{pgfscope}%
\begin{pgfscope}%
\pgfsetbuttcap%
\pgfsetroundjoin%
\definecolor{currentfill}{rgb}{0.000000,0.000000,0.000000}%
\pgfsetfillcolor{currentfill}%
\pgfsetlinewidth{0.602250pt}%
\definecolor{currentstroke}{rgb}{0.000000,0.000000,0.000000}%
\pgfsetstrokecolor{currentstroke}%
\pgfsetdash{}{0pt}%
\pgfsys@defobject{currentmarker}{\pgfqpoint{-0.027778in}{0.000000in}}{\pgfqpoint{-0.000000in}{0.000000in}}{%
\pgfpathmoveto{\pgfqpoint{-0.000000in}{0.000000in}}%
\pgfpathlineto{\pgfqpoint{-0.027778in}{0.000000in}}%
\pgfusepath{stroke,fill}%
}%
\begin{pgfscope}%
\pgfsys@transformshift{0.588387in}{1.698601in}%
\pgfsys@useobject{currentmarker}{}%
\end{pgfscope}%
\end{pgfscope}%
\begin{pgfscope}%
\pgfsetbuttcap%
\pgfsetroundjoin%
\definecolor{currentfill}{rgb}{0.000000,0.000000,0.000000}%
\pgfsetfillcolor{currentfill}%
\pgfsetlinewidth{0.602250pt}%
\definecolor{currentstroke}{rgb}{0.000000,0.000000,0.000000}%
\pgfsetstrokecolor{currentstroke}%
\pgfsetdash{}{0pt}%
\pgfsys@defobject{currentmarker}{\pgfqpoint{-0.027778in}{0.000000in}}{\pgfqpoint{-0.000000in}{0.000000in}}{%
\pgfpathmoveto{\pgfqpoint{-0.000000in}{0.000000in}}%
\pgfpathlineto{\pgfqpoint{-0.027778in}{0.000000in}}%
\pgfusepath{stroke,fill}%
}%
\begin{pgfscope}%
\pgfsys@transformshift{0.588387in}{1.874150in}%
\pgfsys@useobject{currentmarker}{}%
\end{pgfscope}%
\end{pgfscope}%
\begin{pgfscope}%
\pgfsetbuttcap%
\pgfsetroundjoin%
\definecolor{currentfill}{rgb}{0.000000,0.000000,0.000000}%
\pgfsetfillcolor{currentfill}%
\pgfsetlinewidth{0.602250pt}%
\definecolor{currentstroke}{rgb}{0.000000,0.000000,0.000000}%
\pgfsetstrokecolor{currentstroke}%
\pgfsetdash{}{0pt}%
\pgfsys@defobject{currentmarker}{\pgfqpoint{-0.027778in}{0.000000in}}{\pgfqpoint{-0.000000in}{0.000000in}}{%
\pgfpathmoveto{\pgfqpoint{-0.000000in}{0.000000in}}%
\pgfpathlineto{\pgfqpoint{-0.027778in}{0.000000in}}%
\pgfusepath{stroke,fill}%
}%
\begin{pgfscope}%
\pgfsys@transformshift{0.588387in}{2.010316in}%
\pgfsys@useobject{currentmarker}{}%
\end{pgfscope}%
\end{pgfscope}%
\begin{pgfscope}%
\pgfsetbuttcap%
\pgfsetroundjoin%
\definecolor{currentfill}{rgb}{0.000000,0.000000,0.000000}%
\pgfsetfillcolor{currentfill}%
\pgfsetlinewidth{0.602250pt}%
\definecolor{currentstroke}{rgb}{0.000000,0.000000,0.000000}%
\pgfsetstrokecolor{currentstroke}%
\pgfsetdash{}{0pt}%
\pgfsys@defobject{currentmarker}{\pgfqpoint{-0.027778in}{0.000000in}}{\pgfqpoint{-0.000000in}{0.000000in}}{%
\pgfpathmoveto{\pgfqpoint{-0.000000in}{0.000000in}}%
\pgfpathlineto{\pgfqpoint{-0.027778in}{0.000000in}}%
\pgfusepath{stroke,fill}%
}%
\begin{pgfscope}%
\pgfsys@transformshift{0.588387in}{2.121571in}%
\pgfsys@useobject{currentmarker}{}%
\end{pgfscope}%
\end{pgfscope}%
\begin{pgfscope}%
\pgfsetbuttcap%
\pgfsetroundjoin%
\definecolor{currentfill}{rgb}{0.000000,0.000000,0.000000}%
\pgfsetfillcolor{currentfill}%
\pgfsetlinewidth{0.602250pt}%
\definecolor{currentstroke}{rgb}{0.000000,0.000000,0.000000}%
\pgfsetstrokecolor{currentstroke}%
\pgfsetdash{}{0pt}%
\pgfsys@defobject{currentmarker}{\pgfqpoint{-0.027778in}{0.000000in}}{\pgfqpoint{-0.000000in}{0.000000in}}{%
\pgfpathmoveto{\pgfqpoint{-0.000000in}{0.000000in}}%
\pgfpathlineto{\pgfqpoint{-0.027778in}{0.000000in}}%
\pgfusepath{stroke,fill}%
}%
\begin{pgfscope}%
\pgfsys@transformshift{0.588387in}{2.215637in}%
\pgfsys@useobject{currentmarker}{}%
\end{pgfscope}%
\end{pgfscope}%
\begin{pgfscope}%
\pgfsetbuttcap%
\pgfsetroundjoin%
\definecolor{currentfill}{rgb}{0.000000,0.000000,0.000000}%
\pgfsetfillcolor{currentfill}%
\pgfsetlinewidth{0.602250pt}%
\definecolor{currentstroke}{rgb}{0.000000,0.000000,0.000000}%
\pgfsetstrokecolor{currentstroke}%
\pgfsetdash{}{0pt}%
\pgfsys@defobject{currentmarker}{\pgfqpoint{-0.027778in}{0.000000in}}{\pgfqpoint{-0.000000in}{0.000000in}}{%
\pgfpathmoveto{\pgfqpoint{-0.000000in}{0.000000in}}%
\pgfpathlineto{\pgfqpoint{-0.027778in}{0.000000in}}%
\pgfusepath{stroke,fill}%
}%
\begin{pgfscope}%
\pgfsys@transformshift{0.588387in}{2.297120in}%
\pgfsys@useobject{currentmarker}{}%
\end{pgfscope}%
\end{pgfscope}%
\begin{pgfscope}%
\pgfsetbuttcap%
\pgfsetroundjoin%
\definecolor{currentfill}{rgb}{0.000000,0.000000,0.000000}%
\pgfsetfillcolor{currentfill}%
\pgfsetlinewidth{0.602250pt}%
\definecolor{currentstroke}{rgb}{0.000000,0.000000,0.000000}%
\pgfsetstrokecolor{currentstroke}%
\pgfsetdash{}{0pt}%
\pgfsys@defobject{currentmarker}{\pgfqpoint{-0.027778in}{0.000000in}}{\pgfqpoint{-0.000000in}{0.000000in}}{%
\pgfpathmoveto{\pgfqpoint{-0.000000in}{0.000000in}}%
\pgfpathlineto{\pgfqpoint{-0.027778in}{0.000000in}}%
\pgfusepath{stroke,fill}%
}%
\begin{pgfscope}%
\pgfsys@transformshift{0.588387in}{2.368993in}%
\pgfsys@useobject{currentmarker}{}%
\end{pgfscope}%
\end{pgfscope}%
\begin{pgfscope}%
\definecolor{textcolor}{rgb}{0.000000,0.000000,0.000000}%
\pgfsetstrokecolor{textcolor}%
\pgfsetfillcolor{textcolor}%
\pgftext[x=0.234413in,y=1.526746in,,bottom,rotate=90.000000]{\color{textcolor}{\rmfamily\fontsize{10.000000}{12.000000}\selectfont\catcode`\^=\active\def^{\ifmmode\sp\else\^{}\fi}\catcode`\%=\active\def%{\%}Time [ms]}}%
\end{pgfscope}%
\begin{pgfscope}%
\pgfpathrectangle{\pgfqpoint{0.588387in}{0.521603in}}{\pgfqpoint{4.669024in}{2.010285in}}%
\pgfusepath{clip}%
\pgfsetrectcap%
\pgfsetroundjoin%
\pgfsetlinewidth{1.505625pt}%
\pgfsetstrokecolor{currentstroke1}%
\pgfsetdash{}{0pt}%
\pgfpathmoveto{\pgfqpoint{0.800616in}{0.673045in}}%
\pgfpathlineto{\pgfqpoint{0.892889in}{0.744736in}}%
\pgfpathlineto{\pgfqpoint{0.985162in}{0.805200in}}%
\pgfpathlineto{\pgfqpoint{1.077435in}{0.888954in}}%
\pgfpathlineto{\pgfqpoint{1.169708in}{0.957098in}}%
\pgfpathlineto{\pgfqpoint{1.261982in}{1.034281in}}%
\pgfpathlineto{\pgfqpoint{1.354255in}{1.114555in}}%
\pgfpathlineto{\pgfqpoint{1.446528in}{1.162420in}}%
\pgfpathlineto{\pgfqpoint{1.538801in}{1.238739in}}%
\pgfpathlineto{\pgfqpoint{1.631074in}{1.299564in}}%
\pgfpathlineto{\pgfqpoint{1.723348in}{1.315776in}}%
\pgfpathlineto{\pgfqpoint{1.815621in}{1.351649in}}%
\pgfpathlineto{\pgfqpoint{1.907894in}{1.453007in}}%
\pgfpathlineto{\pgfqpoint{2.000167in}{1.438540in}}%
\pgfpathlineto{\pgfqpoint{2.092440in}{1.473867in}}%
\pgfpathlineto{\pgfqpoint{2.184714in}{1.530064in}}%
\pgfpathlineto{\pgfqpoint{2.276987in}{1.561417in}}%
\pgfpathlineto{\pgfqpoint{2.369260in}{1.603959in}}%
\pgfpathlineto{\pgfqpoint{2.461533in}{1.666230in}}%
\pgfpathlineto{\pgfqpoint{2.553806in}{1.707486in}}%
\pgfpathlineto{\pgfqpoint{2.646080in}{1.749319in}}%
\pgfpathlineto{\pgfqpoint{2.738353in}{1.797183in}}%
\pgfpathlineto{\pgfqpoint{2.830626in}{1.816935in}}%
\pgfpathlineto{\pgfqpoint{2.922899in}{1.866084in}}%
\pgfpathlineto{\pgfqpoint{3.015172in}{1.834521in}}%
\pgfpathlineto{\pgfqpoint{3.107446in}{1.897643in}}%
\pgfpathlineto{\pgfqpoint{3.199719in}{1.939413in}}%
\pgfpathlineto{\pgfqpoint{3.291992in}{1.921395in}}%
\pgfpathlineto{\pgfqpoint{3.384265in}{1.998496in}}%
\pgfpathlineto{\pgfqpoint{3.476538in}{2.032930in}}%
\pgfpathlineto{\pgfqpoint{3.568812in}{2.031007in}}%
\pgfpathlineto{\pgfqpoint{3.661085in}{2.048310in}}%
\pgfpathlineto{\pgfqpoint{3.753358in}{2.098776in}}%
\pgfpathlineto{\pgfqpoint{3.845631in}{2.080241in}}%
\pgfpathlineto{\pgfqpoint{3.937904in}{2.098248in}}%
\pgfpathlineto{\pgfqpoint{4.030178in}{2.125667in}}%
\pgfpathlineto{\pgfqpoint{4.122451in}{2.118763in}}%
\pgfpathlineto{\pgfqpoint{4.214724in}{2.170794in}}%
\pgfpathlineto{\pgfqpoint{4.306997in}{2.219236in}}%
\pgfpathlineto{\pgfqpoint{4.399270in}{2.235704in}}%
\pgfpathlineto{\pgfqpoint{4.491544in}{2.264033in}}%
\pgfpathlineto{\pgfqpoint{4.583817in}{2.298063in}}%
\pgfpathlineto{\pgfqpoint{4.676090in}{2.312931in}}%
\pgfpathlineto{\pgfqpoint{4.768363in}{2.297040in}}%
\pgfpathlineto{\pgfqpoint{4.860636in}{2.256833in}}%
\pgfpathlineto{\pgfqpoint{4.952910in}{2.356749in}}%
\pgfpathlineto{\pgfqpoint{5.045183in}{2.406253in}}%
\pgfusepath{stroke}%
\end{pgfscope}%
\begin{pgfscope}%
\pgfpathrectangle{\pgfqpoint{0.588387in}{0.521603in}}{\pgfqpoint{4.669024in}{2.010285in}}%
\pgfusepath{clip}%
\pgfsetrectcap%
\pgfsetroundjoin%
\pgfsetlinewidth{1.505625pt}%
\pgfsetstrokecolor{currentstroke2}%
\pgfsetdash{}{0pt}%
\pgfpathmoveto{\pgfqpoint{0.800616in}{0.671273in}}%
\pgfpathlineto{\pgfqpoint{0.892889in}{0.734428in}}%
\pgfpathlineto{\pgfqpoint{0.985162in}{0.831096in}}%
\pgfpathlineto{\pgfqpoint{1.077435in}{0.954948in}}%
\pgfpathlineto{\pgfqpoint{1.169708in}{0.989152in}}%
\pgfpathlineto{\pgfqpoint{1.261982in}{1.059723in}}%
\pgfpathlineto{\pgfqpoint{1.354255in}{1.114833in}}%
\pgfpathlineto{\pgfqpoint{1.446528in}{1.179324in}}%
\pgfpathlineto{\pgfqpoint{1.538801in}{1.250297in}}%
\pgfpathlineto{\pgfqpoint{1.631074in}{1.334899in}}%
\pgfpathlineto{\pgfqpoint{1.723348in}{1.311924in}}%
\pgfpathlineto{\pgfqpoint{1.815621in}{1.355230in}}%
\pgfpathlineto{\pgfqpoint{1.907894in}{1.454830in}}%
\pgfpathlineto{\pgfqpoint{2.000167in}{1.458157in}}%
\pgfpathlineto{\pgfqpoint{2.092440in}{1.479498in}}%
\pgfpathlineto{\pgfqpoint{2.184714in}{1.524284in}}%
\pgfpathlineto{\pgfqpoint{2.276987in}{1.571019in}}%
\pgfpathlineto{\pgfqpoint{2.369260in}{1.588823in}}%
\pgfpathlineto{\pgfqpoint{2.461533in}{1.672418in}}%
\pgfpathlineto{\pgfqpoint{2.553806in}{1.725461in}}%
\pgfpathlineto{\pgfqpoint{2.646080in}{1.752679in}}%
\pgfpathlineto{\pgfqpoint{2.738353in}{1.782187in}}%
\pgfpathlineto{\pgfqpoint{2.830626in}{1.799946in}}%
\pgfpathlineto{\pgfqpoint{2.922899in}{1.829430in}}%
\pgfpathlineto{\pgfqpoint{3.015172in}{1.836392in}}%
\pgfpathlineto{\pgfqpoint{3.107446in}{1.881654in}}%
\pgfpathlineto{\pgfqpoint{3.199719in}{1.950798in}}%
\pgfpathlineto{\pgfqpoint{3.291992in}{1.921819in}}%
\pgfpathlineto{\pgfqpoint{3.384265in}{1.999270in}}%
\pgfpathlineto{\pgfqpoint{3.476538in}{2.032795in}}%
\pgfpathlineto{\pgfqpoint{3.568812in}{2.035804in}}%
\pgfpathlineto{\pgfqpoint{3.661085in}{2.045174in}}%
\pgfpathlineto{\pgfqpoint{3.753358in}{2.096404in}}%
\pgfpathlineto{\pgfqpoint{3.845631in}{2.085713in}}%
\pgfpathlineto{\pgfqpoint{3.937904in}{2.098459in}}%
\pgfpathlineto{\pgfqpoint{4.030178in}{2.128750in}}%
\pgfpathlineto{\pgfqpoint{4.122451in}{2.118378in}}%
\pgfpathlineto{\pgfqpoint{4.214724in}{2.160763in}}%
\pgfpathlineto{\pgfqpoint{4.306997in}{2.202864in}}%
\pgfpathlineto{\pgfqpoint{4.399270in}{2.231925in}}%
\pgfpathlineto{\pgfqpoint{4.491544in}{2.253035in}}%
\pgfpathlineto{\pgfqpoint{4.583817in}{2.284192in}}%
\pgfpathlineto{\pgfqpoint{4.676090in}{2.300960in}}%
\pgfpathlineto{\pgfqpoint{4.768363in}{2.297906in}}%
\pgfpathlineto{\pgfqpoint{4.860636in}{2.256009in}}%
\pgfpathlineto{\pgfqpoint{4.952910in}{2.354708in}}%
\pgfpathlineto{\pgfqpoint{5.045183in}{2.402215in}}%
\pgfusepath{stroke}%
\end{pgfscope}%
\begin{pgfscope}%
\pgfpathrectangle{\pgfqpoint{0.588387in}{0.521603in}}{\pgfqpoint{4.669024in}{2.010285in}}%
\pgfusepath{clip}%
\pgfsetrectcap%
\pgfsetroundjoin%
\pgfsetlinewidth{1.505625pt}%
\pgfsetstrokecolor{currentstroke3}%
\pgfsetdash{}{0pt}%
\pgfpathmoveto{\pgfqpoint{0.800616in}{0.612980in}}%
\pgfpathlineto{\pgfqpoint{0.892889in}{0.685066in}}%
\pgfpathlineto{\pgfqpoint{0.985162in}{0.730364in}}%
\pgfpathlineto{\pgfqpoint{1.077435in}{0.814259in}}%
\pgfpathlineto{\pgfqpoint{1.169708in}{0.938611in}}%
\pgfpathlineto{\pgfqpoint{1.261982in}{0.867984in}}%
\pgfpathlineto{\pgfqpoint{1.354255in}{0.938948in}}%
\pgfpathlineto{\pgfqpoint{1.446528in}{0.983926in}}%
\pgfpathlineto{\pgfqpoint{1.538801in}{0.993689in}}%
\pgfpathlineto{\pgfqpoint{1.631074in}{1.117198in}}%
\pgfpathlineto{\pgfqpoint{1.723348in}{1.140481in}}%
\pgfpathlineto{\pgfqpoint{1.815621in}{1.160457in}}%
\pgfpathlineto{\pgfqpoint{1.907894in}{1.225021in}}%
\pgfpathlineto{\pgfqpoint{2.000167in}{1.311955in}}%
\pgfpathlineto{\pgfqpoint{2.092440in}{1.356300in}}%
\pgfpathlineto{\pgfqpoint{2.184714in}{1.378696in}}%
\pgfpathlineto{\pgfqpoint{2.276987in}{1.388579in}}%
\pgfpathlineto{\pgfqpoint{2.369260in}{1.451789in}}%
\pgfpathlineto{\pgfqpoint{2.461533in}{1.492466in}}%
\pgfpathlineto{\pgfqpoint{2.553806in}{1.525263in}}%
\pgfpathlineto{\pgfqpoint{2.646080in}{1.523464in}}%
\pgfpathlineto{\pgfqpoint{2.738353in}{1.624512in}}%
\pgfpathlineto{\pgfqpoint{2.830626in}{1.635437in}}%
\pgfpathlineto{\pgfqpoint{2.922899in}{1.670644in}}%
\pgfpathlineto{\pgfqpoint{3.015172in}{1.706884in}}%
\pgfpathlineto{\pgfqpoint{3.107446in}{1.728955in}}%
\pgfpathlineto{\pgfqpoint{3.199719in}{1.767132in}}%
\pgfpathlineto{\pgfqpoint{3.291992in}{1.808669in}}%
\pgfpathlineto{\pgfqpoint{3.384265in}{1.827242in}}%
\pgfpathlineto{\pgfqpoint{3.476538in}{1.871834in}}%
\pgfpathlineto{\pgfqpoint{3.568812in}{1.978887in}}%
\pgfpathlineto{\pgfqpoint{3.661085in}{1.894700in}}%
\pgfpathlineto{\pgfqpoint{3.753358in}{1.934525in}}%
\pgfpathlineto{\pgfqpoint{3.845631in}{1.965507in}}%
\pgfpathlineto{\pgfqpoint{3.937904in}{2.039257in}}%
\pgfpathlineto{\pgfqpoint{4.030178in}{2.009827in}}%
\pgfpathlineto{\pgfqpoint{4.122451in}{1.994240in}}%
\pgfpathlineto{\pgfqpoint{4.214724in}{2.032015in}}%
\pgfpathlineto{\pgfqpoint{4.306997in}{2.098037in}}%
\pgfpathlineto{\pgfqpoint{4.399270in}{2.075864in}}%
\pgfpathlineto{\pgfqpoint{4.491544in}{2.111629in}}%
\pgfpathlineto{\pgfqpoint{4.583817in}{2.252180in}}%
\pgfpathlineto{\pgfqpoint{4.676090in}{2.168252in}}%
\pgfusepath{stroke}%
\end{pgfscope}%
\begin{pgfscope}%
\pgfpathrectangle{\pgfqpoint{0.588387in}{0.521603in}}{\pgfqpoint{4.669024in}{2.010285in}}%
\pgfusepath{clip}%
\pgfsetrectcap%
\pgfsetroundjoin%
\pgfsetlinewidth{1.505625pt}%
\pgfsetstrokecolor{currentstroke4}%
\pgfsetdash{}{0pt}%
\pgfpathmoveto{\pgfqpoint{0.800616in}{0.662545in}}%
\pgfpathlineto{\pgfqpoint{0.892889in}{0.713292in}}%
\pgfpathlineto{\pgfqpoint{0.985162in}{0.800027in}}%
\pgfpathlineto{\pgfqpoint{1.077435in}{0.876831in}}%
\pgfpathlineto{\pgfqpoint{1.169708in}{0.965626in}}%
\pgfpathlineto{\pgfqpoint{1.261982in}{0.987850in}}%
\pgfpathlineto{\pgfqpoint{1.354255in}{1.055069in}}%
\pgfpathlineto{\pgfqpoint{1.446528in}{1.100354in}}%
\pgfpathlineto{\pgfqpoint{1.538801in}{1.164619in}}%
\pgfpathlineto{\pgfqpoint{1.631074in}{1.235054in}}%
\pgfpathlineto{\pgfqpoint{1.723348in}{1.259138in}}%
\pgfpathlineto{\pgfqpoint{1.815621in}{1.295837in}}%
\pgfpathlineto{\pgfqpoint{1.907894in}{1.381951in}}%
\pgfpathlineto{\pgfqpoint{2.000167in}{1.399137in}}%
\pgfpathlineto{\pgfqpoint{2.092440in}{1.448581in}}%
\pgfpathlineto{\pgfqpoint{2.184714in}{1.488281in}}%
\pgfpathlineto{\pgfqpoint{2.276987in}{1.522646in}}%
\pgfpathlineto{\pgfqpoint{2.369260in}{1.562562in}}%
\pgfpathlineto{\pgfqpoint{2.461533in}{1.608692in}}%
\pgfpathlineto{\pgfqpoint{2.553806in}{1.658133in}}%
\pgfpathlineto{\pgfqpoint{2.646080in}{1.665479in}}%
\pgfpathlineto{\pgfqpoint{2.738353in}{1.728812in}}%
\pgfpathlineto{\pgfqpoint{2.830626in}{1.753052in}}%
\pgfpathlineto{\pgfqpoint{2.922899in}{1.784244in}}%
\pgfpathlineto{\pgfqpoint{3.015172in}{1.791008in}}%
\pgfpathlineto{\pgfqpoint{3.107446in}{1.836360in}}%
\pgfpathlineto{\pgfqpoint{3.199719in}{1.881233in}}%
\pgfpathlineto{\pgfqpoint{3.291992in}{1.881655in}}%
\pgfpathlineto{\pgfqpoint{3.384265in}{1.953791in}}%
\pgfpathlineto{\pgfqpoint{3.476538in}{1.987917in}}%
\pgfpathlineto{\pgfqpoint{3.568812in}{2.029259in}}%
\pgfpathlineto{\pgfqpoint{3.661085in}{2.001295in}}%
\pgfpathlineto{\pgfqpoint{3.753358in}{2.041518in}}%
\pgfpathlineto{\pgfqpoint{3.845631in}{2.044592in}}%
\pgfpathlineto{\pgfqpoint{3.937904in}{2.088556in}}%
\pgfpathlineto{\pgfqpoint{4.030178in}{2.091568in}}%
\pgfpathlineto{\pgfqpoint{4.122451in}{2.076989in}}%
\pgfpathlineto{\pgfqpoint{4.214724in}{2.116879in}}%
\pgfpathlineto{\pgfqpoint{4.306997in}{2.175929in}}%
\pgfpathlineto{\pgfqpoint{4.399270in}{2.181654in}}%
\pgfpathlineto{\pgfqpoint{4.491544in}{2.206218in}}%
\pgfpathlineto{\pgfqpoint{4.583817in}{2.291881in}}%
\pgfpathlineto{\pgfqpoint{4.676090in}{2.266024in}}%
\pgfpathlineto{\pgfqpoint{4.768363in}{2.302600in}}%
\pgfpathlineto{\pgfqpoint{4.860636in}{2.253449in}}%
\pgfpathlineto{\pgfqpoint{4.952910in}{2.345433in}}%
\pgfpathlineto{\pgfqpoint{5.045183in}{2.395528in}}%
\pgfusepath{stroke}%
\end{pgfscope}%
\begin{pgfscope}%
\pgfpathrectangle{\pgfqpoint{0.588387in}{0.521603in}}{\pgfqpoint{4.669024in}{2.010285in}}%
\pgfusepath{clip}%
\pgfsetrectcap%
\pgfsetroundjoin%
\pgfsetlinewidth{1.505625pt}%
\pgfsetstrokecolor{currentstroke5}%
\pgfsetdash{}{0pt}%
\pgfpathmoveto{\pgfqpoint{0.800616in}{0.657396in}}%
\pgfpathlineto{\pgfqpoint{0.892889in}{0.718124in}}%
\pgfpathlineto{\pgfqpoint{0.985162in}{0.797713in}}%
\pgfpathlineto{\pgfqpoint{1.077435in}{0.901360in}}%
\pgfpathlineto{\pgfqpoint{1.169708in}{0.970019in}}%
\pgfpathlineto{\pgfqpoint{1.261982in}{1.013698in}}%
\pgfpathlineto{\pgfqpoint{1.354255in}{1.077991in}}%
\pgfpathlineto{\pgfqpoint{1.446528in}{1.127137in}}%
\pgfpathlineto{\pgfqpoint{1.538801in}{1.194177in}}%
\pgfpathlineto{\pgfqpoint{1.631074in}{1.286118in}}%
\pgfpathlineto{\pgfqpoint{1.723348in}{1.260103in}}%
\pgfpathlineto{\pgfqpoint{1.815621in}{1.302036in}}%
\pgfpathlineto{\pgfqpoint{1.907894in}{1.420686in}}%
\pgfpathlineto{\pgfqpoint{2.000167in}{1.392483in}}%
\pgfpathlineto{\pgfqpoint{2.092440in}{1.445971in}}%
\pgfpathlineto{\pgfqpoint{2.184714in}{1.494317in}}%
\pgfpathlineto{\pgfqpoint{2.276987in}{1.521151in}}%
\pgfpathlineto{\pgfqpoint{2.369260in}{1.559862in}}%
\pgfpathlineto{\pgfqpoint{2.461533in}{1.617351in}}%
\pgfpathlineto{\pgfqpoint{2.553806in}{1.661324in}}%
\pgfpathlineto{\pgfqpoint{2.646080in}{1.673137in}}%
\pgfpathlineto{\pgfqpoint{2.738353in}{1.740184in}}%
\pgfpathlineto{\pgfqpoint{2.830626in}{1.759713in}}%
\pgfpathlineto{\pgfqpoint{2.922899in}{1.784868in}}%
\pgfpathlineto{\pgfqpoint{3.015172in}{1.788029in}}%
\pgfpathlineto{\pgfqpoint{3.107446in}{1.838256in}}%
\pgfpathlineto{\pgfqpoint{3.199719in}{1.887335in}}%
\pgfpathlineto{\pgfqpoint{3.291992in}{1.881314in}}%
\pgfpathlineto{\pgfqpoint{3.384265in}{1.950947in}}%
\pgfpathlineto{\pgfqpoint{3.476538in}{1.982923in}}%
\pgfpathlineto{\pgfqpoint{3.568812in}{2.031969in}}%
\pgfpathlineto{\pgfqpoint{3.661085in}{1.993524in}}%
\pgfpathlineto{\pgfqpoint{3.753358in}{2.046103in}}%
\pgfpathlineto{\pgfqpoint{3.845631in}{2.045697in}}%
\pgfpathlineto{\pgfqpoint{3.937904in}{2.092741in}}%
\pgfpathlineto{\pgfqpoint{4.030178in}{2.088234in}}%
\pgfpathlineto{\pgfqpoint{4.122451in}{2.078364in}}%
\pgfpathlineto{\pgfqpoint{4.214724in}{2.125778in}}%
\pgfpathlineto{\pgfqpoint{4.306997in}{2.177368in}}%
\pgfpathlineto{\pgfqpoint{4.399270in}{2.182075in}}%
\pgfpathlineto{\pgfqpoint{4.491544in}{2.224607in}}%
\pgfpathlineto{\pgfqpoint{4.583817in}{2.300922in}}%
\pgfpathlineto{\pgfqpoint{4.676090in}{2.272815in}}%
\pgfpathlineto{\pgfqpoint{4.768363in}{2.321623in}}%
\pgfpathlineto{\pgfqpoint{4.860636in}{2.252233in}}%
\pgfpathlineto{\pgfqpoint{4.952910in}{2.382981in}}%
\pgfpathlineto{\pgfqpoint{5.045183in}{2.440512in}}%
\pgfusepath{stroke}%
\end{pgfscope}%
\begin{pgfscope}%
\pgfpathrectangle{\pgfqpoint{0.588387in}{0.521603in}}{\pgfqpoint{4.669024in}{2.010285in}}%
\pgfusepath{clip}%
\pgfsetrectcap%
\pgfsetroundjoin%
\pgfsetlinewidth{1.505625pt}%
\pgfsetstrokecolor{currentstroke6}%
\pgfsetdash{}{0pt}%
\pgfpathmoveto{\pgfqpoint{0.800616in}{0.672575in}}%
\pgfpathlineto{\pgfqpoint{0.892889in}{0.713151in}}%
\pgfpathlineto{\pgfqpoint{0.985162in}{0.806003in}}%
\pgfpathlineto{\pgfqpoint{1.077435in}{0.876995in}}%
\pgfpathlineto{\pgfqpoint{1.169708in}{0.945752in}}%
\pgfpathlineto{\pgfqpoint{1.261982in}{0.981625in}}%
\pgfpathlineto{\pgfqpoint{1.354255in}{1.059143in}}%
\pgfpathlineto{\pgfqpoint{1.446528in}{1.112166in}}%
\pgfpathlineto{\pgfqpoint{1.538801in}{1.166325in}}%
\pgfpathlineto{\pgfqpoint{1.631074in}{1.232440in}}%
\pgfpathlineto{\pgfqpoint{1.723348in}{1.257044in}}%
\pgfpathlineto{\pgfqpoint{1.815621in}{1.295049in}}%
\pgfpathlineto{\pgfqpoint{1.907894in}{1.381951in}}%
\pgfpathlineto{\pgfqpoint{2.000167in}{1.404265in}}%
\pgfpathlineto{\pgfqpoint{2.092440in}{1.440096in}}%
\pgfpathlineto{\pgfqpoint{2.184714in}{1.510856in}}%
\pgfpathlineto{\pgfqpoint{2.276987in}{1.524137in}}%
\pgfpathlineto{\pgfqpoint{2.369260in}{1.560525in}}%
\pgfpathlineto{\pgfqpoint{2.461533in}{1.616304in}}%
\pgfpathlineto{\pgfqpoint{2.553806in}{1.647743in}}%
\pgfpathlineto{\pgfqpoint{2.646080in}{1.689172in}}%
\pgfpathlineto{\pgfqpoint{2.738353in}{1.733052in}}%
\pgfpathlineto{\pgfqpoint{2.830626in}{1.758700in}}%
\pgfpathlineto{\pgfqpoint{2.922899in}{1.770429in}}%
\pgfpathlineto{\pgfqpoint{3.015172in}{1.795970in}}%
\pgfpathlineto{\pgfqpoint{3.107446in}{1.840597in}}%
\pgfpathlineto{\pgfqpoint{3.199719in}{1.884487in}}%
\pgfpathlineto{\pgfqpoint{3.291992in}{1.880373in}}%
\pgfpathlineto{\pgfqpoint{3.384265in}{1.953208in}}%
\pgfpathlineto{\pgfqpoint{3.476538in}{1.983294in}}%
\pgfpathlineto{\pgfqpoint{3.568812in}{2.028444in}}%
\pgfpathlineto{\pgfqpoint{3.661085in}{1.998300in}}%
\pgfpathlineto{\pgfqpoint{3.753358in}{2.042973in}}%
\pgfpathlineto{\pgfqpoint{3.845631in}{2.046636in}}%
\pgfpathlineto{\pgfqpoint{3.937904in}{2.090592in}}%
\pgfpathlineto{\pgfqpoint{4.030178in}{2.089194in}}%
\pgfpathlineto{\pgfqpoint{4.122451in}{2.078204in}}%
\pgfpathlineto{\pgfqpoint{4.214724in}{2.128771in}}%
\pgfpathlineto{\pgfqpoint{4.306997in}{2.188306in}}%
\pgfpathlineto{\pgfqpoint{4.399270in}{2.181205in}}%
\pgfpathlineto{\pgfqpoint{4.491544in}{2.223011in}}%
\pgfpathlineto{\pgfqpoint{4.583817in}{2.312931in}}%
\pgfpathlineto{\pgfqpoint{4.676090in}{2.279231in}}%
\pgfpathlineto{\pgfqpoint{4.768363in}{2.323611in}}%
\pgfpathlineto{\pgfqpoint{4.860636in}{2.254276in}}%
\pgfpathlineto{\pgfqpoint{4.952910in}{2.368279in}}%
\pgfpathlineto{\pgfqpoint{5.045183in}{2.413470in}}%
\pgfusepath{stroke}%
\end{pgfscope}%
\begin{pgfscope}%
\pgfpathrectangle{\pgfqpoint{0.588387in}{0.521603in}}{\pgfqpoint{4.669024in}{2.010285in}}%
\pgfusepath{clip}%
\pgfsetrectcap%
\pgfsetroundjoin%
\pgfsetlinewidth{1.505625pt}%
\pgfsetstrokecolor{currentstroke7}%
\pgfsetdash{}{0pt}%
\pgfpathmoveto{\pgfqpoint{0.800616in}{0.642233in}}%
\pgfpathlineto{\pgfqpoint{0.892889in}{0.708008in}}%
\pgfpathlineto{\pgfqpoint{0.985162in}{0.803602in}}%
\pgfpathlineto{\pgfqpoint{1.077435in}{0.873071in}}%
\pgfpathlineto{\pgfqpoint{1.169708in}{0.941005in}}%
\pgfpathlineto{\pgfqpoint{1.261982in}{0.973336in}}%
\pgfpathlineto{\pgfqpoint{1.354255in}{1.082753in}}%
\pgfpathlineto{\pgfqpoint{1.446528in}{1.109079in}}%
\pgfpathlineto{\pgfqpoint{1.538801in}{1.182145in}}%
\pgfpathlineto{\pgfqpoint{1.631074in}{1.220693in}}%
\pgfpathlineto{\pgfqpoint{1.723348in}{1.233751in}}%
\pgfpathlineto{\pgfqpoint{1.815621in}{1.289905in}}%
\pgfpathlineto{\pgfqpoint{1.907894in}{1.387229in}}%
\pgfpathlineto{\pgfqpoint{2.000167in}{1.393965in}}%
\pgfpathlineto{\pgfqpoint{2.092440in}{1.423420in}}%
\pgfpathlineto{\pgfqpoint{2.184714in}{1.463142in}}%
\pgfpathlineto{\pgfqpoint{2.276987in}{1.509479in}}%
\pgfpathlineto{\pgfqpoint{2.369260in}{1.559875in}}%
\pgfpathlineto{\pgfqpoint{2.461533in}{1.624282in}}%
\pgfpathlineto{\pgfqpoint{2.553806in}{1.630938in}}%
\pgfpathlineto{\pgfqpoint{2.646080in}{1.679709in}}%
\pgfpathlineto{\pgfqpoint{2.738353in}{1.707507in}}%
\pgfpathlineto{\pgfqpoint{2.830626in}{1.751910in}}%
\pgfpathlineto{\pgfqpoint{2.922899in}{1.783171in}}%
\pgfpathlineto{\pgfqpoint{3.015172in}{1.785741in}}%
\pgfpathlineto{\pgfqpoint{3.107446in}{1.842042in}}%
\pgfpathlineto{\pgfqpoint{3.199719in}{1.877208in}}%
\pgfpathlineto{\pgfqpoint{3.291992in}{1.871076in}}%
\pgfpathlineto{\pgfqpoint{3.384265in}{1.939558in}}%
\pgfpathlineto{\pgfqpoint{3.476538in}{1.969050in}}%
\pgfpathlineto{\pgfqpoint{3.568812in}{2.020485in}}%
\pgfpathlineto{\pgfqpoint{3.661085in}{1.997362in}}%
\pgfpathlineto{\pgfqpoint{3.753358in}{2.034606in}}%
\pgfpathlineto{\pgfqpoint{3.845631in}{2.046457in}}%
\pgfpathlineto{\pgfqpoint{3.937904in}{2.091875in}}%
\pgfpathlineto{\pgfqpoint{4.030178in}{2.089200in}}%
\pgfpathlineto{\pgfqpoint{4.122451in}{2.077088in}}%
\pgfpathlineto{\pgfqpoint{4.214724in}{2.122384in}}%
\pgfpathlineto{\pgfqpoint{4.306997in}{2.182666in}}%
\pgfpathlineto{\pgfqpoint{4.399270in}{2.181474in}}%
\pgfpathlineto{\pgfqpoint{4.491544in}{2.215417in}}%
\pgfpathlineto{\pgfqpoint{4.583817in}{2.290280in}}%
\pgfpathlineto{\pgfqpoint{4.676090in}{2.261607in}}%
\pgfpathlineto{\pgfqpoint{4.768363in}{2.316469in}}%
\pgfpathlineto{\pgfqpoint{4.860636in}{2.255292in}}%
\pgfpathlineto{\pgfqpoint{4.952910in}{2.335608in}}%
\pgfpathlineto{\pgfqpoint{5.045183in}{2.401388in}}%
\pgfusepath{stroke}%
\end{pgfscope}%
\begin{pgfscope}%
\pgfsetrectcap%
\pgfsetmiterjoin%
\pgfsetlinewidth{0.803000pt}%
\definecolor{currentstroke}{rgb}{0.000000,0.000000,0.000000}%
\pgfsetstrokecolor{currentstroke}%
\pgfsetdash{}{0pt}%
\pgfpathmoveto{\pgfqpoint{0.588387in}{0.521603in}}%
\pgfpathlineto{\pgfqpoint{0.588387in}{2.531888in}}%
\pgfusepath{stroke}%
\end{pgfscope}%
\begin{pgfscope}%
\pgfsetrectcap%
\pgfsetmiterjoin%
\pgfsetlinewidth{0.803000pt}%
\definecolor{currentstroke}{rgb}{0.000000,0.000000,0.000000}%
\pgfsetstrokecolor{currentstroke}%
\pgfsetdash{}{0pt}%
\pgfpathmoveto{\pgfqpoint{5.257411in}{0.521603in}}%
\pgfpathlineto{\pgfqpoint{5.257411in}{2.531888in}}%
\pgfusepath{stroke}%
\end{pgfscope}%
\begin{pgfscope}%
\pgfsetrectcap%
\pgfsetmiterjoin%
\pgfsetlinewidth{0.803000pt}%
\definecolor{currentstroke}{rgb}{0.000000,0.000000,0.000000}%
\pgfsetstrokecolor{currentstroke}%
\pgfsetdash{}{0pt}%
\pgfpathmoveto{\pgfqpoint{0.588387in}{0.521603in}}%
\pgfpathlineto{\pgfqpoint{5.257411in}{0.521603in}}%
\pgfusepath{stroke}%
\end{pgfscope}%
\begin{pgfscope}%
\pgfsetrectcap%
\pgfsetmiterjoin%
\pgfsetlinewidth{0.803000pt}%
\definecolor{currentstroke}{rgb}{0.000000,0.000000,0.000000}%
\pgfsetstrokecolor{currentstroke}%
\pgfsetdash{}{0pt}%
\pgfpathmoveto{\pgfqpoint{0.588387in}{2.531888in}}%
\pgfpathlineto{\pgfqpoint{5.257411in}{2.531888in}}%
\pgfusepath{stroke}%
\end{pgfscope}%
\begin{pgfscope}%
\definecolor{textcolor}{rgb}{0.000000,0.000000,0.000000}%
\pgfsetstrokecolor{textcolor}%
\pgfsetfillcolor{textcolor}%
\pgftext[x=2.922899in,y=2.615222in,,base]{\color{textcolor}{\rmfamily\fontsize{12.000000}{14.400000}\selectfont\catcode`\^=\active\def^{\ifmmode\sp\else\^{}\fi}\catcode`\%=\active\def%{\%}Mean}}%
\end{pgfscope}%
\begin{pgfscope}%
\pgfsetbuttcap%
\pgfsetmiterjoin%
\definecolor{currentfill}{rgb}{1.000000,1.000000,1.000000}%
\pgfsetfillcolor{currentfill}%
\pgfsetfillopacity{0.800000}%
\pgfsetlinewidth{1.003750pt}%
\definecolor{currentstroke}{rgb}{0.800000,0.800000,0.800000}%
\pgfsetstrokecolor{currentstroke}%
\pgfsetstrokeopacity{0.800000}%
\pgfsetdash{}{0pt}%
\pgfpathmoveto{\pgfqpoint{5.344911in}{1.133672in}}%
\pgfpathlineto{\pgfqpoint{8.259376in}{1.133672in}}%
\pgfpathquadraticcurveto{\pgfqpoint{8.284376in}{1.133672in}}{\pgfqpoint{8.284376in}{1.158672in}}%
\pgfpathlineto{\pgfqpoint{8.284376in}{2.444388in}}%
\pgfpathquadraticcurveto{\pgfqpoint{8.284376in}{2.469388in}}{\pgfqpoint{8.259376in}{2.469388in}}%
\pgfpathlineto{\pgfqpoint{5.344911in}{2.469388in}}%
\pgfpathquadraticcurveto{\pgfqpoint{5.319911in}{2.469388in}}{\pgfqpoint{5.319911in}{2.444388in}}%
\pgfpathlineto{\pgfqpoint{5.319911in}{1.158672in}}%
\pgfpathquadraticcurveto{\pgfqpoint{5.319911in}{1.133672in}}{\pgfqpoint{5.344911in}{1.133672in}}%
\pgfpathlineto{\pgfqpoint{5.344911in}{1.133672in}}%
\pgfpathclose%
\pgfusepath{stroke,fill}%
\end{pgfscope}%
\begin{pgfscope}%
\pgfsetrectcap%
\pgfsetroundjoin%
\pgfsetlinewidth{1.505625pt}%
\pgfsetstrokecolor{currentstroke3}%
\pgfsetdash{}{0pt}%
\pgfpathmoveto{\pgfqpoint{5.369911in}{2.368168in}}%
\pgfpathlineto{\pgfqpoint{5.494911in}{2.368168in}}%
\pgfpathlineto{\pgfqpoint{5.619911in}{2.368168in}}%
\pgfusepath{stroke}%
\end{pgfscope}%
\begin{pgfscope}%
\definecolor{textcolor}{rgb}{0.000000,0.000000,0.000000}%
\pgfsetstrokecolor{textcolor}%
\pgfsetfillcolor{textcolor}%
\pgftext[x=5.719911in,y=2.324418in,left,base]{\color{textcolor}{\rmfamily\fontsize{9.000000}{10.800000}\selectfont\catcode`\^=\active\def^{\ifmmode\sp\else\^{}\fi}\catcode`\%=\active\def%{\%}\NaiveCycles{}}}%
\end{pgfscope}%
\begin{pgfscope}%
\pgfsetrectcap%
\pgfsetroundjoin%
\pgfsetlinewidth{1.505625pt}%
\pgfsetstrokecolor{currentstroke1}%
\pgfsetdash{}{0pt}%
\pgfpathmoveto{\pgfqpoint{5.369911in}{2.184696in}}%
\pgfpathlineto{\pgfqpoint{5.494911in}{2.184696in}}%
\pgfpathlineto{\pgfqpoint{5.619911in}{2.184696in}}%
\pgfusepath{stroke}%
\end{pgfscope}%
\begin{pgfscope}%
\definecolor{textcolor}{rgb}{0.000000,0.000000,0.000000}%
\pgfsetstrokecolor{textcolor}%
\pgfsetfillcolor{textcolor}%
\pgftext[x=5.719911in,y=2.140946in,left,base]{\color{textcolor}{\rmfamily\fontsize{9.000000}{10.800000}\selectfont\catcode`\^=\active\def^{\ifmmode\sp\else\^{}\fi}\catcode`\%=\active\def%{\%}\CyclesMatchChunks{} \& \MergeLinear{}}}%
\end{pgfscope}%
\begin{pgfscope}%
\pgfsetrectcap%
\pgfsetroundjoin%
\pgfsetlinewidth{1.505625pt}%
\pgfsetstrokecolor{currentstroke2}%
\pgfsetdash{}{0pt}%
\pgfpathmoveto{\pgfqpoint{5.369911in}{1.997746in}}%
\pgfpathlineto{\pgfqpoint{5.494911in}{1.997746in}}%
\pgfpathlineto{\pgfqpoint{5.619911in}{1.997746in}}%
\pgfusepath{stroke}%
\end{pgfscope}%
\begin{pgfscope}%
\definecolor{textcolor}{rgb}{0.000000,0.000000,0.000000}%
\pgfsetstrokecolor{textcolor}%
\pgfsetfillcolor{textcolor}%
\pgftext[x=5.719911in,y=1.953996in,left,base]{\color{textcolor}{\rmfamily\fontsize{9.000000}{10.800000}\selectfont\catcode`\^=\active\def^{\ifmmode\sp\else\^{}\fi}\catcode`\%=\active\def%{\%}\CyclesMatchChunks{} \& \SharedVertices{}}}%
\end{pgfscope}%
\begin{pgfscope}%
\pgfsetrectcap%
\pgfsetroundjoin%
\pgfsetlinewidth{1.505625pt}%
\pgfsetstrokecolor{currentstroke4}%
\pgfsetdash{}{0pt}%
\pgfpathmoveto{\pgfqpoint{5.369911in}{1.810795in}}%
\pgfpathlineto{\pgfqpoint{5.494911in}{1.810795in}}%
\pgfpathlineto{\pgfqpoint{5.619911in}{1.810795in}}%
\pgfusepath{stroke}%
\end{pgfscope}%
\begin{pgfscope}%
\definecolor{textcolor}{rgb}{0.000000,0.000000,0.000000}%
\pgfsetstrokecolor{textcolor}%
\pgfsetfillcolor{textcolor}%
\pgftext[x=5.719911in,y=1.767045in,left,base]{\color{textcolor}{\rmfamily\fontsize{9.000000}{10.800000}\selectfont\catcode`\^=\active\def^{\ifmmode\sp\else\^{}\fi}\catcode`\%=\active\def%{\%}\Neighbors{} \& \MergeLinear{}}}%
\end{pgfscope}%
\begin{pgfscope}%
\pgfsetrectcap%
\pgfsetroundjoin%
\pgfsetlinewidth{1.505625pt}%
\pgfsetstrokecolor{currentstroke5}%
\pgfsetdash{}{0pt}%
\pgfpathmoveto{\pgfqpoint{5.369911in}{1.627324in}}%
\pgfpathlineto{\pgfqpoint{5.494911in}{1.627324in}}%
\pgfpathlineto{\pgfqpoint{5.619911in}{1.627324in}}%
\pgfusepath{stroke}%
\end{pgfscope}%
\begin{pgfscope}%
\definecolor{textcolor}{rgb}{0.000000,0.000000,0.000000}%
\pgfsetstrokecolor{textcolor}%
\pgfsetfillcolor{textcolor}%
\pgftext[x=5.719911in,y=1.583574in,left,base]{\color{textcolor}{\rmfamily\fontsize{9.000000}{10.800000}\selectfont\catcode`\^=\active\def^{\ifmmode\sp\else\^{}\fi}\catcode`\%=\active\def%{\%}\Neighbors{} \& \SharedVertices{}}}%
\end{pgfscope}%
\begin{pgfscope}%
\pgfsetrectcap%
\pgfsetroundjoin%
\pgfsetlinewidth{1.505625pt}%
\pgfsetstrokecolor{currentstroke6}%
\pgfsetdash{}{0pt}%
\pgfpathmoveto{\pgfqpoint{5.369911in}{1.440373in}}%
\pgfpathlineto{\pgfqpoint{5.494911in}{1.440373in}}%
\pgfpathlineto{\pgfqpoint{5.619911in}{1.440373in}}%
\pgfusepath{stroke}%
\end{pgfscope}%
\begin{pgfscope}%
\definecolor{textcolor}{rgb}{0.000000,0.000000,0.000000}%
\pgfsetstrokecolor{textcolor}%
\pgfsetfillcolor{textcolor}%
\pgftext[x=5.719911in,y=1.396623in,left,base]{\color{textcolor}{\rmfamily\fontsize{9.000000}{10.800000}\selectfont\catcode`\^=\active\def^{\ifmmode\sp\else\^{}\fi}\catcode`\%=\active\def%{\%}\None{} \& \MergeLinear{}}}%
\end{pgfscope}%
\begin{pgfscope}%
\pgfsetrectcap%
\pgfsetroundjoin%
\pgfsetlinewidth{1.505625pt}%
\pgfsetstrokecolor{currentstroke7}%
\pgfsetdash{}{0pt}%
\pgfpathmoveto{\pgfqpoint{5.369911in}{1.256902in}}%
\pgfpathlineto{\pgfqpoint{5.494911in}{1.256902in}}%
\pgfpathlineto{\pgfqpoint{5.619911in}{1.256902in}}%
\pgfusepath{stroke}%
\end{pgfscope}%
\begin{pgfscope}%
\definecolor{textcolor}{rgb}{0.000000,0.000000,0.000000}%
\pgfsetstrokecolor{textcolor}%
\pgfsetfillcolor{textcolor}%
\pgftext[x=5.719911in,y=1.213152in,left,base]{\color{textcolor}{\rmfamily\fontsize{9.000000}{10.800000}\selectfont\catcode`\^=\active\def^{\ifmmode\sp\else\^{}\fi}\catcode`\%=\active\def%{\%}\None{} \& \SharedVertices{}}}%
\end{pgfscope}%
\end{pgfpicture}%
\makeatother%
\endgroup%
}
		\caption[Mean runtime for~globally rigid graphs (some)]{
			Mean running time to find some NAC-coloring for~globally rigid graphs.}%
		\label{fig:graph_globally_rigid_first_runtime}
	\end{figure}%
	\begin{figure}[thbp]
		\centering
		\scalebox{\BenchFigureScale}{%% Creator: Matplotlib, PGF backend
%%
%% To include the figure in your LaTeX document, write
%%   \input{<filename>.pgf}
%%
%% Make sure the required packages are loaded in your preamble
%%   \usepackage{pgf}
%%
%% Also ensure that all the required font packages are loaded; for instance,
%% the lmodern package is sometimes necessary when using math font.
%%   \usepackage{lmodern}
%%
%% Figures using additional raster images can only be included by \input if
%% they are in the same directory as the main LaTeX file. For loading figures
%% from other directories you can use the `import` package
%%   \usepackage{import}
%%
%% and then include the figures with
%%   \import{<path to file>}{<filename>.pgf}
%%
%% Matplotlib used the following preamble
%%   \def\mathdefault#1{#1}
%%   \everymath=\expandafter{\the\everymath\displaystyle}
%%   \IfFileExists{scrextend.sty}{
%%     \usepackage[fontsize=10.000000pt]{scrextend}
%%   }{
%%     \renewcommand{\normalsize}{\fontsize{10.000000}{12.000000}\selectfont}
%%     \normalsize
%%   }
%%   
%%   \ifdefined\pdftexversion\else  % non-pdftex case.
%%     \usepackage{fontspec}
%%     \setmainfont{DejaVuSans.ttf}[Path=\detokenize{/home/petr/Projects/PyRigi/.venv/lib/python3.12/site-packages/matplotlib/mpl-data/fonts/ttf/}]
%%     \setsansfont{DejaVuSans.ttf}[Path=\detokenize{/home/petr/Projects/PyRigi/.venv/lib/python3.12/site-packages/matplotlib/mpl-data/fonts/ttf/}]
%%     \setmonofont{DejaVuSansMono.ttf}[Path=\detokenize{/home/petr/Projects/PyRigi/.venv/lib/python3.12/site-packages/matplotlib/mpl-data/fonts/ttf/}]
%%   \fi
%%   \makeatletter\@ifpackageloaded{under\Score{}}{}{\usepackage[strings]{under\Score{}}}\makeatother
%%
\begingroup%
\makeatletter%
\begin{pgfpicture}%
\pgfpathrectangle{\pgfpointorigin}{\pgfqpoint{8.384376in}{2.841853in}}%
\pgfusepath{use as bounding box, clip}%
\begin{pgfscope}%
\pgfsetbuttcap%
\pgfsetmiterjoin%
\definecolor{currentfill}{rgb}{1.000000,1.000000,1.000000}%
\pgfsetfillcolor{currentfill}%
\pgfsetlinewidth{0.000000pt}%
\definecolor{currentstroke}{rgb}{1.000000,1.000000,1.000000}%
\pgfsetstrokecolor{currentstroke}%
\pgfsetdash{}{0pt}%
\pgfpathmoveto{\pgfqpoint{0.000000in}{0.000000in}}%
\pgfpathlineto{\pgfqpoint{8.384376in}{0.000000in}}%
\pgfpathlineto{\pgfqpoint{8.384376in}{2.841853in}}%
\pgfpathlineto{\pgfqpoint{0.000000in}{2.841853in}}%
\pgfpathlineto{\pgfqpoint{0.000000in}{0.000000in}}%
\pgfpathclose%
\pgfusepath{fill}%
\end{pgfscope}%
\begin{pgfscope}%
\pgfsetbuttcap%
\pgfsetmiterjoin%
\definecolor{currentfill}{rgb}{1.000000,1.000000,1.000000}%
\pgfsetfillcolor{currentfill}%
\pgfsetlinewidth{0.000000pt}%
\definecolor{currentstroke}{rgb}{0.000000,0.000000,0.000000}%
\pgfsetstrokecolor{currentstroke}%
\pgfsetstrokeopacity{0.000000}%
\pgfsetdash{}{0pt}%
\pgfpathmoveto{\pgfqpoint{0.588387in}{0.521603in}}%
\pgfpathlineto{\pgfqpoint{4.248423in}{0.521603in}}%
\pgfpathlineto{\pgfqpoint{4.248423in}{2.713741in}}%
\pgfpathlineto{\pgfqpoint{0.588387in}{2.713741in}}%
\pgfpathlineto{\pgfqpoint{0.588387in}{0.521603in}}%
\pgfpathclose%
\pgfusepath{fill}%
\end{pgfscope}%
\begin{pgfscope}%
\pgfsetbuttcap%
\pgfsetroundjoin%
\definecolor{currentfill}{rgb}{0.000000,0.000000,0.000000}%
\pgfsetfillcolor{currentfill}%
\pgfsetlinewidth{0.803000pt}%
\definecolor{currentstroke}{rgb}{0.000000,0.000000,0.000000}%
\pgfsetstrokecolor{currentstroke}%
\pgfsetdash{}{0pt}%
\pgfsys@defobject{currentmarker}{\pgfqpoint{0.000000in}{-0.048611in}}{\pgfqpoint{0.000000in}{0.000000in}}{%
\pgfpathmoveto{\pgfqpoint{0.000000in}{0.000000in}}%
\pgfpathlineto{\pgfqpoint{0.000000in}{-0.048611in}}%
\pgfusepath{stroke,fill}%
}%
\begin{pgfscope}%
\pgfsys@transformshift{0.984222in}{0.521603in}%
\pgfsys@useobject{currentmarker}{}%
\end{pgfscope}%
\end{pgfscope}%
\begin{pgfscope}%
\definecolor{textcolor}{rgb}{0.000000,0.000000,0.000000}%
\pgfsetstrokecolor{textcolor}%
\pgfsetfillcolor{textcolor}%
\pgftext[x=0.984222in,y=0.424381in,,top]{\color{textcolor}{\rmfamily\fontsize{10.000000}{12.000000}\selectfont\catcode`\^=\active\def^{\ifmmode\sp\else\^{}\fi}\catcode`\%=\active\def%{\%}$\mathdefault{4}$}}%
\end{pgfscope}%
\begin{pgfscope}%
\pgfsetbuttcap%
\pgfsetroundjoin%
\definecolor{currentfill}{rgb}{0.000000,0.000000,0.000000}%
\pgfsetfillcolor{currentfill}%
\pgfsetlinewidth{0.803000pt}%
\definecolor{currentstroke}{rgb}{0.000000,0.000000,0.000000}%
\pgfsetstrokecolor{currentstroke}%
\pgfsetdash{}{0pt}%
\pgfsys@defobject{currentmarker}{\pgfqpoint{0.000000in}{-0.048611in}}{\pgfqpoint{0.000000in}{0.000000in}}{%
\pgfpathmoveto{\pgfqpoint{0.000000in}{0.000000in}}%
\pgfpathlineto{\pgfqpoint{0.000000in}{-0.048611in}}%
\pgfusepath{stroke,fill}%
}%
\begin{pgfscope}%
\pgfsys@transformshift{1.443160in}{0.521603in}%
\pgfsys@useobject{currentmarker}{}%
\end{pgfscope}%
\end{pgfscope}%
\begin{pgfscope}%
\definecolor{textcolor}{rgb}{0.000000,0.000000,0.000000}%
\pgfsetstrokecolor{textcolor}%
\pgfsetfillcolor{textcolor}%
\pgftext[x=1.443160in,y=0.424381in,,top]{\color{textcolor}{\rmfamily\fontsize{10.000000}{12.000000}\selectfont\catcode`\^=\active\def^{\ifmmode\sp\else\^{}\fi}\catcode`\%=\active\def%{\%}$\mathdefault{8}$}}%
\end{pgfscope}%
\begin{pgfscope}%
\pgfsetbuttcap%
\pgfsetroundjoin%
\definecolor{currentfill}{rgb}{0.000000,0.000000,0.000000}%
\pgfsetfillcolor{currentfill}%
\pgfsetlinewidth{0.803000pt}%
\definecolor{currentstroke}{rgb}{0.000000,0.000000,0.000000}%
\pgfsetstrokecolor{currentstroke}%
\pgfsetdash{}{0pt}%
\pgfsys@defobject{currentmarker}{\pgfqpoint{0.000000in}{-0.048611in}}{\pgfqpoint{0.000000in}{0.000000in}}{%
\pgfpathmoveto{\pgfqpoint{0.000000in}{0.000000in}}%
\pgfpathlineto{\pgfqpoint{0.000000in}{-0.048611in}}%
\pgfusepath{stroke,fill}%
}%
\begin{pgfscope}%
\pgfsys@transformshift{1.902099in}{0.521603in}%
\pgfsys@useobject{currentmarker}{}%
\end{pgfscope}%
\end{pgfscope}%
\begin{pgfscope}%
\definecolor{textcolor}{rgb}{0.000000,0.000000,0.000000}%
\pgfsetstrokecolor{textcolor}%
\pgfsetfillcolor{textcolor}%
\pgftext[x=1.902099in,y=0.424381in,,top]{\color{textcolor}{\rmfamily\fontsize{10.000000}{12.000000}\selectfont\catcode`\^=\active\def^{\ifmmode\sp\else\^{}\fi}\catcode`\%=\active\def%{\%}$\mathdefault{12}$}}%
\end{pgfscope}%
\begin{pgfscope}%
\pgfsetbuttcap%
\pgfsetroundjoin%
\definecolor{currentfill}{rgb}{0.000000,0.000000,0.000000}%
\pgfsetfillcolor{currentfill}%
\pgfsetlinewidth{0.803000pt}%
\definecolor{currentstroke}{rgb}{0.000000,0.000000,0.000000}%
\pgfsetstrokecolor{currentstroke}%
\pgfsetdash{}{0pt}%
\pgfsys@defobject{currentmarker}{\pgfqpoint{0.000000in}{-0.048611in}}{\pgfqpoint{0.000000in}{0.000000in}}{%
\pgfpathmoveto{\pgfqpoint{0.000000in}{0.000000in}}%
\pgfpathlineto{\pgfqpoint{0.000000in}{-0.048611in}}%
\pgfusepath{stroke,fill}%
}%
\begin{pgfscope}%
\pgfsys@transformshift{2.361038in}{0.521603in}%
\pgfsys@useobject{currentmarker}{}%
\end{pgfscope}%
\end{pgfscope}%
\begin{pgfscope}%
\definecolor{textcolor}{rgb}{0.000000,0.000000,0.000000}%
\pgfsetstrokecolor{textcolor}%
\pgfsetfillcolor{textcolor}%
\pgftext[x=2.361038in,y=0.424381in,,top]{\color{textcolor}{\rmfamily\fontsize{10.000000}{12.000000}\selectfont\catcode`\^=\active\def^{\ifmmode\sp\else\^{}\fi}\catcode`\%=\active\def%{\%}$\mathdefault{16}$}}%
\end{pgfscope}%
\begin{pgfscope}%
\pgfsetbuttcap%
\pgfsetroundjoin%
\definecolor{currentfill}{rgb}{0.000000,0.000000,0.000000}%
\pgfsetfillcolor{currentfill}%
\pgfsetlinewidth{0.803000pt}%
\definecolor{currentstroke}{rgb}{0.000000,0.000000,0.000000}%
\pgfsetstrokecolor{currentstroke}%
\pgfsetdash{}{0pt}%
\pgfsys@defobject{currentmarker}{\pgfqpoint{0.000000in}{-0.048611in}}{\pgfqpoint{0.000000in}{0.000000in}}{%
\pgfpathmoveto{\pgfqpoint{0.000000in}{0.000000in}}%
\pgfpathlineto{\pgfqpoint{0.000000in}{-0.048611in}}%
\pgfusepath{stroke,fill}%
}%
\begin{pgfscope}%
\pgfsys@transformshift{2.819976in}{0.521603in}%
\pgfsys@useobject{currentmarker}{}%
\end{pgfscope}%
\end{pgfscope}%
\begin{pgfscope}%
\definecolor{textcolor}{rgb}{0.000000,0.000000,0.000000}%
\pgfsetstrokecolor{textcolor}%
\pgfsetfillcolor{textcolor}%
\pgftext[x=2.819976in,y=0.424381in,,top]{\color{textcolor}{\rmfamily\fontsize{10.000000}{12.000000}\selectfont\catcode`\^=\active\def^{\ifmmode\sp\else\^{}\fi}\catcode`\%=\active\def%{\%}$\mathdefault{20}$}}%
\end{pgfscope}%
\begin{pgfscope}%
\pgfsetbuttcap%
\pgfsetroundjoin%
\definecolor{currentfill}{rgb}{0.000000,0.000000,0.000000}%
\pgfsetfillcolor{currentfill}%
\pgfsetlinewidth{0.803000pt}%
\definecolor{currentstroke}{rgb}{0.000000,0.000000,0.000000}%
\pgfsetstrokecolor{currentstroke}%
\pgfsetdash{}{0pt}%
\pgfsys@defobject{currentmarker}{\pgfqpoint{0.000000in}{-0.048611in}}{\pgfqpoint{0.000000in}{0.000000in}}{%
\pgfpathmoveto{\pgfqpoint{0.000000in}{0.000000in}}%
\pgfpathlineto{\pgfqpoint{0.000000in}{-0.048611in}}%
\pgfusepath{stroke,fill}%
}%
\begin{pgfscope}%
\pgfsys@transformshift{3.278915in}{0.521603in}%
\pgfsys@useobject{currentmarker}{}%
\end{pgfscope}%
\end{pgfscope}%
\begin{pgfscope}%
\definecolor{textcolor}{rgb}{0.000000,0.000000,0.000000}%
\pgfsetstrokecolor{textcolor}%
\pgfsetfillcolor{textcolor}%
\pgftext[x=3.278915in,y=0.424381in,,top]{\color{textcolor}{\rmfamily\fontsize{10.000000}{12.000000}\selectfont\catcode`\^=\active\def^{\ifmmode\sp\else\^{}\fi}\catcode`\%=\active\def%{\%}$\mathdefault{24}$}}%
\end{pgfscope}%
\begin{pgfscope}%
\pgfsetbuttcap%
\pgfsetroundjoin%
\definecolor{currentfill}{rgb}{0.000000,0.000000,0.000000}%
\pgfsetfillcolor{currentfill}%
\pgfsetlinewidth{0.803000pt}%
\definecolor{currentstroke}{rgb}{0.000000,0.000000,0.000000}%
\pgfsetstrokecolor{currentstroke}%
\pgfsetdash{}{0pt}%
\pgfsys@defobject{currentmarker}{\pgfqpoint{0.000000in}{-0.048611in}}{\pgfqpoint{0.000000in}{0.000000in}}{%
\pgfpathmoveto{\pgfqpoint{0.000000in}{0.000000in}}%
\pgfpathlineto{\pgfqpoint{0.000000in}{-0.048611in}}%
\pgfusepath{stroke,fill}%
}%
\begin{pgfscope}%
\pgfsys@transformshift{3.737854in}{0.521603in}%
\pgfsys@useobject{currentmarker}{}%
\end{pgfscope}%
\end{pgfscope}%
\begin{pgfscope}%
\definecolor{textcolor}{rgb}{0.000000,0.000000,0.000000}%
\pgfsetstrokecolor{textcolor}%
\pgfsetfillcolor{textcolor}%
\pgftext[x=3.737854in,y=0.424381in,,top]{\color{textcolor}{\rmfamily\fontsize{10.000000}{12.000000}\selectfont\catcode`\^=\active\def^{\ifmmode\sp\else\^{}\fi}\catcode`\%=\active\def%{\%}$\mathdefault{28}$}}%
\end{pgfscope}%
\begin{pgfscope}%
\pgfsetbuttcap%
\pgfsetroundjoin%
\definecolor{currentfill}{rgb}{0.000000,0.000000,0.000000}%
\pgfsetfillcolor{currentfill}%
\pgfsetlinewidth{0.803000pt}%
\definecolor{currentstroke}{rgb}{0.000000,0.000000,0.000000}%
\pgfsetstrokecolor{currentstroke}%
\pgfsetdash{}{0pt}%
\pgfsys@defobject{currentmarker}{\pgfqpoint{0.000000in}{-0.048611in}}{\pgfqpoint{0.000000in}{0.000000in}}{%
\pgfpathmoveto{\pgfqpoint{0.000000in}{0.000000in}}%
\pgfpathlineto{\pgfqpoint{0.000000in}{-0.048611in}}%
\pgfusepath{stroke,fill}%
}%
\begin{pgfscope}%
\pgfsys@transformshift{4.196792in}{0.521603in}%
\pgfsys@useobject{currentmarker}{}%
\end{pgfscope}%
\end{pgfscope}%
\begin{pgfscope}%
\definecolor{textcolor}{rgb}{0.000000,0.000000,0.000000}%
\pgfsetstrokecolor{textcolor}%
\pgfsetfillcolor{textcolor}%
\pgftext[x=4.196792in,y=0.424381in,,top]{\color{textcolor}{\rmfamily\fontsize{10.000000}{12.000000}\selectfont\catcode`\^=\active\def^{\ifmmode\sp\else\^{}\fi}\catcode`\%=\active\def%{\%}$\mathdefault{32}$}}%
\end{pgfscope}%
\begin{pgfscope}%
\definecolor{textcolor}{rgb}{0.000000,0.000000,0.000000}%
\pgfsetstrokecolor{textcolor}%
\pgfsetfillcolor{textcolor}%
\pgftext[x=2.418405in,y=0.234413in,,top]{\color{textcolor}{\rmfamily\fontsize{10.000000}{12.000000}\selectfont\catcode`\^=\active\def^{\ifmmode\sp\else\^{}\fi}\catcode`\%=\active\def%{\%}Monochromatic classes}}%
\end{pgfscope}%
\begin{pgfscope}%
\pgfsetbuttcap%
\pgfsetroundjoin%
\definecolor{currentfill}{rgb}{0.000000,0.000000,0.000000}%
\pgfsetfillcolor{currentfill}%
\pgfsetlinewidth{0.803000pt}%
\definecolor{currentstroke}{rgb}{0.000000,0.000000,0.000000}%
\pgfsetstrokecolor{currentstroke}%
\pgfsetdash{}{0pt}%
\pgfsys@defobject{currentmarker}{\pgfqpoint{-0.048611in}{0.000000in}}{\pgfqpoint{-0.000000in}{0.000000in}}{%
\pgfpathmoveto{\pgfqpoint{-0.000000in}{0.000000in}}%
\pgfpathlineto{\pgfqpoint{-0.048611in}{0.000000in}}%
\pgfusepath{stroke,fill}%
}%
\begin{pgfscope}%
\pgfsys@transformshift{0.588387in}{0.670551in}%
\pgfsys@useobject{currentmarker}{}%
\end{pgfscope}%
\end{pgfscope}%
\begin{pgfscope}%
\definecolor{textcolor}{rgb}{0.000000,0.000000,0.000000}%
\pgfsetstrokecolor{textcolor}%
\pgfsetfillcolor{textcolor}%
\pgftext[x=0.289968in, y=0.617790in, left, base]{\color{textcolor}{\rmfamily\fontsize{10.000000}{12.000000}\selectfont\catcode`\^=\active\def^{\ifmmode\sp\else\^{}\fi}\catcode`\%=\active\def%{\%}$\mathdefault{10^{1}}$}}%
\end{pgfscope}%
\begin{pgfscope}%
\pgfsetbuttcap%
\pgfsetroundjoin%
\definecolor{currentfill}{rgb}{0.000000,0.000000,0.000000}%
\pgfsetfillcolor{currentfill}%
\pgfsetlinewidth{0.803000pt}%
\definecolor{currentstroke}{rgb}{0.000000,0.000000,0.000000}%
\pgfsetstrokecolor{currentstroke}%
\pgfsetdash{}{0pt}%
\pgfsys@defobject{currentmarker}{\pgfqpoint{-0.048611in}{0.000000in}}{\pgfqpoint{-0.000000in}{0.000000in}}{%
\pgfpathmoveto{\pgfqpoint{-0.000000in}{0.000000in}}%
\pgfpathlineto{\pgfqpoint{-0.048611in}{0.000000in}}%
\pgfusepath{stroke,fill}%
}%
\begin{pgfscope}%
\pgfsys@transformshift{0.588387in}{1.343398in}%
\pgfsys@useobject{currentmarker}{}%
\end{pgfscope}%
\end{pgfscope}%
\begin{pgfscope}%
\definecolor{textcolor}{rgb}{0.000000,0.000000,0.000000}%
\pgfsetstrokecolor{textcolor}%
\pgfsetfillcolor{textcolor}%
\pgftext[x=0.289968in, y=1.290636in, left, base]{\color{textcolor}{\rmfamily\fontsize{10.000000}{12.000000}\selectfont\catcode`\^=\active\def^{\ifmmode\sp\else\^{}\fi}\catcode`\%=\active\def%{\%}$\mathdefault{10^{2}}$}}%
\end{pgfscope}%
\begin{pgfscope}%
\pgfsetbuttcap%
\pgfsetroundjoin%
\definecolor{currentfill}{rgb}{0.000000,0.000000,0.000000}%
\pgfsetfillcolor{currentfill}%
\pgfsetlinewidth{0.803000pt}%
\definecolor{currentstroke}{rgb}{0.000000,0.000000,0.000000}%
\pgfsetstrokecolor{currentstroke}%
\pgfsetdash{}{0pt}%
\pgfsys@defobject{currentmarker}{\pgfqpoint{-0.048611in}{0.000000in}}{\pgfqpoint{-0.000000in}{0.000000in}}{%
\pgfpathmoveto{\pgfqpoint{-0.000000in}{0.000000in}}%
\pgfpathlineto{\pgfqpoint{-0.048611in}{0.000000in}}%
\pgfusepath{stroke,fill}%
}%
\begin{pgfscope}%
\pgfsys@transformshift{0.588387in}{2.016245in}%
\pgfsys@useobject{currentmarker}{}%
\end{pgfscope}%
\end{pgfscope}%
\begin{pgfscope}%
\definecolor{textcolor}{rgb}{0.000000,0.000000,0.000000}%
\pgfsetstrokecolor{textcolor}%
\pgfsetfillcolor{textcolor}%
\pgftext[x=0.289968in, y=1.963483in, left, base]{\color{textcolor}{\rmfamily\fontsize{10.000000}{12.000000}\selectfont\catcode`\^=\active\def^{\ifmmode\sp\else\^{}\fi}\catcode`\%=\active\def%{\%}$\mathdefault{10^{3}}$}}%
\end{pgfscope}%
\begin{pgfscope}%
\pgfsetbuttcap%
\pgfsetroundjoin%
\definecolor{currentfill}{rgb}{0.000000,0.000000,0.000000}%
\pgfsetfillcolor{currentfill}%
\pgfsetlinewidth{0.803000pt}%
\definecolor{currentstroke}{rgb}{0.000000,0.000000,0.000000}%
\pgfsetstrokecolor{currentstroke}%
\pgfsetdash{}{0pt}%
\pgfsys@defobject{currentmarker}{\pgfqpoint{-0.048611in}{0.000000in}}{\pgfqpoint{-0.000000in}{0.000000in}}{%
\pgfpathmoveto{\pgfqpoint{-0.000000in}{0.000000in}}%
\pgfpathlineto{\pgfqpoint{-0.048611in}{0.000000in}}%
\pgfusepath{stroke,fill}%
}%
\begin{pgfscope}%
\pgfsys@transformshift{0.588387in}{2.689091in}%
\pgfsys@useobject{currentmarker}{}%
\end{pgfscope}%
\end{pgfscope}%
\begin{pgfscope}%
\definecolor{textcolor}{rgb}{0.000000,0.000000,0.000000}%
\pgfsetstrokecolor{textcolor}%
\pgfsetfillcolor{textcolor}%
\pgftext[x=0.289968in, y=2.636330in, left, base]{\color{textcolor}{\rmfamily\fontsize{10.000000}{12.000000}\selectfont\catcode`\^=\active\def^{\ifmmode\sp\else\^{}\fi}\catcode`\%=\active\def%{\%}$\mathdefault{10^{4}}$}}%
\end{pgfscope}%
\begin{pgfscope}%
\pgfsetbuttcap%
\pgfsetroundjoin%
\definecolor{currentfill}{rgb}{0.000000,0.000000,0.000000}%
\pgfsetfillcolor{currentfill}%
\pgfsetlinewidth{0.602250pt}%
\definecolor{currentstroke}{rgb}{0.000000,0.000000,0.000000}%
\pgfsetstrokecolor{currentstroke}%
\pgfsetdash{}{0pt}%
\pgfsys@defobject{currentmarker}{\pgfqpoint{-0.027778in}{0.000000in}}{\pgfqpoint{-0.000000in}{0.000000in}}{%
\pgfpathmoveto{\pgfqpoint{-0.000000in}{0.000000in}}%
\pgfpathlineto{\pgfqpoint{-0.027778in}{0.000000in}}%
\pgfusepath{stroke,fill}%
}%
\begin{pgfscope}%
\pgfsys@transformshift{0.588387in}{0.566326in}%
\pgfsys@useobject{currentmarker}{}%
\end{pgfscope}%
\end{pgfscope}%
\begin{pgfscope}%
\pgfsetbuttcap%
\pgfsetroundjoin%
\definecolor{currentfill}{rgb}{0.000000,0.000000,0.000000}%
\pgfsetfillcolor{currentfill}%
\pgfsetlinewidth{0.602250pt}%
\definecolor{currentstroke}{rgb}{0.000000,0.000000,0.000000}%
\pgfsetstrokecolor{currentstroke}%
\pgfsetdash{}{0pt}%
\pgfsys@defobject{currentmarker}{\pgfqpoint{-0.027778in}{0.000000in}}{\pgfqpoint{-0.000000in}{0.000000in}}{%
\pgfpathmoveto{\pgfqpoint{-0.000000in}{0.000000in}}%
\pgfpathlineto{\pgfqpoint{-0.027778in}{0.000000in}}%
\pgfusepath{stroke,fill}%
}%
\begin{pgfscope}%
\pgfsys@transformshift{0.588387in}{0.605346in}%
\pgfsys@useobject{currentmarker}{}%
\end{pgfscope}%
\end{pgfscope}%
\begin{pgfscope}%
\pgfsetbuttcap%
\pgfsetroundjoin%
\definecolor{currentfill}{rgb}{0.000000,0.000000,0.000000}%
\pgfsetfillcolor{currentfill}%
\pgfsetlinewidth{0.602250pt}%
\definecolor{currentstroke}{rgb}{0.000000,0.000000,0.000000}%
\pgfsetstrokecolor{currentstroke}%
\pgfsetdash{}{0pt}%
\pgfsys@defobject{currentmarker}{\pgfqpoint{-0.027778in}{0.000000in}}{\pgfqpoint{-0.000000in}{0.000000in}}{%
\pgfpathmoveto{\pgfqpoint{-0.000000in}{0.000000in}}%
\pgfpathlineto{\pgfqpoint{-0.027778in}{0.000000in}}%
\pgfusepath{stroke,fill}%
}%
\begin{pgfscope}%
\pgfsys@transformshift{0.588387in}{0.639763in}%
\pgfsys@useobject{currentmarker}{}%
\end{pgfscope}%
\end{pgfscope}%
\begin{pgfscope}%
\pgfsetbuttcap%
\pgfsetroundjoin%
\definecolor{currentfill}{rgb}{0.000000,0.000000,0.000000}%
\pgfsetfillcolor{currentfill}%
\pgfsetlinewidth{0.602250pt}%
\definecolor{currentstroke}{rgb}{0.000000,0.000000,0.000000}%
\pgfsetstrokecolor{currentstroke}%
\pgfsetdash{}{0pt}%
\pgfsys@defobject{currentmarker}{\pgfqpoint{-0.027778in}{0.000000in}}{\pgfqpoint{-0.000000in}{0.000000in}}{%
\pgfpathmoveto{\pgfqpoint{-0.000000in}{0.000000in}}%
\pgfpathlineto{\pgfqpoint{-0.027778in}{0.000000in}}%
\pgfusepath{stroke,fill}%
}%
\begin{pgfscope}%
\pgfsys@transformshift{0.588387in}{0.873098in}%
\pgfsys@useobject{currentmarker}{}%
\end{pgfscope}%
\end{pgfscope}%
\begin{pgfscope}%
\pgfsetbuttcap%
\pgfsetroundjoin%
\definecolor{currentfill}{rgb}{0.000000,0.000000,0.000000}%
\pgfsetfillcolor{currentfill}%
\pgfsetlinewidth{0.602250pt}%
\definecolor{currentstroke}{rgb}{0.000000,0.000000,0.000000}%
\pgfsetstrokecolor{currentstroke}%
\pgfsetdash{}{0pt}%
\pgfsys@defobject{currentmarker}{\pgfqpoint{-0.027778in}{0.000000in}}{\pgfqpoint{-0.000000in}{0.000000in}}{%
\pgfpathmoveto{\pgfqpoint{-0.000000in}{0.000000in}}%
\pgfpathlineto{\pgfqpoint{-0.027778in}{0.000000in}}%
\pgfusepath{stroke,fill}%
}%
\begin{pgfscope}%
\pgfsys@transformshift{0.588387in}{0.991581in}%
\pgfsys@useobject{currentmarker}{}%
\end{pgfscope}%
\end{pgfscope}%
\begin{pgfscope}%
\pgfsetbuttcap%
\pgfsetroundjoin%
\definecolor{currentfill}{rgb}{0.000000,0.000000,0.000000}%
\pgfsetfillcolor{currentfill}%
\pgfsetlinewidth{0.602250pt}%
\definecolor{currentstroke}{rgb}{0.000000,0.000000,0.000000}%
\pgfsetstrokecolor{currentstroke}%
\pgfsetdash{}{0pt}%
\pgfsys@defobject{currentmarker}{\pgfqpoint{-0.027778in}{0.000000in}}{\pgfqpoint{-0.000000in}{0.000000in}}{%
\pgfpathmoveto{\pgfqpoint{-0.000000in}{0.000000in}}%
\pgfpathlineto{\pgfqpoint{-0.027778in}{0.000000in}}%
\pgfusepath{stroke,fill}%
}%
\begin{pgfscope}%
\pgfsys@transformshift{0.588387in}{1.075645in}%
\pgfsys@useobject{currentmarker}{}%
\end{pgfscope}%
\end{pgfscope}%
\begin{pgfscope}%
\pgfsetbuttcap%
\pgfsetroundjoin%
\definecolor{currentfill}{rgb}{0.000000,0.000000,0.000000}%
\pgfsetfillcolor{currentfill}%
\pgfsetlinewidth{0.602250pt}%
\definecolor{currentstroke}{rgb}{0.000000,0.000000,0.000000}%
\pgfsetstrokecolor{currentstroke}%
\pgfsetdash{}{0pt}%
\pgfsys@defobject{currentmarker}{\pgfqpoint{-0.027778in}{0.000000in}}{\pgfqpoint{-0.000000in}{0.000000in}}{%
\pgfpathmoveto{\pgfqpoint{-0.000000in}{0.000000in}}%
\pgfpathlineto{\pgfqpoint{-0.027778in}{0.000000in}}%
\pgfusepath{stroke,fill}%
}%
\begin{pgfscope}%
\pgfsys@transformshift{0.588387in}{1.140851in}%
\pgfsys@useobject{currentmarker}{}%
\end{pgfscope}%
\end{pgfscope}%
\begin{pgfscope}%
\pgfsetbuttcap%
\pgfsetroundjoin%
\definecolor{currentfill}{rgb}{0.000000,0.000000,0.000000}%
\pgfsetfillcolor{currentfill}%
\pgfsetlinewidth{0.602250pt}%
\definecolor{currentstroke}{rgb}{0.000000,0.000000,0.000000}%
\pgfsetstrokecolor{currentstroke}%
\pgfsetdash{}{0pt}%
\pgfsys@defobject{currentmarker}{\pgfqpoint{-0.027778in}{0.000000in}}{\pgfqpoint{-0.000000in}{0.000000in}}{%
\pgfpathmoveto{\pgfqpoint{-0.000000in}{0.000000in}}%
\pgfpathlineto{\pgfqpoint{-0.027778in}{0.000000in}}%
\pgfusepath{stroke,fill}%
}%
\begin{pgfscope}%
\pgfsys@transformshift{0.588387in}{1.194128in}%
\pgfsys@useobject{currentmarker}{}%
\end{pgfscope}%
\end{pgfscope}%
\begin{pgfscope}%
\pgfsetbuttcap%
\pgfsetroundjoin%
\definecolor{currentfill}{rgb}{0.000000,0.000000,0.000000}%
\pgfsetfillcolor{currentfill}%
\pgfsetlinewidth{0.602250pt}%
\definecolor{currentstroke}{rgb}{0.000000,0.000000,0.000000}%
\pgfsetstrokecolor{currentstroke}%
\pgfsetdash{}{0pt}%
\pgfsys@defobject{currentmarker}{\pgfqpoint{-0.027778in}{0.000000in}}{\pgfqpoint{-0.000000in}{0.000000in}}{%
\pgfpathmoveto{\pgfqpoint{-0.000000in}{0.000000in}}%
\pgfpathlineto{\pgfqpoint{-0.027778in}{0.000000in}}%
\pgfusepath{stroke,fill}%
}%
\begin{pgfscope}%
\pgfsys@transformshift{0.588387in}{1.239173in}%
\pgfsys@useobject{currentmarker}{}%
\end{pgfscope}%
\end{pgfscope}%
\begin{pgfscope}%
\pgfsetbuttcap%
\pgfsetroundjoin%
\definecolor{currentfill}{rgb}{0.000000,0.000000,0.000000}%
\pgfsetfillcolor{currentfill}%
\pgfsetlinewidth{0.602250pt}%
\definecolor{currentstroke}{rgb}{0.000000,0.000000,0.000000}%
\pgfsetstrokecolor{currentstroke}%
\pgfsetdash{}{0pt}%
\pgfsys@defobject{currentmarker}{\pgfqpoint{-0.027778in}{0.000000in}}{\pgfqpoint{-0.000000in}{0.000000in}}{%
\pgfpathmoveto{\pgfqpoint{-0.000000in}{0.000000in}}%
\pgfpathlineto{\pgfqpoint{-0.027778in}{0.000000in}}%
\pgfusepath{stroke,fill}%
}%
\begin{pgfscope}%
\pgfsys@transformshift{0.588387in}{1.278192in}%
\pgfsys@useobject{currentmarker}{}%
\end{pgfscope}%
\end{pgfscope}%
\begin{pgfscope}%
\pgfsetbuttcap%
\pgfsetroundjoin%
\definecolor{currentfill}{rgb}{0.000000,0.000000,0.000000}%
\pgfsetfillcolor{currentfill}%
\pgfsetlinewidth{0.602250pt}%
\definecolor{currentstroke}{rgb}{0.000000,0.000000,0.000000}%
\pgfsetstrokecolor{currentstroke}%
\pgfsetdash{}{0pt}%
\pgfsys@defobject{currentmarker}{\pgfqpoint{-0.027778in}{0.000000in}}{\pgfqpoint{-0.000000in}{0.000000in}}{%
\pgfpathmoveto{\pgfqpoint{-0.000000in}{0.000000in}}%
\pgfpathlineto{\pgfqpoint{-0.027778in}{0.000000in}}%
\pgfusepath{stroke,fill}%
}%
\begin{pgfscope}%
\pgfsys@transformshift{0.588387in}{1.312610in}%
\pgfsys@useobject{currentmarker}{}%
\end{pgfscope}%
\end{pgfscope}%
\begin{pgfscope}%
\pgfsetbuttcap%
\pgfsetroundjoin%
\definecolor{currentfill}{rgb}{0.000000,0.000000,0.000000}%
\pgfsetfillcolor{currentfill}%
\pgfsetlinewidth{0.602250pt}%
\definecolor{currentstroke}{rgb}{0.000000,0.000000,0.000000}%
\pgfsetstrokecolor{currentstroke}%
\pgfsetdash{}{0pt}%
\pgfsys@defobject{currentmarker}{\pgfqpoint{-0.027778in}{0.000000in}}{\pgfqpoint{-0.000000in}{0.000000in}}{%
\pgfpathmoveto{\pgfqpoint{-0.000000in}{0.000000in}}%
\pgfpathlineto{\pgfqpoint{-0.027778in}{0.000000in}}%
\pgfusepath{stroke,fill}%
}%
\begin{pgfscope}%
\pgfsys@transformshift{0.588387in}{1.545945in}%
\pgfsys@useobject{currentmarker}{}%
\end{pgfscope}%
\end{pgfscope}%
\begin{pgfscope}%
\pgfsetbuttcap%
\pgfsetroundjoin%
\definecolor{currentfill}{rgb}{0.000000,0.000000,0.000000}%
\pgfsetfillcolor{currentfill}%
\pgfsetlinewidth{0.602250pt}%
\definecolor{currentstroke}{rgb}{0.000000,0.000000,0.000000}%
\pgfsetstrokecolor{currentstroke}%
\pgfsetdash{}{0pt}%
\pgfsys@defobject{currentmarker}{\pgfqpoint{-0.027778in}{0.000000in}}{\pgfqpoint{-0.000000in}{0.000000in}}{%
\pgfpathmoveto{\pgfqpoint{-0.000000in}{0.000000in}}%
\pgfpathlineto{\pgfqpoint{-0.027778in}{0.000000in}}%
\pgfusepath{stroke,fill}%
}%
\begin{pgfscope}%
\pgfsys@transformshift{0.588387in}{1.664427in}%
\pgfsys@useobject{currentmarker}{}%
\end{pgfscope}%
\end{pgfscope}%
\begin{pgfscope}%
\pgfsetbuttcap%
\pgfsetroundjoin%
\definecolor{currentfill}{rgb}{0.000000,0.000000,0.000000}%
\pgfsetfillcolor{currentfill}%
\pgfsetlinewidth{0.602250pt}%
\definecolor{currentstroke}{rgb}{0.000000,0.000000,0.000000}%
\pgfsetstrokecolor{currentstroke}%
\pgfsetdash{}{0pt}%
\pgfsys@defobject{currentmarker}{\pgfqpoint{-0.027778in}{0.000000in}}{\pgfqpoint{-0.000000in}{0.000000in}}{%
\pgfpathmoveto{\pgfqpoint{-0.000000in}{0.000000in}}%
\pgfpathlineto{\pgfqpoint{-0.027778in}{0.000000in}}%
\pgfusepath{stroke,fill}%
}%
\begin{pgfscope}%
\pgfsys@transformshift{0.588387in}{1.748492in}%
\pgfsys@useobject{currentmarker}{}%
\end{pgfscope}%
\end{pgfscope}%
\begin{pgfscope}%
\pgfsetbuttcap%
\pgfsetroundjoin%
\definecolor{currentfill}{rgb}{0.000000,0.000000,0.000000}%
\pgfsetfillcolor{currentfill}%
\pgfsetlinewidth{0.602250pt}%
\definecolor{currentstroke}{rgb}{0.000000,0.000000,0.000000}%
\pgfsetstrokecolor{currentstroke}%
\pgfsetdash{}{0pt}%
\pgfsys@defobject{currentmarker}{\pgfqpoint{-0.027778in}{0.000000in}}{\pgfqpoint{-0.000000in}{0.000000in}}{%
\pgfpathmoveto{\pgfqpoint{-0.000000in}{0.000000in}}%
\pgfpathlineto{\pgfqpoint{-0.027778in}{0.000000in}}%
\pgfusepath{stroke,fill}%
}%
\begin{pgfscope}%
\pgfsys@transformshift{0.588387in}{1.813697in}%
\pgfsys@useobject{currentmarker}{}%
\end{pgfscope}%
\end{pgfscope}%
\begin{pgfscope}%
\pgfsetbuttcap%
\pgfsetroundjoin%
\definecolor{currentfill}{rgb}{0.000000,0.000000,0.000000}%
\pgfsetfillcolor{currentfill}%
\pgfsetlinewidth{0.602250pt}%
\definecolor{currentstroke}{rgb}{0.000000,0.000000,0.000000}%
\pgfsetstrokecolor{currentstroke}%
\pgfsetdash{}{0pt}%
\pgfsys@defobject{currentmarker}{\pgfqpoint{-0.027778in}{0.000000in}}{\pgfqpoint{-0.000000in}{0.000000in}}{%
\pgfpathmoveto{\pgfqpoint{-0.000000in}{0.000000in}}%
\pgfpathlineto{\pgfqpoint{-0.027778in}{0.000000in}}%
\pgfusepath{stroke,fill}%
}%
\begin{pgfscope}%
\pgfsys@transformshift{0.588387in}{1.866974in}%
\pgfsys@useobject{currentmarker}{}%
\end{pgfscope}%
\end{pgfscope}%
\begin{pgfscope}%
\pgfsetbuttcap%
\pgfsetroundjoin%
\definecolor{currentfill}{rgb}{0.000000,0.000000,0.000000}%
\pgfsetfillcolor{currentfill}%
\pgfsetlinewidth{0.602250pt}%
\definecolor{currentstroke}{rgb}{0.000000,0.000000,0.000000}%
\pgfsetstrokecolor{currentstroke}%
\pgfsetdash{}{0pt}%
\pgfsys@defobject{currentmarker}{\pgfqpoint{-0.027778in}{0.000000in}}{\pgfqpoint{-0.000000in}{0.000000in}}{%
\pgfpathmoveto{\pgfqpoint{-0.000000in}{0.000000in}}%
\pgfpathlineto{\pgfqpoint{-0.027778in}{0.000000in}}%
\pgfusepath{stroke,fill}%
}%
\begin{pgfscope}%
\pgfsys@transformshift{0.588387in}{1.912019in}%
\pgfsys@useobject{currentmarker}{}%
\end{pgfscope}%
\end{pgfscope}%
\begin{pgfscope}%
\pgfsetbuttcap%
\pgfsetroundjoin%
\definecolor{currentfill}{rgb}{0.000000,0.000000,0.000000}%
\pgfsetfillcolor{currentfill}%
\pgfsetlinewidth{0.602250pt}%
\definecolor{currentstroke}{rgb}{0.000000,0.000000,0.000000}%
\pgfsetstrokecolor{currentstroke}%
\pgfsetdash{}{0pt}%
\pgfsys@defobject{currentmarker}{\pgfqpoint{-0.027778in}{0.000000in}}{\pgfqpoint{-0.000000in}{0.000000in}}{%
\pgfpathmoveto{\pgfqpoint{-0.000000in}{0.000000in}}%
\pgfpathlineto{\pgfqpoint{-0.027778in}{0.000000in}}%
\pgfusepath{stroke,fill}%
}%
\begin{pgfscope}%
\pgfsys@transformshift{0.588387in}{1.951039in}%
\pgfsys@useobject{currentmarker}{}%
\end{pgfscope}%
\end{pgfscope}%
\begin{pgfscope}%
\pgfsetbuttcap%
\pgfsetroundjoin%
\definecolor{currentfill}{rgb}{0.000000,0.000000,0.000000}%
\pgfsetfillcolor{currentfill}%
\pgfsetlinewidth{0.602250pt}%
\definecolor{currentstroke}{rgb}{0.000000,0.000000,0.000000}%
\pgfsetstrokecolor{currentstroke}%
\pgfsetdash{}{0pt}%
\pgfsys@defobject{currentmarker}{\pgfqpoint{-0.027778in}{0.000000in}}{\pgfqpoint{-0.000000in}{0.000000in}}{%
\pgfpathmoveto{\pgfqpoint{-0.000000in}{0.000000in}}%
\pgfpathlineto{\pgfqpoint{-0.027778in}{0.000000in}}%
\pgfusepath{stroke,fill}%
}%
\begin{pgfscope}%
\pgfsys@transformshift{0.588387in}{1.985457in}%
\pgfsys@useobject{currentmarker}{}%
\end{pgfscope}%
\end{pgfscope}%
\begin{pgfscope}%
\pgfsetbuttcap%
\pgfsetroundjoin%
\definecolor{currentfill}{rgb}{0.000000,0.000000,0.000000}%
\pgfsetfillcolor{currentfill}%
\pgfsetlinewidth{0.602250pt}%
\definecolor{currentstroke}{rgb}{0.000000,0.000000,0.000000}%
\pgfsetstrokecolor{currentstroke}%
\pgfsetdash{}{0pt}%
\pgfsys@defobject{currentmarker}{\pgfqpoint{-0.027778in}{0.000000in}}{\pgfqpoint{-0.000000in}{0.000000in}}{%
\pgfpathmoveto{\pgfqpoint{-0.000000in}{0.000000in}}%
\pgfpathlineto{\pgfqpoint{-0.027778in}{0.000000in}}%
\pgfusepath{stroke,fill}%
}%
\begin{pgfscope}%
\pgfsys@transformshift{0.588387in}{2.218792in}%
\pgfsys@useobject{currentmarker}{}%
\end{pgfscope}%
\end{pgfscope}%
\begin{pgfscope}%
\pgfsetbuttcap%
\pgfsetroundjoin%
\definecolor{currentfill}{rgb}{0.000000,0.000000,0.000000}%
\pgfsetfillcolor{currentfill}%
\pgfsetlinewidth{0.602250pt}%
\definecolor{currentstroke}{rgb}{0.000000,0.000000,0.000000}%
\pgfsetstrokecolor{currentstroke}%
\pgfsetdash{}{0pt}%
\pgfsys@defobject{currentmarker}{\pgfqpoint{-0.027778in}{0.000000in}}{\pgfqpoint{-0.000000in}{0.000000in}}{%
\pgfpathmoveto{\pgfqpoint{-0.000000in}{0.000000in}}%
\pgfpathlineto{\pgfqpoint{-0.027778in}{0.000000in}}%
\pgfusepath{stroke,fill}%
}%
\begin{pgfscope}%
\pgfsys@transformshift{0.588387in}{2.337274in}%
\pgfsys@useobject{currentmarker}{}%
\end{pgfscope}%
\end{pgfscope}%
\begin{pgfscope}%
\pgfsetbuttcap%
\pgfsetroundjoin%
\definecolor{currentfill}{rgb}{0.000000,0.000000,0.000000}%
\pgfsetfillcolor{currentfill}%
\pgfsetlinewidth{0.602250pt}%
\definecolor{currentstroke}{rgb}{0.000000,0.000000,0.000000}%
\pgfsetstrokecolor{currentstroke}%
\pgfsetdash{}{0pt}%
\pgfsys@defobject{currentmarker}{\pgfqpoint{-0.027778in}{0.000000in}}{\pgfqpoint{-0.000000in}{0.000000in}}{%
\pgfpathmoveto{\pgfqpoint{-0.000000in}{0.000000in}}%
\pgfpathlineto{\pgfqpoint{-0.027778in}{0.000000in}}%
\pgfusepath{stroke,fill}%
}%
\begin{pgfscope}%
\pgfsys@transformshift{0.588387in}{2.421339in}%
\pgfsys@useobject{currentmarker}{}%
\end{pgfscope}%
\end{pgfscope}%
\begin{pgfscope}%
\pgfsetbuttcap%
\pgfsetroundjoin%
\definecolor{currentfill}{rgb}{0.000000,0.000000,0.000000}%
\pgfsetfillcolor{currentfill}%
\pgfsetlinewidth{0.602250pt}%
\definecolor{currentstroke}{rgb}{0.000000,0.000000,0.000000}%
\pgfsetstrokecolor{currentstroke}%
\pgfsetdash{}{0pt}%
\pgfsys@defobject{currentmarker}{\pgfqpoint{-0.027778in}{0.000000in}}{\pgfqpoint{-0.000000in}{0.000000in}}{%
\pgfpathmoveto{\pgfqpoint{-0.000000in}{0.000000in}}%
\pgfpathlineto{\pgfqpoint{-0.027778in}{0.000000in}}%
\pgfusepath{stroke,fill}%
}%
\begin{pgfscope}%
\pgfsys@transformshift{0.588387in}{2.486544in}%
\pgfsys@useobject{currentmarker}{}%
\end{pgfscope}%
\end{pgfscope}%
\begin{pgfscope}%
\pgfsetbuttcap%
\pgfsetroundjoin%
\definecolor{currentfill}{rgb}{0.000000,0.000000,0.000000}%
\pgfsetfillcolor{currentfill}%
\pgfsetlinewidth{0.602250pt}%
\definecolor{currentstroke}{rgb}{0.000000,0.000000,0.000000}%
\pgfsetstrokecolor{currentstroke}%
\pgfsetdash{}{0pt}%
\pgfsys@defobject{currentmarker}{\pgfqpoint{-0.027778in}{0.000000in}}{\pgfqpoint{-0.000000in}{0.000000in}}{%
\pgfpathmoveto{\pgfqpoint{-0.000000in}{0.000000in}}%
\pgfpathlineto{\pgfqpoint{-0.027778in}{0.000000in}}%
\pgfusepath{stroke,fill}%
}%
\begin{pgfscope}%
\pgfsys@transformshift{0.588387in}{2.539821in}%
\pgfsys@useobject{currentmarker}{}%
\end{pgfscope}%
\end{pgfscope}%
\begin{pgfscope}%
\pgfsetbuttcap%
\pgfsetroundjoin%
\definecolor{currentfill}{rgb}{0.000000,0.000000,0.000000}%
\pgfsetfillcolor{currentfill}%
\pgfsetlinewidth{0.602250pt}%
\definecolor{currentstroke}{rgb}{0.000000,0.000000,0.000000}%
\pgfsetstrokecolor{currentstroke}%
\pgfsetdash{}{0pt}%
\pgfsys@defobject{currentmarker}{\pgfqpoint{-0.027778in}{0.000000in}}{\pgfqpoint{-0.000000in}{0.000000in}}{%
\pgfpathmoveto{\pgfqpoint{-0.000000in}{0.000000in}}%
\pgfpathlineto{\pgfqpoint{-0.027778in}{0.000000in}}%
\pgfusepath{stroke,fill}%
}%
\begin{pgfscope}%
\pgfsys@transformshift{0.588387in}{2.584866in}%
\pgfsys@useobject{currentmarker}{}%
\end{pgfscope}%
\end{pgfscope}%
\begin{pgfscope}%
\pgfsetbuttcap%
\pgfsetroundjoin%
\definecolor{currentfill}{rgb}{0.000000,0.000000,0.000000}%
\pgfsetfillcolor{currentfill}%
\pgfsetlinewidth{0.602250pt}%
\definecolor{currentstroke}{rgb}{0.000000,0.000000,0.000000}%
\pgfsetstrokecolor{currentstroke}%
\pgfsetdash{}{0pt}%
\pgfsys@defobject{currentmarker}{\pgfqpoint{-0.027778in}{0.000000in}}{\pgfqpoint{-0.000000in}{0.000000in}}{%
\pgfpathmoveto{\pgfqpoint{-0.000000in}{0.000000in}}%
\pgfpathlineto{\pgfqpoint{-0.027778in}{0.000000in}}%
\pgfusepath{stroke,fill}%
}%
\begin{pgfscope}%
\pgfsys@transformshift{0.588387in}{2.623886in}%
\pgfsys@useobject{currentmarker}{}%
\end{pgfscope}%
\end{pgfscope}%
\begin{pgfscope}%
\pgfsetbuttcap%
\pgfsetroundjoin%
\definecolor{currentfill}{rgb}{0.000000,0.000000,0.000000}%
\pgfsetfillcolor{currentfill}%
\pgfsetlinewidth{0.602250pt}%
\definecolor{currentstroke}{rgb}{0.000000,0.000000,0.000000}%
\pgfsetstrokecolor{currentstroke}%
\pgfsetdash{}{0pt}%
\pgfsys@defobject{currentmarker}{\pgfqpoint{-0.027778in}{0.000000in}}{\pgfqpoint{-0.000000in}{0.000000in}}{%
\pgfpathmoveto{\pgfqpoint{-0.000000in}{0.000000in}}%
\pgfpathlineto{\pgfqpoint{-0.027778in}{0.000000in}}%
\pgfusepath{stroke,fill}%
}%
\begin{pgfscope}%
\pgfsys@transformshift{0.588387in}{2.658303in}%
\pgfsys@useobject{currentmarker}{}%
\end{pgfscope}%
\end{pgfscope}%
\begin{pgfscope}%
\definecolor{textcolor}{rgb}{0.000000,0.000000,0.000000}%
\pgfsetstrokecolor{textcolor}%
\pgfsetfillcolor{textcolor}%
\pgftext[x=0.234413in,y=1.617672in,,bottom,rotate=90.000000]{\color{textcolor}{\rmfamily\fontsize{10.000000}{12.000000}\selectfont\catcode`\^=\active\def^{\ifmmode\sp\else\^{}\fi}\catcode`\%=\active\def%{\%}Time [ms]}}%
\end{pgfscope}%
\begin{pgfscope}%
\pgfpathrectangle{\pgfqpoint{0.588387in}{0.521603in}}{\pgfqpoint{3.660036in}{2.192138in}}%
\pgfusepath{clip}%
\pgfsetrectcap%
\pgfsetroundjoin%
\pgfsetlinewidth{1.505625pt}%
\pgfsetstrokecolor{currentstroke1}%
\pgfsetdash{}{0pt}%
\pgfpathmoveto{\pgfqpoint{0.754752in}{0.650495in}}%
\pgfpathlineto{\pgfqpoint{0.869487in}{0.708985in}}%
\pgfpathlineto{\pgfqpoint{0.984222in}{0.681895in}}%
\pgfpathlineto{\pgfqpoint{1.098956in}{0.712675in}}%
\pgfpathlineto{\pgfqpoint{1.213691in}{0.670369in}}%
\pgfpathlineto{\pgfqpoint{1.328426in}{0.669607in}}%
\pgfpathlineto{\pgfqpoint{1.443160in}{0.699589in}}%
\pgfpathlineto{\pgfqpoint{1.557895in}{0.690039in}}%
\pgfpathlineto{\pgfqpoint{1.672630in}{0.708009in}}%
\pgfpathlineto{\pgfqpoint{1.787364in}{0.752784in}}%
\pgfpathlineto{\pgfqpoint{1.902099in}{0.854173in}}%
\pgfpathlineto{\pgfqpoint{2.016834in}{0.868496in}}%
\pgfpathlineto{\pgfqpoint{2.131568in}{0.937271in}}%
\pgfpathlineto{\pgfqpoint{2.246303in}{0.995673in}}%
\pgfpathlineto{\pgfqpoint{2.361038in}{1.073200in}}%
\pgfpathlineto{\pgfqpoint{2.475772in}{1.206990in}}%
\pgfpathlineto{\pgfqpoint{2.590507in}{1.203120in}}%
\pgfpathlineto{\pgfqpoint{2.705242in}{1.221092in}}%
\pgfpathlineto{\pgfqpoint{2.819976in}{1.337196in}}%
\pgfpathlineto{\pgfqpoint{2.934711in}{1.379628in}}%
\pgfpathlineto{\pgfqpoint{3.049446in}{1.442241in}}%
\pgfpathlineto{\pgfqpoint{3.164180in}{1.372574in}}%
\pgfpathlineto{\pgfqpoint{3.278915in}{1.362485in}}%
\pgfpathlineto{\pgfqpoint{3.393650in}{1.427828in}}%
\pgfpathlineto{\pgfqpoint{3.508384in}{1.408603in}}%
\pgfpathlineto{\pgfqpoint{3.623119in}{1.570120in}}%
\pgfpathlineto{\pgfqpoint{3.967323in}{1.814573in}}%
\pgfpathlineto{\pgfqpoint{4.082057in}{2.117887in}}%
\pgfusepath{stroke}%
\end{pgfscope}%
\begin{pgfscope}%
\pgfpathrectangle{\pgfqpoint{0.588387in}{0.521603in}}{\pgfqpoint{3.660036in}{2.192138in}}%
\pgfusepath{clip}%
\pgfsetrectcap%
\pgfsetroundjoin%
\pgfsetlinewidth{1.505625pt}%
\pgfsetstrokecolor{currentstroke2}%
\pgfsetdash{}{0pt}%
\pgfpathmoveto{\pgfqpoint{0.754752in}{0.649984in}}%
\pgfpathlineto{\pgfqpoint{0.869487in}{0.706070in}}%
\pgfpathlineto{\pgfqpoint{0.984222in}{0.679740in}}%
\pgfpathlineto{\pgfqpoint{1.098956in}{0.713475in}}%
\pgfpathlineto{\pgfqpoint{1.213691in}{0.668892in}}%
\pgfpathlineto{\pgfqpoint{1.328426in}{0.671503in}}%
\pgfpathlineto{\pgfqpoint{1.443160in}{0.696049in}}%
\pgfpathlineto{\pgfqpoint{1.557895in}{0.689568in}}%
\pgfpathlineto{\pgfqpoint{1.672630in}{0.710268in}}%
\pgfpathlineto{\pgfqpoint{1.787364in}{0.748819in}}%
\pgfpathlineto{\pgfqpoint{1.902099in}{0.858878in}}%
\pgfpathlineto{\pgfqpoint{2.016834in}{0.862877in}}%
\pgfpathlineto{\pgfqpoint{2.131568in}{0.943753in}}%
\pgfpathlineto{\pgfqpoint{2.246303in}{0.994191in}}%
\pgfpathlineto{\pgfqpoint{2.361038in}{1.064160in}}%
\pgfpathlineto{\pgfqpoint{2.475772in}{1.181183in}}%
\pgfpathlineto{\pgfqpoint{2.590507in}{1.154411in}}%
\pgfpathlineto{\pgfqpoint{2.705242in}{1.230702in}}%
\pgfpathlineto{\pgfqpoint{2.819976in}{1.317121in}}%
\pgfpathlineto{\pgfqpoint{2.934711in}{1.367506in}}%
\pgfpathlineto{\pgfqpoint{3.049446in}{1.442241in}}%
\pgfpathlineto{\pgfqpoint{3.164180in}{1.304378in}}%
\pgfpathlineto{\pgfqpoint{3.278915in}{1.435390in}}%
\pgfpathlineto{\pgfqpoint{3.393650in}{1.409770in}}%
\pgfpathlineto{\pgfqpoint{3.508384in}{1.598612in}}%
\pgfpathlineto{\pgfqpoint{3.623119in}{1.598612in}}%
\pgfpathlineto{\pgfqpoint{3.967323in}{1.749222in}}%
\pgfpathlineto{\pgfqpoint{4.082057in}{1.965991in}}%
\pgfusepath{stroke}%
\end{pgfscope}%
\begin{pgfscope}%
\pgfpathrectangle{\pgfqpoint{0.588387in}{0.521603in}}{\pgfqpoint{3.660036in}{2.192138in}}%
\pgfusepath{clip}%
\pgfsetrectcap%
\pgfsetroundjoin%
\pgfsetlinewidth{1.505625pt}%
\pgfsetstrokecolor{currentstroke3}%
\pgfsetdash{}{0pt}%
\pgfpathmoveto{\pgfqpoint{0.754752in}{0.622189in}}%
\pgfpathlineto{\pgfqpoint{0.869487in}{0.680427in}}%
\pgfpathlineto{\pgfqpoint{0.984222in}{0.645249in}}%
\pgfpathlineto{\pgfqpoint{1.098956in}{0.676747in}}%
\pgfpathlineto{\pgfqpoint{1.213691in}{0.666666in}}%
\pgfpathlineto{\pgfqpoint{1.328426in}{0.621246in}}%
\pgfpathlineto{\pgfqpoint{1.443160in}{0.644913in}}%
\pgfpathlineto{\pgfqpoint{1.557895in}{0.646520in}}%
\pgfpathlineto{\pgfqpoint{1.672630in}{0.642073in}}%
\pgfpathlineto{\pgfqpoint{1.787364in}{0.711392in}}%
\pgfpathlineto{\pgfqpoint{1.902099in}{0.694916in}}%
\pgfpathlineto{\pgfqpoint{2.016834in}{0.737214in}}%
\pgfpathlineto{\pgfqpoint{2.131568in}{0.838540in}}%
\pgfpathlineto{\pgfqpoint{2.246303in}{0.947714in}}%
\pgfpathlineto{\pgfqpoint{2.361038in}{1.105482in}}%
\pgfpathlineto{\pgfqpoint{2.475772in}{1.295908in}}%
\pgfpathlineto{\pgfqpoint{2.590507in}{1.466710in}}%
\pgfpathlineto{\pgfqpoint{2.705242in}{1.643221in}}%
\pgfpathlineto{\pgfqpoint{2.819976in}{1.849411in}}%
\pgfpathlineto{\pgfqpoint{2.934711in}{2.014682in}}%
\pgfpathlineto{\pgfqpoint{3.049446in}{2.227500in}}%
\pgfpathlineto{\pgfqpoint{3.278915in}{2.614099in}}%
\pgfusepath{stroke}%
\end{pgfscope}%
\begin{pgfscope}%
\pgfpathrectangle{\pgfqpoint{0.588387in}{0.521603in}}{\pgfqpoint{3.660036in}{2.192138in}}%
\pgfusepath{clip}%
\pgfsetrectcap%
\pgfsetroundjoin%
\pgfsetlinewidth{1.505625pt}%
\pgfsetstrokecolor{currentstroke4}%
\pgfsetdash{}{0pt}%
\pgfpathmoveto{\pgfqpoint{0.754752in}{0.648839in}}%
\pgfpathlineto{\pgfqpoint{0.869487in}{0.703864in}}%
\pgfpathlineto{\pgfqpoint{0.984222in}{0.677371in}}%
\pgfpathlineto{\pgfqpoint{1.098956in}{0.708157in}}%
\pgfpathlineto{\pgfqpoint{1.213691in}{0.661085in}}%
\pgfpathlineto{\pgfqpoint{1.328426in}{0.668959in}}%
\pgfpathlineto{\pgfqpoint{1.443160in}{0.701206in}}%
\pgfpathlineto{\pgfqpoint{1.557895in}{0.682603in}}%
\pgfpathlineto{\pgfqpoint{1.672630in}{0.692393in}}%
\pgfpathlineto{\pgfqpoint{1.787364in}{0.735757in}}%
\pgfpathlineto{\pgfqpoint{1.902099in}{0.822440in}}%
\pgfpathlineto{\pgfqpoint{2.016834in}{0.842817in}}%
\pgfpathlineto{\pgfqpoint{2.131568in}{0.896223in}}%
\pgfpathlineto{\pgfqpoint{2.246303in}{0.944531in}}%
\pgfpathlineto{\pgfqpoint{2.361038in}{0.999587in}}%
\pgfpathlineto{\pgfqpoint{2.475772in}{1.074181in}}%
\pgfpathlineto{\pgfqpoint{2.590507in}{1.118069in}}%
\pgfpathlineto{\pgfqpoint{2.705242in}{1.153993in}}%
\pgfpathlineto{\pgfqpoint{2.819976in}{1.261275in}}%
\pgfpathlineto{\pgfqpoint{2.934711in}{1.290705in}}%
\pgfpathlineto{\pgfqpoint{3.049446in}{1.399401in}}%
\pgfpathlineto{\pgfqpoint{3.164180in}{1.208385in}}%
\pgfpathlineto{\pgfqpoint{3.278915in}{1.198958in}}%
\pgfpathlineto{\pgfqpoint{3.393650in}{1.546675in}}%
\pgfpathlineto{\pgfqpoint{3.508384in}{1.307699in}}%
\pgfpathlineto{\pgfqpoint{3.623119in}{1.625964in}}%
\pgfpathlineto{\pgfqpoint{3.967323in}{1.542269in}}%
\pgfpathlineto{\pgfqpoint{4.082057in}{1.483466in}}%
\pgfusepath{stroke}%
\end{pgfscope}%
\begin{pgfscope}%
\pgfpathrectangle{\pgfqpoint{0.588387in}{0.521603in}}{\pgfqpoint{3.660036in}{2.192138in}}%
\pgfusepath{clip}%
\pgfsetrectcap%
\pgfsetroundjoin%
\pgfsetlinewidth{1.505625pt}%
\pgfsetstrokecolor{currentstroke5}%
\pgfsetdash{}{0pt}%
\pgfpathmoveto{\pgfqpoint{0.754752in}{0.648732in}}%
\pgfpathlineto{\pgfqpoint{0.869487in}{0.702385in}}%
\pgfpathlineto{\pgfqpoint{0.984222in}{0.678261in}}%
\pgfpathlineto{\pgfqpoint{1.098956in}{0.706527in}}%
\pgfpathlineto{\pgfqpoint{1.213691in}{0.670364in}}%
\pgfpathlineto{\pgfqpoint{1.328426in}{0.694916in}}%
\pgfpathlineto{\pgfqpoint{1.443160in}{0.689547in}}%
\pgfpathlineto{\pgfqpoint{1.557895in}{0.685566in}}%
\pgfpathlineto{\pgfqpoint{1.672630in}{0.696788in}}%
\pgfpathlineto{\pgfqpoint{1.787364in}{0.736590in}}%
\pgfpathlineto{\pgfqpoint{1.902099in}{0.825608in}}%
\pgfpathlineto{\pgfqpoint{2.016834in}{0.840785in}}%
\pgfpathlineto{\pgfqpoint{2.131568in}{0.900558in}}%
\pgfpathlineto{\pgfqpoint{2.246303in}{0.948899in}}%
\pgfpathlineto{\pgfqpoint{2.361038in}{0.998796in}}%
\pgfpathlineto{\pgfqpoint{2.475772in}{1.072708in}}%
\pgfpathlineto{\pgfqpoint{2.590507in}{1.082861in}}%
\pgfpathlineto{\pgfqpoint{2.705242in}{1.111359in}}%
\pgfpathlineto{\pgfqpoint{2.819976in}{1.262819in}}%
\pgfpathlineto{\pgfqpoint{2.934711in}{1.314229in}}%
\pgfpathlineto{\pgfqpoint{3.049446in}{1.367237in}}%
\pgfpathlineto{\pgfqpoint{3.164180in}{1.191682in}}%
\pgfpathlineto{\pgfqpoint{3.278915in}{1.312610in}}%
\pgfpathlineto{\pgfqpoint{3.393650in}{1.473341in}}%
\pgfpathlineto{\pgfqpoint{3.508384in}{1.325317in}}%
\pgfpathlineto{\pgfqpoint{3.623119in}{1.547039in}}%
\pgfpathlineto{\pgfqpoint{3.967323in}{1.535534in}}%
\pgfpathlineto{\pgfqpoint{4.082057in}{1.527086in}}%
\pgfusepath{stroke}%
\end{pgfscope}%
\begin{pgfscope}%
\pgfpathrectangle{\pgfqpoint{0.588387in}{0.521603in}}{\pgfqpoint{3.660036in}{2.192138in}}%
\pgfusepath{clip}%
\pgfsetrectcap%
\pgfsetroundjoin%
\pgfsetlinewidth{1.505625pt}%
\pgfsetstrokecolor{currentstroke6}%
\pgfsetdash{}{0pt}%
\pgfpathmoveto{\pgfqpoint{0.754752in}{0.646306in}}%
\pgfpathlineto{\pgfqpoint{0.869487in}{0.706679in}}%
\pgfpathlineto{\pgfqpoint{0.984222in}{0.676179in}}%
\pgfpathlineto{\pgfqpoint{1.098956in}{0.708482in}}%
\pgfpathlineto{\pgfqpoint{1.213691in}{0.670174in}}%
\pgfpathlineto{\pgfqpoint{1.328426in}{0.660206in}}%
\pgfpathlineto{\pgfqpoint{1.443160in}{0.692798in}}%
\pgfpathlineto{\pgfqpoint{1.557895in}{0.684048in}}%
\pgfpathlineto{\pgfqpoint{1.672630in}{0.696812in}}%
\pgfpathlineto{\pgfqpoint{1.787364in}{0.739902in}}%
\pgfpathlineto{\pgfqpoint{1.902099in}{0.829026in}}%
\pgfpathlineto{\pgfqpoint{2.016834in}{0.848832in}}%
\pgfpathlineto{\pgfqpoint{2.131568in}{0.904444in}}%
\pgfpathlineto{\pgfqpoint{2.246303in}{0.941878in}}%
\pgfpathlineto{\pgfqpoint{2.361038in}{0.999192in}}%
\pgfpathlineto{\pgfqpoint{2.475772in}{1.080715in}}%
\pgfpathlineto{\pgfqpoint{2.590507in}{1.085521in}}%
\pgfpathlineto{\pgfqpoint{2.705242in}{1.163340in}}%
\pgfpathlineto{\pgfqpoint{2.819976in}{1.287184in}}%
\pgfpathlineto{\pgfqpoint{2.934711in}{1.341051in}}%
\pgfpathlineto{\pgfqpoint{3.049446in}{1.449698in}}%
\pgfpathlineto{\pgfqpoint{3.164180in}{1.276360in}}%
\pgfpathlineto{\pgfqpoint{3.278915in}{1.220870in}}%
\pgfpathlineto{\pgfqpoint{3.393650in}{1.674480in}}%
\pgfpathlineto{\pgfqpoint{3.508384in}{1.377816in}}%
\pgfpathlineto{\pgfqpoint{3.623119in}{1.632555in}}%
\pgfpathlineto{\pgfqpoint{3.967323in}{1.649439in}}%
\pgfpathlineto{\pgfqpoint{4.082057in}{1.593070in}}%
\pgfusepath{stroke}%
\end{pgfscope}%
\begin{pgfscope}%
\pgfpathrectangle{\pgfqpoint{0.588387in}{0.521603in}}{\pgfqpoint{3.660036in}{2.192138in}}%
\pgfusepath{clip}%
\pgfsetrectcap%
\pgfsetroundjoin%
\pgfsetlinewidth{1.505625pt}%
\pgfsetstrokecolor{currentstroke7}%
\pgfsetdash{}{0pt}%
\pgfpathmoveto{\pgfqpoint{0.754752in}{0.646776in}}%
\pgfpathlineto{\pgfqpoint{0.869487in}{0.706314in}}%
\pgfpathlineto{\pgfqpoint{0.984222in}{0.677569in}}%
\pgfpathlineto{\pgfqpoint{1.098956in}{0.707506in}}%
\pgfpathlineto{\pgfqpoint{1.213691in}{0.662298in}}%
\pgfpathlineto{\pgfqpoint{1.328426in}{0.663729in}}%
\pgfpathlineto{\pgfqpoint{1.443160in}{0.691890in}}%
\pgfpathlineto{\pgfqpoint{1.557895in}{0.680301in}}%
\pgfpathlineto{\pgfqpoint{1.672630in}{0.695213in}}%
\pgfpathlineto{\pgfqpoint{1.787364in}{0.737422in}}%
\pgfpathlineto{\pgfqpoint{1.902099in}{0.830720in}}%
\pgfpathlineto{\pgfqpoint{2.016834in}{0.841294in}}%
\pgfpathlineto{\pgfqpoint{2.131568in}{0.902896in}}%
\pgfpathlineto{\pgfqpoint{2.246303in}{0.946723in}}%
\pgfpathlineto{\pgfqpoint{2.361038in}{1.002338in}}%
\pgfpathlineto{\pgfqpoint{2.475772in}{1.082147in}}%
\pgfpathlineto{\pgfqpoint{2.590507in}{1.083750in}}%
\pgfpathlineto{\pgfqpoint{2.705242in}{1.167104in}}%
\pgfpathlineto{\pgfqpoint{2.819976in}{1.262819in}}%
\pgfpathlineto{\pgfqpoint{2.934711in}{1.320932in}}%
\pgfpathlineto{\pgfqpoint{3.049446in}{1.420626in}}%
\pgfpathlineto{\pgfqpoint{3.164180in}{1.226373in}}%
\pgfpathlineto{\pgfqpoint{3.278915in}{1.182959in}}%
\pgfpathlineto{\pgfqpoint{3.393650in}{1.555997in}}%
\pgfpathlineto{\pgfqpoint{3.508384in}{1.285408in}}%
\pgfpathlineto{\pgfqpoint{3.623119in}{1.581011in}}%
\pgfpathlineto{\pgfqpoint{3.967323in}{1.526306in}}%
\pgfpathlineto{\pgfqpoint{4.082057in}{1.562972in}}%
\pgfusepath{stroke}%
\end{pgfscope}%
\begin{pgfscope}%
\pgfpathrectangle{\pgfqpoint{0.588387in}{0.521603in}}{\pgfqpoint{3.660036in}{2.192138in}}%
\pgfusepath{clip}%
\pgfsetrectcap%
\pgfsetroundjoin%
\pgfsetlinewidth{1.505625pt}%
\definecolor{currentstroke}{rgb}{0.498039,0.498039,0.498039}%
\pgfsetstrokecolor{currentstroke}%
\pgfsetdash{}{0pt}%
\pgfpathmoveto{\pgfqpoint{0.754752in}{0.653001in}}%
\pgfpathlineto{\pgfqpoint{0.869487in}{0.707288in}}%
\pgfpathlineto{\pgfqpoint{0.984222in}{0.681504in}}%
\pgfpathlineto{\pgfqpoint{1.098956in}{0.710909in}}%
\pgfpathlineto{\pgfqpoint{1.213691in}{0.671131in}}%
\pgfpathlineto{\pgfqpoint{1.328426in}{0.668575in}}%
\pgfpathlineto{\pgfqpoint{1.443160in}{0.694349in}}%
\pgfpathlineto{\pgfqpoint{1.557895in}{0.687578in}}%
\pgfpathlineto{\pgfqpoint{1.672630in}{0.703122in}}%
\pgfpathlineto{\pgfqpoint{1.787364in}{0.754616in}}%
\pgfpathlineto{\pgfqpoint{1.902099in}{0.841498in}}%
\pgfpathlineto{\pgfqpoint{2.016834in}{0.864762in}}%
\pgfpathlineto{\pgfqpoint{2.131568in}{0.933101in}}%
\pgfpathlineto{\pgfqpoint{2.246303in}{1.006906in}}%
\pgfpathlineto{\pgfqpoint{2.361038in}{1.070424in}}%
\pgfpathlineto{\pgfqpoint{2.475772in}{1.204650in}}%
\pgfpathlineto{\pgfqpoint{2.590507in}{1.169365in}}%
\pgfpathlineto{\pgfqpoint{2.705242in}{1.206524in}}%
\pgfpathlineto{\pgfqpoint{2.819976in}{1.434749in}}%
\pgfpathlineto{\pgfqpoint{2.934711in}{1.373893in}}%
\pgfpathlineto{\pgfqpoint{3.049446in}{1.494125in}}%
\pgfpathlineto{\pgfqpoint{3.164180in}{1.186729in}}%
\pgfpathlineto{\pgfqpoint{3.278915in}{1.521580in}}%
\pgfpathlineto{\pgfqpoint{3.393650in}{1.651991in}}%
\pgfpathlineto{\pgfqpoint{3.508384in}{1.463822in}}%
\pgfpathlineto{\pgfqpoint{3.623119in}{1.867582in}}%
\pgfpathlineto{\pgfqpoint{3.967323in}{2.146188in}}%
\pgfpathlineto{\pgfqpoint{4.082057in}{1.946437in}}%
\pgfusepath{stroke}%
\end{pgfscope}%
\begin{pgfscope}%
\pgfpathrectangle{\pgfqpoint{0.588387in}{0.521603in}}{\pgfqpoint{3.660036in}{2.192138in}}%
\pgfusepath{clip}%
\pgfsetrectcap%
\pgfsetroundjoin%
\pgfsetlinewidth{1.505625pt}%
\definecolor{currentstroke}{rgb}{0.737255,0.741176,0.133333}%
\pgfsetstrokecolor{currentstroke}%
\pgfsetdash{}{0pt}%
\pgfpathmoveto{\pgfqpoint{0.754752in}{0.645339in}}%
\pgfpathlineto{\pgfqpoint{0.869487in}{0.701034in}}%
\pgfpathlineto{\pgfqpoint{0.984222in}{0.674081in}}%
\pgfpathlineto{\pgfqpoint{1.098956in}{0.703568in}}%
\pgfpathlineto{\pgfqpoint{1.213691in}{0.667811in}}%
\pgfpathlineto{\pgfqpoint{1.328426in}{0.662319in}}%
\pgfpathlineto{\pgfqpoint{1.443160in}{0.689410in}}%
\pgfpathlineto{\pgfqpoint{1.557895in}{0.689755in}}%
\pgfpathlineto{\pgfqpoint{1.672630in}{0.698402in}}%
\pgfpathlineto{\pgfqpoint{1.787364in}{0.748019in}}%
\pgfpathlineto{\pgfqpoint{1.902099in}{0.834357in}}%
\pgfpathlineto{\pgfqpoint{2.016834in}{0.856181in}}%
\pgfpathlineto{\pgfqpoint{2.131568in}{0.919857in}}%
\pgfpathlineto{\pgfqpoint{2.246303in}{0.990078in}}%
\pgfpathlineto{\pgfqpoint{2.361038in}{1.059050in}}%
\pgfpathlineto{\pgfqpoint{2.475772in}{1.192173in}}%
\pgfpathlineto{\pgfqpoint{2.590507in}{1.131196in}}%
\pgfpathlineto{\pgfqpoint{2.705242in}{1.207456in}}%
\pgfpathlineto{\pgfqpoint{2.819976in}{1.379370in}}%
\pgfpathlineto{\pgfqpoint{2.934711in}{1.343690in}}%
\pgfpathlineto{\pgfqpoint{3.049446in}{1.482560in}}%
\pgfpathlineto{\pgfqpoint{3.164180in}{1.184221in}}%
\pgfpathlineto{\pgfqpoint{3.278915in}{1.444317in}}%
\pgfpathlineto{\pgfqpoint{3.393650in}{1.781934in}}%
\pgfpathlineto{\pgfqpoint{3.508384in}{1.464788in}}%
\pgfpathlineto{\pgfqpoint{3.623119in}{1.799323in}}%
\pgfpathlineto{\pgfqpoint{3.967323in}{1.685107in}}%
\pgfusepath{stroke}%
\end{pgfscope}%
\begin{pgfscope}%
\pgfsetrectcap%
\pgfsetmiterjoin%
\pgfsetlinewidth{0.803000pt}%
\definecolor{currentstroke}{rgb}{0.000000,0.000000,0.000000}%
\pgfsetstrokecolor{currentstroke}%
\pgfsetdash{}{0pt}%
\pgfpathmoveto{\pgfqpoint{0.588387in}{0.521603in}}%
\pgfpathlineto{\pgfqpoint{0.588387in}{2.713741in}}%
\pgfusepath{stroke}%
\end{pgfscope}%
\begin{pgfscope}%
\pgfsetrectcap%
\pgfsetmiterjoin%
\pgfsetlinewidth{0.803000pt}%
\definecolor{currentstroke}{rgb}{0.000000,0.000000,0.000000}%
\pgfsetstrokecolor{currentstroke}%
\pgfsetdash{}{0pt}%
\pgfpathmoveto{\pgfqpoint{4.248423in}{0.521603in}}%
\pgfpathlineto{\pgfqpoint{4.248423in}{2.713741in}}%
\pgfusepath{stroke}%
\end{pgfscope}%
\begin{pgfscope}%
\pgfsetrectcap%
\pgfsetmiterjoin%
\pgfsetlinewidth{0.803000pt}%
\definecolor{currentstroke}{rgb}{0.000000,0.000000,0.000000}%
\pgfsetstrokecolor{currentstroke}%
\pgfsetdash{}{0pt}%
\pgfpathmoveto{\pgfqpoint{0.588387in}{0.521603in}}%
\pgfpathlineto{\pgfqpoint{4.248423in}{0.521603in}}%
\pgfusepath{stroke}%
\end{pgfscope}%
\begin{pgfscope}%
\pgfsetrectcap%
\pgfsetmiterjoin%
\pgfsetlinewidth{0.803000pt}%
\definecolor{currentstroke}{rgb}{0.000000,0.000000,0.000000}%
\pgfsetstrokecolor{currentstroke}%
\pgfsetdash{}{0pt}%
\pgfpathmoveto{\pgfqpoint{0.588387in}{2.713741in}}%
\pgfpathlineto{\pgfqpoint{4.248423in}{2.713741in}}%
\pgfusepath{stroke}%
\end{pgfscope}%
\begin{pgfscope}%
\pgfsetbuttcap%
\pgfsetmiterjoin%
\definecolor{currentfill}{rgb}{1.000000,1.000000,1.000000}%
\pgfsetfillcolor{currentfill}%
\pgfsetfillopacity{0.800000}%
\pgfsetlinewidth{1.003750pt}%
\definecolor{currentstroke}{rgb}{0.800000,0.800000,0.800000}%
\pgfsetstrokecolor{currentstroke}%
\pgfsetstrokeopacity{0.800000}%
\pgfsetdash{}{0pt}%
\pgfpathmoveto{\pgfqpoint{4.365089in}{0.350918in}}%
\pgfpathlineto{\pgfqpoint{8.251043in}{0.350918in}}%
\pgfpathquadraticcurveto{\pgfqpoint{8.284376in}{0.350918in}}{\pgfqpoint{8.284376in}{0.384251in}}%
\pgfpathlineto{\pgfqpoint{8.284376in}{2.597075in}}%
\pgfpathquadraticcurveto{\pgfqpoint{8.284376in}{2.630408in}}{\pgfqpoint{8.251043in}{2.630408in}}%
\pgfpathlineto{\pgfqpoint{4.365089in}{2.630408in}}%
\pgfpathquadraticcurveto{\pgfqpoint{4.331756in}{2.630408in}}{\pgfqpoint{4.331756in}{2.597075in}}%
\pgfpathlineto{\pgfqpoint{4.331756in}{0.384251in}}%
\pgfpathquadraticcurveto{\pgfqpoint{4.331756in}{0.350918in}}{\pgfqpoint{4.365089in}{0.350918in}}%
\pgfpathlineto{\pgfqpoint{4.365089in}{0.350918in}}%
\pgfpathclose%
\pgfusepath{stroke,fill}%
\end{pgfscope}%
\begin{pgfscope}%
\pgfsetrectcap%
\pgfsetroundjoin%
\pgfsetlinewidth{1.505625pt}%
\pgfsetstrokecolor{currentstroke3}%
\pgfsetdash{}{0pt}%
\pgfpathmoveto{\pgfqpoint{4.398423in}{2.495447in}}%
\pgfpathlineto{\pgfqpoint{4.565089in}{2.495447in}}%
\pgfpathlineto{\pgfqpoint{4.731756in}{2.495447in}}%
\pgfusepath{stroke}%
\end{pgfscope}%
\begin{pgfscope}%
\definecolor{textcolor}{rgb}{0.000000,0.000000,0.000000}%
\pgfsetstrokecolor{textcolor}%
\pgfsetfillcolor{textcolor}%
\pgftext[x=4.865089in,y=2.437114in,left,base]{\color{textcolor}{\rmfamily\fontsize{12.000000}{14.400000}\selectfont\catcode`\^=\active\def^{\ifmmode\sp\else\^{}\fi}\catcode`\%=\active\def%{\%}\NaiveCycles{}}}%
\end{pgfscope}%
\begin{pgfscope}%
\pgfsetrectcap%
\pgfsetroundjoin%
\pgfsetlinewidth{1.505625pt}%
\pgfsetstrokecolor{currentstroke1}%
\pgfsetdash{}{0pt}%
\pgfpathmoveto{\pgfqpoint{4.398423in}{2.250818in}}%
\pgfpathlineto{\pgfqpoint{4.565089in}{2.250818in}}%
\pgfpathlineto{\pgfqpoint{4.731756in}{2.250818in}}%
\pgfusepath{stroke}%
\end{pgfscope}%
\begin{pgfscope}%
\definecolor{textcolor}{rgb}{0.000000,0.000000,0.000000}%
\pgfsetstrokecolor{textcolor}%
\pgfsetfillcolor{textcolor}%
\pgftext[x=4.865089in,y=2.192485in,left,base]{\color{textcolor}{\rmfamily\fontsize{12.000000}{14.400000}\selectfont\catcode`\^=\active\def^{\ifmmode\sp\else\^{}\fi}\catcode`\%=\active\def%{\%}\CyclesMatchChunks{} \& \MergeLinear{}}}%
\end{pgfscope}%
\begin{pgfscope}%
\pgfsetrectcap%
\pgfsetroundjoin%
\pgfsetlinewidth{1.505625pt}%
\pgfsetstrokecolor{currentstroke2}%
\pgfsetdash{}{0pt}%
\pgfpathmoveto{\pgfqpoint{4.398423in}{2.001551in}}%
\pgfpathlineto{\pgfqpoint{4.565089in}{2.001551in}}%
\pgfpathlineto{\pgfqpoint{4.731756in}{2.001551in}}%
\pgfusepath{stroke}%
\end{pgfscope}%
\begin{pgfscope}%
\definecolor{textcolor}{rgb}{0.000000,0.000000,0.000000}%
\pgfsetstrokecolor{textcolor}%
\pgfsetfillcolor{textcolor}%
\pgftext[x=4.865089in,y=1.943218in,left,base]{\color{textcolor}{\rmfamily\fontsize{12.000000}{14.400000}\selectfont\catcode`\^=\active\def^{\ifmmode\sp\else\^{}\fi}\catcode`\%=\active\def%{\%}\CyclesMatchChunks{} \& \SharedVertices{}}}%
\end{pgfscope}%
\begin{pgfscope}%
\pgfsetrectcap%
\pgfsetroundjoin%
\pgfsetlinewidth{1.505625pt}%
\pgfsetstrokecolor{currentstroke4}%
\pgfsetdash{}{0pt}%
\pgfpathmoveto{\pgfqpoint{4.398423in}{1.752284in}}%
\pgfpathlineto{\pgfqpoint{4.565089in}{1.752284in}}%
\pgfpathlineto{\pgfqpoint{4.731756in}{1.752284in}}%
\pgfusepath{stroke}%
\end{pgfscope}%
\begin{pgfscope}%
\definecolor{textcolor}{rgb}{0.000000,0.000000,0.000000}%
\pgfsetstrokecolor{textcolor}%
\pgfsetfillcolor{textcolor}%
\pgftext[x=4.865089in,y=1.693950in,left,base]{\color{textcolor}{\rmfamily\fontsize{12.000000}{14.400000}\selectfont\catcode`\^=\active\def^{\ifmmode\sp\else\^{}\fi}\catcode`\%=\active\def%{\%}\Neighbors{} \& \MergeLinear{}}}%
\end{pgfscope}%
\begin{pgfscope}%
\pgfsetrectcap%
\pgfsetroundjoin%
\pgfsetlinewidth{1.505625pt}%
\pgfsetstrokecolor{currentstroke5}%
\pgfsetdash{}{0pt}%
\pgfpathmoveto{\pgfqpoint{4.398423in}{1.507655in}}%
\pgfpathlineto{\pgfqpoint{4.565089in}{1.507655in}}%
\pgfpathlineto{\pgfqpoint{4.731756in}{1.507655in}}%
\pgfusepath{stroke}%
\end{pgfscope}%
\begin{pgfscope}%
\definecolor{textcolor}{rgb}{0.000000,0.000000,0.000000}%
\pgfsetstrokecolor{textcolor}%
\pgfsetfillcolor{textcolor}%
\pgftext[x=4.865089in,y=1.449322in,left,base]{\color{textcolor}{\rmfamily\fontsize{12.000000}{14.400000}\selectfont\catcode`\^=\active\def^{\ifmmode\sp\else\^{}\fi}\catcode`\%=\active\def%{\%}\Neighbors{} \& \SharedVertices{}}}%
\end{pgfscope}%
\begin{pgfscope}%
\pgfsetrectcap%
\pgfsetroundjoin%
\pgfsetlinewidth{1.505625pt}%
\pgfsetstrokecolor{currentstroke6}%
\pgfsetdash{}{0pt}%
\pgfpathmoveto{\pgfqpoint{4.398423in}{1.258387in}}%
\pgfpathlineto{\pgfqpoint{4.565089in}{1.258387in}}%
\pgfpathlineto{\pgfqpoint{4.731756in}{1.258387in}}%
\pgfusepath{stroke}%
\end{pgfscope}%
\begin{pgfscope}%
\definecolor{textcolor}{rgb}{0.000000,0.000000,0.000000}%
\pgfsetstrokecolor{textcolor}%
\pgfsetfillcolor{textcolor}%
\pgftext[x=4.865089in,y=1.200054in,left,base]{\color{textcolor}{\rmfamily\fontsize{12.000000}{14.400000}\selectfont\catcode`\^=\active\def^{\ifmmode\sp\else\^{}\fi}\catcode`\%=\active\def%{\%}\NeighborsDegree{} \& \MergeLinear{}}}%
\end{pgfscope}%
\begin{pgfscope}%
\pgfsetrectcap%
\pgfsetroundjoin%
\pgfsetlinewidth{1.505625pt}%
\pgfsetstrokecolor{currentstroke7}%
\pgfsetdash{}{0pt}%
\pgfpathmoveto{\pgfqpoint{4.398423in}{1.009120in}}%
\pgfpathlineto{\pgfqpoint{4.565089in}{1.009120in}}%
\pgfpathlineto{\pgfqpoint{4.731756in}{1.009120in}}%
\pgfusepath{stroke}%
\end{pgfscope}%
\begin{pgfscope}%
\definecolor{textcolor}{rgb}{0.000000,0.000000,0.000000}%
\pgfsetstrokecolor{textcolor}%
\pgfsetfillcolor{textcolor}%
\pgftext[x=4.865089in,y=0.950787in,left,base]{\color{textcolor}{\rmfamily\fontsize{12.000000}{14.400000}\selectfont\catcode`\^=\active\def^{\ifmmode\sp\else\^{}\fi}\catcode`\%=\active\def%{\%}\NeighborsDegree{} \& \SharedVertices{}}}%
\end{pgfscope}%
\begin{pgfscope}%
\pgfsetrectcap%
\pgfsetroundjoin%
\pgfsetlinewidth{1.505625pt}%
\definecolor{currentstroke}{rgb}{0.498039,0.498039,0.498039}%
\pgfsetstrokecolor{currentstroke}%
\pgfsetdash{}{0pt}%
\pgfpathmoveto{\pgfqpoint{4.398423in}{0.759853in}}%
\pgfpathlineto{\pgfqpoint{4.565089in}{0.759853in}}%
\pgfpathlineto{\pgfqpoint{4.731756in}{0.759853in}}%
\pgfusepath{stroke}%
\end{pgfscope}%
\begin{pgfscope}%
\definecolor{textcolor}{rgb}{0.000000,0.000000,0.000000}%
\pgfsetstrokecolor{textcolor}%
\pgfsetfillcolor{textcolor}%
\pgftext[x=4.865089in,y=0.701519in,left,base]{\color{textcolor}{\rmfamily\fontsize{12.000000}{14.400000}\selectfont\catcode`\^=\active\def^{\ifmmode\sp\else\^{}\fi}\catcode`\%=\active\def%{\%}\None{} \& \MergeLinear{}}}%
\end{pgfscope}%
\begin{pgfscope}%
\pgfsetrectcap%
\pgfsetroundjoin%
\pgfsetlinewidth{1.505625pt}%
\definecolor{currentstroke}{rgb}{0.737255,0.741176,0.133333}%
\pgfsetstrokecolor{currentstroke}%
\pgfsetdash{}{0pt}%
\pgfpathmoveto{\pgfqpoint{4.398423in}{0.515224in}}%
\pgfpathlineto{\pgfqpoint{4.565089in}{0.515224in}}%
\pgfpathlineto{\pgfqpoint{4.731756in}{0.515224in}}%
\pgfusepath{stroke}%
\end{pgfscope}%
\begin{pgfscope}%
\definecolor{textcolor}{rgb}{0.000000,0.000000,0.000000}%
\pgfsetstrokecolor{textcolor}%
\pgfsetfillcolor{textcolor}%
\pgftext[x=4.865089in,y=0.456891in,left,base]{\color{textcolor}{\rmfamily\fontsize{12.000000}{14.400000}\selectfont\catcode`\^=\active\def^{\ifmmode\sp\else\^{}\fi}\catcode`\%=\active\def%{\%}\None{} \& \SharedVertices{}}}%
\end{pgfscope}%
\end{pgfpicture}%
\makeatother%
\endgroup%
}
		\caption[Mean runtime for~globally rigid graphs (all)]{
			Mean running time to find all NAC-colorings for~globally rigid graphs.}%
		\label{fig:graph_globally_rigid_all_runtime}
	\end{figure}%
\end{frame}

\begin{frame}
	\frametitle{Benchmarks summary}

	Compared to the \Naive{} approach:
	%
	\begin{itemize}
		\item
		      The number of cycles/polynomial checks reduced significantly.
		\item
		      Faster except for small graphs (important for graph class exploration).
		\item
		      Other polynomial optimizations introduced.
	\end{itemize}
\end{frame}

\begin{frame}
	\frametitle{NAC-colorings search}
	\begin{itemize}
		\item
		      Extension of our paper from VýLeT.
		\item
		      Code contributed to PyRigi.
		\item
		      I am ready for your questions and discussion.
	\end{itemize}
\end{frame}
\end{document}
