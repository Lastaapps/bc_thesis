\chapter{NP-completeness on graphs with maximum degree five}%
\label{chapter:np}

\begin{chapterabstract}

	It has been previously shown that it is NP-complete to decide
	if a graph has a NAC-coloring.
	One could hope that for graphs with small maximum degree
	the problem becomes somewhat simpler.
	In this chapter we show that this is not the case
	for graphs with maximum degree five --- the problem is still NP-complete.
	A graph is constructed from gadgets and equivalence
	with NAC-coloring existence and 3-SAT satisfiability is proved.
	Later we show the same for graphs with average degree of
	$4+\varepsilon$ of any $\varepsilon > 0$.

\end{chapterabstract}

The main idea of our proof is inspired by
the proof of NAC-coloring NP-completeness~\cite{np_complete},
which is described first in this chapter.
After that, we propose a different gadget construction
that allows us to limit the maximum
degree in the constructed graph to five.

\section{NAC-coloring NP-completeness}

In this section we describe the main idea and gadgets used
in the original proof of NAC-coloring NP-completeness~\cite{np_complete}.

For a reduction from other NP-complete problem
we use well known \emph{3-SAT} problem.
Propositional logic SAT answers the question
whether the formula is satisfiable ---
there exists a truth assignment of variables that satisfies the formula.
3-SAT problem is an alternative formulation of SAT
where the formula must be in the conjunctive normal form (CNF), namely in 3-CNF
--- conjunction of clauses with three literals.
For example, \( (A \lor B \lor \lnot C) \land (\lnot A \lor D \lor \lnot E) \)
is in 3-CNF\@.
% TODO CNF formulas look like a face with a nose in the middle \( (\lor)\land(\lor) \).
It was shown that 3-SAT is NP-complete,
and it is a common tool for NP-completeness proofs~\cite{3-sat}.

First, for a formula \( \phi \) we create a graph \( G_\phi \).
Alongside the graph an edge-map is introduced mapping into the literals
of \( \phi \). Recall that for an elementary clause \( x_i \) the literals
are both \( x_i \) and \( \neg x_i \). We denote \( \lnot x_i \) by \( \bar{x}_i \).
There are also literals \( t \) and \( f \) for \( \true \) and \( \false \).
Note, that the goal is to color literals where \( \true \) is assigned \( \blue \)
and \( \false \) is assigned \( \red \).

To construct \( G_\phi \) for each literal \( x_i, \bar{x}_i\) and \( t, f \),
a \emph{connecting edge} labeled by the literal is added to \( G_\phi \).
For each literal we add a five-cycle \( A_i \) to \( G_\phi \)
such that the edges are labeled by \( x_i, x_i, \bar{x}_i, \bar{x}_i, t \).
Then, also for each literal we add a four-cycle \( B_i \) to \( G_\phi \)
such that the edges are labeled by \( x_i, \bar{x}_i, t, f \).
Lastly, for each clause a seven-cycle \( C_i \) is added to \( G_\phi \)
such that the edges are labeled by
\( \hat{x}_{i,1}, \hat{x}_{i,1}, \hat{x}_{i,2}, \hat{x}_{i,2}, \hat{x}_{i,3}, \hat{x}_{i,3}, t \)
where \( \hat{x}_{i,j} \) is either \( x_{i,j} \) or \( \bar{x}_{i,j} \)
depending on the \( i \)th clause.

The paper~\cite{np_complete} then introduces a \emph{connecting element} gadget.
%
\begin{figure}
	\begin{center}
		\begin{tikzpicture}[scale=2]
			\node[vertex] (11) at (0.5, 0.5) {};
			\node[vertex] (12) at (0.5, 1.0) {};
			\node[vertex] (21) at (1.0, 0.5) {};
			\node[vertex] (22) at (1.0, 1.0) {};
			\node[vertex] (23) at (1.0, 1.5) {};
			\node[vertex] (m2) at (1.15, 0.75) {};
			\node[vertex] (m3) at (1.35, 0.75) {};
			\node[vertex] (31) at (1.5, 0.5) {};
			\node[vertex] (32) at (1.5, 1.0) {};
			\node[vertex] (33) at (1.5, 1.5) {};
			\node[vertex] (41) at (2.0, 0.5) {};
			\node[vertex] (42) at (2.0, 1.0) {};

			% Horizontal
			\draw[edge] (11)edge(21) (31)edge(41);
			\draw[edge] (12)edge(22) (32)edge(42);
			\draw[edge,dotted] (21)edge(31) (22)edge(32);
			% Vertical bars
			\draw[edge] (11)edge(12) (21)edge(22) (31)edge(32) (41)edge(42);
			% Diagonals
			\draw[edge] (12)edge(21) (31)edge(42);
			\draw[edge] (11)edge(22) (32)edge(41);
			% Chimneys
			\draw[edge,dotted] (22)edge(23) (32)edge(33) (23)edge(33) (23)edge(32) (22)edge(33);
			% Center
			\draw[edge] (22)edge(m2) (21)edge(m2) (31)edge(m3) (32)edge(m3);
			\draw[edge,dotted] (m2)edge(m3);

			\node[] at (0.25, 0.75) {$x_i$};
			\node[] at (2.25, 0.75) {$x_i$};
			\node[] at (1.25, 1.75) {$\bar{x}_i$};
		\end{tikzpicture}
	\end{center}
	\caption[Connecting element gadget]{
		\centering Connecting element gadget as proposed in~\cite{np_complete}.
		Note that dashed and filled edges need to share same colors in every NAC-coloring.}%
	\label{fig:np_gadget_connecting}
\end{figure}
%
Now, for each edge in \( G_\phi \) we add a connecting element gadget.
The edge labeled by \( x_i \) is identified with the left most edge
of the gadget in \Cref{fig:np_gadget_connecting},
Then the right most edge is identified with the connecting edge corresponding to literal \( x_i \).
The top most edge is identified with the connecting edge corresponding to literal \( \bar{x}_i \).

We do not provide the full proof from the paper~\cite{np_complete},
still we outline the main idea.
First observe, that for any NAC-coloring and literal \( x_i \),
all the edges labeled by \( x_i \) are colored the same.

First, let us suppose that we have a NAC-coloring \( \delta \) of \( G_\phi \).
The goal is to create a truth assignment~\( \tau \).
W.l.o.g.\ let the connecting edge corresponding to \( \true \) is colored \( \blue \).
The elementary formulas colored by \( \blue \) are assigned \( \true \) in \( \tau \),
the other formulas are assigned \( \false \).
We show that \( \phi \) is satisfiable with assignment \( \tau \).
Because of cycles \( A_i \) and \( B_i \) literals \( x_i \) and \( \bar{x}_i \)
must have different color. In cycles \( C_i \) corresponding to clauses
at least one of \( \hat{x}_{i,1}, \hat{x}_{i,2}, \hat{x}_{i,3} \) must be colored \( \blue \),
otherwise an almost cycle is formed as the last edge labeled by \( \true \) is colored \( \blue \).
Therefore, each in each clause at least one literal is assigned \( \true \)
and therefore all clauses and \( \phi \) itself are satisfied.

Now, let us assume that we are given satisfiable \( \phi \)
with truth assignment \( \tau \).
We want to construct a NAC-coloring \( \delta \) of \( G_\phi \).
We color edges labeled by \( x_i \) with \( \blue \) if \( \tau(x_i) = \true \)
and \( \red \) otherwise.
As there are literals \( \true \) and \( \false \), the coloring is surjective.
Now we show that there are no almost cycles in \( G_\phi \).
It can be easily seen (as discussed in the previous paragraph)
that there are no almost cycles in any of the cycles \( A_i, B_i, C_i \).
So if there is an almost cycle in \( G_\phi \), it must pass through
at least two connecting elements.
If the cycles pass in \Cref{fig:np_gadget_connecting}
from \( x_i \) section to \( \bar{x}_i \) section, at least one \( \blue \)
and \( \red \) edge is used. The same holds if the cycle passes
from \( x_i \) section to \( x_i \) the other \( x_i \) section.
As there are at least two connecting elements, at least two edges of each color are in the cycle.
Therefore, the cycle is not an almost cycle.

Notice high degree of vertices of the connecting edges.
For \( i \)th literal edges base degree is thirteen: one from the connecting edge itself,
eight from \( A_i \) cycle and four from the \( B_i \) cycle.
Now for every occurrence of \( x_i \) in a clause degree is increased by four.
Similar holds for the \( \true \) and \( \false \) connecting edges.
The degrees of vertices in the cycles are bounded by six
and the degrees of vertices inside connecting elements are bounded by seven.


\section{Graphs with maximum degree five}

In the previous proof there are not guaranties about
degree of vertices in the graph. The vertices incident to the connecting edges
have high degree, propositional to the number of clauses.
We propose different gadgets where we bound the maximum degree.
We already presented this proof in our previous paper~\cite{my_paper}.

%
\begin{definition}%
	\label[definition]{def:2-tree}
	A \emph{2-tree} is a graph formed by merging triangles by edge identification.
\end{definition}
%

\begin{theorem}[\cite{my_paper}]%
	\label[theorem]{theorem:nac-deg-5}
	%
	The question whether a graph has a NAC-coloring is NP-complete
	on the class of graphs with maximum degree five.
	%
\end{theorem}
\begin{proof}
	%
	Let $\phi$ be a formula with variables $x_{1}, \dots, x_{n}$
	and clauses $L_1, \dots, L_k$.
	Our goal is to construct a graph $G_\phi$
	of size $O(n+k)$ such that $\phi$ is satisfiable if and only if
	$G_\phi$ has a NAC-coloring.

	We exploit the fact that \trcon{} components
	are monochromatic in every NAC-coloring.
	In particular, every subgraph isomorphic to a ladder graph with diagonals is monochromatic.
	We call a ladder graph such that every 4-cycle has one diagonal a \emph{braced ladder}.
	We build a 2-tree structure called a \emph{train},
	which is a ``horizontal'' braced ladder with other ``vertical'' braced ladders
	glued so that the maximum degree is 5, see \Cref{fig:proof_trains}.
	A train can be extended arbitrarily to connect more braced ladders.

	\begin{figure}[h]
		\centering
		\begin{tikzpicture}[scale=1.5]
			\node[vertex] (11) at (0.5, 0.5) {};
			\node[vertex] (12) at (0.5, 1.0) {};
			\node[vertex] (21) at (1.0, 0.5) {};
			\node[vertex] (22) at (1.0, 1.0) {};
			\node[vertex] (23) at (1.0, 1.5) {};
			\node[vertex] (31) at (1.5, 0.5) {};
			\node[vertex] (32) at (1.5, 1.0) {};
			\node[vertex] (33) at (1.5, 1.5) {};
			\node[vertex] (41) at (2.0, 0.5) {};
			\node[vertex] (42) at (2.0, 1.0) {};
			\node[vertex] (51) at (2.5, 0.5) {};
			\node[vertex] (52) at (2.5, 1.0) {};
			\node[vertex] (53) at (2.5, 1.5) {};
			\node[vertex] (61) at (3.0, 0.5) {};
			\node[vertex] (62) at (3.0, 1.0) {};
			\node[vertex] (63) at (3.0, 1.5) {};
			\node[vertex] (71) at (3.5, 0.5) {};
			\node[vertex] (72) at (3.5, 1.0) {};

			% Horizontal
			\draw[edge] (11)edge(21) (21)edge(31) (31)edge(41) (41)edge(51) (51)edge(61) (61)edge(71);
			\draw[edge] (12)edge(22) (22)edge(32) (32)edge(42) (42)edge(52) (52)edge(62) (62)edge(72);
			% Vertical bars
			\draw[edge] (11)edge(12) (21)edge(22) (31)edge(32) (41)edge(42) (51)edge(52) (61)edge(62) (71)edge(72);
			% Diagonals
			\draw[edge] (12)edge(21) (22)edge(31) (31)edge(42) (42)edge(51) (51)edge(62) (61)edge(72);
			% Chimneys
			\draw[edge] (22)edge(23) (32)edge(33) (23)edge(33) (23)edge(32);
			\draw[edge] (52)edge(53) (62)edge(63) (53)edge(63) (52)edge(63);

			\node[] (11d) at (0.25, 0.5) {};
			\node[] (12d) at (0.25, 1.0) {};
			\node[] (71d) at (3.75, 0.5) {};
			\node[] (72d) at (3.75, 1.0) {};
			\draw[edge] (11d)edge(11) (12d)edge(12) (71d)edge(71) (72d)edge(72);
			\node[] (23d) at (1.0, 1.75) {};
			\node[] (33d) at (1.5, 1.75) {};
			\node[] (53d) at (2.5, 1.75) {};
			\node[] (63d) at (3.0, 1.75) {};
			\draw[edge] (23d)edge(23) (33d)edge(33) (53d)edge(53) (63d)edge(63);
		\end{tikzpicture}
		\qquad
		\begin{tikzpicture}[scale=1.5]
			\node[vertex] (11) at (0.5, 0.5) {};
			\node[vertex] (12) at (0.5, 1.0) {};
			\node[vertex] (21) at (1.0, 0.5) {};
			\node[vertex] (22) at (1.0, 1.0) {};
			\node[vertex] (31) at (1.5, 0.5) {};
			\node[vertex] (32) at (1.5, 1.0) {};
			\node[vertex] (41) at (2.0, 0.5) {};
			\node[vertex] (42) at (2.0, 1.0) {};

			% Horizontal
			\draw[edge] (11)edge(21) (21)edge(31) (31)edge(41);
			\draw[edge] (12)edge(22) (22)edge(32) (32)edge(42);
			% Vertical bars
			\draw[edge] (11)edge(12) (21)edge(22) (31)edge(32) (41)edge(42);
			% Diagonals
			\draw[edge] (12)edge(21) (22)edge(31) (31)edge(42);

			\node[] (11d) at (0.25, 0.5) {};
			\node[] (12d) at (0.25, 1.0) {};
			\node[] (41d) at (2.25, 0.5) {};
			\node[] (42d) at (2.25, 1.0) {};
			\draw[edge] (11d)edge(11) (12d)edge(12) (41d)edge(41) (42d)edge(42);

			\begin{scope}[shift={(-0.25,-1)}]
				\node[] (ar) at (1.5, 1.25) {$\downarrow$};
				\node[vertex] (11) at (0.5, 0.5) {};
				\node[vertex] (12) at (0.5, 1.0) {};
				\node[vertex] (21) at (1.0, 0.5) {};
				\node[vertex] (22) at (1.0, 1.0) {};
				\node[vertexSig] (31) at (1.5, 0.5) {};
				\node[vertexSig] (32) at (1.5, 1.0) {};
				\node[vertex] (41) at (2.0, 0.5) {};
				\node[vertex] (42) at (2.0, 1.0) {};
				\node[vertex] (51) at (2.5, 0.5) {};
				\node[vertex] (52) at (2.5, 1.0) {};

				% Horizontal
				\draw[edge] (11)edge(21) (21)edge(31) (41)edge(51);
				\draw[edge] (12)edge(22) (22)edge(32) (42)edge(52);
				% Vertical bars
				\draw[edge] (11)edge(12) (21)edge(22) (41)edge(42) (51)edge(52);
				% Diagonals
				\draw[edge] (12)edge(21) (22)edge(31) (41)edge(52);
				% Red
				\draw[yedge] (31)edge(42) (31)edge(32) (31)edge(41) (32)edge(42);

				\node[] (11d) at (0.25, 0.5) {};
				\node[] (12d) at (0.25, 1.0) {};
				\node[] (51d) at (2.75, 0.5) {};
				\node[] (52d) at (2.75, 1.0) {};
				\draw[edge] (11d)edge(11) (12d)edge(12) (51d)edge(51) (52d)edge(52);
			\end{scope}
		\end{tikzpicture}
		\caption[Train and its extension]{
			\centering A train (left) is formed by gluing braced ladders so that
			the maximum degree is 5. The right figure shows how it can be extended.}
		\label{fig:proof_trains}
	\end{figure}

	We label the edges of $G_\phi$ with literals $x_i, \bar{x}_i$,
	where the bar denotes negation, for $1 \le i \le n$ and with $t, f$ literals.
	The edges in one \trcon{} component have always the same label.
	We construct the graph so that the edges with the same label
	have the same color in every NAC-coloring:
	eventually, blue edges will correspond to \true{} and red edges to \false{}.

	We take $2n+2$ trains, the edges of each labelled by one literal, to which we will
	link other gadgets using braced ladders.
	Note that an edge of a graph such that its end vertices have degrees at most 3 and 4,
	can be glued to a train via a braced ladder so that the maximum degree is at most 5.

	For each variable $x_i$, we create two gadgets:
	one with cycle $A_i$ in the center
	with the edges of the cycle linked using braced ladders
	to the trains $x_i, \bar{x}_i$ and $t$
	and the other linked to the trains $x_i, \bar{x}_i, t$ and $f$
	according to \Cref{fig:proof_enforce_true_false}.

	\begin{figure}[h]
		\centering
		\begin{tikzpicture}[scale=1.75]
			% for i in range(1,8):
			%       for j in range(1,5):
			%           print(f"\\node[vertex] ({i}{j}) at ({i/2}, {j/2}) {{}};")
			% \node[vertex] (11) at (0.5, 0.5) {};
			\node[vertex]      (22) at (1.25, 1.00) {};
			\node[]           (d22) at (1.00, 1.00) {};
			\node[vertex]      (23) at (1.25, 1.50) {};
			\node[]           (d23) at (1.00, 1.50) {};
			\node[vertex]      (32) at (1.50, 1.00) {};
			\node[vertex]      (33) at (1.50, 1.50) {};
			\node[vertex]      (42) at (2.00, 1.00) {};
			\node[vertex]      (43) at (2.00, 1.50) {};
			\node[vertexSig]   (44) at (2.00, 1.75) {};
			\node[vertex]      (45) at (2.00, 2.25) {};
			\node[vertex]      (46) at (2.00, 2.50) {};
			\node[]           (d46) at (2.00, 2.75) {};
			\node[vertex]      (52) at (2.50, 1.00) {};
			\node[vertex]      (53) at (2.50, 1.50) {};
			\node[vertexSig]   (54) at (2.50, 1.75) {};
			\node[vertex]      (55) at (2.50, 2.25) {};
			\node[vertex]      (56) at (2.50, 2.50) {};
			\node[]           (d56) at (2.50, 2.75) {};
			\node[vertex]      (62) at (3.00, 1.00) {};
			\node[vertex]      (63) at (3.00, 1.50) {};
			\node[vertex]      (72) at (3.25, 1.00) {};
			\node[]           (d72) at (3.50, 1.00) {};
			\node[vertex]      (73) at (3.25, 1.50) {};
			\node[]           (d73) at (3.50, 1.50) {};
			\node[vertex] (special) at (2.25, 0.75) {};

			\node[] at (2.25, 1.5) {$A_i$};

			%%% Left part
			% Bridge to center
			\draw[edge] (32)edge(42) (33)edge(43) (32)edge(33) (32)edge(43);
			\draw[edge] (22)edge(32) (23)edge(33) (22)edge(23) (22)edge(33);
			% Center
			\draw[edge] (32)edge(special) (42)edge(special) (33)edge(44) (43)edge(44) (42)edge(43);
			%%% Decoration
			\draw[edge] (22)edge(d22) (23)edge(d23);
			\node[] at (1.0, 1.25) {$x_i$};
			\node[] at (2.125, 1.25) {$x_i$};

			%%% Right part
			% Bridge to center
			\draw[edge] (52)edge(62) (53)edge(63) (62)edge(63) (53)edge(62);
			\draw[edge] (62)edge(72) (63)edge(73) (72)edge(73) (63)edge(72);
			% Center
			\draw[edge] (62)edge(special) (52)edge(special) (63)edge(54) (53)edge(54) (52)edge(53);
			% Decoration
			\draw[edge] (72)edge(d72) (73)edge(d73);
			\node[] at (3.50,  1.25) {$\bar{x}_i$};
			\node[] at (2.375, 1.25) {$\bar{x}_i$};

			%%% Center piece
			% Center peace and the one above
			\draw[bedge] (44)edge(45) (54)edge(55) (44)edge(55);
			\draw[bedge] (45)edge(46) (55)edge(56) (45)edge(56);
			\draw[bedge] (44)edge(54) (45)edge(55) (46)edge(56);
			%%% Decoration
			\draw[bedge] (46)edge(d46) (56)edge(d56);
			\node[] at (2.25, 2.75)  {$t$};
			\node[] at (2.25, 1.875) {$t$};

			\begin{scope}[xshift=4cm]
				\node[vertex]    (13) at (0.75, 1.50) {};
				\node[]         (d13) at (0.50, 1.50) {};
				\node[vertex]    (14) at (0.75, 2.00) {};
				\node[]         (d14) at (0.50, 2.00) {};
				\node[vertex]    (23) at (1.00, 1.50) {};
				\node[vertex]    (24) at (1.00, 2.00) {};
				\node[vertex]    (31) at (1.50, 0.75) {};
				\node[]         (d31) at (1.50, 0.50) {};
				\node[vertex]    (32) at (1.50, 1.00) {};
				\node[vertexSig] (33) at (1.50, 1.50) {};
				\node[vertexSig] (34) at (1.50, 2.00) {};
				\node[vertex]    (35) at (1.50, 2.50) {};
				\node[vertex]    (36) at (1.50, 2.75) {};
				\node[]         (d36) at (1.50, 3.00) {};
				\node[vertex]    (41) at (2.00, 0.75) {};
				\node[]         (d41) at (2.00, 0.50) {};
				\node[vertex]    (42) at (2.00, 1.00) {};
				\node[vertexSig] (43) at (2.00, 1.50) {};
				\node[vertexSig] (44) at (2.00, 2.00) {};
				\node[vertex]    (45) at (2.00, 2.50) {};
				\node[vertex]    (46) at (2.00, 2.75) {};
				\node[]         (d46) at (2.00, 3.00) {};
				\node[vertex]    (53) at (2.50, 1.50) {};
				\node[vertex]    (54) at (2.50, 2.00) {};
				\node[vertex]    (63) at (2.75, 1.50) {};
				\node[]         (d63) at (3.00, 1.50) {};
				\node[vertex]    (64) at (2.75, 2.00) {};
				\node[]         (d64) at (3.00, 2.00) {};

				%%% Center
				\draw[edge] (33)edge(34);
				\draw[edge] (43)edge(44);
				\draw[bedge] (34)edge(44);
				\draw[redge] (33)edge(43);
				%%% Labels
				\node[] at (1.750, 1.750) {$B_i$};
				\node[] at (1.375, 1.700) {$x_i$};
				\node[] at (2.125, 1.800) {$\bar{x}_i$};
				\node[] at (1.700, 2.125) {$t$};
				\node[] at (1.800, 1.375) {$f$};

				%%% Left
				\draw[edge] (23)edge(33) (24)edge(34) (23)edge(24) (23)edge(34);
				\draw[edge] (13)edge(23) (14)edge(24) (13)edge(14) (13)edge(24);
				\draw[edge] (13)edge(d13) (14)edge(d14);
				\node[] at (0.50, 1.75) {$x_i$};

				%%% Right
				\draw[edge] (43)edge(53) (44)edge(54) (53)edge(54) (43)edge(54);
				\draw[edge] (53)edge(63) (54)edge(64) (63)edge(64) (53)edge(64);
				\draw[edge] (63)edge(d63) (64)edge(d64);
				\node[] at (3.00, 1.75) {$\bar{x}_i$};

				%%% Top
				\draw[bedge] (34)edge(35) (44)edge(45) (35)edge(45) (35)edge(44);
				\draw[bedge] (35)edge(36) (45)edge(46) (36)edge(46) (36)edge(45);
				\draw[bedge] (36)edge(d36) (46)edge(d46);
				\node[] at (1.75, 3.00) {$t$};

				%%% Bottom
				\draw[redge] (32)edge(33) (42)edge(43) (32)edge(42) (33)edge(42);
				\draw[redge] (31)edge(32) (41)edge(42) (31)edge(41) (32)edge(41);
				\draw[redge] (31)edge(d31) (41)edge(d41);
				\node[] at (1.75, 0.50) {$f$};

			\end{scope}
		\end{tikzpicture}
		\caption[Gadgets for literals with cycles \( A_i \) and \( B_i \)]{
			\centering The gadgets for every variable $x_i$.
			For all variables together they enforce that the trains $x_i$ and $\bar{x}_i$ have different colors.}%
		\label{fig:proof_enforce_true_false}
	\end{figure}

	For each clause $L_i$, we create a gadget indicated
	in \Cref{fig:proof_clause} with cycle $C_i$ in the center.
	Let $\hat{x}_{i,1}, \hat{x}_{i,2}, \hat{x}_{i,3}$
	be literals used in $L_i$
	where $\hat{x}_{i,j}$ denotes either $x_{i,j}$ or $\bar{x}_{i,j}$
	depending on~$L_i$. We link the edge labeled $\hat{x}_{i,j}$
	to the appropriate literal trains.
	Since each of the 3-prism subgraphs has only one NAC-coloring up to swapping colors,
	all edges labelled with the same label have the same color in every NAC-coloring.

	\begin{figure}[h]
		\centering
		\begin{tikzpicture}[scale=1.75]
			%%% Center
			\node[vertexSig] (35) at (1.5, 2.5) {};
			\node[vertexSig] (53) at (2.5, 1.5) {};
			\node[vertex]    (57) at (2.5, 3.5) {};
			\node[vertex]    (75) at (3.5, 2.5) {};
			\node[] at (2.5, 2.5) {$C_i$};

			%%%%%%%%%%%%%%%%%%%%%%%%%%%%%%%%%%%%%%%%%%%%%%%%%%%%%%%%%%%%%%%%%%%%%%%%%%%%
			%%% t
			%%%%%%%%%%%%%%%%%%%%%%%%%%%%%%%%%%%%%%%%%%%%%%%%%%%%%%%%%%%%%%%%%%%%%%%%%%%%
			\node[vertex] (13) at (0.5, 1.5) {};
			\node[vertex] (14) at (0.5, 2.0) {};
			\node[vertex] (23) at (1.0, 1.5) {};
			\node[vertex] (24) at (1.0, 2.0) {};
			\draw[bedge] (13)edge(23) (14)edge(24) (13)edge(14) (23)edge(24) (13)edge(24);
			\draw[bedge] (53)edge(35) (53)edge(23) (35)edge(24) (53)edge(24);
			%%%% Extensions
			\node[] (13d) at (0.25, 1.5 ) {};
			\node[] (14d) at (0.25, 2.0 ) {};
			\draw[bedge] (13)edge(13d) (14)edge(14d);
			%%%% Labels
			\node[] at (0.25, 1.75) {$t$};
			\node[] at (2.125, 2.125) {$t$};

			%%%%%%%%%%%%%%%%%%%%%%%%%%%%%%%%%%%%%%%%%%%%%%%%%%%%%%%%%%%%%%%%%%%%%%%%%%%%
			%%% x_1
			%%%%%%%%%%%%%%%%%%%%%%%%%%%%%%%%%%%%%%%%%%%%%%%%%%%%%%%%%%%%%%%%%%%%%%%%%%%%
			\node[vertex]    (06)   at (0.25, 3.0 ) {};
			\node[vertex]    (07)   at (0.25, 3.5 ) {};
			\node[vertexSig] (16)   at (0.5 , 3.0 ) {};
			\node[vertexSig] (17)   at (0.5 , 3.5 ) {};
			\node[vertexSig] (26)   at (1.25, 3.0 ) {};
			\node[vertexSig] (27)   at (1.25, 3.5 ) {};
			\node[vertex]    (36)   at (1.5 , 3.0 ) {};
			\node[vertex]    (37)   at (1.5 , 3.5 ) {};
			\node[vertex]    (46)   at (2.0 , 3.0 ) {};
			\node[vertexSig] (p1m1) at (1.0,  3.25) {}; % prism x_1 middle
			\node[vertexSig] (p1m2) at (0.75, 3.25) {};
			\node[vertex]    (p1t1) at (1.0,  2.75) {}; % prism x_1 true
			\node[vertex]    (p1t2) at (0.75, 2.75) {};
			%%%% Construction
			\draw[edge] (35)edge(46) (46)edge(57) (57)edge(37) (36)edge(37) (46)edge(37);
			\draw[edge] (37)edge(27) (36)edge(26) (37)edge(26) (35)edge(36) (46)edge(36) (27)edge(26);
			%%%% Prism
			\draw[bedge] (26)edge(16) (27)edge(17); % linking horizontal edges
			\draw[edge] (26)edge(p1m1) (27)edge(p1m1); % left triangle
			\draw[edge] (16)edge(p1m2) (17)edge(p1m2); % right triangle
			\draw[bedge] (p1m1)edge(p1m2) (p1t1)edge(p1t2);
			\draw[bedge] (p1m1)edge(p1t1) (p1m2)edge(p1t2) (p1m1)edge(p1t2);
			%%%% Train
			\draw[edge] (16)edge(17) (06)edge(07); % vertical
			\draw[edge] (16)edge(06) (17)edge(07); % horizontal
			\draw[edge] (16)edge(07); % diagonal
			%%%% Extensions
			\node[] (06d) at (0.0 , 3.0) {};
			\node[] (07d) at (0.0 , 3.5) {};
			\draw[edge] (07)edge(07d) (06)edge(06d);
			\node[] (p1t1d) at (1.0 , 2.5) {};
			\node[] (p1t2d) at (0.75, 2.5) {};
			\draw[bedge] (p1t1)edge(p1t1d) (p1t2)edge(p1t2d);
			%%%% Labels
			\node[] at (1.875, 2.625) {$\hat{x}_1$};
			\node[] at (2.375, 3.125) {$\hat{x}_1$};
			\node[] at (0.0  , 3.25 ) {$\hat{x}_1$};
			\node[] at (0.875, 2.625) {$t$};


			%%%%%%%%%%%%%%%%%%%%%%%%%%%%%%%%%%%%%%%%%%%%%%%%%%%%%%%%%%%%%%%%%%%%%%%%%%%%
			%%% x_2
			%%%%%%%%%%%%%%%%%%%%%%%%%%%%%%%%%%%%%%%%%%%%%%%%%%%%%%%%%%%%%%%%%%%%%%%%%%%%
			\node[vertex]    (66)   at (3.0 , 3.0 ) {};
			\node[vertex]    (76)   at (3.5 , 3.0 ) {};
			\node[vertex]    (77)   at (3.5 , 3.5 ) {};
			\node[vertexSig] (86)   at (3.75, 3.0 ) {};
			\node[vertexSig] (87)   at (3.75, 3.5 ) {};
			\node[vertexSig] (96)   at (4.5 , 3.0 ) {};
			\node[vertexSig] (97)   at (4.5 , 3.5 ) {};
			\node[vertex]    (A6)   at (4.75, 3.0 ) {};
			\node[vertex]    (A7)   at (4.75, 3.5 ) {};
			\node[vertexSig] (p2m1) at (4.0,  3.25) {}; % prism 2 middle
			\node[vertexSig] (p2m2) at (4.25, 3.25) {};
			\node[vertex]    (p2t1) at (4.0,  2.75) {}; % prism 2 true
			\node[vertex]    (p2t2) at (4.25, 2.75) {};
			%%%% Construction
			\draw[edge] (75)edge(66) (66)edge(57) (57)edge(77) (76)edge(77) (66)edge(77);
			\draw[edge] (77)edge(87) (76)edge(86) (77)edge(86) (75)edge(76) (66)edge(76) (87)edge(86);
			%%%% Prism
			\draw[bedge] (86)edge(96) (87)edge(97); % linking horizontal edges
			\draw[edge] (86)edge(p2m1) (87)edge(p2m1); % left triangle
			\draw[edge] (96)edge(p2m2) (97)edge(p2m2); % right triangle
			\draw[bedge] (p2m1)edge(p2m2) (p2t1)edge(p2t2);
			\draw[bedge] (p2m1)edge(p2t1) (p2m2)edge(p2t2) (p2m1)edge(p2t2);
			%%%% Train
			\draw[edge] (96)edge(97) (A6)edge(A7); % vertical
			\draw[edge] (96)edge(A6) (97)edge(A7); % horizontal
			\draw[edge] (96)edge(A7); % diagonal
			%%%% Extensions
			\node[] (A6d) at (5.0 , 3.0) {};
			\node[] (A7d) at (5.0 , 3.5) {};
			\draw[edge] (A7)edge(A7d) (A6)edge(A6d);
			\node[] (p2t1d) at (4.0 , 2.5) {};
			\node[] (p2t2d) at (4.25, 2.5) {};
			\draw[bedge] (p2t1)edge(p2t1d) (p2t2)edge(p2t2d);
			%%%% Labels
			\node[] at (2.625, 3.125) {$\hat{x}_2$};
			\node[] at (3.125, 2.625) {$\hat{x}_2$};
			\node[] at (5.00 , 3.25 ) {$\hat{x}_2$};
			\node[] at (4.125, 2.625) {$t$};

			%%%%%%%%%%%%%%%%%%%%%%%%%%%%%%%%%%%%%%%%%%%%%%%%%%%%%%%%%%%%%%%%%%%%%%%%%%%%
			%%% x_3
			%%%%%%%%%%%%%%%%%%%%%%%%%%%%%%%%%%%%%%%%%%%%%%%%%%%%%%%%%%%%%%%%%%%%%%%%%%%%
			\node[vertex]    (64)   at (3.0 , 2.0 ) {};
			\node[vertex]    (74)   at (3.5 , 2.0 ) {};
			\node[vertex]    (73)   at (3.5 , 1.5 ) {};
			\node[vertexSig] (84)   at (3.75, 2.0 ) {};
			\node[vertexSig] (83)   at (3.75, 1.5 ) {};
			\node[vertexSig] (94)   at (4.5 , 2.0 ) {};
			\node[vertexSig] (93)   at (4.5 , 1.5 ) {};
			\node[vertex]    (A4)   at (4.75, 2.0 ) {};
			\node[vertex]    (A3)   at (4.75, 1.5 ) {};
			\node[vertexSig] (p3m1) at (4.0,  1.75) {}; % prism 3 middle
			\node[vertexSig] (p3m2) at (4.25, 1.75) {};
			\node[vertex]    (p3t1) at (4.0,  2.25) {}; % prism 3 true
			\node[vertex]    (p3t2) at (4.25, 2.25) {};
			%%%% Construction
			\draw[edge] (75)edge(64) (64)edge(53) (53)edge(73) (74)edge(73) (64)edge(73);
			\draw[edge] (73)edge(83) (74)edge(84) (73)edge(84) (75)edge(74) (64)edge(74) (83)edge(84);
			%%%% Prism
			\draw[bedge] (84)edge(94) (83)edge(93); % linking horizontal edges
			\draw[edge] (84)edge(p3m1) (83)edge(p3m1); % left triangle
			\draw[edge] (94)edge(p3m2) (93)edge(p3m2); % right triangle
			\draw[bedge] (p3m1)edge(p3m2) (p3t1)edge(p3t2);
			\draw[bedge] (p3m1)edge(p3t1) (p3m2)edge(p3t2) (p3m1)edge(p3t2);
			%%%% Train
			\draw[edge] (94)edge(93) (A4)edge(A3); % vertical
			\draw[edge] (94)edge(A4) (93)edge(A3); % horizontal
			\draw[edge] (94)edge(A3); % diagonal
			%%%% Extensions
			\node[] (A4d) at (5.0 , 2.0) {};
			\node[] (A3d) at (5.0 , 1.5) {};
			\draw[edge] (A3)edge(A3d) (A4)edge(A4d);
			\node[] (p3t1d) at (4.0 , 2.5) {};
			\node[] (p3t2d) at (4.25, 2.5) {};
			\draw[bedge] (p3t1)edge(p3t1d) (p3t2)edge(p3t2d);
			%%%% Labels
			\node[] at (3.125, 2.375) {$\hat{x}_3$};
			\node[] at (2.625, 1.875) {$\hat{x}_3$};
			\node[] at (5.00 , 1.75 ) {$\hat{x}_3$};
			\node[] at (4.125, 2.375) {$t$};

		\end{tikzpicture}
		\caption[Gadget for a clause with the cycle \( C_i \)]{
			\centering The gadget for the clause
			$(\hat{x}_{i,1} \lor \hat{x}_{i,2} \lor \hat{x}_{i,3})$, index $i$ omitted in the labels.}%
		\label{fig:proof_clause}
	\end{figure}

	Note that for the variable gadgets we add
	a fixed number of vertices and edges bounded by $O(n)$ and
	for the clause gadgets the number is bounded by $O(k)$.
	Therefore, the whole graph size is bounded by $O(n+k)$
	and the graph can be constructed in polynomial time.
	Also, note that the maximum degree is five.

	We prove that the graph $G_\phi$ has a NAC-coloring if and only if
	$\phi$ is satisfiable.
	Suppose we first have a NAC-coloring $\delta$ of $G_\phi$.
	Let the train $t$ be blue.
	We derive some properties of the NAC-coloring from the graph.
	We prove that the trains $x_i$ and~$\bar{x}_i$
	are colored with different colors for every $i$
	and the train $f$ is red.

	Assume for contradiction that the train $f$ is blue.
	Then the trains $x_i$ and $\bar{x}_i$ have the same color for all $i$,
	otherwise $B_i$ would form an almost cycle.
	Since every edge is labelled by a literal and NAC-coloring is surjective,
	there is literal $x_j$ such that the trains~$x_j$ and~$\bar{x}_j$ are red.
	But then the cycle $A_j$ is an almost cycle, which is a contradiction.
	Hence, the train $f$ is red.
	From the cycles $B_i$ we also see
	that trains $x_i$ and $\bar{x}_i$ have to be colored with different colors
	for every $i$.

	Now we create the related truth assignment.
	For each variable $x_i$ we assign \true{} if
	the train $x_i$ is blue,
	otherwise \false{}.
	Each cause $L_i$ is satisfied since
	in every cycle $C_i$, at least one of
	the literals $\hat{x}_{i,j}$ corresponds to blue colored
	edges, otherwise an almost red cycle is formed.
	Therefore, a truth assignment for $\phi$ can be obtained
	from a NAC-coloring of~$G_\phi$ in polynomial time.

	Now we prove that for every truth assignment such that $\phi$ evaluates to \true{}, there exist
	a NAC-coloring of $G_\phi$. We define an edge coloring
	$\delta: E(G_\phi) \to \{\red, \blue\}$ as follows:
	the edges labelled with $t$ and $f$ are blue and red respectively,
	and an edge labelled by literals $x_i$, resp.\ $\bar{x}_i$, is blue
	if $x_i$, resp.\ $\bar{x}_i$, evaluates to \true{} in the truth assignment, red otherwise.
	Since $t$ and $f$ have different colors,
	the coloring $\delta$ is surjective.

	Suppose there is an almost cycle $C$.
	Let $e=uv$ be the edge of the almost cycle $C$ that has the opposite color
	than all the other edges of $C$.
	The vertices $u$ and $v$ must be contained in multiple \trcon{} components
	since these are monochromatic.
	In the gadgets, all such possibilities for $e$ are indicated by edges with yellow end vertices.
	The edge~$e$ cannot be in any train since there are no two adjacent vertices that are also
	in some other \trcon{} component.

	Now we use the fact that both $u$ and $v$ must be incident to edges of both colors.
	The gadget in \Cref{fig:proof_enforce_true_false} with cycle $A_i$
	cannot contain $e$ since exactly one of the two yellow vertices is incident only to blue edges
	as either $x_i$ or $\bar{x}_i$ are colored blue.
	The edge $e$ is not in the gadget with cycle $B_i$ either,
	since there is a pair of opposite vertices of the cycle $B_i$
	such that one vertex is incident only to blue edges and the other one only to red ones
	as $x_i$ is blue and $\bar{x}_i$ is red, or the other way around.

	For the third gadget in \Cref{fig:proof_clause},
	suppose first that $e$ is the edge in cycle $C_i$ labelled $t$.
	Since it is blue, the edges labelled $\hat{x}_{i,1}$ and $\hat{x}_{i,3}$ are red.
	The edges labelled $\hat{x}_{i,2}$ are blue since the $i$-th  clause evaluates to \true{}.
	Hence, nor $C_i$ nor any other cycle inside the gadget is an almost red cycle.
	Since every cycle containing $e$ that does not lie entirely in the gadget
	has to pass through some of the 3-prism subgraphs in the gadget, it contains another blue edge labelled by $t$.
	Analogous argument applies also for $e$ being any other edge in the gadget labelled by $t$.
	It remains to consider when $e$ is one of the edges in a 3-cycle of the 3-prism subgraphs.
	But in this case $C$ cannot be an almost cycle either since both triangles in each 3-prism are colored the same.
	%
\end{proof}

It may not be obvious why 3-prisms in clause gadget
(\Cref{fig:proof_clause})
are really needed. We present a counter example.
Let us have a 3-CNF formula \( \phi = (A \lor B \lor C) \land (\lnot A \lor D \lor B) \).
For each satisfiable truth assignment, there should be a NAC-coloring in \( G_\phi \).
%
This formula is satisfiable for example if \( C \) and \( D \) are assigned \( \true \)
and all the other literals are assigned \( \false \).
If we create corresponding \( \red \)-\( \blue \)-coloring in \( G_\phi \)
where the cause gadgets do not have the 3-prisms, an almost cycle is created:
%
the cycle starts at \( A \) in the first cause, goes through connecting
train to \( A \) section in the second cause. There, it uses the \( \blue \)
\( \true \) edge and goes to \( B \). Using a connecting train,
it goes back to the \( B \) segment in the first cause.
Here, it joins back to \( A \) as the segments share a vertex.
All edges corresponding to \( A \) and \( B \)
are \( \red \), and we used only a single \( \blue \) edge.
Therefore, an almost cycle exists, and the coloring is not a NAC-coloring.
If the prisms are used, four more blue edges are visited with an analogous cycle.


\section{Average degree \( 4 + \varepsilon \)}

The previous result can be extended further.
We can extend ladder parts of the graph by adding
edges and vertices with degree four.
By doing this, we can decrease the average degree as close to four as we need.

\begin{theorem}%
	\label[theorem]{theorem:nac-eps}
	%
	For every $\varepsilon>0$,
	the question whether a NAC-coloring exists is NP-complete for graphs with $|E(G)| \leq (2 + \varepsilon) |V(G)|$.
	%
\end{theorem}
\begin{proof}
	%
	Fix $\varepsilon>0$. We can assume $\varepsilon<\frac{1}{2}$.
	The proof of \Cref{theorem:nac-deg-5} applies once we show that
	we can create graph~$G'_\phi$ from the graph $G_\phi$ constructed for a formula $\phi$
	such that $|E(G'_\phi)| \leq (2 + \varepsilon) |V(G'_\phi)|$, and
	the graph $G'_\phi$ has a NAC-coloring if and only if $G_\phi$ has a NAC-coloring.
	We take any braced ladder in a train of $G_\phi$
	and extend it $k$-times as indicated in \Cref{fig:proof_trains}.
	Hence, $|E(G'_\phi)| = |E(G_\phi)|+4k$ and $|V(G'_\phi)| = |V(G_\phi)|+2k$.
	Since we modify only a \trcon{} component,
	there is a bijection between NAC-colorings of $G'_\phi$ and $G_\phi$.
	Using the fact that the maximum degree of $G_\phi$ is 5,
	the required inequality holds for any  $k\geq\left\lceil V\frac{1-2\varepsilon}{4\varepsilon}\right\rceil$.
	%
\end{proof}

Complexity relation to stable cuts is described in \Cref{chapter:stable_cuts}.

