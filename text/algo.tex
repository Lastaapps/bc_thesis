\chapter{Algorithms for NAC-coloring search}%
\label{chapter:algo}

\begin{chapterabstract}

	As shown before, it is NP-complete to decide if a~graph has a~NAC-coloring.
	Therefore, we cannot hope for an~algorithm that can solve this problem in polynomial time.
	The~trivial idea how to search for NAC-colorings is to
	try all the~\( \red \)-\( \blue \)-edge colorings
	and check in polynomial time if a~coloring given is a~NAC-coloring.
	In this chapter, we propose multiple optimizations to the~naive approach
	and an~algorithm that significantly outperforms the naive approach in general case.

\end{chapterabstract}

The~goal of this section is to propose algorithms to find some or all NAC-colorings of a~given graph.
After recalling the~solution used in~\flexrilog{},
we describe an~improvement of the~idea of \trcon{} components in~\Cref{sec:NACvalid}
and checking whether a~coloring is a~NAC-coloring in~\Cref{sec:small_cycles}.
In~\Cref{sec:combining} we sketch the~idea of algorithms that
exploit combining NAC-colorings of subgraphs.
We propose heuristics for decomposing into subgraphs and merging strategies
in~\Cref{sec:decomposition,sec:merging} respectively.
Although the~proposed approaches are written so that all NAC-colorings are generated,
the~existence of a~NAC-coloring can be checked easily by stopping after finding a~first one.

This chapter is based on our previous work in~\cite{my_paper},
here we describe the~approaches used more in detail
and also describe other approaches that do not work as well as the~approaches
shown in the~paper. We evaluate the~performance
of the~algorithms and related heuristics in~\Cref{chapter:benchmarks}.

\section{Naive approach}

A~naive approach to list all NAC-colorings is to consider
all $2^{|E(G)|} - 2$ surjective \( \red \)-\( \blue \)-edge colorings of graph $G$
and check each of them in polynomial time using \Cref{lemma:is_nac_coloring}.
We call this check \IsNACColoring{}$(G, E_r, E_b)$.
To remind the~reader, the~idea is simple
--- first we create subgraphs \( G[E_\red], G[E_\blue] \).
If there is w.l.o.g.\ \( \blue \) edge \( \{u, v\} \)
and vertices \( u, v \) share the~same connected component in~\( G[E_\red] \),
an~almost cycle is formed from a~\( \red \) \( u \)-\( v \)-path in~\( G[E_\red] \)
and the~\( \blue \) edge \( \{u, v\} \).

The~algorithm we call \Naive{} runs in~\( O(2^{|E(G)|}) \) time.
It can be sped up dramatically by considering sets of edges
where it can be easily seen that all the~edges in the~set need to share
the~same color in every NAC-coloring.
%
Now we can color the~whole sets at once
and reduce the~search space and time complexity to \( O(2^{c}) \)
where \( c \) is the~number of described sets.
One example of such a~set are \trcon{} components
as defined in~\Cref{def:triangle_connected_component}.
In the~following section, we extend upon the~idea of \trcon{} components
to reduce the~search space even further.

Another simple but important observation is that the~whole search space \( O(2^{c}) \)
does not need to be traversed, as a~NAC-coloring with colors swapped is again a~NAC-coloring.
Therefore, we fix the~color of one set of edges and additionally generate NAC-colorings with colors swapped.
This cuts runtime in half%
\nohznamka{
	To learn more on the~topic of cutting in half, we recommend
	\href{https://youtu.be/FZ4qeJly1g0}{youtu.be/FZ4qeJly1g0}.
}
to \( O(2^{c-1}) \).


The~algorithm implemented in~\flexrilog{}
considers only colorings such that \trcon{} components are monochromatic,
and also ignores NAC-colorings with colors swapped.

\section{NAC-valid relations}%
\label{sec:NACvalid}

While the~idea of \trcon{} components is that 3-cycles are monochromatic,
there are other cases when two edges have to have the~same color,
which can further reduce the~search space.
Namely, we aim to find a~partition of the~edge set such that each part is necessarily monochromatic
in every NAC-coloring and the~number of parts is as small as possible.

\begin{definition}[NAC-valid~\cite{my_paper}]%
	\label[definition]{def:NACvalid}
	%
	An~equivalence relation $\sim$ on the~edge set
	of a~graph $G$ is called \emph{NAC-valid}
	if for every NAC-coloring $\delta$ of $G$ it holds that
	$\forall e_1, e_2 \in E(G) :
		e_1 \sim e_2 \Rightarrow \delta (e_{1}) = \delta (e_{2})$.
	An~equivalence class of a~NAC-valid relation is called a~\emph{monochromatic class}.
	%
\end{definition}
%
With that, we also define terms used with monochromatic classes:
%
\begin{itemize}
	\item The~\emph{vertices} of a~monochromatic class $M$
	      are the~vertices of subgraph $G[M]$.
	\item We call two distinct monochromatic classes \( M_1, M_2 \) \emph{neighboring},
	      if they share at least one common vertex.
	      % if the~intersection of the~vertices of \( M_1, M_2 \) is non-empty.
	\item The~\emph{degree} of a~monochromatic class \( M \) is the~number of
	      monochromatic classes that are neighbors of \( M \).
\end{itemize}
%
\begin{observation}%
	\label[observation]{observ:monochromatic_classes_graph}
	%
	A~graph can be formed from monochromatic classes:
	vertices are the~monochromatic classes, and there is an~edge incident to
	two monochromatic classes if they are neighboring.
\end{observation}
%

Clearly, the~relation inducing \trcon{} components is NAC-valid.
The~following lemma describes a~way how to construct a~new NAC-valid relation from another one
with possibly less monochromatic classes.

\begin{lemma}[\cite{my_paper}]%
	\label[lemma]{lemma:two_edges_and_component}
	%
	Let $\sim$ be a~NAC-valid relation on $G$ and
	$\sim^\prime$ be a~relation on $E(G)$ given by
	$e_{1} \sim^\prime e_{2}$ if and only if
	$e_{1} \sim e_{2}$ or there exists
	a~cycle $C$ in $G$ such that $e_{1}, e_{2}$
	are edges in $C$
	and all other edges are in the~same monochromatic class of $\sim$.
	Then the~reflexive-transitive closure of $\sim^\prime$ is NAC-valid.
\end{lemma}
%
\begin{proof}
	%
	The~condition in~\Cref{def:NACvalid}
	is preserved by undertaking reflexive-transitive closure,
	so it suffices to check it only for $\sim^\prime$: if $e_1$ and $e_2$ had different colors in some NAC-coloring,
	then $C$ would be an~almost cycle since $C - \{e_1,e_2\}$ is monochromatic as the~relation $\sim$ is NAC-valid.
	%
\end{proof}

Note that various situations can occur:
$e_1$ or $e_2$ is in the~same a~monochromatic class of~$\sim$ as the~rest of $C$,
or neither of them is, see \Cref{fig:two_edges_and_component}.
If each monochromatic class of $\sim$ induces a~connected subgraph,
then the~edges $e_1$ and $e_2$ can be supposed to be incident,
but this is not the~case in general.
For instance, one could introduce a~NAC-valid relation such that both triangles in a 3-prism subgraph
are in the~same monochromatic class, since they have the~same color in all NAC-colorings of the 3-prism.

\begin{figure}[h]
	\centering
	\begin{tikzpicture}[scale=2]
		\node[vertex] (0) at (0, 0) {};
		\node[vertex] (1) at (1, 0.5) {};
		\node[vertex] (2) at (2, 0) {};
		\node[vertex] (3) at (0.5, 0.866) {};
		\node[vertex] (4) at (1.5, 0.866) {};
		\draw[redge] (0)edge(1) (1)edge(2) (0)edge(3) (1)edge(3) (1)edge(4) (2)edge(4) (3)edge(4)  ;
		\draw[edge]  (0)edge(2)  ;
	\end{tikzpicture}
	\qquad
	\begin{tikzpicture}[scale=2]
		\node[vertex] (0) at (0, 0) {};
		\node[vertex] (1) at (1, 0.5) {};
		\node[vertex] (2) at (2, 0) {};
		\node[vertex] (3) at (0.5, 0.866) {};
		\node[vertex] (4) at (1.5, 0.866) {};
		\node[vertex,label={north:$v$}] (5) at (1, 0) {};
		\draw[redge] (0)edge(1) (1)edge(2) (0)edge(3) (1)edge(3) (1)edge(4) (2)edge(4) (3)edge(4)  ;
		\draw[edge]  (0)edge(5) (2)edge(5)  ;
	\end{tikzpicture}
	\caption[Monochromatic classes advantage]{
		An~edge with adjacent vertices in a~monochromatic class (left) and
		two edges with adjacent vertices in a~monochromatic class (right).}%
	\label{fig:two_edges_and_component}
\end{figure}

Notice that application of \Cref{lemma:two_edges_and_component} corresponds
to merging the~monochromatic classes of $e_1$ and $e_2$.
We propose to use the~following NAC-valid relation:
we start with the~\trcon{} components and then apply the~following two steps
as long as there is some change:
%
\begin{enumerate}
	\item If there is an~edge \( \{u, v\} \) such that $u$ and $v$ are vertices of
	      the~same monochromatic class $M$, then merge $M$ with the~monochromatic class of \( \{u, v\} \).
	\item If there are edges \( \{u, v\} \) and \( \{v, w\} \) such that $u$ and $w$ are vertices of
	      the~same monochromatic class, then merge the~monochromatic classes of \( \{u, v\} \) and \( \{v, w\} \).
\end{enumerate}
%
Every \trcon{} component induces a~connected subgraph, and also the~two operations preserve
the~fact that each monochromatic class induces a~connected subgraph.
Hence, the~resulting partition indeed forms monochromatic classes
of a~NAC-valid relation by \Cref{lemma:two_edges_and_component}.
The~construction can be done in polynomial time
and implemented efficiently using a~Union-find data structure
as presented in the~pseudocode of \Cref{alg:create_monochromatic_classes}.

\begin{algorithm}
	\caption[Create Monochromatic classes]{Create Monochromatic classes~\cite{my_paper}}%
	\label{alg:create_monochromatic_classes}
	\begin{algorithmic}[1]
		\Require{} $G$

		\Ensure{} $P \gets ()$
		\Comment{} list of monochromatic classes

		\State{} $U \gets \Call{CreateUnionFind}{E(G)}$
		\Comment{} create Union-find data structure

		\ForAll{$e_{1},e_{2},e_{3} \in \Call{FindAllTriangles}{G}$}
		\State{} $U \gets \Call{Join}{U, e_{1}, e_{2}, e_{3}}$
		\EndFor{}

		\State{} $U^\prime \gets \emptyset$
		\While{$U^\prime \not= U$}
		\State{} $U^\prime \gets U$
		\Comment{} as long as changes happen

		\State{} $\lambda : V(G) \to \mathcal{P}(E(G))$
		\Comment{} monochromatic classes a~vertex is member of,
		\State{}
		\Comment{} classes are represented by a~single edge, defaults to $\emptyset$

		\ForAll{$\{u, v\} \in E(G)$}
		\State{} $e^\prime \gets \Call{Find}{U, \{u, v\}}$
		\State{} $\lambda(u) \gets \lambda(u) \cup \{e^\prime\}$
		\State{} $\lambda(v) \gets \lambda(v) \cup \{e^\prime\}$
		\EndFor{}

		\ForAll{$v \in V(G)$}
		\If{$\exists u, w \in N(v) : \lambda(u) \cap \lambda(w) \not= \emptyset$}
		\State{} $U \gets \Call{Join}{U, \{v, u\}, \{v, w\}}$
		\Comment{} edges are over a~same component
		\EndIf{}
		\EndFor{}
		\EndWhile{}

		\State{} $P \gets \Call{Sets}{U}$

	\end{algorithmic}
\end{algorithm}

With these improvements, the~complexity of the~naive algorithm can often be
reduced even further.

\section{Small cycles}%
\label{sec:small_cycles}

The~check \IsNACColoring{} can become computationally expensive
considering it is called for every possible coloring,
even though it can be implemented as simple breadth-first-search.
An~optimization we use is to keep a~list of shorter cycles in the~graph and
check if they are almost monochromatic cycles before doing the~full check.
We call this improved \Naive{} algorithm \NaiveCycles{}.

The~check for a~single cycle can be done in linear time
in the~number of monochromatic classes using bitwise arithmetic.
One cycle $C$ can reject up to $\frac{1}{|E(C)|}$ colorings,
which can be significant for small cycles.

There are often many cycles of different lengths in the~graph,
and for the~algorithm, only a~subset of them can be chosen.
If we choose too few, the~whole potential of the~approach is not used, if too much,
we spend too much time checking the~masks while saving little
by avoiding \IsNACColoring{} checks and performance degrades.
The~number of cycles should also grow with the~number of edges,
resp.\ monochromatic classes.
As stated above, smaller cycles are preferred.

Following that, we propose the~following strategy.
Given a~graph $G$ and $k,\ell\in \mathbb{N}$,
we adopt the~idea to monochromatic classes as follows:
for every monochromatic class $M$ in $G$,
we consider all edges $\{u, v\} \in M$ such that there is a~path from $u$ to $v$
in $G - M$ that uses at most $k$ monochromatic classes.
We pick up to $l$ cycles constructed from $\{u, v\}$ with the~corresponding
path with the~least number of used monochromatic classes.
In our implementation, we take $k=4$ and $l=2$ cycles for every monochromatic class.

\section{Combining NAC-colorings of subgraphs}%
\label{sec:combining}

In our algorithm called \Subgraphs{}, a~graph is decomposed into subgraphs so that
the~NAC-colorings of the~graph can be obtained from the~NAC-colorings of the~subgraphs.
We introduce notation for this process of combining NAC-colorings of suitable subgraphs.

\begin{definition}[\cite{my_paper}]%
	\label[definition]{def:NACproduct}
	%
	Let $G$ be a~graph with subgraphs $G_1, \dots, G_k$
	s.\ t.\ $\bigcup_{i=1}^k E(G_i) =E(G)$.
	For $1\leq i \leq k$, let $\delta^{i}_{\red}$ and
	$\delta^{i}_{\blue}$ be the~monochromatic colorings of $G_i$.
	The~set
	\[ \Bigl\{
		\text{surjective }\delta : E(G) \to \{\blue, \red\}
		\mid\forall i : \delta |_{E(G_i)} \in
		\nac{G_i} \cup \{\delta^{i}_{\blue}, \delta^{i}_{\red}\}
		\Bigr\}
	\]
	is called the~\emph{NAC-product} of $G_1, \dots, G_k$ and denoted by $\CP(G_1, \dots, G_k)$.
	%
\end{definition}

Since the~restriction of a~NAC-coloring of a~graph
to its subgraph is either a~NAC-coloring or monochromatic,
we have $\nac{G_{1} \cup G_{2}} \subseteq \CP(G_{1}, G_{2})$.
If~$G_1, \dots, G_k$ are the~connected components of $G$,
then it holds that $\nac{G} = \CP(G_1,\ldots, G_k)$.
Since every cycle is contained in a~single
block%
\footnote{
	A~\emph{block} is a~single edge or a~maximal 2-connected subgraph.
}
of a~graph, we have that $\nac{G} = \CP(G_1,\ldots, G_k)$
for $G_1,\ldots, G_k$ being the~blocks of $G$.

In order to design a~faster algorithm for searching for all NAC-colorings,
we exploit the~fact that for edge-disjoint graphs $G_1$ and $G_2$,
it is straightforward to construct $\CP(G_{1}, G_{2})$ once we know $\nac{G_1}$ and $\nac{G_2}$.
Then, we can get $\nac{G_{1} \cup G_{2}}$ by applying \IsNACColoring{}
to each coloring in $\CP(G_{1}, G_{2})$, see also \Cref{alg:coloring_product}.
This can significantly reduce the~number of \IsNACColoring{} calls
compared to testing all red-blue-colorings of $G_{1} \cup G_{2}$.
To apply the~idea on a~graph $G$, we:
%
\begin{enumerate}
	\item decompose $G$ into pairwise edge-disjoint subgraphs $G_1, \ldots, G_\ell$, and
	\item compute the~NAC-product $\CP(G_1, \dots, G_\ell)$ using $\nac{G_1}, \ldots, \nac{G_\ell}$
	      and filter it to get $\nac{G}$.
\end{enumerate}
%
It is also important to note that in case the~merged subgraphs
have no vertices in common, we get the~whole NAC-product.
No filtering is needed in such a~case.
%
In the~following two subsections, we discuss possible heuristics
for these two phases.

\begin{algorithm}
	\caption{NAC-product with filtering}%
	\label{alg:coloring_product}
	\begin{algorithmic}[1]
		\Require{} $G_1, G_2$
		\Comment{} subgraphs
		\Require{} $N_1, N_2$
		\Comment{} NAC-colorings of the~subgraphs (pairs of sets of edges)

		\Ensure{} $G$
		\Comment{} merged subgraphs
		\Ensure{} $S \gets \emptyset$
		\Comment{} the~NAC-colorings of $G$

		\State{} $G = (V(G_1)\cup V(G_2), E(G_1) \cup E(G_2))$

		\State{} $N_1 \gets N_1 \cup \{(E(G_1), \emptyset), (\emptyset, E(G_1))\}$
		\Comment{} the~subgraphs can also be monochromatic
		\State{} $N_2 \gets N_2 \cup \{(E(G_2), \emptyset), (\emptyset, E(G_2))\}$

		\ForAll{$(E_{r_{1}}, E_{b_{1}}) \in N_1$}
		\ForAll{$(E_{r_{2}}, E_{b_{2}}) \in N_2$}
		\State{} $E_r \gets E_{r_{1}} \cup E_{r_{2}}$
		\State{} $E_b \gets E_{b_{1}} \cup E_{b_{2}}$
		\If{$\Call{IsNACColoring}{G, E_r, E_b}$}
		% \Comment{} cycles check can be also used here
		\State{} $S \gets S \cup \{(E_r, E_b)\}$
		\EndIf{}
		\EndFor{}
		\EndFor{}
	\end{algorithmic}
\end{algorithm}

\subsection{Subgraph decomposition}%
\label{sec:decomposition}

On the~input, we assume the~list of monochromatic classes
and an~integer $k\geq 1$, and we output a~list of subgraphs
such that each has at most $k$ monochromatic classes.
We describe some heuristics how to split a~graph into edge disjoint subgraphs.
In contrast with~\cite{my_paper}, we also discuss heuristics
that did not perform well or have other specific limitations.

\subsubsection*{\None{}}

The~strategy with the~least overhead costs called \None{}
is to take the~chunks of $k$ consecutive
monochromatic classes in the~vertex input list.
Some synthetically generated graphs are formed in such a~way
that \None{} ends up performing as good as the~other strategies,
this strategy degrades when the~vertex input list is shuffled.

Our general goal for the~following strategies
is to maximize the~number of suitable cycles in each
subgraph as cycles may form almost cycles and those help us
to reduce the~search space.

\subsubsection*{\Neighbors{}}

We start with a~heuristic that we call \Neighbors{},
its pseudocode is in~\Cref{alg:neighbors}.
The~algorithm aims to mimic breadth-first-search
behavior while respecting monochromatic classes.
The~goal is to find monochromatic classes that are close together.
Let $P$ be the~list of monochromatic classes,
for simplicity we assume that $k$ divides $|P|$.
We will gradually divide the~monochromatic classes into bags.
The~output is a list of subgraphs induced by the~monochromatic classes in each bag.

We start with $\frac{|P|}{k}$ empty bags
and all vertices of the~graph labelled $open$.
First, we add a~random of the~remaining monochromatic classes to a~bag.
All the~vertices of the~monochromatic classes in the~bag are denoted by $used$.
We take all $open$ vertices that are neighbors of the~vertices in $used$ and assign
them a~score. A~vertex $best$ with the~highest score is chosen.
The~monochromatic classes corresponding to edges connecting the~vertex $best$
with vertices in $used$ are then added to the~bag while not exceeding its size.
The~$used$ set is updated, and the~algorithm continues until the~bag is full.
If $open$ is empty, this iteration of the~search also ends.
A~new bag is chosen, and we repeat the~process.

The~first strategy \Neighbors{} takes as the~score the~number of
neighboring vertices of a~vertex in the~$used$ set.
The~other strategy \NeighborsDegree{} is based
on the~first and adds another rule ---
if the~number of neighbors matches, the~vertex with lower degree is chosen.
%
Other variations of this approach were also tried,
but the~performance was not better.


\begin{algorithm}[ht]
	\caption{Heuristic \Neighbors}%
	\label{alg:neighbors}
	\begin{algorithmic}[1]

		\Require{} $G$
		\Comment{} a~graph
		\Require{} $P$
		\Comment{} monochromatic classes of $G$
		\Require{} $k$, $b \gets |P|/k$
		\Comment{} target size of bags, the~number of bags

		\Ensure{} $O \gets (B_1, \ldots, B_b)$
		\Comment{} bags with monochromatic classes

		\While{$P \ne \emptyset$}
		\State{} $p \gets p \in P$
		\State{} $P \gets P \setminus \{p\}$
		\Comment{} random monochromatic class

		\State{} $B \gets \underset{B_i \in O}{\arg\min} |B_i|$
		\Comment{} the~most empty bag

		\State{} $used \gets \{{v \in M} \mid {M \in B } \}$
		\Comment{} vertices already used by this bag
		\State{} $open \gets  \{w \in N(u) \mid u \in used\} \setminus used $
		\Comment{} neighbors, candidates for addition

		\While{$|open| > 0 \land |B| < k$}

		\State{} $best \gets \underset{u \in open}{\arg\max}|used \cap N(u)|$
		\Comment{} vertices with the~largest neighborhood

		\For{$n \in N(best) \land |B| < k$}
		\State{} $p \gets p \in P$ such that $\{best, n\} \in p$
		\State{} $P \gets P \setminus \{p\}$

		\State{} $B \gets B \cup \{p\}$

		\State{} $used \gets used \cup \{v \mid {v \in p} \}$
		\State{} $open \gets  \{w \in N(u) \mid u \in used\} \setminus used $
		\EndFor{}
		\EndWhile{}
		\EndWhile{}
	\end{algorithmic}
\end{algorithm}

\subsubsection*{\CyclesMatchChunks{}}

Monochromatic classes are first ordered descending by their degree.
Then cycles are found for each monochromatic class with the~same procedure
as described in~\Cref{sec:small_cycles}.
The~algorithm, modified version of the~breadth-first-search, goes as follows:
when a~stage starts, we take a~component with the~highest degree
that has not been marked as used yet and add it to a~queue.
While the~queue is non-empty, we pop the~first monochromatic class,
add it the~result list and mark it as used.
Then we traverse all the~cycles (that we obtained before)
corresponding to the~component
and add monochromatic classes from the~cycles to the~queue.
The~stage ends when the~queue is empty or when \( k \) monochromatic classes
were marked as used. The~whole process is repeated until
all monochromatic classes are marked as used.

\subsubsection*{\KernighanLin{}, \Cuts{}}

Kernighan-Lin algorithm~\cite{kernighan_lin} proposes a~heuristic based algorithm
for partitioning a~graph into two parts. It tires to minimize the~edge cut between
the~parts by iteratively swapping pairs of vertices from the~parts.

As noted in~\Cref{observ:monochromatic_classes_graph},
monochromatic classes can be converted into a~graph.
In strategy \KernighanLin{},
Kernighan-Lin bisection algorithm is recursively run to split the~graph
into subgraphs till the~subgraphs are smaller than \( k \) in the~number of vertices.
We hope that smaller-edge cuts will preserve more cycles among
the~monochromatic classes that often also represent cycles in the~real graph.

We also tried approach \Cuts{}, where Kernighan-Lin algorithm is replaced by
more classical flow-based edge cuts to split graph into parts.
With that, we lose the~guarantee that the~parts in a recursive step
are similar in size. Therefore, the~recursive tree may be unbalanced.

Other strategies based on
sorting monochromatic classes by their degree
or breadth-first-search were also tried,
but with no meaningful results.



\subsection{Subgraph merging}%
\label{sec:merging}

After constructing the~ordered list of edge-disjoint subgraphs,
we compute all the~NAC-colorings for each of them using
the~naive algorithm with monochromatic classes
and improved check on cycles.
Then the~results need to be merged to obtain the~NAC-colorings of the~original graph
using the~NAC-product.

Since checking every coloring in the~NAC-product is a~costly operation,
we try to minimize the~work that has to be done.
The~complexity of the~task grows with the~size of
the~merged subgraphs (as \IsNACColoring{} depends on the~size of the~graph checked)
and also with the~number of NAC-colorings found in each subgraph
(which can grow exponentially).
We describe multiple merging strategies.
As before, we extend the~list we presented in~\cite{my_paper}.

\subsubsection*{\MergeLinear{}, \SharedVertices{}}

The~first approach \MergeLinear{} is
to take the~sequence of subgraphs and merge them one by one.
We merge the~first and the~second subgraph, then we merge this result with the
third one and so on.
%
Another approach called \SharedVertices{} always merges the~two subgraphs that
have the~most vertices in common with the~goal
to create as many new cycles as possible.

\subsubsection*{\Log{}}

Next approach called \Log{} tries to merge subgraphs in a~tree-like manner.
For an~ordered list of subgraphs,
for simplicity let its size be \( 2^c \) for some \( c \in \N \).
The~list is split into pairs of consequent subgraphs and NAC-colorings
of the~subgraphs in pairs are merged. With that, a~new list is created.
This continues until we have the~original graph and all its NAC-colorings.

\subsubsection*{\SortedBits{}, \MinMax{}}

For the~following strategies,
we consider the~number of monochromatic classes
that induces a~given subgraph as its size.
%
One heuristic \SortedBits{} is that runtime can be reduced
when we first merge smaller subgraphs as less processing power is
required for smaller graphs.
Subgraphs are always sorted by their size, and two smallest are merged
till we obtain the~original graph.
%
The~other heuristic \MinMax{} based on an~opposite idea
merges small subgraphs with large subgraphs.
First, the~smallest subgraph is merged with the~largest one,
then second smallest with the~second largest and so on.
The~idea is that work is minimized on average
as we prevent merges of large graphs.

\subsubsection*{\PromisingCycles{}}

Following strategy is called \PromisingCycles{}.
The~strategy tries to guess which subgraphs
have the~most potential almost cycles when joined.
For each pair of subgraphs, we find cycles in joined subgraph using
the~same routine as in~\Cref{sec:small_cycles}.
A~joined graph with the~maximal number of cycles is chosen,
otherwise we score as follows:
%
\begin{itemize}
	\item If subgraphs share a~vertex, the~score grows.
	\item Smaller joined graphs get higher priority.
	\item Subgraphs with more common vertices get higher priority.
\end{itemize}
%

\subsubsection*{\SortedSize{}, \Score{}}

All the~previous merging strategies did not use the~number of NAC-colorings
of individual subgraphs in merging decisions.
That is on purpose as we do not know the~number of NAC-colorings in advance.
With that, if only a~single NAC-coloring is requested by a~user,
we can use lazy iterators in our implementation that
finds a~NAC-coloring of a~subgraph and wait till another coloring is requested.
If all the~NAC-colorings are desired or if we expect the~graph to have no NAC-coloring,
all NAC-colorings of subgraphs need to be traversed.
In such cases, we can list all NAC-colorings of subgraphs first
and employ heuristics that take advantage of this additional knowledge.

The~first proposed heuristic is called \SortedSize{}
that sorts initial subgraphs by the~number of NAC-colorings found,
and merges the~subgraphs with the~smallest number of NAC-colorings.
%
Our preferred strategy of this heuristic family is called \Score{}.
In each iteration, a~score is computed for each pair of subgraphs in the~current list.
For subgraphs \( G_1, G_2 \), formula computes the~score
\( \nnac{G_1} \cdot \nnac{G_2} \cdot \text{sizeof}(G_1 \cup G_2) \).
The~score approximates the~amount of work needed.
The~pair with the~lowest score is chosen.

We also tried approaches based on dynamic programming and recursive optimization,
these approached do not scale well to larger instances.

\subsubsection*{Smart split}%
\label{sec:smart_split}

We already talked about the~importance of subgraphs having vertices in common,
while we perform the~filtering of the~NAC-product.
We also stressed the~order of subgraphs and
how do merging strategies interact with them.
The~goal of the~following algorithm is to create and order subgraphs in a~way that they usually
share property that reduces the~output size while running
the~filtering of the~NAC-product.

The~strategies for subgraph creation are generic --- they accept a~graph and
the~list of target subgraphs sizes bounded by \( k \) as input and output final subgraphs.
We exploit this property here.
We use these strategies to divide the~graph in half by giving it appropriate
sizes (the sum of the~first half of subgraph sizes and the~sum of the~rest).
Then we split the~subgraphs further until we reach desired graph sizes.
We order the~created subgraphs in the~tree traversal order.
By doing that, we hope to ensure that most of the~neighboring subgraph pairs share
some vertices and therefore more cycles appear in the~merged subgraph.

Not all merging strategies as described above do use the~order of subgraphs heavily,
for these strategies we do not expect significant gains.
Also note that \Cref{alg:smart_split} has a~compatible interface with
all the~subgraph creation strategies and therefore can be easily added.

\begin{algorithm}
	\caption{Smart Split}%
	\label{alg:smart_split}
	\begin{algorithmic}[1]
		\Require{} $G$
		\Comment{} Subgraph to work on
		\Require{} $P$
		\Comment{} Monochromatic classes of subgraph $G$
		\Require{} $S$, $s \gets |S|$
		\Comment{} List of the~sizes of subgraphs and their number

		\Ensure{} $O \gets ()$
		\Comment{} List of ordered vertices

		\If{$s \le 2$}
		\State{} $O \gets \Call{MergeStrategy}{G, P, S}$
		\State{}
		\Return{}
		\EndIf{}

		\State{} $S_1, S_2 \gets S[:\lfloor\frac{s}{2}\rfloor], S[\lfloor\frac{s}{2}\rfloor:]$
		\State{} $P \gets \Call{SubgraphStrategy}{G, P, (\sum S_{1}, \sum S_{2})}$

		\State{} $P_1 \gets P[:(\sum S_1)]$
		\State{} $P_2 \gets P[(\sum S_1):]$
		\State{} $G_1 \gets \Call{InducedSubgraphOnEdges}{G, \bigcup P_1}$
		\State{} $G_2 \gets \Call{InducedSubgraphOnEdges}{G, \bigcup P_2}$

		\State{} $O_1 \gets \Call{SmartSplit}{G_1, P_1, S_1}$
		\State{} $O_2 \gets \Call{SmartSplit}{G_2, P_2, S_2}$
		\State{} $O \gets O_{1}.O_{2}$
	\end{algorithmic}
\end{algorithm}



\section{Polynomial optimizations}%
\label{sec:polynomial_optimizations}

In this section, we present and recall some polynomial checks
that can detect if a~graph has a~NAC-coloring and often also provide a~certificate.
Unless otherwise stated, a~certificate can be obtained for all the~following approaches.

We start by recalling some statements from \Cref{chapter:preliminaries}.
%
First, if a~graph~\( G \) with no isolated vertices is not connected,
a~NAC-coloring can be found by coloring different connected components differently.
%
If a~graph~\( G \) has an~articulation vertex or
a~single edge as a~block (often called a~bridge),
a~NAC-coloring can be found in~\( O(|V(G)|+|E(G)|) \) time.
Connectivity can be also checked using the~same algorithms.

All algorithms and approaches described in~\Cref{chapter:stable_cuts}
are also relevant here as a~NAC-coloring can be constructed from a~stable cut.
An~important algorithm can be easily obtained from \Cref{cor:vertex_no_triangle_stable_cut}.
It can be run alongside the~search for monochromatic classes (\Cref{alg:create_monochromatic_classes}).
%
If a~graph is flexible (which always holds for \( |E(G)| \le 2|V(G)|-4 \)),
a~stable cut can be found in polynomial time using \Cref{alg:stable_cut_flexible}.
This also holds for minimally rigid graphs that are not 2-trees as shown in~\Cref{lemma:stable_cut_or_2_tree}.
Currently, a~certificate is not obtainable for such graphs
even though the~algorithm can be described and implemented, see \Cref{sec:stable_cut_algo}.
%
As shown in~\cite{legersky_original}, there are no flexible realizations
of graphs with \( |E(G)| > \frac{n(n-1)}{2} - (n-2) \) where \( n = |V(G)| \).
These graphs always form a~\trcon{} component.
%
Another polynomial algorithm goes from \Cref{lemma:weird_four_cycle}
The~algorithm is not implemented as part of the~thesis.

