\chapter{Algorithm for NAC-coloring search}%
\label{chapter:alg}

\begin{chapterabstract}

	As shown before it is NP-complete to decide if a graph has a NAC-coloring.
	The trivial idea how to search for a NAC-coloring is to try all the colorings
	and check in polynomial time if a coloring given is a NAC-coloring.
	In this chapter we propose multiple optimizations
	and an algorithm that significantly outperforms naive approach in general case.

\end{chapterabstract}

% TODO in the correctness section to reference to Never Gonna give you up
% As proved above, it can be seen that the algorithm will never gonna give you up [x].

The goal of this section is to propose algorithms to find one or all NAC-colorings of a given graph.
After recalling the solution used in \flexrilog{},
we describe an improvement of the idea of \trcon{} components in \Cref{sec:NACvalid}
and checking whether a coloring is a NAC-coloring in \Cref{sec:small_cycles}.
In \Cref{sec:combining} we sketch the idea of algorithms that
exploit combining NAC-colorings of subgraphs.
We propose heuristics for decomposing into subgraphs and merging strategies
in \Cref{sec:decomposition,sec:merging} respectively.
Although the proposed approaches are written so that all NAC-colorings are generated,
the existence of a NAC-coloring can be checked easily by stopping after finding a first one.

A naive approach to list all NAC-colorings is to consider
all $2^{|E(G)|}$ surjective edge colorings of graph $G$ by red and blue
and check each of them in polynomial time using the following lemma.
We call this check \IsNACColoring{}$(G, E_r, E_b)$.

\begin{lemma}[{\cite[Lemma~2.4]{GLS2019}}]
	\label{lemma:is_NAC_coloring}
	%
	Let $G$ be a graph and $\delta: E(G) \to \{\red, \blue\}$ be a surjective edge coloring.
	Let $E_r$ and $E_b$ be the red and blue edges of $G$ respectively.
	The coloring~$\delta $ is a NAC-coloring if and only if
	the connected components of $G[E_r]$ and $G[E_b]$\footnote{
		By $G[F]$ for $F\subset E(G)$ we mean the edge-induced subgraph,
		namely, the subgraph of $G$ having the edges $F$ and the vertex set
		being the end vertices of the edges in $F$.}
	are induced subgraphs of $G$.
	%
\end{lemma}

The algorithm implemented in \flexrilog{}
considers only colorings such that \trcon{} components are monochromatic.

\subsection{NAC-valid relations}%
\label{sec:NACvalid}

While the idea of \trcon{} components is that 3-cycles are monochromatic,
there are other cases when two edge have to have the same color, which can further reduce the search space.
Namely, we aim to find a partition of the edge set such that each part is necessarily monochromatic
in every NAC-coloring and the number of parts is as small as possible.

\begin{definition}%
	\label{def:NACvalid}
	%
	An equivalence relation $\sim$ on the edge set
	of a graph $G$ is called \emph{NAC-valid}
	if for every NAC-coloring $\delta$ of $G$ it holds that
	$\forall e_1, e_2 \in E(G) :
		e_1 \sim e_2 \Rightarrow \delta (e_{1}) = \delta (e_{2})$.
	An equivalence class of a NAC-valid relation is called a \emph{monochromatic class}.
	The \emph{vertices} of a monochromatic class $M$ are the vertices of the subgraph $G[M]$.
	%
\end{definition}

Clearly, the relation inducing \trcon{} components is NAC-valid.
The following lemma describes a way how to construct a new NAC-valid relation from another one
with possibly less monochromatic classes.

\begin{lemma}%
	\label{lemma:two_edges_and_component}
	%
	Let $\sim$ be a NAC-valid relation on $G$ and
	$\sim^\prime$ be a relation on $E(G)$ given by
	$e_{1} \sim^\prime e_{2}$ if and only if
	$e_{1} \sim e_{2}$ or there exists
	a cycle $C$ in $G$ such that $e_{1}, e_{2}$
	are edges in $C$
	and all other edges are in the same monochromatic class of $\sim$.
	Then the reflexive-transitive closure of $\sim^\prime$ is NAC-valid.
\end{lemma}
\begin{proof}
	%
	The condition in \Cref{def:NACvalid} is preserved under taking reflexive-transitive closure,
	so it suffices to check it only for $\sim^\prime$: if $e_1$ and $e_2$ had different colors in some NAC-coloring,
	then $C$ would be an almost cycle since $C - \{e_1,e_2\}$ is monochromatic as the relation $\sim$ is NAC-valid.
	%
\end{proof}

Note that various situations can occur: $e_1$ or $e_2$ is in the same a monochromatic class of~$\sim$ as the rest of $C$,
or none if them is.
% , see \Cref{fig:two_edges_and_component}.
If each monochromatic class of $\sim$ induces a connected subgraph,
then the edges $e_1$ and $e_2$ can be supposed to be incident,
but this is not the case in general.
For instance one could introduce a NAC-valid relation such that both triangles in a 3-prism subgraph
are in the same monochromatic class, since they have the same color in all NAC-colorings of the 3-prism.

% \begin{figure}[h]
% 	\centering
% 	\begin{tikzpicture}[scale=2]
% 		\node[vertex] (0) at (0, 0) {};
% 		\node[vertex] (1) at (1, 0.5) {};
% 		\node[vertex] (2) at (2, 0) {};
% 		\node[vertex] (3) at (0.5, 0.866) {};
% 		\node[vertex] (4) at (1.5, 0.866) {};
% 		\draw[redge] (0)edge(1) (1)edge(2) (0)edge(3) (1)edge(3) (1)edge(4) (2)edge(4) (3)edge(4)  ;
% 		\draw[edge]  (0)edge(2)  ;
% 	\end{tikzpicture}
% 	\qquad
% 	\begin{tikzpicture}[scale=2]
% 		\node[vertex] (0) at (0, 0) {};
% 		\node[vertex] (1) at (1, 0.5) {};
% 		\node[vertex] (2) at (2, 0) {};
% 		\node[vertex] (3) at (0.5, 0.866) {};
% 		\node[vertex] (4) at (1.5, 0.866) {};
% 		\node[vertex,label={north:$v$}] (5) at (1, 0) {};
% 		\draw[redge] (0)edge(1) (1)edge(2) (0)edge(3) (1)edge(3) (1)edge(4) (2)edge(4) (3)edge(4)  ;
% 		\draw[edge]  (0)edge(5) (2)edge(5)  ;
% 	\end{tikzpicture}
% 	\caption{An edge with adjacent vertices in a monochromatic class (left) and
% 		two edges with adjacent vertices in a monochromatic class (right).}%
% 	\label{fig:two_edges_and_component}
% \end{figure}

Notice that application of \Cref{lemma:two_edges_and_component} corresponds
to merging the monochromatic classes of $e_1$ and $e_2$.
We propose to use the following NAC-valid relation:
we start with the \trcon{} components and then apply the following two steps
as long as there is some change:
\begin{enumerate}
	\item If there is an edge $uv$ such that $u$ and $v$ are vertices of
	      the same monochromatic class $M$, then merge $M$ with the monochromatic class of $uv$.
	\item If there are edges $uv$ and $vw$ such that $u$ and $w$ are vertices of
	      the same monochromatic class, then merge the monochromatic classes of $uv$ and $vw$.
\end{enumerate}
Every \trcon{} component induces a connected subgraph and also the two operations preserve
the fact that each monochromatic class induces a connected subgraph.
Hence, the resulting partition indeed forms monochromatic classes
of a NAC-valid relation by \Cref{lemma:two_edges_and_component}.
The construction can be done in polynomial time
and implemented efficiently using a Union-find data structure,
see also \Cref{alg:create_monochromatic classes} in \Cref{sec:pseudocode}.

Searching all NAC-colorings naively, we have to test $2^{m-1}$ colorings,
where $m$ is the number of monochromatic classes (we can fix the color of one class).
Hence, using a good NAC-valid relation can reduce the computation time significantly.

\subsection{Small cycles}%
\label{sec:small_cycles}

The check \IsNACColoring{} can be quite computationally expensive
considering it is called for every possible coloring.
An optimization we use is to keep a list of shorter cycles in the graph and
check if they are almost monochromatic cycles before doing the full check.

The check for a single cycle can be done in linear time
in the number of monochromatic classes using bitwise arithmetic.
One cycle $C$ can reject up to $\frac{1}{|E(C)|}$ colorings,
which can be significant for small cycles.

Given a graph $G$ and $k,\ell\in \mathbb{N}$,
we adopt the idea to monochromatic classes as follows:
for every monochromatic class $M$ in $G$,
we consider all edges $uv\in M$ such that there is a path from $u$ to $v$
in $G - M$ that uses at most $k$ monochromatic classes.
We pick up to $l$ cycles constructed from $uv$ with the corresponding
path with the least number of used monochromatic classes.
In our implementation, we take $k=4$ and $l=2$ cycles for every monochromatic class.

\subsection{Combining NAC-colorings of subgraphs}%
\label{sec:combining}

In some cases, a graph can be decomposed into subgraphs so that
the NAC-colorings of the graph can be obtained from the NAC-colorings of the subgraphs.
We introduce notation for this process of combining NAC-colorings of suitable subgraphs.

\begin{definition}
	%
	Let $G$ be a graph with subgraphs $G_1, \dots, G_k$
	s.\ t.\ $\bigcup_{i=1}^k E(G_i) =E(G)$.
	For $1\leq i \leq k$, let $\delta^{i}_{\red}$ and
	$\delta^{i}_{\blue}$ be the monochromatic colorings of $G_i$.
	The set
	\[ \Bigl\{
		\text{surjective }\delta : E(G) \to \{\blue, \red\}
		\mid\forall i : \delta |_{E(G_i)} \in
		\nac{G_i} \cup \{\delta^{i}_{\blue}, \delta^{i}_{\red}\}
		\Bigr\}
	\]
	is called the \emph{NAC-product} of $G_1, \dots, G_k$ and denoted by $\CP(G_1, \dots, G_k)$.
	%
\end{definition}

Since the restriction of a NAC-coloring to a subgraph is a NAC-coloring or monochromatic,
we have $\nac{G_{1} \cup G_{2}} \subseteq \CP(G_{1}, G_{2})$.
If $G_1, \dots, G_k$ are the connected components of $G$,
then $\nac{G} = \CP(G_1,\ldots, G_k)$.
Since every cycle is contained in a single
block\footnote{A \emph{block} is a bridge or a maximal 2-connected subgraph.}
of a graph, we have that $\nac{G} = \CP(G_1,\ldots, G_k)$
for $G_1,\ldots, G_k$ being the blocks of $G$.

In order to design a faster algorithm for searching for all NAC-colorings,
we exploit the fact that for edge-disjoint graphs $G_1$ and $G_2$,
it is straightforward to construct $\CP(G_{1}, G_{2})$ once we know $\nac{G_1}$ and $\nac{G_2}$.
Then, we can get $\nac{G_{1} \cup G_{2}}$ by applying \IsNACColoring{}
to each coloring in $\CP(G_{1}, G_{2})$, see also \Cref{alg:coloring_product} in \Cref{sec:pseudocode}.
This can significantly reduce the number of \IsNACColoring{} calls
comparing to testing all red-blue-colorings of $G_{1} \cup G_{2}$.
To apply the idea on a graph $G$, we:
\begin{enumerate}
	\item decompose $G$ into pairwise edge-disjoint subgraphs $G_1, \ldots, G_\ell$, and
	\item compute the NAC-product $\CP(G_1, \dots, G_\ell)$ using $\nac{G_1}, \ldots, \nac{G_\ell}$
	      and filter it to get $\nac{G}$.
\end{enumerate}
In the following two subsections we discuss possible heuristics
for these two phases.



\subsection{Subgraph decomposition}%
\label{sec:decomposition}

On the input, we assume the list of monochromatic classes
and an integer $k\geq 1$, and we output a list of subgraphs
such that each has at most $k$ monochromatic classes.
We describe some heuristics how to split a graph into edge disjoint subgraphs
that perform well with the idea of the previous subsection.

The strategy with the least overhead costs is to take the chunks of $k$ consecutive
monochromatic classes in the input list (heuristic \None{}).

Our goal is to maximize the number of suitable cycles in each
subgraph as cycles may form almost cycles and those help us to reduce the search
space.
We call the following heuristic \Neighbors{},
its pseudocode is in \Cref{alg:neighbors} in \Cref{sec:pseudocode}.
The algorithm aims to mimic breadth-first-search
behavior while respecting monochromatic classes.
The goal is to find monochromatic classes that are close together.
Let $P$ be the list on monochromatic classes,
for simplicity we assume that $k$ divides $|P|$.
We will gradually divide the monochromatic classes into bags,
the output are the subgraphs induced by the edges in each bag.

We start with $\frac{|P|}{k}$ empty bags
and all vertices of the graph labelled $open$.
First we add a random of the remaining monochromatic classes to a bag.
All the vertices of the monochromatic classes in the bag are denoted by $used$.
We take all $open$ vertices that are neighbors of the vertices in $used$ and assign
them a score. A vertex $best$ with the highest score is chosen.
The monochromatic classes corresponding to edges connecting the vertex $best$
with vertices in $used$ are then added to the bag while not exceeding its size.
The $used$ set is updated, and the algorithm continues until the bag is full.
If $open$ is empty, this iteration of the search also ends.
A new bag is chosen, and we repeat the process.

The first strategy \Neighbors{} takes as the score the number of
neighboring vertices of a vertex in the $used$ set.
The other strategy \NeighborsDegree{} is based
on the first and adds another rule ---
if the number of neighbors match, the vertex with lower degree is chosen.

Several other strategies have been tried, but they performed worse.


\subsection{Subgraph merging}%
\label{sec:merging}

After constructing the list of edge-disjoint subgraphs,
we compute all the NAC-colorings for each of them using
the naive algorithm with monochromatic classes
and improved check on cycles.
Then the results need to be merged to obtain the NAC-colorings of the original graph
using the NAC-product.
Since checking every coloring in the NAC-product is a costly operation,
we try to minimize the work that has to be done.
The complexity of the task grows with the size of
the merged subgraphs (as \IsNACColoring{} depends on the size of the graph checked)
and also with the number of NAC-colorings found in each subgraph.
It is also important to note that in case the merged subgraphs
have no vertices in common, we get the whole NAC-product.

We describe two strategies.
The first approach \MergeLinear{} is
to take the sequence of subgraphs and merge them one by one.
We merge the first and the second subgraph, then we merge this result with the
third one and so on. We have tried also tree-like approach, i.e., to merge consecutive pairs
and then recursively again, but the performance was worse.

Another approach called \SharedVertices{} always merges the two subgraphs that
have the most vertices in common with the goal
to create as many new cycles as possible.

