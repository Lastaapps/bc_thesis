
\chapter{Implementation}%
\label{chapter:impl}

\begin{chapterabstract}

	In~this chapter,
	we first describe the~structure of the~project,
	general description of source code files, and datasets.
	We~list significant libraries used in~the~project,
	mention relation to PyRigi and
	describe the~Python API surface of the~implementation.

\end{chapterabstract}

The~code is available as an~attachment of this thesis,
nevertheless, the~most recent version can be found on GitHub~\cite{my_code}
and soon also in~PyRigi~\cite{pyrigi}.
%
The~code is written in~Python~\cite{python}, the minimal supported version is Python 3.12.
To set up the~project, create a~virtual environment and install packages
from \texttt{requirements.txt}. On NixOS, \texttt{shell.nix} can be used.
See \texttt{README.md} for~additional instructions.
We~go through the~main folders and files of the~project,
see the project file structure in~\nameref{chapter:attachments}.


\section{Files overview}

In~this section, we~describe the~main source code files of the~project
and link them to the~corresponding terms and section
from \Cref{chapter:stable_cuts,chapter:algo}.

\subsection{Code for~benchmarking}

File \texttt{NAC\_presentation.ipynb} shows how the~benchmarks
can be run and analyzed.
Bellow imports, a~user can choose if~precomputed benchmarks should be used
(they need to be decompressed first),
or if~the~benchmarks should be run from scratch.
Note that this takes a~long time
as further discussed in~\Cref{chapter:benchmarks}.
Then follows a~section dedicated to loading graph datasets.
After that, tooling for~running benchmarks and individual cells
for~running benchmarks for~each graph class follow.
Lastly, there are cells for~the~performance analysis.
%
There are many utility functions
imported from \texttt{benchmarks/notebook\_utils.py}
that offer a wide range of functionality.
The~file \texttt{benchmarks/NAC\_benchmark.ipynb}
contains code similar to \texttt{NAC\_presentation.ipynb},
but with additional functionality like exporting figures to \LaTeX{}.

\subsection{NAC-coloring search}

The~code of the~algorithms described in~\Cref{chapter:algo}
is stored in~directory \texttt{nac}.
%
Directory \texttt{util} stores helper functions and classes
like an~implementation of \textsc{UnionFind} data structure.
%
File \texttt{check.py} implements \IsNACColoring{} check.
%
File \texttt{monochromatic\_classes.py} is used to find \trcon{} components
and monochromatic classes in~a~graph. With this, we~can compare performance
between using monochromatic classes, \trcon{} components or just edges.
%
File \texttt{cycle\_detection.py} holds algorithms for~finding cycles
used by \Cref{sec:small_cycles}
and some heuristics.
%
In~\Cref{sec:polynomial_optimizations},
we~present checks that can
sometimes find a~NAC-coloring in~polynomial time
or determine that there is none.
These checks are implemented in~\texttt{existence.py} and
used mostly from \texttt{single.py} that is the~entry-point
for~a~single NAC-coloring search.
%
General NAC-coloring searching is implemented in~\texttt{search.py}
along with parameter parsing, graph vertices normalization, and
optimizations like search for~articulation vertices.
After that, the~correct algorithm from \Naive{}, \NaiveCycles{} or \Subgraphs{}
is chosen and called.
%
These algorithms are implemented in~\texttt{algorithms.py} alongside many helper functions.
Heuristics for~\Subgraphs{} algorithm are stored in~\texttt{strategies\_split.py}
and \texttt{strategies\_merging.py}.


\subsection{Stable cuts}

The~code related to finding stable cuts in~flexible graphs
is implemented in~directory \texttt{stablecut}.
%
The~algorithm described in~\Cref{sec:stable_cut_algo}
is implemented in~\texttt{flexible\_graphs.py}.
Other helper functions related to cuts and stable sets
are implemented in~\texttt{util.py}%
\nohznamka{
	Note that some changes were done
	when the~code was merged into PyRigi~\cite{pyrigi_pr_stable_cuts}
	that were not backported to the~codebase of this thesis.
}.


\subsection{Graph datasets}

In~\texttt{graphs\_store}, we~store datasets used for~benchmarking.
Graphs are either obtained from~\cite{extremal_graphs},
generated using Nauty~\cite{nauty} with a~plugin~\cite{nauty_plugin}
or generated using NetworkX~\cite{networkx} and checks from PyRigi~\cite{pyrigi}.
Graphs are stored in~Graph6 or Sparse6 formats~\cite{graph6}.
%
Files and directories store these graph classes:
minimally rigid graphs, graphs with no three nor four cycles, globally rigid graphs,
and graphs with no NAC-coloring, see~\nameref{chapter:attachments}.
%
The~code for~working with datasets is stored in~directory \texttt{benchmarks}.
The~code for~reading dataset files from the~store can be found in~\texttt{datasets.py},
the~code for~generating random graphs of some graph classes
can be found in~\texttt{generators.py}.
%
In~the~\texttt{precomputed} subdirectory, there are all results of the~benchmarks that
we~use for~evaluation of algorithms.
Individual runs are stored in~a~compressed CSV file.
The~CSV file structure is described in~detail in~\texttt{NAC\_presentation.ipynb}.
Notable columns are:
encoded graph,
the~number of \trcon{} components and monochromatic classes of the~graph,
the~number of NAC-colorings found and mean running time,
both for~search for~a~single NAC-coloring or for~all NAC-colorings,
and the~strategies used and subgraph sizes.


\subsection{Other files}

File \texttt{NAC\_playground.ipynb} presents a~simple case
to show how the~algorithm's API can be used.
This API is further described in~\Cref{sec:impl_python_api}.
%
Tests of both stable cuts and NAC-coloring parts are stored in~directory \texttt{test}.
%
File \texttt{requirements.txt} lists versions of Python libraries used for~development.
\texttt{README.md} comes from our original paper~\cite{my_paper}
and contains additional instructions for~setting up the~project.
File \texttt{shell.nix} can be used for~project setup on NixOS\@.


\section{Python API surface}%
\label{sec:impl_python_api}

In~this section,
we~describe the~main entry points to our library.

% To call algorithm~\Cref{alg:stable_cut_flexible},
% first import function \texttt{stable\_cut\_in\_flexible\_graph}
\Cref{alg:stable_cut_flexible}
is implemented in~\texttt{stable\_cut\_in\_flexible\_graph}
from module \texttt{stablecut.flexible\_graphs}.
The~function accepts a~graph and
optionally vertices that should be avoided and separated by a~cut.

All the~following functions are imported from module \texttt{nac}.
To find monochromatic classes on a~graph,
call function \texttt{find\_monochromatic\_classes} and provide a~graph.
Optionally, \trcon{} components can be found by changing the~second parameter.
%
\IsNACColoring{} check can be run by calling \texttt{is\_nac\_coloring} function.
The~function accepts a~graph and a~NAC-coloring
as a~tuple of containers with edges.
%
To find a~single NAC-coloring, call \texttt{single\_NAC\_coloring} and provide
a~graph. If~there exists any NAC-coloring, it~will be returned.
In~case a~certificate is not needed,
\texttt{has\_NAC\_coloring} should be called instead.
%
To list all NAC-colorings of a~graph, call function \texttt{NAC\_colorings}.
Notable parameters are:
\begin{itemize}
	\item \texttt{graph} is a~graph to search on.
	\item \texttt{algorithm} is a~string that can be either
	      \texttt{naive} for~\Naive{},
	      \texttt{cycles} for~\NaiveCycles{},
	      or \texttt{subgraphs-\{s\}-\{m\}-\{k\}} for~\Subgraphs{},
	      where \texttt{s} stands for~a~splitting strategy,
	      \texttt{m} for~a~merging strategy,
	      and \texttt{k} for~the~size of subgraphs.
	      %
	      Notable split strategies are
	      \texttt{none} for~\None{},
	      \texttt{neighbors} for~\Neighbors{}, and
	      \texttt{neighbors\_degree} for~\NeighborsDegree{},
	      %
	      notable merging strategies are
	      \texttt{linear} for~\MergeLinear{} and
	      \texttt{shared\_vertices} for~\SharedVertices{}.
	      %
	      Names of other strategies can be found in~\texttt{algorithms.py}.
	\item \texttt{monochromatic\_class\_type} chooses
	      between monochromatic classes and \trcon{} components.
	\item \texttt{use\_decompositions} sets
	      whether a~check for~articulation vertices is done.
	\item \texttt{use\_has\_coloring\_check} sets
	      whether \texttt{has\_NAC\_coloring} should be used
	      as if~we can decide the~graph has no NAC-coloring,
	      the~whole check can be skipped.
\end{itemize}
%
An~iterator is returned, so the~colorings can be listed lazily.


\section{Implementation details}

In~this section, we~describe some code specifics
in~case a~reader of this thesis wants to examine the~code.
%
To improve reader's understanding of the~code,
we~first describe common function parameters:
\texttt{graph} represents a (sub)graph where NAC-colorings should be found.
%
\texttt{comp\_graph} is a~graph where vertices are some integer IDs of monochromatic classes
and edges exist if~the~classes are neighboring,
see \Cref{observ:monochromatic_classes_graph}.
%
Monochromatic class IDs also serve as indices into \texttt{component\_to\_edges}
that map an~ID of a~monochromatic class to its edges.
%
NAC-colorings are represented as bit-masks where bit's offset correspond to a~class ID\@.

As \IsNACColoring{} is a~core component of all our algorithms,
we~tried to optimize it~well.
%
In~the~base implementation of \IsNACColoring{},
subgraphs from \( \red \) and \( \blue \) edges are created.
To create such subgraphs in~code, edges can be added to an~empty graph
using NetworkX's function \texttt{add\_edges\_from}.
%
This is rather slow as creating new vertices in~the~empty graph causes noticeable overhead.
Therefore, we~create a~graph with no edges and the~same vertices as the~original graph,
cache it~and reuse it~for the~checks.
Every time only edges are added, the~check is run, and the~edges are removed.
By doing this, the~performance of \IsNACColoring{} is increased by roughly 40\%.
%
Another way how the~performance could be increased is by reserving space in~lists
when the~final size is known.
To our knowledge, this is impossible in~Python.

The~code uses \texttt{Graph} class and related algorithms from NetworkX~\cite{networkx}
as the~base of many operations. We~use some utility functions from PyRigi~\cite{pyrigi}
related to (global) rigidity tests and rigidity components search.
Other than that, the~code is not dependent on PyRigi.
%
Pytest~\cite{pytest} is used for~testing and
Matplotlib~\cite{matplotlib} for~visualizations.

