
\chapter*{Conclusion}\addcontentsline{toc}{chapter}{Conclusion}\markboth{Conclusion}{Conclusion}

%%%%%%%%%%%%%%%%%%%%%%%%%%%%%%%%%%%%%%%%%%%%%%%%%%%%%%%%%%%%%%%%%%%%%%%%%%%%%%%%
% Evaluation
%%%%%%%%%%%%%%%%%%%%%%%%%%%%%%%%%%%%%%%%%%%%%%%%%%%%%%%%%%%%%%%%%%%%%%%%%%%%%%%%

% To remind us of the goals of the thesis,
% we wanted to show that the problem of deciding if a graph has a NAC-coloring
% is NP-complete also on graphs with maximum degree five.
% We wanted to present and evaluate an algorithm for NAC-coloring search
% that can find all the NAC-colorings of a graph.
% Multiple heuristics should have been shown and evaluated.
% We wanted to discuss relation to stable cuts and implement related algorithm
% for stable cut search in a flexible graph.
% Lastly, we wanted to propose an FPT algorithm for NAC-coloring counting
% parametrized by treewidth.

In this thesis,
we studied and summarized basics of Rigidity Theory.
We focused on the question of flexible realizations existence
and its combinatorial characterization using NAC-colorings.
We also studied some cases where NAC-coloring can be found in polynomial time
for some graphs.

We proved that it is NP-complete
to decide if a graph has a NAC-coloring even on graphs with maximum degree five
by a reduction from 3-SAT\@.
This improves the previous known fact that the problem is NP-complete~\cite{np_complete}.
% We designed gadgets from which a graph is constructed for a 3-SAT formula.
% We show equivalence between the NAC-coloring existence and 3-SAT satisfiability.
% The maximum degree of the constructed graph is bounded by five.

We also presented significantly faster algorithm
for NAC-coloring search.
%
First we proposed optimizations to the algorithm
that checks if a coloring is a NAC-coloring on a graph
by finding and checking perspective cycles in the graph
that can be almost cycles.
%
The concept of monochromatic classes reducing the search space
and speeding up the computation significantly for many graph classes
even for the naive approach was also introduced.
%
We proposed and implemented an algorithm using graph decomposition and gradual coloring buildup.
We also implemented multiple heuristics and optimizations of this algorithm.
%
We also implemented multiple checks that can find a NAC-coloring in polynomial time
if only a single NAC-coloring is requested.

The performance of our algorithms was
first compared with the previous naive algorithm implemented in \flexrilog{},
and then the heuristics were compared on multiple significant graph classes
with each other.
%
These results were evaluated and well performing heuristics were chosen.
Our algorithm outperforms the previous implementation by a large margin.
%
With that, all the goals of the thesis are fulfilled.

We extended the scope of the thesis beyond the assignment by
describing an FPT algorithm for NAC-coloring counting
parametrized by treewidth.
It can possibly outperform our proposed algorithm if implemented
in cases where the number of NAC-colorings is requested.

Further, we discussed the relation of our NP-completeness complexity result
with stable cuts.
We also studied and implemented an algorithm that can find
a stable cut for any flexible graph in polynomial time.

We extended our work done in~\cite{my_paper}.
Code implemented in this thesis~\cite{my_code} is already partially
refactored and merged into PyRigi, a library for Rigidity Theory.

%%%%%%%%%%%%%%%%%%%%%%%%%%%%%%%%%%%%%%%%%%%%%%%%%%%%%%%%%%%%%%%%%%%%%%%%%%%%%%%%
% Future work
%%%%%%%%%%%%%%%%%%%%%%%%%%%%%%%%%%%%%%%%%%%%%%%%%%%%%%%%%%%%%%%%%%%%%%%%%%%%%%%%

To further improve the performance of the current solution, the codebase
can be rewritten from Python into some low-level language like C, C++, Rust or Zig.
Also, further heuristics can be tried
and tested on additional graph classes
to reach additional performance gains.
It is expected that future heuristic has to work significantly differently
compared to our solution as we reach some sort of limit
that we did not manage to overcome.

Major gains for some graph classes
can be expected if the FPT algorithm is also implemented.
Expectations are that the FPT algorithm will outperform the current algorithm
for graphs with large number of NAC-colorings if only the number of colorings
is required (without all the colorings itself being materialized),
and for graph classes with small treewidth.
This algorithm could also be parallelized up to some extent,
but this would make sense for large graphs only.
%
Also, it is possible that there is a Monte Carlo algorithm with significantly
lower time complexity compared to the algorithm proposed by us.

For flexible graphs including graphs where \( |E(G)| \le 2|V(G)| - 4 \),
we can find a stable cut in polynomial time using \Cref{alg:stable_cut_flexible},
that is implemented in this thesis.
There is no known implementation of an algorithm
for graphs where \( |E(G)| = 2|V(G)| - 3 \)
even though it can be described and implemented
based on~\cite{stable_cuts_2v_3,stable_cuts_2v_3_revisit}.

