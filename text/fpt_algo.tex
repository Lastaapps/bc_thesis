\chapter{FPT Algorithm for NAC-coloring counting}

\begin{chapterabstract}

	It can be easily shown that the NAC-coloring counting as a NP-complete
	problem can be parametrized by tree width by using \MSO{} logic.
	In this chapter, we present an algorithm that can solve the given problem
	in \({k}^{O(k)} n^{O(1)}\) time, where \(k\) stands for the tree width of the graph.

\end{chapterabstract}

\section{Tree width}

In this section I'll describe following:
\begin{description}
	\item[FPT algorithms]
	\item[Tree width]
	\item[Tree decomposition]
	\item[Other common parameters]
\end{description}

\section{Monadic second-order logic}

In this section I'll describe following:

\begin{description}
	\item[Introduce the reader swiftly into~\MSO]
	\item[Tree width]
	\item[Tree decomposition]
	\item[Other common parameters]
\end{description}

\section{FPT algorithm}

In this section we first introduce the main idea of the algorithm,
go through each type of node in a tree decomposition and
set final complexity of the algorithm.
Lastly we optimize the algorithm by using monochromatic components.

Our algorithm will be somewhat similar to Steiner tree construction algorithm
as described in~\cite{book_parametrized_algorithms} as both the problems require connectivity
among vertex partitions.

First we want to build the intuition for the following algorithm.
Let us have a graph \( G \) and all the NAC-colorings coloring of the graph
\( NACG \). Let us add an edge \( e = \{u, v\}, e \not\in E(G) \) to form graph~\Gprime.
We want to obtain \( \text{NAC} (G^\prime) \) from \( NAC(G) \).
Let us have \( \delta \in \NACG \).
We want to extend it to \( delta^\prime \), suppose \( \delta^\prime(e) = \blue \).
\( \delta^\prime \) is a NAC-coloring unless we formed an almost cycle.
In that case one of these must be true:
\begin{itemize}
	\item Both vertices \( u, v \) share the same connected component in \( G[\Ered] \).
	      An almost cycle is formed
	      from \( u \)-\( v \)-path in the connected component with \( \red \) color
	      and edge \( e \) with \( \blue \) color.
	\item Vertices \( u, v \) lay in different connected components in \( G[\Eblue] \)
	      and there is a \( \red \) colored edge connecting some vertices from the components.
	      An almost cycle is formed from paths in each component, the bridge and edge \( e \).
\end{itemize}
In all other cases an almost cycle cannot be formed by adding an edge.
Note that if a vertex cannot be in more connected component for the given color
as this fact would mean that both the components are not maximal.

\subsection{Dynamic cache}

First we define a state space \( S \) of \emph{configurations}
that will be used by our cache in our dynamic programming
algorithm on a tree decomposition \( \mathcal{T} \) of a graph \( G \) with width \( k \).
For a bag \Xt{} for \( t \in \mathcal{T} \)
with \( l \le k+1 \) vertices we take all the factorizations of \Xt{} denoted by \( \mathcal{F}(X_t) \)
and for each partitioning we take all the possible symmetric irreflexive relations between
partitions. Now we do it again for the other color.
Our configuration is the formed by a four tuple \( s = (P_\red, R_\red, P_\blue, R_\blue), s \in S \)
where \( P_\red\), resp. \( P_\blue \), is a partitioning of \Xt{}
and \( R_\red\), resp. \(R_\blue \), is a relation
on partitions in \( P_\red\), resp. \(P_\blue \).
Partitions \( P_\red, P_\blue \) represent vertices
that share the same connected component in the subgraph
corresponding to the specified color and
relations \( R_\red, R_\blue \) represent
whenever the connected components are connected by an edge of the opposite color.
Note that the state space is somewhat large, but it is expected as we are solving
and NP-complete problem.
We denote cache as function \( c: S \to \N \).


\subsection{Introduce node}

Suppose that \( t \in \mathcal{T} \) is
the only parent of \( t^\prime \in \mathcal{T} \) in \( \mathcal {T} \).
Let \( v \) be the only vertex in \( X_t \setminus X_{t^\prime} \).
Then we define cost function for \( s=(P_\red, R_\red, P_\blue, R_\blue) \)
for \( a \in \{\red, \blue\} \) as:
%
\begin{align*}
	P_a^\prime & \coloneqq P_a \setminus \{\{v\}\}                                                                    \notag \\
	R_a^\prime & \coloneqq R_a \setminus \{ (p_1, p_2) \mid (p_1, p_2) \in R_a, (\{v\} \in p_1 \lor \{v\} \in p_2) \} \notag \\
	c[t, s]    & \coloneqq
	\begin{cases}
		0,                                                                           & \text{if } \exists a : \{v\} \not\in P_a                    \\
		0,                                                                           & \text{if } \exists a \exists p \in P_a : (\{v\}, p) \in R_a \\
		c[t^\prime, (P_\red^\prime, R_\red^\prime, P_\blue^\prime, R_\blue^\prime)], & \text{otherwise}
	\end{cases}
\end{align*}
%
What this means is that we just propagate our already computed values to the next level.
We did not find any new NAC-colorings as we did not add any edges.
Zero is set for components that state to be connected,
but they cannot be as they contain \( v \)
which has no edge connecting it to the other vertices of \( X_{t^\prime} \).
The same idea holds for bridges as no bridge is created.

\subsection{Forget node}

Suppose that \( t \in \mathcal{T} \) is
the only parent of \( t^\prime \in \mathcal{T} \) in \( \mathcal {T} \).
Let \( v \) be the only vertex in \( X_{t^\prime} \setminus X_t \).
Then we define cost function for \( s=(P_\red, R_\red, P_\blue, R_\blue) \)
for \( a \in \{\red, \blue\} \) as:
%
\begin{align}
	m(p)          & \coloneqq p \setminus \{v\}   \notag                                                                                                                    \\
	M(P)          & \coloneqq \{m(p) \mid p \in P \} \setminus \{\emptyset\}   \notag                                                                                       \\
	\mathcal{P}_a & \coloneqq \{ P_a^\prime \mid P_a^\prime \in \mathcal{F}(X_{t^\prime}), M(P_a^\prime) = P_a \}   \notag                                                  \\
	\mathcal{R}_a & \coloneqq \{ R_a^\prime \mid P_a^\prime \in \mathcal{P}_a, R_a^\prime \subseteq P_a^\prime \times P_a^\prime,   \notag                                  \\
	              & \qquad \forall (p_1, p_2) \in R_a^\prime : (m(p_1), m(p_2)) \in R_a \lor m(p_1) \cup m(p_2) = \emptyset \}   \notag                                     \\
	S^\prime      & \coloneqq \{(P_\red^\prime, R_\red^\prime, P_\blue^\prime, R_\blue^\prime) \mid P_a^\prime \in \mathcal{P}_a, R_a^\prime \in \mathcal{R}_a  \}   \notag \\
	c[t, s]       & \coloneqq \sum_{s^\prime \in S^\prime} c[t^\prime, s^\prime]
\end{align}
%
We take all the previous states that collapse into our requested state \( s \)
after removing \( v \).
\todo[inline]{Prove that no coloring is counted twice by checking that each previous state is mapped to exactly one new state}

\subsection{Introduce edge}

Suppose that \( t \in \mathcal{T} \) is
the only parent of \( t^\prime \in \mathcal{T} \) in \( \mathcal {T} \).
It holds that \( X_t = X_{t^\prime} \),
let \( e = \{u, v\} \) be the only edge added in the step, \( e \in E_t \setminus E_{t^\prime} \).
We will discuss all the possible cases that must be fulfilled for both colors,
otherwise the resulting value for the state given is zero.
Interactions between components of different colors are of no interest for us.
We state conditions for a single color, \WLOG{} let the edge \( e \) be \( \blue \):
%
\begin{description}
	\item[Edge lies in a \( \blue \) connected component]
	      In this case there is no risk of an almost cycle being created.
	\item[Edge lies in a \( \red \) connected component]
	      In this case an almost cycle is trivially created,
	      therefore this is not allowed.
	\item[Edge connects two neighboring \( \blue \) connected components]
	      This again causes an almost cycle to be created.
	\item[Edge connects two neighboring \( \red \) connected components]
	      This causes no harm as cycle with two \( \blue \) edges is created.
	\item[Edge connects two non-neighboring \( \blue \) connected components]
	      This operation is not allowed as both the components
	      are now connected using this edge.
	\item[Edge connects two non-neighboring \( \red \) connected components]
	      This is not allowed as bot the components
	      are neighbors with \( e \) being the new bridge.
\end{description}
%
Note that either both conditions need to pass.

Now we state how the cached values is computed for cases when
the previous conditions pass for both the colors.
Resulting cache entry is sum of multiple previous states that
collapse into the new valid state as some components are joined,
and some neighboring relations are created:
%
\begin{description}
	\item[Edge lies in a \( \blue \) connected component]
	      First we query the previous cache state with
	      the same partition and neighbors configuration.
	      Then we also need to add configurations where
	      both the components were not neighbors as no almost cycle was created
	      and both the separate connected components
	      are now joined into a single component.
	      We denote \( \blue \) components of such configuration
	      as \( P_\blue^\prime \) and \( R_\blue^\prime \).
	      Note, that other relations between partitions need to match.
	\item[Edge connects two neighboring \( \red \) connected components]
	      Result is a sum of previous states with the same partitioning as \( P_\red \),
	      for first query we use \( R_\red \),
	      and then we add query of \( R^\prime_\red = R_\red \setminus \{(c(u), c(v)), (c(v), c(u))\} \)
	      (with the neighbor constraint removed).
\end{description}
%

The resulting cache for \( \blue \) color is then computed as follows:
%
\begin{align}
	c_\blue[t, s] & = c[t^\prime, (P_\red, R_\red, P_\blue, R_\blue)]   \notag                      \\
	              & + c[t^\prime, (P_\red, R_\red^\prime, P_\blue, R_\blue)]   \notag               \\
	              & + c[t^\prime, (P_\red, R_\red, P_\blue^\prime, R_\blue^\prime)]   \notag        \\
	              & + c[t^\prime, (P_\red, R_\red^\prime, P_\blue^\prime, R_\blue^\prime)]   \notag
\end{align}
%
If some rule was not matched for \( s \) and \( e \) colored \( \blue \)
the cache entry results into \( c_\blue[t, s] = 0 \).

The final entry is \( c[t, s] = c_\red[t, s] + c_\blue[t, s] \).

\subsection{Join node}

Suppose that \( t \in \mathcal{T} \) is
the only parent of \( t_1, t_2 \in \mathcal{T} \) in \( \mathcal {T} \).
It holds that \( X_t = X_{t_1} = X_{t_2} \)
and \( S = S_1 = S_2 \) where \( S_\cdot \) are corresponding cache state spaces.

We create mappings \( M: S_1 \times S_2 \to S \) and \( N: S_1 \times S_2 \to \{0, 1\} \).
Mapping \( M \) maps to the state when connected components of both the states
are joined together and \( N \) states if the given join is valid.

Let \( {(.)}^* \) denote the transitive closure of a set.
For states let us have
% \( s = (P_\red, R_\red, P_\blue, R_\blue), s \in S \),
\( s_1 = (P_{1,\red}, R_{1,\red}, P_{1,\blue}, R_{1,\blue}), s_1 \in S_1 \) and
\( s_2 = (P_{2,\red}, R_{2,\red}, P_{2,\blue}, R_{2,\blue}), s_2 \in S_2 \).
Let \( a \in \{\red, \blue\} \),
We take original relations \( O_{1, a}, O_{2, a} \)
from which \( P_{1, a}, P_{2, a} \) were factorizes.
We set
%
\begin{align}
	O_a & \coloneqq {(O_{1, a} \cup O_{2, a})}^*   \notag \\
	P_a & \coloneqq \mathcal{F}_{O_a}(X_t)
\end{align}
%
where \( \mathcal{F}_O(X) \) stands for factorization of \( X \)
according to relation \( O \). This operation represents merging connected
components from both the states. No NAC-coloring related checking is performed yet.

We continue with neighbors relation. We drop connections that related components
that were merged (we will get back to them soon).
Let for \( i \in {1, 2} \)
%
\begin{align}
	R_{i,a}^\prime & \coloneqq \{(p_1, p_2) \mid p_1, p_2 \in P_a,   \notag                                                                                                            \\
	               & \qquad \exists p_1^\prime, p_2^\prime \in P_{i,a} : (p_1^\prime, p_2^\prime) \in R_{i,a} \land p_1^\prime \subseteq p_1 \land p_2^\prime \subseteq p_2\}   \notag
\end{align}
%
and \WLOG{} let \( R_a \coloneqq R_{1,a} \).
Then we define mapping \( M \) as \[ M(s_1, s_2) = (P_\red, R_\red, P_\blue, R_\blue). \]

We follow with mapping \( N \) that states if the merge done by \( M \) is valid.
First, sates that provide different guaranties
(\( R_{1,a}^\prime \not= R_{2,a}^\prime \)) about neighboring components
among the components in the final state are unsuitable.
Now we need to validate that we did not create an almost cycle by checking
neighboring relations in \( R_{1,a} \) among components from \( P_{1,a} \)
that will be merged because of \( P_{2,\red}, P_{2,\blue} \).
The same follows the other way.

The rest of the statement follows similar in introduce edge node proof.
The check is done once again for both the colors.
Let us go through cases how can an almost cycle be created.
Suppose the only edge with different color is colored \( \red \).
Then there must exist a \( \blue \) path. If the path spans only using vertices
in \( V_{t_1} \), the cache entry must hold zero as no NAC-coloring can come
from \( s_1 \). This case is of not interest to us.
Therefore, the \( \red \) edge must be
a bridge connecting two \( \blue \) components in \( P_{1,\blue} \) and
the path has to also span trough vertices in \( V_{t_2} \).
Such vertices have to share the same connected \( \blue \) component in \( P_{2,\blue} \).
This happens if and only if there is a neighboring pair of components in \( P_{1,\blue} \)
that can be connected by a \( \blue \) path in \( P_{2,\blue} \).

To formalize let \( a \in \{\red, \blue \}  \) and \( b \in \{\red, \blue \} \setminus \{a\} \).
Let \( \mathcal{P} \subseteq P_{1, a} \)
and \( p \in P_{a} \) such that \( \bigcup_{p^\prime \in \mathcal{P}} p^\prime = p \).
If there exists \( p_1, p_2 \in \mathcal{P} : (p_1, p_2) \in R_{1, a} \),
this configuration is not acceptable as an almost cycle is created
using a \( \red \) bridge connecting \( p_1 \) and \( p_2 \).

This is the only way an almost cycle can be created as no edges are added.
Note that each vertex in \( X_t \) is part of both \( \red \) and \( \blue \)
components even if the components have just a single vertex.
All the checks for both sides (\( 1, 2 \)) and colors are run and
if a single one of them fails, the states pair is considered bad.
We denote set of such bad configurations as \( \mathcal{B}, (s_1, s_2) \in \mathcal{B} \).

%
\begin{align}
	N(s_1, s_2) =
	\begin{cases}
		0, & \text{if } (s_1, s_2) \in \mathcal{B}                  \\
		0, & \text{if } \exists a : R_{1,a}^\prime = R_{2,a}^\prime \\
		1, & \text{otherwise}
	\end{cases}\notag
\end{align}
%

Lastly we show formula to compute cache entries based on previously stated mappings.
It comes from the same idea as coloring product where all the resulting colorings
are NAC-colorings. The size of such product is multiple of sizes of the original subgraphs.
%
\begin{align}
	\mathcal{S}(s) & \coloneqq \{(s_1, s_2) \mid M(s_1, s_2) = s \}   \notag                                                     \\
	c[t, s]        & \coloneqq \sum_{(s_1, s_2) \in \mathcal{S}(s)} N(s_1, s_2) \cdot c[t_1, s_1] \cdot c[t_{2}, s_{2}]   \notag
\end{align}
%

This finishes our basic dynamic programming algorithm
for counting NAC-colorings of the graph.
The number of NAC-colorings is the only value stored in cache
after the last vertex is forgotten.

\todo[inline]{Don't forget to define \( V_t \)}


\subsection{Time complexity}

Bell numbers are defined as the number of factorizations of the set.
They can be upper bounded by \( n^n \), which is what we will use from now on.
\todo[inline]{Reference?}
Recall that \( \forall t \in \mathcal{T} : |X_t| \le k+1 \).

First, size of the state space used is upper bounded.
For each color there is up to \( {(k+1)}^{(k+1)} \) factorizations of a bag and
up to \( 2^{\binom(k, 2)} \) neighbors relations.
Total state space size for each bag is therefore bounded by
\( {\big({(k+1)}^{(k+1)} \cdot 2^{\binom{k}{2}} \big)}^2 = {(k+1)}^{2(k+1)} \cdot 2^{2 \binom{k}{2}} \).

We go through all the nodes and state complexity of the operations performed in the node:
\begin{description}
	\item[Introduce vertex node]
	      We need to fill a state space and copy previous values into the cache
	      which can be done in constant time per operation,
	      so the time complexity corresponds to the state space size:
	      \( {k}^{O(k)} \cdot 2^{O(k)} \cdot O(1) \).
	\item[Forget vertex node]
	      Even if the operation seams complex, we just need to traverse the state space
	      and create target state after \( v \) is removed. This can be done in \( O(k) \) time
	      per state, total complexity therefore is
	      \( {k}^{O(k)} \cdot 2^{O(k)} \cdot O(k) \).
	\item[Introduce edge node]
	      The state space is iterated again and we run few simple checks and cache key constructions
	      that can be run in \( O(k) \) time, therefore the final complexity stays the same
	      \( {k}^{O(k)} \cdot 2^{O(k)} \cdot O(k) \).
	\item[Join node]
	      We iterate through product of state spaces,
	      so the size of the new spate space is power of two
	      of the otherwise used state space size.
	      This hides into the big \( O \) notation.
	      Checks run checking for almost cycles can be run in time \( O(k^2) \).
	      Thus, the final		complexity is
	      \( {k}^{O(k)} \cdot 2^{O(k)} \cdot O(k^2) \).
\end{description}

There are \( O(nk) \) nodes in \( \mathcal{T} \).
The final complexity of the algorithm is therefore
\( {k}^{O(k)} \cdot O(n) \).

\subsection{Introduce node with edges}
\subsection{Monochromatic components}

