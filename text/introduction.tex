
\chapter{Introduction}
% uncomment the following line to create an unnumbered chapter
%\chapter*{Introduction}\addcontentsline{toc}{chapter}{Introduction}\markboth{Introduction}{Introduction}
\setcounter{page}{1}

% Pohádka
% Nepřehánět to s definicemi a citacemi -> spíše teoretická část
% Proč to dělám, význam, návaznost (PyRigi)
% Jak jsem si to vybral
% prodat drony
% Cíle
% - část úvodu, nebo vlastní kapitola
% Obsah jen stručně

The core concept of Rigidity Theory is a \emph{framework} of a simple graph \(G\).
It is a realization of \(G\) into a \(d\)-dimensional plane \(p: V(G) \to \R^d\).
A framework is \emph{\( d \)-flexible} if there exists a transformation
that continuously translates some subset of vertices in such a way that
distances between neighboring vertices are preserved during the transformation.
Otherwise, the framework is called \emph{\( d \)-rigid}.

For \( d = 1 \) all the vertices are on a line
and any vertex deviation necessarily changes distances (if the graph is connected).
%
For \( d \ge 3 \) we can map all the vertices except one onto a line and
rotate the vertex around the line.
%
The interesting case is when \( d = 2 \) --- a mapping into a plane.
It is known that a graph has most of the realizations
either \( 2 \)-flexible, or \( 2 \)-rigid.,
still there may be some realizations of the other type~\cite{generically_rigid_graphs}
--- we talk about paradoxical flexibility.
A graph is \emph{(generically) rigid} if most of the realizations are \( 2 \)-rigid
and \emph{(generically) flexible} if most of the realizations are \( 2 \)-flexible.

An example application of the Rigidity Theory is a squadron of drones
where the drones can measure distance to their close neighbors.
Can we determine the locations of all the drones
down to translation and rotation of the whole squadron
just from such information?

For similar problems in the plane,
one would think that if the graph formed is rigid, the answer is yes, and
for flexible graphs the answer is no.
But as stated above, paradoxically even a rigid graph can have a flexible realization,
and it may just happen that the drones form such a \( 2 \)-flexible framework.

When we focus on the two-dimensional case,
it is already known for some classes of graphs that they are flexible.
Namely, graphs where \( |E(G)| \le 2|V(G)| - 4 \)
are flexible~\cite{stable_cuts_2v_4}.
Another important class of graphs are \emph{minimally rigid} graphs~\cite{laman_1970}.
Those are rigid graph where removal of any edge results into a flexible graph.
This class has been studied a lot as it produces relatively large number of flexible
realizations.

In efforts to find such paradoxical realizations,
a new edge coloring was defined~\cite{legersky_original}.
A \emph{NAC-coloring} is an edge coloring of a graph by \( \red \) and \( \blue \)
such that both the colors are used and there is no almost cycle formed.
An \emph{almost cycle} is a cycle in the colored graph such that exactly one
edge of the cycle is colored \( \red \) or exactly one edge is colored \( \blue \).
Such coloring exists if and only if the graph has a flexible realization.
This provides a simple but strong tool to decide whenever a graph has
a flexible realization even if it is rigid.
As shown later in the thesis, one can check in polynomial time if a coloring
given is a NAC-coloring.
Unfortunately, it is NP-complete to decide if a graph has any NAC-coloring.

\begin{figure}[ht]
	\centering
	\begin{tikzpicture}[rotate=90,scale=1.5]
		\node[vertex] (a) at (0,0) {};
		\node[vertex] (b) at (1,0) {};
		\node[vertex] (c) at (0.5,0.5) {};
		\node[vertex] (d) at (0,1.5) {};
		\node[vertex] (e) at (1,1.5) {};
		\node[vertex] (f) at (0.5,1) {};
		\draw[bedge] (a)edge(b) (b)edge(c) (c)edge(a) (d)edge(e) (e)edge(f) (f)edge(d) ;
		\draw[redge] (a)edge(d) (b)edge(e) (c)edge(f);
	\end{tikzpicture}
	\qquad
	\qquad
	\begin{tikzpicture}[rotate=90,scale=1.5]
		\node[vertex] (a) at (0.00,0) {};
		\node[vertex] (b) at (1.00,0) {};
		\node[vertex] (c) at (0.50,0.5) {};
		\node[vertex] (d) at (0.25,1) {};
		\node[vertex] (e) at (1.25,1) {};
		\node[vertex] (f) at (0.75,1.5) {};
		\draw[edge] (a)edge(b) (b)edge(c) (c)edge(a) (d)edge(e) (e)edge(f) (f)edge(d) ;
		\draw[edge] (a)edge(d) (b)edge(e) (c)edge(f);
	\end{tikzpicture}
	\caption{The $3$-prism is generically $2$-rigid, but has a flexible realization (right).
		It has a unique NAC-coloring modulo swapping colors (left).}%
	\label{fig:3prism}
\end{figure}

A nice example for NAC-coloring showcase is the prism graph~\Cref{fig:3prism},
a graph formed from two triangles with three interconnecting edges.
Such graph is rigid, still it has a NAC-coloring and flexible realizations.
The flexible realizations are all the realization where
the triangles are identical except for translation.
You can check yourself that there are
no other NAC-colorings except the symmetric one.

The goals of this thesis are to provide an algorithm
that can find all the NAC-colorings of a graph.
Multiple heuristics will be used to significantly
reduce the runtime of the algorithm.
We also provide parametrized approach where complexity can be reduced for graphs
with low internal structural complexity.
We also show that the problem of deciding if a graph has a NAC-coloring
is NP-complete even on graphs with maximal degree five.
Lastly we show relation to stable cuts.
Code implemented in this thesis will enrich the existing PyRigi~\cite{pyrigi}
library for Rigidity Theory.
Currently, there are no such algorithms implemented in PyRigi itself,
the only known implementation is rather naive in FlexRiLoG~\cite{flexrilog}.

