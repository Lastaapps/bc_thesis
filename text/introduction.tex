
\chapter*{Introduction}\addcontentsline{toc}{chapter}{Introduction}\markboth{Introduction}{Introduction}
\setcounter{page}{1}

% Pohádka
% Nepřehánět to s definicemi a~citacemi -> spíše teoretická část
% Proč to dělám, význam, návaznost (PyRigi)
% Jak jsem si to vybral
% prodat drony
% Cíle
% - část úvodu, nebo vlastní kapitola
% Obsah jen stručně

A~core concept of Rigidity Theory is a~\emph{framework} which is
a~simple graph~\(G\) with its realization
into a~\(d\)-dimensional space \(p: V(G) \to \R^d\).
A~framework is \emph{\( d \)-flexible} if~there exists
a~continuous deformation of \( p \) such that
the~distances between neighboring vertices are preserved during the~transformation.
Otherwise, the~framework is called \emph{\( d \)-rigid}.

It~is known that for~a~graph either almost all realizations are \( d \)-flexible,
or almost all realizations are \( d \)-rigid~\cite{generically_rigid_graphs}.
A~graph is \emph{(generically) \( d \)-rigid} if~most of the~realizations are \( d \)-rigid
and \emph{(generically) \( d \)-flexible} if~most of the~realizations are \( d \)-flexible.
%
As for~a~rigid graph,
there may be some flexible realizations,
and therefore we~talk about paradoxical flexibility.

An~illustrative application of the~Rigidity Theory is a~squadron of drones
where the~drones can measure distance to their close neighbors.
Can we~determine the~locations of all the~drones
up to translation and rotation of the~whole squadron
just from such information?

For~similar problems in~the~plane,
one would think that if~the~graph formed is rigid, the~answer is yes, and
for~flexible graphs the~answer is no.
But as stated above, paradoxically, even a~rigid graph can have a~flexible realization,
and it~may just happen that the~drones form such a~\( 2 \)-flexible framework.

In~efforts to find such paradoxical realizations,
a~new edge coloring was defined~\cite{legersky_original}.
A~\emph{NAC-coloring} (where NAC stands for~``no almost cycle'')
is an~edge coloring of a~graph by \( \red \) and \( \blue \)
such that both the~colors are used and there is no almost cycle formed.
An~\emph{almost cycle} is a~cycle in~the~colored graph such that exactly one
edge of the~cycle is colored \( \red \) or exactly one edge is colored \( \blue \).
Such coloring exists if~and only if~the~graph has a~flexible realization~\cite{legersky_original}.
This provides a~simple but strong tool to decide whenever a~graph has
a~flexible realization even if~it is (generically) rigid.
As shown later in~the~thesis, one can check in~polynomial time
if~a~given coloring is a~NAC-coloring.
Unfortunately, it~is NP-complete to decide if~a~graph has any NAC-coloring~\cite{np_complete}.

\begin{figure}[ht]
	\centering
	\begin{tikzpicture}[rotate=90,scale=1.5]
		\node[vertex] (a) at (0,0) {};
		\node[vertex] (b) at (1,0) {};
		\node[vertex] (c) at (0.5,0.5) {};
		\node[vertex] (d) at (0,1.5) {};
		\node[vertex] (e) at (1,1.5) {};
		\node[vertex] (f) at (0.5,1) {};
		\draw[bedge] (a)edge(b) (b)edge(c) (c)edge(a) (d)edge(e) (e)edge(f) (f)edge(d) ;
		\draw[redge] (a)edge(d) (b)edge(e) (c)edge(f);
	\end{tikzpicture}
	\qquad
	\qquad
	\begin{tikzpicture}[rotate=90,scale=1.5]
		\node[vertex] (a) at (0.00,0) {};
		\node[vertex] (b) at (1.00,0) {};
		\node[vertex] (c) at (0.50,0.5) {};
		\node[vertex] (d) at (0.25,1) {};
		\node[vertex] (e) at (1.25,1) {};
		\node[vertex] (f) at (0.75,1.5) {};
		\draw[edge] (a)edge(b) (b)edge(c) (c)edge(a) (d)edge(e) (e)edge(f) (f)edge(d) ;
		\draw[edge] (a)edge(d) (b)edge(e) (c)edge(f);
	\end{tikzpicture}
	\caption[Flexible realization of 3-prism]{The $3$-prism is generically $2$-rigid, but has a~flexible realization (right).
		It~has a~unique NAC-coloring modulo swapping colors (left).}%
	\label{fig:3prism}
\end{figure}

A~nice example for~NAC-coloring showcase is the 3-prism graph in~\Cref{fig:3prism},
a~graph formed from two triangles with three interconnecting edges.
Such a~graph is rigid, still it~has a~NAC-coloring and flexible realizations.
The~flexible realizations are all the~realizations where
the~triangles are identical except for~translation.
It~can be easily seen that there are
no other NAC-colorings except for~the~one, where the~colors are swapped.

Among the~goals of this thesis is to
%
study basics of Rigidity theory with the~focus on existence of flexible realizations
and its combinatorial characterization using NAC-colorings.
%
In~\Cref{chapter:np},
we~show that the~problem of deciding if~a~graph has a~NAC-coloring
is NP-complete even on graphs with maximum degree five.
%
We~provide a~parametrized approach in~\Cref{chapter:fpt}
where complexity can be reduced for~graphs
with low internal structural complexity (treewidth).
%
The~relation of NAC-colorings with stable cuts is presented in~\Cref{chapter:stable_cuts},
and an~algorithm for~stable cut search on flexible graphs from~\cite{stable_cuts_legersky}
is described and implemented.
%
In~\Cref{chapter:algo},
we~provide an~algorithm
that can find all the~NAC-colorings of a~graph.
Multiple heuristics are used to significantly
reduce the~runtime of the~algorithm.
%
We~describe the~implementation in~\Cref{chapter:impl}
and compare and evaluate the~performance of the~algorithm
compared to previous implementations,
and also we~compare individual heuristics among each other
in~\Cref{chapter:benchmarks}.
Multiple relevant graph classes are considered.

The~code implemented in~this thesis available at \cite{my_code}
shall enrich PyRigi~\cite{pyrigi},
a library used for~research in~Rigidity Theory.
Currently, there are no such algorithms implemented in~PyRigi itself,
the~only available implementation is rather naive in~FlexRiLoG~\cite{flexrilog}.

In~this thesis, the~results shown in~our previous paper~\cite{my_paper}
(done as part of VýLeT --- Student Summer Research Program)
are extended upon by introducing a~fixed-parameter tractable algorithm,
discussion of relation to stable cuts,
and present other heuristics that do not perform as well.
Here, we~also describe the~remaining topics more in~detail.

