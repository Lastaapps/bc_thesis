
\chapter{Conclusion}

\section{Discussion and Future work}

\todo[inline]{Prokonzultovat - jsou algo na stable cutech na 4-degree -> dává tohle smysl?}
%
It remains open if NAC-coloring is also NP-complete on graphs with
maximum degree four as this would diverge from the stable cuts results.

To further improve the performance of the current solution, the codebase
can be rewritten from Python into some sensible language like C, C++, Rust or Zig.
Also, further heuristics can be tried
and tested on additional graph classes
to reach additional performance gains.

Major gains for some graph classes
can be expected if the FPT algorithm is also implemented.
Expectations are that the FPT algorithm will outperform the current algorithm
for graphs with large number of NAC-colorings if only the number of colorings
is required (without the colorings itself being formed), and
for graph classes with small tree width.
This algorithm could also be parallelized up to some extent,
but this would make sense for large graphs only.
%
Also, it is possible that there is a Monte Carlo algorithm with significantly
lower time complexity compared to the algorithm proposed by us.

For flexible graphs including graphs where \( |E(G)| \le 2|V(G)| - 4 \),
we can find a stable cut in polynomial time using \Cref{alg:stableCutFlexible},
that is implemented in this thesis.
There is no known implementation of an algorithm
for graphs where \( |E(G)| = 2|V(G)| - 3 \)
even though it can be described and implemented
as shown in~\cite{stable_cuts_2v_3,stable_cuts_2v_3_revisit}.

\section{Results}

To remind us of the goals of the thesis,
we wanted to show that the problem of deciding if a graph has a NAC-coloring
is NP-complete also on graphs with maximal degree five.
We wanted to present and evaluate an algorithm for NAC-coloring search
that can find all the NAC-colorings of a graph.
Multiple heuristics should have been shown and evaluated.
We wanted to discuss relation to stable cuts and implement related algorithm
for stable cut search in a flexible graph.
Lastly, we wanted to propose an FPT algorithm for NAC-coloring counting
parametrized by tree width.

And so we did.
In this thesis, we showed that it is NP-complete
to decide if a graph has a NAC-coloring even on graphs with maximal degree five.
We discussed the relation of this result with stable cuts.
Regarding stable cuts, we implemented and described algorithm that can find
a stable cut for any flexible graph.
%
Back to NAC-colorings, we presented a significantly faster algorithm
for NAC-coloring search using graph decomposition and gradual coloring buildup.
%
We also proposed multiple heuristics and optimizations.
The performance of the algorithm was
first compared with the previous naive or simple
algorithms, and then heuristics were compared with each other and
well performing heuristics were chosen.
Our algorithm outperforms the previous implementation by a large margin
and also works fast for non-simple instances.
%
Lastly, we proposed an FPT algorithm for NAC-coloring counting
parametrized by tree width that can
possibly outperform our proposed algorithm if implemented.
%
Code is already partially refactored and merged into PyRigi.
%
With that, all the goals of the thesis have been fulfilled.

