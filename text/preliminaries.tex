
\chapter{Preliminaries}

\begin{chapterabstract}

	In this chapter we define terms from Rigidity theory with the goal to define
	NAC-coloring. We also introduce its relation to Rigidity theory and
	some simple statements considering
	structure of graphs and NAC-coloring existence.

\end{chapterabstract}

\section{Rigidity theory}

Rigidity theory, often also called Structural rigidity,
is a branch of combinatorial theory,
which studies properties of objects formed from flexible hinges and bars.
These objects can be represented as graphs with
a mapping into \( d \)-dimensional space.
Rigidity theory studies if such objects are rigid, or flexible.

For the rest of the thesis,
we consider only simple undirected graphs.
Often, only connected graphs are considered as disconnected graphs
represent uninteresting trivial cases.

%
\begin{definition}[\( d \)-realization~\cite{my_paper,np_complete}]
	\emph{A realization} of a graph \( G \) is a mapping \( p: V(G) \to \R^d \).
\end{definition}
%
\begin{definition}[Framework~\cite{my_paper,np_complete}]
	\emph{A framework} is a pair of a graph \( G \) and its realization.
\end{definition}
%
\begin{definition}[Nontrivial flex~\cite{my_paper,np_complete}]
	For a framework \( (G, p) \), \emph{a nontrivial flex} of
	\( p_0 = p \) such that for all \( 0 < t \le 1 \) we have \( p_t \) such that
	\( \|p_t(u) - p_t(v)\| = \|p(u) - p(v)\|\) for every \( \{u, v\} \in E(G) \),
	but \( \|p_t(u) - p_t(v)\| \ne \|p(u) - p(v)\| \) for some \( u, v \in V (G) \).
\end{definition}
%
\begin{definition}[\( d \)-flexible, \( d \)-rigid~\cite{my_paper}]
	A framework is \( d \)-flexible, if it has a nontrivial flex.
	Otherwise, it is \( d \)-rigid.
\end{definition}
%
To summarize all the previous definitions,
a framework is flexible if it can be transformed while preserving lengths of
edges of the graph.

From now on we consider only \emph{quasi-injective} realizations ---
two neighboring vertices cannot be mapped to the same position.
This is guarantied when assigned edge lengths are positive.
Note that two non-neighboring vertices can be mapped to the same position.
Similar results can be shown for frameworks with \emph{injective} realizations.
In this thesis they are of no interest for us.

It is NP-hard to decide whether a given \( d \)-dimensional framework is
\( d \)-rigid for \( d \ge 2 \)~\cite{d_rigidity_hardness}.
For \( d = 2 \) we can simplify the problem if we talk about \emph{generic}
behavior.
%
\begin{definition}[Flexible \& Rigid graphs~\cite{generically_rigid_graphs}]
	A graph is \( (generically) flexible \) if for almost all of
	its \( d \)-realizations are \( 2 \)-flexible.
	A graph is \( (generically) rigid \) if for almost all of
	its \( d \)-realizations are \( 2 \)-rigid.
\end{definition}
%

From now on we focus only on the \( 2 \)-dimensional case,
it is already known for some classes of graphs that they are flexible.
Namely, graphs where \( |E(G)| \le 2|V(G)| - 4 \)
are flexible~\cite{stable_cuts_2v_4}.

There is a significant graph class studied in Rigidity theory
called minimally rigid graphs or Laman graphs.
%
\begin{definition}[Pollaczek-Geiringer~\cite{polzacek_1927}, Laman~\cite{laman_1970}]
	Graph is \emph{minimally \( d \)-generically rigid} if it is \( d \)-rigid
	and \( (V(G), E(G) \setminus \{e\}) \) is \(d\)-flexible for each \( e \in E(G) \).
\end{definition}
%

A polynomial algorithm for testing \( 2 \)-rigidity was obtained
for minimally rigid graphs in~\cite{polynomial-min-rigid}.
We focus on these graphs in benchmarks.

\section{NAC-coloring}

\begin{definition}[NAC-coloring~\cite{legersky_original}]
	Let \( G \) be a graph and \( \delta: E(G) \to \{ \red, \blue \} \)
	be a coloring of edges:
	%
	\begin{itemize}
		\item A cycle in \( G \) is a \( \red \) cycle, if all its edges are \( \red \).
		\item A cycle in \( G \) is an \emph{almost \( \red \) cycle},
		      if exactly one of its edges is \( \blue \).
		      \emph{Almost \( \blue \) cycle} is defined analogously.
	\end{itemize}
	%
	A coloring \( \delta \) is called a NAC-coloring, if it is surjective
	and there are no almost cycles.
\end{definition}
%

It was shown, that it is NP-complete to decide if a graph has a NAC-coloring~\cite{np_complete}.
We elaborate further on this in later chapters.
Luckily, the check if a coloring is a NAC-coloring can be done in polynomial time.
Let \( E_\red\), resp. \( E_\blue \), be edges colored by \( \red \), resp.~\( \blue \)
in an edge coloring \( \delta \).
%
\begin{lemma}[\cite{legersky_original}]
	Let \( G \) be a graph. If \( \delta: E(G) \to \{ \red, \blue \} \) is a coloring of edges,
	then there are no almost cycles in \( G \) if and only if the connected components
	of \( G[E_\red] \) and \( G[E_\blue] \)%
	\footnote{For \(X subseteq E(G)\) by \(G[X]\) we denote induced subgraph of \( G \) on edges \( X \).}
	are induced subgraphs of \( G \).
\end{lemma}
%
The idea is quite simple --- if there is \WLOG{} \( \blue \) edge \( e \) incident to
two vertices \( u, v \) sharing the same connected component in \( G[E_\red] \),
an almost cycle is formed from a \( u \)-\( v \)-path in \( G[E_\red] \)
and the edge \( e \).

Now we present relation of NAC-colorings to flexible frameworks.
%
\begin{theorem}[\cite{legersky_original}]
	A connected graph \( G \) with at least one edge has a flexible
	quasi-injective \( 2 \)-dimensional realization if and only if it has a NAC-coloring.
\end{theorem}
%
The theorem has a constructive proof from which the related realization can be found.
The proof itself is nontrivial, and we don't show it here.
For us the most interesting questions are whether a graph has a NAC-coloring
and how many NAC-coloring does a graph have.
For an example have a look at~\Cref{fig:3prism}
from Introduction.

Note that for a NAC-coloring \( \delta \) there exists a NAC-coloring of the same
graph where \( \red \) and \( \blue \) colors are swapped.
Therefore, we define a measure where these \emph{congruent} colorings are
counted as one.
%
\begin{definition}
	By \( \nac{G} \) we denote all the NAC-coloring of \( G \).
	By \( \nnac{G} \) we denote the number of NAC-coloring of \( G \)
	while only one coloring is considered from each congruent pair.
	It holds that \( \nnac{G} = | \nac{G} | / 2 \)
\end{definition}
%

\subsection{NAC-coloring constraints}

In algorithms related to NAC-coloring search it is beneficial
to reduce the search space. Often we know that if colors differ for some
set of edges, the resulting coloring cannot be a NAC-coloring.

If there is a \( C_3 \) (a triangle) as a subgraph of \( G \)
all the edges of \( C_3 \) must map to the same color.
Otherwise, two edges of \( C_3 \) are \WLOG{} \( \red \) and one is \( \blue \).
Therefore, \( C_3 \) forms an almost cycle.
This idea can be expanded further to neighboring images.
%
\begin{definition}[\( \triangle \)-connected component~\cite{legersky_original}]
	Let \( G \) be a graph and \( \sim^\prime_\triangle \) be
	a relation on \( E(G) \times E(G) \) such that \( e_1 \sim^\prime_\triangle e_2 \)
	iff there exists a subgraph \( C_3 \) of \( G \)
	such that \( e_1, e_2 \in E(C_3) \).
	Let \( \sim_\triangle \) be the reflex-transitive closure of \( \sim^\prime_\triangle \).
	The graph \( G \) is called \emph{\( \triangle \)-connected} if \( e_1 \sim_\triangle \)
	for all \( e_1, e_2 \in E(G) \).
	A \emph{\( \triangle \)-connected component} is a maximal subgraph \( G^\prime \) of \( G \) such
	that \( G^\prime \) is \( \triangle \)-connected.
\end{definition}
%
\begin{lemma}[\cite{legersky_original}]
	Let \( \delta \) be a coloring of a graph \( G \) such that there are
	no almost cycles. If \( G^\prime \) is
	a \( \triangle \)-connected subgraph of \( G \),
	then \( \delta(e_1) = \delta(e_2) \) for all \( e_1, e_2 \in E(G^\prime) \).
\end{lemma}
%
The naive algorithm that will be described more in detail in the following chapters
iterates through all the possible edge colorings of a graph.
If \( \triangle \)-connected components are employed in the algorithm,
the whole components can be checked instead of individual edges.
Later we extend on the idea of \( \triangle \)-connected components.

A stable cut set is an independent set of vertices that is also a vertex cut.
If there is such set in a graph, a NAC-coloring can be simply found.
We elaborate more on this case in chapter~\Cref{chapter:stable_cuts}.

%
\begin{lemma}[\cite{legersky_original}]
	Let \( G \) be a connected graph, \( |E(G)| \ge 2 \). If there is \( E_c \subseteq E(G) \)
	edge cut in \( G \) such that \( E_c \) separates \( G \) and the subgraph of \( G \)
	induced by \( E_c \) contains no path of length four, then \( G \) has a NAC-coloring.
\end{lemma}
%
To construct a NAC-coloring, first consider \( E_c^\prime \) minimal subset of \( E_c \)
such that it is also an edge cut in \( G \).
We color edges in \( E_c^\prime \) \( \red \) and the other edges \( \blue \).
See~\cite{legersky_original} for a complete proof.

For graph \( G \), some vertex \( v \in G \) is an articulation vertex if
\( \{v\} \) is a vertex cut in \( G \).
When all the articulation vertices are found,
graph can be decomposed into blocks
--- vertex \( 2 \)-connected components.
As no cycles pass through multiple blocks, we can color each block
independently as long as both colors are used.~\cite{my_paper}.
We formalize this observation later.

\section{Other useful terms}

Lastly we define some terms that will be used in the following chapters.
%
\begin{definition}%
	\label{def:2-tree}
	A \emph{2-tree} is a graph formed by merging triangles by edge identification.
\end{definition}

For the following proof about NAC-coloring search NP-completeness we need
well known NP-complete problem for reduction, \emph{3-SAT} is used in our reduction.
For a formula of propositional logic SAT answers the question
whether the formula is satisfiable ---
there exists a truth assignment of literals that satisfies the formula.
3-SAT problem is an alternative formulation of SAT
where the formula must be in 3-CNF --- conjunction of clauses with three literals.
For example, \( (A \lor B \lor \lnot C) \land (\lnot A \lor D \lor \lnot E) \)
is in 3-CNF\@.
% TODO For simper remembering, the formulas look like a face with a nose \( (\lor)\land(\lor) \).
It was shown that 3-SAT is NP-complete,
and it is a common tool for NP-completeness proofs~\cite{3-sat}.

