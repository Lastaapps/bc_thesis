
\chapter{Preliminaries}

\begin{chapterabstract}

	In this chapter we define terminology from Rigidity theory including NAC-coloring.
	We also introduce its relation to Rigidity theory and
	some simple statements considering
	structure of graphs and NAC-coloring existence.

\end{chapterabstract}

\section{Rigidity theory}

Rigidity theory, often also called Structural rigidity,
is a branch of combinatorial theory,
which studies properties of objects formed from flexible joins and rigid bars.
These objects can be represented as graphs with
a mapping into \( d \)-dimensional space.
Rigidity theory studies if such objects are rigid, or flexible.

For the rest of the thesis,
we consider only simple undirected graphs.
Often, only connected graphs are considered as disconnected graphs
represent uninteresting trivial cases.
By \( V(G) \) we denote vertices of a graph \( G \) and
by \( E(G) \) we denote edges of a graph \( G \).
The following definitions are inspired by~\cite{np_complete,my_paper}.

%
\begin{definition}[\( d \)-realization]
	\emph{A \( d \)-realization} of a graph \( G \) is a mapping \( p: V(G) \to \R^d \).
\end{definition}
%
\begin{definition}[Framework]
	\emph{A framework} is a pair of a graph \( G \) and its realization.
\end{definition}
%
\begin{definition}[Nontrivial flex]
	For a framework \( (G, p) \), \emph{a nontrivial flex} of
	\( p_0 = p \) such that for all \( 0 < t \le 1 \) we have \( p_t \) such that
	\( \|p_t(u) - p_t(v)\| = \|p(u) - p(v)\|\) for every \( \{u, v\} \in E(G) \),
	but \( \|p_t(u) - p_t(v)\| \ne \|p(u) - p(v)\| \) for some \( u, v \in V (G) \).
\end{definition}
%
\begin{definition}[\( d \)-flexible, \( d \)-rigid]
	A framework is \( d \)-flexible, if it has a nontrivial flex.
	Otherwise, it is \( d \)-rigid.
\end{definition}
%
To summarize all the previous definitions,
a framework is flexible if it can be deformed while preserving lengths of
edges of the graph.

From now on we consider only \emph{quasi-injective} realizations ---
two neighboring vertices cannot be mapped to the same position.
This is guarantied when distances of neighboring vertices are positive.
Note that two non-neighboring vertices can be mapped to the same position.
Similar results can be shown for frameworks with \emph{injective} realizations
\todo[inline]{Footnote
	% \footnote{
	For example, it is also NP-complete to
	recognize graphs that have a generically rigid injective realization in \( \R^1 \).
	% }.
}
In this thesis they are of no interest for us.

It is NP-hard to decide whether a given \( d \)-dimensional framework is
\( d \)-rigid for \( d \ge 2 \)~\cite{d_rigidity_hardness}.
We can simplify the problem if we talk about \emph{generic}
behavior.
%
\begin{definition}[Flexible \& Rigid graphs~\cite{generically_rigid_graphs}]
	A graph is \( (generically) flexible \) if almost all of
	its \( 2 \)-realizations are \( 2 \)-flexible.
	A graph is \( (generically) rigid \) if almost all of
	its \( 2 \)-realizations are \( 2 \)-rigid.
\end{definition}
%

From now on we focus only on the \( 2 \)-dimensional case,
in which there is a combinatorial characterization of rigid graphs.

There is a significant graph class studied in Rigidity theory
called minimally rigid graphs or Laman graphs.
This class was described by Pollaczek-Geiringer~\cite{polzacek_1927}
and Laman~\cite{laman_1970} independently.
%
\begin{definition}
	Graph is \emph{minimally generically \( d \)-rigid} if it is \( d \)-rigid
	and \( G - e \)\footnote{
		\( G - e \) denotes a graph formed from \( G \) by removing an edge \( e \).
		Formally, \( G - e = (V(G), E(G) \setminus \{e\}) \).
	}
	is \(d\)-flexible for each \( e \in E(G) \).
\end{definition}
%
% \begin{theorem}[Pollaczek-Geiringer/Laman]
% 	There is a polynomial time algorithm for deciding if a graph is minimally
% \end{theorem}

It is already known for some classes of graphs that they are flexible.
Namely, graphs where \( |E(G)| \le 2|V(G)| - 4 \)
are flexible~\cite{stable_cuts_2v_4}.

A polynomial algorithm for testing \( 2 \)-rigidity was obtained
for minimally rigid graphs in~\cite{polynomial-min-rigid}.
We focus on these graphs in benchmarks.

\section{NAC-coloring}

It is algorithmically hard to find flexible realizations of a graph
just by trying realizations and checking if they are flexible
as the plane is continuous.
In~\cite{legersky_original} a new edge coloring is proposed
that corresponds with existence of a flexible realization.

\begin{definition}[NAC-coloring~\cite{legersky_original}]
	Let \( G \) be a graph and \( \delta: E(G) \to \{ \red, \blue \} \)
	be a coloring of edges:
	%
	\begin{itemize}
		\item A cycle in \( G \) is a \( \red \) cycle, if all its edges are \( \red \).
		\item A cycle in \( G \) is an \emph{almost \( \red \) cycle},
		      if exactly one of its edges is \( \blue \).
		      \emph{Almost \( \blue \) cycle} is defined analogously.
	\end{itemize}
	%
	By \emph{an almost cycle} we denote both almost \( \red \) and almost \( \blue \) cycles.
	A coloring \( \delta \) is called a NAC-coloring, if it is surjective
	and there are no almost cycles.
\end{definition}
%

It was shown in~\cite{np_complete}, that it is NP-complete to decide if a graph has a NAC-coloring.
We elaborate further on this in later chapters.
Thanks to the following \Cref{lemma:is_nac_coloring},
the check if a coloring is a NAC-coloring can be done in polynomial time.
Let \( E_\red\), resp. \( E_\blue \), be edges colored by \( \red \), resp.~\( \blue \),
in an edge coloring \( \delta \).
%
\begin{lemma}[\cite{legersky_original}]%
	\label{lemma:is_nac_coloring}
	Let \( G \) be a graph. If \( \delta: E(G) \to \{ \red, \blue \} \) is a coloring of edges,
	then there are no almost cycles in \( G \) if and only if the connected components
	of \( G[E_\red] \) and \( G[E_\blue] \)%
	\footnote{
		For \(X \subseteq V(G)\), by \(G[X]\) we denote
		the induced subgraph of \( G \) on vertices \( X \),
		\( G[X] = (X, \{ \{u, v\} \mid u, v \in X\}) \).
		%
		For \(Y \subseteq E(G)\), by \(G[Y]\) we denote
		the induced subgraph of \( G \) on edges \( Y \),
		\( G[Y] = (\{ v \mid \exists u \in V(G) : \{u, v\} \in Y\}, Y) \).
	}
	are induced subgraphs of \( G \).
\end{lemma}
%
The idea is quite simple --- if there is w.l.o.g.\ \( \blue \) edge \( \{u, v\} \)
and \( \{u, v\} \) share the same connected component in \( G[E_\red] \),
an almost cycle is formed from a \( u \)-\( v \)-path in \( G[E_\red] \)
and the edge \( \{u, v\} \).

Now we present relation of NAC-colorings and flexible frameworks.
%
\begin{theorem}[\cite{legersky_original}]
	A connected graph \( G \) with at least one edge has a flexible
	quasi-injective \( 2 \)-dimensional realization if and only if it has a NAC-coloring.
\end{theorem}
%
The theorem has a constructive proof from which a flexible realization
can be found for a given NAC-coloring.
The proof itself is nontrivial, and we do not show it here.
For us the most interesting questions are whether a graph has a NAC-coloring
and how many NAC-coloring does a graph have.
For an example have a look at~\Cref{fig:3prism}
from Introduction.

Note that for each NAC-coloring \( \delta \) on graph \( G \)
there exists a NAC-coloring \( \delta^\prime \) on \( G \)
where \( \red \) and \( \blue \) colors are swapped.
%
\begin{definition}
	By \( \nac{G} \) we denote the set of all the NAC-coloring of graph \( G \).
	By \( \nnac{G} \) we denote the number of NAC-colorings of \( G \)
	up to swapping the colors.
	That is \( \nnac{G} = | \nac{G} | / 2 \)
\end{definition}
%

\subsection{NAC-coloring constraints}

In algorithms related to NAC-coloring search it is beneficial
to reduce the search space.
It can be often seen from the graph's structure
that if colors differ for some set of edges,
the resulting coloring cannot be a NAC-coloring.

All the edges of a 3-cycle subgraph \( C_3 \) must map to the same color.
Otherwise, two edges of \( C_3 \) are w.l.o.g.\ \( \red \) and one is \( \blue \).
Therefore, \( C_3 \) forms an almost cycle.
This idea can be expanded further to neighboring triangles.
%
\begin{definition}[\( \triangle \)-connected component~\cite{legersky_original}]
	Let \( G \) be a graph and \( \sim^\prime_\triangle \) be
	a relation on \( E(G) \times E(G) \) such that \( e_1 \sim^\prime_\triangle e_2 \)
	iff there exists a 3-cycle subgraph \( C_3 \) of \( G \)
	such that \( e_1, e_2 \in E(C_3) \).
	Let \( \sim_\triangle \) be the reflexive-transitive closure of \( \sim^\prime_\triangle \).
	The graph \( G \) is called \emph{\( \triangle \)-connected} if \( e_1 \sim_\triangle \)
	for all \( e_1, e_2 \in E(G) \).
	A \emph{\( \triangle \)-connected component} is a maximal subgraph \( G^\prime \) of \( G \) such
	that \( G^\prime \) is \( \triangle \)-connected.
\end{definition}
%
\begin{lemma}[\cite{legersky_original}]
	Let \( \delta \) be a coloring of a graph \( G \) such that there are
	no almost cycles. If \( G^\prime \) is
	a \( \triangle \)-connected subgraph of \( G \),
	then \( \delta(e_1) = \delta(e_2) \) for all \( e_1, e_2 \in E(G^\prime) \).
\end{lemma}
%
The naive algorithm that is described in more detail in the following chapters
iterates through all the possible edge colorings of a graph.
If \( \triangle \)-connected components are employed in the algorithm,
the whole \( \triangle \)-connected components can be checked instead of individual edges.
Later we further improve the idea of \( \triangle \)-connected components.

Now we define common terminology with the goal to define stable cuts in a graph.
%
\begin{definition}[Independent set]
	\emph{An independent set} of a graph \( G \) is a set \( I \subseteq V(G) \) such that
	\( \forall u, v \in I : \{u, v\} \not\in E(G) \).
\end{definition}
%
\begin{definition}[Vertex cut]
	\emph{A vertex cut} of a graph \( G \) is a set \( I \subseteq V(G) \) such that
	\( G \setminus I \) (\( G \) with \( I \) removed) is a disconnected graph.
\end{definition}
%
\begin{definition}[Edge cut]
	\emph{An edge cut} of a graph \( G \) is a set \( I \subseteq E(G) \) such that
	\( G \setminus I \) (\( G \) with \( I \) removed) is a disconnected graph.
\end{definition}
%
\begin{definition}[Partitions of a vertex cut]
	For a vertex cut \( I \subseteq V(G) \),
	let \( G^\prime \) be a subgraph of \( G \), \( G^\prime = G \setminus I \).
	Nonempty sets \( A, B \subsetneq V(G^\prime) \) are called \emph{partitions} of the cut
	if \( A \cup B = V(G), A \cap B = S \) and for each \( v \in A \setminus S, u \in B \setminus S \)
	there exists no \( u \)--\( v \)--path in \( G^\prime \).
\end{definition}
%
\begin{definition}[Stable cut]
	\emph{A stable cut} is an independent set of vertices that is also a vertex cut.
\end{definition}
%
If there is a stable cut in a graph, a NAC-coloring can be simply found.
We elaborate more on this case in chapter~\Cref{chapter:stable_cuts}.

We continue by other lemmas that are useful in our algorithm.
They are able to find a NAC-coloring of a graph in polynomial time for some graph classes.
%
\begin{lemma}[\cite{legersky_original}]
	Let \( G \) be a connected graph, \( |E(G)| \ge 2 \). If there is \( E_c \subseteq E(G) \)
	edge cut in \( G \) such that \( E_c \) separates \( G \) and the subgraph of \( G \)
	induced by \( E_c \) contains no path of length four, then \( G \) has a NAC-coloring.
\end{lemma}
%
The idea is to construct a NAC-coloring, first consider \( E_c^\prime \) minimal subset of \( E_c \)
such that it is also an edge cut in \( G \).
We color edges in \( E_c^\prime \) \( \red \) and the other edges \( \blue \).
See~\cite{legersky_original} for a complete proof.

For graph \( G \), some vertex \( v \in G \) is \emph{an articulation} vertex if
\( \{v\} \) is a vertex cut in \( G \).
When all the articulation vertices are found,
graph can be decomposed into blocks
--- vertex \( 2 \)-connected components.
As no cycles pass through multiple blocks, we can color each block
independently as long as both colors are used.~\cite{my_paper}.
We formalize this observation later.

\section{Other useful terminology}

Lastly, we define some terms that will be used in the following chapters.

For the proof about NAC-coloring search NP-completeness,
a NP-complete problem for reduction is needed.
We use well known \emph{3-SAT} problem in our reduction.
Propositional logic SAT answers the question
whether the formula is satisfiable ---
there exists a truth assignment of variables that satisfies the formula.
3-SAT problem is an alternative formulation of SAT
where the formula must be in the conjunctive normal form (CNF), namely in 3-CNF
--- conjunction of clauses with three literals.
For example, \( (A \lor B \lor \lnot C) \land (\lnot A \lor D \lor \lnot E) \)
is in 3-CNF\@.
% TODO CNF formulas look like a face with a nose in the middle \( (\lor)\land(\lor) \).
It was shown that 3-SAT is NP-complete,
and it is a common tool for NP-completeness proofs~\cite{3-sat}.

There are other useful graph properties related to stable cuts
that will be introduced later in \Cref{chapter:stable_cuts}.
