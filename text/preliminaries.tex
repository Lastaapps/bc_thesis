
\chapter{Preliminaries}%
\label{chapter:preliminaries}

\begin{chapterabstract}

	In this chapter, we define terminology from Rigidity theory including NAC-colorings.
	We also introduce its relation to Rigidity theory and
	some simple statements considering
	the structure of graphs and NAC-coloring existence.

\end{chapterabstract}

\section{Rigidity theory}

Rigidity theory, often also called Structural rigidity,
is an area combining among the others
combinatorics and discrete and computational geometry,
which studies properties of objects formed from joints and rigid bars.
These objects can be represented as graphs with
a mapping into \( d \)-dimensional space.
Rigidity theory studies if such objects are rigid or flexible.

For the rest of the thesis,
we consider only simple undirected graphs.
Often, only connected graphs are considered as disconnected graphs
represent uninteresting trivial cases.
By \( V(G) \) we denote the vertices of a graph \( G \) and
by \( E(G) \) we denote the edges of~\( G \).

By \( G - v \), resp.\ \( G - e \),
where \( v \in V(G), e \in E(G) \) we denote the graph
formed from \( G \) by removing vertex \( v \), resp.\ edge \( e \).
Formally,
\( G - v = (V(G) \setminus \{v\}, E(G) \setminus \{\{u, v\} \mid u \in V(G)\}) \)
and \( G - e = (V(G), E(G) \setminus \{e\}) \).
%
By \( G \setminus S \), resp.\ \( G \setminus F \),
where \( S \subseteq V(G), F \subseteq E(G) \) we denote the graph
formed from \( G \) by removing vertices \( S \), resp.\ edges \( F \).
Formally,
\( G \setminus S = (V(G) \setminus S, E(G) \setminus \{\{u, v\} \mid v \in S, u \in V(G)\}) \)
and \( G \setminus F = (V(G), E(G) \setminus F) \).

The following definitions are taken from~\cite{np_complete,my_paper}.

%
\begin{definition}[\( d \)-realization]
	A \emph{\( d \)-realization} of a graph \( G \) is a mapping \( p: V(G) \to \R^d \).
\end{definition}
%
\begin{definition}[Framework]
	A \emph{framework} is a pair of a graph \( G \) and its realization.
\end{definition}
%
\begin{definition}[Nontrivial flex]
	A \emph{nontrivial flex} of framework \( (G, p) \) is a continuous curve of realizations \( p_t \)
	for \( 0 \le t \le 1\) such that
	\( p_0 = p \) and for all \( 0 < t \le 1 \)
	we have that
	\( \|p_t(u) - p_t(v)\| = \|p(u) - p(v)\|\) for every \( \{u, v\} \in E(G) \),
	but \( \|p_t(u) - p_t(v)\| \ne \|p(u) - p(v)\| \) for some \( u, v \in V (G) \).
\end{definition}
%
\begin{definition}[\( d \)-flexible, \( d \)-rigid]
	A framework is \emph{\( d \)-flexible}, if it has a nontrivial flex.
	Otherwise, it is \emph{\( d \)-rigid}.
\end{definition}
%
Realizations that differ only by translation,
rotation or reflection are called \emph{congruent}.
%
For a \( d \)-rigid realization, there may still be noncongruent realizations
with the same distances between vertices connected by edges.
%
If there are no such realizations for a \( d \)-rigid realization,
we call such realization \emph{globally \( d \)-rigid}.
To summarize all the previous definitions,
a framework is \( d \)-flexible if it can be continuously deformed
while preserving the lengths of the edges of the graph.

From now on, we consider only \emph{quasi-injective} realizations ---
two neighboring vertices cannot be mapped to the same position.
This is equivalent to the distances between adjacent vertices being positive.
Note that two non-neighboring vertices can be mapped to the same position.
The existence of \emph{injective} flexible realizations
has been also studied~\cite{injective_realizations},
but they are of no interest for us in this thesis
as there is no known combinatorial characterization of them.

For \( d = 1 \) or \( d \ge 3 \),
the question whether a graph has a flexible \( d \)-framework is simple.
%
For \( d = 1 \), all the vertices are on a line
and any vertex deviation necessarily changes distances (if the graph is connected).
Therefore, there is no flexible realization if and only if the graph is connected.
%
For \( d \ge 3 \), we can map all the vertices except two nonadjacent ones
onto a line and rotate one of them around the line.
Therefore, flexible \( d \)-framework exists if and only if the graph is not complete.

It is NP-hard to decide whether a given \( d \)-dimensional framework is
\( d \)-rigid for \( d \ge 2 \)~\cite{d_rigidity_hardness}.
We can simplify the problem if we talk about \emph{generic} behavior.
%
For rigid graphs, it holds that almost all%
\nohznamka{
	If a condition holds for almost all elements of a set,
	then the condition holds for all elements except for a subset of measure zero.
}
or all the realizations are
either \( d \)-rigid.
As there may be few \( d \)-flexible realizations,
we talk about paradoxical flexibility.
%
\begin{definition}[Flexible \& rigid graphs~\cite{generically_rigid_graphs}]
	A graph is \emph{(generically) \( d \)-flexible} if almost all of
	its \( d \)-realizations are \( d \)-flexible.
	A graph is \emph{(generically) \( d \)-rigid} if almost all of
	its \( d \)-realizations are \( d \)-rigid.
\end{definition}
%


There is a significant graph class studied in Rigidity theory
called minimally rigid graphs or Laman graphs.
%
\begin{definition}
	A graph \( G \) is \emph{minimally generically \( d \)-rigid} if it is \( d \)-rigid
	and \( G - e \) is \(d\)-flexible for each \( e \in E(G) \).
\end{definition}
%
From now on, we focus only on the \( 2 \)-dimensional case,
in which there is a combinatorial characterization of rigid graphs.
When we use \emph{rigid} or \emph{flexible} we mean \( 2 \)-rigid or \( 2 \)-flexible.

Pollaczek-Geiringer and later Laman gave characterization of minimally $2$-rigid graphs.
%
\begin{theorem}[\cite{polzacek_1927,laman_1970}]%
	\label[theorem]{theorem:laman_characterization}
	%
	A graph \( G \) is minimally \( 2 \)-rigid if and only if
	\( |E(G)| = 2|V(G)| - 3\) and
	\( |E(H)| \le 2|V(H)| - 3 \) for any \( H \) subgraph of \( G \).
\end{theorem}
%
\begin{corollary}[\cite{polzacek_1927,laman_1970}]
	For a graph \( G \) where \( |E(G)| < 2|V(G)| - 3 \) or
	where \( |E(G)| = 2|V(G)| - 3 \) and there exists a subgraph \( H \) such that \( |E(H)| > 2|V(H)| - 3 \),
	there exists a flexible realization of \( G \).
\end{corollary}
%
Based on \Cref{theorem:laman_characterization},
a polynomial algorithm
for testing $2$-rigidity was obtained~\cite{polynomial-min-rigid}.
A combinatorial classification of $d$-rigid graphs
for $d \geq 3$ remains an open problem.

It is already known for some classes of graphs that they are flexible.
Namely, graphs where \( |E(G)| \le 2|V(G)| - 4 \) are flexible.
This can be seen form \Cref{theorem:laman_characterization}.
Every rigid graph contains a spanning minimally rigid subgraph
(obtained by repeatably removing edges),
such that it has exactly \( 2|V(G)| - 3 \) edges.
So a graph with less than \( 2|V(G)| - 3 \) edges cannot be rigid.

% It can be derived from the fact that for \( |V(G)| \) vertices,
% there are \( 2|V(G)| \) degrees of freedom in the \( 2 \)-dimensional space.
% We can fix translation, rotation and reflection of the graph
% by fixing three vertices and by using \( 3 \) edges.
% %
% By adding a single, we fix one degree of freedom.
% Therefore, we can fix the remaining \( |V(G)| - 3 \) vertices
% (\( 2|V(G)| - 6 \) degrees of freedom)
% by using at least \( 2|V(G)| - 6 \) edges.
% But when there are less than \( 2|V(G)| - 3 \) edges in total,
% we cannot ever fix enough degrees of freedom.


There are polynomial algorithms for testing \( 2 \)-rigidity
for minimally rigid graphs, see for example~\cite{polynomial-min-rigid}.
We focus more on these graphs in benchmarks.


\section{NAC-colorings}

It is algorithmically hard to find flexible realizations of a graph
just by trying realizations and checking if they are flexible.
In~\cite{legersky_original} a new edge coloring is proposed
that corresponds with the existence of a flexible realization.

\todo[inline]{
	Picture, maybe prism again?
	Vykrást https://arxiv.org/abs/2412.16018
}

\begin{definition}[NAC-coloring~\cite{legersky_original}]
	Let \( G \) be a graph and \( \delta: E(G) \to \{ \red, \blue \} \)
	be a coloring of edges:
	%
	\begin{itemize}
		\item A cycle in \( G \) is a \( \red \) cycle, if all its edges are \( \red \),
		      analogously for a \( \blue \) cycles.
		\item A cycle in \( G \) is an \emph{almost \( \red \) cycle},
		      if exactly one of its edges is \( \blue \).
		      \emph{Almost \( \blue \) cycle} is defined analogously.
	\end{itemize}
	%
	By \emph{an almost cycle} we denote both almost \( \red \) and almost \( \blue \) cycles.
	The coloring~\( \delta \) is called a \emph{NAC-coloring}, if it is surjective
	and there are no almost cycles.
\end{definition}
%

As shown in~\cite{np_complete}, it is NP-complete to decide if a graph has a NAC-coloring.
We elaborate further on this in later chapters.
Let \( E_\red\), resp.\ \( E_\blue \), be edges colored by \( \red \), resp.~\( \blue \),
in an edge coloring \( \delta \).
%
\begin{lemma}[\cite{legersky_original}]%
	\label[lemma]{lemma:is_nac_coloring}
	Let \( G \) be a graph. If \( \delta: E(G) \to \{ \red, \blue \} \) is a coloring of edges,
	then there are no almost cycles in \( G \) if and only if the connected components
	of \( G[E_\red] \) and \( G[E_\blue] \)%
	\footnote{
		For \(Y \subseteq E(G)\), by \(G[Y]\) we denote
		the induced subgraph of \( G \) on edges \( Y \),
		\( G[Y] = (\{ v \mid \exists u \in V(G) : \{u, v\} \in Y\}, Y) \).
		%
		We also later use a similar notation for vertex-induced components:
		for \(X \subseteq V(G)\), by \(G[X]\) we denote
		the induced subgraph of \( G \) on vertices \( X \),
		\( G[X] = (X, \{ \{u, v\} \mid u, v \in X\}) \).
	}
	are induced subgraphs of \( G \).
\end{lemma}
%
The idea is quite simple --- if there is w.l.o.g.\ \( \blue \) edge \( \{u, v\} \)
and \( \{u, v\} \) share the same connected component in \( G[E_\red] \),
an almost cycle is formed from a \( u \)-\( v \)-path in \( G[E_\red] \)
and the edge \( \{u, v\} \).
The check can be done in polynomial (linear) time.

Now we present relation of NAC-colorings and flexible frameworks.
%
\begin{theorem}[\cite{legersky_original}]
	A connected graph \( G \) with at least one edge has a flexible
	quasi-injective \( 2 \)-dimensional realization if and only if it has a NAC-coloring.
\end{theorem}
%
The theorem has a constructive proof from which a flexible realization
can be found for a given NAC-coloring.
The proof itself is nontrivial, and we do not show it here.
For us, the most interesting questions are whether a graph has a NAC-coloring
and how many NAC-coloring does a graph have.
For example, have a look at~\Cref{fig:3prism}
from Introduction.

Note that for each NAC-coloring \( \delta \) on a graph \( G \),
there exists a NAC-coloring~\( \delta^\prime \) on~\( G \)
where \( \red \) and \( \blue \) colors are swapped.
%
\begin{definition}
	By \( \nac{G} \) we denote the set of all the NAC-coloring of graph \( G \).
	By \( \nnac{G} \) we denote the number of NAC-colorings of \( G \)
	up to swapping the colors.
	That is \( \nnac{G} = | \nac{G} | / 2 \)
\end{definition}
%

In algorithms related to NAC-coloring search, it is beneficial
to reduce the search space.
It can often be seen from the graph's structure
that if colors differ for some set of edges,
the resulting coloring cannot be a NAC-coloring.
%
For instance, all the edges of a 3-cycle subgraph \( C_3 \) must map to the same color.
Otherwise, two edges of \( C_3 \) are w.l.o.g.\ \( \red \) and one is \( \blue \).
Therefore, \( C_3 \) forms an almost cycle.
This idea can be expanded further to neighboring triangles.
%
\begin{definition}[\( \triangle \)-connected component~\cite{legersky_original}]
	\label[definition]{def:triangle_connected_component}
	Let \( G \) be a graph and \( \sim^\prime_\triangle \) be
	a relation on \( E(G) \times E(G) \) such that \( e_1 \sim^\prime_\triangle e_2 \)
	if and only if there exists a 3-cycle subgraph \( C_3 \) of \( G \)
	such that \( e_1, e_2 \in E(C_3) \).
	Let \( \sim_\triangle \) be the reflexive-transitive closure of \( \sim^\prime_\triangle \).
	The graph \( G \) is called \emph{\trcon{}} if \( e_1 \sim_\triangle \)
	for all \( e_1, e_2 \in E(G) \).
	A \emph{\trcon{} component} is a maximal subgraph \( G^\prime \) of \( G \) such
	that \( G^\prime \) is \( \triangle \)-connected.
\end{definition}
%
\begin{lemma}[\cite{legersky_original}]
	Let \( \delta \) be a coloring of a graph \( G \) such that there are
	no almost cycles. If \( G^\prime \) is
	a \( \triangle \)-connected subgraph of \( G \),
	then \( \delta(e_1) = \delta(e_2) \) for all \( e_1, e_2 \in E(G^\prime) \).
\end{lemma}
%
The naive algorithm that is described in more detail in the following chapters
iterates through all the possible edge colorings of a graph.
If \( \triangle \)-connected components are employed in the algorithm,
the whole \( \triangle \)-connected components can be checked instead of individual edges.
We further improve the idea of \( \triangle \)-connected components later.

Now we define common terminology with the goal to define stable cuts in a graph.

%
\begin{definition}[Stable set]
	A \emph{stable set} of a graph \( G \) is a set \( S \subseteq V(G) \) such that
	\( \forall u, v \in S : \{u, v\} \not\in E(G) \).
\end{definition}
%
Stable set is also often called an \emph{independent set}.
%
\begin{definition}[Vertex cut]
	A \emph{vertex cut} of a graph \( G \) is a set \( S \subseteq V(G) \) such that
	\( G \setminus S \) is a disconnected graph.
	%
	Vertices \( u, v \in V(G) \) are separated by \( S \)
	if \( u, v \not\in S \)
	and there is no path from \( u \) to \( v \) in \( G \setminus S \).
\end{definition}
%
\begin{definition}[Edge cut]
	An \emph{edge cut} of a graph \( G \) is a set \( F \subseteq E(G) \) such that
	\( G \setminus F \) is a disconnected graph.
\end{definition}
%
\begin{definition}[Stable cut]%
	\label[definition]{def:stable_cut}
	A \emph{stable cut} is a stable set of vertices that is also a vertex cut.
\end{definition}
%
If there is a stable cut in a graph, a NAC-coloring can be simply found.
We elaborate more on this case in~\Cref{chapter:stable_cuts}.

We continue with other lemmas that are useful in our algorithm.
They are able to find a NAC-coloring of a graph in polynomial time for some graph classes.
%
\begin{lemma}[\cite{legersky_original}]%
	\label[lemma]{lemma:weird_four_cycle}
	%
	Let \( G \) be a connected graph such that \( |E(G)| \ge 2 \). If there is an edge cut \( E_c \subseteq E(G) \)
	in~\( G \) and the subgraph of \( G \)
	induced by \( E_c \) contains no path of length four, then \( G \) has a NAC-coloring.
\end{lemma}
%
The original proof constructs a NAC-coloring by first considering
\( E_c^\prime \) minimal subset of \( E_c \)
such that it is also an edge cut in \( G \).
We color edges in \( E_c^\prime \) \( \red \) and the other edges \( \blue \).
See~\cite{legersky_original} for a complete proof.

For graph \( G \), some vertex \( v \in G \) is \emph{an articulation} vertex if
\( \{v\} \) is a vertex cut in \( G \).
When all the articulation vertices are found,
the graph can be decomposed into blocks
--- vertex \( 2 \)-connected components.
As no cycles pass through multiple blocks, we can color each block
independently as long as both colors are used~\cite{my_paper}.
We formalize this observation later.

There are other useful graph properties related to stable cuts
that will be introduced later in \Cref{chapter:stable_cuts}.

