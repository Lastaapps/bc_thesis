\chapter{Implementation \& Benchmarks}%
\label{chapter:benchmarks}

\begin{chapterabstract}

	In this chapter we first describe the structure of the project
	and discuss some of the design choices.
	After that, we evaluate performance of the algorithms
	proposed in \Cref{chapter:alg}.
	First, we compare the approaches with previous approaches,
	and then we compare heuristics with each other
	for different use-cases.
	We show reduction both in runtime and in the number
	of \IsNACColoring{} checks performed.

\end{chapterabstract}

\section{Implementation}

\todo[inline]{Popsat, jak/kde je kód přiložen}

In this section we first describe the structure of the project containing
the code of the algorithm.
Next, we mention libraries, relation to PyRigi and
some worth mentioning implementation details.

The code is written in Python, minimal supported version is Python 3.12.
To set up the project, create a virtual environment and install packages
from \texttt{requirements.txt}. On NixOS, you can use \texttt{shell.nix}.
See \texttt{README.md} for additional instructions.
We go through the main folders and files of the project.

In \texttt{graphs\_store} we store datasets used for benchmarking.
Graphs are either obtained from~\cite{extremal_graphs},
generated using Nauty~\cite{nauty} with a plugin~\cite{nauty_plugin}
or generated using NetworkX~\cite{networkx} and checks from PyRigi~\cite{pyrigi}.
Graph are usually stored in Graph6 format.
Code for reading graphs from the store can be found in \texttt{benchmarks/dataset.py},
code for generating some graph classes can be found in  \texttt{benchmarks/generators.py},
In the \texttt{benchmark} directory, there are all result of the benchmarks that we run
during the project development. The most relevant for a reader is the CSV file
stored in \texttt{precomputed} directory containing individual benchmark results.
The base directory contains tooling for running, visualizing and exporting benchmarks.
File \texttt{NAC\_playground.ipynb} in the root directory presents a simple case
to visualize how the Python API can be used.
File \texttt{NAC\_presentation.ipynb} shows how the basic benchmarks can be run.

The code of the algorithm described in~\Cref{sec:stable_cuts_implementation}
with additional helper functions is implemented in directory \texttt{stablecut}.
Note that some changes were done when the code was merged into PyRigi.
The code of the algorithm described in~\Cref{chapter:alg}
is stored in directory \texttt{nac}.
Directory \texttt{nac/util} stores helper functions and classes
like an implementation of the \textsc{UnionFind} data structure.
File \texttt{check.py} implements \IsNACColoring{} check.
File \texttt{monochromatic\_classes.py} is used to find \trcon{} components
and monochromatic classes in a graph. With this, we can compare performance
between using monochromatic classes, \trcon{} components or just edges.
File \texttt{cycle\_detection.py} holds algorithms for finding cycles
used in \Cref{sec:small_cycles}
and related heuristics as described in \Cref{chapter:alg}.
In \Cref{sec:polynomial_optimizations}
we presented checks that can sometimes find
a NAC-coloring or determine that there is none in polynomial time.
These checks are implemented in \texttt{existence.py} and
used mostly from \texttt{single.py} that is the entry-point for finding a single NAC-coloring of a graph.
General NAC-coloring searching is implemented in \texttt{search.py}
along with parameter processing, graph vertices relabeling,
optimizations like search for articulation vertices are performed.
After that, the correct algorithm (\Naive{}, \NaiveCycles{} or \Subgraphs{})
is chosen and called.
These algorithms are implemented in \texttt{algorithms.py} alongside many helper functions.
Heuristics for \Subgraphs{} algorithm are stored in \texttt{strategies.py}.
Tests of the whole code are stored in directory \texttt{test}.

Common function parameters are:
\textsc{graph} repressing the subgraph where NAC-colorings should be found,
\textsc{comp\_graph} where vertices are some integer IDs of monochromatic classes
and edges exists if the classes are neighboring.
An ID of a monochromatic class also serves as index into \textsc{component\_to\_edges}
that maps an ID of a monochromatic class to its edges.
NAC-colorings are represented as bit-masks where bit's offset correspond to a component ID.

As \IsNACColoring{} is a core component of all our algorithms, we optimize it as much as we could.
In the implementation of \IsNACColoring{}, subgraphs from \( \red \) and \( \blue \) edges are created.
To create such subgraphs in code, edges can be added to an empty graph
using NetworkX's function \textsc{add\_edges\_from}.
This is rather slow as creating new vertices in the empty graph causes noticeable overhead.
Therefore, we create a graph with no edges and the same vertices as the original graph,
cache it and reuse it for the checks. First edges are added, the check is run, and the edges are cleared.
By doing this, the performance of \IsNACColoring{} is increased by roughly 40\%.
Another way how the performance could be increased is by reserving space in lists
when the final size is known. This is unfortunately not possible in Python.

The code uses \textsc{Graph} class and related algorithms from NetworkX~\cite{networkx}
as the base of all operations. We use some utility functions from PyRigi~\cite{pyrigi}
related to rigidity tests and rigidity components search.
Otherwise, the code is not dependent on PyRigi.
Pytest is used for testing.

\section{Benchmarks}

In this section we first set meaningful targets for our benchmarks,
then we compare the performance of our algorithm with the previous implementations
and show running time and internal search optimizations for various graph classes.

The main question regarding NAC-coloring search is whether a graph has a NAC-coloring.
We usually ask the algorithm to not only answer yes, but to also provide a certificate.
For flexible graphs, it is usually algorithmically quite simple to find a NAC-coloring,
so this question is more interesting for rigid graphs.
For flexible graphs, it is more interesting to ask for the number of NAC-colorings
of a graph.
Note, that for larger flexible graphs with around thirty vertices
the number of NAC-colorings is huge as it often grows exponentially.
This slows our algorithm down as just materializing exponential
number of NAC-colorings takes exponential time.
For such cases, the FPT algorithm described in \Cref{chapter:fpt}
is a better fit as it does not materialize all the NAC-coloring on subgraphs.

The benchmarks comparing our algorithm with the previous implementations
were run on Linux on a laptop with Intel i7 of the 11th generation
with CPython 3.12 and SageMath 10.4.
The remaining benchmarks were run a laptop with Intel i5 of the 6th generation
using CPython 3.12.
On modern hardware, the times can be easily cut in half.

\subsection{Improvement over previous solutions}

\Cref{tab:all_min_rigid}
shows the time required for finding all the NAC-colorings
of all minimally rigid graphs with given size (generated using Nauty~\cite{nauty}
with a corresponding plugin~\cite{nauty_plugin}).
We show results of the implementation in \flexrilog{}~\cite{flexrilog} run in SageMath
and compare them to our implementation of the same \Naive{} algorithm
using $\triangle$-connected components
and monochromatic classes as described in \Cref{sec:NACvalid}.
Next column shows \NaiveCycles{} from \Cref{sec:small_cycles}
using monochromatic classes.
The last column is for the \NeighborsDegree{} (each initial subgraph has $k=4$ monochromatic classes)
with \MergeLinear{} merging strategy.
For twelve vertices, \Neighbors{} algorithm took around four hours for over 800k minimally rigid graphs.
In every case, our algorithms are significantly faster than implementation in \flexrilog{}~\cite{flexrilog}.
Notice also huge advantage gained by using monochromatic classes instead of \trcon{} components,
that are also used by \flexrilog{}.
%
\begin{table}[ht]
	\caption[Running times on graphs.]{
		The time (in seconds) needed to find all NAC-colorings for all graphs with a given size. Run by us.
		\textsc{FRLG} stands for \flexrilog{}, \textsc{ND} for \NeighborsDegree{}.}%
	\label{tab:all_min_rigid}
	\vspace{0.3cm}
	\centering
	\begin{tabular}{ccccccc}
		\hline
		\,$|V(G)|$\, & \,\#graphs\, & \,FRLG\, & \,$\triangle$-comps.\, & \,monochr.\, & \,cycles\, & \,\textsc{ND}\, \\
		\hline
		% 5        & 3           & 0.007 s      & 0.002 s            & 0.001 s       & 0.001 s & 0.002 s          \\
		% 6        & 13          & 0.063 s      & 0.030 s            & 0.010 s       & 0.005 s & 0.007 s          \\
		% 7        & 70          & 0.57 s       & 0.052 s            & 0.047 s       & 0.029 s & 0.041 s          \\
		8            & 608          & 14       & 1.09                   & 0.97         & 0.36       & 0.49            \\
		9            & 7\,222       & 509      & 34                     & 29           & 5.8        & 8.6             \\
		10           & 110\,132     & 27k      & 1\,725                 & 1\,446       & 151        & 213             \\
		11           & 2\,039\,273  & -        & -                      & -            & 5\,440     & 6\,650          \\
		\hline
	\end{tabular}
\end{table}

\Cref{fig:graph_time_minimally_rigid}
shows timings to compute all NAC-colorings of minimally rigid graphs
depending on the strategy used.
We did not list all NAC-coloring for minimally rigid graphs with more than twelve vertices
as there is too many such graphs.

The following dataset has been randomly generated
using NetworkX~\cite{networkx} and PyRigi~\cite{pyrigi}.
%
You can see that for graphs up to around fourteen vertices the \NaiveCycles{} algorithm
is still faster than \Subgraphs{}.
For graphs with more than eighteen vertices,
the growing advantage of \Subgraphs{} is already significant.

\begin{figure}[ht]
	\centering
	\scalebox{0.5}{%% Creator: Matplotlib, PGF backend
%%
%% To include the figure in your LaTeX document, write
%%   \input{<filename>.pgf}
%%
%% Make sure the required packages are loaded in your preamble
%%   \usepackage{pgf}
%%
%% Also ensure that all the required font packages are loaded; for instance,
%% the lmodern package is sometimes necessary when using math font.
%%   \usepackage{lmodern}
%%
%% Figures using additional raster images can only be included by \input if
%% they are in the same directory as the main LaTeX file. For loading figures
%% from other directories you can use the `import` package
%%   \usepackage{import}
%%
%% and then include the figures with
%%   \import{<path to file>}{<filename>.pgf}
%%
%% Matplotlib used the following preamble
%%   \def\mathdefault#1{#1}
%%   \everymath=\expandafter{\the\everymath\displaystyle}
%%   \IfFileExists{scrextend.sty}{
%%     \usepackage[fontsize=10.000000pt]{scrextend}
%%   }{
%%     \renewcommand{\normalsize}{\fontsize{10.000000}{12.000000}\selectfont}
%%     \normalsize
%%   }
%%   
%%   \ifdefined\pdftexversion\else  % non-pdftex case.
%%     \usepackage{fontspec}
%%     \setmainfont{DejaVuSans.ttf}[Path=\detokenize{/home/petr/Projects/PyRigi/.venv/lib/python3.12/site-packages/matplotlib/mpl-data/fonts/ttf/}]
%%     \setsansfont{DejaVuSans.ttf}[Path=\detokenize{/home/petr/Projects/PyRigi/.venv/lib/python3.12/site-packages/matplotlib/mpl-data/fonts/ttf/}]
%%     \setmonofont{DejaVuSansMono.ttf}[Path=\detokenize{/home/petr/Projects/PyRigi/.venv/lib/python3.12/site-packages/matplotlib/mpl-data/fonts/ttf/}]
%%   \fi
%%   \makeatletter\@ifpackageloaded{under\Score{}}{}{\usepackage[strings]{under\Score{}}}\makeatother
%%
\begingroup%
\makeatletter%
\begin{pgfpicture}%
\pgfpathrectangle{\pgfpointorigin}{\pgfqpoint{8.384376in}{2.841849in}}%
\pgfusepath{use as bounding box, clip}%
\begin{pgfscope}%
\pgfsetbuttcap%
\pgfsetmiterjoin%
\definecolor{currentfill}{rgb}{1.000000,1.000000,1.000000}%
\pgfsetfillcolor{currentfill}%
\pgfsetlinewidth{0.000000pt}%
\definecolor{currentstroke}{rgb}{1.000000,1.000000,1.000000}%
\pgfsetstrokecolor{currentstroke}%
\pgfsetdash{}{0pt}%
\pgfpathmoveto{\pgfqpoint{0.000000in}{0.000000in}}%
\pgfpathlineto{\pgfqpoint{8.384376in}{0.000000in}}%
\pgfpathlineto{\pgfqpoint{8.384376in}{2.841849in}}%
\pgfpathlineto{\pgfqpoint{0.000000in}{2.841849in}}%
\pgfpathlineto{\pgfqpoint{0.000000in}{0.000000in}}%
\pgfpathclose%
\pgfusepath{fill}%
\end{pgfscope}%
\begin{pgfscope}%
\pgfsetbuttcap%
\pgfsetmiterjoin%
\definecolor{currentfill}{rgb}{1.000000,1.000000,1.000000}%
\pgfsetfillcolor{currentfill}%
\pgfsetlinewidth{0.000000pt}%
\definecolor{currentstroke}{rgb}{0.000000,0.000000,0.000000}%
\pgfsetstrokecolor{currentstroke}%
\pgfsetstrokeopacity{0.000000}%
\pgfsetdash{}{0pt}%
\pgfpathmoveto{\pgfqpoint{0.588387in}{0.521603in}}%
\pgfpathlineto{\pgfqpoint{5.988063in}{0.521603in}}%
\pgfpathlineto{\pgfqpoint{5.988063in}{2.531888in}}%
\pgfpathlineto{\pgfqpoint{0.588387in}{2.531888in}}%
\pgfpathlineto{\pgfqpoint{0.588387in}{0.521603in}}%
\pgfpathclose%
\pgfusepath{fill}%
\end{pgfscope}%
\begin{pgfscope}%
\pgfsetbuttcap%
\pgfsetroundjoin%
\definecolor{currentfill}{rgb}{0.000000,0.000000,0.000000}%
\pgfsetfillcolor{currentfill}%
\pgfsetlinewidth{0.803000pt}%
\definecolor{currentstroke}{rgb}{0.000000,0.000000,0.000000}%
\pgfsetstrokecolor{currentstroke}%
\pgfsetdash{}{0pt}%
\pgfsys@defobject{currentmarker}{\pgfqpoint{0.000000in}{-0.048611in}}{\pgfqpoint{0.000000in}{0.000000in}}{%
\pgfpathmoveto{\pgfqpoint{0.000000in}{0.000000in}}%
\pgfpathlineto{\pgfqpoint{0.000000in}{-0.048611in}}%
\pgfusepath{stroke,fill}%
}%
\begin{pgfscope}%
\pgfsys@transformshift{1.015634in}{0.521603in}%
\pgfsys@useobject{currentmarker}{}%
\end{pgfscope}%
\end{pgfscope}%
\begin{pgfscope}%
\definecolor{textcolor}{rgb}{0.000000,0.000000,0.000000}%
\pgfsetstrokecolor{textcolor}%
\pgfsetfillcolor{textcolor}%
\pgftext[x=1.015634in,y=0.424381in,,top]{\color{textcolor}{\rmfamily\fontsize{10.000000}{12.000000}\selectfont\catcode`\^=\active\def^{\ifmmode\sp\else\^{}\fi}\catcode`\%=\active\def%{\%}$\mathdefault{3}$}}%
\end{pgfscope}%
\begin{pgfscope}%
\pgfsetbuttcap%
\pgfsetroundjoin%
\definecolor{currentfill}{rgb}{0.000000,0.000000,0.000000}%
\pgfsetfillcolor{currentfill}%
\pgfsetlinewidth{0.803000pt}%
\definecolor{currentstroke}{rgb}{0.000000,0.000000,0.000000}%
\pgfsetstrokecolor{currentstroke}%
\pgfsetdash{}{0pt}%
\pgfsys@defobject{currentmarker}{\pgfqpoint{0.000000in}{-0.048611in}}{\pgfqpoint{0.000000in}{0.000000in}}{%
\pgfpathmoveto{\pgfqpoint{0.000000in}{0.000000in}}%
\pgfpathlineto{\pgfqpoint{0.000000in}{-0.048611in}}%
\pgfusepath{stroke,fill}%
}%
\begin{pgfscope}%
\pgfsys@transformshift{1.561056in}{0.521603in}%
\pgfsys@useobject{currentmarker}{}%
\end{pgfscope}%
\end{pgfscope}%
\begin{pgfscope}%
\definecolor{textcolor}{rgb}{0.000000,0.000000,0.000000}%
\pgfsetstrokecolor{textcolor}%
\pgfsetfillcolor{textcolor}%
\pgftext[x=1.561056in,y=0.424381in,,top]{\color{textcolor}{\rmfamily\fontsize{10.000000}{12.000000}\selectfont\catcode`\^=\active\def^{\ifmmode\sp\else\^{}\fi}\catcode`\%=\active\def%{\%}$\mathdefault{6}$}}%
\end{pgfscope}%
\begin{pgfscope}%
\pgfsetbuttcap%
\pgfsetroundjoin%
\definecolor{currentfill}{rgb}{0.000000,0.000000,0.000000}%
\pgfsetfillcolor{currentfill}%
\pgfsetlinewidth{0.803000pt}%
\definecolor{currentstroke}{rgb}{0.000000,0.000000,0.000000}%
\pgfsetstrokecolor{currentstroke}%
\pgfsetdash{}{0pt}%
\pgfsys@defobject{currentmarker}{\pgfqpoint{0.000000in}{-0.048611in}}{\pgfqpoint{0.000000in}{0.000000in}}{%
\pgfpathmoveto{\pgfqpoint{0.000000in}{0.000000in}}%
\pgfpathlineto{\pgfqpoint{0.000000in}{-0.048611in}}%
\pgfusepath{stroke,fill}%
}%
\begin{pgfscope}%
\pgfsys@transformshift{2.106478in}{0.521603in}%
\pgfsys@useobject{currentmarker}{}%
\end{pgfscope}%
\end{pgfscope}%
\begin{pgfscope}%
\definecolor{textcolor}{rgb}{0.000000,0.000000,0.000000}%
\pgfsetstrokecolor{textcolor}%
\pgfsetfillcolor{textcolor}%
\pgftext[x=2.106478in,y=0.424381in,,top]{\color{textcolor}{\rmfamily\fontsize{10.000000}{12.000000}\selectfont\catcode`\^=\active\def^{\ifmmode\sp\else\^{}\fi}\catcode`\%=\active\def%{\%}$\mathdefault{9}$}}%
\end{pgfscope}%
\begin{pgfscope}%
\pgfsetbuttcap%
\pgfsetroundjoin%
\definecolor{currentfill}{rgb}{0.000000,0.000000,0.000000}%
\pgfsetfillcolor{currentfill}%
\pgfsetlinewidth{0.803000pt}%
\definecolor{currentstroke}{rgb}{0.000000,0.000000,0.000000}%
\pgfsetstrokecolor{currentstroke}%
\pgfsetdash{}{0pt}%
\pgfsys@defobject{currentmarker}{\pgfqpoint{0.000000in}{-0.048611in}}{\pgfqpoint{0.000000in}{0.000000in}}{%
\pgfpathmoveto{\pgfqpoint{0.000000in}{0.000000in}}%
\pgfpathlineto{\pgfqpoint{0.000000in}{-0.048611in}}%
\pgfusepath{stroke,fill}%
}%
\begin{pgfscope}%
\pgfsys@transformshift{2.651900in}{0.521603in}%
\pgfsys@useobject{currentmarker}{}%
\end{pgfscope}%
\end{pgfscope}%
\begin{pgfscope}%
\definecolor{textcolor}{rgb}{0.000000,0.000000,0.000000}%
\pgfsetstrokecolor{textcolor}%
\pgfsetfillcolor{textcolor}%
\pgftext[x=2.651900in,y=0.424381in,,top]{\color{textcolor}{\rmfamily\fontsize{10.000000}{12.000000}\selectfont\catcode`\^=\active\def^{\ifmmode\sp\else\^{}\fi}\catcode`\%=\active\def%{\%}$\mathdefault{12}$}}%
\end{pgfscope}%
\begin{pgfscope}%
\pgfsetbuttcap%
\pgfsetroundjoin%
\definecolor{currentfill}{rgb}{0.000000,0.000000,0.000000}%
\pgfsetfillcolor{currentfill}%
\pgfsetlinewidth{0.803000pt}%
\definecolor{currentstroke}{rgb}{0.000000,0.000000,0.000000}%
\pgfsetstrokecolor{currentstroke}%
\pgfsetdash{}{0pt}%
\pgfsys@defobject{currentmarker}{\pgfqpoint{0.000000in}{-0.048611in}}{\pgfqpoint{0.000000in}{0.000000in}}{%
\pgfpathmoveto{\pgfqpoint{0.000000in}{0.000000in}}%
\pgfpathlineto{\pgfqpoint{0.000000in}{-0.048611in}}%
\pgfusepath{stroke,fill}%
}%
\begin{pgfscope}%
\pgfsys@transformshift{3.197321in}{0.521603in}%
\pgfsys@useobject{currentmarker}{}%
\end{pgfscope}%
\end{pgfscope}%
\begin{pgfscope}%
\definecolor{textcolor}{rgb}{0.000000,0.000000,0.000000}%
\pgfsetstrokecolor{textcolor}%
\pgfsetfillcolor{textcolor}%
\pgftext[x=3.197321in,y=0.424381in,,top]{\color{textcolor}{\rmfamily\fontsize{10.000000}{12.000000}\selectfont\catcode`\^=\active\def^{\ifmmode\sp\else\^{}\fi}\catcode`\%=\active\def%{\%}$\mathdefault{15}$}}%
\end{pgfscope}%
\begin{pgfscope}%
\pgfsetbuttcap%
\pgfsetroundjoin%
\definecolor{currentfill}{rgb}{0.000000,0.000000,0.000000}%
\pgfsetfillcolor{currentfill}%
\pgfsetlinewidth{0.803000pt}%
\definecolor{currentstroke}{rgb}{0.000000,0.000000,0.000000}%
\pgfsetstrokecolor{currentstroke}%
\pgfsetdash{}{0pt}%
\pgfsys@defobject{currentmarker}{\pgfqpoint{0.000000in}{-0.048611in}}{\pgfqpoint{0.000000in}{0.000000in}}{%
\pgfpathmoveto{\pgfqpoint{0.000000in}{0.000000in}}%
\pgfpathlineto{\pgfqpoint{0.000000in}{-0.048611in}}%
\pgfusepath{stroke,fill}%
}%
\begin{pgfscope}%
\pgfsys@transformshift{3.742743in}{0.521603in}%
\pgfsys@useobject{currentmarker}{}%
\end{pgfscope}%
\end{pgfscope}%
\begin{pgfscope}%
\definecolor{textcolor}{rgb}{0.000000,0.000000,0.000000}%
\pgfsetstrokecolor{textcolor}%
\pgfsetfillcolor{textcolor}%
\pgftext[x=3.742743in,y=0.424381in,,top]{\color{textcolor}{\rmfamily\fontsize{10.000000}{12.000000}\selectfont\catcode`\^=\active\def^{\ifmmode\sp\else\^{}\fi}\catcode`\%=\active\def%{\%}$\mathdefault{18}$}}%
\end{pgfscope}%
\begin{pgfscope}%
\pgfsetbuttcap%
\pgfsetroundjoin%
\definecolor{currentfill}{rgb}{0.000000,0.000000,0.000000}%
\pgfsetfillcolor{currentfill}%
\pgfsetlinewidth{0.803000pt}%
\definecolor{currentstroke}{rgb}{0.000000,0.000000,0.000000}%
\pgfsetstrokecolor{currentstroke}%
\pgfsetdash{}{0pt}%
\pgfsys@defobject{currentmarker}{\pgfqpoint{0.000000in}{-0.048611in}}{\pgfqpoint{0.000000in}{0.000000in}}{%
\pgfpathmoveto{\pgfqpoint{0.000000in}{0.000000in}}%
\pgfpathlineto{\pgfqpoint{0.000000in}{-0.048611in}}%
\pgfusepath{stroke,fill}%
}%
\begin{pgfscope}%
\pgfsys@transformshift{4.288165in}{0.521603in}%
\pgfsys@useobject{currentmarker}{}%
\end{pgfscope}%
\end{pgfscope}%
\begin{pgfscope}%
\definecolor{textcolor}{rgb}{0.000000,0.000000,0.000000}%
\pgfsetstrokecolor{textcolor}%
\pgfsetfillcolor{textcolor}%
\pgftext[x=4.288165in,y=0.424381in,,top]{\color{textcolor}{\rmfamily\fontsize{10.000000}{12.000000}\selectfont\catcode`\^=\active\def^{\ifmmode\sp\else\^{}\fi}\catcode`\%=\active\def%{\%}$\mathdefault{21}$}}%
\end{pgfscope}%
\begin{pgfscope}%
\pgfsetbuttcap%
\pgfsetroundjoin%
\definecolor{currentfill}{rgb}{0.000000,0.000000,0.000000}%
\pgfsetfillcolor{currentfill}%
\pgfsetlinewidth{0.803000pt}%
\definecolor{currentstroke}{rgb}{0.000000,0.000000,0.000000}%
\pgfsetstrokecolor{currentstroke}%
\pgfsetdash{}{0pt}%
\pgfsys@defobject{currentmarker}{\pgfqpoint{0.000000in}{-0.048611in}}{\pgfqpoint{0.000000in}{0.000000in}}{%
\pgfpathmoveto{\pgfqpoint{0.000000in}{0.000000in}}%
\pgfpathlineto{\pgfqpoint{0.000000in}{-0.048611in}}%
\pgfusepath{stroke,fill}%
}%
\begin{pgfscope}%
\pgfsys@transformshift{4.833587in}{0.521603in}%
\pgfsys@useobject{currentmarker}{}%
\end{pgfscope}%
\end{pgfscope}%
\begin{pgfscope}%
\definecolor{textcolor}{rgb}{0.000000,0.000000,0.000000}%
\pgfsetstrokecolor{textcolor}%
\pgfsetfillcolor{textcolor}%
\pgftext[x=4.833587in,y=0.424381in,,top]{\color{textcolor}{\rmfamily\fontsize{10.000000}{12.000000}\selectfont\catcode`\^=\active\def^{\ifmmode\sp\else\^{}\fi}\catcode`\%=\active\def%{\%}$\mathdefault{24}$}}%
\end{pgfscope}%
\begin{pgfscope}%
\pgfsetbuttcap%
\pgfsetroundjoin%
\definecolor{currentfill}{rgb}{0.000000,0.000000,0.000000}%
\pgfsetfillcolor{currentfill}%
\pgfsetlinewidth{0.803000pt}%
\definecolor{currentstroke}{rgb}{0.000000,0.000000,0.000000}%
\pgfsetstrokecolor{currentstroke}%
\pgfsetdash{}{0pt}%
\pgfsys@defobject{currentmarker}{\pgfqpoint{0.000000in}{-0.048611in}}{\pgfqpoint{0.000000in}{0.000000in}}{%
\pgfpathmoveto{\pgfqpoint{0.000000in}{0.000000in}}%
\pgfpathlineto{\pgfqpoint{0.000000in}{-0.048611in}}%
\pgfusepath{stroke,fill}%
}%
\begin{pgfscope}%
\pgfsys@transformshift{5.379009in}{0.521603in}%
\pgfsys@useobject{currentmarker}{}%
\end{pgfscope}%
\end{pgfscope}%
\begin{pgfscope}%
\definecolor{textcolor}{rgb}{0.000000,0.000000,0.000000}%
\pgfsetstrokecolor{textcolor}%
\pgfsetfillcolor{textcolor}%
\pgftext[x=5.379009in,y=0.424381in,,top]{\color{textcolor}{\rmfamily\fontsize{10.000000}{12.000000}\selectfont\catcode`\^=\active\def^{\ifmmode\sp\else\^{}\fi}\catcode`\%=\active\def%{\%}$\mathdefault{27}$}}%
\end{pgfscope}%
\begin{pgfscope}%
\pgfsetbuttcap%
\pgfsetroundjoin%
\definecolor{currentfill}{rgb}{0.000000,0.000000,0.000000}%
\pgfsetfillcolor{currentfill}%
\pgfsetlinewidth{0.803000pt}%
\definecolor{currentstroke}{rgb}{0.000000,0.000000,0.000000}%
\pgfsetstrokecolor{currentstroke}%
\pgfsetdash{}{0pt}%
\pgfsys@defobject{currentmarker}{\pgfqpoint{0.000000in}{-0.048611in}}{\pgfqpoint{0.000000in}{0.000000in}}{%
\pgfpathmoveto{\pgfqpoint{0.000000in}{0.000000in}}%
\pgfpathlineto{\pgfqpoint{0.000000in}{-0.048611in}}%
\pgfusepath{stroke,fill}%
}%
\begin{pgfscope}%
\pgfsys@transformshift{5.924430in}{0.521603in}%
\pgfsys@useobject{currentmarker}{}%
\end{pgfscope}%
\end{pgfscope}%
\begin{pgfscope}%
\definecolor{textcolor}{rgb}{0.000000,0.000000,0.000000}%
\pgfsetstrokecolor{textcolor}%
\pgfsetfillcolor{textcolor}%
\pgftext[x=5.924430in,y=0.424381in,,top]{\color{textcolor}{\rmfamily\fontsize{10.000000}{12.000000}\selectfont\catcode`\^=\active\def^{\ifmmode\sp\else\^{}\fi}\catcode`\%=\active\def%{\%}$\mathdefault{30}$}}%
\end{pgfscope}%
\begin{pgfscope}%
\definecolor{textcolor}{rgb}{0.000000,0.000000,0.000000}%
\pgfsetstrokecolor{textcolor}%
\pgfsetfillcolor{textcolor}%
\pgftext[x=3.288225in,y=0.234413in,,top]{\color{textcolor}{\rmfamily\fontsize{10.000000}{12.000000}\selectfont\catcode`\^=\active\def^{\ifmmode\sp\else\^{}\fi}\catcode`\%=\active\def%{\%}Monochromatic classes}}%
\end{pgfscope}%
\begin{pgfscope}%
\pgfsetbuttcap%
\pgfsetroundjoin%
\definecolor{currentfill}{rgb}{0.000000,0.000000,0.000000}%
\pgfsetfillcolor{currentfill}%
\pgfsetlinewidth{0.803000pt}%
\definecolor{currentstroke}{rgb}{0.000000,0.000000,0.000000}%
\pgfsetstrokecolor{currentstroke}%
\pgfsetdash{}{0pt}%
\pgfsys@defobject{currentmarker}{\pgfqpoint{-0.048611in}{0.000000in}}{\pgfqpoint{-0.000000in}{0.000000in}}{%
\pgfpathmoveto{\pgfqpoint{-0.000000in}{0.000000in}}%
\pgfpathlineto{\pgfqpoint{-0.048611in}{0.000000in}}%
\pgfusepath{stroke,fill}%
}%
\begin{pgfscope}%
\pgfsys@transformshift{0.588387in}{0.918569in}%
\pgfsys@useobject{currentmarker}{}%
\end{pgfscope}%
\end{pgfscope}%
\begin{pgfscope}%
\definecolor{textcolor}{rgb}{0.000000,0.000000,0.000000}%
\pgfsetstrokecolor{textcolor}%
\pgfsetfillcolor{textcolor}%
\pgftext[x=0.289968in, y=0.865807in, left, base]{\color{textcolor}{\rmfamily\fontsize{10.000000}{12.000000}\selectfont\catcode`\^=\active\def^{\ifmmode\sp\else\^{}\fi}\catcode`\%=\active\def%{\%}$\mathdefault{10^{1}}$}}%
\end{pgfscope}%
\begin{pgfscope}%
\pgfsetbuttcap%
\pgfsetroundjoin%
\definecolor{currentfill}{rgb}{0.000000,0.000000,0.000000}%
\pgfsetfillcolor{currentfill}%
\pgfsetlinewidth{0.803000pt}%
\definecolor{currentstroke}{rgb}{0.000000,0.000000,0.000000}%
\pgfsetstrokecolor{currentstroke}%
\pgfsetdash{}{0pt}%
\pgfsys@defobject{currentmarker}{\pgfqpoint{-0.048611in}{0.000000in}}{\pgfqpoint{-0.000000in}{0.000000in}}{%
\pgfpathmoveto{\pgfqpoint{-0.000000in}{0.000000in}}%
\pgfpathlineto{\pgfqpoint{-0.048611in}{0.000000in}}%
\pgfusepath{stroke,fill}%
}%
\begin{pgfscope}%
\pgfsys@transformshift{0.588387in}{1.492526in}%
\pgfsys@useobject{currentmarker}{}%
\end{pgfscope}%
\end{pgfscope}%
\begin{pgfscope}%
\definecolor{textcolor}{rgb}{0.000000,0.000000,0.000000}%
\pgfsetstrokecolor{textcolor}%
\pgfsetfillcolor{textcolor}%
\pgftext[x=0.289968in, y=1.439764in, left, base]{\color{textcolor}{\rmfamily\fontsize{10.000000}{12.000000}\selectfont\catcode`\^=\active\def^{\ifmmode\sp\else\^{}\fi}\catcode`\%=\active\def%{\%}$\mathdefault{10^{2}}$}}%
\end{pgfscope}%
\begin{pgfscope}%
\pgfsetbuttcap%
\pgfsetroundjoin%
\definecolor{currentfill}{rgb}{0.000000,0.000000,0.000000}%
\pgfsetfillcolor{currentfill}%
\pgfsetlinewidth{0.803000pt}%
\definecolor{currentstroke}{rgb}{0.000000,0.000000,0.000000}%
\pgfsetstrokecolor{currentstroke}%
\pgfsetdash{}{0pt}%
\pgfsys@defobject{currentmarker}{\pgfqpoint{-0.048611in}{0.000000in}}{\pgfqpoint{-0.000000in}{0.000000in}}{%
\pgfpathmoveto{\pgfqpoint{-0.000000in}{0.000000in}}%
\pgfpathlineto{\pgfqpoint{-0.048611in}{0.000000in}}%
\pgfusepath{stroke,fill}%
}%
\begin{pgfscope}%
\pgfsys@transformshift{0.588387in}{2.066483in}%
\pgfsys@useobject{currentmarker}{}%
\end{pgfscope}%
\end{pgfscope}%
\begin{pgfscope}%
\definecolor{textcolor}{rgb}{0.000000,0.000000,0.000000}%
\pgfsetstrokecolor{textcolor}%
\pgfsetfillcolor{textcolor}%
\pgftext[x=0.289968in, y=2.013722in, left, base]{\color{textcolor}{\rmfamily\fontsize{10.000000}{12.000000}\selectfont\catcode`\^=\active\def^{\ifmmode\sp\else\^{}\fi}\catcode`\%=\active\def%{\%}$\mathdefault{10^{3}}$}}%
\end{pgfscope}%
\begin{pgfscope}%
\pgfsetbuttcap%
\pgfsetroundjoin%
\definecolor{currentfill}{rgb}{0.000000,0.000000,0.000000}%
\pgfsetfillcolor{currentfill}%
\pgfsetlinewidth{0.602250pt}%
\definecolor{currentstroke}{rgb}{0.000000,0.000000,0.000000}%
\pgfsetstrokecolor{currentstroke}%
\pgfsetdash{}{0pt}%
\pgfsys@defobject{currentmarker}{\pgfqpoint{-0.027778in}{0.000000in}}{\pgfqpoint{-0.000000in}{0.000000in}}{%
\pgfpathmoveto{\pgfqpoint{-0.000000in}{0.000000in}}%
\pgfpathlineto{\pgfqpoint{-0.027778in}{0.000000in}}%
\pgfusepath{stroke,fill}%
}%
\begin{pgfscope}%
\pgfsys@transformshift{0.588387in}{0.618459in}%
\pgfsys@useobject{currentmarker}{}%
\end{pgfscope}%
\end{pgfscope}%
\begin{pgfscope}%
\pgfsetbuttcap%
\pgfsetroundjoin%
\definecolor{currentfill}{rgb}{0.000000,0.000000,0.000000}%
\pgfsetfillcolor{currentfill}%
\pgfsetlinewidth{0.602250pt}%
\definecolor{currentstroke}{rgb}{0.000000,0.000000,0.000000}%
\pgfsetstrokecolor{currentstroke}%
\pgfsetdash{}{0pt}%
\pgfsys@defobject{currentmarker}{\pgfqpoint{-0.027778in}{0.000000in}}{\pgfqpoint{-0.000000in}{0.000000in}}{%
\pgfpathmoveto{\pgfqpoint{-0.000000in}{0.000000in}}%
\pgfpathlineto{\pgfqpoint{-0.027778in}{0.000000in}}%
\pgfusepath{stroke,fill}%
}%
\begin{pgfscope}%
\pgfsys@transformshift{0.588387in}{0.690168in}%
\pgfsys@useobject{currentmarker}{}%
\end{pgfscope}%
\end{pgfscope}%
\begin{pgfscope}%
\pgfsetbuttcap%
\pgfsetroundjoin%
\definecolor{currentfill}{rgb}{0.000000,0.000000,0.000000}%
\pgfsetfillcolor{currentfill}%
\pgfsetlinewidth{0.602250pt}%
\definecolor{currentstroke}{rgb}{0.000000,0.000000,0.000000}%
\pgfsetstrokecolor{currentstroke}%
\pgfsetdash{}{0pt}%
\pgfsys@defobject{currentmarker}{\pgfqpoint{-0.027778in}{0.000000in}}{\pgfqpoint{-0.000000in}{0.000000in}}{%
\pgfpathmoveto{\pgfqpoint{-0.000000in}{0.000000in}}%
\pgfpathlineto{\pgfqpoint{-0.027778in}{0.000000in}}%
\pgfusepath{stroke,fill}%
}%
\begin{pgfscope}%
\pgfsys@transformshift{0.588387in}{0.745790in}%
\pgfsys@useobject{currentmarker}{}%
\end{pgfscope}%
\end{pgfscope}%
\begin{pgfscope}%
\pgfsetbuttcap%
\pgfsetroundjoin%
\definecolor{currentfill}{rgb}{0.000000,0.000000,0.000000}%
\pgfsetfillcolor{currentfill}%
\pgfsetlinewidth{0.602250pt}%
\definecolor{currentstroke}{rgb}{0.000000,0.000000,0.000000}%
\pgfsetstrokecolor{currentstroke}%
\pgfsetdash{}{0pt}%
\pgfsys@defobject{currentmarker}{\pgfqpoint{-0.027778in}{0.000000in}}{\pgfqpoint{-0.000000in}{0.000000in}}{%
\pgfpathmoveto{\pgfqpoint{-0.000000in}{0.000000in}}%
\pgfpathlineto{\pgfqpoint{-0.027778in}{0.000000in}}%
\pgfusepath{stroke,fill}%
}%
\begin{pgfscope}%
\pgfsys@transformshift{0.588387in}{0.791237in}%
\pgfsys@useobject{currentmarker}{}%
\end{pgfscope}%
\end{pgfscope}%
\begin{pgfscope}%
\pgfsetbuttcap%
\pgfsetroundjoin%
\definecolor{currentfill}{rgb}{0.000000,0.000000,0.000000}%
\pgfsetfillcolor{currentfill}%
\pgfsetlinewidth{0.602250pt}%
\definecolor{currentstroke}{rgb}{0.000000,0.000000,0.000000}%
\pgfsetstrokecolor{currentstroke}%
\pgfsetdash{}{0pt}%
\pgfsys@defobject{currentmarker}{\pgfqpoint{-0.027778in}{0.000000in}}{\pgfqpoint{-0.000000in}{0.000000in}}{%
\pgfpathmoveto{\pgfqpoint{-0.000000in}{0.000000in}}%
\pgfpathlineto{\pgfqpoint{-0.027778in}{0.000000in}}%
\pgfusepath{stroke,fill}%
}%
\begin{pgfscope}%
\pgfsys@transformshift{0.588387in}{0.829662in}%
\pgfsys@useobject{currentmarker}{}%
\end{pgfscope}%
\end{pgfscope}%
\begin{pgfscope}%
\pgfsetbuttcap%
\pgfsetroundjoin%
\definecolor{currentfill}{rgb}{0.000000,0.000000,0.000000}%
\pgfsetfillcolor{currentfill}%
\pgfsetlinewidth{0.602250pt}%
\definecolor{currentstroke}{rgb}{0.000000,0.000000,0.000000}%
\pgfsetstrokecolor{currentstroke}%
\pgfsetdash{}{0pt}%
\pgfsys@defobject{currentmarker}{\pgfqpoint{-0.027778in}{0.000000in}}{\pgfqpoint{-0.000000in}{0.000000in}}{%
\pgfpathmoveto{\pgfqpoint{-0.000000in}{0.000000in}}%
\pgfpathlineto{\pgfqpoint{-0.027778in}{0.000000in}}%
\pgfusepath{stroke,fill}%
}%
\begin{pgfscope}%
\pgfsys@transformshift{0.588387in}{0.862946in}%
\pgfsys@useobject{currentmarker}{}%
\end{pgfscope}%
\end{pgfscope}%
\begin{pgfscope}%
\pgfsetbuttcap%
\pgfsetroundjoin%
\definecolor{currentfill}{rgb}{0.000000,0.000000,0.000000}%
\pgfsetfillcolor{currentfill}%
\pgfsetlinewidth{0.602250pt}%
\definecolor{currentstroke}{rgb}{0.000000,0.000000,0.000000}%
\pgfsetstrokecolor{currentstroke}%
\pgfsetdash{}{0pt}%
\pgfsys@defobject{currentmarker}{\pgfqpoint{-0.027778in}{0.000000in}}{\pgfqpoint{-0.000000in}{0.000000in}}{%
\pgfpathmoveto{\pgfqpoint{-0.000000in}{0.000000in}}%
\pgfpathlineto{\pgfqpoint{-0.027778in}{0.000000in}}%
\pgfusepath{stroke,fill}%
}%
\begin{pgfscope}%
\pgfsys@transformshift{0.588387in}{0.892306in}%
\pgfsys@useobject{currentmarker}{}%
\end{pgfscope}%
\end{pgfscope}%
\begin{pgfscope}%
\pgfsetbuttcap%
\pgfsetroundjoin%
\definecolor{currentfill}{rgb}{0.000000,0.000000,0.000000}%
\pgfsetfillcolor{currentfill}%
\pgfsetlinewidth{0.602250pt}%
\definecolor{currentstroke}{rgb}{0.000000,0.000000,0.000000}%
\pgfsetstrokecolor{currentstroke}%
\pgfsetdash{}{0pt}%
\pgfsys@defobject{currentmarker}{\pgfqpoint{-0.027778in}{0.000000in}}{\pgfqpoint{-0.000000in}{0.000000in}}{%
\pgfpathmoveto{\pgfqpoint{-0.000000in}{0.000000in}}%
\pgfpathlineto{\pgfqpoint{-0.027778in}{0.000000in}}%
\pgfusepath{stroke,fill}%
}%
\begin{pgfscope}%
\pgfsys@transformshift{0.588387in}{1.091347in}%
\pgfsys@useobject{currentmarker}{}%
\end{pgfscope}%
\end{pgfscope}%
\begin{pgfscope}%
\pgfsetbuttcap%
\pgfsetroundjoin%
\definecolor{currentfill}{rgb}{0.000000,0.000000,0.000000}%
\pgfsetfillcolor{currentfill}%
\pgfsetlinewidth{0.602250pt}%
\definecolor{currentstroke}{rgb}{0.000000,0.000000,0.000000}%
\pgfsetstrokecolor{currentstroke}%
\pgfsetdash{}{0pt}%
\pgfsys@defobject{currentmarker}{\pgfqpoint{-0.027778in}{0.000000in}}{\pgfqpoint{-0.000000in}{0.000000in}}{%
\pgfpathmoveto{\pgfqpoint{-0.000000in}{0.000000in}}%
\pgfpathlineto{\pgfqpoint{-0.027778in}{0.000000in}}%
\pgfusepath{stroke,fill}%
}%
\begin{pgfscope}%
\pgfsys@transformshift{0.588387in}{1.192416in}%
\pgfsys@useobject{currentmarker}{}%
\end{pgfscope}%
\end{pgfscope}%
\begin{pgfscope}%
\pgfsetbuttcap%
\pgfsetroundjoin%
\definecolor{currentfill}{rgb}{0.000000,0.000000,0.000000}%
\pgfsetfillcolor{currentfill}%
\pgfsetlinewidth{0.602250pt}%
\definecolor{currentstroke}{rgb}{0.000000,0.000000,0.000000}%
\pgfsetstrokecolor{currentstroke}%
\pgfsetdash{}{0pt}%
\pgfsys@defobject{currentmarker}{\pgfqpoint{-0.027778in}{0.000000in}}{\pgfqpoint{-0.000000in}{0.000000in}}{%
\pgfpathmoveto{\pgfqpoint{-0.000000in}{0.000000in}}%
\pgfpathlineto{\pgfqpoint{-0.027778in}{0.000000in}}%
\pgfusepath{stroke,fill}%
}%
\begin{pgfscope}%
\pgfsys@transformshift{0.588387in}{1.264125in}%
\pgfsys@useobject{currentmarker}{}%
\end{pgfscope}%
\end{pgfscope}%
\begin{pgfscope}%
\pgfsetbuttcap%
\pgfsetroundjoin%
\definecolor{currentfill}{rgb}{0.000000,0.000000,0.000000}%
\pgfsetfillcolor{currentfill}%
\pgfsetlinewidth{0.602250pt}%
\definecolor{currentstroke}{rgb}{0.000000,0.000000,0.000000}%
\pgfsetstrokecolor{currentstroke}%
\pgfsetdash{}{0pt}%
\pgfsys@defobject{currentmarker}{\pgfqpoint{-0.027778in}{0.000000in}}{\pgfqpoint{-0.000000in}{0.000000in}}{%
\pgfpathmoveto{\pgfqpoint{-0.000000in}{0.000000in}}%
\pgfpathlineto{\pgfqpoint{-0.027778in}{0.000000in}}%
\pgfusepath{stroke,fill}%
}%
\begin{pgfscope}%
\pgfsys@transformshift{0.588387in}{1.319748in}%
\pgfsys@useobject{currentmarker}{}%
\end{pgfscope}%
\end{pgfscope}%
\begin{pgfscope}%
\pgfsetbuttcap%
\pgfsetroundjoin%
\definecolor{currentfill}{rgb}{0.000000,0.000000,0.000000}%
\pgfsetfillcolor{currentfill}%
\pgfsetlinewidth{0.602250pt}%
\definecolor{currentstroke}{rgb}{0.000000,0.000000,0.000000}%
\pgfsetstrokecolor{currentstroke}%
\pgfsetdash{}{0pt}%
\pgfsys@defobject{currentmarker}{\pgfqpoint{-0.027778in}{0.000000in}}{\pgfqpoint{-0.000000in}{0.000000in}}{%
\pgfpathmoveto{\pgfqpoint{-0.000000in}{0.000000in}}%
\pgfpathlineto{\pgfqpoint{-0.027778in}{0.000000in}}%
\pgfusepath{stroke,fill}%
}%
\begin{pgfscope}%
\pgfsys@transformshift{0.588387in}{1.365194in}%
\pgfsys@useobject{currentmarker}{}%
\end{pgfscope}%
\end{pgfscope}%
\begin{pgfscope}%
\pgfsetbuttcap%
\pgfsetroundjoin%
\definecolor{currentfill}{rgb}{0.000000,0.000000,0.000000}%
\pgfsetfillcolor{currentfill}%
\pgfsetlinewidth{0.602250pt}%
\definecolor{currentstroke}{rgb}{0.000000,0.000000,0.000000}%
\pgfsetstrokecolor{currentstroke}%
\pgfsetdash{}{0pt}%
\pgfsys@defobject{currentmarker}{\pgfqpoint{-0.027778in}{0.000000in}}{\pgfqpoint{-0.000000in}{0.000000in}}{%
\pgfpathmoveto{\pgfqpoint{-0.000000in}{0.000000in}}%
\pgfpathlineto{\pgfqpoint{-0.027778in}{0.000000in}}%
\pgfusepath{stroke,fill}%
}%
\begin{pgfscope}%
\pgfsys@transformshift{0.588387in}{1.403619in}%
\pgfsys@useobject{currentmarker}{}%
\end{pgfscope}%
\end{pgfscope}%
\begin{pgfscope}%
\pgfsetbuttcap%
\pgfsetroundjoin%
\definecolor{currentfill}{rgb}{0.000000,0.000000,0.000000}%
\pgfsetfillcolor{currentfill}%
\pgfsetlinewidth{0.602250pt}%
\definecolor{currentstroke}{rgb}{0.000000,0.000000,0.000000}%
\pgfsetstrokecolor{currentstroke}%
\pgfsetdash{}{0pt}%
\pgfsys@defobject{currentmarker}{\pgfqpoint{-0.027778in}{0.000000in}}{\pgfqpoint{-0.000000in}{0.000000in}}{%
\pgfpathmoveto{\pgfqpoint{-0.000000in}{0.000000in}}%
\pgfpathlineto{\pgfqpoint{-0.027778in}{0.000000in}}%
\pgfusepath{stroke,fill}%
}%
\begin{pgfscope}%
\pgfsys@transformshift{0.588387in}{1.436904in}%
\pgfsys@useobject{currentmarker}{}%
\end{pgfscope}%
\end{pgfscope}%
\begin{pgfscope}%
\pgfsetbuttcap%
\pgfsetroundjoin%
\definecolor{currentfill}{rgb}{0.000000,0.000000,0.000000}%
\pgfsetfillcolor{currentfill}%
\pgfsetlinewidth{0.602250pt}%
\definecolor{currentstroke}{rgb}{0.000000,0.000000,0.000000}%
\pgfsetstrokecolor{currentstroke}%
\pgfsetdash{}{0pt}%
\pgfsys@defobject{currentmarker}{\pgfqpoint{-0.027778in}{0.000000in}}{\pgfqpoint{-0.000000in}{0.000000in}}{%
\pgfpathmoveto{\pgfqpoint{-0.000000in}{0.000000in}}%
\pgfpathlineto{\pgfqpoint{-0.027778in}{0.000000in}}%
\pgfusepath{stroke,fill}%
}%
\begin{pgfscope}%
\pgfsys@transformshift{0.588387in}{1.466263in}%
\pgfsys@useobject{currentmarker}{}%
\end{pgfscope}%
\end{pgfscope}%
\begin{pgfscope}%
\pgfsetbuttcap%
\pgfsetroundjoin%
\definecolor{currentfill}{rgb}{0.000000,0.000000,0.000000}%
\pgfsetfillcolor{currentfill}%
\pgfsetlinewidth{0.602250pt}%
\definecolor{currentstroke}{rgb}{0.000000,0.000000,0.000000}%
\pgfsetstrokecolor{currentstroke}%
\pgfsetdash{}{0pt}%
\pgfsys@defobject{currentmarker}{\pgfqpoint{-0.027778in}{0.000000in}}{\pgfqpoint{-0.000000in}{0.000000in}}{%
\pgfpathmoveto{\pgfqpoint{-0.000000in}{0.000000in}}%
\pgfpathlineto{\pgfqpoint{-0.027778in}{0.000000in}}%
\pgfusepath{stroke,fill}%
}%
\begin{pgfscope}%
\pgfsys@transformshift{0.588387in}{1.665304in}%
\pgfsys@useobject{currentmarker}{}%
\end{pgfscope}%
\end{pgfscope}%
\begin{pgfscope}%
\pgfsetbuttcap%
\pgfsetroundjoin%
\definecolor{currentfill}{rgb}{0.000000,0.000000,0.000000}%
\pgfsetfillcolor{currentfill}%
\pgfsetlinewidth{0.602250pt}%
\definecolor{currentstroke}{rgb}{0.000000,0.000000,0.000000}%
\pgfsetstrokecolor{currentstroke}%
\pgfsetdash{}{0pt}%
\pgfsys@defobject{currentmarker}{\pgfqpoint{-0.027778in}{0.000000in}}{\pgfqpoint{-0.000000in}{0.000000in}}{%
\pgfpathmoveto{\pgfqpoint{-0.000000in}{0.000000in}}%
\pgfpathlineto{\pgfqpoint{-0.027778in}{0.000000in}}%
\pgfusepath{stroke,fill}%
}%
\begin{pgfscope}%
\pgfsys@transformshift{0.588387in}{1.766373in}%
\pgfsys@useobject{currentmarker}{}%
\end{pgfscope}%
\end{pgfscope}%
\begin{pgfscope}%
\pgfsetbuttcap%
\pgfsetroundjoin%
\definecolor{currentfill}{rgb}{0.000000,0.000000,0.000000}%
\pgfsetfillcolor{currentfill}%
\pgfsetlinewidth{0.602250pt}%
\definecolor{currentstroke}{rgb}{0.000000,0.000000,0.000000}%
\pgfsetstrokecolor{currentstroke}%
\pgfsetdash{}{0pt}%
\pgfsys@defobject{currentmarker}{\pgfqpoint{-0.027778in}{0.000000in}}{\pgfqpoint{-0.000000in}{0.000000in}}{%
\pgfpathmoveto{\pgfqpoint{-0.000000in}{0.000000in}}%
\pgfpathlineto{\pgfqpoint{-0.027778in}{0.000000in}}%
\pgfusepath{stroke,fill}%
}%
\begin{pgfscope}%
\pgfsys@transformshift{0.588387in}{1.838083in}%
\pgfsys@useobject{currentmarker}{}%
\end{pgfscope}%
\end{pgfscope}%
\begin{pgfscope}%
\pgfsetbuttcap%
\pgfsetroundjoin%
\definecolor{currentfill}{rgb}{0.000000,0.000000,0.000000}%
\pgfsetfillcolor{currentfill}%
\pgfsetlinewidth{0.602250pt}%
\definecolor{currentstroke}{rgb}{0.000000,0.000000,0.000000}%
\pgfsetstrokecolor{currentstroke}%
\pgfsetdash{}{0pt}%
\pgfsys@defobject{currentmarker}{\pgfqpoint{-0.027778in}{0.000000in}}{\pgfqpoint{-0.000000in}{0.000000in}}{%
\pgfpathmoveto{\pgfqpoint{-0.000000in}{0.000000in}}%
\pgfpathlineto{\pgfqpoint{-0.027778in}{0.000000in}}%
\pgfusepath{stroke,fill}%
}%
\begin{pgfscope}%
\pgfsys@transformshift{0.588387in}{1.893705in}%
\pgfsys@useobject{currentmarker}{}%
\end{pgfscope}%
\end{pgfscope}%
\begin{pgfscope}%
\pgfsetbuttcap%
\pgfsetroundjoin%
\definecolor{currentfill}{rgb}{0.000000,0.000000,0.000000}%
\pgfsetfillcolor{currentfill}%
\pgfsetlinewidth{0.602250pt}%
\definecolor{currentstroke}{rgb}{0.000000,0.000000,0.000000}%
\pgfsetstrokecolor{currentstroke}%
\pgfsetdash{}{0pt}%
\pgfsys@defobject{currentmarker}{\pgfqpoint{-0.027778in}{0.000000in}}{\pgfqpoint{-0.000000in}{0.000000in}}{%
\pgfpathmoveto{\pgfqpoint{-0.000000in}{0.000000in}}%
\pgfpathlineto{\pgfqpoint{-0.027778in}{0.000000in}}%
\pgfusepath{stroke,fill}%
}%
\begin{pgfscope}%
\pgfsys@transformshift{0.588387in}{1.939152in}%
\pgfsys@useobject{currentmarker}{}%
\end{pgfscope}%
\end{pgfscope}%
\begin{pgfscope}%
\pgfsetbuttcap%
\pgfsetroundjoin%
\definecolor{currentfill}{rgb}{0.000000,0.000000,0.000000}%
\pgfsetfillcolor{currentfill}%
\pgfsetlinewidth{0.602250pt}%
\definecolor{currentstroke}{rgb}{0.000000,0.000000,0.000000}%
\pgfsetstrokecolor{currentstroke}%
\pgfsetdash{}{0pt}%
\pgfsys@defobject{currentmarker}{\pgfqpoint{-0.027778in}{0.000000in}}{\pgfqpoint{-0.000000in}{0.000000in}}{%
\pgfpathmoveto{\pgfqpoint{-0.000000in}{0.000000in}}%
\pgfpathlineto{\pgfqpoint{-0.027778in}{0.000000in}}%
\pgfusepath{stroke,fill}%
}%
\begin{pgfscope}%
\pgfsys@transformshift{0.588387in}{1.977576in}%
\pgfsys@useobject{currentmarker}{}%
\end{pgfscope}%
\end{pgfscope}%
\begin{pgfscope}%
\pgfsetbuttcap%
\pgfsetroundjoin%
\definecolor{currentfill}{rgb}{0.000000,0.000000,0.000000}%
\pgfsetfillcolor{currentfill}%
\pgfsetlinewidth{0.602250pt}%
\definecolor{currentstroke}{rgb}{0.000000,0.000000,0.000000}%
\pgfsetstrokecolor{currentstroke}%
\pgfsetdash{}{0pt}%
\pgfsys@defobject{currentmarker}{\pgfqpoint{-0.027778in}{0.000000in}}{\pgfqpoint{-0.000000in}{0.000000in}}{%
\pgfpathmoveto{\pgfqpoint{-0.000000in}{0.000000in}}%
\pgfpathlineto{\pgfqpoint{-0.027778in}{0.000000in}}%
\pgfusepath{stroke,fill}%
}%
\begin{pgfscope}%
\pgfsys@transformshift{0.588387in}{2.010861in}%
\pgfsys@useobject{currentmarker}{}%
\end{pgfscope}%
\end{pgfscope}%
\begin{pgfscope}%
\pgfsetbuttcap%
\pgfsetroundjoin%
\definecolor{currentfill}{rgb}{0.000000,0.000000,0.000000}%
\pgfsetfillcolor{currentfill}%
\pgfsetlinewidth{0.602250pt}%
\definecolor{currentstroke}{rgb}{0.000000,0.000000,0.000000}%
\pgfsetstrokecolor{currentstroke}%
\pgfsetdash{}{0pt}%
\pgfsys@defobject{currentmarker}{\pgfqpoint{-0.027778in}{0.000000in}}{\pgfqpoint{-0.000000in}{0.000000in}}{%
\pgfpathmoveto{\pgfqpoint{-0.000000in}{0.000000in}}%
\pgfpathlineto{\pgfqpoint{-0.027778in}{0.000000in}}%
\pgfusepath{stroke,fill}%
}%
\begin{pgfscope}%
\pgfsys@transformshift{0.588387in}{2.040221in}%
\pgfsys@useobject{currentmarker}{}%
\end{pgfscope}%
\end{pgfscope}%
\begin{pgfscope}%
\pgfsetbuttcap%
\pgfsetroundjoin%
\definecolor{currentfill}{rgb}{0.000000,0.000000,0.000000}%
\pgfsetfillcolor{currentfill}%
\pgfsetlinewidth{0.602250pt}%
\definecolor{currentstroke}{rgb}{0.000000,0.000000,0.000000}%
\pgfsetstrokecolor{currentstroke}%
\pgfsetdash{}{0pt}%
\pgfsys@defobject{currentmarker}{\pgfqpoint{-0.027778in}{0.000000in}}{\pgfqpoint{-0.000000in}{0.000000in}}{%
\pgfpathmoveto{\pgfqpoint{-0.000000in}{0.000000in}}%
\pgfpathlineto{\pgfqpoint{-0.027778in}{0.000000in}}%
\pgfusepath{stroke,fill}%
}%
\begin{pgfscope}%
\pgfsys@transformshift{0.588387in}{2.239262in}%
\pgfsys@useobject{currentmarker}{}%
\end{pgfscope}%
\end{pgfscope}%
\begin{pgfscope}%
\pgfsetbuttcap%
\pgfsetroundjoin%
\definecolor{currentfill}{rgb}{0.000000,0.000000,0.000000}%
\pgfsetfillcolor{currentfill}%
\pgfsetlinewidth{0.602250pt}%
\definecolor{currentstroke}{rgb}{0.000000,0.000000,0.000000}%
\pgfsetstrokecolor{currentstroke}%
\pgfsetdash{}{0pt}%
\pgfsys@defobject{currentmarker}{\pgfqpoint{-0.027778in}{0.000000in}}{\pgfqpoint{-0.000000in}{0.000000in}}{%
\pgfpathmoveto{\pgfqpoint{-0.000000in}{0.000000in}}%
\pgfpathlineto{\pgfqpoint{-0.027778in}{0.000000in}}%
\pgfusepath{stroke,fill}%
}%
\begin{pgfscope}%
\pgfsys@transformshift{0.588387in}{2.340331in}%
\pgfsys@useobject{currentmarker}{}%
\end{pgfscope}%
\end{pgfscope}%
\begin{pgfscope}%
\pgfsetbuttcap%
\pgfsetroundjoin%
\definecolor{currentfill}{rgb}{0.000000,0.000000,0.000000}%
\pgfsetfillcolor{currentfill}%
\pgfsetlinewidth{0.602250pt}%
\definecolor{currentstroke}{rgb}{0.000000,0.000000,0.000000}%
\pgfsetstrokecolor{currentstroke}%
\pgfsetdash{}{0pt}%
\pgfsys@defobject{currentmarker}{\pgfqpoint{-0.027778in}{0.000000in}}{\pgfqpoint{-0.000000in}{0.000000in}}{%
\pgfpathmoveto{\pgfqpoint{-0.000000in}{0.000000in}}%
\pgfpathlineto{\pgfqpoint{-0.027778in}{0.000000in}}%
\pgfusepath{stroke,fill}%
}%
\begin{pgfscope}%
\pgfsys@transformshift{0.588387in}{2.412040in}%
\pgfsys@useobject{currentmarker}{}%
\end{pgfscope}%
\end{pgfscope}%
\begin{pgfscope}%
\pgfsetbuttcap%
\pgfsetroundjoin%
\definecolor{currentfill}{rgb}{0.000000,0.000000,0.000000}%
\pgfsetfillcolor{currentfill}%
\pgfsetlinewidth{0.602250pt}%
\definecolor{currentstroke}{rgb}{0.000000,0.000000,0.000000}%
\pgfsetstrokecolor{currentstroke}%
\pgfsetdash{}{0pt}%
\pgfsys@defobject{currentmarker}{\pgfqpoint{-0.027778in}{0.000000in}}{\pgfqpoint{-0.000000in}{0.000000in}}{%
\pgfpathmoveto{\pgfqpoint{-0.000000in}{0.000000in}}%
\pgfpathlineto{\pgfqpoint{-0.027778in}{0.000000in}}%
\pgfusepath{stroke,fill}%
}%
\begin{pgfscope}%
\pgfsys@transformshift{0.588387in}{2.467662in}%
\pgfsys@useobject{currentmarker}{}%
\end{pgfscope}%
\end{pgfscope}%
\begin{pgfscope}%
\pgfsetbuttcap%
\pgfsetroundjoin%
\definecolor{currentfill}{rgb}{0.000000,0.000000,0.000000}%
\pgfsetfillcolor{currentfill}%
\pgfsetlinewidth{0.602250pt}%
\definecolor{currentstroke}{rgb}{0.000000,0.000000,0.000000}%
\pgfsetstrokecolor{currentstroke}%
\pgfsetdash{}{0pt}%
\pgfsys@defobject{currentmarker}{\pgfqpoint{-0.027778in}{0.000000in}}{\pgfqpoint{-0.000000in}{0.000000in}}{%
\pgfpathmoveto{\pgfqpoint{-0.000000in}{0.000000in}}%
\pgfpathlineto{\pgfqpoint{-0.027778in}{0.000000in}}%
\pgfusepath{stroke,fill}%
}%
\begin{pgfscope}%
\pgfsys@transformshift{0.588387in}{2.513109in}%
\pgfsys@useobject{currentmarker}{}%
\end{pgfscope}%
\end{pgfscope}%
\begin{pgfscope}%
\definecolor{textcolor}{rgb}{0.000000,0.000000,0.000000}%
\pgfsetstrokecolor{textcolor}%
\pgfsetfillcolor{textcolor}%
\pgftext[x=0.234413in,y=1.526746in,,bottom,rotate=90.000000]{\color{textcolor}{\rmfamily\fontsize{10.000000}{12.000000}\selectfont\catcode`\^=\active\def^{\ifmmode\sp\else\^{}\fi}\catcode`\%=\active\def%{\%}Time [ms]}}%
\end{pgfscope}%
\begin{pgfscope}%
\pgfpathrectangle{\pgfqpoint{0.588387in}{0.521603in}}{\pgfqpoint{5.399676in}{2.010285in}}%
\pgfusepath{clip}%
\pgfsetrectcap%
\pgfsetroundjoin%
\pgfsetlinewidth{1.505625pt}%
\pgfsetstrokecolor{currentstroke1}%
\pgfsetdash{}{0pt}%
\pgfpathmoveto{\pgfqpoint{0.833827in}{0.716431in}}%
\pgfpathlineto{\pgfqpoint{1.015634in}{0.725838in}}%
\pgfpathlineto{\pgfqpoint{1.197442in}{0.692648in}}%
\pgfpathlineto{\pgfqpoint{1.379249in}{0.640532in}}%
\pgfpathlineto{\pgfqpoint{1.561056in}{0.612980in}}%
\pgfpathlineto{\pgfqpoint{1.742863in}{0.637496in}}%
\pgfpathlineto{\pgfqpoint{1.924671in}{0.640467in}}%
\pgfpathlineto{\pgfqpoint{2.106478in}{0.709805in}}%
\pgfpathlineto{\pgfqpoint{2.288285in}{0.764263in}}%
\pgfpathlineto{\pgfqpoint{2.470092in}{0.832563in}}%
\pgfpathlineto{\pgfqpoint{2.651900in}{0.942326in}}%
\pgfpathlineto{\pgfqpoint{2.833707in}{1.029077in}}%
\pgfpathlineto{\pgfqpoint{3.015514in}{1.171758in}}%
\pgfpathlineto{\pgfqpoint{3.197321in}{1.244307in}}%
\pgfpathlineto{\pgfqpoint{3.379129in}{1.436395in}}%
\pgfpathlineto{\pgfqpoint{3.560936in}{1.526073in}}%
\pgfpathlineto{\pgfqpoint{3.742743in}{1.689696in}}%
\pgfpathlineto{\pgfqpoint{3.924551in}{1.799645in}}%
\pgfpathlineto{\pgfqpoint{4.106358in}{1.988190in}}%
\pgfpathlineto{\pgfqpoint{4.288165in}{2.115539in}}%
\pgfpathlineto{\pgfqpoint{4.469972in}{2.302938in}}%
\pgfpathlineto{\pgfqpoint{4.651780in}{2.436405in}}%
\pgfusepath{stroke}%
\end{pgfscope}%
\begin{pgfscope}%
\pgfpathrectangle{\pgfqpoint{0.588387in}{0.521603in}}{\pgfqpoint{5.399676in}{2.010285in}}%
\pgfusepath{clip}%
\pgfsetrectcap%
\pgfsetroundjoin%
\pgfsetlinewidth{1.505625pt}%
\pgfsetstrokecolor{currentstroke2}%
\pgfsetdash{}{0pt}%
\pgfpathmoveto{\pgfqpoint{0.833827in}{0.767479in}}%
\pgfpathlineto{\pgfqpoint{1.015634in}{0.754482in}}%
\pgfpathlineto{\pgfqpoint{1.197442in}{0.746220in}}%
\pgfpathlineto{\pgfqpoint{1.379249in}{0.675658in}}%
\pgfpathlineto{\pgfqpoint{1.561056in}{0.662138in}}%
\pgfpathlineto{\pgfqpoint{1.742863in}{0.682637in}}%
\pgfpathlineto{\pgfqpoint{1.924671in}{0.700276in}}%
\pgfpathlineto{\pgfqpoint{2.106478in}{0.768626in}}%
\pgfpathlineto{\pgfqpoint{2.288285in}{0.822901in}}%
\pgfpathlineto{\pgfqpoint{2.470092in}{0.876566in}}%
\pgfpathlineto{\pgfqpoint{2.651900in}{1.041880in}}%
\pgfpathlineto{\pgfqpoint{2.833707in}{1.084243in}}%
\pgfpathlineto{\pgfqpoint{3.015514in}{1.202232in}}%
\pgfpathlineto{\pgfqpoint{3.197321in}{1.228048in}}%
\pgfpathlineto{\pgfqpoint{3.379129in}{1.371193in}}%
\pgfpathlineto{\pgfqpoint{3.560936in}{1.375557in}}%
\pgfpathlineto{\pgfqpoint{3.742743in}{1.536113in}}%
\pgfpathlineto{\pgfqpoint{3.924551in}{1.536090in}}%
\pgfpathlineto{\pgfqpoint{4.106358in}{1.715774in}}%
\pgfpathlineto{\pgfqpoint{4.288165in}{1.726309in}}%
\pgfpathlineto{\pgfqpoint{4.469972in}{1.967197in}}%
\pgfpathlineto{\pgfqpoint{4.651780in}{1.851695in}}%
\pgfpathlineto{\pgfqpoint{4.833587in}{2.097703in}}%
\pgfpathlineto{\pgfqpoint{5.015394in}{1.999210in}}%
\pgfpathlineto{\pgfqpoint{5.197201in}{2.203879in}}%
\pgfpathlineto{\pgfqpoint{5.379009in}{2.123844in}}%
\pgfpathlineto{\pgfqpoint{5.560816in}{2.418863in}}%
\pgfpathlineto{\pgfqpoint{5.742623in}{2.310329in}}%
\pgfusepath{stroke}%
\end{pgfscope}%
\begin{pgfscope}%
\pgfpathrectangle{\pgfqpoint{0.588387in}{0.521603in}}{\pgfqpoint{5.399676in}{2.010285in}}%
\pgfusepath{clip}%
\pgfsetrectcap%
\pgfsetroundjoin%
\pgfsetlinewidth{1.505625pt}%
\pgfsetstrokecolor{currentstroke3}%
\pgfsetdash{}{0pt}%
\pgfpathmoveto{\pgfqpoint{0.833827in}{0.722586in}}%
\pgfpathlineto{\pgfqpoint{1.015634in}{0.753344in}}%
\pgfpathlineto{\pgfqpoint{1.197442in}{0.716181in}}%
\pgfpathlineto{\pgfqpoint{1.379249in}{0.666583in}}%
\pgfpathlineto{\pgfqpoint{1.561056in}{0.656883in}}%
\pgfpathlineto{\pgfqpoint{1.742863in}{0.684161in}}%
\pgfpathlineto{\pgfqpoint{1.924671in}{0.686303in}}%
\pgfpathlineto{\pgfqpoint{2.106478in}{0.750646in}}%
\pgfpathlineto{\pgfqpoint{2.288285in}{0.809115in}}%
\pgfpathlineto{\pgfqpoint{2.470092in}{0.863457in}}%
\pgfpathlineto{\pgfqpoint{2.651900in}{1.029296in}}%
\pgfpathlineto{\pgfqpoint{2.833707in}{1.096058in}}%
\pgfpathlineto{\pgfqpoint{3.015514in}{1.206720in}}%
\pgfpathlineto{\pgfqpoint{3.197321in}{1.223196in}}%
\pgfpathlineto{\pgfqpoint{3.379129in}{1.407369in}}%
\pgfpathlineto{\pgfqpoint{3.560936in}{1.426295in}}%
\pgfpathlineto{\pgfqpoint{3.742743in}{1.547069in}}%
\pgfpathlineto{\pgfqpoint{3.924551in}{1.553620in}}%
\pgfpathlineto{\pgfqpoint{4.106358in}{1.757863in}}%
\pgfpathlineto{\pgfqpoint{4.288165in}{1.741832in}}%
\pgfpathlineto{\pgfqpoint{4.469972in}{1.958720in}}%
\pgfpathlineto{\pgfqpoint{4.651780in}{1.877821in}}%
\pgfpathlineto{\pgfqpoint{4.833587in}{2.124833in}}%
\pgfpathlineto{\pgfqpoint{5.015394in}{2.013121in}}%
\pgfpathlineto{\pgfqpoint{5.197201in}{2.198996in}}%
\pgfpathlineto{\pgfqpoint{5.379009in}{2.098488in}}%
\pgfpathlineto{\pgfqpoint{5.560816in}{2.403611in}}%
\pgfpathlineto{\pgfqpoint{5.742623in}{2.271354in}}%
\pgfusepath{stroke}%
\end{pgfscope}%
\begin{pgfscope}%
\pgfpathrectangle{\pgfqpoint{0.588387in}{0.521603in}}{\pgfqpoint{5.399676in}{2.010285in}}%
\pgfusepath{clip}%
\pgfsetrectcap%
\pgfsetroundjoin%
\pgfsetlinewidth{1.505625pt}%
\pgfsetstrokecolor{currentstroke4}%
\pgfsetdash{}{0pt}%
\pgfpathmoveto{\pgfqpoint{0.833827in}{0.725744in}}%
\pgfpathlineto{\pgfqpoint{1.015634in}{0.751356in}}%
\pgfpathlineto{\pgfqpoint{1.197442in}{0.743632in}}%
\pgfpathlineto{\pgfqpoint{1.379249in}{0.674476in}}%
\pgfpathlineto{\pgfqpoint{1.561056in}{0.659987in}}%
\pgfpathlineto{\pgfqpoint{1.742863in}{0.688680in}}%
\pgfpathlineto{\pgfqpoint{1.924671in}{0.700882in}}%
\pgfpathlineto{\pgfqpoint{2.106478in}{0.766848in}}%
\pgfpathlineto{\pgfqpoint{2.288285in}{0.820782in}}%
\pgfpathlineto{\pgfqpoint{2.470092in}{0.873086in}}%
\pgfpathlineto{\pgfqpoint{2.651900in}{1.039428in}}%
\pgfpathlineto{\pgfqpoint{2.833707in}{1.089222in}}%
\pgfpathlineto{\pgfqpoint{3.015514in}{1.202340in}}%
\pgfpathlineto{\pgfqpoint{3.197321in}{1.229134in}}%
\pgfpathlineto{\pgfqpoint{3.379129in}{1.369169in}}%
\pgfpathlineto{\pgfqpoint{3.560936in}{1.365084in}}%
\pgfpathlineto{\pgfqpoint{3.742743in}{1.538329in}}%
\pgfpathlineto{\pgfqpoint{3.924551in}{1.549948in}}%
\pgfpathlineto{\pgfqpoint{4.106358in}{1.732399in}}%
\pgfpathlineto{\pgfqpoint{4.288165in}{1.772209in}}%
\pgfpathlineto{\pgfqpoint{4.651780in}{1.876605in}}%
\pgfpathlineto{\pgfqpoint{4.833587in}{2.112345in}}%
\pgfpathlineto{\pgfqpoint{5.015394in}{1.990020in}}%
\pgfpathlineto{\pgfqpoint{5.379009in}{2.228387in}}%
\pgfusepath{stroke}%
\end{pgfscope}%
\begin{pgfscope}%
\pgfpathrectangle{\pgfqpoint{0.588387in}{0.521603in}}{\pgfqpoint{5.399676in}{2.010285in}}%
\pgfusepath{clip}%
\pgfsetrectcap%
\pgfsetroundjoin%
\pgfsetlinewidth{1.505625pt}%
\pgfsetstrokecolor{currentstroke5}%
\pgfsetdash{}{0pt}%
\pgfpathmoveto{\pgfqpoint{0.833827in}{0.726969in}}%
\pgfpathlineto{\pgfqpoint{1.015634in}{0.748191in}}%
\pgfpathlineto{\pgfqpoint{1.197442in}{0.722376in}}%
\pgfpathlineto{\pgfqpoint{1.379249in}{0.676051in}}%
\pgfpathlineto{\pgfqpoint{1.561056in}{0.651835in}}%
\pgfpathlineto{\pgfqpoint{1.742863in}{0.676443in}}%
\pgfpathlineto{\pgfqpoint{1.924671in}{0.689536in}}%
\pgfpathlineto{\pgfqpoint{2.106478in}{0.758521in}}%
\pgfpathlineto{\pgfqpoint{2.288285in}{0.820689in}}%
\pgfpathlineto{\pgfqpoint{2.470092in}{0.872362in}}%
\pgfpathlineto{\pgfqpoint{2.651900in}{1.062007in}}%
\pgfpathlineto{\pgfqpoint{2.833707in}{1.121291in}}%
\pgfpathlineto{\pgfqpoint{3.015514in}{1.246234in}}%
\pgfpathlineto{\pgfqpoint{3.197321in}{1.285912in}}%
\pgfpathlineto{\pgfqpoint{3.379129in}{1.433074in}}%
\pgfpathlineto{\pgfqpoint{3.560936in}{1.484146in}}%
\pgfpathlineto{\pgfqpoint{3.742743in}{1.629246in}}%
\pgfpathlineto{\pgfqpoint{3.924551in}{1.670884in}}%
\pgfpathlineto{\pgfqpoint{4.106358in}{1.880472in}}%
\pgfpathlineto{\pgfqpoint{4.288165in}{1.878874in}}%
\pgfpathlineto{\pgfqpoint{4.469972in}{2.136790in}}%
\pgfpathlineto{\pgfqpoint{4.651780in}{2.064926in}}%
\pgfpathlineto{\pgfqpoint{4.833587in}{2.181134in}}%
\pgfpathlineto{\pgfqpoint{5.015394in}{2.232848in}}%
\pgfpathlineto{\pgfqpoint{5.197201in}{2.372383in}}%
\pgfpathlineto{\pgfqpoint{5.379009in}{2.262660in}}%
\pgfpathlineto{\pgfqpoint{5.742623in}{2.418114in}}%
\pgfusepath{stroke}%
\end{pgfscope}%
\begin{pgfscope}%
\pgfpathrectangle{\pgfqpoint{0.588387in}{0.521603in}}{\pgfqpoint{5.399676in}{2.010285in}}%
\pgfusepath{clip}%
\pgfsetrectcap%
\pgfsetroundjoin%
\pgfsetlinewidth{1.505625pt}%
\pgfsetstrokecolor{currentstroke6}%
\pgfsetdash{}{0pt}%
\pgfpathmoveto{\pgfqpoint{0.833827in}{0.716431in}}%
\pgfpathlineto{\pgfqpoint{1.015634in}{0.749596in}}%
\pgfpathlineto{\pgfqpoint{1.197442in}{0.713926in}}%
\pgfpathlineto{\pgfqpoint{1.379249in}{0.660809in}}%
\pgfpathlineto{\pgfqpoint{1.561056in}{0.640899in}}%
\pgfpathlineto{\pgfqpoint{1.742863in}{0.665221in}}%
\pgfpathlineto{\pgfqpoint{1.924671in}{0.691125in}}%
\pgfpathlineto{\pgfqpoint{2.106478in}{0.735785in}}%
\pgfpathlineto{\pgfqpoint{2.288285in}{0.796924in}}%
\pgfpathlineto{\pgfqpoint{2.470092in}{0.855164in}}%
\pgfpathlineto{\pgfqpoint{2.651900in}{1.049748in}}%
\pgfpathlineto{\pgfqpoint{2.833707in}{1.119244in}}%
\pgfpathlineto{\pgfqpoint{3.015514in}{1.244574in}}%
\pgfpathlineto{\pgfqpoint{3.197321in}{1.292930in}}%
\pgfpathlineto{\pgfqpoint{3.379129in}{1.451956in}}%
\pgfpathlineto{\pgfqpoint{3.560936in}{1.511073in}}%
\pgfpathlineto{\pgfqpoint{3.742743in}{1.610677in}}%
\pgfpathlineto{\pgfqpoint{3.924551in}{1.648766in}}%
\pgfpathlineto{\pgfqpoint{4.106358in}{1.869553in}}%
\pgfpathlineto{\pgfqpoint{4.288165in}{1.834586in}}%
\pgfpathlineto{\pgfqpoint{4.469972in}{2.015996in}}%
\pgfpathlineto{\pgfqpoint{4.651780in}{2.042555in}}%
\pgfpathlineto{\pgfqpoint{4.833587in}{2.156075in}}%
\pgfpathlineto{\pgfqpoint{5.015394in}{2.179838in}}%
\pgfpathlineto{\pgfqpoint{5.197201in}{2.407787in}}%
\pgfpathlineto{\pgfqpoint{5.379009in}{2.222287in}}%
\pgfpathlineto{\pgfqpoint{5.560816in}{2.440512in}}%
\pgfpathlineto{\pgfqpoint{5.742623in}{2.413035in}}%
\pgfusepath{stroke}%
\end{pgfscope}%
\begin{pgfscope}%
\pgfsetrectcap%
\pgfsetmiterjoin%
\pgfsetlinewidth{0.803000pt}%
\definecolor{currentstroke}{rgb}{0.000000,0.000000,0.000000}%
\pgfsetstrokecolor{currentstroke}%
\pgfsetdash{}{0pt}%
\pgfpathmoveto{\pgfqpoint{0.588387in}{0.521603in}}%
\pgfpathlineto{\pgfqpoint{0.588387in}{2.531888in}}%
\pgfusepath{stroke}%
\end{pgfscope}%
\begin{pgfscope}%
\pgfsetrectcap%
\pgfsetmiterjoin%
\pgfsetlinewidth{0.803000pt}%
\definecolor{currentstroke}{rgb}{0.000000,0.000000,0.000000}%
\pgfsetstrokecolor{currentstroke}%
\pgfsetdash{}{0pt}%
\pgfpathmoveto{\pgfqpoint{5.988063in}{0.521603in}}%
\pgfpathlineto{\pgfqpoint{5.988063in}{2.531888in}}%
\pgfusepath{stroke}%
\end{pgfscope}%
\begin{pgfscope}%
\pgfsetrectcap%
\pgfsetmiterjoin%
\pgfsetlinewidth{0.803000pt}%
\definecolor{currentstroke}{rgb}{0.000000,0.000000,0.000000}%
\pgfsetstrokecolor{currentstroke}%
\pgfsetdash{}{0pt}%
\pgfpathmoveto{\pgfqpoint{0.588387in}{0.521603in}}%
\pgfpathlineto{\pgfqpoint{5.988063in}{0.521603in}}%
\pgfusepath{stroke}%
\end{pgfscope}%
\begin{pgfscope}%
\pgfsetrectcap%
\pgfsetmiterjoin%
\pgfsetlinewidth{0.803000pt}%
\definecolor{currentstroke}{rgb}{0.000000,0.000000,0.000000}%
\pgfsetstrokecolor{currentstroke}%
\pgfsetdash{}{0pt}%
\pgfpathmoveto{\pgfqpoint{0.588387in}{2.531888in}}%
\pgfpathlineto{\pgfqpoint{5.988063in}{2.531888in}}%
\pgfusepath{stroke}%
\end{pgfscope}%
\begin{pgfscope}%
\definecolor{textcolor}{rgb}{0.000000,0.000000,0.000000}%
\pgfsetstrokecolor{textcolor}%
\pgfsetfillcolor{textcolor}%
\pgftext[x=3.288225in,y=2.615222in,,base]{\color{textcolor}{\rmfamily\fontsize{12.000000}{14.400000}\selectfont\catcode`\^=\active\def^{\ifmmode\sp\else\^{}\fi}\catcode`\%=\active\def%{\%}Mean}}%
\end{pgfscope}%
\begin{pgfscope}%
\pgfsetbuttcap%
\pgfsetmiterjoin%
\definecolor{currentfill}{rgb}{1.000000,1.000000,1.000000}%
\pgfsetfillcolor{currentfill}%
\pgfsetfillopacity{0.800000}%
\pgfsetlinewidth{1.003750pt}%
\definecolor{currentstroke}{rgb}{0.800000,0.800000,0.800000}%
\pgfsetstrokecolor{currentstroke}%
\pgfsetstrokeopacity{0.800000}%
\pgfsetdash{}{0pt}%
\pgfpathmoveto{\pgfqpoint{6.075563in}{1.320622in}}%
\pgfpathlineto{\pgfqpoint{8.259376in}{1.320622in}}%
\pgfpathquadraticcurveto{\pgfqpoint{8.284376in}{1.320622in}}{\pgfqpoint{8.284376in}{1.345622in}}%
\pgfpathlineto{\pgfqpoint{8.284376in}{2.444388in}}%
\pgfpathquadraticcurveto{\pgfqpoint{8.284376in}{2.469388in}}{\pgfqpoint{8.259376in}{2.469388in}}%
\pgfpathlineto{\pgfqpoint{6.075563in}{2.469388in}}%
\pgfpathquadraticcurveto{\pgfqpoint{6.050563in}{2.469388in}}{\pgfqpoint{6.050563in}{2.444388in}}%
\pgfpathlineto{\pgfqpoint{6.050563in}{1.345622in}}%
\pgfpathquadraticcurveto{\pgfqpoint{6.050563in}{1.320622in}}{\pgfqpoint{6.075563in}{1.320622in}}%
\pgfpathlineto{\pgfqpoint{6.075563in}{1.320622in}}%
\pgfpathclose%
\pgfusepath{stroke,fill}%
\end{pgfscope}%
\begin{pgfscope}%
\pgfsetrectcap%
\pgfsetroundjoin%
\pgfsetlinewidth{1.505625pt}%
\pgfsetstrokecolor{currentstroke1}%
\pgfsetdash{}{0pt}%
\pgfpathmoveto{\pgfqpoint{6.100563in}{2.368168in}}%
\pgfpathlineto{\pgfqpoint{6.225563in}{2.368168in}}%
\pgfpathlineto{\pgfqpoint{6.350563in}{2.368168in}}%
\pgfusepath{stroke}%
\end{pgfscope}%
\begin{pgfscope}%
\definecolor{textcolor}{rgb}{0.000000,0.000000,0.000000}%
\pgfsetstrokecolor{textcolor}%
\pgfsetfillcolor{textcolor}%
\pgftext[x=6.450563in,y=2.324418in,left,base]{\color{textcolor}{\rmfamily\fontsize{9.000000}{10.800000}\selectfont\catcode`\^=\active\def^{\ifmmode\sp\else\^{}\fi}\catcode`\%=\active\def%{\%}\NaiveCycles{}}}%
\end{pgfscope}%
\begin{pgfscope}%
\pgfsetrectcap%
\pgfsetroundjoin%
\pgfsetlinewidth{1.505625pt}%
\pgfsetstrokecolor{currentstroke2}%
\pgfsetdash{}{0pt}%
\pgfpathmoveto{\pgfqpoint{6.100563in}{2.184696in}}%
\pgfpathlineto{\pgfqpoint{6.225563in}{2.184696in}}%
\pgfpathlineto{\pgfqpoint{6.350563in}{2.184696in}}%
\pgfusepath{stroke}%
\end{pgfscope}%
\begin{pgfscope}%
\definecolor{textcolor}{rgb}{0.000000,0.000000,0.000000}%
\pgfsetstrokecolor{textcolor}%
\pgfsetfillcolor{textcolor}%
\pgftext[x=6.450563in,y=2.140946in,left,base]{\color{textcolor}{\rmfamily\fontsize{9.000000}{10.800000}\selectfont\catcode`\^=\active\def^{\ifmmode\sp\else\^{}\fi}\catcode`\%=\active\def%{\%}\Neighbors{} \& \MergeLinear{}}}%
\end{pgfscope}%
\begin{pgfscope}%
\pgfsetrectcap%
\pgfsetroundjoin%
\pgfsetlinewidth{1.505625pt}%
\pgfsetstrokecolor{currentstroke3}%
\pgfsetdash{}{0pt}%
\pgfpathmoveto{\pgfqpoint{6.100563in}{2.001225in}}%
\pgfpathlineto{\pgfqpoint{6.225563in}{2.001225in}}%
\pgfpathlineto{\pgfqpoint{6.350563in}{2.001225in}}%
\pgfusepath{stroke}%
\end{pgfscope}%
\begin{pgfscope}%
\definecolor{textcolor}{rgb}{0.000000,0.000000,0.000000}%
\pgfsetstrokecolor{textcolor}%
\pgfsetfillcolor{textcolor}%
\pgftext[x=6.450563in,y=1.957475in,left,base]{\color{textcolor}{\rmfamily\fontsize{9.000000}{10.800000}\selectfont\catcode`\^=\active\def^{\ifmmode\sp\else\^{}\fi}\catcode`\%=\active\def%{\%}\Neighbors{} \& \SharedVertices{}}}%
\end{pgfscope}%
\begin{pgfscope}%
\pgfsetrectcap%
\pgfsetroundjoin%
\pgfsetlinewidth{1.505625pt}%
\pgfsetstrokecolor{currentstroke4}%
\pgfsetdash{}{0pt}%
\pgfpathmoveto{\pgfqpoint{6.100563in}{1.814274in}}%
\pgfpathlineto{\pgfqpoint{6.225563in}{1.814274in}}%
\pgfpathlineto{\pgfqpoint{6.350563in}{1.814274in}}%
\pgfusepath{stroke}%
\end{pgfscope}%
\begin{pgfscope}%
\definecolor{textcolor}{rgb}{0.000000,0.000000,0.000000}%
\pgfsetstrokecolor{textcolor}%
\pgfsetfillcolor{textcolor}%
\pgftext[x=6.450563in,y=1.770524in,left,base]{\color{textcolor}{\rmfamily\fontsize{9.000000}{10.800000}\selectfont\catcode`\^=\active\def^{\ifmmode\sp\else\^{}\fi}\catcode`\%=\active\def%{\%}\NeighborsDegree{} \& \MergeLinear{}}}%
\end{pgfscope}%
\begin{pgfscope}%
\pgfsetrectcap%
\pgfsetroundjoin%
\pgfsetlinewidth{1.505625pt}%
\pgfsetstrokecolor{currentstroke5}%
\pgfsetdash{}{0pt}%
\pgfpathmoveto{\pgfqpoint{6.100563in}{1.627324in}}%
\pgfpathlineto{\pgfqpoint{6.225563in}{1.627324in}}%
\pgfpathlineto{\pgfqpoint{6.350563in}{1.627324in}}%
\pgfusepath{stroke}%
\end{pgfscope}%
\begin{pgfscope}%
\definecolor{textcolor}{rgb}{0.000000,0.000000,0.000000}%
\pgfsetstrokecolor{textcolor}%
\pgfsetfillcolor{textcolor}%
\pgftext[x=6.450563in,y=1.583574in,left,base]{\color{textcolor}{\rmfamily\fontsize{9.000000}{10.800000}\selectfont\catcode`\^=\active\def^{\ifmmode\sp\else\^{}\fi}\catcode`\%=\active\def%{\%}\None{} \& \MergeLinear{}}}%
\end{pgfscope}%
\begin{pgfscope}%
\pgfsetrectcap%
\pgfsetroundjoin%
\pgfsetlinewidth{1.505625pt}%
\pgfsetstrokecolor{currentstroke6}%
\pgfsetdash{}{0pt}%
\pgfpathmoveto{\pgfqpoint{6.100563in}{1.443852in}}%
\pgfpathlineto{\pgfqpoint{6.225563in}{1.443852in}}%
\pgfpathlineto{\pgfqpoint{6.350563in}{1.443852in}}%
\pgfusepath{stroke}%
\end{pgfscope}%
\begin{pgfscope}%
\definecolor{textcolor}{rgb}{0.000000,0.000000,0.000000}%
\pgfsetstrokecolor{textcolor}%
\pgfsetfillcolor{textcolor}%
\pgftext[x=6.450563in,y=1.400102in,left,base]{\color{textcolor}{\rmfamily\fontsize{9.000000}{10.800000}\selectfont\catcode`\^=\active\def^{\ifmmode\sp\else\^{}\fi}\catcode`\%=\active\def%{\%}\None{} \& \SharedVertices{}}}%
\end{pgfscope}%
\end{pgfpicture}%
\makeatother%
\endgroup%
}
	\caption[Running time for minimally rigid graphs.]{
		Mean running time (ms) to find all NAC-colorings for minimally rigid graphs.}%
	\label{fig:graph_time_minimally_rigid}
\end{figure}

If we analyze the number of \IsNACColoring{} calls performed by \NaiveCycles{} and \Subgraphs{} algorithms
as shown in \Cref{fig:graph_count_minimally_rigid},
you can see that the number of \IsNACColoring{} calls is reduced already for graphs
with eleven vertices,
even though the \NaiveCycles{} algorithm is still faster for these graphs.

\begin{figure}[ht]
	\centering
	\scalebox{0.5}{%% Creator: Matplotlib, PGF backend
%%
%% To include the figure in your LaTeX document, write
%%   \input{<filename>.pgf}
%%
%% Make sure the required packages are loaded in your preamble
%%   \usepackage{pgf}
%%
%% Also ensure that all the required font packages are loaded; for instance,
%% the lmodern package is sometimes necessary when using math font.
%%   \usepackage{lmodern}
%%
%% Figures using additional raster images can only be included by \input if
%% they are in the same directory as the main LaTeX file. For loading figures
%% from other directories you can use the `import` package
%%   \usepackage{import}
%%
%% and then include the figures with
%%   \import{<path to file>}{<filename>.pgf}
%%
%% Matplotlib used the following preamble
%%   \def\mathdefault#1{#1}
%%   \everymath=\expandafter{\the\everymath\displaystyle}
%%   \IfFileExists{scrextend.sty}{
%%     \usepackage[fontsize=10.000000pt]{scrextend}
%%   }{
%%     \renewcommand{\normalsize}{\fontsize{10.000000}{12.000000}\selectfont}
%%     \normalsize
%%   }
%%   
%%   \ifdefined\pdftexversion\else  % non-pdftex case.
%%     \usepackage{fontspec}
%%     \setmainfont{DejaVuSans.ttf}[Path=\detokenize{/home/petr/Projects/PyRigi/.venv/lib/python3.12/site-packages/matplotlib/mpl-data/fonts/ttf/}]
%%     \setsansfont{DejaVuSans.ttf}[Path=\detokenize{/home/petr/Projects/PyRigi/.venv/lib/python3.12/site-packages/matplotlib/mpl-data/fonts/ttf/}]
%%     \setmonofont{DejaVuSansMono.ttf}[Path=\detokenize{/home/petr/Projects/PyRigi/.venv/lib/python3.12/site-packages/matplotlib/mpl-data/fonts/ttf/}]
%%   \fi
%%   \makeatletter\@ifpackageloaded{underscore}{}{\usepackage[strings]{underscore}}\makeatother
%%
\begingroup%
\makeatletter%
\begin{pgfpicture}%
\pgfpathrectangle{\pgfpointorigin}{\pgfqpoint{8.384376in}{2.841849in}}%
\pgfusepath{use as bounding box, clip}%
\begin{pgfscope}%
\pgfsetbuttcap%
\pgfsetmiterjoin%
\definecolor{currentfill}{rgb}{1.000000,1.000000,1.000000}%
\pgfsetfillcolor{currentfill}%
\pgfsetlinewidth{0.000000pt}%
\definecolor{currentstroke}{rgb}{1.000000,1.000000,1.000000}%
\pgfsetstrokecolor{currentstroke}%
\pgfsetdash{}{0pt}%
\pgfpathmoveto{\pgfqpoint{0.000000in}{0.000000in}}%
\pgfpathlineto{\pgfqpoint{8.384376in}{0.000000in}}%
\pgfpathlineto{\pgfqpoint{8.384376in}{2.841849in}}%
\pgfpathlineto{\pgfqpoint{0.000000in}{2.841849in}}%
\pgfpathlineto{\pgfqpoint{0.000000in}{0.000000in}}%
\pgfpathclose%
\pgfusepath{fill}%
\end{pgfscope}%
\begin{pgfscope}%
\pgfsetbuttcap%
\pgfsetmiterjoin%
\definecolor{currentfill}{rgb}{1.000000,1.000000,1.000000}%
\pgfsetfillcolor{currentfill}%
\pgfsetlinewidth{0.000000pt}%
\definecolor{currentstroke}{rgb}{0.000000,0.000000,0.000000}%
\pgfsetstrokecolor{currentstroke}%
\pgfsetstrokeopacity{0.000000}%
\pgfsetdash{}{0pt}%
\pgfpathmoveto{\pgfqpoint{0.588387in}{0.521603in}}%
\pgfpathlineto{\pgfqpoint{5.988063in}{0.521603in}}%
\pgfpathlineto{\pgfqpoint{5.988063in}{2.531888in}}%
\pgfpathlineto{\pgfqpoint{0.588387in}{2.531888in}}%
\pgfpathlineto{\pgfqpoint{0.588387in}{0.521603in}}%
\pgfpathclose%
\pgfusepath{fill}%
\end{pgfscope}%
\begin{pgfscope}%
\pgfsetbuttcap%
\pgfsetroundjoin%
\definecolor{currentfill}{rgb}{0.000000,0.000000,0.000000}%
\pgfsetfillcolor{currentfill}%
\pgfsetlinewidth{0.803000pt}%
\definecolor{currentstroke}{rgb}{0.000000,0.000000,0.000000}%
\pgfsetstrokecolor{currentstroke}%
\pgfsetdash{}{0pt}%
\pgfsys@defobject{currentmarker}{\pgfqpoint{0.000000in}{-0.048611in}}{\pgfqpoint{0.000000in}{0.000000in}}{%
\pgfpathmoveto{\pgfqpoint{0.000000in}{0.000000in}}%
\pgfpathlineto{\pgfqpoint{0.000000in}{-0.048611in}}%
\pgfusepath{stroke,fill}%
}%
\begin{pgfscope}%
\pgfsys@transformshift{1.015634in}{0.521603in}%
\pgfsys@useobject{currentmarker}{}%
\end{pgfscope}%
\end{pgfscope}%
\begin{pgfscope}%
\definecolor{textcolor}{rgb}{0.000000,0.000000,0.000000}%
\pgfsetstrokecolor{textcolor}%
\pgfsetfillcolor{textcolor}%
\pgftext[x=1.015634in,y=0.424381in,,top]{\color{textcolor}{\rmfamily\fontsize{10.000000}{12.000000}\selectfont\catcode`\^=\active\def^{\ifmmode\sp\else\^{}\fi}\catcode`\%=\active\def%{\%}$\mathdefault{3}$}}%
\end{pgfscope}%
\begin{pgfscope}%
\pgfsetbuttcap%
\pgfsetroundjoin%
\definecolor{currentfill}{rgb}{0.000000,0.000000,0.000000}%
\pgfsetfillcolor{currentfill}%
\pgfsetlinewidth{0.803000pt}%
\definecolor{currentstroke}{rgb}{0.000000,0.000000,0.000000}%
\pgfsetstrokecolor{currentstroke}%
\pgfsetdash{}{0pt}%
\pgfsys@defobject{currentmarker}{\pgfqpoint{0.000000in}{-0.048611in}}{\pgfqpoint{0.000000in}{0.000000in}}{%
\pgfpathmoveto{\pgfqpoint{0.000000in}{0.000000in}}%
\pgfpathlineto{\pgfqpoint{0.000000in}{-0.048611in}}%
\pgfusepath{stroke,fill}%
}%
\begin{pgfscope}%
\pgfsys@transformshift{1.561056in}{0.521603in}%
\pgfsys@useobject{currentmarker}{}%
\end{pgfscope}%
\end{pgfscope}%
\begin{pgfscope}%
\definecolor{textcolor}{rgb}{0.000000,0.000000,0.000000}%
\pgfsetstrokecolor{textcolor}%
\pgfsetfillcolor{textcolor}%
\pgftext[x=1.561056in,y=0.424381in,,top]{\color{textcolor}{\rmfamily\fontsize{10.000000}{12.000000}\selectfont\catcode`\^=\active\def^{\ifmmode\sp\else\^{}\fi}\catcode`\%=\active\def%{\%}$\mathdefault{6}$}}%
\end{pgfscope}%
\begin{pgfscope}%
\pgfsetbuttcap%
\pgfsetroundjoin%
\definecolor{currentfill}{rgb}{0.000000,0.000000,0.000000}%
\pgfsetfillcolor{currentfill}%
\pgfsetlinewidth{0.803000pt}%
\definecolor{currentstroke}{rgb}{0.000000,0.000000,0.000000}%
\pgfsetstrokecolor{currentstroke}%
\pgfsetdash{}{0pt}%
\pgfsys@defobject{currentmarker}{\pgfqpoint{0.000000in}{-0.048611in}}{\pgfqpoint{0.000000in}{0.000000in}}{%
\pgfpathmoveto{\pgfqpoint{0.000000in}{0.000000in}}%
\pgfpathlineto{\pgfqpoint{0.000000in}{-0.048611in}}%
\pgfusepath{stroke,fill}%
}%
\begin{pgfscope}%
\pgfsys@transformshift{2.106478in}{0.521603in}%
\pgfsys@useobject{currentmarker}{}%
\end{pgfscope}%
\end{pgfscope}%
\begin{pgfscope}%
\definecolor{textcolor}{rgb}{0.000000,0.000000,0.000000}%
\pgfsetstrokecolor{textcolor}%
\pgfsetfillcolor{textcolor}%
\pgftext[x=2.106478in,y=0.424381in,,top]{\color{textcolor}{\rmfamily\fontsize{10.000000}{12.000000}\selectfont\catcode`\^=\active\def^{\ifmmode\sp\else\^{}\fi}\catcode`\%=\active\def%{\%}$\mathdefault{9}$}}%
\end{pgfscope}%
\begin{pgfscope}%
\pgfsetbuttcap%
\pgfsetroundjoin%
\definecolor{currentfill}{rgb}{0.000000,0.000000,0.000000}%
\pgfsetfillcolor{currentfill}%
\pgfsetlinewidth{0.803000pt}%
\definecolor{currentstroke}{rgb}{0.000000,0.000000,0.000000}%
\pgfsetstrokecolor{currentstroke}%
\pgfsetdash{}{0pt}%
\pgfsys@defobject{currentmarker}{\pgfqpoint{0.000000in}{-0.048611in}}{\pgfqpoint{0.000000in}{0.000000in}}{%
\pgfpathmoveto{\pgfqpoint{0.000000in}{0.000000in}}%
\pgfpathlineto{\pgfqpoint{0.000000in}{-0.048611in}}%
\pgfusepath{stroke,fill}%
}%
\begin{pgfscope}%
\pgfsys@transformshift{2.651900in}{0.521603in}%
\pgfsys@useobject{currentmarker}{}%
\end{pgfscope}%
\end{pgfscope}%
\begin{pgfscope}%
\definecolor{textcolor}{rgb}{0.000000,0.000000,0.000000}%
\pgfsetstrokecolor{textcolor}%
\pgfsetfillcolor{textcolor}%
\pgftext[x=2.651900in,y=0.424381in,,top]{\color{textcolor}{\rmfamily\fontsize{10.000000}{12.000000}\selectfont\catcode`\^=\active\def^{\ifmmode\sp\else\^{}\fi}\catcode`\%=\active\def%{\%}$\mathdefault{12}$}}%
\end{pgfscope}%
\begin{pgfscope}%
\pgfsetbuttcap%
\pgfsetroundjoin%
\definecolor{currentfill}{rgb}{0.000000,0.000000,0.000000}%
\pgfsetfillcolor{currentfill}%
\pgfsetlinewidth{0.803000pt}%
\definecolor{currentstroke}{rgb}{0.000000,0.000000,0.000000}%
\pgfsetstrokecolor{currentstroke}%
\pgfsetdash{}{0pt}%
\pgfsys@defobject{currentmarker}{\pgfqpoint{0.000000in}{-0.048611in}}{\pgfqpoint{0.000000in}{0.000000in}}{%
\pgfpathmoveto{\pgfqpoint{0.000000in}{0.000000in}}%
\pgfpathlineto{\pgfqpoint{0.000000in}{-0.048611in}}%
\pgfusepath{stroke,fill}%
}%
\begin{pgfscope}%
\pgfsys@transformshift{3.197321in}{0.521603in}%
\pgfsys@useobject{currentmarker}{}%
\end{pgfscope}%
\end{pgfscope}%
\begin{pgfscope}%
\definecolor{textcolor}{rgb}{0.000000,0.000000,0.000000}%
\pgfsetstrokecolor{textcolor}%
\pgfsetfillcolor{textcolor}%
\pgftext[x=3.197321in,y=0.424381in,,top]{\color{textcolor}{\rmfamily\fontsize{10.000000}{12.000000}\selectfont\catcode`\^=\active\def^{\ifmmode\sp\else\^{}\fi}\catcode`\%=\active\def%{\%}$\mathdefault{15}$}}%
\end{pgfscope}%
\begin{pgfscope}%
\pgfsetbuttcap%
\pgfsetroundjoin%
\definecolor{currentfill}{rgb}{0.000000,0.000000,0.000000}%
\pgfsetfillcolor{currentfill}%
\pgfsetlinewidth{0.803000pt}%
\definecolor{currentstroke}{rgb}{0.000000,0.000000,0.000000}%
\pgfsetstrokecolor{currentstroke}%
\pgfsetdash{}{0pt}%
\pgfsys@defobject{currentmarker}{\pgfqpoint{0.000000in}{-0.048611in}}{\pgfqpoint{0.000000in}{0.000000in}}{%
\pgfpathmoveto{\pgfqpoint{0.000000in}{0.000000in}}%
\pgfpathlineto{\pgfqpoint{0.000000in}{-0.048611in}}%
\pgfusepath{stroke,fill}%
}%
\begin{pgfscope}%
\pgfsys@transformshift{3.742743in}{0.521603in}%
\pgfsys@useobject{currentmarker}{}%
\end{pgfscope}%
\end{pgfscope}%
\begin{pgfscope}%
\definecolor{textcolor}{rgb}{0.000000,0.000000,0.000000}%
\pgfsetstrokecolor{textcolor}%
\pgfsetfillcolor{textcolor}%
\pgftext[x=3.742743in,y=0.424381in,,top]{\color{textcolor}{\rmfamily\fontsize{10.000000}{12.000000}\selectfont\catcode`\^=\active\def^{\ifmmode\sp\else\^{}\fi}\catcode`\%=\active\def%{\%}$\mathdefault{18}$}}%
\end{pgfscope}%
\begin{pgfscope}%
\pgfsetbuttcap%
\pgfsetroundjoin%
\definecolor{currentfill}{rgb}{0.000000,0.000000,0.000000}%
\pgfsetfillcolor{currentfill}%
\pgfsetlinewidth{0.803000pt}%
\definecolor{currentstroke}{rgb}{0.000000,0.000000,0.000000}%
\pgfsetstrokecolor{currentstroke}%
\pgfsetdash{}{0pt}%
\pgfsys@defobject{currentmarker}{\pgfqpoint{0.000000in}{-0.048611in}}{\pgfqpoint{0.000000in}{0.000000in}}{%
\pgfpathmoveto{\pgfqpoint{0.000000in}{0.000000in}}%
\pgfpathlineto{\pgfqpoint{0.000000in}{-0.048611in}}%
\pgfusepath{stroke,fill}%
}%
\begin{pgfscope}%
\pgfsys@transformshift{4.288165in}{0.521603in}%
\pgfsys@useobject{currentmarker}{}%
\end{pgfscope}%
\end{pgfscope}%
\begin{pgfscope}%
\definecolor{textcolor}{rgb}{0.000000,0.000000,0.000000}%
\pgfsetstrokecolor{textcolor}%
\pgfsetfillcolor{textcolor}%
\pgftext[x=4.288165in,y=0.424381in,,top]{\color{textcolor}{\rmfamily\fontsize{10.000000}{12.000000}\selectfont\catcode`\^=\active\def^{\ifmmode\sp\else\^{}\fi}\catcode`\%=\active\def%{\%}$\mathdefault{21}$}}%
\end{pgfscope}%
\begin{pgfscope}%
\pgfsetbuttcap%
\pgfsetroundjoin%
\definecolor{currentfill}{rgb}{0.000000,0.000000,0.000000}%
\pgfsetfillcolor{currentfill}%
\pgfsetlinewidth{0.803000pt}%
\definecolor{currentstroke}{rgb}{0.000000,0.000000,0.000000}%
\pgfsetstrokecolor{currentstroke}%
\pgfsetdash{}{0pt}%
\pgfsys@defobject{currentmarker}{\pgfqpoint{0.000000in}{-0.048611in}}{\pgfqpoint{0.000000in}{0.000000in}}{%
\pgfpathmoveto{\pgfqpoint{0.000000in}{0.000000in}}%
\pgfpathlineto{\pgfqpoint{0.000000in}{-0.048611in}}%
\pgfusepath{stroke,fill}%
}%
\begin{pgfscope}%
\pgfsys@transformshift{4.833587in}{0.521603in}%
\pgfsys@useobject{currentmarker}{}%
\end{pgfscope}%
\end{pgfscope}%
\begin{pgfscope}%
\definecolor{textcolor}{rgb}{0.000000,0.000000,0.000000}%
\pgfsetstrokecolor{textcolor}%
\pgfsetfillcolor{textcolor}%
\pgftext[x=4.833587in,y=0.424381in,,top]{\color{textcolor}{\rmfamily\fontsize{10.000000}{12.000000}\selectfont\catcode`\^=\active\def^{\ifmmode\sp\else\^{}\fi}\catcode`\%=\active\def%{\%}$\mathdefault{24}$}}%
\end{pgfscope}%
\begin{pgfscope}%
\pgfsetbuttcap%
\pgfsetroundjoin%
\definecolor{currentfill}{rgb}{0.000000,0.000000,0.000000}%
\pgfsetfillcolor{currentfill}%
\pgfsetlinewidth{0.803000pt}%
\definecolor{currentstroke}{rgb}{0.000000,0.000000,0.000000}%
\pgfsetstrokecolor{currentstroke}%
\pgfsetdash{}{0pt}%
\pgfsys@defobject{currentmarker}{\pgfqpoint{0.000000in}{-0.048611in}}{\pgfqpoint{0.000000in}{0.000000in}}{%
\pgfpathmoveto{\pgfqpoint{0.000000in}{0.000000in}}%
\pgfpathlineto{\pgfqpoint{0.000000in}{-0.048611in}}%
\pgfusepath{stroke,fill}%
}%
\begin{pgfscope}%
\pgfsys@transformshift{5.379009in}{0.521603in}%
\pgfsys@useobject{currentmarker}{}%
\end{pgfscope}%
\end{pgfscope}%
\begin{pgfscope}%
\definecolor{textcolor}{rgb}{0.000000,0.000000,0.000000}%
\pgfsetstrokecolor{textcolor}%
\pgfsetfillcolor{textcolor}%
\pgftext[x=5.379009in,y=0.424381in,,top]{\color{textcolor}{\rmfamily\fontsize{10.000000}{12.000000}\selectfont\catcode`\^=\active\def^{\ifmmode\sp\else\^{}\fi}\catcode`\%=\active\def%{\%}$\mathdefault{27}$}}%
\end{pgfscope}%
\begin{pgfscope}%
\pgfsetbuttcap%
\pgfsetroundjoin%
\definecolor{currentfill}{rgb}{0.000000,0.000000,0.000000}%
\pgfsetfillcolor{currentfill}%
\pgfsetlinewidth{0.803000pt}%
\definecolor{currentstroke}{rgb}{0.000000,0.000000,0.000000}%
\pgfsetstrokecolor{currentstroke}%
\pgfsetdash{}{0pt}%
\pgfsys@defobject{currentmarker}{\pgfqpoint{0.000000in}{-0.048611in}}{\pgfqpoint{0.000000in}{0.000000in}}{%
\pgfpathmoveto{\pgfqpoint{0.000000in}{0.000000in}}%
\pgfpathlineto{\pgfqpoint{0.000000in}{-0.048611in}}%
\pgfusepath{stroke,fill}%
}%
\begin{pgfscope}%
\pgfsys@transformshift{5.924430in}{0.521603in}%
\pgfsys@useobject{currentmarker}{}%
\end{pgfscope}%
\end{pgfscope}%
\begin{pgfscope}%
\definecolor{textcolor}{rgb}{0.000000,0.000000,0.000000}%
\pgfsetstrokecolor{textcolor}%
\pgfsetfillcolor{textcolor}%
\pgftext[x=5.924430in,y=0.424381in,,top]{\color{textcolor}{\rmfamily\fontsize{10.000000}{12.000000}\selectfont\catcode`\^=\active\def^{\ifmmode\sp\else\^{}\fi}\catcode`\%=\active\def%{\%}$\mathdefault{30}$}}%
\end{pgfscope}%
\begin{pgfscope}%
\definecolor{textcolor}{rgb}{0.000000,0.000000,0.000000}%
\pgfsetstrokecolor{textcolor}%
\pgfsetfillcolor{textcolor}%
\pgftext[x=3.288225in,y=0.234413in,,top]{\color{textcolor}{\rmfamily\fontsize{10.000000}{12.000000}\selectfont\catcode`\^=\active\def^{\ifmmode\sp\else\^{}\fi}\catcode`\%=\active\def%{\%}Monochromatic classes}}%
\end{pgfscope}%
\begin{pgfscope}%
\pgfsetbuttcap%
\pgfsetroundjoin%
\definecolor{currentfill}{rgb}{0.000000,0.000000,0.000000}%
\pgfsetfillcolor{currentfill}%
\pgfsetlinewidth{0.803000pt}%
\definecolor{currentstroke}{rgb}{0.000000,0.000000,0.000000}%
\pgfsetstrokecolor{currentstroke}%
\pgfsetdash{}{0pt}%
\pgfsys@defobject{currentmarker}{\pgfqpoint{-0.048611in}{0.000000in}}{\pgfqpoint{-0.000000in}{0.000000in}}{%
\pgfpathmoveto{\pgfqpoint{-0.000000in}{0.000000in}}%
\pgfpathlineto{\pgfqpoint{-0.048611in}{0.000000in}}%
\pgfusepath{stroke,fill}%
}%
\begin{pgfscope}%
\pgfsys@transformshift{0.588387in}{0.612980in}%
\pgfsys@useobject{currentmarker}{}%
\end{pgfscope}%
\end{pgfscope}%
\begin{pgfscope}%
\definecolor{textcolor}{rgb}{0.000000,0.000000,0.000000}%
\pgfsetstrokecolor{textcolor}%
\pgfsetfillcolor{textcolor}%
\pgftext[x=0.289968in, y=0.560218in, left, base]{\color{textcolor}{\rmfamily\fontsize{10.000000}{12.000000}\selectfont\catcode`\^=\active\def^{\ifmmode\sp\else\^{}\fi}\catcode`\%=\active\def%{\%}$\mathdefault{10^{0}}$}}%
\end{pgfscope}%
\begin{pgfscope}%
\pgfsetbuttcap%
\pgfsetroundjoin%
\definecolor{currentfill}{rgb}{0.000000,0.000000,0.000000}%
\pgfsetfillcolor{currentfill}%
\pgfsetlinewidth{0.803000pt}%
\definecolor{currentstroke}{rgb}{0.000000,0.000000,0.000000}%
\pgfsetstrokecolor{currentstroke}%
\pgfsetdash{}{0pt}%
\pgfsys@defobject{currentmarker}{\pgfqpoint{-0.048611in}{0.000000in}}{\pgfqpoint{-0.000000in}{0.000000in}}{%
\pgfpathmoveto{\pgfqpoint{-0.000000in}{0.000000in}}%
\pgfpathlineto{\pgfqpoint{-0.048611in}{0.000000in}}%
\pgfusepath{stroke,fill}%
}%
\begin{pgfscope}%
\pgfsys@transformshift{0.588387in}{0.888931in}%
\pgfsys@useobject{currentmarker}{}%
\end{pgfscope}%
\end{pgfscope}%
\begin{pgfscope}%
\definecolor{textcolor}{rgb}{0.000000,0.000000,0.000000}%
\pgfsetstrokecolor{textcolor}%
\pgfsetfillcolor{textcolor}%
\pgftext[x=0.289968in, y=0.836170in, left, base]{\color{textcolor}{\rmfamily\fontsize{10.000000}{12.000000}\selectfont\catcode`\^=\active\def^{\ifmmode\sp\else\^{}\fi}\catcode`\%=\active\def%{\%}$\mathdefault{10^{1}}$}}%
\end{pgfscope}%
\begin{pgfscope}%
\pgfsetbuttcap%
\pgfsetroundjoin%
\definecolor{currentfill}{rgb}{0.000000,0.000000,0.000000}%
\pgfsetfillcolor{currentfill}%
\pgfsetlinewidth{0.803000pt}%
\definecolor{currentstroke}{rgb}{0.000000,0.000000,0.000000}%
\pgfsetstrokecolor{currentstroke}%
\pgfsetdash{}{0pt}%
\pgfsys@defobject{currentmarker}{\pgfqpoint{-0.048611in}{0.000000in}}{\pgfqpoint{-0.000000in}{0.000000in}}{%
\pgfpathmoveto{\pgfqpoint{-0.000000in}{0.000000in}}%
\pgfpathlineto{\pgfqpoint{-0.048611in}{0.000000in}}%
\pgfusepath{stroke,fill}%
}%
\begin{pgfscope}%
\pgfsys@transformshift{0.588387in}{1.164883in}%
\pgfsys@useobject{currentmarker}{}%
\end{pgfscope}%
\end{pgfscope}%
\begin{pgfscope}%
\definecolor{textcolor}{rgb}{0.000000,0.000000,0.000000}%
\pgfsetstrokecolor{textcolor}%
\pgfsetfillcolor{textcolor}%
\pgftext[x=0.289968in, y=1.112121in, left, base]{\color{textcolor}{\rmfamily\fontsize{10.000000}{12.000000}\selectfont\catcode`\^=\active\def^{\ifmmode\sp\else\^{}\fi}\catcode`\%=\active\def%{\%}$\mathdefault{10^{2}}$}}%
\end{pgfscope}%
\begin{pgfscope}%
\pgfsetbuttcap%
\pgfsetroundjoin%
\definecolor{currentfill}{rgb}{0.000000,0.000000,0.000000}%
\pgfsetfillcolor{currentfill}%
\pgfsetlinewidth{0.803000pt}%
\definecolor{currentstroke}{rgb}{0.000000,0.000000,0.000000}%
\pgfsetstrokecolor{currentstroke}%
\pgfsetdash{}{0pt}%
\pgfsys@defobject{currentmarker}{\pgfqpoint{-0.048611in}{0.000000in}}{\pgfqpoint{-0.000000in}{0.000000in}}{%
\pgfpathmoveto{\pgfqpoint{-0.000000in}{0.000000in}}%
\pgfpathlineto{\pgfqpoint{-0.048611in}{0.000000in}}%
\pgfusepath{stroke,fill}%
}%
\begin{pgfscope}%
\pgfsys@transformshift{0.588387in}{1.440834in}%
\pgfsys@useobject{currentmarker}{}%
\end{pgfscope}%
\end{pgfscope}%
\begin{pgfscope}%
\definecolor{textcolor}{rgb}{0.000000,0.000000,0.000000}%
\pgfsetstrokecolor{textcolor}%
\pgfsetfillcolor{textcolor}%
\pgftext[x=0.289968in, y=1.388072in, left, base]{\color{textcolor}{\rmfamily\fontsize{10.000000}{12.000000}\selectfont\catcode`\^=\active\def^{\ifmmode\sp\else\^{}\fi}\catcode`\%=\active\def%{\%}$\mathdefault{10^{3}}$}}%
\end{pgfscope}%
\begin{pgfscope}%
\pgfsetbuttcap%
\pgfsetroundjoin%
\definecolor{currentfill}{rgb}{0.000000,0.000000,0.000000}%
\pgfsetfillcolor{currentfill}%
\pgfsetlinewidth{0.803000pt}%
\definecolor{currentstroke}{rgb}{0.000000,0.000000,0.000000}%
\pgfsetstrokecolor{currentstroke}%
\pgfsetdash{}{0pt}%
\pgfsys@defobject{currentmarker}{\pgfqpoint{-0.048611in}{0.000000in}}{\pgfqpoint{-0.000000in}{0.000000in}}{%
\pgfpathmoveto{\pgfqpoint{-0.000000in}{0.000000in}}%
\pgfpathlineto{\pgfqpoint{-0.048611in}{0.000000in}}%
\pgfusepath{stroke,fill}%
}%
\begin{pgfscope}%
\pgfsys@transformshift{0.588387in}{1.716785in}%
\pgfsys@useobject{currentmarker}{}%
\end{pgfscope}%
\end{pgfscope}%
\begin{pgfscope}%
\definecolor{textcolor}{rgb}{0.000000,0.000000,0.000000}%
\pgfsetstrokecolor{textcolor}%
\pgfsetfillcolor{textcolor}%
\pgftext[x=0.289968in, y=1.664024in, left, base]{\color{textcolor}{\rmfamily\fontsize{10.000000}{12.000000}\selectfont\catcode`\^=\active\def^{\ifmmode\sp\else\^{}\fi}\catcode`\%=\active\def%{\%}$\mathdefault{10^{4}}$}}%
\end{pgfscope}%
\begin{pgfscope}%
\pgfsetbuttcap%
\pgfsetroundjoin%
\definecolor{currentfill}{rgb}{0.000000,0.000000,0.000000}%
\pgfsetfillcolor{currentfill}%
\pgfsetlinewidth{0.803000pt}%
\definecolor{currentstroke}{rgb}{0.000000,0.000000,0.000000}%
\pgfsetstrokecolor{currentstroke}%
\pgfsetdash{}{0pt}%
\pgfsys@defobject{currentmarker}{\pgfqpoint{-0.048611in}{0.000000in}}{\pgfqpoint{-0.000000in}{0.000000in}}{%
\pgfpathmoveto{\pgfqpoint{-0.000000in}{0.000000in}}%
\pgfpathlineto{\pgfqpoint{-0.048611in}{0.000000in}}%
\pgfusepath{stroke,fill}%
}%
\begin{pgfscope}%
\pgfsys@transformshift{0.588387in}{1.992737in}%
\pgfsys@useobject{currentmarker}{}%
\end{pgfscope}%
\end{pgfscope}%
\begin{pgfscope}%
\definecolor{textcolor}{rgb}{0.000000,0.000000,0.000000}%
\pgfsetstrokecolor{textcolor}%
\pgfsetfillcolor{textcolor}%
\pgftext[x=0.289968in, y=1.939975in, left, base]{\color{textcolor}{\rmfamily\fontsize{10.000000}{12.000000}\selectfont\catcode`\^=\active\def^{\ifmmode\sp\else\^{}\fi}\catcode`\%=\active\def%{\%}$\mathdefault{10^{5}}$}}%
\end{pgfscope}%
\begin{pgfscope}%
\pgfsetbuttcap%
\pgfsetroundjoin%
\definecolor{currentfill}{rgb}{0.000000,0.000000,0.000000}%
\pgfsetfillcolor{currentfill}%
\pgfsetlinewidth{0.803000pt}%
\definecolor{currentstroke}{rgb}{0.000000,0.000000,0.000000}%
\pgfsetstrokecolor{currentstroke}%
\pgfsetdash{}{0pt}%
\pgfsys@defobject{currentmarker}{\pgfqpoint{-0.048611in}{0.000000in}}{\pgfqpoint{-0.000000in}{0.000000in}}{%
\pgfpathmoveto{\pgfqpoint{-0.000000in}{0.000000in}}%
\pgfpathlineto{\pgfqpoint{-0.048611in}{0.000000in}}%
\pgfusepath{stroke,fill}%
}%
\begin{pgfscope}%
\pgfsys@transformshift{0.588387in}{2.268688in}%
\pgfsys@useobject{currentmarker}{}%
\end{pgfscope}%
\end{pgfscope}%
\begin{pgfscope}%
\definecolor{textcolor}{rgb}{0.000000,0.000000,0.000000}%
\pgfsetstrokecolor{textcolor}%
\pgfsetfillcolor{textcolor}%
\pgftext[x=0.289968in, y=2.215926in, left, base]{\color{textcolor}{\rmfamily\fontsize{10.000000}{12.000000}\selectfont\catcode`\^=\active\def^{\ifmmode\sp\else\^{}\fi}\catcode`\%=\active\def%{\%}$\mathdefault{10^{6}}$}}%
\end{pgfscope}%
\begin{pgfscope}%
\pgfsetbuttcap%
\pgfsetroundjoin%
\definecolor{currentfill}{rgb}{0.000000,0.000000,0.000000}%
\pgfsetfillcolor{currentfill}%
\pgfsetlinewidth{0.602250pt}%
\definecolor{currentstroke}{rgb}{0.000000,0.000000,0.000000}%
\pgfsetstrokecolor{currentstroke}%
\pgfsetdash{}{0pt}%
\pgfsys@defobject{currentmarker}{\pgfqpoint{-0.027778in}{0.000000in}}{\pgfqpoint{-0.000000in}{0.000000in}}{%
\pgfpathmoveto{\pgfqpoint{-0.000000in}{0.000000in}}%
\pgfpathlineto{\pgfqpoint{-0.027778in}{0.000000in}}%
\pgfusepath{stroke,fill}%
}%
\begin{pgfscope}%
\pgfsys@transformshift{0.588387in}{0.529910in}%
\pgfsys@useobject{currentmarker}{}%
\end{pgfscope}%
\end{pgfscope}%
\begin{pgfscope}%
\pgfsetbuttcap%
\pgfsetroundjoin%
\definecolor{currentfill}{rgb}{0.000000,0.000000,0.000000}%
\pgfsetfillcolor{currentfill}%
\pgfsetlinewidth{0.602250pt}%
\definecolor{currentstroke}{rgb}{0.000000,0.000000,0.000000}%
\pgfsetstrokecolor{currentstroke}%
\pgfsetdash{}{0pt}%
\pgfsys@defobject{currentmarker}{\pgfqpoint{-0.027778in}{0.000000in}}{\pgfqpoint{-0.000000in}{0.000000in}}{%
\pgfpathmoveto{\pgfqpoint{-0.000000in}{0.000000in}}%
\pgfpathlineto{\pgfqpoint{-0.027778in}{0.000000in}}%
\pgfusepath{stroke,fill}%
}%
\begin{pgfscope}%
\pgfsys@transformshift{0.588387in}{0.551760in}%
\pgfsys@useobject{currentmarker}{}%
\end{pgfscope}%
\end{pgfscope}%
\begin{pgfscope}%
\pgfsetbuttcap%
\pgfsetroundjoin%
\definecolor{currentfill}{rgb}{0.000000,0.000000,0.000000}%
\pgfsetfillcolor{currentfill}%
\pgfsetlinewidth{0.602250pt}%
\definecolor{currentstroke}{rgb}{0.000000,0.000000,0.000000}%
\pgfsetstrokecolor{currentstroke}%
\pgfsetdash{}{0pt}%
\pgfsys@defobject{currentmarker}{\pgfqpoint{-0.027778in}{0.000000in}}{\pgfqpoint{-0.000000in}{0.000000in}}{%
\pgfpathmoveto{\pgfqpoint{-0.000000in}{0.000000in}}%
\pgfpathlineto{\pgfqpoint{-0.027778in}{0.000000in}}%
\pgfusepath{stroke,fill}%
}%
\begin{pgfscope}%
\pgfsys@transformshift{0.588387in}{0.570235in}%
\pgfsys@useobject{currentmarker}{}%
\end{pgfscope}%
\end{pgfscope}%
\begin{pgfscope}%
\pgfsetbuttcap%
\pgfsetroundjoin%
\definecolor{currentfill}{rgb}{0.000000,0.000000,0.000000}%
\pgfsetfillcolor{currentfill}%
\pgfsetlinewidth{0.602250pt}%
\definecolor{currentstroke}{rgb}{0.000000,0.000000,0.000000}%
\pgfsetstrokecolor{currentstroke}%
\pgfsetdash{}{0pt}%
\pgfsys@defobject{currentmarker}{\pgfqpoint{-0.027778in}{0.000000in}}{\pgfqpoint{-0.000000in}{0.000000in}}{%
\pgfpathmoveto{\pgfqpoint{-0.000000in}{0.000000in}}%
\pgfpathlineto{\pgfqpoint{-0.027778in}{0.000000in}}%
\pgfusepath{stroke,fill}%
}%
\begin{pgfscope}%
\pgfsys@transformshift{0.588387in}{0.586237in}%
\pgfsys@useobject{currentmarker}{}%
\end{pgfscope}%
\end{pgfscope}%
\begin{pgfscope}%
\pgfsetbuttcap%
\pgfsetroundjoin%
\definecolor{currentfill}{rgb}{0.000000,0.000000,0.000000}%
\pgfsetfillcolor{currentfill}%
\pgfsetlinewidth{0.602250pt}%
\definecolor{currentstroke}{rgb}{0.000000,0.000000,0.000000}%
\pgfsetstrokecolor{currentstroke}%
\pgfsetdash{}{0pt}%
\pgfsys@defobject{currentmarker}{\pgfqpoint{-0.027778in}{0.000000in}}{\pgfqpoint{-0.000000in}{0.000000in}}{%
\pgfpathmoveto{\pgfqpoint{-0.000000in}{0.000000in}}%
\pgfpathlineto{\pgfqpoint{-0.027778in}{0.000000in}}%
\pgfusepath{stroke,fill}%
}%
\begin{pgfscope}%
\pgfsys@transformshift{0.588387in}{0.600353in}%
\pgfsys@useobject{currentmarker}{}%
\end{pgfscope}%
\end{pgfscope}%
\begin{pgfscope}%
\pgfsetbuttcap%
\pgfsetroundjoin%
\definecolor{currentfill}{rgb}{0.000000,0.000000,0.000000}%
\pgfsetfillcolor{currentfill}%
\pgfsetlinewidth{0.602250pt}%
\definecolor{currentstroke}{rgb}{0.000000,0.000000,0.000000}%
\pgfsetstrokecolor{currentstroke}%
\pgfsetdash{}{0pt}%
\pgfsys@defobject{currentmarker}{\pgfqpoint{-0.027778in}{0.000000in}}{\pgfqpoint{-0.000000in}{0.000000in}}{%
\pgfpathmoveto{\pgfqpoint{-0.000000in}{0.000000in}}%
\pgfpathlineto{\pgfqpoint{-0.027778in}{0.000000in}}%
\pgfusepath{stroke,fill}%
}%
\begin{pgfscope}%
\pgfsys@transformshift{0.588387in}{0.696050in}%
\pgfsys@useobject{currentmarker}{}%
\end{pgfscope}%
\end{pgfscope}%
\begin{pgfscope}%
\pgfsetbuttcap%
\pgfsetroundjoin%
\definecolor{currentfill}{rgb}{0.000000,0.000000,0.000000}%
\pgfsetfillcolor{currentfill}%
\pgfsetlinewidth{0.602250pt}%
\definecolor{currentstroke}{rgb}{0.000000,0.000000,0.000000}%
\pgfsetstrokecolor{currentstroke}%
\pgfsetdash{}{0pt}%
\pgfsys@defobject{currentmarker}{\pgfqpoint{-0.027778in}{0.000000in}}{\pgfqpoint{-0.000000in}{0.000000in}}{%
\pgfpathmoveto{\pgfqpoint{-0.000000in}{0.000000in}}%
\pgfpathlineto{\pgfqpoint{-0.027778in}{0.000000in}}%
\pgfusepath{stroke,fill}%
}%
\begin{pgfscope}%
\pgfsys@transformshift{0.588387in}{0.744642in}%
\pgfsys@useobject{currentmarker}{}%
\end{pgfscope}%
\end{pgfscope}%
\begin{pgfscope}%
\pgfsetbuttcap%
\pgfsetroundjoin%
\definecolor{currentfill}{rgb}{0.000000,0.000000,0.000000}%
\pgfsetfillcolor{currentfill}%
\pgfsetlinewidth{0.602250pt}%
\definecolor{currentstroke}{rgb}{0.000000,0.000000,0.000000}%
\pgfsetstrokecolor{currentstroke}%
\pgfsetdash{}{0pt}%
\pgfsys@defobject{currentmarker}{\pgfqpoint{-0.027778in}{0.000000in}}{\pgfqpoint{-0.000000in}{0.000000in}}{%
\pgfpathmoveto{\pgfqpoint{-0.000000in}{0.000000in}}%
\pgfpathlineto{\pgfqpoint{-0.027778in}{0.000000in}}%
\pgfusepath{stroke,fill}%
}%
\begin{pgfscope}%
\pgfsys@transformshift{0.588387in}{0.779119in}%
\pgfsys@useobject{currentmarker}{}%
\end{pgfscope}%
\end{pgfscope}%
\begin{pgfscope}%
\pgfsetbuttcap%
\pgfsetroundjoin%
\definecolor{currentfill}{rgb}{0.000000,0.000000,0.000000}%
\pgfsetfillcolor{currentfill}%
\pgfsetlinewidth{0.602250pt}%
\definecolor{currentstroke}{rgb}{0.000000,0.000000,0.000000}%
\pgfsetstrokecolor{currentstroke}%
\pgfsetdash{}{0pt}%
\pgfsys@defobject{currentmarker}{\pgfqpoint{-0.027778in}{0.000000in}}{\pgfqpoint{-0.000000in}{0.000000in}}{%
\pgfpathmoveto{\pgfqpoint{-0.000000in}{0.000000in}}%
\pgfpathlineto{\pgfqpoint{-0.027778in}{0.000000in}}%
\pgfusepath{stroke,fill}%
}%
\begin{pgfscope}%
\pgfsys@transformshift{0.588387in}{0.805862in}%
\pgfsys@useobject{currentmarker}{}%
\end{pgfscope}%
\end{pgfscope}%
\begin{pgfscope}%
\pgfsetbuttcap%
\pgfsetroundjoin%
\definecolor{currentfill}{rgb}{0.000000,0.000000,0.000000}%
\pgfsetfillcolor{currentfill}%
\pgfsetlinewidth{0.602250pt}%
\definecolor{currentstroke}{rgb}{0.000000,0.000000,0.000000}%
\pgfsetstrokecolor{currentstroke}%
\pgfsetdash{}{0pt}%
\pgfsys@defobject{currentmarker}{\pgfqpoint{-0.027778in}{0.000000in}}{\pgfqpoint{-0.000000in}{0.000000in}}{%
\pgfpathmoveto{\pgfqpoint{-0.000000in}{0.000000in}}%
\pgfpathlineto{\pgfqpoint{-0.027778in}{0.000000in}}%
\pgfusepath{stroke,fill}%
}%
\begin{pgfscope}%
\pgfsys@transformshift{0.588387in}{0.827712in}%
\pgfsys@useobject{currentmarker}{}%
\end{pgfscope}%
\end{pgfscope}%
\begin{pgfscope}%
\pgfsetbuttcap%
\pgfsetroundjoin%
\definecolor{currentfill}{rgb}{0.000000,0.000000,0.000000}%
\pgfsetfillcolor{currentfill}%
\pgfsetlinewidth{0.602250pt}%
\definecolor{currentstroke}{rgb}{0.000000,0.000000,0.000000}%
\pgfsetstrokecolor{currentstroke}%
\pgfsetdash{}{0pt}%
\pgfsys@defobject{currentmarker}{\pgfqpoint{-0.027778in}{0.000000in}}{\pgfqpoint{-0.000000in}{0.000000in}}{%
\pgfpathmoveto{\pgfqpoint{-0.000000in}{0.000000in}}%
\pgfpathlineto{\pgfqpoint{-0.027778in}{0.000000in}}%
\pgfusepath{stroke,fill}%
}%
\begin{pgfscope}%
\pgfsys@transformshift{0.588387in}{0.846186in}%
\pgfsys@useobject{currentmarker}{}%
\end{pgfscope}%
\end{pgfscope}%
\begin{pgfscope}%
\pgfsetbuttcap%
\pgfsetroundjoin%
\definecolor{currentfill}{rgb}{0.000000,0.000000,0.000000}%
\pgfsetfillcolor{currentfill}%
\pgfsetlinewidth{0.602250pt}%
\definecolor{currentstroke}{rgb}{0.000000,0.000000,0.000000}%
\pgfsetstrokecolor{currentstroke}%
\pgfsetdash{}{0pt}%
\pgfsys@defobject{currentmarker}{\pgfqpoint{-0.027778in}{0.000000in}}{\pgfqpoint{-0.000000in}{0.000000in}}{%
\pgfpathmoveto{\pgfqpoint{-0.000000in}{0.000000in}}%
\pgfpathlineto{\pgfqpoint{-0.027778in}{0.000000in}}%
\pgfusepath{stroke,fill}%
}%
\begin{pgfscope}%
\pgfsys@transformshift{0.588387in}{0.862189in}%
\pgfsys@useobject{currentmarker}{}%
\end{pgfscope}%
\end{pgfscope}%
\begin{pgfscope}%
\pgfsetbuttcap%
\pgfsetroundjoin%
\definecolor{currentfill}{rgb}{0.000000,0.000000,0.000000}%
\pgfsetfillcolor{currentfill}%
\pgfsetlinewidth{0.602250pt}%
\definecolor{currentstroke}{rgb}{0.000000,0.000000,0.000000}%
\pgfsetstrokecolor{currentstroke}%
\pgfsetdash{}{0pt}%
\pgfsys@defobject{currentmarker}{\pgfqpoint{-0.027778in}{0.000000in}}{\pgfqpoint{-0.000000in}{0.000000in}}{%
\pgfpathmoveto{\pgfqpoint{-0.000000in}{0.000000in}}%
\pgfpathlineto{\pgfqpoint{-0.027778in}{0.000000in}}%
\pgfusepath{stroke,fill}%
}%
\begin{pgfscope}%
\pgfsys@transformshift{0.588387in}{0.876304in}%
\pgfsys@useobject{currentmarker}{}%
\end{pgfscope}%
\end{pgfscope}%
\begin{pgfscope}%
\pgfsetbuttcap%
\pgfsetroundjoin%
\definecolor{currentfill}{rgb}{0.000000,0.000000,0.000000}%
\pgfsetfillcolor{currentfill}%
\pgfsetlinewidth{0.602250pt}%
\definecolor{currentstroke}{rgb}{0.000000,0.000000,0.000000}%
\pgfsetstrokecolor{currentstroke}%
\pgfsetdash{}{0pt}%
\pgfsys@defobject{currentmarker}{\pgfqpoint{-0.027778in}{0.000000in}}{\pgfqpoint{-0.000000in}{0.000000in}}{%
\pgfpathmoveto{\pgfqpoint{-0.000000in}{0.000000in}}%
\pgfpathlineto{\pgfqpoint{-0.027778in}{0.000000in}}%
\pgfusepath{stroke,fill}%
}%
\begin{pgfscope}%
\pgfsys@transformshift{0.588387in}{0.972001in}%
\pgfsys@useobject{currentmarker}{}%
\end{pgfscope}%
\end{pgfscope}%
\begin{pgfscope}%
\pgfsetbuttcap%
\pgfsetroundjoin%
\definecolor{currentfill}{rgb}{0.000000,0.000000,0.000000}%
\pgfsetfillcolor{currentfill}%
\pgfsetlinewidth{0.602250pt}%
\definecolor{currentstroke}{rgb}{0.000000,0.000000,0.000000}%
\pgfsetstrokecolor{currentstroke}%
\pgfsetdash{}{0pt}%
\pgfsys@defobject{currentmarker}{\pgfqpoint{-0.027778in}{0.000000in}}{\pgfqpoint{-0.000000in}{0.000000in}}{%
\pgfpathmoveto{\pgfqpoint{-0.000000in}{0.000000in}}%
\pgfpathlineto{\pgfqpoint{-0.027778in}{0.000000in}}%
\pgfusepath{stroke,fill}%
}%
\begin{pgfscope}%
\pgfsys@transformshift{0.588387in}{1.020593in}%
\pgfsys@useobject{currentmarker}{}%
\end{pgfscope}%
\end{pgfscope}%
\begin{pgfscope}%
\pgfsetbuttcap%
\pgfsetroundjoin%
\definecolor{currentfill}{rgb}{0.000000,0.000000,0.000000}%
\pgfsetfillcolor{currentfill}%
\pgfsetlinewidth{0.602250pt}%
\definecolor{currentstroke}{rgb}{0.000000,0.000000,0.000000}%
\pgfsetstrokecolor{currentstroke}%
\pgfsetdash{}{0pt}%
\pgfsys@defobject{currentmarker}{\pgfqpoint{-0.027778in}{0.000000in}}{\pgfqpoint{-0.000000in}{0.000000in}}{%
\pgfpathmoveto{\pgfqpoint{-0.000000in}{0.000000in}}%
\pgfpathlineto{\pgfqpoint{-0.027778in}{0.000000in}}%
\pgfusepath{stroke,fill}%
}%
\begin{pgfscope}%
\pgfsys@transformshift{0.588387in}{1.055070in}%
\pgfsys@useobject{currentmarker}{}%
\end{pgfscope}%
\end{pgfscope}%
\begin{pgfscope}%
\pgfsetbuttcap%
\pgfsetroundjoin%
\definecolor{currentfill}{rgb}{0.000000,0.000000,0.000000}%
\pgfsetfillcolor{currentfill}%
\pgfsetlinewidth{0.602250pt}%
\definecolor{currentstroke}{rgb}{0.000000,0.000000,0.000000}%
\pgfsetstrokecolor{currentstroke}%
\pgfsetdash{}{0pt}%
\pgfsys@defobject{currentmarker}{\pgfqpoint{-0.027778in}{0.000000in}}{\pgfqpoint{-0.000000in}{0.000000in}}{%
\pgfpathmoveto{\pgfqpoint{-0.000000in}{0.000000in}}%
\pgfpathlineto{\pgfqpoint{-0.027778in}{0.000000in}}%
\pgfusepath{stroke,fill}%
}%
\begin{pgfscope}%
\pgfsys@transformshift{0.588387in}{1.081813in}%
\pgfsys@useobject{currentmarker}{}%
\end{pgfscope}%
\end{pgfscope}%
\begin{pgfscope}%
\pgfsetbuttcap%
\pgfsetroundjoin%
\definecolor{currentfill}{rgb}{0.000000,0.000000,0.000000}%
\pgfsetfillcolor{currentfill}%
\pgfsetlinewidth{0.602250pt}%
\definecolor{currentstroke}{rgb}{0.000000,0.000000,0.000000}%
\pgfsetstrokecolor{currentstroke}%
\pgfsetdash{}{0pt}%
\pgfsys@defobject{currentmarker}{\pgfqpoint{-0.027778in}{0.000000in}}{\pgfqpoint{-0.000000in}{0.000000in}}{%
\pgfpathmoveto{\pgfqpoint{-0.000000in}{0.000000in}}%
\pgfpathlineto{\pgfqpoint{-0.027778in}{0.000000in}}%
\pgfusepath{stroke,fill}%
}%
\begin{pgfscope}%
\pgfsys@transformshift{0.588387in}{1.103663in}%
\pgfsys@useobject{currentmarker}{}%
\end{pgfscope}%
\end{pgfscope}%
\begin{pgfscope}%
\pgfsetbuttcap%
\pgfsetroundjoin%
\definecolor{currentfill}{rgb}{0.000000,0.000000,0.000000}%
\pgfsetfillcolor{currentfill}%
\pgfsetlinewidth{0.602250pt}%
\definecolor{currentstroke}{rgb}{0.000000,0.000000,0.000000}%
\pgfsetstrokecolor{currentstroke}%
\pgfsetdash{}{0pt}%
\pgfsys@defobject{currentmarker}{\pgfqpoint{-0.027778in}{0.000000in}}{\pgfqpoint{-0.000000in}{0.000000in}}{%
\pgfpathmoveto{\pgfqpoint{-0.000000in}{0.000000in}}%
\pgfpathlineto{\pgfqpoint{-0.027778in}{0.000000in}}%
\pgfusepath{stroke,fill}%
}%
\begin{pgfscope}%
\pgfsys@transformshift{0.588387in}{1.122137in}%
\pgfsys@useobject{currentmarker}{}%
\end{pgfscope}%
\end{pgfscope}%
\begin{pgfscope}%
\pgfsetbuttcap%
\pgfsetroundjoin%
\definecolor{currentfill}{rgb}{0.000000,0.000000,0.000000}%
\pgfsetfillcolor{currentfill}%
\pgfsetlinewidth{0.602250pt}%
\definecolor{currentstroke}{rgb}{0.000000,0.000000,0.000000}%
\pgfsetstrokecolor{currentstroke}%
\pgfsetdash{}{0pt}%
\pgfsys@defobject{currentmarker}{\pgfqpoint{-0.027778in}{0.000000in}}{\pgfqpoint{-0.000000in}{0.000000in}}{%
\pgfpathmoveto{\pgfqpoint{-0.000000in}{0.000000in}}%
\pgfpathlineto{\pgfqpoint{-0.027778in}{0.000000in}}%
\pgfusepath{stroke,fill}%
}%
\begin{pgfscope}%
\pgfsys@transformshift{0.588387in}{1.138140in}%
\pgfsys@useobject{currentmarker}{}%
\end{pgfscope}%
\end{pgfscope}%
\begin{pgfscope}%
\pgfsetbuttcap%
\pgfsetroundjoin%
\definecolor{currentfill}{rgb}{0.000000,0.000000,0.000000}%
\pgfsetfillcolor{currentfill}%
\pgfsetlinewidth{0.602250pt}%
\definecolor{currentstroke}{rgb}{0.000000,0.000000,0.000000}%
\pgfsetstrokecolor{currentstroke}%
\pgfsetdash{}{0pt}%
\pgfsys@defobject{currentmarker}{\pgfqpoint{-0.027778in}{0.000000in}}{\pgfqpoint{-0.000000in}{0.000000in}}{%
\pgfpathmoveto{\pgfqpoint{-0.000000in}{0.000000in}}%
\pgfpathlineto{\pgfqpoint{-0.027778in}{0.000000in}}%
\pgfusepath{stroke,fill}%
}%
\begin{pgfscope}%
\pgfsys@transformshift{0.588387in}{1.152256in}%
\pgfsys@useobject{currentmarker}{}%
\end{pgfscope}%
\end{pgfscope}%
\begin{pgfscope}%
\pgfsetbuttcap%
\pgfsetroundjoin%
\definecolor{currentfill}{rgb}{0.000000,0.000000,0.000000}%
\pgfsetfillcolor{currentfill}%
\pgfsetlinewidth{0.602250pt}%
\definecolor{currentstroke}{rgb}{0.000000,0.000000,0.000000}%
\pgfsetstrokecolor{currentstroke}%
\pgfsetdash{}{0pt}%
\pgfsys@defobject{currentmarker}{\pgfqpoint{-0.027778in}{0.000000in}}{\pgfqpoint{-0.000000in}{0.000000in}}{%
\pgfpathmoveto{\pgfqpoint{-0.000000in}{0.000000in}}%
\pgfpathlineto{\pgfqpoint{-0.027778in}{0.000000in}}%
\pgfusepath{stroke,fill}%
}%
\begin{pgfscope}%
\pgfsys@transformshift{0.588387in}{1.247952in}%
\pgfsys@useobject{currentmarker}{}%
\end{pgfscope}%
\end{pgfscope}%
\begin{pgfscope}%
\pgfsetbuttcap%
\pgfsetroundjoin%
\definecolor{currentfill}{rgb}{0.000000,0.000000,0.000000}%
\pgfsetfillcolor{currentfill}%
\pgfsetlinewidth{0.602250pt}%
\definecolor{currentstroke}{rgb}{0.000000,0.000000,0.000000}%
\pgfsetstrokecolor{currentstroke}%
\pgfsetdash{}{0pt}%
\pgfsys@defobject{currentmarker}{\pgfqpoint{-0.027778in}{0.000000in}}{\pgfqpoint{-0.000000in}{0.000000in}}{%
\pgfpathmoveto{\pgfqpoint{-0.000000in}{0.000000in}}%
\pgfpathlineto{\pgfqpoint{-0.027778in}{0.000000in}}%
\pgfusepath{stroke,fill}%
}%
\begin{pgfscope}%
\pgfsys@transformshift{0.588387in}{1.296545in}%
\pgfsys@useobject{currentmarker}{}%
\end{pgfscope}%
\end{pgfscope}%
\begin{pgfscope}%
\pgfsetbuttcap%
\pgfsetroundjoin%
\definecolor{currentfill}{rgb}{0.000000,0.000000,0.000000}%
\pgfsetfillcolor{currentfill}%
\pgfsetlinewidth{0.602250pt}%
\definecolor{currentstroke}{rgb}{0.000000,0.000000,0.000000}%
\pgfsetstrokecolor{currentstroke}%
\pgfsetdash{}{0pt}%
\pgfsys@defobject{currentmarker}{\pgfqpoint{-0.027778in}{0.000000in}}{\pgfqpoint{-0.000000in}{0.000000in}}{%
\pgfpathmoveto{\pgfqpoint{-0.000000in}{0.000000in}}%
\pgfpathlineto{\pgfqpoint{-0.027778in}{0.000000in}}%
\pgfusepath{stroke,fill}%
}%
\begin{pgfscope}%
\pgfsys@transformshift{0.588387in}{1.331022in}%
\pgfsys@useobject{currentmarker}{}%
\end{pgfscope}%
\end{pgfscope}%
\begin{pgfscope}%
\pgfsetbuttcap%
\pgfsetroundjoin%
\definecolor{currentfill}{rgb}{0.000000,0.000000,0.000000}%
\pgfsetfillcolor{currentfill}%
\pgfsetlinewidth{0.602250pt}%
\definecolor{currentstroke}{rgb}{0.000000,0.000000,0.000000}%
\pgfsetstrokecolor{currentstroke}%
\pgfsetdash{}{0pt}%
\pgfsys@defobject{currentmarker}{\pgfqpoint{-0.027778in}{0.000000in}}{\pgfqpoint{-0.000000in}{0.000000in}}{%
\pgfpathmoveto{\pgfqpoint{-0.000000in}{0.000000in}}%
\pgfpathlineto{\pgfqpoint{-0.027778in}{0.000000in}}%
\pgfusepath{stroke,fill}%
}%
\begin{pgfscope}%
\pgfsys@transformshift{0.588387in}{1.357764in}%
\pgfsys@useobject{currentmarker}{}%
\end{pgfscope}%
\end{pgfscope}%
\begin{pgfscope}%
\pgfsetbuttcap%
\pgfsetroundjoin%
\definecolor{currentfill}{rgb}{0.000000,0.000000,0.000000}%
\pgfsetfillcolor{currentfill}%
\pgfsetlinewidth{0.602250pt}%
\definecolor{currentstroke}{rgb}{0.000000,0.000000,0.000000}%
\pgfsetstrokecolor{currentstroke}%
\pgfsetdash{}{0pt}%
\pgfsys@defobject{currentmarker}{\pgfqpoint{-0.027778in}{0.000000in}}{\pgfqpoint{-0.000000in}{0.000000in}}{%
\pgfpathmoveto{\pgfqpoint{-0.000000in}{0.000000in}}%
\pgfpathlineto{\pgfqpoint{-0.027778in}{0.000000in}}%
\pgfusepath{stroke,fill}%
}%
\begin{pgfscope}%
\pgfsys@transformshift{0.588387in}{1.379614in}%
\pgfsys@useobject{currentmarker}{}%
\end{pgfscope}%
\end{pgfscope}%
\begin{pgfscope}%
\pgfsetbuttcap%
\pgfsetroundjoin%
\definecolor{currentfill}{rgb}{0.000000,0.000000,0.000000}%
\pgfsetfillcolor{currentfill}%
\pgfsetlinewidth{0.602250pt}%
\definecolor{currentstroke}{rgb}{0.000000,0.000000,0.000000}%
\pgfsetstrokecolor{currentstroke}%
\pgfsetdash{}{0pt}%
\pgfsys@defobject{currentmarker}{\pgfqpoint{-0.027778in}{0.000000in}}{\pgfqpoint{-0.000000in}{0.000000in}}{%
\pgfpathmoveto{\pgfqpoint{-0.000000in}{0.000000in}}%
\pgfpathlineto{\pgfqpoint{-0.027778in}{0.000000in}}%
\pgfusepath{stroke,fill}%
}%
\begin{pgfscope}%
\pgfsys@transformshift{0.588387in}{1.398088in}%
\pgfsys@useobject{currentmarker}{}%
\end{pgfscope}%
\end{pgfscope}%
\begin{pgfscope}%
\pgfsetbuttcap%
\pgfsetroundjoin%
\definecolor{currentfill}{rgb}{0.000000,0.000000,0.000000}%
\pgfsetfillcolor{currentfill}%
\pgfsetlinewidth{0.602250pt}%
\definecolor{currentstroke}{rgb}{0.000000,0.000000,0.000000}%
\pgfsetstrokecolor{currentstroke}%
\pgfsetdash{}{0pt}%
\pgfsys@defobject{currentmarker}{\pgfqpoint{-0.027778in}{0.000000in}}{\pgfqpoint{-0.000000in}{0.000000in}}{%
\pgfpathmoveto{\pgfqpoint{-0.000000in}{0.000000in}}%
\pgfpathlineto{\pgfqpoint{-0.027778in}{0.000000in}}%
\pgfusepath{stroke,fill}%
}%
\begin{pgfscope}%
\pgfsys@transformshift{0.588387in}{1.414091in}%
\pgfsys@useobject{currentmarker}{}%
\end{pgfscope}%
\end{pgfscope}%
\begin{pgfscope}%
\pgfsetbuttcap%
\pgfsetroundjoin%
\definecolor{currentfill}{rgb}{0.000000,0.000000,0.000000}%
\pgfsetfillcolor{currentfill}%
\pgfsetlinewidth{0.602250pt}%
\definecolor{currentstroke}{rgb}{0.000000,0.000000,0.000000}%
\pgfsetstrokecolor{currentstroke}%
\pgfsetdash{}{0pt}%
\pgfsys@defobject{currentmarker}{\pgfqpoint{-0.027778in}{0.000000in}}{\pgfqpoint{-0.000000in}{0.000000in}}{%
\pgfpathmoveto{\pgfqpoint{-0.000000in}{0.000000in}}%
\pgfpathlineto{\pgfqpoint{-0.027778in}{0.000000in}}%
\pgfusepath{stroke,fill}%
}%
\begin{pgfscope}%
\pgfsys@transformshift{0.588387in}{1.428207in}%
\pgfsys@useobject{currentmarker}{}%
\end{pgfscope}%
\end{pgfscope}%
\begin{pgfscope}%
\pgfsetbuttcap%
\pgfsetroundjoin%
\definecolor{currentfill}{rgb}{0.000000,0.000000,0.000000}%
\pgfsetfillcolor{currentfill}%
\pgfsetlinewidth{0.602250pt}%
\definecolor{currentstroke}{rgb}{0.000000,0.000000,0.000000}%
\pgfsetstrokecolor{currentstroke}%
\pgfsetdash{}{0pt}%
\pgfsys@defobject{currentmarker}{\pgfqpoint{-0.027778in}{0.000000in}}{\pgfqpoint{-0.000000in}{0.000000in}}{%
\pgfpathmoveto{\pgfqpoint{-0.000000in}{0.000000in}}%
\pgfpathlineto{\pgfqpoint{-0.027778in}{0.000000in}}%
\pgfusepath{stroke,fill}%
}%
\begin{pgfscope}%
\pgfsys@transformshift{0.588387in}{1.523904in}%
\pgfsys@useobject{currentmarker}{}%
\end{pgfscope}%
\end{pgfscope}%
\begin{pgfscope}%
\pgfsetbuttcap%
\pgfsetroundjoin%
\definecolor{currentfill}{rgb}{0.000000,0.000000,0.000000}%
\pgfsetfillcolor{currentfill}%
\pgfsetlinewidth{0.602250pt}%
\definecolor{currentstroke}{rgb}{0.000000,0.000000,0.000000}%
\pgfsetstrokecolor{currentstroke}%
\pgfsetdash{}{0pt}%
\pgfsys@defobject{currentmarker}{\pgfqpoint{-0.027778in}{0.000000in}}{\pgfqpoint{-0.000000in}{0.000000in}}{%
\pgfpathmoveto{\pgfqpoint{-0.000000in}{0.000000in}}%
\pgfpathlineto{\pgfqpoint{-0.027778in}{0.000000in}}%
\pgfusepath{stroke,fill}%
}%
\begin{pgfscope}%
\pgfsys@transformshift{0.588387in}{1.572496in}%
\pgfsys@useobject{currentmarker}{}%
\end{pgfscope}%
\end{pgfscope}%
\begin{pgfscope}%
\pgfsetbuttcap%
\pgfsetroundjoin%
\definecolor{currentfill}{rgb}{0.000000,0.000000,0.000000}%
\pgfsetfillcolor{currentfill}%
\pgfsetlinewidth{0.602250pt}%
\definecolor{currentstroke}{rgb}{0.000000,0.000000,0.000000}%
\pgfsetstrokecolor{currentstroke}%
\pgfsetdash{}{0pt}%
\pgfsys@defobject{currentmarker}{\pgfqpoint{-0.027778in}{0.000000in}}{\pgfqpoint{-0.000000in}{0.000000in}}{%
\pgfpathmoveto{\pgfqpoint{-0.000000in}{0.000000in}}%
\pgfpathlineto{\pgfqpoint{-0.027778in}{0.000000in}}%
\pgfusepath{stroke,fill}%
}%
\begin{pgfscope}%
\pgfsys@transformshift{0.588387in}{1.606973in}%
\pgfsys@useobject{currentmarker}{}%
\end{pgfscope}%
\end{pgfscope}%
\begin{pgfscope}%
\pgfsetbuttcap%
\pgfsetroundjoin%
\definecolor{currentfill}{rgb}{0.000000,0.000000,0.000000}%
\pgfsetfillcolor{currentfill}%
\pgfsetlinewidth{0.602250pt}%
\definecolor{currentstroke}{rgb}{0.000000,0.000000,0.000000}%
\pgfsetstrokecolor{currentstroke}%
\pgfsetdash{}{0pt}%
\pgfsys@defobject{currentmarker}{\pgfqpoint{-0.027778in}{0.000000in}}{\pgfqpoint{-0.000000in}{0.000000in}}{%
\pgfpathmoveto{\pgfqpoint{-0.000000in}{0.000000in}}%
\pgfpathlineto{\pgfqpoint{-0.027778in}{0.000000in}}%
\pgfusepath{stroke,fill}%
}%
\begin{pgfscope}%
\pgfsys@transformshift{0.588387in}{1.633716in}%
\pgfsys@useobject{currentmarker}{}%
\end{pgfscope}%
\end{pgfscope}%
\begin{pgfscope}%
\pgfsetbuttcap%
\pgfsetroundjoin%
\definecolor{currentfill}{rgb}{0.000000,0.000000,0.000000}%
\pgfsetfillcolor{currentfill}%
\pgfsetlinewidth{0.602250pt}%
\definecolor{currentstroke}{rgb}{0.000000,0.000000,0.000000}%
\pgfsetstrokecolor{currentstroke}%
\pgfsetdash{}{0pt}%
\pgfsys@defobject{currentmarker}{\pgfqpoint{-0.027778in}{0.000000in}}{\pgfqpoint{-0.000000in}{0.000000in}}{%
\pgfpathmoveto{\pgfqpoint{-0.000000in}{0.000000in}}%
\pgfpathlineto{\pgfqpoint{-0.027778in}{0.000000in}}%
\pgfusepath{stroke,fill}%
}%
\begin{pgfscope}%
\pgfsys@transformshift{0.588387in}{1.655566in}%
\pgfsys@useobject{currentmarker}{}%
\end{pgfscope}%
\end{pgfscope}%
\begin{pgfscope}%
\pgfsetbuttcap%
\pgfsetroundjoin%
\definecolor{currentfill}{rgb}{0.000000,0.000000,0.000000}%
\pgfsetfillcolor{currentfill}%
\pgfsetlinewidth{0.602250pt}%
\definecolor{currentstroke}{rgb}{0.000000,0.000000,0.000000}%
\pgfsetstrokecolor{currentstroke}%
\pgfsetdash{}{0pt}%
\pgfsys@defobject{currentmarker}{\pgfqpoint{-0.027778in}{0.000000in}}{\pgfqpoint{-0.000000in}{0.000000in}}{%
\pgfpathmoveto{\pgfqpoint{-0.000000in}{0.000000in}}%
\pgfpathlineto{\pgfqpoint{-0.027778in}{0.000000in}}%
\pgfusepath{stroke,fill}%
}%
\begin{pgfscope}%
\pgfsys@transformshift{0.588387in}{1.674040in}%
\pgfsys@useobject{currentmarker}{}%
\end{pgfscope}%
\end{pgfscope}%
\begin{pgfscope}%
\pgfsetbuttcap%
\pgfsetroundjoin%
\definecolor{currentfill}{rgb}{0.000000,0.000000,0.000000}%
\pgfsetfillcolor{currentfill}%
\pgfsetlinewidth{0.602250pt}%
\definecolor{currentstroke}{rgb}{0.000000,0.000000,0.000000}%
\pgfsetstrokecolor{currentstroke}%
\pgfsetdash{}{0pt}%
\pgfsys@defobject{currentmarker}{\pgfqpoint{-0.027778in}{0.000000in}}{\pgfqpoint{-0.000000in}{0.000000in}}{%
\pgfpathmoveto{\pgfqpoint{-0.000000in}{0.000000in}}%
\pgfpathlineto{\pgfqpoint{-0.027778in}{0.000000in}}%
\pgfusepath{stroke,fill}%
}%
\begin{pgfscope}%
\pgfsys@transformshift{0.588387in}{1.690043in}%
\pgfsys@useobject{currentmarker}{}%
\end{pgfscope}%
\end{pgfscope}%
\begin{pgfscope}%
\pgfsetbuttcap%
\pgfsetroundjoin%
\definecolor{currentfill}{rgb}{0.000000,0.000000,0.000000}%
\pgfsetfillcolor{currentfill}%
\pgfsetlinewidth{0.602250pt}%
\definecolor{currentstroke}{rgb}{0.000000,0.000000,0.000000}%
\pgfsetstrokecolor{currentstroke}%
\pgfsetdash{}{0pt}%
\pgfsys@defobject{currentmarker}{\pgfqpoint{-0.027778in}{0.000000in}}{\pgfqpoint{-0.000000in}{0.000000in}}{%
\pgfpathmoveto{\pgfqpoint{-0.000000in}{0.000000in}}%
\pgfpathlineto{\pgfqpoint{-0.027778in}{0.000000in}}%
\pgfusepath{stroke,fill}%
}%
\begin{pgfscope}%
\pgfsys@transformshift{0.588387in}{1.704158in}%
\pgfsys@useobject{currentmarker}{}%
\end{pgfscope}%
\end{pgfscope}%
\begin{pgfscope}%
\pgfsetbuttcap%
\pgfsetroundjoin%
\definecolor{currentfill}{rgb}{0.000000,0.000000,0.000000}%
\pgfsetfillcolor{currentfill}%
\pgfsetlinewidth{0.602250pt}%
\definecolor{currentstroke}{rgb}{0.000000,0.000000,0.000000}%
\pgfsetstrokecolor{currentstroke}%
\pgfsetdash{}{0pt}%
\pgfsys@defobject{currentmarker}{\pgfqpoint{-0.027778in}{0.000000in}}{\pgfqpoint{-0.000000in}{0.000000in}}{%
\pgfpathmoveto{\pgfqpoint{-0.000000in}{0.000000in}}%
\pgfpathlineto{\pgfqpoint{-0.027778in}{0.000000in}}%
\pgfusepath{stroke,fill}%
}%
\begin{pgfscope}%
\pgfsys@transformshift{0.588387in}{1.799855in}%
\pgfsys@useobject{currentmarker}{}%
\end{pgfscope}%
\end{pgfscope}%
\begin{pgfscope}%
\pgfsetbuttcap%
\pgfsetroundjoin%
\definecolor{currentfill}{rgb}{0.000000,0.000000,0.000000}%
\pgfsetfillcolor{currentfill}%
\pgfsetlinewidth{0.602250pt}%
\definecolor{currentstroke}{rgb}{0.000000,0.000000,0.000000}%
\pgfsetstrokecolor{currentstroke}%
\pgfsetdash{}{0pt}%
\pgfsys@defobject{currentmarker}{\pgfqpoint{-0.027778in}{0.000000in}}{\pgfqpoint{-0.000000in}{0.000000in}}{%
\pgfpathmoveto{\pgfqpoint{-0.000000in}{0.000000in}}%
\pgfpathlineto{\pgfqpoint{-0.027778in}{0.000000in}}%
\pgfusepath{stroke,fill}%
}%
\begin{pgfscope}%
\pgfsys@transformshift{0.588387in}{1.848447in}%
\pgfsys@useobject{currentmarker}{}%
\end{pgfscope}%
\end{pgfscope}%
\begin{pgfscope}%
\pgfsetbuttcap%
\pgfsetroundjoin%
\definecolor{currentfill}{rgb}{0.000000,0.000000,0.000000}%
\pgfsetfillcolor{currentfill}%
\pgfsetlinewidth{0.602250pt}%
\definecolor{currentstroke}{rgb}{0.000000,0.000000,0.000000}%
\pgfsetstrokecolor{currentstroke}%
\pgfsetdash{}{0pt}%
\pgfsys@defobject{currentmarker}{\pgfqpoint{-0.027778in}{0.000000in}}{\pgfqpoint{-0.000000in}{0.000000in}}{%
\pgfpathmoveto{\pgfqpoint{-0.000000in}{0.000000in}}%
\pgfpathlineto{\pgfqpoint{-0.027778in}{0.000000in}}%
\pgfusepath{stroke,fill}%
}%
\begin{pgfscope}%
\pgfsys@transformshift{0.588387in}{1.882924in}%
\pgfsys@useobject{currentmarker}{}%
\end{pgfscope}%
\end{pgfscope}%
\begin{pgfscope}%
\pgfsetbuttcap%
\pgfsetroundjoin%
\definecolor{currentfill}{rgb}{0.000000,0.000000,0.000000}%
\pgfsetfillcolor{currentfill}%
\pgfsetlinewidth{0.602250pt}%
\definecolor{currentstroke}{rgb}{0.000000,0.000000,0.000000}%
\pgfsetstrokecolor{currentstroke}%
\pgfsetdash{}{0pt}%
\pgfsys@defobject{currentmarker}{\pgfqpoint{-0.027778in}{0.000000in}}{\pgfqpoint{-0.000000in}{0.000000in}}{%
\pgfpathmoveto{\pgfqpoint{-0.000000in}{0.000000in}}%
\pgfpathlineto{\pgfqpoint{-0.027778in}{0.000000in}}%
\pgfusepath{stroke,fill}%
}%
\begin{pgfscope}%
\pgfsys@transformshift{0.588387in}{1.909667in}%
\pgfsys@useobject{currentmarker}{}%
\end{pgfscope}%
\end{pgfscope}%
\begin{pgfscope}%
\pgfsetbuttcap%
\pgfsetroundjoin%
\definecolor{currentfill}{rgb}{0.000000,0.000000,0.000000}%
\pgfsetfillcolor{currentfill}%
\pgfsetlinewidth{0.602250pt}%
\definecolor{currentstroke}{rgb}{0.000000,0.000000,0.000000}%
\pgfsetstrokecolor{currentstroke}%
\pgfsetdash{}{0pt}%
\pgfsys@defobject{currentmarker}{\pgfqpoint{-0.027778in}{0.000000in}}{\pgfqpoint{-0.000000in}{0.000000in}}{%
\pgfpathmoveto{\pgfqpoint{-0.000000in}{0.000000in}}%
\pgfpathlineto{\pgfqpoint{-0.027778in}{0.000000in}}%
\pgfusepath{stroke,fill}%
}%
\begin{pgfscope}%
\pgfsys@transformshift{0.588387in}{1.931517in}%
\pgfsys@useobject{currentmarker}{}%
\end{pgfscope}%
\end{pgfscope}%
\begin{pgfscope}%
\pgfsetbuttcap%
\pgfsetroundjoin%
\definecolor{currentfill}{rgb}{0.000000,0.000000,0.000000}%
\pgfsetfillcolor{currentfill}%
\pgfsetlinewidth{0.602250pt}%
\definecolor{currentstroke}{rgb}{0.000000,0.000000,0.000000}%
\pgfsetstrokecolor{currentstroke}%
\pgfsetdash{}{0pt}%
\pgfsys@defobject{currentmarker}{\pgfqpoint{-0.027778in}{0.000000in}}{\pgfqpoint{-0.000000in}{0.000000in}}{%
\pgfpathmoveto{\pgfqpoint{-0.000000in}{0.000000in}}%
\pgfpathlineto{\pgfqpoint{-0.027778in}{0.000000in}}%
\pgfusepath{stroke,fill}%
}%
\begin{pgfscope}%
\pgfsys@transformshift{0.588387in}{1.949991in}%
\pgfsys@useobject{currentmarker}{}%
\end{pgfscope}%
\end{pgfscope}%
\begin{pgfscope}%
\pgfsetbuttcap%
\pgfsetroundjoin%
\definecolor{currentfill}{rgb}{0.000000,0.000000,0.000000}%
\pgfsetfillcolor{currentfill}%
\pgfsetlinewidth{0.602250pt}%
\definecolor{currentstroke}{rgb}{0.000000,0.000000,0.000000}%
\pgfsetstrokecolor{currentstroke}%
\pgfsetdash{}{0pt}%
\pgfsys@defobject{currentmarker}{\pgfqpoint{-0.027778in}{0.000000in}}{\pgfqpoint{-0.000000in}{0.000000in}}{%
\pgfpathmoveto{\pgfqpoint{-0.000000in}{0.000000in}}%
\pgfpathlineto{\pgfqpoint{-0.027778in}{0.000000in}}%
\pgfusepath{stroke,fill}%
}%
\begin{pgfscope}%
\pgfsys@transformshift{0.588387in}{1.965994in}%
\pgfsys@useobject{currentmarker}{}%
\end{pgfscope}%
\end{pgfscope}%
\begin{pgfscope}%
\pgfsetbuttcap%
\pgfsetroundjoin%
\definecolor{currentfill}{rgb}{0.000000,0.000000,0.000000}%
\pgfsetfillcolor{currentfill}%
\pgfsetlinewidth{0.602250pt}%
\definecolor{currentstroke}{rgb}{0.000000,0.000000,0.000000}%
\pgfsetstrokecolor{currentstroke}%
\pgfsetdash{}{0pt}%
\pgfsys@defobject{currentmarker}{\pgfqpoint{-0.027778in}{0.000000in}}{\pgfqpoint{-0.000000in}{0.000000in}}{%
\pgfpathmoveto{\pgfqpoint{-0.000000in}{0.000000in}}%
\pgfpathlineto{\pgfqpoint{-0.027778in}{0.000000in}}%
\pgfusepath{stroke,fill}%
}%
\begin{pgfscope}%
\pgfsys@transformshift{0.588387in}{1.980110in}%
\pgfsys@useobject{currentmarker}{}%
\end{pgfscope}%
\end{pgfscope}%
\begin{pgfscope}%
\pgfsetbuttcap%
\pgfsetroundjoin%
\definecolor{currentfill}{rgb}{0.000000,0.000000,0.000000}%
\pgfsetfillcolor{currentfill}%
\pgfsetlinewidth{0.602250pt}%
\definecolor{currentstroke}{rgb}{0.000000,0.000000,0.000000}%
\pgfsetstrokecolor{currentstroke}%
\pgfsetdash{}{0pt}%
\pgfsys@defobject{currentmarker}{\pgfqpoint{-0.027778in}{0.000000in}}{\pgfqpoint{-0.000000in}{0.000000in}}{%
\pgfpathmoveto{\pgfqpoint{-0.000000in}{0.000000in}}%
\pgfpathlineto{\pgfqpoint{-0.027778in}{0.000000in}}%
\pgfusepath{stroke,fill}%
}%
\begin{pgfscope}%
\pgfsys@transformshift{0.588387in}{2.075806in}%
\pgfsys@useobject{currentmarker}{}%
\end{pgfscope}%
\end{pgfscope}%
\begin{pgfscope}%
\pgfsetbuttcap%
\pgfsetroundjoin%
\definecolor{currentfill}{rgb}{0.000000,0.000000,0.000000}%
\pgfsetfillcolor{currentfill}%
\pgfsetlinewidth{0.602250pt}%
\definecolor{currentstroke}{rgb}{0.000000,0.000000,0.000000}%
\pgfsetstrokecolor{currentstroke}%
\pgfsetdash{}{0pt}%
\pgfsys@defobject{currentmarker}{\pgfqpoint{-0.027778in}{0.000000in}}{\pgfqpoint{-0.000000in}{0.000000in}}{%
\pgfpathmoveto{\pgfqpoint{-0.000000in}{0.000000in}}%
\pgfpathlineto{\pgfqpoint{-0.027778in}{0.000000in}}%
\pgfusepath{stroke,fill}%
}%
\begin{pgfscope}%
\pgfsys@transformshift{0.588387in}{2.124399in}%
\pgfsys@useobject{currentmarker}{}%
\end{pgfscope}%
\end{pgfscope}%
\begin{pgfscope}%
\pgfsetbuttcap%
\pgfsetroundjoin%
\definecolor{currentfill}{rgb}{0.000000,0.000000,0.000000}%
\pgfsetfillcolor{currentfill}%
\pgfsetlinewidth{0.602250pt}%
\definecolor{currentstroke}{rgb}{0.000000,0.000000,0.000000}%
\pgfsetstrokecolor{currentstroke}%
\pgfsetdash{}{0pt}%
\pgfsys@defobject{currentmarker}{\pgfqpoint{-0.027778in}{0.000000in}}{\pgfqpoint{-0.000000in}{0.000000in}}{%
\pgfpathmoveto{\pgfqpoint{-0.000000in}{0.000000in}}%
\pgfpathlineto{\pgfqpoint{-0.027778in}{0.000000in}}%
\pgfusepath{stroke,fill}%
}%
\begin{pgfscope}%
\pgfsys@transformshift{0.588387in}{2.158876in}%
\pgfsys@useobject{currentmarker}{}%
\end{pgfscope}%
\end{pgfscope}%
\begin{pgfscope}%
\pgfsetbuttcap%
\pgfsetroundjoin%
\definecolor{currentfill}{rgb}{0.000000,0.000000,0.000000}%
\pgfsetfillcolor{currentfill}%
\pgfsetlinewidth{0.602250pt}%
\definecolor{currentstroke}{rgb}{0.000000,0.000000,0.000000}%
\pgfsetstrokecolor{currentstroke}%
\pgfsetdash{}{0pt}%
\pgfsys@defobject{currentmarker}{\pgfqpoint{-0.027778in}{0.000000in}}{\pgfqpoint{-0.000000in}{0.000000in}}{%
\pgfpathmoveto{\pgfqpoint{-0.000000in}{0.000000in}}%
\pgfpathlineto{\pgfqpoint{-0.027778in}{0.000000in}}%
\pgfusepath{stroke,fill}%
}%
\begin{pgfscope}%
\pgfsys@transformshift{0.588387in}{2.185618in}%
\pgfsys@useobject{currentmarker}{}%
\end{pgfscope}%
\end{pgfscope}%
\begin{pgfscope}%
\pgfsetbuttcap%
\pgfsetroundjoin%
\definecolor{currentfill}{rgb}{0.000000,0.000000,0.000000}%
\pgfsetfillcolor{currentfill}%
\pgfsetlinewidth{0.602250pt}%
\definecolor{currentstroke}{rgb}{0.000000,0.000000,0.000000}%
\pgfsetstrokecolor{currentstroke}%
\pgfsetdash{}{0pt}%
\pgfsys@defobject{currentmarker}{\pgfqpoint{-0.027778in}{0.000000in}}{\pgfqpoint{-0.000000in}{0.000000in}}{%
\pgfpathmoveto{\pgfqpoint{-0.000000in}{0.000000in}}%
\pgfpathlineto{\pgfqpoint{-0.027778in}{0.000000in}}%
\pgfusepath{stroke,fill}%
}%
\begin{pgfscope}%
\pgfsys@transformshift{0.588387in}{2.207468in}%
\pgfsys@useobject{currentmarker}{}%
\end{pgfscope}%
\end{pgfscope}%
\begin{pgfscope}%
\pgfsetbuttcap%
\pgfsetroundjoin%
\definecolor{currentfill}{rgb}{0.000000,0.000000,0.000000}%
\pgfsetfillcolor{currentfill}%
\pgfsetlinewidth{0.602250pt}%
\definecolor{currentstroke}{rgb}{0.000000,0.000000,0.000000}%
\pgfsetstrokecolor{currentstroke}%
\pgfsetdash{}{0pt}%
\pgfsys@defobject{currentmarker}{\pgfqpoint{-0.027778in}{0.000000in}}{\pgfqpoint{-0.000000in}{0.000000in}}{%
\pgfpathmoveto{\pgfqpoint{-0.000000in}{0.000000in}}%
\pgfpathlineto{\pgfqpoint{-0.027778in}{0.000000in}}%
\pgfusepath{stroke,fill}%
}%
\begin{pgfscope}%
\pgfsys@transformshift{0.588387in}{2.225942in}%
\pgfsys@useobject{currentmarker}{}%
\end{pgfscope}%
\end{pgfscope}%
\begin{pgfscope}%
\pgfsetbuttcap%
\pgfsetroundjoin%
\definecolor{currentfill}{rgb}{0.000000,0.000000,0.000000}%
\pgfsetfillcolor{currentfill}%
\pgfsetlinewidth{0.602250pt}%
\definecolor{currentstroke}{rgb}{0.000000,0.000000,0.000000}%
\pgfsetstrokecolor{currentstroke}%
\pgfsetdash{}{0pt}%
\pgfsys@defobject{currentmarker}{\pgfqpoint{-0.027778in}{0.000000in}}{\pgfqpoint{-0.000000in}{0.000000in}}{%
\pgfpathmoveto{\pgfqpoint{-0.000000in}{0.000000in}}%
\pgfpathlineto{\pgfqpoint{-0.027778in}{0.000000in}}%
\pgfusepath{stroke,fill}%
}%
\begin{pgfscope}%
\pgfsys@transformshift{0.588387in}{2.241945in}%
\pgfsys@useobject{currentmarker}{}%
\end{pgfscope}%
\end{pgfscope}%
\begin{pgfscope}%
\pgfsetbuttcap%
\pgfsetroundjoin%
\definecolor{currentfill}{rgb}{0.000000,0.000000,0.000000}%
\pgfsetfillcolor{currentfill}%
\pgfsetlinewidth{0.602250pt}%
\definecolor{currentstroke}{rgb}{0.000000,0.000000,0.000000}%
\pgfsetstrokecolor{currentstroke}%
\pgfsetdash{}{0pt}%
\pgfsys@defobject{currentmarker}{\pgfqpoint{-0.027778in}{0.000000in}}{\pgfqpoint{-0.000000in}{0.000000in}}{%
\pgfpathmoveto{\pgfqpoint{-0.000000in}{0.000000in}}%
\pgfpathlineto{\pgfqpoint{-0.027778in}{0.000000in}}%
\pgfusepath{stroke,fill}%
}%
\begin{pgfscope}%
\pgfsys@transformshift{0.588387in}{2.256061in}%
\pgfsys@useobject{currentmarker}{}%
\end{pgfscope}%
\end{pgfscope}%
\begin{pgfscope}%
\pgfsetbuttcap%
\pgfsetroundjoin%
\definecolor{currentfill}{rgb}{0.000000,0.000000,0.000000}%
\pgfsetfillcolor{currentfill}%
\pgfsetlinewidth{0.602250pt}%
\definecolor{currentstroke}{rgb}{0.000000,0.000000,0.000000}%
\pgfsetstrokecolor{currentstroke}%
\pgfsetdash{}{0pt}%
\pgfsys@defobject{currentmarker}{\pgfqpoint{-0.027778in}{0.000000in}}{\pgfqpoint{-0.000000in}{0.000000in}}{%
\pgfpathmoveto{\pgfqpoint{-0.000000in}{0.000000in}}%
\pgfpathlineto{\pgfqpoint{-0.027778in}{0.000000in}}%
\pgfusepath{stroke,fill}%
}%
\begin{pgfscope}%
\pgfsys@transformshift{0.588387in}{2.351757in}%
\pgfsys@useobject{currentmarker}{}%
\end{pgfscope}%
\end{pgfscope}%
\begin{pgfscope}%
\pgfsetbuttcap%
\pgfsetroundjoin%
\definecolor{currentfill}{rgb}{0.000000,0.000000,0.000000}%
\pgfsetfillcolor{currentfill}%
\pgfsetlinewidth{0.602250pt}%
\definecolor{currentstroke}{rgb}{0.000000,0.000000,0.000000}%
\pgfsetstrokecolor{currentstroke}%
\pgfsetdash{}{0pt}%
\pgfsys@defobject{currentmarker}{\pgfqpoint{-0.027778in}{0.000000in}}{\pgfqpoint{-0.000000in}{0.000000in}}{%
\pgfpathmoveto{\pgfqpoint{-0.000000in}{0.000000in}}%
\pgfpathlineto{\pgfqpoint{-0.027778in}{0.000000in}}%
\pgfusepath{stroke,fill}%
}%
\begin{pgfscope}%
\pgfsys@transformshift{0.588387in}{2.400350in}%
\pgfsys@useobject{currentmarker}{}%
\end{pgfscope}%
\end{pgfscope}%
\begin{pgfscope}%
\pgfsetbuttcap%
\pgfsetroundjoin%
\definecolor{currentfill}{rgb}{0.000000,0.000000,0.000000}%
\pgfsetfillcolor{currentfill}%
\pgfsetlinewidth{0.602250pt}%
\definecolor{currentstroke}{rgb}{0.000000,0.000000,0.000000}%
\pgfsetstrokecolor{currentstroke}%
\pgfsetdash{}{0pt}%
\pgfsys@defobject{currentmarker}{\pgfqpoint{-0.027778in}{0.000000in}}{\pgfqpoint{-0.000000in}{0.000000in}}{%
\pgfpathmoveto{\pgfqpoint{-0.000000in}{0.000000in}}%
\pgfpathlineto{\pgfqpoint{-0.027778in}{0.000000in}}%
\pgfusepath{stroke,fill}%
}%
\begin{pgfscope}%
\pgfsys@transformshift{0.588387in}{2.434827in}%
\pgfsys@useobject{currentmarker}{}%
\end{pgfscope}%
\end{pgfscope}%
\begin{pgfscope}%
\pgfsetbuttcap%
\pgfsetroundjoin%
\definecolor{currentfill}{rgb}{0.000000,0.000000,0.000000}%
\pgfsetfillcolor{currentfill}%
\pgfsetlinewidth{0.602250pt}%
\definecolor{currentstroke}{rgb}{0.000000,0.000000,0.000000}%
\pgfsetstrokecolor{currentstroke}%
\pgfsetdash{}{0pt}%
\pgfsys@defobject{currentmarker}{\pgfqpoint{-0.027778in}{0.000000in}}{\pgfqpoint{-0.000000in}{0.000000in}}{%
\pgfpathmoveto{\pgfqpoint{-0.000000in}{0.000000in}}%
\pgfpathlineto{\pgfqpoint{-0.027778in}{0.000000in}}%
\pgfusepath{stroke,fill}%
}%
\begin{pgfscope}%
\pgfsys@transformshift{0.588387in}{2.461570in}%
\pgfsys@useobject{currentmarker}{}%
\end{pgfscope}%
\end{pgfscope}%
\begin{pgfscope}%
\pgfsetbuttcap%
\pgfsetroundjoin%
\definecolor{currentfill}{rgb}{0.000000,0.000000,0.000000}%
\pgfsetfillcolor{currentfill}%
\pgfsetlinewidth{0.602250pt}%
\definecolor{currentstroke}{rgb}{0.000000,0.000000,0.000000}%
\pgfsetstrokecolor{currentstroke}%
\pgfsetdash{}{0pt}%
\pgfsys@defobject{currentmarker}{\pgfqpoint{-0.027778in}{0.000000in}}{\pgfqpoint{-0.000000in}{0.000000in}}{%
\pgfpathmoveto{\pgfqpoint{-0.000000in}{0.000000in}}%
\pgfpathlineto{\pgfqpoint{-0.027778in}{0.000000in}}%
\pgfusepath{stroke,fill}%
}%
\begin{pgfscope}%
\pgfsys@transformshift{0.588387in}{2.483420in}%
\pgfsys@useobject{currentmarker}{}%
\end{pgfscope}%
\end{pgfscope}%
\begin{pgfscope}%
\pgfsetbuttcap%
\pgfsetroundjoin%
\definecolor{currentfill}{rgb}{0.000000,0.000000,0.000000}%
\pgfsetfillcolor{currentfill}%
\pgfsetlinewidth{0.602250pt}%
\definecolor{currentstroke}{rgb}{0.000000,0.000000,0.000000}%
\pgfsetstrokecolor{currentstroke}%
\pgfsetdash{}{0pt}%
\pgfsys@defobject{currentmarker}{\pgfqpoint{-0.027778in}{0.000000in}}{\pgfqpoint{-0.000000in}{0.000000in}}{%
\pgfpathmoveto{\pgfqpoint{-0.000000in}{0.000000in}}%
\pgfpathlineto{\pgfqpoint{-0.027778in}{0.000000in}}%
\pgfusepath{stroke,fill}%
}%
\begin{pgfscope}%
\pgfsys@transformshift{0.588387in}{2.501894in}%
\pgfsys@useobject{currentmarker}{}%
\end{pgfscope}%
\end{pgfscope}%
\begin{pgfscope}%
\pgfsetbuttcap%
\pgfsetroundjoin%
\definecolor{currentfill}{rgb}{0.000000,0.000000,0.000000}%
\pgfsetfillcolor{currentfill}%
\pgfsetlinewidth{0.602250pt}%
\definecolor{currentstroke}{rgb}{0.000000,0.000000,0.000000}%
\pgfsetstrokecolor{currentstroke}%
\pgfsetdash{}{0pt}%
\pgfsys@defobject{currentmarker}{\pgfqpoint{-0.027778in}{0.000000in}}{\pgfqpoint{-0.000000in}{0.000000in}}{%
\pgfpathmoveto{\pgfqpoint{-0.000000in}{0.000000in}}%
\pgfpathlineto{\pgfqpoint{-0.027778in}{0.000000in}}%
\pgfusepath{stroke,fill}%
}%
\begin{pgfscope}%
\pgfsys@transformshift{0.588387in}{2.517897in}%
\pgfsys@useobject{currentmarker}{}%
\end{pgfscope}%
\end{pgfscope}%
\begin{pgfscope}%
\definecolor{textcolor}{rgb}{0.000000,0.000000,0.000000}%
\pgfsetstrokecolor{textcolor}%
\pgfsetfillcolor{textcolor}%
\pgftext[x=0.234413in,y=1.526746in,,bottom,rotate=90.000000]{\color{textcolor}{\rmfamily\fontsize{10.000000}{12.000000}\selectfont\catcode`\^=\active\def^{\ifmmode\sp\else\^{}\fi}\catcode`\%=\active\def%{\%}Checks [call]}}%
\end{pgfscope}%
\begin{pgfscope}%
\pgfpathrectangle{\pgfqpoint{0.588387in}{0.521603in}}{\pgfqpoint{5.399676in}{2.010285in}}%
\pgfusepath{clip}%
\pgfsetrectcap%
\pgfsetroundjoin%
\pgfsetlinewidth{1.505625pt}%
\pgfsetstrokecolor{currentstroke1}%
\pgfsetdash{}{0pt}%
\pgfpathmoveto{\pgfqpoint{0.833827in}{0.612980in}}%
\pgfpathlineto{\pgfqpoint{1.015634in}{0.744642in}}%
\pgfpathlineto{\pgfqpoint{1.197442in}{0.846186in}}%
\pgfpathlineto{\pgfqpoint{1.379249in}{0.937524in}}%
\pgfpathlineto{\pgfqpoint{1.561056in}{1.024523in}}%
\pgfpathlineto{\pgfqpoint{1.742863in}{1.109510in}}%
\pgfpathlineto{\pgfqpoint{1.924671in}{1.193527in}}%
\pgfpathlineto{\pgfqpoint{2.106478in}{1.277068in}}%
\pgfpathlineto{\pgfqpoint{2.288285in}{1.360372in}}%
\pgfpathlineto{\pgfqpoint{2.470092in}{1.443559in}}%
\pgfpathlineto{\pgfqpoint{2.651900in}{1.526687in}}%
\pgfpathlineto{\pgfqpoint{2.833707in}{1.609786in}}%
\pgfpathlineto{\pgfqpoint{3.015514in}{1.692870in}}%
\pgfpathlineto{\pgfqpoint{3.197321in}{1.775947in}}%
\pgfpathlineto{\pgfqpoint{3.379129in}{1.859021in}}%
\pgfpathlineto{\pgfqpoint{3.560936in}{1.942092in}}%
\pgfpathlineto{\pgfqpoint{3.742743in}{2.025163in}}%
\pgfpathlineto{\pgfqpoint{3.924551in}{2.108233in}}%
\pgfpathlineto{\pgfqpoint{4.106358in}{2.191303in}}%
\pgfpathlineto{\pgfqpoint{4.288165in}{2.274372in}}%
\pgfpathlineto{\pgfqpoint{4.469972in}{2.357442in}}%
\pgfpathlineto{\pgfqpoint{4.651780in}{2.440512in}}%
\pgfusepath{stroke}%
\end{pgfscope}%
\begin{pgfscope}%
\pgfpathrectangle{\pgfqpoint{0.588387in}{0.521603in}}{\pgfqpoint{5.399676in}{2.010285in}}%
\pgfusepath{clip}%
\pgfsetrectcap%
\pgfsetroundjoin%
\pgfsetlinewidth{1.505625pt}%
\pgfsetstrokecolor{currentstroke2}%
\pgfsetdash{}{0pt}%
\pgfpathmoveto{\pgfqpoint{0.833827in}{0.696050in}}%
\pgfpathlineto{\pgfqpoint{1.015634in}{0.779119in}}%
\pgfpathlineto{\pgfqpoint{1.197442in}{0.862189in}}%
\pgfpathlineto{\pgfqpoint{1.379249in}{0.945258in}}%
\pgfpathlineto{\pgfqpoint{1.561056in}{1.028328in}}%
\pgfpathlineto{\pgfqpoint{1.742863in}{1.111398in}}%
\pgfpathlineto{\pgfqpoint{1.924671in}{1.190871in}}%
\pgfpathlineto{\pgfqpoint{2.106478in}{1.272254in}}%
\pgfpathlineto{\pgfqpoint{2.288285in}{1.353538in}}%
\pgfpathlineto{\pgfqpoint{2.470092in}{1.432870in}}%
\pgfpathlineto{\pgfqpoint{2.651900in}{1.409495in}}%
\pgfpathlineto{\pgfqpoint{2.833707in}{1.457391in}}%
\pgfpathlineto{\pgfqpoint{3.015514in}{1.540050in}}%
\pgfpathlineto{\pgfqpoint{3.197321in}{1.574131in}}%
\pgfpathlineto{\pgfqpoint{3.379129in}{1.652532in}}%
\pgfpathlineto{\pgfqpoint{3.560936in}{1.674018in}}%
\pgfpathlineto{\pgfqpoint{3.742743in}{1.718390in}}%
\pgfpathlineto{\pgfqpoint{3.924551in}{1.742347in}}%
\pgfpathlineto{\pgfqpoint{4.106358in}{1.812228in}}%
\pgfpathlineto{\pgfqpoint{4.288165in}{1.875221in}}%
\pgfpathlineto{\pgfqpoint{4.469972in}{1.950711in}}%
\pgfpathlineto{\pgfqpoint{4.651780in}{1.949623in}}%
\pgfpathlineto{\pgfqpoint{4.833587in}{2.000987in}}%
\pgfpathlineto{\pgfqpoint{5.015394in}{1.967519in}}%
\pgfpathlineto{\pgfqpoint{5.197201in}{2.034384in}}%
\pgfpathlineto{\pgfqpoint{5.379009in}{2.039886in}}%
\pgfpathlineto{\pgfqpoint{5.560816in}{2.226789in}}%
\pgfpathlineto{\pgfqpoint{5.742623in}{2.188602in}}%
\pgfusepath{stroke}%
\end{pgfscope}%
\begin{pgfscope}%
\pgfpathrectangle{\pgfqpoint{0.588387in}{0.521603in}}{\pgfqpoint{5.399676in}{2.010285in}}%
\pgfusepath{clip}%
\pgfsetrectcap%
\pgfsetroundjoin%
\pgfsetlinewidth{1.505625pt}%
\pgfsetstrokecolor{currentstroke3}%
\pgfsetdash{}{0pt}%
\pgfpathmoveto{\pgfqpoint{0.833827in}{0.696050in}}%
\pgfpathlineto{\pgfqpoint{1.015634in}{0.779119in}}%
\pgfpathlineto{\pgfqpoint{1.197442in}{0.862189in}}%
\pgfpathlineto{\pgfqpoint{1.379249in}{0.945258in}}%
\pgfpathlineto{\pgfqpoint{1.561056in}{1.028328in}}%
\pgfpathlineto{\pgfqpoint{1.742863in}{1.111398in}}%
\pgfpathlineto{\pgfqpoint{1.924671in}{1.194467in}}%
\pgfpathlineto{\pgfqpoint{2.106478in}{1.277537in}}%
\pgfpathlineto{\pgfqpoint{2.288285in}{1.360607in}}%
\pgfpathlineto{\pgfqpoint{2.470092in}{1.443676in}}%
\pgfpathlineto{\pgfqpoint{2.651900in}{1.408457in}}%
\pgfpathlineto{\pgfqpoint{2.833707in}{1.468516in}}%
\pgfpathlineto{\pgfqpoint{3.015514in}{1.544337in}}%
\pgfpathlineto{\pgfqpoint{3.197321in}{1.570558in}}%
\pgfpathlineto{\pgfqpoint{3.379129in}{1.671607in}}%
\pgfpathlineto{\pgfqpoint{3.560936in}{1.707936in}}%
\pgfpathlineto{\pgfqpoint{3.742743in}{1.694656in}}%
\pgfpathlineto{\pgfqpoint{3.924551in}{1.721683in}}%
\pgfpathlineto{\pgfqpoint{4.106358in}{1.830062in}}%
\pgfpathlineto{\pgfqpoint{4.288165in}{1.839337in}}%
\pgfpathlineto{\pgfqpoint{4.469972in}{1.941976in}}%
\pgfpathlineto{\pgfqpoint{4.651780in}{1.921040in}}%
\pgfpathlineto{\pgfqpoint{4.833587in}{1.992218in}}%
\pgfpathlineto{\pgfqpoint{5.015394in}{1.921789in}}%
\pgfpathlineto{\pgfqpoint{5.197201in}{1.996697in}}%
\pgfpathlineto{\pgfqpoint{5.379009in}{2.016232in}}%
\pgfpathlineto{\pgfqpoint{5.560816in}{2.133847in}}%
\pgfpathlineto{\pgfqpoint{5.742623in}{2.120298in}}%
\pgfusepath{stroke}%
\end{pgfscope}%
\begin{pgfscope}%
\pgfpathrectangle{\pgfqpoint{0.588387in}{0.521603in}}{\pgfqpoint{5.399676in}{2.010285in}}%
\pgfusepath{clip}%
\pgfsetrectcap%
\pgfsetroundjoin%
\pgfsetlinewidth{1.505625pt}%
\pgfsetstrokecolor{currentstroke4}%
\pgfsetdash{}{0pt}%
\pgfpathmoveto{\pgfqpoint{0.833827in}{0.696050in}}%
\pgfpathlineto{\pgfqpoint{1.015634in}{0.779119in}}%
\pgfpathlineto{\pgfqpoint{1.197442in}{0.862189in}}%
\pgfpathlineto{\pgfqpoint{1.379249in}{0.945258in}}%
\pgfpathlineto{\pgfqpoint{1.561056in}{1.028328in}}%
\pgfpathlineto{\pgfqpoint{1.742863in}{1.111398in}}%
\pgfpathlineto{\pgfqpoint{1.924671in}{1.190975in}}%
\pgfpathlineto{\pgfqpoint{2.106478in}{1.272118in}}%
\pgfpathlineto{\pgfqpoint{2.288285in}{1.352368in}}%
\pgfpathlineto{\pgfqpoint{2.470092in}{1.432780in}}%
\pgfpathlineto{\pgfqpoint{2.651900in}{1.425431in}}%
\pgfpathlineto{\pgfqpoint{2.833707in}{1.462049in}}%
\pgfpathlineto{\pgfqpoint{3.015514in}{1.540076in}}%
\pgfpathlineto{\pgfqpoint{3.197321in}{1.573856in}}%
\pgfpathlineto{\pgfqpoint{3.379129in}{1.651819in}}%
\pgfpathlineto{\pgfqpoint{3.560936in}{1.689050in}}%
\pgfpathlineto{\pgfqpoint{3.742743in}{1.742599in}}%
\pgfpathlineto{\pgfqpoint{3.924551in}{1.782890in}}%
\pgfpathlineto{\pgfqpoint{4.106358in}{1.884255in}}%
\pgfpathlineto{\pgfqpoint{4.288165in}{1.920574in}}%
\pgfpathlineto{\pgfqpoint{4.651780in}{1.980516in}}%
\pgfpathlineto{\pgfqpoint{4.833587in}{2.009595in}}%
\pgfpathlineto{\pgfqpoint{5.015394in}{1.994231in}}%
\pgfpathlineto{\pgfqpoint{5.379009in}{2.125265in}}%
\pgfusepath{stroke}%
\end{pgfscope}%
\begin{pgfscope}%
\pgfpathrectangle{\pgfqpoint{0.588387in}{0.521603in}}{\pgfqpoint{5.399676in}{2.010285in}}%
\pgfusepath{clip}%
\pgfsetrectcap%
\pgfsetroundjoin%
\pgfsetlinewidth{1.505625pt}%
\pgfsetstrokecolor{currentstroke5}%
\pgfsetdash{}{0pt}%
\pgfpathmoveto{\pgfqpoint{0.833827in}{0.696050in}}%
\pgfpathlineto{\pgfqpoint{1.015634in}{0.779119in}}%
\pgfpathlineto{\pgfqpoint{1.197442in}{0.862189in}}%
\pgfpathlineto{\pgfqpoint{1.379249in}{0.945258in}}%
\pgfpathlineto{\pgfqpoint{1.561056in}{1.028328in}}%
\pgfpathlineto{\pgfqpoint{1.742863in}{1.111398in}}%
\pgfpathlineto{\pgfqpoint{1.924671in}{1.193604in}}%
\pgfpathlineto{\pgfqpoint{2.106478in}{1.277896in}}%
\pgfpathlineto{\pgfqpoint{2.288285in}{1.358992in}}%
\pgfpathlineto{\pgfqpoint{2.470092in}{1.437994in}}%
\pgfpathlineto{\pgfqpoint{2.651900in}{1.480213in}}%
\pgfpathlineto{\pgfqpoint{2.833707in}{1.532936in}}%
\pgfpathlineto{\pgfqpoint{3.015514in}{1.621742in}}%
\pgfpathlineto{\pgfqpoint{3.197321in}{1.674502in}}%
\pgfpathlineto{\pgfqpoint{3.379129in}{1.755334in}}%
\pgfpathlineto{\pgfqpoint{3.560936in}{1.814411in}}%
\pgfpathlineto{\pgfqpoint{3.742743in}{1.831133in}}%
\pgfpathlineto{\pgfqpoint{3.924551in}{1.874157in}}%
\pgfpathlineto{\pgfqpoint{4.106358in}{1.977624in}}%
\pgfpathlineto{\pgfqpoint{4.288165in}{2.006365in}}%
\pgfpathlineto{\pgfqpoint{4.469972in}{2.131768in}}%
\pgfpathlineto{\pgfqpoint{4.651780in}{2.107317in}}%
\pgfpathlineto{\pgfqpoint{4.833587in}{2.090411in}}%
\pgfpathlineto{\pgfqpoint{5.015394in}{2.152189in}}%
\pgfpathlineto{\pgfqpoint{5.197201in}{2.190000in}}%
\pgfpathlineto{\pgfqpoint{5.379009in}{2.167840in}}%
\pgfpathlineto{\pgfqpoint{5.742623in}{2.254665in}}%
\pgfusepath{stroke}%
\end{pgfscope}%
\begin{pgfscope}%
\pgfpathrectangle{\pgfqpoint{0.588387in}{0.521603in}}{\pgfqpoint{5.399676in}{2.010285in}}%
\pgfusepath{clip}%
\pgfsetrectcap%
\pgfsetroundjoin%
\pgfsetlinewidth{1.505625pt}%
\pgfsetstrokecolor{currentstroke6}%
\pgfsetdash{}{0pt}%
\pgfpathmoveto{\pgfqpoint{0.833827in}{0.696050in}}%
\pgfpathlineto{\pgfqpoint{1.015634in}{0.779119in}}%
\pgfpathlineto{\pgfqpoint{1.197442in}{0.862189in}}%
\pgfpathlineto{\pgfqpoint{1.379249in}{0.945258in}}%
\pgfpathlineto{\pgfqpoint{1.561056in}{1.028328in}}%
\pgfpathlineto{\pgfqpoint{1.742863in}{1.111398in}}%
\pgfpathlineto{\pgfqpoint{1.924671in}{1.194467in}}%
\pgfpathlineto{\pgfqpoint{2.106478in}{1.277537in}}%
\pgfpathlineto{\pgfqpoint{2.288285in}{1.360607in}}%
\pgfpathlineto{\pgfqpoint{2.470092in}{1.443676in}}%
\pgfpathlineto{\pgfqpoint{2.651900in}{1.482269in}}%
\pgfpathlineto{\pgfqpoint{2.833707in}{1.540494in}}%
\pgfpathlineto{\pgfqpoint{3.015514in}{1.628739in}}%
\pgfpathlineto{\pgfqpoint{3.197321in}{1.681559in}}%
\pgfpathlineto{\pgfqpoint{3.379129in}{1.768891in}}%
\pgfpathlineto{\pgfqpoint{3.560936in}{1.830739in}}%
\pgfpathlineto{\pgfqpoint{3.742743in}{1.808037in}}%
\pgfpathlineto{\pgfqpoint{3.924551in}{1.843636in}}%
\pgfpathlineto{\pgfqpoint{4.106358in}{1.952992in}}%
\pgfpathlineto{\pgfqpoint{4.288165in}{1.962593in}}%
\pgfpathlineto{\pgfqpoint{4.469972in}{2.036122in}}%
\pgfpathlineto{\pgfqpoint{4.651780in}{2.097894in}}%
\pgfpathlineto{\pgfqpoint{4.833587in}{2.055922in}}%
\pgfpathlineto{\pgfqpoint{5.015394in}{2.123077in}}%
\pgfpathlineto{\pgfqpoint{5.197201in}{2.226982in}}%
\pgfpathlineto{\pgfqpoint{5.379009in}{2.121261in}}%
\pgfpathlineto{\pgfqpoint{5.560816in}{2.150333in}}%
\pgfpathlineto{\pgfqpoint{5.742623in}{2.254665in}}%
\pgfusepath{stroke}%
\end{pgfscope}%
\begin{pgfscope}%
\pgfsetrectcap%
\pgfsetmiterjoin%
\pgfsetlinewidth{0.803000pt}%
\definecolor{currentstroke}{rgb}{0.000000,0.000000,0.000000}%
\pgfsetstrokecolor{currentstroke}%
\pgfsetdash{}{0pt}%
\pgfpathmoveto{\pgfqpoint{0.588387in}{0.521603in}}%
\pgfpathlineto{\pgfqpoint{0.588387in}{2.531888in}}%
\pgfusepath{stroke}%
\end{pgfscope}%
\begin{pgfscope}%
\pgfsetrectcap%
\pgfsetmiterjoin%
\pgfsetlinewidth{0.803000pt}%
\definecolor{currentstroke}{rgb}{0.000000,0.000000,0.000000}%
\pgfsetstrokecolor{currentstroke}%
\pgfsetdash{}{0pt}%
\pgfpathmoveto{\pgfqpoint{5.988063in}{0.521603in}}%
\pgfpathlineto{\pgfqpoint{5.988063in}{2.531888in}}%
\pgfusepath{stroke}%
\end{pgfscope}%
\begin{pgfscope}%
\pgfsetrectcap%
\pgfsetmiterjoin%
\pgfsetlinewidth{0.803000pt}%
\definecolor{currentstroke}{rgb}{0.000000,0.000000,0.000000}%
\pgfsetstrokecolor{currentstroke}%
\pgfsetdash{}{0pt}%
\pgfpathmoveto{\pgfqpoint{0.588387in}{0.521603in}}%
\pgfpathlineto{\pgfqpoint{5.988063in}{0.521603in}}%
\pgfusepath{stroke}%
\end{pgfscope}%
\begin{pgfscope}%
\pgfsetrectcap%
\pgfsetmiterjoin%
\pgfsetlinewidth{0.803000pt}%
\definecolor{currentstroke}{rgb}{0.000000,0.000000,0.000000}%
\pgfsetstrokecolor{currentstroke}%
\pgfsetdash{}{0pt}%
\pgfpathmoveto{\pgfqpoint{0.588387in}{2.531888in}}%
\pgfpathlineto{\pgfqpoint{5.988063in}{2.531888in}}%
\pgfusepath{stroke}%
\end{pgfscope}%
\begin{pgfscope}%
\definecolor{textcolor}{rgb}{0.000000,0.000000,0.000000}%
\pgfsetstrokecolor{textcolor}%
\pgfsetfillcolor{textcolor}%
\pgftext[x=3.288225in,y=2.615222in,,base]{\color{textcolor}{\rmfamily\fontsize{12.000000}{14.400000}\selectfont\catcode`\^=\active\def^{\ifmmode\sp\else\^{}\fi}\catcode`\%=\active\def%{\%}Mean}}%
\end{pgfscope}%
\begin{pgfscope}%
\pgfsetbuttcap%
\pgfsetmiterjoin%
\definecolor{currentfill}{rgb}{1.000000,1.000000,1.000000}%
\pgfsetfillcolor{currentfill}%
\pgfsetfillopacity{0.800000}%
\pgfsetlinewidth{1.003750pt}%
\definecolor{currentstroke}{rgb}{0.800000,0.800000,0.800000}%
\pgfsetstrokecolor{currentstroke}%
\pgfsetstrokeopacity{0.800000}%
\pgfsetdash{}{0pt}%
\pgfpathmoveto{\pgfqpoint{6.075563in}{1.320622in}}%
\pgfpathlineto{\pgfqpoint{8.259376in}{1.320622in}}%
\pgfpathquadraticcurveto{\pgfqpoint{8.284376in}{1.320622in}}{\pgfqpoint{8.284376in}{1.345622in}}%
\pgfpathlineto{\pgfqpoint{8.284376in}{2.444388in}}%
\pgfpathquadraticcurveto{\pgfqpoint{8.284376in}{2.469388in}}{\pgfqpoint{8.259376in}{2.469388in}}%
\pgfpathlineto{\pgfqpoint{6.075563in}{2.469388in}}%
\pgfpathquadraticcurveto{\pgfqpoint{6.050563in}{2.469388in}}{\pgfqpoint{6.050563in}{2.444388in}}%
\pgfpathlineto{\pgfqpoint{6.050563in}{1.345622in}}%
\pgfpathquadraticcurveto{\pgfqpoint{6.050563in}{1.320622in}}{\pgfqpoint{6.075563in}{1.320622in}}%
\pgfpathlineto{\pgfqpoint{6.075563in}{1.320622in}}%
\pgfpathclose%
\pgfusepath{stroke,fill}%
\end{pgfscope}%
\begin{pgfscope}%
\pgfsetrectcap%
\pgfsetroundjoin%
\pgfsetlinewidth{1.505625pt}%
\pgfsetstrokecolor{currentstroke1}%
\pgfsetdash{}{0pt}%
\pgfpathmoveto{\pgfqpoint{6.100563in}{2.368168in}}%
\pgfpathlineto{\pgfqpoint{6.225563in}{2.368168in}}%
\pgfpathlineto{\pgfqpoint{6.350563in}{2.368168in}}%
\pgfusepath{stroke}%
\end{pgfscope}%
\begin{pgfscope}%
\definecolor{textcolor}{rgb}{0.000000,0.000000,0.000000}%
\pgfsetstrokecolor{textcolor}%
\pgfsetfillcolor{textcolor}%
\pgftext[x=6.450563in,y=2.324418in,left,base]{\color{textcolor}{\rmfamily\fontsize{9.000000}{10.800000}\selectfont\catcode`\^=\active\def^{\ifmmode\sp\else\^{}\fi}\catcode`\%=\active\def%{\%}\NaiveCycles{}}}%
\end{pgfscope}%
\begin{pgfscope}%
\pgfsetrectcap%
\pgfsetroundjoin%
\pgfsetlinewidth{1.505625pt}%
\pgfsetstrokecolor{currentstroke2}%
\pgfsetdash{}{0pt}%
\pgfpathmoveto{\pgfqpoint{6.100563in}{2.184696in}}%
\pgfpathlineto{\pgfqpoint{6.225563in}{2.184696in}}%
\pgfpathlineto{\pgfqpoint{6.350563in}{2.184696in}}%
\pgfusepath{stroke}%
\end{pgfscope}%
\begin{pgfscope}%
\definecolor{textcolor}{rgb}{0.000000,0.000000,0.000000}%
\pgfsetstrokecolor{textcolor}%
\pgfsetfillcolor{textcolor}%
\pgftext[x=6.450563in,y=2.140946in,left,base]{\color{textcolor}{\rmfamily\fontsize{9.000000}{10.800000}\selectfont\catcode`\^=\active\def^{\ifmmode\sp\else\^{}\fi}\catcode`\%=\active\def%{\%}\Neighbors{} \& \MergeLinear{}}}%
\end{pgfscope}%
\begin{pgfscope}%
\pgfsetrectcap%
\pgfsetroundjoin%
\pgfsetlinewidth{1.505625pt}%
\pgfsetstrokecolor{currentstroke3}%
\pgfsetdash{}{0pt}%
\pgfpathmoveto{\pgfqpoint{6.100563in}{2.001225in}}%
\pgfpathlineto{\pgfqpoint{6.225563in}{2.001225in}}%
\pgfpathlineto{\pgfqpoint{6.350563in}{2.001225in}}%
\pgfusepath{stroke}%
\end{pgfscope}%
\begin{pgfscope}%
\definecolor{textcolor}{rgb}{0.000000,0.000000,0.000000}%
\pgfsetstrokecolor{textcolor}%
\pgfsetfillcolor{textcolor}%
\pgftext[x=6.450563in,y=1.957475in,left,base]{\color{textcolor}{\rmfamily\fontsize{9.000000}{10.800000}\selectfont\catcode`\^=\active\def^{\ifmmode\sp\else\^{}\fi}\catcode`\%=\active\def%{\%}\Neighbors{} \& \SharedVertices{}}}%
\end{pgfscope}%
\begin{pgfscope}%
\pgfsetrectcap%
\pgfsetroundjoin%
\pgfsetlinewidth{1.505625pt}%
\pgfsetstrokecolor{currentstroke4}%
\pgfsetdash{}{0pt}%
\pgfpathmoveto{\pgfqpoint{6.100563in}{1.814274in}}%
\pgfpathlineto{\pgfqpoint{6.225563in}{1.814274in}}%
\pgfpathlineto{\pgfqpoint{6.350563in}{1.814274in}}%
\pgfusepath{stroke}%
\end{pgfscope}%
\begin{pgfscope}%
\definecolor{textcolor}{rgb}{0.000000,0.000000,0.000000}%
\pgfsetstrokecolor{textcolor}%
\pgfsetfillcolor{textcolor}%
\pgftext[x=6.450563in,y=1.770524in,left,base]{\color{textcolor}{\rmfamily\fontsize{9.000000}{10.800000}\selectfont\catcode`\^=\active\def^{\ifmmode\sp\else\^{}\fi}\catcode`\%=\active\def%{\%}\NeighborsDegree{} \& \MergeLinear{}}}%
\end{pgfscope}%
\begin{pgfscope}%
\pgfsetrectcap%
\pgfsetroundjoin%
\pgfsetlinewidth{1.505625pt}%
\pgfsetstrokecolor{currentstroke5}%
\pgfsetdash{}{0pt}%
\pgfpathmoveto{\pgfqpoint{6.100563in}{1.627324in}}%
\pgfpathlineto{\pgfqpoint{6.225563in}{1.627324in}}%
\pgfpathlineto{\pgfqpoint{6.350563in}{1.627324in}}%
\pgfusepath{stroke}%
\end{pgfscope}%
\begin{pgfscope}%
\definecolor{textcolor}{rgb}{0.000000,0.000000,0.000000}%
\pgfsetstrokecolor{textcolor}%
\pgfsetfillcolor{textcolor}%
\pgftext[x=6.450563in,y=1.583574in,left,base]{\color{textcolor}{\rmfamily\fontsize{9.000000}{10.800000}\selectfont\catcode`\^=\active\def^{\ifmmode\sp\else\^{}\fi}\catcode`\%=\active\def%{\%}\None{} \& \MergeLinear{}}}%
\end{pgfscope}%
\begin{pgfscope}%
\pgfsetrectcap%
\pgfsetroundjoin%
\pgfsetlinewidth{1.505625pt}%
\pgfsetstrokecolor{currentstroke6}%
\pgfsetdash{}{0pt}%
\pgfpathmoveto{\pgfqpoint{6.100563in}{1.443852in}}%
\pgfpathlineto{\pgfqpoint{6.225563in}{1.443852in}}%
\pgfpathlineto{\pgfqpoint{6.350563in}{1.443852in}}%
\pgfusepath{stroke}%
\end{pgfscope}%
\begin{pgfscope}%
\definecolor{textcolor}{rgb}{0.000000,0.000000,0.000000}%
\pgfsetstrokecolor{textcolor}%
\pgfsetfillcolor{textcolor}%
\pgftext[x=6.450563in,y=1.400102in,left,base]{\color{textcolor}{\rmfamily\fontsize{9.000000}{10.800000}\selectfont\catcode`\^=\active\def^{\ifmmode\sp\else\^{}\fi}\catcode`\%=\active\def%{\%}\None{} \& \SharedVertices{}}}%
\end{pgfscope}%
\end{pgfpicture}%
\makeatother%
\endgroup%
}
	\caption[Checks performed for minimally rigid graphs.]{
		The number of checks performed to find all NAC-colorings for minimally rigid graphs.}%
	\label{fig:graph_count_minimally_rigid}
\end{figure}

In \Cref{fig:graph_summary}
we show the relation between the number of \IsNACColoring{} checks that
would \Naive{} algorithm perform compared to our solution.
The values are similar for graphs with few monochromatic classes,
which explains why the \NaiveCycles{} algorithm outperformed
the \NeighborsDegree{}\&\MergeLinear{} algorithm in \Cref{tab:all_min_rigid}. This should improve quickly for larger graphs.
We can also see how the use of \CycleMask{} routine
reduces the number of more expensive \IsNACColoring{} calls,
since these are called only when the small cycles check \CycleMask{} passes
(\CycleMask{} is called every time).

\begin{figure}[ht]
	\centering
	\scalebox{0.5}{%% Creator: Matplotlib, PGF backend
%%
%% To include the figure in your LaTeX document, write
%%   \input{<filename>.pgf}
%%
%% Make sure the required packages are loaded in your preamble
%%   \usepackage{pgf}
%%
%% Also ensure that all the required font packages are loaded; for instance,
%% the lmodern package is sometimes necessary when using math font.
%%   \usepackage{lmodern}
%%
%% Figures using additional raster images can only be included by \input if
%% they are in the same directory as the main LaTeX file. For loading figures
%% from other directories you can use the `import` package
%%   \usepackage{import}
%%
%% and then include the figures with
%%   \import{<path to file>}{<filename>.pgf}
%%
%% Matplotlib used the following preamble
%%   \def\mathdefault#1{#1}
%%   \everymath=\expandafter{\the\everymath\displaystyle}
%%   \IfFileExists{scrextend.sty}{
%%     \usepackage[fontsize=10.000000pt]{scrextend}
%%   }{
%%     \renewcommand{\normalsize}{\fontsize{10.000000}{12.000000}\selectfont}
%%     \normalsize
%%   }
%%   
%%   \ifdefined\pdftexversion\else  % non-pdftex case.
%%     \usepackage{fontspec}
%%     \setmainfont{DejaVuSans.ttf}[Path=\detokenize{/home/petr/Projects/PyRigi/.venv/lib/python3.12/site-packages/matplotlib/mpl-data/fonts/ttf/}]
%%     \setsansfont{DejaVuSans.ttf}[Path=\detokenize{/home/petr/Projects/PyRigi/.venv/lib/python3.12/site-packages/matplotlib/mpl-data/fonts/ttf/}]
%%     \setmonofont{DejaVuSansMono.ttf}[Path=\detokenize{/home/petr/Projects/PyRigi/.venv/lib/python3.12/site-packages/matplotlib/mpl-data/fonts/ttf/}]
%%   \fi
%%   \makeatletter\@ifpackageloaded{under\Score{}}{}{\usepackage[strings]{under\Score{}}}\makeatother
%%
\begingroup%
\makeatletter%
\begin{pgfpicture}%
\pgfpathrectangle{\pgfpointorigin}{\pgfqpoint{8.384376in}{2.841849in}}%
\pgfusepath{use as bounding box, clip}%
\begin{pgfscope}%
\pgfsetbuttcap%
\pgfsetmiterjoin%
\definecolor{currentfill}{rgb}{1.000000,1.000000,1.000000}%
\pgfsetfillcolor{currentfill}%
\pgfsetlinewidth{0.000000pt}%
\definecolor{currentstroke}{rgb}{1.000000,1.000000,1.000000}%
\pgfsetstrokecolor{currentstroke}%
\pgfsetdash{}{0pt}%
\pgfpathmoveto{\pgfqpoint{0.000000in}{0.000000in}}%
\pgfpathlineto{\pgfqpoint{8.384376in}{0.000000in}}%
\pgfpathlineto{\pgfqpoint{8.384376in}{2.841849in}}%
\pgfpathlineto{\pgfqpoint{0.000000in}{2.841849in}}%
\pgfpathlineto{\pgfqpoint{0.000000in}{0.000000in}}%
\pgfpathclose%
\pgfusepath{fill}%
\end{pgfscope}%
\begin{pgfscope}%
\pgfsetbuttcap%
\pgfsetmiterjoin%
\definecolor{currentfill}{rgb}{1.000000,1.000000,1.000000}%
\pgfsetfillcolor{currentfill}%
\pgfsetlinewidth{0.000000pt}%
\definecolor{currentstroke}{rgb}{0.000000,0.000000,0.000000}%
\pgfsetstrokecolor{currentstroke}%
\pgfsetstrokeopacity{0.000000}%
\pgfsetdash{}{0pt}%
\pgfpathmoveto{\pgfqpoint{0.643750in}{0.521603in}}%
\pgfpathlineto{\pgfqpoint{8.284376in}{0.521603in}}%
\pgfpathlineto{\pgfqpoint{8.284376in}{2.741849in}}%
\pgfpathlineto{\pgfqpoint{0.643750in}{2.741849in}}%
\pgfpathlineto{\pgfqpoint{0.643750in}{0.521603in}}%
\pgfpathclose%
\pgfusepath{fill}%
\end{pgfscope}%
\begin{pgfscope}%
\pgfsetbuttcap%
\pgfsetroundjoin%
\definecolor{currentfill}{rgb}{0.000000,0.000000,0.000000}%
\pgfsetfillcolor{currentfill}%
\pgfsetlinewidth{0.803000pt}%
\definecolor{currentstroke}{rgb}{0.000000,0.000000,0.000000}%
\pgfsetstrokecolor{currentstroke}%
\pgfsetdash{}{0pt}%
\pgfsys@defobject{currentmarker}{\pgfqpoint{0.000000in}{-0.048611in}}{\pgfqpoint{0.000000in}{0.000000in}}{%
\pgfpathmoveto{\pgfqpoint{0.000000in}{0.000000in}}%
\pgfpathlineto{\pgfqpoint{0.000000in}{-0.048611in}}%
\pgfusepath{stroke,fill}%
}%
\begin{pgfscope}%
\pgfsys@transformshift{1.425178in}{0.521603in}%
\pgfsys@useobject{currentmarker}{}%
\end{pgfscope}%
\end{pgfscope}%
\begin{pgfscope}%
\definecolor{textcolor}{rgb}{0.000000,0.000000,0.000000}%
\pgfsetstrokecolor{textcolor}%
\pgfsetfillcolor{textcolor}%
\pgftext[x=1.425178in,y=0.424381in,,top]{\color{textcolor}{\rmfamily\fontsize{10.000000}{12.000000}\selectfont\catcode`\^=\active\def^{\ifmmode\sp\else\^{}\fi}\catcode`\%=\active\def%{\%}$\mathdefault{4}$}}%
\end{pgfscope}%
\begin{pgfscope}%
\pgfsetbuttcap%
\pgfsetroundjoin%
\definecolor{currentfill}{rgb}{0.000000,0.000000,0.000000}%
\pgfsetfillcolor{currentfill}%
\pgfsetlinewidth{0.803000pt}%
\definecolor{currentstroke}{rgb}{0.000000,0.000000,0.000000}%
\pgfsetstrokecolor{currentstroke}%
\pgfsetdash{}{0pt}%
\pgfsys@defobject{currentmarker}{\pgfqpoint{0.000000in}{-0.048611in}}{\pgfqpoint{0.000000in}{0.000000in}}{%
\pgfpathmoveto{\pgfqpoint{0.000000in}{0.000000in}}%
\pgfpathlineto{\pgfqpoint{0.000000in}{-0.048611in}}%
\pgfusepath{stroke,fill}%
}%
\begin{pgfscope}%
\pgfsys@transformshift{2.293431in}{0.521603in}%
\pgfsys@useobject{currentmarker}{}%
\end{pgfscope}%
\end{pgfscope}%
\begin{pgfscope}%
\definecolor{textcolor}{rgb}{0.000000,0.000000,0.000000}%
\pgfsetstrokecolor{textcolor}%
\pgfsetfillcolor{textcolor}%
\pgftext[x=2.293431in,y=0.424381in,,top]{\color{textcolor}{\rmfamily\fontsize{10.000000}{12.000000}\selectfont\catcode`\^=\active\def^{\ifmmode\sp\else\^{}\fi}\catcode`\%=\active\def%{\%}$\mathdefault{8}$}}%
\end{pgfscope}%
\begin{pgfscope}%
\pgfsetbuttcap%
\pgfsetroundjoin%
\definecolor{currentfill}{rgb}{0.000000,0.000000,0.000000}%
\pgfsetfillcolor{currentfill}%
\pgfsetlinewidth{0.803000pt}%
\definecolor{currentstroke}{rgb}{0.000000,0.000000,0.000000}%
\pgfsetstrokecolor{currentstroke}%
\pgfsetdash{}{0pt}%
\pgfsys@defobject{currentmarker}{\pgfqpoint{0.000000in}{-0.048611in}}{\pgfqpoint{0.000000in}{0.000000in}}{%
\pgfpathmoveto{\pgfqpoint{0.000000in}{0.000000in}}%
\pgfpathlineto{\pgfqpoint{0.000000in}{-0.048611in}}%
\pgfusepath{stroke,fill}%
}%
\begin{pgfscope}%
\pgfsys@transformshift{3.161684in}{0.521603in}%
\pgfsys@useobject{currentmarker}{}%
\end{pgfscope}%
\end{pgfscope}%
\begin{pgfscope}%
\definecolor{textcolor}{rgb}{0.000000,0.000000,0.000000}%
\pgfsetstrokecolor{textcolor}%
\pgfsetfillcolor{textcolor}%
\pgftext[x=3.161684in,y=0.424381in,,top]{\color{textcolor}{\rmfamily\fontsize{10.000000}{12.000000}\selectfont\catcode`\^=\active\def^{\ifmmode\sp\else\^{}\fi}\catcode`\%=\active\def%{\%}$\mathdefault{12}$}}%
\end{pgfscope}%
\begin{pgfscope}%
\pgfsetbuttcap%
\pgfsetroundjoin%
\definecolor{currentfill}{rgb}{0.000000,0.000000,0.000000}%
\pgfsetfillcolor{currentfill}%
\pgfsetlinewidth{0.803000pt}%
\definecolor{currentstroke}{rgb}{0.000000,0.000000,0.000000}%
\pgfsetstrokecolor{currentstroke}%
\pgfsetdash{}{0pt}%
\pgfsys@defobject{currentmarker}{\pgfqpoint{0.000000in}{-0.048611in}}{\pgfqpoint{0.000000in}{0.000000in}}{%
\pgfpathmoveto{\pgfqpoint{0.000000in}{0.000000in}}%
\pgfpathlineto{\pgfqpoint{0.000000in}{-0.048611in}}%
\pgfusepath{stroke,fill}%
}%
\begin{pgfscope}%
\pgfsys@transformshift{4.029937in}{0.521603in}%
\pgfsys@useobject{currentmarker}{}%
\end{pgfscope}%
\end{pgfscope}%
\begin{pgfscope}%
\definecolor{textcolor}{rgb}{0.000000,0.000000,0.000000}%
\pgfsetstrokecolor{textcolor}%
\pgfsetfillcolor{textcolor}%
\pgftext[x=4.029937in,y=0.424381in,,top]{\color{textcolor}{\rmfamily\fontsize{10.000000}{12.000000}\selectfont\catcode`\^=\active\def^{\ifmmode\sp\else\^{}\fi}\catcode`\%=\active\def%{\%}$\mathdefault{16}$}}%
\end{pgfscope}%
\begin{pgfscope}%
\pgfsetbuttcap%
\pgfsetroundjoin%
\definecolor{currentfill}{rgb}{0.000000,0.000000,0.000000}%
\pgfsetfillcolor{currentfill}%
\pgfsetlinewidth{0.803000pt}%
\definecolor{currentstroke}{rgb}{0.000000,0.000000,0.000000}%
\pgfsetstrokecolor{currentstroke}%
\pgfsetdash{}{0pt}%
\pgfsys@defobject{currentmarker}{\pgfqpoint{0.000000in}{-0.048611in}}{\pgfqpoint{0.000000in}{0.000000in}}{%
\pgfpathmoveto{\pgfqpoint{0.000000in}{0.000000in}}%
\pgfpathlineto{\pgfqpoint{0.000000in}{-0.048611in}}%
\pgfusepath{stroke,fill}%
}%
\begin{pgfscope}%
\pgfsys@transformshift{4.898190in}{0.521603in}%
\pgfsys@useobject{currentmarker}{}%
\end{pgfscope}%
\end{pgfscope}%
\begin{pgfscope}%
\definecolor{textcolor}{rgb}{0.000000,0.000000,0.000000}%
\pgfsetstrokecolor{textcolor}%
\pgfsetfillcolor{textcolor}%
\pgftext[x=4.898190in,y=0.424381in,,top]{\color{textcolor}{\rmfamily\fontsize{10.000000}{12.000000}\selectfont\catcode`\^=\active\def^{\ifmmode\sp\else\^{}\fi}\catcode`\%=\active\def%{\%}$\mathdefault{20}$}}%
\end{pgfscope}%
\begin{pgfscope}%
\pgfsetbuttcap%
\pgfsetroundjoin%
\definecolor{currentfill}{rgb}{0.000000,0.000000,0.000000}%
\pgfsetfillcolor{currentfill}%
\pgfsetlinewidth{0.803000pt}%
\definecolor{currentstroke}{rgb}{0.000000,0.000000,0.000000}%
\pgfsetstrokecolor{currentstroke}%
\pgfsetdash{}{0pt}%
\pgfsys@defobject{currentmarker}{\pgfqpoint{0.000000in}{-0.048611in}}{\pgfqpoint{0.000000in}{0.000000in}}{%
\pgfpathmoveto{\pgfqpoint{0.000000in}{0.000000in}}%
\pgfpathlineto{\pgfqpoint{0.000000in}{-0.048611in}}%
\pgfusepath{stroke,fill}%
}%
\begin{pgfscope}%
\pgfsys@transformshift{5.766443in}{0.521603in}%
\pgfsys@useobject{currentmarker}{}%
\end{pgfscope}%
\end{pgfscope}%
\begin{pgfscope}%
\definecolor{textcolor}{rgb}{0.000000,0.000000,0.000000}%
\pgfsetstrokecolor{textcolor}%
\pgfsetfillcolor{textcolor}%
\pgftext[x=5.766443in,y=0.424381in,,top]{\color{textcolor}{\rmfamily\fontsize{10.000000}{12.000000}\selectfont\catcode`\^=\active\def^{\ifmmode\sp\else\^{}\fi}\catcode`\%=\active\def%{\%}$\mathdefault{24}$}}%
\end{pgfscope}%
\begin{pgfscope}%
\pgfsetbuttcap%
\pgfsetroundjoin%
\definecolor{currentfill}{rgb}{0.000000,0.000000,0.000000}%
\pgfsetfillcolor{currentfill}%
\pgfsetlinewidth{0.803000pt}%
\definecolor{currentstroke}{rgb}{0.000000,0.000000,0.000000}%
\pgfsetstrokecolor{currentstroke}%
\pgfsetdash{}{0pt}%
\pgfsys@defobject{currentmarker}{\pgfqpoint{0.000000in}{-0.048611in}}{\pgfqpoint{0.000000in}{0.000000in}}{%
\pgfpathmoveto{\pgfqpoint{0.000000in}{0.000000in}}%
\pgfpathlineto{\pgfqpoint{0.000000in}{-0.048611in}}%
\pgfusepath{stroke,fill}%
}%
\begin{pgfscope}%
\pgfsys@transformshift{6.634696in}{0.521603in}%
\pgfsys@useobject{currentmarker}{}%
\end{pgfscope}%
\end{pgfscope}%
\begin{pgfscope}%
\definecolor{textcolor}{rgb}{0.000000,0.000000,0.000000}%
\pgfsetstrokecolor{textcolor}%
\pgfsetfillcolor{textcolor}%
\pgftext[x=6.634696in,y=0.424381in,,top]{\color{textcolor}{\rmfamily\fontsize{10.000000}{12.000000}\selectfont\catcode`\^=\active\def^{\ifmmode\sp\else\^{}\fi}\catcode`\%=\active\def%{\%}$\mathdefault{28}$}}%
\end{pgfscope}%
\begin{pgfscope}%
\pgfsetbuttcap%
\pgfsetroundjoin%
\definecolor{currentfill}{rgb}{0.000000,0.000000,0.000000}%
\pgfsetfillcolor{currentfill}%
\pgfsetlinewidth{0.803000pt}%
\definecolor{currentstroke}{rgb}{0.000000,0.000000,0.000000}%
\pgfsetstrokecolor{currentstroke}%
\pgfsetdash{}{0pt}%
\pgfsys@defobject{currentmarker}{\pgfqpoint{0.000000in}{-0.048611in}}{\pgfqpoint{0.000000in}{0.000000in}}{%
\pgfpathmoveto{\pgfqpoint{0.000000in}{0.000000in}}%
\pgfpathlineto{\pgfqpoint{0.000000in}{-0.048611in}}%
\pgfusepath{stroke,fill}%
}%
\begin{pgfscope}%
\pgfsys@transformshift{7.502949in}{0.521603in}%
\pgfsys@useobject{currentmarker}{}%
\end{pgfscope}%
\end{pgfscope}%
\begin{pgfscope}%
\definecolor{textcolor}{rgb}{0.000000,0.000000,0.000000}%
\pgfsetstrokecolor{textcolor}%
\pgfsetfillcolor{textcolor}%
\pgftext[x=7.502949in,y=0.424381in,,top]{\color{textcolor}{\rmfamily\fontsize{10.000000}{12.000000}\selectfont\catcode`\^=\active\def^{\ifmmode\sp\else\^{}\fi}\catcode`\%=\active\def%{\%}$\mathdefault{32}$}}%
\end{pgfscope}%
\begin{pgfscope}%
\definecolor{textcolor}{rgb}{0.000000,0.000000,0.000000}%
\pgfsetstrokecolor{textcolor}%
\pgfsetfillcolor{textcolor}%
\pgftext[x=4.464063in,y=0.234413in,,top]{\color{textcolor}{\rmfamily\fontsize{10.000000}{12.000000}\selectfont\catcode`\^=\active\def^{\ifmmode\sp\else\^{}\fi}\catcode`\%=\active\def%{\%}Monochromatic classes}}%
\end{pgfscope}%
\begin{pgfscope}%
\pgfsetbuttcap%
\pgfsetroundjoin%
\definecolor{currentfill}{rgb}{0.000000,0.000000,0.000000}%
\pgfsetfillcolor{currentfill}%
\pgfsetlinewidth{0.803000pt}%
\definecolor{currentstroke}{rgb}{0.000000,0.000000,0.000000}%
\pgfsetstrokecolor{currentstroke}%
\pgfsetdash{}{0pt}%
\pgfsys@defobject{currentmarker}{\pgfqpoint{-0.048611in}{0.000000in}}{\pgfqpoint{-0.000000in}{0.000000in}}{%
\pgfpathmoveto{\pgfqpoint{-0.000000in}{0.000000in}}%
\pgfpathlineto{\pgfqpoint{-0.048611in}{0.000000in}}%
\pgfusepath{stroke,fill}%
}%
\begin{pgfscope}%
\pgfsys@transformshift{0.643750in}{0.838318in}%
\pgfsys@useobject{currentmarker}{}%
\end{pgfscope}%
\end{pgfscope}%
\begin{pgfscope}%
\definecolor{textcolor}{rgb}{0.000000,0.000000,0.000000}%
\pgfsetstrokecolor{textcolor}%
\pgfsetfillcolor{textcolor}%
\pgftext[x=0.345331in, y=0.785556in, left, base]{\color{textcolor}{\rmfamily\fontsize{10.000000}{12.000000}\selectfont\catcode`\^=\active\def^{\ifmmode\sp\else\^{}\fi}\catcode`\%=\active\def%{\%}$\mathdefault{10^{2}}$}}%
\end{pgfscope}%
\begin{pgfscope}%
\pgfsetbuttcap%
\pgfsetroundjoin%
\definecolor{currentfill}{rgb}{0.000000,0.000000,0.000000}%
\pgfsetfillcolor{currentfill}%
\pgfsetlinewidth{0.803000pt}%
\definecolor{currentstroke}{rgb}{0.000000,0.000000,0.000000}%
\pgfsetstrokecolor{currentstroke}%
\pgfsetdash{}{0pt}%
\pgfsys@defobject{currentmarker}{\pgfqpoint{-0.048611in}{0.000000in}}{\pgfqpoint{-0.000000in}{0.000000in}}{%
\pgfpathmoveto{\pgfqpoint{-0.000000in}{0.000000in}}%
\pgfpathlineto{\pgfqpoint{-0.048611in}{0.000000in}}%
\pgfusepath{stroke,fill}%
}%
\begin{pgfscope}%
\pgfsys@transformshift{0.643750in}{1.219362in}%
\pgfsys@useobject{currentmarker}{}%
\end{pgfscope}%
\end{pgfscope}%
\begin{pgfscope}%
\definecolor{textcolor}{rgb}{0.000000,0.000000,0.000000}%
\pgfsetstrokecolor{textcolor}%
\pgfsetfillcolor{textcolor}%
\pgftext[x=0.345331in, y=1.166601in, left, base]{\color{textcolor}{\rmfamily\fontsize{10.000000}{12.000000}\selectfont\catcode`\^=\active\def^{\ifmmode\sp\else\^{}\fi}\catcode`\%=\active\def%{\%}$\mathdefault{10^{5}}$}}%
\end{pgfscope}%
\begin{pgfscope}%
\pgfsetbuttcap%
\pgfsetroundjoin%
\definecolor{currentfill}{rgb}{0.000000,0.000000,0.000000}%
\pgfsetfillcolor{currentfill}%
\pgfsetlinewidth{0.803000pt}%
\definecolor{currentstroke}{rgb}{0.000000,0.000000,0.000000}%
\pgfsetstrokecolor{currentstroke}%
\pgfsetdash{}{0pt}%
\pgfsys@defobject{currentmarker}{\pgfqpoint{-0.048611in}{0.000000in}}{\pgfqpoint{-0.000000in}{0.000000in}}{%
\pgfpathmoveto{\pgfqpoint{-0.000000in}{0.000000in}}%
\pgfpathlineto{\pgfqpoint{-0.048611in}{0.000000in}}%
\pgfusepath{stroke,fill}%
}%
\begin{pgfscope}%
\pgfsys@transformshift{0.643750in}{1.600407in}%
\pgfsys@useobject{currentmarker}{}%
\end{pgfscope}%
\end{pgfscope}%
\begin{pgfscope}%
\definecolor{textcolor}{rgb}{0.000000,0.000000,0.000000}%
\pgfsetstrokecolor{textcolor}%
\pgfsetfillcolor{textcolor}%
\pgftext[x=0.345331in, y=1.547645in, left, base]{\color{textcolor}{\rmfamily\fontsize{10.000000}{12.000000}\selectfont\catcode`\^=\active\def^{\ifmmode\sp\else\^{}\fi}\catcode`\%=\active\def%{\%}$\mathdefault{10^{8}}$}}%
\end{pgfscope}%
\begin{pgfscope}%
\pgfsetbuttcap%
\pgfsetroundjoin%
\definecolor{currentfill}{rgb}{0.000000,0.000000,0.000000}%
\pgfsetfillcolor{currentfill}%
\pgfsetlinewidth{0.803000pt}%
\definecolor{currentstroke}{rgb}{0.000000,0.000000,0.000000}%
\pgfsetstrokecolor{currentstroke}%
\pgfsetdash{}{0pt}%
\pgfsys@defobject{currentmarker}{\pgfqpoint{-0.048611in}{0.000000in}}{\pgfqpoint{-0.000000in}{0.000000in}}{%
\pgfpathmoveto{\pgfqpoint{-0.000000in}{0.000000in}}%
\pgfpathlineto{\pgfqpoint{-0.048611in}{0.000000in}}%
\pgfusepath{stroke,fill}%
}%
\begin{pgfscope}%
\pgfsys@transformshift{0.643750in}{1.981451in}%
\pgfsys@useobject{currentmarker}{}%
\end{pgfscope}%
\end{pgfscope}%
\begin{pgfscope}%
\definecolor{textcolor}{rgb}{0.000000,0.000000,0.000000}%
\pgfsetstrokecolor{textcolor}%
\pgfsetfillcolor{textcolor}%
\pgftext[x=0.289968in, y=1.928689in, left, base]{\color{textcolor}{\rmfamily\fontsize{10.000000}{12.000000}\selectfont\catcode`\^=\active\def^{\ifmmode\sp\else\^{}\fi}\catcode`\%=\active\def%{\%}$\mathdefault{10^{11}}$}}%
\end{pgfscope}%
\begin{pgfscope}%
\pgfsetbuttcap%
\pgfsetroundjoin%
\definecolor{currentfill}{rgb}{0.000000,0.000000,0.000000}%
\pgfsetfillcolor{currentfill}%
\pgfsetlinewidth{0.803000pt}%
\definecolor{currentstroke}{rgb}{0.000000,0.000000,0.000000}%
\pgfsetstrokecolor{currentstroke}%
\pgfsetdash{}{0pt}%
\pgfsys@defobject{currentmarker}{\pgfqpoint{-0.048611in}{0.000000in}}{\pgfqpoint{-0.000000in}{0.000000in}}{%
\pgfpathmoveto{\pgfqpoint{-0.000000in}{0.000000in}}%
\pgfpathlineto{\pgfqpoint{-0.048611in}{0.000000in}}%
\pgfusepath{stroke,fill}%
}%
\begin{pgfscope}%
\pgfsys@transformshift{0.643750in}{2.362495in}%
\pgfsys@useobject{currentmarker}{}%
\end{pgfscope}%
\end{pgfscope}%
\begin{pgfscope}%
\definecolor{textcolor}{rgb}{0.000000,0.000000,0.000000}%
\pgfsetstrokecolor{textcolor}%
\pgfsetfillcolor{textcolor}%
\pgftext[x=0.289968in, y=2.309734in, left, base]{\color{textcolor}{\rmfamily\fontsize{10.000000}{12.000000}\selectfont\catcode`\^=\active\def^{\ifmmode\sp\else\^{}\fi}\catcode`\%=\active\def%{\%}$\mathdefault{10^{14}}$}}%
\end{pgfscope}%
\begin{pgfscope}%
\definecolor{textcolor}{rgb}{0.000000,0.000000,0.000000}%
\pgfsetstrokecolor{textcolor}%
\pgfsetfillcolor{textcolor}%
\pgftext[x=0.234413in,y=1.631726in,,bottom,rotate=90.000000]{\color{textcolor}{\rmfamily\fontsize{10.000000}{12.000000}\selectfont\catcode`\^=\active\def^{\ifmmode\sp\else\^{}\fi}\catcode`\%=\active\def%{\%}Checks [call]}}%
\end{pgfscope}%
\begin{pgfscope}%
\pgfpathrectangle{\pgfqpoint{0.643750in}{0.521603in}}{\pgfqpoint{7.640626in}{2.220246in}}%
\pgfusepath{clip}%
\pgfsetrectcap%
\pgfsetroundjoin%
\pgfsetlinewidth{1.505625pt}%
\pgfsetstrokecolor{currentstroke1}%
\pgfsetdash{}{0pt}%
\pgfpathmoveto{\pgfqpoint{0.991051in}{2.618289in}}%
\pgfpathlineto{\pgfqpoint{1.208115in}{2.582972in}}%
\pgfpathlineto{\pgfqpoint{1.425178in}{2.640929in}}%
\pgfpathlineto{\pgfqpoint{1.642241in}{2.450208in}}%
\pgfpathlineto{\pgfqpoint{1.859304in}{2.556654in}}%
\pgfpathlineto{\pgfqpoint{2.076368in}{2.569786in}}%
\pgfpathlineto{\pgfqpoint{2.293431in}{2.405350in}}%
\pgfpathlineto{\pgfqpoint{2.510494in}{2.463250in}}%
\pgfpathlineto{\pgfqpoint{2.727557in}{2.353674in}}%
\pgfpathlineto{\pgfqpoint{2.944621in}{2.103172in}}%
\pgfpathlineto{\pgfqpoint{3.161684in}{2.428318in}}%
\pgfpathlineto{\pgfqpoint{3.378747in}{1.966533in}}%
\pgfpathlineto{\pgfqpoint{3.595810in}{2.221514in}}%
\pgfpathlineto{\pgfqpoint{3.812874in}{2.144665in}}%
\pgfpathlineto{\pgfqpoint{4.029937in}{2.262275in}}%
\pgfpathlineto{\pgfqpoint{4.247000in}{2.137084in}}%
\pgfpathlineto{\pgfqpoint{4.464063in}{2.114293in}}%
\pgfpathlineto{\pgfqpoint{4.681127in}{1.815053in}}%
\pgfpathlineto{\pgfqpoint{4.898190in}{1.956060in}}%
\pgfpathlineto{\pgfqpoint{5.115253in}{2.179567in}}%
\pgfpathlineto{\pgfqpoint{5.332316in}{1.986927in}}%
\pgfpathlineto{\pgfqpoint{5.549380in}{2.189365in}}%
\pgfpathlineto{\pgfqpoint{5.766443in}{2.061992in}}%
\pgfpathlineto{\pgfqpoint{5.983506in}{1.890173in}}%
\pgfpathlineto{\pgfqpoint{6.200569in}{2.160996in}}%
\pgfpathlineto{\pgfqpoint{6.417633in}{2.073458in}}%
\pgfpathlineto{\pgfqpoint{6.634696in}{1.724291in}}%
\pgfpathlineto{\pgfqpoint{6.851759in}{1.857906in}}%
\pgfpathlineto{\pgfqpoint{7.068822in}{2.304875in}}%
\pgfpathlineto{\pgfqpoint{7.285886in}{2.203264in}}%
\pgfpathlineto{\pgfqpoint{7.720012in}{1.884287in}}%
\pgfpathlineto{\pgfqpoint{7.937075in}{1.846052in}}%
\pgfusepath{stroke}%
\end{pgfscope}%
\begin{pgfscope}%
\pgfpathrectangle{\pgfqpoint{0.643750in}{0.521603in}}{\pgfqpoint{7.640626in}{2.220246in}}%
\pgfusepath{clip}%
\pgfsetrectcap%
\pgfsetroundjoin%
\pgfsetlinewidth{1.505625pt}%
\pgfsetstrokecolor{currentstroke2}%
\pgfsetdash{}{0pt}%
\pgfpathmoveto{\pgfqpoint{0.991051in}{1.151352in}}%
\pgfpathlineto{\pgfqpoint{1.208115in}{1.225999in}}%
\pgfpathlineto{\pgfqpoint{1.425178in}{1.201836in}}%
\pgfpathlineto{\pgfqpoint{1.642241in}{1.250919in}}%
\pgfpathlineto{\pgfqpoint{1.859304in}{1.228600in}}%
\pgfpathlineto{\pgfqpoint{2.076368in}{1.279886in}}%
\pgfpathlineto{\pgfqpoint{2.293431in}{1.214487in}}%
\pgfpathlineto{\pgfqpoint{2.510494in}{1.287904in}}%
\pgfpathlineto{\pgfqpoint{2.727557in}{1.318506in}}%
\pgfpathlineto{\pgfqpoint{2.944621in}{1.245160in}}%
\pgfpathlineto{\pgfqpoint{3.161684in}{1.331327in}}%
\pgfpathlineto{\pgfqpoint{3.378747in}{1.197408in}}%
\pgfpathlineto{\pgfqpoint{3.595810in}{1.316669in}}%
\pgfpathlineto{\pgfqpoint{3.812874in}{1.440288in}}%
\pgfpathlineto{\pgfqpoint{4.029937in}{1.450348in}}%
\pgfpathlineto{\pgfqpoint{4.247000in}{1.347073in}}%
\pgfpathlineto{\pgfqpoint{4.464063in}{1.357126in}}%
\pgfpathlineto{\pgfqpoint{4.681127in}{1.350706in}}%
\pgfpathlineto{\pgfqpoint{4.898190in}{1.377639in}}%
\pgfpathlineto{\pgfqpoint{5.115253in}{1.431670in}}%
\pgfpathlineto{\pgfqpoint{5.332316in}{1.443325in}}%
\pgfpathlineto{\pgfqpoint{5.549380in}{1.444025in}}%
\pgfpathlineto{\pgfqpoint{5.766443in}{1.519739in}}%
\pgfpathlineto{\pgfqpoint{5.983506in}{1.513442in}}%
\pgfpathlineto{\pgfqpoint{6.200569in}{1.641319in}}%
\pgfpathlineto{\pgfqpoint{6.417633in}{1.602465in}}%
\pgfpathlineto{\pgfqpoint{6.634696in}{1.619914in}}%
\pgfpathlineto{\pgfqpoint{6.851759in}{1.654876in}}%
\pgfpathlineto{\pgfqpoint{7.068822in}{1.846052in}}%
\pgfpathlineto{\pgfqpoint{7.285886in}{1.739458in}}%
\pgfpathlineto{\pgfqpoint{7.720012in}{1.807817in}}%
\pgfpathlineto{\pgfqpoint{7.937075in}{1.846052in}}%
\pgfusepath{stroke}%
\end{pgfscope}%
\begin{pgfscope}%
\pgfpathrectangle{\pgfqpoint{0.643750in}{0.521603in}}{\pgfqpoint{7.640626in}{2.220246in}}%
\pgfusepath{clip}%
\pgfsetrectcap%
\pgfsetroundjoin%
\pgfsetlinewidth{1.505625pt}%
\pgfsetstrokecolor{currentstroke3}%
\pgfsetdash{}{0pt}%
\pgfpathmoveto{\pgfqpoint{0.991051in}{0.622524in}}%
\pgfpathlineto{\pgfqpoint{1.208115in}{0.660759in}}%
\pgfpathlineto{\pgfqpoint{1.425178in}{0.698994in}}%
\pgfpathlineto{\pgfqpoint{1.642241in}{0.737229in}}%
\pgfpathlineto{\pgfqpoint{1.859304in}{0.775465in}}%
\pgfpathlineto{\pgfqpoint{2.076368in}{0.813700in}}%
\pgfpathlineto{\pgfqpoint{2.293431in}{0.851935in}}%
\pgfpathlineto{\pgfqpoint{2.510494in}{0.890170in}}%
\pgfpathlineto{\pgfqpoint{2.727557in}{0.928406in}}%
\pgfpathlineto{\pgfqpoint{2.944621in}{0.966641in}}%
\pgfpathlineto{\pgfqpoint{3.161684in}{1.004876in}}%
\pgfpathlineto{\pgfqpoint{3.378747in}{1.043111in}}%
\pgfpathlineto{\pgfqpoint{3.595810in}{1.081347in}}%
\pgfpathlineto{\pgfqpoint{3.812874in}{1.119582in}}%
\pgfpathlineto{\pgfqpoint{4.029937in}{1.157817in}}%
\pgfpathlineto{\pgfqpoint{4.247000in}{1.196053in}}%
\pgfpathlineto{\pgfqpoint{4.464063in}{1.234288in}}%
\pgfpathlineto{\pgfqpoint{4.681127in}{1.272523in}}%
\pgfpathlineto{\pgfqpoint{4.898190in}{1.310758in}}%
\pgfpathlineto{\pgfqpoint{5.115253in}{1.348994in}}%
\pgfpathlineto{\pgfqpoint{5.332316in}{1.387229in}}%
\pgfpathlineto{\pgfqpoint{5.549380in}{1.425464in}}%
\pgfpathlineto{\pgfqpoint{5.766443in}{1.463699in}}%
\pgfpathlineto{\pgfqpoint{5.983506in}{1.501935in}}%
\pgfpathlineto{\pgfqpoint{6.200569in}{1.540170in}}%
\pgfpathlineto{\pgfqpoint{6.417633in}{1.578405in}}%
\pgfpathlineto{\pgfqpoint{6.634696in}{1.616640in}}%
\pgfpathlineto{\pgfqpoint{6.851759in}{1.654876in}}%
\pgfpathlineto{\pgfqpoint{7.068822in}{1.693111in}}%
\pgfpathlineto{\pgfqpoint{7.285886in}{1.731346in}}%
\pgfpathlineto{\pgfqpoint{7.720012in}{1.807817in}}%
\pgfpathlineto{\pgfqpoint{7.937075in}{1.846052in}}%
\pgfusepath{stroke}%
\end{pgfscope}%
\begin{pgfscope}%
\pgfpathrectangle{\pgfqpoint{0.643750in}{0.521603in}}{\pgfqpoint{7.640626in}{2.220246in}}%
\pgfusepath{clip}%
\pgfsetrectcap%
\pgfsetroundjoin%
\pgfsetlinewidth{1.505625pt}%
\pgfsetstrokecolor{currentstroke4}%
\pgfsetdash{}{0pt}%
\pgfpathmoveto{\pgfqpoint{0.991051in}{0.622524in}}%
\pgfpathlineto{\pgfqpoint{1.208115in}{0.660759in}}%
\pgfpathlineto{\pgfqpoint{1.425178in}{0.698994in}}%
\pgfpathlineto{\pgfqpoint{1.642241in}{0.737229in}}%
\pgfpathlineto{\pgfqpoint{1.859304in}{0.775239in}}%
\pgfpathlineto{\pgfqpoint{2.076368in}{0.813700in}}%
\pgfpathlineto{\pgfqpoint{2.293431in}{0.845197in}}%
\pgfpathlineto{\pgfqpoint{2.510494in}{0.875033in}}%
\pgfpathlineto{\pgfqpoint{2.727557in}{0.910532in}}%
\pgfpathlineto{\pgfqpoint{2.944621in}{0.935624in}}%
\pgfpathlineto{\pgfqpoint{3.161684in}{0.950745in}}%
\pgfpathlineto{\pgfqpoint{3.378747in}{0.972291in}}%
\pgfpathlineto{\pgfqpoint{3.595810in}{0.998560in}}%
\pgfpathlineto{\pgfqpoint{3.812874in}{1.029885in}}%
\pgfpathlineto{\pgfqpoint{4.029937in}{1.052894in}}%
\pgfpathlineto{\pgfqpoint{4.247000in}{1.062798in}}%
\pgfpathlineto{\pgfqpoint{4.464063in}{1.112928in}}%
\pgfpathlineto{\pgfqpoint{4.681127in}{1.099172in}}%
\pgfpathlineto{\pgfqpoint{4.898190in}{1.132661in}}%
\pgfpathlineto{\pgfqpoint{5.115253in}{1.146570in}}%
\pgfpathlineto{\pgfqpoint{5.332316in}{1.166922in}}%
\pgfpathlineto{\pgfqpoint{5.549380in}{1.206202in}}%
\pgfpathlineto{\pgfqpoint{5.766443in}{1.209798in}}%
\pgfpathlineto{\pgfqpoint{5.983506in}{1.216637in}}%
\pgfpathlineto{\pgfqpoint{6.200569in}{1.257050in}}%
\pgfpathlineto{\pgfqpoint{6.417633in}{1.232813in}}%
\pgfpathlineto{\pgfqpoint{6.634696in}{1.328293in}}%
\pgfpathlineto{\pgfqpoint{6.851759in}{1.256856in}}%
\pgfpathlineto{\pgfqpoint{7.068822in}{1.197566in}}%
\pgfpathlineto{\pgfqpoint{7.285886in}{1.315866in}}%
\pgfpathlineto{\pgfqpoint{7.720012in}{1.245938in}}%
\pgfpathlineto{\pgfqpoint{7.937075in}{1.335354in}}%
\pgfusepath{stroke}%
\end{pgfscope}%
\begin{pgfscope}%
\pgfpathrectangle{\pgfqpoint{0.643750in}{0.521603in}}{\pgfqpoint{7.640626in}{2.220246in}}%
\pgfusepath{clip}%
\pgfsetrectcap%
\pgfsetroundjoin%
\pgfsetlinewidth{1.505625pt}%
\pgfsetstrokecolor{currentstroke5}%
\pgfsetdash{}{0pt}%
\pgfpathmoveto{\pgfqpoint{0.991051in}{0.622524in}}%
\pgfpathlineto{\pgfqpoint{1.208115in}{0.653568in}}%
\pgfpathlineto{\pgfqpoint{1.425178in}{0.664465in}}%
\pgfpathlineto{\pgfqpoint{1.642241in}{0.681651in}}%
\pgfpathlineto{\pgfqpoint{1.859304in}{0.700034in}}%
\pgfpathlineto{\pgfqpoint{2.076368in}{0.712586in}}%
\pgfpathlineto{\pgfqpoint{2.293431in}{0.765953in}}%
\pgfpathlineto{\pgfqpoint{2.510494in}{0.779809in}}%
\pgfpathlineto{\pgfqpoint{2.727557in}{0.798189in}}%
\pgfpathlineto{\pgfqpoint{2.944621in}{0.818015in}}%
\pgfpathlineto{\pgfqpoint{3.161684in}{0.846318in}}%
\pgfpathlineto{\pgfqpoint{3.378747in}{0.864005in}}%
\pgfpathlineto{\pgfqpoint{3.595810in}{0.883651in}}%
\pgfpathlineto{\pgfqpoint{3.812874in}{0.897401in}}%
\pgfpathlineto{\pgfqpoint{4.029937in}{0.945573in}}%
\pgfpathlineto{\pgfqpoint{4.247000in}{0.948404in}}%
\pgfpathlineto{\pgfqpoint{4.464063in}{1.003491in}}%
\pgfpathlineto{\pgfqpoint{4.681127in}{0.979329in}}%
\pgfpathlineto{\pgfqpoint{4.898190in}{1.017992in}}%
\pgfpathlineto{\pgfqpoint{5.115253in}{1.028499in}}%
\pgfpathlineto{\pgfqpoint{5.332316in}{1.053422in}}%
\pgfpathlineto{\pgfqpoint{5.549380in}{1.074575in}}%
\pgfpathlineto{\pgfqpoint{5.766443in}{1.098611in}}%
\pgfpathlineto{\pgfqpoint{5.983506in}{1.102044in}}%
\pgfpathlineto{\pgfqpoint{6.200569in}{1.128364in}}%
\pgfpathlineto{\pgfqpoint{6.417633in}{1.120351in}}%
\pgfpathlineto{\pgfqpoint{6.634696in}{1.178232in}}%
\pgfpathlineto{\pgfqpoint{6.851759in}{1.147348in}}%
\pgfpathlineto{\pgfqpoint{7.068822in}{1.023951in}}%
\pgfpathlineto{\pgfqpoint{7.285886in}{1.174764in}}%
\pgfpathlineto{\pgfqpoint{7.720012in}{1.177433in}}%
\pgfpathlineto{\pgfqpoint{7.937075in}{1.189897in}}%
\pgfusepath{stroke}%
\end{pgfscope}%
\begin{pgfscope}%
\pgfsetrectcap%
\pgfsetmiterjoin%
\pgfsetlinewidth{0.803000pt}%
\definecolor{currentstroke}{rgb}{0.000000,0.000000,0.000000}%
\pgfsetstrokecolor{currentstroke}%
\pgfsetdash{}{0pt}%
\pgfpathmoveto{\pgfqpoint{0.643750in}{0.521603in}}%
\pgfpathlineto{\pgfqpoint{0.643750in}{2.741849in}}%
\pgfusepath{stroke}%
\end{pgfscope}%
\begin{pgfscope}%
\pgfsetrectcap%
\pgfsetmiterjoin%
\pgfsetlinewidth{0.803000pt}%
\definecolor{currentstroke}{rgb}{0.000000,0.000000,0.000000}%
\pgfsetstrokecolor{currentstroke}%
\pgfsetdash{}{0pt}%
\pgfpathmoveto{\pgfqpoint{8.284376in}{0.521603in}}%
\pgfpathlineto{\pgfqpoint{8.284376in}{2.741849in}}%
\pgfusepath{stroke}%
\end{pgfscope}%
\begin{pgfscope}%
\pgfsetrectcap%
\pgfsetmiterjoin%
\pgfsetlinewidth{0.803000pt}%
\definecolor{currentstroke}{rgb}{0.000000,0.000000,0.000000}%
\pgfsetstrokecolor{currentstroke}%
\pgfsetdash{}{0pt}%
\pgfpathmoveto{\pgfqpoint{0.643750in}{0.521603in}}%
\pgfpathlineto{\pgfqpoint{8.284376in}{0.521603in}}%
\pgfusepath{stroke}%
\end{pgfscope}%
\begin{pgfscope}%
\pgfsetrectcap%
\pgfsetmiterjoin%
\pgfsetlinewidth{0.803000pt}%
\definecolor{currentstroke}{rgb}{0.000000,0.000000,0.000000}%
\pgfsetstrokecolor{currentstroke}%
\pgfsetdash{}{0pt}%
\pgfpathmoveto{\pgfqpoint{0.643750in}{2.741849in}}%
\pgfpathlineto{\pgfqpoint{8.284376in}{2.741849in}}%
\pgfusepath{stroke}%
\end{pgfscope}%
\begin{pgfscope}%
\pgfsetbuttcap%
\pgfsetmiterjoin%
\definecolor{currentfill}{rgb}{1.000000,1.000000,1.000000}%
\pgfsetfillcolor{currentfill}%
\pgfsetfillopacity{0.800000}%
\pgfsetlinewidth{1.003750pt}%
\definecolor{currentstroke}{rgb}{0.800000,0.800000,0.800000}%
\pgfsetstrokecolor{currentstroke}%
\pgfsetstrokeopacity{0.800000}%
\pgfsetdash{}{0pt}%
\pgfpathmoveto{\pgfqpoint{0.760417in}{1.385372in}}%
\pgfpathlineto{\pgfqpoint{4.097823in}{1.385372in}}%
\pgfpathquadraticcurveto{\pgfqpoint{4.131156in}{1.385372in}}{\pgfqpoint{4.131156in}{1.418705in}}%
\pgfpathlineto{\pgfqpoint{4.131156in}{2.625183in}}%
\pgfpathquadraticcurveto{\pgfqpoint{4.131156in}{2.658516in}}{\pgfqpoint{4.097823in}{2.658516in}}%
\pgfpathlineto{\pgfqpoint{0.760417in}{2.658516in}}%
\pgfpathquadraticcurveto{\pgfqpoint{0.727083in}{2.658516in}}{\pgfqpoint{0.727083in}{2.625183in}}%
\pgfpathlineto{\pgfqpoint{0.727083in}{1.418705in}}%
\pgfpathquadraticcurveto{\pgfqpoint{0.727083in}{1.385372in}}{\pgfqpoint{0.760417in}{1.385372in}}%
\pgfpathlineto{\pgfqpoint{0.760417in}{1.385372in}}%
\pgfpathclose%
\pgfusepath{stroke,fill}%
\end{pgfscope}%
\begin{pgfscope}%
\pgfsetrectcap%
\pgfsetroundjoin%
\pgfsetlinewidth{1.505625pt}%
\pgfsetstrokecolor{currentstroke1}%
\pgfsetdash{}{0pt}%
\pgfpathmoveto{\pgfqpoint{0.793750in}{2.523555in}}%
\pgfpathlineto{\pgfqpoint{0.960417in}{2.523555in}}%
\pgfpathlineto{\pgfqpoint{1.127083in}{2.523555in}}%
\pgfusepath{stroke}%
\end{pgfscope}%
\begin{pgfscope}%
\definecolor{textcolor}{rgb}{0.000000,0.000000,0.000000}%
\pgfsetstrokecolor{textcolor}%
\pgfsetfillcolor{textcolor}%
\pgftext[x=1.260417in,y=2.465222in,left,base]{\color{textcolor}{\rmfamily\fontsize{12.000000}{14.400000}\selectfont\catcode`\^=\active\def^{\ifmmode\sp\else\^{}\fi}\catcode`\%=\active\def%{\%}Naive - Edges}}%
\end{pgfscope}%
\begin{pgfscope}%
\pgfsetrectcap%
\pgfsetroundjoin%
\pgfsetlinewidth{1.505625pt}%
\pgfsetstrokecolor{currentstroke2}%
\pgfsetdash{}{0pt}%
\pgfpathmoveto{\pgfqpoint{0.793750in}{2.278926in}}%
\pgfpathlineto{\pgfqpoint{0.960417in}{2.278926in}}%
\pgfpathlineto{\pgfqpoint{1.127083in}{2.278926in}}%
\pgfusepath{stroke}%
\end{pgfscope}%
\begin{pgfscope}%
\definecolor{textcolor}{rgb}{0.000000,0.000000,0.000000}%
\pgfsetstrokecolor{textcolor}%
\pgfsetfillcolor{textcolor}%
\pgftext[x=1.260417in,y=2.220593in,left,base]{\color{textcolor}{\rmfamily\fontsize{12.000000}{14.400000}\selectfont\catcode`\^=\active\def^{\ifmmode\sp\else\^{}\fi}\catcode`\%=\active\def%{\%}Naive - $\triangle$-connected components}}%
\end{pgfscope}%
\begin{pgfscope}%
\pgfsetrectcap%
\pgfsetroundjoin%
\pgfsetlinewidth{1.505625pt}%
\pgfsetstrokecolor{currentstroke3}%
\pgfsetdash{}{0pt}%
\pgfpathmoveto{\pgfqpoint{0.793750in}{2.034297in}}%
\pgfpathlineto{\pgfqpoint{0.960417in}{2.034297in}}%
\pgfpathlineto{\pgfqpoint{1.127083in}{2.034297in}}%
\pgfusepath{stroke}%
\end{pgfscope}%
\begin{pgfscope}%
\definecolor{textcolor}{rgb}{0.000000,0.000000,0.000000}%
\pgfsetstrokecolor{textcolor}%
\pgfsetfillcolor{textcolor}%
\pgftext[x=1.260417in,y=1.975964in,left,base]{\color{textcolor}{\rmfamily\fontsize{12.000000}{14.400000}\selectfont\catcode`\^=\active\def^{\ifmmode\sp\else\^{}\fi}\catcode`\%=\active\def%{\%}Naive - Monochromatic classes}}%
\end{pgfscope}%
\begin{pgfscope}%
\pgfsetrectcap%
\pgfsetroundjoin%
\pgfsetlinewidth{1.505625pt}%
\pgfsetstrokecolor{currentstroke4}%
\pgfsetdash{}{0pt}%
\pgfpathmoveto{\pgfqpoint{0.793750in}{1.789669in}}%
\pgfpathlineto{\pgfqpoint{0.960417in}{1.789669in}}%
\pgfpathlineto{\pgfqpoint{1.127083in}{1.789669in}}%
\pgfusepath{stroke}%
\end{pgfscope}%
\begin{pgfscope}%
\definecolor{textcolor}{rgb}{0.000000,0.000000,0.000000}%
\pgfsetstrokecolor{textcolor}%
\pgfsetfillcolor{textcolor}%
\pgftext[x=1.260417in,y=1.731335in,left,base]{\color{textcolor}{\rmfamily\fontsize{12.000000}{14.400000}\selectfont\catcode`\^=\active\def^{\ifmmode\sp\else\^{}\fi}\catcode`\%=\active\def%{\%}Subgraphs - \CycleMask{}}}%
\end{pgfscope}%
\begin{pgfscope}%
\pgfsetrectcap%
\pgfsetroundjoin%
\pgfsetlinewidth{1.505625pt}%
\pgfsetstrokecolor{currentstroke5}%
\pgfsetdash{}{0pt}%
\pgfpathmoveto{\pgfqpoint{0.793750in}{1.545040in}}%
\pgfpathlineto{\pgfqpoint{0.960417in}{1.545040in}}%
\pgfpathlineto{\pgfqpoint{1.127083in}{1.545040in}}%
\pgfusepath{stroke}%
\end{pgfscope}%
\begin{pgfscope}%
\definecolor{textcolor}{rgb}{0.000000,0.000000,0.000000}%
\pgfsetstrokecolor{textcolor}%
\pgfsetfillcolor{textcolor}%
\pgftext[x=1.260417in,y=1.486707in,left,base]{\color{textcolor}{\rmfamily\fontsize{12.000000}{14.400000}\selectfont\catcode`\^=\active\def^{\ifmmode\sp\else\^{}\fi}\catcode`\%=\active\def%{\%}Subgraphs - \IsNACColoring{}}}%
\end{pgfscope}%
\end{pgfpicture}%
\makeatother%
\endgroup%
}
	\caption[The number of \IsNACColoring{} calls.]{
		The number of \IsNACColoring{} calls with respect to the number of monochromatic classes
		over all graphs used for benchmarking.}%
	\label{fig:graph_summary}
\end{figure}






\subsection{Performance on graph classes}%

\todo[inline]{Consider Laman deg 3+}
\todo[inline]{Consider line graphs of 3 nor 4 cycles}

Each benchmark was run two or three times and the mean was taken.
The graphs are grouped either by the number of vertices
monochromatic classes or \trcon{} components, see respective \(x\)-axis.
Overall, over 430k configurations were run
on over 28k graphs from multiple graph classes.
First, we only show strategies that performed well,
we show the others later in \Cref{sec:failing_strategies}.
% If a strategy did not finish in time for a graph,
% we replace the runtime field with of 60 second.
These graphs are excluded from the number of check graphs.

% Laman random first
In the previous section we showed performance of the algorithm for listing
all NAC-colorings of minimally rigid graphs.
In \Cref{fig:graph_minimally_rigid_first_runtime,fig:graph_minimally_rigid_first_checks}
we focus of finding some NAC-coloring of a minimally rigid graph.
Minimally rigid graphs as mostly flexible graphs have also
large number of NAC-colorings, therefore it is simple for both \NaiveCycles{}
and \Subgraphs{} algorithms to find some NAC-coloring.
It can be seen from the graphs, that for larger graphs, the required runtime
does not grow significantly.
Note that minimally rigid graphs have no NAC-coloring if and only if they are formed from
a single \trcon{} component, therefore such instances do not worsen runtime performance.
\NaiveCycles{} is faster as it has lower internal overhead.
The number of \IsNACColoring{} checks is also lower,
that is probably because \Subgraphs{} strategies do additional checks
while merging, which are not needed for \NaiveCycles{}.

\begin{figure}[p]
	\centering
	\scalebox{0.5}{%% Creator: Matplotlib, PGF backend
%%
%% To include the figure in your LaTeX document, write
%%   \input{<filename>.pgf}
%%
%% Make sure the required packages are loaded in your preamble
%%   \usepackage{pgf}
%%
%% Also ensure that all the required font packages are loaded; for instance,
%% the lmodern package is sometimes necessary when using math font.
%%   \usepackage{lmodern}
%%
%% Figures using additional raster images can only be included by \input if
%% they are in the same directory as the main LaTeX file. For loading figures
%% from other directories you can use the `import` package
%%   \usepackage{import}
%%
%% and then include the figures with
%%   \import{<path to file>}{<filename>.pgf}
%%
%% Matplotlib used the following preamble
%%   \def\mathdefault#1{#1}
%%   \everymath=\expandafter{\the\everymath\displaystyle}
%%   \IfFileExists{scrextend.sty}{
%%     \usepackage[fontsize=10.000000pt]{scrextend}
%%   }{
%%     \renewcommand{\normalsize}{\fontsize{10.000000}{12.000000}\selectfont}
%%     \normalsize
%%   }
%%   
%%   \ifdefined\pdftexversion\else  % non-pdftex case.
%%     \usepackage{fontspec}
%%     \setmainfont{DejaVuSans.ttf}[Path=\detokenize{/home/petr/Projects/PyRigi/.venv/lib/python3.12/site-packages/matplotlib/mpl-data/fonts/ttf/}]
%%     \setsansfont{DejaVuSans.ttf}[Path=\detokenize{/home/petr/Projects/PyRigi/.venv/lib/python3.12/site-packages/matplotlib/mpl-data/fonts/ttf/}]
%%     \setmonofont{DejaVuSansMono.ttf}[Path=\detokenize{/home/petr/Projects/PyRigi/.venv/lib/python3.12/site-packages/matplotlib/mpl-data/fonts/ttf/}]
%%   \fi
%%   \makeatletter\@ifpackageloaded{under\Score{}}{}{\usepackage[strings]{under\Score{}}}\makeatother
%%
\begingroup%
\makeatletter%
\begin{pgfpicture}%
\pgfpathrectangle{\pgfpointorigin}{\pgfqpoint{8.384376in}{2.841849in}}%
\pgfusepath{use as bounding box, clip}%
\begin{pgfscope}%
\pgfsetbuttcap%
\pgfsetmiterjoin%
\definecolor{currentfill}{rgb}{1.000000,1.000000,1.000000}%
\pgfsetfillcolor{currentfill}%
\pgfsetlinewidth{0.000000pt}%
\definecolor{currentstroke}{rgb}{1.000000,1.000000,1.000000}%
\pgfsetstrokecolor{currentstroke}%
\pgfsetdash{}{0pt}%
\pgfpathmoveto{\pgfqpoint{0.000000in}{0.000000in}}%
\pgfpathlineto{\pgfqpoint{8.384376in}{0.000000in}}%
\pgfpathlineto{\pgfqpoint{8.384376in}{2.841849in}}%
\pgfpathlineto{\pgfqpoint{0.000000in}{2.841849in}}%
\pgfpathlineto{\pgfqpoint{0.000000in}{0.000000in}}%
\pgfpathclose%
\pgfusepath{fill}%
\end{pgfscope}%
\begin{pgfscope}%
\pgfsetbuttcap%
\pgfsetmiterjoin%
\definecolor{currentfill}{rgb}{1.000000,1.000000,1.000000}%
\pgfsetfillcolor{currentfill}%
\pgfsetlinewidth{0.000000pt}%
\definecolor{currentstroke}{rgb}{0.000000,0.000000,0.000000}%
\pgfsetstrokecolor{currentstroke}%
\pgfsetstrokeopacity{0.000000}%
\pgfsetdash{}{0pt}%
\pgfpathmoveto{\pgfqpoint{0.588387in}{0.521603in}}%
\pgfpathlineto{\pgfqpoint{5.988063in}{0.521603in}}%
\pgfpathlineto{\pgfqpoint{5.988063in}{2.531888in}}%
\pgfpathlineto{\pgfqpoint{0.588387in}{2.531888in}}%
\pgfpathlineto{\pgfqpoint{0.588387in}{0.521603in}}%
\pgfpathclose%
\pgfusepath{fill}%
\end{pgfscope}%
\begin{pgfscope}%
\pgfsetbuttcap%
\pgfsetroundjoin%
\definecolor{currentfill}{rgb}{0.000000,0.000000,0.000000}%
\pgfsetfillcolor{currentfill}%
\pgfsetlinewidth{0.803000pt}%
\definecolor{currentstroke}{rgb}{0.000000,0.000000,0.000000}%
\pgfsetstrokecolor{currentstroke}%
\pgfsetdash{}{0pt}%
\pgfsys@defobject{currentmarker}{\pgfqpoint{0.000000in}{-0.048611in}}{\pgfqpoint{0.000000in}{0.000000in}}{%
\pgfpathmoveto{\pgfqpoint{0.000000in}{0.000000in}}%
\pgfpathlineto{\pgfqpoint{0.000000in}{-0.048611in}}%
\pgfusepath{stroke,fill}%
}%
\begin{pgfscope}%
\pgfsys@transformshift{0.738511in}{0.521603in}%
\pgfsys@useobject{currentmarker}{}%
\end{pgfscope}%
\end{pgfscope}%
\begin{pgfscope}%
\definecolor{textcolor}{rgb}{0.000000,0.000000,0.000000}%
\pgfsetstrokecolor{textcolor}%
\pgfsetfillcolor{textcolor}%
\pgftext[x=0.738511in,y=0.424381in,,top]{\color{textcolor}{\rmfamily\fontsize{10.000000}{12.000000}\selectfont\catcode`\^=\active\def^{\ifmmode\sp\else\^{}\fi}\catcode`\%=\active\def%{\%}$\mathdefault{0}$}}%
\end{pgfscope}%
\begin{pgfscope}%
\pgfsetbuttcap%
\pgfsetroundjoin%
\definecolor{currentfill}{rgb}{0.000000,0.000000,0.000000}%
\pgfsetfillcolor{currentfill}%
\pgfsetlinewidth{0.803000pt}%
\definecolor{currentstroke}{rgb}{0.000000,0.000000,0.000000}%
\pgfsetstrokecolor{currentstroke}%
\pgfsetdash{}{0pt}%
\pgfsys@defobject{currentmarker}{\pgfqpoint{0.000000in}{-0.048611in}}{\pgfqpoint{0.000000in}{0.000000in}}{%
\pgfpathmoveto{\pgfqpoint{0.000000in}{0.000000in}}%
\pgfpathlineto{\pgfqpoint{0.000000in}{-0.048611in}}%
\pgfusepath{stroke,fill}%
}%
\begin{pgfscope}%
\pgfsys@transformshift{1.453384in}{0.521603in}%
\pgfsys@useobject{currentmarker}{}%
\end{pgfscope}%
\end{pgfscope}%
\begin{pgfscope}%
\definecolor{textcolor}{rgb}{0.000000,0.000000,0.000000}%
\pgfsetstrokecolor{textcolor}%
\pgfsetfillcolor{textcolor}%
\pgftext[x=1.453384in,y=0.424381in,,top]{\color{textcolor}{\rmfamily\fontsize{10.000000}{12.000000}\selectfont\catcode`\^=\active\def^{\ifmmode\sp\else\^{}\fi}\catcode`\%=\active\def%{\%}$\mathdefault{15}$}}%
\end{pgfscope}%
\begin{pgfscope}%
\pgfsetbuttcap%
\pgfsetroundjoin%
\definecolor{currentfill}{rgb}{0.000000,0.000000,0.000000}%
\pgfsetfillcolor{currentfill}%
\pgfsetlinewidth{0.803000pt}%
\definecolor{currentstroke}{rgb}{0.000000,0.000000,0.000000}%
\pgfsetstrokecolor{currentstroke}%
\pgfsetdash{}{0pt}%
\pgfsys@defobject{currentmarker}{\pgfqpoint{0.000000in}{-0.048611in}}{\pgfqpoint{0.000000in}{0.000000in}}{%
\pgfpathmoveto{\pgfqpoint{0.000000in}{0.000000in}}%
\pgfpathlineto{\pgfqpoint{0.000000in}{-0.048611in}}%
\pgfusepath{stroke,fill}%
}%
\begin{pgfscope}%
\pgfsys@transformshift{2.168257in}{0.521603in}%
\pgfsys@useobject{currentmarker}{}%
\end{pgfscope}%
\end{pgfscope}%
\begin{pgfscope}%
\definecolor{textcolor}{rgb}{0.000000,0.000000,0.000000}%
\pgfsetstrokecolor{textcolor}%
\pgfsetfillcolor{textcolor}%
\pgftext[x=2.168257in,y=0.424381in,,top]{\color{textcolor}{\rmfamily\fontsize{10.000000}{12.000000}\selectfont\catcode`\^=\active\def^{\ifmmode\sp\else\^{}\fi}\catcode`\%=\active\def%{\%}$\mathdefault{30}$}}%
\end{pgfscope}%
\begin{pgfscope}%
\pgfsetbuttcap%
\pgfsetroundjoin%
\definecolor{currentfill}{rgb}{0.000000,0.000000,0.000000}%
\pgfsetfillcolor{currentfill}%
\pgfsetlinewidth{0.803000pt}%
\definecolor{currentstroke}{rgb}{0.000000,0.000000,0.000000}%
\pgfsetstrokecolor{currentstroke}%
\pgfsetdash{}{0pt}%
\pgfsys@defobject{currentmarker}{\pgfqpoint{0.000000in}{-0.048611in}}{\pgfqpoint{0.000000in}{0.000000in}}{%
\pgfpathmoveto{\pgfqpoint{0.000000in}{0.000000in}}%
\pgfpathlineto{\pgfqpoint{0.000000in}{-0.048611in}}%
\pgfusepath{stroke,fill}%
}%
\begin{pgfscope}%
\pgfsys@transformshift{2.883130in}{0.521603in}%
\pgfsys@useobject{currentmarker}{}%
\end{pgfscope}%
\end{pgfscope}%
\begin{pgfscope}%
\definecolor{textcolor}{rgb}{0.000000,0.000000,0.000000}%
\pgfsetstrokecolor{textcolor}%
\pgfsetfillcolor{textcolor}%
\pgftext[x=2.883130in,y=0.424381in,,top]{\color{textcolor}{\rmfamily\fontsize{10.000000}{12.000000}\selectfont\catcode`\^=\active\def^{\ifmmode\sp\else\^{}\fi}\catcode`\%=\active\def%{\%}$\mathdefault{45}$}}%
\end{pgfscope}%
\begin{pgfscope}%
\pgfsetbuttcap%
\pgfsetroundjoin%
\definecolor{currentfill}{rgb}{0.000000,0.000000,0.000000}%
\pgfsetfillcolor{currentfill}%
\pgfsetlinewidth{0.803000pt}%
\definecolor{currentstroke}{rgb}{0.000000,0.000000,0.000000}%
\pgfsetstrokecolor{currentstroke}%
\pgfsetdash{}{0pt}%
\pgfsys@defobject{currentmarker}{\pgfqpoint{0.000000in}{-0.048611in}}{\pgfqpoint{0.000000in}{0.000000in}}{%
\pgfpathmoveto{\pgfqpoint{0.000000in}{0.000000in}}%
\pgfpathlineto{\pgfqpoint{0.000000in}{-0.048611in}}%
\pgfusepath{stroke,fill}%
}%
\begin{pgfscope}%
\pgfsys@transformshift{3.598004in}{0.521603in}%
\pgfsys@useobject{currentmarker}{}%
\end{pgfscope}%
\end{pgfscope}%
\begin{pgfscope}%
\definecolor{textcolor}{rgb}{0.000000,0.000000,0.000000}%
\pgfsetstrokecolor{textcolor}%
\pgfsetfillcolor{textcolor}%
\pgftext[x=3.598004in,y=0.424381in,,top]{\color{textcolor}{\rmfamily\fontsize{10.000000}{12.000000}\selectfont\catcode`\^=\active\def^{\ifmmode\sp\else\^{}\fi}\catcode`\%=\active\def%{\%}$\mathdefault{60}$}}%
\end{pgfscope}%
\begin{pgfscope}%
\pgfsetbuttcap%
\pgfsetroundjoin%
\definecolor{currentfill}{rgb}{0.000000,0.000000,0.000000}%
\pgfsetfillcolor{currentfill}%
\pgfsetlinewidth{0.803000pt}%
\definecolor{currentstroke}{rgb}{0.000000,0.000000,0.000000}%
\pgfsetstrokecolor{currentstroke}%
\pgfsetdash{}{0pt}%
\pgfsys@defobject{currentmarker}{\pgfqpoint{0.000000in}{-0.048611in}}{\pgfqpoint{0.000000in}{0.000000in}}{%
\pgfpathmoveto{\pgfqpoint{0.000000in}{0.000000in}}%
\pgfpathlineto{\pgfqpoint{0.000000in}{-0.048611in}}%
\pgfusepath{stroke,fill}%
}%
\begin{pgfscope}%
\pgfsys@transformshift{4.312877in}{0.521603in}%
\pgfsys@useobject{currentmarker}{}%
\end{pgfscope}%
\end{pgfscope}%
\begin{pgfscope}%
\definecolor{textcolor}{rgb}{0.000000,0.000000,0.000000}%
\pgfsetstrokecolor{textcolor}%
\pgfsetfillcolor{textcolor}%
\pgftext[x=4.312877in,y=0.424381in,,top]{\color{textcolor}{\rmfamily\fontsize{10.000000}{12.000000}\selectfont\catcode`\^=\active\def^{\ifmmode\sp\else\^{}\fi}\catcode`\%=\active\def%{\%}$\mathdefault{75}$}}%
\end{pgfscope}%
\begin{pgfscope}%
\pgfsetbuttcap%
\pgfsetroundjoin%
\definecolor{currentfill}{rgb}{0.000000,0.000000,0.000000}%
\pgfsetfillcolor{currentfill}%
\pgfsetlinewidth{0.803000pt}%
\definecolor{currentstroke}{rgb}{0.000000,0.000000,0.000000}%
\pgfsetstrokecolor{currentstroke}%
\pgfsetdash{}{0pt}%
\pgfsys@defobject{currentmarker}{\pgfqpoint{0.000000in}{-0.048611in}}{\pgfqpoint{0.000000in}{0.000000in}}{%
\pgfpathmoveto{\pgfqpoint{0.000000in}{0.000000in}}%
\pgfpathlineto{\pgfqpoint{0.000000in}{-0.048611in}}%
\pgfusepath{stroke,fill}%
}%
\begin{pgfscope}%
\pgfsys@transformshift{5.027750in}{0.521603in}%
\pgfsys@useobject{currentmarker}{}%
\end{pgfscope}%
\end{pgfscope}%
\begin{pgfscope}%
\definecolor{textcolor}{rgb}{0.000000,0.000000,0.000000}%
\pgfsetstrokecolor{textcolor}%
\pgfsetfillcolor{textcolor}%
\pgftext[x=5.027750in,y=0.424381in,,top]{\color{textcolor}{\rmfamily\fontsize{10.000000}{12.000000}\selectfont\catcode`\^=\active\def^{\ifmmode\sp\else\^{}\fi}\catcode`\%=\active\def%{\%}$\mathdefault{90}$}}%
\end{pgfscope}%
\begin{pgfscope}%
\pgfsetbuttcap%
\pgfsetroundjoin%
\definecolor{currentfill}{rgb}{0.000000,0.000000,0.000000}%
\pgfsetfillcolor{currentfill}%
\pgfsetlinewidth{0.803000pt}%
\definecolor{currentstroke}{rgb}{0.000000,0.000000,0.000000}%
\pgfsetstrokecolor{currentstroke}%
\pgfsetdash{}{0pt}%
\pgfsys@defobject{currentmarker}{\pgfqpoint{0.000000in}{-0.048611in}}{\pgfqpoint{0.000000in}{0.000000in}}{%
\pgfpathmoveto{\pgfqpoint{0.000000in}{0.000000in}}%
\pgfpathlineto{\pgfqpoint{0.000000in}{-0.048611in}}%
\pgfusepath{stroke,fill}%
}%
\begin{pgfscope}%
\pgfsys@transformshift{5.742623in}{0.521603in}%
\pgfsys@useobject{currentmarker}{}%
\end{pgfscope}%
\end{pgfscope}%
\begin{pgfscope}%
\definecolor{textcolor}{rgb}{0.000000,0.000000,0.000000}%
\pgfsetstrokecolor{textcolor}%
\pgfsetfillcolor{textcolor}%
\pgftext[x=5.742623in,y=0.424381in,,top]{\color{textcolor}{\rmfamily\fontsize{10.000000}{12.000000}\selectfont\catcode`\^=\active\def^{\ifmmode\sp\else\^{}\fi}\catcode`\%=\active\def%{\%}$\mathdefault{105}$}}%
\end{pgfscope}%
\begin{pgfscope}%
\definecolor{textcolor}{rgb}{0.000000,0.000000,0.000000}%
\pgfsetstrokecolor{textcolor}%
\pgfsetfillcolor{textcolor}%
\pgftext[x=3.288225in,y=0.234413in,,top]{\color{textcolor}{\rmfamily\fontsize{10.000000}{12.000000}\selectfont\catcode`\^=\active\def^{\ifmmode\sp\else\^{}\fi}\catcode`\%=\active\def%{\%}Monochromatic classes}}%
\end{pgfscope}%
\begin{pgfscope}%
\pgfsetbuttcap%
\pgfsetroundjoin%
\definecolor{currentfill}{rgb}{0.000000,0.000000,0.000000}%
\pgfsetfillcolor{currentfill}%
\pgfsetlinewidth{0.803000pt}%
\definecolor{currentstroke}{rgb}{0.000000,0.000000,0.000000}%
\pgfsetstrokecolor{currentstroke}%
\pgfsetdash{}{0pt}%
\pgfsys@defobject{currentmarker}{\pgfqpoint{-0.048611in}{0.000000in}}{\pgfqpoint{-0.000000in}{0.000000in}}{%
\pgfpathmoveto{\pgfqpoint{-0.000000in}{0.000000in}}%
\pgfpathlineto{\pgfqpoint{-0.048611in}{0.000000in}}%
\pgfusepath{stroke,fill}%
}%
\begin{pgfscope}%
\pgfsys@transformshift{0.588387in}{1.002015in}%
\pgfsys@useobject{currentmarker}{}%
\end{pgfscope}%
\end{pgfscope}%
\begin{pgfscope}%
\definecolor{textcolor}{rgb}{0.000000,0.000000,0.000000}%
\pgfsetstrokecolor{textcolor}%
\pgfsetfillcolor{textcolor}%
\pgftext[x=0.289968in, y=0.949253in, left, base]{\color{textcolor}{\rmfamily\fontsize{10.000000}{12.000000}\selectfont\catcode`\^=\active\def^{\ifmmode\sp\else\^{}\fi}\catcode`\%=\active\def%{\%}$\mathdefault{10^{1}}$}}%
\end{pgfscope}%
\begin{pgfscope}%
\pgfsetbuttcap%
\pgfsetroundjoin%
\definecolor{currentfill}{rgb}{0.000000,0.000000,0.000000}%
\pgfsetfillcolor{currentfill}%
\pgfsetlinewidth{0.803000pt}%
\definecolor{currentstroke}{rgb}{0.000000,0.000000,0.000000}%
\pgfsetstrokecolor{currentstroke}%
\pgfsetdash{}{0pt}%
\pgfsys@defobject{currentmarker}{\pgfqpoint{-0.048611in}{0.000000in}}{\pgfqpoint{-0.000000in}{0.000000in}}{%
\pgfpathmoveto{\pgfqpoint{-0.000000in}{0.000000in}}%
\pgfpathlineto{\pgfqpoint{-0.048611in}{0.000000in}}%
\pgfusepath{stroke,fill}%
}%
\begin{pgfscope}%
\pgfsys@transformshift{0.588387in}{1.532229in}%
\pgfsys@useobject{currentmarker}{}%
\end{pgfscope}%
\end{pgfscope}%
\begin{pgfscope}%
\definecolor{textcolor}{rgb}{0.000000,0.000000,0.000000}%
\pgfsetstrokecolor{textcolor}%
\pgfsetfillcolor{textcolor}%
\pgftext[x=0.289968in, y=1.479467in, left, base]{\color{textcolor}{\rmfamily\fontsize{10.000000}{12.000000}\selectfont\catcode`\^=\active\def^{\ifmmode\sp\else\^{}\fi}\catcode`\%=\active\def%{\%}$\mathdefault{10^{2}}$}}%
\end{pgfscope}%
\begin{pgfscope}%
\pgfsetbuttcap%
\pgfsetroundjoin%
\definecolor{currentfill}{rgb}{0.000000,0.000000,0.000000}%
\pgfsetfillcolor{currentfill}%
\pgfsetlinewidth{0.803000pt}%
\definecolor{currentstroke}{rgb}{0.000000,0.000000,0.000000}%
\pgfsetstrokecolor{currentstroke}%
\pgfsetdash{}{0pt}%
\pgfsys@defobject{currentmarker}{\pgfqpoint{-0.048611in}{0.000000in}}{\pgfqpoint{-0.000000in}{0.000000in}}{%
\pgfpathmoveto{\pgfqpoint{-0.000000in}{0.000000in}}%
\pgfpathlineto{\pgfqpoint{-0.048611in}{0.000000in}}%
\pgfusepath{stroke,fill}%
}%
\begin{pgfscope}%
\pgfsys@transformshift{0.588387in}{2.062443in}%
\pgfsys@useobject{currentmarker}{}%
\end{pgfscope}%
\end{pgfscope}%
\begin{pgfscope}%
\definecolor{textcolor}{rgb}{0.000000,0.000000,0.000000}%
\pgfsetstrokecolor{textcolor}%
\pgfsetfillcolor{textcolor}%
\pgftext[x=0.289968in, y=2.009681in, left, base]{\color{textcolor}{\rmfamily\fontsize{10.000000}{12.000000}\selectfont\catcode`\^=\active\def^{\ifmmode\sp\else\^{}\fi}\catcode`\%=\active\def%{\%}$\mathdefault{10^{3}}$}}%
\end{pgfscope}%
\begin{pgfscope}%
\pgfsetbuttcap%
\pgfsetroundjoin%
\definecolor{currentfill}{rgb}{0.000000,0.000000,0.000000}%
\pgfsetfillcolor{currentfill}%
\pgfsetlinewidth{0.602250pt}%
\definecolor{currentstroke}{rgb}{0.000000,0.000000,0.000000}%
\pgfsetstrokecolor{currentstroke}%
\pgfsetdash{}{0pt}%
\pgfsys@defobject{currentmarker}{\pgfqpoint{-0.027778in}{0.000000in}}{\pgfqpoint{-0.000000in}{0.000000in}}{%
\pgfpathmoveto{\pgfqpoint{-0.000000in}{0.000000in}}%
\pgfpathlineto{\pgfqpoint{-0.027778in}{0.000000in}}%
\pgfusepath{stroke,fill}%
}%
\begin{pgfscope}%
\pgfsys@transformshift{0.588387in}{0.631411in}%
\pgfsys@useobject{currentmarker}{}%
\end{pgfscope}%
\end{pgfscope}%
\begin{pgfscope}%
\pgfsetbuttcap%
\pgfsetroundjoin%
\definecolor{currentfill}{rgb}{0.000000,0.000000,0.000000}%
\pgfsetfillcolor{currentfill}%
\pgfsetlinewidth{0.602250pt}%
\definecolor{currentstroke}{rgb}{0.000000,0.000000,0.000000}%
\pgfsetstrokecolor{currentstroke}%
\pgfsetdash{}{0pt}%
\pgfsys@defobject{currentmarker}{\pgfqpoint{-0.027778in}{0.000000in}}{\pgfqpoint{-0.000000in}{0.000000in}}{%
\pgfpathmoveto{\pgfqpoint{-0.000000in}{0.000000in}}%
\pgfpathlineto{\pgfqpoint{-0.027778in}{0.000000in}}%
\pgfusepath{stroke,fill}%
}%
\begin{pgfscope}%
\pgfsys@transformshift{0.588387in}{0.724777in}%
\pgfsys@useobject{currentmarker}{}%
\end{pgfscope}%
\end{pgfscope}%
\begin{pgfscope}%
\pgfsetbuttcap%
\pgfsetroundjoin%
\definecolor{currentfill}{rgb}{0.000000,0.000000,0.000000}%
\pgfsetfillcolor{currentfill}%
\pgfsetlinewidth{0.602250pt}%
\definecolor{currentstroke}{rgb}{0.000000,0.000000,0.000000}%
\pgfsetstrokecolor{currentstroke}%
\pgfsetdash{}{0pt}%
\pgfsys@defobject{currentmarker}{\pgfqpoint{-0.027778in}{0.000000in}}{\pgfqpoint{-0.000000in}{0.000000in}}{%
\pgfpathmoveto{\pgfqpoint{-0.000000in}{0.000000in}}%
\pgfpathlineto{\pgfqpoint{-0.027778in}{0.000000in}}%
\pgfusepath{stroke,fill}%
}%
\begin{pgfscope}%
\pgfsys@transformshift{0.588387in}{0.791022in}%
\pgfsys@useobject{currentmarker}{}%
\end{pgfscope}%
\end{pgfscope}%
\begin{pgfscope}%
\pgfsetbuttcap%
\pgfsetroundjoin%
\definecolor{currentfill}{rgb}{0.000000,0.000000,0.000000}%
\pgfsetfillcolor{currentfill}%
\pgfsetlinewidth{0.602250pt}%
\definecolor{currentstroke}{rgb}{0.000000,0.000000,0.000000}%
\pgfsetstrokecolor{currentstroke}%
\pgfsetdash{}{0pt}%
\pgfsys@defobject{currentmarker}{\pgfqpoint{-0.027778in}{0.000000in}}{\pgfqpoint{-0.000000in}{0.000000in}}{%
\pgfpathmoveto{\pgfqpoint{-0.000000in}{0.000000in}}%
\pgfpathlineto{\pgfqpoint{-0.027778in}{0.000000in}}%
\pgfusepath{stroke,fill}%
}%
\begin{pgfscope}%
\pgfsys@transformshift{0.588387in}{0.842405in}%
\pgfsys@useobject{currentmarker}{}%
\end{pgfscope}%
\end{pgfscope}%
\begin{pgfscope}%
\pgfsetbuttcap%
\pgfsetroundjoin%
\definecolor{currentfill}{rgb}{0.000000,0.000000,0.000000}%
\pgfsetfillcolor{currentfill}%
\pgfsetlinewidth{0.602250pt}%
\definecolor{currentstroke}{rgb}{0.000000,0.000000,0.000000}%
\pgfsetstrokecolor{currentstroke}%
\pgfsetdash{}{0pt}%
\pgfsys@defobject{currentmarker}{\pgfqpoint{-0.027778in}{0.000000in}}{\pgfqpoint{-0.000000in}{0.000000in}}{%
\pgfpathmoveto{\pgfqpoint{-0.000000in}{0.000000in}}%
\pgfpathlineto{\pgfqpoint{-0.027778in}{0.000000in}}%
\pgfusepath{stroke,fill}%
}%
\begin{pgfscope}%
\pgfsys@transformshift{0.588387in}{0.884388in}%
\pgfsys@useobject{currentmarker}{}%
\end{pgfscope}%
\end{pgfscope}%
\begin{pgfscope}%
\pgfsetbuttcap%
\pgfsetroundjoin%
\definecolor{currentfill}{rgb}{0.000000,0.000000,0.000000}%
\pgfsetfillcolor{currentfill}%
\pgfsetlinewidth{0.602250pt}%
\definecolor{currentstroke}{rgb}{0.000000,0.000000,0.000000}%
\pgfsetstrokecolor{currentstroke}%
\pgfsetdash{}{0pt}%
\pgfsys@defobject{currentmarker}{\pgfqpoint{-0.027778in}{0.000000in}}{\pgfqpoint{-0.000000in}{0.000000in}}{%
\pgfpathmoveto{\pgfqpoint{-0.000000in}{0.000000in}}%
\pgfpathlineto{\pgfqpoint{-0.027778in}{0.000000in}}%
\pgfusepath{stroke,fill}%
}%
\begin{pgfscope}%
\pgfsys@transformshift{0.588387in}{0.919884in}%
\pgfsys@useobject{currentmarker}{}%
\end{pgfscope}%
\end{pgfscope}%
\begin{pgfscope}%
\pgfsetbuttcap%
\pgfsetroundjoin%
\definecolor{currentfill}{rgb}{0.000000,0.000000,0.000000}%
\pgfsetfillcolor{currentfill}%
\pgfsetlinewidth{0.602250pt}%
\definecolor{currentstroke}{rgb}{0.000000,0.000000,0.000000}%
\pgfsetstrokecolor{currentstroke}%
\pgfsetdash{}{0pt}%
\pgfsys@defobject{currentmarker}{\pgfqpoint{-0.027778in}{0.000000in}}{\pgfqpoint{-0.000000in}{0.000000in}}{%
\pgfpathmoveto{\pgfqpoint{-0.000000in}{0.000000in}}%
\pgfpathlineto{\pgfqpoint{-0.027778in}{0.000000in}}%
\pgfusepath{stroke,fill}%
}%
\begin{pgfscope}%
\pgfsys@transformshift{0.588387in}{0.950632in}%
\pgfsys@useobject{currentmarker}{}%
\end{pgfscope}%
\end{pgfscope}%
\begin{pgfscope}%
\pgfsetbuttcap%
\pgfsetroundjoin%
\definecolor{currentfill}{rgb}{0.000000,0.000000,0.000000}%
\pgfsetfillcolor{currentfill}%
\pgfsetlinewidth{0.602250pt}%
\definecolor{currentstroke}{rgb}{0.000000,0.000000,0.000000}%
\pgfsetstrokecolor{currentstroke}%
\pgfsetdash{}{0pt}%
\pgfsys@defobject{currentmarker}{\pgfqpoint{-0.027778in}{0.000000in}}{\pgfqpoint{-0.000000in}{0.000000in}}{%
\pgfpathmoveto{\pgfqpoint{-0.000000in}{0.000000in}}%
\pgfpathlineto{\pgfqpoint{-0.027778in}{0.000000in}}%
\pgfusepath{stroke,fill}%
}%
\begin{pgfscope}%
\pgfsys@transformshift{0.588387in}{0.977754in}%
\pgfsys@useobject{currentmarker}{}%
\end{pgfscope}%
\end{pgfscope}%
\begin{pgfscope}%
\pgfsetbuttcap%
\pgfsetroundjoin%
\definecolor{currentfill}{rgb}{0.000000,0.000000,0.000000}%
\pgfsetfillcolor{currentfill}%
\pgfsetlinewidth{0.602250pt}%
\definecolor{currentstroke}{rgb}{0.000000,0.000000,0.000000}%
\pgfsetstrokecolor{currentstroke}%
\pgfsetdash{}{0pt}%
\pgfsys@defobject{currentmarker}{\pgfqpoint{-0.027778in}{0.000000in}}{\pgfqpoint{-0.000000in}{0.000000in}}{%
\pgfpathmoveto{\pgfqpoint{-0.000000in}{0.000000in}}%
\pgfpathlineto{\pgfqpoint{-0.027778in}{0.000000in}}%
\pgfusepath{stroke,fill}%
}%
\begin{pgfscope}%
\pgfsys@transformshift{0.588387in}{1.161625in}%
\pgfsys@useobject{currentmarker}{}%
\end{pgfscope}%
\end{pgfscope}%
\begin{pgfscope}%
\pgfsetbuttcap%
\pgfsetroundjoin%
\definecolor{currentfill}{rgb}{0.000000,0.000000,0.000000}%
\pgfsetfillcolor{currentfill}%
\pgfsetlinewidth{0.602250pt}%
\definecolor{currentstroke}{rgb}{0.000000,0.000000,0.000000}%
\pgfsetstrokecolor{currentstroke}%
\pgfsetdash{}{0pt}%
\pgfsys@defobject{currentmarker}{\pgfqpoint{-0.027778in}{0.000000in}}{\pgfqpoint{-0.000000in}{0.000000in}}{%
\pgfpathmoveto{\pgfqpoint{-0.000000in}{0.000000in}}%
\pgfpathlineto{\pgfqpoint{-0.027778in}{0.000000in}}%
\pgfusepath{stroke,fill}%
}%
\begin{pgfscope}%
\pgfsys@transformshift{0.588387in}{1.254991in}%
\pgfsys@useobject{currentmarker}{}%
\end{pgfscope}%
\end{pgfscope}%
\begin{pgfscope}%
\pgfsetbuttcap%
\pgfsetroundjoin%
\definecolor{currentfill}{rgb}{0.000000,0.000000,0.000000}%
\pgfsetfillcolor{currentfill}%
\pgfsetlinewidth{0.602250pt}%
\definecolor{currentstroke}{rgb}{0.000000,0.000000,0.000000}%
\pgfsetstrokecolor{currentstroke}%
\pgfsetdash{}{0pt}%
\pgfsys@defobject{currentmarker}{\pgfqpoint{-0.027778in}{0.000000in}}{\pgfqpoint{-0.000000in}{0.000000in}}{%
\pgfpathmoveto{\pgfqpoint{-0.000000in}{0.000000in}}%
\pgfpathlineto{\pgfqpoint{-0.027778in}{0.000000in}}%
\pgfusepath{stroke,fill}%
}%
\begin{pgfscope}%
\pgfsys@transformshift{0.588387in}{1.321236in}%
\pgfsys@useobject{currentmarker}{}%
\end{pgfscope}%
\end{pgfscope}%
\begin{pgfscope}%
\pgfsetbuttcap%
\pgfsetroundjoin%
\definecolor{currentfill}{rgb}{0.000000,0.000000,0.000000}%
\pgfsetfillcolor{currentfill}%
\pgfsetlinewidth{0.602250pt}%
\definecolor{currentstroke}{rgb}{0.000000,0.000000,0.000000}%
\pgfsetstrokecolor{currentstroke}%
\pgfsetdash{}{0pt}%
\pgfsys@defobject{currentmarker}{\pgfqpoint{-0.027778in}{0.000000in}}{\pgfqpoint{-0.000000in}{0.000000in}}{%
\pgfpathmoveto{\pgfqpoint{-0.000000in}{0.000000in}}%
\pgfpathlineto{\pgfqpoint{-0.027778in}{0.000000in}}%
\pgfusepath{stroke,fill}%
}%
\begin{pgfscope}%
\pgfsys@transformshift{0.588387in}{1.372619in}%
\pgfsys@useobject{currentmarker}{}%
\end{pgfscope}%
\end{pgfscope}%
\begin{pgfscope}%
\pgfsetbuttcap%
\pgfsetroundjoin%
\definecolor{currentfill}{rgb}{0.000000,0.000000,0.000000}%
\pgfsetfillcolor{currentfill}%
\pgfsetlinewidth{0.602250pt}%
\definecolor{currentstroke}{rgb}{0.000000,0.000000,0.000000}%
\pgfsetstrokecolor{currentstroke}%
\pgfsetdash{}{0pt}%
\pgfsys@defobject{currentmarker}{\pgfqpoint{-0.027778in}{0.000000in}}{\pgfqpoint{-0.000000in}{0.000000in}}{%
\pgfpathmoveto{\pgfqpoint{-0.000000in}{0.000000in}}%
\pgfpathlineto{\pgfqpoint{-0.027778in}{0.000000in}}%
\pgfusepath{stroke,fill}%
}%
\begin{pgfscope}%
\pgfsys@transformshift{0.588387in}{1.414602in}%
\pgfsys@useobject{currentmarker}{}%
\end{pgfscope}%
\end{pgfscope}%
\begin{pgfscope}%
\pgfsetbuttcap%
\pgfsetroundjoin%
\definecolor{currentfill}{rgb}{0.000000,0.000000,0.000000}%
\pgfsetfillcolor{currentfill}%
\pgfsetlinewidth{0.602250pt}%
\definecolor{currentstroke}{rgb}{0.000000,0.000000,0.000000}%
\pgfsetstrokecolor{currentstroke}%
\pgfsetdash{}{0pt}%
\pgfsys@defobject{currentmarker}{\pgfqpoint{-0.027778in}{0.000000in}}{\pgfqpoint{-0.000000in}{0.000000in}}{%
\pgfpathmoveto{\pgfqpoint{-0.000000in}{0.000000in}}%
\pgfpathlineto{\pgfqpoint{-0.027778in}{0.000000in}}%
\pgfusepath{stroke,fill}%
}%
\begin{pgfscope}%
\pgfsys@transformshift{0.588387in}{1.450098in}%
\pgfsys@useobject{currentmarker}{}%
\end{pgfscope}%
\end{pgfscope}%
\begin{pgfscope}%
\pgfsetbuttcap%
\pgfsetroundjoin%
\definecolor{currentfill}{rgb}{0.000000,0.000000,0.000000}%
\pgfsetfillcolor{currentfill}%
\pgfsetlinewidth{0.602250pt}%
\definecolor{currentstroke}{rgb}{0.000000,0.000000,0.000000}%
\pgfsetstrokecolor{currentstroke}%
\pgfsetdash{}{0pt}%
\pgfsys@defobject{currentmarker}{\pgfqpoint{-0.027778in}{0.000000in}}{\pgfqpoint{-0.000000in}{0.000000in}}{%
\pgfpathmoveto{\pgfqpoint{-0.000000in}{0.000000in}}%
\pgfpathlineto{\pgfqpoint{-0.027778in}{0.000000in}}%
\pgfusepath{stroke,fill}%
}%
\begin{pgfscope}%
\pgfsys@transformshift{0.588387in}{1.480846in}%
\pgfsys@useobject{currentmarker}{}%
\end{pgfscope}%
\end{pgfscope}%
\begin{pgfscope}%
\pgfsetbuttcap%
\pgfsetroundjoin%
\definecolor{currentfill}{rgb}{0.000000,0.000000,0.000000}%
\pgfsetfillcolor{currentfill}%
\pgfsetlinewidth{0.602250pt}%
\definecolor{currentstroke}{rgb}{0.000000,0.000000,0.000000}%
\pgfsetstrokecolor{currentstroke}%
\pgfsetdash{}{0pt}%
\pgfsys@defobject{currentmarker}{\pgfqpoint{-0.027778in}{0.000000in}}{\pgfqpoint{-0.000000in}{0.000000in}}{%
\pgfpathmoveto{\pgfqpoint{-0.000000in}{0.000000in}}%
\pgfpathlineto{\pgfqpoint{-0.027778in}{0.000000in}}%
\pgfusepath{stroke,fill}%
}%
\begin{pgfscope}%
\pgfsys@transformshift{0.588387in}{1.507968in}%
\pgfsys@useobject{currentmarker}{}%
\end{pgfscope}%
\end{pgfscope}%
\begin{pgfscope}%
\pgfsetbuttcap%
\pgfsetroundjoin%
\definecolor{currentfill}{rgb}{0.000000,0.000000,0.000000}%
\pgfsetfillcolor{currentfill}%
\pgfsetlinewidth{0.602250pt}%
\definecolor{currentstroke}{rgb}{0.000000,0.000000,0.000000}%
\pgfsetstrokecolor{currentstroke}%
\pgfsetdash{}{0pt}%
\pgfsys@defobject{currentmarker}{\pgfqpoint{-0.027778in}{0.000000in}}{\pgfqpoint{-0.000000in}{0.000000in}}{%
\pgfpathmoveto{\pgfqpoint{-0.000000in}{0.000000in}}%
\pgfpathlineto{\pgfqpoint{-0.027778in}{0.000000in}}%
\pgfusepath{stroke,fill}%
}%
\begin{pgfscope}%
\pgfsys@transformshift{0.588387in}{1.691839in}%
\pgfsys@useobject{currentmarker}{}%
\end{pgfscope}%
\end{pgfscope}%
\begin{pgfscope}%
\pgfsetbuttcap%
\pgfsetroundjoin%
\definecolor{currentfill}{rgb}{0.000000,0.000000,0.000000}%
\pgfsetfillcolor{currentfill}%
\pgfsetlinewidth{0.602250pt}%
\definecolor{currentstroke}{rgb}{0.000000,0.000000,0.000000}%
\pgfsetstrokecolor{currentstroke}%
\pgfsetdash{}{0pt}%
\pgfsys@defobject{currentmarker}{\pgfqpoint{-0.027778in}{0.000000in}}{\pgfqpoint{-0.000000in}{0.000000in}}{%
\pgfpathmoveto{\pgfqpoint{-0.000000in}{0.000000in}}%
\pgfpathlineto{\pgfqpoint{-0.027778in}{0.000000in}}%
\pgfusepath{stroke,fill}%
}%
\begin{pgfscope}%
\pgfsys@transformshift{0.588387in}{1.785205in}%
\pgfsys@useobject{currentmarker}{}%
\end{pgfscope}%
\end{pgfscope}%
\begin{pgfscope}%
\pgfsetbuttcap%
\pgfsetroundjoin%
\definecolor{currentfill}{rgb}{0.000000,0.000000,0.000000}%
\pgfsetfillcolor{currentfill}%
\pgfsetlinewidth{0.602250pt}%
\definecolor{currentstroke}{rgb}{0.000000,0.000000,0.000000}%
\pgfsetstrokecolor{currentstroke}%
\pgfsetdash{}{0pt}%
\pgfsys@defobject{currentmarker}{\pgfqpoint{-0.027778in}{0.000000in}}{\pgfqpoint{-0.000000in}{0.000000in}}{%
\pgfpathmoveto{\pgfqpoint{-0.000000in}{0.000000in}}%
\pgfpathlineto{\pgfqpoint{-0.027778in}{0.000000in}}%
\pgfusepath{stroke,fill}%
}%
\begin{pgfscope}%
\pgfsys@transformshift{0.588387in}{1.851450in}%
\pgfsys@useobject{currentmarker}{}%
\end{pgfscope}%
\end{pgfscope}%
\begin{pgfscope}%
\pgfsetbuttcap%
\pgfsetroundjoin%
\definecolor{currentfill}{rgb}{0.000000,0.000000,0.000000}%
\pgfsetfillcolor{currentfill}%
\pgfsetlinewidth{0.602250pt}%
\definecolor{currentstroke}{rgb}{0.000000,0.000000,0.000000}%
\pgfsetstrokecolor{currentstroke}%
\pgfsetdash{}{0pt}%
\pgfsys@defobject{currentmarker}{\pgfqpoint{-0.027778in}{0.000000in}}{\pgfqpoint{-0.000000in}{0.000000in}}{%
\pgfpathmoveto{\pgfqpoint{-0.000000in}{0.000000in}}%
\pgfpathlineto{\pgfqpoint{-0.027778in}{0.000000in}}%
\pgfusepath{stroke,fill}%
}%
\begin{pgfscope}%
\pgfsys@transformshift{0.588387in}{1.902833in}%
\pgfsys@useobject{currentmarker}{}%
\end{pgfscope}%
\end{pgfscope}%
\begin{pgfscope}%
\pgfsetbuttcap%
\pgfsetroundjoin%
\definecolor{currentfill}{rgb}{0.000000,0.000000,0.000000}%
\pgfsetfillcolor{currentfill}%
\pgfsetlinewidth{0.602250pt}%
\definecolor{currentstroke}{rgb}{0.000000,0.000000,0.000000}%
\pgfsetstrokecolor{currentstroke}%
\pgfsetdash{}{0pt}%
\pgfsys@defobject{currentmarker}{\pgfqpoint{-0.027778in}{0.000000in}}{\pgfqpoint{-0.000000in}{0.000000in}}{%
\pgfpathmoveto{\pgfqpoint{-0.000000in}{0.000000in}}%
\pgfpathlineto{\pgfqpoint{-0.027778in}{0.000000in}}%
\pgfusepath{stroke,fill}%
}%
\begin{pgfscope}%
\pgfsys@transformshift{0.588387in}{1.944816in}%
\pgfsys@useobject{currentmarker}{}%
\end{pgfscope}%
\end{pgfscope}%
\begin{pgfscope}%
\pgfsetbuttcap%
\pgfsetroundjoin%
\definecolor{currentfill}{rgb}{0.000000,0.000000,0.000000}%
\pgfsetfillcolor{currentfill}%
\pgfsetlinewidth{0.602250pt}%
\definecolor{currentstroke}{rgb}{0.000000,0.000000,0.000000}%
\pgfsetstrokecolor{currentstroke}%
\pgfsetdash{}{0pt}%
\pgfsys@defobject{currentmarker}{\pgfqpoint{-0.027778in}{0.000000in}}{\pgfqpoint{-0.000000in}{0.000000in}}{%
\pgfpathmoveto{\pgfqpoint{-0.000000in}{0.000000in}}%
\pgfpathlineto{\pgfqpoint{-0.027778in}{0.000000in}}%
\pgfusepath{stroke,fill}%
}%
\begin{pgfscope}%
\pgfsys@transformshift{0.588387in}{1.980312in}%
\pgfsys@useobject{currentmarker}{}%
\end{pgfscope}%
\end{pgfscope}%
\begin{pgfscope}%
\pgfsetbuttcap%
\pgfsetroundjoin%
\definecolor{currentfill}{rgb}{0.000000,0.000000,0.000000}%
\pgfsetfillcolor{currentfill}%
\pgfsetlinewidth{0.602250pt}%
\definecolor{currentstroke}{rgb}{0.000000,0.000000,0.000000}%
\pgfsetstrokecolor{currentstroke}%
\pgfsetdash{}{0pt}%
\pgfsys@defobject{currentmarker}{\pgfqpoint{-0.027778in}{0.000000in}}{\pgfqpoint{-0.000000in}{0.000000in}}{%
\pgfpathmoveto{\pgfqpoint{-0.000000in}{0.000000in}}%
\pgfpathlineto{\pgfqpoint{-0.027778in}{0.000000in}}%
\pgfusepath{stroke,fill}%
}%
\begin{pgfscope}%
\pgfsys@transformshift{0.588387in}{2.011060in}%
\pgfsys@useobject{currentmarker}{}%
\end{pgfscope}%
\end{pgfscope}%
\begin{pgfscope}%
\pgfsetbuttcap%
\pgfsetroundjoin%
\definecolor{currentfill}{rgb}{0.000000,0.000000,0.000000}%
\pgfsetfillcolor{currentfill}%
\pgfsetlinewidth{0.602250pt}%
\definecolor{currentstroke}{rgb}{0.000000,0.000000,0.000000}%
\pgfsetstrokecolor{currentstroke}%
\pgfsetdash{}{0pt}%
\pgfsys@defobject{currentmarker}{\pgfqpoint{-0.027778in}{0.000000in}}{\pgfqpoint{-0.000000in}{0.000000in}}{%
\pgfpathmoveto{\pgfqpoint{-0.000000in}{0.000000in}}%
\pgfpathlineto{\pgfqpoint{-0.027778in}{0.000000in}}%
\pgfusepath{stroke,fill}%
}%
\begin{pgfscope}%
\pgfsys@transformshift{0.588387in}{2.038182in}%
\pgfsys@useobject{currentmarker}{}%
\end{pgfscope}%
\end{pgfscope}%
\begin{pgfscope}%
\pgfsetbuttcap%
\pgfsetroundjoin%
\definecolor{currentfill}{rgb}{0.000000,0.000000,0.000000}%
\pgfsetfillcolor{currentfill}%
\pgfsetlinewidth{0.602250pt}%
\definecolor{currentstroke}{rgb}{0.000000,0.000000,0.000000}%
\pgfsetstrokecolor{currentstroke}%
\pgfsetdash{}{0pt}%
\pgfsys@defobject{currentmarker}{\pgfqpoint{-0.027778in}{0.000000in}}{\pgfqpoint{-0.000000in}{0.000000in}}{%
\pgfpathmoveto{\pgfqpoint{-0.000000in}{0.000000in}}%
\pgfpathlineto{\pgfqpoint{-0.027778in}{0.000000in}}%
\pgfusepath{stroke,fill}%
}%
\begin{pgfscope}%
\pgfsys@transformshift{0.588387in}{2.222053in}%
\pgfsys@useobject{currentmarker}{}%
\end{pgfscope}%
\end{pgfscope}%
\begin{pgfscope}%
\pgfsetbuttcap%
\pgfsetroundjoin%
\definecolor{currentfill}{rgb}{0.000000,0.000000,0.000000}%
\pgfsetfillcolor{currentfill}%
\pgfsetlinewidth{0.602250pt}%
\definecolor{currentstroke}{rgb}{0.000000,0.000000,0.000000}%
\pgfsetstrokecolor{currentstroke}%
\pgfsetdash{}{0pt}%
\pgfsys@defobject{currentmarker}{\pgfqpoint{-0.027778in}{0.000000in}}{\pgfqpoint{-0.000000in}{0.000000in}}{%
\pgfpathmoveto{\pgfqpoint{-0.000000in}{0.000000in}}%
\pgfpathlineto{\pgfqpoint{-0.027778in}{0.000000in}}%
\pgfusepath{stroke,fill}%
}%
\begin{pgfscope}%
\pgfsys@transformshift{0.588387in}{2.315419in}%
\pgfsys@useobject{currentmarker}{}%
\end{pgfscope}%
\end{pgfscope}%
\begin{pgfscope}%
\pgfsetbuttcap%
\pgfsetroundjoin%
\definecolor{currentfill}{rgb}{0.000000,0.000000,0.000000}%
\pgfsetfillcolor{currentfill}%
\pgfsetlinewidth{0.602250pt}%
\definecolor{currentstroke}{rgb}{0.000000,0.000000,0.000000}%
\pgfsetstrokecolor{currentstroke}%
\pgfsetdash{}{0pt}%
\pgfsys@defobject{currentmarker}{\pgfqpoint{-0.027778in}{0.000000in}}{\pgfqpoint{-0.000000in}{0.000000in}}{%
\pgfpathmoveto{\pgfqpoint{-0.000000in}{0.000000in}}%
\pgfpathlineto{\pgfqpoint{-0.027778in}{0.000000in}}%
\pgfusepath{stroke,fill}%
}%
\begin{pgfscope}%
\pgfsys@transformshift{0.588387in}{2.381664in}%
\pgfsys@useobject{currentmarker}{}%
\end{pgfscope}%
\end{pgfscope}%
\begin{pgfscope}%
\pgfsetbuttcap%
\pgfsetroundjoin%
\definecolor{currentfill}{rgb}{0.000000,0.000000,0.000000}%
\pgfsetfillcolor{currentfill}%
\pgfsetlinewidth{0.602250pt}%
\definecolor{currentstroke}{rgb}{0.000000,0.000000,0.000000}%
\pgfsetstrokecolor{currentstroke}%
\pgfsetdash{}{0pt}%
\pgfsys@defobject{currentmarker}{\pgfqpoint{-0.027778in}{0.000000in}}{\pgfqpoint{-0.000000in}{0.000000in}}{%
\pgfpathmoveto{\pgfqpoint{-0.000000in}{0.000000in}}%
\pgfpathlineto{\pgfqpoint{-0.027778in}{0.000000in}}%
\pgfusepath{stroke,fill}%
}%
\begin{pgfscope}%
\pgfsys@transformshift{0.588387in}{2.433047in}%
\pgfsys@useobject{currentmarker}{}%
\end{pgfscope}%
\end{pgfscope}%
\begin{pgfscope}%
\pgfsetbuttcap%
\pgfsetroundjoin%
\definecolor{currentfill}{rgb}{0.000000,0.000000,0.000000}%
\pgfsetfillcolor{currentfill}%
\pgfsetlinewidth{0.602250pt}%
\definecolor{currentstroke}{rgb}{0.000000,0.000000,0.000000}%
\pgfsetstrokecolor{currentstroke}%
\pgfsetdash{}{0pt}%
\pgfsys@defobject{currentmarker}{\pgfqpoint{-0.027778in}{0.000000in}}{\pgfqpoint{-0.000000in}{0.000000in}}{%
\pgfpathmoveto{\pgfqpoint{-0.000000in}{0.000000in}}%
\pgfpathlineto{\pgfqpoint{-0.027778in}{0.000000in}}%
\pgfusepath{stroke,fill}%
}%
\begin{pgfscope}%
\pgfsys@transformshift{0.588387in}{2.475030in}%
\pgfsys@useobject{currentmarker}{}%
\end{pgfscope}%
\end{pgfscope}%
\begin{pgfscope}%
\pgfsetbuttcap%
\pgfsetroundjoin%
\definecolor{currentfill}{rgb}{0.000000,0.000000,0.000000}%
\pgfsetfillcolor{currentfill}%
\pgfsetlinewidth{0.602250pt}%
\definecolor{currentstroke}{rgb}{0.000000,0.000000,0.000000}%
\pgfsetstrokecolor{currentstroke}%
\pgfsetdash{}{0pt}%
\pgfsys@defobject{currentmarker}{\pgfqpoint{-0.027778in}{0.000000in}}{\pgfqpoint{-0.000000in}{0.000000in}}{%
\pgfpathmoveto{\pgfqpoint{-0.000000in}{0.000000in}}%
\pgfpathlineto{\pgfqpoint{-0.027778in}{0.000000in}}%
\pgfusepath{stroke,fill}%
}%
\begin{pgfscope}%
\pgfsys@transformshift{0.588387in}{2.510526in}%
\pgfsys@useobject{currentmarker}{}%
\end{pgfscope}%
\end{pgfscope}%
\begin{pgfscope}%
\definecolor{textcolor}{rgb}{0.000000,0.000000,0.000000}%
\pgfsetstrokecolor{textcolor}%
\pgfsetfillcolor{textcolor}%
\pgftext[x=0.234413in,y=1.526746in,,bottom,rotate=90.000000]{\color{textcolor}{\rmfamily\fontsize{10.000000}{12.000000}\selectfont\catcode`\^=\active\def^{\ifmmode\sp\else\^{}\fi}\catcode`\%=\active\def%{\%}Time [ms]}}%
\end{pgfscope}%
\begin{pgfscope}%
\pgfpathrectangle{\pgfqpoint{0.588387in}{0.521603in}}{\pgfqpoint{5.399676in}{2.010285in}}%
\pgfusepath{clip}%
\pgfsetrectcap%
\pgfsetroundjoin%
\pgfsetlinewidth{1.505625pt}%
\pgfsetstrokecolor{currentstroke1}%
\pgfsetdash{}{0pt}%
\pgfpathmoveto{\pgfqpoint{0.833827in}{0.815283in}}%
\pgfpathlineto{\pgfqpoint{0.881485in}{0.820103in}}%
\pgfpathlineto{\pgfqpoint{0.929143in}{0.774311in}}%
\pgfpathlineto{\pgfqpoint{0.976802in}{0.711616in}}%
\pgfpathlineto{\pgfqpoint{1.024460in}{0.652903in}}%
\pgfpathlineto{\pgfqpoint{1.072118in}{0.669245in}}%
\pgfpathlineto{\pgfqpoint{1.119776in}{0.612980in}}%
\pgfpathlineto{\pgfqpoint{1.167435in}{0.637005in}}%
\pgfpathlineto{\pgfqpoint{1.215093in}{0.653041in}}%
\pgfpathlineto{\pgfqpoint{1.262751in}{0.669588in}}%
\pgfpathlineto{\pgfqpoint{1.310409in}{0.687855in}}%
\pgfpathlineto{\pgfqpoint{1.358067in}{0.681579in}}%
\pgfpathlineto{\pgfqpoint{1.405726in}{0.733709in}}%
\pgfpathlineto{\pgfqpoint{1.453384in}{0.718290in}}%
\pgfpathlineto{\pgfqpoint{1.501042in}{0.746089in}}%
\pgfpathlineto{\pgfqpoint{1.548700in}{0.754500in}}%
\pgfpathlineto{\pgfqpoint{1.596358in}{0.782797in}}%
\pgfpathlineto{\pgfqpoint{1.644017in}{0.789355in}}%
\pgfpathlineto{\pgfqpoint{1.691675in}{0.802766in}}%
\pgfpathlineto{\pgfqpoint{1.739333in}{0.802506in}}%
\pgfpathlineto{\pgfqpoint{1.786991in}{0.823204in}}%
\pgfpathlineto{\pgfqpoint{1.834650in}{0.819601in}}%
\pgfpathlineto{\pgfqpoint{1.882308in}{0.845265in}}%
\pgfpathlineto{\pgfqpoint{1.929966in}{0.842405in}}%
\pgfpathlineto{\pgfqpoint{1.977624in}{0.852265in}}%
\pgfpathlineto{\pgfqpoint{2.025282in}{0.843449in}}%
\pgfpathlineto{\pgfqpoint{2.072941in}{0.877302in}}%
\pgfpathlineto{\pgfqpoint{2.120599in}{0.912531in}}%
\pgfpathlineto{\pgfqpoint{2.168257in}{0.895318in}}%
\pgfpathlineto{\pgfqpoint{2.215915in}{0.883648in}}%
\pgfpathlineto{\pgfqpoint{2.263573in}{0.948221in}}%
\pgfpathlineto{\pgfqpoint{2.311232in}{0.902442in}}%
\pgfpathlineto{\pgfqpoint{2.358890in}{0.943321in}}%
\pgfpathlineto{\pgfqpoint{2.406548in}{0.926125in}}%
\pgfpathlineto{\pgfqpoint{2.454206in}{0.962649in}}%
\pgfpathlineto{\pgfqpoint{2.501865in}{0.940604in}}%
\pgfpathlineto{\pgfqpoint{2.549523in}{0.934745in}}%
\pgfpathlineto{\pgfqpoint{2.597181in}{0.944527in}}%
\pgfpathlineto{\pgfqpoint{2.644839in}{1.056015in}}%
\pgfpathlineto{\pgfqpoint{2.692497in}{0.954116in}}%
\pgfpathlineto{\pgfqpoint{2.740156in}{0.975416in}}%
\pgfpathlineto{\pgfqpoint{2.787814in}{0.949962in}}%
\pgfpathlineto{\pgfqpoint{2.835472in}{0.980930in}}%
\pgfpathlineto{\pgfqpoint{2.883130in}{0.981811in}}%
\pgfpathlineto{\pgfqpoint{2.930789in}{0.981657in}}%
\pgfpathlineto{\pgfqpoint{2.978447in}{0.992924in}}%
\pgfpathlineto{\pgfqpoint{3.026105in}{1.195802in}}%
\pgfpathlineto{\pgfqpoint{3.073763in}{0.992215in}}%
\pgfpathlineto{\pgfqpoint{3.121421in}{1.002015in}}%
\pgfpathlineto{\pgfqpoint{3.169080in}{1.003224in}}%
\pgfpathlineto{\pgfqpoint{3.216738in}{1.023962in}}%
\pgfpathlineto{\pgfqpoint{3.264396in}{1.014092in}}%
\pgfpathlineto{\pgfqpoint{3.312054in}{1.051548in}}%
\pgfpathlineto{\pgfqpoint{3.359712in}{1.020292in}}%
\pgfpathlineto{\pgfqpoint{3.407371in}{1.029137in}}%
\pgfpathlineto{\pgfqpoint{3.455029in}{1.058394in}}%
\pgfpathlineto{\pgfqpoint{3.502687in}{1.039694in}}%
\pgfpathlineto{\pgfqpoint{3.550345in}{1.036364in}}%
\pgfpathlineto{\pgfqpoint{3.598004in}{1.064946in}}%
\pgfpathlineto{\pgfqpoint{3.645662in}{1.043998in}}%
\pgfpathlineto{\pgfqpoint{3.693320in}{1.095381in}}%
\pgfpathlineto{\pgfqpoint{3.740978in}{1.076573in}}%
\pgfpathlineto{\pgfqpoint{3.788636in}{1.110242in}}%
\pgfpathlineto{\pgfqpoint{3.836295in}{1.059688in}}%
\pgfpathlineto{\pgfqpoint{3.931611in}{1.102931in}}%
\pgfpathlineto{\pgfqpoint{4.026927in}{1.072335in}}%
\pgfpathlineto{\pgfqpoint{4.122244in}{1.097925in}}%
\pgfpathlineto{\pgfqpoint{4.217560in}{1.110242in}}%
\pgfusepath{stroke}%
\end{pgfscope}%
\begin{pgfscope}%
\pgfpathrectangle{\pgfqpoint{0.588387in}{0.521603in}}{\pgfqpoint{5.399676in}{2.010285in}}%
\pgfusepath{clip}%
\pgfsetrectcap%
\pgfsetroundjoin%
\pgfsetlinewidth{1.505625pt}%
\pgfsetstrokecolor{currentstroke2}%
\pgfsetdash{}{0pt}%
\pgfpathmoveto{\pgfqpoint{0.833827in}{0.862441in}}%
\pgfpathlineto{\pgfqpoint{0.881485in}{0.843146in}}%
\pgfpathlineto{\pgfqpoint{0.929143in}{0.825950in}}%
\pgfpathlineto{\pgfqpoint{0.976802in}{0.752710in}}%
\pgfpathlineto{\pgfqpoint{1.024460in}{0.726454in}}%
\pgfpathlineto{\pgfqpoint{1.072118in}{0.706133in}}%
\pgfpathlineto{\pgfqpoint{1.119776in}{0.675202in}}%
\pgfpathlineto{\pgfqpoint{1.167435in}{0.705504in}}%
\pgfpathlineto{\pgfqpoint{1.215093in}{0.737702in}}%
\pgfpathlineto{\pgfqpoint{1.262751in}{0.742328in}}%
\pgfpathlineto{\pgfqpoint{1.310409in}{0.887641in}}%
\pgfpathlineto{\pgfqpoint{1.358067in}{0.890626in}}%
\pgfpathlineto{\pgfqpoint{1.405726in}{0.959212in}}%
\pgfpathlineto{\pgfqpoint{1.453384in}{0.967525in}}%
\pgfpathlineto{\pgfqpoint{1.501042in}{1.058540in}}%
\pgfpathlineto{\pgfqpoint{1.548700in}{1.061231in}}%
\pgfpathlineto{\pgfqpoint{1.596358in}{1.037018in}}%
\pgfpathlineto{\pgfqpoint{1.644017in}{1.032960in}}%
\pgfpathlineto{\pgfqpoint{1.691675in}{1.079128in}}%
\pgfpathlineto{\pgfqpoint{1.739333in}{1.107716in}}%
\pgfpathlineto{\pgfqpoint{1.786991in}{1.126113in}}%
\pgfpathlineto{\pgfqpoint{1.834650in}{1.153129in}}%
\pgfpathlineto{\pgfqpoint{1.882308in}{1.134953in}}%
\pgfpathlineto{\pgfqpoint{1.929966in}{1.135302in}}%
\pgfpathlineto{\pgfqpoint{1.977624in}{1.170362in}}%
\pgfpathlineto{\pgfqpoint{2.025282in}{1.194546in}}%
\pgfpathlineto{\pgfqpoint{2.072941in}{1.238074in}}%
\pgfpathlineto{\pgfqpoint{2.120599in}{1.231836in}}%
\pgfpathlineto{\pgfqpoint{2.168257in}{1.227959in}}%
\pgfpathlineto{\pgfqpoint{2.215915in}{1.219557in}}%
\pgfpathlineto{\pgfqpoint{2.263573in}{1.258089in}}%
\pgfpathlineto{\pgfqpoint{2.311232in}{1.274775in}}%
\pgfpathlineto{\pgfqpoint{2.358890in}{1.299558in}}%
\pgfpathlineto{\pgfqpoint{2.406548in}{1.313753in}}%
\pgfpathlineto{\pgfqpoint{2.454206in}{1.289152in}}%
\pgfpathlineto{\pgfqpoint{2.501865in}{1.432731in}}%
\pgfpathlineto{\pgfqpoint{2.549523in}{1.321044in}}%
\pgfpathlineto{\pgfqpoint{2.597181in}{1.335573in}}%
\pgfpathlineto{\pgfqpoint{2.644839in}{1.358895in}}%
\pgfpathlineto{\pgfqpoint{2.692497in}{1.370333in}}%
\pgfpathlineto{\pgfqpoint{2.740156in}{1.348590in}}%
\pgfpathlineto{\pgfqpoint{2.787814in}{1.351078in}}%
\pgfpathlineto{\pgfqpoint{2.835472in}{1.390340in}}%
\pgfpathlineto{\pgfqpoint{2.883130in}{1.384160in}}%
\pgfpathlineto{\pgfqpoint{2.930789in}{1.412526in}}%
\pgfpathlineto{\pgfqpoint{2.978447in}{1.421342in}}%
\pgfpathlineto{\pgfqpoint{3.026105in}{1.400047in}}%
\pgfpathlineto{\pgfqpoint{3.073763in}{1.408142in}}%
\pgfpathlineto{\pgfqpoint{3.121421in}{1.417056in}}%
\pgfpathlineto{\pgfqpoint{3.169080in}{1.443823in}}%
\pgfpathlineto{\pgfqpoint{3.216738in}{1.468926in}}%
\pgfpathlineto{\pgfqpoint{3.264396in}{1.468801in}}%
\pgfpathlineto{\pgfqpoint{3.312054in}{1.461333in}}%
\pgfpathlineto{\pgfqpoint{3.359712in}{1.461877in}}%
\pgfpathlineto{\pgfqpoint{3.407371in}{1.465216in}}%
\pgfpathlineto{\pgfqpoint{3.455029in}{1.481310in}}%
\pgfpathlineto{\pgfqpoint{3.502687in}{1.502866in}}%
\pgfpathlineto{\pgfqpoint{3.550345in}{1.507494in}}%
\pgfpathlineto{\pgfqpoint{3.598004in}{1.509958in}}%
\pgfpathlineto{\pgfqpoint{3.645662in}{1.499533in}}%
\pgfpathlineto{\pgfqpoint{3.693320in}{1.533377in}}%
\pgfpathlineto{\pgfqpoint{3.740978in}{1.525545in}}%
\pgfpathlineto{\pgfqpoint{3.788636in}{1.562401in}}%
\pgfpathlineto{\pgfqpoint{3.836295in}{1.544339in}}%
\pgfpathlineto{\pgfqpoint{3.883953in}{1.542915in}}%
\pgfpathlineto{\pgfqpoint{3.931611in}{1.542915in}}%
\pgfpathlineto{\pgfqpoint{3.979269in}{1.572043in}}%
\pgfpathlineto{\pgfqpoint{4.026927in}{1.555457in}}%
\pgfpathlineto{\pgfqpoint{4.074586in}{1.588172in}}%
\pgfpathlineto{\pgfqpoint{4.122244in}{1.581484in}}%
\pgfpathlineto{\pgfqpoint{4.169902in}{1.554176in}}%
\pgfpathlineto{\pgfqpoint{4.217560in}{1.583213in}}%
\pgfpathlineto{\pgfqpoint{4.265219in}{1.589073in}}%
\pgfpathlineto{\pgfqpoint{4.312877in}{1.591534in}}%
\pgfpathlineto{\pgfqpoint{4.408193in}{1.623053in}}%
\pgfpathlineto{\pgfqpoint{4.455851in}{1.602355in}}%
\pgfpathlineto{\pgfqpoint{4.503510in}{1.612772in}}%
\pgfpathlineto{\pgfqpoint{4.598826in}{1.627713in}}%
\pgfpathlineto{\pgfqpoint{4.646484in}{1.639374in}}%
\pgfpathlineto{\pgfqpoint{4.694142in}{1.651828in}}%
\pgfpathlineto{\pgfqpoint{4.741801in}{1.638650in}}%
\pgfpathlineto{\pgfqpoint{4.789459in}{1.637111in}}%
\pgfpathlineto{\pgfqpoint{4.837117in}{1.647542in}}%
\pgfpathlineto{\pgfqpoint{4.884775in}{1.657277in}}%
\pgfpathlineto{\pgfqpoint{4.980092in}{1.672505in}}%
\pgfpathlineto{\pgfqpoint{5.027750in}{1.682963in}}%
\pgfpathlineto{\pgfqpoint{5.075408in}{1.664543in}}%
\pgfpathlineto{\pgfqpoint{5.123066in}{1.679421in}}%
\pgfpathlineto{\pgfqpoint{5.170725in}{1.690974in}}%
\pgfpathlineto{\pgfqpoint{5.218383in}{1.691263in}}%
\pgfpathlineto{\pgfqpoint{5.266041in}{1.716388in}}%
\pgfpathlineto{\pgfqpoint{5.361357in}{1.691311in}}%
\pgfpathlineto{\pgfqpoint{5.409016in}{1.709694in}}%
\pgfpathlineto{\pgfqpoint{5.456674in}{1.724647in}}%
\pgfpathlineto{\pgfqpoint{5.551990in}{1.719642in}}%
\pgfpathlineto{\pgfqpoint{5.647307in}{1.746651in}}%
\pgfpathlineto{\pgfqpoint{5.694965in}{1.701975in}}%
\pgfpathlineto{\pgfqpoint{5.742623in}{1.762961in}}%
\pgfusepath{stroke}%
\end{pgfscope}%
\begin{pgfscope}%
\pgfpathrectangle{\pgfqpoint{0.588387in}{0.521603in}}{\pgfqpoint{5.399676in}{2.010285in}}%
\pgfusepath{clip}%
\pgfsetrectcap%
\pgfsetroundjoin%
\pgfsetlinewidth{1.505625pt}%
\pgfsetstrokecolor{currentstroke3}%
\pgfsetdash{}{0pt}%
\pgfpathmoveto{\pgfqpoint{0.833827in}{0.820969in}}%
\pgfpathlineto{\pgfqpoint{0.881485in}{0.842405in}}%
\pgfpathlineto{\pgfqpoint{0.929143in}{0.797828in}}%
\pgfpathlineto{\pgfqpoint{0.976802in}{0.745168in}}%
\pgfpathlineto{\pgfqpoint{1.024460in}{0.718004in}}%
\pgfpathlineto{\pgfqpoint{1.072118in}{0.711616in}}%
\pgfpathlineto{\pgfqpoint{1.119776in}{0.668923in}}%
\pgfpathlineto{\pgfqpoint{1.167435in}{0.690953in}}%
\pgfpathlineto{\pgfqpoint{1.215093in}{0.724777in}}%
\pgfpathlineto{\pgfqpoint{1.262751in}{0.726032in}}%
\pgfpathlineto{\pgfqpoint{1.310409in}{0.884388in}}%
\pgfpathlineto{\pgfqpoint{1.358067in}{0.896678in}}%
\pgfpathlineto{\pgfqpoint{1.405726in}{0.978187in}}%
\pgfpathlineto{\pgfqpoint{1.453384in}{0.967582in}}%
\pgfpathlineto{\pgfqpoint{1.501042in}{1.073127in}}%
\pgfpathlineto{\pgfqpoint{1.548700in}{1.081242in}}%
\pgfpathlineto{\pgfqpoint{1.596358in}{1.149601in}}%
\pgfpathlineto{\pgfqpoint{1.644017in}{1.125371in}}%
\pgfpathlineto{\pgfqpoint{1.691675in}{1.259586in}}%
\pgfpathlineto{\pgfqpoint{1.739333in}{1.320646in}}%
\pgfpathlineto{\pgfqpoint{1.786991in}{1.344574in}}%
\pgfpathlineto{\pgfqpoint{1.834650in}{1.374584in}}%
\pgfpathlineto{\pgfqpoint{1.882308in}{1.373194in}}%
\pgfpathlineto{\pgfqpoint{1.929966in}{1.401686in}}%
\pgfpathlineto{\pgfqpoint{1.977624in}{1.208650in}}%
\pgfpathlineto{\pgfqpoint{2.025282in}{1.489134in}}%
\pgfpathlineto{\pgfqpoint{2.072941in}{1.716999in}}%
\pgfpathlineto{\pgfqpoint{2.120599in}{1.639504in}}%
\pgfpathlineto{\pgfqpoint{2.168257in}{1.754384in}}%
\pgfpathlineto{\pgfqpoint{2.215915in}{1.662579in}}%
\pgfpathlineto{\pgfqpoint{2.263573in}{1.621073in}}%
\pgfpathlineto{\pgfqpoint{2.311232in}{1.697883in}}%
\pgfpathlineto{\pgfqpoint{2.358890in}{1.731249in}}%
\pgfpathlineto{\pgfqpoint{2.406548in}{1.914497in}}%
\pgfpathlineto{\pgfqpoint{2.454206in}{1.965530in}}%
\pgfpathlineto{\pgfqpoint{2.501865in}{1.882538in}}%
\pgfpathlineto{\pgfqpoint{2.549523in}{1.776869in}}%
\pgfpathlineto{\pgfqpoint{2.597181in}{1.935731in}}%
\pgfpathlineto{\pgfqpoint{2.644839in}{2.014286in}}%
\pgfpathlineto{\pgfqpoint{2.692497in}{2.015013in}}%
\pgfpathlineto{\pgfqpoint{2.740156in}{1.906488in}}%
\pgfpathlineto{\pgfqpoint{2.787814in}{1.851583in}}%
\pgfpathlineto{\pgfqpoint{2.835472in}{1.793590in}}%
\pgfpathlineto{\pgfqpoint{2.883130in}{1.912289in}}%
\pgfpathlineto{\pgfqpoint{2.930789in}{2.041523in}}%
\pgfpathlineto{\pgfqpoint{2.978447in}{2.119084in}}%
\pgfpathlineto{\pgfqpoint{3.026105in}{1.997301in}}%
\pgfpathlineto{\pgfqpoint{3.073763in}{1.820628in}}%
\pgfpathlineto{\pgfqpoint{3.121421in}{2.092730in}}%
\pgfpathlineto{\pgfqpoint{3.169080in}{1.878975in}}%
\pgfpathlineto{\pgfqpoint{3.216738in}{1.976473in}}%
\pgfpathlineto{\pgfqpoint{3.264396in}{2.113861in}}%
\pgfpathlineto{\pgfqpoint{3.312054in}{1.836343in}}%
\pgfpathlineto{\pgfqpoint{3.359712in}{1.966064in}}%
\pgfpathlineto{\pgfqpoint{3.407371in}{1.490016in}}%
\pgfpathlineto{\pgfqpoint{3.455029in}{2.017117in}}%
\pgfpathlineto{\pgfqpoint{3.502687in}{1.669418in}}%
\pgfpathlineto{\pgfqpoint{3.550345in}{2.029083in}}%
\pgfpathlineto{\pgfqpoint{3.598004in}{2.021986in}}%
\pgfpathlineto{\pgfqpoint{3.645662in}{1.749062in}}%
\pgfpathlineto{\pgfqpoint{3.693320in}{1.882025in}}%
\pgfpathlineto{\pgfqpoint{3.740978in}{2.118024in}}%
\pgfpathlineto{\pgfqpoint{3.788636in}{1.704168in}}%
\pgfpathlineto{\pgfqpoint{3.836295in}{2.031467in}}%
\pgfpathlineto{\pgfqpoint{3.883953in}{1.579898in}}%
\pgfpathlineto{\pgfqpoint{3.931611in}{2.035026in}}%
\pgfpathlineto{\pgfqpoint{3.979269in}{1.681839in}}%
\pgfpathlineto{\pgfqpoint{4.026927in}{2.075267in}}%
\pgfpathlineto{\pgfqpoint{4.074586in}{1.729301in}}%
\pgfpathlineto{\pgfqpoint{4.122244in}{2.037721in}}%
\pgfpathlineto{\pgfqpoint{4.169902in}{1.620420in}}%
\pgfpathlineto{\pgfqpoint{4.217560in}{2.025543in}}%
\pgfpathlineto{\pgfqpoint{4.265219in}{2.140743in}}%
\pgfpathlineto{\pgfqpoint{4.312877in}{1.965392in}}%
\pgfpathlineto{\pgfqpoint{4.408193in}{2.128792in}}%
\pgfpathlineto{\pgfqpoint{4.455851in}{2.237698in}}%
\pgfpathlineto{\pgfqpoint{4.503510in}{2.200125in}}%
\pgfpathlineto{\pgfqpoint{4.598826in}{2.167933in}}%
\pgfpathlineto{\pgfqpoint{4.646484in}{1.676363in}}%
\pgfpathlineto{\pgfqpoint{4.694142in}{2.121650in}}%
\pgfpathlineto{\pgfqpoint{4.741801in}{2.144563in}}%
\pgfpathlineto{\pgfqpoint{4.789459in}{2.146864in}}%
\pgfpathlineto{\pgfqpoint{4.837117in}{1.689525in}}%
\pgfpathlineto{\pgfqpoint{4.884775in}{2.294460in}}%
\pgfpathlineto{\pgfqpoint{4.980092in}{2.084898in}}%
\pgfpathlineto{\pgfqpoint{5.027750in}{2.161794in}}%
\pgfpathlineto{\pgfqpoint{5.075408in}{2.273265in}}%
\pgfpathlineto{\pgfqpoint{5.123066in}{1.905352in}}%
\pgfpathlineto{\pgfqpoint{5.170725in}{2.156919in}}%
\pgfpathlineto{\pgfqpoint{5.218383in}{2.304576in}}%
\pgfpathlineto{\pgfqpoint{5.266041in}{2.222769in}}%
\pgfpathlineto{\pgfqpoint{5.361357in}{2.211652in}}%
\pgfpathlineto{\pgfqpoint{5.409016in}{1.734301in}}%
\pgfpathlineto{\pgfqpoint{5.456674in}{2.338584in}}%
\pgfpathlineto{\pgfqpoint{5.551990in}{2.293871in}}%
\pgfpathlineto{\pgfqpoint{5.647307in}{1.891021in}}%
\pgfpathlineto{\pgfqpoint{5.694965in}{1.741373in}}%
\pgfpathlineto{\pgfqpoint{5.742623in}{2.294250in}}%
\pgfusepath{stroke}%
\end{pgfscope}%
\begin{pgfscope}%
\pgfpathrectangle{\pgfqpoint{0.588387in}{0.521603in}}{\pgfqpoint{5.399676in}{2.010285in}}%
\pgfusepath{clip}%
\pgfsetrectcap%
\pgfsetroundjoin%
\pgfsetlinewidth{1.505625pt}%
\pgfsetstrokecolor{currentstroke4}%
\pgfsetdash{}{0pt}%
\pgfpathmoveto{\pgfqpoint{0.833827in}{0.823886in}}%
\pgfpathlineto{\pgfqpoint{0.881485in}{0.841661in}}%
\pgfpathlineto{\pgfqpoint{0.929143in}{0.822945in}}%
\pgfpathlineto{\pgfqpoint{0.976802in}{0.750269in}}%
\pgfpathlineto{\pgfqpoint{1.024460in}{0.721386in}}%
\pgfpathlineto{\pgfqpoint{1.072118in}{0.715834in}}%
\pgfpathlineto{\pgfqpoint{1.119776in}{0.676165in}}%
\pgfpathlineto{\pgfqpoint{1.167435in}{0.707928in}}%
\pgfpathlineto{\pgfqpoint{1.215093in}{0.735318in}}%
\pgfpathlineto{\pgfqpoint{1.262751in}{0.745085in}}%
\pgfpathlineto{\pgfqpoint{1.310409in}{0.880613in}}%
\pgfpathlineto{\pgfqpoint{1.358067in}{0.891790in}}%
\pgfpathlineto{\pgfqpoint{1.405726in}{0.955484in}}%
\pgfpathlineto{\pgfqpoint{1.453384in}{0.963024in}}%
\pgfpathlineto{\pgfqpoint{1.501042in}{1.040096in}}%
\pgfpathlineto{\pgfqpoint{1.548700in}{1.049695in}}%
\pgfpathlineto{\pgfqpoint{1.596358in}{1.032059in}}%
\pgfpathlineto{\pgfqpoint{1.644017in}{1.025707in}}%
\pgfpathlineto{\pgfqpoint{1.691675in}{1.075252in}}%
\pgfpathlineto{\pgfqpoint{1.739333in}{1.100323in}}%
\pgfpathlineto{\pgfqpoint{1.786991in}{1.117328in}}%
\pgfpathlineto{\pgfqpoint{1.834650in}{1.151743in}}%
\pgfpathlineto{\pgfqpoint{1.882308in}{1.135759in}}%
\pgfpathlineto{\pgfqpoint{1.929966in}{1.135687in}}%
\pgfpathlineto{\pgfqpoint{1.977624in}{1.166784in}}%
\pgfpathlineto{\pgfqpoint{2.025282in}{1.191004in}}%
\pgfpathlineto{\pgfqpoint{2.072941in}{1.236662in}}%
\pgfpathlineto{\pgfqpoint{2.120599in}{1.222570in}}%
\pgfpathlineto{\pgfqpoint{2.168257in}{1.227239in}}%
\pgfpathlineto{\pgfqpoint{2.215915in}{1.219083in}}%
\pgfpathlineto{\pgfqpoint{2.263573in}{1.257694in}}%
\pgfpathlineto{\pgfqpoint{2.311232in}{1.277864in}}%
\pgfpathlineto{\pgfqpoint{2.358890in}{1.289113in}}%
\pgfpathlineto{\pgfqpoint{2.406548in}{1.309453in}}%
\pgfpathlineto{\pgfqpoint{2.454206in}{1.288679in}}%
\pgfpathlineto{\pgfqpoint{2.501865in}{1.299720in}}%
\pgfpathlineto{\pgfqpoint{2.549523in}{1.323717in}}%
\pgfpathlineto{\pgfqpoint{2.597181in}{1.328159in}}%
\pgfpathlineto{\pgfqpoint{2.644839in}{1.356350in}}%
\pgfpathlineto{\pgfqpoint{2.692497in}{1.367134in}}%
\pgfpathlineto{\pgfqpoint{2.740156in}{1.347542in}}%
\pgfpathlineto{\pgfqpoint{2.787814in}{1.346415in}}%
\pgfpathlineto{\pgfqpoint{2.835472in}{1.380123in}}%
\pgfpathlineto{\pgfqpoint{2.883130in}{1.384467in}}%
\pgfpathlineto{\pgfqpoint{2.930789in}{1.414970in}}%
\pgfpathlineto{\pgfqpoint{2.978447in}{1.419350in}}%
\pgfpathlineto{\pgfqpoint{3.026105in}{1.400252in}}%
\pgfpathlineto{\pgfqpoint{3.073763in}{1.403658in}}%
\pgfpathlineto{\pgfqpoint{3.121421in}{1.415012in}}%
\pgfpathlineto{\pgfqpoint{3.169080in}{1.435997in}}%
\pgfpathlineto{\pgfqpoint{3.216738in}{1.469899in}}%
\pgfpathlineto{\pgfqpoint{3.264396in}{1.462325in}}%
\pgfpathlineto{\pgfqpoint{3.312054in}{1.453526in}}%
\pgfpathlineto{\pgfqpoint{3.359712in}{1.455993in}}%
\pgfpathlineto{\pgfqpoint{3.407371in}{1.461333in}}%
\pgfpathlineto{\pgfqpoint{3.455029in}{1.478958in}}%
\pgfpathlineto{\pgfqpoint{3.502687in}{1.502356in}}%
\pgfpathlineto{\pgfqpoint{3.550345in}{1.503418in}}%
\pgfpathlineto{\pgfqpoint{3.598004in}{1.511774in}}%
\pgfpathlineto{\pgfqpoint{3.645662in}{1.498413in}}%
\pgfpathlineto{\pgfqpoint{3.693320in}{1.514152in}}%
\pgfpathlineto{\pgfqpoint{3.740978in}{1.523793in}}%
\pgfpathlineto{\pgfqpoint{3.788636in}{1.549417in}}%
\pgfpathlineto{\pgfqpoint{3.836295in}{1.539259in}}%
\pgfpathlineto{\pgfqpoint{3.883953in}{1.529041in}}%
\pgfpathlineto{\pgfqpoint{3.931611in}{1.541295in}}%
\pgfpathlineto{\pgfqpoint{3.979269in}{1.579898in}}%
\pgfpathlineto{\pgfqpoint{4.026927in}{1.557576in}}%
\pgfpathlineto{\pgfqpoint{4.074586in}{1.580987in}}%
\pgfpathlineto{\pgfqpoint{4.122244in}{1.579304in}}%
\pgfpathlineto{\pgfqpoint{4.169902in}{1.562232in}}%
\pgfpathlineto{\pgfqpoint{4.217560in}{1.575933in}}%
\pgfpathlineto{\pgfqpoint{4.265219in}{1.582458in}}%
\pgfpathlineto{\pgfqpoint{4.312877in}{1.587795in}}%
\pgfpathlineto{\pgfqpoint{4.408193in}{1.614401in}}%
\pgfpathlineto{\pgfqpoint{4.455851in}{1.604552in}}%
\pgfpathlineto{\pgfqpoint{4.503510in}{1.606985in}}%
\pgfpathlineto{\pgfqpoint{4.598826in}{1.623879in}}%
\pgfpathlineto{\pgfqpoint{4.646484in}{1.629778in}}%
\pgfpathlineto{\pgfqpoint{4.694142in}{1.641557in}}%
\pgfpathlineto{\pgfqpoint{4.741801in}{1.637924in}}%
\pgfpathlineto{\pgfqpoint{4.789459in}{1.635787in}}%
\pgfpathlineto{\pgfqpoint{4.837117in}{1.636830in}}%
\pgfpathlineto{\pgfqpoint{4.884775in}{1.650489in}}%
\pgfpathlineto{\pgfqpoint{4.980092in}{1.663010in}}%
\pgfpathlineto{\pgfqpoint{5.027750in}{1.662648in}}%
\pgfpathlineto{\pgfqpoint{5.075408in}{1.660997in}}%
\pgfpathlineto{\pgfqpoint{5.123066in}{1.665651in}}%
\pgfpathlineto{\pgfqpoint{5.170725in}{1.681839in}}%
\pgfpathlineto{\pgfqpoint{5.218383in}{1.691695in}}%
\pgfpathlineto{\pgfqpoint{5.266041in}{1.709443in}}%
\pgfpathlineto{\pgfqpoint{5.361357in}{1.681839in}}%
\pgfpathlineto{\pgfqpoint{5.409016in}{1.701837in}}%
\pgfpathlineto{\pgfqpoint{5.456674in}{1.710027in}}%
\pgfpathlineto{\pgfqpoint{5.551990in}{1.715913in}}%
\pgfpathlineto{\pgfqpoint{5.647307in}{1.724022in}}%
\pgfpathlineto{\pgfqpoint{5.694965in}{1.701975in}}%
\pgfpathlineto{\pgfqpoint{5.742623in}{1.736920in}}%
\pgfusepath{stroke}%
\end{pgfscope}%
\begin{pgfscope}%
\pgfpathrectangle{\pgfqpoint{0.588387in}{0.521603in}}{\pgfqpoint{5.399676in}{2.010285in}}%
\pgfusepath{clip}%
\pgfsetrectcap%
\pgfsetroundjoin%
\pgfsetlinewidth{1.505625pt}%
\pgfsetstrokecolor{currentstroke5}%
\pgfsetdash{}{0pt}%
\pgfpathmoveto{\pgfqpoint{0.833827in}{0.822749in}}%
\pgfpathlineto{\pgfqpoint{0.881485in}{0.839414in}}%
\pgfpathlineto{\pgfqpoint{0.929143in}{0.800739in}}%
\pgfpathlineto{\pgfqpoint{0.976802in}{0.747802in}}%
\pgfpathlineto{\pgfqpoint{1.024460in}{0.713037in}}%
\pgfpathlineto{\pgfqpoint{1.072118in}{0.701731in}}%
\pgfpathlineto{\pgfqpoint{1.119776in}{0.661624in}}%
\pgfpathlineto{\pgfqpoint{1.167435in}{0.690399in}}%
\pgfpathlineto{\pgfqpoint{1.215093in}{0.716347in}}%
\pgfpathlineto{\pgfqpoint{1.262751in}{0.719547in}}%
\pgfpathlineto{\pgfqpoint{1.310409in}{0.861268in}}%
\pgfpathlineto{\pgfqpoint{1.358067in}{0.876288in}}%
\pgfpathlineto{\pgfqpoint{1.405726in}{0.936452in}}%
\pgfpathlineto{\pgfqpoint{1.453384in}{0.942654in}}%
\pgfpathlineto{\pgfqpoint{1.501042in}{1.031191in}}%
\pgfpathlineto{\pgfqpoint{1.548700in}{1.036763in}}%
\pgfpathlineto{\pgfqpoint{1.596358in}{1.018119in}}%
\pgfpathlineto{\pgfqpoint{1.644017in}{1.015987in}}%
\pgfpathlineto{\pgfqpoint{1.691675in}{1.080770in}}%
\pgfpathlineto{\pgfqpoint{1.739333in}{1.100839in}}%
\pgfpathlineto{\pgfqpoint{1.786991in}{1.118488in}}%
\pgfpathlineto{\pgfqpoint{1.834650in}{1.170498in}}%
\pgfpathlineto{\pgfqpoint{1.882308in}{1.123035in}}%
\pgfpathlineto{\pgfqpoint{1.929966in}{1.123138in}}%
\pgfpathlineto{\pgfqpoint{1.977624in}{1.170708in}}%
\pgfpathlineto{\pgfqpoint{2.025282in}{1.194830in}}%
\pgfpathlineto{\pgfqpoint{2.072941in}{1.235241in}}%
\pgfpathlineto{\pgfqpoint{2.120599in}{1.229833in}}%
\pgfpathlineto{\pgfqpoint{2.168257in}{1.211661in}}%
\pgfpathlineto{\pgfqpoint{2.215915in}{1.205309in}}%
\pgfpathlineto{\pgfqpoint{2.263573in}{1.250388in}}%
\pgfpathlineto{\pgfqpoint{2.311232in}{1.277494in}}%
\pgfpathlineto{\pgfqpoint{2.358890in}{1.296040in}}%
\pgfpathlineto{\pgfqpoint{2.406548in}{1.314068in}}%
\pgfpathlineto{\pgfqpoint{2.454206in}{1.276606in}}%
\pgfpathlineto{\pgfqpoint{2.501865in}{1.285308in}}%
\pgfpathlineto{\pgfqpoint{2.549523in}{1.315504in}}%
\pgfpathlineto{\pgfqpoint{2.597181in}{1.329034in}}%
\pgfpathlineto{\pgfqpoint{2.644839in}{1.359766in}}%
\pgfpathlineto{\pgfqpoint{2.692497in}{1.371803in}}%
\pgfpathlineto{\pgfqpoint{2.740156in}{1.342945in}}%
\pgfpathlineto{\pgfqpoint{2.787814in}{1.331352in}}%
\pgfpathlineto{\pgfqpoint{2.835472in}{1.373194in}}%
\pgfpathlineto{\pgfqpoint{2.883130in}{1.382004in}}%
\pgfpathlineto{\pgfqpoint{2.930789in}{1.407862in}}%
\pgfpathlineto{\pgfqpoint{2.978447in}{1.424210in}}%
\pgfpathlineto{\pgfqpoint{3.026105in}{1.379872in}}%
\pgfpathlineto{\pgfqpoint{3.073763in}{1.390222in}}%
\pgfpathlineto{\pgfqpoint{3.121421in}{1.409614in}}%
\pgfpathlineto{\pgfqpoint{3.169080in}{1.438332in}}%
\pgfpathlineto{\pgfqpoint{3.216738in}{1.467950in}}%
\pgfpathlineto{\pgfqpoint{3.264396in}{1.465364in}}%
\pgfpathlineto{\pgfqpoint{3.312054in}{1.437245in}}%
\pgfpathlineto{\pgfqpoint{3.359712in}{1.440385in}}%
\pgfpathlineto{\pgfqpoint{3.407371in}{1.451123in}}%
\pgfpathlineto{\pgfqpoint{3.455029in}{1.470399in}}%
\pgfpathlineto{\pgfqpoint{3.502687in}{1.489477in}}%
\pgfpathlineto{\pgfqpoint{3.550345in}{1.506517in}}%
\pgfpathlineto{\pgfqpoint{3.598004in}{1.481205in}}%
\pgfpathlineto{\pgfqpoint{3.645662in}{1.479288in}}%
\pgfpathlineto{\pgfqpoint{3.693320in}{1.506427in}}%
\pgfpathlineto{\pgfqpoint{3.740978in}{1.516273in}}%
\pgfpathlineto{\pgfqpoint{3.788636in}{1.543464in}}%
\pgfpathlineto{\pgfqpoint{3.836295in}{1.539184in}}%
\pgfpathlineto{\pgfqpoint{3.883953in}{1.523428in}}%
\pgfpathlineto{\pgfqpoint{3.931611in}{1.519203in}}%
\pgfpathlineto{\pgfqpoint{3.979269in}{1.551809in}}%
\pgfpathlineto{\pgfqpoint{4.026927in}{1.546976in}}%
\pgfpathlineto{\pgfqpoint{4.074586in}{1.568710in}}%
\pgfpathlineto{\pgfqpoint{4.122244in}{1.570537in}}%
\pgfpathlineto{\pgfqpoint{4.169902in}{1.526596in}}%
\pgfpathlineto{\pgfqpoint{4.217560in}{1.557501in}}%
\pgfpathlineto{\pgfqpoint{4.265219in}{1.570951in}}%
\pgfpathlineto{\pgfqpoint{4.312877in}{1.583458in}}%
\pgfpathlineto{\pgfqpoint{4.408193in}{1.615889in}}%
\pgfpathlineto{\pgfqpoint{4.455851in}{1.579148in}}%
\pgfpathlineto{\pgfqpoint{4.503510in}{1.581762in}}%
\pgfpathlineto{\pgfqpoint{4.598826in}{1.610752in}}%
\pgfpathlineto{\pgfqpoint{4.646484in}{1.616993in}}%
\pgfpathlineto{\pgfqpoint{4.694142in}{1.637438in}}%
\pgfpathlineto{\pgfqpoint{4.741801in}{1.610734in}}%
\pgfpathlineto{\pgfqpoint{4.789459in}{1.619795in}}%
\pgfpathlineto{\pgfqpoint{4.837117in}{1.624826in}}%
\pgfpathlineto{\pgfqpoint{4.884775in}{1.647542in}}%
\pgfpathlineto{\pgfqpoint{4.980092in}{1.657587in}}%
\pgfpathlineto{\pgfqpoint{5.027750in}{1.666536in}}%
\pgfpathlineto{\pgfqpoint{5.075408in}{1.640764in}}%
\pgfpathlineto{\pgfqpoint{5.123066in}{1.653398in}}%
\pgfpathlineto{\pgfqpoint{5.170725in}{1.674043in}}%
\pgfpathlineto{\pgfqpoint{5.218383in}{1.696540in}}%
\pgfpathlineto{\pgfqpoint{5.266041in}{1.698335in}}%
\pgfpathlineto{\pgfqpoint{5.361357in}{1.663979in}}%
\pgfpathlineto{\pgfqpoint{5.409016in}{1.684231in}}%
\pgfpathlineto{\pgfqpoint{5.456674in}{1.695763in}}%
\pgfpathlineto{\pgfqpoint{5.551990in}{1.700131in}}%
\pgfpathlineto{\pgfqpoint{5.647307in}{1.721252in}}%
\pgfpathlineto{\pgfqpoint{5.694965in}{1.685122in}}%
\pgfpathlineto{\pgfqpoint{5.742623in}{1.736446in}}%
\pgfusepath{stroke}%
\end{pgfscope}%
\begin{pgfscope}%
\pgfpathrectangle{\pgfqpoint{0.588387in}{0.521603in}}{\pgfqpoint{5.399676in}{2.010285in}}%
\pgfusepath{clip}%
\pgfsetrectcap%
\pgfsetroundjoin%
\pgfsetlinewidth{1.505625pt}%
\pgfsetstrokecolor{currentstroke6}%
\pgfsetdash{}{0pt}%
\pgfpathmoveto{\pgfqpoint{0.833827in}{0.815283in}}%
\pgfpathlineto{\pgfqpoint{0.881485in}{0.835208in}}%
\pgfpathlineto{\pgfqpoint{0.929143in}{0.786370in}}%
\pgfpathlineto{\pgfqpoint{0.976802in}{0.741232in}}%
\pgfpathlineto{\pgfqpoint{1.024460in}{0.712790in}}%
\pgfpathlineto{\pgfqpoint{1.072118in}{0.695590in}}%
\pgfpathlineto{\pgfqpoint{1.119776in}{0.676328in}}%
\pgfpathlineto{\pgfqpoint{1.167435in}{0.669588in}}%
\pgfpathlineto{\pgfqpoint{1.215093in}{0.700257in}}%
\pgfpathlineto{\pgfqpoint{1.262751in}{0.702326in}}%
\pgfpathlineto{\pgfqpoint{1.310409in}{0.861229in}}%
\pgfpathlineto{\pgfqpoint{1.358067in}{0.873531in}}%
\pgfpathlineto{\pgfqpoint{1.405726in}{0.950144in}}%
\pgfpathlineto{\pgfqpoint{1.453384in}{0.945375in}}%
\pgfpathlineto{\pgfqpoint{1.501042in}{1.043649in}}%
\pgfpathlineto{\pgfqpoint{1.548700in}{1.044764in}}%
\pgfpathlineto{\pgfqpoint{1.596358in}{1.178185in}}%
\pgfpathlineto{\pgfqpoint{1.644017in}{1.123158in}}%
\pgfpathlineto{\pgfqpoint{1.691675in}{1.266990in}}%
\pgfpathlineto{\pgfqpoint{1.739333in}{1.334208in}}%
\pgfpathlineto{\pgfqpoint{1.786991in}{1.354417in}}%
\pgfpathlineto{\pgfqpoint{1.834650in}{1.444821in}}%
\pgfpathlineto{\pgfqpoint{1.882308in}{1.566839in}}%
\pgfpathlineto{\pgfqpoint{1.929966in}{1.540252in}}%
\pgfpathlineto{\pgfqpoint{1.977624in}{1.731229in}}%
\pgfpathlineto{\pgfqpoint{2.025282in}{1.918197in}}%
\pgfpathlineto{\pgfqpoint{2.072941in}{2.113968in}}%
\pgfpathlineto{\pgfqpoint{2.120599in}{1.866792in}}%
\pgfpathlineto{\pgfqpoint{2.168257in}{2.158334in}}%
\pgfpathlineto{\pgfqpoint{2.215915in}{2.069649in}}%
\pgfpathlineto{\pgfqpoint{2.263573in}{2.135940in}}%
\pgfpathlineto{\pgfqpoint{2.311232in}{2.133060in}}%
\pgfpathlineto{\pgfqpoint{2.358890in}{2.302998in}}%
\pgfpathlineto{\pgfqpoint{2.406548in}{2.216618in}}%
\pgfpathlineto{\pgfqpoint{2.454206in}{2.209340in}}%
\pgfpathlineto{\pgfqpoint{2.501865in}{2.178446in}}%
\pgfpathlineto{\pgfqpoint{2.549523in}{2.281266in}}%
\pgfpathlineto{\pgfqpoint{2.597181in}{2.189523in}}%
\pgfpathlineto{\pgfqpoint{2.644839in}{2.192925in}}%
\pgfpathlineto{\pgfqpoint{2.692497in}{2.303831in}}%
\pgfpathlineto{\pgfqpoint{2.740156in}{2.330686in}}%
\pgfpathlineto{\pgfqpoint{2.787814in}{2.284187in}}%
\pgfpathlineto{\pgfqpoint{2.835472in}{2.127699in}}%
\pgfpathlineto{\pgfqpoint{2.883130in}{2.191479in}}%
\pgfpathlineto{\pgfqpoint{2.930789in}{2.314277in}}%
\pgfpathlineto{\pgfqpoint{2.978447in}{2.276916in}}%
\pgfpathlineto{\pgfqpoint{3.026105in}{2.234622in}}%
\pgfpathlineto{\pgfqpoint{3.073763in}{2.245021in}}%
\pgfpathlineto{\pgfqpoint{3.121421in}{2.312251in}}%
\pgfpathlineto{\pgfqpoint{3.169080in}{2.253804in}}%
\pgfpathlineto{\pgfqpoint{3.216738in}{2.316076in}}%
\pgfpathlineto{\pgfqpoint{3.264396in}{2.293563in}}%
\pgfpathlineto{\pgfqpoint{3.312054in}{2.322300in}}%
\pgfpathlineto{\pgfqpoint{3.359712in}{2.306914in}}%
\pgfpathlineto{\pgfqpoint{3.407371in}{2.295906in}}%
\pgfpathlineto{\pgfqpoint{3.455029in}{2.293621in}}%
\pgfpathlineto{\pgfqpoint{3.502687in}{2.212226in}}%
\pgfpathlineto{\pgfqpoint{3.550345in}{2.339742in}}%
\pgfpathlineto{\pgfqpoint{3.598004in}{2.402837in}}%
\pgfpathlineto{\pgfqpoint{3.645662in}{2.365837in}}%
\pgfpathlineto{\pgfqpoint{3.693320in}{2.385296in}}%
\pgfpathlineto{\pgfqpoint{3.740978in}{2.311484in}}%
\pgfpathlineto{\pgfqpoint{3.836295in}{2.387776in}}%
\pgfpathlineto{\pgfqpoint{3.883953in}{1.543464in}}%
\pgfpathlineto{\pgfqpoint{3.931611in}{2.382221in}}%
\pgfpathlineto{\pgfqpoint{4.026927in}{2.281467in}}%
\pgfpathlineto{\pgfqpoint{4.074586in}{2.365818in}}%
\pgfpathlineto{\pgfqpoint{4.122244in}{2.365180in}}%
\pgfpathlineto{\pgfqpoint{4.217560in}{2.345367in}}%
\pgfpathlineto{\pgfqpoint{4.265219in}{2.386491in}}%
\pgfpathlineto{\pgfqpoint{4.312877in}{2.348041in}}%
\pgfpathlineto{\pgfqpoint{4.408193in}{2.346894in}}%
\pgfpathlineto{\pgfqpoint{4.598826in}{2.366344in}}%
\pgfpathlineto{\pgfqpoint{4.646484in}{1.910532in}}%
\pgfpathlineto{\pgfqpoint{4.694142in}{2.389214in}}%
\pgfpathlineto{\pgfqpoint{4.741801in}{2.279885in}}%
\pgfpathlineto{\pgfqpoint{4.789459in}{2.356698in}}%
\pgfpathlineto{\pgfqpoint{4.884775in}{2.369352in}}%
\pgfpathlineto{\pgfqpoint{4.980092in}{2.399050in}}%
\pgfpathlineto{\pgfqpoint{5.075408in}{2.314573in}}%
\pgfpathlineto{\pgfqpoint{5.170725in}{2.337715in}}%
\pgfpathlineto{\pgfqpoint{5.266041in}{2.381580in}}%
\pgfpathlineto{\pgfqpoint{5.456674in}{2.440512in}}%
\pgfpathlineto{\pgfqpoint{5.551990in}{2.322462in}}%
\pgfusepath{stroke}%
\end{pgfscope}%
\begin{pgfscope}%
\pgfsetrectcap%
\pgfsetmiterjoin%
\pgfsetlinewidth{0.803000pt}%
\definecolor{currentstroke}{rgb}{0.000000,0.000000,0.000000}%
\pgfsetstrokecolor{currentstroke}%
\pgfsetdash{}{0pt}%
\pgfpathmoveto{\pgfqpoint{0.588387in}{0.521603in}}%
\pgfpathlineto{\pgfqpoint{0.588387in}{2.531888in}}%
\pgfusepath{stroke}%
\end{pgfscope}%
\begin{pgfscope}%
\pgfsetrectcap%
\pgfsetmiterjoin%
\pgfsetlinewidth{0.803000pt}%
\definecolor{currentstroke}{rgb}{0.000000,0.000000,0.000000}%
\pgfsetstrokecolor{currentstroke}%
\pgfsetdash{}{0pt}%
\pgfpathmoveto{\pgfqpoint{5.988063in}{0.521603in}}%
\pgfpathlineto{\pgfqpoint{5.988063in}{2.531888in}}%
\pgfusepath{stroke}%
\end{pgfscope}%
\begin{pgfscope}%
\pgfsetrectcap%
\pgfsetmiterjoin%
\pgfsetlinewidth{0.803000pt}%
\definecolor{currentstroke}{rgb}{0.000000,0.000000,0.000000}%
\pgfsetstrokecolor{currentstroke}%
\pgfsetdash{}{0pt}%
\pgfpathmoveto{\pgfqpoint{0.588387in}{0.521603in}}%
\pgfpathlineto{\pgfqpoint{5.988063in}{0.521603in}}%
\pgfusepath{stroke}%
\end{pgfscope}%
\begin{pgfscope}%
\pgfsetrectcap%
\pgfsetmiterjoin%
\pgfsetlinewidth{0.803000pt}%
\definecolor{currentstroke}{rgb}{0.000000,0.000000,0.000000}%
\pgfsetstrokecolor{currentstroke}%
\pgfsetdash{}{0pt}%
\pgfpathmoveto{\pgfqpoint{0.588387in}{2.531888in}}%
\pgfpathlineto{\pgfqpoint{5.988063in}{2.531888in}}%
\pgfusepath{stroke}%
\end{pgfscope}%
\begin{pgfscope}%
\definecolor{textcolor}{rgb}{0.000000,0.000000,0.000000}%
\pgfsetstrokecolor{textcolor}%
\pgfsetfillcolor{textcolor}%
\pgftext[x=3.288225in,y=2.615222in,,base]{\color{textcolor}{\rmfamily\fontsize{12.000000}{14.400000}\selectfont\catcode`\^=\active\def^{\ifmmode\sp\else\^{}\fi}\catcode`\%=\active\def%{\%}Mean}}%
\end{pgfscope}%
\begin{pgfscope}%
\pgfsetbuttcap%
\pgfsetmiterjoin%
\definecolor{currentfill}{rgb}{1.000000,1.000000,1.000000}%
\pgfsetfillcolor{currentfill}%
\pgfsetfillopacity{0.800000}%
\pgfsetlinewidth{1.003750pt}%
\definecolor{currentstroke}{rgb}{0.800000,0.800000,0.800000}%
\pgfsetstrokecolor{currentstroke}%
\pgfsetstrokeopacity{0.800000}%
\pgfsetdash{}{0pt}%
\pgfpathmoveto{\pgfqpoint{6.075563in}{1.320622in}}%
\pgfpathlineto{\pgfqpoint{8.259376in}{1.320622in}}%
\pgfpathquadraticcurveto{\pgfqpoint{8.284376in}{1.320622in}}{\pgfqpoint{8.284376in}{1.345622in}}%
\pgfpathlineto{\pgfqpoint{8.284376in}{2.444388in}}%
\pgfpathquadraticcurveto{\pgfqpoint{8.284376in}{2.469388in}}{\pgfqpoint{8.259376in}{2.469388in}}%
\pgfpathlineto{\pgfqpoint{6.075563in}{2.469388in}}%
\pgfpathquadraticcurveto{\pgfqpoint{6.050563in}{2.469388in}}{\pgfqpoint{6.050563in}{2.444388in}}%
\pgfpathlineto{\pgfqpoint{6.050563in}{1.345622in}}%
\pgfpathquadraticcurveto{\pgfqpoint{6.050563in}{1.320622in}}{\pgfqpoint{6.075563in}{1.320622in}}%
\pgfpathlineto{\pgfqpoint{6.075563in}{1.320622in}}%
\pgfpathclose%
\pgfusepath{stroke,fill}%
\end{pgfscope}%
\begin{pgfscope}%
\pgfsetrectcap%
\pgfsetroundjoin%
\pgfsetlinewidth{1.505625pt}%
\pgfsetstrokecolor{currentstroke1}%
\pgfsetdash{}{0pt}%
\pgfpathmoveto{\pgfqpoint{6.100563in}{2.368168in}}%
\pgfpathlineto{\pgfqpoint{6.225563in}{2.368168in}}%
\pgfpathlineto{\pgfqpoint{6.350563in}{2.368168in}}%
\pgfusepath{stroke}%
\end{pgfscope}%
\begin{pgfscope}%
\definecolor{textcolor}{rgb}{0.000000,0.000000,0.000000}%
\pgfsetstrokecolor{textcolor}%
\pgfsetfillcolor{textcolor}%
\pgftext[x=6.450563in,y=2.324418in,left,base]{\color{textcolor}{\rmfamily\fontsize{9.000000}{10.800000}\selectfont\catcode`\^=\active\def^{\ifmmode\sp\else\^{}\fi}\catcode`\%=\active\def%{\%}\NaiveCycles{}}}%
\end{pgfscope}%
\begin{pgfscope}%
\pgfsetrectcap%
\pgfsetroundjoin%
\pgfsetlinewidth{1.505625pt}%
\pgfsetstrokecolor{currentstroke2}%
\pgfsetdash{}{0pt}%
\pgfpathmoveto{\pgfqpoint{6.100563in}{2.184696in}}%
\pgfpathlineto{\pgfqpoint{6.225563in}{2.184696in}}%
\pgfpathlineto{\pgfqpoint{6.350563in}{2.184696in}}%
\pgfusepath{stroke}%
\end{pgfscope}%
\begin{pgfscope}%
\definecolor{textcolor}{rgb}{0.000000,0.000000,0.000000}%
\pgfsetstrokecolor{textcolor}%
\pgfsetfillcolor{textcolor}%
\pgftext[x=6.450563in,y=2.140946in,left,base]{\color{textcolor}{\rmfamily\fontsize{9.000000}{10.800000}\selectfont\catcode`\^=\active\def^{\ifmmode\sp\else\^{}\fi}\catcode`\%=\active\def%{\%}\Neighbors{} \& \MergeLinear{}}}%
\end{pgfscope}%
\begin{pgfscope}%
\pgfsetrectcap%
\pgfsetroundjoin%
\pgfsetlinewidth{1.505625pt}%
\pgfsetstrokecolor{currentstroke3}%
\pgfsetdash{}{0pt}%
\pgfpathmoveto{\pgfqpoint{6.100563in}{2.001225in}}%
\pgfpathlineto{\pgfqpoint{6.225563in}{2.001225in}}%
\pgfpathlineto{\pgfqpoint{6.350563in}{2.001225in}}%
\pgfusepath{stroke}%
\end{pgfscope}%
\begin{pgfscope}%
\definecolor{textcolor}{rgb}{0.000000,0.000000,0.000000}%
\pgfsetstrokecolor{textcolor}%
\pgfsetfillcolor{textcolor}%
\pgftext[x=6.450563in,y=1.957475in,left,base]{\color{textcolor}{\rmfamily\fontsize{9.000000}{10.800000}\selectfont\catcode`\^=\active\def^{\ifmmode\sp\else\^{}\fi}\catcode`\%=\active\def%{\%}\Neighbors{} \& \SharedVertices{}}}%
\end{pgfscope}%
\begin{pgfscope}%
\pgfsetrectcap%
\pgfsetroundjoin%
\pgfsetlinewidth{1.505625pt}%
\pgfsetstrokecolor{currentstroke4}%
\pgfsetdash{}{0pt}%
\pgfpathmoveto{\pgfqpoint{6.100563in}{1.814274in}}%
\pgfpathlineto{\pgfqpoint{6.225563in}{1.814274in}}%
\pgfpathlineto{\pgfqpoint{6.350563in}{1.814274in}}%
\pgfusepath{stroke}%
\end{pgfscope}%
\begin{pgfscope}%
\definecolor{textcolor}{rgb}{0.000000,0.000000,0.000000}%
\pgfsetstrokecolor{textcolor}%
\pgfsetfillcolor{textcolor}%
\pgftext[x=6.450563in,y=1.770524in,left,base]{\color{textcolor}{\rmfamily\fontsize{9.000000}{10.800000}\selectfont\catcode`\^=\active\def^{\ifmmode\sp\else\^{}\fi}\catcode`\%=\active\def%{\%}\NeighborsDegree{} \& \MergeLinear{}}}%
\end{pgfscope}%
\begin{pgfscope}%
\pgfsetrectcap%
\pgfsetroundjoin%
\pgfsetlinewidth{1.505625pt}%
\pgfsetstrokecolor{currentstroke5}%
\pgfsetdash{}{0pt}%
\pgfpathmoveto{\pgfqpoint{6.100563in}{1.627324in}}%
\pgfpathlineto{\pgfqpoint{6.225563in}{1.627324in}}%
\pgfpathlineto{\pgfqpoint{6.350563in}{1.627324in}}%
\pgfusepath{stroke}%
\end{pgfscope}%
\begin{pgfscope}%
\definecolor{textcolor}{rgb}{0.000000,0.000000,0.000000}%
\pgfsetstrokecolor{textcolor}%
\pgfsetfillcolor{textcolor}%
\pgftext[x=6.450563in,y=1.583574in,left,base]{\color{textcolor}{\rmfamily\fontsize{9.000000}{10.800000}\selectfont\catcode`\^=\active\def^{\ifmmode\sp\else\^{}\fi}\catcode`\%=\active\def%{\%}\None{} \& \MergeLinear{}}}%
\end{pgfscope}%
\begin{pgfscope}%
\pgfsetrectcap%
\pgfsetroundjoin%
\pgfsetlinewidth{1.505625pt}%
\pgfsetstrokecolor{currentstroke6}%
\pgfsetdash{}{0pt}%
\pgfpathmoveto{\pgfqpoint{6.100563in}{1.443852in}}%
\pgfpathlineto{\pgfqpoint{6.225563in}{1.443852in}}%
\pgfpathlineto{\pgfqpoint{6.350563in}{1.443852in}}%
\pgfusepath{stroke}%
\end{pgfscope}%
\begin{pgfscope}%
\definecolor{textcolor}{rgb}{0.000000,0.000000,0.000000}%
\pgfsetstrokecolor{textcolor}%
\pgfsetfillcolor{textcolor}%
\pgftext[x=6.450563in,y=1.400102in,left,base]{\color{textcolor}{\rmfamily\fontsize{9.000000}{10.800000}\selectfont\catcode`\^=\active\def^{\ifmmode\sp\else\^{}\fi}\catcode`\%=\active\def%{\%}\None{} \& \SharedVertices{}}}%
\end{pgfscope}%
\end{pgfpicture}%
\makeatother%
\endgroup%
}
	\caption[Mean runtime for minimally rigid graphs (some).]{
		Mean running time (ms) to find all NAC-colorings for minimally rigid graphs.}%
	\label{fig:graph_minimally_rigid_first_runtime}
\end{figure}
\begin{figure}[p]
	\centering
	\scalebox{0.5}{%% Creator: Matplotlib, PGF backend
%%
%% To include the figure in your LaTeX document, write
%%   \input{<filename>.pgf}
%%
%% Make sure the required packages are loaded in your preamble
%%   \usepackage{pgf}
%%
%% Also ensure that all the required font packages are loaded; for instance,
%% the lmodern package is sometimes necessary when using math font.
%%   \usepackage{lmodern}
%%
%% Figures using additional raster images can only be included by \input if
%% they are in the same directory as the main LaTeX file. For loading figures
%% from other directories you can use the `import` package
%%   \usepackage{import}
%%
%% and then include the figures with
%%   \import{<path to file>}{<filename>.pgf}
%%
%% Matplotlib used the following preamble
%%   \def\mathdefault#1{#1}
%%   \everymath=\expandafter{\the\everymath\displaystyle}
%%   \IfFileExists{scrextend.sty}{
%%     \usepackage[fontsize=10.000000pt]{scrextend}
%%   }{
%%     \renewcommand{\normalsize}{\fontsize{10.000000}{12.000000}\selectfont}
%%     \normalsize
%%   }
%%   
%%   \ifdefined\pdftexversion\else  % non-pdftex case.
%%     \usepackage{fontspec}
%%     \setmainfont{DejaVuSans.ttf}[Path=\detokenize{/home/petr/Projects/PyRigi/.venv/lib/python3.12/site-packages/matplotlib/mpl-data/fonts/ttf/}]
%%     \setsansfont{DejaVuSans.ttf}[Path=\detokenize{/home/petr/Projects/PyRigi/.venv/lib/python3.12/site-packages/matplotlib/mpl-data/fonts/ttf/}]
%%     \setmonofont{DejaVuSansMono.ttf}[Path=\detokenize{/home/petr/Projects/PyRigi/.venv/lib/python3.12/site-packages/matplotlib/mpl-data/fonts/ttf/}]
%%   \fi
%%   \makeatletter\@ifpackageloaded{under\Score{}}{}{\usepackage[strings]{under\Score{}}}\makeatother
%%
\begingroup%
\makeatletter%
\begin{pgfpicture}%
\pgfpathrectangle{\pgfpointorigin}{\pgfqpoint{8.384376in}{2.841849in}}%
\pgfusepath{use as bounding box, clip}%
\begin{pgfscope}%
\pgfsetbuttcap%
\pgfsetmiterjoin%
\definecolor{currentfill}{rgb}{1.000000,1.000000,1.000000}%
\pgfsetfillcolor{currentfill}%
\pgfsetlinewidth{0.000000pt}%
\definecolor{currentstroke}{rgb}{1.000000,1.000000,1.000000}%
\pgfsetstrokecolor{currentstroke}%
\pgfsetdash{}{0pt}%
\pgfpathmoveto{\pgfqpoint{0.000000in}{0.000000in}}%
\pgfpathlineto{\pgfqpoint{8.384376in}{0.000000in}}%
\pgfpathlineto{\pgfqpoint{8.384376in}{2.841849in}}%
\pgfpathlineto{\pgfqpoint{0.000000in}{2.841849in}}%
\pgfpathlineto{\pgfqpoint{0.000000in}{0.000000in}}%
\pgfpathclose%
\pgfusepath{fill}%
\end{pgfscope}%
\begin{pgfscope}%
\pgfsetbuttcap%
\pgfsetmiterjoin%
\definecolor{currentfill}{rgb}{1.000000,1.000000,1.000000}%
\pgfsetfillcolor{currentfill}%
\pgfsetlinewidth{0.000000pt}%
\definecolor{currentstroke}{rgb}{0.000000,0.000000,0.000000}%
\pgfsetstrokecolor{currentstroke}%
\pgfsetstrokeopacity{0.000000}%
\pgfsetdash{}{0pt}%
\pgfpathmoveto{\pgfqpoint{0.588387in}{0.521603in}}%
\pgfpathlineto{\pgfqpoint{5.988063in}{0.521603in}}%
\pgfpathlineto{\pgfqpoint{5.988063in}{2.531888in}}%
\pgfpathlineto{\pgfqpoint{0.588387in}{2.531888in}}%
\pgfpathlineto{\pgfqpoint{0.588387in}{0.521603in}}%
\pgfpathclose%
\pgfusepath{fill}%
\end{pgfscope}%
\begin{pgfscope}%
\pgfsetbuttcap%
\pgfsetroundjoin%
\definecolor{currentfill}{rgb}{0.000000,0.000000,0.000000}%
\pgfsetfillcolor{currentfill}%
\pgfsetlinewidth{0.803000pt}%
\definecolor{currentstroke}{rgb}{0.000000,0.000000,0.000000}%
\pgfsetstrokecolor{currentstroke}%
\pgfsetdash{}{0pt}%
\pgfsys@defobject{currentmarker}{\pgfqpoint{0.000000in}{-0.048611in}}{\pgfqpoint{0.000000in}{0.000000in}}{%
\pgfpathmoveto{\pgfqpoint{0.000000in}{0.000000in}}%
\pgfpathlineto{\pgfqpoint{0.000000in}{-0.048611in}}%
\pgfusepath{stroke,fill}%
}%
\begin{pgfscope}%
\pgfsys@transformshift{0.738511in}{0.521603in}%
\pgfsys@useobject{currentmarker}{}%
\end{pgfscope}%
\end{pgfscope}%
\begin{pgfscope}%
\definecolor{textcolor}{rgb}{0.000000,0.000000,0.000000}%
\pgfsetstrokecolor{textcolor}%
\pgfsetfillcolor{textcolor}%
\pgftext[x=0.738511in,y=0.424381in,,top]{\color{textcolor}{\rmfamily\fontsize{10.000000}{12.000000}\selectfont\catcode`\^=\active\def^{\ifmmode\sp\else\^{}\fi}\catcode`\%=\active\def%{\%}$\mathdefault{0}$}}%
\end{pgfscope}%
\begin{pgfscope}%
\pgfsetbuttcap%
\pgfsetroundjoin%
\definecolor{currentfill}{rgb}{0.000000,0.000000,0.000000}%
\pgfsetfillcolor{currentfill}%
\pgfsetlinewidth{0.803000pt}%
\definecolor{currentstroke}{rgb}{0.000000,0.000000,0.000000}%
\pgfsetstrokecolor{currentstroke}%
\pgfsetdash{}{0pt}%
\pgfsys@defobject{currentmarker}{\pgfqpoint{0.000000in}{-0.048611in}}{\pgfqpoint{0.000000in}{0.000000in}}{%
\pgfpathmoveto{\pgfqpoint{0.000000in}{0.000000in}}%
\pgfpathlineto{\pgfqpoint{0.000000in}{-0.048611in}}%
\pgfusepath{stroke,fill}%
}%
\begin{pgfscope}%
\pgfsys@transformshift{1.453384in}{0.521603in}%
\pgfsys@useobject{currentmarker}{}%
\end{pgfscope}%
\end{pgfscope}%
\begin{pgfscope}%
\definecolor{textcolor}{rgb}{0.000000,0.000000,0.000000}%
\pgfsetstrokecolor{textcolor}%
\pgfsetfillcolor{textcolor}%
\pgftext[x=1.453384in,y=0.424381in,,top]{\color{textcolor}{\rmfamily\fontsize{10.000000}{12.000000}\selectfont\catcode`\^=\active\def^{\ifmmode\sp\else\^{}\fi}\catcode`\%=\active\def%{\%}$\mathdefault{15}$}}%
\end{pgfscope}%
\begin{pgfscope}%
\pgfsetbuttcap%
\pgfsetroundjoin%
\definecolor{currentfill}{rgb}{0.000000,0.000000,0.000000}%
\pgfsetfillcolor{currentfill}%
\pgfsetlinewidth{0.803000pt}%
\definecolor{currentstroke}{rgb}{0.000000,0.000000,0.000000}%
\pgfsetstrokecolor{currentstroke}%
\pgfsetdash{}{0pt}%
\pgfsys@defobject{currentmarker}{\pgfqpoint{0.000000in}{-0.048611in}}{\pgfqpoint{0.000000in}{0.000000in}}{%
\pgfpathmoveto{\pgfqpoint{0.000000in}{0.000000in}}%
\pgfpathlineto{\pgfqpoint{0.000000in}{-0.048611in}}%
\pgfusepath{stroke,fill}%
}%
\begin{pgfscope}%
\pgfsys@transformshift{2.168257in}{0.521603in}%
\pgfsys@useobject{currentmarker}{}%
\end{pgfscope}%
\end{pgfscope}%
\begin{pgfscope}%
\definecolor{textcolor}{rgb}{0.000000,0.000000,0.000000}%
\pgfsetstrokecolor{textcolor}%
\pgfsetfillcolor{textcolor}%
\pgftext[x=2.168257in,y=0.424381in,,top]{\color{textcolor}{\rmfamily\fontsize{10.000000}{12.000000}\selectfont\catcode`\^=\active\def^{\ifmmode\sp\else\^{}\fi}\catcode`\%=\active\def%{\%}$\mathdefault{30}$}}%
\end{pgfscope}%
\begin{pgfscope}%
\pgfsetbuttcap%
\pgfsetroundjoin%
\definecolor{currentfill}{rgb}{0.000000,0.000000,0.000000}%
\pgfsetfillcolor{currentfill}%
\pgfsetlinewidth{0.803000pt}%
\definecolor{currentstroke}{rgb}{0.000000,0.000000,0.000000}%
\pgfsetstrokecolor{currentstroke}%
\pgfsetdash{}{0pt}%
\pgfsys@defobject{currentmarker}{\pgfqpoint{0.000000in}{-0.048611in}}{\pgfqpoint{0.000000in}{0.000000in}}{%
\pgfpathmoveto{\pgfqpoint{0.000000in}{0.000000in}}%
\pgfpathlineto{\pgfqpoint{0.000000in}{-0.048611in}}%
\pgfusepath{stroke,fill}%
}%
\begin{pgfscope}%
\pgfsys@transformshift{2.883130in}{0.521603in}%
\pgfsys@useobject{currentmarker}{}%
\end{pgfscope}%
\end{pgfscope}%
\begin{pgfscope}%
\definecolor{textcolor}{rgb}{0.000000,0.000000,0.000000}%
\pgfsetstrokecolor{textcolor}%
\pgfsetfillcolor{textcolor}%
\pgftext[x=2.883130in,y=0.424381in,,top]{\color{textcolor}{\rmfamily\fontsize{10.000000}{12.000000}\selectfont\catcode`\^=\active\def^{\ifmmode\sp\else\^{}\fi}\catcode`\%=\active\def%{\%}$\mathdefault{45}$}}%
\end{pgfscope}%
\begin{pgfscope}%
\pgfsetbuttcap%
\pgfsetroundjoin%
\definecolor{currentfill}{rgb}{0.000000,0.000000,0.000000}%
\pgfsetfillcolor{currentfill}%
\pgfsetlinewidth{0.803000pt}%
\definecolor{currentstroke}{rgb}{0.000000,0.000000,0.000000}%
\pgfsetstrokecolor{currentstroke}%
\pgfsetdash{}{0pt}%
\pgfsys@defobject{currentmarker}{\pgfqpoint{0.000000in}{-0.048611in}}{\pgfqpoint{0.000000in}{0.000000in}}{%
\pgfpathmoveto{\pgfqpoint{0.000000in}{0.000000in}}%
\pgfpathlineto{\pgfqpoint{0.000000in}{-0.048611in}}%
\pgfusepath{stroke,fill}%
}%
\begin{pgfscope}%
\pgfsys@transformshift{3.598004in}{0.521603in}%
\pgfsys@useobject{currentmarker}{}%
\end{pgfscope}%
\end{pgfscope}%
\begin{pgfscope}%
\definecolor{textcolor}{rgb}{0.000000,0.000000,0.000000}%
\pgfsetstrokecolor{textcolor}%
\pgfsetfillcolor{textcolor}%
\pgftext[x=3.598004in,y=0.424381in,,top]{\color{textcolor}{\rmfamily\fontsize{10.000000}{12.000000}\selectfont\catcode`\^=\active\def^{\ifmmode\sp\else\^{}\fi}\catcode`\%=\active\def%{\%}$\mathdefault{60}$}}%
\end{pgfscope}%
\begin{pgfscope}%
\pgfsetbuttcap%
\pgfsetroundjoin%
\definecolor{currentfill}{rgb}{0.000000,0.000000,0.000000}%
\pgfsetfillcolor{currentfill}%
\pgfsetlinewidth{0.803000pt}%
\definecolor{currentstroke}{rgb}{0.000000,0.000000,0.000000}%
\pgfsetstrokecolor{currentstroke}%
\pgfsetdash{}{0pt}%
\pgfsys@defobject{currentmarker}{\pgfqpoint{0.000000in}{-0.048611in}}{\pgfqpoint{0.000000in}{0.000000in}}{%
\pgfpathmoveto{\pgfqpoint{0.000000in}{0.000000in}}%
\pgfpathlineto{\pgfqpoint{0.000000in}{-0.048611in}}%
\pgfusepath{stroke,fill}%
}%
\begin{pgfscope}%
\pgfsys@transformshift{4.312877in}{0.521603in}%
\pgfsys@useobject{currentmarker}{}%
\end{pgfscope}%
\end{pgfscope}%
\begin{pgfscope}%
\definecolor{textcolor}{rgb}{0.000000,0.000000,0.000000}%
\pgfsetstrokecolor{textcolor}%
\pgfsetfillcolor{textcolor}%
\pgftext[x=4.312877in,y=0.424381in,,top]{\color{textcolor}{\rmfamily\fontsize{10.000000}{12.000000}\selectfont\catcode`\^=\active\def^{\ifmmode\sp\else\^{}\fi}\catcode`\%=\active\def%{\%}$\mathdefault{75}$}}%
\end{pgfscope}%
\begin{pgfscope}%
\pgfsetbuttcap%
\pgfsetroundjoin%
\definecolor{currentfill}{rgb}{0.000000,0.000000,0.000000}%
\pgfsetfillcolor{currentfill}%
\pgfsetlinewidth{0.803000pt}%
\definecolor{currentstroke}{rgb}{0.000000,0.000000,0.000000}%
\pgfsetstrokecolor{currentstroke}%
\pgfsetdash{}{0pt}%
\pgfsys@defobject{currentmarker}{\pgfqpoint{0.000000in}{-0.048611in}}{\pgfqpoint{0.000000in}{0.000000in}}{%
\pgfpathmoveto{\pgfqpoint{0.000000in}{0.000000in}}%
\pgfpathlineto{\pgfqpoint{0.000000in}{-0.048611in}}%
\pgfusepath{stroke,fill}%
}%
\begin{pgfscope}%
\pgfsys@transformshift{5.027750in}{0.521603in}%
\pgfsys@useobject{currentmarker}{}%
\end{pgfscope}%
\end{pgfscope}%
\begin{pgfscope}%
\definecolor{textcolor}{rgb}{0.000000,0.000000,0.000000}%
\pgfsetstrokecolor{textcolor}%
\pgfsetfillcolor{textcolor}%
\pgftext[x=5.027750in,y=0.424381in,,top]{\color{textcolor}{\rmfamily\fontsize{10.000000}{12.000000}\selectfont\catcode`\^=\active\def^{\ifmmode\sp\else\^{}\fi}\catcode`\%=\active\def%{\%}$\mathdefault{90}$}}%
\end{pgfscope}%
\begin{pgfscope}%
\pgfsetbuttcap%
\pgfsetroundjoin%
\definecolor{currentfill}{rgb}{0.000000,0.000000,0.000000}%
\pgfsetfillcolor{currentfill}%
\pgfsetlinewidth{0.803000pt}%
\definecolor{currentstroke}{rgb}{0.000000,0.000000,0.000000}%
\pgfsetstrokecolor{currentstroke}%
\pgfsetdash{}{0pt}%
\pgfsys@defobject{currentmarker}{\pgfqpoint{0.000000in}{-0.048611in}}{\pgfqpoint{0.000000in}{0.000000in}}{%
\pgfpathmoveto{\pgfqpoint{0.000000in}{0.000000in}}%
\pgfpathlineto{\pgfqpoint{0.000000in}{-0.048611in}}%
\pgfusepath{stroke,fill}%
}%
\begin{pgfscope}%
\pgfsys@transformshift{5.742623in}{0.521603in}%
\pgfsys@useobject{currentmarker}{}%
\end{pgfscope}%
\end{pgfscope}%
\begin{pgfscope}%
\definecolor{textcolor}{rgb}{0.000000,0.000000,0.000000}%
\pgfsetstrokecolor{textcolor}%
\pgfsetfillcolor{textcolor}%
\pgftext[x=5.742623in,y=0.424381in,,top]{\color{textcolor}{\rmfamily\fontsize{10.000000}{12.000000}\selectfont\catcode`\^=\active\def^{\ifmmode\sp\else\^{}\fi}\catcode`\%=\active\def%{\%}$\mathdefault{105}$}}%
\end{pgfscope}%
\begin{pgfscope}%
\definecolor{textcolor}{rgb}{0.000000,0.000000,0.000000}%
\pgfsetstrokecolor{textcolor}%
\pgfsetfillcolor{textcolor}%
\pgftext[x=3.288225in,y=0.234413in,,top]{\color{textcolor}{\rmfamily\fontsize{10.000000}{12.000000}\selectfont\catcode`\^=\active\def^{\ifmmode\sp\else\^{}\fi}\catcode`\%=\active\def%{\%}Monochromatic classes}}%
\end{pgfscope}%
\begin{pgfscope}%
\pgfsetbuttcap%
\pgfsetroundjoin%
\definecolor{currentfill}{rgb}{0.000000,0.000000,0.000000}%
\pgfsetfillcolor{currentfill}%
\pgfsetlinewidth{0.803000pt}%
\definecolor{currentstroke}{rgb}{0.000000,0.000000,0.000000}%
\pgfsetstrokecolor{currentstroke}%
\pgfsetdash{}{0pt}%
\pgfsys@defobject{currentmarker}{\pgfqpoint{-0.048611in}{0.000000in}}{\pgfqpoint{-0.000000in}{0.000000in}}{%
\pgfpathmoveto{\pgfqpoint{-0.000000in}{0.000000in}}%
\pgfpathlineto{\pgfqpoint{-0.048611in}{0.000000in}}%
\pgfusepath{stroke,fill}%
}%
\begin{pgfscope}%
\pgfsys@transformshift{0.588387in}{0.612980in}%
\pgfsys@useobject{currentmarker}{}%
\end{pgfscope}%
\end{pgfscope}%
\begin{pgfscope}%
\definecolor{textcolor}{rgb}{0.000000,0.000000,0.000000}%
\pgfsetstrokecolor{textcolor}%
\pgfsetfillcolor{textcolor}%
\pgftext[x=0.289968in, y=0.560218in, left, base]{\color{textcolor}{\rmfamily\fontsize{10.000000}{12.000000}\selectfont\catcode`\^=\active\def^{\ifmmode\sp\else\^{}\fi}\catcode`\%=\active\def%{\%}$\mathdefault{10^{0}}$}}%
\end{pgfscope}%
\begin{pgfscope}%
\pgfsetbuttcap%
\pgfsetroundjoin%
\definecolor{currentfill}{rgb}{0.000000,0.000000,0.000000}%
\pgfsetfillcolor{currentfill}%
\pgfsetlinewidth{0.803000pt}%
\definecolor{currentstroke}{rgb}{0.000000,0.000000,0.000000}%
\pgfsetstrokecolor{currentstroke}%
\pgfsetdash{}{0pt}%
\pgfsys@defobject{currentmarker}{\pgfqpoint{-0.048611in}{0.000000in}}{\pgfqpoint{-0.000000in}{0.000000in}}{%
\pgfpathmoveto{\pgfqpoint{-0.000000in}{0.000000in}}%
\pgfpathlineto{\pgfqpoint{-0.048611in}{0.000000in}}%
\pgfusepath{stroke,fill}%
}%
\begin{pgfscope}%
\pgfsys@transformshift{0.588387in}{0.969461in}%
\pgfsys@useobject{currentmarker}{}%
\end{pgfscope}%
\end{pgfscope}%
\begin{pgfscope}%
\definecolor{textcolor}{rgb}{0.000000,0.000000,0.000000}%
\pgfsetstrokecolor{textcolor}%
\pgfsetfillcolor{textcolor}%
\pgftext[x=0.289968in, y=0.916700in, left, base]{\color{textcolor}{\rmfamily\fontsize{10.000000}{12.000000}\selectfont\catcode`\^=\active\def^{\ifmmode\sp\else\^{}\fi}\catcode`\%=\active\def%{\%}$\mathdefault{10^{1}}$}}%
\end{pgfscope}%
\begin{pgfscope}%
\pgfsetbuttcap%
\pgfsetroundjoin%
\definecolor{currentfill}{rgb}{0.000000,0.000000,0.000000}%
\pgfsetfillcolor{currentfill}%
\pgfsetlinewidth{0.803000pt}%
\definecolor{currentstroke}{rgb}{0.000000,0.000000,0.000000}%
\pgfsetstrokecolor{currentstroke}%
\pgfsetdash{}{0pt}%
\pgfsys@defobject{currentmarker}{\pgfqpoint{-0.048611in}{0.000000in}}{\pgfqpoint{-0.000000in}{0.000000in}}{%
\pgfpathmoveto{\pgfqpoint{-0.000000in}{0.000000in}}%
\pgfpathlineto{\pgfqpoint{-0.048611in}{0.000000in}}%
\pgfusepath{stroke,fill}%
}%
\begin{pgfscope}%
\pgfsys@transformshift{0.588387in}{1.325942in}%
\pgfsys@useobject{currentmarker}{}%
\end{pgfscope}%
\end{pgfscope}%
\begin{pgfscope}%
\definecolor{textcolor}{rgb}{0.000000,0.000000,0.000000}%
\pgfsetstrokecolor{textcolor}%
\pgfsetfillcolor{textcolor}%
\pgftext[x=0.289968in, y=1.273181in, left, base]{\color{textcolor}{\rmfamily\fontsize{10.000000}{12.000000}\selectfont\catcode`\^=\active\def^{\ifmmode\sp\else\^{}\fi}\catcode`\%=\active\def%{\%}$\mathdefault{10^{2}}$}}%
\end{pgfscope}%
\begin{pgfscope}%
\pgfsetbuttcap%
\pgfsetroundjoin%
\definecolor{currentfill}{rgb}{0.000000,0.000000,0.000000}%
\pgfsetfillcolor{currentfill}%
\pgfsetlinewidth{0.803000pt}%
\definecolor{currentstroke}{rgb}{0.000000,0.000000,0.000000}%
\pgfsetstrokecolor{currentstroke}%
\pgfsetdash{}{0pt}%
\pgfsys@defobject{currentmarker}{\pgfqpoint{-0.048611in}{0.000000in}}{\pgfqpoint{-0.000000in}{0.000000in}}{%
\pgfpathmoveto{\pgfqpoint{-0.000000in}{0.000000in}}%
\pgfpathlineto{\pgfqpoint{-0.048611in}{0.000000in}}%
\pgfusepath{stroke,fill}%
}%
\begin{pgfscope}%
\pgfsys@transformshift{0.588387in}{1.682424in}%
\pgfsys@useobject{currentmarker}{}%
\end{pgfscope}%
\end{pgfscope}%
\begin{pgfscope}%
\definecolor{textcolor}{rgb}{0.000000,0.000000,0.000000}%
\pgfsetstrokecolor{textcolor}%
\pgfsetfillcolor{textcolor}%
\pgftext[x=0.289968in, y=1.629662in, left, base]{\color{textcolor}{\rmfamily\fontsize{10.000000}{12.000000}\selectfont\catcode`\^=\active\def^{\ifmmode\sp\else\^{}\fi}\catcode`\%=\active\def%{\%}$\mathdefault{10^{3}}$}}%
\end{pgfscope}%
\begin{pgfscope}%
\pgfsetbuttcap%
\pgfsetroundjoin%
\definecolor{currentfill}{rgb}{0.000000,0.000000,0.000000}%
\pgfsetfillcolor{currentfill}%
\pgfsetlinewidth{0.803000pt}%
\definecolor{currentstroke}{rgb}{0.000000,0.000000,0.000000}%
\pgfsetstrokecolor{currentstroke}%
\pgfsetdash{}{0pt}%
\pgfsys@defobject{currentmarker}{\pgfqpoint{-0.048611in}{0.000000in}}{\pgfqpoint{-0.000000in}{0.000000in}}{%
\pgfpathmoveto{\pgfqpoint{-0.000000in}{0.000000in}}%
\pgfpathlineto{\pgfqpoint{-0.048611in}{0.000000in}}%
\pgfusepath{stroke,fill}%
}%
\begin{pgfscope}%
\pgfsys@transformshift{0.588387in}{2.038905in}%
\pgfsys@useobject{currentmarker}{}%
\end{pgfscope}%
\end{pgfscope}%
\begin{pgfscope}%
\definecolor{textcolor}{rgb}{0.000000,0.000000,0.000000}%
\pgfsetstrokecolor{textcolor}%
\pgfsetfillcolor{textcolor}%
\pgftext[x=0.289968in, y=1.986143in, left, base]{\color{textcolor}{\rmfamily\fontsize{10.000000}{12.000000}\selectfont\catcode`\^=\active\def^{\ifmmode\sp\else\^{}\fi}\catcode`\%=\active\def%{\%}$\mathdefault{10^{4}}$}}%
\end{pgfscope}%
\begin{pgfscope}%
\pgfsetbuttcap%
\pgfsetroundjoin%
\definecolor{currentfill}{rgb}{0.000000,0.000000,0.000000}%
\pgfsetfillcolor{currentfill}%
\pgfsetlinewidth{0.803000pt}%
\definecolor{currentstroke}{rgb}{0.000000,0.000000,0.000000}%
\pgfsetstrokecolor{currentstroke}%
\pgfsetdash{}{0pt}%
\pgfsys@defobject{currentmarker}{\pgfqpoint{-0.048611in}{0.000000in}}{\pgfqpoint{-0.000000in}{0.000000in}}{%
\pgfpathmoveto{\pgfqpoint{-0.000000in}{0.000000in}}%
\pgfpathlineto{\pgfqpoint{-0.048611in}{0.000000in}}%
\pgfusepath{stroke,fill}%
}%
\begin{pgfscope}%
\pgfsys@transformshift{0.588387in}{2.395386in}%
\pgfsys@useobject{currentmarker}{}%
\end{pgfscope}%
\end{pgfscope}%
\begin{pgfscope}%
\definecolor{textcolor}{rgb}{0.000000,0.000000,0.000000}%
\pgfsetstrokecolor{textcolor}%
\pgfsetfillcolor{textcolor}%
\pgftext[x=0.289968in, y=2.342625in, left, base]{\color{textcolor}{\rmfamily\fontsize{10.000000}{12.000000}\selectfont\catcode`\^=\active\def^{\ifmmode\sp\else\^{}\fi}\catcode`\%=\active\def%{\%}$\mathdefault{10^{5}}$}}%
\end{pgfscope}%
\begin{pgfscope}%
\pgfsetbuttcap%
\pgfsetroundjoin%
\definecolor{currentfill}{rgb}{0.000000,0.000000,0.000000}%
\pgfsetfillcolor{currentfill}%
\pgfsetlinewidth{0.602250pt}%
\definecolor{currentstroke}{rgb}{0.000000,0.000000,0.000000}%
\pgfsetstrokecolor{currentstroke}%
\pgfsetdash{}{0pt}%
\pgfsys@defobject{currentmarker}{\pgfqpoint{-0.027778in}{0.000000in}}{\pgfqpoint{-0.000000in}{0.000000in}}{%
\pgfpathmoveto{\pgfqpoint{-0.000000in}{0.000000in}}%
\pgfpathlineto{\pgfqpoint{-0.027778in}{0.000000in}}%
\pgfusepath{stroke,fill}%
}%
\begin{pgfscope}%
\pgfsys@transformshift{0.588387in}{0.533895in}%
\pgfsys@useobject{currentmarker}{}%
\end{pgfscope}%
\end{pgfscope}%
\begin{pgfscope}%
\pgfsetbuttcap%
\pgfsetroundjoin%
\definecolor{currentfill}{rgb}{0.000000,0.000000,0.000000}%
\pgfsetfillcolor{currentfill}%
\pgfsetlinewidth{0.602250pt}%
\definecolor{currentstroke}{rgb}{0.000000,0.000000,0.000000}%
\pgfsetstrokecolor{currentstroke}%
\pgfsetdash{}{0pt}%
\pgfsys@defobject{currentmarker}{\pgfqpoint{-0.027778in}{0.000000in}}{\pgfqpoint{-0.000000in}{0.000000in}}{%
\pgfpathmoveto{\pgfqpoint{-0.000000in}{0.000000in}}%
\pgfpathlineto{\pgfqpoint{-0.027778in}{0.000000in}}%
\pgfusepath{stroke,fill}%
}%
\begin{pgfscope}%
\pgfsys@transformshift{0.588387in}{0.557760in}%
\pgfsys@useobject{currentmarker}{}%
\end{pgfscope}%
\end{pgfscope}%
\begin{pgfscope}%
\pgfsetbuttcap%
\pgfsetroundjoin%
\definecolor{currentfill}{rgb}{0.000000,0.000000,0.000000}%
\pgfsetfillcolor{currentfill}%
\pgfsetlinewidth{0.602250pt}%
\definecolor{currentstroke}{rgb}{0.000000,0.000000,0.000000}%
\pgfsetstrokecolor{currentstroke}%
\pgfsetdash{}{0pt}%
\pgfsys@defobject{currentmarker}{\pgfqpoint{-0.027778in}{0.000000in}}{\pgfqpoint{-0.000000in}{0.000000in}}{%
\pgfpathmoveto{\pgfqpoint{-0.000000in}{0.000000in}}%
\pgfpathlineto{\pgfqpoint{-0.027778in}{0.000000in}}%
\pgfusepath{stroke,fill}%
}%
\begin{pgfscope}%
\pgfsys@transformshift{0.588387in}{0.578433in}%
\pgfsys@useobject{currentmarker}{}%
\end{pgfscope}%
\end{pgfscope}%
\begin{pgfscope}%
\pgfsetbuttcap%
\pgfsetroundjoin%
\definecolor{currentfill}{rgb}{0.000000,0.000000,0.000000}%
\pgfsetfillcolor{currentfill}%
\pgfsetlinewidth{0.602250pt}%
\definecolor{currentstroke}{rgb}{0.000000,0.000000,0.000000}%
\pgfsetstrokecolor{currentstroke}%
\pgfsetdash{}{0pt}%
\pgfsys@defobject{currentmarker}{\pgfqpoint{-0.027778in}{0.000000in}}{\pgfqpoint{-0.000000in}{0.000000in}}{%
\pgfpathmoveto{\pgfqpoint{-0.000000in}{0.000000in}}%
\pgfpathlineto{\pgfqpoint{-0.027778in}{0.000000in}}%
\pgfusepath{stroke,fill}%
}%
\begin{pgfscope}%
\pgfsys@transformshift{0.588387in}{0.596668in}%
\pgfsys@useobject{currentmarker}{}%
\end{pgfscope}%
\end{pgfscope}%
\begin{pgfscope}%
\pgfsetbuttcap%
\pgfsetroundjoin%
\definecolor{currentfill}{rgb}{0.000000,0.000000,0.000000}%
\pgfsetfillcolor{currentfill}%
\pgfsetlinewidth{0.602250pt}%
\definecolor{currentstroke}{rgb}{0.000000,0.000000,0.000000}%
\pgfsetstrokecolor{currentstroke}%
\pgfsetdash{}{0pt}%
\pgfsys@defobject{currentmarker}{\pgfqpoint{-0.027778in}{0.000000in}}{\pgfqpoint{-0.000000in}{0.000000in}}{%
\pgfpathmoveto{\pgfqpoint{-0.000000in}{0.000000in}}%
\pgfpathlineto{\pgfqpoint{-0.027778in}{0.000000in}}%
\pgfusepath{stroke,fill}%
}%
\begin{pgfscope}%
\pgfsys@transformshift{0.588387in}{0.720291in}%
\pgfsys@useobject{currentmarker}{}%
\end{pgfscope}%
\end{pgfscope}%
\begin{pgfscope}%
\pgfsetbuttcap%
\pgfsetroundjoin%
\definecolor{currentfill}{rgb}{0.000000,0.000000,0.000000}%
\pgfsetfillcolor{currentfill}%
\pgfsetlinewidth{0.602250pt}%
\definecolor{currentstroke}{rgb}{0.000000,0.000000,0.000000}%
\pgfsetstrokecolor{currentstroke}%
\pgfsetdash{}{0pt}%
\pgfsys@defobject{currentmarker}{\pgfqpoint{-0.027778in}{0.000000in}}{\pgfqpoint{-0.000000in}{0.000000in}}{%
\pgfpathmoveto{\pgfqpoint{-0.000000in}{0.000000in}}%
\pgfpathlineto{\pgfqpoint{-0.027778in}{0.000000in}}%
\pgfusepath{stroke,fill}%
}%
\begin{pgfscope}%
\pgfsys@transformshift{0.588387in}{0.783065in}%
\pgfsys@useobject{currentmarker}{}%
\end{pgfscope}%
\end{pgfscope}%
\begin{pgfscope}%
\pgfsetbuttcap%
\pgfsetroundjoin%
\definecolor{currentfill}{rgb}{0.000000,0.000000,0.000000}%
\pgfsetfillcolor{currentfill}%
\pgfsetlinewidth{0.602250pt}%
\definecolor{currentstroke}{rgb}{0.000000,0.000000,0.000000}%
\pgfsetstrokecolor{currentstroke}%
\pgfsetdash{}{0pt}%
\pgfsys@defobject{currentmarker}{\pgfqpoint{-0.027778in}{0.000000in}}{\pgfqpoint{-0.000000in}{0.000000in}}{%
\pgfpathmoveto{\pgfqpoint{-0.000000in}{0.000000in}}%
\pgfpathlineto{\pgfqpoint{-0.027778in}{0.000000in}}%
\pgfusepath{stroke,fill}%
}%
\begin{pgfscope}%
\pgfsys@transformshift{0.588387in}{0.827603in}%
\pgfsys@useobject{currentmarker}{}%
\end{pgfscope}%
\end{pgfscope}%
\begin{pgfscope}%
\pgfsetbuttcap%
\pgfsetroundjoin%
\definecolor{currentfill}{rgb}{0.000000,0.000000,0.000000}%
\pgfsetfillcolor{currentfill}%
\pgfsetlinewidth{0.602250pt}%
\definecolor{currentstroke}{rgb}{0.000000,0.000000,0.000000}%
\pgfsetstrokecolor{currentstroke}%
\pgfsetdash{}{0pt}%
\pgfsys@defobject{currentmarker}{\pgfqpoint{-0.027778in}{0.000000in}}{\pgfqpoint{-0.000000in}{0.000000in}}{%
\pgfpathmoveto{\pgfqpoint{-0.000000in}{0.000000in}}%
\pgfpathlineto{\pgfqpoint{-0.027778in}{0.000000in}}%
\pgfusepath{stroke,fill}%
}%
\begin{pgfscope}%
\pgfsys@transformshift{0.588387in}{0.862150in}%
\pgfsys@useobject{currentmarker}{}%
\end{pgfscope}%
\end{pgfscope}%
\begin{pgfscope}%
\pgfsetbuttcap%
\pgfsetroundjoin%
\definecolor{currentfill}{rgb}{0.000000,0.000000,0.000000}%
\pgfsetfillcolor{currentfill}%
\pgfsetlinewidth{0.602250pt}%
\definecolor{currentstroke}{rgb}{0.000000,0.000000,0.000000}%
\pgfsetstrokecolor{currentstroke}%
\pgfsetdash{}{0pt}%
\pgfsys@defobject{currentmarker}{\pgfqpoint{-0.027778in}{0.000000in}}{\pgfqpoint{-0.000000in}{0.000000in}}{%
\pgfpathmoveto{\pgfqpoint{-0.000000in}{0.000000in}}%
\pgfpathlineto{\pgfqpoint{-0.027778in}{0.000000in}}%
\pgfusepath{stroke,fill}%
}%
\begin{pgfscope}%
\pgfsys@transformshift{0.588387in}{0.890376in}%
\pgfsys@useobject{currentmarker}{}%
\end{pgfscope}%
\end{pgfscope}%
\begin{pgfscope}%
\pgfsetbuttcap%
\pgfsetroundjoin%
\definecolor{currentfill}{rgb}{0.000000,0.000000,0.000000}%
\pgfsetfillcolor{currentfill}%
\pgfsetlinewidth{0.602250pt}%
\definecolor{currentstroke}{rgb}{0.000000,0.000000,0.000000}%
\pgfsetstrokecolor{currentstroke}%
\pgfsetdash{}{0pt}%
\pgfsys@defobject{currentmarker}{\pgfqpoint{-0.027778in}{0.000000in}}{\pgfqpoint{-0.000000in}{0.000000in}}{%
\pgfpathmoveto{\pgfqpoint{-0.000000in}{0.000000in}}%
\pgfpathlineto{\pgfqpoint{-0.027778in}{0.000000in}}%
\pgfusepath{stroke,fill}%
}%
\begin{pgfscope}%
\pgfsys@transformshift{0.588387in}{0.914242in}%
\pgfsys@useobject{currentmarker}{}%
\end{pgfscope}%
\end{pgfscope}%
\begin{pgfscope}%
\pgfsetbuttcap%
\pgfsetroundjoin%
\definecolor{currentfill}{rgb}{0.000000,0.000000,0.000000}%
\pgfsetfillcolor{currentfill}%
\pgfsetlinewidth{0.602250pt}%
\definecolor{currentstroke}{rgb}{0.000000,0.000000,0.000000}%
\pgfsetstrokecolor{currentstroke}%
\pgfsetdash{}{0pt}%
\pgfsys@defobject{currentmarker}{\pgfqpoint{-0.027778in}{0.000000in}}{\pgfqpoint{-0.000000in}{0.000000in}}{%
\pgfpathmoveto{\pgfqpoint{-0.000000in}{0.000000in}}%
\pgfpathlineto{\pgfqpoint{-0.027778in}{0.000000in}}%
\pgfusepath{stroke,fill}%
}%
\begin{pgfscope}%
\pgfsys@transformshift{0.588387in}{0.934915in}%
\pgfsys@useobject{currentmarker}{}%
\end{pgfscope}%
\end{pgfscope}%
\begin{pgfscope}%
\pgfsetbuttcap%
\pgfsetroundjoin%
\definecolor{currentfill}{rgb}{0.000000,0.000000,0.000000}%
\pgfsetfillcolor{currentfill}%
\pgfsetlinewidth{0.602250pt}%
\definecolor{currentstroke}{rgb}{0.000000,0.000000,0.000000}%
\pgfsetstrokecolor{currentstroke}%
\pgfsetdash{}{0pt}%
\pgfsys@defobject{currentmarker}{\pgfqpoint{-0.027778in}{0.000000in}}{\pgfqpoint{-0.000000in}{0.000000in}}{%
\pgfpathmoveto{\pgfqpoint{-0.000000in}{0.000000in}}%
\pgfpathlineto{\pgfqpoint{-0.027778in}{0.000000in}}%
\pgfusepath{stroke,fill}%
}%
\begin{pgfscope}%
\pgfsys@transformshift{0.588387in}{0.953149in}%
\pgfsys@useobject{currentmarker}{}%
\end{pgfscope}%
\end{pgfscope}%
\begin{pgfscope}%
\pgfsetbuttcap%
\pgfsetroundjoin%
\definecolor{currentfill}{rgb}{0.000000,0.000000,0.000000}%
\pgfsetfillcolor{currentfill}%
\pgfsetlinewidth{0.602250pt}%
\definecolor{currentstroke}{rgb}{0.000000,0.000000,0.000000}%
\pgfsetstrokecolor{currentstroke}%
\pgfsetdash{}{0pt}%
\pgfsys@defobject{currentmarker}{\pgfqpoint{-0.027778in}{0.000000in}}{\pgfqpoint{-0.000000in}{0.000000in}}{%
\pgfpathmoveto{\pgfqpoint{-0.000000in}{0.000000in}}%
\pgfpathlineto{\pgfqpoint{-0.027778in}{0.000000in}}%
\pgfusepath{stroke,fill}%
}%
\begin{pgfscope}%
\pgfsys@transformshift{0.588387in}{1.076773in}%
\pgfsys@useobject{currentmarker}{}%
\end{pgfscope}%
\end{pgfscope}%
\begin{pgfscope}%
\pgfsetbuttcap%
\pgfsetroundjoin%
\definecolor{currentfill}{rgb}{0.000000,0.000000,0.000000}%
\pgfsetfillcolor{currentfill}%
\pgfsetlinewidth{0.602250pt}%
\definecolor{currentstroke}{rgb}{0.000000,0.000000,0.000000}%
\pgfsetstrokecolor{currentstroke}%
\pgfsetdash{}{0pt}%
\pgfsys@defobject{currentmarker}{\pgfqpoint{-0.027778in}{0.000000in}}{\pgfqpoint{-0.000000in}{0.000000in}}{%
\pgfpathmoveto{\pgfqpoint{-0.000000in}{0.000000in}}%
\pgfpathlineto{\pgfqpoint{-0.027778in}{0.000000in}}%
\pgfusepath{stroke,fill}%
}%
\begin{pgfscope}%
\pgfsys@transformshift{0.588387in}{1.139546in}%
\pgfsys@useobject{currentmarker}{}%
\end{pgfscope}%
\end{pgfscope}%
\begin{pgfscope}%
\pgfsetbuttcap%
\pgfsetroundjoin%
\definecolor{currentfill}{rgb}{0.000000,0.000000,0.000000}%
\pgfsetfillcolor{currentfill}%
\pgfsetlinewidth{0.602250pt}%
\definecolor{currentstroke}{rgb}{0.000000,0.000000,0.000000}%
\pgfsetstrokecolor{currentstroke}%
\pgfsetdash{}{0pt}%
\pgfsys@defobject{currentmarker}{\pgfqpoint{-0.027778in}{0.000000in}}{\pgfqpoint{-0.000000in}{0.000000in}}{%
\pgfpathmoveto{\pgfqpoint{-0.000000in}{0.000000in}}%
\pgfpathlineto{\pgfqpoint{-0.027778in}{0.000000in}}%
\pgfusepath{stroke,fill}%
}%
\begin{pgfscope}%
\pgfsys@transformshift{0.588387in}{1.184084in}%
\pgfsys@useobject{currentmarker}{}%
\end{pgfscope}%
\end{pgfscope}%
\begin{pgfscope}%
\pgfsetbuttcap%
\pgfsetroundjoin%
\definecolor{currentfill}{rgb}{0.000000,0.000000,0.000000}%
\pgfsetfillcolor{currentfill}%
\pgfsetlinewidth{0.602250pt}%
\definecolor{currentstroke}{rgb}{0.000000,0.000000,0.000000}%
\pgfsetstrokecolor{currentstroke}%
\pgfsetdash{}{0pt}%
\pgfsys@defobject{currentmarker}{\pgfqpoint{-0.027778in}{0.000000in}}{\pgfqpoint{-0.000000in}{0.000000in}}{%
\pgfpathmoveto{\pgfqpoint{-0.000000in}{0.000000in}}%
\pgfpathlineto{\pgfqpoint{-0.027778in}{0.000000in}}%
\pgfusepath{stroke,fill}%
}%
\begin{pgfscope}%
\pgfsys@transformshift{0.588387in}{1.218631in}%
\pgfsys@useobject{currentmarker}{}%
\end{pgfscope}%
\end{pgfscope}%
\begin{pgfscope}%
\pgfsetbuttcap%
\pgfsetroundjoin%
\definecolor{currentfill}{rgb}{0.000000,0.000000,0.000000}%
\pgfsetfillcolor{currentfill}%
\pgfsetlinewidth{0.602250pt}%
\definecolor{currentstroke}{rgb}{0.000000,0.000000,0.000000}%
\pgfsetstrokecolor{currentstroke}%
\pgfsetdash{}{0pt}%
\pgfsys@defobject{currentmarker}{\pgfqpoint{-0.027778in}{0.000000in}}{\pgfqpoint{-0.000000in}{0.000000in}}{%
\pgfpathmoveto{\pgfqpoint{-0.000000in}{0.000000in}}%
\pgfpathlineto{\pgfqpoint{-0.027778in}{0.000000in}}%
\pgfusepath{stroke,fill}%
}%
\begin{pgfscope}%
\pgfsys@transformshift{0.588387in}{1.246857in}%
\pgfsys@useobject{currentmarker}{}%
\end{pgfscope}%
\end{pgfscope}%
\begin{pgfscope}%
\pgfsetbuttcap%
\pgfsetroundjoin%
\definecolor{currentfill}{rgb}{0.000000,0.000000,0.000000}%
\pgfsetfillcolor{currentfill}%
\pgfsetlinewidth{0.602250pt}%
\definecolor{currentstroke}{rgb}{0.000000,0.000000,0.000000}%
\pgfsetstrokecolor{currentstroke}%
\pgfsetdash{}{0pt}%
\pgfsys@defobject{currentmarker}{\pgfqpoint{-0.027778in}{0.000000in}}{\pgfqpoint{-0.000000in}{0.000000in}}{%
\pgfpathmoveto{\pgfqpoint{-0.000000in}{0.000000in}}%
\pgfpathlineto{\pgfqpoint{-0.027778in}{0.000000in}}%
\pgfusepath{stroke,fill}%
}%
\begin{pgfscope}%
\pgfsys@transformshift{0.588387in}{1.270723in}%
\pgfsys@useobject{currentmarker}{}%
\end{pgfscope}%
\end{pgfscope}%
\begin{pgfscope}%
\pgfsetbuttcap%
\pgfsetroundjoin%
\definecolor{currentfill}{rgb}{0.000000,0.000000,0.000000}%
\pgfsetfillcolor{currentfill}%
\pgfsetlinewidth{0.602250pt}%
\definecolor{currentstroke}{rgb}{0.000000,0.000000,0.000000}%
\pgfsetstrokecolor{currentstroke}%
\pgfsetdash{}{0pt}%
\pgfsys@defobject{currentmarker}{\pgfqpoint{-0.027778in}{0.000000in}}{\pgfqpoint{-0.000000in}{0.000000in}}{%
\pgfpathmoveto{\pgfqpoint{-0.000000in}{0.000000in}}%
\pgfpathlineto{\pgfqpoint{-0.027778in}{0.000000in}}%
\pgfusepath{stroke,fill}%
}%
\begin{pgfscope}%
\pgfsys@transformshift{0.588387in}{1.291396in}%
\pgfsys@useobject{currentmarker}{}%
\end{pgfscope}%
\end{pgfscope}%
\begin{pgfscope}%
\pgfsetbuttcap%
\pgfsetroundjoin%
\definecolor{currentfill}{rgb}{0.000000,0.000000,0.000000}%
\pgfsetfillcolor{currentfill}%
\pgfsetlinewidth{0.602250pt}%
\definecolor{currentstroke}{rgb}{0.000000,0.000000,0.000000}%
\pgfsetstrokecolor{currentstroke}%
\pgfsetdash{}{0pt}%
\pgfsys@defobject{currentmarker}{\pgfqpoint{-0.027778in}{0.000000in}}{\pgfqpoint{-0.000000in}{0.000000in}}{%
\pgfpathmoveto{\pgfqpoint{-0.000000in}{0.000000in}}%
\pgfpathlineto{\pgfqpoint{-0.027778in}{0.000000in}}%
\pgfusepath{stroke,fill}%
}%
\begin{pgfscope}%
\pgfsys@transformshift{0.588387in}{1.309631in}%
\pgfsys@useobject{currentmarker}{}%
\end{pgfscope}%
\end{pgfscope}%
\begin{pgfscope}%
\pgfsetbuttcap%
\pgfsetroundjoin%
\definecolor{currentfill}{rgb}{0.000000,0.000000,0.000000}%
\pgfsetfillcolor{currentfill}%
\pgfsetlinewidth{0.602250pt}%
\definecolor{currentstroke}{rgb}{0.000000,0.000000,0.000000}%
\pgfsetstrokecolor{currentstroke}%
\pgfsetdash{}{0pt}%
\pgfsys@defobject{currentmarker}{\pgfqpoint{-0.027778in}{0.000000in}}{\pgfqpoint{-0.000000in}{0.000000in}}{%
\pgfpathmoveto{\pgfqpoint{-0.000000in}{0.000000in}}%
\pgfpathlineto{\pgfqpoint{-0.027778in}{0.000000in}}%
\pgfusepath{stroke,fill}%
}%
\begin{pgfscope}%
\pgfsys@transformshift{0.588387in}{1.433254in}%
\pgfsys@useobject{currentmarker}{}%
\end{pgfscope}%
\end{pgfscope}%
\begin{pgfscope}%
\pgfsetbuttcap%
\pgfsetroundjoin%
\definecolor{currentfill}{rgb}{0.000000,0.000000,0.000000}%
\pgfsetfillcolor{currentfill}%
\pgfsetlinewidth{0.602250pt}%
\definecolor{currentstroke}{rgb}{0.000000,0.000000,0.000000}%
\pgfsetstrokecolor{currentstroke}%
\pgfsetdash{}{0pt}%
\pgfsys@defobject{currentmarker}{\pgfqpoint{-0.027778in}{0.000000in}}{\pgfqpoint{-0.000000in}{0.000000in}}{%
\pgfpathmoveto{\pgfqpoint{-0.000000in}{0.000000in}}%
\pgfpathlineto{\pgfqpoint{-0.027778in}{0.000000in}}%
\pgfusepath{stroke,fill}%
}%
\begin{pgfscope}%
\pgfsys@transformshift{0.588387in}{1.496027in}%
\pgfsys@useobject{currentmarker}{}%
\end{pgfscope}%
\end{pgfscope}%
\begin{pgfscope}%
\pgfsetbuttcap%
\pgfsetroundjoin%
\definecolor{currentfill}{rgb}{0.000000,0.000000,0.000000}%
\pgfsetfillcolor{currentfill}%
\pgfsetlinewidth{0.602250pt}%
\definecolor{currentstroke}{rgb}{0.000000,0.000000,0.000000}%
\pgfsetstrokecolor{currentstroke}%
\pgfsetdash{}{0pt}%
\pgfsys@defobject{currentmarker}{\pgfqpoint{-0.027778in}{0.000000in}}{\pgfqpoint{-0.000000in}{0.000000in}}{%
\pgfpathmoveto{\pgfqpoint{-0.000000in}{0.000000in}}%
\pgfpathlineto{\pgfqpoint{-0.027778in}{0.000000in}}%
\pgfusepath{stroke,fill}%
}%
\begin{pgfscope}%
\pgfsys@transformshift{0.588387in}{1.540566in}%
\pgfsys@useobject{currentmarker}{}%
\end{pgfscope}%
\end{pgfscope}%
\begin{pgfscope}%
\pgfsetbuttcap%
\pgfsetroundjoin%
\definecolor{currentfill}{rgb}{0.000000,0.000000,0.000000}%
\pgfsetfillcolor{currentfill}%
\pgfsetlinewidth{0.602250pt}%
\definecolor{currentstroke}{rgb}{0.000000,0.000000,0.000000}%
\pgfsetstrokecolor{currentstroke}%
\pgfsetdash{}{0pt}%
\pgfsys@defobject{currentmarker}{\pgfqpoint{-0.027778in}{0.000000in}}{\pgfqpoint{-0.000000in}{0.000000in}}{%
\pgfpathmoveto{\pgfqpoint{-0.000000in}{0.000000in}}%
\pgfpathlineto{\pgfqpoint{-0.027778in}{0.000000in}}%
\pgfusepath{stroke,fill}%
}%
\begin{pgfscope}%
\pgfsys@transformshift{0.588387in}{1.575112in}%
\pgfsys@useobject{currentmarker}{}%
\end{pgfscope}%
\end{pgfscope}%
\begin{pgfscope}%
\pgfsetbuttcap%
\pgfsetroundjoin%
\definecolor{currentfill}{rgb}{0.000000,0.000000,0.000000}%
\pgfsetfillcolor{currentfill}%
\pgfsetlinewidth{0.602250pt}%
\definecolor{currentstroke}{rgb}{0.000000,0.000000,0.000000}%
\pgfsetstrokecolor{currentstroke}%
\pgfsetdash{}{0pt}%
\pgfsys@defobject{currentmarker}{\pgfqpoint{-0.027778in}{0.000000in}}{\pgfqpoint{-0.000000in}{0.000000in}}{%
\pgfpathmoveto{\pgfqpoint{-0.000000in}{0.000000in}}%
\pgfpathlineto{\pgfqpoint{-0.027778in}{0.000000in}}%
\pgfusepath{stroke,fill}%
}%
\begin{pgfscope}%
\pgfsys@transformshift{0.588387in}{1.603339in}%
\pgfsys@useobject{currentmarker}{}%
\end{pgfscope}%
\end{pgfscope}%
\begin{pgfscope}%
\pgfsetbuttcap%
\pgfsetroundjoin%
\definecolor{currentfill}{rgb}{0.000000,0.000000,0.000000}%
\pgfsetfillcolor{currentfill}%
\pgfsetlinewidth{0.602250pt}%
\definecolor{currentstroke}{rgb}{0.000000,0.000000,0.000000}%
\pgfsetstrokecolor{currentstroke}%
\pgfsetdash{}{0pt}%
\pgfsys@defobject{currentmarker}{\pgfqpoint{-0.027778in}{0.000000in}}{\pgfqpoint{-0.000000in}{0.000000in}}{%
\pgfpathmoveto{\pgfqpoint{-0.000000in}{0.000000in}}%
\pgfpathlineto{\pgfqpoint{-0.027778in}{0.000000in}}%
\pgfusepath{stroke,fill}%
}%
\begin{pgfscope}%
\pgfsys@transformshift{0.588387in}{1.627204in}%
\pgfsys@useobject{currentmarker}{}%
\end{pgfscope}%
\end{pgfscope}%
\begin{pgfscope}%
\pgfsetbuttcap%
\pgfsetroundjoin%
\definecolor{currentfill}{rgb}{0.000000,0.000000,0.000000}%
\pgfsetfillcolor{currentfill}%
\pgfsetlinewidth{0.602250pt}%
\definecolor{currentstroke}{rgb}{0.000000,0.000000,0.000000}%
\pgfsetstrokecolor{currentstroke}%
\pgfsetdash{}{0pt}%
\pgfsys@defobject{currentmarker}{\pgfqpoint{-0.027778in}{0.000000in}}{\pgfqpoint{-0.000000in}{0.000000in}}{%
\pgfpathmoveto{\pgfqpoint{-0.000000in}{0.000000in}}%
\pgfpathlineto{\pgfqpoint{-0.027778in}{0.000000in}}%
\pgfusepath{stroke,fill}%
}%
\begin{pgfscope}%
\pgfsys@transformshift{0.588387in}{1.647877in}%
\pgfsys@useobject{currentmarker}{}%
\end{pgfscope}%
\end{pgfscope}%
\begin{pgfscope}%
\pgfsetbuttcap%
\pgfsetroundjoin%
\definecolor{currentfill}{rgb}{0.000000,0.000000,0.000000}%
\pgfsetfillcolor{currentfill}%
\pgfsetlinewidth{0.602250pt}%
\definecolor{currentstroke}{rgb}{0.000000,0.000000,0.000000}%
\pgfsetstrokecolor{currentstroke}%
\pgfsetdash{}{0pt}%
\pgfsys@defobject{currentmarker}{\pgfqpoint{-0.027778in}{0.000000in}}{\pgfqpoint{-0.000000in}{0.000000in}}{%
\pgfpathmoveto{\pgfqpoint{-0.000000in}{0.000000in}}%
\pgfpathlineto{\pgfqpoint{-0.027778in}{0.000000in}}%
\pgfusepath{stroke,fill}%
}%
\begin{pgfscope}%
\pgfsys@transformshift{0.588387in}{1.666112in}%
\pgfsys@useobject{currentmarker}{}%
\end{pgfscope}%
\end{pgfscope}%
\begin{pgfscope}%
\pgfsetbuttcap%
\pgfsetroundjoin%
\definecolor{currentfill}{rgb}{0.000000,0.000000,0.000000}%
\pgfsetfillcolor{currentfill}%
\pgfsetlinewidth{0.602250pt}%
\definecolor{currentstroke}{rgb}{0.000000,0.000000,0.000000}%
\pgfsetstrokecolor{currentstroke}%
\pgfsetdash{}{0pt}%
\pgfsys@defobject{currentmarker}{\pgfqpoint{-0.027778in}{0.000000in}}{\pgfqpoint{-0.000000in}{0.000000in}}{%
\pgfpathmoveto{\pgfqpoint{-0.000000in}{0.000000in}}%
\pgfpathlineto{\pgfqpoint{-0.027778in}{0.000000in}}%
\pgfusepath{stroke,fill}%
}%
\begin{pgfscope}%
\pgfsys@transformshift{0.588387in}{1.789735in}%
\pgfsys@useobject{currentmarker}{}%
\end{pgfscope}%
\end{pgfscope}%
\begin{pgfscope}%
\pgfsetbuttcap%
\pgfsetroundjoin%
\definecolor{currentfill}{rgb}{0.000000,0.000000,0.000000}%
\pgfsetfillcolor{currentfill}%
\pgfsetlinewidth{0.602250pt}%
\definecolor{currentstroke}{rgb}{0.000000,0.000000,0.000000}%
\pgfsetstrokecolor{currentstroke}%
\pgfsetdash{}{0pt}%
\pgfsys@defobject{currentmarker}{\pgfqpoint{-0.027778in}{0.000000in}}{\pgfqpoint{-0.000000in}{0.000000in}}{%
\pgfpathmoveto{\pgfqpoint{-0.000000in}{0.000000in}}%
\pgfpathlineto{\pgfqpoint{-0.027778in}{0.000000in}}%
\pgfusepath{stroke,fill}%
}%
\begin{pgfscope}%
\pgfsys@transformshift{0.588387in}{1.852508in}%
\pgfsys@useobject{currentmarker}{}%
\end{pgfscope}%
\end{pgfscope}%
\begin{pgfscope}%
\pgfsetbuttcap%
\pgfsetroundjoin%
\definecolor{currentfill}{rgb}{0.000000,0.000000,0.000000}%
\pgfsetfillcolor{currentfill}%
\pgfsetlinewidth{0.602250pt}%
\definecolor{currentstroke}{rgb}{0.000000,0.000000,0.000000}%
\pgfsetstrokecolor{currentstroke}%
\pgfsetdash{}{0pt}%
\pgfsys@defobject{currentmarker}{\pgfqpoint{-0.027778in}{0.000000in}}{\pgfqpoint{-0.000000in}{0.000000in}}{%
\pgfpathmoveto{\pgfqpoint{-0.000000in}{0.000000in}}%
\pgfpathlineto{\pgfqpoint{-0.027778in}{0.000000in}}%
\pgfusepath{stroke,fill}%
}%
\begin{pgfscope}%
\pgfsys@transformshift{0.588387in}{1.897047in}%
\pgfsys@useobject{currentmarker}{}%
\end{pgfscope}%
\end{pgfscope}%
\begin{pgfscope}%
\pgfsetbuttcap%
\pgfsetroundjoin%
\definecolor{currentfill}{rgb}{0.000000,0.000000,0.000000}%
\pgfsetfillcolor{currentfill}%
\pgfsetlinewidth{0.602250pt}%
\definecolor{currentstroke}{rgb}{0.000000,0.000000,0.000000}%
\pgfsetstrokecolor{currentstroke}%
\pgfsetdash{}{0pt}%
\pgfsys@defobject{currentmarker}{\pgfqpoint{-0.027778in}{0.000000in}}{\pgfqpoint{-0.000000in}{0.000000in}}{%
\pgfpathmoveto{\pgfqpoint{-0.000000in}{0.000000in}}%
\pgfpathlineto{\pgfqpoint{-0.027778in}{0.000000in}}%
\pgfusepath{stroke,fill}%
}%
\begin{pgfscope}%
\pgfsys@transformshift{0.588387in}{1.931593in}%
\pgfsys@useobject{currentmarker}{}%
\end{pgfscope}%
\end{pgfscope}%
\begin{pgfscope}%
\pgfsetbuttcap%
\pgfsetroundjoin%
\definecolor{currentfill}{rgb}{0.000000,0.000000,0.000000}%
\pgfsetfillcolor{currentfill}%
\pgfsetlinewidth{0.602250pt}%
\definecolor{currentstroke}{rgb}{0.000000,0.000000,0.000000}%
\pgfsetstrokecolor{currentstroke}%
\pgfsetdash{}{0pt}%
\pgfsys@defobject{currentmarker}{\pgfqpoint{-0.027778in}{0.000000in}}{\pgfqpoint{-0.000000in}{0.000000in}}{%
\pgfpathmoveto{\pgfqpoint{-0.000000in}{0.000000in}}%
\pgfpathlineto{\pgfqpoint{-0.027778in}{0.000000in}}%
\pgfusepath{stroke,fill}%
}%
\begin{pgfscope}%
\pgfsys@transformshift{0.588387in}{1.959820in}%
\pgfsys@useobject{currentmarker}{}%
\end{pgfscope}%
\end{pgfscope}%
\begin{pgfscope}%
\pgfsetbuttcap%
\pgfsetroundjoin%
\definecolor{currentfill}{rgb}{0.000000,0.000000,0.000000}%
\pgfsetfillcolor{currentfill}%
\pgfsetlinewidth{0.602250pt}%
\definecolor{currentstroke}{rgb}{0.000000,0.000000,0.000000}%
\pgfsetstrokecolor{currentstroke}%
\pgfsetdash{}{0pt}%
\pgfsys@defobject{currentmarker}{\pgfqpoint{-0.027778in}{0.000000in}}{\pgfqpoint{-0.000000in}{0.000000in}}{%
\pgfpathmoveto{\pgfqpoint{-0.000000in}{0.000000in}}%
\pgfpathlineto{\pgfqpoint{-0.027778in}{0.000000in}}%
\pgfusepath{stroke,fill}%
}%
\begin{pgfscope}%
\pgfsys@transformshift{0.588387in}{1.983685in}%
\pgfsys@useobject{currentmarker}{}%
\end{pgfscope}%
\end{pgfscope}%
\begin{pgfscope}%
\pgfsetbuttcap%
\pgfsetroundjoin%
\definecolor{currentfill}{rgb}{0.000000,0.000000,0.000000}%
\pgfsetfillcolor{currentfill}%
\pgfsetlinewidth{0.602250pt}%
\definecolor{currentstroke}{rgb}{0.000000,0.000000,0.000000}%
\pgfsetstrokecolor{currentstroke}%
\pgfsetdash{}{0pt}%
\pgfsys@defobject{currentmarker}{\pgfqpoint{-0.027778in}{0.000000in}}{\pgfqpoint{-0.000000in}{0.000000in}}{%
\pgfpathmoveto{\pgfqpoint{-0.000000in}{0.000000in}}%
\pgfpathlineto{\pgfqpoint{-0.027778in}{0.000000in}}%
\pgfusepath{stroke,fill}%
}%
\begin{pgfscope}%
\pgfsys@transformshift{0.588387in}{2.004358in}%
\pgfsys@useobject{currentmarker}{}%
\end{pgfscope}%
\end{pgfscope}%
\begin{pgfscope}%
\pgfsetbuttcap%
\pgfsetroundjoin%
\definecolor{currentfill}{rgb}{0.000000,0.000000,0.000000}%
\pgfsetfillcolor{currentfill}%
\pgfsetlinewidth{0.602250pt}%
\definecolor{currentstroke}{rgb}{0.000000,0.000000,0.000000}%
\pgfsetstrokecolor{currentstroke}%
\pgfsetdash{}{0pt}%
\pgfsys@defobject{currentmarker}{\pgfqpoint{-0.027778in}{0.000000in}}{\pgfqpoint{-0.000000in}{0.000000in}}{%
\pgfpathmoveto{\pgfqpoint{-0.000000in}{0.000000in}}%
\pgfpathlineto{\pgfqpoint{-0.027778in}{0.000000in}}%
\pgfusepath{stroke,fill}%
}%
\begin{pgfscope}%
\pgfsys@transformshift{0.588387in}{2.022593in}%
\pgfsys@useobject{currentmarker}{}%
\end{pgfscope}%
\end{pgfscope}%
\begin{pgfscope}%
\pgfsetbuttcap%
\pgfsetroundjoin%
\definecolor{currentfill}{rgb}{0.000000,0.000000,0.000000}%
\pgfsetfillcolor{currentfill}%
\pgfsetlinewidth{0.602250pt}%
\definecolor{currentstroke}{rgb}{0.000000,0.000000,0.000000}%
\pgfsetstrokecolor{currentstroke}%
\pgfsetdash{}{0pt}%
\pgfsys@defobject{currentmarker}{\pgfqpoint{-0.027778in}{0.000000in}}{\pgfqpoint{-0.000000in}{0.000000in}}{%
\pgfpathmoveto{\pgfqpoint{-0.000000in}{0.000000in}}%
\pgfpathlineto{\pgfqpoint{-0.027778in}{0.000000in}}%
\pgfusepath{stroke,fill}%
}%
\begin{pgfscope}%
\pgfsys@transformshift{0.588387in}{2.146216in}%
\pgfsys@useobject{currentmarker}{}%
\end{pgfscope}%
\end{pgfscope}%
\begin{pgfscope}%
\pgfsetbuttcap%
\pgfsetroundjoin%
\definecolor{currentfill}{rgb}{0.000000,0.000000,0.000000}%
\pgfsetfillcolor{currentfill}%
\pgfsetlinewidth{0.602250pt}%
\definecolor{currentstroke}{rgb}{0.000000,0.000000,0.000000}%
\pgfsetstrokecolor{currentstroke}%
\pgfsetdash{}{0pt}%
\pgfsys@defobject{currentmarker}{\pgfqpoint{-0.027778in}{0.000000in}}{\pgfqpoint{-0.000000in}{0.000000in}}{%
\pgfpathmoveto{\pgfqpoint{-0.000000in}{0.000000in}}%
\pgfpathlineto{\pgfqpoint{-0.027778in}{0.000000in}}%
\pgfusepath{stroke,fill}%
}%
\begin{pgfscope}%
\pgfsys@transformshift{0.588387in}{2.208990in}%
\pgfsys@useobject{currentmarker}{}%
\end{pgfscope}%
\end{pgfscope}%
\begin{pgfscope}%
\pgfsetbuttcap%
\pgfsetroundjoin%
\definecolor{currentfill}{rgb}{0.000000,0.000000,0.000000}%
\pgfsetfillcolor{currentfill}%
\pgfsetlinewidth{0.602250pt}%
\definecolor{currentstroke}{rgb}{0.000000,0.000000,0.000000}%
\pgfsetstrokecolor{currentstroke}%
\pgfsetdash{}{0pt}%
\pgfsys@defobject{currentmarker}{\pgfqpoint{-0.027778in}{0.000000in}}{\pgfqpoint{-0.000000in}{0.000000in}}{%
\pgfpathmoveto{\pgfqpoint{-0.000000in}{0.000000in}}%
\pgfpathlineto{\pgfqpoint{-0.027778in}{0.000000in}}%
\pgfusepath{stroke,fill}%
}%
\begin{pgfscope}%
\pgfsys@transformshift{0.588387in}{2.253528in}%
\pgfsys@useobject{currentmarker}{}%
\end{pgfscope}%
\end{pgfscope}%
\begin{pgfscope}%
\pgfsetbuttcap%
\pgfsetroundjoin%
\definecolor{currentfill}{rgb}{0.000000,0.000000,0.000000}%
\pgfsetfillcolor{currentfill}%
\pgfsetlinewidth{0.602250pt}%
\definecolor{currentstroke}{rgb}{0.000000,0.000000,0.000000}%
\pgfsetstrokecolor{currentstroke}%
\pgfsetdash{}{0pt}%
\pgfsys@defobject{currentmarker}{\pgfqpoint{-0.027778in}{0.000000in}}{\pgfqpoint{-0.000000in}{0.000000in}}{%
\pgfpathmoveto{\pgfqpoint{-0.000000in}{0.000000in}}%
\pgfpathlineto{\pgfqpoint{-0.027778in}{0.000000in}}%
\pgfusepath{stroke,fill}%
}%
\begin{pgfscope}%
\pgfsys@transformshift{0.588387in}{2.288075in}%
\pgfsys@useobject{currentmarker}{}%
\end{pgfscope}%
\end{pgfscope}%
\begin{pgfscope}%
\pgfsetbuttcap%
\pgfsetroundjoin%
\definecolor{currentfill}{rgb}{0.000000,0.000000,0.000000}%
\pgfsetfillcolor{currentfill}%
\pgfsetlinewidth{0.602250pt}%
\definecolor{currentstroke}{rgb}{0.000000,0.000000,0.000000}%
\pgfsetstrokecolor{currentstroke}%
\pgfsetdash{}{0pt}%
\pgfsys@defobject{currentmarker}{\pgfqpoint{-0.027778in}{0.000000in}}{\pgfqpoint{-0.000000in}{0.000000in}}{%
\pgfpathmoveto{\pgfqpoint{-0.000000in}{0.000000in}}%
\pgfpathlineto{\pgfqpoint{-0.027778in}{0.000000in}}%
\pgfusepath{stroke,fill}%
}%
\begin{pgfscope}%
\pgfsys@transformshift{0.588387in}{2.316301in}%
\pgfsys@useobject{currentmarker}{}%
\end{pgfscope}%
\end{pgfscope}%
\begin{pgfscope}%
\pgfsetbuttcap%
\pgfsetroundjoin%
\definecolor{currentfill}{rgb}{0.000000,0.000000,0.000000}%
\pgfsetfillcolor{currentfill}%
\pgfsetlinewidth{0.602250pt}%
\definecolor{currentstroke}{rgb}{0.000000,0.000000,0.000000}%
\pgfsetstrokecolor{currentstroke}%
\pgfsetdash{}{0pt}%
\pgfsys@defobject{currentmarker}{\pgfqpoint{-0.027778in}{0.000000in}}{\pgfqpoint{-0.000000in}{0.000000in}}{%
\pgfpathmoveto{\pgfqpoint{-0.000000in}{0.000000in}}%
\pgfpathlineto{\pgfqpoint{-0.027778in}{0.000000in}}%
\pgfusepath{stroke,fill}%
}%
\begin{pgfscope}%
\pgfsys@transformshift{0.588387in}{2.340167in}%
\pgfsys@useobject{currentmarker}{}%
\end{pgfscope}%
\end{pgfscope}%
\begin{pgfscope}%
\pgfsetbuttcap%
\pgfsetroundjoin%
\definecolor{currentfill}{rgb}{0.000000,0.000000,0.000000}%
\pgfsetfillcolor{currentfill}%
\pgfsetlinewidth{0.602250pt}%
\definecolor{currentstroke}{rgb}{0.000000,0.000000,0.000000}%
\pgfsetstrokecolor{currentstroke}%
\pgfsetdash{}{0pt}%
\pgfsys@defobject{currentmarker}{\pgfqpoint{-0.027778in}{0.000000in}}{\pgfqpoint{-0.000000in}{0.000000in}}{%
\pgfpathmoveto{\pgfqpoint{-0.000000in}{0.000000in}}%
\pgfpathlineto{\pgfqpoint{-0.027778in}{0.000000in}}%
\pgfusepath{stroke,fill}%
}%
\begin{pgfscope}%
\pgfsys@transformshift{0.588387in}{2.360840in}%
\pgfsys@useobject{currentmarker}{}%
\end{pgfscope}%
\end{pgfscope}%
\begin{pgfscope}%
\pgfsetbuttcap%
\pgfsetroundjoin%
\definecolor{currentfill}{rgb}{0.000000,0.000000,0.000000}%
\pgfsetfillcolor{currentfill}%
\pgfsetlinewidth{0.602250pt}%
\definecolor{currentstroke}{rgb}{0.000000,0.000000,0.000000}%
\pgfsetstrokecolor{currentstroke}%
\pgfsetdash{}{0pt}%
\pgfsys@defobject{currentmarker}{\pgfqpoint{-0.027778in}{0.000000in}}{\pgfqpoint{-0.000000in}{0.000000in}}{%
\pgfpathmoveto{\pgfqpoint{-0.000000in}{0.000000in}}%
\pgfpathlineto{\pgfqpoint{-0.027778in}{0.000000in}}%
\pgfusepath{stroke,fill}%
}%
\begin{pgfscope}%
\pgfsys@transformshift{0.588387in}{2.379074in}%
\pgfsys@useobject{currentmarker}{}%
\end{pgfscope}%
\end{pgfscope}%
\begin{pgfscope}%
\pgfsetbuttcap%
\pgfsetroundjoin%
\definecolor{currentfill}{rgb}{0.000000,0.000000,0.000000}%
\pgfsetfillcolor{currentfill}%
\pgfsetlinewidth{0.602250pt}%
\definecolor{currentstroke}{rgb}{0.000000,0.000000,0.000000}%
\pgfsetstrokecolor{currentstroke}%
\pgfsetdash{}{0pt}%
\pgfsys@defobject{currentmarker}{\pgfqpoint{-0.027778in}{0.000000in}}{\pgfqpoint{-0.000000in}{0.000000in}}{%
\pgfpathmoveto{\pgfqpoint{-0.000000in}{0.000000in}}%
\pgfpathlineto{\pgfqpoint{-0.027778in}{0.000000in}}%
\pgfusepath{stroke,fill}%
}%
\begin{pgfscope}%
\pgfsys@transformshift{0.588387in}{2.502698in}%
\pgfsys@useobject{currentmarker}{}%
\end{pgfscope}%
\end{pgfscope}%
\begin{pgfscope}%
\definecolor{textcolor}{rgb}{0.000000,0.000000,0.000000}%
\pgfsetstrokecolor{textcolor}%
\pgfsetfillcolor{textcolor}%
\pgftext[x=0.234413in,y=1.526746in,,bottom,rotate=90.000000]{\color{textcolor}{\rmfamily\fontsize{10.000000}{12.000000}\selectfont\catcode`\^=\active\def^{\ifmmode\sp\else\^{}\fi}\catcode`\%=\active\def%{\%}Checks [call]}}%
\end{pgfscope}%
\begin{pgfscope}%
\pgfpathrectangle{\pgfqpoint{0.588387in}{0.521603in}}{\pgfqpoint{5.399676in}{2.010285in}}%
\pgfusepath{clip}%
\pgfsetrectcap%
\pgfsetroundjoin%
\pgfsetlinewidth{1.505625pt}%
\pgfsetstrokecolor{currentstroke1}%
\pgfsetdash{}{0pt}%
\pgfpathmoveto{\pgfqpoint{0.833827in}{0.612980in}}%
\pgfpathlineto{\pgfqpoint{0.881485in}{0.624453in}}%
\pgfpathlineto{\pgfqpoint{0.929143in}{0.735047in}}%
\pgfpathlineto{\pgfqpoint{0.976802in}{0.777222in}}%
\pgfpathlineto{\pgfqpoint{1.024460in}{0.842817in}}%
\pgfpathlineto{\pgfqpoint{1.072118in}{0.909428in}}%
\pgfpathlineto{\pgfqpoint{1.119776in}{0.917940in}}%
\pgfpathlineto{\pgfqpoint{1.167435in}{1.002272in}}%
\pgfpathlineto{\pgfqpoint{1.215093in}{1.031921in}}%
\pgfpathlineto{\pgfqpoint{1.262751in}{1.116412in}}%
\pgfpathlineto{\pgfqpoint{1.310409in}{1.094753in}}%
\pgfpathlineto{\pgfqpoint{1.358067in}{1.160414in}}%
\pgfpathlineto{\pgfqpoint{1.405726in}{1.087419in}}%
\pgfpathlineto{\pgfqpoint{1.453384in}{1.180989in}}%
\pgfpathlineto{\pgfqpoint{1.501042in}{1.188319in}}%
\pgfpathlineto{\pgfqpoint{1.548700in}{1.295696in}}%
\pgfpathlineto{\pgfqpoint{1.596358in}{1.291328in}}%
\pgfpathlineto{\pgfqpoint{1.644017in}{1.393484in}}%
\pgfpathlineto{\pgfqpoint{1.691675in}{1.334317in}}%
\pgfpathlineto{\pgfqpoint{1.739333in}{1.384097in}}%
\pgfpathlineto{\pgfqpoint{1.786991in}{1.341073in}}%
\pgfpathlineto{\pgfqpoint{1.834650in}{1.395777in}}%
\pgfpathlineto{\pgfqpoint{1.882308in}{1.331921in}}%
\pgfpathlineto{\pgfqpoint{1.929966in}{1.514436in}}%
\pgfpathlineto{\pgfqpoint{1.977624in}{1.311341in}}%
\pgfpathlineto{\pgfqpoint{2.025282in}{1.403812in}}%
\pgfpathlineto{\pgfqpoint{2.072941in}{1.274693in}}%
\pgfpathlineto{\pgfqpoint{2.120599in}{1.729426in}}%
\pgfpathlineto{\pgfqpoint{2.168257in}{1.266803in}}%
\pgfpathlineto{\pgfqpoint{2.215915in}{1.399561in}}%
\pgfpathlineto{\pgfqpoint{2.263573in}{1.691737in}}%
\pgfpathlineto{\pgfqpoint{2.311232in}{1.526795in}}%
\pgfpathlineto{\pgfqpoint{2.358890in}{1.329828in}}%
\pgfpathlineto{\pgfqpoint{2.406548in}{1.582297in}}%
\pgfpathlineto{\pgfqpoint{2.454206in}{1.721532in}}%
\pgfpathlineto{\pgfqpoint{2.501865in}{1.345795in}}%
\pgfpathlineto{\pgfqpoint{2.549523in}{1.207950in}}%
\pgfpathlineto{\pgfqpoint{2.597181in}{1.513390in}}%
\pgfpathlineto{\pgfqpoint{2.644839in}{1.517022in}}%
\pgfpathlineto{\pgfqpoint{2.692497in}{1.471339in}}%
\pgfpathlineto{\pgfqpoint{2.740156in}{1.087642in}}%
\pgfpathlineto{\pgfqpoint{2.787814in}{1.223521in}}%
\pgfpathlineto{\pgfqpoint{2.835472in}{1.229827in}}%
\pgfpathlineto{\pgfqpoint{2.883130in}{1.387347in}}%
\pgfpathlineto{\pgfqpoint{2.930789in}{1.283141in}}%
\pgfpathlineto{\pgfqpoint{2.978447in}{1.389744in}}%
\pgfpathlineto{\pgfqpoint{3.026105in}{2.038284in}}%
\pgfpathlineto{\pgfqpoint{3.073763in}{1.070632in}}%
\pgfpathlineto{\pgfqpoint{3.121421in}{1.280220in}}%
\pgfpathlineto{\pgfqpoint{3.169080in}{1.200864in}}%
\pgfpathlineto{\pgfqpoint{3.216738in}{1.138064in}}%
\pgfpathlineto{\pgfqpoint{3.264396in}{1.251646in}}%
\pgfpathlineto{\pgfqpoint{3.312054in}{1.294462in}}%
\pgfpathlineto{\pgfqpoint{3.359712in}{1.413157in}}%
\pgfpathlineto{\pgfqpoint{3.407371in}{1.372690in}}%
\pgfpathlineto{\pgfqpoint{3.455029in}{1.218331in}}%
\pgfpathlineto{\pgfqpoint{3.502687in}{0.962423in}}%
\pgfpathlineto{\pgfqpoint{3.550345in}{1.354673in}}%
\pgfpathlineto{\pgfqpoint{3.598004in}{1.356909in}}%
\pgfpathlineto{\pgfqpoint{3.645662in}{1.313823in}}%
\pgfpathlineto{\pgfqpoint{3.693320in}{0.839995in}}%
\pgfpathlineto{\pgfqpoint{3.740978in}{1.751962in}}%
\pgfpathlineto{\pgfqpoint{3.788636in}{0.783065in}}%
\pgfpathlineto{\pgfqpoint{3.836295in}{1.120116in}}%
\pgfpathlineto{\pgfqpoint{3.931611in}{1.446862in}}%
\pgfpathlineto{\pgfqpoint{4.026927in}{1.038023in}}%
\pgfpathlineto{\pgfqpoint{4.122244in}{1.388544in}}%
\pgfpathlineto{\pgfqpoint{4.217560in}{1.144622in}}%
\pgfusepath{stroke}%
\end{pgfscope}%
\begin{pgfscope}%
\pgfpathrectangle{\pgfqpoint{0.588387in}{0.521603in}}{\pgfqpoint{5.399676in}{2.010285in}}%
\pgfusepath{clip}%
\pgfsetrectcap%
\pgfsetroundjoin%
\pgfsetlinewidth{1.505625pt}%
\pgfsetstrokecolor{currentstroke2}%
\pgfsetdash{}{0pt}%
\pgfpathmoveto{\pgfqpoint{0.833827in}{0.720291in}}%
\pgfpathlineto{\pgfqpoint{0.881485in}{0.720291in}}%
\pgfpathlineto{\pgfqpoint{0.929143in}{0.789171in}}%
\pgfpathlineto{\pgfqpoint{0.976802in}{0.831695in}}%
\pgfpathlineto{\pgfqpoint{1.024460in}{0.883787in}}%
\pgfpathlineto{\pgfqpoint{1.072118in}{0.931655in}}%
\pgfpathlineto{\pgfqpoint{1.119776in}{0.969357in}}%
\pgfpathlineto{\pgfqpoint{1.167435in}{1.003782in}}%
\pgfpathlineto{\pgfqpoint{1.215093in}{1.048333in}}%
\pgfpathlineto{\pgfqpoint{1.262751in}{1.116523in}}%
\pgfpathlineto{\pgfqpoint{1.310409in}{1.182407in}}%
\pgfpathlineto{\pgfqpoint{1.358067in}{1.188635in}}%
\pgfpathlineto{\pgfqpoint{1.405726in}{1.269638in}}%
\pgfpathlineto{\pgfqpoint{1.453384in}{1.282833in}}%
\pgfpathlineto{\pgfqpoint{1.501042in}{1.363868in}}%
\pgfpathlineto{\pgfqpoint{1.548700in}{1.368601in}}%
\pgfpathlineto{\pgfqpoint{1.596358in}{1.281591in}}%
\pgfpathlineto{\pgfqpoint{1.644017in}{1.289435in}}%
\pgfpathlineto{\pgfqpoint{1.691675in}{1.338557in}}%
\pgfpathlineto{\pgfqpoint{1.739333in}{1.376106in}}%
\pgfpathlineto{\pgfqpoint{1.786991in}{1.383512in}}%
\pgfpathlineto{\pgfqpoint{1.834650in}{1.435740in}}%
\pgfpathlineto{\pgfqpoint{1.882308in}{1.348973in}}%
\pgfpathlineto{\pgfqpoint{1.929966in}{1.344064in}}%
\pgfpathlineto{\pgfqpoint{1.977624in}{1.380443in}}%
\pgfpathlineto{\pgfqpoint{2.025282in}{1.414320in}}%
\pgfpathlineto{\pgfqpoint{2.072941in}{1.438393in}}%
\pgfpathlineto{\pgfqpoint{2.120599in}{1.446059in}}%
\pgfpathlineto{\pgfqpoint{2.168257in}{1.384266in}}%
\pgfpathlineto{\pgfqpoint{2.215915in}{1.386567in}}%
\pgfpathlineto{\pgfqpoint{2.263573in}{1.415629in}}%
\pgfpathlineto{\pgfqpoint{2.311232in}{1.443525in}}%
\pgfpathlineto{\pgfqpoint{2.358890in}{1.464416in}}%
\pgfpathlineto{\pgfqpoint{2.406548in}{1.487169in}}%
\pgfpathlineto{\pgfqpoint{2.454206in}{1.419491in}}%
\pgfpathlineto{\pgfqpoint{2.501865in}{1.421132in}}%
\pgfpathlineto{\pgfqpoint{2.549523in}{1.444006in}}%
\pgfpathlineto{\pgfqpoint{2.597181in}{1.466081in}}%
\pgfpathlineto{\pgfqpoint{2.644839in}{1.484345in}}%
\pgfpathlineto{\pgfqpoint{2.692497in}{1.502628in}}%
\pgfpathlineto{\pgfqpoint{2.740156in}{1.447208in}}%
\pgfpathlineto{\pgfqpoint{2.787814in}{1.446930in}}%
\pgfpathlineto{\pgfqpoint{2.835472in}{1.468842in}}%
\pgfpathlineto{\pgfqpoint{2.883130in}{1.485445in}}%
\pgfpathlineto{\pgfqpoint{2.930789in}{1.504269in}}%
\pgfpathlineto{\pgfqpoint{2.978447in}{1.518048in}}%
\pgfpathlineto{\pgfqpoint{3.026105in}{1.466588in}}%
\pgfpathlineto{\pgfqpoint{3.073763in}{1.468534in}}%
\pgfpathlineto{\pgfqpoint{3.121421in}{1.486526in}}%
\pgfpathlineto{\pgfqpoint{3.169080in}{1.504606in}}%
\pgfpathlineto{\pgfqpoint{3.216738in}{1.517079in}}%
\pgfpathlineto{\pgfqpoint{3.264396in}{1.533581in}}%
\pgfpathlineto{\pgfqpoint{3.312054in}{1.486558in}}%
\pgfpathlineto{\pgfqpoint{3.359712in}{1.486639in}}%
\pgfpathlineto{\pgfqpoint{3.407371in}{1.507703in}}%
\pgfpathlineto{\pgfqpoint{3.455029in}{1.522874in}}%
\pgfpathlineto{\pgfqpoint{3.502687in}{1.531123in}}%
\pgfpathlineto{\pgfqpoint{3.550345in}{1.545227in}}%
\pgfpathlineto{\pgfqpoint{3.598004in}{1.502703in}}%
\pgfpathlineto{\pgfqpoint{3.645662in}{1.507665in}}%
\pgfpathlineto{\pgfqpoint{3.693320in}{1.527593in}}%
\pgfpathlineto{\pgfqpoint{3.740978in}{1.536591in}}%
\pgfpathlineto{\pgfqpoint{3.788636in}{1.544907in}}%
\pgfpathlineto{\pgfqpoint{3.836295in}{1.558709in}}%
\pgfpathlineto{\pgfqpoint{3.883953in}{1.523985in}}%
\pgfpathlineto{\pgfqpoint{3.931611in}{1.523385in}}%
\pgfpathlineto{\pgfqpoint{3.979269in}{1.530780in}}%
\pgfpathlineto{\pgfqpoint{4.026927in}{1.547447in}}%
\pgfpathlineto{\pgfqpoint{4.074586in}{1.555760in}}%
\pgfpathlineto{\pgfqpoint{4.122244in}{1.569313in}}%
\pgfpathlineto{\pgfqpoint{4.169902in}{1.534615in}}%
\pgfpathlineto{\pgfqpoint{4.217560in}{1.535324in}}%
\pgfpathlineto{\pgfqpoint{4.265219in}{1.546568in}}%
\pgfpathlineto{\pgfqpoint{4.312877in}{1.558600in}}%
\pgfpathlineto{\pgfqpoint{4.408193in}{1.579454in}}%
\pgfpathlineto{\pgfqpoint{4.455851in}{1.546470in}}%
\pgfpathlineto{\pgfqpoint{4.503510in}{1.547287in}}%
\pgfpathlineto{\pgfqpoint{4.598826in}{1.573478in}}%
\pgfpathlineto{\pgfqpoint{4.646484in}{1.577035in}}%
\pgfpathlineto{\pgfqpoint{4.694142in}{1.588714in}}%
\pgfpathlineto{\pgfqpoint{4.741801in}{1.559658in}}%
\pgfpathlineto{\pgfqpoint{4.789459in}{1.560583in}}%
\pgfpathlineto{\pgfqpoint{4.837117in}{1.567659in}}%
\pgfpathlineto{\pgfqpoint{4.884775in}{1.583401in}}%
\pgfpathlineto{\pgfqpoint{4.980092in}{1.597335in}}%
\pgfpathlineto{\pgfqpoint{5.027750in}{1.568247in}}%
\pgfpathlineto{\pgfqpoint{5.075408in}{1.571295in}}%
\pgfpathlineto{\pgfqpoint{5.123066in}{1.584461in}}%
\pgfpathlineto{\pgfqpoint{5.170725in}{1.590026in}}%
\pgfpathlineto{\pgfqpoint{5.218383in}{1.596514in}}%
\pgfpathlineto{\pgfqpoint{5.266041in}{1.607085in}}%
\pgfpathlineto{\pgfqpoint{5.361357in}{1.580874in}}%
\pgfpathlineto{\pgfqpoint{5.409016in}{1.589480in}}%
\pgfpathlineto{\pgfqpoint{5.456674in}{1.599980in}}%
\pgfpathlineto{\pgfqpoint{5.551990in}{1.613401in}}%
\pgfpathlineto{\pgfqpoint{5.647307in}{1.585007in}}%
\pgfpathlineto{\pgfqpoint{5.694965in}{1.609038in}}%
\pgfpathlineto{\pgfqpoint{5.742623in}{1.609844in}}%
\pgfusepath{stroke}%
\end{pgfscope}%
\begin{pgfscope}%
\pgfpathrectangle{\pgfqpoint{0.588387in}{0.521603in}}{\pgfqpoint{5.399676in}{2.010285in}}%
\pgfusepath{clip}%
\pgfsetrectcap%
\pgfsetroundjoin%
\pgfsetlinewidth{1.505625pt}%
\pgfsetstrokecolor{currentstroke3}%
\pgfsetdash{}{0pt}%
\pgfpathmoveto{\pgfqpoint{0.833827in}{0.720291in}}%
\pgfpathlineto{\pgfqpoint{0.881485in}{0.720291in}}%
\pgfpathlineto{\pgfqpoint{0.929143in}{0.794980in}}%
\pgfpathlineto{\pgfqpoint{0.976802in}{0.837997in}}%
\pgfpathlineto{\pgfqpoint{1.024460in}{0.893706in}}%
\pgfpathlineto{\pgfqpoint{1.072118in}{0.929284in}}%
\pgfpathlineto{\pgfqpoint{1.119776in}{0.956034in}}%
\pgfpathlineto{\pgfqpoint{1.167435in}{1.007521in}}%
\pgfpathlineto{\pgfqpoint{1.215093in}{1.026012in}}%
\pgfpathlineto{\pgfqpoint{1.262751in}{1.137245in}}%
\pgfpathlineto{\pgfqpoint{1.310409in}{1.191296in}}%
\pgfpathlineto{\pgfqpoint{1.358067in}{1.195328in}}%
\pgfpathlineto{\pgfqpoint{1.405726in}{1.281684in}}%
\pgfpathlineto{\pgfqpoint{1.453384in}{1.280397in}}%
\pgfpathlineto{\pgfqpoint{1.501042in}{1.374948in}}%
\pgfpathlineto{\pgfqpoint{1.548700in}{1.381591in}}%
\pgfpathlineto{\pgfqpoint{1.596358in}{1.469002in}}%
\pgfpathlineto{\pgfqpoint{1.644017in}{1.457121in}}%
\pgfpathlineto{\pgfqpoint{1.691675in}{1.617789in}}%
\pgfpathlineto{\pgfqpoint{1.739333in}{1.709497in}}%
\pgfpathlineto{\pgfqpoint{1.786991in}{1.720731in}}%
\pgfpathlineto{\pgfqpoint{1.834650in}{1.690074in}}%
\pgfpathlineto{\pgfqpoint{1.882308in}{1.727199in}}%
\pgfpathlineto{\pgfqpoint{1.929966in}{1.836375in}}%
\pgfpathlineto{\pgfqpoint{1.977624in}{1.410857in}}%
\pgfpathlineto{\pgfqpoint{2.025282in}{1.761451in}}%
\pgfpathlineto{\pgfqpoint{2.072941in}{2.040251in}}%
\pgfpathlineto{\pgfqpoint{2.120599in}{2.080853in}}%
\pgfpathlineto{\pgfqpoint{2.168257in}{1.592982in}}%
\pgfpathlineto{\pgfqpoint{2.215915in}{1.897766in}}%
\pgfpathlineto{\pgfqpoint{2.263573in}{1.882832in}}%
\pgfpathlineto{\pgfqpoint{2.311232in}{2.038196in}}%
\pgfpathlineto{\pgfqpoint{2.358890in}{1.969661in}}%
\pgfpathlineto{\pgfqpoint{2.406548in}{1.959571in}}%
\pgfpathlineto{\pgfqpoint{2.454206in}{1.796842in}}%
\pgfpathlineto{\pgfqpoint{2.501865in}{1.786451in}}%
\pgfpathlineto{\pgfqpoint{2.549523in}{1.947525in}}%
\pgfpathlineto{\pgfqpoint{2.597181in}{1.805345in}}%
\pgfpathlineto{\pgfqpoint{2.644839in}{1.737572in}}%
\pgfpathlineto{\pgfqpoint{2.692497in}{1.858110in}}%
\pgfpathlineto{\pgfqpoint{2.740156in}{1.580318in}}%
\pgfpathlineto{\pgfqpoint{2.787814in}{1.983640in}}%
\pgfpathlineto{\pgfqpoint{2.835472in}{1.873736in}}%
\pgfpathlineto{\pgfqpoint{2.883130in}{1.770384in}}%
\pgfpathlineto{\pgfqpoint{2.930789in}{1.838976in}}%
\pgfpathlineto{\pgfqpoint{2.978447in}{2.085872in}}%
\pgfpathlineto{\pgfqpoint{3.026105in}{1.865249in}}%
\pgfpathlineto{\pgfqpoint{3.073763in}{1.824953in}}%
\pgfpathlineto{\pgfqpoint{3.121421in}{2.068203in}}%
\pgfpathlineto{\pgfqpoint{3.169080in}{1.697687in}}%
\pgfpathlineto{\pgfqpoint{3.216738in}{1.811993in}}%
\pgfpathlineto{\pgfqpoint{3.264396in}{1.832600in}}%
\pgfpathlineto{\pgfqpoint{3.312054in}{2.028887in}}%
\pgfpathlineto{\pgfqpoint{3.359712in}{1.698441in}}%
\pgfpathlineto{\pgfqpoint{3.407371in}{1.498332in}}%
\pgfpathlineto{\pgfqpoint{3.455029in}{1.992131in}}%
\pgfpathlineto{\pgfqpoint{3.502687in}{1.682079in}}%
\pgfpathlineto{\pgfqpoint{3.550345in}{1.927947in}}%
\pgfpathlineto{\pgfqpoint{3.598004in}{1.708572in}}%
\pgfpathlineto{\pgfqpoint{3.645662in}{1.647459in}}%
\pgfpathlineto{\pgfqpoint{3.693320in}{2.003854in}}%
\pgfpathlineto{\pgfqpoint{3.740978in}{1.647974in}}%
\pgfpathlineto{\pgfqpoint{3.788636in}{1.737201in}}%
\pgfpathlineto{\pgfqpoint{3.836295in}{1.802442in}}%
\pgfpathlineto{\pgfqpoint{3.883953in}{1.516539in}}%
\pgfpathlineto{\pgfqpoint{3.931611in}{1.998492in}}%
\pgfpathlineto{\pgfqpoint{3.979269in}{1.540566in}}%
\pgfpathlineto{\pgfqpoint{4.026927in}{1.677956in}}%
\pgfpathlineto{\pgfqpoint{4.074586in}{1.699369in}}%
\pgfpathlineto{\pgfqpoint{4.122244in}{1.759917in}}%
\pgfpathlineto{\pgfqpoint{4.169902in}{1.527095in}}%
\pgfpathlineto{\pgfqpoint{4.217560in}{1.719662in}}%
\pgfpathlineto{\pgfqpoint{4.265219in}{1.695386in}}%
\pgfpathlineto{\pgfqpoint{4.312877in}{1.751530in}}%
\pgfpathlineto{\pgfqpoint{4.408193in}{1.990445in}}%
\pgfpathlineto{\pgfqpoint{4.455851in}{1.756025in}}%
\pgfpathlineto{\pgfqpoint{4.503510in}{1.763730in}}%
\pgfpathlineto{\pgfqpoint{4.598826in}{1.705718in}}%
\pgfpathlineto{\pgfqpoint{4.646484in}{1.574492in}}%
\pgfpathlineto{\pgfqpoint{4.694142in}{1.784553in}}%
\pgfpathlineto{\pgfqpoint{4.741801in}{2.216251in}}%
\pgfpathlineto{\pgfqpoint{4.789459in}{1.885624in}}%
\pgfpathlineto{\pgfqpoint{4.837117in}{1.570396in}}%
\pgfpathlineto{\pgfqpoint{4.884775in}{1.909677in}}%
\pgfpathlineto{\pgfqpoint{4.980092in}{1.784273in}}%
\pgfpathlineto{\pgfqpoint{5.027750in}{1.757366in}}%
\pgfpathlineto{\pgfqpoint{5.075408in}{1.744628in}}%
\pgfpathlineto{\pgfqpoint{5.123066in}{1.918617in}}%
\pgfpathlineto{\pgfqpoint{5.170725in}{1.849047in}}%
\pgfpathlineto{\pgfqpoint{5.218383in}{1.921486in}}%
\pgfpathlineto{\pgfqpoint{5.266041in}{2.023417in}}%
\pgfpathlineto{\pgfqpoint{5.361357in}{1.640698in}}%
\pgfpathlineto{\pgfqpoint{5.409016in}{1.585876in}}%
\pgfpathlineto{\pgfqpoint{5.456674in}{2.133992in}}%
\pgfpathlineto{\pgfqpoint{5.551990in}{1.920303in}}%
\pgfpathlineto{\pgfqpoint{5.647307in}{1.784940in}}%
\pgfpathlineto{\pgfqpoint{5.694965in}{1.593759in}}%
\pgfpathlineto{\pgfqpoint{5.742623in}{1.832893in}}%
\pgfusepath{stroke}%
\end{pgfscope}%
\begin{pgfscope}%
\pgfpathrectangle{\pgfqpoint{0.588387in}{0.521603in}}{\pgfqpoint{5.399676in}{2.010285in}}%
\pgfusepath{clip}%
\pgfsetrectcap%
\pgfsetroundjoin%
\pgfsetlinewidth{1.505625pt}%
\pgfsetstrokecolor{currentstroke4}%
\pgfsetdash{}{0pt}%
\pgfpathmoveto{\pgfqpoint{0.833827in}{0.720291in}}%
\pgfpathlineto{\pgfqpoint{0.881485in}{0.720291in}}%
\pgfpathlineto{\pgfqpoint{0.929143in}{0.789171in}}%
\pgfpathlineto{\pgfqpoint{0.976802in}{0.827603in}}%
\pgfpathlineto{\pgfqpoint{1.024460in}{0.884729in}}%
\pgfpathlineto{\pgfqpoint{1.072118in}{0.934268in}}%
\pgfpathlineto{\pgfqpoint{1.119776in}{0.957391in}}%
\pgfpathlineto{\pgfqpoint{1.167435in}{0.988822in}}%
\pgfpathlineto{\pgfqpoint{1.215093in}{1.038060in}}%
\pgfpathlineto{\pgfqpoint{1.262751in}{1.120816in}}%
\pgfpathlineto{\pgfqpoint{1.310409in}{1.190193in}}%
\pgfpathlineto{\pgfqpoint{1.358067in}{1.198460in}}%
\pgfpathlineto{\pgfqpoint{1.405726in}{1.269423in}}%
\pgfpathlineto{\pgfqpoint{1.453384in}{1.286602in}}%
\pgfpathlineto{\pgfqpoint{1.501042in}{1.362341in}}%
\pgfpathlineto{\pgfqpoint{1.548700in}{1.378102in}}%
\pgfpathlineto{\pgfqpoint{1.596358in}{1.302741in}}%
\pgfpathlineto{\pgfqpoint{1.644017in}{1.302846in}}%
\pgfpathlineto{\pgfqpoint{1.691675in}{1.345608in}}%
\pgfpathlineto{\pgfqpoint{1.739333in}{1.389174in}}%
\pgfpathlineto{\pgfqpoint{1.786991in}{1.397428in}}%
\pgfpathlineto{\pgfqpoint{1.834650in}{1.441917in}}%
\pgfpathlineto{\pgfqpoint{1.882308in}{1.355763in}}%
\pgfpathlineto{\pgfqpoint{1.929966in}{1.362030in}}%
\pgfpathlineto{\pgfqpoint{1.977624in}{1.394275in}}%
\pgfpathlineto{\pgfqpoint{2.025282in}{1.427145in}}%
\pgfpathlineto{\pgfqpoint{2.072941in}{1.447561in}}%
\pgfpathlineto{\pgfqpoint{2.120599in}{1.452685in}}%
\pgfpathlineto{\pgfqpoint{2.168257in}{1.402727in}}%
\pgfpathlineto{\pgfqpoint{2.215915in}{1.410737in}}%
\pgfpathlineto{\pgfqpoint{2.263573in}{1.428937in}}%
\pgfpathlineto{\pgfqpoint{2.311232in}{1.459877in}}%
\pgfpathlineto{\pgfqpoint{2.358890in}{1.473108in}}%
\pgfpathlineto{\pgfqpoint{2.406548in}{1.499591in}}%
\pgfpathlineto{\pgfqpoint{2.454206in}{1.430201in}}%
\pgfpathlineto{\pgfqpoint{2.501865in}{1.437604in}}%
\pgfpathlineto{\pgfqpoint{2.549523in}{1.461771in}}%
\pgfpathlineto{\pgfqpoint{2.597181in}{1.478615in}}%
\pgfpathlineto{\pgfqpoint{2.644839in}{1.498060in}}%
\pgfpathlineto{\pgfqpoint{2.692497in}{1.513971in}}%
\pgfpathlineto{\pgfqpoint{2.740156in}{1.450846in}}%
\pgfpathlineto{\pgfqpoint{2.787814in}{1.465003in}}%
\pgfpathlineto{\pgfqpoint{2.835472in}{1.489253in}}%
\pgfpathlineto{\pgfqpoint{2.883130in}{1.499199in}}%
\pgfpathlineto{\pgfqpoint{2.930789in}{1.510648in}}%
\pgfpathlineto{\pgfqpoint{2.978447in}{1.532795in}}%
\pgfpathlineto{\pgfqpoint{3.026105in}{1.479500in}}%
\pgfpathlineto{\pgfqpoint{3.073763in}{1.482255in}}%
\pgfpathlineto{\pgfqpoint{3.121421in}{1.495196in}}%
\pgfpathlineto{\pgfqpoint{3.169080in}{1.515243in}}%
\pgfpathlineto{\pgfqpoint{3.216738in}{1.533907in}}%
\pgfpathlineto{\pgfqpoint{3.264396in}{1.543197in}}%
\pgfpathlineto{\pgfqpoint{3.312054in}{1.490645in}}%
\pgfpathlineto{\pgfqpoint{3.359712in}{1.504465in}}%
\pgfpathlineto{\pgfqpoint{3.407371in}{1.524092in}}%
\pgfpathlineto{\pgfqpoint{3.455029in}{1.533901in}}%
\pgfpathlineto{\pgfqpoint{3.502687in}{1.541669in}}%
\pgfpathlineto{\pgfqpoint{3.550345in}{1.554362in}}%
\pgfpathlineto{\pgfqpoint{3.598004in}{1.515049in}}%
\pgfpathlineto{\pgfqpoint{3.645662in}{1.521370in}}%
\pgfpathlineto{\pgfqpoint{3.693320in}{1.537753in}}%
\pgfpathlineto{\pgfqpoint{3.740978in}{1.551356in}}%
\pgfpathlineto{\pgfqpoint{3.788636in}{1.557505in}}%
\pgfpathlineto{\pgfqpoint{3.836295in}{1.567117in}}%
\pgfpathlineto{\pgfqpoint{3.883953in}{1.542537in}}%
\pgfpathlineto{\pgfqpoint{3.931611in}{1.540729in}}%
\pgfpathlineto{\pgfqpoint{3.979269in}{1.535950in}}%
\pgfpathlineto{\pgfqpoint{4.026927in}{1.564315in}}%
\pgfpathlineto{\pgfqpoint{4.074586in}{1.573374in}}%
\pgfpathlineto{\pgfqpoint{4.122244in}{1.576187in}}%
\pgfpathlineto{\pgfqpoint{4.169902in}{1.546047in}}%
\pgfpathlineto{\pgfqpoint{4.217560in}{1.546122in}}%
\pgfpathlineto{\pgfqpoint{4.265219in}{1.556417in}}%
\pgfpathlineto{\pgfqpoint{4.312877in}{1.567280in}}%
\pgfpathlineto{\pgfqpoint{4.408193in}{1.589229in}}%
\pgfpathlineto{\pgfqpoint{4.455851in}{1.559932in}}%
\pgfpathlineto{\pgfqpoint{4.503510in}{1.558908in}}%
\pgfpathlineto{\pgfqpoint{4.598826in}{1.579093in}}%
\pgfpathlineto{\pgfqpoint{4.646484in}{1.585515in}}%
\pgfpathlineto{\pgfqpoint{4.694142in}{1.596839in}}%
\pgfpathlineto{\pgfqpoint{4.741801in}{1.570037in}}%
\pgfpathlineto{\pgfqpoint{4.789459in}{1.574713in}}%
\pgfpathlineto{\pgfqpoint{4.837117in}{1.569034in}}%
\pgfpathlineto{\pgfqpoint{4.884775in}{1.591895in}}%
\pgfpathlineto{\pgfqpoint{4.980092in}{1.606043in}}%
\pgfpathlineto{\pgfqpoint{5.027750in}{1.585188in}}%
\pgfpathlineto{\pgfqpoint{5.075408in}{1.578568in}}%
\pgfpathlineto{\pgfqpoint{5.123066in}{1.589692in}}%
\pgfpathlineto{\pgfqpoint{5.170725in}{1.602482in}}%
\pgfpathlineto{\pgfqpoint{5.218383in}{1.607978in}}%
\pgfpathlineto{\pgfqpoint{5.266041in}{1.615876in}}%
\pgfpathlineto{\pgfqpoint{5.361357in}{1.595669in}}%
\pgfpathlineto{\pgfqpoint{5.409016in}{1.593656in}}%
\pgfpathlineto{\pgfqpoint{5.456674in}{1.607130in}}%
\pgfpathlineto{\pgfqpoint{5.551990in}{1.622593in}}%
\pgfpathlineto{\pgfqpoint{5.647307in}{1.600343in}}%
\pgfpathlineto{\pgfqpoint{5.694965in}{1.609535in}}%
\pgfpathlineto{\pgfqpoint{5.742623in}{1.611414in}}%
\pgfusepath{stroke}%
\end{pgfscope}%
\begin{pgfscope}%
\pgfpathrectangle{\pgfqpoint{0.588387in}{0.521603in}}{\pgfqpoint{5.399676in}{2.010285in}}%
\pgfusepath{clip}%
\pgfsetrectcap%
\pgfsetroundjoin%
\pgfsetlinewidth{1.505625pt}%
\pgfsetstrokecolor{currentstroke5}%
\pgfsetdash{}{0pt}%
\pgfpathmoveto{\pgfqpoint{0.833827in}{0.720291in}}%
\pgfpathlineto{\pgfqpoint{0.881485in}{0.725207in}}%
\pgfpathlineto{\pgfqpoint{0.929143in}{0.789171in}}%
\pgfpathlineto{\pgfqpoint{0.976802in}{0.823872in}}%
\pgfpathlineto{\pgfqpoint{1.024460in}{0.874667in}}%
\pgfpathlineto{\pgfqpoint{1.072118in}{0.927361in}}%
\pgfpathlineto{\pgfqpoint{1.119776in}{0.956938in}}%
\pgfpathlineto{\pgfqpoint{1.167435in}{1.028189in}}%
\pgfpathlineto{\pgfqpoint{1.215093in}{1.062693in}}%
\pgfpathlineto{\pgfqpoint{1.262751in}{1.129842in}}%
\pgfpathlineto{\pgfqpoint{1.310409in}{1.234322in}}%
\pgfpathlineto{\pgfqpoint{1.358067in}{1.274582in}}%
\pgfpathlineto{\pgfqpoint{1.405726in}{1.308839in}}%
\pgfpathlineto{\pgfqpoint{1.453384in}{1.353929in}}%
\pgfpathlineto{\pgfqpoint{1.501042in}{1.386164in}}%
\pgfpathlineto{\pgfqpoint{1.548700in}{1.426106in}}%
\pgfpathlineto{\pgfqpoint{1.596358in}{1.319356in}}%
\pgfpathlineto{\pgfqpoint{1.644017in}{1.338242in}}%
\pgfpathlineto{\pgfqpoint{1.691675in}{1.367813in}}%
\pgfpathlineto{\pgfqpoint{1.739333in}{1.423146in}}%
\pgfpathlineto{\pgfqpoint{1.786991in}{1.425313in}}%
\pgfpathlineto{\pgfqpoint{1.834650in}{1.473524in}}%
\pgfpathlineto{\pgfqpoint{1.882308in}{1.381384in}}%
\pgfpathlineto{\pgfqpoint{1.929966in}{1.405758in}}%
\pgfpathlineto{\pgfqpoint{1.977624in}{1.414940in}}%
\pgfpathlineto{\pgfqpoint{2.025282in}{1.442491in}}%
\pgfpathlineto{\pgfqpoint{2.072941in}{1.470618in}}%
\pgfpathlineto{\pgfqpoint{2.120599in}{1.480013in}}%
\pgfpathlineto{\pgfqpoint{2.168257in}{1.425990in}}%
\pgfpathlineto{\pgfqpoint{2.215915in}{1.420329in}}%
\pgfpathlineto{\pgfqpoint{2.263573in}{1.444030in}}%
\pgfpathlineto{\pgfqpoint{2.311232in}{1.483856in}}%
\pgfpathlineto{\pgfqpoint{2.358890in}{1.479357in}}%
\pgfpathlineto{\pgfqpoint{2.406548in}{1.521104in}}%
\pgfpathlineto{\pgfqpoint{2.454206in}{1.451161in}}%
\pgfpathlineto{\pgfqpoint{2.501865in}{1.459815in}}%
\pgfpathlineto{\pgfqpoint{2.549523in}{1.472236in}}%
\pgfpathlineto{\pgfqpoint{2.597181in}{1.485975in}}%
\pgfpathlineto{\pgfqpoint{2.644839in}{1.500281in}}%
\pgfpathlineto{\pgfqpoint{2.692497in}{1.526544in}}%
\pgfpathlineto{\pgfqpoint{2.740156in}{1.476290in}}%
\pgfpathlineto{\pgfqpoint{2.787814in}{1.497995in}}%
\pgfpathlineto{\pgfqpoint{2.835472in}{1.487814in}}%
\pgfpathlineto{\pgfqpoint{2.883130in}{1.515040in}}%
\pgfpathlineto{\pgfqpoint{2.930789in}{1.529202in}}%
\pgfpathlineto{\pgfqpoint{2.978447in}{1.538378in}}%
\pgfpathlineto{\pgfqpoint{3.026105in}{1.489330in}}%
\pgfpathlineto{\pgfqpoint{3.073763in}{1.502375in}}%
\pgfpathlineto{\pgfqpoint{3.121421in}{1.514818in}}%
\pgfpathlineto{\pgfqpoint{3.169080in}{1.526835in}}%
\pgfpathlineto{\pgfqpoint{3.216738in}{1.534303in}}%
\pgfpathlineto{\pgfqpoint{3.264396in}{1.550636in}}%
\pgfpathlineto{\pgfqpoint{3.312054in}{1.519538in}}%
\pgfpathlineto{\pgfqpoint{3.359712in}{1.516588in}}%
\pgfpathlineto{\pgfqpoint{3.407371in}{1.507942in}}%
\pgfpathlineto{\pgfqpoint{3.455029in}{1.546913in}}%
\pgfpathlineto{\pgfqpoint{3.502687in}{1.560625in}}%
\pgfpathlineto{\pgfqpoint{3.550345in}{1.565990in}}%
\pgfpathlineto{\pgfqpoint{3.598004in}{1.523877in}}%
\pgfpathlineto{\pgfqpoint{3.645662in}{1.537521in}}%
\pgfpathlineto{\pgfqpoint{3.693320in}{1.562539in}}%
\pgfpathlineto{\pgfqpoint{3.740978in}{1.555204in}}%
\pgfpathlineto{\pgfqpoint{3.788636in}{1.542297in}}%
\pgfpathlineto{\pgfqpoint{3.836295in}{1.580029in}}%
\pgfpathlineto{\pgfqpoint{3.883953in}{1.533842in}}%
\pgfpathlineto{\pgfqpoint{3.931611in}{1.541145in}}%
\pgfpathlineto{\pgfqpoint{3.979269in}{1.562539in}}%
\pgfpathlineto{\pgfqpoint{4.026927in}{1.564421in}}%
\pgfpathlineto{\pgfqpoint{4.074586in}{1.571351in}}%
\pgfpathlineto{\pgfqpoint{4.122244in}{1.592362in}}%
\pgfpathlineto{\pgfqpoint{4.169902in}{1.549221in}}%
\pgfpathlineto{\pgfqpoint{4.217560in}{1.560942in}}%
\pgfpathlineto{\pgfqpoint{4.265219in}{1.578102in}}%
\pgfpathlineto{\pgfqpoint{4.312877in}{1.582124in}}%
\pgfpathlineto{\pgfqpoint{4.408193in}{1.607088in}}%
\pgfpathlineto{\pgfqpoint{4.455851in}{1.555392in}}%
\pgfpathlineto{\pgfqpoint{4.503510in}{1.573112in}}%
\pgfpathlineto{\pgfqpoint{4.598826in}{1.589432in}}%
\pgfpathlineto{\pgfqpoint{4.646484in}{1.587885in}}%
\pgfpathlineto{\pgfqpoint{4.694142in}{1.604128in}}%
\pgfpathlineto{\pgfqpoint{4.741801in}{1.579538in}}%
\pgfpathlineto{\pgfqpoint{4.789459in}{1.583616in}}%
\pgfpathlineto{\pgfqpoint{4.837117in}{1.578178in}}%
\pgfpathlineto{\pgfqpoint{4.884775in}{1.596750in}}%
\pgfpathlineto{\pgfqpoint{4.980092in}{1.616410in}}%
\pgfpathlineto{\pgfqpoint{5.027750in}{1.600277in}}%
\pgfpathlineto{\pgfqpoint{5.075408in}{1.599684in}}%
\pgfpathlineto{\pgfqpoint{5.123066in}{1.583987in}}%
\pgfpathlineto{\pgfqpoint{5.170725in}{1.661441in}}%
\pgfpathlineto{\pgfqpoint{5.218383in}{1.624752in}}%
\pgfpathlineto{\pgfqpoint{5.266041in}{1.620140in}}%
\pgfpathlineto{\pgfqpoint{5.361357in}{1.599551in}}%
\pgfpathlineto{\pgfqpoint{5.409016in}{1.599947in}}%
\pgfpathlineto{\pgfqpoint{5.456674in}{1.615969in}}%
\pgfpathlineto{\pgfqpoint{5.551990in}{1.628635in}}%
\pgfpathlineto{\pgfqpoint{5.647307in}{1.629836in}}%
\pgfpathlineto{\pgfqpoint{5.694965in}{1.618094in}}%
\pgfpathlineto{\pgfqpoint{5.742623in}{1.623624in}}%
\pgfusepath{stroke}%
\end{pgfscope}%
\begin{pgfscope}%
\pgfpathrectangle{\pgfqpoint{0.588387in}{0.521603in}}{\pgfqpoint{5.399676in}{2.010285in}}%
\pgfusepath{clip}%
\pgfsetrectcap%
\pgfsetroundjoin%
\pgfsetlinewidth{1.505625pt}%
\pgfsetstrokecolor{currentstroke6}%
\pgfsetdash{}{0pt}%
\pgfpathmoveto{\pgfqpoint{0.833827in}{0.720291in}}%
\pgfpathlineto{\pgfqpoint{0.881485in}{0.726134in}}%
\pgfpathlineto{\pgfqpoint{0.929143in}{0.793056in}}%
\pgfpathlineto{\pgfqpoint{0.976802in}{0.823242in}}%
\pgfpathlineto{\pgfqpoint{1.024460in}{0.874438in}}%
\pgfpathlineto{\pgfqpoint{1.072118in}{0.930711in}}%
\pgfpathlineto{\pgfqpoint{1.119776in}{0.938155in}}%
\pgfpathlineto{\pgfqpoint{1.167435in}{1.014316in}}%
\pgfpathlineto{\pgfqpoint{1.215093in}{1.041933in}}%
\pgfpathlineto{\pgfqpoint{1.262751in}{1.122291in}}%
\pgfpathlineto{\pgfqpoint{1.310409in}{1.241856in}}%
\pgfpathlineto{\pgfqpoint{1.358067in}{1.280969in}}%
\pgfpathlineto{\pgfqpoint{1.405726in}{1.316865in}}%
\pgfpathlineto{\pgfqpoint{1.453384in}{1.355062in}}%
\pgfpathlineto{\pgfqpoint{1.501042in}{1.399742in}}%
\pgfpathlineto{\pgfqpoint{1.548700in}{1.435111in}}%
\pgfpathlineto{\pgfqpoint{1.596358in}{1.589329in}}%
\pgfpathlineto{\pgfqpoint{1.644017in}{1.540301in}}%
\pgfpathlineto{\pgfqpoint{1.691675in}{1.687115in}}%
\pgfpathlineto{\pgfqpoint{1.739333in}{1.808361in}}%
\pgfpathlineto{\pgfqpoint{1.786991in}{1.773444in}}%
\pgfpathlineto{\pgfqpoint{1.834650in}{1.875195in}}%
\pgfpathlineto{\pgfqpoint{1.882308in}{1.987574in}}%
\pgfpathlineto{\pgfqpoint{1.929966in}{1.953743in}}%
\pgfpathlineto{\pgfqpoint{1.977624in}{2.091003in}}%
\pgfpathlineto{\pgfqpoint{2.025282in}{2.262322in}}%
\pgfpathlineto{\pgfqpoint{2.072941in}{2.344423in}}%
\pgfpathlineto{\pgfqpoint{2.120599in}{2.299269in}}%
\pgfpathlineto{\pgfqpoint{2.168257in}{2.271708in}}%
\pgfpathlineto{\pgfqpoint{2.215915in}{2.153299in}}%
\pgfpathlineto{\pgfqpoint{2.263573in}{2.139371in}}%
\pgfpathlineto{\pgfqpoint{2.311232in}{2.200946in}}%
\pgfpathlineto{\pgfqpoint{2.358890in}{2.360401in}}%
\pgfpathlineto{\pgfqpoint{2.406548in}{2.236505in}}%
\pgfpathlineto{\pgfqpoint{2.454206in}{2.134104in}}%
\pgfpathlineto{\pgfqpoint{2.501865in}{2.053719in}}%
\pgfpathlineto{\pgfqpoint{2.549523in}{1.978373in}}%
\pgfpathlineto{\pgfqpoint{2.597181in}{2.111968in}}%
\pgfpathlineto{\pgfqpoint{2.644839in}{2.283820in}}%
\pgfpathlineto{\pgfqpoint{2.692497in}{2.227765in}}%
\pgfpathlineto{\pgfqpoint{2.740156in}{1.633860in}}%
\pgfpathlineto{\pgfqpoint{2.787814in}{2.281551in}}%
\pgfpathlineto{\pgfqpoint{2.835472in}{1.673529in}}%
\pgfpathlineto{\pgfqpoint{2.883130in}{1.824914in}}%
\pgfpathlineto{\pgfqpoint{2.930789in}{1.944807in}}%
\pgfpathlineto{\pgfqpoint{2.978447in}{1.809604in}}%
\pgfpathlineto{\pgfqpoint{3.026105in}{2.060805in}}%
\pgfpathlineto{\pgfqpoint{3.073763in}{1.945849in}}%
\pgfpathlineto{\pgfqpoint{3.121421in}{1.760109in}}%
\pgfpathlineto{\pgfqpoint{3.169080in}{1.947570in}}%
\pgfpathlineto{\pgfqpoint{3.216738in}{1.857001in}}%
\pgfpathlineto{\pgfqpoint{3.264396in}{1.892187in}}%
\pgfpathlineto{\pgfqpoint{3.312054in}{1.722266in}}%
\pgfpathlineto{\pgfqpoint{3.359712in}{1.673551in}}%
\pgfpathlineto{\pgfqpoint{3.407371in}{1.871567in}}%
\pgfpathlineto{\pgfqpoint{3.455029in}{1.801440in}}%
\pgfpathlineto{\pgfqpoint{3.502687in}{1.841974in}}%
\pgfpathlineto{\pgfqpoint{3.550345in}{1.769195in}}%
\pgfpathlineto{\pgfqpoint{3.598004in}{1.573869in}}%
\pgfpathlineto{\pgfqpoint{3.645662in}{2.098359in}}%
\pgfpathlineto{\pgfqpoint{3.693320in}{1.820964in}}%
\pgfpathlineto{\pgfqpoint{3.740978in}{1.836111in}}%
\pgfpathlineto{\pgfqpoint{3.836295in}{1.918089in}}%
\pgfpathlineto{\pgfqpoint{3.883953in}{1.523607in}}%
\pgfpathlineto{\pgfqpoint{3.931611in}{2.174070in}}%
\pgfpathlineto{\pgfqpoint{4.026927in}{1.904919in}}%
\pgfpathlineto{\pgfqpoint{4.074586in}{2.015986in}}%
\pgfpathlineto{\pgfqpoint{4.122244in}{1.684194in}}%
\pgfpathlineto{\pgfqpoint{4.217560in}{1.712216in}}%
\pgfpathlineto{\pgfqpoint{4.265219in}{1.973446in}}%
\pgfpathlineto{\pgfqpoint{4.312877in}{1.827772in}}%
\pgfpathlineto{\pgfqpoint{4.408193in}{2.066205in}}%
\pgfpathlineto{\pgfqpoint{4.598826in}{1.776236in}}%
\pgfpathlineto{\pgfqpoint{4.646484in}{1.829518in}}%
\pgfpathlineto{\pgfqpoint{4.694142in}{1.866904in}}%
\pgfpathlineto{\pgfqpoint{4.741801in}{1.557071in}}%
\pgfpathlineto{\pgfqpoint{4.789459in}{1.824467in}}%
\pgfpathlineto{\pgfqpoint{4.884775in}{1.610892in}}%
\pgfpathlineto{\pgfqpoint{4.980092in}{1.593759in}}%
\pgfpathlineto{\pgfqpoint{5.075408in}{1.710263in}}%
\pgfpathlineto{\pgfqpoint{5.170725in}{1.977146in}}%
\pgfpathlineto{\pgfqpoint{5.266041in}{1.801363in}}%
\pgfpathlineto{\pgfqpoint{5.456674in}{2.440512in}}%
\pgfpathlineto{\pgfqpoint{5.551990in}{1.944518in}}%
\pgfusepath{stroke}%
\end{pgfscope}%
\begin{pgfscope}%
\pgfsetrectcap%
\pgfsetmiterjoin%
\pgfsetlinewidth{0.803000pt}%
\definecolor{currentstroke}{rgb}{0.000000,0.000000,0.000000}%
\pgfsetstrokecolor{currentstroke}%
\pgfsetdash{}{0pt}%
\pgfpathmoveto{\pgfqpoint{0.588387in}{0.521603in}}%
\pgfpathlineto{\pgfqpoint{0.588387in}{2.531888in}}%
\pgfusepath{stroke}%
\end{pgfscope}%
\begin{pgfscope}%
\pgfsetrectcap%
\pgfsetmiterjoin%
\pgfsetlinewidth{0.803000pt}%
\definecolor{currentstroke}{rgb}{0.000000,0.000000,0.000000}%
\pgfsetstrokecolor{currentstroke}%
\pgfsetdash{}{0pt}%
\pgfpathmoveto{\pgfqpoint{5.988063in}{0.521603in}}%
\pgfpathlineto{\pgfqpoint{5.988063in}{2.531888in}}%
\pgfusepath{stroke}%
\end{pgfscope}%
\begin{pgfscope}%
\pgfsetrectcap%
\pgfsetmiterjoin%
\pgfsetlinewidth{0.803000pt}%
\definecolor{currentstroke}{rgb}{0.000000,0.000000,0.000000}%
\pgfsetstrokecolor{currentstroke}%
\pgfsetdash{}{0pt}%
\pgfpathmoveto{\pgfqpoint{0.588387in}{0.521603in}}%
\pgfpathlineto{\pgfqpoint{5.988063in}{0.521603in}}%
\pgfusepath{stroke}%
\end{pgfscope}%
\begin{pgfscope}%
\pgfsetrectcap%
\pgfsetmiterjoin%
\pgfsetlinewidth{0.803000pt}%
\definecolor{currentstroke}{rgb}{0.000000,0.000000,0.000000}%
\pgfsetstrokecolor{currentstroke}%
\pgfsetdash{}{0pt}%
\pgfpathmoveto{\pgfqpoint{0.588387in}{2.531888in}}%
\pgfpathlineto{\pgfqpoint{5.988063in}{2.531888in}}%
\pgfusepath{stroke}%
\end{pgfscope}%
\begin{pgfscope}%
\definecolor{textcolor}{rgb}{0.000000,0.000000,0.000000}%
\pgfsetstrokecolor{textcolor}%
\pgfsetfillcolor{textcolor}%
\pgftext[x=3.288225in,y=2.615222in,,base]{\color{textcolor}{\rmfamily\fontsize{12.000000}{14.400000}\selectfont\catcode`\^=\active\def^{\ifmmode\sp\else\^{}\fi}\catcode`\%=\active\def%{\%}Mean}}%
\end{pgfscope}%
\begin{pgfscope}%
\pgfsetbuttcap%
\pgfsetmiterjoin%
\definecolor{currentfill}{rgb}{1.000000,1.000000,1.000000}%
\pgfsetfillcolor{currentfill}%
\pgfsetfillopacity{0.800000}%
\pgfsetlinewidth{1.003750pt}%
\definecolor{currentstroke}{rgb}{0.800000,0.800000,0.800000}%
\pgfsetstrokecolor{currentstroke}%
\pgfsetstrokeopacity{0.800000}%
\pgfsetdash{}{0pt}%
\pgfpathmoveto{\pgfqpoint{6.075563in}{1.320622in}}%
\pgfpathlineto{\pgfqpoint{8.259376in}{1.320622in}}%
\pgfpathquadraticcurveto{\pgfqpoint{8.284376in}{1.320622in}}{\pgfqpoint{8.284376in}{1.345622in}}%
\pgfpathlineto{\pgfqpoint{8.284376in}{2.444388in}}%
\pgfpathquadraticcurveto{\pgfqpoint{8.284376in}{2.469388in}}{\pgfqpoint{8.259376in}{2.469388in}}%
\pgfpathlineto{\pgfqpoint{6.075563in}{2.469388in}}%
\pgfpathquadraticcurveto{\pgfqpoint{6.050563in}{2.469388in}}{\pgfqpoint{6.050563in}{2.444388in}}%
\pgfpathlineto{\pgfqpoint{6.050563in}{1.345622in}}%
\pgfpathquadraticcurveto{\pgfqpoint{6.050563in}{1.320622in}}{\pgfqpoint{6.075563in}{1.320622in}}%
\pgfpathlineto{\pgfqpoint{6.075563in}{1.320622in}}%
\pgfpathclose%
\pgfusepath{stroke,fill}%
\end{pgfscope}%
\begin{pgfscope}%
\pgfsetrectcap%
\pgfsetroundjoin%
\pgfsetlinewidth{1.505625pt}%
\pgfsetstrokecolor{currentstroke1}%
\pgfsetdash{}{0pt}%
\pgfpathmoveto{\pgfqpoint{6.100563in}{2.368168in}}%
\pgfpathlineto{\pgfqpoint{6.225563in}{2.368168in}}%
\pgfpathlineto{\pgfqpoint{6.350563in}{2.368168in}}%
\pgfusepath{stroke}%
\end{pgfscope}%
\begin{pgfscope}%
\definecolor{textcolor}{rgb}{0.000000,0.000000,0.000000}%
\pgfsetstrokecolor{textcolor}%
\pgfsetfillcolor{textcolor}%
\pgftext[x=6.450563in,y=2.324418in,left,base]{\color{textcolor}{\rmfamily\fontsize{9.000000}{10.800000}\selectfont\catcode`\^=\active\def^{\ifmmode\sp\else\^{}\fi}\catcode`\%=\active\def%{\%}\NaiveCycles{}}}%
\end{pgfscope}%
\begin{pgfscope}%
\pgfsetrectcap%
\pgfsetroundjoin%
\pgfsetlinewidth{1.505625pt}%
\pgfsetstrokecolor{currentstroke2}%
\pgfsetdash{}{0pt}%
\pgfpathmoveto{\pgfqpoint{6.100563in}{2.184696in}}%
\pgfpathlineto{\pgfqpoint{6.225563in}{2.184696in}}%
\pgfpathlineto{\pgfqpoint{6.350563in}{2.184696in}}%
\pgfusepath{stroke}%
\end{pgfscope}%
\begin{pgfscope}%
\definecolor{textcolor}{rgb}{0.000000,0.000000,0.000000}%
\pgfsetstrokecolor{textcolor}%
\pgfsetfillcolor{textcolor}%
\pgftext[x=6.450563in,y=2.140946in,left,base]{\color{textcolor}{\rmfamily\fontsize{9.000000}{10.800000}\selectfont\catcode`\^=\active\def^{\ifmmode\sp\else\^{}\fi}\catcode`\%=\active\def%{\%}\Neighbors{} \& \MergeLinear{}}}%
\end{pgfscope}%
\begin{pgfscope}%
\pgfsetrectcap%
\pgfsetroundjoin%
\pgfsetlinewidth{1.505625pt}%
\pgfsetstrokecolor{currentstroke3}%
\pgfsetdash{}{0pt}%
\pgfpathmoveto{\pgfqpoint{6.100563in}{2.001225in}}%
\pgfpathlineto{\pgfqpoint{6.225563in}{2.001225in}}%
\pgfpathlineto{\pgfqpoint{6.350563in}{2.001225in}}%
\pgfusepath{stroke}%
\end{pgfscope}%
\begin{pgfscope}%
\definecolor{textcolor}{rgb}{0.000000,0.000000,0.000000}%
\pgfsetstrokecolor{textcolor}%
\pgfsetfillcolor{textcolor}%
\pgftext[x=6.450563in,y=1.957475in,left,base]{\color{textcolor}{\rmfamily\fontsize{9.000000}{10.800000}\selectfont\catcode`\^=\active\def^{\ifmmode\sp\else\^{}\fi}\catcode`\%=\active\def%{\%}\Neighbors{} \& \SharedVertices{}}}%
\end{pgfscope}%
\begin{pgfscope}%
\pgfsetrectcap%
\pgfsetroundjoin%
\pgfsetlinewidth{1.505625pt}%
\pgfsetstrokecolor{currentstroke4}%
\pgfsetdash{}{0pt}%
\pgfpathmoveto{\pgfqpoint{6.100563in}{1.814274in}}%
\pgfpathlineto{\pgfqpoint{6.225563in}{1.814274in}}%
\pgfpathlineto{\pgfqpoint{6.350563in}{1.814274in}}%
\pgfusepath{stroke}%
\end{pgfscope}%
\begin{pgfscope}%
\definecolor{textcolor}{rgb}{0.000000,0.000000,0.000000}%
\pgfsetstrokecolor{textcolor}%
\pgfsetfillcolor{textcolor}%
\pgftext[x=6.450563in,y=1.770524in,left,base]{\color{textcolor}{\rmfamily\fontsize{9.000000}{10.800000}\selectfont\catcode`\^=\active\def^{\ifmmode\sp\else\^{}\fi}\catcode`\%=\active\def%{\%}\NeighborsDegree{} \& \MergeLinear{}}}%
\end{pgfscope}%
\begin{pgfscope}%
\pgfsetrectcap%
\pgfsetroundjoin%
\pgfsetlinewidth{1.505625pt}%
\pgfsetstrokecolor{currentstroke5}%
\pgfsetdash{}{0pt}%
\pgfpathmoveto{\pgfqpoint{6.100563in}{1.627324in}}%
\pgfpathlineto{\pgfqpoint{6.225563in}{1.627324in}}%
\pgfpathlineto{\pgfqpoint{6.350563in}{1.627324in}}%
\pgfusepath{stroke}%
\end{pgfscope}%
\begin{pgfscope}%
\definecolor{textcolor}{rgb}{0.000000,0.000000,0.000000}%
\pgfsetstrokecolor{textcolor}%
\pgfsetfillcolor{textcolor}%
\pgftext[x=6.450563in,y=1.583574in,left,base]{\color{textcolor}{\rmfamily\fontsize{9.000000}{10.800000}\selectfont\catcode`\^=\active\def^{\ifmmode\sp\else\^{}\fi}\catcode`\%=\active\def%{\%}\None{} \& \MergeLinear{}}}%
\end{pgfscope}%
\begin{pgfscope}%
\pgfsetrectcap%
\pgfsetroundjoin%
\pgfsetlinewidth{1.505625pt}%
\pgfsetstrokecolor{currentstroke6}%
\pgfsetdash{}{0pt}%
\pgfpathmoveto{\pgfqpoint{6.100563in}{1.443852in}}%
\pgfpathlineto{\pgfqpoint{6.225563in}{1.443852in}}%
\pgfpathlineto{\pgfqpoint{6.350563in}{1.443852in}}%
\pgfusepath{stroke}%
\end{pgfscope}%
\begin{pgfscope}%
\definecolor{textcolor}{rgb}{0.000000,0.000000,0.000000}%
\pgfsetstrokecolor{textcolor}%
\pgfsetfillcolor{textcolor}%
\pgftext[x=6.450563in,y=1.400102in,left,base]{\color{textcolor}{\rmfamily\fontsize{9.000000}{10.800000}\selectfont\catcode`\^=\active\def^{\ifmmode\sp\else\^{}\fi}\catcode`\%=\active\def%{\%}\None{} \& \SharedVertices{}}}%
\end{pgfscope}%
\end{pgfpicture}%
\makeatother%
\endgroup%
}
	\caption[Checks performed for minimally rigid graphs (some).]{
		The number of checks performed to find all NAC-colorings for minimally rigid graphs.}%
	\label{fig:graph_minimally_rigid_first_checks}
\end{figure}


% No 3 nor 4 cycles
From~\cite{extremal_graphs} we obtained all graphs with up to 52 vertices
that have no three nor four cycles. This class of graphs is interesting for us
as \trcon{} components and monochromatic classes optimizations cannot be used.
These graphs have many NAC-colorings.
Therefore, as seen in \Cref{fig:graph_count_no_3_nor_4_cycles_first_runtime},
\NaiveCycles{} are again faster for finding some NAC-coloring
for the similar reasons as for minimally rigid graphs.
For listing all NAC-colorings shown in \Cref{fig:graph_count_no_3_nor_4_cycles_all_runtime},
\Subgraphs{} are once again significantly faster.
It can be seen in \Cref{fig:graph_count_no_3_nor_4_cycles_all_checks}
that \Neighbors{} and \CyclesMatchChunks{} strategies are not faster than \None{},
but the number of \IsNACColoring{} check calls is reduced.
The difference grows for larger graphs, therefore it can be expected
that \Neighbors{} and \CyclesMatchChunks{} will eventually become faster
for larger graphs then \None{}.

\begin{figure}[p]
	\centering
	\scalebox{0.5}{%% Creator: Matplotlib, PGF backend
%%
%% To include the figure in your LaTeX document, write
%%   \input{<filename>.pgf}
%%
%% Make sure the required packages are loaded in your preamble
%%   \usepackage{pgf}
%%
%% Also ensure that all the required font packages are loaded; for instance,
%% the lmodern package is sometimes necessary when using math font.
%%   \usepackage{lmodern}
%%
%% Figures using additional raster images can only be included by \input if
%% they are in the same directory as the main LaTeX file. For loading figures
%% from other directories you can use the `import` package
%%   \usepackage{import}
%%
%% and then include the figures with
%%   \import{<path to file>}{<filename>.pgf}
%%
%% Matplotlib used the following preamble
%%   \def\mathdefault#1{#1}
%%   \everymath=\expandafter{\the\everymath\displaystyle}
%%   \IfFileExists{scrextend.sty}{
%%     \usepackage[fontsize=10.000000pt]{scrextend}
%%   }{
%%     \renewcommand{\normalsize}{\fontsize{10.000000}{12.000000}\selectfont}
%%     \normalsize
%%   }
%%   
%%   \ifdefined\pdftexversion\else  % non-pdftex case.
%%     \usepackage{fontspec}
%%     \setmainfont{DejaVuSans.ttf}[Path=\detokenize{/home/petr/Projects/PyRigi/.venv/lib/python3.12/site-packages/matplotlib/mpl-data/fonts/ttf/}]
%%     \setsansfont{DejaVuSans.ttf}[Path=\detokenize{/home/petr/Projects/PyRigi/.venv/lib/python3.12/site-packages/matplotlib/mpl-data/fonts/ttf/}]
%%     \setmonofont{DejaVuSansMono.ttf}[Path=\detokenize{/home/petr/Projects/PyRigi/.venv/lib/python3.12/site-packages/matplotlib/mpl-data/fonts/ttf/}]
%%   \fi
%%   \makeatletter\@ifpackageloaded{underscore}{}{\usepackage[strings]{underscore}}\makeatother
%%
\begingroup%
\makeatletter%
\begin{pgfpicture}%
\pgfpathrectangle{\pgfpointorigin}{\pgfqpoint{8.384376in}{2.841849in}}%
\pgfusepath{use as bounding box, clip}%
\begin{pgfscope}%
\pgfsetbuttcap%
\pgfsetmiterjoin%
\definecolor{currentfill}{rgb}{1.000000,1.000000,1.000000}%
\pgfsetfillcolor{currentfill}%
\pgfsetlinewidth{0.000000pt}%
\definecolor{currentstroke}{rgb}{1.000000,1.000000,1.000000}%
\pgfsetstrokecolor{currentstroke}%
\pgfsetdash{}{0pt}%
\pgfpathmoveto{\pgfqpoint{0.000000in}{0.000000in}}%
\pgfpathlineto{\pgfqpoint{8.384376in}{0.000000in}}%
\pgfpathlineto{\pgfqpoint{8.384376in}{2.841849in}}%
\pgfpathlineto{\pgfqpoint{0.000000in}{2.841849in}}%
\pgfpathlineto{\pgfqpoint{0.000000in}{0.000000in}}%
\pgfpathclose%
\pgfusepath{fill}%
\end{pgfscope}%
\begin{pgfscope}%
\pgfsetbuttcap%
\pgfsetmiterjoin%
\definecolor{currentfill}{rgb}{1.000000,1.000000,1.000000}%
\pgfsetfillcolor{currentfill}%
\pgfsetlinewidth{0.000000pt}%
\definecolor{currentstroke}{rgb}{0.000000,0.000000,0.000000}%
\pgfsetstrokecolor{currentstroke}%
\pgfsetstrokeopacity{0.000000}%
\pgfsetdash{}{0pt}%
\pgfpathmoveto{\pgfqpoint{0.588387in}{0.521603in}}%
\pgfpathlineto{\pgfqpoint{5.257411in}{0.521603in}}%
\pgfpathlineto{\pgfqpoint{5.257411in}{2.531888in}}%
\pgfpathlineto{\pgfqpoint{0.588387in}{2.531888in}}%
\pgfpathlineto{\pgfqpoint{0.588387in}{0.521603in}}%
\pgfpathclose%
\pgfusepath{fill}%
\end{pgfscope}%
\begin{pgfscope}%
\pgfsetbuttcap%
\pgfsetroundjoin%
\definecolor{currentfill}{rgb}{0.000000,0.000000,0.000000}%
\pgfsetfillcolor{currentfill}%
\pgfsetlinewidth{0.803000pt}%
\definecolor{currentstroke}{rgb}{0.000000,0.000000,0.000000}%
\pgfsetstrokecolor{currentstroke}%
\pgfsetdash{}{0pt}%
\pgfsys@defobject{currentmarker}{\pgfqpoint{0.000000in}{-0.048611in}}{\pgfqpoint{0.000000in}{0.000000in}}{%
\pgfpathmoveto{\pgfqpoint{0.000000in}{0.000000in}}%
\pgfpathlineto{\pgfqpoint{0.000000in}{-0.048611in}}%
\pgfusepath{stroke,fill}%
}%
\begin{pgfscope}%
\pgfsys@transformshift{0.677940in}{0.521603in}%
\pgfsys@useobject{currentmarker}{}%
\end{pgfscope}%
\end{pgfscope}%
\begin{pgfscope}%
\definecolor{textcolor}{rgb}{0.000000,0.000000,0.000000}%
\pgfsetstrokecolor{textcolor}%
\pgfsetfillcolor{textcolor}%
\pgftext[x=0.677940in,y=0.424381in,,top]{\color{textcolor}{\rmfamily\fontsize{10.000000}{12.000000}\selectfont\catcode`\^=\active\def^{\ifmmode\sp\else\^{}\fi}\catcode`\%=\active\def%{\%}$\mathdefault{0}$}}%
\end{pgfscope}%
\begin{pgfscope}%
\pgfsetbuttcap%
\pgfsetroundjoin%
\definecolor{currentfill}{rgb}{0.000000,0.000000,0.000000}%
\pgfsetfillcolor{currentfill}%
\pgfsetlinewidth{0.803000pt}%
\definecolor{currentstroke}{rgb}{0.000000,0.000000,0.000000}%
\pgfsetstrokecolor{currentstroke}%
\pgfsetdash{}{0pt}%
\pgfsys@defobject{currentmarker}{\pgfqpoint{0.000000in}{-0.048611in}}{\pgfqpoint{0.000000in}{0.000000in}}{%
\pgfpathmoveto{\pgfqpoint{0.000000in}{0.000000in}}%
\pgfpathlineto{\pgfqpoint{0.000000in}{-0.048611in}}%
\pgfusepath{stroke,fill}%
}%
\begin{pgfscope}%
\pgfsys@transformshift{1.168642in}{0.521603in}%
\pgfsys@useobject{currentmarker}{}%
\end{pgfscope}%
\end{pgfscope}%
\begin{pgfscope}%
\definecolor{textcolor}{rgb}{0.000000,0.000000,0.000000}%
\pgfsetstrokecolor{textcolor}%
\pgfsetfillcolor{textcolor}%
\pgftext[x=1.168642in,y=0.424381in,,top]{\color{textcolor}{\rmfamily\fontsize{10.000000}{12.000000}\selectfont\catcode`\^=\active\def^{\ifmmode\sp\else\^{}\fi}\catcode`\%=\active\def%{\%}$\mathdefault{20}$}}%
\end{pgfscope}%
\begin{pgfscope}%
\pgfsetbuttcap%
\pgfsetroundjoin%
\definecolor{currentfill}{rgb}{0.000000,0.000000,0.000000}%
\pgfsetfillcolor{currentfill}%
\pgfsetlinewidth{0.803000pt}%
\definecolor{currentstroke}{rgb}{0.000000,0.000000,0.000000}%
\pgfsetstrokecolor{currentstroke}%
\pgfsetdash{}{0pt}%
\pgfsys@defobject{currentmarker}{\pgfqpoint{0.000000in}{-0.048611in}}{\pgfqpoint{0.000000in}{0.000000in}}{%
\pgfpathmoveto{\pgfqpoint{0.000000in}{0.000000in}}%
\pgfpathlineto{\pgfqpoint{0.000000in}{-0.048611in}}%
\pgfusepath{stroke,fill}%
}%
\begin{pgfscope}%
\pgfsys@transformshift{1.659343in}{0.521603in}%
\pgfsys@useobject{currentmarker}{}%
\end{pgfscope}%
\end{pgfscope}%
\begin{pgfscope}%
\definecolor{textcolor}{rgb}{0.000000,0.000000,0.000000}%
\pgfsetstrokecolor{textcolor}%
\pgfsetfillcolor{textcolor}%
\pgftext[x=1.659343in,y=0.424381in,,top]{\color{textcolor}{\rmfamily\fontsize{10.000000}{12.000000}\selectfont\catcode`\^=\active\def^{\ifmmode\sp\else\^{}\fi}\catcode`\%=\active\def%{\%}$\mathdefault{40}$}}%
\end{pgfscope}%
\begin{pgfscope}%
\pgfsetbuttcap%
\pgfsetroundjoin%
\definecolor{currentfill}{rgb}{0.000000,0.000000,0.000000}%
\pgfsetfillcolor{currentfill}%
\pgfsetlinewidth{0.803000pt}%
\definecolor{currentstroke}{rgb}{0.000000,0.000000,0.000000}%
\pgfsetstrokecolor{currentstroke}%
\pgfsetdash{}{0pt}%
\pgfsys@defobject{currentmarker}{\pgfqpoint{0.000000in}{-0.048611in}}{\pgfqpoint{0.000000in}{0.000000in}}{%
\pgfpathmoveto{\pgfqpoint{0.000000in}{0.000000in}}%
\pgfpathlineto{\pgfqpoint{0.000000in}{-0.048611in}}%
\pgfusepath{stroke,fill}%
}%
\begin{pgfscope}%
\pgfsys@transformshift{2.150044in}{0.521603in}%
\pgfsys@useobject{currentmarker}{}%
\end{pgfscope}%
\end{pgfscope}%
\begin{pgfscope}%
\definecolor{textcolor}{rgb}{0.000000,0.000000,0.000000}%
\pgfsetstrokecolor{textcolor}%
\pgfsetfillcolor{textcolor}%
\pgftext[x=2.150044in,y=0.424381in,,top]{\color{textcolor}{\rmfamily\fontsize{10.000000}{12.000000}\selectfont\catcode`\^=\active\def^{\ifmmode\sp\else\^{}\fi}\catcode`\%=\active\def%{\%}$\mathdefault{60}$}}%
\end{pgfscope}%
\begin{pgfscope}%
\pgfsetbuttcap%
\pgfsetroundjoin%
\definecolor{currentfill}{rgb}{0.000000,0.000000,0.000000}%
\pgfsetfillcolor{currentfill}%
\pgfsetlinewidth{0.803000pt}%
\definecolor{currentstroke}{rgb}{0.000000,0.000000,0.000000}%
\pgfsetstrokecolor{currentstroke}%
\pgfsetdash{}{0pt}%
\pgfsys@defobject{currentmarker}{\pgfqpoint{0.000000in}{-0.048611in}}{\pgfqpoint{0.000000in}{0.000000in}}{%
\pgfpathmoveto{\pgfqpoint{0.000000in}{0.000000in}}%
\pgfpathlineto{\pgfqpoint{0.000000in}{-0.048611in}}%
\pgfusepath{stroke,fill}%
}%
\begin{pgfscope}%
\pgfsys@transformshift{2.640746in}{0.521603in}%
\pgfsys@useobject{currentmarker}{}%
\end{pgfscope}%
\end{pgfscope}%
\begin{pgfscope}%
\definecolor{textcolor}{rgb}{0.000000,0.000000,0.000000}%
\pgfsetstrokecolor{textcolor}%
\pgfsetfillcolor{textcolor}%
\pgftext[x=2.640746in,y=0.424381in,,top]{\color{textcolor}{\rmfamily\fontsize{10.000000}{12.000000}\selectfont\catcode`\^=\active\def^{\ifmmode\sp\else\^{}\fi}\catcode`\%=\active\def%{\%}$\mathdefault{80}$}}%
\end{pgfscope}%
\begin{pgfscope}%
\pgfsetbuttcap%
\pgfsetroundjoin%
\definecolor{currentfill}{rgb}{0.000000,0.000000,0.000000}%
\pgfsetfillcolor{currentfill}%
\pgfsetlinewidth{0.803000pt}%
\definecolor{currentstroke}{rgb}{0.000000,0.000000,0.000000}%
\pgfsetstrokecolor{currentstroke}%
\pgfsetdash{}{0pt}%
\pgfsys@defobject{currentmarker}{\pgfqpoint{0.000000in}{-0.048611in}}{\pgfqpoint{0.000000in}{0.000000in}}{%
\pgfpathmoveto{\pgfqpoint{0.000000in}{0.000000in}}%
\pgfpathlineto{\pgfqpoint{0.000000in}{-0.048611in}}%
\pgfusepath{stroke,fill}%
}%
\begin{pgfscope}%
\pgfsys@transformshift{3.131447in}{0.521603in}%
\pgfsys@useobject{currentmarker}{}%
\end{pgfscope}%
\end{pgfscope}%
\begin{pgfscope}%
\definecolor{textcolor}{rgb}{0.000000,0.000000,0.000000}%
\pgfsetstrokecolor{textcolor}%
\pgfsetfillcolor{textcolor}%
\pgftext[x=3.131447in,y=0.424381in,,top]{\color{textcolor}{\rmfamily\fontsize{10.000000}{12.000000}\selectfont\catcode`\^=\active\def^{\ifmmode\sp\else\^{}\fi}\catcode`\%=\active\def%{\%}$\mathdefault{100}$}}%
\end{pgfscope}%
\begin{pgfscope}%
\pgfsetbuttcap%
\pgfsetroundjoin%
\definecolor{currentfill}{rgb}{0.000000,0.000000,0.000000}%
\pgfsetfillcolor{currentfill}%
\pgfsetlinewidth{0.803000pt}%
\definecolor{currentstroke}{rgb}{0.000000,0.000000,0.000000}%
\pgfsetstrokecolor{currentstroke}%
\pgfsetdash{}{0pt}%
\pgfsys@defobject{currentmarker}{\pgfqpoint{0.000000in}{-0.048611in}}{\pgfqpoint{0.000000in}{0.000000in}}{%
\pgfpathmoveto{\pgfqpoint{0.000000in}{0.000000in}}%
\pgfpathlineto{\pgfqpoint{0.000000in}{-0.048611in}}%
\pgfusepath{stroke,fill}%
}%
\begin{pgfscope}%
\pgfsys@transformshift{3.622149in}{0.521603in}%
\pgfsys@useobject{currentmarker}{}%
\end{pgfscope}%
\end{pgfscope}%
\begin{pgfscope}%
\definecolor{textcolor}{rgb}{0.000000,0.000000,0.000000}%
\pgfsetstrokecolor{textcolor}%
\pgfsetfillcolor{textcolor}%
\pgftext[x=3.622149in,y=0.424381in,,top]{\color{textcolor}{\rmfamily\fontsize{10.000000}{12.000000}\selectfont\catcode`\^=\active\def^{\ifmmode\sp\else\^{}\fi}\catcode`\%=\active\def%{\%}$\mathdefault{120}$}}%
\end{pgfscope}%
\begin{pgfscope}%
\pgfsetbuttcap%
\pgfsetroundjoin%
\definecolor{currentfill}{rgb}{0.000000,0.000000,0.000000}%
\pgfsetfillcolor{currentfill}%
\pgfsetlinewidth{0.803000pt}%
\definecolor{currentstroke}{rgb}{0.000000,0.000000,0.000000}%
\pgfsetstrokecolor{currentstroke}%
\pgfsetdash{}{0pt}%
\pgfsys@defobject{currentmarker}{\pgfqpoint{0.000000in}{-0.048611in}}{\pgfqpoint{0.000000in}{0.000000in}}{%
\pgfpathmoveto{\pgfqpoint{0.000000in}{0.000000in}}%
\pgfpathlineto{\pgfqpoint{0.000000in}{-0.048611in}}%
\pgfusepath{stroke,fill}%
}%
\begin{pgfscope}%
\pgfsys@transformshift{4.112850in}{0.521603in}%
\pgfsys@useobject{currentmarker}{}%
\end{pgfscope}%
\end{pgfscope}%
\begin{pgfscope}%
\definecolor{textcolor}{rgb}{0.000000,0.000000,0.000000}%
\pgfsetstrokecolor{textcolor}%
\pgfsetfillcolor{textcolor}%
\pgftext[x=4.112850in,y=0.424381in,,top]{\color{textcolor}{\rmfamily\fontsize{10.000000}{12.000000}\selectfont\catcode`\^=\active\def^{\ifmmode\sp\else\^{}\fi}\catcode`\%=\active\def%{\%}$\mathdefault{140}$}}%
\end{pgfscope}%
\begin{pgfscope}%
\pgfsetbuttcap%
\pgfsetroundjoin%
\definecolor{currentfill}{rgb}{0.000000,0.000000,0.000000}%
\pgfsetfillcolor{currentfill}%
\pgfsetlinewidth{0.803000pt}%
\definecolor{currentstroke}{rgb}{0.000000,0.000000,0.000000}%
\pgfsetstrokecolor{currentstroke}%
\pgfsetdash{}{0pt}%
\pgfsys@defobject{currentmarker}{\pgfqpoint{0.000000in}{-0.048611in}}{\pgfqpoint{0.000000in}{0.000000in}}{%
\pgfpathmoveto{\pgfqpoint{0.000000in}{0.000000in}}%
\pgfpathlineto{\pgfqpoint{0.000000in}{-0.048611in}}%
\pgfusepath{stroke,fill}%
}%
\begin{pgfscope}%
\pgfsys@transformshift{4.603552in}{0.521603in}%
\pgfsys@useobject{currentmarker}{}%
\end{pgfscope}%
\end{pgfscope}%
\begin{pgfscope}%
\definecolor{textcolor}{rgb}{0.000000,0.000000,0.000000}%
\pgfsetstrokecolor{textcolor}%
\pgfsetfillcolor{textcolor}%
\pgftext[x=4.603552in,y=0.424381in,,top]{\color{textcolor}{\rmfamily\fontsize{10.000000}{12.000000}\selectfont\catcode`\^=\active\def^{\ifmmode\sp\else\^{}\fi}\catcode`\%=\active\def%{\%}$\mathdefault{160}$}}%
\end{pgfscope}%
\begin{pgfscope}%
\pgfsetbuttcap%
\pgfsetroundjoin%
\definecolor{currentfill}{rgb}{0.000000,0.000000,0.000000}%
\pgfsetfillcolor{currentfill}%
\pgfsetlinewidth{0.803000pt}%
\definecolor{currentstroke}{rgb}{0.000000,0.000000,0.000000}%
\pgfsetstrokecolor{currentstroke}%
\pgfsetdash{}{0pt}%
\pgfsys@defobject{currentmarker}{\pgfqpoint{0.000000in}{-0.048611in}}{\pgfqpoint{0.000000in}{0.000000in}}{%
\pgfpathmoveto{\pgfqpoint{0.000000in}{0.000000in}}%
\pgfpathlineto{\pgfqpoint{0.000000in}{-0.048611in}}%
\pgfusepath{stroke,fill}%
}%
\begin{pgfscope}%
\pgfsys@transformshift{5.094253in}{0.521603in}%
\pgfsys@useobject{currentmarker}{}%
\end{pgfscope}%
\end{pgfscope}%
\begin{pgfscope}%
\definecolor{textcolor}{rgb}{0.000000,0.000000,0.000000}%
\pgfsetstrokecolor{textcolor}%
\pgfsetfillcolor{textcolor}%
\pgftext[x=5.094253in,y=0.424381in,,top]{\color{textcolor}{\rmfamily\fontsize{10.000000}{12.000000}\selectfont\catcode`\^=\active\def^{\ifmmode\sp\else\^{}\fi}\catcode`\%=\active\def%{\%}$\mathdefault{180}$}}%
\end{pgfscope}%
\begin{pgfscope}%
\definecolor{textcolor}{rgb}{0.000000,0.000000,0.000000}%
\pgfsetstrokecolor{textcolor}%
\pgfsetfillcolor{textcolor}%
\pgftext[x=2.922899in,y=0.234413in,,top]{\color{textcolor}{\rmfamily\fontsize{10.000000}{12.000000}\selectfont\catcode`\^=\active\def^{\ifmmode\sp\else\^{}\fi}\catcode`\%=\active\def%{\%}Monochromatic classes}}%
\end{pgfscope}%
\begin{pgfscope}%
\pgfsetbuttcap%
\pgfsetroundjoin%
\definecolor{currentfill}{rgb}{0.000000,0.000000,0.000000}%
\pgfsetfillcolor{currentfill}%
\pgfsetlinewidth{0.803000pt}%
\definecolor{currentstroke}{rgb}{0.000000,0.000000,0.000000}%
\pgfsetstrokecolor{currentstroke}%
\pgfsetdash{}{0pt}%
\pgfsys@defobject{currentmarker}{\pgfqpoint{-0.048611in}{0.000000in}}{\pgfqpoint{-0.000000in}{0.000000in}}{%
\pgfpathmoveto{\pgfqpoint{-0.000000in}{0.000000in}}%
\pgfpathlineto{\pgfqpoint{-0.048611in}{0.000000in}}%
\pgfusepath{stroke,fill}%
}%
\begin{pgfscope}%
\pgfsys@transformshift{0.588387in}{0.612980in}%
\pgfsys@useobject{currentmarker}{}%
\end{pgfscope}%
\end{pgfscope}%
\begin{pgfscope}%
\definecolor{textcolor}{rgb}{0.000000,0.000000,0.000000}%
\pgfsetstrokecolor{textcolor}%
\pgfsetfillcolor{textcolor}%
\pgftext[x=0.289968in, y=0.560218in, left, base]{\color{textcolor}{\rmfamily\fontsize{10.000000}{12.000000}\selectfont\catcode`\^=\active\def^{\ifmmode\sp\else\^{}\fi}\catcode`\%=\active\def%{\%}$\mathdefault{10^{0}}$}}%
\end{pgfscope}%
\begin{pgfscope}%
\pgfsetbuttcap%
\pgfsetroundjoin%
\definecolor{currentfill}{rgb}{0.000000,0.000000,0.000000}%
\pgfsetfillcolor{currentfill}%
\pgfsetlinewidth{0.803000pt}%
\definecolor{currentstroke}{rgb}{0.000000,0.000000,0.000000}%
\pgfsetstrokecolor{currentstroke}%
\pgfsetdash{}{0pt}%
\pgfsys@defobject{currentmarker}{\pgfqpoint{-0.048611in}{0.000000in}}{\pgfqpoint{-0.000000in}{0.000000in}}{%
\pgfpathmoveto{\pgfqpoint{-0.000000in}{0.000000in}}%
\pgfpathlineto{\pgfqpoint{-0.048611in}{0.000000in}}%
\pgfusepath{stroke,fill}%
}%
\begin{pgfscope}%
\pgfsys@transformshift{0.588387in}{1.107607in}%
\pgfsys@useobject{currentmarker}{}%
\end{pgfscope}%
\end{pgfscope}%
\begin{pgfscope}%
\definecolor{textcolor}{rgb}{0.000000,0.000000,0.000000}%
\pgfsetstrokecolor{textcolor}%
\pgfsetfillcolor{textcolor}%
\pgftext[x=0.289968in, y=1.054846in, left, base]{\color{textcolor}{\rmfamily\fontsize{10.000000}{12.000000}\selectfont\catcode`\^=\active\def^{\ifmmode\sp\else\^{}\fi}\catcode`\%=\active\def%{\%}$\mathdefault{10^{1}}$}}%
\end{pgfscope}%
\begin{pgfscope}%
\pgfsetbuttcap%
\pgfsetroundjoin%
\definecolor{currentfill}{rgb}{0.000000,0.000000,0.000000}%
\pgfsetfillcolor{currentfill}%
\pgfsetlinewidth{0.803000pt}%
\definecolor{currentstroke}{rgb}{0.000000,0.000000,0.000000}%
\pgfsetstrokecolor{currentstroke}%
\pgfsetdash{}{0pt}%
\pgfsys@defobject{currentmarker}{\pgfqpoint{-0.048611in}{0.000000in}}{\pgfqpoint{-0.000000in}{0.000000in}}{%
\pgfpathmoveto{\pgfqpoint{-0.000000in}{0.000000in}}%
\pgfpathlineto{\pgfqpoint{-0.048611in}{0.000000in}}%
\pgfusepath{stroke,fill}%
}%
\begin{pgfscope}%
\pgfsys@transformshift{0.588387in}{1.602235in}%
\pgfsys@useobject{currentmarker}{}%
\end{pgfscope}%
\end{pgfscope}%
\begin{pgfscope}%
\definecolor{textcolor}{rgb}{0.000000,0.000000,0.000000}%
\pgfsetstrokecolor{textcolor}%
\pgfsetfillcolor{textcolor}%
\pgftext[x=0.289968in, y=1.549473in, left, base]{\color{textcolor}{\rmfamily\fontsize{10.000000}{12.000000}\selectfont\catcode`\^=\active\def^{\ifmmode\sp\else\^{}\fi}\catcode`\%=\active\def%{\%}$\mathdefault{10^{2}}$}}%
\end{pgfscope}%
\begin{pgfscope}%
\pgfsetbuttcap%
\pgfsetroundjoin%
\definecolor{currentfill}{rgb}{0.000000,0.000000,0.000000}%
\pgfsetfillcolor{currentfill}%
\pgfsetlinewidth{0.803000pt}%
\definecolor{currentstroke}{rgb}{0.000000,0.000000,0.000000}%
\pgfsetstrokecolor{currentstroke}%
\pgfsetdash{}{0pt}%
\pgfsys@defobject{currentmarker}{\pgfqpoint{-0.048611in}{0.000000in}}{\pgfqpoint{-0.000000in}{0.000000in}}{%
\pgfpathmoveto{\pgfqpoint{-0.000000in}{0.000000in}}%
\pgfpathlineto{\pgfqpoint{-0.048611in}{0.000000in}}%
\pgfusepath{stroke,fill}%
}%
\begin{pgfscope}%
\pgfsys@transformshift{0.588387in}{2.096862in}%
\pgfsys@useobject{currentmarker}{}%
\end{pgfscope}%
\end{pgfscope}%
\begin{pgfscope}%
\definecolor{textcolor}{rgb}{0.000000,0.000000,0.000000}%
\pgfsetstrokecolor{textcolor}%
\pgfsetfillcolor{textcolor}%
\pgftext[x=0.289968in, y=2.044100in, left, base]{\color{textcolor}{\rmfamily\fontsize{10.000000}{12.000000}\selectfont\catcode`\^=\active\def^{\ifmmode\sp\else\^{}\fi}\catcode`\%=\active\def%{\%}$\mathdefault{10^{3}}$}}%
\end{pgfscope}%
\begin{pgfscope}%
\pgfsetbuttcap%
\pgfsetroundjoin%
\definecolor{currentfill}{rgb}{0.000000,0.000000,0.000000}%
\pgfsetfillcolor{currentfill}%
\pgfsetlinewidth{0.602250pt}%
\definecolor{currentstroke}{rgb}{0.000000,0.000000,0.000000}%
\pgfsetstrokecolor{currentstroke}%
\pgfsetdash{}{0pt}%
\pgfsys@defobject{currentmarker}{\pgfqpoint{-0.027778in}{0.000000in}}{\pgfqpoint{-0.000000in}{0.000000in}}{%
\pgfpathmoveto{\pgfqpoint{-0.000000in}{0.000000in}}%
\pgfpathlineto{\pgfqpoint{-0.027778in}{0.000000in}}%
\pgfusepath{stroke,fill}%
}%
\begin{pgfscope}%
\pgfsys@transformshift{0.588387in}{0.536361in}%
\pgfsys@useobject{currentmarker}{}%
\end{pgfscope}%
\end{pgfscope}%
\begin{pgfscope}%
\pgfsetbuttcap%
\pgfsetroundjoin%
\definecolor{currentfill}{rgb}{0.000000,0.000000,0.000000}%
\pgfsetfillcolor{currentfill}%
\pgfsetlinewidth{0.602250pt}%
\definecolor{currentstroke}{rgb}{0.000000,0.000000,0.000000}%
\pgfsetstrokecolor{currentstroke}%
\pgfsetdash{}{0pt}%
\pgfsys@defobject{currentmarker}{\pgfqpoint{-0.027778in}{0.000000in}}{\pgfqpoint{-0.000000in}{0.000000in}}{%
\pgfpathmoveto{\pgfqpoint{-0.000000in}{0.000000in}}%
\pgfpathlineto{\pgfqpoint{-0.027778in}{0.000000in}}%
\pgfusepath{stroke,fill}%
}%
\begin{pgfscope}%
\pgfsys@transformshift{0.588387in}{0.565046in}%
\pgfsys@useobject{currentmarker}{}%
\end{pgfscope}%
\end{pgfscope}%
\begin{pgfscope}%
\pgfsetbuttcap%
\pgfsetroundjoin%
\definecolor{currentfill}{rgb}{0.000000,0.000000,0.000000}%
\pgfsetfillcolor{currentfill}%
\pgfsetlinewidth{0.602250pt}%
\definecolor{currentstroke}{rgb}{0.000000,0.000000,0.000000}%
\pgfsetstrokecolor{currentstroke}%
\pgfsetdash{}{0pt}%
\pgfsys@defobject{currentmarker}{\pgfqpoint{-0.027778in}{0.000000in}}{\pgfqpoint{-0.000000in}{0.000000in}}{%
\pgfpathmoveto{\pgfqpoint{-0.000000in}{0.000000in}}%
\pgfpathlineto{\pgfqpoint{-0.027778in}{0.000000in}}%
\pgfusepath{stroke,fill}%
}%
\begin{pgfscope}%
\pgfsys@transformshift{0.588387in}{0.590347in}%
\pgfsys@useobject{currentmarker}{}%
\end{pgfscope}%
\end{pgfscope}%
\begin{pgfscope}%
\pgfsetbuttcap%
\pgfsetroundjoin%
\definecolor{currentfill}{rgb}{0.000000,0.000000,0.000000}%
\pgfsetfillcolor{currentfill}%
\pgfsetlinewidth{0.602250pt}%
\definecolor{currentstroke}{rgb}{0.000000,0.000000,0.000000}%
\pgfsetstrokecolor{currentstroke}%
\pgfsetdash{}{0pt}%
\pgfsys@defobject{currentmarker}{\pgfqpoint{-0.027778in}{0.000000in}}{\pgfqpoint{-0.000000in}{0.000000in}}{%
\pgfpathmoveto{\pgfqpoint{-0.000000in}{0.000000in}}%
\pgfpathlineto{\pgfqpoint{-0.027778in}{0.000000in}}%
\pgfusepath{stroke,fill}%
}%
\begin{pgfscope}%
\pgfsys@transformshift{0.588387in}{0.761878in}%
\pgfsys@useobject{currentmarker}{}%
\end{pgfscope}%
\end{pgfscope}%
\begin{pgfscope}%
\pgfsetbuttcap%
\pgfsetroundjoin%
\definecolor{currentfill}{rgb}{0.000000,0.000000,0.000000}%
\pgfsetfillcolor{currentfill}%
\pgfsetlinewidth{0.602250pt}%
\definecolor{currentstroke}{rgb}{0.000000,0.000000,0.000000}%
\pgfsetstrokecolor{currentstroke}%
\pgfsetdash{}{0pt}%
\pgfsys@defobject{currentmarker}{\pgfqpoint{-0.027778in}{0.000000in}}{\pgfqpoint{-0.000000in}{0.000000in}}{%
\pgfpathmoveto{\pgfqpoint{-0.000000in}{0.000000in}}%
\pgfpathlineto{\pgfqpoint{-0.027778in}{0.000000in}}%
\pgfusepath{stroke,fill}%
}%
\begin{pgfscope}%
\pgfsys@transformshift{0.588387in}{0.848977in}%
\pgfsys@useobject{currentmarker}{}%
\end{pgfscope}%
\end{pgfscope}%
\begin{pgfscope}%
\pgfsetbuttcap%
\pgfsetroundjoin%
\definecolor{currentfill}{rgb}{0.000000,0.000000,0.000000}%
\pgfsetfillcolor{currentfill}%
\pgfsetlinewidth{0.602250pt}%
\definecolor{currentstroke}{rgb}{0.000000,0.000000,0.000000}%
\pgfsetstrokecolor{currentstroke}%
\pgfsetdash{}{0pt}%
\pgfsys@defobject{currentmarker}{\pgfqpoint{-0.027778in}{0.000000in}}{\pgfqpoint{-0.000000in}{0.000000in}}{%
\pgfpathmoveto{\pgfqpoint{-0.000000in}{0.000000in}}%
\pgfpathlineto{\pgfqpoint{-0.027778in}{0.000000in}}%
\pgfusepath{stroke,fill}%
}%
\begin{pgfscope}%
\pgfsys@transformshift{0.588387in}{0.910775in}%
\pgfsys@useobject{currentmarker}{}%
\end{pgfscope}%
\end{pgfscope}%
\begin{pgfscope}%
\pgfsetbuttcap%
\pgfsetroundjoin%
\definecolor{currentfill}{rgb}{0.000000,0.000000,0.000000}%
\pgfsetfillcolor{currentfill}%
\pgfsetlinewidth{0.602250pt}%
\definecolor{currentstroke}{rgb}{0.000000,0.000000,0.000000}%
\pgfsetstrokecolor{currentstroke}%
\pgfsetdash{}{0pt}%
\pgfsys@defobject{currentmarker}{\pgfqpoint{-0.027778in}{0.000000in}}{\pgfqpoint{-0.000000in}{0.000000in}}{%
\pgfpathmoveto{\pgfqpoint{-0.000000in}{0.000000in}}%
\pgfpathlineto{\pgfqpoint{-0.027778in}{0.000000in}}%
\pgfusepath{stroke,fill}%
}%
\begin{pgfscope}%
\pgfsys@transformshift{0.588387in}{0.958710in}%
\pgfsys@useobject{currentmarker}{}%
\end{pgfscope}%
\end{pgfscope}%
\begin{pgfscope}%
\pgfsetbuttcap%
\pgfsetroundjoin%
\definecolor{currentfill}{rgb}{0.000000,0.000000,0.000000}%
\pgfsetfillcolor{currentfill}%
\pgfsetlinewidth{0.602250pt}%
\definecolor{currentstroke}{rgb}{0.000000,0.000000,0.000000}%
\pgfsetstrokecolor{currentstroke}%
\pgfsetdash{}{0pt}%
\pgfsys@defobject{currentmarker}{\pgfqpoint{-0.027778in}{0.000000in}}{\pgfqpoint{-0.000000in}{0.000000in}}{%
\pgfpathmoveto{\pgfqpoint{-0.000000in}{0.000000in}}%
\pgfpathlineto{\pgfqpoint{-0.027778in}{0.000000in}}%
\pgfusepath{stroke,fill}%
}%
\begin{pgfscope}%
\pgfsys@transformshift{0.588387in}{0.997875in}%
\pgfsys@useobject{currentmarker}{}%
\end{pgfscope}%
\end{pgfscope}%
\begin{pgfscope}%
\pgfsetbuttcap%
\pgfsetroundjoin%
\definecolor{currentfill}{rgb}{0.000000,0.000000,0.000000}%
\pgfsetfillcolor{currentfill}%
\pgfsetlinewidth{0.602250pt}%
\definecolor{currentstroke}{rgb}{0.000000,0.000000,0.000000}%
\pgfsetstrokecolor{currentstroke}%
\pgfsetdash{}{0pt}%
\pgfsys@defobject{currentmarker}{\pgfqpoint{-0.027778in}{0.000000in}}{\pgfqpoint{-0.000000in}{0.000000in}}{%
\pgfpathmoveto{\pgfqpoint{-0.000000in}{0.000000in}}%
\pgfpathlineto{\pgfqpoint{-0.027778in}{0.000000in}}%
\pgfusepath{stroke,fill}%
}%
\begin{pgfscope}%
\pgfsys@transformshift{0.588387in}{1.030989in}%
\pgfsys@useobject{currentmarker}{}%
\end{pgfscope}%
\end{pgfscope}%
\begin{pgfscope}%
\pgfsetbuttcap%
\pgfsetroundjoin%
\definecolor{currentfill}{rgb}{0.000000,0.000000,0.000000}%
\pgfsetfillcolor{currentfill}%
\pgfsetlinewidth{0.602250pt}%
\definecolor{currentstroke}{rgb}{0.000000,0.000000,0.000000}%
\pgfsetstrokecolor{currentstroke}%
\pgfsetdash{}{0pt}%
\pgfsys@defobject{currentmarker}{\pgfqpoint{-0.027778in}{0.000000in}}{\pgfqpoint{-0.000000in}{0.000000in}}{%
\pgfpathmoveto{\pgfqpoint{-0.000000in}{0.000000in}}%
\pgfpathlineto{\pgfqpoint{-0.027778in}{0.000000in}}%
\pgfusepath{stroke,fill}%
}%
\begin{pgfscope}%
\pgfsys@transformshift{0.588387in}{1.059673in}%
\pgfsys@useobject{currentmarker}{}%
\end{pgfscope}%
\end{pgfscope}%
\begin{pgfscope}%
\pgfsetbuttcap%
\pgfsetroundjoin%
\definecolor{currentfill}{rgb}{0.000000,0.000000,0.000000}%
\pgfsetfillcolor{currentfill}%
\pgfsetlinewidth{0.602250pt}%
\definecolor{currentstroke}{rgb}{0.000000,0.000000,0.000000}%
\pgfsetstrokecolor{currentstroke}%
\pgfsetdash{}{0pt}%
\pgfsys@defobject{currentmarker}{\pgfqpoint{-0.027778in}{0.000000in}}{\pgfqpoint{-0.000000in}{0.000000in}}{%
\pgfpathmoveto{\pgfqpoint{-0.000000in}{0.000000in}}%
\pgfpathlineto{\pgfqpoint{-0.027778in}{0.000000in}}%
\pgfusepath{stroke,fill}%
}%
\begin{pgfscope}%
\pgfsys@transformshift{0.588387in}{1.084974in}%
\pgfsys@useobject{currentmarker}{}%
\end{pgfscope}%
\end{pgfscope}%
\begin{pgfscope}%
\pgfsetbuttcap%
\pgfsetroundjoin%
\definecolor{currentfill}{rgb}{0.000000,0.000000,0.000000}%
\pgfsetfillcolor{currentfill}%
\pgfsetlinewidth{0.602250pt}%
\definecolor{currentstroke}{rgb}{0.000000,0.000000,0.000000}%
\pgfsetstrokecolor{currentstroke}%
\pgfsetdash{}{0pt}%
\pgfsys@defobject{currentmarker}{\pgfqpoint{-0.027778in}{0.000000in}}{\pgfqpoint{-0.000000in}{0.000000in}}{%
\pgfpathmoveto{\pgfqpoint{-0.000000in}{0.000000in}}%
\pgfpathlineto{\pgfqpoint{-0.027778in}{0.000000in}}%
\pgfusepath{stroke,fill}%
}%
\begin{pgfscope}%
\pgfsys@transformshift{0.588387in}{1.256505in}%
\pgfsys@useobject{currentmarker}{}%
\end{pgfscope}%
\end{pgfscope}%
\begin{pgfscope}%
\pgfsetbuttcap%
\pgfsetroundjoin%
\definecolor{currentfill}{rgb}{0.000000,0.000000,0.000000}%
\pgfsetfillcolor{currentfill}%
\pgfsetlinewidth{0.602250pt}%
\definecolor{currentstroke}{rgb}{0.000000,0.000000,0.000000}%
\pgfsetstrokecolor{currentstroke}%
\pgfsetdash{}{0pt}%
\pgfsys@defobject{currentmarker}{\pgfqpoint{-0.027778in}{0.000000in}}{\pgfqpoint{-0.000000in}{0.000000in}}{%
\pgfpathmoveto{\pgfqpoint{-0.000000in}{0.000000in}}%
\pgfpathlineto{\pgfqpoint{-0.027778in}{0.000000in}}%
\pgfusepath{stroke,fill}%
}%
\begin{pgfscope}%
\pgfsys@transformshift{0.588387in}{1.343605in}%
\pgfsys@useobject{currentmarker}{}%
\end{pgfscope}%
\end{pgfscope}%
\begin{pgfscope}%
\pgfsetbuttcap%
\pgfsetroundjoin%
\definecolor{currentfill}{rgb}{0.000000,0.000000,0.000000}%
\pgfsetfillcolor{currentfill}%
\pgfsetlinewidth{0.602250pt}%
\definecolor{currentstroke}{rgb}{0.000000,0.000000,0.000000}%
\pgfsetstrokecolor{currentstroke}%
\pgfsetdash{}{0pt}%
\pgfsys@defobject{currentmarker}{\pgfqpoint{-0.027778in}{0.000000in}}{\pgfqpoint{-0.000000in}{0.000000in}}{%
\pgfpathmoveto{\pgfqpoint{-0.000000in}{0.000000in}}%
\pgfpathlineto{\pgfqpoint{-0.027778in}{0.000000in}}%
\pgfusepath{stroke,fill}%
}%
\begin{pgfscope}%
\pgfsys@transformshift{0.588387in}{1.405403in}%
\pgfsys@useobject{currentmarker}{}%
\end{pgfscope}%
\end{pgfscope}%
\begin{pgfscope}%
\pgfsetbuttcap%
\pgfsetroundjoin%
\definecolor{currentfill}{rgb}{0.000000,0.000000,0.000000}%
\pgfsetfillcolor{currentfill}%
\pgfsetlinewidth{0.602250pt}%
\definecolor{currentstroke}{rgb}{0.000000,0.000000,0.000000}%
\pgfsetstrokecolor{currentstroke}%
\pgfsetdash{}{0pt}%
\pgfsys@defobject{currentmarker}{\pgfqpoint{-0.027778in}{0.000000in}}{\pgfqpoint{-0.000000in}{0.000000in}}{%
\pgfpathmoveto{\pgfqpoint{-0.000000in}{0.000000in}}%
\pgfpathlineto{\pgfqpoint{-0.027778in}{0.000000in}}%
\pgfusepath{stroke,fill}%
}%
\begin{pgfscope}%
\pgfsys@transformshift{0.588387in}{1.453337in}%
\pgfsys@useobject{currentmarker}{}%
\end{pgfscope}%
\end{pgfscope}%
\begin{pgfscope}%
\pgfsetbuttcap%
\pgfsetroundjoin%
\definecolor{currentfill}{rgb}{0.000000,0.000000,0.000000}%
\pgfsetfillcolor{currentfill}%
\pgfsetlinewidth{0.602250pt}%
\definecolor{currentstroke}{rgb}{0.000000,0.000000,0.000000}%
\pgfsetstrokecolor{currentstroke}%
\pgfsetdash{}{0pt}%
\pgfsys@defobject{currentmarker}{\pgfqpoint{-0.027778in}{0.000000in}}{\pgfqpoint{-0.000000in}{0.000000in}}{%
\pgfpathmoveto{\pgfqpoint{-0.000000in}{0.000000in}}%
\pgfpathlineto{\pgfqpoint{-0.027778in}{0.000000in}}%
\pgfusepath{stroke,fill}%
}%
\begin{pgfscope}%
\pgfsys@transformshift{0.588387in}{1.492502in}%
\pgfsys@useobject{currentmarker}{}%
\end{pgfscope}%
\end{pgfscope}%
\begin{pgfscope}%
\pgfsetbuttcap%
\pgfsetroundjoin%
\definecolor{currentfill}{rgb}{0.000000,0.000000,0.000000}%
\pgfsetfillcolor{currentfill}%
\pgfsetlinewidth{0.602250pt}%
\definecolor{currentstroke}{rgb}{0.000000,0.000000,0.000000}%
\pgfsetstrokecolor{currentstroke}%
\pgfsetdash{}{0pt}%
\pgfsys@defobject{currentmarker}{\pgfqpoint{-0.027778in}{0.000000in}}{\pgfqpoint{-0.000000in}{0.000000in}}{%
\pgfpathmoveto{\pgfqpoint{-0.000000in}{0.000000in}}%
\pgfpathlineto{\pgfqpoint{-0.027778in}{0.000000in}}%
\pgfusepath{stroke,fill}%
}%
\begin{pgfscope}%
\pgfsys@transformshift{0.588387in}{1.525616in}%
\pgfsys@useobject{currentmarker}{}%
\end{pgfscope}%
\end{pgfscope}%
\begin{pgfscope}%
\pgfsetbuttcap%
\pgfsetroundjoin%
\definecolor{currentfill}{rgb}{0.000000,0.000000,0.000000}%
\pgfsetfillcolor{currentfill}%
\pgfsetlinewidth{0.602250pt}%
\definecolor{currentstroke}{rgb}{0.000000,0.000000,0.000000}%
\pgfsetstrokecolor{currentstroke}%
\pgfsetdash{}{0pt}%
\pgfsys@defobject{currentmarker}{\pgfqpoint{-0.027778in}{0.000000in}}{\pgfqpoint{-0.000000in}{0.000000in}}{%
\pgfpathmoveto{\pgfqpoint{-0.000000in}{0.000000in}}%
\pgfpathlineto{\pgfqpoint{-0.027778in}{0.000000in}}%
\pgfusepath{stroke,fill}%
}%
\begin{pgfscope}%
\pgfsys@transformshift{0.588387in}{1.554300in}%
\pgfsys@useobject{currentmarker}{}%
\end{pgfscope}%
\end{pgfscope}%
\begin{pgfscope}%
\pgfsetbuttcap%
\pgfsetroundjoin%
\definecolor{currentfill}{rgb}{0.000000,0.000000,0.000000}%
\pgfsetfillcolor{currentfill}%
\pgfsetlinewidth{0.602250pt}%
\definecolor{currentstroke}{rgb}{0.000000,0.000000,0.000000}%
\pgfsetstrokecolor{currentstroke}%
\pgfsetdash{}{0pt}%
\pgfsys@defobject{currentmarker}{\pgfqpoint{-0.027778in}{0.000000in}}{\pgfqpoint{-0.000000in}{0.000000in}}{%
\pgfpathmoveto{\pgfqpoint{-0.000000in}{0.000000in}}%
\pgfpathlineto{\pgfqpoint{-0.027778in}{0.000000in}}%
\pgfusepath{stroke,fill}%
}%
\begin{pgfscope}%
\pgfsys@transformshift{0.588387in}{1.579602in}%
\pgfsys@useobject{currentmarker}{}%
\end{pgfscope}%
\end{pgfscope}%
\begin{pgfscope}%
\pgfsetbuttcap%
\pgfsetroundjoin%
\definecolor{currentfill}{rgb}{0.000000,0.000000,0.000000}%
\pgfsetfillcolor{currentfill}%
\pgfsetlinewidth{0.602250pt}%
\definecolor{currentstroke}{rgb}{0.000000,0.000000,0.000000}%
\pgfsetstrokecolor{currentstroke}%
\pgfsetdash{}{0pt}%
\pgfsys@defobject{currentmarker}{\pgfqpoint{-0.027778in}{0.000000in}}{\pgfqpoint{-0.000000in}{0.000000in}}{%
\pgfpathmoveto{\pgfqpoint{-0.000000in}{0.000000in}}%
\pgfpathlineto{\pgfqpoint{-0.027778in}{0.000000in}}%
\pgfusepath{stroke,fill}%
}%
\begin{pgfscope}%
\pgfsys@transformshift{0.588387in}{1.751132in}%
\pgfsys@useobject{currentmarker}{}%
\end{pgfscope}%
\end{pgfscope}%
\begin{pgfscope}%
\pgfsetbuttcap%
\pgfsetroundjoin%
\definecolor{currentfill}{rgb}{0.000000,0.000000,0.000000}%
\pgfsetfillcolor{currentfill}%
\pgfsetlinewidth{0.602250pt}%
\definecolor{currentstroke}{rgb}{0.000000,0.000000,0.000000}%
\pgfsetstrokecolor{currentstroke}%
\pgfsetdash{}{0pt}%
\pgfsys@defobject{currentmarker}{\pgfqpoint{-0.027778in}{0.000000in}}{\pgfqpoint{-0.000000in}{0.000000in}}{%
\pgfpathmoveto{\pgfqpoint{-0.000000in}{0.000000in}}%
\pgfpathlineto{\pgfqpoint{-0.027778in}{0.000000in}}%
\pgfusepath{stroke,fill}%
}%
\begin{pgfscope}%
\pgfsys@transformshift{0.588387in}{1.838232in}%
\pgfsys@useobject{currentmarker}{}%
\end{pgfscope}%
\end{pgfscope}%
\begin{pgfscope}%
\pgfsetbuttcap%
\pgfsetroundjoin%
\definecolor{currentfill}{rgb}{0.000000,0.000000,0.000000}%
\pgfsetfillcolor{currentfill}%
\pgfsetlinewidth{0.602250pt}%
\definecolor{currentstroke}{rgb}{0.000000,0.000000,0.000000}%
\pgfsetstrokecolor{currentstroke}%
\pgfsetdash{}{0pt}%
\pgfsys@defobject{currentmarker}{\pgfqpoint{-0.027778in}{0.000000in}}{\pgfqpoint{-0.000000in}{0.000000in}}{%
\pgfpathmoveto{\pgfqpoint{-0.000000in}{0.000000in}}%
\pgfpathlineto{\pgfqpoint{-0.027778in}{0.000000in}}%
\pgfusepath{stroke,fill}%
}%
\begin{pgfscope}%
\pgfsys@transformshift{0.588387in}{1.900030in}%
\pgfsys@useobject{currentmarker}{}%
\end{pgfscope}%
\end{pgfscope}%
\begin{pgfscope}%
\pgfsetbuttcap%
\pgfsetroundjoin%
\definecolor{currentfill}{rgb}{0.000000,0.000000,0.000000}%
\pgfsetfillcolor{currentfill}%
\pgfsetlinewidth{0.602250pt}%
\definecolor{currentstroke}{rgb}{0.000000,0.000000,0.000000}%
\pgfsetstrokecolor{currentstroke}%
\pgfsetdash{}{0pt}%
\pgfsys@defobject{currentmarker}{\pgfqpoint{-0.027778in}{0.000000in}}{\pgfqpoint{-0.000000in}{0.000000in}}{%
\pgfpathmoveto{\pgfqpoint{-0.000000in}{0.000000in}}%
\pgfpathlineto{\pgfqpoint{-0.027778in}{0.000000in}}%
\pgfusepath{stroke,fill}%
}%
\begin{pgfscope}%
\pgfsys@transformshift{0.588387in}{1.947964in}%
\pgfsys@useobject{currentmarker}{}%
\end{pgfscope}%
\end{pgfscope}%
\begin{pgfscope}%
\pgfsetbuttcap%
\pgfsetroundjoin%
\definecolor{currentfill}{rgb}{0.000000,0.000000,0.000000}%
\pgfsetfillcolor{currentfill}%
\pgfsetlinewidth{0.602250pt}%
\definecolor{currentstroke}{rgb}{0.000000,0.000000,0.000000}%
\pgfsetstrokecolor{currentstroke}%
\pgfsetdash{}{0pt}%
\pgfsys@defobject{currentmarker}{\pgfqpoint{-0.027778in}{0.000000in}}{\pgfqpoint{-0.000000in}{0.000000in}}{%
\pgfpathmoveto{\pgfqpoint{-0.000000in}{0.000000in}}%
\pgfpathlineto{\pgfqpoint{-0.027778in}{0.000000in}}%
\pgfusepath{stroke,fill}%
}%
\begin{pgfscope}%
\pgfsys@transformshift{0.588387in}{1.987130in}%
\pgfsys@useobject{currentmarker}{}%
\end{pgfscope}%
\end{pgfscope}%
\begin{pgfscope}%
\pgfsetbuttcap%
\pgfsetroundjoin%
\definecolor{currentfill}{rgb}{0.000000,0.000000,0.000000}%
\pgfsetfillcolor{currentfill}%
\pgfsetlinewidth{0.602250pt}%
\definecolor{currentstroke}{rgb}{0.000000,0.000000,0.000000}%
\pgfsetstrokecolor{currentstroke}%
\pgfsetdash{}{0pt}%
\pgfsys@defobject{currentmarker}{\pgfqpoint{-0.027778in}{0.000000in}}{\pgfqpoint{-0.000000in}{0.000000in}}{%
\pgfpathmoveto{\pgfqpoint{-0.000000in}{0.000000in}}%
\pgfpathlineto{\pgfqpoint{-0.027778in}{0.000000in}}%
\pgfusepath{stroke,fill}%
}%
\begin{pgfscope}%
\pgfsys@transformshift{0.588387in}{2.020243in}%
\pgfsys@useobject{currentmarker}{}%
\end{pgfscope}%
\end{pgfscope}%
\begin{pgfscope}%
\pgfsetbuttcap%
\pgfsetroundjoin%
\definecolor{currentfill}{rgb}{0.000000,0.000000,0.000000}%
\pgfsetfillcolor{currentfill}%
\pgfsetlinewidth{0.602250pt}%
\definecolor{currentstroke}{rgb}{0.000000,0.000000,0.000000}%
\pgfsetstrokecolor{currentstroke}%
\pgfsetdash{}{0pt}%
\pgfsys@defobject{currentmarker}{\pgfqpoint{-0.027778in}{0.000000in}}{\pgfqpoint{-0.000000in}{0.000000in}}{%
\pgfpathmoveto{\pgfqpoint{-0.000000in}{0.000000in}}%
\pgfpathlineto{\pgfqpoint{-0.027778in}{0.000000in}}%
\pgfusepath{stroke,fill}%
}%
\begin{pgfscope}%
\pgfsys@transformshift{0.588387in}{2.048928in}%
\pgfsys@useobject{currentmarker}{}%
\end{pgfscope}%
\end{pgfscope}%
\begin{pgfscope}%
\pgfsetbuttcap%
\pgfsetroundjoin%
\definecolor{currentfill}{rgb}{0.000000,0.000000,0.000000}%
\pgfsetfillcolor{currentfill}%
\pgfsetlinewidth{0.602250pt}%
\definecolor{currentstroke}{rgb}{0.000000,0.000000,0.000000}%
\pgfsetstrokecolor{currentstroke}%
\pgfsetdash{}{0pt}%
\pgfsys@defobject{currentmarker}{\pgfqpoint{-0.027778in}{0.000000in}}{\pgfqpoint{-0.000000in}{0.000000in}}{%
\pgfpathmoveto{\pgfqpoint{-0.000000in}{0.000000in}}%
\pgfpathlineto{\pgfqpoint{-0.027778in}{0.000000in}}%
\pgfusepath{stroke,fill}%
}%
\begin{pgfscope}%
\pgfsys@transformshift{0.588387in}{2.074229in}%
\pgfsys@useobject{currentmarker}{}%
\end{pgfscope}%
\end{pgfscope}%
\begin{pgfscope}%
\pgfsetbuttcap%
\pgfsetroundjoin%
\definecolor{currentfill}{rgb}{0.000000,0.000000,0.000000}%
\pgfsetfillcolor{currentfill}%
\pgfsetlinewidth{0.602250pt}%
\definecolor{currentstroke}{rgb}{0.000000,0.000000,0.000000}%
\pgfsetstrokecolor{currentstroke}%
\pgfsetdash{}{0pt}%
\pgfsys@defobject{currentmarker}{\pgfqpoint{-0.027778in}{0.000000in}}{\pgfqpoint{-0.000000in}{0.000000in}}{%
\pgfpathmoveto{\pgfqpoint{-0.000000in}{0.000000in}}%
\pgfpathlineto{\pgfqpoint{-0.027778in}{0.000000in}}%
\pgfusepath{stroke,fill}%
}%
\begin{pgfscope}%
\pgfsys@transformshift{0.588387in}{2.245760in}%
\pgfsys@useobject{currentmarker}{}%
\end{pgfscope}%
\end{pgfscope}%
\begin{pgfscope}%
\pgfsetbuttcap%
\pgfsetroundjoin%
\definecolor{currentfill}{rgb}{0.000000,0.000000,0.000000}%
\pgfsetfillcolor{currentfill}%
\pgfsetlinewidth{0.602250pt}%
\definecolor{currentstroke}{rgb}{0.000000,0.000000,0.000000}%
\pgfsetstrokecolor{currentstroke}%
\pgfsetdash{}{0pt}%
\pgfsys@defobject{currentmarker}{\pgfqpoint{-0.027778in}{0.000000in}}{\pgfqpoint{-0.000000in}{0.000000in}}{%
\pgfpathmoveto{\pgfqpoint{-0.000000in}{0.000000in}}%
\pgfpathlineto{\pgfqpoint{-0.027778in}{0.000000in}}%
\pgfusepath{stroke,fill}%
}%
\begin{pgfscope}%
\pgfsys@transformshift{0.588387in}{2.332859in}%
\pgfsys@useobject{currentmarker}{}%
\end{pgfscope}%
\end{pgfscope}%
\begin{pgfscope}%
\pgfsetbuttcap%
\pgfsetroundjoin%
\definecolor{currentfill}{rgb}{0.000000,0.000000,0.000000}%
\pgfsetfillcolor{currentfill}%
\pgfsetlinewidth{0.602250pt}%
\definecolor{currentstroke}{rgb}{0.000000,0.000000,0.000000}%
\pgfsetstrokecolor{currentstroke}%
\pgfsetdash{}{0pt}%
\pgfsys@defobject{currentmarker}{\pgfqpoint{-0.027778in}{0.000000in}}{\pgfqpoint{-0.000000in}{0.000000in}}{%
\pgfpathmoveto{\pgfqpoint{-0.000000in}{0.000000in}}%
\pgfpathlineto{\pgfqpoint{-0.027778in}{0.000000in}}%
\pgfusepath{stroke,fill}%
}%
\begin{pgfscope}%
\pgfsys@transformshift{0.588387in}{2.394657in}%
\pgfsys@useobject{currentmarker}{}%
\end{pgfscope}%
\end{pgfscope}%
\begin{pgfscope}%
\pgfsetbuttcap%
\pgfsetroundjoin%
\definecolor{currentfill}{rgb}{0.000000,0.000000,0.000000}%
\pgfsetfillcolor{currentfill}%
\pgfsetlinewidth{0.602250pt}%
\definecolor{currentstroke}{rgb}{0.000000,0.000000,0.000000}%
\pgfsetstrokecolor{currentstroke}%
\pgfsetdash{}{0pt}%
\pgfsys@defobject{currentmarker}{\pgfqpoint{-0.027778in}{0.000000in}}{\pgfqpoint{-0.000000in}{0.000000in}}{%
\pgfpathmoveto{\pgfqpoint{-0.000000in}{0.000000in}}%
\pgfpathlineto{\pgfqpoint{-0.027778in}{0.000000in}}%
\pgfusepath{stroke,fill}%
}%
\begin{pgfscope}%
\pgfsys@transformshift{0.588387in}{2.442592in}%
\pgfsys@useobject{currentmarker}{}%
\end{pgfscope}%
\end{pgfscope}%
\begin{pgfscope}%
\pgfsetbuttcap%
\pgfsetroundjoin%
\definecolor{currentfill}{rgb}{0.000000,0.000000,0.000000}%
\pgfsetfillcolor{currentfill}%
\pgfsetlinewidth{0.602250pt}%
\definecolor{currentstroke}{rgb}{0.000000,0.000000,0.000000}%
\pgfsetstrokecolor{currentstroke}%
\pgfsetdash{}{0pt}%
\pgfsys@defobject{currentmarker}{\pgfqpoint{-0.027778in}{0.000000in}}{\pgfqpoint{-0.000000in}{0.000000in}}{%
\pgfpathmoveto{\pgfqpoint{-0.000000in}{0.000000in}}%
\pgfpathlineto{\pgfqpoint{-0.027778in}{0.000000in}}%
\pgfusepath{stroke,fill}%
}%
\begin{pgfscope}%
\pgfsys@transformshift{0.588387in}{2.481757in}%
\pgfsys@useobject{currentmarker}{}%
\end{pgfscope}%
\end{pgfscope}%
\begin{pgfscope}%
\pgfsetbuttcap%
\pgfsetroundjoin%
\definecolor{currentfill}{rgb}{0.000000,0.000000,0.000000}%
\pgfsetfillcolor{currentfill}%
\pgfsetlinewidth{0.602250pt}%
\definecolor{currentstroke}{rgb}{0.000000,0.000000,0.000000}%
\pgfsetstrokecolor{currentstroke}%
\pgfsetdash{}{0pt}%
\pgfsys@defobject{currentmarker}{\pgfqpoint{-0.027778in}{0.000000in}}{\pgfqpoint{-0.000000in}{0.000000in}}{%
\pgfpathmoveto{\pgfqpoint{-0.000000in}{0.000000in}}%
\pgfpathlineto{\pgfqpoint{-0.027778in}{0.000000in}}%
\pgfusepath{stroke,fill}%
}%
\begin{pgfscope}%
\pgfsys@transformshift{0.588387in}{2.514871in}%
\pgfsys@useobject{currentmarker}{}%
\end{pgfscope}%
\end{pgfscope}%
\begin{pgfscope}%
\definecolor{textcolor}{rgb}{0.000000,0.000000,0.000000}%
\pgfsetstrokecolor{textcolor}%
\pgfsetfillcolor{textcolor}%
\pgftext[x=0.234413in,y=1.526746in,,bottom,rotate=90.000000]{\color{textcolor}{\rmfamily\fontsize{10.000000}{12.000000}\selectfont\catcode`\^=\active\def^{\ifmmode\sp\else\^{}\fi}\catcode`\%=\active\def%{\%}Time [ms]}}%
\end{pgfscope}%
\begin{pgfscope}%
\pgfpathrectangle{\pgfqpoint{0.588387in}{0.521603in}}{\pgfqpoint{4.669024in}{2.010285in}}%
\pgfusepath{clip}%
\pgfsetrectcap%
\pgfsetroundjoin%
\pgfsetlinewidth{1.505625pt}%
\pgfsetstrokecolor{currentstroke1}%
\pgfsetdash{}{0pt}%
\pgfpathmoveto{\pgfqpoint{0.800616in}{0.612980in}}%
\pgfpathlineto{\pgfqpoint{0.825151in}{0.612980in}}%
\pgfpathlineto{\pgfqpoint{0.874221in}{0.761878in}}%
\pgfpathlineto{\pgfqpoint{0.923291in}{0.848977in}}%
\pgfpathlineto{\pgfqpoint{0.972361in}{0.910775in}}%
\pgfpathlineto{\pgfqpoint{1.045966in}{0.997875in}}%
\pgfpathlineto{\pgfqpoint{1.070501in}{0.927970in}}%
\pgfpathlineto{\pgfqpoint{1.119572in}{1.055802in}}%
\pgfpathlineto{\pgfqpoint{1.193177in}{1.084974in}}%
\pgfpathlineto{\pgfqpoint{1.242247in}{1.146772in}}%
\pgfpathlineto{\pgfqpoint{1.315852in}{1.208571in}}%
\pgfpathlineto{\pgfqpoint{1.364922in}{1.229488in}}%
\pgfpathlineto{\pgfqpoint{1.438527in}{1.278369in}}%
\pgfpathlineto{\pgfqpoint{1.512133in}{1.319908in}}%
\pgfpathlineto{\pgfqpoint{1.610273in}{1.376718in}}%
\pgfpathlineto{\pgfqpoint{1.683878in}{1.420938in}}%
\pgfpathlineto{\pgfqpoint{1.757483in}{1.443071in}}%
\pgfpathlineto{\pgfqpoint{1.831089in}{1.492502in}}%
\pgfpathlineto{\pgfqpoint{1.904694in}{1.506845in}}%
\pgfpathlineto{\pgfqpoint{2.002834in}{1.548862in}}%
\pgfpathlineto{\pgfqpoint{2.076439in}{1.562208in}}%
\pgfpathlineto{\pgfqpoint{2.174580in}{1.596796in}}%
\pgfpathlineto{\pgfqpoint{2.272720in}{1.626579in}}%
\pgfpathlineto{\pgfqpoint{2.346325in}{1.652731in}}%
\pgfpathlineto{\pgfqpoint{2.444465in}{1.714954in}}%
\pgfpathlineto{\pgfqpoint{2.542606in}{1.712396in}}%
\pgfpathlineto{\pgfqpoint{2.640746in}{1.755912in}}%
\pgfpathlineto{\pgfqpoint{2.763421in}{1.772580in}}%
\pgfpathlineto{\pgfqpoint{2.812491in}{1.788048in}}%
\pgfpathlineto{\pgfqpoint{2.886097in}{1.804681in}}%
\pgfpathlineto{\pgfqpoint{3.008772in}{1.831541in}}%
\pgfpathlineto{\pgfqpoint{3.106912in}{1.857382in}}%
\pgfpathlineto{\pgfqpoint{3.229588in}{1.887309in}}%
\pgfpathlineto{\pgfqpoint{3.352263in}{1.904547in}}%
\pgfpathlineto{\pgfqpoint{3.474938in}{1.933297in}}%
\pgfpathlineto{\pgfqpoint{3.622149in}{1.963300in}}%
\pgfpathlineto{\pgfqpoint{3.720289in}{1.994173in}}%
\pgfpathlineto{\pgfqpoint{3.842964in}{2.021467in}}%
\pgfpathlineto{\pgfqpoint{3.965640in}{2.031308in}}%
\pgfpathlineto{\pgfqpoint{4.088315in}{2.064837in}}%
\pgfpathlineto{\pgfqpoint{4.235526in}{2.082884in}}%
\pgfpathlineto{\pgfqpoint{4.358201in}{2.110692in}}%
\pgfpathlineto{\pgfqpoint{4.505411in}{2.126885in}}%
\pgfpathlineto{\pgfqpoint{4.652622in}{2.157961in}}%
\pgfpathlineto{\pgfqpoint{4.799832in}{2.284118in}}%
\pgfpathlineto{\pgfqpoint{4.971578in}{2.204826in}}%
\pgfpathlineto{\pgfqpoint{4.996113in}{2.236607in}}%
\pgfpathlineto{\pgfqpoint{5.045183in}{2.222990in}}%
\pgfusepath{stroke}%
\end{pgfscope}%
\begin{pgfscope}%
\pgfpathrectangle{\pgfqpoint{0.588387in}{0.521603in}}{\pgfqpoint{4.669024in}{2.010285in}}%
\pgfusepath{clip}%
\pgfsetrectcap%
\pgfsetroundjoin%
\pgfsetlinewidth{1.505625pt}%
\pgfsetstrokecolor{currentstroke2}%
\pgfsetdash{}{0pt}%
\pgfpathmoveto{\pgfqpoint{0.800616in}{0.612980in}}%
\pgfpathlineto{\pgfqpoint{0.825151in}{0.612980in}}%
\pgfpathlineto{\pgfqpoint{0.874221in}{0.612980in}}%
\pgfpathlineto{\pgfqpoint{0.923291in}{0.848977in}}%
\pgfpathlineto{\pgfqpoint{0.972361in}{0.910775in}}%
\pgfpathlineto{\pgfqpoint{1.045966in}{0.997875in}}%
\pgfpathlineto{\pgfqpoint{1.070501in}{1.020508in}}%
\pgfpathlineto{\pgfqpoint{1.119572in}{1.538381in}}%
\pgfpathlineto{\pgfqpoint{1.193177in}{1.107607in}}%
\pgfpathlineto{\pgfqpoint{1.242247in}{2.057353in}}%
\pgfpathlineto{\pgfqpoint{1.315852in}{2.346185in}}%
\pgfpathlineto{\pgfqpoint{1.364922in}{1.891660in}}%
\pgfpathlineto{\pgfqpoint{1.438527in}{2.244860in}}%
\pgfpathlineto{\pgfqpoint{1.512133in}{2.231636in}}%
\pgfpathlineto{\pgfqpoint{1.831089in}{2.357310in}}%
\pgfpathlineto{\pgfqpoint{1.904694in}{2.404289in}}%
\pgfpathlineto{\pgfqpoint{2.076439in}{2.404164in}}%
\pgfpathlineto{\pgfqpoint{2.174580in}{2.404344in}}%
\pgfpathlineto{\pgfqpoint{2.346325in}{2.196883in}}%
\pgfpathlineto{\pgfqpoint{2.640746in}{2.315957in}}%
\pgfpathlineto{\pgfqpoint{2.812491in}{2.391492in}}%
\pgfpathlineto{\pgfqpoint{2.886097in}{2.416464in}}%
\pgfpathlineto{\pgfqpoint{3.008772in}{2.404072in}}%
\pgfpathlineto{\pgfqpoint{3.106912in}{2.420560in}}%
\pgfpathlineto{\pgfqpoint{3.229588in}{2.380089in}}%
\pgfpathlineto{\pgfqpoint{4.996113in}{2.366411in}}%
\pgfpathlineto{\pgfqpoint{5.045183in}{2.434149in}}%
\pgfusepath{stroke}%
\end{pgfscope}%
\begin{pgfscope}%
\pgfpathrectangle{\pgfqpoint{0.588387in}{0.521603in}}{\pgfqpoint{4.669024in}{2.010285in}}%
\pgfusepath{clip}%
\pgfsetrectcap%
\pgfsetroundjoin%
\pgfsetlinewidth{1.505625pt}%
\pgfsetstrokecolor{currentstroke3}%
\pgfsetdash{}{0pt}%
\pgfpathmoveto{\pgfqpoint{0.800616in}{0.722712in}}%
\pgfpathlineto{\pgfqpoint{0.825151in}{0.612980in}}%
\pgfpathlineto{\pgfqpoint{0.874221in}{0.674778in}}%
\pgfpathlineto{\pgfqpoint{0.923291in}{0.700079in}}%
\pgfpathlineto{\pgfqpoint{0.972361in}{0.910775in}}%
\pgfpathlineto{\pgfqpoint{1.045966in}{0.979184in}}%
\pgfpathlineto{\pgfqpoint{1.070501in}{0.991823in}}%
\pgfpathlineto{\pgfqpoint{1.119572in}{1.055802in}}%
\pgfpathlineto{\pgfqpoint{1.193177in}{1.128081in}}%
\pgfpathlineto{\pgfqpoint{1.242247in}{1.189879in}}%
\pgfpathlineto{\pgfqpoint{1.315852in}{1.208571in}}%
\pgfpathlineto{\pgfqpoint{1.364922in}{1.255363in}}%
\pgfpathlineto{\pgfqpoint{1.438527in}{1.282034in}}%
\pgfpathlineto{\pgfqpoint{1.512133in}{1.352641in}}%
\pgfpathlineto{\pgfqpoint{1.610273in}{1.376718in}}%
\pgfpathlineto{\pgfqpoint{1.683878in}{1.436976in}}%
\pgfpathlineto{\pgfqpoint{1.757483in}{1.448021in}}%
\pgfpathlineto{\pgfqpoint{1.831089in}{1.488081in}}%
\pgfpathlineto{\pgfqpoint{1.904694in}{1.503793in}}%
\pgfpathlineto{\pgfqpoint{2.002834in}{1.581975in}}%
\pgfpathlineto{\pgfqpoint{2.076439in}{1.563499in}}%
\pgfpathlineto{\pgfqpoint{2.174580in}{1.598260in}}%
\pgfpathlineto{\pgfqpoint{2.272720in}{1.631634in}}%
\pgfpathlineto{\pgfqpoint{2.346325in}{1.655124in}}%
\pgfpathlineto{\pgfqpoint{2.444465in}{1.685966in}}%
\pgfpathlineto{\pgfqpoint{2.542606in}{1.721217in}}%
\pgfpathlineto{\pgfqpoint{2.640746in}{1.742550in}}%
\pgfpathlineto{\pgfqpoint{2.763421in}{1.771281in}}%
\pgfpathlineto{\pgfqpoint{2.812491in}{1.781622in}}%
\pgfpathlineto{\pgfqpoint{2.886097in}{1.795438in}}%
\pgfpathlineto{\pgfqpoint{3.008772in}{1.820009in}}%
\pgfpathlineto{\pgfqpoint{3.106912in}{1.834311in}}%
\pgfpathlineto{\pgfqpoint{3.229588in}{1.857493in}}%
\pgfpathlineto{\pgfqpoint{3.352263in}{1.898683in}}%
\pgfpathlineto{\pgfqpoint{3.474938in}{1.904635in}}%
\pgfpathlineto{\pgfqpoint{3.622149in}{1.936644in}}%
\pgfpathlineto{\pgfqpoint{3.720289in}{1.961559in}}%
\pgfpathlineto{\pgfqpoint{3.842964in}{1.987010in}}%
\pgfpathlineto{\pgfqpoint{3.965640in}{2.044862in}}%
\pgfpathlineto{\pgfqpoint{4.088315in}{2.066576in}}%
\pgfpathlineto{\pgfqpoint{4.235526in}{2.092741in}}%
\pgfpathlineto{\pgfqpoint{4.358201in}{2.119183in}}%
\pgfpathlineto{\pgfqpoint{4.505411in}{2.146849in}}%
\pgfpathlineto{\pgfqpoint{4.652622in}{2.175192in}}%
\pgfpathlineto{\pgfqpoint{4.799832in}{2.199430in}}%
\pgfpathlineto{\pgfqpoint{4.971578in}{2.222529in}}%
\pgfpathlineto{\pgfqpoint{4.996113in}{2.193857in}}%
\pgfpathlineto{\pgfqpoint{5.045183in}{2.232669in}}%
\pgfusepath{stroke}%
\end{pgfscope}%
\begin{pgfscope}%
\pgfpathrectangle{\pgfqpoint{0.588387in}{0.521603in}}{\pgfqpoint{4.669024in}{2.010285in}}%
\pgfusepath{clip}%
\pgfsetrectcap%
\pgfsetroundjoin%
\pgfsetlinewidth{1.505625pt}%
\pgfsetstrokecolor{currentstroke4}%
\pgfsetdash{}{0pt}%
\pgfpathmoveto{\pgfqpoint{0.800616in}{0.612980in}}%
\pgfpathlineto{\pgfqpoint{0.825151in}{0.612980in}}%
\pgfpathlineto{\pgfqpoint{0.874221in}{0.700079in}}%
\pgfpathlineto{\pgfqpoint{0.923291in}{0.761878in}}%
\pgfpathlineto{\pgfqpoint{0.972361in}{0.958710in}}%
\pgfpathlineto{\pgfqpoint{1.045966in}{0.997875in}}%
\pgfpathlineto{\pgfqpoint{1.070501in}{1.009489in}}%
\pgfpathlineto{\pgfqpoint{1.119572in}{1.237255in}}%
\pgfpathlineto{\pgfqpoint{1.193177in}{1.442318in}}%
\pgfpathlineto{\pgfqpoint{1.242247in}{1.563499in}}%
\pgfpathlineto{\pgfqpoint{1.315852in}{1.266986in}}%
\pgfpathlineto{\pgfqpoint{1.364922in}{1.282023in}}%
\pgfpathlineto{\pgfqpoint{1.438527in}{1.351882in}}%
\pgfpathlineto{\pgfqpoint{1.512133in}{1.363427in}}%
\pgfpathlineto{\pgfqpoint{1.610273in}{2.009225in}}%
\pgfpathlineto{\pgfqpoint{1.683878in}{1.716221in}}%
\pgfpathlineto{\pgfqpoint{1.757483in}{1.507482in}}%
\pgfpathlineto{\pgfqpoint{1.831089in}{1.563927in}}%
\pgfpathlineto{\pgfqpoint{1.904694in}{1.560910in}}%
\pgfpathlineto{\pgfqpoint{2.002834in}{1.558291in}}%
\pgfpathlineto{\pgfqpoint{2.076439in}{1.583152in}}%
\pgfpathlineto{\pgfqpoint{2.174580in}{1.604903in}}%
\pgfpathlineto{\pgfqpoint{2.272720in}{1.648444in}}%
\pgfpathlineto{\pgfqpoint{2.346325in}{1.669272in}}%
\pgfpathlineto{\pgfqpoint{2.444465in}{1.734965in}}%
\pgfpathlineto{\pgfqpoint{2.542606in}{1.726702in}}%
\pgfpathlineto{\pgfqpoint{2.640746in}{1.910255in}}%
\pgfpathlineto{\pgfqpoint{2.763421in}{1.811989in}}%
\pgfpathlineto{\pgfqpoint{2.812491in}{1.962063in}}%
\pgfpathlineto{\pgfqpoint{2.886097in}{2.132048in}}%
\pgfpathlineto{\pgfqpoint{3.008772in}{1.914413in}}%
\pgfpathlineto{\pgfqpoint{3.106912in}{1.935717in}}%
\pgfpathlineto{\pgfqpoint{3.229588in}{1.963928in}}%
\pgfpathlineto{\pgfqpoint{3.352263in}{1.931913in}}%
\pgfpathlineto{\pgfqpoint{3.474938in}{1.937172in}}%
\pgfpathlineto{\pgfqpoint{3.622149in}{1.985151in}}%
\pgfpathlineto{\pgfqpoint{3.720289in}{1.992957in}}%
\pgfpathlineto{\pgfqpoint{3.842964in}{2.035779in}}%
\pgfpathlineto{\pgfqpoint{4.088315in}{2.102794in}}%
\pgfpathlineto{\pgfqpoint{4.358201in}{2.126136in}}%
\pgfpathlineto{\pgfqpoint{4.652622in}{2.178008in}}%
\pgfpathlineto{\pgfqpoint{4.971578in}{2.228896in}}%
\pgfpathlineto{\pgfqpoint{4.996113in}{2.242154in}}%
\pgfpathlineto{\pgfqpoint{5.045183in}{2.254068in}}%
\pgfusepath{stroke}%
\end{pgfscope}%
\begin{pgfscope}%
\pgfpathrectangle{\pgfqpoint{0.588387in}{0.521603in}}{\pgfqpoint{4.669024in}{2.010285in}}%
\pgfusepath{clip}%
\pgfsetrectcap%
\pgfsetroundjoin%
\pgfsetlinewidth{1.505625pt}%
\pgfsetstrokecolor{currentstroke5}%
\pgfsetdash{}{0pt}%
\pgfpathmoveto{\pgfqpoint{0.800616in}{0.612980in}}%
\pgfpathlineto{\pgfqpoint{0.825151in}{0.612980in}}%
\pgfpathlineto{\pgfqpoint{0.874221in}{0.612980in}}%
\pgfpathlineto{\pgfqpoint{0.923291in}{0.700079in}}%
\pgfpathlineto{\pgfqpoint{0.972361in}{0.910775in}}%
\pgfpathlineto{\pgfqpoint{1.045966in}{0.988732in}}%
\pgfpathlineto{\pgfqpoint{1.070501in}{0.969190in}}%
\pgfpathlineto{\pgfqpoint{1.119572in}{1.049862in}}%
\pgfpathlineto{\pgfqpoint{1.193177in}{1.132909in}}%
\pgfpathlineto{\pgfqpoint{1.242247in}{1.203476in}}%
\pgfpathlineto{\pgfqpoint{1.315852in}{1.201751in}}%
\pgfpathlineto{\pgfqpoint{1.364922in}{1.267218in}}%
\pgfpathlineto{\pgfqpoint{1.438527in}{1.281465in}}%
\pgfpathlineto{\pgfqpoint{1.512133in}{1.365592in}}%
\pgfpathlineto{\pgfqpoint{1.610273in}{1.391539in}}%
\pgfpathlineto{\pgfqpoint{1.683878in}{1.437748in}}%
\pgfpathlineto{\pgfqpoint{1.757483in}{1.443071in}}%
\pgfpathlineto{\pgfqpoint{1.831089in}{1.496639in}}%
\pgfpathlineto{\pgfqpoint{1.904694in}{1.501517in}}%
\pgfpathlineto{\pgfqpoint{2.002834in}{1.540437in}}%
\pgfpathlineto{\pgfqpoint{2.076439in}{1.565844in}}%
\pgfpathlineto{\pgfqpoint{2.174580in}{1.594024in}}%
\pgfpathlineto{\pgfqpoint{2.272720in}{1.631321in}}%
\pgfpathlineto{\pgfqpoint{2.346325in}{1.648444in}}%
\pgfpathlineto{\pgfqpoint{2.444465in}{1.684994in}}%
\pgfpathlineto{\pgfqpoint{2.542606in}{1.703198in}}%
\pgfpathlineto{\pgfqpoint{2.640746in}{1.718421in}}%
\pgfpathlineto{\pgfqpoint{2.763421in}{1.749515in}}%
\pgfpathlineto{\pgfqpoint{2.812491in}{1.761400in}}%
\pgfpathlineto{\pgfqpoint{2.886097in}{1.780153in}}%
\pgfpathlineto{\pgfqpoint{3.008772in}{1.811908in}}%
\pgfpathlineto{\pgfqpoint{3.106912in}{1.821468in}}%
\pgfpathlineto{\pgfqpoint{3.229588in}{1.838436in}}%
\pgfpathlineto{\pgfqpoint{3.352263in}{1.872875in}}%
\pgfpathlineto{\pgfqpoint{3.474938in}{1.887594in}}%
\pgfpathlineto{\pgfqpoint{3.622149in}{1.925093in}}%
\pgfpathlineto{\pgfqpoint{3.720289in}{1.944281in}}%
\pgfpathlineto{\pgfqpoint{3.842964in}{1.968243in}}%
\pgfusepath{stroke}%
\end{pgfscope}%
\begin{pgfscope}%
\pgfpathrectangle{\pgfqpoint{0.588387in}{0.521603in}}{\pgfqpoint{4.669024in}{2.010285in}}%
\pgfusepath{clip}%
\pgfsetrectcap%
\pgfsetroundjoin%
\pgfsetlinewidth{1.505625pt}%
\pgfsetstrokecolor{currentstroke6}%
\pgfsetdash{}{0pt}%
\pgfpathmoveto{\pgfqpoint{0.800616in}{0.700079in}}%
\pgfpathlineto{\pgfqpoint{0.825151in}{0.612980in}}%
\pgfpathlineto{\pgfqpoint{0.874221in}{0.674778in}}%
\pgfpathlineto{\pgfqpoint{0.923291in}{0.700079in}}%
\pgfpathlineto{\pgfqpoint{0.972361in}{0.882091in}}%
\pgfpathlineto{\pgfqpoint{1.045966in}{0.958710in}}%
\pgfpathlineto{\pgfqpoint{1.070501in}{0.958710in}}%
\pgfpathlineto{\pgfqpoint{1.119572in}{1.026559in}}%
\pgfpathlineto{\pgfqpoint{1.193177in}{1.102169in}}%
\pgfpathlineto{\pgfqpoint{1.242247in}{1.148259in}}%
\pgfpathlineto{\pgfqpoint{1.315852in}{1.184941in}}%
\pgfpathlineto{\pgfqpoint{1.364922in}{1.235493in}}%
\pgfpathlineto{\pgfqpoint{1.438527in}{1.254449in}}%
\pgfpathlineto{\pgfqpoint{1.512133in}{1.327072in}}%
\pgfpathlineto{\pgfqpoint{1.610273in}{1.350648in}}%
\pgfpathlineto{\pgfqpoint{1.683878in}{1.412446in}}%
\pgfpathlineto{\pgfqpoint{1.757483in}{1.422597in}}%
\pgfpathlineto{\pgfqpoint{1.831089in}{1.472068in}}%
\pgfpathlineto{\pgfqpoint{1.904694in}{1.479682in}}%
\pgfpathlineto{\pgfqpoint{2.002834in}{1.543282in}}%
\pgfpathlineto{\pgfqpoint{2.076439in}{1.544222in}}%
\pgfpathlineto{\pgfqpoint{2.174580in}{1.569419in}}%
\pgfpathlineto{\pgfqpoint{2.272720in}{1.621403in}}%
\pgfpathlineto{\pgfqpoint{2.346325in}{1.623359in}}%
\pgfpathlineto{\pgfqpoint{2.444465in}{1.650169in}}%
\pgfpathlineto{\pgfqpoint{2.542606in}{1.687416in}}%
\pgfpathlineto{\pgfqpoint{2.640746in}{1.716852in}}%
\pgfpathlineto{\pgfqpoint{2.763421in}{1.750774in}}%
\pgfpathlineto{\pgfqpoint{2.812491in}{1.773470in}}%
\pgfpathlineto{\pgfqpoint{2.886097in}{1.781349in}}%
\pgfpathlineto{\pgfqpoint{3.008772in}{1.801742in}}%
\pgfpathlineto{\pgfqpoint{3.106912in}{1.820356in}}%
\pgfpathlineto{\pgfqpoint{3.229588in}{1.833857in}}%
\pgfpathlineto{\pgfqpoint{3.352263in}{1.862789in}}%
\pgfpathlineto{\pgfqpoint{3.474938in}{1.904810in}}%
\pgfpathlineto{\pgfqpoint{3.622149in}{1.911022in}}%
\pgfpathlineto{\pgfqpoint{3.720289in}{1.940385in}}%
\pgfpathlineto{\pgfqpoint{3.842964in}{1.966608in}}%
\pgfpathlineto{\pgfqpoint{3.965640in}{2.027781in}}%
\pgfpathlineto{\pgfqpoint{4.088315in}{2.133053in}}%
\pgfpathlineto{\pgfqpoint{4.235526in}{2.084482in}}%
\pgfpathlineto{\pgfqpoint{4.358201in}{2.098999in}}%
\pgfpathlineto{\pgfqpoint{4.505411in}{2.168526in}}%
\pgfpathlineto{\pgfqpoint{4.652622in}{2.154375in}}%
\pgfpathlineto{\pgfqpoint{4.799832in}{2.181803in}}%
\pgfpathlineto{\pgfqpoint{4.971578in}{2.201948in}}%
\pgfpathlineto{\pgfqpoint{4.996113in}{2.181120in}}%
\pgfpathlineto{\pgfqpoint{5.045183in}{2.317788in}}%
\pgfusepath{stroke}%
\end{pgfscope}%
\begin{pgfscope}%
\pgfpathrectangle{\pgfqpoint{0.588387in}{0.521603in}}{\pgfqpoint{4.669024in}{2.010285in}}%
\pgfusepath{clip}%
\pgfsetrectcap%
\pgfsetroundjoin%
\pgfsetlinewidth{1.505625pt}%
\pgfsetstrokecolor{currentstroke7}%
\pgfsetdash{}{0pt}%
\pgfpathmoveto{\pgfqpoint{0.800616in}{0.612980in}}%
\pgfpathlineto{\pgfqpoint{0.825151in}{0.612980in}}%
\pgfpathlineto{\pgfqpoint{0.874221in}{0.612980in}}%
\pgfpathlineto{\pgfqpoint{0.923291in}{0.700079in}}%
\pgfpathlineto{\pgfqpoint{0.972361in}{0.958710in}}%
\pgfpathlineto{\pgfqpoint{1.045966in}{1.084974in}}%
\pgfpathlineto{\pgfqpoint{1.070501in}{1.223689in}}%
\pgfpathlineto{\pgfqpoint{1.119572in}{1.382770in}}%
\pgfpathlineto{\pgfqpoint{1.193177in}{1.580792in}}%
\pgfpathlineto{\pgfqpoint{1.242247in}{1.192909in}}%
\pgfpathlineto{\pgfqpoint{1.315852in}{1.905858in}}%
\pgfpathlineto{\pgfqpoint{1.364922in}{2.323640in}}%
\pgfpathlineto{\pgfqpoint{1.438527in}{2.229025in}}%
\pgfpathlineto{\pgfqpoint{1.512133in}{2.326903in}}%
\pgfpathlineto{\pgfqpoint{1.831089in}{2.408051in}}%
\pgfpathlineto{\pgfqpoint{2.812491in}{2.433936in}}%
\pgfpathlineto{\pgfqpoint{2.886097in}{2.439624in}}%
\pgfpathlineto{\pgfqpoint{3.106912in}{2.440512in}}%
\pgfpathlineto{\pgfqpoint{3.229588in}{2.427829in}}%
\pgfpathlineto{\pgfqpoint{4.996113in}{2.409557in}}%
\pgfusepath{stroke}%
\end{pgfscope}%
\begin{pgfscope}%
\pgfpathrectangle{\pgfqpoint{0.588387in}{0.521603in}}{\pgfqpoint{4.669024in}{2.010285in}}%
\pgfusepath{clip}%
\pgfsetrectcap%
\pgfsetroundjoin%
\pgfsetlinewidth{1.505625pt}%
\definecolor{currentstroke}{rgb}{0.498039,0.498039,0.498039}%
\pgfsetstrokecolor{currentstroke}%
\pgfsetdash{}{0pt}%
\pgfpathmoveto{\pgfqpoint{0.923291in}{0.612980in}}%
\pgfpathlineto{\pgfqpoint{0.972361in}{0.612980in}}%
\pgfpathlineto{\pgfqpoint{1.045966in}{0.761878in}}%
\pgfpathlineto{\pgfqpoint{1.070501in}{0.674778in}}%
\pgfpathlineto{\pgfqpoint{1.119572in}{0.761878in}}%
\pgfpathlineto{\pgfqpoint{1.193177in}{0.761878in}}%
\pgfpathlineto{\pgfqpoint{1.242247in}{0.848977in}}%
\pgfpathlineto{\pgfqpoint{1.315852in}{0.848977in}}%
\pgfpathlineto{\pgfqpoint{1.364922in}{0.864664in}}%
\pgfpathlineto{\pgfqpoint{1.438527in}{0.914577in}}%
\pgfpathlineto{\pgfqpoint{1.512133in}{0.949940in}}%
\pgfpathlineto{\pgfqpoint{1.610273in}{0.958710in}}%
\pgfpathlineto{\pgfqpoint{1.683878in}{1.030989in}}%
\pgfpathlineto{\pgfqpoint{1.757483in}{1.059673in}}%
\pgfpathlineto{\pgfqpoint{1.831089in}{1.059673in}}%
\pgfpathlineto{\pgfqpoint{1.904694in}{1.098198in}}%
\pgfpathlineto{\pgfqpoint{2.002834in}{1.128081in}}%
\pgfpathlineto{\pgfqpoint{2.076439in}{1.146772in}}%
\pgfpathlineto{\pgfqpoint{2.174580in}{1.179886in}}%
\pgfpathlineto{\pgfqpoint{2.272720in}{1.194707in}}%
\pgfpathlineto{\pgfqpoint{2.346325in}{1.215181in}}%
\pgfpathlineto{\pgfqpoint{2.444465in}{1.233872in}}%
\pgfpathlineto{\pgfqpoint{2.542606in}{1.266986in}}%
\pgfpathlineto{\pgfqpoint{2.640746in}{1.286528in}}%
\pgfpathlineto{\pgfqpoint{2.763421in}{1.312864in}}%
\pgfpathlineto{\pgfqpoint{2.812491in}{1.318973in}}%
\pgfpathlineto{\pgfqpoint{2.886097in}{1.328525in}}%
\pgfpathlineto{\pgfqpoint{3.008772in}{1.350648in}}%
\pgfpathlineto{\pgfqpoint{3.106912in}{1.369259in}}%
\pgfpathlineto{\pgfqpoint{3.229588in}{1.394384in}}%
\pgfpathlineto{\pgfqpoint{3.352263in}{1.415883in}}%
\pgfpathlineto{\pgfqpoint{3.474938in}{1.435425in}}%
\pgfpathlineto{\pgfqpoint{3.622149in}{1.457591in}}%
\pgfpathlineto{\pgfqpoint{3.720289in}{1.473811in}}%
\pgfpathlineto{\pgfqpoint{3.842964in}{1.492502in}}%
\pgfpathlineto{\pgfqpoint{3.965640in}{1.519389in}}%
\pgfpathlineto{\pgfqpoint{4.088315in}{1.533154in}}%
\pgfpathlineto{\pgfqpoint{4.235526in}{1.551598in}}%
\pgfpathlineto{\pgfqpoint{4.358201in}{1.571081in}}%
\pgfpathlineto{\pgfqpoint{4.505411in}{1.593465in}}%
\pgfpathlineto{\pgfqpoint{4.652622in}{1.610660in}}%
\pgfpathlineto{\pgfqpoint{4.799832in}{1.630381in}}%
\pgfpathlineto{\pgfqpoint{4.971578in}{1.650169in}}%
\pgfpathlineto{\pgfqpoint{4.996113in}{1.616482in}}%
\pgfpathlineto{\pgfqpoint{5.045183in}{1.653098in}}%
\pgfusepath{stroke}%
\end{pgfscope}%
\begin{pgfscope}%
\pgfsetrectcap%
\pgfsetmiterjoin%
\pgfsetlinewidth{0.803000pt}%
\definecolor{currentstroke}{rgb}{0.000000,0.000000,0.000000}%
\pgfsetstrokecolor{currentstroke}%
\pgfsetdash{}{0pt}%
\pgfpathmoveto{\pgfqpoint{0.588387in}{0.521603in}}%
\pgfpathlineto{\pgfqpoint{0.588387in}{2.531888in}}%
\pgfusepath{stroke}%
\end{pgfscope}%
\begin{pgfscope}%
\pgfsetrectcap%
\pgfsetmiterjoin%
\pgfsetlinewidth{0.803000pt}%
\definecolor{currentstroke}{rgb}{0.000000,0.000000,0.000000}%
\pgfsetstrokecolor{currentstroke}%
\pgfsetdash{}{0pt}%
\pgfpathmoveto{\pgfqpoint{5.257411in}{0.521603in}}%
\pgfpathlineto{\pgfqpoint{5.257411in}{2.531888in}}%
\pgfusepath{stroke}%
\end{pgfscope}%
\begin{pgfscope}%
\pgfsetrectcap%
\pgfsetmiterjoin%
\pgfsetlinewidth{0.803000pt}%
\definecolor{currentstroke}{rgb}{0.000000,0.000000,0.000000}%
\pgfsetstrokecolor{currentstroke}%
\pgfsetdash{}{0pt}%
\pgfpathmoveto{\pgfqpoint{0.588387in}{0.521603in}}%
\pgfpathlineto{\pgfqpoint{5.257411in}{0.521603in}}%
\pgfusepath{stroke}%
\end{pgfscope}%
\begin{pgfscope}%
\pgfsetrectcap%
\pgfsetmiterjoin%
\pgfsetlinewidth{0.803000pt}%
\definecolor{currentstroke}{rgb}{0.000000,0.000000,0.000000}%
\pgfsetstrokecolor{currentstroke}%
\pgfsetdash{}{0pt}%
\pgfpathmoveto{\pgfqpoint{0.588387in}{2.531888in}}%
\pgfpathlineto{\pgfqpoint{5.257411in}{2.531888in}}%
\pgfusepath{stroke}%
\end{pgfscope}%
\begin{pgfscope}%
\definecolor{textcolor}{rgb}{0.000000,0.000000,0.000000}%
\pgfsetstrokecolor{textcolor}%
\pgfsetfillcolor{textcolor}%
\pgftext[x=2.922899in,y=2.615222in,,base]{\color{textcolor}{\rmfamily\fontsize{12.000000}{14.400000}\selectfont\catcode`\^=\active\def^{\ifmmode\sp\else\^{}\fi}\catcode`\%=\active\def%{\%}Mean}}%
\end{pgfscope}%
\begin{pgfscope}%
\pgfsetbuttcap%
\pgfsetmiterjoin%
\definecolor{currentfill}{rgb}{1.000000,1.000000,1.000000}%
\pgfsetfillcolor{currentfill}%
\pgfsetfillopacity{0.800000}%
\pgfsetlinewidth{1.003750pt}%
\definecolor{currentstroke}{rgb}{0.800000,0.800000,0.800000}%
\pgfsetstrokecolor{currentstroke}%
\pgfsetstrokeopacity{0.800000}%
\pgfsetdash{}{0pt}%
\pgfpathmoveto{\pgfqpoint{5.344911in}{0.946722in}}%
\pgfpathlineto{\pgfqpoint{8.259376in}{0.946722in}}%
\pgfpathquadraticcurveto{\pgfqpoint{8.284376in}{0.946722in}}{\pgfqpoint{8.284376in}{0.971722in}}%
\pgfpathlineto{\pgfqpoint{8.284376in}{2.444388in}}%
\pgfpathquadraticcurveto{\pgfqpoint{8.284376in}{2.469388in}}{\pgfqpoint{8.259376in}{2.469388in}}%
\pgfpathlineto{\pgfqpoint{5.344911in}{2.469388in}}%
\pgfpathquadraticcurveto{\pgfqpoint{5.319911in}{2.469388in}}{\pgfqpoint{5.319911in}{2.444388in}}%
\pgfpathlineto{\pgfqpoint{5.319911in}{0.971722in}}%
\pgfpathquadraticcurveto{\pgfqpoint{5.319911in}{0.946722in}}{\pgfqpoint{5.344911in}{0.946722in}}%
\pgfpathlineto{\pgfqpoint{5.344911in}{0.946722in}}%
\pgfpathclose%
\pgfusepath{stroke,fill}%
\end{pgfscope}%
\begin{pgfscope}%
\pgfsetrectcap%
\pgfsetroundjoin%
\pgfsetlinewidth{1.505625pt}%
\definecolor{currentstroke}{rgb}{0.498039,0.498039,0.498039}%
\pgfsetstrokecolor{currentstroke}%
\pgfsetdash{}{0pt}%
\pgfpathmoveto{\pgfqpoint{5.369911in}{2.368168in}}%
\pgfpathlineto{\pgfqpoint{5.494911in}{2.368168in}}%
\pgfpathlineto{\pgfqpoint{5.619911in}{2.368168in}}%
\pgfusepath{stroke}%
\end{pgfscope}%
\begin{pgfscope}%
\definecolor{textcolor}{rgb}{0.000000,0.000000,0.000000}%
\pgfsetstrokecolor{textcolor}%
\pgfsetfillcolor{textcolor}%
\pgftext[x=5.719911in,y=2.324418in,left,base]{\color{textcolor}{\rmfamily\fontsize{9.000000}{10.800000}\selectfont\catcode`\^=\active\def^{\ifmmode\sp\else\^{}\fi}\catcode`\%=\active\def%{\%}\NaiveCycles{}}}%
\end{pgfscope}%
\begin{pgfscope}%
\pgfsetrectcap%
\pgfsetroundjoin%
\pgfsetlinewidth{1.505625pt}%
\pgfsetstrokecolor{currentstroke1}%
\pgfsetdash{}{0pt}%
\pgfpathmoveto{\pgfqpoint{5.369911in}{2.184696in}}%
\pgfpathlineto{\pgfqpoint{5.494911in}{2.184696in}}%
\pgfpathlineto{\pgfqpoint{5.619911in}{2.184696in}}%
\pgfusepath{stroke}%
\end{pgfscope}%
\begin{pgfscope}%
\definecolor{textcolor}{rgb}{0.000000,0.000000,0.000000}%
\pgfsetstrokecolor{textcolor}%
\pgfsetfillcolor{textcolor}%
\pgftext[x=5.719911in,y=2.140946in,left,base]{\color{textcolor}{\rmfamily\fontsize{9.000000}{10.800000}\selectfont\catcode`\^=\active\def^{\ifmmode\sp\else\^{}\fi}\catcode`\%=\active\def%{\%}\CyclesMatchChunks{} \& \MergeLinear{}}}%
\end{pgfscope}%
\begin{pgfscope}%
\pgfsetrectcap%
\pgfsetroundjoin%
\pgfsetlinewidth{1.505625pt}%
\pgfsetstrokecolor{currentstroke2}%
\pgfsetdash{}{0pt}%
\pgfpathmoveto{\pgfqpoint{5.369911in}{1.997746in}}%
\pgfpathlineto{\pgfqpoint{5.494911in}{1.997746in}}%
\pgfpathlineto{\pgfqpoint{5.619911in}{1.997746in}}%
\pgfusepath{stroke}%
\end{pgfscope}%
\begin{pgfscope}%
\definecolor{textcolor}{rgb}{0.000000,0.000000,0.000000}%
\pgfsetstrokecolor{textcolor}%
\pgfsetfillcolor{textcolor}%
\pgftext[x=5.719911in,y=1.953996in,left,base]{\color{textcolor}{\rmfamily\fontsize{9.000000}{10.800000}\selectfont\catcode`\^=\active\def^{\ifmmode\sp\else\^{}\fi}\catcode`\%=\active\def%{\%}\CyclesMatchChunks{} \& \SharedVertices{}}}%
\end{pgfscope}%
\begin{pgfscope}%
\pgfsetrectcap%
\pgfsetroundjoin%
\pgfsetlinewidth{1.505625pt}%
\pgfsetstrokecolor{currentstroke3}%
\pgfsetdash{}{0pt}%
\pgfpathmoveto{\pgfqpoint{5.369911in}{1.810795in}}%
\pgfpathlineto{\pgfqpoint{5.494911in}{1.810795in}}%
\pgfpathlineto{\pgfqpoint{5.619911in}{1.810795in}}%
\pgfusepath{stroke}%
\end{pgfscope}%
\begin{pgfscope}%
\definecolor{textcolor}{rgb}{0.000000,0.000000,0.000000}%
\pgfsetstrokecolor{textcolor}%
\pgfsetfillcolor{textcolor}%
\pgftext[x=5.719911in,y=1.767045in,left,base]{\color{textcolor}{\rmfamily\fontsize{9.000000}{10.800000}\selectfont\catcode`\^=\active\def^{\ifmmode\sp\else\^{}\fi}\catcode`\%=\active\def%{\%}\Neighbors{} \& \MergeLinear{}}}%
\end{pgfscope}%
\begin{pgfscope}%
\pgfsetrectcap%
\pgfsetroundjoin%
\pgfsetlinewidth{1.505625pt}%
\pgfsetstrokecolor{currentstroke4}%
\pgfsetdash{}{0pt}%
\pgfpathmoveto{\pgfqpoint{5.369911in}{1.627324in}}%
\pgfpathlineto{\pgfqpoint{5.494911in}{1.627324in}}%
\pgfpathlineto{\pgfqpoint{5.619911in}{1.627324in}}%
\pgfusepath{stroke}%
\end{pgfscope}%
\begin{pgfscope}%
\definecolor{textcolor}{rgb}{0.000000,0.000000,0.000000}%
\pgfsetstrokecolor{textcolor}%
\pgfsetfillcolor{textcolor}%
\pgftext[x=5.719911in,y=1.583574in,left,base]{\color{textcolor}{\rmfamily\fontsize{9.000000}{10.800000}\selectfont\catcode`\^=\active\def^{\ifmmode\sp\else\^{}\fi}\catcode`\%=\active\def%{\%}\Neighbors{} \& \SharedVertices{}}}%
\end{pgfscope}%
\begin{pgfscope}%
\pgfsetrectcap%
\pgfsetroundjoin%
\pgfsetlinewidth{1.505625pt}%
\pgfsetstrokecolor{currentstroke5}%
\pgfsetdash{}{0pt}%
\pgfpathmoveto{\pgfqpoint{5.369911in}{1.440373in}}%
\pgfpathlineto{\pgfqpoint{5.494911in}{1.440373in}}%
\pgfpathlineto{\pgfqpoint{5.619911in}{1.440373in}}%
\pgfusepath{stroke}%
\end{pgfscope}%
\begin{pgfscope}%
\definecolor{textcolor}{rgb}{0.000000,0.000000,0.000000}%
\pgfsetstrokecolor{textcolor}%
\pgfsetfillcolor{textcolor}%
\pgftext[x=5.719911in,y=1.396623in,left,base]{\color{textcolor}{\rmfamily\fontsize{9.000000}{10.800000}\selectfont\catcode`\^=\active\def^{\ifmmode\sp\else\^{}\fi}\catcode`\%=\active\def%{\%}\NeighborsDegree{} \& \MergeLinear{}}}%
\end{pgfscope}%
\begin{pgfscope}%
\pgfsetrectcap%
\pgfsetroundjoin%
\pgfsetlinewidth{1.505625pt}%
\pgfsetstrokecolor{currentstroke6}%
\pgfsetdash{}{0pt}%
\pgfpathmoveto{\pgfqpoint{5.369911in}{1.253423in}}%
\pgfpathlineto{\pgfqpoint{5.494911in}{1.253423in}}%
\pgfpathlineto{\pgfqpoint{5.619911in}{1.253423in}}%
\pgfusepath{stroke}%
\end{pgfscope}%
\begin{pgfscope}%
\definecolor{textcolor}{rgb}{0.000000,0.000000,0.000000}%
\pgfsetstrokecolor{textcolor}%
\pgfsetfillcolor{textcolor}%
\pgftext[x=5.719911in,y=1.209673in,left,base]{\color{textcolor}{\rmfamily\fontsize{9.000000}{10.800000}\selectfont\catcode`\^=\active\def^{\ifmmode\sp\else\^{}\fi}\catcode`\%=\active\def%{\%}\None{} \& \MergeLinear{}}}%
\end{pgfscope}%
\begin{pgfscope}%
\pgfsetrectcap%
\pgfsetroundjoin%
\pgfsetlinewidth{1.505625pt}%
\pgfsetstrokecolor{currentstroke7}%
\pgfsetdash{}{0pt}%
\pgfpathmoveto{\pgfqpoint{5.369911in}{1.069951in}}%
\pgfpathlineto{\pgfqpoint{5.494911in}{1.069951in}}%
\pgfpathlineto{\pgfqpoint{5.619911in}{1.069951in}}%
\pgfusepath{stroke}%
\end{pgfscope}%
\begin{pgfscope}%
\definecolor{textcolor}{rgb}{0.000000,0.000000,0.000000}%
\pgfsetstrokecolor{textcolor}%
\pgfsetfillcolor{textcolor}%
\pgftext[x=5.719911in,y=1.026201in,left,base]{\color{textcolor}{\rmfamily\fontsize{9.000000}{10.800000}\selectfont\catcode`\^=\active\def^{\ifmmode\sp\else\^{}\fi}\catcode`\%=\active\def%{\%}\None{} \& \SharedVertices{}}}%
\end{pgfscope}%
\end{pgfpicture}%
\makeatother%
\endgroup%
}
	\caption[Mean runtime for graphs with no 3 nor 4 cycles (some).]{
		Mean running time (ms) to find all NAC-colorings for graphs with no 3 nor 4 cycles.}%
	\label{fig:graph_count_no_3_nor_4_cycles_first_runtime}
\end{figure}
% \begin{figure}[p]
% 	\centering
% 	\scalebox{0.5}{\input{./figures/graph_export_no_3_nor_4_cycles_first_monochromatic_checks_split_merging_mean.pgf}}
% 	\caption[Checks performed for graphs with no 3 nor 4 cycles (some).]{
% 		The number of checks performed to find all NAC-colorings for graphs with no 3 nor 4 cycles.}%
% 	\label{fig:graph_count_no_3_nor_4_cycles_first_checks}
% \end{figure}
\begin{figure}[p]
	\centering
	\scalebox{0.5}{%% Creator: Matplotlib, PGF backend
%%
%% To include the figure in your LaTeX document, write
%%   \input{<filename>.pgf}
%%
%% Make sure the required packages are loaded in your preamble
%%   \usepackage{pgf}
%%
%% Also ensure that all the required font packages are loaded; for instance,
%% the lmodern package is sometimes necessary when using math font.
%%   \usepackage{lmodern}
%%
%% Figures using additional raster images can only be included by \input if
%% they are in the same directory as the main LaTeX file. For loading figures
%% from other directories you can use the `import` package
%%   \usepackage{import}
%%
%% and then include the figures with
%%   \import{<path to file>}{<filename>.pgf}
%%
%% Matplotlib used the following preamble
%%   \def\mathdefault#1{#1}
%%   \everymath=\expandafter{\the\everymath\displaystyle}
%%   \IfFileExists{scrextend.sty}{
%%     \usepackage[fontsize=10.000000pt]{scrextend}
%%   }{
%%     \renewcommand{\normalsize}{\fontsize{10.000000}{12.000000}\selectfont}
%%     \normalsize
%%   }
%%   
%%   \ifdefined\pdftexversion\else  % non-pdftex case.
%%     \usepackage{fontspec}
%%     \setmainfont{DejaVuSans.ttf}[Path=\detokenize{/home/petr/Projects/PyRigi/.venv/lib/python3.12/site-packages/matplotlib/mpl-data/fonts/ttf/}]
%%     \setsansfont{DejaVuSans.ttf}[Path=\detokenize{/home/petr/Projects/PyRigi/.venv/lib/python3.12/site-packages/matplotlib/mpl-data/fonts/ttf/}]
%%     \setmonofont{DejaVuSansMono.ttf}[Path=\detokenize{/home/petr/Projects/PyRigi/.venv/lib/python3.12/site-packages/matplotlib/mpl-data/fonts/ttf/}]
%%   \fi
%%   \makeatletter\@ifpackageloaded{under\Score{}}{}{\usepackage[strings]{under\Score{}}}\makeatother
%%
\begingroup%
\makeatletter%
\begin{pgfpicture}%
\pgfpathrectangle{\pgfpointorigin}{\pgfqpoint{8.384376in}{2.841860in}}%
\pgfusepath{use as bounding box, clip}%
\begin{pgfscope}%
\pgfsetbuttcap%
\pgfsetmiterjoin%
\definecolor{currentfill}{rgb}{1.000000,1.000000,1.000000}%
\pgfsetfillcolor{currentfill}%
\pgfsetlinewidth{0.000000pt}%
\definecolor{currentstroke}{rgb}{1.000000,1.000000,1.000000}%
\pgfsetstrokecolor{currentstroke}%
\pgfsetdash{}{0pt}%
\pgfpathmoveto{\pgfqpoint{0.000000in}{0.000000in}}%
\pgfpathlineto{\pgfqpoint{8.384376in}{0.000000in}}%
\pgfpathlineto{\pgfqpoint{8.384376in}{2.841860in}}%
\pgfpathlineto{\pgfqpoint{0.000000in}{2.841860in}}%
\pgfpathlineto{\pgfqpoint{0.000000in}{0.000000in}}%
\pgfpathclose%
\pgfusepath{fill}%
\end{pgfscope}%
\begin{pgfscope}%
\pgfsetbuttcap%
\pgfsetmiterjoin%
\definecolor{currentfill}{rgb}{1.000000,1.000000,1.000000}%
\pgfsetfillcolor{currentfill}%
\pgfsetlinewidth{0.000000pt}%
\definecolor{currentstroke}{rgb}{0.000000,0.000000,0.000000}%
\pgfsetstrokecolor{currentstroke}%
\pgfsetstrokeopacity{0.000000}%
\pgfsetdash{}{0pt}%
\pgfpathmoveto{\pgfqpoint{0.588387in}{0.521603in}}%
\pgfpathlineto{\pgfqpoint{4.248423in}{0.521603in}}%
\pgfpathlineto{\pgfqpoint{4.248423in}{2.741376in}}%
\pgfpathlineto{\pgfqpoint{0.588387in}{2.741376in}}%
\pgfpathlineto{\pgfqpoint{0.588387in}{0.521603in}}%
\pgfpathclose%
\pgfusepath{fill}%
\end{pgfscope}%
\begin{pgfscope}%
\pgfsetbuttcap%
\pgfsetroundjoin%
\definecolor{currentfill}{rgb}{0.000000,0.000000,0.000000}%
\pgfsetfillcolor{currentfill}%
\pgfsetlinewidth{0.803000pt}%
\definecolor{currentstroke}{rgb}{0.000000,0.000000,0.000000}%
\pgfsetstrokecolor{currentstroke}%
\pgfsetdash{}{0pt}%
\pgfsys@defobject{currentmarker}{\pgfqpoint{0.000000in}{-0.048611in}}{\pgfqpoint{0.000000in}{0.000000in}}{%
\pgfpathmoveto{\pgfqpoint{0.000000in}{0.000000in}}%
\pgfpathlineto{\pgfqpoint{0.000000in}{-0.048611in}}%
\pgfusepath{stroke,fill}%
}%
\begin{pgfscope}%
\pgfsys@transformshift{0.882726in}{0.521603in}%
\pgfsys@useobject{currentmarker}{}%
\end{pgfscope}%
\end{pgfscope}%
\begin{pgfscope}%
\definecolor{textcolor}{rgb}{0.000000,0.000000,0.000000}%
\pgfsetstrokecolor{textcolor}%
\pgfsetfillcolor{textcolor}%
\pgftext[x=0.882726in,y=0.424381in,,top]{\color{textcolor}{\rmfamily\fontsize{10.000000}{12.000000}\selectfont\catcode`\^=\active\def^{\ifmmode\sp\else\^{}\fi}\catcode`\%=\active\def%{\%}$\mathdefault{9}$}}%
\end{pgfscope}%
\begin{pgfscope}%
\pgfsetbuttcap%
\pgfsetroundjoin%
\definecolor{currentfill}{rgb}{0.000000,0.000000,0.000000}%
\pgfsetfillcolor{currentfill}%
\pgfsetlinewidth{0.803000pt}%
\definecolor{currentstroke}{rgb}{0.000000,0.000000,0.000000}%
\pgfsetstrokecolor{currentstroke}%
\pgfsetdash{}{0pt}%
\pgfsys@defobject{currentmarker}{\pgfqpoint{0.000000in}{-0.048611in}}{\pgfqpoint{0.000000in}{0.000000in}}{%
\pgfpathmoveto{\pgfqpoint{0.000000in}{0.000000in}}%
\pgfpathlineto{\pgfqpoint{0.000000in}{-0.048611in}}%
\pgfusepath{stroke,fill}%
}%
\begin{pgfscope}%
\pgfsys@transformshift{1.266646in}{0.521603in}%
\pgfsys@useobject{currentmarker}{}%
\end{pgfscope}%
\end{pgfscope}%
\begin{pgfscope}%
\definecolor{textcolor}{rgb}{0.000000,0.000000,0.000000}%
\pgfsetstrokecolor{textcolor}%
\pgfsetfillcolor{textcolor}%
\pgftext[x=1.266646in,y=0.424381in,,top]{\color{textcolor}{\rmfamily\fontsize{10.000000}{12.000000}\selectfont\catcode`\^=\active\def^{\ifmmode\sp\else\^{}\fi}\catcode`\%=\active\def%{\%}$\mathdefault{12}$}}%
\end{pgfscope}%
\begin{pgfscope}%
\pgfsetbuttcap%
\pgfsetroundjoin%
\definecolor{currentfill}{rgb}{0.000000,0.000000,0.000000}%
\pgfsetfillcolor{currentfill}%
\pgfsetlinewidth{0.803000pt}%
\definecolor{currentstroke}{rgb}{0.000000,0.000000,0.000000}%
\pgfsetstrokecolor{currentstroke}%
\pgfsetdash{}{0pt}%
\pgfsys@defobject{currentmarker}{\pgfqpoint{0.000000in}{-0.048611in}}{\pgfqpoint{0.000000in}{0.000000in}}{%
\pgfpathmoveto{\pgfqpoint{0.000000in}{0.000000in}}%
\pgfpathlineto{\pgfqpoint{0.000000in}{-0.048611in}}%
\pgfusepath{stroke,fill}%
}%
\begin{pgfscope}%
\pgfsys@transformshift{1.650565in}{0.521603in}%
\pgfsys@useobject{currentmarker}{}%
\end{pgfscope}%
\end{pgfscope}%
\begin{pgfscope}%
\definecolor{textcolor}{rgb}{0.000000,0.000000,0.000000}%
\pgfsetstrokecolor{textcolor}%
\pgfsetfillcolor{textcolor}%
\pgftext[x=1.650565in,y=0.424381in,,top]{\color{textcolor}{\rmfamily\fontsize{10.000000}{12.000000}\selectfont\catcode`\^=\active\def^{\ifmmode\sp\else\^{}\fi}\catcode`\%=\active\def%{\%}$\mathdefault{15}$}}%
\end{pgfscope}%
\begin{pgfscope}%
\pgfsetbuttcap%
\pgfsetroundjoin%
\definecolor{currentfill}{rgb}{0.000000,0.000000,0.000000}%
\pgfsetfillcolor{currentfill}%
\pgfsetlinewidth{0.803000pt}%
\definecolor{currentstroke}{rgb}{0.000000,0.000000,0.000000}%
\pgfsetstrokecolor{currentstroke}%
\pgfsetdash{}{0pt}%
\pgfsys@defobject{currentmarker}{\pgfqpoint{0.000000in}{-0.048611in}}{\pgfqpoint{0.000000in}{0.000000in}}{%
\pgfpathmoveto{\pgfqpoint{0.000000in}{0.000000in}}%
\pgfpathlineto{\pgfqpoint{0.000000in}{-0.048611in}}%
\pgfusepath{stroke,fill}%
}%
\begin{pgfscope}%
\pgfsys@transformshift{2.034485in}{0.521603in}%
\pgfsys@useobject{currentmarker}{}%
\end{pgfscope}%
\end{pgfscope}%
\begin{pgfscope}%
\definecolor{textcolor}{rgb}{0.000000,0.000000,0.000000}%
\pgfsetstrokecolor{textcolor}%
\pgfsetfillcolor{textcolor}%
\pgftext[x=2.034485in,y=0.424381in,,top]{\color{textcolor}{\rmfamily\fontsize{10.000000}{12.000000}\selectfont\catcode`\^=\active\def^{\ifmmode\sp\else\^{}\fi}\catcode`\%=\active\def%{\%}$\mathdefault{18}$}}%
\end{pgfscope}%
\begin{pgfscope}%
\pgfsetbuttcap%
\pgfsetroundjoin%
\definecolor{currentfill}{rgb}{0.000000,0.000000,0.000000}%
\pgfsetfillcolor{currentfill}%
\pgfsetlinewidth{0.803000pt}%
\definecolor{currentstroke}{rgb}{0.000000,0.000000,0.000000}%
\pgfsetstrokecolor{currentstroke}%
\pgfsetdash{}{0pt}%
\pgfsys@defobject{currentmarker}{\pgfqpoint{0.000000in}{-0.048611in}}{\pgfqpoint{0.000000in}{0.000000in}}{%
\pgfpathmoveto{\pgfqpoint{0.000000in}{0.000000in}}%
\pgfpathlineto{\pgfqpoint{0.000000in}{-0.048611in}}%
\pgfusepath{stroke,fill}%
}%
\begin{pgfscope}%
\pgfsys@transformshift{2.418405in}{0.521603in}%
\pgfsys@useobject{currentmarker}{}%
\end{pgfscope}%
\end{pgfscope}%
\begin{pgfscope}%
\definecolor{textcolor}{rgb}{0.000000,0.000000,0.000000}%
\pgfsetstrokecolor{textcolor}%
\pgfsetfillcolor{textcolor}%
\pgftext[x=2.418405in,y=0.424381in,,top]{\color{textcolor}{\rmfamily\fontsize{10.000000}{12.000000}\selectfont\catcode`\^=\active\def^{\ifmmode\sp\else\^{}\fi}\catcode`\%=\active\def%{\%}$\mathdefault{21}$}}%
\end{pgfscope}%
\begin{pgfscope}%
\pgfsetbuttcap%
\pgfsetroundjoin%
\definecolor{currentfill}{rgb}{0.000000,0.000000,0.000000}%
\pgfsetfillcolor{currentfill}%
\pgfsetlinewidth{0.803000pt}%
\definecolor{currentstroke}{rgb}{0.000000,0.000000,0.000000}%
\pgfsetstrokecolor{currentstroke}%
\pgfsetdash{}{0pt}%
\pgfsys@defobject{currentmarker}{\pgfqpoint{0.000000in}{-0.048611in}}{\pgfqpoint{0.000000in}{0.000000in}}{%
\pgfpathmoveto{\pgfqpoint{0.000000in}{0.000000in}}%
\pgfpathlineto{\pgfqpoint{0.000000in}{-0.048611in}}%
\pgfusepath{stroke,fill}%
}%
\begin{pgfscope}%
\pgfsys@transformshift{2.802325in}{0.521603in}%
\pgfsys@useobject{currentmarker}{}%
\end{pgfscope}%
\end{pgfscope}%
\begin{pgfscope}%
\definecolor{textcolor}{rgb}{0.000000,0.000000,0.000000}%
\pgfsetstrokecolor{textcolor}%
\pgfsetfillcolor{textcolor}%
\pgftext[x=2.802325in,y=0.424381in,,top]{\color{textcolor}{\rmfamily\fontsize{10.000000}{12.000000}\selectfont\catcode`\^=\active\def^{\ifmmode\sp\else\^{}\fi}\catcode`\%=\active\def%{\%}$\mathdefault{24}$}}%
\end{pgfscope}%
\begin{pgfscope}%
\pgfsetbuttcap%
\pgfsetroundjoin%
\definecolor{currentfill}{rgb}{0.000000,0.000000,0.000000}%
\pgfsetfillcolor{currentfill}%
\pgfsetlinewidth{0.803000pt}%
\definecolor{currentstroke}{rgb}{0.000000,0.000000,0.000000}%
\pgfsetstrokecolor{currentstroke}%
\pgfsetdash{}{0pt}%
\pgfsys@defobject{currentmarker}{\pgfqpoint{0.000000in}{-0.048611in}}{\pgfqpoint{0.000000in}{0.000000in}}{%
\pgfpathmoveto{\pgfqpoint{0.000000in}{0.000000in}}%
\pgfpathlineto{\pgfqpoint{0.000000in}{-0.048611in}}%
\pgfusepath{stroke,fill}%
}%
\begin{pgfscope}%
\pgfsys@transformshift{3.186245in}{0.521603in}%
\pgfsys@useobject{currentmarker}{}%
\end{pgfscope}%
\end{pgfscope}%
\begin{pgfscope}%
\definecolor{textcolor}{rgb}{0.000000,0.000000,0.000000}%
\pgfsetstrokecolor{textcolor}%
\pgfsetfillcolor{textcolor}%
\pgftext[x=3.186245in,y=0.424381in,,top]{\color{textcolor}{\rmfamily\fontsize{10.000000}{12.000000}\selectfont\catcode`\^=\active\def^{\ifmmode\sp\else\^{}\fi}\catcode`\%=\active\def%{\%}$\mathdefault{27}$}}%
\end{pgfscope}%
\begin{pgfscope}%
\pgfsetbuttcap%
\pgfsetroundjoin%
\definecolor{currentfill}{rgb}{0.000000,0.000000,0.000000}%
\pgfsetfillcolor{currentfill}%
\pgfsetlinewidth{0.803000pt}%
\definecolor{currentstroke}{rgb}{0.000000,0.000000,0.000000}%
\pgfsetstrokecolor{currentstroke}%
\pgfsetdash{}{0pt}%
\pgfsys@defobject{currentmarker}{\pgfqpoint{0.000000in}{-0.048611in}}{\pgfqpoint{0.000000in}{0.000000in}}{%
\pgfpathmoveto{\pgfqpoint{0.000000in}{0.000000in}}%
\pgfpathlineto{\pgfqpoint{0.000000in}{-0.048611in}}%
\pgfusepath{stroke,fill}%
}%
\begin{pgfscope}%
\pgfsys@transformshift{3.570164in}{0.521603in}%
\pgfsys@useobject{currentmarker}{}%
\end{pgfscope}%
\end{pgfscope}%
\begin{pgfscope}%
\definecolor{textcolor}{rgb}{0.000000,0.000000,0.000000}%
\pgfsetstrokecolor{textcolor}%
\pgfsetfillcolor{textcolor}%
\pgftext[x=3.570164in,y=0.424381in,,top]{\color{textcolor}{\rmfamily\fontsize{10.000000}{12.000000}\selectfont\catcode`\^=\active\def^{\ifmmode\sp\else\^{}\fi}\catcode`\%=\active\def%{\%}$\mathdefault{30}$}}%
\end{pgfscope}%
\begin{pgfscope}%
\pgfsetbuttcap%
\pgfsetroundjoin%
\definecolor{currentfill}{rgb}{0.000000,0.000000,0.000000}%
\pgfsetfillcolor{currentfill}%
\pgfsetlinewidth{0.803000pt}%
\definecolor{currentstroke}{rgb}{0.000000,0.000000,0.000000}%
\pgfsetstrokecolor{currentstroke}%
\pgfsetdash{}{0pt}%
\pgfsys@defobject{currentmarker}{\pgfqpoint{0.000000in}{-0.048611in}}{\pgfqpoint{0.000000in}{0.000000in}}{%
\pgfpathmoveto{\pgfqpoint{0.000000in}{0.000000in}}%
\pgfpathlineto{\pgfqpoint{0.000000in}{-0.048611in}}%
\pgfusepath{stroke,fill}%
}%
\begin{pgfscope}%
\pgfsys@transformshift{3.954084in}{0.521603in}%
\pgfsys@useobject{currentmarker}{}%
\end{pgfscope}%
\end{pgfscope}%
\begin{pgfscope}%
\definecolor{textcolor}{rgb}{0.000000,0.000000,0.000000}%
\pgfsetstrokecolor{textcolor}%
\pgfsetfillcolor{textcolor}%
\pgftext[x=3.954084in,y=0.424381in,,top]{\color{textcolor}{\rmfamily\fontsize{10.000000}{12.000000}\selectfont\catcode`\^=\active\def^{\ifmmode\sp\else\^{}\fi}\catcode`\%=\active\def%{\%}$\mathdefault{33}$}}%
\end{pgfscope}%
\begin{pgfscope}%
\definecolor{textcolor}{rgb}{0.000000,0.000000,0.000000}%
\pgfsetstrokecolor{textcolor}%
\pgfsetfillcolor{textcolor}%
\pgftext[x=2.418405in,y=0.234413in,,top]{\color{textcolor}{\rmfamily\fontsize{10.000000}{12.000000}\selectfont\catcode`\^=\active\def^{\ifmmode\sp\else\^{}\fi}\catcode`\%=\active\def%{\%}Monochromatic classes}}%
\end{pgfscope}%
\begin{pgfscope}%
\pgfsetbuttcap%
\pgfsetroundjoin%
\definecolor{currentfill}{rgb}{0.000000,0.000000,0.000000}%
\pgfsetfillcolor{currentfill}%
\pgfsetlinewidth{0.803000pt}%
\definecolor{currentstroke}{rgb}{0.000000,0.000000,0.000000}%
\pgfsetstrokecolor{currentstroke}%
\pgfsetdash{}{0pt}%
\pgfsys@defobject{currentmarker}{\pgfqpoint{-0.048611in}{0.000000in}}{\pgfqpoint{-0.000000in}{0.000000in}}{%
\pgfpathmoveto{\pgfqpoint{-0.000000in}{0.000000in}}%
\pgfpathlineto{\pgfqpoint{-0.048611in}{0.000000in}}%
\pgfusepath{stroke,fill}%
}%
\begin{pgfscope}%
\pgfsys@transformshift{0.588387in}{0.810961in}%
\pgfsys@useobject{currentmarker}{}%
\end{pgfscope}%
\end{pgfscope}%
\begin{pgfscope}%
\definecolor{textcolor}{rgb}{0.000000,0.000000,0.000000}%
\pgfsetstrokecolor{textcolor}%
\pgfsetfillcolor{textcolor}%
\pgftext[x=0.289968in, y=0.758199in, left, base]{\color{textcolor}{\rmfamily\fontsize{10.000000}{12.000000}\selectfont\catcode`\^=\active\def^{\ifmmode\sp\else\^{}\fi}\catcode`\%=\active\def%{\%}$\mathdefault{10^{1}}$}}%
\end{pgfscope}%
\begin{pgfscope}%
\pgfsetbuttcap%
\pgfsetroundjoin%
\definecolor{currentfill}{rgb}{0.000000,0.000000,0.000000}%
\pgfsetfillcolor{currentfill}%
\pgfsetlinewidth{0.803000pt}%
\definecolor{currentstroke}{rgb}{0.000000,0.000000,0.000000}%
\pgfsetstrokecolor{currentstroke}%
\pgfsetdash{}{0pt}%
\pgfsys@defobject{currentmarker}{\pgfqpoint{-0.048611in}{0.000000in}}{\pgfqpoint{-0.000000in}{0.000000in}}{%
\pgfpathmoveto{\pgfqpoint{-0.000000in}{0.000000in}}%
\pgfpathlineto{\pgfqpoint{-0.048611in}{0.000000in}}%
\pgfusepath{stroke,fill}%
}%
\begin{pgfscope}%
\pgfsys@transformshift{0.588387in}{1.437007in}%
\pgfsys@useobject{currentmarker}{}%
\end{pgfscope}%
\end{pgfscope}%
\begin{pgfscope}%
\definecolor{textcolor}{rgb}{0.000000,0.000000,0.000000}%
\pgfsetstrokecolor{textcolor}%
\pgfsetfillcolor{textcolor}%
\pgftext[x=0.289968in, y=1.384245in, left, base]{\color{textcolor}{\rmfamily\fontsize{10.000000}{12.000000}\selectfont\catcode`\^=\active\def^{\ifmmode\sp\else\^{}\fi}\catcode`\%=\active\def%{\%}$\mathdefault{10^{2}}$}}%
\end{pgfscope}%
\begin{pgfscope}%
\pgfsetbuttcap%
\pgfsetroundjoin%
\definecolor{currentfill}{rgb}{0.000000,0.000000,0.000000}%
\pgfsetfillcolor{currentfill}%
\pgfsetlinewidth{0.803000pt}%
\definecolor{currentstroke}{rgb}{0.000000,0.000000,0.000000}%
\pgfsetstrokecolor{currentstroke}%
\pgfsetdash{}{0pt}%
\pgfsys@defobject{currentmarker}{\pgfqpoint{-0.048611in}{0.000000in}}{\pgfqpoint{-0.000000in}{0.000000in}}{%
\pgfpathmoveto{\pgfqpoint{-0.000000in}{0.000000in}}%
\pgfpathlineto{\pgfqpoint{-0.048611in}{0.000000in}}%
\pgfusepath{stroke,fill}%
}%
\begin{pgfscope}%
\pgfsys@transformshift{0.588387in}{2.063053in}%
\pgfsys@useobject{currentmarker}{}%
\end{pgfscope}%
\end{pgfscope}%
\begin{pgfscope}%
\definecolor{textcolor}{rgb}{0.000000,0.000000,0.000000}%
\pgfsetstrokecolor{textcolor}%
\pgfsetfillcolor{textcolor}%
\pgftext[x=0.289968in, y=2.010291in, left, base]{\color{textcolor}{\rmfamily\fontsize{10.000000}{12.000000}\selectfont\catcode`\^=\active\def^{\ifmmode\sp\else\^{}\fi}\catcode`\%=\active\def%{\%}$\mathdefault{10^{3}}$}}%
\end{pgfscope}%
\begin{pgfscope}%
\pgfsetbuttcap%
\pgfsetroundjoin%
\definecolor{currentfill}{rgb}{0.000000,0.000000,0.000000}%
\pgfsetfillcolor{currentfill}%
\pgfsetlinewidth{0.803000pt}%
\definecolor{currentstroke}{rgb}{0.000000,0.000000,0.000000}%
\pgfsetstrokecolor{currentstroke}%
\pgfsetdash{}{0pt}%
\pgfsys@defobject{currentmarker}{\pgfqpoint{-0.048611in}{0.000000in}}{\pgfqpoint{-0.000000in}{0.000000in}}{%
\pgfpathmoveto{\pgfqpoint{-0.000000in}{0.000000in}}%
\pgfpathlineto{\pgfqpoint{-0.048611in}{0.000000in}}%
\pgfusepath{stroke,fill}%
}%
\begin{pgfscope}%
\pgfsys@transformshift{0.588387in}{2.689099in}%
\pgfsys@useobject{currentmarker}{}%
\end{pgfscope}%
\end{pgfscope}%
\begin{pgfscope}%
\definecolor{textcolor}{rgb}{0.000000,0.000000,0.000000}%
\pgfsetstrokecolor{textcolor}%
\pgfsetfillcolor{textcolor}%
\pgftext[x=0.289968in, y=2.636337in, left, base]{\color{textcolor}{\rmfamily\fontsize{10.000000}{12.000000}\selectfont\catcode`\^=\active\def^{\ifmmode\sp\else\^{}\fi}\catcode`\%=\active\def%{\%}$\mathdefault{10^{4}}$}}%
\end{pgfscope}%
\begin{pgfscope}%
\pgfsetbuttcap%
\pgfsetroundjoin%
\definecolor{currentfill}{rgb}{0.000000,0.000000,0.000000}%
\pgfsetfillcolor{currentfill}%
\pgfsetlinewidth{0.602250pt}%
\definecolor{currentstroke}{rgb}{0.000000,0.000000,0.000000}%
\pgfsetstrokecolor{currentstroke}%
\pgfsetdash{}{0pt}%
\pgfsys@defobject{currentmarker}{\pgfqpoint{-0.027778in}{0.000000in}}{\pgfqpoint{-0.000000in}{0.000000in}}{%
\pgfpathmoveto{\pgfqpoint{-0.000000in}{0.000000in}}%
\pgfpathlineto{\pgfqpoint{-0.027778in}{0.000000in}}%
\pgfusepath{stroke,fill}%
}%
\begin{pgfscope}%
\pgfsys@transformshift{0.588387in}{0.561832in}%
\pgfsys@useobject{currentmarker}{}%
\end{pgfscope}%
\end{pgfscope}%
\begin{pgfscope}%
\pgfsetbuttcap%
\pgfsetroundjoin%
\definecolor{currentfill}{rgb}{0.000000,0.000000,0.000000}%
\pgfsetfillcolor{currentfill}%
\pgfsetlinewidth{0.602250pt}%
\definecolor{currentstroke}{rgb}{0.000000,0.000000,0.000000}%
\pgfsetstrokecolor{currentstroke}%
\pgfsetdash{}{0pt}%
\pgfsys@defobject{currentmarker}{\pgfqpoint{-0.027778in}{0.000000in}}{\pgfqpoint{-0.000000in}{0.000000in}}{%
\pgfpathmoveto{\pgfqpoint{-0.000000in}{0.000000in}}%
\pgfpathlineto{\pgfqpoint{-0.027778in}{0.000000in}}%
\pgfusepath{stroke,fill}%
}%
\begin{pgfscope}%
\pgfsys@transformshift{0.588387in}{0.622502in}%
\pgfsys@useobject{currentmarker}{}%
\end{pgfscope}%
\end{pgfscope}%
\begin{pgfscope}%
\pgfsetbuttcap%
\pgfsetroundjoin%
\definecolor{currentfill}{rgb}{0.000000,0.000000,0.000000}%
\pgfsetfillcolor{currentfill}%
\pgfsetlinewidth{0.602250pt}%
\definecolor{currentstroke}{rgb}{0.000000,0.000000,0.000000}%
\pgfsetstrokecolor{currentstroke}%
\pgfsetdash{}{0pt}%
\pgfsys@defobject{currentmarker}{\pgfqpoint{-0.027778in}{0.000000in}}{\pgfqpoint{-0.000000in}{0.000000in}}{%
\pgfpathmoveto{\pgfqpoint{-0.000000in}{0.000000in}}%
\pgfpathlineto{\pgfqpoint{-0.027778in}{0.000000in}}%
\pgfusepath{stroke,fill}%
}%
\begin{pgfscope}%
\pgfsys@transformshift{0.588387in}{0.672073in}%
\pgfsys@useobject{currentmarker}{}%
\end{pgfscope}%
\end{pgfscope}%
\begin{pgfscope}%
\pgfsetbuttcap%
\pgfsetroundjoin%
\definecolor{currentfill}{rgb}{0.000000,0.000000,0.000000}%
\pgfsetfillcolor{currentfill}%
\pgfsetlinewidth{0.602250pt}%
\definecolor{currentstroke}{rgb}{0.000000,0.000000,0.000000}%
\pgfsetstrokecolor{currentstroke}%
\pgfsetdash{}{0pt}%
\pgfsys@defobject{currentmarker}{\pgfqpoint{-0.027778in}{0.000000in}}{\pgfqpoint{-0.000000in}{0.000000in}}{%
\pgfpathmoveto{\pgfqpoint{-0.000000in}{0.000000in}}%
\pgfpathlineto{\pgfqpoint{-0.027778in}{0.000000in}}%
\pgfusepath{stroke,fill}%
}%
\begin{pgfscope}%
\pgfsys@transformshift{0.588387in}{0.713985in}%
\pgfsys@useobject{currentmarker}{}%
\end{pgfscope}%
\end{pgfscope}%
\begin{pgfscope}%
\pgfsetbuttcap%
\pgfsetroundjoin%
\definecolor{currentfill}{rgb}{0.000000,0.000000,0.000000}%
\pgfsetfillcolor{currentfill}%
\pgfsetlinewidth{0.602250pt}%
\definecolor{currentstroke}{rgb}{0.000000,0.000000,0.000000}%
\pgfsetstrokecolor{currentstroke}%
\pgfsetdash{}{0pt}%
\pgfsys@defobject{currentmarker}{\pgfqpoint{-0.027778in}{0.000000in}}{\pgfqpoint{-0.000000in}{0.000000in}}{%
\pgfpathmoveto{\pgfqpoint{-0.000000in}{0.000000in}}%
\pgfpathlineto{\pgfqpoint{-0.027778in}{0.000000in}}%
\pgfusepath{stroke,fill}%
}%
\begin{pgfscope}%
\pgfsys@transformshift{0.588387in}{0.750291in}%
\pgfsys@useobject{currentmarker}{}%
\end{pgfscope}%
\end{pgfscope}%
\begin{pgfscope}%
\pgfsetbuttcap%
\pgfsetroundjoin%
\definecolor{currentfill}{rgb}{0.000000,0.000000,0.000000}%
\pgfsetfillcolor{currentfill}%
\pgfsetlinewidth{0.602250pt}%
\definecolor{currentstroke}{rgb}{0.000000,0.000000,0.000000}%
\pgfsetstrokecolor{currentstroke}%
\pgfsetdash{}{0pt}%
\pgfsys@defobject{currentmarker}{\pgfqpoint{-0.027778in}{0.000000in}}{\pgfqpoint{-0.000000in}{0.000000in}}{%
\pgfpathmoveto{\pgfqpoint{-0.000000in}{0.000000in}}%
\pgfpathlineto{\pgfqpoint{-0.027778in}{0.000000in}}%
\pgfusepath{stroke,fill}%
}%
\begin{pgfscope}%
\pgfsys@transformshift{0.588387in}{0.782314in}%
\pgfsys@useobject{currentmarker}{}%
\end{pgfscope}%
\end{pgfscope}%
\begin{pgfscope}%
\pgfsetbuttcap%
\pgfsetroundjoin%
\definecolor{currentfill}{rgb}{0.000000,0.000000,0.000000}%
\pgfsetfillcolor{currentfill}%
\pgfsetlinewidth{0.602250pt}%
\definecolor{currentstroke}{rgb}{0.000000,0.000000,0.000000}%
\pgfsetstrokecolor{currentstroke}%
\pgfsetdash{}{0pt}%
\pgfsys@defobject{currentmarker}{\pgfqpoint{-0.027778in}{0.000000in}}{\pgfqpoint{-0.000000in}{0.000000in}}{%
\pgfpathmoveto{\pgfqpoint{-0.000000in}{0.000000in}}%
\pgfpathlineto{\pgfqpoint{-0.027778in}{0.000000in}}%
\pgfusepath{stroke,fill}%
}%
\begin{pgfscope}%
\pgfsys@transformshift{0.588387in}{0.999419in}%
\pgfsys@useobject{currentmarker}{}%
\end{pgfscope}%
\end{pgfscope}%
\begin{pgfscope}%
\pgfsetbuttcap%
\pgfsetroundjoin%
\definecolor{currentfill}{rgb}{0.000000,0.000000,0.000000}%
\pgfsetfillcolor{currentfill}%
\pgfsetlinewidth{0.602250pt}%
\definecolor{currentstroke}{rgb}{0.000000,0.000000,0.000000}%
\pgfsetstrokecolor{currentstroke}%
\pgfsetdash{}{0pt}%
\pgfsys@defobject{currentmarker}{\pgfqpoint{-0.027778in}{0.000000in}}{\pgfqpoint{-0.000000in}{0.000000in}}{%
\pgfpathmoveto{\pgfqpoint{-0.000000in}{0.000000in}}%
\pgfpathlineto{\pgfqpoint{-0.027778in}{0.000000in}}%
\pgfusepath{stroke,fill}%
}%
\begin{pgfscope}%
\pgfsys@transformshift{0.588387in}{1.109661in}%
\pgfsys@useobject{currentmarker}{}%
\end{pgfscope}%
\end{pgfscope}%
\begin{pgfscope}%
\pgfsetbuttcap%
\pgfsetroundjoin%
\definecolor{currentfill}{rgb}{0.000000,0.000000,0.000000}%
\pgfsetfillcolor{currentfill}%
\pgfsetlinewidth{0.602250pt}%
\definecolor{currentstroke}{rgb}{0.000000,0.000000,0.000000}%
\pgfsetstrokecolor{currentstroke}%
\pgfsetdash{}{0pt}%
\pgfsys@defobject{currentmarker}{\pgfqpoint{-0.027778in}{0.000000in}}{\pgfqpoint{-0.000000in}{0.000000in}}{%
\pgfpathmoveto{\pgfqpoint{-0.000000in}{0.000000in}}%
\pgfpathlineto{\pgfqpoint{-0.027778in}{0.000000in}}%
\pgfusepath{stroke,fill}%
}%
\begin{pgfscope}%
\pgfsys@transformshift{0.588387in}{1.187878in}%
\pgfsys@useobject{currentmarker}{}%
\end{pgfscope}%
\end{pgfscope}%
\begin{pgfscope}%
\pgfsetbuttcap%
\pgfsetroundjoin%
\definecolor{currentfill}{rgb}{0.000000,0.000000,0.000000}%
\pgfsetfillcolor{currentfill}%
\pgfsetlinewidth{0.602250pt}%
\definecolor{currentstroke}{rgb}{0.000000,0.000000,0.000000}%
\pgfsetstrokecolor{currentstroke}%
\pgfsetdash{}{0pt}%
\pgfsys@defobject{currentmarker}{\pgfqpoint{-0.027778in}{0.000000in}}{\pgfqpoint{-0.000000in}{0.000000in}}{%
\pgfpathmoveto{\pgfqpoint{-0.000000in}{0.000000in}}%
\pgfpathlineto{\pgfqpoint{-0.027778in}{0.000000in}}%
\pgfusepath{stroke,fill}%
}%
\begin{pgfscope}%
\pgfsys@transformshift{0.588387in}{1.248548in}%
\pgfsys@useobject{currentmarker}{}%
\end{pgfscope}%
\end{pgfscope}%
\begin{pgfscope}%
\pgfsetbuttcap%
\pgfsetroundjoin%
\definecolor{currentfill}{rgb}{0.000000,0.000000,0.000000}%
\pgfsetfillcolor{currentfill}%
\pgfsetlinewidth{0.602250pt}%
\definecolor{currentstroke}{rgb}{0.000000,0.000000,0.000000}%
\pgfsetstrokecolor{currentstroke}%
\pgfsetdash{}{0pt}%
\pgfsys@defobject{currentmarker}{\pgfqpoint{-0.027778in}{0.000000in}}{\pgfqpoint{-0.000000in}{0.000000in}}{%
\pgfpathmoveto{\pgfqpoint{-0.000000in}{0.000000in}}%
\pgfpathlineto{\pgfqpoint{-0.027778in}{0.000000in}}%
\pgfusepath{stroke,fill}%
}%
\begin{pgfscope}%
\pgfsys@transformshift{0.588387in}{1.298119in}%
\pgfsys@useobject{currentmarker}{}%
\end{pgfscope}%
\end{pgfscope}%
\begin{pgfscope}%
\pgfsetbuttcap%
\pgfsetroundjoin%
\definecolor{currentfill}{rgb}{0.000000,0.000000,0.000000}%
\pgfsetfillcolor{currentfill}%
\pgfsetlinewidth{0.602250pt}%
\definecolor{currentstroke}{rgb}{0.000000,0.000000,0.000000}%
\pgfsetstrokecolor{currentstroke}%
\pgfsetdash{}{0pt}%
\pgfsys@defobject{currentmarker}{\pgfqpoint{-0.027778in}{0.000000in}}{\pgfqpoint{-0.000000in}{0.000000in}}{%
\pgfpathmoveto{\pgfqpoint{-0.000000in}{0.000000in}}%
\pgfpathlineto{\pgfqpoint{-0.027778in}{0.000000in}}%
\pgfusepath{stroke,fill}%
}%
\begin{pgfscope}%
\pgfsys@transformshift{0.588387in}{1.340031in}%
\pgfsys@useobject{currentmarker}{}%
\end{pgfscope}%
\end{pgfscope}%
\begin{pgfscope}%
\pgfsetbuttcap%
\pgfsetroundjoin%
\definecolor{currentfill}{rgb}{0.000000,0.000000,0.000000}%
\pgfsetfillcolor{currentfill}%
\pgfsetlinewidth{0.602250pt}%
\definecolor{currentstroke}{rgb}{0.000000,0.000000,0.000000}%
\pgfsetstrokecolor{currentstroke}%
\pgfsetdash{}{0pt}%
\pgfsys@defobject{currentmarker}{\pgfqpoint{-0.027778in}{0.000000in}}{\pgfqpoint{-0.000000in}{0.000000in}}{%
\pgfpathmoveto{\pgfqpoint{-0.000000in}{0.000000in}}%
\pgfpathlineto{\pgfqpoint{-0.027778in}{0.000000in}}%
\pgfusepath{stroke,fill}%
}%
\begin{pgfscope}%
\pgfsys@transformshift{0.588387in}{1.376337in}%
\pgfsys@useobject{currentmarker}{}%
\end{pgfscope}%
\end{pgfscope}%
\begin{pgfscope}%
\pgfsetbuttcap%
\pgfsetroundjoin%
\definecolor{currentfill}{rgb}{0.000000,0.000000,0.000000}%
\pgfsetfillcolor{currentfill}%
\pgfsetlinewidth{0.602250pt}%
\definecolor{currentstroke}{rgb}{0.000000,0.000000,0.000000}%
\pgfsetstrokecolor{currentstroke}%
\pgfsetdash{}{0pt}%
\pgfsys@defobject{currentmarker}{\pgfqpoint{-0.027778in}{0.000000in}}{\pgfqpoint{-0.000000in}{0.000000in}}{%
\pgfpathmoveto{\pgfqpoint{-0.000000in}{0.000000in}}%
\pgfpathlineto{\pgfqpoint{-0.027778in}{0.000000in}}%
\pgfusepath{stroke,fill}%
}%
\begin{pgfscope}%
\pgfsys@transformshift{0.588387in}{1.408360in}%
\pgfsys@useobject{currentmarker}{}%
\end{pgfscope}%
\end{pgfscope}%
\begin{pgfscope}%
\pgfsetbuttcap%
\pgfsetroundjoin%
\definecolor{currentfill}{rgb}{0.000000,0.000000,0.000000}%
\pgfsetfillcolor{currentfill}%
\pgfsetlinewidth{0.602250pt}%
\definecolor{currentstroke}{rgb}{0.000000,0.000000,0.000000}%
\pgfsetstrokecolor{currentstroke}%
\pgfsetdash{}{0pt}%
\pgfsys@defobject{currentmarker}{\pgfqpoint{-0.027778in}{0.000000in}}{\pgfqpoint{-0.000000in}{0.000000in}}{%
\pgfpathmoveto{\pgfqpoint{-0.000000in}{0.000000in}}%
\pgfpathlineto{\pgfqpoint{-0.027778in}{0.000000in}}%
\pgfusepath{stroke,fill}%
}%
\begin{pgfscope}%
\pgfsys@transformshift{0.588387in}{1.625465in}%
\pgfsys@useobject{currentmarker}{}%
\end{pgfscope}%
\end{pgfscope}%
\begin{pgfscope}%
\pgfsetbuttcap%
\pgfsetroundjoin%
\definecolor{currentfill}{rgb}{0.000000,0.000000,0.000000}%
\pgfsetfillcolor{currentfill}%
\pgfsetlinewidth{0.602250pt}%
\definecolor{currentstroke}{rgb}{0.000000,0.000000,0.000000}%
\pgfsetstrokecolor{currentstroke}%
\pgfsetdash{}{0pt}%
\pgfsys@defobject{currentmarker}{\pgfqpoint{-0.027778in}{0.000000in}}{\pgfqpoint{-0.000000in}{0.000000in}}{%
\pgfpathmoveto{\pgfqpoint{-0.000000in}{0.000000in}}%
\pgfpathlineto{\pgfqpoint{-0.027778in}{0.000000in}}%
\pgfusepath{stroke,fill}%
}%
\begin{pgfscope}%
\pgfsys@transformshift{0.588387in}{1.735707in}%
\pgfsys@useobject{currentmarker}{}%
\end{pgfscope}%
\end{pgfscope}%
\begin{pgfscope}%
\pgfsetbuttcap%
\pgfsetroundjoin%
\definecolor{currentfill}{rgb}{0.000000,0.000000,0.000000}%
\pgfsetfillcolor{currentfill}%
\pgfsetlinewidth{0.602250pt}%
\definecolor{currentstroke}{rgb}{0.000000,0.000000,0.000000}%
\pgfsetstrokecolor{currentstroke}%
\pgfsetdash{}{0pt}%
\pgfsys@defobject{currentmarker}{\pgfqpoint{-0.027778in}{0.000000in}}{\pgfqpoint{-0.000000in}{0.000000in}}{%
\pgfpathmoveto{\pgfqpoint{-0.000000in}{0.000000in}}%
\pgfpathlineto{\pgfqpoint{-0.027778in}{0.000000in}}%
\pgfusepath{stroke,fill}%
}%
\begin{pgfscope}%
\pgfsys@transformshift{0.588387in}{1.813924in}%
\pgfsys@useobject{currentmarker}{}%
\end{pgfscope}%
\end{pgfscope}%
\begin{pgfscope}%
\pgfsetbuttcap%
\pgfsetroundjoin%
\definecolor{currentfill}{rgb}{0.000000,0.000000,0.000000}%
\pgfsetfillcolor{currentfill}%
\pgfsetlinewidth{0.602250pt}%
\definecolor{currentstroke}{rgb}{0.000000,0.000000,0.000000}%
\pgfsetstrokecolor{currentstroke}%
\pgfsetdash{}{0pt}%
\pgfsys@defobject{currentmarker}{\pgfqpoint{-0.027778in}{0.000000in}}{\pgfqpoint{-0.000000in}{0.000000in}}{%
\pgfpathmoveto{\pgfqpoint{-0.000000in}{0.000000in}}%
\pgfpathlineto{\pgfqpoint{-0.027778in}{0.000000in}}%
\pgfusepath{stroke,fill}%
}%
\begin{pgfscope}%
\pgfsys@transformshift{0.588387in}{1.874594in}%
\pgfsys@useobject{currentmarker}{}%
\end{pgfscope}%
\end{pgfscope}%
\begin{pgfscope}%
\pgfsetbuttcap%
\pgfsetroundjoin%
\definecolor{currentfill}{rgb}{0.000000,0.000000,0.000000}%
\pgfsetfillcolor{currentfill}%
\pgfsetlinewidth{0.602250pt}%
\definecolor{currentstroke}{rgb}{0.000000,0.000000,0.000000}%
\pgfsetstrokecolor{currentstroke}%
\pgfsetdash{}{0pt}%
\pgfsys@defobject{currentmarker}{\pgfqpoint{-0.027778in}{0.000000in}}{\pgfqpoint{-0.000000in}{0.000000in}}{%
\pgfpathmoveto{\pgfqpoint{-0.000000in}{0.000000in}}%
\pgfpathlineto{\pgfqpoint{-0.027778in}{0.000000in}}%
\pgfusepath{stroke,fill}%
}%
\begin{pgfscope}%
\pgfsys@transformshift{0.588387in}{1.924165in}%
\pgfsys@useobject{currentmarker}{}%
\end{pgfscope}%
\end{pgfscope}%
\begin{pgfscope}%
\pgfsetbuttcap%
\pgfsetroundjoin%
\definecolor{currentfill}{rgb}{0.000000,0.000000,0.000000}%
\pgfsetfillcolor{currentfill}%
\pgfsetlinewidth{0.602250pt}%
\definecolor{currentstroke}{rgb}{0.000000,0.000000,0.000000}%
\pgfsetstrokecolor{currentstroke}%
\pgfsetdash{}{0pt}%
\pgfsys@defobject{currentmarker}{\pgfqpoint{-0.027778in}{0.000000in}}{\pgfqpoint{-0.000000in}{0.000000in}}{%
\pgfpathmoveto{\pgfqpoint{-0.000000in}{0.000000in}}%
\pgfpathlineto{\pgfqpoint{-0.027778in}{0.000000in}}%
\pgfusepath{stroke,fill}%
}%
\begin{pgfscope}%
\pgfsys@transformshift{0.588387in}{1.966077in}%
\pgfsys@useobject{currentmarker}{}%
\end{pgfscope}%
\end{pgfscope}%
\begin{pgfscope}%
\pgfsetbuttcap%
\pgfsetroundjoin%
\definecolor{currentfill}{rgb}{0.000000,0.000000,0.000000}%
\pgfsetfillcolor{currentfill}%
\pgfsetlinewidth{0.602250pt}%
\definecolor{currentstroke}{rgb}{0.000000,0.000000,0.000000}%
\pgfsetstrokecolor{currentstroke}%
\pgfsetdash{}{0pt}%
\pgfsys@defobject{currentmarker}{\pgfqpoint{-0.027778in}{0.000000in}}{\pgfqpoint{-0.000000in}{0.000000in}}{%
\pgfpathmoveto{\pgfqpoint{-0.000000in}{0.000000in}}%
\pgfpathlineto{\pgfqpoint{-0.027778in}{0.000000in}}%
\pgfusepath{stroke,fill}%
}%
\begin{pgfscope}%
\pgfsys@transformshift{0.588387in}{2.002383in}%
\pgfsys@useobject{currentmarker}{}%
\end{pgfscope}%
\end{pgfscope}%
\begin{pgfscope}%
\pgfsetbuttcap%
\pgfsetroundjoin%
\definecolor{currentfill}{rgb}{0.000000,0.000000,0.000000}%
\pgfsetfillcolor{currentfill}%
\pgfsetlinewidth{0.602250pt}%
\definecolor{currentstroke}{rgb}{0.000000,0.000000,0.000000}%
\pgfsetstrokecolor{currentstroke}%
\pgfsetdash{}{0pt}%
\pgfsys@defobject{currentmarker}{\pgfqpoint{-0.027778in}{0.000000in}}{\pgfqpoint{-0.000000in}{0.000000in}}{%
\pgfpathmoveto{\pgfqpoint{-0.000000in}{0.000000in}}%
\pgfpathlineto{\pgfqpoint{-0.027778in}{0.000000in}}%
\pgfusepath{stroke,fill}%
}%
\begin{pgfscope}%
\pgfsys@transformshift{0.588387in}{2.034407in}%
\pgfsys@useobject{currentmarker}{}%
\end{pgfscope}%
\end{pgfscope}%
\begin{pgfscope}%
\pgfsetbuttcap%
\pgfsetroundjoin%
\definecolor{currentfill}{rgb}{0.000000,0.000000,0.000000}%
\pgfsetfillcolor{currentfill}%
\pgfsetlinewidth{0.602250pt}%
\definecolor{currentstroke}{rgb}{0.000000,0.000000,0.000000}%
\pgfsetstrokecolor{currentstroke}%
\pgfsetdash{}{0pt}%
\pgfsys@defobject{currentmarker}{\pgfqpoint{-0.027778in}{0.000000in}}{\pgfqpoint{-0.000000in}{0.000000in}}{%
\pgfpathmoveto{\pgfqpoint{-0.000000in}{0.000000in}}%
\pgfpathlineto{\pgfqpoint{-0.027778in}{0.000000in}}%
\pgfusepath{stroke,fill}%
}%
\begin{pgfscope}%
\pgfsys@transformshift{0.588387in}{2.251511in}%
\pgfsys@useobject{currentmarker}{}%
\end{pgfscope}%
\end{pgfscope}%
\begin{pgfscope}%
\pgfsetbuttcap%
\pgfsetroundjoin%
\definecolor{currentfill}{rgb}{0.000000,0.000000,0.000000}%
\pgfsetfillcolor{currentfill}%
\pgfsetlinewidth{0.602250pt}%
\definecolor{currentstroke}{rgb}{0.000000,0.000000,0.000000}%
\pgfsetstrokecolor{currentstroke}%
\pgfsetdash{}{0pt}%
\pgfsys@defobject{currentmarker}{\pgfqpoint{-0.027778in}{0.000000in}}{\pgfqpoint{-0.000000in}{0.000000in}}{%
\pgfpathmoveto{\pgfqpoint{-0.000000in}{0.000000in}}%
\pgfpathlineto{\pgfqpoint{-0.027778in}{0.000000in}}%
\pgfusepath{stroke,fill}%
}%
\begin{pgfscope}%
\pgfsys@transformshift{0.588387in}{2.361753in}%
\pgfsys@useobject{currentmarker}{}%
\end{pgfscope}%
\end{pgfscope}%
\begin{pgfscope}%
\pgfsetbuttcap%
\pgfsetroundjoin%
\definecolor{currentfill}{rgb}{0.000000,0.000000,0.000000}%
\pgfsetfillcolor{currentfill}%
\pgfsetlinewidth{0.602250pt}%
\definecolor{currentstroke}{rgb}{0.000000,0.000000,0.000000}%
\pgfsetstrokecolor{currentstroke}%
\pgfsetdash{}{0pt}%
\pgfsys@defobject{currentmarker}{\pgfqpoint{-0.027778in}{0.000000in}}{\pgfqpoint{-0.000000in}{0.000000in}}{%
\pgfpathmoveto{\pgfqpoint{-0.000000in}{0.000000in}}%
\pgfpathlineto{\pgfqpoint{-0.027778in}{0.000000in}}%
\pgfusepath{stroke,fill}%
}%
\begin{pgfscope}%
\pgfsys@transformshift{0.588387in}{2.439970in}%
\pgfsys@useobject{currentmarker}{}%
\end{pgfscope}%
\end{pgfscope}%
\begin{pgfscope}%
\pgfsetbuttcap%
\pgfsetroundjoin%
\definecolor{currentfill}{rgb}{0.000000,0.000000,0.000000}%
\pgfsetfillcolor{currentfill}%
\pgfsetlinewidth{0.602250pt}%
\definecolor{currentstroke}{rgb}{0.000000,0.000000,0.000000}%
\pgfsetstrokecolor{currentstroke}%
\pgfsetdash{}{0pt}%
\pgfsys@defobject{currentmarker}{\pgfqpoint{-0.027778in}{0.000000in}}{\pgfqpoint{-0.000000in}{0.000000in}}{%
\pgfpathmoveto{\pgfqpoint{-0.000000in}{0.000000in}}%
\pgfpathlineto{\pgfqpoint{-0.027778in}{0.000000in}}%
\pgfusepath{stroke,fill}%
}%
\begin{pgfscope}%
\pgfsys@transformshift{0.588387in}{2.500640in}%
\pgfsys@useobject{currentmarker}{}%
\end{pgfscope}%
\end{pgfscope}%
\begin{pgfscope}%
\pgfsetbuttcap%
\pgfsetroundjoin%
\definecolor{currentfill}{rgb}{0.000000,0.000000,0.000000}%
\pgfsetfillcolor{currentfill}%
\pgfsetlinewidth{0.602250pt}%
\definecolor{currentstroke}{rgb}{0.000000,0.000000,0.000000}%
\pgfsetstrokecolor{currentstroke}%
\pgfsetdash{}{0pt}%
\pgfsys@defobject{currentmarker}{\pgfqpoint{-0.027778in}{0.000000in}}{\pgfqpoint{-0.000000in}{0.000000in}}{%
\pgfpathmoveto{\pgfqpoint{-0.000000in}{0.000000in}}%
\pgfpathlineto{\pgfqpoint{-0.027778in}{0.000000in}}%
\pgfusepath{stroke,fill}%
}%
\begin{pgfscope}%
\pgfsys@transformshift{0.588387in}{2.550211in}%
\pgfsys@useobject{currentmarker}{}%
\end{pgfscope}%
\end{pgfscope}%
\begin{pgfscope}%
\pgfsetbuttcap%
\pgfsetroundjoin%
\definecolor{currentfill}{rgb}{0.000000,0.000000,0.000000}%
\pgfsetfillcolor{currentfill}%
\pgfsetlinewidth{0.602250pt}%
\definecolor{currentstroke}{rgb}{0.000000,0.000000,0.000000}%
\pgfsetstrokecolor{currentstroke}%
\pgfsetdash{}{0pt}%
\pgfsys@defobject{currentmarker}{\pgfqpoint{-0.027778in}{0.000000in}}{\pgfqpoint{-0.000000in}{0.000000in}}{%
\pgfpathmoveto{\pgfqpoint{-0.000000in}{0.000000in}}%
\pgfpathlineto{\pgfqpoint{-0.027778in}{0.000000in}}%
\pgfusepath{stroke,fill}%
}%
\begin{pgfscope}%
\pgfsys@transformshift{0.588387in}{2.592123in}%
\pgfsys@useobject{currentmarker}{}%
\end{pgfscope}%
\end{pgfscope}%
\begin{pgfscope}%
\pgfsetbuttcap%
\pgfsetroundjoin%
\definecolor{currentfill}{rgb}{0.000000,0.000000,0.000000}%
\pgfsetfillcolor{currentfill}%
\pgfsetlinewidth{0.602250pt}%
\definecolor{currentstroke}{rgb}{0.000000,0.000000,0.000000}%
\pgfsetstrokecolor{currentstroke}%
\pgfsetdash{}{0pt}%
\pgfsys@defobject{currentmarker}{\pgfqpoint{-0.027778in}{0.000000in}}{\pgfqpoint{-0.000000in}{0.000000in}}{%
\pgfpathmoveto{\pgfqpoint{-0.000000in}{0.000000in}}%
\pgfpathlineto{\pgfqpoint{-0.027778in}{0.000000in}}%
\pgfusepath{stroke,fill}%
}%
\begin{pgfscope}%
\pgfsys@transformshift{0.588387in}{2.628429in}%
\pgfsys@useobject{currentmarker}{}%
\end{pgfscope}%
\end{pgfscope}%
\begin{pgfscope}%
\pgfsetbuttcap%
\pgfsetroundjoin%
\definecolor{currentfill}{rgb}{0.000000,0.000000,0.000000}%
\pgfsetfillcolor{currentfill}%
\pgfsetlinewidth{0.602250pt}%
\definecolor{currentstroke}{rgb}{0.000000,0.000000,0.000000}%
\pgfsetstrokecolor{currentstroke}%
\pgfsetdash{}{0pt}%
\pgfsys@defobject{currentmarker}{\pgfqpoint{-0.027778in}{0.000000in}}{\pgfqpoint{-0.000000in}{0.000000in}}{%
\pgfpathmoveto{\pgfqpoint{-0.000000in}{0.000000in}}%
\pgfpathlineto{\pgfqpoint{-0.027778in}{0.000000in}}%
\pgfusepath{stroke,fill}%
}%
\begin{pgfscope}%
\pgfsys@transformshift{0.588387in}{2.660453in}%
\pgfsys@useobject{currentmarker}{}%
\end{pgfscope}%
\end{pgfscope}%
\begin{pgfscope}%
\definecolor{textcolor}{rgb}{0.000000,0.000000,0.000000}%
\pgfsetstrokecolor{textcolor}%
\pgfsetfillcolor{textcolor}%
\pgftext[x=0.234413in,y=1.631490in,,bottom,rotate=90.000000]{\color{textcolor}{\rmfamily\fontsize{10.000000}{12.000000}\selectfont\catcode`\^=\active\def^{\ifmmode\sp\else\^{}\fi}\catcode`\%=\active\def%{\%}Time [ms]}}%
\end{pgfscope}%
\begin{pgfscope}%
\pgfpathrectangle{\pgfqpoint{0.588387in}{0.521603in}}{\pgfqpoint{3.660036in}{2.219773in}}%
\pgfusepath{clip}%
\pgfsetrectcap%
\pgfsetroundjoin%
\pgfsetlinewidth{1.505625pt}%
\pgfsetstrokecolor{currentstroke1}%
\pgfsetdash{}{0pt}%
\pgfpathmoveto{\pgfqpoint{0.754752in}{0.622502in}}%
\pgfpathlineto{\pgfqpoint{1.010699in}{0.836874in}}%
\pgfpathlineto{\pgfqpoint{1.266646in}{1.048990in}}%
\pgfpathlineto{\pgfqpoint{1.650565in}{1.259212in}}%
\pgfpathlineto{\pgfqpoint{1.778539in}{1.577656in}}%
\pgfpathlineto{\pgfqpoint{2.034485in}{1.813000in}}%
\pgfpathlineto{\pgfqpoint{2.418405in}{2.040235in}}%
\pgfpathlineto{\pgfqpoint{2.674352in}{2.329559in}}%
\pgfpathlineto{\pgfqpoint{3.058271in}{2.618319in}}%
\pgfpathlineto{\pgfqpoint{3.314218in}{2.599732in}}%
\pgfusepath{stroke}%
\end{pgfscope}%
\begin{pgfscope}%
\pgfpathrectangle{\pgfqpoint{0.588387in}{0.521603in}}{\pgfqpoint{3.660036in}{2.219773in}}%
\pgfusepath{clip}%
\pgfsetrectcap%
\pgfsetroundjoin%
\pgfsetlinewidth{1.505625pt}%
\pgfsetstrokecolor{currentstroke2}%
\pgfsetdash{}{0pt}%
\pgfpathmoveto{\pgfqpoint{0.754752in}{0.622502in}}%
\pgfpathlineto{\pgfqpoint{1.010699in}{0.810961in}}%
\pgfpathlineto{\pgfqpoint{1.266646in}{1.054597in}}%
\pgfpathlineto{\pgfqpoint{1.650565in}{1.307034in}}%
\pgfpathlineto{\pgfqpoint{1.778539in}{1.549953in}}%
\pgfpathlineto{\pgfqpoint{2.034485in}{1.816052in}}%
\pgfpathlineto{\pgfqpoint{2.418405in}{2.035311in}}%
\pgfpathlineto{\pgfqpoint{2.674352in}{2.253189in}}%
\pgfpathlineto{\pgfqpoint{3.058271in}{2.611858in}}%
\pgfpathlineto{\pgfqpoint{3.314218in}{2.620278in}}%
\pgfpathlineto{\pgfqpoint{3.698138in}{2.526646in}}%
\pgfusepath{stroke}%
\end{pgfscope}%
\begin{pgfscope}%
\pgfpathrectangle{\pgfqpoint{0.588387in}{0.521603in}}{\pgfqpoint{3.660036in}{2.219773in}}%
\pgfusepath{clip}%
\pgfsetrectcap%
\pgfsetroundjoin%
\pgfsetlinewidth{1.505625pt}%
\pgfsetstrokecolor{currentstroke3}%
\pgfsetdash{}{0pt}%
\pgfpathmoveto{\pgfqpoint{0.754752in}{0.672073in}}%
\pgfpathlineto{\pgfqpoint{1.010699in}{0.810961in}}%
\pgfpathlineto{\pgfqpoint{1.266646in}{1.031443in}}%
\pgfpathlineto{\pgfqpoint{1.650565in}{1.259212in}}%
\pgfpathlineto{\pgfqpoint{1.778539in}{1.517593in}}%
\pgfpathlineto{\pgfqpoint{2.034485in}{1.780708in}}%
\pgfpathlineto{\pgfqpoint{2.418405in}{1.892990in}}%
\pgfpathlineto{\pgfqpoint{2.674352in}{2.211668in}}%
\pgfpathlineto{\pgfqpoint{3.058271in}{2.119116in}}%
\pgfpathlineto{\pgfqpoint{3.314218in}{2.454711in}}%
\pgfpathlineto{\pgfqpoint{3.698138in}{2.579553in}}%
\pgfpathlineto{\pgfqpoint{4.082057in}{2.605148in}}%
\pgfusepath{stroke}%
\end{pgfscope}%
\begin{pgfscope}%
\pgfpathrectangle{\pgfqpoint{0.588387in}{0.521603in}}{\pgfqpoint{3.660036in}{2.219773in}}%
\pgfusepath{clip}%
\pgfsetrectcap%
\pgfsetroundjoin%
\pgfsetlinewidth{1.505625pt}%
\pgfsetstrokecolor{currentstroke4}%
\pgfsetdash{}{0pt}%
\pgfpathmoveto{\pgfqpoint{0.754752in}{0.622502in}}%
\pgfpathlineto{\pgfqpoint{1.010699in}{0.810961in}}%
\pgfpathlineto{\pgfqpoint{1.266646in}{1.054597in}}%
\pgfpathlineto{\pgfqpoint{1.650565in}{1.251254in}}%
\pgfpathlineto{\pgfqpoint{1.778539in}{1.524085in}}%
\pgfpathlineto{\pgfqpoint{2.034485in}{1.773425in}}%
\pgfpathlineto{\pgfqpoint{2.418405in}{1.883158in}}%
\pgfpathlineto{\pgfqpoint{2.674352in}{2.147378in}}%
\pgfpathlineto{\pgfqpoint{3.058271in}{2.102585in}}%
\pgfpathlineto{\pgfqpoint{3.314218in}{2.444862in}}%
\pgfpathlineto{\pgfqpoint{3.698138in}{2.561125in}}%
\pgfpathlineto{\pgfqpoint{4.082057in}{2.613114in}}%
\pgfusepath{stroke}%
\end{pgfscope}%
\begin{pgfscope}%
\pgfpathrectangle{\pgfqpoint{0.588387in}{0.521603in}}{\pgfqpoint{3.660036in}{2.219773in}}%
\pgfusepath{clip}%
\pgfsetrectcap%
\pgfsetroundjoin%
\pgfsetlinewidth{1.505625pt}%
\pgfsetstrokecolor{currentstroke5}%
\pgfsetdash{}{0pt}%
\pgfpathmoveto{\pgfqpoint{0.754752in}{0.622502in}}%
\pgfpathlineto{\pgfqpoint{1.010699in}{0.810961in}}%
\pgfpathlineto{\pgfqpoint{1.266646in}{1.043266in}}%
\pgfpathlineto{\pgfqpoint{1.650565in}{1.253932in}}%
\pgfpathlineto{\pgfqpoint{1.778539in}{1.531387in}}%
\pgfpathlineto{\pgfqpoint{2.034485in}{1.775278in}}%
\pgfpathlineto{\pgfqpoint{2.418405in}{2.002553in}}%
\pgfpathlineto{\pgfqpoint{2.674352in}{2.200502in}}%
\pgfpathlineto{\pgfqpoint{3.058271in}{2.187765in}}%
\pgfpathlineto{\pgfqpoint{3.314218in}{2.516349in}}%
\pgfpathlineto{\pgfqpoint{3.698138in}{2.588396in}}%
\pgfpathlineto{\pgfqpoint{4.082057in}{2.595573in}}%
\pgfusepath{stroke}%
\end{pgfscope}%
\begin{pgfscope}%
\pgfpathrectangle{\pgfqpoint{0.588387in}{0.521603in}}{\pgfqpoint{3.660036in}{2.219773in}}%
\pgfusepath{clip}%
\pgfsetrectcap%
\pgfsetroundjoin%
\pgfsetlinewidth{1.505625pt}%
\pgfsetstrokecolor{currentstroke6}%
\pgfsetdash{}{0pt}%
\pgfpathmoveto{\pgfqpoint{0.754752in}{0.622502in}}%
\pgfpathlineto{\pgfqpoint{1.010699in}{0.810961in}}%
\pgfpathlineto{\pgfqpoint{1.266646in}{1.048990in}}%
\pgfpathlineto{\pgfqpoint{1.650565in}{1.245816in}}%
\pgfpathlineto{\pgfqpoint{1.778539in}{1.529459in}}%
\pgfpathlineto{\pgfqpoint{2.034485in}{1.765825in}}%
\pgfpathlineto{\pgfqpoint{2.418405in}{1.896775in}}%
\pgfpathlineto{\pgfqpoint{2.674352in}{2.183478in}}%
\pgfpathlineto{\pgfqpoint{3.058271in}{2.105624in}}%
\pgfpathlineto{\pgfqpoint{3.314218in}{2.479310in}}%
\pgfpathlineto{\pgfqpoint{3.698138in}{2.573163in}}%
\pgfpathlineto{\pgfqpoint{4.082057in}{2.633702in}}%
\pgfusepath{stroke}%
\end{pgfscope}%
\begin{pgfscope}%
\pgfpathrectangle{\pgfqpoint{0.588387in}{0.521603in}}{\pgfqpoint{3.660036in}{2.219773in}}%
\pgfusepath{clip}%
\pgfsetrectcap%
\pgfsetroundjoin%
\pgfsetlinewidth{1.505625pt}%
\pgfsetstrokecolor{currentstroke7}%
\pgfsetdash{}{0pt}%
\pgfpathmoveto{\pgfqpoint{0.754752in}{0.672073in}}%
\pgfpathlineto{\pgfqpoint{1.010699in}{0.810961in}}%
\pgfpathlineto{\pgfqpoint{1.266646in}{1.048990in}}%
\pgfpathlineto{\pgfqpoint{1.650565in}{1.274462in}}%
\pgfpathlineto{\pgfqpoint{1.778539in}{1.548153in}}%
\pgfpathlineto{\pgfqpoint{2.034485in}{1.837666in}}%
\pgfpathlineto{\pgfqpoint{2.418405in}{2.007933in}}%
\pgfpathlineto{\pgfqpoint{2.674352in}{2.183565in}}%
\pgfpathlineto{\pgfqpoint{3.058271in}{2.425702in}}%
\pgfpathlineto{\pgfqpoint{3.314218in}{2.614459in}}%
\pgfpathlineto{\pgfqpoint{3.698138in}{2.640478in}}%
\pgfusepath{stroke}%
\end{pgfscope}%
\begin{pgfscope}%
\pgfpathrectangle{\pgfqpoint{0.588387in}{0.521603in}}{\pgfqpoint{3.660036in}{2.219773in}}%
\pgfusepath{clip}%
\pgfsetrectcap%
\pgfsetroundjoin%
\pgfsetlinewidth{1.505625pt}%
\definecolor{currentstroke}{rgb}{0.498039,0.498039,0.498039}%
\pgfsetstrokecolor{currentstroke}%
\pgfsetdash{}{0pt}%
\pgfpathmoveto{\pgfqpoint{1.010699in}{0.810961in}}%
\pgfpathlineto{\pgfqpoint{1.266646in}{1.031443in}}%
\pgfpathlineto{\pgfqpoint{1.650565in}{1.243055in}}%
\pgfpathlineto{\pgfqpoint{1.778539in}{1.540829in}}%
\pgfpathlineto{\pgfqpoint{2.034485in}{1.784088in}}%
\pgfpathlineto{\pgfqpoint{2.418405in}{1.964714in}}%
\pgfpathlineto{\pgfqpoint{2.674352in}{2.188837in}}%
\pgfpathlineto{\pgfqpoint{3.058271in}{2.415483in}}%
\pgfpathlineto{\pgfqpoint{3.314218in}{2.537460in}}%
\pgfpathlineto{\pgfqpoint{3.698138in}{2.637042in}}%
\pgfpathlineto{\pgfqpoint{4.082057in}{2.547890in}}%
\pgfusepath{stroke}%
\end{pgfscope}%
\begin{pgfscope}%
\pgfpathrectangle{\pgfqpoint{0.588387in}{0.521603in}}{\pgfqpoint{3.660036in}{2.219773in}}%
\pgfusepath{clip}%
\pgfsetrectcap%
\pgfsetroundjoin%
\pgfsetlinewidth{1.505625pt}%
\definecolor{currentstroke}{rgb}{0.737255,0.741176,0.133333}%
\pgfsetstrokecolor{currentstroke}%
\pgfsetdash{}{0pt}%
\pgfpathmoveto{\pgfqpoint{2.034485in}{1.825613in}}%
\pgfpathlineto{\pgfqpoint{2.418405in}{2.194722in}}%
\pgfpathlineto{\pgfqpoint{2.674352in}{2.565251in}}%
\pgfusepath{stroke}%
\end{pgfscope}%
\begin{pgfscope}%
\pgfsetrectcap%
\pgfsetmiterjoin%
\pgfsetlinewidth{0.803000pt}%
\definecolor{currentstroke}{rgb}{0.000000,0.000000,0.000000}%
\pgfsetstrokecolor{currentstroke}%
\pgfsetdash{}{0pt}%
\pgfpathmoveto{\pgfqpoint{0.588387in}{0.521603in}}%
\pgfpathlineto{\pgfqpoint{0.588387in}{2.741376in}}%
\pgfusepath{stroke}%
\end{pgfscope}%
\begin{pgfscope}%
\pgfsetrectcap%
\pgfsetmiterjoin%
\pgfsetlinewidth{0.803000pt}%
\definecolor{currentstroke}{rgb}{0.000000,0.000000,0.000000}%
\pgfsetstrokecolor{currentstroke}%
\pgfsetdash{}{0pt}%
\pgfpathmoveto{\pgfqpoint{4.248423in}{0.521603in}}%
\pgfpathlineto{\pgfqpoint{4.248423in}{2.741376in}}%
\pgfusepath{stroke}%
\end{pgfscope}%
\begin{pgfscope}%
\pgfsetrectcap%
\pgfsetmiterjoin%
\pgfsetlinewidth{0.803000pt}%
\definecolor{currentstroke}{rgb}{0.000000,0.000000,0.000000}%
\pgfsetstrokecolor{currentstroke}%
\pgfsetdash{}{0pt}%
\pgfpathmoveto{\pgfqpoint{0.588387in}{0.521603in}}%
\pgfpathlineto{\pgfqpoint{4.248423in}{0.521603in}}%
\pgfusepath{stroke}%
\end{pgfscope}%
\begin{pgfscope}%
\pgfsetrectcap%
\pgfsetmiterjoin%
\pgfsetlinewidth{0.803000pt}%
\definecolor{currentstroke}{rgb}{0.000000,0.000000,0.000000}%
\pgfsetstrokecolor{currentstroke}%
\pgfsetdash{}{0pt}%
\pgfpathmoveto{\pgfqpoint{0.588387in}{2.741376in}}%
\pgfpathlineto{\pgfqpoint{4.248423in}{2.741376in}}%
\pgfusepath{stroke}%
\end{pgfscope}%
\begin{pgfscope}%
\pgfsetbuttcap%
\pgfsetmiterjoin%
\definecolor{currentfill}{rgb}{1.000000,1.000000,1.000000}%
\pgfsetfillcolor{currentfill}%
\pgfsetfillopacity{0.800000}%
\pgfsetlinewidth{1.003750pt}%
\definecolor{currentstroke}{rgb}{0.800000,0.800000,0.800000}%
\pgfsetstrokecolor{currentstroke}%
\pgfsetstrokeopacity{0.800000}%
\pgfsetdash{}{0pt}%
\pgfpathmoveto{\pgfqpoint{4.365089in}{0.378553in}}%
\pgfpathlineto{\pgfqpoint{8.251043in}{0.378553in}}%
\pgfpathquadraticcurveto{\pgfqpoint{8.284376in}{0.378553in}}{\pgfqpoint{8.284376in}{0.411886in}}%
\pgfpathlineto{\pgfqpoint{8.284376in}{2.624710in}}%
\pgfpathquadraticcurveto{\pgfqpoint{8.284376in}{2.658043in}}{\pgfqpoint{8.251043in}{2.658043in}}%
\pgfpathlineto{\pgfqpoint{4.365089in}{2.658043in}}%
\pgfpathquadraticcurveto{\pgfqpoint{4.331756in}{2.658043in}}{\pgfqpoint{4.331756in}{2.624710in}}%
\pgfpathlineto{\pgfqpoint{4.331756in}{0.411886in}}%
\pgfpathquadraticcurveto{\pgfqpoint{4.331756in}{0.378553in}}{\pgfqpoint{4.365089in}{0.378553in}}%
\pgfpathlineto{\pgfqpoint{4.365089in}{0.378553in}}%
\pgfpathclose%
\pgfusepath{stroke,fill}%
\end{pgfscope}%
\begin{pgfscope}%
\pgfsetrectcap%
\pgfsetroundjoin%
\pgfsetlinewidth{1.505625pt}%
\definecolor{currentstroke}{rgb}{0.737255,0.741176,0.133333}%
\pgfsetstrokecolor{currentstroke}%
\pgfsetdash{}{0pt}%
\pgfpathmoveto{\pgfqpoint{4.398423in}{2.523082in}}%
\pgfpathlineto{\pgfqpoint{4.565089in}{2.523082in}}%
\pgfpathlineto{\pgfqpoint{4.731756in}{2.523082in}}%
\pgfusepath{stroke}%
\end{pgfscope}%
\begin{pgfscope}%
\definecolor{textcolor}{rgb}{0.000000,0.000000,0.000000}%
\pgfsetstrokecolor{textcolor}%
\pgfsetfillcolor{textcolor}%
\pgftext[x=4.865089in,y=2.464749in,left,base]{\color{textcolor}{\rmfamily\fontsize{12.000000}{14.400000}\selectfont\catcode`\^=\active\def^{\ifmmode\sp\else\^{}\fi}\catcode`\%=\active\def%{\%}\NaiveCycles{}}}%
\end{pgfscope}%
\begin{pgfscope}%
\pgfsetrectcap%
\pgfsetroundjoin%
\pgfsetlinewidth{1.505625pt}%
\pgfsetstrokecolor{currentstroke1}%
\pgfsetdash{}{0pt}%
\pgfpathmoveto{\pgfqpoint{4.398423in}{2.278453in}}%
\pgfpathlineto{\pgfqpoint{4.565089in}{2.278453in}}%
\pgfpathlineto{\pgfqpoint{4.731756in}{2.278453in}}%
\pgfusepath{stroke}%
\end{pgfscope}%
\begin{pgfscope}%
\definecolor{textcolor}{rgb}{0.000000,0.000000,0.000000}%
\pgfsetstrokecolor{textcolor}%
\pgfsetfillcolor{textcolor}%
\pgftext[x=4.865089in,y=2.220120in,left,base]{\color{textcolor}{\rmfamily\fontsize{12.000000}{14.400000}\selectfont\catcode`\^=\active\def^{\ifmmode\sp\else\^{}\fi}\catcode`\%=\active\def%{\%}\CyclesMatchChunks{} \& \MergeLinear{}}}%
\end{pgfscope}%
\begin{pgfscope}%
\pgfsetrectcap%
\pgfsetroundjoin%
\pgfsetlinewidth{1.505625pt}%
\pgfsetstrokecolor{currentstroke2}%
\pgfsetdash{}{0pt}%
\pgfpathmoveto{\pgfqpoint{4.398423in}{2.029186in}}%
\pgfpathlineto{\pgfqpoint{4.565089in}{2.029186in}}%
\pgfpathlineto{\pgfqpoint{4.731756in}{2.029186in}}%
\pgfusepath{stroke}%
\end{pgfscope}%
\begin{pgfscope}%
\definecolor{textcolor}{rgb}{0.000000,0.000000,0.000000}%
\pgfsetstrokecolor{textcolor}%
\pgfsetfillcolor{textcolor}%
\pgftext[x=4.865089in,y=1.970853in,left,base]{\color{textcolor}{\rmfamily\fontsize{12.000000}{14.400000}\selectfont\catcode`\^=\active\def^{\ifmmode\sp\else\^{}\fi}\catcode`\%=\active\def%{\%}\CyclesMatchChunks{} \& \SharedVertices{}}}%
\end{pgfscope}%
\begin{pgfscope}%
\pgfsetrectcap%
\pgfsetroundjoin%
\pgfsetlinewidth{1.505625pt}%
\pgfsetstrokecolor{currentstroke3}%
\pgfsetdash{}{0pt}%
\pgfpathmoveto{\pgfqpoint{4.398423in}{1.779919in}}%
\pgfpathlineto{\pgfqpoint{4.565089in}{1.779919in}}%
\pgfpathlineto{\pgfqpoint{4.731756in}{1.779919in}}%
\pgfusepath{stroke}%
\end{pgfscope}%
\begin{pgfscope}%
\definecolor{textcolor}{rgb}{0.000000,0.000000,0.000000}%
\pgfsetstrokecolor{textcolor}%
\pgfsetfillcolor{textcolor}%
\pgftext[x=4.865089in,y=1.721585in,left,base]{\color{textcolor}{\rmfamily\fontsize{12.000000}{14.400000}\selectfont\catcode`\^=\active\def^{\ifmmode\sp\else\^{}\fi}\catcode`\%=\active\def%{\%}\Neighbors{} \& \MergeLinear{}}}%
\end{pgfscope}%
\begin{pgfscope}%
\pgfsetrectcap%
\pgfsetroundjoin%
\pgfsetlinewidth{1.505625pt}%
\pgfsetstrokecolor{currentstroke4}%
\pgfsetdash{}{0pt}%
\pgfpathmoveto{\pgfqpoint{4.398423in}{1.535290in}}%
\pgfpathlineto{\pgfqpoint{4.565089in}{1.535290in}}%
\pgfpathlineto{\pgfqpoint{4.731756in}{1.535290in}}%
\pgfusepath{stroke}%
\end{pgfscope}%
\begin{pgfscope}%
\definecolor{textcolor}{rgb}{0.000000,0.000000,0.000000}%
\pgfsetstrokecolor{textcolor}%
\pgfsetfillcolor{textcolor}%
\pgftext[x=4.865089in,y=1.476957in,left,base]{\color{textcolor}{\rmfamily\fontsize{12.000000}{14.400000}\selectfont\catcode`\^=\active\def^{\ifmmode\sp\else\^{}\fi}\catcode`\%=\active\def%{\%}\Neighbors{} \& \SharedVertices{}}}%
\end{pgfscope}%
\begin{pgfscope}%
\pgfsetrectcap%
\pgfsetroundjoin%
\pgfsetlinewidth{1.505625pt}%
\pgfsetstrokecolor{currentstroke5}%
\pgfsetdash{}{0pt}%
\pgfpathmoveto{\pgfqpoint{4.398423in}{1.286023in}}%
\pgfpathlineto{\pgfqpoint{4.565089in}{1.286023in}}%
\pgfpathlineto{\pgfqpoint{4.731756in}{1.286023in}}%
\pgfusepath{stroke}%
\end{pgfscope}%
\begin{pgfscope}%
\definecolor{textcolor}{rgb}{0.000000,0.000000,0.000000}%
\pgfsetstrokecolor{textcolor}%
\pgfsetfillcolor{textcolor}%
\pgftext[x=4.865089in,y=1.227689in,left,base]{\color{textcolor}{\rmfamily\fontsize{12.000000}{14.400000}\selectfont\catcode`\^=\active\def^{\ifmmode\sp\else\^{}\fi}\catcode`\%=\active\def%{\%}\NeighborsDegree{} \& \MergeLinear{}}}%
\end{pgfscope}%
\begin{pgfscope}%
\pgfsetrectcap%
\pgfsetroundjoin%
\pgfsetlinewidth{1.505625pt}%
\pgfsetstrokecolor{currentstroke6}%
\pgfsetdash{}{0pt}%
\pgfpathmoveto{\pgfqpoint{4.398423in}{1.036755in}}%
\pgfpathlineto{\pgfqpoint{4.565089in}{1.036755in}}%
\pgfpathlineto{\pgfqpoint{4.731756in}{1.036755in}}%
\pgfusepath{stroke}%
\end{pgfscope}%
\begin{pgfscope}%
\definecolor{textcolor}{rgb}{0.000000,0.000000,0.000000}%
\pgfsetstrokecolor{textcolor}%
\pgfsetfillcolor{textcolor}%
\pgftext[x=4.865089in,y=0.978422in,left,base]{\color{textcolor}{\rmfamily\fontsize{12.000000}{14.400000}\selectfont\catcode`\^=\active\def^{\ifmmode\sp\else\^{}\fi}\catcode`\%=\active\def%{\%}\NeighborsDegree{} \& \SharedVertices{}}}%
\end{pgfscope}%
\begin{pgfscope}%
\pgfsetrectcap%
\pgfsetroundjoin%
\pgfsetlinewidth{1.505625pt}%
\pgfsetstrokecolor{currentstroke7}%
\pgfsetdash{}{0pt}%
\pgfpathmoveto{\pgfqpoint{4.398423in}{0.787488in}}%
\pgfpathlineto{\pgfqpoint{4.565089in}{0.787488in}}%
\pgfpathlineto{\pgfqpoint{4.731756in}{0.787488in}}%
\pgfusepath{stroke}%
\end{pgfscope}%
\begin{pgfscope}%
\definecolor{textcolor}{rgb}{0.000000,0.000000,0.000000}%
\pgfsetstrokecolor{textcolor}%
\pgfsetfillcolor{textcolor}%
\pgftext[x=4.865089in,y=0.729155in,left,base]{\color{textcolor}{\rmfamily\fontsize{12.000000}{14.400000}\selectfont\catcode`\^=\active\def^{\ifmmode\sp\else\^{}\fi}\catcode`\%=\active\def%{\%}\None{} \& \MergeLinear{}}}%
\end{pgfscope}%
\begin{pgfscope}%
\pgfsetrectcap%
\pgfsetroundjoin%
\pgfsetlinewidth{1.505625pt}%
\definecolor{currentstroke}{rgb}{0.498039,0.498039,0.498039}%
\pgfsetstrokecolor{currentstroke}%
\pgfsetdash{}{0pt}%
\pgfpathmoveto{\pgfqpoint{4.398423in}{0.542859in}}%
\pgfpathlineto{\pgfqpoint{4.565089in}{0.542859in}}%
\pgfpathlineto{\pgfqpoint{4.731756in}{0.542859in}}%
\pgfusepath{stroke}%
\end{pgfscope}%
\begin{pgfscope}%
\definecolor{textcolor}{rgb}{0.000000,0.000000,0.000000}%
\pgfsetstrokecolor{textcolor}%
\pgfsetfillcolor{textcolor}%
\pgftext[x=4.865089in,y=0.484526in,left,base]{\color{textcolor}{\rmfamily\fontsize{12.000000}{14.400000}\selectfont\catcode`\^=\active\def^{\ifmmode\sp\else\^{}\fi}\catcode`\%=\active\def%{\%}\None{} \& \SharedVertices{}}}%
\end{pgfscope}%
\end{pgfpicture}%
\makeatother%
\endgroup%
}
	\caption[Mean runtime for graphs with no 3 nor 4 cycles (all).]{
		Mean running time (ms) to find all NAC-colorings for graphs with no 3 nor 4 cycles.}%
	\label{fig:graph_count_no_3_nor_4_cycles_all_runtime}
\end{figure}
\begin{figure}[p]
	\centering
	\scalebox{0.5}{%% Creator: Matplotlib, PGF backend
%%
%% To include the figure in your LaTeX document, write
%%   \input{<filename>.pgf}
%%
%% Make sure the required packages are loaded in your preamble
%%   \usepackage{pgf}
%%
%% Also ensure that all the required font packages are loaded; for instance,
%% the lmodern package is sometimes necessary when using math font.
%%   \usepackage{lmodern}
%%
%% Figures using additional raster images can only be included by \input if
%% they are in the same directory as the main LaTeX file. For loading figures
%% from other directories you can use the `import` package
%%   \usepackage{import}
%%
%% and then include the figures with
%%   \import{<path to file>}{<filename>.pgf}
%%
%% Matplotlib used the following preamble
%%   \def\mathdefault#1{#1}
%%   \everymath=\expandafter{\the\everymath\displaystyle}
%%   \IfFileExists{scrextend.sty}{
%%     \usepackage[fontsize=10.000000pt]{scrextend}
%%   }{
%%     \renewcommand{\normalsize}{\fontsize{10.000000}{12.000000}\selectfont}
%%     \normalsize
%%   }
%%   
%%   \ifdefined\pdftexversion\else  % non-pdftex case.
%%     \usepackage{fontspec}
%%     \setmainfont{DejaVuSans.ttf}[Path=\detokenize{/home/petr/Projects/PyRigi/.venv/lib/python3.12/site-packages/matplotlib/mpl-data/fonts/ttf/}]
%%     \setsansfont{DejaVuSans.ttf}[Path=\detokenize{/home/petr/Projects/PyRigi/.venv/lib/python3.12/site-packages/matplotlib/mpl-data/fonts/ttf/}]
%%     \setmonofont{DejaVuSansMono.ttf}[Path=\detokenize{/home/petr/Projects/PyRigi/.venv/lib/python3.12/site-packages/matplotlib/mpl-data/fonts/ttf/}]
%%   \fi
%%   \makeatletter\@ifpackageloaded{under\Score{}}{}{\usepackage[strings]{under\Score{}}}\makeatother
%%
\begingroup%
\makeatletter%
\begin{pgfpicture}%
\pgfpathrectangle{\pgfpointorigin}{\pgfqpoint{8.384376in}{2.841849in}}%
\pgfusepath{use as bounding box, clip}%
\begin{pgfscope}%
\pgfsetbuttcap%
\pgfsetmiterjoin%
\definecolor{currentfill}{rgb}{1.000000,1.000000,1.000000}%
\pgfsetfillcolor{currentfill}%
\pgfsetlinewidth{0.000000pt}%
\definecolor{currentstroke}{rgb}{1.000000,1.000000,1.000000}%
\pgfsetstrokecolor{currentstroke}%
\pgfsetdash{}{0pt}%
\pgfpathmoveto{\pgfqpoint{0.000000in}{0.000000in}}%
\pgfpathlineto{\pgfqpoint{8.384376in}{0.000000in}}%
\pgfpathlineto{\pgfqpoint{8.384376in}{2.841849in}}%
\pgfpathlineto{\pgfqpoint{0.000000in}{2.841849in}}%
\pgfpathlineto{\pgfqpoint{0.000000in}{0.000000in}}%
\pgfpathclose%
\pgfusepath{fill}%
\end{pgfscope}%
\begin{pgfscope}%
\pgfsetbuttcap%
\pgfsetmiterjoin%
\definecolor{currentfill}{rgb}{1.000000,1.000000,1.000000}%
\pgfsetfillcolor{currentfill}%
\pgfsetlinewidth{0.000000pt}%
\definecolor{currentstroke}{rgb}{0.000000,0.000000,0.000000}%
\pgfsetstrokecolor{currentstroke}%
\pgfsetstrokeopacity{0.000000}%
\pgfsetdash{}{0pt}%
\pgfpathmoveto{\pgfqpoint{0.588387in}{0.521603in}}%
\pgfpathlineto{\pgfqpoint{5.257411in}{0.521603in}}%
\pgfpathlineto{\pgfqpoint{5.257411in}{2.741849in}}%
\pgfpathlineto{\pgfqpoint{0.588387in}{2.741849in}}%
\pgfpathlineto{\pgfqpoint{0.588387in}{0.521603in}}%
\pgfpathclose%
\pgfusepath{fill}%
\end{pgfscope}%
\begin{pgfscope}%
\pgfsetbuttcap%
\pgfsetroundjoin%
\definecolor{currentfill}{rgb}{0.000000,0.000000,0.000000}%
\pgfsetfillcolor{currentfill}%
\pgfsetlinewidth{0.803000pt}%
\definecolor{currentstroke}{rgb}{0.000000,0.000000,0.000000}%
\pgfsetstrokecolor{currentstroke}%
\pgfsetdash{}{0pt}%
\pgfsys@defobject{currentmarker}{\pgfqpoint{0.000000in}{-0.048611in}}{\pgfqpoint{0.000000in}{0.000000in}}{%
\pgfpathmoveto{\pgfqpoint{0.000000in}{0.000000in}}%
\pgfpathlineto{\pgfqpoint{0.000000in}{-0.048611in}}%
\pgfusepath{stroke,fill}%
}%
\begin{pgfscope}%
\pgfsys@transformshift{0.654251in}{0.521603in}%
\pgfsys@useobject{currentmarker}{}%
\end{pgfscope}%
\end{pgfscope}%
\begin{pgfscope}%
\definecolor{textcolor}{rgb}{0.000000,0.000000,0.000000}%
\pgfsetstrokecolor{textcolor}%
\pgfsetfillcolor{textcolor}%
\pgftext[x=0.654251in,y=0.424381in,,top]{\color{textcolor}{\rmfamily\fontsize{10.000000}{12.000000}\selectfont\catcode`\^=\active\def^{\ifmmode\sp\else\^{}\fi}\catcode`\%=\active\def%{\%}$\mathdefault{4}$}}%
\end{pgfscope}%
\begin{pgfscope}%
\pgfsetbuttcap%
\pgfsetroundjoin%
\definecolor{currentfill}{rgb}{0.000000,0.000000,0.000000}%
\pgfsetfillcolor{currentfill}%
\pgfsetlinewidth{0.803000pt}%
\definecolor{currentstroke}{rgb}{0.000000,0.000000,0.000000}%
\pgfsetstrokecolor{currentstroke}%
\pgfsetdash{}{0pt}%
\pgfsys@defobject{currentmarker}{\pgfqpoint{0.000000in}{-0.048611in}}{\pgfqpoint{0.000000in}{0.000000in}}{%
\pgfpathmoveto{\pgfqpoint{0.000000in}{0.000000in}}%
\pgfpathlineto{\pgfqpoint{0.000000in}{-0.048611in}}%
\pgfusepath{stroke,fill}%
}%
\begin{pgfscope}%
\pgfsys@transformshift{1.239709in}{0.521603in}%
\pgfsys@useobject{currentmarker}{}%
\end{pgfscope}%
\end{pgfscope}%
\begin{pgfscope}%
\definecolor{textcolor}{rgb}{0.000000,0.000000,0.000000}%
\pgfsetstrokecolor{textcolor}%
\pgfsetfillcolor{textcolor}%
\pgftext[x=1.239709in,y=0.424381in,,top]{\color{textcolor}{\rmfamily\fontsize{10.000000}{12.000000}\selectfont\catcode`\^=\active\def^{\ifmmode\sp\else\^{}\fi}\catcode`\%=\active\def%{\%}$\mathdefault{8}$}}%
\end{pgfscope}%
\begin{pgfscope}%
\pgfsetbuttcap%
\pgfsetroundjoin%
\definecolor{currentfill}{rgb}{0.000000,0.000000,0.000000}%
\pgfsetfillcolor{currentfill}%
\pgfsetlinewidth{0.803000pt}%
\definecolor{currentstroke}{rgb}{0.000000,0.000000,0.000000}%
\pgfsetstrokecolor{currentstroke}%
\pgfsetdash{}{0pt}%
\pgfsys@defobject{currentmarker}{\pgfqpoint{0.000000in}{-0.048611in}}{\pgfqpoint{0.000000in}{0.000000in}}{%
\pgfpathmoveto{\pgfqpoint{0.000000in}{0.000000in}}%
\pgfpathlineto{\pgfqpoint{0.000000in}{-0.048611in}}%
\pgfusepath{stroke,fill}%
}%
\begin{pgfscope}%
\pgfsys@transformshift{1.825166in}{0.521603in}%
\pgfsys@useobject{currentmarker}{}%
\end{pgfscope}%
\end{pgfscope}%
\begin{pgfscope}%
\definecolor{textcolor}{rgb}{0.000000,0.000000,0.000000}%
\pgfsetstrokecolor{textcolor}%
\pgfsetfillcolor{textcolor}%
\pgftext[x=1.825166in,y=0.424381in,,top]{\color{textcolor}{\rmfamily\fontsize{10.000000}{12.000000}\selectfont\catcode`\^=\active\def^{\ifmmode\sp\else\^{}\fi}\catcode`\%=\active\def%{\%}$\mathdefault{12}$}}%
\end{pgfscope}%
\begin{pgfscope}%
\pgfsetbuttcap%
\pgfsetroundjoin%
\definecolor{currentfill}{rgb}{0.000000,0.000000,0.000000}%
\pgfsetfillcolor{currentfill}%
\pgfsetlinewidth{0.803000pt}%
\definecolor{currentstroke}{rgb}{0.000000,0.000000,0.000000}%
\pgfsetstrokecolor{currentstroke}%
\pgfsetdash{}{0pt}%
\pgfsys@defobject{currentmarker}{\pgfqpoint{0.000000in}{-0.048611in}}{\pgfqpoint{0.000000in}{0.000000in}}{%
\pgfpathmoveto{\pgfqpoint{0.000000in}{0.000000in}}%
\pgfpathlineto{\pgfqpoint{0.000000in}{-0.048611in}}%
\pgfusepath{stroke,fill}%
}%
\begin{pgfscope}%
\pgfsys@transformshift{2.410624in}{0.521603in}%
\pgfsys@useobject{currentmarker}{}%
\end{pgfscope}%
\end{pgfscope}%
\begin{pgfscope}%
\definecolor{textcolor}{rgb}{0.000000,0.000000,0.000000}%
\pgfsetstrokecolor{textcolor}%
\pgfsetfillcolor{textcolor}%
\pgftext[x=2.410624in,y=0.424381in,,top]{\color{textcolor}{\rmfamily\fontsize{10.000000}{12.000000}\selectfont\catcode`\^=\active\def^{\ifmmode\sp\else\^{}\fi}\catcode`\%=\active\def%{\%}$\mathdefault{16}$}}%
\end{pgfscope}%
\begin{pgfscope}%
\pgfsetbuttcap%
\pgfsetroundjoin%
\definecolor{currentfill}{rgb}{0.000000,0.000000,0.000000}%
\pgfsetfillcolor{currentfill}%
\pgfsetlinewidth{0.803000pt}%
\definecolor{currentstroke}{rgb}{0.000000,0.000000,0.000000}%
\pgfsetstrokecolor{currentstroke}%
\pgfsetdash{}{0pt}%
\pgfsys@defobject{currentmarker}{\pgfqpoint{0.000000in}{-0.048611in}}{\pgfqpoint{0.000000in}{0.000000in}}{%
\pgfpathmoveto{\pgfqpoint{0.000000in}{0.000000in}}%
\pgfpathlineto{\pgfqpoint{0.000000in}{-0.048611in}}%
\pgfusepath{stroke,fill}%
}%
\begin{pgfscope}%
\pgfsys@transformshift{2.996081in}{0.521603in}%
\pgfsys@useobject{currentmarker}{}%
\end{pgfscope}%
\end{pgfscope}%
\begin{pgfscope}%
\definecolor{textcolor}{rgb}{0.000000,0.000000,0.000000}%
\pgfsetstrokecolor{textcolor}%
\pgfsetfillcolor{textcolor}%
\pgftext[x=2.996081in,y=0.424381in,,top]{\color{textcolor}{\rmfamily\fontsize{10.000000}{12.000000}\selectfont\catcode`\^=\active\def^{\ifmmode\sp\else\^{}\fi}\catcode`\%=\active\def%{\%}$\mathdefault{20}$}}%
\end{pgfscope}%
\begin{pgfscope}%
\pgfsetbuttcap%
\pgfsetroundjoin%
\definecolor{currentfill}{rgb}{0.000000,0.000000,0.000000}%
\pgfsetfillcolor{currentfill}%
\pgfsetlinewidth{0.803000pt}%
\definecolor{currentstroke}{rgb}{0.000000,0.000000,0.000000}%
\pgfsetstrokecolor{currentstroke}%
\pgfsetdash{}{0pt}%
\pgfsys@defobject{currentmarker}{\pgfqpoint{0.000000in}{-0.048611in}}{\pgfqpoint{0.000000in}{0.000000in}}{%
\pgfpathmoveto{\pgfqpoint{0.000000in}{0.000000in}}%
\pgfpathlineto{\pgfqpoint{0.000000in}{-0.048611in}}%
\pgfusepath{stroke,fill}%
}%
\begin{pgfscope}%
\pgfsys@transformshift{3.581539in}{0.521603in}%
\pgfsys@useobject{currentmarker}{}%
\end{pgfscope}%
\end{pgfscope}%
\begin{pgfscope}%
\definecolor{textcolor}{rgb}{0.000000,0.000000,0.000000}%
\pgfsetstrokecolor{textcolor}%
\pgfsetfillcolor{textcolor}%
\pgftext[x=3.581539in,y=0.424381in,,top]{\color{textcolor}{\rmfamily\fontsize{10.000000}{12.000000}\selectfont\catcode`\^=\active\def^{\ifmmode\sp\else\^{}\fi}\catcode`\%=\active\def%{\%}$\mathdefault{24}$}}%
\end{pgfscope}%
\begin{pgfscope}%
\pgfsetbuttcap%
\pgfsetroundjoin%
\definecolor{currentfill}{rgb}{0.000000,0.000000,0.000000}%
\pgfsetfillcolor{currentfill}%
\pgfsetlinewidth{0.803000pt}%
\definecolor{currentstroke}{rgb}{0.000000,0.000000,0.000000}%
\pgfsetstrokecolor{currentstroke}%
\pgfsetdash{}{0pt}%
\pgfsys@defobject{currentmarker}{\pgfqpoint{0.000000in}{-0.048611in}}{\pgfqpoint{0.000000in}{0.000000in}}{%
\pgfpathmoveto{\pgfqpoint{0.000000in}{0.000000in}}%
\pgfpathlineto{\pgfqpoint{0.000000in}{-0.048611in}}%
\pgfusepath{stroke,fill}%
}%
\begin{pgfscope}%
\pgfsys@transformshift{4.166997in}{0.521603in}%
\pgfsys@useobject{currentmarker}{}%
\end{pgfscope}%
\end{pgfscope}%
\begin{pgfscope}%
\definecolor{textcolor}{rgb}{0.000000,0.000000,0.000000}%
\pgfsetstrokecolor{textcolor}%
\pgfsetfillcolor{textcolor}%
\pgftext[x=4.166997in,y=0.424381in,,top]{\color{textcolor}{\rmfamily\fontsize{10.000000}{12.000000}\selectfont\catcode`\^=\active\def^{\ifmmode\sp\else\^{}\fi}\catcode`\%=\active\def%{\%}$\mathdefault{28}$}}%
\end{pgfscope}%
\begin{pgfscope}%
\pgfsetbuttcap%
\pgfsetroundjoin%
\definecolor{currentfill}{rgb}{0.000000,0.000000,0.000000}%
\pgfsetfillcolor{currentfill}%
\pgfsetlinewidth{0.803000pt}%
\definecolor{currentstroke}{rgb}{0.000000,0.000000,0.000000}%
\pgfsetstrokecolor{currentstroke}%
\pgfsetdash{}{0pt}%
\pgfsys@defobject{currentmarker}{\pgfqpoint{0.000000in}{-0.048611in}}{\pgfqpoint{0.000000in}{0.000000in}}{%
\pgfpathmoveto{\pgfqpoint{0.000000in}{0.000000in}}%
\pgfpathlineto{\pgfqpoint{0.000000in}{-0.048611in}}%
\pgfusepath{stroke,fill}%
}%
\begin{pgfscope}%
\pgfsys@transformshift{4.752454in}{0.521603in}%
\pgfsys@useobject{currentmarker}{}%
\end{pgfscope}%
\end{pgfscope}%
\begin{pgfscope}%
\definecolor{textcolor}{rgb}{0.000000,0.000000,0.000000}%
\pgfsetstrokecolor{textcolor}%
\pgfsetfillcolor{textcolor}%
\pgftext[x=4.752454in,y=0.424381in,,top]{\color{textcolor}{\rmfamily\fontsize{10.000000}{12.000000}\selectfont\catcode`\^=\active\def^{\ifmmode\sp\else\^{}\fi}\catcode`\%=\active\def%{\%}$\mathdefault{32}$}}%
\end{pgfscope}%
\begin{pgfscope}%
\definecolor{textcolor}{rgb}{0.000000,0.000000,0.000000}%
\pgfsetstrokecolor{textcolor}%
\pgfsetfillcolor{textcolor}%
\pgftext[x=2.922899in,y=0.234413in,,top]{\color{textcolor}{\rmfamily\fontsize{10.000000}{12.000000}\selectfont\catcode`\^=\active\def^{\ifmmode\sp\else\^{}\fi}\catcode`\%=\active\def%{\%}Monochromatic classes}}%
\end{pgfscope}%
\begin{pgfscope}%
\pgfsetbuttcap%
\pgfsetroundjoin%
\definecolor{currentfill}{rgb}{0.000000,0.000000,0.000000}%
\pgfsetfillcolor{currentfill}%
\pgfsetlinewidth{0.803000pt}%
\definecolor{currentstroke}{rgb}{0.000000,0.000000,0.000000}%
\pgfsetstrokecolor{currentstroke}%
\pgfsetdash{}{0pt}%
\pgfsys@defobject{currentmarker}{\pgfqpoint{-0.048611in}{0.000000in}}{\pgfqpoint{-0.000000in}{0.000000in}}{%
\pgfpathmoveto{\pgfqpoint{-0.000000in}{0.000000in}}%
\pgfpathlineto{\pgfqpoint{-0.048611in}{0.000000in}}%
\pgfusepath{stroke,fill}%
}%
\begin{pgfscope}%
\pgfsys@transformshift{0.588387in}{0.546489in}%
\pgfsys@useobject{currentmarker}{}%
\end{pgfscope}%
\end{pgfscope}%
\begin{pgfscope}%
\definecolor{textcolor}{rgb}{0.000000,0.000000,0.000000}%
\pgfsetstrokecolor{textcolor}%
\pgfsetfillcolor{textcolor}%
\pgftext[x=0.289968in, y=0.493727in, left, base]{\color{textcolor}{\rmfamily\fontsize{10.000000}{12.000000}\selectfont\catcode`\^=\active\def^{\ifmmode\sp\else\^{}\fi}\catcode`\%=\active\def%{\%}$\mathdefault{10^{1}}$}}%
\end{pgfscope}%
\begin{pgfscope}%
\pgfsetbuttcap%
\pgfsetroundjoin%
\definecolor{currentfill}{rgb}{0.000000,0.000000,0.000000}%
\pgfsetfillcolor{currentfill}%
\pgfsetlinewidth{0.803000pt}%
\definecolor{currentstroke}{rgb}{0.000000,0.000000,0.000000}%
\pgfsetstrokecolor{currentstroke}%
\pgfsetdash{}{0pt}%
\pgfsys@defobject{currentmarker}{\pgfqpoint{-0.048611in}{0.000000in}}{\pgfqpoint{-0.000000in}{0.000000in}}{%
\pgfpathmoveto{\pgfqpoint{-0.000000in}{0.000000in}}%
\pgfpathlineto{\pgfqpoint{-0.048611in}{0.000000in}}%
\pgfusepath{stroke,fill}%
}%
\begin{pgfscope}%
\pgfsys@transformshift{0.588387in}{0.918989in}%
\pgfsys@useobject{currentmarker}{}%
\end{pgfscope}%
\end{pgfscope}%
\begin{pgfscope}%
\definecolor{textcolor}{rgb}{0.000000,0.000000,0.000000}%
\pgfsetstrokecolor{textcolor}%
\pgfsetfillcolor{textcolor}%
\pgftext[x=0.289968in, y=0.866227in, left, base]{\color{textcolor}{\rmfamily\fontsize{10.000000}{12.000000}\selectfont\catcode`\^=\active\def^{\ifmmode\sp\else\^{}\fi}\catcode`\%=\active\def%{\%}$\mathdefault{10^{2}}$}}%
\end{pgfscope}%
\begin{pgfscope}%
\pgfsetbuttcap%
\pgfsetroundjoin%
\definecolor{currentfill}{rgb}{0.000000,0.000000,0.000000}%
\pgfsetfillcolor{currentfill}%
\pgfsetlinewidth{0.803000pt}%
\definecolor{currentstroke}{rgb}{0.000000,0.000000,0.000000}%
\pgfsetstrokecolor{currentstroke}%
\pgfsetdash{}{0pt}%
\pgfsys@defobject{currentmarker}{\pgfqpoint{-0.048611in}{0.000000in}}{\pgfqpoint{-0.000000in}{0.000000in}}{%
\pgfpathmoveto{\pgfqpoint{-0.000000in}{0.000000in}}%
\pgfpathlineto{\pgfqpoint{-0.048611in}{0.000000in}}%
\pgfusepath{stroke,fill}%
}%
\begin{pgfscope}%
\pgfsys@transformshift{0.588387in}{1.291489in}%
\pgfsys@useobject{currentmarker}{}%
\end{pgfscope}%
\end{pgfscope}%
\begin{pgfscope}%
\definecolor{textcolor}{rgb}{0.000000,0.000000,0.000000}%
\pgfsetstrokecolor{textcolor}%
\pgfsetfillcolor{textcolor}%
\pgftext[x=0.289968in, y=1.238727in, left, base]{\color{textcolor}{\rmfamily\fontsize{10.000000}{12.000000}\selectfont\catcode`\^=\active\def^{\ifmmode\sp\else\^{}\fi}\catcode`\%=\active\def%{\%}$\mathdefault{10^{3}}$}}%
\end{pgfscope}%
\begin{pgfscope}%
\pgfsetbuttcap%
\pgfsetroundjoin%
\definecolor{currentfill}{rgb}{0.000000,0.000000,0.000000}%
\pgfsetfillcolor{currentfill}%
\pgfsetlinewidth{0.803000pt}%
\definecolor{currentstroke}{rgb}{0.000000,0.000000,0.000000}%
\pgfsetstrokecolor{currentstroke}%
\pgfsetdash{}{0pt}%
\pgfsys@defobject{currentmarker}{\pgfqpoint{-0.048611in}{0.000000in}}{\pgfqpoint{-0.000000in}{0.000000in}}{%
\pgfpathmoveto{\pgfqpoint{-0.000000in}{0.000000in}}%
\pgfpathlineto{\pgfqpoint{-0.048611in}{0.000000in}}%
\pgfusepath{stroke,fill}%
}%
\begin{pgfscope}%
\pgfsys@transformshift{0.588387in}{1.663989in}%
\pgfsys@useobject{currentmarker}{}%
\end{pgfscope}%
\end{pgfscope}%
\begin{pgfscope}%
\definecolor{textcolor}{rgb}{0.000000,0.000000,0.000000}%
\pgfsetstrokecolor{textcolor}%
\pgfsetfillcolor{textcolor}%
\pgftext[x=0.289968in, y=1.611227in, left, base]{\color{textcolor}{\rmfamily\fontsize{10.000000}{12.000000}\selectfont\catcode`\^=\active\def^{\ifmmode\sp\else\^{}\fi}\catcode`\%=\active\def%{\%}$\mathdefault{10^{4}}$}}%
\end{pgfscope}%
\begin{pgfscope}%
\pgfsetbuttcap%
\pgfsetroundjoin%
\definecolor{currentfill}{rgb}{0.000000,0.000000,0.000000}%
\pgfsetfillcolor{currentfill}%
\pgfsetlinewidth{0.803000pt}%
\definecolor{currentstroke}{rgb}{0.000000,0.000000,0.000000}%
\pgfsetstrokecolor{currentstroke}%
\pgfsetdash{}{0pt}%
\pgfsys@defobject{currentmarker}{\pgfqpoint{-0.048611in}{0.000000in}}{\pgfqpoint{-0.000000in}{0.000000in}}{%
\pgfpathmoveto{\pgfqpoint{-0.000000in}{0.000000in}}%
\pgfpathlineto{\pgfqpoint{-0.048611in}{0.000000in}}%
\pgfusepath{stroke,fill}%
}%
\begin{pgfscope}%
\pgfsys@transformshift{0.588387in}{2.036488in}%
\pgfsys@useobject{currentmarker}{}%
\end{pgfscope}%
\end{pgfscope}%
\begin{pgfscope}%
\definecolor{textcolor}{rgb}{0.000000,0.000000,0.000000}%
\pgfsetstrokecolor{textcolor}%
\pgfsetfillcolor{textcolor}%
\pgftext[x=0.289968in, y=1.983727in, left, base]{\color{textcolor}{\rmfamily\fontsize{10.000000}{12.000000}\selectfont\catcode`\^=\active\def^{\ifmmode\sp\else\^{}\fi}\catcode`\%=\active\def%{\%}$\mathdefault{10^{5}}$}}%
\end{pgfscope}%
\begin{pgfscope}%
\pgfsetbuttcap%
\pgfsetroundjoin%
\definecolor{currentfill}{rgb}{0.000000,0.000000,0.000000}%
\pgfsetfillcolor{currentfill}%
\pgfsetlinewidth{0.803000pt}%
\definecolor{currentstroke}{rgb}{0.000000,0.000000,0.000000}%
\pgfsetstrokecolor{currentstroke}%
\pgfsetdash{}{0pt}%
\pgfsys@defobject{currentmarker}{\pgfqpoint{-0.048611in}{0.000000in}}{\pgfqpoint{-0.000000in}{0.000000in}}{%
\pgfpathmoveto{\pgfqpoint{-0.000000in}{0.000000in}}%
\pgfpathlineto{\pgfqpoint{-0.048611in}{0.000000in}}%
\pgfusepath{stroke,fill}%
}%
\begin{pgfscope}%
\pgfsys@transformshift{0.588387in}{2.408988in}%
\pgfsys@useobject{currentmarker}{}%
\end{pgfscope}%
\end{pgfscope}%
\begin{pgfscope}%
\definecolor{textcolor}{rgb}{0.000000,0.000000,0.000000}%
\pgfsetstrokecolor{textcolor}%
\pgfsetfillcolor{textcolor}%
\pgftext[x=0.289968in, y=2.356227in, left, base]{\color{textcolor}{\rmfamily\fontsize{10.000000}{12.000000}\selectfont\catcode`\^=\active\def^{\ifmmode\sp\else\^{}\fi}\catcode`\%=\active\def%{\%}$\mathdefault{10^{6}}$}}%
\end{pgfscope}%
\begin{pgfscope}%
\pgfsetbuttcap%
\pgfsetroundjoin%
\definecolor{currentfill}{rgb}{0.000000,0.000000,0.000000}%
\pgfsetfillcolor{currentfill}%
\pgfsetlinewidth{0.602250pt}%
\definecolor{currentstroke}{rgb}{0.000000,0.000000,0.000000}%
\pgfsetstrokecolor{currentstroke}%
\pgfsetdash{}{0pt}%
\pgfsys@defobject{currentmarker}{\pgfqpoint{-0.027778in}{0.000000in}}{\pgfqpoint{-0.000000in}{0.000000in}}{%
\pgfpathmoveto{\pgfqpoint{-0.000000in}{0.000000in}}%
\pgfpathlineto{\pgfqpoint{-0.027778in}{0.000000in}}%
\pgfusepath{stroke,fill}%
}%
\begin{pgfscope}%
\pgfsys@transformshift{0.588387in}{0.529444in}%
\pgfsys@useobject{currentmarker}{}%
\end{pgfscope}%
\end{pgfscope}%
\begin{pgfscope}%
\pgfsetbuttcap%
\pgfsetroundjoin%
\definecolor{currentfill}{rgb}{0.000000,0.000000,0.000000}%
\pgfsetfillcolor{currentfill}%
\pgfsetlinewidth{0.602250pt}%
\definecolor{currentstroke}{rgb}{0.000000,0.000000,0.000000}%
\pgfsetstrokecolor{currentstroke}%
\pgfsetdash{}{0pt}%
\pgfsys@defobject{currentmarker}{\pgfqpoint{-0.027778in}{0.000000in}}{\pgfqpoint{-0.000000in}{0.000000in}}{%
\pgfpathmoveto{\pgfqpoint{-0.000000in}{0.000000in}}%
\pgfpathlineto{\pgfqpoint{-0.027778in}{0.000000in}}%
\pgfusepath{stroke,fill}%
}%
\begin{pgfscope}%
\pgfsys@transformshift{0.588387in}{0.658623in}%
\pgfsys@useobject{currentmarker}{}%
\end{pgfscope}%
\end{pgfscope}%
\begin{pgfscope}%
\pgfsetbuttcap%
\pgfsetroundjoin%
\definecolor{currentfill}{rgb}{0.000000,0.000000,0.000000}%
\pgfsetfillcolor{currentfill}%
\pgfsetlinewidth{0.602250pt}%
\definecolor{currentstroke}{rgb}{0.000000,0.000000,0.000000}%
\pgfsetstrokecolor{currentstroke}%
\pgfsetdash{}{0pt}%
\pgfsys@defobject{currentmarker}{\pgfqpoint{-0.027778in}{0.000000in}}{\pgfqpoint{-0.000000in}{0.000000in}}{%
\pgfpathmoveto{\pgfqpoint{-0.000000in}{0.000000in}}%
\pgfpathlineto{\pgfqpoint{-0.027778in}{0.000000in}}%
\pgfusepath{stroke,fill}%
}%
\begin{pgfscope}%
\pgfsys@transformshift{0.588387in}{0.724217in}%
\pgfsys@useobject{currentmarker}{}%
\end{pgfscope}%
\end{pgfscope}%
\begin{pgfscope}%
\pgfsetbuttcap%
\pgfsetroundjoin%
\definecolor{currentfill}{rgb}{0.000000,0.000000,0.000000}%
\pgfsetfillcolor{currentfill}%
\pgfsetlinewidth{0.602250pt}%
\definecolor{currentstroke}{rgb}{0.000000,0.000000,0.000000}%
\pgfsetstrokecolor{currentstroke}%
\pgfsetdash{}{0pt}%
\pgfsys@defobject{currentmarker}{\pgfqpoint{-0.027778in}{0.000000in}}{\pgfqpoint{-0.000000in}{0.000000in}}{%
\pgfpathmoveto{\pgfqpoint{-0.000000in}{0.000000in}}%
\pgfpathlineto{\pgfqpoint{-0.027778in}{0.000000in}}%
\pgfusepath{stroke,fill}%
}%
\begin{pgfscope}%
\pgfsys@transformshift{0.588387in}{0.770756in}%
\pgfsys@useobject{currentmarker}{}%
\end{pgfscope}%
\end{pgfscope}%
\begin{pgfscope}%
\pgfsetbuttcap%
\pgfsetroundjoin%
\definecolor{currentfill}{rgb}{0.000000,0.000000,0.000000}%
\pgfsetfillcolor{currentfill}%
\pgfsetlinewidth{0.602250pt}%
\definecolor{currentstroke}{rgb}{0.000000,0.000000,0.000000}%
\pgfsetstrokecolor{currentstroke}%
\pgfsetdash{}{0pt}%
\pgfsys@defobject{currentmarker}{\pgfqpoint{-0.027778in}{0.000000in}}{\pgfqpoint{-0.000000in}{0.000000in}}{%
\pgfpathmoveto{\pgfqpoint{-0.000000in}{0.000000in}}%
\pgfpathlineto{\pgfqpoint{-0.027778in}{0.000000in}}%
\pgfusepath{stroke,fill}%
}%
\begin{pgfscope}%
\pgfsys@transformshift{0.588387in}{0.806855in}%
\pgfsys@useobject{currentmarker}{}%
\end{pgfscope}%
\end{pgfscope}%
\begin{pgfscope}%
\pgfsetbuttcap%
\pgfsetroundjoin%
\definecolor{currentfill}{rgb}{0.000000,0.000000,0.000000}%
\pgfsetfillcolor{currentfill}%
\pgfsetlinewidth{0.602250pt}%
\definecolor{currentstroke}{rgb}{0.000000,0.000000,0.000000}%
\pgfsetstrokecolor{currentstroke}%
\pgfsetdash{}{0pt}%
\pgfsys@defobject{currentmarker}{\pgfqpoint{-0.027778in}{0.000000in}}{\pgfqpoint{-0.000000in}{0.000000in}}{%
\pgfpathmoveto{\pgfqpoint{-0.000000in}{0.000000in}}%
\pgfpathlineto{\pgfqpoint{-0.027778in}{0.000000in}}%
\pgfusepath{stroke,fill}%
}%
\begin{pgfscope}%
\pgfsys@transformshift{0.588387in}{0.836350in}%
\pgfsys@useobject{currentmarker}{}%
\end{pgfscope}%
\end{pgfscope}%
\begin{pgfscope}%
\pgfsetbuttcap%
\pgfsetroundjoin%
\definecolor{currentfill}{rgb}{0.000000,0.000000,0.000000}%
\pgfsetfillcolor{currentfill}%
\pgfsetlinewidth{0.602250pt}%
\definecolor{currentstroke}{rgb}{0.000000,0.000000,0.000000}%
\pgfsetstrokecolor{currentstroke}%
\pgfsetdash{}{0pt}%
\pgfsys@defobject{currentmarker}{\pgfqpoint{-0.027778in}{0.000000in}}{\pgfqpoint{-0.000000in}{0.000000in}}{%
\pgfpathmoveto{\pgfqpoint{-0.000000in}{0.000000in}}%
\pgfpathlineto{\pgfqpoint{-0.027778in}{0.000000in}}%
\pgfusepath{stroke,fill}%
}%
\begin{pgfscope}%
\pgfsys@transformshift{0.588387in}{0.861288in}%
\pgfsys@useobject{currentmarker}{}%
\end{pgfscope}%
\end{pgfscope}%
\begin{pgfscope}%
\pgfsetbuttcap%
\pgfsetroundjoin%
\definecolor{currentfill}{rgb}{0.000000,0.000000,0.000000}%
\pgfsetfillcolor{currentfill}%
\pgfsetlinewidth{0.602250pt}%
\definecolor{currentstroke}{rgb}{0.000000,0.000000,0.000000}%
\pgfsetstrokecolor{currentstroke}%
\pgfsetdash{}{0pt}%
\pgfsys@defobject{currentmarker}{\pgfqpoint{-0.027778in}{0.000000in}}{\pgfqpoint{-0.000000in}{0.000000in}}{%
\pgfpathmoveto{\pgfqpoint{-0.000000in}{0.000000in}}%
\pgfpathlineto{\pgfqpoint{-0.027778in}{0.000000in}}%
\pgfusepath{stroke,fill}%
}%
\begin{pgfscope}%
\pgfsys@transformshift{0.588387in}{0.882890in}%
\pgfsys@useobject{currentmarker}{}%
\end{pgfscope}%
\end{pgfscope}%
\begin{pgfscope}%
\pgfsetbuttcap%
\pgfsetroundjoin%
\definecolor{currentfill}{rgb}{0.000000,0.000000,0.000000}%
\pgfsetfillcolor{currentfill}%
\pgfsetlinewidth{0.602250pt}%
\definecolor{currentstroke}{rgb}{0.000000,0.000000,0.000000}%
\pgfsetstrokecolor{currentstroke}%
\pgfsetdash{}{0pt}%
\pgfsys@defobject{currentmarker}{\pgfqpoint{-0.027778in}{0.000000in}}{\pgfqpoint{-0.000000in}{0.000000in}}{%
\pgfpathmoveto{\pgfqpoint{-0.000000in}{0.000000in}}%
\pgfpathlineto{\pgfqpoint{-0.027778in}{0.000000in}}%
\pgfusepath{stroke,fill}%
}%
\begin{pgfscope}%
\pgfsys@transformshift{0.588387in}{0.901944in}%
\pgfsys@useobject{currentmarker}{}%
\end{pgfscope}%
\end{pgfscope}%
\begin{pgfscope}%
\pgfsetbuttcap%
\pgfsetroundjoin%
\definecolor{currentfill}{rgb}{0.000000,0.000000,0.000000}%
\pgfsetfillcolor{currentfill}%
\pgfsetlinewidth{0.602250pt}%
\definecolor{currentstroke}{rgb}{0.000000,0.000000,0.000000}%
\pgfsetstrokecolor{currentstroke}%
\pgfsetdash{}{0pt}%
\pgfsys@defobject{currentmarker}{\pgfqpoint{-0.027778in}{0.000000in}}{\pgfqpoint{-0.000000in}{0.000000in}}{%
\pgfpathmoveto{\pgfqpoint{-0.000000in}{0.000000in}}%
\pgfpathlineto{\pgfqpoint{-0.027778in}{0.000000in}}%
\pgfusepath{stroke,fill}%
}%
\begin{pgfscope}%
\pgfsys@transformshift{0.588387in}{1.031122in}%
\pgfsys@useobject{currentmarker}{}%
\end{pgfscope}%
\end{pgfscope}%
\begin{pgfscope}%
\pgfsetbuttcap%
\pgfsetroundjoin%
\definecolor{currentfill}{rgb}{0.000000,0.000000,0.000000}%
\pgfsetfillcolor{currentfill}%
\pgfsetlinewidth{0.602250pt}%
\definecolor{currentstroke}{rgb}{0.000000,0.000000,0.000000}%
\pgfsetstrokecolor{currentstroke}%
\pgfsetdash{}{0pt}%
\pgfsys@defobject{currentmarker}{\pgfqpoint{-0.027778in}{0.000000in}}{\pgfqpoint{-0.000000in}{0.000000in}}{%
\pgfpathmoveto{\pgfqpoint{-0.000000in}{0.000000in}}%
\pgfpathlineto{\pgfqpoint{-0.027778in}{0.000000in}}%
\pgfusepath{stroke,fill}%
}%
\begin{pgfscope}%
\pgfsys@transformshift{0.588387in}{1.096716in}%
\pgfsys@useobject{currentmarker}{}%
\end{pgfscope}%
\end{pgfscope}%
\begin{pgfscope}%
\pgfsetbuttcap%
\pgfsetroundjoin%
\definecolor{currentfill}{rgb}{0.000000,0.000000,0.000000}%
\pgfsetfillcolor{currentfill}%
\pgfsetlinewidth{0.602250pt}%
\definecolor{currentstroke}{rgb}{0.000000,0.000000,0.000000}%
\pgfsetstrokecolor{currentstroke}%
\pgfsetdash{}{0pt}%
\pgfsys@defobject{currentmarker}{\pgfqpoint{-0.027778in}{0.000000in}}{\pgfqpoint{-0.000000in}{0.000000in}}{%
\pgfpathmoveto{\pgfqpoint{-0.000000in}{0.000000in}}%
\pgfpathlineto{\pgfqpoint{-0.027778in}{0.000000in}}%
\pgfusepath{stroke,fill}%
}%
\begin{pgfscope}%
\pgfsys@transformshift{0.588387in}{1.143256in}%
\pgfsys@useobject{currentmarker}{}%
\end{pgfscope}%
\end{pgfscope}%
\begin{pgfscope}%
\pgfsetbuttcap%
\pgfsetroundjoin%
\definecolor{currentfill}{rgb}{0.000000,0.000000,0.000000}%
\pgfsetfillcolor{currentfill}%
\pgfsetlinewidth{0.602250pt}%
\definecolor{currentstroke}{rgb}{0.000000,0.000000,0.000000}%
\pgfsetstrokecolor{currentstroke}%
\pgfsetdash{}{0pt}%
\pgfsys@defobject{currentmarker}{\pgfqpoint{-0.027778in}{0.000000in}}{\pgfqpoint{-0.000000in}{0.000000in}}{%
\pgfpathmoveto{\pgfqpoint{-0.000000in}{0.000000in}}%
\pgfpathlineto{\pgfqpoint{-0.027778in}{0.000000in}}%
\pgfusepath{stroke,fill}%
}%
\begin{pgfscope}%
\pgfsys@transformshift{0.588387in}{1.179355in}%
\pgfsys@useobject{currentmarker}{}%
\end{pgfscope}%
\end{pgfscope}%
\begin{pgfscope}%
\pgfsetbuttcap%
\pgfsetroundjoin%
\definecolor{currentfill}{rgb}{0.000000,0.000000,0.000000}%
\pgfsetfillcolor{currentfill}%
\pgfsetlinewidth{0.602250pt}%
\definecolor{currentstroke}{rgb}{0.000000,0.000000,0.000000}%
\pgfsetstrokecolor{currentstroke}%
\pgfsetdash{}{0pt}%
\pgfsys@defobject{currentmarker}{\pgfqpoint{-0.027778in}{0.000000in}}{\pgfqpoint{-0.000000in}{0.000000in}}{%
\pgfpathmoveto{\pgfqpoint{-0.000000in}{0.000000in}}%
\pgfpathlineto{\pgfqpoint{-0.027778in}{0.000000in}}%
\pgfusepath{stroke,fill}%
}%
\begin{pgfscope}%
\pgfsys@transformshift{0.588387in}{1.208850in}%
\pgfsys@useobject{currentmarker}{}%
\end{pgfscope}%
\end{pgfscope}%
\begin{pgfscope}%
\pgfsetbuttcap%
\pgfsetroundjoin%
\definecolor{currentfill}{rgb}{0.000000,0.000000,0.000000}%
\pgfsetfillcolor{currentfill}%
\pgfsetlinewidth{0.602250pt}%
\definecolor{currentstroke}{rgb}{0.000000,0.000000,0.000000}%
\pgfsetstrokecolor{currentstroke}%
\pgfsetdash{}{0pt}%
\pgfsys@defobject{currentmarker}{\pgfqpoint{-0.027778in}{0.000000in}}{\pgfqpoint{-0.000000in}{0.000000in}}{%
\pgfpathmoveto{\pgfqpoint{-0.000000in}{0.000000in}}%
\pgfpathlineto{\pgfqpoint{-0.027778in}{0.000000in}}%
\pgfusepath{stroke,fill}%
}%
\begin{pgfscope}%
\pgfsys@transformshift{0.588387in}{1.233788in}%
\pgfsys@useobject{currentmarker}{}%
\end{pgfscope}%
\end{pgfscope}%
\begin{pgfscope}%
\pgfsetbuttcap%
\pgfsetroundjoin%
\definecolor{currentfill}{rgb}{0.000000,0.000000,0.000000}%
\pgfsetfillcolor{currentfill}%
\pgfsetlinewidth{0.602250pt}%
\definecolor{currentstroke}{rgb}{0.000000,0.000000,0.000000}%
\pgfsetstrokecolor{currentstroke}%
\pgfsetdash{}{0pt}%
\pgfsys@defobject{currentmarker}{\pgfqpoint{-0.027778in}{0.000000in}}{\pgfqpoint{-0.000000in}{0.000000in}}{%
\pgfpathmoveto{\pgfqpoint{-0.000000in}{0.000000in}}%
\pgfpathlineto{\pgfqpoint{-0.027778in}{0.000000in}}%
\pgfusepath{stroke,fill}%
}%
\begin{pgfscope}%
\pgfsys@transformshift{0.588387in}{1.255390in}%
\pgfsys@useobject{currentmarker}{}%
\end{pgfscope}%
\end{pgfscope}%
\begin{pgfscope}%
\pgfsetbuttcap%
\pgfsetroundjoin%
\definecolor{currentfill}{rgb}{0.000000,0.000000,0.000000}%
\pgfsetfillcolor{currentfill}%
\pgfsetlinewidth{0.602250pt}%
\definecolor{currentstroke}{rgb}{0.000000,0.000000,0.000000}%
\pgfsetstrokecolor{currentstroke}%
\pgfsetdash{}{0pt}%
\pgfsys@defobject{currentmarker}{\pgfqpoint{-0.027778in}{0.000000in}}{\pgfqpoint{-0.000000in}{0.000000in}}{%
\pgfpathmoveto{\pgfqpoint{-0.000000in}{0.000000in}}%
\pgfpathlineto{\pgfqpoint{-0.027778in}{0.000000in}}%
\pgfusepath{stroke,fill}%
}%
\begin{pgfscope}%
\pgfsys@transformshift{0.588387in}{1.274444in}%
\pgfsys@useobject{currentmarker}{}%
\end{pgfscope}%
\end{pgfscope}%
\begin{pgfscope}%
\pgfsetbuttcap%
\pgfsetroundjoin%
\definecolor{currentfill}{rgb}{0.000000,0.000000,0.000000}%
\pgfsetfillcolor{currentfill}%
\pgfsetlinewidth{0.602250pt}%
\definecolor{currentstroke}{rgb}{0.000000,0.000000,0.000000}%
\pgfsetstrokecolor{currentstroke}%
\pgfsetdash{}{0pt}%
\pgfsys@defobject{currentmarker}{\pgfqpoint{-0.027778in}{0.000000in}}{\pgfqpoint{-0.000000in}{0.000000in}}{%
\pgfpathmoveto{\pgfqpoint{-0.000000in}{0.000000in}}%
\pgfpathlineto{\pgfqpoint{-0.027778in}{0.000000in}}%
\pgfusepath{stroke,fill}%
}%
\begin{pgfscope}%
\pgfsys@transformshift{0.588387in}{1.403622in}%
\pgfsys@useobject{currentmarker}{}%
\end{pgfscope}%
\end{pgfscope}%
\begin{pgfscope}%
\pgfsetbuttcap%
\pgfsetroundjoin%
\definecolor{currentfill}{rgb}{0.000000,0.000000,0.000000}%
\pgfsetfillcolor{currentfill}%
\pgfsetlinewidth{0.602250pt}%
\definecolor{currentstroke}{rgb}{0.000000,0.000000,0.000000}%
\pgfsetstrokecolor{currentstroke}%
\pgfsetdash{}{0pt}%
\pgfsys@defobject{currentmarker}{\pgfqpoint{-0.027778in}{0.000000in}}{\pgfqpoint{-0.000000in}{0.000000in}}{%
\pgfpathmoveto{\pgfqpoint{-0.000000in}{0.000000in}}%
\pgfpathlineto{\pgfqpoint{-0.027778in}{0.000000in}}%
\pgfusepath{stroke,fill}%
}%
\begin{pgfscope}%
\pgfsys@transformshift{0.588387in}{1.469216in}%
\pgfsys@useobject{currentmarker}{}%
\end{pgfscope}%
\end{pgfscope}%
\begin{pgfscope}%
\pgfsetbuttcap%
\pgfsetroundjoin%
\definecolor{currentfill}{rgb}{0.000000,0.000000,0.000000}%
\pgfsetfillcolor{currentfill}%
\pgfsetlinewidth{0.602250pt}%
\definecolor{currentstroke}{rgb}{0.000000,0.000000,0.000000}%
\pgfsetstrokecolor{currentstroke}%
\pgfsetdash{}{0pt}%
\pgfsys@defobject{currentmarker}{\pgfqpoint{-0.027778in}{0.000000in}}{\pgfqpoint{-0.000000in}{0.000000in}}{%
\pgfpathmoveto{\pgfqpoint{-0.000000in}{0.000000in}}%
\pgfpathlineto{\pgfqpoint{-0.027778in}{0.000000in}}%
\pgfusepath{stroke,fill}%
}%
\begin{pgfscope}%
\pgfsys@transformshift{0.588387in}{1.515756in}%
\pgfsys@useobject{currentmarker}{}%
\end{pgfscope}%
\end{pgfscope}%
\begin{pgfscope}%
\pgfsetbuttcap%
\pgfsetroundjoin%
\definecolor{currentfill}{rgb}{0.000000,0.000000,0.000000}%
\pgfsetfillcolor{currentfill}%
\pgfsetlinewidth{0.602250pt}%
\definecolor{currentstroke}{rgb}{0.000000,0.000000,0.000000}%
\pgfsetstrokecolor{currentstroke}%
\pgfsetdash{}{0pt}%
\pgfsys@defobject{currentmarker}{\pgfqpoint{-0.027778in}{0.000000in}}{\pgfqpoint{-0.000000in}{0.000000in}}{%
\pgfpathmoveto{\pgfqpoint{-0.000000in}{0.000000in}}%
\pgfpathlineto{\pgfqpoint{-0.027778in}{0.000000in}}%
\pgfusepath{stroke,fill}%
}%
\begin{pgfscope}%
\pgfsys@transformshift{0.588387in}{1.551855in}%
\pgfsys@useobject{currentmarker}{}%
\end{pgfscope}%
\end{pgfscope}%
\begin{pgfscope}%
\pgfsetbuttcap%
\pgfsetroundjoin%
\definecolor{currentfill}{rgb}{0.000000,0.000000,0.000000}%
\pgfsetfillcolor{currentfill}%
\pgfsetlinewidth{0.602250pt}%
\definecolor{currentstroke}{rgb}{0.000000,0.000000,0.000000}%
\pgfsetstrokecolor{currentstroke}%
\pgfsetdash{}{0pt}%
\pgfsys@defobject{currentmarker}{\pgfqpoint{-0.027778in}{0.000000in}}{\pgfqpoint{-0.000000in}{0.000000in}}{%
\pgfpathmoveto{\pgfqpoint{-0.000000in}{0.000000in}}%
\pgfpathlineto{\pgfqpoint{-0.027778in}{0.000000in}}%
\pgfusepath{stroke,fill}%
}%
\begin{pgfscope}%
\pgfsys@transformshift{0.588387in}{1.581350in}%
\pgfsys@useobject{currentmarker}{}%
\end{pgfscope}%
\end{pgfscope}%
\begin{pgfscope}%
\pgfsetbuttcap%
\pgfsetroundjoin%
\definecolor{currentfill}{rgb}{0.000000,0.000000,0.000000}%
\pgfsetfillcolor{currentfill}%
\pgfsetlinewidth{0.602250pt}%
\definecolor{currentstroke}{rgb}{0.000000,0.000000,0.000000}%
\pgfsetstrokecolor{currentstroke}%
\pgfsetdash{}{0pt}%
\pgfsys@defobject{currentmarker}{\pgfqpoint{-0.027778in}{0.000000in}}{\pgfqpoint{-0.000000in}{0.000000in}}{%
\pgfpathmoveto{\pgfqpoint{-0.000000in}{0.000000in}}%
\pgfpathlineto{\pgfqpoint{-0.027778in}{0.000000in}}%
\pgfusepath{stroke,fill}%
}%
\begin{pgfscope}%
\pgfsys@transformshift{0.588387in}{1.606288in}%
\pgfsys@useobject{currentmarker}{}%
\end{pgfscope}%
\end{pgfscope}%
\begin{pgfscope}%
\pgfsetbuttcap%
\pgfsetroundjoin%
\definecolor{currentfill}{rgb}{0.000000,0.000000,0.000000}%
\pgfsetfillcolor{currentfill}%
\pgfsetlinewidth{0.602250pt}%
\definecolor{currentstroke}{rgb}{0.000000,0.000000,0.000000}%
\pgfsetstrokecolor{currentstroke}%
\pgfsetdash{}{0pt}%
\pgfsys@defobject{currentmarker}{\pgfqpoint{-0.027778in}{0.000000in}}{\pgfqpoint{-0.000000in}{0.000000in}}{%
\pgfpathmoveto{\pgfqpoint{-0.000000in}{0.000000in}}%
\pgfpathlineto{\pgfqpoint{-0.027778in}{0.000000in}}%
\pgfusepath{stroke,fill}%
}%
\begin{pgfscope}%
\pgfsys@transformshift{0.588387in}{1.627890in}%
\pgfsys@useobject{currentmarker}{}%
\end{pgfscope}%
\end{pgfscope}%
\begin{pgfscope}%
\pgfsetbuttcap%
\pgfsetroundjoin%
\definecolor{currentfill}{rgb}{0.000000,0.000000,0.000000}%
\pgfsetfillcolor{currentfill}%
\pgfsetlinewidth{0.602250pt}%
\definecolor{currentstroke}{rgb}{0.000000,0.000000,0.000000}%
\pgfsetstrokecolor{currentstroke}%
\pgfsetdash{}{0pt}%
\pgfsys@defobject{currentmarker}{\pgfqpoint{-0.027778in}{0.000000in}}{\pgfqpoint{-0.000000in}{0.000000in}}{%
\pgfpathmoveto{\pgfqpoint{-0.000000in}{0.000000in}}%
\pgfpathlineto{\pgfqpoint{-0.027778in}{0.000000in}}%
\pgfusepath{stroke,fill}%
}%
\begin{pgfscope}%
\pgfsys@transformshift{0.588387in}{1.646944in}%
\pgfsys@useobject{currentmarker}{}%
\end{pgfscope}%
\end{pgfscope}%
\begin{pgfscope}%
\pgfsetbuttcap%
\pgfsetroundjoin%
\definecolor{currentfill}{rgb}{0.000000,0.000000,0.000000}%
\pgfsetfillcolor{currentfill}%
\pgfsetlinewidth{0.602250pt}%
\definecolor{currentstroke}{rgb}{0.000000,0.000000,0.000000}%
\pgfsetstrokecolor{currentstroke}%
\pgfsetdash{}{0pt}%
\pgfsys@defobject{currentmarker}{\pgfqpoint{-0.027778in}{0.000000in}}{\pgfqpoint{-0.000000in}{0.000000in}}{%
\pgfpathmoveto{\pgfqpoint{-0.000000in}{0.000000in}}%
\pgfpathlineto{\pgfqpoint{-0.027778in}{0.000000in}}%
\pgfusepath{stroke,fill}%
}%
\begin{pgfscope}%
\pgfsys@transformshift{0.588387in}{1.776122in}%
\pgfsys@useobject{currentmarker}{}%
\end{pgfscope}%
\end{pgfscope}%
\begin{pgfscope}%
\pgfsetbuttcap%
\pgfsetroundjoin%
\definecolor{currentfill}{rgb}{0.000000,0.000000,0.000000}%
\pgfsetfillcolor{currentfill}%
\pgfsetlinewidth{0.602250pt}%
\definecolor{currentstroke}{rgb}{0.000000,0.000000,0.000000}%
\pgfsetstrokecolor{currentstroke}%
\pgfsetdash{}{0pt}%
\pgfsys@defobject{currentmarker}{\pgfqpoint{-0.027778in}{0.000000in}}{\pgfqpoint{-0.000000in}{0.000000in}}{%
\pgfpathmoveto{\pgfqpoint{-0.000000in}{0.000000in}}%
\pgfpathlineto{\pgfqpoint{-0.027778in}{0.000000in}}%
\pgfusepath{stroke,fill}%
}%
\begin{pgfscope}%
\pgfsys@transformshift{0.588387in}{1.841716in}%
\pgfsys@useobject{currentmarker}{}%
\end{pgfscope}%
\end{pgfscope}%
\begin{pgfscope}%
\pgfsetbuttcap%
\pgfsetroundjoin%
\definecolor{currentfill}{rgb}{0.000000,0.000000,0.000000}%
\pgfsetfillcolor{currentfill}%
\pgfsetlinewidth{0.602250pt}%
\definecolor{currentstroke}{rgb}{0.000000,0.000000,0.000000}%
\pgfsetstrokecolor{currentstroke}%
\pgfsetdash{}{0pt}%
\pgfsys@defobject{currentmarker}{\pgfqpoint{-0.027778in}{0.000000in}}{\pgfqpoint{-0.000000in}{0.000000in}}{%
\pgfpathmoveto{\pgfqpoint{-0.000000in}{0.000000in}}%
\pgfpathlineto{\pgfqpoint{-0.027778in}{0.000000in}}%
\pgfusepath{stroke,fill}%
}%
\begin{pgfscope}%
\pgfsys@transformshift{0.588387in}{1.888256in}%
\pgfsys@useobject{currentmarker}{}%
\end{pgfscope}%
\end{pgfscope}%
\begin{pgfscope}%
\pgfsetbuttcap%
\pgfsetroundjoin%
\definecolor{currentfill}{rgb}{0.000000,0.000000,0.000000}%
\pgfsetfillcolor{currentfill}%
\pgfsetlinewidth{0.602250pt}%
\definecolor{currentstroke}{rgb}{0.000000,0.000000,0.000000}%
\pgfsetstrokecolor{currentstroke}%
\pgfsetdash{}{0pt}%
\pgfsys@defobject{currentmarker}{\pgfqpoint{-0.027778in}{0.000000in}}{\pgfqpoint{-0.000000in}{0.000000in}}{%
\pgfpathmoveto{\pgfqpoint{-0.000000in}{0.000000in}}%
\pgfpathlineto{\pgfqpoint{-0.027778in}{0.000000in}}%
\pgfusepath{stroke,fill}%
}%
\begin{pgfscope}%
\pgfsys@transformshift{0.588387in}{1.924355in}%
\pgfsys@useobject{currentmarker}{}%
\end{pgfscope}%
\end{pgfscope}%
\begin{pgfscope}%
\pgfsetbuttcap%
\pgfsetroundjoin%
\definecolor{currentfill}{rgb}{0.000000,0.000000,0.000000}%
\pgfsetfillcolor{currentfill}%
\pgfsetlinewidth{0.602250pt}%
\definecolor{currentstroke}{rgb}{0.000000,0.000000,0.000000}%
\pgfsetstrokecolor{currentstroke}%
\pgfsetdash{}{0pt}%
\pgfsys@defobject{currentmarker}{\pgfqpoint{-0.027778in}{0.000000in}}{\pgfqpoint{-0.000000in}{0.000000in}}{%
\pgfpathmoveto{\pgfqpoint{-0.000000in}{0.000000in}}%
\pgfpathlineto{\pgfqpoint{-0.027778in}{0.000000in}}%
\pgfusepath{stroke,fill}%
}%
\begin{pgfscope}%
\pgfsys@transformshift{0.588387in}{1.953850in}%
\pgfsys@useobject{currentmarker}{}%
\end{pgfscope}%
\end{pgfscope}%
\begin{pgfscope}%
\pgfsetbuttcap%
\pgfsetroundjoin%
\definecolor{currentfill}{rgb}{0.000000,0.000000,0.000000}%
\pgfsetfillcolor{currentfill}%
\pgfsetlinewidth{0.602250pt}%
\definecolor{currentstroke}{rgb}{0.000000,0.000000,0.000000}%
\pgfsetstrokecolor{currentstroke}%
\pgfsetdash{}{0pt}%
\pgfsys@defobject{currentmarker}{\pgfqpoint{-0.027778in}{0.000000in}}{\pgfqpoint{-0.000000in}{0.000000in}}{%
\pgfpathmoveto{\pgfqpoint{-0.000000in}{0.000000in}}%
\pgfpathlineto{\pgfqpoint{-0.027778in}{0.000000in}}%
\pgfusepath{stroke,fill}%
}%
\begin{pgfscope}%
\pgfsys@transformshift{0.588387in}{1.978787in}%
\pgfsys@useobject{currentmarker}{}%
\end{pgfscope}%
\end{pgfscope}%
\begin{pgfscope}%
\pgfsetbuttcap%
\pgfsetroundjoin%
\definecolor{currentfill}{rgb}{0.000000,0.000000,0.000000}%
\pgfsetfillcolor{currentfill}%
\pgfsetlinewidth{0.602250pt}%
\definecolor{currentstroke}{rgb}{0.000000,0.000000,0.000000}%
\pgfsetstrokecolor{currentstroke}%
\pgfsetdash{}{0pt}%
\pgfsys@defobject{currentmarker}{\pgfqpoint{-0.027778in}{0.000000in}}{\pgfqpoint{-0.000000in}{0.000000in}}{%
\pgfpathmoveto{\pgfqpoint{-0.000000in}{0.000000in}}%
\pgfpathlineto{\pgfqpoint{-0.027778in}{0.000000in}}%
\pgfusepath{stroke,fill}%
}%
\begin{pgfscope}%
\pgfsys@transformshift{0.588387in}{2.000389in}%
\pgfsys@useobject{currentmarker}{}%
\end{pgfscope}%
\end{pgfscope}%
\begin{pgfscope}%
\pgfsetbuttcap%
\pgfsetroundjoin%
\definecolor{currentfill}{rgb}{0.000000,0.000000,0.000000}%
\pgfsetfillcolor{currentfill}%
\pgfsetlinewidth{0.602250pt}%
\definecolor{currentstroke}{rgb}{0.000000,0.000000,0.000000}%
\pgfsetstrokecolor{currentstroke}%
\pgfsetdash{}{0pt}%
\pgfsys@defobject{currentmarker}{\pgfqpoint{-0.027778in}{0.000000in}}{\pgfqpoint{-0.000000in}{0.000000in}}{%
\pgfpathmoveto{\pgfqpoint{-0.000000in}{0.000000in}}%
\pgfpathlineto{\pgfqpoint{-0.027778in}{0.000000in}}%
\pgfusepath{stroke,fill}%
}%
\begin{pgfscope}%
\pgfsys@transformshift{0.588387in}{2.019444in}%
\pgfsys@useobject{currentmarker}{}%
\end{pgfscope}%
\end{pgfscope}%
\begin{pgfscope}%
\pgfsetbuttcap%
\pgfsetroundjoin%
\definecolor{currentfill}{rgb}{0.000000,0.000000,0.000000}%
\pgfsetfillcolor{currentfill}%
\pgfsetlinewidth{0.602250pt}%
\definecolor{currentstroke}{rgb}{0.000000,0.000000,0.000000}%
\pgfsetstrokecolor{currentstroke}%
\pgfsetdash{}{0pt}%
\pgfsys@defobject{currentmarker}{\pgfqpoint{-0.027778in}{0.000000in}}{\pgfqpoint{-0.000000in}{0.000000in}}{%
\pgfpathmoveto{\pgfqpoint{-0.000000in}{0.000000in}}%
\pgfpathlineto{\pgfqpoint{-0.027778in}{0.000000in}}%
\pgfusepath{stroke,fill}%
}%
\begin{pgfscope}%
\pgfsys@transformshift{0.588387in}{2.148622in}%
\pgfsys@useobject{currentmarker}{}%
\end{pgfscope}%
\end{pgfscope}%
\begin{pgfscope}%
\pgfsetbuttcap%
\pgfsetroundjoin%
\definecolor{currentfill}{rgb}{0.000000,0.000000,0.000000}%
\pgfsetfillcolor{currentfill}%
\pgfsetlinewidth{0.602250pt}%
\definecolor{currentstroke}{rgb}{0.000000,0.000000,0.000000}%
\pgfsetstrokecolor{currentstroke}%
\pgfsetdash{}{0pt}%
\pgfsys@defobject{currentmarker}{\pgfqpoint{-0.027778in}{0.000000in}}{\pgfqpoint{-0.000000in}{0.000000in}}{%
\pgfpathmoveto{\pgfqpoint{-0.000000in}{0.000000in}}%
\pgfpathlineto{\pgfqpoint{-0.027778in}{0.000000in}}%
\pgfusepath{stroke,fill}%
}%
\begin{pgfscope}%
\pgfsys@transformshift{0.588387in}{2.214216in}%
\pgfsys@useobject{currentmarker}{}%
\end{pgfscope}%
\end{pgfscope}%
\begin{pgfscope}%
\pgfsetbuttcap%
\pgfsetroundjoin%
\definecolor{currentfill}{rgb}{0.000000,0.000000,0.000000}%
\pgfsetfillcolor{currentfill}%
\pgfsetlinewidth{0.602250pt}%
\definecolor{currentstroke}{rgb}{0.000000,0.000000,0.000000}%
\pgfsetstrokecolor{currentstroke}%
\pgfsetdash{}{0pt}%
\pgfsys@defobject{currentmarker}{\pgfqpoint{-0.027778in}{0.000000in}}{\pgfqpoint{-0.000000in}{0.000000in}}{%
\pgfpathmoveto{\pgfqpoint{-0.000000in}{0.000000in}}%
\pgfpathlineto{\pgfqpoint{-0.027778in}{0.000000in}}%
\pgfusepath{stroke,fill}%
}%
\begin{pgfscope}%
\pgfsys@transformshift{0.588387in}{2.260756in}%
\pgfsys@useobject{currentmarker}{}%
\end{pgfscope}%
\end{pgfscope}%
\begin{pgfscope}%
\pgfsetbuttcap%
\pgfsetroundjoin%
\definecolor{currentfill}{rgb}{0.000000,0.000000,0.000000}%
\pgfsetfillcolor{currentfill}%
\pgfsetlinewidth{0.602250pt}%
\definecolor{currentstroke}{rgb}{0.000000,0.000000,0.000000}%
\pgfsetstrokecolor{currentstroke}%
\pgfsetdash{}{0pt}%
\pgfsys@defobject{currentmarker}{\pgfqpoint{-0.027778in}{0.000000in}}{\pgfqpoint{-0.000000in}{0.000000in}}{%
\pgfpathmoveto{\pgfqpoint{-0.000000in}{0.000000in}}%
\pgfpathlineto{\pgfqpoint{-0.027778in}{0.000000in}}%
\pgfusepath{stroke,fill}%
}%
\begin{pgfscope}%
\pgfsys@transformshift{0.588387in}{2.296855in}%
\pgfsys@useobject{currentmarker}{}%
\end{pgfscope}%
\end{pgfscope}%
\begin{pgfscope}%
\pgfsetbuttcap%
\pgfsetroundjoin%
\definecolor{currentfill}{rgb}{0.000000,0.000000,0.000000}%
\pgfsetfillcolor{currentfill}%
\pgfsetlinewidth{0.602250pt}%
\definecolor{currentstroke}{rgb}{0.000000,0.000000,0.000000}%
\pgfsetstrokecolor{currentstroke}%
\pgfsetdash{}{0pt}%
\pgfsys@defobject{currentmarker}{\pgfqpoint{-0.027778in}{0.000000in}}{\pgfqpoint{-0.000000in}{0.000000in}}{%
\pgfpathmoveto{\pgfqpoint{-0.000000in}{0.000000in}}%
\pgfpathlineto{\pgfqpoint{-0.027778in}{0.000000in}}%
\pgfusepath{stroke,fill}%
}%
\begin{pgfscope}%
\pgfsys@transformshift{0.588387in}{2.326350in}%
\pgfsys@useobject{currentmarker}{}%
\end{pgfscope}%
\end{pgfscope}%
\begin{pgfscope}%
\pgfsetbuttcap%
\pgfsetroundjoin%
\definecolor{currentfill}{rgb}{0.000000,0.000000,0.000000}%
\pgfsetfillcolor{currentfill}%
\pgfsetlinewidth{0.602250pt}%
\definecolor{currentstroke}{rgb}{0.000000,0.000000,0.000000}%
\pgfsetstrokecolor{currentstroke}%
\pgfsetdash{}{0pt}%
\pgfsys@defobject{currentmarker}{\pgfqpoint{-0.027778in}{0.000000in}}{\pgfqpoint{-0.000000in}{0.000000in}}{%
\pgfpathmoveto{\pgfqpoint{-0.000000in}{0.000000in}}%
\pgfpathlineto{\pgfqpoint{-0.027778in}{0.000000in}}%
\pgfusepath{stroke,fill}%
}%
\begin{pgfscope}%
\pgfsys@transformshift{0.588387in}{2.351287in}%
\pgfsys@useobject{currentmarker}{}%
\end{pgfscope}%
\end{pgfscope}%
\begin{pgfscope}%
\pgfsetbuttcap%
\pgfsetroundjoin%
\definecolor{currentfill}{rgb}{0.000000,0.000000,0.000000}%
\pgfsetfillcolor{currentfill}%
\pgfsetlinewidth{0.602250pt}%
\definecolor{currentstroke}{rgb}{0.000000,0.000000,0.000000}%
\pgfsetstrokecolor{currentstroke}%
\pgfsetdash{}{0pt}%
\pgfsys@defobject{currentmarker}{\pgfqpoint{-0.027778in}{0.000000in}}{\pgfqpoint{-0.000000in}{0.000000in}}{%
\pgfpathmoveto{\pgfqpoint{-0.000000in}{0.000000in}}%
\pgfpathlineto{\pgfqpoint{-0.027778in}{0.000000in}}%
\pgfusepath{stroke,fill}%
}%
\begin{pgfscope}%
\pgfsys@transformshift{0.588387in}{2.372889in}%
\pgfsys@useobject{currentmarker}{}%
\end{pgfscope}%
\end{pgfscope}%
\begin{pgfscope}%
\pgfsetbuttcap%
\pgfsetroundjoin%
\definecolor{currentfill}{rgb}{0.000000,0.000000,0.000000}%
\pgfsetfillcolor{currentfill}%
\pgfsetlinewidth{0.602250pt}%
\definecolor{currentstroke}{rgb}{0.000000,0.000000,0.000000}%
\pgfsetstrokecolor{currentstroke}%
\pgfsetdash{}{0pt}%
\pgfsys@defobject{currentmarker}{\pgfqpoint{-0.027778in}{0.000000in}}{\pgfqpoint{-0.000000in}{0.000000in}}{%
\pgfpathmoveto{\pgfqpoint{-0.000000in}{0.000000in}}%
\pgfpathlineto{\pgfqpoint{-0.027778in}{0.000000in}}%
\pgfusepath{stroke,fill}%
}%
\begin{pgfscope}%
\pgfsys@transformshift{0.588387in}{2.391944in}%
\pgfsys@useobject{currentmarker}{}%
\end{pgfscope}%
\end{pgfscope}%
\begin{pgfscope}%
\pgfsetbuttcap%
\pgfsetroundjoin%
\definecolor{currentfill}{rgb}{0.000000,0.000000,0.000000}%
\pgfsetfillcolor{currentfill}%
\pgfsetlinewidth{0.602250pt}%
\definecolor{currentstroke}{rgb}{0.000000,0.000000,0.000000}%
\pgfsetstrokecolor{currentstroke}%
\pgfsetdash{}{0pt}%
\pgfsys@defobject{currentmarker}{\pgfqpoint{-0.027778in}{0.000000in}}{\pgfqpoint{-0.000000in}{0.000000in}}{%
\pgfpathmoveto{\pgfqpoint{-0.000000in}{0.000000in}}%
\pgfpathlineto{\pgfqpoint{-0.027778in}{0.000000in}}%
\pgfusepath{stroke,fill}%
}%
\begin{pgfscope}%
\pgfsys@transformshift{0.588387in}{2.521122in}%
\pgfsys@useobject{currentmarker}{}%
\end{pgfscope}%
\end{pgfscope}%
\begin{pgfscope}%
\pgfsetbuttcap%
\pgfsetroundjoin%
\definecolor{currentfill}{rgb}{0.000000,0.000000,0.000000}%
\pgfsetfillcolor{currentfill}%
\pgfsetlinewidth{0.602250pt}%
\definecolor{currentstroke}{rgb}{0.000000,0.000000,0.000000}%
\pgfsetstrokecolor{currentstroke}%
\pgfsetdash{}{0pt}%
\pgfsys@defobject{currentmarker}{\pgfqpoint{-0.027778in}{0.000000in}}{\pgfqpoint{-0.000000in}{0.000000in}}{%
\pgfpathmoveto{\pgfqpoint{-0.000000in}{0.000000in}}%
\pgfpathlineto{\pgfqpoint{-0.027778in}{0.000000in}}%
\pgfusepath{stroke,fill}%
}%
\begin{pgfscope}%
\pgfsys@transformshift{0.588387in}{2.586716in}%
\pgfsys@useobject{currentmarker}{}%
\end{pgfscope}%
\end{pgfscope}%
\begin{pgfscope}%
\pgfsetbuttcap%
\pgfsetroundjoin%
\definecolor{currentfill}{rgb}{0.000000,0.000000,0.000000}%
\pgfsetfillcolor{currentfill}%
\pgfsetlinewidth{0.602250pt}%
\definecolor{currentstroke}{rgb}{0.000000,0.000000,0.000000}%
\pgfsetstrokecolor{currentstroke}%
\pgfsetdash{}{0pt}%
\pgfsys@defobject{currentmarker}{\pgfqpoint{-0.027778in}{0.000000in}}{\pgfqpoint{-0.000000in}{0.000000in}}{%
\pgfpathmoveto{\pgfqpoint{-0.000000in}{0.000000in}}%
\pgfpathlineto{\pgfqpoint{-0.027778in}{0.000000in}}%
\pgfusepath{stroke,fill}%
}%
\begin{pgfscope}%
\pgfsys@transformshift{0.588387in}{2.633256in}%
\pgfsys@useobject{currentmarker}{}%
\end{pgfscope}%
\end{pgfscope}%
\begin{pgfscope}%
\pgfsetbuttcap%
\pgfsetroundjoin%
\definecolor{currentfill}{rgb}{0.000000,0.000000,0.000000}%
\pgfsetfillcolor{currentfill}%
\pgfsetlinewidth{0.602250pt}%
\definecolor{currentstroke}{rgb}{0.000000,0.000000,0.000000}%
\pgfsetstrokecolor{currentstroke}%
\pgfsetdash{}{0pt}%
\pgfsys@defobject{currentmarker}{\pgfqpoint{-0.027778in}{0.000000in}}{\pgfqpoint{-0.000000in}{0.000000in}}{%
\pgfpathmoveto{\pgfqpoint{-0.000000in}{0.000000in}}%
\pgfpathlineto{\pgfqpoint{-0.027778in}{0.000000in}}%
\pgfusepath{stroke,fill}%
}%
\begin{pgfscope}%
\pgfsys@transformshift{0.588387in}{2.669354in}%
\pgfsys@useobject{currentmarker}{}%
\end{pgfscope}%
\end{pgfscope}%
\begin{pgfscope}%
\pgfsetbuttcap%
\pgfsetroundjoin%
\definecolor{currentfill}{rgb}{0.000000,0.000000,0.000000}%
\pgfsetfillcolor{currentfill}%
\pgfsetlinewidth{0.602250pt}%
\definecolor{currentstroke}{rgb}{0.000000,0.000000,0.000000}%
\pgfsetstrokecolor{currentstroke}%
\pgfsetdash{}{0pt}%
\pgfsys@defobject{currentmarker}{\pgfqpoint{-0.027778in}{0.000000in}}{\pgfqpoint{-0.000000in}{0.000000in}}{%
\pgfpathmoveto{\pgfqpoint{-0.000000in}{0.000000in}}%
\pgfpathlineto{\pgfqpoint{-0.027778in}{0.000000in}}%
\pgfusepath{stroke,fill}%
}%
\begin{pgfscope}%
\pgfsys@transformshift{0.588387in}{2.698849in}%
\pgfsys@useobject{currentmarker}{}%
\end{pgfscope}%
\end{pgfscope}%
\begin{pgfscope}%
\pgfsetbuttcap%
\pgfsetroundjoin%
\definecolor{currentfill}{rgb}{0.000000,0.000000,0.000000}%
\pgfsetfillcolor{currentfill}%
\pgfsetlinewidth{0.602250pt}%
\definecolor{currentstroke}{rgb}{0.000000,0.000000,0.000000}%
\pgfsetstrokecolor{currentstroke}%
\pgfsetdash{}{0pt}%
\pgfsys@defobject{currentmarker}{\pgfqpoint{-0.027778in}{0.000000in}}{\pgfqpoint{-0.000000in}{0.000000in}}{%
\pgfpathmoveto{\pgfqpoint{-0.000000in}{0.000000in}}%
\pgfpathlineto{\pgfqpoint{-0.027778in}{0.000000in}}%
\pgfusepath{stroke,fill}%
}%
\begin{pgfscope}%
\pgfsys@transformshift{0.588387in}{2.723787in}%
\pgfsys@useobject{currentmarker}{}%
\end{pgfscope}%
\end{pgfscope}%
\begin{pgfscope}%
\definecolor{textcolor}{rgb}{0.000000,0.000000,0.000000}%
\pgfsetstrokecolor{textcolor}%
\pgfsetfillcolor{textcolor}%
\pgftext[x=0.234413in,y=1.631726in,,bottom,rotate=90.000000]{\color{textcolor}{\rmfamily\fontsize{10.000000}{12.000000}\selectfont\catcode`\^=\active\def^{\ifmmode\sp\else\^{}\fi}\catcode`\%=\active\def%{\%}Checks [call]}}%
\end{pgfscope}%
\begin{pgfscope}%
\pgfpathrectangle{\pgfqpoint{0.588387in}{0.521603in}}{\pgfqpoint{4.669024in}{2.220246in}}%
\pgfusepath{clip}%
\pgfsetrectcap%
\pgfsetroundjoin%
\pgfsetlinewidth{1.505625pt}%
\pgfsetstrokecolor{currentstroke1}%
\pgfsetdash{}{0pt}%
\pgfpathmoveto{\pgfqpoint{0.800616in}{0.622524in}}%
\pgfpathlineto{\pgfqpoint{0.946980in}{0.707172in}}%
\pgfpathlineto{\pgfqpoint{1.239709in}{0.968732in}}%
\pgfpathlineto{\pgfqpoint{1.532438in}{1.161590in}}%
\pgfpathlineto{\pgfqpoint{1.825166in}{1.358799in}}%
\pgfpathlineto{\pgfqpoint{2.264259in}{1.617708in}}%
\pgfpathlineto{\pgfqpoint{2.410624in}{1.702098in}}%
\pgfpathlineto{\pgfqpoint{2.703353in}{1.904705in}}%
\pgfpathlineto{\pgfqpoint{3.142446in}{2.153165in}}%
\pgfpathlineto{\pgfqpoint{3.435175in}{2.282590in}}%
\pgfpathlineto{\pgfqpoint{3.874268in}{2.445825in}}%
\pgfpathlineto{\pgfqpoint{4.166997in}{2.465487in}}%
\pgfusepath{stroke}%
\end{pgfscope}%
\begin{pgfscope}%
\pgfpathrectangle{\pgfqpoint{0.588387in}{0.521603in}}{\pgfqpoint{4.669024in}{2.220246in}}%
\pgfusepath{clip}%
\pgfsetrectcap%
\pgfsetroundjoin%
\pgfsetlinewidth{1.505625pt}%
\pgfsetstrokecolor{currentstroke2}%
\pgfsetdash{}{0pt}%
\pgfpathmoveto{\pgfqpoint{0.800616in}{0.622524in}}%
\pgfpathlineto{\pgfqpoint{0.946980in}{0.734657in}}%
\pgfpathlineto{\pgfqpoint{1.239709in}{0.968732in}}%
\pgfpathlineto{\pgfqpoint{1.532438in}{1.161590in}}%
\pgfpathlineto{\pgfqpoint{1.825166in}{1.371912in}}%
\pgfpathlineto{\pgfqpoint{2.264259in}{1.621454in}}%
\pgfpathlineto{\pgfqpoint{2.410624in}{1.701244in}}%
\pgfpathlineto{\pgfqpoint{2.703353in}{1.873410in}}%
\pgfpathlineto{\pgfqpoint{3.142446in}{2.069799in}}%
\pgfpathlineto{\pgfqpoint{3.435175in}{2.195290in}}%
\pgfpathlineto{\pgfqpoint{3.874268in}{2.519169in}}%
\pgfpathlineto{\pgfqpoint{4.166997in}{2.427642in}}%
\pgfpathlineto{\pgfqpoint{4.606090in}{2.408004in}}%
\pgfusepath{stroke}%
\end{pgfscope}%
\begin{pgfscope}%
\pgfpathrectangle{\pgfqpoint{0.588387in}{0.521603in}}{\pgfqpoint{4.669024in}{2.220246in}}%
\pgfusepath{clip}%
\pgfsetrectcap%
\pgfsetroundjoin%
\pgfsetlinewidth{1.505625pt}%
\pgfsetstrokecolor{currentstroke3}%
\pgfsetdash{}{0pt}%
\pgfpathmoveto{\pgfqpoint{0.800616in}{0.622524in}}%
\pgfpathlineto{\pgfqpoint{0.946980in}{0.734657in}}%
\pgfpathlineto{\pgfqpoint{1.239709in}{0.968732in}}%
\pgfpathlineto{\pgfqpoint{1.532438in}{1.161590in}}%
\pgfpathlineto{\pgfqpoint{1.825166in}{1.358799in}}%
\pgfpathlineto{\pgfqpoint{2.264259in}{1.621454in}}%
\pgfpathlineto{\pgfqpoint{2.410624in}{1.665454in}}%
\pgfpathlineto{\pgfqpoint{2.703353in}{1.829611in}}%
\pgfpathlineto{\pgfqpoint{3.142446in}{2.033655in}}%
\pgfpathlineto{\pgfqpoint{3.435175in}{2.178512in}}%
\pgfpathlineto{\pgfqpoint{3.874268in}{2.089417in}}%
\pgfpathlineto{\pgfqpoint{4.166997in}{2.320008in}}%
\pgfpathlineto{\pgfqpoint{4.606090in}{2.372161in}}%
\pgfpathlineto{\pgfqpoint{5.045183in}{2.393370in}}%
\pgfusepath{stroke}%
\end{pgfscope}%
\begin{pgfscope}%
\pgfpathrectangle{\pgfqpoint{0.588387in}{0.521603in}}{\pgfqpoint{4.669024in}{2.220246in}}%
\pgfusepath{clip}%
\pgfsetrectcap%
\pgfsetroundjoin%
\pgfsetlinewidth{1.505625pt}%
\pgfsetstrokecolor{currentstroke4}%
\pgfsetdash{}{0pt}%
\pgfpathmoveto{\pgfqpoint{0.800616in}{0.622524in}}%
\pgfpathlineto{\pgfqpoint{0.946980in}{0.734657in}}%
\pgfpathlineto{\pgfqpoint{1.239709in}{0.968732in}}%
\pgfpathlineto{\pgfqpoint{1.532438in}{1.161590in}}%
\pgfpathlineto{\pgfqpoint{1.825166in}{1.358799in}}%
\pgfpathlineto{\pgfqpoint{2.264259in}{1.613872in}}%
\pgfpathlineto{\pgfqpoint{2.410624in}{1.671331in}}%
\pgfpathlineto{\pgfqpoint{2.703353in}{1.817775in}}%
\pgfpathlineto{\pgfqpoint{3.142446in}{2.027111in}}%
\pgfpathlineto{\pgfqpoint{3.435175in}{2.166331in}}%
\pgfpathlineto{\pgfqpoint{3.874268in}{2.089361in}}%
\pgfpathlineto{\pgfqpoint{4.166997in}{2.325428in}}%
\pgfpathlineto{\pgfqpoint{4.606090in}{2.355525in}}%
\pgfpathlineto{\pgfqpoint{5.045183in}{2.370393in}}%
\pgfusepath{stroke}%
\end{pgfscope}%
\begin{pgfscope}%
\pgfpathrectangle{\pgfqpoint{0.588387in}{0.521603in}}{\pgfqpoint{4.669024in}{2.220246in}}%
\pgfusepath{clip}%
\pgfsetrectcap%
\pgfsetroundjoin%
\pgfsetlinewidth{1.505625pt}%
\pgfsetstrokecolor{currentstroke5}%
\pgfsetdash{}{0pt}%
\pgfpathmoveto{\pgfqpoint{0.800616in}{0.622524in}}%
\pgfpathlineto{\pgfqpoint{0.946980in}{0.734657in}}%
\pgfpathlineto{\pgfqpoint{1.239709in}{0.968732in}}%
\pgfpathlineto{\pgfqpoint{1.532438in}{1.161590in}}%
\pgfpathlineto{\pgfqpoint{1.825166in}{1.358799in}}%
\pgfpathlineto{\pgfqpoint{2.264259in}{1.621454in}}%
\pgfpathlineto{\pgfqpoint{2.410624in}{1.692948in}}%
\pgfpathlineto{\pgfqpoint{2.703353in}{1.885327in}}%
\pgfpathlineto{\pgfqpoint{3.142446in}{2.125519in}}%
\pgfpathlineto{\pgfqpoint{3.435175in}{2.224578in}}%
\pgfpathlineto{\pgfqpoint{3.874268in}{2.179042in}}%
\pgfpathlineto{\pgfqpoint{4.166997in}{2.361945in}}%
\pgfpathlineto{\pgfqpoint{4.606090in}{2.301688in}}%
\pgfpathlineto{\pgfqpoint{5.045183in}{2.374612in}}%
\pgfusepath{stroke}%
\end{pgfscope}%
\begin{pgfscope}%
\pgfpathrectangle{\pgfqpoint{0.588387in}{0.521603in}}{\pgfqpoint{4.669024in}{2.220246in}}%
\pgfusepath{clip}%
\pgfsetrectcap%
\pgfsetroundjoin%
\pgfsetlinewidth{1.505625pt}%
\pgfsetstrokecolor{currentstroke6}%
\pgfsetdash{}{0pt}%
\pgfpathmoveto{\pgfqpoint{0.800616in}{0.622524in}}%
\pgfpathlineto{\pgfqpoint{0.946980in}{0.734657in}}%
\pgfpathlineto{\pgfqpoint{1.239709in}{0.968732in}}%
\pgfpathlineto{\pgfqpoint{1.532438in}{1.161590in}}%
\pgfpathlineto{\pgfqpoint{1.825166in}{1.358799in}}%
\pgfpathlineto{\pgfqpoint{2.264259in}{1.613872in}}%
\pgfpathlineto{\pgfqpoint{2.410624in}{1.682673in}}%
\pgfpathlineto{\pgfqpoint{2.703353in}{1.834185in}}%
\pgfpathlineto{\pgfqpoint{3.142446in}{2.032439in}}%
\pgfpathlineto{\pgfqpoint{3.435175in}{2.186508in}}%
\pgfpathlineto{\pgfqpoint{3.874268in}{2.092320in}}%
\pgfpathlineto{\pgfqpoint{4.166997in}{2.327833in}}%
\pgfpathlineto{\pgfqpoint{4.606090in}{2.323026in}}%
\pgfpathlineto{\pgfqpoint{5.045183in}{2.344278in}}%
\pgfusepath{stroke}%
\end{pgfscope}%
\begin{pgfscope}%
\pgfpathrectangle{\pgfqpoint{0.588387in}{0.521603in}}{\pgfqpoint{4.669024in}{2.220246in}}%
\pgfusepath{clip}%
\pgfsetrectcap%
\pgfsetroundjoin%
\pgfsetlinewidth{1.505625pt}%
\pgfsetstrokecolor{currentstroke7}%
\pgfsetdash{}{0pt}%
\pgfpathmoveto{\pgfqpoint{0.800616in}{0.622524in}}%
\pgfpathlineto{\pgfqpoint{0.946980in}{0.707172in}}%
\pgfpathlineto{\pgfqpoint{1.239709in}{0.968732in}}%
\pgfpathlineto{\pgfqpoint{1.532438in}{1.161590in}}%
\pgfpathlineto{\pgfqpoint{1.825166in}{1.358799in}}%
\pgfpathlineto{\pgfqpoint{2.264259in}{1.630298in}}%
\pgfpathlineto{\pgfqpoint{2.410624in}{1.715596in}}%
\pgfpathlineto{\pgfqpoint{2.703353in}{1.907655in}}%
\pgfpathlineto{\pgfqpoint{3.142446in}{2.109114in}}%
\pgfpathlineto{\pgfqpoint{3.435175in}{2.165087in}}%
\pgfpathlineto{\pgfqpoint{3.874268in}{2.332761in}}%
\pgfpathlineto{\pgfqpoint{4.166997in}{2.417984in}}%
\pgfpathlineto{\pgfqpoint{4.606090in}{2.432487in}}%
\pgfusepath{stroke}%
\end{pgfscope}%
\begin{pgfscope}%
\pgfpathrectangle{\pgfqpoint{0.588387in}{0.521603in}}{\pgfqpoint{4.669024in}{2.220246in}}%
\pgfusepath{clip}%
\pgfsetrectcap%
\pgfsetroundjoin%
\pgfsetlinewidth{1.505625pt}%
\definecolor{currentstroke}{rgb}{0.498039,0.498039,0.498039}%
\pgfsetstrokecolor{currentstroke}%
\pgfsetdash{}{0pt}%
\pgfpathmoveto{\pgfqpoint{0.800616in}{0.622524in}}%
\pgfpathlineto{\pgfqpoint{0.946980in}{0.707172in}}%
\pgfpathlineto{\pgfqpoint{1.239709in}{0.968732in}}%
\pgfpathlineto{\pgfqpoint{1.532438in}{1.161590in}}%
\pgfpathlineto{\pgfqpoint{1.825166in}{1.327846in}}%
\pgfpathlineto{\pgfqpoint{2.264259in}{1.579833in}}%
\pgfpathlineto{\pgfqpoint{2.410624in}{1.701620in}}%
\pgfpathlineto{\pgfqpoint{2.703353in}{1.850303in}}%
\pgfpathlineto{\pgfqpoint{3.142446in}{2.082292in}}%
\pgfpathlineto{\pgfqpoint{3.435175in}{2.196495in}}%
\pgfpathlineto{\pgfqpoint{3.874268in}{2.306788in}}%
\pgfpathlineto{\pgfqpoint{4.166997in}{2.387798in}}%
\pgfpathlineto{\pgfqpoint{4.606090in}{2.454052in}}%
\pgfpathlineto{\pgfqpoint{5.045183in}{2.472561in}}%
\pgfusepath{stroke}%
\end{pgfscope}%
\begin{pgfscope}%
\pgfpathrectangle{\pgfqpoint{0.588387in}{0.521603in}}{\pgfqpoint{4.669024in}{2.220246in}}%
\pgfusepath{clip}%
\pgfsetrectcap%
\pgfsetroundjoin%
\pgfsetlinewidth{1.505625pt}%
\definecolor{currentstroke}{rgb}{0.737255,0.741176,0.133333}%
\pgfsetstrokecolor{currentstroke}%
\pgfsetdash{}{0pt}%
\pgfpathmoveto{\pgfqpoint{1.532438in}{1.182875in}}%
\pgfpathlineto{\pgfqpoint{1.825166in}{1.407380in}}%
\pgfpathlineto{\pgfqpoint{2.264259in}{1.743850in}}%
\pgfpathlineto{\pgfqpoint{2.410624in}{1.826493in}}%
\pgfpathlineto{\pgfqpoint{2.703353in}{2.080260in}}%
\pgfpathlineto{\pgfqpoint{3.142446in}{2.416662in}}%
\pgfpathlineto{\pgfqpoint{3.435175in}{2.640929in}}%
\pgfusepath{stroke}%
\end{pgfscope}%
\begin{pgfscope}%
\pgfsetrectcap%
\pgfsetmiterjoin%
\pgfsetlinewidth{0.803000pt}%
\definecolor{currentstroke}{rgb}{0.000000,0.000000,0.000000}%
\pgfsetstrokecolor{currentstroke}%
\pgfsetdash{}{0pt}%
\pgfpathmoveto{\pgfqpoint{0.588387in}{0.521603in}}%
\pgfpathlineto{\pgfqpoint{0.588387in}{2.741849in}}%
\pgfusepath{stroke}%
\end{pgfscope}%
\begin{pgfscope}%
\pgfsetrectcap%
\pgfsetmiterjoin%
\pgfsetlinewidth{0.803000pt}%
\definecolor{currentstroke}{rgb}{0.000000,0.000000,0.000000}%
\pgfsetstrokecolor{currentstroke}%
\pgfsetdash{}{0pt}%
\pgfpathmoveto{\pgfqpoint{5.257411in}{0.521603in}}%
\pgfpathlineto{\pgfqpoint{5.257411in}{2.741849in}}%
\pgfusepath{stroke}%
\end{pgfscope}%
\begin{pgfscope}%
\pgfsetrectcap%
\pgfsetmiterjoin%
\pgfsetlinewidth{0.803000pt}%
\definecolor{currentstroke}{rgb}{0.000000,0.000000,0.000000}%
\pgfsetstrokecolor{currentstroke}%
\pgfsetdash{}{0pt}%
\pgfpathmoveto{\pgfqpoint{0.588387in}{0.521603in}}%
\pgfpathlineto{\pgfqpoint{5.257411in}{0.521603in}}%
\pgfusepath{stroke}%
\end{pgfscope}%
\begin{pgfscope}%
\pgfsetrectcap%
\pgfsetmiterjoin%
\pgfsetlinewidth{0.803000pt}%
\definecolor{currentstroke}{rgb}{0.000000,0.000000,0.000000}%
\pgfsetstrokecolor{currentstroke}%
\pgfsetdash{}{0pt}%
\pgfpathmoveto{\pgfqpoint{0.588387in}{2.741849in}}%
\pgfpathlineto{\pgfqpoint{5.257411in}{2.741849in}}%
\pgfusepath{stroke}%
\end{pgfscope}%
\begin{pgfscope}%
\pgfsetbuttcap%
\pgfsetmiterjoin%
\definecolor{currentfill}{rgb}{1.000000,1.000000,1.000000}%
\pgfsetfillcolor{currentfill}%
\pgfsetfillopacity{0.800000}%
\pgfsetlinewidth{1.003750pt}%
\definecolor{currentstroke}{rgb}{0.800000,0.800000,0.800000}%
\pgfsetstrokecolor{currentstroke}%
\pgfsetstrokeopacity{0.800000}%
\pgfsetdash{}{0pt}%
\pgfpathmoveto{\pgfqpoint{5.344911in}{0.969732in}}%
\pgfpathlineto{\pgfqpoint{8.259376in}{0.969732in}}%
\pgfpathquadraticcurveto{\pgfqpoint{8.284376in}{0.969732in}}{\pgfqpoint{8.284376in}{0.994732in}}%
\pgfpathlineto{\pgfqpoint{8.284376in}{2.654349in}}%
\pgfpathquadraticcurveto{\pgfqpoint{8.284376in}{2.679349in}}{\pgfqpoint{8.259376in}{2.679349in}}%
\pgfpathlineto{\pgfqpoint{5.344911in}{2.679349in}}%
\pgfpathquadraticcurveto{\pgfqpoint{5.319911in}{2.679349in}}{\pgfqpoint{5.319911in}{2.654349in}}%
\pgfpathlineto{\pgfqpoint{5.319911in}{0.994732in}}%
\pgfpathquadraticcurveto{\pgfqpoint{5.319911in}{0.969732in}}{\pgfqpoint{5.344911in}{0.969732in}}%
\pgfpathlineto{\pgfqpoint{5.344911in}{0.969732in}}%
\pgfpathclose%
\pgfusepath{stroke,fill}%
\end{pgfscope}%
\begin{pgfscope}%
\pgfsetrectcap%
\pgfsetroundjoin%
\pgfsetlinewidth{1.505625pt}%
\definecolor{currentstroke}{rgb}{0.737255,0.741176,0.133333}%
\pgfsetstrokecolor{currentstroke}%
\pgfsetdash{}{0pt}%
\pgfpathmoveto{\pgfqpoint{5.369911in}{2.578129in}}%
\pgfpathlineto{\pgfqpoint{5.494911in}{2.578129in}}%
\pgfpathlineto{\pgfqpoint{5.619911in}{2.578129in}}%
\pgfusepath{stroke}%
\end{pgfscope}%
\begin{pgfscope}%
\definecolor{textcolor}{rgb}{0.000000,0.000000,0.000000}%
\pgfsetstrokecolor{textcolor}%
\pgfsetfillcolor{textcolor}%
\pgftext[x=5.719911in,y=2.534379in,left,base]{\color{textcolor}{\rmfamily\fontsize{9.000000}{10.800000}\selectfont\catcode`\^=\active\def^{\ifmmode\sp\else\^{}\fi}\catcode`\%=\active\def%{\%}\NaiveCycles{}}}%
\end{pgfscope}%
\begin{pgfscope}%
\pgfsetrectcap%
\pgfsetroundjoin%
\pgfsetlinewidth{1.505625pt}%
\pgfsetstrokecolor{currentstroke1}%
\pgfsetdash{}{0pt}%
\pgfpathmoveto{\pgfqpoint{5.369911in}{2.394657in}}%
\pgfpathlineto{\pgfqpoint{5.494911in}{2.394657in}}%
\pgfpathlineto{\pgfqpoint{5.619911in}{2.394657in}}%
\pgfusepath{stroke}%
\end{pgfscope}%
\begin{pgfscope}%
\definecolor{textcolor}{rgb}{0.000000,0.000000,0.000000}%
\pgfsetstrokecolor{textcolor}%
\pgfsetfillcolor{textcolor}%
\pgftext[x=5.719911in,y=2.350907in,left,base]{\color{textcolor}{\rmfamily\fontsize{9.000000}{10.800000}\selectfont\catcode`\^=\active\def^{\ifmmode\sp\else\^{}\fi}\catcode`\%=\active\def%{\%}\CyclesMatchChunks{} \& \MergeLinear{}}}%
\end{pgfscope}%
\begin{pgfscope}%
\pgfsetrectcap%
\pgfsetroundjoin%
\pgfsetlinewidth{1.505625pt}%
\pgfsetstrokecolor{currentstroke2}%
\pgfsetdash{}{0pt}%
\pgfpathmoveto{\pgfqpoint{5.369911in}{2.207707in}}%
\pgfpathlineto{\pgfqpoint{5.494911in}{2.207707in}}%
\pgfpathlineto{\pgfqpoint{5.619911in}{2.207707in}}%
\pgfusepath{stroke}%
\end{pgfscope}%
\begin{pgfscope}%
\definecolor{textcolor}{rgb}{0.000000,0.000000,0.000000}%
\pgfsetstrokecolor{textcolor}%
\pgfsetfillcolor{textcolor}%
\pgftext[x=5.719911in,y=2.163957in,left,base]{\color{textcolor}{\rmfamily\fontsize{9.000000}{10.800000}\selectfont\catcode`\^=\active\def^{\ifmmode\sp\else\^{}\fi}\catcode`\%=\active\def%{\%}\CyclesMatchChunks{} \& \SharedVertices{}}}%
\end{pgfscope}%
\begin{pgfscope}%
\pgfsetrectcap%
\pgfsetroundjoin%
\pgfsetlinewidth{1.505625pt}%
\pgfsetstrokecolor{currentstroke3}%
\pgfsetdash{}{0pt}%
\pgfpathmoveto{\pgfqpoint{5.369911in}{2.020756in}}%
\pgfpathlineto{\pgfqpoint{5.494911in}{2.020756in}}%
\pgfpathlineto{\pgfqpoint{5.619911in}{2.020756in}}%
\pgfusepath{stroke}%
\end{pgfscope}%
\begin{pgfscope}%
\definecolor{textcolor}{rgb}{0.000000,0.000000,0.000000}%
\pgfsetstrokecolor{textcolor}%
\pgfsetfillcolor{textcolor}%
\pgftext[x=5.719911in,y=1.977006in,left,base]{\color{textcolor}{\rmfamily\fontsize{9.000000}{10.800000}\selectfont\catcode`\^=\active\def^{\ifmmode\sp\else\^{}\fi}\catcode`\%=\active\def%{\%}\Neighbors{} \& \MergeLinear{}}}%
\end{pgfscope}%
\begin{pgfscope}%
\pgfsetrectcap%
\pgfsetroundjoin%
\pgfsetlinewidth{1.505625pt}%
\pgfsetstrokecolor{currentstroke4}%
\pgfsetdash{}{0pt}%
\pgfpathmoveto{\pgfqpoint{5.369911in}{1.837285in}}%
\pgfpathlineto{\pgfqpoint{5.494911in}{1.837285in}}%
\pgfpathlineto{\pgfqpoint{5.619911in}{1.837285in}}%
\pgfusepath{stroke}%
\end{pgfscope}%
\begin{pgfscope}%
\definecolor{textcolor}{rgb}{0.000000,0.000000,0.000000}%
\pgfsetstrokecolor{textcolor}%
\pgfsetfillcolor{textcolor}%
\pgftext[x=5.719911in,y=1.793535in,left,base]{\color{textcolor}{\rmfamily\fontsize{9.000000}{10.800000}\selectfont\catcode`\^=\active\def^{\ifmmode\sp\else\^{}\fi}\catcode`\%=\active\def%{\%}\Neighbors{} \& \SharedVertices{}}}%
\end{pgfscope}%
\begin{pgfscope}%
\pgfsetrectcap%
\pgfsetroundjoin%
\pgfsetlinewidth{1.505625pt}%
\pgfsetstrokecolor{currentstroke5}%
\pgfsetdash{}{0pt}%
\pgfpathmoveto{\pgfqpoint{5.369911in}{1.650334in}}%
\pgfpathlineto{\pgfqpoint{5.494911in}{1.650334in}}%
\pgfpathlineto{\pgfqpoint{5.619911in}{1.650334in}}%
\pgfusepath{stroke}%
\end{pgfscope}%
\begin{pgfscope}%
\definecolor{textcolor}{rgb}{0.000000,0.000000,0.000000}%
\pgfsetstrokecolor{textcolor}%
\pgfsetfillcolor{textcolor}%
\pgftext[x=5.719911in,y=1.606584in,left,base]{\color{textcolor}{\rmfamily\fontsize{9.000000}{10.800000}\selectfont\catcode`\^=\active\def^{\ifmmode\sp\else\^{}\fi}\catcode`\%=\active\def%{\%}\NeighborsDegree{} \& \MergeLinear{}}}%
\end{pgfscope}%
\begin{pgfscope}%
\pgfsetrectcap%
\pgfsetroundjoin%
\pgfsetlinewidth{1.505625pt}%
\pgfsetstrokecolor{currentstroke6}%
\pgfsetdash{}{0pt}%
\pgfpathmoveto{\pgfqpoint{5.369911in}{1.463384in}}%
\pgfpathlineto{\pgfqpoint{5.494911in}{1.463384in}}%
\pgfpathlineto{\pgfqpoint{5.619911in}{1.463384in}}%
\pgfusepath{stroke}%
\end{pgfscope}%
\begin{pgfscope}%
\definecolor{textcolor}{rgb}{0.000000,0.000000,0.000000}%
\pgfsetstrokecolor{textcolor}%
\pgfsetfillcolor{textcolor}%
\pgftext[x=5.719911in,y=1.419634in,left,base]{\color{textcolor}{\rmfamily\fontsize{9.000000}{10.800000}\selectfont\catcode`\^=\active\def^{\ifmmode\sp\else\^{}\fi}\catcode`\%=\active\def%{\%}\NeighborsDegree{} \& \SharedVertices{}}}%
\end{pgfscope}%
\begin{pgfscope}%
\pgfsetrectcap%
\pgfsetroundjoin%
\pgfsetlinewidth{1.505625pt}%
\pgfsetstrokecolor{currentstroke7}%
\pgfsetdash{}{0pt}%
\pgfpathmoveto{\pgfqpoint{5.369911in}{1.276433in}}%
\pgfpathlineto{\pgfqpoint{5.494911in}{1.276433in}}%
\pgfpathlineto{\pgfqpoint{5.619911in}{1.276433in}}%
\pgfusepath{stroke}%
\end{pgfscope}%
\begin{pgfscope}%
\definecolor{textcolor}{rgb}{0.000000,0.000000,0.000000}%
\pgfsetstrokecolor{textcolor}%
\pgfsetfillcolor{textcolor}%
\pgftext[x=5.719911in,y=1.232683in,left,base]{\color{textcolor}{\rmfamily\fontsize{9.000000}{10.800000}\selectfont\catcode`\^=\active\def^{\ifmmode\sp\else\^{}\fi}\catcode`\%=\active\def%{\%}\None{} \& \MergeLinear{}}}%
\end{pgfscope}%
\begin{pgfscope}%
\pgfsetrectcap%
\pgfsetroundjoin%
\pgfsetlinewidth{1.505625pt}%
\definecolor{currentstroke}{rgb}{0.498039,0.498039,0.498039}%
\pgfsetstrokecolor{currentstroke}%
\pgfsetdash{}{0pt}%
\pgfpathmoveto{\pgfqpoint{5.369911in}{1.092962in}}%
\pgfpathlineto{\pgfqpoint{5.494911in}{1.092962in}}%
\pgfpathlineto{\pgfqpoint{5.619911in}{1.092962in}}%
\pgfusepath{stroke}%
\end{pgfscope}%
\begin{pgfscope}%
\definecolor{textcolor}{rgb}{0.000000,0.000000,0.000000}%
\pgfsetstrokecolor{textcolor}%
\pgfsetfillcolor{textcolor}%
\pgftext[x=5.719911in,y=1.049212in,left,base]{\color{textcolor}{\rmfamily\fontsize{9.000000}{10.800000}\selectfont\catcode`\^=\active\def^{\ifmmode\sp\else\^{}\fi}\catcode`\%=\active\def%{\%}\None{} \& \SharedVertices{}}}%
\end{pgfscope}%
\end{pgfpicture}%
\makeatother%
\endgroup%
}
	\caption[Checks performed for graphs with no 3 nor 4 cycles (all).]{
		The number of checks performed to find all NAC-colorings for graphs with no 3 nor 4 cycles.}%
	\label{fig:graph_count_no_3_nor_4_cycles_all_checks}
\end{figure}


We also randomly generated a dataset of globally rigid graphs
up to 57 vertices.
We used a formula from yet unpublished work of John Haslegrave
that for a number of vertices gives a number of edges,
such that graphs have no or just few NAC-colorings.
For such random graphs, we checked if they are globally rigid using PyRigi~\cite{pyrigi}.
%
The idea of monochromatic classes is so effective
that even large graphs collapse into just a few monochromatic classes.
Most of the graphs in this dataset have a NAC-coloring,
and the other graphs often have only a single monochromatic class.
The statements from other graph classes in this section hold
--- \NaiveCycles{} is faster for finding some NAC-coloring
as shown in \Cref{fig:graph_globally_rigid_first_runtime}
and significantly slower when we list all NAC-colorings
as shown in \Cref{fig:graph_globally_rigid_all_runtime}.
It can be also seen in \Cref{fig:graph_globally_rigid_all_checks}
that the number of checks performed
by \NaiveCycles{} is not consisted while it is for \Subgraphs{}.

\begin{figure}[p]
	\centering
	\scalebox{0.5}{%% Creator: Matplotlib, PGF backend
%%
%% To include the figure in your LaTeX document, write
%%   \input{<filename>.pgf}
%%
%% Make sure the required packages are loaded in your preamble
%%   \usepackage{pgf}
%%
%% Also ensure that all the required font packages are loaded; for instance,
%% the lmodern package is sometimes necessary when using math font.
%%   \usepackage{lmodern}
%%
%% Figures using additional raster images can only be included by \input if
%% they are in the same directory as the main LaTeX file. For loading figures
%% from other directories you can use the `import` package
%%   \usepackage{import}
%%
%% and then include the figures with
%%   \import{<path to file>}{<filename>.pgf}
%%
%% Matplotlib used the following preamble
%%   \def\mathdefault#1{#1}
%%   \everymath=\expandafter{\the\everymath\displaystyle}
%%   \IfFileExists{scrextend.sty}{
%%     \usepackage[fontsize=10.000000pt]{scrextend}
%%   }{
%%     \renewcommand{\normalsize}{\fontsize{10.000000}{12.000000}\selectfont}
%%     \normalsize
%%   }
%%   
%%   \ifdefined\pdftexversion\else  % non-pdftex case.
%%     \usepackage{fontspec}
%%     \setmainfont{DejaVuSans.ttf}[Path=\detokenize{/home/petr/Projects/PyRigi/.venv/lib/python3.12/site-packages/matplotlib/mpl-data/fonts/ttf/}]
%%     \setsansfont{DejaVuSans.ttf}[Path=\detokenize{/home/petr/Projects/PyRigi/.venv/lib/python3.12/site-packages/matplotlib/mpl-data/fonts/ttf/}]
%%     \setmonofont{DejaVuSansMono.ttf}[Path=\detokenize{/home/petr/Projects/PyRigi/.venv/lib/python3.12/site-packages/matplotlib/mpl-data/fonts/ttf/}]
%%   \fi
%%   \makeatletter\@ifpackageloaded{under\Score{}}{}{\usepackage[strings]{under\Score{}}}\makeatother
%%
\begingroup%
\makeatletter%
\begin{pgfpicture}%
\pgfpathrectangle{\pgfpointorigin}{\pgfqpoint{8.384376in}{2.841849in}}%
\pgfusepath{use as bounding box, clip}%
\begin{pgfscope}%
\pgfsetbuttcap%
\pgfsetmiterjoin%
\definecolor{currentfill}{rgb}{1.000000,1.000000,1.000000}%
\pgfsetfillcolor{currentfill}%
\pgfsetlinewidth{0.000000pt}%
\definecolor{currentstroke}{rgb}{1.000000,1.000000,1.000000}%
\pgfsetstrokecolor{currentstroke}%
\pgfsetdash{}{0pt}%
\pgfpathmoveto{\pgfqpoint{0.000000in}{0.000000in}}%
\pgfpathlineto{\pgfqpoint{8.384376in}{0.000000in}}%
\pgfpathlineto{\pgfqpoint{8.384376in}{2.841849in}}%
\pgfpathlineto{\pgfqpoint{0.000000in}{2.841849in}}%
\pgfpathlineto{\pgfqpoint{0.000000in}{0.000000in}}%
\pgfpathclose%
\pgfusepath{fill}%
\end{pgfscope}%
\begin{pgfscope}%
\pgfsetbuttcap%
\pgfsetmiterjoin%
\definecolor{currentfill}{rgb}{1.000000,1.000000,1.000000}%
\pgfsetfillcolor{currentfill}%
\pgfsetlinewidth{0.000000pt}%
\definecolor{currentstroke}{rgb}{0.000000,0.000000,0.000000}%
\pgfsetstrokecolor{currentstroke}%
\pgfsetstrokeopacity{0.000000}%
\pgfsetdash{}{0pt}%
\pgfpathmoveto{\pgfqpoint{0.588387in}{0.521603in}}%
\pgfpathlineto{\pgfqpoint{4.248423in}{0.521603in}}%
\pgfpathlineto{\pgfqpoint{4.248423in}{2.741849in}}%
\pgfpathlineto{\pgfqpoint{0.588387in}{2.741849in}}%
\pgfpathlineto{\pgfqpoint{0.588387in}{0.521603in}}%
\pgfpathclose%
\pgfusepath{fill}%
\end{pgfscope}%
\begin{pgfscope}%
\pgfsetbuttcap%
\pgfsetroundjoin%
\definecolor{currentfill}{rgb}{0.000000,0.000000,0.000000}%
\pgfsetfillcolor{currentfill}%
\pgfsetlinewidth{0.803000pt}%
\definecolor{currentstroke}{rgb}{0.000000,0.000000,0.000000}%
\pgfsetstrokecolor{currentstroke}%
\pgfsetdash{}{0pt}%
\pgfsys@defobject{currentmarker}{\pgfqpoint{0.000000in}{-0.048611in}}{\pgfqpoint{0.000000in}{0.000000in}}{%
\pgfpathmoveto{\pgfqpoint{0.000000in}{0.000000in}}%
\pgfpathlineto{\pgfqpoint{0.000000in}{-0.048611in}}%
\pgfusepath{stroke,fill}%
}%
\begin{pgfscope}%
\pgfsys@transformshift{0.896340in}{0.521603in}%
\pgfsys@useobject{currentmarker}{}%
\end{pgfscope}%
\end{pgfscope}%
\begin{pgfscope}%
\definecolor{textcolor}{rgb}{0.000000,0.000000,0.000000}%
\pgfsetstrokecolor{textcolor}%
\pgfsetfillcolor{textcolor}%
\pgftext[x=0.896340in,y=0.424381in,,top]{\color{textcolor}{\rmfamily\fontsize{10.000000}{12.000000}\selectfont\catcode`\^=\active\def^{\ifmmode\sp\else\^{}\fi}\catcode`\%=\active\def%{\%}$\mathdefault{12}$}}%
\end{pgfscope}%
\begin{pgfscope}%
\pgfsetbuttcap%
\pgfsetroundjoin%
\definecolor{currentfill}{rgb}{0.000000,0.000000,0.000000}%
\pgfsetfillcolor{currentfill}%
\pgfsetlinewidth{0.803000pt}%
\definecolor{currentstroke}{rgb}{0.000000,0.000000,0.000000}%
\pgfsetstrokecolor{currentstroke}%
\pgfsetdash{}{0pt}%
\pgfsys@defobject{currentmarker}{\pgfqpoint{0.000000in}{-0.048611in}}{\pgfqpoint{0.000000in}{0.000000in}}{%
\pgfpathmoveto{\pgfqpoint{0.000000in}{0.000000in}}%
\pgfpathlineto{\pgfqpoint{0.000000in}{-0.048611in}}%
\pgfusepath{stroke,fill}%
}%
\begin{pgfscope}%
\pgfsys@transformshift{1.321102in}{0.521603in}%
\pgfsys@useobject{currentmarker}{}%
\end{pgfscope}%
\end{pgfscope}%
\begin{pgfscope}%
\definecolor{textcolor}{rgb}{0.000000,0.000000,0.000000}%
\pgfsetstrokecolor{textcolor}%
\pgfsetfillcolor{textcolor}%
\pgftext[x=1.321102in,y=0.424381in,,top]{\color{textcolor}{\rmfamily\fontsize{10.000000}{12.000000}\selectfont\catcode`\^=\active\def^{\ifmmode\sp\else\^{}\fi}\catcode`\%=\active\def%{\%}$\mathdefault{18}$}}%
\end{pgfscope}%
\begin{pgfscope}%
\pgfsetbuttcap%
\pgfsetroundjoin%
\definecolor{currentfill}{rgb}{0.000000,0.000000,0.000000}%
\pgfsetfillcolor{currentfill}%
\pgfsetlinewidth{0.803000pt}%
\definecolor{currentstroke}{rgb}{0.000000,0.000000,0.000000}%
\pgfsetstrokecolor{currentstroke}%
\pgfsetdash{}{0pt}%
\pgfsys@defobject{currentmarker}{\pgfqpoint{0.000000in}{-0.048611in}}{\pgfqpoint{0.000000in}{0.000000in}}{%
\pgfpathmoveto{\pgfqpoint{0.000000in}{0.000000in}}%
\pgfpathlineto{\pgfqpoint{0.000000in}{-0.048611in}}%
\pgfusepath{stroke,fill}%
}%
\begin{pgfscope}%
\pgfsys@transformshift{1.745865in}{0.521603in}%
\pgfsys@useobject{currentmarker}{}%
\end{pgfscope}%
\end{pgfscope}%
\begin{pgfscope}%
\definecolor{textcolor}{rgb}{0.000000,0.000000,0.000000}%
\pgfsetstrokecolor{textcolor}%
\pgfsetfillcolor{textcolor}%
\pgftext[x=1.745865in,y=0.424381in,,top]{\color{textcolor}{\rmfamily\fontsize{10.000000}{12.000000}\selectfont\catcode`\^=\active\def^{\ifmmode\sp\else\^{}\fi}\catcode`\%=\active\def%{\%}$\mathdefault{24}$}}%
\end{pgfscope}%
\begin{pgfscope}%
\pgfsetbuttcap%
\pgfsetroundjoin%
\definecolor{currentfill}{rgb}{0.000000,0.000000,0.000000}%
\pgfsetfillcolor{currentfill}%
\pgfsetlinewidth{0.803000pt}%
\definecolor{currentstroke}{rgb}{0.000000,0.000000,0.000000}%
\pgfsetstrokecolor{currentstroke}%
\pgfsetdash{}{0pt}%
\pgfsys@defobject{currentmarker}{\pgfqpoint{0.000000in}{-0.048611in}}{\pgfqpoint{0.000000in}{0.000000in}}{%
\pgfpathmoveto{\pgfqpoint{0.000000in}{0.000000in}}%
\pgfpathlineto{\pgfqpoint{0.000000in}{-0.048611in}}%
\pgfusepath{stroke,fill}%
}%
\begin{pgfscope}%
\pgfsys@transformshift{2.170627in}{0.521603in}%
\pgfsys@useobject{currentmarker}{}%
\end{pgfscope}%
\end{pgfscope}%
\begin{pgfscope}%
\definecolor{textcolor}{rgb}{0.000000,0.000000,0.000000}%
\pgfsetstrokecolor{textcolor}%
\pgfsetfillcolor{textcolor}%
\pgftext[x=2.170627in,y=0.424381in,,top]{\color{textcolor}{\rmfamily\fontsize{10.000000}{12.000000}\selectfont\catcode`\^=\active\def^{\ifmmode\sp\else\^{}\fi}\catcode`\%=\active\def%{\%}$\mathdefault{30}$}}%
\end{pgfscope}%
\begin{pgfscope}%
\pgfsetbuttcap%
\pgfsetroundjoin%
\definecolor{currentfill}{rgb}{0.000000,0.000000,0.000000}%
\pgfsetfillcolor{currentfill}%
\pgfsetlinewidth{0.803000pt}%
\definecolor{currentstroke}{rgb}{0.000000,0.000000,0.000000}%
\pgfsetstrokecolor{currentstroke}%
\pgfsetdash{}{0pt}%
\pgfsys@defobject{currentmarker}{\pgfqpoint{0.000000in}{-0.048611in}}{\pgfqpoint{0.000000in}{0.000000in}}{%
\pgfpathmoveto{\pgfqpoint{0.000000in}{0.000000in}}%
\pgfpathlineto{\pgfqpoint{0.000000in}{-0.048611in}}%
\pgfusepath{stroke,fill}%
}%
\begin{pgfscope}%
\pgfsys@transformshift{2.595389in}{0.521603in}%
\pgfsys@useobject{currentmarker}{}%
\end{pgfscope}%
\end{pgfscope}%
\begin{pgfscope}%
\definecolor{textcolor}{rgb}{0.000000,0.000000,0.000000}%
\pgfsetstrokecolor{textcolor}%
\pgfsetfillcolor{textcolor}%
\pgftext[x=2.595389in,y=0.424381in,,top]{\color{textcolor}{\rmfamily\fontsize{10.000000}{12.000000}\selectfont\catcode`\^=\active\def^{\ifmmode\sp\else\^{}\fi}\catcode`\%=\active\def%{\%}$\mathdefault{36}$}}%
\end{pgfscope}%
\begin{pgfscope}%
\pgfsetbuttcap%
\pgfsetroundjoin%
\definecolor{currentfill}{rgb}{0.000000,0.000000,0.000000}%
\pgfsetfillcolor{currentfill}%
\pgfsetlinewidth{0.803000pt}%
\definecolor{currentstroke}{rgb}{0.000000,0.000000,0.000000}%
\pgfsetstrokecolor{currentstroke}%
\pgfsetdash{}{0pt}%
\pgfsys@defobject{currentmarker}{\pgfqpoint{0.000000in}{-0.048611in}}{\pgfqpoint{0.000000in}{0.000000in}}{%
\pgfpathmoveto{\pgfqpoint{0.000000in}{0.000000in}}%
\pgfpathlineto{\pgfqpoint{0.000000in}{-0.048611in}}%
\pgfusepath{stroke,fill}%
}%
\begin{pgfscope}%
\pgfsys@transformshift{3.020152in}{0.521603in}%
\pgfsys@useobject{currentmarker}{}%
\end{pgfscope}%
\end{pgfscope}%
\begin{pgfscope}%
\definecolor{textcolor}{rgb}{0.000000,0.000000,0.000000}%
\pgfsetstrokecolor{textcolor}%
\pgfsetfillcolor{textcolor}%
\pgftext[x=3.020152in,y=0.424381in,,top]{\color{textcolor}{\rmfamily\fontsize{10.000000}{12.000000}\selectfont\catcode`\^=\active\def^{\ifmmode\sp\else\^{}\fi}\catcode`\%=\active\def%{\%}$\mathdefault{42}$}}%
\end{pgfscope}%
\begin{pgfscope}%
\pgfsetbuttcap%
\pgfsetroundjoin%
\definecolor{currentfill}{rgb}{0.000000,0.000000,0.000000}%
\pgfsetfillcolor{currentfill}%
\pgfsetlinewidth{0.803000pt}%
\definecolor{currentstroke}{rgb}{0.000000,0.000000,0.000000}%
\pgfsetstrokecolor{currentstroke}%
\pgfsetdash{}{0pt}%
\pgfsys@defobject{currentmarker}{\pgfqpoint{0.000000in}{-0.048611in}}{\pgfqpoint{0.000000in}{0.000000in}}{%
\pgfpathmoveto{\pgfqpoint{0.000000in}{0.000000in}}%
\pgfpathlineto{\pgfqpoint{0.000000in}{-0.048611in}}%
\pgfusepath{stroke,fill}%
}%
\begin{pgfscope}%
\pgfsys@transformshift{3.444914in}{0.521603in}%
\pgfsys@useobject{currentmarker}{}%
\end{pgfscope}%
\end{pgfscope}%
\begin{pgfscope}%
\definecolor{textcolor}{rgb}{0.000000,0.000000,0.000000}%
\pgfsetstrokecolor{textcolor}%
\pgfsetfillcolor{textcolor}%
\pgftext[x=3.444914in,y=0.424381in,,top]{\color{textcolor}{\rmfamily\fontsize{10.000000}{12.000000}\selectfont\catcode`\^=\active\def^{\ifmmode\sp\else\^{}\fi}\catcode`\%=\active\def%{\%}$\mathdefault{48}$}}%
\end{pgfscope}%
\begin{pgfscope}%
\pgfsetbuttcap%
\pgfsetroundjoin%
\definecolor{currentfill}{rgb}{0.000000,0.000000,0.000000}%
\pgfsetfillcolor{currentfill}%
\pgfsetlinewidth{0.803000pt}%
\definecolor{currentstroke}{rgb}{0.000000,0.000000,0.000000}%
\pgfsetstrokecolor{currentstroke}%
\pgfsetdash{}{0pt}%
\pgfsys@defobject{currentmarker}{\pgfqpoint{0.000000in}{-0.048611in}}{\pgfqpoint{0.000000in}{0.000000in}}{%
\pgfpathmoveto{\pgfqpoint{0.000000in}{0.000000in}}%
\pgfpathlineto{\pgfqpoint{0.000000in}{-0.048611in}}%
\pgfusepath{stroke,fill}%
}%
\begin{pgfscope}%
\pgfsys@transformshift{3.869676in}{0.521603in}%
\pgfsys@useobject{currentmarker}{}%
\end{pgfscope}%
\end{pgfscope}%
\begin{pgfscope}%
\definecolor{textcolor}{rgb}{0.000000,0.000000,0.000000}%
\pgfsetstrokecolor{textcolor}%
\pgfsetfillcolor{textcolor}%
\pgftext[x=3.869676in,y=0.424381in,,top]{\color{textcolor}{\rmfamily\fontsize{10.000000}{12.000000}\selectfont\catcode`\^=\active\def^{\ifmmode\sp\else\^{}\fi}\catcode`\%=\active\def%{\%}$\mathdefault{54}$}}%
\end{pgfscope}%
\begin{pgfscope}%
\definecolor{textcolor}{rgb}{0.000000,0.000000,0.000000}%
\pgfsetstrokecolor{textcolor}%
\pgfsetfillcolor{textcolor}%
\pgftext[x=2.418405in,y=0.234413in,,top]{\color{textcolor}{\rmfamily\fontsize{10.000000}{12.000000}\selectfont\catcode`\^=\active\def^{\ifmmode\sp\else\^{}\fi}\catcode`\%=\active\def%{\%}Vertices}}%
\end{pgfscope}%
\begin{pgfscope}%
\pgfsetbuttcap%
\pgfsetroundjoin%
\definecolor{currentfill}{rgb}{0.000000,0.000000,0.000000}%
\pgfsetfillcolor{currentfill}%
\pgfsetlinewidth{0.803000pt}%
\definecolor{currentstroke}{rgb}{0.000000,0.000000,0.000000}%
\pgfsetstrokecolor{currentstroke}%
\pgfsetdash{}{0pt}%
\pgfsys@defobject{currentmarker}{\pgfqpoint{-0.048611in}{0.000000in}}{\pgfqpoint{-0.000000in}{0.000000in}}{%
\pgfpathmoveto{\pgfqpoint{-0.000000in}{0.000000in}}%
\pgfpathlineto{\pgfqpoint{-0.048611in}{0.000000in}}%
\pgfusepath{stroke,fill}%
}%
\begin{pgfscope}%
\pgfsys@transformshift{0.588387in}{1.110268in}%
\pgfsys@useobject{currentmarker}{}%
\end{pgfscope}%
\end{pgfscope}%
\begin{pgfscope}%
\definecolor{textcolor}{rgb}{0.000000,0.000000,0.000000}%
\pgfsetstrokecolor{textcolor}%
\pgfsetfillcolor{textcolor}%
\pgftext[x=0.289968in, y=1.057506in, left, base]{\color{textcolor}{\rmfamily\fontsize{10.000000}{12.000000}\selectfont\catcode`\^=\active\def^{\ifmmode\sp\else\^{}\fi}\catcode`\%=\active\def%{\%}$\mathdefault{10^{1}}$}}%
\end{pgfscope}%
\begin{pgfscope}%
\pgfsetbuttcap%
\pgfsetroundjoin%
\definecolor{currentfill}{rgb}{0.000000,0.000000,0.000000}%
\pgfsetfillcolor{currentfill}%
\pgfsetlinewidth{0.803000pt}%
\definecolor{currentstroke}{rgb}{0.000000,0.000000,0.000000}%
\pgfsetstrokecolor{currentstroke}%
\pgfsetdash{}{0pt}%
\pgfsys@defobject{currentmarker}{\pgfqpoint{-0.048611in}{0.000000in}}{\pgfqpoint{-0.000000in}{0.000000in}}{%
\pgfpathmoveto{\pgfqpoint{-0.000000in}{0.000000in}}%
\pgfpathlineto{\pgfqpoint{-0.048611in}{0.000000in}}%
\pgfusepath{stroke,fill}%
}%
\begin{pgfscope}%
\pgfsys@transformshift{0.588387in}{2.667503in}%
\pgfsys@useobject{currentmarker}{}%
\end{pgfscope}%
\end{pgfscope}%
\begin{pgfscope}%
\definecolor{textcolor}{rgb}{0.000000,0.000000,0.000000}%
\pgfsetstrokecolor{textcolor}%
\pgfsetfillcolor{textcolor}%
\pgftext[x=0.289968in, y=2.614741in, left, base]{\color{textcolor}{\rmfamily\fontsize{10.000000}{12.000000}\selectfont\catcode`\^=\active\def^{\ifmmode\sp\else\^{}\fi}\catcode`\%=\active\def%{\%}$\mathdefault{10^{2}}$}}%
\end{pgfscope}%
\begin{pgfscope}%
\pgfsetbuttcap%
\pgfsetroundjoin%
\definecolor{currentfill}{rgb}{0.000000,0.000000,0.000000}%
\pgfsetfillcolor{currentfill}%
\pgfsetlinewidth{0.602250pt}%
\definecolor{currentstroke}{rgb}{0.000000,0.000000,0.000000}%
\pgfsetstrokecolor{currentstroke}%
\pgfsetdash{}{0pt}%
\pgfsys@defobject{currentmarker}{\pgfqpoint{-0.027778in}{0.000000in}}{\pgfqpoint{-0.000000in}{0.000000in}}{%
\pgfpathmoveto{\pgfqpoint{-0.000000in}{0.000000in}}%
\pgfpathlineto{\pgfqpoint{-0.027778in}{0.000000in}}%
\pgfusepath{stroke,fill}%
}%
\begin{pgfscope}%
\pgfsys@transformshift{0.588387in}{0.641493in}%
\pgfsys@useobject{currentmarker}{}%
\end{pgfscope}%
\end{pgfscope}%
\begin{pgfscope}%
\pgfsetbuttcap%
\pgfsetroundjoin%
\definecolor{currentfill}{rgb}{0.000000,0.000000,0.000000}%
\pgfsetfillcolor{currentfill}%
\pgfsetlinewidth{0.602250pt}%
\definecolor{currentstroke}{rgb}{0.000000,0.000000,0.000000}%
\pgfsetstrokecolor{currentstroke}%
\pgfsetdash{}{0pt}%
\pgfsys@defobject{currentmarker}{\pgfqpoint{-0.027778in}{0.000000in}}{\pgfqpoint{-0.000000in}{0.000000in}}{%
\pgfpathmoveto{\pgfqpoint{-0.000000in}{0.000000in}}%
\pgfpathlineto{\pgfqpoint{-0.027778in}{0.000000in}}%
\pgfusepath{stroke,fill}%
}%
\begin{pgfscope}%
\pgfsys@transformshift{0.588387in}{0.764797in}%
\pgfsys@useobject{currentmarker}{}%
\end{pgfscope}%
\end{pgfscope}%
\begin{pgfscope}%
\pgfsetbuttcap%
\pgfsetroundjoin%
\definecolor{currentfill}{rgb}{0.000000,0.000000,0.000000}%
\pgfsetfillcolor{currentfill}%
\pgfsetlinewidth{0.602250pt}%
\definecolor{currentstroke}{rgb}{0.000000,0.000000,0.000000}%
\pgfsetstrokecolor{currentstroke}%
\pgfsetdash{}{0pt}%
\pgfsys@defobject{currentmarker}{\pgfqpoint{-0.027778in}{0.000000in}}{\pgfqpoint{-0.000000in}{0.000000in}}{%
\pgfpathmoveto{\pgfqpoint{-0.000000in}{0.000000in}}%
\pgfpathlineto{\pgfqpoint{-0.027778in}{0.000000in}}%
\pgfusepath{stroke,fill}%
}%
\begin{pgfscope}%
\pgfsys@transformshift{0.588387in}{0.869049in}%
\pgfsys@useobject{currentmarker}{}%
\end{pgfscope}%
\end{pgfscope}%
\begin{pgfscope}%
\pgfsetbuttcap%
\pgfsetroundjoin%
\definecolor{currentfill}{rgb}{0.000000,0.000000,0.000000}%
\pgfsetfillcolor{currentfill}%
\pgfsetlinewidth{0.602250pt}%
\definecolor{currentstroke}{rgb}{0.000000,0.000000,0.000000}%
\pgfsetstrokecolor{currentstroke}%
\pgfsetdash{}{0pt}%
\pgfsys@defobject{currentmarker}{\pgfqpoint{-0.027778in}{0.000000in}}{\pgfqpoint{-0.000000in}{0.000000in}}{%
\pgfpathmoveto{\pgfqpoint{-0.000000in}{0.000000in}}%
\pgfpathlineto{\pgfqpoint{-0.027778in}{0.000000in}}%
\pgfusepath{stroke,fill}%
}%
\begin{pgfscope}%
\pgfsys@transformshift{0.588387in}{0.959356in}%
\pgfsys@useobject{currentmarker}{}%
\end{pgfscope}%
\end{pgfscope}%
\begin{pgfscope}%
\pgfsetbuttcap%
\pgfsetroundjoin%
\definecolor{currentfill}{rgb}{0.000000,0.000000,0.000000}%
\pgfsetfillcolor{currentfill}%
\pgfsetlinewidth{0.602250pt}%
\definecolor{currentstroke}{rgb}{0.000000,0.000000,0.000000}%
\pgfsetstrokecolor{currentstroke}%
\pgfsetdash{}{0pt}%
\pgfsys@defobject{currentmarker}{\pgfqpoint{-0.027778in}{0.000000in}}{\pgfqpoint{-0.000000in}{0.000000in}}{%
\pgfpathmoveto{\pgfqpoint{-0.000000in}{0.000000in}}%
\pgfpathlineto{\pgfqpoint{-0.027778in}{0.000000in}}%
\pgfusepath{stroke,fill}%
}%
\begin{pgfscope}%
\pgfsys@transformshift{0.588387in}{1.039013in}%
\pgfsys@useobject{currentmarker}{}%
\end{pgfscope}%
\end{pgfscope}%
\begin{pgfscope}%
\pgfsetbuttcap%
\pgfsetroundjoin%
\definecolor{currentfill}{rgb}{0.000000,0.000000,0.000000}%
\pgfsetfillcolor{currentfill}%
\pgfsetlinewidth{0.602250pt}%
\definecolor{currentstroke}{rgb}{0.000000,0.000000,0.000000}%
\pgfsetstrokecolor{currentstroke}%
\pgfsetdash{}{0pt}%
\pgfsys@defobject{currentmarker}{\pgfqpoint{-0.027778in}{0.000000in}}{\pgfqpoint{-0.000000in}{0.000000in}}{%
\pgfpathmoveto{\pgfqpoint{-0.000000in}{0.000000in}}%
\pgfpathlineto{\pgfqpoint{-0.027778in}{0.000000in}}%
\pgfusepath{stroke,fill}%
}%
\begin{pgfscope}%
\pgfsys@transformshift{0.588387in}{1.579042in}%
\pgfsys@useobject{currentmarker}{}%
\end{pgfscope}%
\end{pgfscope}%
\begin{pgfscope}%
\pgfsetbuttcap%
\pgfsetroundjoin%
\definecolor{currentfill}{rgb}{0.000000,0.000000,0.000000}%
\pgfsetfillcolor{currentfill}%
\pgfsetlinewidth{0.602250pt}%
\definecolor{currentstroke}{rgb}{0.000000,0.000000,0.000000}%
\pgfsetstrokecolor{currentstroke}%
\pgfsetdash{}{0pt}%
\pgfsys@defobject{currentmarker}{\pgfqpoint{-0.027778in}{0.000000in}}{\pgfqpoint{-0.000000in}{0.000000in}}{%
\pgfpathmoveto{\pgfqpoint{-0.000000in}{0.000000in}}%
\pgfpathlineto{\pgfqpoint{-0.027778in}{0.000000in}}%
\pgfusepath{stroke,fill}%
}%
\begin{pgfscope}%
\pgfsys@transformshift{0.588387in}{1.853258in}%
\pgfsys@useobject{currentmarker}{}%
\end{pgfscope}%
\end{pgfscope}%
\begin{pgfscope}%
\pgfsetbuttcap%
\pgfsetroundjoin%
\definecolor{currentfill}{rgb}{0.000000,0.000000,0.000000}%
\pgfsetfillcolor{currentfill}%
\pgfsetlinewidth{0.602250pt}%
\definecolor{currentstroke}{rgb}{0.000000,0.000000,0.000000}%
\pgfsetstrokecolor{currentstroke}%
\pgfsetdash{}{0pt}%
\pgfsys@defobject{currentmarker}{\pgfqpoint{-0.027778in}{0.000000in}}{\pgfqpoint{-0.000000in}{0.000000in}}{%
\pgfpathmoveto{\pgfqpoint{-0.000000in}{0.000000in}}%
\pgfpathlineto{\pgfqpoint{-0.027778in}{0.000000in}}%
\pgfusepath{stroke,fill}%
}%
\begin{pgfscope}%
\pgfsys@transformshift{0.588387in}{2.047817in}%
\pgfsys@useobject{currentmarker}{}%
\end{pgfscope}%
\end{pgfscope}%
\begin{pgfscope}%
\pgfsetbuttcap%
\pgfsetroundjoin%
\definecolor{currentfill}{rgb}{0.000000,0.000000,0.000000}%
\pgfsetfillcolor{currentfill}%
\pgfsetlinewidth{0.602250pt}%
\definecolor{currentstroke}{rgb}{0.000000,0.000000,0.000000}%
\pgfsetstrokecolor{currentstroke}%
\pgfsetdash{}{0pt}%
\pgfsys@defobject{currentmarker}{\pgfqpoint{-0.027778in}{0.000000in}}{\pgfqpoint{-0.000000in}{0.000000in}}{%
\pgfpathmoveto{\pgfqpoint{-0.000000in}{0.000000in}}%
\pgfpathlineto{\pgfqpoint{-0.027778in}{0.000000in}}%
\pgfusepath{stroke,fill}%
}%
\begin{pgfscope}%
\pgfsys@transformshift{0.588387in}{2.198728in}%
\pgfsys@useobject{currentmarker}{}%
\end{pgfscope}%
\end{pgfscope}%
\begin{pgfscope}%
\pgfsetbuttcap%
\pgfsetroundjoin%
\definecolor{currentfill}{rgb}{0.000000,0.000000,0.000000}%
\pgfsetfillcolor{currentfill}%
\pgfsetlinewidth{0.602250pt}%
\definecolor{currentstroke}{rgb}{0.000000,0.000000,0.000000}%
\pgfsetstrokecolor{currentstroke}%
\pgfsetdash{}{0pt}%
\pgfsys@defobject{currentmarker}{\pgfqpoint{-0.027778in}{0.000000in}}{\pgfqpoint{-0.000000in}{0.000000in}}{%
\pgfpathmoveto{\pgfqpoint{-0.000000in}{0.000000in}}%
\pgfpathlineto{\pgfqpoint{-0.027778in}{0.000000in}}%
\pgfusepath{stroke,fill}%
}%
\begin{pgfscope}%
\pgfsys@transformshift{0.588387in}{2.322032in}%
\pgfsys@useobject{currentmarker}{}%
\end{pgfscope}%
\end{pgfscope}%
\begin{pgfscope}%
\pgfsetbuttcap%
\pgfsetroundjoin%
\definecolor{currentfill}{rgb}{0.000000,0.000000,0.000000}%
\pgfsetfillcolor{currentfill}%
\pgfsetlinewidth{0.602250pt}%
\definecolor{currentstroke}{rgb}{0.000000,0.000000,0.000000}%
\pgfsetstrokecolor{currentstroke}%
\pgfsetdash{}{0pt}%
\pgfsys@defobject{currentmarker}{\pgfqpoint{-0.027778in}{0.000000in}}{\pgfqpoint{-0.000000in}{0.000000in}}{%
\pgfpathmoveto{\pgfqpoint{-0.000000in}{0.000000in}}%
\pgfpathlineto{\pgfqpoint{-0.027778in}{0.000000in}}%
\pgfusepath{stroke,fill}%
}%
\begin{pgfscope}%
\pgfsys@transformshift{0.588387in}{2.426284in}%
\pgfsys@useobject{currentmarker}{}%
\end{pgfscope}%
\end{pgfscope}%
\begin{pgfscope}%
\pgfsetbuttcap%
\pgfsetroundjoin%
\definecolor{currentfill}{rgb}{0.000000,0.000000,0.000000}%
\pgfsetfillcolor{currentfill}%
\pgfsetlinewidth{0.602250pt}%
\definecolor{currentstroke}{rgb}{0.000000,0.000000,0.000000}%
\pgfsetstrokecolor{currentstroke}%
\pgfsetdash{}{0pt}%
\pgfsys@defobject{currentmarker}{\pgfqpoint{-0.027778in}{0.000000in}}{\pgfqpoint{-0.000000in}{0.000000in}}{%
\pgfpathmoveto{\pgfqpoint{-0.000000in}{0.000000in}}%
\pgfpathlineto{\pgfqpoint{-0.027778in}{0.000000in}}%
\pgfusepath{stroke,fill}%
}%
\begin{pgfscope}%
\pgfsys@transformshift{0.588387in}{2.516591in}%
\pgfsys@useobject{currentmarker}{}%
\end{pgfscope}%
\end{pgfscope}%
\begin{pgfscope}%
\pgfsetbuttcap%
\pgfsetroundjoin%
\definecolor{currentfill}{rgb}{0.000000,0.000000,0.000000}%
\pgfsetfillcolor{currentfill}%
\pgfsetlinewidth{0.602250pt}%
\definecolor{currentstroke}{rgb}{0.000000,0.000000,0.000000}%
\pgfsetstrokecolor{currentstroke}%
\pgfsetdash{}{0pt}%
\pgfsys@defobject{currentmarker}{\pgfqpoint{-0.027778in}{0.000000in}}{\pgfqpoint{-0.000000in}{0.000000in}}{%
\pgfpathmoveto{\pgfqpoint{-0.000000in}{0.000000in}}%
\pgfpathlineto{\pgfqpoint{-0.027778in}{0.000000in}}%
\pgfusepath{stroke,fill}%
}%
\begin{pgfscope}%
\pgfsys@transformshift{0.588387in}{2.596248in}%
\pgfsys@useobject{currentmarker}{}%
\end{pgfscope}%
\end{pgfscope}%
\begin{pgfscope}%
\definecolor{textcolor}{rgb}{0.000000,0.000000,0.000000}%
\pgfsetstrokecolor{textcolor}%
\pgfsetfillcolor{textcolor}%
\pgftext[x=0.234413in,y=1.631726in,,bottom,rotate=90.000000]{\color{textcolor}{\rmfamily\fontsize{10.000000}{12.000000}\selectfont\catcode`\^=\active\def^{\ifmmode\sp\else\^{}\fi}\catcode`\%=\active\def%{\%}Time [ms]}}%
\end{pgfscope}%
\begin{pgfscope}%
\pgfpathrectangle{\pgfqpoint{0.588387in}{0.521603in}}{\pgfqpoint{3.660036in}{2.220246in}}%
\pgfusepath{clip}%
\pgfsetrectcap%
\pgfsetroundjoin%
\pgfsetlinewidth{1.505625pt}%
\pgfsetstrokecolor{currentstroke1}%
\pgfsetdash{}{0pt}%
\pgfpathmoveto{\pgfqpoint{0.754752in}{0.703411in}}%
\pgfpathlineto{\pgfqpoint{0.825546in}{0.764207in}}%
\pgfpathlineto{\pgfqpoint{0.896340in}{0.854526in}}%
\pgfpathlineto{\pgfqpoint{0.967134in}{0.945554in}}%
\pgfpathlineto{\pgfqpoint{1.037927in}{1.019574in}}%
\pgfpathlineto{\pgfqpoint{1.108721in}{1.103129in}}%
\pgfpathlineto{\pgfqpoint{1.179515in}{1.194716in}}%
\pgfpathlineto{\pgfqpoint{1.250309in}{1.245581in}}%
\pgfpathlineto{\pgfqpoint{1.321102in}{1.309208in}}%
\pgfpathlineto{\pgfqpoint{1.391896in}{1.374262in}}%
\pgfpathlineto{\pgfqpoint{1.462690in}{1.392815in}}%
\pgfpathlineto{\pgfqpoint{1.533483in}{1.453639in}}%
\pgfpathlineto{\pgfqpoint{1.604277in}{1.567618in}}%
\pgfpathlineto{\pgfqpoint{1.675071in}{1.568134in}}%
\pgfpathlineto{\pgfqpoint{1.745865in}{1.614623in}}%
\pgfpathlineto{\pgfqpoint{1.816658in}{1.668546in}}%
\pgfpathlineto{\pgfqpoint{1.887452in}{1.702487in}}%
\pgfpathlineto{\pgfqpoint{1.958246in}{1.740289in}}%
\pgfpathlineto{\pgfqpoint{2.029039in}{1.804421in}}%
\pgfpathlineto{\pgfqpoint{2.099833in}{1.864104in}}%
\pgfpathlineto{\pgfqpoint{2.170627in}{1.885610in}}%
\pgfpathlineto{\pgfqpoint{2.241421in}{1.956833in}}%
\pgfpathlineto{\pgfqpoint{2.312214in}{1.969100in}}%
\pgfpathlineto{\pgfqpoint{2.383008in}{2.016161in}}%
\pgfpathlineto{\pgfqpoint{2.453802in}{2.000124in}}%
\pgfpathlineto{\pgfqpoint{2.524596in}{2.060380in}}%
\pgfpathlineto{\pgfqpoint{2.595389in}{2.099422in}}%
\pgfpathlineto{\pgfqpoint{2.666183in}{2.103803in}}%
\pgfpathlineto{\pgfqpoint{2.736977in}{2.180969in}}%
\pgfpathlineto{\pgfqpoint{2.807770in}{2.196724in}}%
\pgfpathlineto{\pgfqpoint{2.878564in}{2.201552in}}%
\pgfpathlineto{\pgfqpoint{2.949358in}{2.221480in}}%
\pgfpathlineto{\pgfqpoint{3.020152in}{2.314666in}}%
\pgfpathlineto{\pgfqpoint{3.090945in}{2.286503in}}%
\pgfpathlineto{\pgfqpoint{3.161739in}{2.299105in}}%
\pgfpathlineto{\pgfqpoint{3.232533in}{2.329244in}}%
\pgfpathlineto{\pgfqpoint{3.303327in}{2.337978in}}%
\pgfpathlineto{\pgfqpoint{3.374120in}{2.389176in}}%
\pgfpathlineto{\pgfqpoint{3.444914in}{2.449302in}}%
\pgfpathlineto{\pgfqpoint{3.515708in}{2.449776in}}%
\pgfpathlineto{\pgfqpoint{3.586501in}{2.470720in}}%
\pgfpathlineto{\pgfqpoint{3.657295in}{2.491459in}}%
\pgfpathlineto{\pgfqpoint{3.728089in}{2.550124in}}%
\pgfpathlineto{\pgfqpoint{3.798883in}{2.527628in}}%
\pgfpathlineto{\pgfqpoint{3.869676in}{2.498544in}}%
\pgfpathlineto{\pgfqpoint{3.940470in}{2.581480in}}%
\pgfpathlineto{\pgfqpoint{4.011264in}{2.629696in}}%
\pgfpathlineto{\pgfqpoint{4.082057in}{2.629244in}}%
\pgfusepath{stroke}%
\end{pgfscope}%
\begin{pgfscope}%
\pgfpathrectangle{\pgfqpoint{0.588387in}{0.521603in}}{\pgfqpoint{3.660036in}{2.220246in}}%
\pgfusepath{clip}%
\pgfsetrectcap%
\pgfsetroundjoin%
\pgfsetlinewidth{1.505625pt}%
\pgfsetstrokecolor{currentstroke2}%
\pgfsetdash{}{0pt}%
\pgfpathmoveto{\pgfqpoint{0.754752in}{0.691366in}}%
\pgfpathlineto{\pgfqpoint{0.825546in}{0.761228in}}%
\pgfpathlineto{\pgfqpoint{0.896340in}{0.864024in}}%
\pgfpathlineto{\pgfqpoint{0.967134in}{0.989419in}}%
\pgfpathlineto{\pgfqpoint{1.037927in}{1.036377in}}%
\pgfpathlineto{\pgfqpoint{1.108721in}{1.136142in}}%
\pgfpathlineto{\pgfqpoint{1.179515in}{1.227884in}}%
\pgfpathlineto{\pgfqpoint{1.250309in}{1.249721in}}%
\pgfpathlineto{\pgfqpoint{1.321102in}{1.313479in}}%
\pgfpathlineto{\pgfqpoint{1.391896in}{1.400523in}}%
\pgfpathlineto{\pgfqpoint{1.462690in}{1.397053in}}%
\pgfpathlineto{\pgfqpoint{1.533483in}{1.453231in}}%
\pgfpathlineto{\pgfqpoint{1.604277in}{1.561658in}}%
\pgfpathlineto{\pgfqpoint{1.675071in}{1.572554in}}%
\pgfpathlineto{\pgfqpoint{1.745865in}{1.614771in}}%
\pgfpathlineto{\pgfqpoint{1.816658in}{1.652337in}}%
\pgfpathlineto{\pgfqpoint{1.887452in}{1.697514in}}%
\pgfpathlineto{\pgfqpoint{1.958246in}{1.720599in}}%
\pgfpathlineto{\pgfqpoint{2.029039in}{1.807201in}}%
\pgfpathlineto{\pgfqpoint{2.099833in}{1.869737in}}%
\pgfpathlineto{\pgfqpoint{2.170627in}{1.886576in}}%
\pgfpathlineto{\pgfqpoint{2.241421in}{1.935994in}}%
\pgfpathlineto{\pgfqpoint{2.312214in}{1.958668in}}%
\pgfpathlineto{\pgfqpoint{2.383008in}{1.994999in}}%
\pgfpathlineto{\pgfqpoint{2.453802in}{1.997737in}}%
\pgfpathlineto{\pgfqpoint{2.524596in}{2.052136in}}%
\pgfpathlineto{\pgfqpoint{2.595389in}{2.105644in}}%
\pgfpathlineto{\pgfqpoint{2.666183in}{2.102131in}}%
\pgfpathlineto{\pgfqpoint{2.736977in}{2.179879in}}%
\pgfpathlineto{\pgfqpoint{2.807770in}{2.197201in}}%
\pgfpathlineto{\pgfqpoint{2.878564in}{2.203199in}}%
\pgfpathlineto{\pgfqpoint{2.949358in}{2.219779in}}%
\pgfpathlineto{\pgfqpoint{3.020152in}{2.309166in}}%
\pgfpathlineto{\pgfqpoint{3.090945in}{2.289676in}}%
\pgfpathlineto{\pgfqpoint{3.161739in}{2.296827in}}%
\pgfpathlineto{\pgfqpoint{3.232533in}{2.329988in}}%
\pgfpathlineto{\pgfqpoint{3.303327in}{2.332483in}}%
\pgfpathlineto{\pgfqpoint{3.374120in}{2.379310in}}%
\pgfpathlineto{\pgfqpoint{3.444914in}{2.432418in}}%
\pgfpathlineto{\pgfqpoint{3.515708in}{2.451697in}}%
\pgfpathlineto{\pgfqpoint{3.586501in}{2.462395in}}%
\pgfpathlineto{\pgfqpoint{3.657295in}{2.485434in}}%
\pgfpathlineto{\pgfqpoint{3.728089in}{2.546204in}}%
\pgfpathlineto{\pgfqpoint{3.798883in}{2.519027in}}%
\pgfpathlineto{\pgfqpoint{3.869676in}{2.497441in}}%
\pgfpathlineto{\pgfqpoint{3.940470in}{2.580693in}}%
\pgfpathlineto{\pgfqpoint{4.011264in}{2.625432in}}%
\pgfpathlineto{\pgfqpoint{4.082057in}{2.614109in}}%
\pgfusepath{stroke}%
\end{pgfscope}%
\begin{pgfscope}%
\pgfpathrectangle{\pgfqpoint{0.588387in}{0.521603in}}{\pgfqpoint{3.660036in}{2.220246in}}%
\pgfusepath{clip}%
\pgfsetrectcap%
\pgfsetroundjoin%
\pgfsetlinewidth{1.505625pt}%
\pgfsetstrokecolor{currentstroke3}%
\pgfsetdash{}{0pt}%
\pgfpathmoveto{\pgfqpoint{0.754752in}{0.622524in}}%
\pgfpathlineto{\pgfqpoint{0.825546in}{0.732009in}}%
\pgfpathlineto{\pgfqpoint{0.896340in}{0.772616in}}%
\pgfpathlineto{\pgfqpoint{0.967134in}{0.860502in}}%
\pgfpathlineto{\pgfqpoint{1.037927in}{0.962020in}}%
\pgfpathlineto{\pgfqpoint{1.108721in}{0.970367in}}%
\pgfpathlineto{\pgfqpoint{1.179515in}{1.073439in}}%
\pgfpathlineto{\pgfqpoint{1.250309in}{1.149037in}}%
\pgfpathlineto{\pgfqpoint{1.321102in}{1.157919in}}%
\pgfpathlineto{\pgfqpoint{1.391896in}{1.288224in}}%
\pgfpathlineto{\pgfqpoint{1.533483in}{1.345986in}}%
\pgfpathlineto{\pgfqpoint{1.604277in}{1.411981in}}%
\pgfpathlineto{\pgfqpoint{1.675071in}{1.500990in}}%
\pgfpathlineto{\pgfqpoint{1.745865in}{1.549318in}}%
\pgfpathlineto{\pgfqpoint{1.816658in}{1.576671in}}%
\pgfpathlineto{\pgfqpoint{1.887452in}{1.582079in}}%
\pgfpathlineto{\pgfqpoint{1.958246in}{1.641345in}}%
\pgfpathlineto{\pgfqpoint{2.029039in}{1.705158in}}%
\pgfpathlineto{\pgfqpoint{2.099833in}{1.744158in}}%
\pgfpathlineto{\pgfqpoint{2.170627in}{1.746687in}}%
\pgfpathlineto{\pgfqpoint{2.241421in}{1.849868in}}%
\pgfpathlineto{\pgfqpoint{2.312214in}{1.854159in}}%
\pgfpathlineto{\pgfqpoint{2.383008in}{1.896905in}}%
\pgfpathlineto{\pgfqpoint{2.453802in}{1.928491in}}%
\pgfpathlineto{\pgfqpoint{2.524596in}{1.970713in}}%
\pgfpathlineto{\pgfqpoint{2.595389in}{2.012593in}}%
\pgfpathlineto{\pgfqpoint{2.666183in}{2.057719in}}%
\pgfpathlineto{\pgfqpoint{2.736977in}{2.072548in}}%
\pgfpathlineto{\pgfqpoint{2.807770in}{2.099610in}}%
\pgfpathlineto{\pgfqpoint{2.878564in}{2.118090in}}%
\pgfpathlineto{\pgfqpoint{2.949358in}{2.137172in}}%
\pgfusepath{stroke}%
\end{pgfscope}%
\begin{pgfscope}%
\pgfpathrectangle{\pgfqpoint{0.588387in}{0.521603in}}{\pgfqpoint{3.660036in}{2.220246in}}%
\pgfusepath{clip}%
\pgfsetrectcap%
\pgfsetroundjoin%
\pgfsetlinewidth{1.505625pt}%
\pgfsetstrokecolor{currentstroke4}%
\pgfsetdash{}{0pt}%
\pgfpathmoveto{\pgfqpoint{0.754752in}{0.705005in}}%
\pgfpathlineto{\pgfqpoint{0.825546in}{0.761247in}}%
\pgfpathlineto{\pgfqpoint{0.896340in}{0.857375in}}%
\pgfpathlineto{\pgfqpoint{0.967134in}{0.942496in}}%
\pgfpathlineto{\pgfqpoint{1.037927in}{1.040908in}}%
\pgfpathlineto{\pgfqpoint{1.108721in}{1.065538in}}%
\pgfpathlineto{\pgfqpoint{1.179515in}{1.140036in}}%
\pgfpathlineto{\pgfqpoint{1.250309in}{1.190225in}}%
\pgfpathlineto{\pgfqpoint{1.321102in}{1.285881in}}%
\pgfpathlineto{\pgfqpoint{1.391896in}{1.354523in}}%
\pgfpathlineto{\pgfqpoint{1.462690in}{1.385834in}}%
\pgfpathlineto{\pgfqpoint{1.533483in}{1.406877in}}%
\pgfpathlineto{\pgfqpoint{1.604277in}{1.502317in}}%
\pgfpathlineto{\pgfqpoint{1.675071in}{1.521364in}}%
\pgfpathlineto{\pgfqpoint{1.745865in}{1.576162in}}%
\pgfpathlineto{\pgfqpoint{1.816658in}{1.620161in}}%
\pgfpathlineto{\pgfqpoint{1.887452in}{1.658248in}}%
\pgfpathlineto{\pgfqpoint{1.958246in}{1.702487in}}%
\pgfpathlineto{\pgfqpoint{2.029039in}{1.753611in}}%
\pgfpathlineto{\pgfqpoint{2.099833in}{1.808407in}}%
\pgfpathlineto{\pgfqpoint{2.170627in}{1.816548in}}%
\pgfpathlineto{\pgfqpoint{2.241421in}{1.886740in}}%
\pgfpathlineto{\pgfqpoint{2.312214in}{1.913605in}}%
\pgfpathlineto{\pgfqpoint{2.383008in}{1.958577in}}%
\pgfpathlineto{\pgfqpoint{2.453802in}{1.955671in}}%
\pgfpathlineto{\pgfqpoint{2.524596in}{2.005934in}}%
\pgfpathlineto{\pgfqpoint{2.595389in}{2.055667in}}%
\pgfpathlineto{\pgfqpoint{2.666183in}{2.056134in}}%
\pgfpathlineto{\pgfqpoint{2.736977in}{2.136083in}}%
\pgfpathlineto{\pgfqpoint{2.807770in}{2.173904in}}%
\pgfpathlineto{\pgfqpoint{2.878564in}{2.219723in}}%
\pgfpathlineto{\pgfqpoint{2.949358in}{2.188731in}}%
\pgfpathlineto{\pgfqpoint{3.020152in}{2.233309in}}%
\pgfpathlineto{\pgfqpoint{3.090945in}{2.236716in}}%
\pgfpathlineto{\pgfqpoint{3.161739in}{2.285442in}}%
\pgfpathlineto{\pgfqpoint{3.232533in}{2.288779in}}%
\pgfpathlineto{\pgfqpoint{3.303327in}{2.272622in}}%
\pgfpathlineto{\pgfqpoint{3.374120in}{2.316832in}}%
\pgfpathlineto{\pgfqpoint{3.444914in}{2.382276in}}%
\pgfpathlineto{\pgfqpoint{3.515708in}{2.388622in}}%
\pgfpathlineto{\pgfqpoint{3.586501in}{2.419412in}}%
\pgfpathlineto{\pgfqpoint{3.657295in}{2.513490in}}%
\pgfpathlineto{\pgfqpoint{3.728089in}{2.482127in}}%
\pgfpathlineto{\pgfqpoint{3.798883in}{2.524691in}}%
\pgfpathlineto{\pgfqpoint{3.869676in}{2.496602in}}%
\pgfpathlineto{\pgfqpoint{3.940470in}{2.581326in}}%
\pgfpathlineto{\pgfqpoint{4.011264in}{2.614389in}}%
\pgfpathlineto{\pgfqpoint{4.082057in}{2.615585in}}%
\pgfusepath{stroke}%
\end{pgfscope}%
\begin{pgfscope}%
\pgfpathrectangle{\pgfqpoint{0.588387in}{0.521603in}}{\pgfqpoint{3.660036in}{2.220246in}}%
\pgfusepath{clip}%
\pgfsetrectcap%
\pgfsetroundjoin%
\pgfsetlinewidth{1.505625pt}%
\pgfsetstrokecolor{currentstroke5}%
\pgfsetdash{}{0pt}%
\pgfpathmoveto{\pgfqpoint{0.754752in}{0.699298in}}%
\pgfpathlineto{\pgfqpoint{0.825546in}{0.766603in}}%
\pgfpathlineto{\pgfqpoint{0.896340in}{0.854811in}}%
\pgfpathlineto{\pgfqpoint{0.967134in}{0.969681in}}%
\pgfpathlineto{\pgfqpoint{1.037927in}{1.045776in}}%
\pgfpathlineto{\pgfqpoint{1.108721in}{1.094185in}}%
\pgfpathlineto{\pgfqpoint{1.179515in}{1.165440in}}%
\pgfpathlineto{\pgfqpoint{1.250309in}{1.219909in}}%
\pgfpathlineto{\pgfqpoint{1.321102in}{1.329028in}}%
\pgfpathlineto{\pgfqpoint{1.391896in}{1.391213in}}%
\pgfpathlineto{\pgfqpoint{1.462690in}{1.384934in}}%
\pgfpathlineto{\pgfqpoint{1.533483in}{1.413748in}}%
\pgfpathlineto{\pgfqpoint{1.604277in}{1.545246in}}%
\pgfpathlineto{\pgfqpoint{1.675071in}{1.513989in}}%
\pgfpathlineto{\pgfqpoint{1.745865in}{1.573269in}}%
\pgfpathlineto{\pgfqpoint{1.816658in}{1.626851in}}%
\pgfpathlineto{\pgfqpoint{1.887452in}{1.656591in}}%
\pgfpathlineto{\pgfqpoint{1.958246in}{1.699494in}}%
\pgfpathlineto{\pgfqpoint{2.029039in}{1.763208in}}%
\pgfpathlineto{\pgfqpoint{2.099833in}{1.811944in}}%
\pgfpathlineto{\pgfqpoint{2.170627in}{1.825036in}}%
\pgfpathlineto{\pgfqpoint{2.241421in}{1.899343in}}%
\pgfpathlineto{\pgfqpoint{2.312214in}{1.920987in}}%
\pgfpathlineto{\pgfqpoint{2.383008in}{1.948867in}}%
\pgfpathlineto{\pgfqpoint{2.453802in}{1.952370in}}%
\pgfpathlineto{\pgfqpoint{2.524596in}{2.008036in}}%
\pgfpathlineto{\pgfqpoint{2.595389in}{2.062430in}}%
\pgfpathlineto{\pgfqpoint{2.666183in}{2.055756in}}%
\pgfpathlineto{\pgfqpoint{2.736977in}{2.132930in}}%
\pgfpathlineto{\pgfqpoint{2.807770in}{2.168369in}}%
\pgfpathlineto{\pgfqpoint{2.878564in}{2.222727in}}%
\pgfpathlineto{\pgfqpoint{2.949358in}{2.180118in}}%
\pgfpathlineto{\pgfqpoint{3.020152in}{2.238391in}}%
\pgfpathlineto{\pgfqpoint{3.090945in}{2.237941in}}%
\pgfpathlineto{\pgfqpoint{3.161739in}{2.290080in}}%
\pgfpathlineto{\pgfqpoint{3.232533in}{2.285084in}}%
\pgfpathlineto{\pgfqpoint{3.303327in}{2.274145in}}%
\pgfpathlineto{\pgfqpoint{3.374120in}{2.326694in}}%
\pgfpathlineto{\pgfqpoint{3.444914in}{2.383871in}}%
\pgfpathlineto{\pgfqpoint{3.515708in}{2.389088in}}%
\pgfpathlineto{\pgfqpoint{3.586501in}{2.436226in}}%
\pgfpathlineto{\pgfqpoint{3.657295in}{2.523417in}}%
\pgfpathlineto{\pgfqpoint{3.728089in}{2.489655in}}%
\pgfpathlineto{\pgfqpoint{3.798883in}{2.535830in}}%
\pgfpathlineto{\pgfqpoint{3.869676in}{2.497963in}}%
\pgfpathlineto{\pgfqpoint{3.940470in}{2.594735in}}%
\pgfpathlineto{\pgfqpoint{4.011264in}{2.640929in}}%
\pgfpathlineto{\pgfqpoint{4.082057in}{2.623457in}}%
\pgfusepath{stroke}%
\end{pgfscope}%
\begin{pgfscope}%
\pgfpathrectangle{\pgfqpoint{0.588387in}{0.521603in}}{\pgfqpoint{3.660036in}{2.220246in}}%
\pgfusepath{clip}%
\pgfsetrectcap%
\pgfsetroundjoin%
\pgfsetlinewidth{1.505625pt}%
\pgfsetstrokecolor{currentstroke6}%
\pgfsetdash{}{0pt}%
\pgfpathmoveto{\pgfqpoint{0.754752in}{0.667069in}}%
\pgfpathlineto{\pgfqpoint{0.825546in}{0.746893in}}%
\pgfpathlineto{\pgfqpoint{0.896340in}{0.829682in}}%
\pgfpathlineto{\pgfqpoint{0.967134in}{0.919549in}}%
\pgfpathlineto{\pgfqpoint{1.037927in}{0.997566in}}%
\pgfpathlineto{\pgfqpoint{1.108721in}{1.067119in}}%
\pgfpathlineto{\pgfqpoint{1.179515in}{1.169170in}}%
\pgfpathlineto{\pgfqpoint{1.250309in}{1.226205in}}%
\pgfpathlineto{\pgfqpoint{1.321102in}{1.288484in}}%
\pgfpathlineto{\pgfqpoint{1.391896in}{1.352872in}}%
\pgfpathlineto{\pgfqpoint{1.462690in}{1.383355in}}%
\pgfpathlineto{\pgfqpoint{1.533483in}{1.430872in}}%
\pgfpathlineto{\pgfqpoint{1.604277in}{1.532503in}}%
\pgfpathlineto{\pgfqpoint{1.675071in}{1.557571in}}%
\pgfpathlineto{\pgfqpoint{1.745865in}{1.608163in}}%
\pgfpathlineto{\pgfqpoint{1.816658in}{1.657646in}}%
\pgfpathlineto{\pgfqpoint{1.887452in}{1.689545in}}%
\pgfpathlineto{\pgfqpoint{1.958246in}{1.724113in}}%
\pgfpathlineto{\pgfqpoint{2.029039in}{1.787643in}}%
\pgfpathlineto{\pgfqpoint{2.099833in}{1.845663in}}%
\pgfpathlineto{\pgfqpoint{2.170627in}{1.844865in}}%
\pgfpathlineto{\pgfqpoint{2.241421in}{1.939594in}}%
\pgfpathlineto{\pgfqpoint{2.312214in}{1.942267in}}%
\pgfpathlineto{\pgfqpoint{2.383008in}{1.976561in}}%
\pgfpathlineto{\pgfqpoint{2.453802in}{1.986631in}}%
\pgfpathlineto{\pgfqpoint{2.524596in}{2.036049in}}%
\pgfpathlineto{\pgfqpoint{2.595389in}{2.067923in}}%
\pgfpathlineto{\pgfqpoint{2.666183in}{2.099865in}}%
\pgfpathlineto{\pgfqpoint{2.736977in}{2.164402in}}%
\pgfpathlineto{\pgfqpoint{2.807770in}{2.170622in}}%
\pgfpathlineto{\pgfqpoint{2.878564in}{2.172492in}}%
\pgfpathlineto{\pgfqpoint{2.949358in}{2.195647in}}%
\pgfpathlineto{\pgfqpoint{3.020152in}{2.309977in}}%
\pgfpathlineto{\pgfqpoint{3.090945in}{2.288185in}}%
\pgfpathlineto{\pgfqpoint{3.161739in}{2.292955in}}%
\pgfpathlineto{\pgfqpoint{3.232533in}{2.321919in}}%
\pgfpathlineto{\pgfqpoint{3.303327in}{2.347615in}}%
\pgfpathlineto{\pgfqpoint{3.374120in}{2.392313in}}%
\pgfpathlineto{\pgfqpoint{3.444914in}{2.456034in}}%
\pgfpathlineto{\pgfqpoint{3.515708in}{2.460471in}}%
\pgfpathlineto{\pgfqpoint{3.586501in}{2.449946in}}%
\pgfpathlineto{\pgfqpoint{3.657295in}{2.460859in}}%
\pgfpathlineto{\pgfqpoint{3.728089in}{2.565662in}}%
\pgfpathlineto{\pgfqpoint{3.798883in}{2.516634in}}%
\pgfpathlineto{\pgfqpoint{3.869676in}{2.526678in}}%
\pgfpathlineto{\pgfqpoint{3.940470in}{2.576819in}}%
\pgfpathlineto{\pgfqpoint{4.011264in}{2.608008in}}%
\pgfpathlineto{\pgfqpoint{4.082057in}{2.626192in}}%
\pgfusepath{stroke}%
\end{pgfscope}%
\begin{pgfscope}%
\pgfpathrectangle{\pgfqpoint{0.588387in}{0.521603in}}{\pgfqpoint{3.660036in}{2.220246in}}%
\pgfusepath{clip}%
\pgfsetrectcap%
\pgfsetroundjoin%
\pgfsetlinewidth{1.505625pt}%
\pgfsetstrokecolor{currentstroke7}%
\pgfsetdash{}{0pt}%
\pgfpathmoveto{\pgfqpoint{0.754752in}{0.682494in}}%
\pgfpathlineto{\pgfqpoint{0.825546in}{0.749475in}}%
\pgfpathlineto{\pgfqpoint{0.896340in}{0.833258in}}%
\pgfpathlineto{\pgfqpoint{0.967134in}{0.949031in}}%
\pgfpathlineto{\pgfqpoint{1.037927in}{1.009445in}}%
\pgfpathlineto{\pgfqpoint{1.108721in}{1.091409in}}%
\pgfpathlineto{\pgfqpoint{1.179515in}{1.178099in}}%
\pgfpathlineto{\pgfqpoint{1.250309in}{1.238345in}}%
\pgfpathlineto{\pgfqpoint{1.321102in}{1.310216in}}%
\pgfpathlineto{\pgfqpoint{1.391896in}{1.371510in}}%
\pgfpathlineto{\pgfqpoint{1.462690in}{1.393886in}}%
\pgfpathlineto{\pgfqpoint{1.533483in}{1.443179in}}%
\pgfpathlineto{\pgfqpoint{1.604277in}{1.591668in}}%
\pgfpathlineto{\pgfqpoint{1.675071in}{1.567146in}}%
\pgfpathlineto{\pgfqpoint{1.745865in}{1.620679in}}%
\pgfpathlineto{\pgfqpoint{1.816658in}{1.651526in}}%
\pgfpathlineto{\pgfqpoint{1.887452in}{1.686070in}}%
\pgfpathlineto{\pgfqpoint{1.958246in}{1.710208in}}%
\pgfpathlineto{\pgfqpoint{2.029039in}{1.781749in}}%
\pgfpathlineto{\pgfqpoint{2.099833in}{1.877394in}}%
\pgfpathlineto{\pgfqpoint{2.170627in}{1.859976in}}%
\pgfpathlineto{\pgfqpoint{2.241421in}{1.941832in}}%
\pgfpathlineto{\pgfqpoint{2.312214in}{1.927834in}}%
\pgfpathlineto{\pgfqpoint{2.383008in}{1.968352in}}%
\pgfpathlineto{\pgfqpoint{2.453802in}{1.983511in}}%
\pgfpathlineto{\pgfqpoint{2.524596in}{2.025747in}}%
\pgfpathlineto{\pgfqpoint{2.595389in}{2.070983in}}%
\pgfpathlineto{\pgfqpoint{2.666183in}{2.091491in}}%
\pgfpathlineto{\pgfqpoint{2.736977in}{2.179248in}}%
\pgfpathlineto{\pgfqpoint{2.807770in}{2.165807in}}%
\pgfpathlineto{\pgfqpoint{2.878564in}{2.175991in}}%
\pgfpathlineto{\pgfqpoint{2.949358in}{2.202904in}}%
\pgfpathlineto{\pgfqpoint{3.020152in}{2.310549in}}%
\pgfpathlineto{\pgfqpoint{3.090945in}{2.289867in}}%
\pgfpathlineto{\pgfqpoint{3.161739in}{2.296710in}}%
\pgfpathlineto{\pgfqpoint{3.232533in}{2.322772in}}%
\pgfpathlineto{\pgfqpoint{3.303327in}{2.346665in}}%
\pgfpathlineto{\pgfqpoint{3.374120in}{2.392057in}}%
\pgfpathlineto{\pgfqpoint{3.444914in}{2.454808in}}%
\pgfpathlineto{\pgfqpoint{3.515708in}{2.453409in}}%
\pgfpathlineto{\pgfqpoint{3.586501in}{2.456767in}}%
\pgfpathlineto{\pgfqpoint{3.657295in}{2.464881in}}%
\pgfpathlineto{\pgfqpoint{3.728089in}{2.563543in}}%
\pgfpathlineto{\pgfqpoint{3.798883in}{2.517281in}}%
\pgfpathlineto{\pgfqpoint{3.869676in}{2.520761in}}%
\pgfpathlineto{\pgfqpoint{3.940470in}{2.584842in}}%
\pgfpathlineto{\pgfqpoint{4.011264in}{2.611298in}}%
\pgfpathlineto{\pgfqpoint{4.082057in}{2.626739in}}%
\pgfusepath{stroke}%
\end{pgfscope}%
\begin{pgfscope}%
\pgfpathrectangle{\pgfqpoint{0.588387in}{0.521603in}}{\pgfqpoint{3.660036in}{2.220246in}}%
\pgfusepath{clip}%
\pgfsetrectcap%
\pgfsetroundjoin%
\pgfsetlinewidth{1.505625pt}%
\definecolor{currentstroke}{rgb}{0.498039,0.498039,0.498039}%
\pgfsetstrokecolor{currentstroke}%
\pgfsetdash{}{0pt}%
\pgfpathmoveto{\pgfqpoint{0.754752in}{0.716122in}}%
\pgfpathlineto{\pgfqpoint{0.825546in}{0.761091in}}%
\pgfpathlineto{\pgfqpoint{0.896340in}{0.863998in}}%
\pgfpathlineto{\pgfqpoint{0.967134in}{0.942678in}}%
\pgfpathlineto{\pgfqpoint{1.037927in}{1.018881in}}%
\pgfpathlineto{\pgfqpoint{1.108721in}{1.058638in}}%
\pgfpathlineto{\pgfqpoint{1.179515in}{1.144551in}}%
\pgfpathlineto{\pgfqpoint{1.250309in}{1.203317in}}%
\pgfpathlineto{\pgfqpoint{1.321102in}{1.280907in}}%
\pgfpathlineto{\pgfqpoint{1.391896in}{1.361089in}}%
\pgfpathlineto{\pgfqpoint{1.462690in}{1.380640in}}%
\pgfpathlineto{\pgfqpoint{1.533483in}{1.406004in}}%
\pgfpathlineto{\pgfqpoint{1.604277in}{1.502317in}}%
\pgfpathlineto{\pgfqpoint{1.675071in}{1.527048in}}%
\pgfpathlineto{\pgfqpoint{1.745865in}{1.566758in}}%
\pgfpathlineto{\pgfqpoint{1.816658in}{1.645181in}}%
\pgfpathlineto{\pgfqpoint{1.887452in}{1.659900in}}%
\pgfpathlineto{\pgfqpoint{1.958246in}{1.700229in}}%
\pgfpathlineto{\pgfqpoint{2.029039in}{1.762048in}}%
\pgfpathlineto{\pgfqpoint{2.099833in}{1.796892in}}%
\pgfpathlineto{\pgfqpoint{2.170627in}{1.842807in}}%
\pgfpathlineto{\pgfqpoint{2.241421in}{1.902932in}}%
\pgfpathlineto{\pgfqpoint{2.312214in}{1.919864in}}%
\pgfpathlineto{\pgfqpoint{2.383008in}{1.943480in}}%
\pgfpathlineto{\pgfqpoint{2.453802in}{1.961171in}}%
\pgfpathlineto{\pgfqpoint{2.524596in}{2.010631in}}%
\pgfpathlineto{\pgfqpoint{2.595389in}{2.059274in}}%
\pgfpathlineto{\pgfqpoint{2.666183in}{2.054713in}}%
\pgfpathlineto{\pgfqpoint{2.736977in}{2.135436in}}%
\pgfpathlineto{\pgfqpoint{2.807770in}{2.168780in}}%
\pgfpathlineto{\pgfqpoint{2.878564in}{2.218820in}}%
\pgfpathlineto{\pgfqpoint{2.949358in}{2.185412in}}%
\pgfpathlineto{\pgfqpoint{3.020152in}{2.234922in}}%
\pgfpathlineto{\pgfqpoint{3.090945in}{2.238981in}}%
\pgfpathlineto{\pgfqpoint{3.161739in}{2.287698in}}%
\pgfpathlineto{\pgfqpoint{3.232533in}{2.286149in}}%
\pgfpathlineto{\pgfqpoint{3.303327in}{2.273968in}}%
\pgfpathlineto{\pgfqpoint{3.374120in}{2.330012in}}%
\pgfpathlineto{\pgfqpoint{3.444914in}{2.395994in}}%
\pgfpathlineto{\pgfqpoint{3.515708in}{2.392052in}}%
\pgfpathlineto{\pgfqpoint{3.586501in}{2.434457in}}%
\pgfpathlineto{\pgfqpoint{3.657295in}{2.534115in}}%
\pgfpathlineto{\pgfqpoint{3.728089in}{2.496765in}}%
\pgfpathlineto{\pgfqpoint{3.798883in}{2.532643in}}%
\pgfpathlineto{\pgfqpoint{3.869676in}{2.500927in}}%
\pgfpathlineto{\pgfqpoint{3.940470in}{2.585992in}}%
\pgfpathlineto{\pgfqpoint{4.011264in}{2.624146in}}%
\pgfpathlineto{\pgfqpoint{4.082057in}{2.605049in}}%
\pgfusepath{stroke}%
\end{pgfscope}%
\begin{pgfscope}%
\pgfpathrectangle{\pgfqpoint{0.588387in}{0.521603in}}{\pgfqpoint{3.660036in}{2.220246in}}%
\pgfusepath{clip}%
\pgfsetrectcap%
\pgfsetroundjoin%
\pgfsetlinewidth{1.505625pt}%
\definecolor{currentstroke}{rgb}{0.737255,0.741176,0.133333}%
\pgfsetstrokecolor{currentstroke}%
\pgfsetdash{}{0pt}%
\pgfpathmoveto{\pgfqpoint{0.754752in}{0.682494in}}%
\pgfpathlineto{\pgfqpoint{0.825546in}{0.755391in}}%
\pgfpathlineto{\pgfqpoint{0.896340in}{0.861337in}}%
\pgfpathlineto{\pgfqpoint{0.967134in}{0.938330in}}%
\pgfpathlineto{\pgfqpoint{1.037927in}{1.013620in}}%
\pgfpathlineto{\pgfqpoint{1.108721in}{1.049452in}}%
\pgfpathlineto{\pgfqpoint{1.179515in}{1.170718in}}%
\pgfpathlineto{\pgfqpoint{1.250309in}{1.199895in}}%
\pgfpathlineto{\pgfqpoint{1.321102in}{1.305506in}}%
\pgfpathlineto{\pgfqpoint{1.391896in}{1.345067in}}%
\pgfpathlineto{\pgfqpoint{1.462690in}{1.362732in}}%
\pgfpathlineto{\pgfqpoint{1.533483in}{1.400303in}}%
\pgfpathlineto{\pgfqpoint{1.604277in}{1.508166in}}%
\pgfpathlineto{\pgfqpoint{1.675071in}{1.515632in}}%
\pgfpathlineto{\pgfqpoint{1.745865in}{1.548276in}}%
\pgfpathlineto{\pgfqpoint{1.816658in}{1.592300in}}%
\pgfpathlineto{\pgfqpoint{1.887452in}{1.643655in}}%
\pgfpathlineto{\pgfqpoint{1.958246in}{1.699508in}}%
\pgfpathlineto{\pgfqpoint{2.029039in}{1.770891in}}%
\pgfpathlineto{\pgfqpoint{2.099833in}{1.778267in}}%
\pgfpathlineto{\pgfqpoint{2.170627in}{1.832319in}}%
\pgfpathlineto{\pgfqpoint{2.241421in}{1.863127in}}%
\pgfpathlineto{\pgfqpoint{2.312214in}{1.912339in}}%
\pgfpathlineto{\pgfqpoint{2.383008in}{1.946986in}}%
\pgfpathlineto{\pgfqpoint{2.453802in}{1.949834in}}%
\pgfpathlineto{\pgfqpoint{2.524596in}{2.012232in}}%
\pgfpathlineto{\pgfqpoint{2.595389in}{2.051207in}}%
\pgfpathlineto{\pgfqpoint{2.666183in}{2.044410in}}%
\pgfpathlineto{\pgfqpoint{2.736977in}{2.127781in}}%
\pgfpathlineto{\pgfqpoint{2.807770in}{2.152994in}}%
\pgfpathlineto{\pgfqpoint{2.878564in}{2.209999in}}%
\pgfpathlineto{\pgfqpoint{2.949358in}{2.184371in}}%
\pgfpathlineto{\pgfqpoint{3.020152in}{2.225649in}}%
\pgfpathlineto{\pgfqpoint{3.090945in}{2.238783in}}%
\pgfpathlineto{\pgfqpoint{3.161739in}{2.289120in}}%
\pgfpathlineto{\pgfqpoint{3.232533in}{2.286155in}}%
\pgfpathlineto{\pgfqpoint{3.303327in}{2.272731in}}%
\pgfpathlineto{\pgfqpoint{3.374120in}{2.322933in}}%
\pgfpathlineto{\pgfqpoint{3.444914in}{2.389743in}}%
\pgfpathlineto{\pgfqpoint{3.515708in}{2.388422in}}%
\pgfpathlineto{\pgfqpoint{3.586501in}{2.426040in}}%
\pgfpathlineto{\pgfqpoint{3.657295in}{2.514258in}}%
\pgfpathlineto{\pgfqpoint{3.728089in}{2.477233in}}%
\pgfpathlineto{\pgfqpoint{3.798883in}{2.528290in}}%
\pgfpathlineto{\pgfqpoint{3.869676in}{2.488253in}}%
\pgfpathlineto{\pgfqpoint{3.940470in}{2.568399in}}%
\pgfpathlineto{\pgfqpoint{4.011264in}{2.619309in}}%
\pgfpathlineto{\pgfqpoint{4.082057in}{2.594694in}}%
\pgfusepath{stroke}%
\end{pgfscope}%
\begin{pgfscope}%
\pgfsetrectcap%
\pgfsetmiterjoin%
\pgfsetlinewidth{0.803000pt}%
\definecolor{currentstroke}{rgb}{0.000000,0.000000,0.000000}%
\pgfsetstrokecolor{currentstroke}%
\pgfsetdash{}{0pt}%
\pgfpathmoveto{\pgfqpoint{0.588387in}{0.521603in}}%
\pgfpathlineto{\pgfqpoint{0.588387in}{2.741849in}}%
\pgfusepath{stroke}%
\end{pgfscope}%
\begin{pgfscope}%
\pgfsetrectcap%
\pgfsetmiterjoin%
\pgfsetlinewidth{0.803000pt}%
\definecolor{currentstroke}{rgb}{0.000000,0.000000,0.000000}%
\pgfsetstrokecolor{currentstroke}%
\pgfsetdash{}{0pt}%
\pgfpathmoveto{\pgfqpoint{4.248423in}{0.521603in}}%
\pgfpathlineto{\pgfqpoint{4.248423in}{2.741849in}}%
\pgfusepath{stroke}%
\end{pgfscope}%
\begin{pgfscope}%
\pgfsetrectcap%
\pgfsetmiterjoin%
\pgfsetlinewidth{0.803000pt}%
\definecolor{currentstroke}{rgb}{0.000000,0.000000,0.000000}%
\pgfsetstrokecolor{currentstroke}%
\pgfsetdash{}{0pt}%
\pgfpathmoveto{\pgfqpoint{0.588387in}{0.521603in}}%
\pgfpathlineto{\pgfqpoint{4.248423in}{0.521603in}}%
\pgfusepath{stroke}%
\end{pgfscope}%
\begin{pgfscope}%
\pgfsetrectcap%
\pgfsetmiterjoin%
\pgfsetlinewidth{0.803000pt}%
\definecolor{currentstroke}{rgb}{0.000000,0.000000,0.000000}%
\pgfsetstrokecolor{currentstroke}%
\pgfsetdash{}{0pt}%
\pgfpathmoveto{\pgfqpoint{0.588387in}{2.741849in}}%
\pgfpathlineto{\pgfqpoint{4.248423in}{2.741849in}}%
\pgfusepath{stroke}%
\end{pgfscope}%
\begin{pgfscope}%
\pgfsetbuttcap%
\pgfsetmiterjoin%
\definecolor{currentfill}{rgb}{1.000000,1.000000,1.000000}%
\pgfsetfillcolor{currentfill}%
\pgfsetfillopacity{0.800000}%
\pgfsetlinewidth{1.003750pt}%
\definecolor{currentstroke}{rgb}{0.800000,0.800000,0.800000}%
\pgfsetstrokecolor{currentstroke}%
\pgfsetstrokeopacity{0.800000}%
\pgfsetdash{}{0pt}%
\pgfpathmoveto{\pgfqpoint{4.365089in}{0.379025in}}%
\pgfpathlineto{\pgfqpoint{8.251043in}{0.379025in}}%
\pgfpathquadraticcurveto{\pgfqpoint{8.284376in}{0.379025in}}{\pgfqpoint{8.284376in}{0.412359in}}%
\pgfpathlineto{\pgfqpoint{8.284376in}{2.625183in}}%
\pgfpathquadraticcurveto{\pgfqpoint{8.284376in}{2.658516in}}{\pgfqpoint{8.251043in}{2.658516in}}%
\pgfpathlineto{\pgfqpoint{4.365089in}{2.658516in}}%
\pgfpathquadraticcurveto{\pgfqpoint{4.331756in}{2.658516in}}{\pgfqpoint{4.331756in}{2.625183in}}%
\pgfpathlineto{\pgfqpoint{4.331756in}{0.412359in}}%
\pgfpathquadraticcurveto{\pgfqpoint{4.331756in}{0.379025in}}{\pgfqpoint{4.365089in}{0.379025in}}%
\pgfpathlineto{\pgfqpoint{4.365089in}{0.379025in}}%
\pgfpathclose%
\pgfusepath{stroke,fill}%
\end{pgfscope}%
\begin{pgfscope}%
\pgfsetrectcap%
\pgfsetroundjoin%
\pgfsetlinewidth{1.505625pt}%
\pgfsetstrokecolor{currentstroke3}%
\pgfsetdash{}{0pt}%
\pgfpathmoveto{\pgfqpoint{4.398423in}{2.523555in}}%
\pgfpathlineto{\pgfqpoint{4.565089in}{2.523555in}}%
\pgfpathlineto{\pgfqpoint{4.731756in}{2.523555in}}%
\pgfusepath{stroke}%
\end{pgfscope}%
\begin{pgfscope}%
\definecolor{textcolor}{rgb}{0.000000,0.000000,0.000000}%
\pgfsetstrokecolor{textcolor}%
\pgfsetfillcolor{textcolor}%
\pgftext[x=4.865089in,y=2.465222in,left,base]{\color{textcolor}{\rmfamily\fontsize{12.000000}{14.400000}\selectfont\catcode`\^=\active\def^{\ifmmode\sp\else\^{}\fi}\catcode`\%=\active\def%{\%}\NaiveCycles{}}}%
\end{pgfscope}%
\begin{pgfscope}%
\pgfsetrectcap%
\pgfsetroundjoin%
\pgfsetlinewidth{1.505625pt}%
\pgfsetstrokecolor{currentstroke1}%
\pgfsetdash{}{0pt}%
\pgfpathmoveto{\pgfqpoint{4.398423in}{2.278926in}}%
\pgfpathlineto{\pgfqpoint{4.565089in}{2.278926in}}%
\pgfpathlineto{\pgfqpoint{4.731756in}{2.278926in}}%
\pgfusepath{stroke}%
\end{pgfscope}%
\begin{pgfscope}%
\definecolor{textcolor}{rgb}{0.000000,0.000000,0.000000}%
\pgfsetstrokecolor{textcolor}%
\pgfsetfillcolor{textcolor}%
\pgftext[x=4.865089in,y=2.220593in,left,base]{\color{textcolor}{\rmfamily\fontsize{12.000000}{14.400000}\selectfont\catcode`\^=\active\def^{\ifmmode\sp\else\^{}\fi}\catcode`\%=\active\def%{\%}\CyclesMatchChunks{} \& \MergeLinear{}}}%
\end{pgfscope}%
\begin{pgfscope}%
\pgfsetrectcap%
\pgfsetroundjoin%
\pgfsetlinewidth{1.505625pt}%
\pgfsetstrokecolor{currentstroke2}%
\pgfsetdash{}{0pt}%
\pgfpathmoveto{\pgfqpoint{4.398423in}{2.029659in}}%
\pgfpathlineto{\pgfqpoint{4.565089in}{2.029659in}}%
\pgfpathlineto{\pgfqpoint{4.731756in}{2.029659in}}%
\pgfusepath{stroke}%
\end{pgfscope}%
\begin{pgfscope}%
\definecolor{textcolor}{rgb}{0.000000,0.000000,0.000000}%
\pgfsetstrokecolor{textcolor}%
\pgfsetfillcolor{textcolor}%
\pgftext[x=4.865089in,y=1.971325in,left,base]{\color{textcolor}{\rmfamily\fontsize{12.000000}{14.400000}\selectfont\catcode`\^=\active\def^{\ifmmode\sp\else\^{}\fi}\catcode`\%=\active\def%{\%}\CyclesMatchChunks{} \& \SharedVertices{}}}%
\end{pgfscope}%
\begin{pgfscope}%
\pgfsetrectcap%
\pgfsetroundjoin%
\pgfsetlinewidth{1.505625pt}%
\pgfsetstrokecolor{currentstroke4}%
\pgfsetdash{}{0pt}%
\pgfpathmoveto{\pgfqpoint{4.398423in}{1.780391in}}%
\pgfpathlineto{\pgfqpoint{4.565089in}{1.780391in}}%
\pgfpathlineto{\pgfqpoint{4.731756in}{1.780391in}}%
\pgfusepath{stroke}%
\end{pgfscope}%
\begin{pgfscope}%
\definecolor{textcolor}{rgb}{0.000000,0.000000,0.000000}%
\pgfsetstrokecolor{textcolor}%
\pgfsetfillcolor{textcolor}%
\pgftext[x=4.865089in,y=1.722058in,left,base]{\color{textcolor}{\rmfamily\fontsize{12.000000}{14.400000}\selectfont\catcode`\^=\active\def^{\ifmmode\sp\else\^{}\fi}\catcode`\%=\active\def%{\%}\Neighbors{} \& \MergeLinear{}}}%
\end{pgfscope}%
\begin{pgfscope}%
\pgfsetrectcap%
\pgfsetroundjoin%
\pgfsetlinewidth{1.505625pt}%
\pgfsetstrokecolor{currentstroke5}%
\pgfsetdash{}{0pt}%
\pgfpathmoveto{\pgfqpoint{4.398423in}{1.535763in}}%
\pgfpathlineto{\pgfqpoint{4.565089in}{1.535763in}}%
\pgfpathlineto{\pgfqpoint{4.731756in}{1.535763in}}%
\pgfusepath{stroke}%
\end{pgfscope}%
\begin{pgfscope}%
\definecolor{textcolor}{rgb}{0.000000,0.000000,0.000000}%
\pgfsetstrokecolor{textcolor}%
\pgfsetfillcolor{textcolor}%
\pgftext[x=4.865089in,y=1.477429in,left,base]{\color{textcolor}{\rmfamily\fontsize{12.000000}{14.400000}\selectfont\catcode`\^=\active\def^{\ifmmode\sp\else\^{}\fi}\catcode`\%=\active\def%{\%}\Neighbors{} \& \SharedVertices{}}}%
\end{pgfscope}%
\begin{pgfscope}%
\pgfsetrectcap%
\pgfsetroundjoin%
\pgfsetlinewidth{1.505625pt}%
\pgfsetstrokecolor{currentstroke6}%
\pgfsetdash{}{0pt}%
\pgfpathmoveto{\pgfqpoint{4.398423in}{1.286495in}}%
\pgfpathlineto{\pgfqpoint{4.565089in}{1.286495in}}%
\pgfpathlineto{\pgfqpoint{4.731756in}{1.286495in}}%
\pgfusepath{stroke}%
\end{pgfscope}%
\begin{pgfscope}%
\definecolor{textcolor}{rgb}{0.000000,0.000000,0.000000}%
\pgfsetstrokecolor{textcolor}%
\pgfsetfillcolor{textcolor}%
\pgftext[x=4.865089in,y=1.228162in,left,base]{\color{textcolor}{\rmfamily\fontsize{12.000000}{14.400000}\selectfont\catcode`\^=\active\def^{\ifmmode\sp\else\^{}\fi}\catcode`\%=\active\def%{\%}\NeighborsDegree{} \& \MergeLinear{}}}%
\end{pgfscope}%
\begin{pgfscope}%
\pgfsetrectcap%
\pgfsetroundjoin%
\pgfsetlinewidth{1.505625pt}%
\pgfsetstrokecolor{currentstroke7}%
\pgfsetdash{}{0pt}%
\pgfpathmoveto{\pgfqpoint{4.398423in}{1.037228in}}%
\pgfpathlineto{\pgfqpoint{4.565089in}{1.037228in}}%
\pgfpathlineto{\pgfqpoint{4.731756in}{1.037228in}}%
\pgfusepath{stroke}%
\end{pgfscope}%
\begin{pgfscope}%
\definecolor{textcolor}{rgb}{0.000000,0.000000,0.000000}%
\pgfsetstrokecolor{textcolor}%
\pgfsetfillcolor{textcolor}%
\pgftext[x=4.865089in,y=0.978895in,left,base]{\color{textcolor}{\rmfamily\fontsize{12.000000}{14.400000}\selectfont\catcode`\^=\active\def^{\ifmmode\sp\else\^{}\fi}\catcode`\%=\active\def%{\%}\NeighborsDegree{} \& \SharedVertices{}}}%
\end{pgfscope}%
\begin{pgfscope}%
\pgfsetrectcap%
\pgfsetroundjoin%
\pgfsetlinewidth{1.505625pt}%
\definecolor{currentstroke}{rgb}{0.498039,0.498039,0.498039}%
\pgfsetstrokecolor{currentstroke}%
\pgfsetdash{}{0pt}%
\pgfpathmoveto{\pgfqpoint{4.398423in}{0.787961in}}%
\pgfpathlineto{\pgfqpoint{4.565089in}{0.787961in}}%
\pgfpathlineto{\pgfqpoint{4.731756in}{0.787961in}}%
\pgfusepath{stroke}%
\end{pgfscope}%
\begin{pgfscope}%
\definecolor{textcolor}{rgb}{0.000000,0.000000,0.000000}%
\pgfsetstrokecolor{textcolor}%
\pgfsetfillcolor{textcolor}%
\pgftext[x=4.865089in,y=0.729627in,left,base]{\color{textcolor}{\rmfamily\fontsize{12.000000}{14.400000}\selectfont\catcode`\^=\active\def^{\ifmmode\sp\else\^{}\fi}\catcode`\%=\active\def%{\%}\None{} \& \MergeLinear{}}}%
\end{pgfscope}%
\begin{pgfscope}%
\pgfsetrectcap%
\pgfsetroundjoin%
\pgfsetlinewidth{1.505625pt}%
\definecolor{currentstroke}{rgb}{0.737255,0.741176,0.133333}%
\pgfsetstrokecolor{currentstroke}%
\pgfsetdash{}{0pt}%
\pgfpathmoveto{\pgfqpoint{4.398423in}{0.543332in}}%
\pgfpathlineto{\pgfqpoint{4.565089in}{0.543332in}}%
\pgfpathlineto{\pgfqpoint{4.731756in}{0.543332in}}%
\pgfusepath{stroke}%
\end{pgfscope}%
\begin{pgfscope}%
\definecolor{textcolor}{rgb}{0.000000,0.000000,0.000000}%
\pgfsetstrokecolor{textcolor}%
\pgfsetfillcolor{textcolor}%
\pgftext[x=4.865089in,y=0.484999in,left,base]{\color{textcolor}{\rmfamily\fontsize{12.000000}{14.400000}\selectfont\catcode`\^=\active\def^{\ifmmode\sp\else\^{}\fi}\catcode`\%=\active\def%{\%}\None{} \& \SharedVertices{}}}%
\end{pgfscope}%
\end{pgfpicture}%
\makeatother%
\endgroup%
}
	\caption[Mean runtime for globally rigid graphs (some).]{
		Mean running time (ms) to find all NAC-colorings for globally rigid graphs.}%
	\label{fig:graph_globally_rigid_first_runtime}
\end{figure}
% \begin{figure}[p]
% 	\centering
% 	\scalebox{0.5}{%% Creator: Matplotlib, PGF backend
%%
%% To include the figure in your LaTeX document, write
%%   \input{<filename>.pgf}
%%
%% Make sure the required packages are loaded in your preamble
%%   \usepackage{pgf}
%%
%% Also ensure that all the required font packages are loaded; for instance,
%% the lmodern package is sometimes necessary when using math font.
%%   \usepackage{lmodern}
%%
%% Figures using additional raster images can only be included by \input if
%% they are in the same directory as the main LaTeX file. For loading figures
%% from other directories you can use the `import` package
%%   \usepackage{import}
%%
%% and then include the figures with
%%   \import{<path to file>}{<filename>.pgf}
%%
%% Matplotlib used the following preamble
%%   \def\mathdefault#1{#1}
%%   \everymath=\expandafter{\the\everymath\displaystyle}
%%   \IfFileExists{scrextend.sty}{
%%     \usepackage[fontsize=10.000000pt]{scrextend}
%%   }{
%%     \renewcommand{\normalsize}{\fontsize{10.000000}{12.000000}\selectfont}
%%     \normalsize
%%   }
%%   
%%   \ifdefined\pdftexversion\else  % non-pdftex case.
%%     \usepackage{fontspec}
%%     \setmainfont{DejaVuSans.ttf}[Path=\detokenize{/home/petr/Projects/PyRigi/.venv/lib/python3.12/site-packages/matplotlib/mpl-data/fonts/ttf/}]
%%     \setsansfont{DejaVuSans.ttf}[Path=\detokenize{/home/petr/Projects/PyRigi/.venv/lib/python3.12/site-packages/matplotlib/mpl-data/fonts/ttf/}]
%%     \setmonofont{DejaVuSansMono.ttf}[Path=\detokenize{/home/petr/Projects/PyRigi/.venv/lib/python3.12/site-packages/matplotlib/mpl-data/fonts/ttf/}]
%%   \fi
%%   \makeatletter\@ifpackageloaded{under\Score{}}{}{\usepackage[strings]{under\Score{}}}\makeatother
%%
\begingroup%
\makeatletter%
\begin{pgfpicture}%
\pgfpathrectangle{\pgfpointorigin}{\pgfqpoint{8.384376in}{2.841849in}}%
\pgfusepath{use as bounding box, clip}%
\begin{pgfscope}%
\pgfsetbuttcap%
\pgfsetmiterjoin%
\definecolor{currentfill}{rgb}{1.000000,1.000000,1.000000}%
\pgfsetfillcolor{currentfill}%
\pgfsetlinewidth{0.000000pt}%
\definecolor{currentstroke}{rgb}{1.000000,1.000000,1.000000}%
\pgfsetstrokecolor{currentstroke}%
\pgfsetdash{}{0pt}%
\pgfpathmoveto{\pgfqpoint{0.000000in}{0.000000in}}%
\pgfpathlineto{\pgfqpoint{8.384376in}{0.000000in}}%
\pgfpathlineto{\pgfqpoint{8.384376in}{2.841849in}}%
\pgfpathlineto{\pgfqpoint{0.000000in}{2.841849in}}%
\pgfpathlineto{\pgfqpoint{0.000000in}{0.000000in}}%
\pgfpathclose%
\pgfusepath{fill}%
\end{pgfscope}%
\begin{pgfscope}%
\pgfsetbuttcap%
\pgfsetmiterjoin%
\definecolor{currentfill}{rgb}{1.000000,1.000000,1.000000}%
\pgfsetfillcolor{currentfill}%
\pgfsetlinewidth{0.000000pt}%
\definecolor{currentstroke}{rgb}{0.000000,0.000000,0.000000}%
\pgfsetstrokecolor{currentstroke}%
\pgfsetstrokeopacity{0.000000}%
\pgfsetdash{}{0pt}%
\pgfpathmoveto{\pgfqpoint{0.588387in}{0.521603in}}%
\pgfpathlineto{\pgfqpoint{4.248423in}{0.521603in}}%
\pgfpathlineto{\pgfqpoint{4.248423in}{2.741849in}}%
\pgfpathlineto{\pgfqpoint{0.588387in}{2.741849in}}%
\pgfpathlineto{\pgfqpoint{0.588387in}{0.521603in}}%
\pgfpathclose%
\pgfusepath{fill}%
\end{pgfscope}%
\begin{pgfscope}%
\pgfsetbuttcap%
\pgfsetroundjoin%
\definecolor{currentfill}{rgb}{0.000000,0.000000,0.000000}%
\pgfsetfillcolor{currentfill}%
\pgfsetlinewidth{0.803000pt}%
\definecolor{currentstroke}{rgb}{0.000000,0.000000,0.000000}%
\pgfsetstrokecolor{currentstroke}%
\pgfsetdash{}{0pt}%
\pgfsys@defobject{currentmarker}{\pgfqpoint{0.000000in}{-0.048611in}}{\pgfqpoint{0.000000in}{0.000000in}}{%
\pgfpathmoveto{\pgfqpoint{0.000000in}{0.000000in}}%
\pgfpathlineto{\pgfqpoint{0.000000in}{-0.048611in}}%
\pgfusepath{stroke,fill}%
}%
\begin{pgfscope}%
\pgfsys@transformshift{0.655430in}{0.521603in}%
\pgfsys@useobject{currentmarker}{}%
\end{pgfscope}%
\end{pgfscope}%
\begin{pgfscope}%
\definecolor{textcolor}{rgb}{0.000000,0.000000,0.000000}%
\pgfsetstrokecolor{textcolor}%
\pgfsetfillcolor{textcolor}%
\pgftext[x=0.655430in,y=0.424381in,,top]{\color{textcolor}{\rmfamily\fontsize{10.000000}{12.000000}\selectfont\catcode`\^=\active\def^{\ifmmode\sp\else\^{}\fi}\catcode`\%=\active\def%{\%}$\mathdefault{0}$}}%
\end{pgfscope}%
\begin{pgfscope}%
\pgfsetbuttcap%
\pgfsetroundjoin%
\definecolor{currentfill}{rgb}{0.000000,0.000000,0.000000}%
\pgfsetfillcolor{currentfill}%
\pgfsetlinewidth{0.803000pt}%
\definecolor{currentstroke}{rgb}{0.000000,0.000000,0.000000}%
\pgfsetstrokecolor{currentstroke}%
\pgfsetdash{}{0pt}%
\pgfsys@defobject{currentmarker}{\pgfqpoint{0.000000in}{-0.048611in}}{\pgfqpoint{0.000000in}{0.000000in}}{%
\pgfpathmoveto{\pgfqpoint{0.000000in}{0.000000in}}%
\pgfpathlineto{\pgfqpoint{0.000000in}{-0.048611in}}%
\pgfusepath{stroke,fill}%
}%
\begin{pgfscope}%
\pgfsys@transformshift{1.052720in}{0.521603in}%
\pgfsys@useobject{currentmarker}{}%
\end{pgfscope}%
\end{pgfscope}%
\begin{pgfscope}%
\definecolor{textcolor}{rgb}{0.000000,0.000000,0.000000}%
\pgfsetstrokecolor{textcolor}%
\pgfsetfillcolor{textcolor}%
\pgftext[x=1.052720in,y=0.424381in,,top]{\color{textcolor}{\rmfamily\fontsize{10.000000}{12.000000}\selectfont\catcode`\^=\active\def^{\ifmmode\sp\else\^{}\fi}\catcode`\%=\active\def%{\%}$\mathdefault{8}$}}%
\end{pgfscope}%
\begin{pgfscope}%
\pgfsetbuttcap%
\pgfsetroundjoin%
\definecolor{currentfill}{rgb}{0.000000,0.000000,0.000000}%
\pgfsetfillcolor{currentfill}%
\pgfsetlinewidth{0.803000pt}%
\definecolor{currentstroke}{rgb}{0.000000,0.000000,0.000000}%
\pgfsetstrokecolor{currentstroke}%
\pgfsetdash{}{0pt}%
\pgfsys@defobject{currentmarker}{\pgfqpoint{0.000000in}{-0.048611in}}{\pgfqpoint{0.000000in}{0.000000in}}{%
\pgfpathmoveto{\pgfqpoint{0.000000in}{0.000000in}}%
\pgfpathlineto{\pgfqpoint{0.000000in}{-0.048611in}}%
\pgfusepath{stroke,fill}%
}%
\begin{pgfscope}%
\pgfsys@transformshift{1.450010in}{0.521603in}%
\pgfsys@useobject{currentmarker}{}%
\end{pgfscope}%
\end{pgfscope}%
\begin{pgfscope}%
\definecolor{textcolor}{rgb}{0.000000,0.000000,0.000000}%
\pgfsetstrokecolor{textcolor}%
\pgfsetfillcolor{textcolor}%
\pgftext[x=1.450010in,y=0.424381in,,top]{\color{textcolor}{\rmfamily\fontsize{10.000000}{12.000000}\selectfont\catcode`\^=\active\def^{\ifmmode\sp\else\^{}\fi}\catcode`\%=\active\def%{\%}$\mathdefault{16}$}}%
\end{pgfscope}%
\begin{pgfscope}%
\pgfsetbuttcap%
\pgfsetroundjoin%
\definecolor{currentfill}{rgb}{0.000000,0.000000,0.000000}%
\pgfsetfillcolor{currentfill}%
\pgfsetlinewidth{0.803000pt}%
\definecolor{currentstroke}{rgb}{0.000000,0.000000,0.000000}%
\pgfsetstrokecolor{currentstroke}%
\pgfsetdash{}{0pt}%
\pgfsys@defobject{currentmarker}{\pgfqpoint{0.000000in}{-0.048611in}}{\pgfqpoint{0.000000in}{0.000000in}}{%
\pgfpathmoveto{\pgfqpoint{0.000000in}{0.000000in}}%
\pgfpathlineto{\pgfqpoint{0.000000in}{-0.048611in}}%
\pgfusepath{stroke,fill}%
}%
\begin{pgfscope}%
\pgfsys@transformshift{1.847300in}{0.521603in}%
\pgfsys@useobject{currentmarker}{}%
\end{pgfscope}%
\end{pgfscope}%
\begin{pgfscope}%
\definecolor{textcolor}{rgb}{0.000000,0.000000,0.000000}%
\pgfsetstrokecolor{textcolor}%
\pgfsetfillcolor{textcolor}%
\pgftext[x=1.847300in,y=0.424381in,,top]{\color{textcolor}{\rmfamily\fontsize{10.000000}{12.000000}\selectfont\catcode`\^=\active\def^{\ifmmode\sp\else\^{}\fi}\catcode`\%=\active\def%{\%}$\mathdefault{24}$}}%
\end{pgfscope}%
\begin{pgfscope}%
\pgfsetbuttcap%
\pgfsetroundjoin%
\definecolor{currentfill}{rgb}{0.000000,0.000000,0.000000}%
\pgfsetfillcolor{currentfill}%
\pgfsetlinewidth{0.803000pt}%
\definecolor{currentstroke}{rgb}{0.000000,0.000000,0.000000}%
\pgfsetstrokecolor{currentstroke}%
\pgfsetdash{}{0pt}%
\pgfsys@defobject{currentmarker}{\pgfqpoint{0.000000in}{-0.048611in}}{\pgfqpoint{0.000000in}{0.000000in}}{%
\pgfpathmoveto{\pgfqpoint{0.000000in}{0.000000in}}%
\pgfpathlineto{\pgfqpoint{0.000000in}{-0.048611in}}%
\pgfusepath{stroke,fill}%
}%
\begin{pgfscope}%
\pgfsys@transformshift{2.244591in}{0.521603in}%
\pgfsys@useobject{currentmarker}{}%
\end{pgfscope}%
\end{pgfscope}%
\begin{pgfscope}%
\definecolor{textcolor}{rgb}{0.000000,0.000000,0.000000}%
\pgfsetstrokecolor{textcolor}%
\pgfsetfillcolor{textcolor}%
\pgftext[x=2.244591in,y=0.424381in,,top]{\color{textcolor}{\rmfamily\fontsize{10.000000}{12.000000}\selectfont\catcode`\^=\active\def^{\ifmmode\sp\else\^{}\fi}\catcode`\%=\active\def%{\%}$\mathdefault{32}$}}%
\end{pgfscope}%
\begin{pgfscope}%
\pgfsetbuttcap%
\pgfsetroundjoin%
\definecolor{currentfill}{rgb}{0.000000,0.000000,0.000000}%
\pgfsetfillcolor{currentfill}%
\pgfsetlinewidth{0.803000pt}%
\definecolor{currentstroke}{rgb}{0.000000,0.000000,0.000000}%
\pgfsetstrokecolor{currentstroke}%
\pgfsetdash{}{0pt}%
\pgfsys@defobject{currentmarker}{\pgfqpoint{0.000000in}{-0.048611in}}{\pgfqpoint{0.000000in}{0.000000in}}{%
\pgfpathmoveto{\pgfqpoint{0.000000in}{0.000000in}}%
\pgfpathlineto{\pgfqpoint{0.000000in}{-0.048611in}}%
\pgfusepath{stroke,fill}%
}%
\begin{pgfscope}%
\pgfsys@transformshift{2.641881in}{0.521603in}%
\pgfsys@useobject{currentmarker}{}%
\end{pgfscope}%
\end{pgfscope}%
\begin{pgfscope}%
\definecolor{textcolor}{rgb}{0.000000,0.000000,0.000000}%
\pgfsetstrokecolor{textcolor}%
\pgfsetfillcolor{textcolor}%
\pgftext[x=2.641881in,y=0.424381in,,top]{\color{textcolor}{\rmfamily\fontsize{10.000000}{12.000000}\selectfont\catcode`\^=\active\def^{\ifmmode\sp\else\^{}\fi}\catcode`\%=\active\def%{\%}$\mathdefault{40}$}}%
\end{pgfscope}%
\begin{pgfscope}%
\pgfsetbuttcap%
\pgfsetroundjoin%
\definecolor{currentfill}{rgb}{0.000000,0.000000,0.000000}%
\pgfsetfillcolor{currentfill}%
\pgfsetlinewidth{0.803000pt}%
\definecolor{currentstroke}{rgb}{0.000000,0.000000,0.000000}%
\pgfsetstrokecolor{currentstroke}%
\pgfsetdash{}{0pt}%
\pgfsys@defobject{currentmarker}{\pgfqpoint{0.000000in}{-0.048611in}}{\pgfqpoint{0.000000in}{0.000000in}}{%
\pgfpathmoveto{\pgfqpoint{0.000000in}{0.000000in}}%
\pgfpathlineto{\pgfqpoint{0.000000in}{-0.048611in}}%
\pgfusepath{stroke,fill}%
}%
\begin{pgfscope}%
\pgfsys@transformshift{3.039171in}{0.521603in}%
\pgfsys@useobject{currentmarker}{}%
\end{pgfscope}%
\end{pgfscope}%
\begin{pgfscope}%
\definecolor{textcolor}{rgb}{0.000000,0.000000,0.000000}%
\pgfsetstrokecolor{textcolor}%
\pgfsetfillcolor{textcolor}%
\pgftext[x=3.039171in,y=0.424381in,,top]{\color{textcolor}{\rmfamily\fontsize{10.000000}{12.000000}\selectfont\catcode`\^=\active\def^{\ifmmode\sp\else\^{}\fi}\catcode`\%=\active\def%{\%}$\mathdefault{48}$}}%
\end{pgfscope}%
\begin{pgfscope}%
\pgfsetbuttcap%
\pgfsetroundjoin%
\definecolor{currentfill}{rgb}{0.000000,0.000000,0.000000}%
\pgfsetfillcolor{currentfill}%
\pgfsetlinewidth{0.803000pt}%
\definecolor{currentstroke}{rgb}{0.000000,0.000000,0.000000}%
\pgfsetstrokecolor{currentstroke}%
\pgfsetdash{}{0pt}%
\pgfsys@defobject{currentmarker}{\pgfqpoint{0.000000in}{-0.048611in}}{\pgfqpoint{0.000000in}{0.000000in}}{%
\pgfpathmoveto{\pgfqpoint{0.000000in}{0.000000in}}%
\pgfpathlineto{\pgfqpoint{0.000000in}{-0.048611in}}%
\pgfusepath{stroke,fill}%
}%
\begin{pgfscope}%
\pgfsys@transformshift{3.436461in}{0.521603in}%
\pgfsys@useobject{currentmarker}{}%
\end{pgfscope}%
\end{pgfscope}%
\begin{pgfscope}%
\definecolor{textcolor}{rgb}{0.000000,0.000000,0.000000}%
\pgfsetstrokecolor{textcolor}%
\pgfsetfillcolor{textcolor}%
\pgftext[x=3.436461in,y=0.424381in,,top]{\color{textcolor}{\rmfamily\fontsize{10.000000}{12.000000}\selectfont\catcode`\^=\active\def^{\ifmmode\sp\else\^{}\fi}\catcode`\%=\active\def%{\%}$\mathdefault{56}$}}%
\end{pgfscope}%
\begin{pgfscope}%
\pgfsetbuttcap%
\pgfsetroundjoin%
\definecolor{currentfill}{rgb}{0.000000,0.000000,0.000000}%
\pgfsetfillcolor{currentfill}%
\pgfsetlinewidth{0.803000pt}%
\definecolor{currentstroke}{rgb}{0.000000,0.000000,0.000000}%
\pgfsetstrokecolor{currentstroke}%
\pgfsetdash{}{0pt}%
\pgfsys@defobject{currentmarker}{\pgfqpoint{0.000000in}{-0.048611in}}{\pgfqpoint{0.000000in}{0.000000in}}{%
\pgfpathmoveto{\pgfqpoint{0.000000in}{0.000000in}}%
\pgfpathlineto{\pgfqpoint{0.000000in}{-0.048611in}}%
\pgfusepath{stroke,fill}%
}%
\begin{pgfscope}%
\pgfsys@transformshift{3.833751in}{0.521603in}%
\pgfsys@useobject{currentmarker}{}%
\end{pgfscope}%
\end{pgfscope}%
\begin{pgfscope}%
\definecolor{textcolor}{rgb}{0.000000,0.000000,0.000000}%
\pgfsetstrokecolor{textcolor}%
\pgfsetfillcolor{textcolor}%
\pgftext[x=3.833751in,y=0.424381in,,top]{\color{textcolor}{\rmfamily\fontsize{10.000000}{12.000000}\selectfont\catcode`\^=\active\def^{\ifmmode\sp\else\^{}\fi}\catcode`\%=\active\def%{\%}$\mathdefault{64}$}}%
\end{pgfscope}%
\begin{pgfscope}%
\pgfsetbuttcap%
\pgfsetroundjoin%
\definecolor{currentfill}{rgb}{0.000000,0.000000,0.000000}%
\pgfsetfillcolor{currentfill}%
\pgfsetlinewidth{0.803000pt}%
\definecolor{currentstroke}{rgb}{0.000000,0.000000,0.000000}%
\pgfsetstrokecolor{currentstroke}%
\pgfsetdash{}{0pt}%
\pgfsys@defobject{currentmarker}{\pgfqpoint{0.000000in}{-0.048611in}}{\pgfqpoint{0.000000in}{0.000000in}}{%
\pgfpathmoveto{\pgfqpoint{0.000000in}{0.000000in}}%
\pgfpathlineto{\pgfqpoint{0.000000in}{-0.048611in}}%
\pgfusepath{stroke,fill}%
}%
\begin{pgfscope}%
\pgfsys@transformshift{4.231041in}{0.521603in}%
\pgfsys@useobject{currentmarker}{}%
\end{pgfscope}%
\end{pgfscope}%
\begin{pgfscope}%
\definecolor{textcolor}{rgb}{0.000000,0.000000,0.000000}%
\pgfsetstrokecolor{textcolor}%
\pgfsetfillcolor{textcolor}%
\pgftext[x=4.231041in,y=0.424381in,,top]{\color{textcolor}{\rmfamily\fontsize{10.000000}{12.000000}\selectfont\catcode`\^=\active\def^{\ifmmode\sp\else\^{}\fi}\catcode`\%=\active\def%{\%}$\mathdefault{72}$}}%
\end{pgfscope}%
\begin{pgfscope}%
\definecolor{textcolor}{rgb}{0.000000,0.000000,0.000000}%
\pgfsetstrokecolor{textcolor}%
\pgfsetfillcolor{textcolor}%
\pgftext[x=2.418405in,y=0.234413in,,top]{\color{textcolor}{\rmfamily\fontsize{10.000000}{12.000000}\selectfont\catcode`\^=\active\def^{\ifmmode\sp\else\^{}\fi}\catcode`\%=\active\def%{\%}Monochromatic classes}}%
\end{pgfscope}%
\begin{pgfscope}%
\pgfsetbuttcap%
\pgfsetroundjoin%
\definecolor{currentfill}{rgb}{0.000000,0.000000,0.000000}%
\pgfsetfillcolor{currentfill}%
\pgfsetlinewidth{0.803000pt}%
\definecolor{currentstroke}{rgb}{0.000000,0.000000,0.000000}%
\pgfsetstrokecolor{currentstroke}%
\pgfsetdash{}{0pt}%
\pgfsys@defobject{currentmarker}{\pgfqpoint{-0.048611in}{0.000000in}}{\pgfqpoint{-0.000000in}{0.000000in}}{%
\pgfpathmoveto{\pgfqpoint{-0.000000in}{0.000000in}}%
\pgfpathlineto{\pgfqpoint{-0.048611in}{0.000000in}}%
\pgfusepath{stroke,fill}%
}%
\begin{pgfscope}%
\pgfsys@transformshift{0.588387in}{0.622524in}%
\pgfsys@useobject{currentmarker}{}%
\end{pgfscope}%
\end{pgfscope}%
\begin{pgfscope}%
\definecolor{textcolor}{rgb}{0.000000,0.000000,0.000000}%
\pgfsetstrokecolor{textcolor}%
\pgfsetfillcolor{textcolor}%
\pgftext[x=0.289968in, y=0.569762in, left, base]{\color{textcolor}{\rmfamily\fontsize{10.000000}{12.000000}\selectfont\catcode`\^=\active\def^{\ifmmode\sp\else\^{}\fi}\catcode`\%=\active\def%{\%}$\mathdefault{10^{0}}$}}%
\end{pgfscope}%
\begin{pgfscope}%
\pgfsetbuttcap%
\pgfsetroundjoin%
\definecolor{currentfill}{rgb}{0.000000,0.000000,0.000000}%
\pgfsetfillcolor{currentfill}%
\pgfsetlinewidth{0.803000pt}%
\definecolor{currentstroke}{rgb}{0.000000,0.000000,0.000000}%
\pgfsetstrokecolor{currentstroke}%
\pgfsetdash{}{0pt}%
\pgfsys@defobject{currentmarker}{\pgfqpoint{-0.048611in}{0.000000in}}{\pgfqpoint{-0.000000in}{0.000000in}}{%
\pgfpathmoveto{\pgfqpoint{-0.000000in}{0.000000in}}%
\pgfpathlineto{\pgfqpoint{-0.048611in}{0.000000in}}%
\pgfusepath{stroke,fill}%
}%
\begin{pgfscope}%
\pgfsys@transformshift{0.588387in}{1.072296in}%
\pgfsys@useobject{currentmarker}{}%
\end{pgfscope}%
\end{pgfscope}%
\begin{pgfscope}%
\definecolor{textcolor}{rgb}{0.000000,0.000000,0.000000}%
\pgfsetstrokecolor{textcolor}%
\pgfsetfillcolor{textcolor}%
\pgftext[x=0.289968in, y=1.019534in, left, base]{\color{textcolor}{\rmfamily\fontsize{10.000000}{12.000000}\selectfont\catcode`\^=\active\def^{\ifmmode\sp\else\^{}\fi}\catcode`\%=\active\def%{\%}$\mathdefault{10^{1}}$}}%
\end{pgfscope}%
\begin{pgfscope}%
\pgfsetbuttcap%
\pgfsetroundjoin%
\definecolor{currentfill}{rgb}{0.000000,0.000000,0.000000}%
\pgfsetfillcolor{currentfill}%
\pgfsetlinewidth{0.803000pt}%
\definecolor{currentstroke}{rgb}{0.000000,0.000000,0.000000}%
\pgfsetstrokecolor{currentstroke}%
\pgfsetdash{}{0pt}%
\pgfsys@defobject{currentmarker}{\pgfqpoint{-0.048611in}{0.000000in}}{\pgfqpoint{-0.000000in}{0.000000in}}{%
\pgfpathmoveto{\pgfqpoint{-0.000000in}{0.000000in}}%
\pgfpathlineto{\pgfqpoint{-0.048611in}{0.000000in}}%
\pgfusepath{stroke,fill}%
}%
\begin{pgfscope}%
\pgfsys@transformshift{0.588387in}{1.522068in}%
\pgfsys@useobject{currentmarker}{}%
\end{pgfscope}%
\end{pgfscope}%
\begin{pgfscope}%
\definecolor{textcolor}{rgb}{0.000000,0.000000,0.000000}%
\pgfsetstrokecolor{textcolor}%
\pgfsetfillcolor{textcolor}%
\pgftext[x=0.289968in, y=1.469306in, left, base]{\color{textcolor}{\rmfamily\fontsize{10.000000}{12.000000}\selectfont\catcode`\^=\active\def^{\ifmmode\sp\else\^{}\fi}\catcode`\%=\active\def%{\%}$\mathdefault{10^{2}}$}}%
\end{pgfscope}%
\begin{pgfscope}%
\pgfsetbuttcap%
\pgfsetroundjoin%
\definecolor{currentfill}{rgb}{0.000000,0.000000,0.000000}%
\pgfsetfillcolor{currentfill}%
\pgfsetlinewidth{0.803000pt}%
\definecolor{currentstroke}{rgb}{0.000000,0.000000,0.000000}%
\pgfsetstrokecolor{currentstroke}%
\pgfsetdash{}{0pt}%
\pgfsys@defobject{currentmarker}{\pgfqpoint{-0.048611in}{0.000000in}}{\pgfqpoint{-0.000000in}{0.000000in}}{%
\pgfpathmoveto{\pgfqpoint{-0.000000in}{0.000000in}}%
\pgfpathlineto{\pgfqpoint{-0.048611in}{0.000000in}}%
\pgfusepath{stroke,fill}%
}%
\begin{pgfscope}%
\pgfsys@transformshift{0.588387in}{1.971840in}%
\pgfsys@useobject{currentmarker}{}%
\end{pgfscope}%
\end{pgfscope}%
\begin{pgfscope}%
\definecolor{textcolor}{rgb}{0.000000,0.000000,0.000000}%
\pgfsetstrokecolor{textcolor}%
\pgfsetfillcolor{textcolor}%
\pgftext[x=0.289968in, y=1.919078in, left, base]{\color{textcolor}{\rmfamily\fontsize{10.000000}{12.000000}\selectfont\catcode`\^=\active\def^{\ifmmode\sp\else\^{}\fi}\catcode`\%=\active\def%{\%}$\mathdefault{10^{3}}$}}%
\end{pgfscope}%
\begin{pgfscope}%
\pgfsetbuttcap%
\pgfsetroundjoin%
\definecolor{currentfill}{rgb}{0.000000,0.000000,0.000000}%
\pgfsetfillcolor{currentfill}%
\pgfsetlinewidth{0.803000pt}%
\definecolor{currentstroke}{rgb}{0.000000,0.000000,0.000000}%
\pgfsetstrokecolor{currentstroke}%
\pgfsetdash{}{0pt}%
\pgfsys@defobject{currentmarker}{\pgfqpoint{-0.048611in}{0.000000in}}{\pgfqpoint{-0.000000in}{0.000000in}}{%
\pgfpathmoveto{\pgfqpoint{-0.000000in}{0.000000in}}%
\pgfpathlineto{\pgfqpoint{-0.048611in}{0.000000in}}%
\pgfusepath{stroke,fill}%
}%
\begin{pgfscope}%
\pgfsys@transformshift{0.588387in}{2.421612in}%
\pgfsys@useobject{currentmarker}{}%
\end{pgfscope}%
\end{pgfscope}%
\begin{pgfscope}%
\definecolor{textcolor}{rgb}{0.000000,0.000000,0.000000}%
\pgfsetstrokecolor{textcolor}%
\pgfsetfillcolor{textcolor}%
\pgftext[x=0.289968in, y=2.368850in, left, base]{\color{textcolor}{\rmfamily\fontsize{10.000000}{12.000000}\selectfont\catcode`\^=\active\def^{\ifmmode\sp\else\^{}\fi}\catcode`\%=\active\def%{\%}$\mathdefault{10^{4}}$}}%
\end{pgfscope}%
\begin{pgfscope}%
\pgfsetbuttcap%
\pgfsetroundjoin%
\definecolor{currentfill}{rgb}{0.000000,0.000000,0.000000}%
\pgfsetfillcolor{currentfill}%
\pgfsetlinewidth{0.602250pt}%
\definecolor{currentstroke}{rgb}{0.000000,0.000000,0.000000}%
\pgfsetstrokecolor{currentstroke}%
\pgfsetdash{}{0pt}%
\pgfsys@defobject{currentmarker}{\pgfqpoint{-0.027778in}{0.000000in}}{\pgfqpoint{-0.000000in}{0.000000in}}{%
\pgfpathmoveto{\pgfqpoint{-0.000000in}{0.000000in}}%
\pgfpathlineto{\pgfqpoint{-0.027778in}{0.000000in}}%
\pgfusepath{stroke,fill}%
}%
\begin{pgfscope}%
\pgfsys@transformshift{0.588387in}{0.522742in}%
\pgfsys@useobject{currentmarker}{}%
\end{pgfscope}%
\end{pgfscope}%
\begin{pgfscope}%
\pgfsetbuttcap%
\pgfsetroundjoin%
\definecolor{currentfill}{rgb}{0.000000,0.000000,0.000000}%
\pgfsetfillcolor{currentfill}%
\pgfsetlinewidth{0.602250pt}%
\definecolor{currentstroke}{rgb}{0.000000,0.000000,0.000000}%
\pgfsetstrokecolor{currentstroke}%
\pgfsetdash{}{0pt}%
\pgfsys@defobject{currentmarker}{\pgfqpoint{-0.027778in}{0.000000in}}{\pgfqpoint{-0.000000in}{0.000000in}}{%
\pgfpathmoveto{\pgfqpoint{-0.000000in}{0.000000in}}%
\pgfpathlineto{\pgfqpoint{-0.027778in}{0.000000in}}%
\pgfusepath{stroke,fill}%
}%
\begin{pgfscope}%
\pgfsys@transformshift{0.588387in}{0.552853in}%
\pgfsys@useobject{currentmarker}{}%
\end{pgfscope}%
\end{pgfscope}%
\begin{pgfscope}%
\pgfsetbuttcap%
\pgfsetroundjoin%
\definecolor{currentfill}{rgb}{0.000000,0.000000,0.000000}%
\pgfsetfillcolor{currentfill}%
\pgfsetlinewidth{0.602250pt}%
\definecolor{currentstroke}{rgb}{0.000000,0.000000,0.000000}%
\pgfsetstrokecolor{currentstroke}%
\pgfsetdash{}{0pt}%
\pgfsys@defobject{currentmarker}{\pgfqpoint{-0.027778in}{0.000000in}}{\pgfqpoint{-0.000000in}{0.000000in}}{%
\pgfpathmoveto{\pgfqpoint{-0.000000in}{0.000000in}}%
\pgfpathlineto{\pgfqpoint{-0.027778in}{0.000000in}}%
\pgfusepath{stroke,fill}%
}%
\begin{pgfscope}%
\pgfsys@transformshift{0.588387in}{0.578936in}%
\pgfsys@useobject{currentmarker}{}%
\end{pgfscope}%
\end{pgfscope}%
\begin{pgfscope}%
\pgfsetbuttcap%
\pgfsetroundjoin%
\definecolor{currentfill}{rgb}{0.000000,0.000000,0.000000}%
\pgfsetfillcolor{currentfill}%
\pgfsetlinewidth{0.602250pt}%
\definecolor{currentstroke}{rgb}{0.000000,0.000000,0.000000}%
\pgfsetstrokecolor{currentstroke}%
\pgfsetdash{}{0pt}%
\pgfsys@defobject{currentmarker}{\pgfqpoint{-0.027778in}{0.000000in}}{\pgfqpoint{-0.000000in}{0.000000in}}{%
\pgfpathmoveto{\pgfqpoint{-0.000000in}{0.000000in}}%
\pgfpathlineto{\pgfqpoint{-0.027778in}{0.000000in}}%
\pgfusepath{stroke,fill}%
}%
\begin{pgfscope}%
\pgfsys@transformshift{0.588387in}{0.601943in}%
\pgfsys@useobject{currentmarker}{}%
\end{pgfscope}%
\end{pgfscope}%
\begin{pgfscope}%
\pgfsetbuttcap%
\pgfsetroundjoin%
\definecolor{currentfill}{rgb}{0.000000,0.000000,0.000000}%
\pgfsetfillcolor{currentfill}%
\pgfsetlinewidth{0.602250pt}%
\definecolor{currentstroke}{rgb}{0.000000,0.000000,0.000000}%
\pgfsetstrokecolor{currentstroke}%
\pgfsetdash{}{0pt}%
\pgfsys@defobject{currentmarker}{\pgfqpoint{-0.027778in}{0.000000in}}{\pgfqpoint{-0.000000in}{0.000000in}}{%
\pgfpathmoveto{\pgfqpoint{-0.000000in}{0.000000in}}%
\pgfpathlineto{\pgfqpoint{-0.027778in}{0.000000in}}%
\pgfusepath{stroke,fill}%
}%
\begin{pgfscope}%
\pgfsys@transformshift{0.588387in}{0.757918in}%
\pgfsys@useobject{currentmarker}{}%
\end{pgfscope}%
\end{pgfscope}%
\begin{pgfscope}%
\pgfsetbuttcap%
\pgfsetroundjoin%
\definecolor{currentfill}{rgb}{0.000000,0.000000,0.000000}%
\pgfsetfillcolor{currentfill}%
\pgfsetlinewidth{0.602250pt}%
\definecolor{currentstroke}{rgb}{0.000000,0.000000,0.000000}%
\pgfsetstrokecolor{currentstroke}%
\pgfsetdash{}{0pt}%
\pgfsys@defobject{currentmarker}{\pgfqpoint{-0.027778in}{0.000000in}}{\pgfqpoint{-0.000000in}{0.000000in}}{%
\pgfpathmoveto{\pgfqpoint{-0.000000in}{0.000000in}}%
\pgfpathlineto{\pgfqpoint{-0.027778in}{0.000000in}}%
\pgfusepath{stroke,fill}%
}%
\begin{pgfscope}%
\pgfsys@transformshift{0.588387in}{0.837119in}%
\pgfsys@useobject{currentmarker}{}%
\end{pgfscope}%
\end{pgfscope}%
\begin{pgfscope}%
\pgfsetbuttcap%
\pgfsetroundjoin%
\definecolor{currentfill}{rgb}{0.000000,0.000000,0.000000}%
\pgfsetfillcolor{currentfill}%
\pgfsetlinewidth{0.602250pt}%
\definecolor{currentstroke}{rgb}{0.000000,0.000000,0.000000}%
\pgfsetstrokecolor{currentstroke}%
\pgfsetdash{}{0pt}%
\pgfsys@defobject{currentmarker}{\pgfqpoint{-0.027778in}{0.000000in}}{\pgfqpoint{-0.000000in}{0.000000in}}{%
\pgfpathmoveto{\pgfqpoint{-0.000000in}{0.000000in}}%
\pgfpathlineto{\pgfqpoint{-0.027778in}{0.000000in}}%
\pgfusepath{stroke,fill}%
}%
\begin{pgfscope}%
\pgfsys@transformshift{0.588387in}{0.893313in}%
\pgfsys@useobject{currentmarker}{}%
\end{pgfscope}%
\end{pgfscope}%
\begin{pgfscope}%
\pgfsetbuttcap%
\pgfsetroundjoin%
\definecolor{currentfill}{rgb}{0.000000,0.000000,0.000000}%
\pgfsetfillcolor{currentfill}%
\pgfsetlinewidth{0.602250pt}%
\definecolor{currentstroke}{rgb}{0.000000,0.000000,0.000000}%
\pgfsetstrokecolor{currentstroke}%
\pgfsetdash{}{0pt}%
\pgfsys@defobject{currentmarker}{\pgfqpoint{-0.027778in}{0.000000in}}{\pgfqpoint{-0.000000in}{0.000000in}}{%
\pgfpathmoveto{\pgfqpoint{-0.000000in}{0.000000in}}%
\pgfpathlineto{\pgfqpoint{-0.027778in}{0.000000in}}%
\pgfusepath{stroke,fill}%
}%
\begin{pgfscope}%
\pgfsys@transformshift{0.588387in}{0.936901in}%
\pgfsys@useobject{currentmarker}{}%
\end{pgfscope}%
\end{pgfscope}%
\begin{pgfscope}%
\pgfsetbuttcap%
\pgfsetroundjoin%
\definecolor{currentfill}{rgb}{0.000000,0.000000,0.000000}%
\pgfsetfillcolor{currentfill}%
\pgfsetlinewidth{0.602250pt}%
\definecolor{currentstroke}{rgb}{0.000000,0.000000,0.000000}%
\pgfsetstrokecolor{currentstroke}%
\pgfsetdash{}{0pt}%
\pgfsys@defobject{currentmarker}{\pgfqpoint{-0.027778in}{0.000000in}}{\pgfqpoint{-0.000000in}{0.000000in}}{%
\pgfpathmoveto{\pgfqpoint{-0.000000in}{0.000000in}}%
\pgfpathlineto{\pgfqpoint{-0.027778in}{0.000000in}}%
\pgfusepath{stroke,fill}%
}%
\begin{pgfscope}%
\pgfsys@transformshift{0.588387in}{0.972514in}%
\pgfsys@useobject{currentmarker}{}%
\end{pgfscope}%
\end{pgfscope}%
\begin{pgfscope}%
\pgfsetbuttcap%
\pgfsetroundjoin%
\definecolor{currentfill}{rgb}{0.000000,0.000000,0.000000}%
\pgfsetfillcolor{currentfill}%
\pgfsetlinewidth{0.602250pt}%
\definecolor{currentstroke}{rgb}{0.000000,0.000000,0.000000}%
\pgfsetstrokecolor{currentstroke}%
\pgfsetdash{}{0pt}%
\pgfsys@defobject{currentmarker}{\pgfqpoint{-0.027778in}{0.000000in}}{\pgfqpoint{-0.000000in}{0.000000in}}{%
\pgfpathmoveto{\pgfqpoint{-0.000000in}{0.000000in}}%
\pgfpathlineto{\pgfqpoint{-0.027778in}{0.000000in}}%
\pgfusepath{stroke,fill}%
}%
\begin{pgfscope}%
\pgfsys@transformshift{0.588387in}{1.002625in}%
\pgfsys@useobject{currentmarker}{}%
\end{pgfscope}%
\end{pgfscope}%
\begin{pgfscope}%
\pgfsetbuttcap%
\pgfsetroundjoin%
\definecolor{currentfill}{rgb}{0.000000,0.000000,0.000000}%
\pgfsetfillcolor{currentfill}%
\pgfsetlinewidth{0.602250pt}%
\definecolor{currentstroke}{rgb}{0.000000,0.000000,0.000000}%
\pgfsetstrokecolor{currentstroke}%
\pgfsetdash{}{0pt}%
\pgfsys@defobject{currentmarker}{\pgfqpoint{-0.027778in}{0.000000in}}{\pgfqpoint{-0.000000in}{0.000000in}}{%
\pgfpathmoveto{\pgfqpoint{-0.000000in}{0.000000in}}%
\pgfpathlineto{\pgfqpoint{-0.027778in}{0.000000in}}%
\pgfusepath{stroke,fill}%
}%
\begin{pgfscope}%
\pgfsys@transformshift{0.588387in}{1.028708in}%
\pgfsys@useobject{currentmarker}{}%
\end{pgfscope}%
\end{pgfscope}%
\begin{pgfscope}%
\pgfsetbuttcap%
\pgfsetroundjoin%
\definecolor{currentfill}{rgb}{0.000000,0.000000,0.000000}%
\pgfsetfillcolor{currentfill}%
\pgfsetlinewidth{0.602250pt}%
\definecolor{currentstroke}{rgb}{0.000000,0.000000,0.000000}%
\pgfsetstrokecolor{currentstroke}%
\pgfsetdash{}{0pt}%
\pgfsys@defobject{currentmarker}{\pgfqpoint{-0.027778in}{0.000000in}}{\pgfqpoint{-0.000000in}{0.000000in}}{%
\pgfpathmoveto{\pgfqpoint{-0.000000in}{0.000000in}}%
\pgfpathlineto{\pgfqpoint{-0.027778in}{0.000000in}}%
\pgfusepath{stroke,fill}%
}%
\begin{pgfscope}%
\pgfsys@transformshift{0.588387in}{1.051715in}%
\pgfsys@useobject{currentmarker}{}%
\end{pgfscope}%
\end{pgfscope}%
\begin{pgfscope}%
\pgfsetbuttcap%
\pgfsetroundjoin%
\definecolor{currentfill}{rgb}{0.000000,0.000000,0.000000}%
\pgfsetfillcolor{currentfill}%
\pgfsetlinewidth{0.602250pt}%
\definecolor{currentstroke}{rgb}{0.000000,0.000000,0.000000}%
\pgfsetstrokecolor{currentstroke}%
\pgfsetdash{}{0pt}%
\pgfsys@defobject{currentmarker}{\pgfqpoint{-0.027778in}{0.000000in}}{\pgfqpoint{-0.000000in}{0.000000in}}{%
\pgfpathmoveto{\pgfqpoint{-0.000000in}{0.000000in}}%
\pgfpathlineto{\pgfqpoint{-0.027778in}{0.000000in}}%
\pgfusepath{stroke,fill}%
}%
\begin{pgfscope}%
\pgfsys@transformshift{0.588387in}{1.207690in}%
\pgfsys@useobject{currentmarker}{}%
\end{pgfscope}%
\end{pgfscope}%
\begin{pgfscope}%
\pgfsetbuttcap%
\pgfsetroundjoin%
\definecolor{currentfill}{rgb}{0.000000,0.000000,0.000000}%
\pgfsetfillcolor{currentfill}%
\pgfsetlinewidth{0.602250pt}%
\definecolor{currentstroke}{rgb}{0.000000,0.000000,0.000000}%
\pgfsetstrokecolor{currentstroke}%
\pgfsetdash{}{0pt}%
\pgfsys@defobject{currentmarker}{\pgfqpoint{-0.027778in}{0.000000in}}{\pgfqpoint{-0.000000in}{0.000000in}}{%
\pgfpathmoveto{\pgfqpoint{-0.000000in}{0.000000in}}%
\pgfpathlineto{\pgfqpoint{-0.027778in}{0.000000in}}%
\pgfusepath{stroke,fill}%
}%
\begin{pgfscope}%
\pgfsys@transformshift{0.588387in}{1.286891in}%
\pgfsys@useobject{currentmarker}{}%
\end{pgfscope}%
\end{pgfscope}%
\begin{pgfscope}%
\pgfsetbuttcap%
\pgfsetroundjoin%
\definecolor{currentfill}{rgb}{0.000000,0.000000,0.000000}%
\pgfsetfillcolor{currentfill}%
\pgfsetlinewidth{0.602250pt}%
\definecolor{currentstroke}{rgb}{0.000000,0.000000,0.000000}%
\pgfsetstrokecolor{currentstroke}%
\pgfsetdash{}{0pt}%
\pgfsys@defobject{currentmarker}{\pgfqpoint{-0.027778in}{0.000000in}}{\pgfqpoint{-0.000000in}{0.000000in}}{%
\pgfpathmoveto{\pgfqpoint{-0.000000in}{0.000000in}}%
\pgfpathlineto{\pgfqpoint{-0.027778in}{0.000000in}}%
\pgfusepath{stroke,fill}%
}%
\begin{pgfscope}%
\pgfsys@transformshift{0.588387in}{1.343085in}%
\pgfsys@useobject{currentmarker}{}%
\end{pgfscope}%
\end{pgfscope}%
\begin{pgfscope}%
\pgfsetbuttcap%
\pgfsetroundjoin%
\definecolor{currentfill}{rgb}{0.000000,0.000000,0.000000}%
\pgfsetfillcolor{currentfill}%
\pgfsetlinewidth{0.602250pt}%
\definecolor{currentstroke}{rgb}{0.000000,0.000000,0.000000}%
\pgfsetstrokecolor{currentstroke}%
\pgfsetdash{}{0pt}%
\pgfsys@defobject{currentmarker}{\pgfqpoint{-0.027778in}{0.000000in}}{\pgfqpoint{-0.000000in}{0.000000in}}{%
\pgfpathmoveto{\pgfqpoint{-0.000000in}{0.000000in}}%
\pgfpathlineto{\pgfqpoint{-0.027778in}{0.000000in}}%
\pgfusepath{stroke,fill}%
}%
\begin{pgfscope}%
\pgfsys@transformshift{0.588387in}{1.386673in}%
\pgfsys@useobject{currentmarker}{}%
\end{pgfscope}%
\end{pgfscope}%
\begin{pgfscope}%
\pgfsetbuttcap%
\pgfsetroundjoin%
\definecolor{currentfill}{rgb}{0.000000,0.000000,0.000000}%
\pgfsetfillcolor{currentfill}%
\pgfsetlinewidth{0.602250pt}%
\definecolor{currentstroke}{rgb}{0.000000,0.000000,0.000000}%
\pgfsetstrokecolor{currentstroke}%
\pgfsetdash{}{0pt}%
\pgfsys@defobject{currentmarker}{\pgfqpoint{-0.027778in}{0.000000in}}{\pgfqpoint{-0.000000in}{0.000000in}}{%
\pgfpathmoveto{\pgfqpoint{-0.000000in}{0.000000in}}%
\pgfpathlineto{\pgfqpoint{-0.027778in}{0.000000in}}%
\pgfusepath{stroke,fill}%
}%
\begin{pgfscope}%
\pgfsys@transformshift{0.588387in}{1.422286in}%
\pgfsys@useobject{currentmarker}{}%
\end{pgfscope}%
\end{pgfscope}%
\begin{pgfscope}%
\pgfsetbuttcap%
\pgfsetroundjoin%
\definecolor{currentfill}{rgb}{0.000000,0.000000,0.000000}%
\pgfsetfillcolor{currentfill}%
\pgfsetlinewidth{0.602250pt}%
\definecolor{currentstroke}{rgb}{0.000000,0.000000,0.000000}%
\pgfsetstrokecolor{currentstroke}%
\pgfsetdash{}{0pt}%
\pgfsys@defobject{currentmarker}{\pgfqpoint{-0.027778in}{0.000000in}}{\pgfqpoint{-0.000000in}{0.000000in}}{%
\pgfpathmoveto{\pgfqpoint{-0.000000in}{0.000000in}}%
\pgfpathlineto{\pgfqpoint{-0.027778in}{0.000000in}}%
\pgfusepath{stroke,fill}%
}%
\begin{pgfscope}%
\pgfsys@transformshift{0.588387in}{1.452397in}%
\pgfsys@useobject{currentmarker}{}%
\end{pgfscope}%
\end{pgfscope}%
\begin{pgfscope}%
\pgfsetbuttcap%
\pgfsetroundjoin%
\definecolor{currentfill}{rgb}{0.000000,0.000000,0.000000}%
\pgfsetfillcolor{currentfill}%
\pgfsetlinewidth{0.602250pt}%
\definecolor{currentstroke}{rgb}{0.000000,0.000000,0.000000}%
\pgfsetstrokecolor{currentstroke}%
\pgfsetdash{}{0pt}%
\pgfsys@defobject{currentmarker}{\pgfqpoint{-0.027778in}{0.000000in}}{\pgfqpoint{-0.000000in}{0.000000in}}{%
\pgfpathmoveto{\pgfqpoint{-0.000000in}{0.000000in}}%
\pgfpathlineto{\pgfqpoint{-0.027778in}{0.000000in}}%
\pgfusepath{stroke,fill}%
}%
\begin{pgfscope}%
\pgfsys@transformshift{0.588387in}{1.478480in}%
\pgfsys@useobject{currentmarker}{}%
\end{pgfscope}%
\end{pgfscope}%
\begin{pgfscope}%
\pgfsetbuttcap%
\pgfsetroundjoin%
\definecolor{currentfill}{rgb}{0.000000,0.000000,0.000000}%
\pgfsetfillcolor{currentfill}%
\pgfsetlinewidth{0.602250pt}%
\definecolor{currentstroke}{rgb}{0.000000,0.000000,0.000000}%
\pgfsetstrokecolor{currentstroke}%
\pgfsetdash{}{0pt}%
\pgfsys@defobject{currentmarker}{\pgfqpoint{-0.027778in}{0.000000in}}{\pgfqpoint{-0.000000in}{0.000000in}}{%
\pgfpathmoveto{\pgfqpoint{-0.000000in}{0.000000in}}%
\pgfpathlineto{\pgfqpoint{-0.027778in}{0.000000in}}%
\pgfusepath{stroke,fill}%
}%
\begin{pgfscope}%
\pgfsys@transformshift{0.588387in}{1.501487in}%
\pgfsys@useobject{currentmarker}{}%
\end{pgfscope}%
\end{pgfscope}%
\begin{pgfscope}%
\pgfsetbuttcap%
\pgfsetroundjoin%
\definecolor{currentfill}{rgb}{0.000000,0.000000,0.000000}%
\pgfsetfillcolor{currentfill}%
\pgfsetlinewidth{0.602250pt}%
\definecolor{currentstroke}{rgb}{0.000000,0.000000,0.000000}%
\pgfsetstrokecolor{currentstroke}%
\pgfsetdash{}{0pt}%
\pgfsys@defobject{currentmarker}{\pgfqpoint{-0.027778in}{0.000000in}}{\pgfqpoint{-0.000000in}{0.000000in}}{%
\pgfpathmoveto{\pgfqpoint{-0.000000in}{0.000000in}}%
\pgfpathlineto{\pgfqpoint{-0.027778in}{0.000000in}}%
\pgfusepath{stroke,fill}%
}%
\begin{pgfscope}%
\pgfsys@transformshift{0.588387in}{1.657462in}%
\pgfsys@useobject{currentmarker}{}%
\end{pgfscope}%
\end{pgfscope}%
\begin{pgfscope}%
\pgfsetbuttcap%
\pgfsetroundjoin%
\definecolor{currentfill}{rgb}{0.000000,0.000000,0.000000}%
\pgfsetfillcolor{currentfill}%
\pgfsetlinewidth{0.602250pt}%
\definecolor{currentstroke}{rgb}{0.000000,0.000000,0.000000}%
\pgfsetstrokecolor{currentstroke}%
\pgfsetdash{}{0pt}%
\pgfsys@defobject{currentmarker}{\pgfqpoint{-0.027778in}{0.000000in}}{\pgfqpoint{-0.000000in}{0.000000in}}{%
\pgfpathmoveto{\pgfqpoint{-0.000000in}{0.000000in}}%
\pgfpathlineto{\pgfqpoint{-0.027778in}{0.000000in}}%
\pgfusepath{stroke,fill}%
}%
\begin{pgfscope}%
\pgfsys@transformshift{0.588387in}{1.736663in}%
\pgfsys@useobject{currentmarker}{}%
\end{pgfscope}%
\end{pgfscope}%
\begin{pgfscope}%
\pgfsetbuttcap%
\pgfsetroundjoin%
\definecolor{currentfill}{rgb}{0.000000,0.000000,0.000000}%
\pgfsetfillcolor{currentfill}%
\pgfsetlinewidth{0.602250pt}%
\definecolor{currentstroke}{rgb}{0.000000,0.000000,0.000000}%
\pgfsetstrokecolor{currentstroke}%
\pgfsetdash{}{0pt}%
\pgfsys@defobject{currentmarker}{\pgfqpoint{-0.027778in}{0.000000in}}{\pgfqpoint{-0.000000in}{0.000000in}}{%
\pgfpathmoveto{\pgfqpoint{-0.000000in}{0.000000in}}%
\pgfpathlineto{\pgfqpoint{-0.027778in}{0.000000in}}%
\pgfusepath{stroke,fill}%
}%
\begin{pgfscope}%
\pgfsys@transformshift{0.588387in}{1.792857in}%
\pgfsys@useobject{currentmarker}{}%
\end{pgfscope}%
\end{pgfscope}%
\begin{pgfscope}%
\pgfsetbuttcap%
\pgfsetroundjoin%
\definecolor{currentfill}{rgb}{0.000000,0.000000,0.000000}%
\pgfsetfillcolor{currentfill}%
\pgfsetlinewidth{0.602250pt}%
\definecolor{currentstroke}{rgb}{0.000000,0.000000,0.000000}%
\pgfsetstrokecolor{currentstroke}%
\pgfsetdash{}{0pt}%
\pgfsys@defobject{currentmarker}{\pgfqpoint{-0.027778in}{0.000000in}}{\pgfqpoint{-0.000000in}{0.000000in}}{%
\pgfpathmoveto{\pgfqpoint{-0.000000in}{0.000000in}}%
\pgfpathlineto{\pgfqpoint{-0.027778in}{0.000000in}}%
\pgfusepath{stroke,fill}%
}%
\begin{pgfscope}%
\pgfsys@transformshift{0.588387in}{1.836445in}%
\pgfsys@useobject{currentmarker}{}%
\end{pgfscope}%
\end{pgfscope}%
\begin{pgfscope}%
\pgfsetbuttcap%
\pgfsetroundjoin%
\definecolor{currentfill}{rgb}{0.000000,0.000000,0.000000}%
\pgfsetfillcolor{currentfill}%
\pgfsetlinewidth{0.602250pt}%
\definecolor{currentstroke}{rgb}{0.000000,0.000000,0.000000}%
\pgfsetstrokecolor{currentstroke}%
\pgfsetdash{}{0pt}%
\pgfsys@defobject{currentmarker}{\pgfqpoint{-0.027778in}{0.000000in}}{\pgfqpoint{-0.000000in}{0.000000in}}{%
\pgfpathmoveto{\pgfqpoint{-0.000000in}{0.000000in}}%
\pgfpathlineto{\pgfqpoint{-0.027778in}{0.000000in}}%
\pgfusepath{stroke,fill}%
}%
\begin{pgfscope}%
\pgfsys@transformshift{0.588387in}{1.872058in}%
\pgfsys@useobject{currentmarker}{}%
\end{pgfscope}%
\end{pgfscope}%
\begin{pgfscope}%
\pgfsetbuttcap%
\pgfsetroundjoin%
\definecolor{currentfill}{rgb}{0.000000,0.000000,0.000000}%
\pgfsetfillcolor{currentfill}%
\pgfsetlinewidth{0.602250pt}%
\definecolor{currentstroke}{rgb}{0.000000,0.000000,0.000000}%
\pgfsetstrokecolor{currentstroke}%
\pgfsetdash{}{0pt}%
\pgfsys@defobject{currentmarker}{\pgfqpoint{-0.027778in}{0.000000in}}{\pgfqpoint{-0.000000in}{0.000000in}}{%
\pgfpathmoveto{\pgfqpoint{-0.000000in}{0.000000in}}%
\pgfpathlineto{\pgfqpoint{-0.027778in}{0.000000in}}%
\pgfusepath{stroke,fill}%
}%
\begin{pgfscope}%
\pgfsys@transformshift{0.588387in}{1.902169in}%
\pgfsys@useobject{currentmarker}{}%
\end{pgfscope}%
\end{pgfscope}%
\begin{pgfscope}%
\pgfsetbuttcap%
\pgfsetroundjoin%
\definecolor{currentfill}{rgb}{0.000000,0.000000,0.000000}%
\pgfsetfillcolor{currentfill}%
\pgfsetlinewidth{0.602250pt}%
\definecolor{currentstroke}{rgb}{0.000000,0.000000,0.000000}%
\pgfsetstrokecolor{currentstroke}%
\pgfsetdash{}{0pt}%
\pgfsys@defobject{currentmarker}{\pgfqpoint{-0.027778in}{0.000000in}}{\pgfqpoint{-0.000000in}{0.000000in}}{%
\pgfpathmoveto{\pgfqpoint{-0.000000in}{0.000000in}}%
\pgfpathlineto{\pgfqpoint{-0.027778in}{0.000000in}}%
\pgfusepath{stroke,fill}%
}%
\begin{pgfscope}%
\pgfsys@transformshift{0.588387in}{1.928252in}%
\pgfsys@useobject{currentmarker}{}%
\end{pgfscope}%
\end{pgfscope}%
\begin{pgfscope}%
\pgfsetbuttcap%
\pgfsetroundjoin%
\definecolor{currentfill}{rgb}{0.000000,0.000000,0.000000}%
\pgfsetfillcolor{currentfill}%
\pgfsetlinewidth{0.602250pt}%
\definecolor{currentstroke}{rgb}{0.000000,0.000000,0.000000}%
\pgfsetstrokecolor{currentstroke}%
\pgfsetdash{}{0pt}%
\pgfsys@defobject{currentmarker}{\pgfqpoint{-0.027778in}{0.000000in}}{\pgfqpoint{-0.000000in}{0.000000in}}{%
\pgfpathmoveto{\pgfqpoint{-0.000000in}{0.000000in}}%
\pgfpathlineto{\pgfqpoint{-0.027778in}{0.000000in}}%
\pgfusepath{stroke,fill}%
}%
\begin{pgfscope}%
\pgfsys@transformshift{0.588387in}{1.951259in}%
\pgfsys@useobject{currentmarker}{}%
\end{pgfscope}%
\end{pgfscope}%
\begin{pgfscope}%
\pgfsetbuttcap%
\pgfsetroundjoin%
\definecolor{currentfill}{rgb}{0.000000,0.000000,0.000000}%
\pgfsetfillcolor{currentfill}%
\pgfsetlinewidth{0.602250pt}%
\definecolor{currentstroke}{rgb}{0.000000,0.000000,0.000000}%
\pgfsetstrokecolor{currentstroke}%
\pgfsetdash{}{0pt}%
\pgfsys@defobject{currentmarker}{\pgfqpoint{-0.027778in}{0.000000in}}{\pgfqpoint{-0.000000in}{0.000000in}}{%
\pgfpathmoveto{\pgfqpoint{-0.000000in}{0.000000in}}%
\pgfpathlineto{\pgfqpoint{-0.027778in}{0.000000in}}%
\pgfusepath{stroke,fill}%
}%
\begin{pgfscope}%
\pgfsys@transformshift{0.588387in}{2.107234in}%
\pgfsys@useobject{currentmarker}{}%
\end{pgfscope}%
\end{pgfscope}%
\begin{pgfscope}%
\pgfsetbuttcap%
\pgfsetroundjoin%
\definecolor{currentfill}{rgb}{0.000000,0.000000,0.000000}%
\pgfsetfillcolor{currentfill}%
\pgfsetlinewidth{0.602250pt}%
\definecolor{currentstroke}{rgb}{0.000000,0.000000,0.000000}%
\pgfsetstrokecolor{currentstroke}%
\pgfsetdash{}{0pt}%
\pgfsys@defobject{currentmarker}{\pgfqpoint{-0.027778in}{0.000000in}}{\pgfqpoint{-0.000000in}{0.000000in}}{%
\pgfpathmoveto{\pgfqpoint{-0.000000in}{0.000000in}}%
\pgfpathlineto{\pgfqpoint{-0.027778in}{0.000000in}}%
\pgfusepath{stroke,fill}%
}%
\begin{pgfscope}%
\pgfsys@transformshift{0.588387in}{2.186435in}%
\pgfsys@useobject{currentmarker}{}%
\end{pgfscope}%
\end{pgfscope}%
\begin{pgfscope}%
\pgfsetbuttcap%
\pgfsetroundjoin%
\definecolor{currentfill}{rgb}{0.000000,0.000000,0.000000}%
\pgfsetfillcolor{currentfill}%
\pgfsetlinewidth{0.602250pt}%
\definecolor{currentstroke}{rgb}{0.000000,0.000000,0.000000}%
\pgfsetstrokecolor{currentstroke}%
\pgfsetdash{}{0pt}%
\pgfsys@defobject{currentmarker}{\pgfqpoint{-0.027778in}{0.000000in}}{\pgfqpoint{-0.000000in}{0.000000in}}{%
\pgfpathmoveto{\pgfqpoint{-0.000000in}{0.000000in}}%
\pgfpathlineto{\pgfqpoint{-0.027778in}{0.000000in}}%
\pgfusepath{stroke,fill}%
}%
\begin{pgfscope}%
\pgfsys@transformshift{0.588387in}{2.242629in}%
\pgfsys@useobject{currentmarker}{}%
\end{pgfscope}%
\end{pgfscope}%
\begin{pgfscope}%
\pgfsetbuttcap%
\pgfsetroundjoin%
\definecolor{currentfill}{rgb}{0.000000,0.000000,0.000000}%
\pgfsetfillcolor{currentfill}%
\pgfsetlinewidth{0.602250pt}%
\definecolor{currentstroke}{rgb}{0.000000,0.000000,0.000000}%
\pgfsetstrokecolor{currentstroke}%
\pgfsetdash{}{0pt}%
\pgfsys@defobject{currentmarker}{\pgfqpoint{-0.027778in}{0.000000in}}{\pgfqpoint{-0.000000in}{0.000000in}}{%
\pgfpathmoveto{\pgfqpoint{-0.000000in}{0.000000in}}%
\pgfpathlineto{\pgfqpoint{-0.027778in}{0.000000in}}%
\pgfusepath{stroke,fill}%
}%
\begin{pgfscope}%
\pgfsys@transformshift{0.588387in}{2.286217in}%
\pgfsys@useobject{currentmarker}{}%
\end{pgfscope}%
\end{pgfscope}%
\begin{pgfscope}%
\pgfsetbuttcap%
\pgfsetroundjoin%
\definecolor{currentfill}{rgb}{0.000000,0.000000,0.000000}%
\pgfsetfillcolor{currentfill}%
\pgfsetlinewidth{0.602250pt}%
\definecolor{currentstroke}{rgb}{0.000000,0.000000,0.000000}%
\pgfsetstrokecolor{currentstroke}%
\pgfsetdash{}{0pt}%
\pgfsys@defobject{currentmarker}{\pgfqpoint{-0.027778in}{0.000000in}}{\pgfqpoint{-0.000000in}{0.000000in}}{%
\pgfpathmoveto{\pgfqpoint{-0.000000in}{0.000000in}}%
\pgfpathlineto{\pgfqpoint{-0.027778in}{0.000000in}}%
\pgfusepath{stroke,fill}%
}%
\begin{pgfscope}%
\pgfsys@transformshift{0.588387in}{2.321830in}%
\pgfsys@useobject{currentmarker}{}%
\end{pgfscope}%
\end{pgfscope}%
\begin{pgfscope}%
\pgfsetbuttcap%
\pgfsetroundjoin%
\definecolor{currentfill}{rgb}{0.000000,0.000000,0.000000}%
\pgfsetfillcolor{currentfill}%
\pgfsetlinewidth{0.602250pt}%
\definecolor{currentstroke}{rgb}{0.000000,0.000000,0.000000}%
\pgfsetstrokecolor{currentstroke}%
\pgfsetdash{}{0pt}%
\pgfsys@defobject{currentmarker}{\pgfqpoint{-0.027778in}{0.000000in}}{\pgfqpoint{-0.000000in}{0.000000in}}{%
\pgfpathmoveto{\pgfqpoint{-0.000000in}{0.000000in}}%
\pgfpathlineto{\pgfqpoint{-0.027778in}{0.000000in}}%
\pgfusepath{stroke,fill}%
}%
\begin{pgfscope}%
\pgfsys@transformshift{0.588387in}{2.351941in}%
\pgfsys@useobject{currentmarker}{}%
\end{pgfscope}%
\end{pgfscope}%
\begin{pgfscope}%
\pgfsetbuttcap%
\pgfsetroundjoin%
\definecolor{currentfill}{rgb}{0.000000,0.000000,0.000000}%
\pgfsetfillcolor{currentfill}%
\pgfsetlinewidth{0.602250pt}%
\definecolor{currentstroke}{rgb}{0.000000,0.000000,0.000000}%
\pgfsetstrokecolor{currentstroke}%
\pgfsetdash{}{0pt}%
\pgfsys@defobject{currentmarker}{\pgfqpoint{-0.027778in}{0.000000in}}{\pgfqpoint{-0.000000in}{0.000000in}}{%
\pgfpathmoveto{\pgfqpoint{-0.000000in}{0.000000in}}%
\pgfpathlineto{\pgfqpoint{-0.027778in}{0.000000in}}%
\pgfusepath{stroke,fill}%
}%
\begin{pgfscope}%
\pgfsys@transformshift{0.588387in}{2.378024in}%
\pgfsys@useobject{currentmarker}{}%
\end{pgfscope}%
\end{pgfscope}%
\begin{pgfscope}%
\pgfsetbuttcap%
\pgfsetroundjoin%
\definecolor{currentfill}{rgb}{0.000000,0.000000,0.000000}%
\pgfsetfillcolor{currentfill}%
\pgfsetlinewidth{0.602250pt}%
\definecolor{currentstroke}{rgb}{0.000000,0.000000,0.000000}%
\pgfsetstrokecolor{currentstroke}%
\pgfsetdash{}{0pt}%
\pgfsys@defobject{currentmarker}{\pgfqpoint{-0.027778in}{0.000000in}}{\pgfqpoint{-0.000000in}{0.000000in}}{%
\pgfpathmoveto{\pgfqpoint{-0.000000in}{0.000000in}}%
\pgfpathlineto{\pgfqpoint{-0.027778in}{0.000000in}}%
\pgfusepath{stroke,fill}%
}%
\begin{pgfscope}%
\pgfsys@transformshift{0.588387in}{2.401031in}%
\pgfsys@useobject{currentmarker}{}%
\end{pgfscope}%
\end{pgfscope}%
\begin{pgfscope}%
\pgfsetbuttcap%
\pgfsetroundjoin%
\definecolor{currentfill}{rgb}{0.000000,0.000000,0.000000}%
\pgfsetfillcolor{currentfill}%
\pgfsetlinewidth{0.602250pt}%
\definecolor{currentstroke}{rgb}{0.000000,0.000000,0.000000}%
\pgfsetstrokecolor{currentstroke}%
\pgfsetdash{}{0pt}%
\pgfsys@defobject{currentmarker}{\pgfqpoint{-0.027778in}{0.000000in}}{\pgfqpoint{-0.000000in}{0.000000in}}{%
\pgfpathmoveto{\pgfqpoint{-0.000000in}{0.000000in}}%
\pgfpathlineto{\pgfqpoint{-0.027778in}{0.000000in}}%
\pgfusepath{stroke,fill}%
}%
\begin{pgfscope}%
\pgfsys@transformshift{0.588387in}{2.557006in}%
\pgfsys@useobject{currentmarker}{}%
\end{pgfscope}%
\end{pgfscope}%
\begin{pgfscope}%
\pgfsetbuttcap%
\pgfsetroundjoin%
\definecolor{currentfill}{rgb}{0.000000,0.000000,0.000000}%
\pgfsetfillcolor{currentfill}%
\pgfsetlinewidth{0.602250pt}%
\definecolor{currentstroke}{rgb}{0.000000,0.000000,0.000000}%
\pgfsetstrokecolor{currentstroke}%
\pgfsetdash{}{0pt}%
\pgfsys@defobject{currentmarker}{\pgfqpoint{-0.027778in}{0.000000in}}{\pgfqpoint{-0.000000in}{0.000000in}}{%
\pgfpathmoveto{\pgfqpoint{-0.000000in}{0.000000in}}%
\pgfpathlineto{\pgfqpoint{-0.027778in}{0.000000in}}%
\pgfusepath{stroke,fill}%
}%
\begin{pgfscope}%
\pgfsys@transformshift{0.588387in}{2.636207in}%
\pgfsys@useobject{currentmarker}{}%
\end{pgfscope}%
\end{pgfscope}%
\begin{pgfscope}%
\pgfsetbuttcap%
\pgfsetroundjoin%
\definecolor{currentfill}{rgb}{0.000000,0.000000,0.000000}%
\pgfsetfillcolor{currentfill}%
\pgfsetlinewidth{0.602250pt}%
\definecolor{currentstroke}{rgb}{0.000000,0.000000,0.000000}%
\pgfsetstrokecolor{currentstroke}%
\pgfsetdash{}{0pt}%
\pgfsys@defobject{currentmarker}{\pgfqpoint{-0.027778in}{0.000000in}}{\pgfqpoint{-0.000000in}{0.000000in}}{%
\pgfpathmoveto{\pgfqpoint{-0.000000in}{0.000000in}}%
\pgfpathlineto{\pgfqpoint{-0.027778in}{0.000000in}}%
\pgfusepath{stroke,fill}%
}%
\begin{pgfscope}%
\pgfsys@transformshift{0.588387in}{2.692401in}%
\pgfsys@useobject{currentmarker}{}%
\end{pgfscope}%
\end{pgfscope}%
\begin{pgfscope}%
\pgfsetbuttcap%
\pgfsetroundjoin%
\definecolor{currentfill}{rgb}{0.000000,0.000000,0.000000}%
\pgfsetfillcolor{currentfill}%
\pgfsetlinewidth{0.602250pt}%
\definecolor{currentstroke}{rgb}{0.000000,0.000000,0.000000}%
\pgfsetstrokecolor{currentstroke}%
\pgfsetdash{}{0pt}%
\pgfsys@defobject{currentmarker}{\pgfqpoint{-0.027778in}{0.000000in}}{\pgfqpoint{-0.000000in}{0.000000in}}{%
\pgfpathmoveto{\pgfqpoint{-0.000000in}{0.000000in}}%
\pgfpathlineto{\pgfqpoint{-0.027778in}{0.000000in}}%
\pgfusepath{stroke,fill}%
}%
\begin{pgfscope}%
\pgfsys@transformshift{0.588387in}{2.735989in}%
\pgfsys@useobject{currentmarker}{}%
\end{pgfscope}%
\end{pgfscope}%
\begin{pgfscope}%
\definecolor{textcolor}{rgb}{0.000000,0.000000,0.000000}%
\pgfsetstrokecolor{textcolor}%
\pgfsetfillcolor{textcolor}%
\pgftext[x=0.234413in,y=1.631726in,,bottom,rotate=90.000000]{\color{textcolor}{\rmfamily\fontsize{10.000000}{12.000000}\selectfont\catcode`\^=\active\def^{\ifmmode\sp\else\^{}\fi}\catcode`\%=\active\def%{\%}Checks [call]}}%
\end{pgfscope}%
\begin{pgfscope}%
\pgfpathrectangle{\pgfqpoint{0.588387in}{0.521603in}}{\pgfqpoint{3.660036in}{2.220246in}}%
\pgfusepath{clip}%
\pgfsetrectcap%
\pgfsetroundjoin%
\pgfsetlinewidth{1.505625pt}%
\pgfsetstrokecolor{currentstroke1}%
\pgfsetdash{}{0pt}%
\pgfpathmoveto{\pgfqpoint{0.754752in}{0.757918in}}%
\pgfpathlineto{\pgfqpoint{0.804414in}{0.787671in}}%
\pgfpathlineto{\pgfqpoint{0.854075in}{0.849456in}}%
\pgfpathlineto{\pgfqpoint{0.903736in}{0.883651in}}%
\pgfpathlineto{\pgfqpoint{0.953398in}{0.943638in}}%
\pgfpathlineto{\pgfqpoint{1.003059in}{1.007369in}}%
\pgfpathlineto{\pgfqpoint{1.052720in}{1.100530in}}%
\pgfpathlineto{\pgfqpoint{1.102381in}{1.168276in}}%
\pgfpathlineto{\pgfqpoint{1.152043in}{1.206775in}}%
\pgfpathlineto{\pgfqpoint{1.201704in}{1.247378in}}%
\pgfpathlineto{\pgfqpoint{1.251365in}{1.333712in}}%
\pgfpathlineto{\pgfqpoint{1.301026in}{1.328792in}}%
\pgfpathlineto{\pgfqpoint{1.350688in}{1.430543in}}%
\pgfpathlineto{\pgfqpoint{1.400349in}{1.448124in}}%
\pgfpathlineto{\pgfqpoint{1.450010in}{1.513866in}}%
\pgfpathlineto{\pgfqpoint{1.499672in}{1.518695in}}%
\pgfpathlineto{\pgfqpoint{1.549333in}{1.439689in}}%
\pgfpathlineto{\pgfqpoint{1.598994in}{1.492390in}}%
\pgfpathlineto{\pgfqpoint{1.648655in}{1.511424in}}%
\pgfpathlineto{\pgfqpoint{1.698317in}{1.535952in}}%
\pgfpathlineto{\pgfqpoint{1.747978in}{1.563635in}}%
\pgfpathlineto{\pgfqpoint{1.797639in}{1.746442in}}%
\pgfpathlineto{\pgfqpoint{1.847300in}{1.533818in}}%
\pgfpathlineto{\pgfqpoint{1.896962in}{1.494661in}}%
\pgfpathlineto{\pgfqpoint{1.946623in}{1.626806in}}%
\pgfpathlineto{\pgfqpoint{1.996284in}{1.639933in}}%
\pgfpathlineto{\pgfqpoint{2.045945in}{1.631007in}}%
\pgfpathlineto{\pgfqpoint{2.095607in}{1.641529in}}%
\pgfpathlineto{\pgfqpoint{2.145268in}{1.640784in}}%
\pgfpathlineto{\pgfqpoint{2.194929in}{1.608538in}}%
\pgfpathlineto{\pgfqpoint{2.244591in}{1.637559in}}%
\pgfpathlineto{\pgfqpoint{2.294252in}{1.636701in}}%
\pgfpathlineto{\pgfqpoint{2.343913in}{1.638503in}}%
\pgfpathlineto{\pgfqpoint{2.393574in}{1.726986in}}%
\pgfpathlineto{\pgfqpoint{2.443236in}{1.691851in}}%
\pgfpathlineto{\pgfqpoint{2.492897in}{1.593295in}}%
\pgfpathlineto{\pgfqpoint{2.542558in}{1.641705in}}%
\pgfpathlineto{\pgfqpoint{2.592219in}{1.714631in}}%
\pgfpathlineto{\pgfqpoint{2.641881in}{1.692872in}}%
\pgfpathlineto{\pgfqpoint{2.691542in}{1.690825in}}%
\pgfpathlineto{\pgfqpoint{2.741203in}{1.653516in}}%
\pgfpathlineto{\pgfqpoint{2.890187in}{1.688652in}}%
\pgfpathlineto{\pgfqpoint{3.039171in}{1.681336in}}%
\pgfpathlineto{\pgfqpoint{3.138493in}{1.760537in}}%
\pgfpathlineto{\pgfqpoint{3.188155in}{1.718241in}}%
\pgfpathlineto{\pgfqpoint{3.436461in}{1.761686in}}%
\pgfusepath{stroke}%
\end{pgfscope}%
\begin{pgfscope}%
\pgfpathrectangle{\pgfqpoint{0.588387in}{0.521603in}}{\pgfqpoint{3.660036in}{2.220246in}}%
\pgfusepath{clip}%
\pgfsetrectcap%
\pgfsetroundjoin%
\pgfsetlinewidth{1.505625pt}%
\pgfsetstrokecolor{currentstroke2}%
\pgfsetdash{}{0pt}%
\pgfpathmoveto{\pgfqpoint{0.754752in}{0.757918in}}%
\pgfpathlineto{\pgfqpoint{0.804414in}{0.787582in}}%
\pgfpathlineto{\pgfqpoint{0.854075in}{0.849510in}}%
\pgfpathlineto{\pgfqpoint{0.903736in}{0.883777in}}%
\pgfpathlineto{\pgfqpoint{0.953398in}{0.944470in}}%
\pgfpathlineto{\pgfqpoint{1.003059in}{1.006984in}}%
\pgfpathlineto{\pgfqpoint{1.052720in}{1.100720in}}%
\pgfpathlineto{\pgfqpoint{1.102381in}{1.167494in}}%
\pgfpathlineto{\pgfqpoint{1.152043in}{1.206493in}}%
\pgfpathlineto{\pgfqpoint{1.201704in}{1.247424in}}%
\pgfpathlineto{\pgfqpoint{1.251365in}{1.342423in}}%
\pgfpathlineto{\pgfqpoint{1.301026in}{1.334859in}}%
\pgfpathlineto{\pgfqpoint{1.350688in}{1.431601in}}%
\pgfpathlineto{\pgfqpoint{1.400349in}{1.453375in}}%
\pgfpathlineto{\pgfqpoint{1.450010in}{1.539781in}}%
\pgfpathlineto{\pgfqpoint{1.499672in}{1.520975in}}%
\pgfpathlineto{\pgfqpoint{1.549333in}{1.440145in}}%
\pgfpathlineto{\pgfqpoint{1.598994in}{1.553354in}}%
\pgfpathlineto{\pgfqpoint{1.648655in}{1.687075in}}%
\pgfpathlineto{\pgfqpoint{1.698317in}{1.633985in}}%
\pgfpathlineto{\pgfqpoint{1.747978in}{1.562504in}}%
\pgfpathlineto{\pgfqpoint{1.797639in}{1.716396in}}%
\pgfpathlineto{\pgfqpoint{1.847300in}{1.874865in}}%
\pgfpathlineto{\pgfqpoint{1.896962in}{1.508844in}}%
\pgfpathlineto{\pgfqpoint{1.946623in}{1.656238in}}%
\pgfpathlineto{\pgfqpoint{1.996284in}{1.798726in}}%
\pgfpathlineto{\pgfqpoint{2.045945in}{1.610937in}}%
\pgfpathlineto{\pgfqpoint{2.095607in}{1.643112in}}%
\pgfpathlineto{\pgfqpoint{2.145268in}{1.719742in}}%
\pgfpathlineto{\pgfqpoint{2.194929in}{1.655499in}}%
\pgfpathlineto{\pgfqpoint{2.244591in}{1.630540in}}%
\pgfpathlineto{\pgfqpoint{2.294252in}{1.625334in}}%
\pgfpathlineto{\pgfqpoint{2.343913in}{1.654014in}}%
\pgfpathlineto{\pgfqpoint{2.393574in}{1.722628in}}%
\pgfpathlineto{\pgfqpoint{2.443236in}{1.739465in}}%
\pgfpathlineto{\pgfqpoint{2.492897in}{1.629416in}}%
\pgfpathlineto{\pgfqpoint{2.542558in}{1.655217in}}%
\pgfpathlineto{\pgfqpoint{2.641881in}{1.683698in}}%
\pgfpathlineto{\pgfqpoint{2.691542in}{1.692872in}}%
\pgfpathlineto{\pgfqpoint{2.741203in}{1.613875in}}%
\pgfpathlineto{\pgfqpoint{2.890187in}{1.734151in}}%
\pgfpathlineto{\pgfqpoint{3.039171in}{1.726644in}}%
\pgfpathlineto{\pgfqpoint{3.138493in}{1.756166in}}%
\pgfpathlineto{\pgfqpoint{3.436461in}{1.742437in}}%
\pgfusepath{stroke}%
\end{pgfscope}%
\begin{pgfscope}%
\pgfpathrectangle{\pgfqpoint{0.588387in}{0.521603in}}{\pgfqpoint{3.660036in}{2.220246in}}%
\pgfusepath{clip}%
\pgfsetrectcap%
\pgfsetroundjoin%
\pgfsetlinewidth{1.505625pt}%
\pgfsetstrokecolor{currentstroke3}%
\pgfsetdash{}{0pt}%
\pgfpathmoveto{\pgfqpoint{0.754752in}{0.622524in}}%
\pgfpathlineto{\pgfqpoint{0.804414in}{0.680439in}}%
\pgfpathlineto{\pgfqpoint{0.854075in}{0.785871in}}%
\pgfpathlineto{\pgfqpoint{0.903736in}{0.830424in}}%
\pgfpathlineto{\pgfqpoint{0.953398in}{0.928890in}}%
\pgfpathlineto{\pgfqpoint{1.003059in}{0.992758in}}%
\pgfpathlineto{\pgfqpoint{1.052720in}{1.066940in}}%
\pgfpathlineto{\pgfqpoint{1.102381in}{1.153574in}}%
\pgfpathlineto{\pgfqpoint{1.152043in}{1.172759in}}%
\pgfpathlineto{\pgfqpoint{1.201704in}{1.171294in}}%
\pgfpathlineto{\pgfqpoint{1.251365in}{1.361279in}}%
\pgfpathlineto{\pgfqpoint{1.301026in}{1.392025in}}%
\pgfpathlineto{\pgfqpoint{1.350688in}{1.614844in}}%
\pgfpathlineto{\pgfqpoint{1.400349in}{1.600633in}}%
\pgfpathlineto{\pgfqpoint{1.450010in}{1.870672in}}%
\pgfpathlineto{\pgfqpoint{1.499672in}{1.830217in}}%
\pgfpathlineto{\pgfqpoint{1.549333in}{1.297826in}}%
\pgfpathlineto{\pgfqpoint{1.598994in}{2.315293in}}%
\pgfpathlineto{\pgfqpoint{1.648655in}{1.869023in}}%
\pgfpathlineto{\pgfqpoint{1.698317in}{2.130848in}}%
\pgfpathlineto{\pgfqpoint{1.747978in}{1.430572in}}%
\pgfpathlineto{\pgfqpoint{1.797639in}{2.034551in}}%
\pgfpathlineto{\pgfqpoint{1.847300in}{2.252477in}}%
\pgfpathlineto{\pgfqpoint{1.896962in}{1.371796in}}%
\pgfpathlineto{\pgfqpoint{1.946623in}{2.069362in}}%
\pgfpathlineto{\pgfqpoint{1.996284in}{2.236720in}}%
\pgfpathlineto{\pgfqpoint{2.045945in}{2.583252in}}%
\pgfpathlineto{\pgfqpoint{2.095607in}{2.078755in}}%
\pgfpathlineto{\pgfqpoint{2.145268in}{2.156704in}}%
\pgfpathlineto{\pgfqpoint{2.194929in}{2.078001in}}%
\pgfpathlineto{\pgfqpoint{2.244591in}{1.956188in}}%
\pgfpathlineto{\pgfqpoint{2.294252in}{2.396649in}}%
\pgfpathlineto{\pgfqpoint{2.343913in}{2.370050in}}%
\pgfpathlineto{\pgfqpoint{2.393574in}{1.701310in}}%
\pgfpathlineto{\pgfqpoint{2.443236in}{1.343085in}}%
\pgfpathlineto{\pgfqpoint{2.492897in}{1.002625in}}%
\pgfpathlineto{\pgfqpoint{2.542558in}{1.424445in}}%
\pgfpathlineto{\pgfqpoint{2.592219in}{2.259463in}}%
\pgfpathlineto{\pgfqpoint{2.641881in}{1.456539in}}%
\pgfpathlineto{\pgfqpoint{2.691542in}{1.815864in}}%
\pgfpathlineto{\pgfqpoint{2.741203in}{1.151497in}}%
\pgfpathlineto{\pgfqpoint{2.890187in}{2.113356in}}%
\pgfpathlineto{\pgfqpoint{3.039171in}{2.640929in}}%
\pgfpathlineto{\pgfqpoint{3.138493in}{1.151497in}}%
\pgfpathlineto{\pgfqpoint{3.188155in}{1.319773in}}%
\pgfpathlineto{\pgfqpoint{3.287477in}{2.099261in}}%
\pgfpathlineto{\pgfqpoint{3.486122in}{2.105665in}}%
\pgfpathlineto{\pgfqpoint{3.784090in}{1.557681in}}%
\pgfpathlineto{\pgfqpoint{4.082057in}{1.293296in}}%
\pgfusepath{stroke}%
\end{pgfscope}%
\begin{pgfscope}%
\pgfpathrectangle{\pgfqpoint{0.588387in}{0.521603in}}{\pgfqpoint{3.660036in}{2.220246in}}%
\pgfusepath{clip}%
\pgfsetrectcap%
\pgfsetroundjoin%
\pgfsetlinewidth{1.505625pt}%
\pgfsetstrokecolor{currentstroke4}%
\pgfsetdash{}{0pt}%
\pgfpathmoveto{\pgfqpoint{0.754752in}{0.757918in}}%
\pgfpathlineto{\pgfqpoint{0.804414in}{0.759169in}}%
\pgfpathlineto{\pgfqpoint{0.854075in}{0.828949in}}%
\pgfpathlineto{\pgfqpoint{0.903736in}{0.855004in}}%
\pgfpathlineto{\pgfqpoint{0.953398in}{0.925385in}}%
\pgfpathlineto{\pgfqpoint{1.003059in}{0.968649in}}%
\pgfpathlineto{\pgfqpoint{1.052720in}{1.086439in}}%
\pgfpathlineto{\pgfqpoint{1.102381in}{1.180922in}}%
\pgfpathlineto{\pgfqpoint{1.152043in}{1.199748in}}%
\pgfpathlineto{\pgfqpoint{1.201704in}{1.243651in}}%
\pgfpathlineto{\pgfqpoint{1.251365in}{1.340199in}}%
\pgfpathlineto{\pgfqpoint{1.301026in}{1.316375in}}%
\pgfpathlineto{\pgfqpoint{1.350688in}{1.416361in}}%
\pgfpathlineto{\pgfqpoint{1.400349in}{1.436547in}}%
\pgfpathlineto{\pgfqpoint{1.450010in}{1.516198in}}%
\pgfpathlineto{\pgfqpoint{1.499672in}{1.505290in}}%
\pgfpathlineto{\pgfqpoint{1.549333in}{1.427331in}}%
\pgfpathlineto{\pgfqpoint{1.598994in}{1.448376in}}%
\pgfpathlineto{\pgfqpoint{1.648655in}{1.486170in}}%
\pgfpathlineto{\pgfqpoint{1.698317in}{1.535469in}}%
\pgfpathlineto{\pgfqpoint{1.747978in}{1.538408in}}%
\pgfpathlineto{\pgfqpoint{1.797639in}{1.781134in}}%
\pgfpathlineto{\pgfqpoint{1.847300in}{1.491239in}}%
\pgfpathlineto{\pgfqpoint{1.896962in}{1.496491in}}%
\pgfpathlineto{\pgfqpoint{1.946623in}{1.550526in}}%
\pgfpathlineto{\pgfqpoint{1.996284in}{1.584391in}}%
\pgfpathlineto{\pgfqpoint{2.045945in}{1.593521in}}%
\pgfpathlineto{\pgfqpoint{2.095607in}{1.597211in}}%
\pgfpathlineto{\pgfqpoint{2.145268in}{1.554315in}}%
\pgfpathlineto{\pgfqpoint{2.194929in}{1.579145in}}%
\pgfpathlineto{\pgfqpoint{2.244591in}{1.633461in}}%
\pgfpathlineto{\pgfqpoint{2.294252in}{1.649827in}}%
\pgfpathlineto{\pgfqpoint{2.343913in}{1.710208in}}%
\pgfpathlineto{\pgfqpoint{2.393574in}{1.659544in}}%
\pgfpathlineto{\pgfqpoint{2.443236in}{1.612650in}}%
\pgfpathlineto{\pgfqpoint{2.492897in}{1.602243in}}%
\pgfpathlineto{\pgfqpoint{2.542558in}{1.616603in}}%
\pgfpathlineto{\pgfqpoint{2.592219in}{1.686874in}}%
\pgfpathlineto{\pgfqpoint{2.641881in}{1.662286in}}%
\pgfpathlineto{\pgfqpoint{2.691542in}{1.791755in}}%
\pgfpathlineto{\pgfqpoint{2.741203in}{1.608303in}}%
\pgfpathlineto{\pgfqpoint{2.890187in}{1.650250in}}%
\pgfpathlineto{\pgfqpoint{3.039171in}{1.635521in}}%
\pgfpathlineto{\pgfqpoint{3.138493in}{1.733601in}}%
\pgfpathlineto{\pgfqpoint{3.188155in}{1.748964in}}%
\pgfpathlineto{\pgfqpoint{3.436461in}{1.744324in}}%
\pgfusepath{stroke}%
\end{pgfscope}%
\begin{pgfscope}%
\pgfpathrectangle{\pgfqpoint{0.588387in}{0.521603in}}{\pgfqpoint{3.660036in}{2.220246in}}%
\pgfusepath{clip}%
\pgfsetrectcap%
\pgfsetroundjoin%
\pgfsetlinewidth{1.505625pt}%
\pgfsetstrokecolor{currentstroke5}%
\pgfsetdash{}{0pt}%
\pgfpathmoveto{\pgfqpoint{0.754752in}{0.757918in}}%
\pgfpathlineto{\pgfqpoint{0.804414in}{0.759168in}}%
\pgfpathlineto{\pgfqpoint{0.854075in}{0.829206in}}%
\pgfpathlineto{\pgfqpoint{0.903736in}{0.855157in}}%
\pgfpathlineto{\pgfqpoint{0.953398in}{0.924649in}}%
\pgfpathlineto{\pgfqpoint{1.003059in}{0.969099in}}%
\pgfpathlineto{\pgfqpoint{1.052720in}{1.086816in}}%
\pgfpathlineto{\pgfqpoint{1.102381in}{1.195512in}}%
\pgfpathlineto{\pgfqpoint{1.152043in}{1.197624in}}%
\pgfpathlineto{\pgfqpoint{1.201704in}{1.243209in}}%
\pgfpathlineto{\pgfqpoint{1.251365in}{1.358972in}}%
\pgfpathlineto{\pgfqpoint{1.301026in}{1.333575in}}%
\pgfpathlineto{\pgfqpoint{1.350688in}{1.420522in}}%
\pgfpathlineto{\pgfqpoint{1.400349in}{1.471798in}}%
\pgfpathlineto{\pgfqpoint{1.450010in}{1.525162in}}%
\pgfpathlineto{\pgfqpoint{1.499672in}{1.515478in}}%
\pgfpathlineto{\pgfqpoint{1.549333in}{1.578862in}}%
\pgfpathlineto{\pgfqpoint{1.598994in}{1.510278in}}%
\pgfpathlineto{\pgfqpoint{1.648655in}{1.510296in}}%
\pgfpathlineto{\pgfqpoint{1.698317in}{1.607715in}}%
\pgfpathlineto{\pgfqpoint{1.747978in}{1.694736in}}%
\pgfpathlineto{\pgfqpoint{1.797639in}{1.799294in}}%
\pgfpathlineto{\pgfqpoint{1.847300in}{2.116476in}}%
\pgfpathlineto{\pgfqpoint{1.896962in}{1.547662in}}%
\pgfpathlineto{\pgfqpoint{1.946623in}{1.572524in}}%
\pgfpathlineto{\pgfqpoint{1.996284in}{1.593664in}}%
\pgfpathlineto{\pgfqpoint{2.045945in}{1.774399in}}%
\pgfpathlineto{\pgfqpoint{2.095607in}{1.593972in}}%
\pgfpathlineto{\pgfqpoint{2.145268in}{1.650503in}}%
\pgfpathlineto{\pgfqpoint{2.194929in}{1.730570in}}%
\pgfpathlineto{\pgfqpoint{2.244591in}{1.666644in}}%
\pgfpathlineto{\pgfqpoint{2.294252in}{1.665593in}}%
\pgfpathlineto{\pgfqpoint{2.343913in}{1.677847in}}%
\pgfpathlineto{\pgfqpoint{2.393574in}{1.696038in}}%
\pgfpathlineto{\pgfqpoint{2.443236in}{1.620870in}}%
\pgfpathlineto{\pgfqpoint{2.492897in}{1.603534in}}%
\pgfpathlineto{\pgfqpoint{2.542558in}{1.580977in}}%
\pgfpathlineto{\pgfqpoint{2.592219in}{1.685610in}}%
\pgfpathlineto{\pgfqpoint{2.641881in}{1.695703in}}%
\pgfpathlineto{\pgfqpoint{2.691542in}{1.741962in}}%
\pgfpathlineto{\pgfqpoint{2.741203in}{1.540685in}}%
\pgfpathlineto{\pgfqpoint{2.890187in}{1.617504in}}%
\pgfpathlineto{\pgfqpoint{3.039171in}{1.684763in}}%
\pgfpathlineto{\pgfqpoint{3.138493in}{1.724115in}}%
\pgfpathlineto{\pgfqpoint{3.188155in}{1.805616in}}%
\pgfpathlineto{\pgfqpoint{3.436461in}{1.744324in}}%
\pgfusepath{stroke}%
\end{pgfscope}%
\begin{pgfscope}%
\pgfpathrectangle{\pgfqpoint{0.588387in}{0.521603in}}{\pgfqpoint{3.660036in}{2.220246in}}%
\pgfusepath{clip}%
\pgfsetrectcap%
\pgfsetroundjoin%
\pgfsetlinewidth{1.505625pt}%
\pgfsetstrokecolor{currentstroke6}%
\pgfsetdash{}{0pt}%
\pgfpathmoveto{\pgfqpoint{0.754752in}{0.757918in}}%
\pgfpathlineto{\pgfqpoint{0.804414in}{0.759165in}}%
\pgfpathlineto{\pgfqpoint{0.854075in}{0.828159in}}%
\pgfpathlineto{\pgfqpoint{0.903736in}{0.852822in}}%
\pgfpathlineto{\pgfqpoint{0.953398in}{0.926983in}}%
\pgfpathlineto{\pgfqpoint{1.003059in}{0.964655in}}%
\pgfpathlineto{\pgfqpoint{1.052720in}{1.089243in}}%
\pgfpathlineto{\pgfqpoint{1.102381in}{1.132351in}}%
\pgfpathlineto{\pgfqpoint{1.152043in}{1.196753in}}%
\pgfpathlineto{\pgfqpoint{1.201704in}{1.229268in}}%
\pgfpathlineto{\pgfqpoint{1.251365in}{1.333270in}}%
\pgfpathlineto{\pgfqpoint{1.301026in}{1.322933in}}%
\pgfpathlineto{\pgfqpoint{1.350688in}{1.424049in}}%
\pgfpathlineto{\pgfqpoint{1.400349in}{1.448242in}}%
\pgfpathlineto{\pgfqpoint{1.450010in}{1.516472in}}%
\pgfpathlineto{\pgfqpoint{1.499672in}{1.516466in}}%
\pgfpathlineto{\pgfqpoint{1.549333in}{1.438163in}}%
\pgfpathlineto{\pgfqpoint{1.598994in}{1.490788in}}%
\pgfpathlineto{\pgfqpoint{1.648655in}{1.514217in}}%
\pgfpathlineto{\pgfqpoint{1.698317in}{1.547774in}}%
\pgfpathlineto{\pgfqpoint{1.747978in}{1.607058in}}%
\pgfpathlineto{\pgfqpoint{1.797639in}{1.704943in}}%
\pgfpathlineto{\pgfqpoint{1.847300in}{1.530010in}}%
\pgfpathlineto{\pgfqpoint{1.896962in}{1.497097in}}%
\pgfpathlineto{\pgfqpoint{1.946623in}{1.534552in}}%
\pgfpathlineto{\pgfqpoint{1.996284in}{1.621653in}}%
\pgfpathlineto{\pgfqpoint{2.045945in}{1.656919in}}%
\pgfpathlineto{\pgfqpoint{2.095607in}{1.603534in}}%
\pgfpathlineto{\pgfqpoint{2.145268in}{1.568294in}}%
\pgfpathlineto{\pgfqpoint{2.194929in}{1.687054in}}%
\pgfpathlineto{\pgfqpoint{2.244591in}{1.668036in}}%
\pgfpathlineto{\pgfqpoint{2.294252in}{1.611005in}}%
\pgfpathlineto{\pgfqpoint{2.343913in}{1.760537in}}%
\pgfpathlineto{\pgfqpoint{2.393574in}{1.838278in}}%
\pgfpathlineto{\pgfqpoint{2.443236in}{1.596212in}}%
\pgfpathlineto{\pgfqpoint{2.492897in}{1.578139in}}%
\pgfpathlineto{\pgfqpoint{2.542558in}{1.634425in}}%
\pgfpathlineto{\pgfqpoint{2.592219in}{1.629698in}}%
\pgfpathlineto{\pgfqpoint{2.641881in}{1.679381in}}%
\pgfpathlineto{\pgfqpoint{2.691542in}{1.673173in}}%
\pgfpathlineto{\pgfqpoint{2.741203in}{1.749270in}}%
\pgfpathlineto{\pgfqpoint{2.890187in}{1.671362in}}%
\pgfpathlineto{\pgfqpoint{3.039171in}{1.689379in}}%
\pgfpathlineto{\pgfqpoint{3.138493in}{1.690206in}}%
\pgfpathlineto{\pgfqpoint{3.188155in}{1.707958in}}%
\pgfpathlineto{\pgfqpoint{3.436461in}{1.779206in}}%
\pgfusepath{stroke}%
\end{pgfscope}%
\begin{pgfscope}%
\pgfpathrectangle{\pgfqpoint{0.588387in}{0.521603in}}{\pgfqpoint{3.660036in}{2.220246in}}%
\pgfusepath{clip}%
\pgfsetrectcap%
\pgfsetroundjoin%
\pgfsetlinewidth{1.505625pt}%
\pgfsetstrokecolor{currentstroke7}%
\pgfsetdash{}{0pt}%
\pgfpathmoveto{\pgfqpoint{0.754752in}{0.757918in}}%
\pgfpathlineto{\pgfqpoint{0.804414in}{0.759164in}}%
\pgfpathlineto{\pgfqpoint{0.854075in}{0.828421in}}%
\pgfpathlineto{\pgfqpoint{0.903736in}{0.852923in}}%
\pgfpathlineto{\pgfqpoint{0.953398in}{0.927837in}}%
\pgfpathlineto{\pgfqpoint{1.003059in}{0.963852in}}%
\pgfpathlineto{\pgfqpoint{1.052720in}{1.089609in}}%
\pgfpathlineto{\pgfqpoint{1.102381in}{1.132449in}}%
\pgfpathlineto{\pgfqpoint{1.152043in}{1.196551in}}%
\pgfpathlineto{\pgfqpoint{1.201704in}{1.229111in}}%
\pgfpathlineto{\pgfqpoint{1.251365in}{1.386711in}}%
\pgfpathlineto{\pgfqpoint{1.301026in}{1.364135in}}%
\pgfpathlineto{\pgfqpoint{1.350688in}{1.438643in}}%
\pgfpathlineto{\pgfqpoint{1.400349in}{1.511951in}}%
\pgfpathlineto{\pgfqpoint{1.450010in}{1.590047in}}%
\pgfpathlineto{\pgfqpoint{1.499672in}{1.590036in}}%
\pgfpathlineto{\pgfqpoint{1.549333in}{1.614649in}}%
\pgfpathlineto{\pgfqpoint{1.598994in}{1.653111in}}%
\pgfpathlineto{\pgfqpoint{1.648655in}{1.638768in}}%
\pgfpathlineto{\pgfqpoint{1.698317in}{1.778544in}}%
\pgfpathlineto{\pgfqpoint{1.747978in}{1.742729in}}%
\pgfpathlineto{\pgfqpoint{1.797639in}{1.954815in}}%
\pgfpathlineto{\pgfqpoint{1.847300in}{1.704966in}}%
\pgfpathlineto{\pgfqpoint{1.896962in}{1.535739in}}%
\pgfpathlineto{\pgfqpoint{1.946623in}{1.605437in}}%
\pgfpathlineto{\pgfqpoint{1.996284in}{1.731607in}}%
\pgfpathlineto{\pgfqpoint{2.045945in}{1.799323in}}%
\pgfpathlineto{\pgfqpoint{2.095607in}{1.580029in}}%
\pgfpathlineto{\pgfqpoint{2.145268in}{1.570211in}}%
\pgfpathlineto{\pgfqpoint{2.194929in}{1.731229in}}%
\pgfpathlineto{\pgfqpoint{2.244591in}{1.623405in}}%
\pgfpathlineto{\pgfqpoint{2.294252in}{2.124783in}}%
\pgfpathlineto{\pgfqpoint{2.343913in}{1.750487in}}%
\pgfpathlineto{\pgfqpoint{2.393574in}{1.807438in}}%
\pgfpathlineto{\pgfqpoint{2.443236in}{1.618301in}}%
\pgfpathlineto{\pgfqpoint{2.492897in}{1.653267in}}%
\pgfpathlineto{\pgfqpoint{2.542558in}{1.661011in}}%
\pgfpathlineto{\pgfqpoint{2.592219in}{1.529729in}}%
\pgfpathlineto{\pgfqpoint{2.641881in}{1.626863in}}%
\pgfpathlineto{\pgfqpoint{2.691542in}{1.593972in}}%
\pgfpathlineto{\pgfqpoint{2.741203in}{2.032618in}}%
\pgfpathlineto{\pgfqpoint{2.890187in}{1.825291in}}%
\pgfpathlineto{\pgfqpoint{3.138493in}{1.732052in}}%
\pgfpathlineto{\pgfqpoint{3.188155in}{1.786908in}}%
\pgfpathlineto{\pgfqpoint{3.436461in}{1.761686in}}%
\pgfusepath{stroke}%
\end{pgfscope}%
\begin{pgfscope}%
\pgfpathrectangle{\pgfqpoint{0.588387in}{0.521603in}}{\pgfqpoint{3.660036in}{2.220246in}}%
\pgfusepath{clip}%
\pgfsetrectcap%
\pgfsetroundjoin%
\pgfsetlinewidth{1.505625pt}%
\definecolor{currentstroke}{rgb}{0.498039,0.498039,0.498039}%
\pgfsetstrokecolor{currentstroke}%
\pgfsetdash{}{0pt}%
\pgfpathmoveto{\pgfqpoint{0.754752in}{0.757918in}}%
\pgfpathlineto{\pgfqpoint{0.804414in}{0.787760in}}%
\pgfpathlineto{\pgfqpoint{0.854075in}{0.847726in}}%
\pgfpathlineto{\pgfqpoint{0.903736in}{0.877055in}}%
\pgfpathlineto{\pgfqpoint{0.953398in}{0.941855in}}%
\pgfpathlineto{\pgfqpoint{1.003059in}{0.985600in}}%
\pgfpathlineto{\pgfqpoint{1.052720in}{1.115139in}}%
\pgfpathlineto{\pgfqpoint{1.102381in}{1.206716in}}%
\pgfpathlineto{\pgfqpoint{1.152043in}{1.215335in}}%
\pgfpathlineto{\pgfqpoint{1.201704in}{1.229538in}}%
\pgfpathlineto{\pgfqpoint{1.251365in}{1.351206in}}%
\pgfpathlineto{\pgfqpoint{1.301026in}{1.377617in}}%
\pgfpathlineto{\pgfqpoint{1.350688in}{1.468195in}}%
\pgfpathlineto{\pgfqpoint{1.400349in}{1.468277in}}%
\pgfpathlineto{\pgfqpoint{1.450010in}{1.540583in}}%
\pgfpathlineto{\pgfqpoint{1.499672in}{1.564872in}}%
\pgfpathlineto{\pgfqpoint{1.549333in}{1.494456in}}%
\pgfpathlineto{\pgfqpoint{1.598994in}{1.491037in}}%
\pgfpathlineto{\pgfqpoint{1.648655in}{1.556865in}}%
\pgfpathlineto{\pgfqpoint{1.698317in}{1.595125in}}%
\pgfpathlineto{\pgfqpoint{1.747978in}{1.592031in}}%
\pgfpathlineto{\pgfqpoint{1.797639in}{1.790614in}}%
\pgfpathlineto{\pgfqpoint{1.847300in}{1.768208in}}%
\pgfpathlineto{\pgfqpoint{1.896962in}{1.585805in}}%
\pgfpathlineto{\pgfqpoint{1.946623in}{1.564637in}}%
\pgfpathlineto{\pgfqpoint{1.996284in}{1.623017in}}%
\pgfpathlineto{\pgfqpoint{2.045945in}{1.680104in}}%
\pgfpathlineto{\pgfqpoint{2.095607in}{1.679381in}}%
\pgfpathlineto{\pgfqpoint{2.145268in}{1.639683in}}%
\pgfpathlineto{\pgfqpoint{2.194929in}{1.703225in}}%
\pgfpathlineto{\pgfqpoint{2.244591in}{1.676523in}}%
\pgfpathlineto{\pgfqpoint{2.294252in}{1.651848in}}%
\pgfpathlineto{\pgfqpoint{2.343913in}{1.751093in}}%
\pgfpathlineto{\pgfqpoint{2.393574in}{1.897985in}}%
\pgfpathlineto{\pgfqpoint{2.443236in}{1.648469in}}%
\pgfpathlineto{\pgfqpoint{2.492897in}{1.664182in}}%
\pgfpathlineto{\pgfqpoint{2.542558in}{1.693584in}}%
\pgfpathlineto{\pgfqpoint{2.592219in}{1.688547in}}%
\pgfpathlineto{\pgfqpoint{2.641881in}{1.750790in}}%
\pgfpathlineto{\pgfqpoint{2.691542in}{1.826220in}}%
\pgfpathlineto{\pgfqpoint{2.741203in}{1.631937in}}%
\pgfpathlineto{\pgfqpoint{2.890187in}{1.676586in}}%
\pgfpathlineto{\pgfqpoint{3.039171in}{1.804814in}}%
\pgfpathlineto{\pgfqpoint{3.138493in}{1.704535in}}%
\pgfpathlineto{\pgfqpoint{3.188155in}{1.867613in}}%
\pgfpathlineto{\pgfqpoint{3.436461in}{1.948417in}}%
\pgfpathlineto{\pgfqpoint{3.486122in}{1.827655in}}%
\pgfpathlineto{\pgfqpoint{3.585445in}{1.825601in}}%
\pgfpathlineto{\pgfqpoint{3.784090in}{1.937783in}}%
\pgfusepath{stroke}%
\end{pgfscope}%
\begin{pgfscope}%
\pgfpathrectangle{\pgfqpoint{0.588387in}{0.521603in}}{\pgfqpoint{3.660036in}{2.220246in}}%
\pgfusepath{clip}%
\pgfsetrectcap%
\pgfsetroundjoin%
\pgfsetlinewidth{1.505625pt}%
\definecolor{currentstroke}{rgb}{0.737255,0.741176,0.133333}%
\pgfsetstrokecolor{currentstroke}%
\pgfsetdash{}{0pt}%
\pgfpathmoveto{\pgfqpoint{0.754752in}{0.757918in}}%
\pgfpathlineto{\pgfqpoint{0.804414in}{0.787790in}}%
\pgfpathlineto{\pgfqpoint{0.854075in}{0.847562in}}%
\pgfpathlineto{\pgfqpoint{0.903736in}{0.877033in}}%
\pgfpathlineto{\pgfqpoint{0.953398in}{0.940831in}}%
\pgfpathlineto{\pgfqpoint{1.003059in}{0.984923in}}%
\pgfpathlineto{\pgfqpoint{1.052720in}{1.114915in}}%
\pgfpathlineto{\pgfqpoint{1.102381in}{1.205834in}}%
\pgfpathlineto{\pgfqpoint{1.152043in}{1.216368in}}%
\pgfpathlineto{\pgfqpoint{1.201704in}{1.230664in}}%
\pgfpathlineto{\pgfqpoint{1.251365in}{1.363206in}}%
\pgfpathlineto{\pgfqpoint{1.301026in}{1.380930in}}%
\pgfpathlineto{\pgfqpoint{1.350688in}{1.483587in}}%
\pgfpathlineto{\pgfqpoint{1.400349in}{1.500912in}}%
\pgfpathlineto{\pgfqpoint{1.450010in}{1.554608in}}%
\pgfpathlineto{\pgfqpoint{1.499672in}{1.587631in}}%
\pgfpathlineto{\pgfqpoint{1.549333in}{1.500887in}}%
\pgfpathlineto{\pgfqpoint{1.598994in}{1.518219in}}%
\pgfpathlineto{\pgfqpoint{1.648655in}{1.704337in}}%
\pgfpathlineto{\pgfqpoint{1.698317in}{1.619025in}}%
\pgfpathlineto{\pgfqpoint{1.747978in}{1.756102in}}%
\pgfpathlineto{\pgfqpoint{1.797639in}{1.819690in}}%
\pgfpathlineto{\pgfqpoint{1.847300in}{1.966805in}}%
\pgfpathlineto{\pgfqpoint{1.896962in}{1.737638in}}%
\pgfpathlineto{\pgfqpoint{1.946623in}{1.777490in}}%
\pgfpathlineto{\pgfqpoint{1.996284in}{1.954041in}}%
\pgfpathlineto{\pgfqpoint{2.045945in}{1.904080in}}%
\pgfpathlineto{\pgfqpoint{2.095607in}{1.662286in}}%
\pgfpathlineto{\pgfqpoint{2.145268in}{1.753165in}}%
\pgfpathlineto{\pgfqpoint{2.194929in}{1.850837in}}%
\pgfpathlineto{\pgfqpoint{2.244591in}{1.675190in}}%
\pgfpathlineto{\pgfqpoint{2.294252in}{1.858116in}}%
\pgfpathlineto{\pgfqpoint{2.343913in}{1.707958in}}%
\pgfpathlineto{\pgfqpoint{2.393574in}{1.659406in}}%
\pgfpathlineto{\pgfqpoint{2.443236in}{1.721927in}}%
\pgfpathlineto{\pgfqpoint{2.492897in}{1.636882in}}%
\pgfpathlineto{\pgfqpoint{2.542558in}{1.684763in}}%
\pgfpathlineto{\pgfqpoint{2.641881in}{1.707455in}}%
\pgfpathlineto{\pgfqpoint{2.691542in}{1.939115in}}%
\pgfpathlineto{\pgfqpoint{2.890187in}{1.812139in}}%
\pgfpathlineto{\pgfqpoint{3.039171in}{1.727328in}}%
\pgfpathlineto{\pgfqpoint{3.138493in}{1.809691in}}%
\pgfpathlineto{\pgfqpoint{3.188155in}{1.746194in}}%
\pgfpathlineto{\pgfqpoint{3.436461in}{1.778156in}}%
\pgfpathlineto{\pgfqpoint{3.486122in}{1.874002in}}%
\pgfpathlineto{\pgfqpoint{3.784090in}{1.892150in}}%
\pgfusepath{stroke}%
\end{pgfscope}%
\begin{pgfscope}%
\pgfsetrectcap%
\pgfsetmiterjoin%
\pgfsetlinewidth{0.803000pt}%
\definecolor{currentstroke}{rgb}{0.000000,0.000000,0.000000}%
\pgfsetstrokecolor{currentstroke}%
\pgfsetdash{}{0pt}%
\pgfpathmoveto{\pgfqpoint{0.588387in}{0.521603in}}%
\pgfpathlineto{\pgfqpoint{0.588387in}{2.741849in}}%
\pgfusepath{stroke}%
\end{pgfscope}%
\begin{pgfscope}%
\pgfsetrectcap%
\pgfsetmiterjoin%
\pgfsetlinewidth{0.803000pt}%
\definecolor{currentstroke}{rgb}{0.000000,0.000000,0.000000}%
\pgfsetstrokecolor{currentstroke}%
\pgfsetdash{}{0pt}%
\pgfpathmoveto{\pgfqpoint{4.248423in}{0.521603in}}%
\pgfpathlineto{\pgfqpoint{4.248423in}{2.741849in}}%
\pgfusepath{stroke}%
\end{pgfscope}%
\begin{pgfscope}%
\pgfsetrectcap%
\pgfsetmiterjoin%
\pgfsetlinewidth{0.803000pt}%
\definecolor{currentstroke}{rgb}{0.000000,0.000000,0.000000}%
\pgfsetstrokecolor{currentstroke}%
\pgfsetdash{}{0pt}%
\pgfpathmoveto{\pgfqpoint{0.588387in}{0.521603in}}%
\pgfpathlineto{\pgfqpoint{4.248423in}{0.521603in}}%
\pgfusepath{stroke}%
\end{pgfscope}%
\begin{pgfscope}%
\pgfsetrectcap%
\pgfsetmiterjoin%
\pgfsetlinewidth{0.803000pt}%
\definecolor{currentstroke}{rgb}{0.000000,0.000000,0.000000}%
\pgfsetstrokecolor{currentstroke}%
\pgfsetdash{}{0pt}%
\pgfpathmoveto{\pgfqpoint{0.588387in}{2.741849in}}%
\pgfpathlineto{\pgfqpoint{4.248423in}{2.741849in}}%
\pgfusepath{stroke}%
\end{pgfscope}%
\begin{pgfscope}%
\pgfsetbuttcap%
\pgfsetmiterjoin%
\definecolor{currentfill}{rgb}{1.000000,1.000000,1.000000}%
\pgfsetfillcolor{currentfill}%
\pgfsetfillopacity{0.800000}%
\pgfsetlinewidth{1.003750pt}%
\definecolor{currentstroke}{rgb}{0.800000,0.800000,0.800000}%
\pgfsetstrokecolor{currentstroke}%
\pgfsetstrokeopacity{0.800000}%
\pgfsetdash{}{0pt}%
\pgfpathmoveto{\pgfqpoint{4.365089in}{0.379025in}}%
\pgfpathlineto{\pgfqpoint{8.251043in}{0.379025in}}%
\pgfpathquadraticcurveto{\pgfqpoint{8.284376in}{0.379025in}}{\pgfqpoint{8.284376in}{0.412359in}}%
\pgfpathlineto{\pgfqpoint{8.284376in}{2.625183in}}%
\pgfpathquadraticcurveto{\pgfqpoint{8.284376in}{2.658516in}}{\pgfqpoint{8.251043in}{2.658516in}}%
\pgfpathlineto{\pgfqpoint{4.365089in}{2.658516in}}%
\pgfpathquadraticcurveto{\pgfqpoint{4.331756in}{2.658516in}}{\pgfqpoint{4.331756in}{2.625183in}}%
\pgfpathlineto{\pgfqpoint{4.331756in}{0.412359in}}%
\pgfpathquadraticcurveto{\pgfqpoint{4.331756in}{0.379025in}}{\pgfqpoint{4.365089in}{0.379025in}}%
\pgfpathlineto{\pgfqpoint{4.365089in}{0.379025in}}%
\pgfpathclose%
\pgfusepath{stroke,fill}%
\end{pgfscope}%
\begin{pgfscope}%
\pgfsetrectcap%
\pgfsetroundjoin%
\pgfsetlinewidth{1.505625pt}%
\pgfsetstrokecolor{currentstroke3}%
\pgfsetdash{}{0pt}%
\pgfpathmoveto{\pgfqpoint{4.398423in}{2.523555in}}%
\pgfpathlineto{\pgfqpoint{4.565089in}{2.523555in}}%
\pgfpathlineto{\pgfqpoint{4.731756in}{2.523555in}}%
\pgfusepath{stroke}%
\end{pgfscope}%
\begin{pgfscope}%
\definecolor{textcolor}{rgb}{0.000000,0.000000,0.000000}%
\pgfsetstrokecolor{textcolor}%
\pgfsetfillcolor{textcolor}%
\pgftext[x=4.865089in,y=2.465222in,left,base]{\color{textcolor}{\rmfamily\fontsize{12.000000}{14.400000}\selectfont\catcode`\^=\active\def^{\ifmmode\sp\else\^{}\fi}\catcode`\%=\active\def%{\%}\NaiveCycles{}}}%
\end{pgfscope}%
\begin{pgfscope}%
\pgfsetrectcap%
\pgfsetroundjoin%
\pgfsetlinewidth{1.505625pt}%
\pgfsetstrokecolor{currentstroke1}%
\pgfsetdash{}{0pt}%
\pgfpathmoveto{\pgfqpoint{4.398423in}{2.278926in}}%
\pgfpathlineto{\pgfqpoint{4.565089in}{2.278926in}}%
\pgfpathlineto{\pgfqpoint{4.731756in}{2.278926in}}%
\pgfusepath{stroke}%
\end{pgfscope}%
\begin{pgfscope}%
\definecolor{textcolor}{rgb}{0.000000,0.000000,0.000000}%
\pgfsetstrokecolor{textcolor}%
\pgfsetfillcolor{textcolor}%
\pgftext[x=4.865089in,y=2.220593in,left,base]{\color{textcolor}{\rmfamily\fontsize{12.000000}{14.400000}\selectfont\catcode`\^=\active\def^{\ifmmode\sp\else\^{}\fi}\catcode`\%=\active\def%{\%}\CyclesMatchChunks{} \& \MergeLinear{}}}%
\end{pgfscope}%
\begin{pgfscope}%
\pgfsetrectcap%
\pgfsetroundjoin%
\pgfsetlinewidth{1.505625pt}%
\pgfsetstrokecolor{currentstroke2}%
\pgfsetdash{}{0pt}%
\pgfpathmoveto{\pgfqpoint{4.398423in}{2.029659in}}%
\pgfpathlineto{\pgfqpoint{4.565089in}{2.029659in}}%
\pgfpathlineto{\pgfqpoint{4.731756in}{2.029659in}}%
\pgfusepath{stroke}%
\end{pgfscope}%
\begin{pgfscope}%
\definecolor{textcolor}{rgb}{0.000000,0.000000,0.000000}%
\pgfsetstrokecolor{textcolor}%
\pgfsetfillcolor{textcolor}%
\pgftext[x=4.865089in,y=1.971325in,left,base]{\color{textcolor}{\rmfamily\fontsize{12.000000}{14.400000}\selectfont\catcode`\^=\active\def^{\ifmmode\sp\else\^{}\fi}\catcode`\%=\active\def%{\%}\CyclesMatchChunks{} \& \SharedVertices{}}}%
\end{pgfscope}%
\begin{pgfscope}%
\pgfsetrectcap%
\pgfsetroundjoin%
\pgfsetlinewidth{1.505625pt}%
\pgfsetstrokecolor{currentstroke4}%
\pgfsetdash{}{0pt}%
\pgfpathmoveto{\pgfqpoint{4.398423in}{1.780391in}}%
\pgfpathlineto{\pgfqpoint{4.565089in}{1.780391in}}%
\pgfpathlineto{\pgfqpoint{4.731756in}{1.780391in}}%
\pgfusepath{stroke}%
\end{pgfscope}%
\begin{pgfscope}%
\definecolor{textcolor}{rgb}{0.000000,0.000000,0.000000}%
\pgfsetstrokecolor{textcolor}%
\pgfsetfillcolor{textcolor}%
\pgftext[x=4.865089in,y=1.722058in,left,base]{\color{textcolor}{\rmfamily\fontsize{12.000000}{14.400000}\selectfont\catcode`\^=\active\def^{\ifmmode\sp\else\^{}\fi}\catcode`\%=\active\def%{\%}\Neighbors{} \& \MergeLinear{}}}%
\end{pgfscope}%
\begin{pgfscope}%
\pgfsetrectcap%
\pgfsetroundjoin%
\pgfsetlinewidth{1.505625pt}%
\pgfsetstrokecolor{currentstroke5}%
\pgfsetdash{}{0pt}%
\pgfpathmoveto{\pgfqpoint{4.398423in}{1.535763in}}%
\pgfpathlineto{\pgfqpoint{4.565089in}{1.535763in}}%
\pgfpathlineto{\pgfqpoint{4.731756in}{1.535763in}}%
\pgfusepath{stroke}%
\end{pgfscope}%
\begin{pgfscope}%
\definecolor{textcolor}{rgb}{0.000000,0.000000,0.000000}%
\pgfsetstrokecolor{textcolor}%
\pgfsetfillcolor{textcolor}%
\pgftext[x=4.865089in,y=1.477429in,left,base]{\color{textcolor}{\rmfamily\fontsize{12.000000}{14.400000}\selectfont\catcode`\^=\active\def^{\ifmmode\sp\else\^{}\fi}\catcode`\%=\active\def%{\%}\Neighbors{} \& \SharedVertices{}}}%
\end{pgfscope}%
\begin{pgfscope}%
\pgfsetrectcap%
\pgfsetroundjoin%
\pgfsetlinewidth{1.505625pt}%
\pgfsetstrokecolor{currentstroke6}%
\pgfsetdash{}{0pt}%
\pgfpathmoveto{\pgfqpoint{4.398423in}{1.286495in}}%
\pgfpathlineto{\pgfqpoint{4.565089in}{1.286495in}}%
\pgfpathlineto{\pgfqpoint{4.731756in}{1.286495in}}%
\pgfusepath{stroke}%
\end{pgfscope}%
\begin{pgfscope}%
\definecolor{textcolor}{rgb}{0.000000,0.000000,0.000000}%
\pgfsetstrokecolor{textcolor}%
\pgfsetfillcolor{textcolor}%
\pgftext[x=4.865089in,y=1.228162in,left,base]{\color{textcolor}{\rmfamily\fontsize{12.000000}{14.400000}\selectfont\catcode`\^=\active\def^{\ifmmode\sp\else\^{}\fi}\catcode`\%=\active\def%{\%}\NeighborsDegree{} \& \MergeLinear{}}}%
\end{pgfscope}%
\begin{pgfscope}%
\pgfsetrectcap%
\pgfsetroundjoin%
\pgfsetlinewidth{1.505625pt}%
\pgfsetstrokecolor{currentstroke7}%
\pgfsetdash{}{0pt}%
\pgfpathmoveto{\pgfqpoint{4.398423in}{1.037228in}}%
\pgfpathlineto{\pgfqpoint{4.565089in}{1.037228in}}%
\pgfpathlineto{\pgfqpoint{4.731756in}{1.037228in}}%
\pgfusepath{stroke}%
\end{pgfscope}%
\begin{pgfscope}%
\definecolor{textcolor}{rgb}{0.000000,0.000000,0.000000}%
\pgfsetstrokecolor{textcolor}%
\pgfsetfillcolor{textcolor}%
\pgftext[x=4.865089in,y=0.978895in,left,base]{\color{textcolor}{\rmfamily\fontsize{12.000000}{14.400000}\selectfont\catcode`\^=\active\def^{\ifmmode\sp\else\^{}\fi}\catcode`\%=\active\def%{\%}\NeighborsDegree{} \& \SharedVertices{}}}%
\end{pgfscope}%
\begin{pgfscope}%
\pgfsetrectcap%
\pgfsetroundjoin%
\pgfsetlinewidth{1.505625pt}%
\definecolor{currentstroke}{rgb}{0.498039,0.498039,0.498039}%
\pgfsetstrokecolor{currentstroke}%
\pgfsetdash{}{0pt}%
\pgfpathmoveto{\pgfqpoint{4.398423in}{0.787961in}}%
\pgfpathlineto{\pgfqpoint{4.565089in}{0.787961in}}%
\pgfpathlineto{\pgfqpoint{4.731756in}{0.787961in}}%
\pgfusepath{stroke}%
\end{pgfscope}%
\begin{pgfscope}%
\definecolor{textcolor}{rgb}{0.000000,0.000000,0.000000}%
\pgfsetstrokecolor{textcolor}%
\pgfsetfillcolor{textcolor}%
\pgftext[x=4.865089in,y=0.729627in,left,base]{\color{textcolor}{\rmfamily\fontsize{12.000000}{14.400000}\selectfont\catcode`\^=\active\def^{\ifmmode\sp\else\^{}\fi}\catcode`\%=\active\def%{\%}\None{} \& \MergeLinear{}}}%
\end{pgfscope}%
\begin{pgfscope}%
\pgfsetrectcap%
\pgfsetroundjoin%
\pgfsetlinewidth{1.505625pt}%
\definecolor{currentstroke}{rgb}{0.737255,0.741176,0.133333}%
\pgfsetstrokecolor{currentstroke}%
\pgfsetdash{}{0pt}%
\pgfpathmoveto{\pgfqpoint{4.398423in}{0.543332in}}%
\pgfpathlineto{\pgfqpoint{4.565089in}{0.543332in}}%
\pgfpathlineto{\pgfqpoint{4.731756in}{0.543332in}}%
\pgfusepath{stroke}%
\end{pgfscope}%
\begin{pgfscope}%
\definecolor{textcolor}{rgb}{0.000000,0.000000,0.000000}%
\pgfsetstrokecolor{textcolor}%
\pgfsetfillcolor{textcolor}%
\pgftext[x=4.865089in,y=0.484999in,left,base]{\color{textcolor}{\rmfamily\fontsize{12.000000}{14.400000}\selectfont\catcode`\^=\active\def^{\ifmmode\sp\else\^{}\fi}\catcode`\%=\active\def%{\%}\None{} \& \SharedVertices{}}}%
\end{pgfscope}%
\end{pgfpicture}%
\makeatother%
\endgroup%
}
% 	\caption[Checks performed for globally rigid graphs (some).]{
% 		The number of checks performed to find all NAC-colorings for globally rigid graphs.}%
% 	\label{fig:graph_globally_rigid_first_checks}
% \end{figure}
\begin{figure}[p]
	\centering
	\scalebox{0.5}{%% Creator: Matplotlib, PGF backend
%%
%% To include the figure in your LaTeX document, write
%%   \input{<filename>.pgf}
%%
%% Make sure the required packages are loaded in your preamble
%%   \usepackage{pgf}
%%
%% Also ensure that all the required font packages are loaded; for instance,
%% the lmodern package is sometimes necessary when using math font.
%%   \usepackage{lmodern}
%%
%% Figures using additional raster images can only be included by \input if
%% they are in the same directory as the main LaTeX file. For loading figures
%% from other directories you can use the `import` package
%%   \usepackage{import}
%%
%% and then include the figures with
%%   \import{<path to file>}{<filename>.pgf}
%%
%% Matplotlib used the following preamble
%%   \def\mathdefault#1{#1}
%%   \everymath=\expandafter{\the\everymath\displaystyle}
%%   \IfFileExists{scrextend.sty}{
%%     \usepackage[fontsize=10.000000pt]{scrextend}
%%   }{
%%     \renewcommand{\normalsize}{\fontsize{10.000000}{12.000000}\selectfont}
%%     \normalsize
%%   }
%%   
%%   \ifdefined\pdftexversion\else  % non-pdftex case.
%%     \usepackage{fontspec}
%%     \setmainfont{DejaVuSans.ttf}[Path=\detokenize{/home/petr/Projects/PyRigi/.venv/lib/python3.12/site-packages/matplotlib/mpl-data/fonts/ttf/}]
%%     \setsansfont{DejaVuSans.ttf}[Path=\detokenize{/home/petr/Projects/PyRigi/.venv/lib/python3.12/site-packages/matplotlib/mpl-data/fonts/ttf/}]
%%     \setmonofont{DejaVuSansMono.ttf}[Path=\detokenize{/home/petr/Projects/PyRigi/.venv/lib/python3.12/site-packages/matplotlib/mpl-data/fonts/ttf/}]
%%   \fi
%%   \makeatletter\@ifpackageloaded{under\Score{}}{}{\usepackage[strings]{under\Score{}}}\makeatother
%%
\begingroup%
\makeatletter%
\begin{pgfpicture}%
\pgfpathrectangle{\pgfpointorigin}{\pgfqpoint{8.384376in}{2.841853in}}%
\pgfusepath{use as bounding box, clip}%
\begin{pgfscope}%
\pgfsetbuttcap%
\pgfsetmiterjoin%
\definecolor{currentfill}{rgb}{1.000000,1.000000,1.000000}%
\pgfsetfillcolor{currentfill}%
\pgfsetlinewidth{0.000000pt}%
\definecolor{currentstroke}{rgb}{1.000000,1.000000,1.000000}%
\pgfsetstrokecolor{currentstroke}%
\pgfsetdash{}{0pt}%
\pgfpathmoveto{\pgfqpoint{0.000000in}{0.000000in}}%
\pgfpathlineto{\pgfqpoint{8.384376in}{0.000000in}}%
\pgfpathlineto{\pgfqpoint{8.384376in}{2.841853in}}%
\pgfpathlineto{\pgfqpoint{0.000000in}{2.841853in}}%
\pgfpathlineto{\pgfqpoint{0.000000in}{0.000000in}}%
\pgfpathclose%
\pgfusepath{fill}%
\end{pgfscope}%
\begin{pgfscope}%
\pgfsetbuttcap%
\pgfsetmiterjoin%
\definecolor{currentfill}{rgb}{1.000000,1.000000,1.000000}%
\pgfsetfillcolor{currentfill}%
\pgfsetlinewidth{0.000000pt}%
\definecolor{currentstroke}{rgb}{0.000000,0.000000,0.000000}%
\pgfsetstrokecolor{currentstroke}%
\pgfsetstrokeopacity{0.000000}%
\pgfsetdash{}{0pt}%
\pgfpathmoveto{\pgfqpoint{0.588387in}{0.521603in}}%
\pgfpathlineto{\pgfqpoint{5.257411in}{0.521603in}}%
\pgfpathlineto{\pgfqpoint{5.257411in}{2.713741in}}%
\pgfpathlineto{\pgfqpoint{0.588387in}{2.713741in}}%
\pgfpathlineto{\pgfqpoint{0.588387in}{0.521603in}}%
\pgfpathclose%
\pgfusepath{fill}%
\end{pgfscope}%
\begin{pgfscope}%
\pgfsetbuttcap%
\pgfsetroundjoin%
\definecolor{currentfill}{rgb}{0.000000,0.000000,0.000000}%
\pgfsetfillcolor{currentfill}%
\pgfsetlinewidth{0.803000pt}%
\definecolor{currentstroke}{rgb}{0.000000,0.000000,0.000000}%
\pgfsetstrokecolor{currentstroke}%
\pgfsetdash{}{0pt}%
\pgfsys@defobject{currentmarker}{\pgfqpoint{0.000000in}{-0.048611in}}{\pgfqpoint{0.000000in}{0.000000in}}{%
\pgfpathmoveto{\pgfqpoint{0.000000in}{0.000000in}}%
\pgfpathlineto{\pgfqpoint{0.000000in}{-0.048611in}}%
\pgfusepath{stroke,fill}%
}%
\begin{pgfscope}%
\pgfsys@transformshift{1.093344in}{0.521603in}%
\pgfsys@useobject{currentmarker}{}%
\end{pgfscope}%
\end{pgfscope}%
\begin{pgfscope}%
\definecolor{textcolor}{rgb}{0.000000,0.000000,0.000000}%
\pgfsetstrokecolor{textcolor}%
\pgfsetfillcolor{textcolor}%
\pgftext[x=1.093344in,y=0.424381in,,top]{\color{textcolor}{\rmfamily\fontsize{10.000000}{12.000000}\selectfont\catcode`\^=\active\def^{\ifmmode\sp\else\^{}\fi}\catcode`\%=\active\def%{\%}$\mathdefault{4}$}}%
\end{pgfscope}%
\begin{pgfscope}%
\pgfsetbuttcap%
\pgfsetroundjoin%
\definecolor{currentfill}{rgb}{0.000000,0.000000,0.000000}%
\pgfsetfillcolor{currentfill}%
\pgfsetlinewidth{0.803000pt}%
\definecolor{currentstroke}{rgb}{0.000000,0.000000,0.000000}%
\pgfsetstrokecolor{currentstroke}%
\pgfsetdash{}{0pt}%
\pgfsys@defobject{currentmarker}{\pgfqpoint{0.000000in}{-0.048611in}}{\pgfqpoint{0.000000in}{0.000000in}}{%
\pgfpathmoveto{\pgfqpoint{0.000000in}{0.000000in}}%
\pgfpathlineto{\pgfqpoint{0.000000in}{-0.048611in}}%
\pgfusepath{stroke,fill}%
}%
\begin{pgfscope}%
\pgfsys@transformshift{1.678802in}{0.521603in}%
\pgfsys@useobject{currentmarker}{}%
\end{pgfscope}%
\end{pgfscope}%
\begin{pgfscope}%
\definecolor{textcolor}{rgb}{0.000000,0.000000,0.000000}%
\pgfsetstrokecolor{textcolor}%
\pgfsetfillcolor{textcolor}%
\pgftext[x=1.678802in,y=0.424381in,,top]{\color{textcolor}{\rmfamily\fontsize{10.000000}{12.000000}\selectfont\catcode`\^=\active\def^{\ifmmode\sp\else\^{}\fi}\catcode`\%=\active\def%{\%}$\mathdefault{8}$}}%
\end{pgfscope}%
\begin{pgfscope}%
\pgfsetbuttcap%
\pgfsetroundjoin%
\definecolor{currentfill}{rgb}{0.000000,0.000000,0.000000}%
\pgfsetfillcolor{currentfill}%
\pgfsetlinewidth{0.803000pt}%
\definecolor{currentstroke}{rgb}{0.000000,0.000000,0.000000}%
\pgfsetstrokecolor{currentstroke}%
\pgfsetdash{}{0pt}%
\pgfsys@defobject{currentmarker}{\pgfqpoint{0.000000in}{-0.048611in}}{\pgfqpoint{0.000000in}{0.000000in}}{%
\pgfpathmoveto{\pgfqpoint{0.000000in}{0.000000in}}%
\pgfpathlineto{\pgfqpoint{0.000000in}{-0.048611in}}%
\pgfusepath{stroke,fill}%
}%
\begin{pgfscope}%
\pgfsys@transformshift{2.264259in}{0.521603in}%
\pgfsys@useobject{currentmarker}{}%
\end{pgfscope}%
\end{pgfscope}%
\begin{pgfscope}%
\definecolor{textcolor}{rgb}{0.000000,0.000000,0.000000}%
\pgfsetstrokecolor{textcolor}%
\pgfsetfillcolor{textcolor}%
\pgftext[x=2.264259in,y=0.424381in,,top]{\color{textcolor}{\rmfamily\fontsize{10.000000}{12.000000}\selectfont\catcode`\^=\active\def^{\ifmmode\sp\else\^{}\fi}\catcode`\%=\active\def%{\%}$\mathdefault{12}$}}%
\end{pgfscope}%
\begin{pgfscope}%
\pgfsetbuttcap%
\pgfsetroundjoin%
\definecolor{currentfill}{rgb}{0.000000,0.000000,0.000000}%
\pgfsetfillcolor{currentfill}%
\pgfsetlinewidth{0.803000pt}%
\definecolor{currentstroke}{rgb}{0.000000,0.000000,0.000000}%
\pgfsetstrokecolor{currentstroke}%
\pgfsetdash{}{0pt}%
\pgfsys@defobject{currentmarker}{\pgfqpoint{0.000000in}{-0.048611in}}{\pgfqpoint{0.000000in}{0.000000in}}{%
\pgfpathmoveto{\pgfqpoint{0.000000in}{0.000000in}}%
\pgfpathlineto{\pgfqpoint{0.000000in}{-0.048611in}}%
\pgfusepath{stroke,fill}%
}%
\begin{pgfscope}%
\pgfsys@transformshift{2.849717in}{0.521603in}%
\pgfsys@useobject{currentmarker}{}%
\end{pgfscope}%
\end{pgfscope}%
\begin{pgfscope}%
\definecolor{textcolor}{rgb}{0.000000,0.000000,0.000000}%
\pgfsetstrokecolor{textcolor}%
\pgfsetfillcolor{textcolor}%
\pgftext[x=2.849717in,y=0.424381in,,top]{\color{textcolor}{\rmfamily\fontsize{10.000000}{12.000000}\selectfont\catcode`\^=\active\def^{\ifmmode\sp\else\^{}\fi}\catcode`\%=\active\def%{\%}$\mathdefault{16}$}}%
\end{pgfscope}%
\begin{pgfscope}%
\pgfsetbuttcap%
\pgfsetroundjoin%
\definecolor{currentfill}{rgb}{0.000000,0.000000,0.000000}%
\pgfsetfillcolor{currentfill}%
\pgfsetlinewidth{0.803000pt}%
\definecolor{currentstroke}{rgb}{0.000000,0.000000,0.000000}%
\pgfsetstrokecolor{currentstroke}%
\pgfsetdash{}{0pt}%
\pgfsys@defobject{currentmarker}{\pgfqpoint{0.000000in}{-0.048611in}}{\pgfqpoint{0.000000in}{0.000000in}}{%
\pgfpathmoveto{\pgfqpoint{0.000000in}{0.000000in}}%
\pgfpathlineto{\pgfqpoint{0.000000in}{-0.048611in}}%
\pgfusepath{stroke,fill}%
}%
\begin{pgfscope}%
\pgfsys@transformshift{3.435175in}{0.521603in}%
\pgfsys@useobject{currentmarker}{}%
\end{pgfscope}%
\end{pgfscope}%
\begin{pgfscope}%
\definecolor{textcolor}{rgb}{0.000000,0.000000,0.000000}%
\pgfsetstrokecolor{textcolor}%
\pgfsetfillcolor{textcolor}%
\pgftext[x=3.435175in,y=0.424381in,,top]{\color{textcolor}{\rmfamily\fontsize{10.000000}{12.000000}\selectfont\catcode`\^=\active\def^{\ifmmode\sp\else\^{}\fi}\catcode`\%=\active\def%{\%}$\mathdefault{20}$}}%
\end{pgfscope}%
\begin{pgfscope}%
\pgfsetbuttcap%
\pgfsetroundjoin%
\definecolor{currentfill}{rgb}{0.000000,0.000000,0.000000}%
\pgfsetfillcolor{currentfill}%
\pgfsetlinewidth{0.803000pt}%
\definecolor{currentstroke}{rgb}{0.000000,0.000000,0.000000}%
\pgfsetstrokecolor{currentstroke}%
\pgfsetdash{}{0pt}%
\pgfsys@defobject{currentmarker}{\pgfqpoint{0.000000in}{-0.048611in}}{\pgfqpoint{0.000000in}{0.000000in}}{%
\pgfpathmoveto{\pgfqpoint{0.000000in}{0.000000in}}%
\pgfpathlineto{\pgfqpoint{0.000000in}{-0.048611in}}%
\pgfusepath{stroke,fill}%
}%
\begin{pgfscope}%
\pgfsys@transformshift{4.020632in}{0.521603in}%
\pgfsys@useobject{currentmarker}{}%
\end{pgfscope}%
\end{pgfscope}%
\begin{pgfscope}%
\definecolor{textcolor}{rgb}{0.000000,0.000000,0.000000}%
\pgfsetstrokecolor{textcolor}%
\pgfsetfillcolor{textcolor}%
\pgftext[x=4.020632in,y=0.424381in,,top]{\color{textcolor}{\rmfamily\fontsize{10.000000}{12.000000}\selectfont\catcode`\^=\active\def^{\ifmmode\sp\else\^{}\fi}\catcode`\%=\active\def%{\%}$\mathdefault{24}$}}%
\end{pgfscope}%
\begin{pgfscope}%
\pgfsetbuttcap%
\pgfsetroundjoin%
\definecolor{currentfill}{rgb}{0.000000,0.000000,0.000000}%
\pgfsetfillcolor{currentfill}%
\pgfsetlinewidth{0.803000pt}%
\definecolor{currentstroke}{rgb}{0.000000,0.000000,0.000000}%
\pgfsetstrokecolor{currentstroke}%
\pgfsetdash{}{0pt}%
\pgfsys@defobject{currentmarker}{\pgfqpoint{0.000000in}{-0.048611in}}{\pgfqpoint{0.000000in}{0.000000in}}{%
\pgfpathmoveto{\pgfqpoint{0.000000in}{0.000000in}}%
\pgfpathlineto{\pgfqpoint{0.000000in}{-0.048611in}}%
\pgfusepath{stroke,fill}%
}%
\begin{pgfscope}%
\pgfsys@transformshift{4.606090in}{0.521603in}%
\pgfsys@useobject{currentmarker}{}%
\end{pgfscope}%
\end{pgfscope}%
\begin{pgfscope}%
\definecolor{textcolor}{rgb}{0.000000,0.000000,0.000000}%
\pgfsetstrokecolor{textcolor}%
\pgfsetfillcolor{textcolor}%
\pgftext[x=4.606090in,y=0.424381in,,top]{\color{textcolor}{\rmfamily\fontsize{10.000000}{12.000000}\selectfont\catcode`\^=\active\def^{\ifmmode\sp\else\^{}\fi}\catcode`\%=\active\def%{\%}$\mathdefault{28}$}}%
\end{pgfscope}%
\begin{pgfscope}%
\pgfsetbuttcap%
\pgfsetroundjoin%
\definecolor{currentfill}{rgb}{0.000000,0.000000,0.000000}%
\pgfsetfillcolor{currentfill}%
\pgfsetlinewidth{0.803000pt}%
\definecolor{currentstroke}{rgb}{0.000000,0.000000,0.000000}%
\pgfsetstrokecolor{currentstroke}%
\pgfsetdash{}{0pt}%
\pgfsys@defobject{currentmarker}{\pgfqpoint{0.000000in}{-0.048611in}}{\pgfqpoint{0.000000in}{0.000000in}}{%
\pgfpathmoveto{\pgfqpoint{0.000000in}{0.000000in}}%
\pgfpathlineto{\pgfqpoint{0.000000in}{-0.048611in}}%
\pgfusepath{stroke,fill}%
}%
\begin{pgfscope}%
\pgfsys@transformshift{5.191547in}{0.521603in}%
\pgfsys@useobject{currentmarker}{}%
\end{pgfscope}%
\end{pgfscope}%
\begin{pgfscope}%
\definecolor{textcolor}{rgb}{0.000000,0.000000,0.000000}%
\pgfsetstrokecolor{textcolor}%
\pgfsetfillcolor{textcolor}%
\pgftext[x=5.191547in,y=0.424381in,,top]{\color{textcolor}{\rmfamily\fontsize{10.000000}{12.000000}\selectfont\catcode`\^=\active\def^{\ifmmode\sp\else\^{}\fi}\catcode`\%=\active\def%{\%}$\mathdefault{32}$}}%
\end{pgfscope}%
\begin{pgfscope}%
\definecolor{textcolor}{rgb}{0.000000,0.000000,0.000000}%
\pgfsetstrokecolor{textcolor}%
\pgfsetfillcolor{textcolor}%
\pgftext[x=2.922899in,y=0.234413in,,top]{\color{textcolor}{\rmfamily\fontsize{10.000000}{12.000000}\selectfont\catcode`\^=\active\def^{\ifmmode\sp\else\^{}\fi}\catcode`\%=\active\def%{\%}Monochromatic classes}}%
\end{pgfscope}%
\begin{pgfscope}%
\pgfsetbuttcap%
\pgfsetroundjoin%
\definecolor{currentfill}{rgb}{0.000000,0.000000,0.000000}%
\pgfsetfillcolor{currentfill}%
\pgfsetlinewidth{0.803000pt}%
\definecolor{currentstroke}{rgb}{0.000000,0.000000,0.000000}%
\pgfsetstrokecolor{currentstroke}%
\pgfsetdash{}{0pt}%
\pgfsys@defobject{currentmarker}{\pgfqpoint{-0.048611in}{0.000000in}}{\pgfqpoint{-0.000000in}{0.000000in}}{%
\pgfpathmoveto{\pgfqpoint{-0.000000in}{0.000000in}}%
\pgfpathlineto{\pgfqpoint{-0.048611in}{0.000000in}}%
\pgfusepath{stroke,fill}%
}%
\begin{pgfscope}%
\pgfsys@transformshift{0.588387in}{0.670551in}%
\pgfsys@useobject{currentmarker}{}%
\end{pgfscope}%
\end{pgfscope}%
\begin{pgfscope}%
\definecolor{textcolor}{rgb}{0.000000,0.000000,0.000000}%
\pgfsetstrokecolor{textcolor}%
\pgfsetfillcolor{textcolor}%
\pgftext[x=0.289968in, y=0.617790in, left, base]{\color{textcolor}{\rmfamily\fontsize{10.000000}{12.000000}\selectfont\catcode`\^=\active\def^{\ifmmode\sp\else\^{}\fi}\catcode`\%=\active\def%{\%}$\mathdefault{10^{1}}$}}%
\end{pgfscope}%
\begin{pgfscope}%
\pgfsetbuttcap%
\pgfsetroundjoin%
\definecolor{currentfill}{rgb}{0.000000,0.000000,0.000000}%
\pgfsetfillcolor{currentfill}%
\pgfsetlinewidth{0.803000pt}%
\definecolor{currentstroke}{rgb}{0.000000,0.000000,0.000000}%
\pgfsetstrokecolor{currentstroke}%
\pgfsetdash{}{0pt}%
\pgfsys@defobject{currentmarker}{\pgfqpoint{-0.048611in}{0.000000in}}{\pgfqpoint{-0.000000in}{0.000000in}}{%
\pgfpathmoveto{\pgfqpoint{-0.000000in}{0.000000in}}%
\pgfpathlineto{\pgfqpoint{-0.048611in}{0.000000in}}%
\pgfusepath{stroke,fill}%
}%
\begin{pgfscope}%
\pgfsys@transformshift{0.588387in}{1.343398in}%
\pgfsys@useobject{currentmarker}{}%
\end{pgfscope}%
\end{pgfscope}%
\begin{pgfscope}%
\definecolor{textcolor}{rgb}{0.000000,0.000000,0.000000}%
\pgfsetstrokecolor{textcolor}%
\pgfsetfillcolor{textcolor}%
\pgftext[x=0.289968in, y=1.290636in, left, base]{\color{textcolor}{\rmfamily\fontsize{10.000000}{12.000000}\selectfont\catcode`\^=\active\def^{\ifmmode\sp\else\^{}\fi}\catcode`\%=\active\def%{\%}$\mathdefault{10^{2}}$}}%
\end{pgfscope}%
\begin{pgfscope}%
\pgfsetbuttcap%
\pgfsetroundjoin%
\definecolor{currentfill}{rgb}{0.000000,0.000000,0.000000}%
\pgfsetfillcolor{currentfill}%
\pgfsetlinewidth{0.803000pt}%
\definecolor{currentstroke}{rgb}{0.000000,0.000000,0.000000}%
\pgfsetstrokecolor{currentstroke}%
\pgfsetdash{}{0pt}%
\pgfsys@defobject{currentmarker}{\pgfqpoint{-0.048611in}{0.000000in}}{\pgfqpoint{-0.000000in}{0.000000in}}{%
\pgfpathmoveto{\pgfqpoint{-0.000000in}{0.000000in}}%
\pgfpathlineto{\pgfqpoint{-0.048611in}{0.000000in}}%
\pgfusepath{stroke,fill}%
}%
\begin{pgfscope}%
\pgfsys@transformshift{0.588387in}{2.016245in}%
\pgfsys@useobject{currentmarker}{}%
\end{pgfscope}%
\end{pgfscope}%
\begin{pgfscope}%
\definecolor{textcolor}{rgb}{0.000000,0.000000,0.000000}%
\pgfsetstrokecolor{textcolor}%
\pgfsetfillcolor{textcolor}%
\pgftext[x=0.289968in, y=1.963483in, left, base]{\color{textcolor}{\rmfamily\fontsize{10.000000}{12.000000}\selectfont\catcode`\^=\active\def^{\ifmmode\sp\else\^{}\fi}\catcode`\%=\active\def%{\%}$\mathdefault{10^{3}}$}}%
\end{pgfscope}%
\begin{pgfscope}%
\pgfsetbuttcap%
\pgfsetroundjoin%
\definecolor{currentfill}{rgb}{0.000000,0.000000,0.000000}%
\pgfsetfillcolor{currentfill}%
\pgfsetlinewidth{0.803000pt}%
\definecolor{currentstroke}{rgb}{0.000000,0.000000,0.000000}%
\pgfsetstrokecolor{currentstroke}%
\pgfsetdash{}{0pt}%
\pgfsys@defobject{currentmarker}{\pgfqpoint{-0.048611in}{0.000000in}}{\pgfqpoint{-0.000000in}{0.000000in}}{%
\pgfpathmoveto{\pgfqpoint{-0.000000in}{0.000000in}}%
\pgfpathlineto{\pgfqpoint{-0.048611in}{0.000000in}}%
\pgfusepath{stroke,fill}%
}%
\begin{pgfscope}%
\pgfsys@transformshift{0.588387in}{2.689091in}%
\pgfsys@useobject{currentmarker}{}%
\end{pgfscope}%
\end{pgfscope}%
\begin{pgfscope}%
\definecolor{textcolor}{rgb}{0.000000,0.000000,0.000000}%
\pgfsetstrokecolor{textcolor}%
\pgfsetfillcolor{textcolor}%
\pgftext[x=0.289968in, y=2.636330in, left, base]{\color{textcolor}{\rmfamily\fontsize{10.000000}{12.000000}\selectfont\catcode`\^=\active\def^{\ifmmode\sp\else\^{}\fi}\catcode`\%=\active\def%{\%}$\mathdefault{10^{4}}$}}%
\end{pgfscope}%
\begin{pgfscope}%
\pgfsetbuttcap%
\pgfsetroundjoin%
\definecolor{currentfill}{rgb}{0.000000,0.000000,0.000000}%
\pgfsetfillcolor{currentfill}%
\pgfsetlinewidth{0.602250pt}%
\definecolor{currentstroke}{rgb}{0.000000,0.000000,0.000000}%
\pgfsetstrokecolor{currentstroke}%
\pgfsetdash{}{0pt}%
\pgfsys@defobject{currentmarker}{\pgfqpoint{-0.027778in}{0.000000in}}{\pgfqpoint{-0.000000in}{0.000000in}}{%
\pgfpathmoveto{\pgfqpoint{-0.000000in}{0.000000in}}%
\pgfpathlineto{\pgfqpoint{-0.027778in}{0.000000in}}%
\pgfusepath{stroke,fill}%
}%
\begin{pgfscope}%
\pgfsys@transformshift{0.588387in}{0.566326in}%
\pgfsys@useobject{currentmarker}{}%
\end{pgfscope}%
\end{pgfscope}%
\begin{pgfscope}%
\pgfsetbuttcap%
\pgfsetroundjoin%
\definecolor{currentfill}{rgb}{0.000000,0.000000,0.000000}%
\pgfsetfillcolor{currentfill}%
\pgfsetlinewidth{0.602250pt}%
\definecolor{currentstroke}{rgb}{0.000000,0.000000,0.000000}%
\pgfsetstrokecolor{currentstroke}%
\pgfsetdash{}{0pt}%
\pgfsys@defobject{currentmarker}{\pgfqpoint{-0.027778in}{0.000000in}}{\pgfqpoint{-0.000000in}{0.000000in}}{%
\pgfpathmoveto{\pgfqpoint{-0.000000in}{0.000000in}}%
\pgfpathlineto{\pgfqpoint{-0.027778in}{0.000000in}}%
\pgfusepath{stroke,fill}%
}%
\begin{pgfscope}%
\pgfsys@transformshift{0.588387in}{0.605346in}%
\pgfsys@useobject{currentmarker}{}%
\end{pgfscope}%
\end{pgfscope}%
\begin{pgfscope}%
\pgfsetbuttcap%
\pgfsetroundjoin%
\definecolor{currentfill}{rgb}{0.000000,0.000000,0.000000}%
\pgfsetfillcolor{currentfill}%
\pgfsetlinewidth{0.602250pt}%
\definecolor{currentstroke}{rgb}{0.000000,0.000000,0.000000}%
\pgfsetstrokecolor{currentstroke}%
\pgfsetdash{}{0pt}%
\pgfsys@defobject{currentmarker}{\pgfqpoint{-0.027778in}{0.000000in}}{\pgfqpoint{-0.000000in}{0.000000in}}{%
\pgfpathmoveto{\pgfqpoint{-0.000000in}{0.000000in}}%
\pgfpathlineto{\pgfqpoint{-0.027778in}{0.000000in}}%
\pgfusepath{stroke,fill}%
}%
\begin{pgfscope}%
\pgfsys@transformshift{0.588387in}{0.639763in}%
\pgfsys@useobject{currentmarker}{}%
\end{pgfscope}%
\end{pgfscope}%
\begin{pgfscope}%
\pgfsetbuttcap%
\pgfsetroundjoin%
\definecolor{currentfill}{rgb}{0.000000,0.000000,0.000000}%
\pgfsetfillcolor{currentfill}%
\pgfsetlinewidth{0.602250pt}%
\definecolor{currentstroke}{rgb}{0.000000,0.000000,0.000000}%
\pgfsetstrokecolor{currentstroke}%
\pgfsetdash{}{0pt}%
\pgfsys@defobject{currentmarker}{\pgfqpoint{-0.027778in}{0.000000in}}{\pgfqpoint{-0.000000in}{0.000000in}}{%
\pgfpathmoveto{\pgfqpoint{-0.000000in}{0.000000in}}%
\pgfpathlineto{\pgfqpoint{-0.027778in}{0.000000in}}%
\pgfusepath{stroke,fill}%
}%
\begin{pgfscope}%
\pgfsys@transformshift{0.588387in}{0.873098in}%
\pgfsys@useobject{currentmarker}{}%
\end{pgfscope}%
\end{pgfscope}%
\begin{pgfscope}%
\pgfsetbuttcap%
\pgfsetroundjoin%
\definecolor{currentfill}{rgb}{0.000000,0.000000,0.000000}%
\pgfsetfillcolor{currentfill}%
\pgfsetlinewidth{0.602250pt}%
\definecolor{currentstroke}{rgb}{0.000000,0.000000,0.000000}%
\pgfsetstrokecolor{currentstroke}%
\pgfsetdash{}{0pt}%
\pgfsys@defobject{currentmarker}{\pgfqpoint{-0.027778in}{0.000000in}}{\pgfqpoint{-0.000000in}{0.000000in}}{%
\pgfpathmoveto{\pgfqpoint{-0.000000in}{0.000000in}}%
\pgfpathlineto{\pgfqpoint{-0.027778in}{0.000000in}}%
\pgfusepath{stroke,fill}%
}%
\begin{pgfscope}%
\pgfsys@transformshift{0.588387in}{0.991581in}%
\pgfsys@useobject{currentmarker}{}%
\end{pgfscope}%
\end{pgfscope}%
\begin{pgfscope}%
\pgfsetbuttcap%
\pgfsetroundjoin%
\definecolor{currentfill}{rgb}{0.000000,0.000000,0.000000}%
\pgfsetfillcolor{currentfill}%
\pgfsetlinewidth{0.602250pt}%
\definecolor{currentstroke}{rgb}{0.000000,0.000000,0.000000}%
\pgfsetstrokecolor{currentstroke}%
\pgfsetdash{}{0pt}%
\pgfsys@defobject{currentmarker}{\pgfqpoint{-0.027778in}{0.000000in}}{\pgfqpoint{-0.000000in}{0.000000in}}{%
\pgfpathmoveto{\pgfqpoint{-0.000000in}{0.000000in}}%
\pgfpathlineto{\pgfqpoint{-0.027778in}{0.000000in}}%
\pgfusepath{stroke,fill}%
}%
\begin{pgfscope}%
\pgfsys@transformshift{0.588387in}{1.075645in}%
\pgfsys@useobject{currentmarker}{}%
\end{pgfscope}%
\end{pgfscope}%
\begin{pgfscope}%
\pgfsetbuttcap%
\pgfsetroundjoin%
\definecolor{currentfill}{rgb}{0.000000,0.000000,0.000000}%
\pgfsetfillcolor{currentfill}%
\pgfsetlinewidth{0.602250pt}%
\definecolor{currentstroke}{rgb}{0.000000,0.000000,0.000000}%
\pgfsetstrokecolor{currentstroke}%
\pgfsetdash{}{0pt}%
\pgfsys@defobject{currentmarker}{\pgfqpoint{-0.027778in}{0.000000in}}{\pgfqpoint{-0.000000in}{0.000000in}}{%
\pgfpathmoveto{\pgfqpoint{-0.000000in}{0.000000in}}%
\pgfpathlineto{\pgfqpoint{-0.027778in}{0.000000in}}%
\pgfusepath{stroke,fill}%
}%
\begin{pgfscope}%
\pgfsys@transformshift{0.588387in}{1.140851in}%
\pgfsys@useobject{currentmarker}{}%
\end{pgfscope}%
\end{pgfscope}%
\begin{pgfscope}%
\pgfsetbuttcap%
\pgfsetroundjoin%
\definecolor{currentfill}{rgb}{0.000000,0.000000,0.000000}%
\pgfsetfillcolor{currentfill}%
\pgfsetlinewidth{0.602250pt}%
\definecolor{currentstroke}{rgb}{0.000000,0.000000,0.000000}%
\pgfsetstrokecolor{currentstroke}%
\pgfsetdash{}{0pt}%
\pgfsys@defobject{currentmarker}{\pgfqpoint{-0.027778in}{0.000000in}}{\pgfqpoint{-0.000000in}{0.000000in}}{%
\pgfpathmoveto{\pgfqpoint{-0.000000in}{0.000000in}}%
\pgfpathlineto{\pgfqpoint{-0.027778in}{0.000000in}}%
\pgfusepath{stroke,fill}%
}%
\begin{pgfscope}%
\pgfsys@transformshift{0.588387in}{1.194128in}%
\pgfsys@useobject{currentmarker}{}%
\end{pgfscope}%
\end{pgfscope}%
\begin{pgfscope}%
\pgfsetbuttcap%
\pgfsetroundjoin%
\definecolor{currentfill}{rgb}{0.000000,0.000000,0.000000}%
\pgfsetfillcolor{currentfill}%
\pgfsetlinewidth{0.602250pt}%
\definecolor{currentstroke}{rgb}{0.000000,0.000000,0.000000}%
\pgfsetstrokecolor{currentstroke}%
\pgfsetdash{}{0pt}%
\pgfsys@defobject{currentmarker}{\pgfqpoint{-0.027778in}{0.000000in}}{\pgfqpoint{-0.000000in}{0.000000in}}{%
\pgfpathmoveto{\pgfqpoint{-0.000000in}{0.000000in}}%
\pgfpathlineto{\pgfqpoint{-0.027778in}{0.000000in}}%
\pgfusepath{stroke,fill}%
}%
\begin{pgfscope}%
\pgfsys@transformshift{0.588387in}{1.239173in}%
\pgfsys@useobject{currentmarker}{}%
\end{pgfscope}%
\end{pgfscope}%
\begin{pgfscope}%
\pgfsetbuttcap%
\pgfsetroundjoin%
\definecolor{currentfill}{rgb}{0.000000,0.000000,0.000000}%
\pgfsetfillcolor{currentfill}%
\pgfsetlinewidth{0.602250pt}%
\definecolor{currentstroke}{rgb}{0.000000,0.000000,0.000000}%
\pgfsetstrokecolor{currentstroke}%
\pgfsetdash{}{0pt}%
\pgfsys@defobject{currentmarker}{\pgfqpoint{-0.027778in}{0.000000in}}{\pgfqpoint{-0.000000in}{0.000000in}}{%
\pgfpathmoveto{\pgfqpoint{-0.000000in}{0.000000in}}%
\pgfpathlineto{\pgfqpoint{-0.027778in}{0.000000in}}%
\pgfusepath{stroke,fill}%
}%
\begin{pgfscope}%
\pgfsys@transformshift{0.588387in}{1.278192in}%
\pgfsys@useobject{currentmarker}{}%
\end{pgfscope}%
\end{pgfscope}%
\begin{pgfscope}%
\pgfsetbuttcap%
\pgfsetroundjoin%
\definecolor{currentfill}{rgb}{0.000000,0.000000,0.000000}%
\pgfsetfillcolor{currentfill}%
\pgfsetlinewidth{0.602250pt}%
\definecolor{currentstroke}{rgb}{0.000000,0.000000,0.000000}%
\pgfsetstrokecolor{currentstroke}%
\pgfsetdash{}{0pt}%
\pgfsys@defobject{currentmarker}{\pgfqpoint{-0.027778in}{0.000000in}}{\pgfqpoint{-0.000000in}{0.000000in}}{%
\pgfpathmoveto{\pgfqpoint{-0.000000in}{0.000000in}}%
\pgfpathlineto{\pgfqpoint{-0.027778in}{0.000000in}}%
\pgfusepath{stroke,fill}%
}%
\begin{pgfscope}%
\pgfsys@transformshift{0.588387in}{1.312610in}%
\pgfsys@useobject{currentmarker}{}%
\end{pgfscope}%
\end{pgfscope}%
\begin{pgfscope}%
\pgfsetbuttcap%
\pgfsetroundjoin%
\definecolor{currentfill}{rgb}{0.000000,0.000000,0.000000}%
\pgfsetfillcolor{currentfill}%
\pgfsetlinewidth{0.602250pt}%
\definecolor{currentstroke}{rgb}{0.000000,0.000000,0.000000}%
\pgfsetstrokecolor{currentstroke}%
\pgfsetdash{}{0pt}%
\pgfsys@defobject{currentmarker}{\pgfqpoint{-0.027778in}{0.000000in}}{\pgfqpoint{-0.000000in}{0.000000in}}{%
\pgfpathmoveto{\pgfqpoint{-0.000000in}{0.000000in}}%
\pgfpathlineto{\pgfqpoint{-0.027778in}{0.000000in}}%
\pgfusepath{stroke,fill}%
}%
\begin{pgfscope}%
\pgfsys@transformshift{0.588387in}{1.545945in}%
\pgfsys@useobject{currentmarker}{}%
\end{pgfscope}%
\end{pgfscope}%
\begin{pgfscope}%
\pgfsetbuttcap%
\pgfsetroundjoin%
\definecolor{currentfill}{rgb}{0.000000,0.000000,0.000000}%
\pgfsetfillcolor{currentfill}%
\pgfsetlinewidth{0.602250pt}%
\definecolor{currentstroke}{rgb}{0.000000,0.000000,0.000000}%
\pgfsetstrokecolor{currentstroke}%
\pgfsetdash{}{0pt}%
\pgfsys@defobject{currentmarker}{\pgfqpoint{-0.027778in}{0.000000in}}{\pgfqpoint{-0.000000in}{0.000000in}}{%
\pgfpathmoveto{\pgfqpoint{-0.000000in}{0.000000in}}%
\pgfpathlineto{\pgfqpoint{-0.027778in}{0.000000in}}%
\pgfusepath{stroke,fill}%
}%
\begin{pgfscope}%
\pgfsys@transformshift{0.588387in}{1.664427in}%
\pgfsys@useobject{currentmarker}{}%
\end{pgfscope}%
\end{pgfscope}%
\begin{pgfscope}%
\pgfsetbuttcap%
\pgfsetroundjoin%
\definecolor{currentfill}{rgb}{0.000000,0.000000,0.000000}%
\pgfsetfillcolor{currentfill}%
\pgfsetlinewidth{0.602250pt}%
\definecolor{currentstroke}{rgb}{0.000000,0.000000,0.000000}%
\pgfsetstrokecolor{currentstroke}%
\pgfsetdash{}{0pt}%
\pgfsys@defobject{currentmarker}{\pgfqpoint{-0.027778in}{0.000000in}}{\pgfqpoint{-0.000000in}{0.000000in}}{%
\pgfpathmoveto{\pgfqpoint{-0.000000in}{0.000000in}}%
\pgfpathlineto{\pgfqpoint{-0.027778in}{0.000000in}}%
\pgfusepath{stroke,fill}%
}%
\begin{pgfscope}%
\pgfsys@transformshift{0.588387in}{1.748492in}%
\pgfsys@useobject{currentmarker}{}%
\end{pgfscope}%
\end{pgfscope}%
\begin{pgfscope}%
\pgfsetbuttcap%
\pgfsetroundjoin%
\definecolor{currentfill}{rgb}{0.000000,0.000000,0.000000}%
\pgfsetfillcolor{currentfill}%
\pgfsetlinewidth{0.602250pt}%
\definecolor{currentstroke}{rgb}{0.000000,0.000000,0.000000}%
\pgfsetstrokecolor{currentstroke}%
\pgfsetdash{}{0pt}%
\pgfsys@defobject{currentmarker}{\pgfqpoint{-0.027778in}{0.000000in}}{\pgfqpoint{-0.000000in}{0.000000in}}{%
\pgfpathmoveto{\pgfqpoint{-0.000000in}{0.000000in}}%
\pgfpathlineto{\pgfqpoint{-0.027778in}{0.000000in}}%
\pgfusepath{stroke,fill}%
}%
\begin{pgfscope}%
\pgfsys@transformshift{0.588387in}{1.813697in}%
\pgfsys@useobject{currentmarker}{}%
\end{pgfscope}%
\end{pgfscope}%
\begin{pgfscope}%
\pgfsetbuttcap%
\pgfsetroundjoin%
\definecolor{currentfill}{rgb}{0.000000,0.000000,0.000000}%
\pgfsetfillcolor{currentfill}%
\pgfsetlinewidth{0.602250pt}%
\definecolor{currentstroke}{rgb}{0.000000,0.000000,0.000000}%
\pgfsetstrokecolor{currentstroke}%
\pgfsetdash{}{0pt}%
\pgfsys@defobject{currentmarker}{\pgfqpoint{-0.027778in}{0.000000in}}{\pgfqpoint{-0.000000in}{0.000000in}}{%
\pgfpathmoveto{\pgfqpoint{-0.000000in}{0.000000in}}%
\pgfpathlineto{\pgfqpoint{-0.027778in}{0.000000in}}%
\pgfusepath{stroke,fill}%
}%
\begin{pgfscope}%
\pgfsys@transformshift{0.588387in}{1.866974in}%
\pgfsys@useobject{currentmarker}{}%
\end{pgfscope}%
\end{pgfscope}%
\begin{pgfscope}%
\pgfsetbuttcap%
\pgfsetroundjoin%
\definecolor{currentfill}{rgb}{0.000000,0.000000,0.000000}%
\pgfsetfillcolor{currentfill}%
\pgfsetlinewidth{0.602250pt}%
\definecolor{currentstroke}{rgb}{0.000000,0.000000,0.000000}%
\pgfsetstrokecolor{currentstroke}%
\pgfsetdash{}{0pt}%
\pgfsys@defobject{currentmarker}{\pgfqpoint{-0.027778in}{0.000000in}}{\pgfqpoint{-0.000000in}{0.000000in}}{%
\pgfpathmoveto{\pgfqpoint{-0.000000in}{0.000000in}}%
\pgfpathlineto{\pgfqpoint{-0.027778in}{0.000000in}}%
\pgfusepath{stroke,fill}%
}%
\begin{pgfscope}%
\pgfsys@transformshift{0.588387in}{1.912019in}%
\pgfsys@useobject{currentmarker}{}%
\end{pgfscope}%
\end{pgfscope}%
\begin{pgfscope}%
\pgfsetbuttcap%
\pgfsetroundjoin%
\definecolor{currentfill}{rgb}{0.000000,0.000000,0.000000}%
\pgfsetfillcolor{currentfill}%
\pgfsetlinewidth{0.602250pt}%
\definecolor{currentstroke}{rgb}{0.000000,0.000000,0.000000}%
\pgfsetstrokecolor{currentstroke}%
\pgfsetdash{}{0pt}%
\pgfsys@defobject{currentmarker}{\pgfqpoint{-0.027778in}{0.000000in}}{\pgfqpoint{-0.000000in}{0.000000in}}{%
\pgfpathmoveto{\pgfqpoint{-0.000000in}{0.000000in}}%
\pgfpathlineto{\pgfqpoint{-0.027778in}{0.000000in}}%
\pgfusepath{stroke,fill}%
}%
\begin{pgfscope}%
\pgfsys@transformshift{0.588387in}{1.951039in}%
\pgfsys@useobject{currentmarker}{}%
\end{pgfscope}%
\end{pgfscope}%
\begin{pgfscope}%
\pgfsetbuttcap%
\pgfsetroundjoin%
\definecolor{currentfill}{rgb}{0.000000,0.000000,0.000000}%
\pgfsetfillcolor{currentfill}%
\pgfsetlinewidth{0.602250pt}%
\definecolor{currentstroke}{rgb}{0.000000,0.000000,0.000000}%
\pgfsetstrokecolor{currentstroke}%
\pgfsetdash{}{0pt}%
\pgfsys@defobject{currentmarker}{\pgfqpoint{-0.027778in}{0.000000in}}{\pgfqpoint{-0.000000in}{0.000000in}}{%
\pgfpathmoveto{\pgfqpoint{-0.000000in}{0.000000in}}%
\pgfpathlineto{\pgfqpoint{-0.027778in}{0.000000in}}%
\pgfusepath{stroke,fill}%
}%
\begin{pgfscope}%
\pgfsys@transformshift{0.588387in}{1.985457in}%
\pgfsys@useobject{currentmarker}{}%
\end{pgfscope}%
\end{pgfscope}%
\begin{pgfscope}%
\pgfsetbuttcap%
\pgfsetroundjoin%
\definecolor{currentfill}{rgb}{0.000000,0.000000,0.000000}%
\pgfsetfillcolor{currentfill}%
\pgfsetlinewidth{0.602250pt}%
\definecolor{currentstroke}{rgb}{0.000000,0.000000,0.000000}%
\pgfsetstrokecolor{currentstroke}%
\pgfsetdash{}{0pt}%
\pgfsys@defobject{currentmarker}{\pgfqpoint{-0.027778in}{0.000000in}}{\pgfqpoint{-0.000000in}{0.000000in}}{%
\pgfpathmoveto{\pgfqpoint{-0.000000in}{0.000000in}}%
\pgfpathlineto{\pgfqpoint{-0.027778in}{0.000000in}}%
\pgfusepath{stroke,fill}%
}%
\begin{pgfscope}%
\pgfsys@transformshift{0.588387in}{2.218792in}%
\pgfsys@useobject{currentmarker}{}%
\end{pgfscope}%
\end{pgfscope}%
\begin{pgfscope}%
\pgfsetbuttcap%
\pgfsetroundjoin%
\definecolor{currentfill}{rgb}{0.000000,0.000000,0.000000}%
\pgfsetfillcolor{currentfill}%
\pgfsetlinewidth{0.602250pt}%
\definecolor{currentstroke}{rgb}{0.000000,0.000000,0.000000}%
\pgfsetstrokecolor{currentstroke}%
\pgfsetdash{}{0pt}%
\pgfsys@defobject{currentmarker}{\pgfqpoint{-0.027778in}{0.000000in}}{\pgfqpoint{-0.000000in}{0.000000in}}{%
\pgfpathmoveto{\pgfqpoint{-0.000000in}{0.000000in}}%
\pgfpathlineto{\pgfqpoint{-0.027778in}{0.000000in}}%
\pgfusepath{stroke,fill}%
}%
\begin{pgfscope}%
\pgfsys@transformshift{0.588387in}{2.337274in}%
\pgfsys@useobject{currentmarker}{}%
\end{pgfscope}%
\end{pgfscope}%
\begin{pgfscope}%
\pgfsetbuttcap%
\pgfsetroundjoin%
\definecolor{currentfill}{rgb}{0.000000,0.000000,0.000000}%
\pgfsetfillcolor{currentfill}%
\pgfsetlinewidth{0.602250pt}%
\definecolor{currentstroke}{rgb}{0.000000,0.000000,0.000000}%
\pgfsetstrokecolor{currentstroke}%
\pgfsetdash{}{0pt}%
\pgfsys@defobject{currentmarker}{\pgfqpoint{-0.027778in}{0.000000in}}{\pgfqpoint{-0.000000in}{0.000000in}}{%
\pgfpathmoveto{\pgfqpoint{-0.000000in}{0.000000in}}%
\pgfpathlineto{\pgfqpoint{-0.027778in}{0.000000in}}%
\pgfusepath{stroke,fill}%
}%
\begin{pgfscope}%
\pgfsys@transformshift{0.588387in}{2.421339in}%
\pgfsys@useobject{currentmarker}{}%
\end{pgfscope}%
\end{pgfscope}%
\begin{pgfscope}%
\pgfsetbuttcap%
\pgfsetroundjoin%
\definecolor{currentfill}{rgb}{0.000000,0.000000,0.000000}%
\pgfsetfillcolor{currentfill}%
\pgfsetlinewidth{0.602250pt}%
\definecolor{currentstroke}{rgb}{0.000000,0.000000,0.000000}%
\pgfsetstrokecolor{currentstroke}%
\pgfsetdash{}{0pt}%
\pgfsys@defobject{currentmarker}{\pgfqpoint{-0.027778in}{0.000000in}}{\pgfqpoint{-0.000000in}{0.000000in}}{%
\pgfpathmoveto{\pgfqpoint{-0.000000in}{0.000000in}}%
\pgfpathlineto{\pgfqpoint{-0.027778in}{0.000000in}}%
\pgfusepath{stroke,fill}%
}%
\begin{pgfscope}%
\pgfsys@transformshift{0.588387in}{2.486544in}%
\pgfsys@useobject{currentmarker}{}%
\end{pgfscope}%
\end{pgfscope}%
\begin{pgfscope}%
\pgfsetbuttcap%
\pgfsetroundjoin%
\definecolor{currentfill}{rgb}{0.000000,0.000000,0.000000}%
\pgfsetfillcolor{currentfill}%
\pgfsetlinewidth{0.602250pt}%
\definecolor{currentstroke}{rgb}{0.000000,0.000000,0.000000}%
\pgfsetstrokecolor{currentstroke}%
\pgfsetdash{}{0pt}%
\pgfsys@defobject{currentmarker}{\pgfqpoint{-0.027778in}{0.000000in}}{\pgfqpoint{-0.000000in}{0.000000in}}{%
\pgfpathmoveto{\pgfqpoint{-0.000000in}{0.000000in}}%
\pgfpathlineto{\pgfqpoint{-0.027778in}{0.000000in}}%
\pgfusepath{stroke,fill}%
}%
\begin{pgfscope}%
\pgfsys@transformshift{0.588387in}{2.539821in}%
\pgfsys@useobject{currentmarker}{}%
\end{pgfscope}%
\end{pgfscope}%
\begin{pgfscope}%
\pgfsetbuttcap%
\pgfsetroundjoin%
\definecolor{currentfill}{rgb}{0.000000,0.000000,0.000000}%
\pgfsetfillcolor{currentfill}%
\pgfsetlinewidth{0.602250pt}%
\definecolor{currentstroke}{rgb}{0.000000,0.000000,0.000000}%
\pgfsetstrokecolor{currentstroke}%
\pgfsetdash{}{0pt}%
\pgfsys@defobject{currentmarker}{\pgfqpoint{-0.027778in}{0.000000in}}{\pgfqpoint{-0.000000in}{0.000000in}}{%
\pgfpathmoveto{\pgfqpoint{-0.000000in}{0.000000in}}%
\pgfpathlineto{\pgfqpoint{-0.027778in}{0.000000in}}%
\pgfusepath{stroke,fill}%
}%
\begin{pgfscope}%
\pgfsys@transformshift{0.588387in}{2.584866in}%
\pgfsys@useobject{currentmarker}{}%
\end{pgfscope}%
\end{pgfscope}%
\begin{pgfscope}%
\pgfsetbuttcap%
\pgfsetroundjoin%
\definecolor{currentfill}{rgb}{0.000000,0.000000,0.000000}%
\pgfsetfillcolor{currentfill}%
\pgfsetlinewidth{0.602250pt}%
\definecolor{currentstroke}{rgb}{0.000000,0.000000,0.000000}%
\pgfsetstrokecolor{currentstroke}%
\pgfsetdash{}{0pt}%
\pgfsys@defobject{currentmarker}{\pgfqpoint{-0.027778in}{0.000000in}}{\pgfqpoint{-0.000000in}{0.000000in}}{%
\pgfpathmoveto{\pgfqpoint{-0.000000in}{0.000000in}}%
\pgfpathlineto{\pgfqpoint{-0.027778in}{0.000000in}}%
\pgfusepath{stroke,fill}%
}%
\begin{pgfscope}%
\pgfsys@transformshift{0.588387in}{2.623886in}%
\pgfsys@useobject{currentmarker}{}%
\end{pgfscope}%
\end{pgfscope}%
\begin{pgfscope}%
\pgfsetbuttcap%
\pgfsetroundjoin%
\definecolor{currentfill}{rgb}{0.000000,0.000000,0.000000}%
\pgfsetfillcolor{currentfill}%
\pgfsetlinewidth{0.602250pt}%
\definecolor{currentstroke}{rgb}{0.000000,0.000000,0.000000}%
\pgfsetstrokecolor{currentstroke}%
\pgfsetdash{}{0pt}%
\pgfsys@defobject{currentmarker}{\pgfqpoint{-0.027778in}{0.000000in}}{\pgfqpoint{-0.000000in}{0.000000in}}{%
\pgfpathmoveto{\pgfqpoint{-0.000000in}{0.000000in}}%
\pgfpathlineto{\pgfqpoint{-0.027778in}{0.000000in}}%
\pgfusepath{stroke,fill}%
}%
\begin{pgfscope}%
\pgfsys@transformshift{0.588387in}{2.658303in}%
\pgfsys@useobject{currentmarker}{}%
\end{pgfscope}%
\end{pgfscope}%
\begin{pgfscope}%
\definecolor{textcolor}{rgb}{0.000000,0.000000,0.000000}%
\pgfsetstrokecolor{textcolor}%
\pgfsetfillcolor{textcolor}%
\pgftext[x=0.234413in,y=1.617672in,,bottom,rotate=90.000000]{\color{textcolor}{\rmfamily\fontsize{10.000000}{12.000000}\selectfont\catcode`\^=\active\def^{\ifmmode\sp\else\^{}\fi}\catcode`\%=\active\def%{\%}Time [ms]}}%
\end{pgfscope}%
\begin{pgfscope}%
\pgfpathrectangle{\pgfqpoint{0.588387in}{0.521603in}}{\pgfqpoint{4.669024in}{2.192138in}}%
\pgfusepath{clip}%
\pgfsetrectcap%
\pgfsetroundjoin%
\pgfsetlinewidth{1.505625pt}%
\pgfsetstrokecolor{currentstroke1}%
\pgfsetdash{}{0pt}%
\pgfpathmoveto{\pgfqpoint{0.800616in}{0.650495in}}%
\pgfpathlineto{\pgfqpoint{0.946980in}{0.708985in}}%
\pgfpathlineto{\pgfqpoint{1.093344in}{0.681895in}}%
\pgfpathlineto{\pgfqpoint{1.239709in}{0.712675in}}%
\pgfpathlineto{\pgfqpoint{1.386073in}{0.670369in}}%
\pgfpathlineto{\pgfqpoint{1.532438in}{0.669607in}}%
\pgfpathlineto{\pgfqpoint{1.678802in}{0.699589in}}%
\pgfpathlineto{\pgfqpoint{1.825166in}{0.690039in}}%
\pgfpathlineto{\pgfqpoint{1.971531in}{0.708009in}}%
\pgfpathlineto{\pgfqpoint{2.117895in}{0.752784in}}%
\pgfpathlineto{\pgfqpoint{2.264259in}{0.854173in}}%
\pgfpathlineto{\pgfqpoint{2.410624in}{0.868496in}}%
\pgfpathlineto{\pgfqpoint{2.556988in}{0.937271in}}%
\pgfpathlineto{\pgfqpoint{2.703353in}{0.995673in}}%
\pgfpathlineto{\pgfqpoint{2.849717in}{1.073200in}}%
\pgfpathlineto{\pgfqpoint{2.996081in}{1.206990in}}%
\pgfpathlineto{\pgfqpoint{3.142446in}{1.203120in}}%
\pgfpathlineto{\pgfqpoint{3.288810in}{1.221092in}}%
\pgfpathlineto{\pgfqpoint{3.435175in}{1.337196in}}%
\pgfpathlineto{\pgfqpoint{3.581539in}{1.379628in}}%
\pgfpathlineto{\pgfqpoint{3.727903in}{1.442241in}}%
\pgfpathlineto{\pgfqpoint{3.874268in}{1.372574in}}%
\pgfpathlineto{\pgfqpoint{4.020632in}{1.362485in}}%
\pgfpathlineto{\pgfqpoint{4.166997in}{1.427828in}}%
\pgfpathlineto{\pgfqpoint{4.313361in}{1.408603in}}%
\pgfpathlineto{\pgfqpoint{4.459725in}{1.570120in}}%
\pgfpathlineto{\pgfqpoint{4.898818in}{1.814573in}}%
\pgfpathlineto{\pgfqpoint{5.045183in}{2.117887in}}%
\pgfusepath{stroke}%
\end{pgfscope}%
\begin{pgfscope}%
\pgfpathrectangle{\pgfqpoint{0.588387in}{0.521603in}}{\pgfqpoint{4.669024in}{2.192138in}}%
\pgfusepath{clip}%
\pgfsetrectcap%
\pgfsetroundjoin%
\pgfsetlinewidth{1.505625pt}%
\pgfsetstrokecolor{currentstroke2}%
\pgfsetdash{}{0pt}%
\pgfpathmoveto{\pgfqpoint{0.800616in}{0.649984in}}%
\pgfpathlineto{\pgfqpoint{0.946980in}{0.706070in}}%
\pgfpathlineto{\pgfqpoint{1.093344in}{0.679740in}}%
\pgfpathlineto{\pgfqpoint{1.239709in}{0.713475in}}%
\pgfpathlineto{\pgfqpoint{1.386073in}{0.668892in}}%
\pgfpathlineto{\pgfqpoint{1.532438in}{0.671503in}}%
\pgfpathlineto{\pgfqpoint{1.678802in}{0.696049in}}%
\pgfpathlineto{\pgfqpoint{1.825166in}{0.689568in}}%
\pgfpathlineto{\pgfqpoint{1.971531in}{0.710268in}}%
\pgfpathlineto{\pgfqpoint{2.117895in}{0.748819in}}%
\pgfpathlineto{\pgfqpoint{2.264259in}{0.858878in}}%
\pgfpathlineto{\pgfqpoint{2.410624in}{0.862877in}}%
\pgfpathlineto{\pgfqpoint{2.556988in}{0.943753in}}%
\pgfpathlineto{\pgfqpoint{2.703353in}{0.994191in}}%
\pgfpathlineto{\pgfqpoint{2.849717in}{1.064160in}}%
\pgfpathlineto{\pgfqpoint{2.996081in}{1.181183in}}%
\pgfpathlineto{\pgfqpoint{3.142446in}{1.154411in}}%
\pgfpathlineto{\pgfqpoint{3.288810in}{1.230702in}}%
\pgfpathlineto{\pgfqpoint{3.435175in}{1.317121in}}%
\pgfpathlineto{\pgfqpoint{3.581539in}{1.367506in}}%
\pgfpathlineto{\pgfqpoint{3.727903in}{1.442241in}}%
\pgfpathlineto{\pgfqpoint{3.874268in}{1.304378in}}%
\pgfpathlineto{\pgfqpoint{4.020632in}{1.435390in}}%
\pgfpathlineto{\pgfqpoint{4.166997in}{1.409770in}}%
\pgfpathlineto{\pgfqpoint{4.313361in}{1.598612in}}%
\pgfpathlineto{\pgfqpoint{4.459725in}{1.598612in}}%
\pgfpathlineto{\pgfqpoint{4.898818in}{1.749222in}}%
\pgfpathlineto{\pgfqpoint{5.045183in}{1.965991in}}%
\pgfusepath{stroke}%
\end{pgfscope}%
\begin{pgfscope}%
\pgfpathrectangle{\pgfqpoint{0.588387in}{0.521603in}}{\pgfqpoint{4.669024in}{2.192138in}}%
\pgfusepath{clip}%
\pgfsetrectcap%
\pgfsetroundjoin%
\pgfsetlinewidth{1.505625pt}%
\pgfsetstrokecolor{currentstroke3}%
\pgfsetdash{}{0pt}%
\pgfpathmoveto{\pgfqpoint{0.800616in}{0.622189in}}%
\pgfpathlineto{\pgfqpoint{0.946980in}{0.680427in}}%
\pgfpathlineto{\pgfqpoint{1.093344in}{0.645249in}}%
\pgfpathlineto{\pgfqpoint{1.239709in}{0.676747in}}%
\pgfpathlineto{\pgfqpoint{1.386073in}{0.666666in}}%
\pgfpathlineto{\pgfqpoint{1.532438in}{0.621246in}}%
\pgfpathlineto{\pgfqpoint{1.678802in}{0.644913in}}%
\pgfpathlineto{\pgfqpoint{1.825166in}{0.646520in}}%
\pgfpathlineto{\pgfqpoint{1.971531in}{0.642073in}}%
\pgfpathlineto{\pgfqpoint{2.117895in}{0.711392in}}%
\pgfpathlineto{\pgfqpoint{2.264259in}{0.694916in}}%
\pgfpathlineto{\pgfqpoint{2.410624in}{0.737214in}}%
\pgfpathlineto{\pgfqpoint{2.556988in}{0.838540in}}%
\pgfpathlineto{\pgfqpoint{2.703353in}{0.947714in}}%
\pgfpathlineto{\pgfqpoint{2.849717in}{1.105482in}}%
\pgfpathlineto{\pgfqpoint{2.996081in}{1.295908in}}%
\pgfpathlineto{\pgfqpoint{3.142446in}{1.466710in}}%
\pgfpathlineto{\pgfqpoint{3.288810in}{1.643221in}}%
\pgfpathlineto{\pgfqpoint{3.435175in}{1.849411in}}%
\pgfpathlineto{\pgfqpoint{3.581539in}{2.014682in}}%
\pgfpathlineto{\pgfqpoint{3.727903in}{2.227500in}}%
\pgfpathlineto{\pgfqpoint{4.020632in}{2.614099in}}%
\pgfusepath{stroke}%
\end{pgfscope}%
\begin{pgfscope}%
\pgfpathrectangle{\pgfqpoint{0.588387in}{0.521603in}}{\pgfqpoint{4.669024in}{2.192138in}}%
\pgfusepath{clip}%
\pgfsetrectcap%
\pgfsetroundjoin%
\pgfsetlinewidth{1.505625pt}%
\pgfsetstrokecolor{currentstroke4}%
\pgfsetdash{}{0pt}%
\pgfpathmoveto{\pgfqpoint{0.800616in}{0.648839in}}%
\pgfpathlineto{\pgfqpoint{0.946980in}{0.703864in}}%
\pgfpathlineto{\pgfqpoint{1.093344in}{0.677371in}}%
\pgfpathlineto{\pgfqpoint{1.239709in}{0.708157in}}%
\pgfpathlineto{\pgfqpoint{1.386073in}{0.661085in}}%
\pgfpathlineto{\pgfqpoint{1.532438in}{0.668959in}}%
\pgfpathlineto{\pgfqpoint{1.678802in}{0.701206in}}%
\pgfpathlineto{\pgfqpoint{1.825166in}{0.682603in}}%
\pgfpathlineto{\pgfqpoint{1.971531in}{0.692393in}}%
\pgfpathlineto{\pgfqpoint{2.117895in}{0.735757in}}%
\pgfpathlineto{\pgfqpoint{2.264259in}{0.822440in}}%
\pgfpathlineto{\pgfqpoint{2.410624in}{0.842817in}}%
\pgfpathlineto{\pgfqpoint{2.556988in}{0.896223in}}%
\pgfpathlineto{\pgfqpoint{2.703353in}{0.944531in}}%
\pgfpathlineto{\pgfqpoint{2.849717in}{0.999587in}}%
\pgfpathlineto{\pgfqpoint{2.996081in}{1.074181in}}%
\pgfpathlineto{\pgfqpoint{3.142446in}{1.118069in}}%
\pgfpathlineto{\pgfqpoint{3.288810in}{1.153993in}}%
\pgfpathlineto{\pgfqpoint{3.435175in}{1.261275in}}%
\pgfpathlineto{\pgfqpoint{3.581539in}{1.290705in}}%
\pgfpathlineto{\pgfqpoint{3.727903in}{1.399401in}}%
\pgfpathlineto{\pgfqpoint{3.874268in}{1.208385in}}%
\pgfpathlineto{\pgfqpoint{4.020632in}{1.198958in}}%
\pgfpathlineto{\pgfqpoint{4.166997in}{1.546675in}}%
\pgfpathlineto{\pgfqpoint{4.313361in}{1.307699in}}%
\pgfpathlineto{\pgfqpoint{4.459725in}{1.625964in}}%
\pgfpathlineto{\pgfqpoint{4.898818in}{1.542269in}}%
\pgfpathlineto{\pgfqpoint{5.045183in}{1.483466in}}%
\pgfusepath{stroke}%
\end{pgfscope}%
\begin{pgfscope}%
\pgfpathrectangle{\pgfqpoint{0.588387in}{0.521603in}}{\pgfqpoint{4.669024in}{2.192138in}}%
\pgfusepath{clip}%
\pgfsetrectcap%
\pgfsetroundjoin%
\pgfsetlinewidth{1.505625pt}%
\pgfsetstrokecolor{currentstroke5}%
\pgfsetdash{}{0pt}%
\pgfpathmoveto{\pgfqpoint{0.800616in}{0.648732in}}%
\pgfpathlineto{\pgfqpoint{0.946980in}{0.702385in}}%
\pgfpathlineto{\pgfqpoint{1.093344in}{0.678261in}}%
\pgfpathlineto{\pgfqpoint{1.239709in}{0.706527in}}%
\pgfpathlineto{\pgfqpoint{1.386073in}{0.670364in}}%
\pgfpathlineto{\pgfqpoint{1.532438in}{0.694916in}}%
\pgfpathlineto{\pgfqpoint{1.678802in}{0.689547in}}%
\pgfpathlineto{\pgfqpoint{1.825166in}{0.685566in}}%
\pgfpathlineto{\pgfqpoint{1.971531in}{0.696788in}}%
\pgfpathlineto{\pgfqpoint{2.117895in}{0.736590in}}%
\pgfpathlineto{\pgfqpoint{2.264259in}{0.825608in}}%
\pgfpathlineto{\pgfqpoint{2.410624in}{0.840785in}}%
\pgfpathlineto{\pgfqpoint{2.556988in}{0.900558in}}%
\pgfpathlineto{\pgfqpoint{2.703353in}{0.948899in}}%
\pgfpathlineto{\pgfqpoint{2.849717in}{0.998796in}}%
\pgfpathlineto{\pgfqpoint{2.996081in}{1.072708in}}%
\pgfpathlineto{\pgfqpoint{3.142446in}{1.082861in}}%
\pgfpathlineto{\pgfqpoint{3.288810in}{1.111359in}}%
\pgfpathlineto{\pgfqpoint{3.435175in}{1.262819in}}%
\pgfpathlineto{\pgfqpoint{3.581539in}{1.314229in}}%
\pgfpathlineto{\pgfqpoint{3.727903in}{1.367237in}}%
\pgfpathlineto{\pgfqpoint{3.874268in}{1.191682in}}%
\pgfpathlineto{\pgfqpoint{4.020632in}{1.312610in}}%
\pgfpathlineto{\pgfqpoint{4.166997in}{1.473341in}}%
\pgfpathlineto{\pgfqpoint{4.313361in}{1.325317in}}%
\pgfpathlineto{\pgfqpoint{4.459725in}{1.547039in}}%
\pgfpathlineto{\pgfqpoint{4.898818in}{1.535534in}}%
\pgfpathlineto{\pgfqpoint{5.045183in}{1.527086in}}%
\pgfusepath{stroke}%
\end{pgfscope}%
\begin{pgfscope}%
\pgfpathrectangle{\pgfqpoint{0.588387in}{0.521603in}}{\pgfqpoint{4.669024in}{2.192138in}}%
\pgfusepath{clip}%
\pgfsetrectcap%
\pgfsetroundjoin%
\pgfsetlinewidth{1.505625pt}%
\pgfsetstrokecolor{currentstroke6}%
\pgfsetdash{}{0pt}%
\pgfpathmoveto{\pgfqpoint{0.800616in}{0.646306in}}%
\pgfpathlineto{\pgfqpoint{0.946980in}{0.706679in}}%
\pgfpathlineto{\pgfqpoint{1.093344in}{0.676179in}}%
\pgfpathlineto{\pgfqpoint{1.239709in}{0.708482in}}%
\pgfpathlineto{\pgfqpoint{1.386073in}{0.670174in}}%
\pgfpathlineto{\pgfqpoint{1.532438in}{0.660206in}}%
\pgfpathlineto{\pgfqpoint{1.678802in}{0.692798in}}%
\pgfpathlineto{\pgfqpoint{1.825166in}{0.684048in}}%
\pgfpathlineto{\pgfqpoint{1.971531in}{0.696812in}}%
\pgfpathlineto{\pgfqpoint{2.117895in}{0.739902in}}%
\pgfpathlineto{\pgfqpoint{2.264259in}{0.829026in}}%
\pgfpathlineto{\pgfqpoint{2.410624in}{0.848832in}}%
\pgfpathlineto{\pgfqpoint{2.556988in}{0.904444in}}%
\pgfpathlineto{\pgfqpoint{2.703353in}{0.941878in}}%
\pgfpathlineto{\pgfqpoint{2.849717in}{0.999192in}}%
\pgfpathlineto{\pgfqpoint{2.996081in}{1.080715in}}%
\pgfpathlineto{\pgfqpoint{3.142446in}{1.085521in}}%
\pgfpathlineto{\pgfqpoint{3.288810in}{1.163340in}}%
\pgfpathlineto{\pgfqpoint{3.435175in}{1.287184in}}%
\pgfpathlineto{\pgfqpoint{3.581539in}{1.341051in}}%
\pgfpathlineto{\pgfqpoint{3.727903in}{1.449698in}}%
\pgfpathlineto{\pgfqpoint{3.874268in}{1.276360in}}%
\pgfpathlineto{\pgfqpoint{4.020632in}{1.220870in}}%
\pgfpathlineto{\pgfqpoint{4.166997in}{1.674480in}}%
\pgfpathlineto{\pgfqpoint{4.313361in}{1.377816in}}%
\pgfpathlineto{\pgfqpoint{4.459725in}{1.632555in}}%
\pgfpathlineto{\pgfqpoint{4.898818in}{1.649439in}}%
\pgfpathlineto{\pgfqpoint{5.045183in}{1.593070in}}%
\pgfusepath{stroke}%
\end{pgfscope}%
\begin{pgfscope}%
\pgfpathrectangle{\pgfqpoint{0.588387in}{0.521603in}}{\pgfqpoint{4.669024in}{2.192138in}}%
\pgfusepath{clip}%
\pgfsetrectcap%
\pgfsetroundjoin%
\pgfsetlinewidth{1.505625pt}%
\pgfsetstrokecolor{currentstroke7}%
\pgfsetdash{}{0pt}%
\pgfpathmoveto{\pgfqpoint{0.800616in}{0.646776in}}%
\pgfpathlineto{\pgfqpoint{0.946980in}{0.706314in}}%
\pgfpathlineto{\pgfqpoint{1.093344in}{0.677569in}}%
\pgfpathlineto{\pgfqpoint{1.239709in}{0.707506in}}%
\pgfpathlineto{\pgfqpoint{1.386073in}{0.662298in}}%
\pgfpathlineto{\pgfqpoint{1.532438in}{0.663729in}}%
\pgfpathlineto{\pgfqpoint{1.678802in}{0.691890in}}%
\pgfpathlineto{\pgfqpoint{1.825166in}{0.680301in}}%
\pgfpathlineto{\pgfqpoint{1.971531in}{0.695213in}}%
\pgfpathlineto{\pgfqpoint{2.117895in}{0.737422in}}%
\pgfpathlineto{\pgfqpoint{2.264259in}{0.830720in}}%
\pgfpathlineto{\pgfqpoint{2.410624in}{0.841294in}}%
\pgfpathlineto{\pgfqpoint{2.556988in}{0.902896in}}%
\pgfpathlineto{\pgfqpoint{2.703353in}{0.946723in}}%
\pgfpathlineto{\pgfqpoint{2.849717in}{1.002338in}}%
\pgfpathlineto{\pgfqpoint{2.996081in}{1.082147in}}%
\pgfpathlineto{\pgfqpoint{3.142446in}{1.083750in}}%
\pgfpathlineto{\pgfqpoint{3.288810in}{1.167104in}}%
\pgfpathlineto{\pgfqpoint{3.435175in}{1.262819in}}%
\pgfpathlineto{\pgfqpoint{3.581539in}{1.320932in}}%
\pgfpathlineto{\pgfqpoint{3.727903in}{1.420626in}}%
\pgfpathlineto{\pgfqpoint{3.874268in}{1.226373in}}%
\pgfpathlineto{\pgfqpoint{4.020632in}{1.182959in}}%
\pgfpathlineto{\pgfqpoint{4.166997in}{1.555997in}}%
\pgfpathlineto{\pgfqpoint{4.313361in}{1.285408in}}%
\pgfpathlineto{\pgfqpoint{4.459725in}{1.581011in}}%
\pgfpathlineto{\pgfqpoint{4.898818in}{1.526306in}}%
\pgfpathlineto{\pgfqpoint{5.045183in}{1.562972in}}%
\pgfusepath{stroke}%
\end{pgfscope}%
\begin{pgfscope}%
\pgfpathrectangle{\pgfqpoint{0.588387in}{0.521603in}}{\pgfqpoint{4.669024in}{2.192138in}}%
\pgfusepath{clip}%
\pgfsetrectcap%
\pgfsetroundjoin%
\pgfsetlinewidth{1.505625pt}%
\definecolor{currentstroke}{rgb}{0.498039,0.498039,0.498039}%
\pgfsetstrokecolor{currentstroke}%
\pgfsetdash{}{0pt}%
\pgfpathmoveto{\pgfqpoint{0.800616in}{0.653001in}}%
\pgfpathlineto{\pgfqpoint{0.946980in}{0.707288in}}%
\pgfpathlineto{\pgfqpoint{1.093344in}{0.681504in}}%
\pgfpathlineto{\pgfqpoint{1.239709in}{0.710909in}}%
\pgfpathlineto{\pgfqpoint{1.386073in}{0.671131in}}%
\pgfpathlineto{\pgfqpoint{1.532438in}{0.668575in}}%
\pgfpathlineto{\pgfqpoint{1.678802in}{0.694349in}}%
\pgfpathlineto{\pgfqpoint{1.825166in}{0.687578in}}%
\pgfpathlineto{\pgfqpoint{1.971531in}{0.703122in}}%
\pgfpathlineto{\pgfqpoint{2.117895in}{0.754616in}}%
\pgfpathlineto{\pgfqpoint{2.264259in}{0.841498in}}%
\pgfpathlineto{\pgfqpoint{2.410624in}{0.864762in}}%
\pgfpathlineto{\pgfqpoint{2.556988in}{0.933101in}}%
\pgfpathlineto{\pgfqpoint{2.703353in}{1.006906in}}%
\pgfpathlineto{\pgfqpoint{2.849717in}{1.070424in}}%
\pgfpathlineto{\pgfqpoint{2.996081in}{1.204650in}}%
\pgfpathlineto{\pgfqpoint{3.142446in}{1.169365in}}%
\pgfpathlineto{\pgfqpoint{3.288810in}{1.206524in}}%
\pgfpathlineto{\pgfqpoint{3.435175in}{1.434749in}}%
\pgfpathlineto{\pgfqpoint{3.581539in}{1.373893in}}%
\pgfpathlineto{\pgfqpoint{3.727903in}{1.494125in}}%
\pgfpathlineto{\pgfqpoint{3.874268in}{1.186729in}}%
\pgfpathlineto{\pgfqpoint{4.020632in}{1.521580in}}%
\pgfpathlineto{\pgfqpoint{4.166997in}{1.651991in}}%
\pgfpathlineto{\pgfqpoint{4.313361in}{1.463822in}}%
\pgfpathlineto{\pgfqpoint{4.459725in}{1.867582in}}%
\pgfpathlineto{\pgfqpoint{4.898818in}{2.146188in}}%
\pgfpathlineto{\pgfqpoint{5.045183in}{1.946437in}}%
\pgfusepath{stroke}%
\end{pgfscope}%
\begin{pgfscope}%
\pgfpathrectangle{\pgfqpoint{0.588387in}{0.521603in}}{\pgfqpoint{4.669024in}{2.192138in}}%
\pgfusepath{clip}%
\pgfsetrectcap%
\pgfsetroundjoin%
\pgfsetlinewidth{1.505625pt}%
\definecolor{currentstroke}{rgb}{0.737255,0.741176,0.133333}%
\pgfsetstrokecolor{currentstroke}%
\pgfsetdash{}{0pt}%
\pgfpathmoveto{\pgfqpoint{0.800616in}{0.645339in}}%
\pgfpathlineto{\pgfqpoint{0.946980in}{0.701034in}}%
\pgfpathlineto{\pgfqpoint{1.093344in}{0.674081in}}%
\pgfpathlineto{\pgfqpoint{1.239709in}{0.703568in}}%
\pgfpathlineto{\pgfqpoint{1.386073in}{0.667811in}}%
\pgfpathlineto{\pgfqpoint{1.532438in}{0.662319in}}%
\pgfpathlineto{\pgfqpoint{1.678802in}{0.689410in}}%
\pgfpathlineto{\pgfqpoint{1.825166in}{0.689755in}}%
\pgfpathlineto{\pgfqpoint{1.971531in}{0.698402in}}%
\pgfpathlineto{\pgfqpoint{2.117895in}{0.748019in}}%
\pgfpathlineto{\pgfqpoint{2.264259in}{0.834357in}}%
\pgfpathlineto{\pgfqpoint{2.410624in}{0.856181in}}%
\pgfpathlineto{\pgfqpoint{2.556988in}{0.919857in}}%
\pgfpathlineto{\pgfqpoint{2.703353in}{0.990078in}}%
\pgfpathlineto{\pgfqpoint{2.849717in}{1.059050in}}%
\pgfpathlineto{\pgfqpoint{2.996081in}{1.192173in}}%
\pgfpathlineto{\pgfqpoint{3.142446in}{1.131196in}}%
\pgfpathlineto{\pgfqpoint{3.288810in}{1.207456in}}%
\pgfpathlineto{\pgfqpoint{3.435175in}{1.379370in}}%
\pgfpathlineto{\pgfqpoint{3.581539in}{1.343690in}}%
\pgfpathlineto{\pgfqpoint{3.727903in}{1.482560in}}%
\pgfpathlineto{\pgfqpoint{3.874268in}{1.184221in}}%
\pgfpathlineto{\pgfqpoint{4.020632in}{1.444317in}}%
\pgfpathlineto{\pgfqpoint{4.166997in}{1.781934in}}%
\pgfpathlineto{\pgfqpoint{4.313361in}{1.464788in}}%
\pgfpathlineto{\pgfqpoint{4.459725in}{1.799323in}}%
\pgfpathlineto{\pgfqpoint{4.898818in}{1.685107in}}%
\pgfusepath{stroke}%
\end{pgfscope}%
\begin{pgfscope}%
\pgfsetrectcap%
\pgfsetmiterjoin%
\pgfsetlinewidth{0.803000pt}%
\definecolor{currentstroke}{rgb}{0.000000,0.000000,0.000000}%
\pgfsetstrokecolor{currentstroke}%
\pgfsetdash{}{0pt}%
\pgfpathmoveto{\pgfqpoint{0.588387in}{0.521603in}}%
\pgfpathlineto{\pgfqpoint{0.588387in}{2.713741in}}%
\pgfusepath{stroke}%
\end{pgfscope}%
\begin{pgfscope}%
\pgfsetrectcap%
\pgfsetmiterjoin%
\pgfsetlinewidth{0.803000pt}%
\definecolor{currentstroke}{rgb}{0.000000,0.000000,0.000000}%
\pgfsetstrokecolor{currentstroke}%
\pgfsetdash{}{0pt}%
\pgfpathmoveto{\pgfqpoint{5.257411in}{0.521603in}}%
\pgfpathlineto{\pgfqpoint{5.257411in}{2.713741in}}%
\pgfusepath{stroke}%
\end{pgfscope}%
\begin{pgfscope}%
\pgfsetrectcap%
\pgfsetmiterjoin%
\pgfsetlinewidth{0.803000pt}%
\definecolor{currentstroke}{rgb}{0.000000,0.000000,0.000000}%
\pgfsetstrokecolor{currentstroke}%
\pgfsetdash{}{0pt}%
\pgfpathmoveto{\pgfqpoint{0.588387in}{0.521603in}}%
\pgfpathlineto{\pgfqpoint{5.257411in}{0.521603in}}%
\pgfusepath{stroke}%
\end{pgfscope}%
\begin{pgfscope}%
\pgfsetrectcap%
\pgfsetmiterjoin%
\pgfsetlinewidth{0.803000pt}%
\definecolor{currentstroke}{rgb}{0.000000,0.000000,0.000000}%
\pgfsetstrokecolor{currentstroke}%
\pgfsetdash{}{0pt}%
\pgfpathmoveto{\pgfqpoint{0.588387in}{2.713741in}}%
\pgfpathlineto{\pgfqpoint{5.257411in}{2.713741in}}%
\pgfusepath{stroke}%
\end{pgfscope}%
\begin{pgfscope}%
\pgfsetbuttcap%
\pgfsetmiterjoin%
\definecolor{currentfill}{rgb}{1.000000,1.000000,1.000000}%
\pgfsetfillcolor{currentfill}%
\pgfsetfillopacity{0.800000}%
\pgfsetlinewidth{1.003750pt}%
\definecolor{currentstroke}{rgb}{0.800000,0.800000,0.800000}%
\pgfsetstrokecolor{currentstroke}%
\pgfsetstrokeopacity{0.800000}%
\pgfsetdash{}{0pt}%
\pgfpathmoveto{\pgfqpoint{5.344911in}{0.941624in}}%
\pgfpathlineto{\pgfqpoint{8.259376in}{0.941624in}}%
\pgfpathquadraticcurveto{\pgfqpoint{8.284376in}{0.941624in}}{\pgfqpoint{8.284376in}{0.966624in}}%
\pgfpathlineto{\pgfqpoint{8.284376in}{2.626241in}}%
\pgfpathquadraticcurveto{\pgfqpoint{8.284376in}{2.651241in}}{\pgfqpoint{8.259376in}{2.651241in}}%
\pgfpathlineto{\pgfqpoint{5.344911in}{2.651241in}}%
\pgfpathquadraticcurveto{\pgfqpoint{5.319911in}{2.651241in}}{\pgfqpoint{5.319911in}{2.626241in}}%
\pgfpathlineto{\pgfqpoint{5.319911in}{0.966624in}}%
\pgfpathquadraticcurveto{\pgfqpoint{5.319911in}{0.941624in}}{\pgfqpoint{5.344911in}{0.941624in}}%
\pgfpathlineto{\pgfqpoint{5.344911in}{0.941624in}}%
\pgfpathclose%
\pgfusepath{stroke,fill}%
\end{pgfscope}%
\begin{pgfscope}%
\pgfsetrectcap%
\pgfsetroundjoin%
\pgfsetlinewidth{1.505625pt}%
\pgfsetstrokecolor{currentstroke3}%
\pgfsetdash{}{0pt}%
\pgfpathmoveto{\pgfqpoint{5.369911in}{2.550021in}}%
\pgfpathlineto{\pgfqpoint{5.494911in}{2.550021in}}%
\pgfpathlineto{\pgfqpoint{5.619911in}{2.550021in}}%
\pgfusepath{stroke}%
\end{pgfscope}%
\begin{pgfscope}%
\definecolor{textcolor}{rgb}{0.000000,0.000000,0.000000}%
\pgfsetstrokecolor{textcolor}%
\pgfsetfillcolor{textcolor}%
\pgftext[x=5.719911in,y=2.506271in,left,base]{\color{textcolor}{\rmfamily\fontsize{9.000000}{10.800000}\selectfont\catcode`\^=\active\def^{\ifmmode\sp\else\^{}\fi}\catcode`\%=\active\def%{\%}\NaiveCycles{}}}%
\end{pgfscope}%
\begin{pgfscope}%
\pgfsetrectcap%
\pgfsetroundjoin%
\pgfsetlinewidth{1.505625pt}%
\pgfsetstrokecolor{currentstroke1}%
\pgfsetdash{}{0pt}%
\pgfpathmoveto{\pgfqpoint{5.369911in}{2.366549in}}%
\pgfpathlineto{\pgfqpoint{5.494911in}{2.366549in}}%
\pgfpathlineto{\pgfqpoint{5.619911in}{2.366549in}}%
\pgfusepath{stroke}%
\end{pgfscope}%
\begin{pgfscope}%
\definecolor{textcolor}{rgb}{0.000000,0.000000,0.000000}%
\pgfsetstrokecolor{textcolor}%
\pgfsetfillcolor{textcolor}%
\pgftext[x=5.719911in,y=2.322799in,left,base]{\color{textcolor}{\rmfamily\fontsize{9.000000}{10.800000}\selectfont\catcode`\^=\active\def^{\ifmmode\sp\else\^{}\fi}\catcode`\%=\active\def%{\%}\CyclesMatchChunks{} \& \MergeLinear{}}}%
\end{pgfscope}%
\begin{pgfscope}%
\pgfsetrectcap%
\pgfsetroundjoin%
\pgfsetlinewidth{1.505625pt}%
\pgfsetstrokecolor{currentstroke2}%
\pgfsetdash{}{0pt}%
\pgfpathmoveto{\pgfqpoint{5.369911in}{2.179599in}}%
\pgfpathlineto{\pgfqpoint{5.494911in}{2.179599in}}%
\pgfpathlineto{\pgfqpoint{5.619911in}{2.179599in}}%
\pgfusepath{stroke}%
\end{pgfscope}%
\begin{pgfscope}%
\definecolor{textcolor}{rgb}{0.000000,0.000000,0.000000}%
\pgfsetstrokecolor{textcolor}%
\pgfsetfillcolor{textcolor}%
\pgftext[x=5.719911in,y=2.135849in,left,base]{\color{textcolor}{\rmfamily\fontsize{9.000000}{10.800000}\selectfont\catcode`\^=\active\def^{\ifmmode\sp\else\^{}\fi}\catcode`\%=\active\def%{\%}\CyclesMatchChunks{} \& \SharedVertices{}}}%
\end{pgfscope}%
\begin{pgfscope}%
\pgfsetrectcap%
\pgfsetroundjoin%
\pgfsetlinewidth{1.505625pt}%
\pgfsetstrokecolor{currentstroke4}%
\pgfsetdash{}{0pt}%
\pgfpathmoveto{\pgfqpoint{5.369911in}{1.992648in}}%
\pgfpathlineto{\pgfqpoint{5.494911in}{1.992648in}}%
\pgfpathlineto{\pgfqpoint{5.619911in}{1.992648in}}%
\pgfusepath{stroke}%
\end{pgfscope}%
\begin{pgfscope}%
\definecolor{textcolor}{rgb}{0.000000,0.000000,0.000000}%
\pgfsetstrokecolor{textcolor}%
\pgfsetfillcolor{textcolor}%
\pgftext[x=5.719911in,y=1.948898in,left,base]{\color{textcolor}{\rmfamily\fontsize{9.000000}{10.800000}\selectfont\catcode`\^=\active\def^{\ifmmode\sp\else\^{}\fi}\catcode`\%=\active\def%{\%}\Neighbors{} \& \MergeLinear{}}}%
\end{pgfscope}%
\begin{pgfscope}%
\pgfsetrectcap%
\pgfsetroundjoin%
\pgfsetlinewidth{1.505625pt}%
\pgfsetstrokecolor{currentstroke5}%
\pgfsetdash{}{0pt}%
\pgfpathmoveto{\pgfqpoint{5.369911in}{1.809177in}}%
\pgfpathlineto{\pgfqpoint{5.494911in}{1.809177in}}%
\pgfpathlineto{\pgfqpoint{5.619911in}{1.809177in}}%
\pgfusepath{stroke}%
\end{pgfscope}%
\begin{pgfscope}%
\definecolor{textcolor}{rgb}{0.000000,0.000000,0.000000}%
\pgfsetstrokecolor{textcolor}%
\pgfsetfillcolor{textcolor}%
\pgftext[x=5.719911in,y=1.765427in,left,base]{\color{textcolor}{\rmfamily\fontsize{9.000000}{10.800000}\selectfont\catcode`\^=\active\def^{\ifmmode\sp\else\^{}\fi}\catcode`\%=\active\def%{\%}\Neighbors{} \& \SharedVertices{}}}%
\end{pgfscope}%
\begin{pgfscope}%
\pgfsetrectcap%
\pgfsetroundjoin%
\pgfsetlinewidth{1.505625pt}%
\pgfsetstrokecolor{currentstroke6}%
\pgfsetdash{}{0pt}%
\pgfpathmoveto{\pgfqpoint{5.369911in}{1.622226in}}%
\pgfpathlineto{\pgfqpoint{5.494911in}{1.622226in}}%
\pgfpathlineto{\pgfqpoint{5.619911in}{1.622226in}}%
\pgfusepath{stroke}%
\end{pgfscope}%
\begin{pgfscope}%
\definecolor{textcolor}{rgb}{0.000000,0.000000,0.000000}%
\pgfsetstrokecolor{textcolor}%
\pgfsetfillcolor{textcolor}%
\pgftext[x=5.719911in,y=1.578476in,left,base]{\color{textcolor}{\rmfamily\fontsize{9.000000}{10.800000}\selectfont\catcode`\^=\active\def^{\ifmmode\sp\else\^{}\fi}\catcode`\%=\active\def%{\%}\NeighborsDegree{} \& \MergeLinear{}}}%
\end{pgfscope}%
\begin{pgfscope}%
\pgfsetrectcap%
\pgfsetroundjoin%
\pgfsetlinewidth{1.505625pt}%
\pgfsetstrokecolor{currentstroke7}%
\pgfsetdash{}{0pt}%
\pgfpathmoveto{\pgfqpoint{5.369911in}{1.435276in}}%
\pgfpathlineto{\pgfqpoint{5.494911in}{1.435276in}}%
\pgfpathlineto{\pgfqpoint{5.619911in}{1.435276in}}%
\pgfusepath{stroke}%
\end{pgfscope}%
\begin{pgfscope}%
\definecolor{textcolor}{rgb}{0.000000,0.000000,0.000000}%
\pgfsetstrokecolor{textcolor}%
\pgfsetfillcolor{textcolor}%
\pgftext[x=5.719911in,y=1.391526in,left,base]{\color{textcolor}{\rmfamily\fontsize{9.000000}{10.800000}\selectfont\catcode`\^=\active\def^{\ifmmode\sp\else\^{}\fi}\catcode`\%=\active\def%{\%}\NeighborsDegree{} \& \SharedVertices{}}}%
\end{pgfscope}%
\begin{pgfscope}%
\pgfsetrectcap%
\pgfsetroundjoin%
\pgfsetlinewidth{1.505625pt}%
\definecolor{currentstroke}{rgb}{0.498039,0.498039,0.498039}%
\pgfsetstrokecolor{currentstroke}%
\pgfsetdash{}{0pt}%
\pgfpathmoveto{\pgfqpoint{5.369911in}{1.248326in}}%
\pgfpathlineto{\pgfqpoint{5.494911in}{1.248326in}}%
\pgfpathlineto{\pgfqpoint{5.619911in}{1.248326in}}%
\pgfusepath{stroke}%
\end{pgfscope}%
\begin{pgfscope}%
\definecolor{textcolor}{rgb}{0.000000,0.000000,0.000000}%
\pgfsetstrokecolor{textcolor}%
\pgfsetfillcolor{textcolor}%
\pgftext[x=5.719911in,y=1.204576in,left,base]{\color{textcolor}{\rmfamily\fontsize{9.000000}{10.800000}\selectfont\catcode`\^=\active\def^{\ifmmode\sp\else\^{}\fi}\catcode`\%=\active\def%{\%}\None{} \& \MergeLinear{}}}%
\end{pgfscope}%
\begin{pgfscope}%
\pgfsetrectcap%
\pgfsetroundjoin%
\pgfsetlinewidth{1.505625pt}%
\definecolor{currentstroke}{rgb}{0.737255,0.741176,0.133333}%
\pgfsetstrokecolor{currentstroke}%
\pgfsetdash{}{0pt}%
\pgfpathmoveto{\pgfqpoint{5.369911in}{1.064854in}}%
\pgfpathlineto{\pgfqpoint{5.494911in}{1.064854in}}%
\pgfpathlineto{\pgfqpoint{5.619911in}{1.064854in}}%
\pgfusepath{stroke}%
\end{pgfscope}%
\begin{pgfscope}%
\definecolor{textcolor}{rgb}{0.000000,0.000000,0.000000}%
\pgfsetstrokecolor{textcolor}%
\pgfsetfillcolor{textcolor}%
\pgftext[x=5.719911in,y=1.021104in,left,base]{\color{textcolor}{\rmfamily\fontsize{9.000000}{10.800000}\selectfont\catcode`\^=\active\def^{\ifmmode\sp\else\^{}\fi}\catcode`\%=\active\def%{\%}\None{} \& \SharedVertices{}}}%
\end{pgfscope}%
\end{pgfpicture}%
\makeatother%
\endgroup%
}
	\caption[Mean runtime for globally rigid graphs (all).]{
		Mean running time (ms) to find all NAC-colorings for globally rigid graphs.}%
	\label{fig:graph_globally_rigid_all_runtime}
\end{figure}
\begin{figure}[p]
	\centering
	\scalebox{0.5}{%% Creator: Matplotlib, PGF backend
%%
%% To include the figure in your LaTeX document, write
%%   \input{<filename>.pgf}
%%
%% Make sure the required packages are loaded in your preamble
%%   \usepackage{pgf}
%%
%% Also ensure that all the required font packages are loaded; for instance,
%% the lmodern package is sometimes necessary when using math font.
%%   \usepackage{lmodern}
%%
%% Figures using additional raster images can only be included by \input if
%% they are in the same directory as the main LaTeX file. For loading figures
%% from other directories you can use the `import` package
%%   \usepackage{import}
%%
%% and then include the figures with
%%   \import{<path to file>}{<filename>.pgf}
%%
%% Matplotlib used the following preamble
%%   \def\mathdefault#1{#1}
%%   \everymath=\expandafter{\the\everymath\displaystyle}
%%   \IfFileExists{scrextend.sty}{
%%     \usepackage[fontsize=10.000000pt]{scrextend}
%%   }{
%%     \renewcommand{\normalsize}{\fontsize{10.000000}{12.000000}\selectfont}
%%     \normalsize
%%   }
%%   
%%   \ifdefined\pdftexversion\else  % non-pdftex case.
%%     \usepackage{fontspec}
%%     \setmainfont{DejaVuSans.ttf}[Path=\detokenize{/home/petr/Projects/PyRigi/.venv/lib/python3.12/site-packages/matplotlib/mpl-data/fonts/ttf/}]
%%     \setsansfont{DejaVuSans.ttf}[Path=\detokenize{/home/petr/Projects/PyRigi/.venv/lib/python3.12/site-packages/matplotlib/mpl-data/fonts/ttf/}]
%%     \setmonofont{DejaVuSansMono.ttf}[Path=\detokenize{/home/petr/Projects/PyRigi/.venv/lib/python3.12/site-packages/matplotlib/mpl-data/fonts/ttf/}]
%%   \fi
%%   \makeatletter\@ifpackageloaded{under\Score{}}{}{\usepackage[strings]{under\Score{}}}\makeatother
%%
\begingroup%
\makeatletter%
\begin{pgfpicture}%
\pgfpathrectangle{\pgfpointorigin}{\pgfqpoint{8.384376in}{2.841849in}}%
\pgfusepath{use as bounding box, clip}%
\begin{pgfscope}%
\pgfsetbuttcap%
\pgfsetmiterjoin%
\definecolor{currentfill}{rgb}{1.000000,1.000000,1.000000}%
\pgfsetfillcolor{currentfill}%
\pgfsetlinewidth{0.000000pt}%
\definecolor{currentstroke}{rgb}{1.000000,1.000000,1.000000}%
\pgfsetstrokecolor{currentstroke}%
\pgfsetdash{}{0pt}%
\pgfpathmoveto{\pgfqpoint{0.000000in}{0.000000in}}%
\pgfpathlineto{\pgfqpoint{8.384376in}{0.000000in}}%
\pgfpathlineto{\pgfqpoint{8.384376in}{2.841849in}}%
\pgfpathlineto{\pgfqpoint{0.000000in}{2.841849in}}%
\pgfpathlineto{\pgfqpoint{0.000000in}{0.000000in}}%
\pgfpathclose%
\pgfusepath{fill}%
\end{pgfscope}%
\begin{pgfscope}%
\pgfsetbuttcap%
\pgfsetmiterjoin%
\definecolor{currentfill}{rgb}{1.000000,1.000000,1.000000}%
\pgfsetfillcolor{currentfill}%
\pgfsetlinewidth{0.000000pt}%
\definecolor{currentstroke}{rgb}{0.000000,0.000000,0.000000}%
\pgfsetstrokecolor{currentstroke}%
\pgfsetstrokeopacity{0.000000}%
\pgfsetdash{}{0pt}%
\pgfpathmoveto{\pgfqpoint{0.588387in}{0.521603in}}%
\pgfpathlineto{\pgfqpoint{5.257411in}{0.521603in}}%
\pgfpathlineto{\pgfqpoint{5.257411in}{2.531888in}}%
\pgfpathlineto{\pgfqpoint{0.588387in}{2.531888in}}%
\pgfpathlineto{\pgfqpoint{0.588387in}{0.521603in}}%
\pgfpathclose%
\pgfusepath{fill}%
\end{pgfscope}%
\begin{pgfscope}%
\pgfsetbuttcap%
\pgfsetroundjoin%
\definecolor{currentfill}{rgb}{0.000000,0.000000,0.000000}%
\pgfsetfillcolor{currentfill}%
\pgfsetlinewidth{0.803000pt}%
\definecolor{currentstroke}{rgb}{0.000000,0.000000,0.000000}%
\pgfsetstrokecolor{currentstroke}%
\pgfsetdash{}{0pt}%
\pgfsys@defobject{currentmarker}{\pgfqpoint{0.000000in}{-0.048611in}}{\pgfqpoint{0.000000in}{0.000000in}}{%
\pgfpathmoveto{\pgfqpoint{0.000000in}{0.000000in}}%
\pgfpathlineto{\pgfqpoint{0.000000in}{-0.048611in}}%
\pgfusepath{stroke,fill}%
}%
\begin{pgfscope}%
\pgfsys@transformshift{0.993550in}{0.521603in}%
\pgfsys@useobject{currentmarker}{}%
\end{pgfscope}%
\end{pgfscope}%
\begin{pgfscope}%
\definecolor{textcolor}{rgb}{0.000000,0.000000,0.000000}%
\pgfsetstrokecolor{textcolor}%
\pgfsetfillcolor{textcolor}%
\pgftext[x=0.993550in,y=0.424381in,,top]{\color{textcolor}{\rmfamily\fontsize{10.000000}{12.000000}\selectfont\catcode`\^=\active\def^{\ifmmode\sp\else\^{}\fi}\catcode`\%=\active\def%{\%}$\mathdefault{3}$}}%
\end{pgfscope}%
\begin{pgfscope}%
\pgfsetbuttcap%
\pgfsetroundjoin%
\definecolor{currentfill}{rgb}{0.000000,0.000000,0.000000}%
\pgfsetfillcolor{currentfill}%
\pgfsetlinewidth{0.803000pt}%
\definecolor{currentstroke}{rgb}{0.000000,0.000000,0.000000}%
\pgfsetstrokecolor{currentstroke}%
\pgfsetdash{}{0pt}%
\pgfsys@defobject{currentmarker}{\pgfqpoint{0.000000in}{-0.048611in}}{\pgfqpoint{0.000000in}{0.000000in}}{%
\pgfpathmoveto{\pgfqpoint{0.000000in}{0.000000in}}%
\pgfpathlineto{\pgfqpoint{0.000000in}{-0.048611in}}%
\pgfusepath{stroke,fill}%
}%
\begin{pgfscope}%
\pgfsys@transformshift{1.572355in}{0.521603in}%
\pgfsys@useobject{currentmarker}{}%
\end{pgfscope}%
\end{pgfscope}%
\begin{pgfscope}%
\definecolor{textcolor}{rgb}{0.000000,0.000000,0.000000}%
\pgfsetstrokecolor{textcolor}%
\pgfsetfillcolor{textcolor}%
\pgftext[x=1.572355in,y=0.424381in,,top]{\color{textcolor}{\rmfamily\fontsize{10.000000}{12.000000}\selectfont\catcode`\^=\active\def^{\ifmmode\sp\else\^{}\fi}\catcode`\%=\active\def%{\%}$\mathdefault{6}$}}%
\end{pgfscope}%
\begin{pgfscope}%
\pgfsetbuttcap%
\pgfsetroundjoin%
\definecolor{currentfill}{rgb}{0.000000,0.000000,0.000000}%
\pgfsetfillcolor{currentfill}%
\pgfsetlinewidth{0.803000pt}%
\definecolor{currentstroke}{rgb}{0.000000,0.000000,0.000000}%
\pgfsetstrokecolor{currentstroke}%
\pgfsetdash{}{0pt}%
\pgfsys@defobject{currentmarker}{\pgfqpoint{0.000000in}{-0.048611in}}{\pgfqpoint{0.000000in}{0.000000in}}{%
\pgfpathmoveto{\pgfqpoint{0.000000in}{0.000000in}}%
\pgfpathlineto{\pgfqpoint{0.000000in}{-0.048611in}}%
\pgfusepath{stroke,fill}%
}%
\begin{pgfscope}%
\pgfsys@transformshift{2.151160in}{0.521603in}%
\pgfsys@useobject{currentmarker}{}%
\end{pgfscope}%
\end{pgfscope}%
\begin{pgfscope}%
\definecolor{textcolor}{rgb}{0.000000,0.000000,0.000000}%
\pgfsetstrokecolor{textcolor}%
\pgfsetfillcolor{textcolor}%
\pgftext[x=2.151160in,y=0.424381in,,top]{\color{textcolor}{\rmfamily\fontsize{10.000000}{12.000000}\selectfont\catcode`\^=\active\def^{\ifmmode\sp\else\^{}\fi}\catcode`\%=\active\def%{\%}$\mathdefault{9}$}}%
\end{pgfscope}%
\begin{pgfscope}%
\pgfsetbuttcap%
\pgfsetroundjoin%
\definecolor{currentfill}{rgb}{0.000000,0.000000,0.000000}%
\pgfsetfillcolor{currentfill}%
\pgfsetlinewidth{0.803000pt}%
\definecolor{currentstroke}{rgb}{0.000000,0.000000,0.000000}%
\pgfsetstrokecolor{currentstroke}%
\pgfsetdash{}{0pt}%
\pgfsys@defobject{currentmarker}{\pgfqpoint{0.000000in}{-0.048611in}}{\pgfqpoint{0.000000in}{0.000000in}}{%
\pgfpathmoveto{\pgfqpoint{0.000000in}{0.000000in}}%
\pgfpathlineto{\pgfqpoint{0.000000in}{-0.048611in}}%
\pgfusepath{stroke,fill}%
}%
\begin{pgfscope}%
\pgfsys@transformshift{2.729964in}{0.521603in}%
\pgfsys@useobject{currentmarker}{}%
\end{pgfscope}%
\end{pgfscope}%
\begin{pgfscope}%
\definecolor{textcolor}{rgb}{0.000000,0.000000,0.000000}%
\pgfsetstrokecolor{textcolor}%
\pgfsetfillcolor{textcolor}%
\pgftext[x=2.729964in,y=0.424381in,,top]{\color{textcolor}{\rmfamily\fontsize{10.000000}{12.000000}\selectfont\catcode`\^=\active\def^{\ifmmode\sp\else\^{}\fi}\catcode`\%=\active\def%{\%}$\mathdefault{12}$}}%
\end{pgfscope}%
\begin{pgfscope}%
\pgfsetbuttcap%
\pgfsetroundjoin%
\definecolor{currentfill}{rgb}{0.000000,0.000000,0.000000}%
\pgfsetfillcolor{currentfill}%
\pgfsetlinewidth{0.803000pt}%
\definecolor{currentstroke}{rgb}{0.000000,0.000000,0.000000}%
\pgfsetstrokecolor{currentstroke}%
\pgfsetdash{}{0pt}%
\pgfsys@defobject{currentmarker}{\pgfqpoint{0.000000in}{-0.048611in}}{\pgfqpoint{0.000000in}{0.000000in}}{%
\pgfpathmoveto{\pgfqpoint{0.000000in}{0.000000in}}%
\pgfpathlineto{\pgfqpoint{0.000000in}{-0.048611in}}%
\pgfusepath{stroke,fill}%
}%
\begin{pgfscope}%
\pgfsys@transformshift{3.308769in}{0.521603in}%
\pgfsys@useobject{currentmarker}{}%
\end{pgfscope}%
\end{pgfscope}%
\begin{pgfscope}%
\definecolor{textcolor}{rgb}{0.000000,0.000000,0.000000}%
\pgfsetstrokecolor{textcolor}%
\pgfsetfillcolor{textcolor}%
\pgftext[x=3.308769in,y=0.424381in,,top]{\color{textcolor}{\rmfamily\fontsize{10.000000}{12.000000}\selectfont\catcode`\^=\active\def^{\ifmmode\sp\else\^{}\fi}\catcode`\%=\active\def%{\%}$\mathdefault{15}$}}%
\end{pgfscope}%
\begin{pgfscope}%
\pgfsetbuttcap%
\pgfsetroundjoin%
\definecolor{currentfill}{rgb}{0.000000,0.000000,0.000000}%
\pgfsetfillcolor{currentfill}%
\pgfsetlinewidth{0.803000pt}%
\definecolor{currentstroke}{rgb}{0.000000,0.000000,0.000000}%
\pgfsetstrokecolor{currentstroke}%
\pgfsetdash{}{0pt}%
\pgfsys@defobject{currentmarker}{\pgfqpoint{0.000000in}{-0.048611in}}{\pgfqpoint{0.000000in}{0.000000in}}{%
\pgfpathmoveto{\pgfqpoint{0.000000in}{0.000000in}}%
\pgfpathlineto{\pgfqpoint{0.000000in}{-0.048611in}}%
\pgfusepath{stroke,fill}%
}%
\begin{pgfscope}%
\pgfsys@transformshift{3.887574in}{0.521603in}%
\pgfsys@useobject{currentmarker}{}%
\end{pgfscope}%
\end{pgfscope}%
\begin{pgfscope}%
\definecolor{textcolor}{rgb}{0.000000,0.000000,0.000000}%
\pgfsetstrokecolor{textcolor}%
\pgfsetfillcolor{textcolor}%
\pgftext[x=3.887574in,y=0.424381in,,top]{\color{textcolor}{\rmfamily\fontsize{10.000000}{12.000000}\selectfont\catcode`\^=\active\def^{\ifmmode\sp\else\^{}\fi}\catcode`\%=\active\def%{\%}$\mathdefault{18}$}}%
\end{pgfscope}%
\begin{pgfscope}%
\pgfsetbuttcap%
\pgfsetroundjoin%
\definecolor{currentfill}{rgb}{0.000000,0.000000,0.000000}%
\pgfsetfillcolor{currentfill}%
\pgfsetlinewidth{0.803000pt}%
\definecolor{currentstroke}{rgb}{0.000000,0.000000,0.000000}%
\pgfsetstrokecolor{currentstroke}%
\pgfsetdash{}{0pt}%
\pgfsys@defobject{currentmarker}{\pgfqpoint{0.000000in}{-0.048611in}}{\pgfqpoint{0.000000in}{0.000000in}}{%
\pgfpathmoveto{\pgfqpoint{0.000000in}{0.000000in}}%
\pgfpathlineto{\pgfqpoint{0.000000in}{-0.048611in}}%
\pgfusepath{stroke,fill}%
}%
\begin{pgfscope}%
\pgfsys@transformshift{4.466378in}{0.521603in}%
\pgfsys@useobject{currentmarker}{}%
\end{pgfscope}%
\end{pgfscope}%
\begin{pgfscope}%
\definecolor{textcolor}{rgb}{0.000000,0.000000,0.000000}%
\pgfsetstrokecolor{textcolor}%
\pgfsetfillcolor{textcolor}%
\pgftext[x=4.466378in,y=0.424381in,,top]{\color{textcolor}{\rmfamily\fontsize{10.000000}{12.000000}\selectfont\catcode`\^=\active\def^{\ifmmode\sp\else\^{}\fi}\catcode`\%=\active\def%{\%}$\mathdefault{21}$}}%
\end{pgfscope}%
\begin{pgfscope}%
\pgfsetbuttcap%
\pgfsetroundjoin%
\definecolor{currentfill}{rgb}{0.000000,0.000000,0.000000}%
\pgfsetfillcolor{currentfill}%
\pgfsetlinewidth{0.803000pt}%
\definecolor{currentstroke}{rgb}{0.000000,0.000000,0.000000}%
\pgfsetstrokecolor{currentstroke}%
\pgfsetdash{}{0pt}%
\pgfsys@defobject{currentmarker}{\pgfqpoint{0.000000in}{-0.048611in}}{\pgfqpoint{0.000000in}{0.000000in}}{%
\pgfpathmoveto{\pgfqpoint{0.000000in}{0.000000in}}%
\pgfpathlineto{\pgfqpoint{0.000000in}{-0.048611in}}%
\pgfusepath{stroke,fill}%
}%
\begin{pgfscope}%
\pgfsys@transformshift{5.045183in}{0.521603in}%
\pgfsys@useobject{currentmarker}{}%
\end{pgfscope}%
\end{pgfscope}%
\begin{pgfscope}%
\definecolor{textcolor}{rgb}{0.000000,0.000000,0.000000}%
\pgfsetstrokecolor{textcolor}%
\pgfsetfillcolor{textcolor}%
\pgftext[x=5.045183in,y=0.424381in,,top]{\color{textcolor}{\rmfamily\fontsize{10.000000}{12.000000}\selectfont\catcode`\^=\active\def^{\ifmmode\sp\else\^{}\fi}\catcode`\%=\active\def%{\%}$\mathdefault{24}$}}%
\end{pgfscope}%
\begin{pgfscope}%
\definecolor{textcolor}{rgb}{0.000000,0.000000,0.000000}%
\pgfsetstrokecolor{textcolor}%
\pgfsetfillcolor{textcolor}%
\pgftext[x=2.922899in,y=0.234413in,,top]{\color{textcolor}{\rmfamily\fontsize{10.000000}{12.000000}\selectfont\catcode`\^=\active\def^{\ifmmode\sp\else\^{}\fi}\catcode`\%=\active\def%{\%}Monochromatic classes}}%
\end{pgfscope}%
\begin{pgfscope}%
\pgfsetbuttcap%
\pgfsetroundjoin%
\definecolor{currentfill}{rgb}{0.000000,0.000000,0.000000}%
\pgfsetfillcolor{currentfill}%
\pgfsetlinewidth{0.803000pt}%
\definecolor{currentstroke}{rgb}{0.000000,0.000000,0.000000}%
\pgfsetstrokecolor{currentstroke}%
\pgfsetdash{}{0pt}%
\pgfsys@defobject{currentmarker}{\pgfqpoint{-0.048611in}{0.000000in}}{\pgfqpoint{-0.000000in}{0.000000in}}{%
\pgfpathmoveto{\pgfqpoint{-0.000000in}{0.000000in}}%
\pgfpathlineto{\pgfqpoint{-0.048611in}{0.000000in}}%
\pgfusepath{stroke,fill}%
}%
\begin{pgfscope}%
\pgfsys@transformshift{0.588387in}{0.876933in}%
\pgfsys@useobject{currentmarker}{}%
\end{pgfscope}%
\end{pgfscope}%
\begin{pgfscope}%
\definecolor{textcolor}{rgb}{0.000000,0.000000,0.000000}%
\pgfsetstrokecolor{textcolor}%
\pgfsetfillcolor{textcolor}%
\pgftext[x=0.289968in, y=0.824172in, left, base]{\color{textcolor}{\rmfamily\fontsize{10.000000}{12.000000}\selectfont\catcode`\^=\active\def^{\ifmmode\sp\else\^{}\fi}\catcode`\%=\active\def%{\%}$\mathdefault{10^{1}}$}}%
\end{pgfscope}%
\begin{pgfscope}%
\pgfsetbuttcap%
\pgfsetroundjoin%
\definecolor{currentfill}{rgb}{0.000000,0.000000,0.000000}%
\pgfsetfillcolor{currentfill}%
\pgfsetlinewidth{0.803000pt}%
\definecolor{currentstroke}{rgb}{0.000000,0.000000,0.000000}%
\pgfsetstrokecolor{currentstroke}%
\pgfsetdash{}{0pt}%
\pgfsys@defobject{currentmarker}{\pgfqpoint{-0.048611in}{0.000000in}}{\pgfqpoint{-0.000000in}{0.000000in}}{%
\pgfpathmoveto{\pgfqpoint{-0.000000in}{0.000000in}}%
\pgfpathlineto{\pgfqpoint{-0.048611in}{0.000000in}}%
\pgfusepath{stroke,fill}%
}%
\begin{pgfscope}%
\pgfsys@transformshift{0.588387in}{1.404840in}%
\pgfsys@useobject{currentmarker}{}%
\end{pgfscope}%
\end{pgfscope}%
\begin{pgfscope}%
\definecolor{textcolor}{rgb}{0.000000,0.000000,0.000000}%
\pgfsetstrokecolor{textcolor}%
\pgfsetfillcolor{textcolor}%
\pgftext[x=0.289968in, y=1.352079in, left, base]{\color{textcolor}{\rmfamily\fontsize{10.000000}{12.000000}\selectfont\catcode`\^=\active\def^{\ifmmode\sp\else\^{}\fi}\catcode`\%=\active\def%{\%}$\mathdefault{10^{3}}$}}%
\end{pgfscope}%
\begin{pgfscope}%
\pgfsetbuttcap%
\pgfsetroundjoin%
\definecolor{currentfill}{rgb}{0.000000,0.000000,0.000000}%
\pgfsetfillcolor{currentfill}%
\pgfsetlinewidth{0.803000pt}%
\definecolor{currentstroke}{rgb}{0.000000,0.000000,0.000000}%
\pgfsetstrokecolor{currentstroke}%
\pgfsetdash{}{0pt}%
\pgfsys@defobject{currentmarker}{\pgfqpoint{-0.048611in}{0.000000in}}{\pgfqpoint{-0.000000in}{0.000000in}}{%
\pgfpathmoveto{\pgfqpoint{-0.000000in}{0.000000in}}%
\pgfpathlineto{\pgfqpoint{-0.048611in}{0.000000in}}%
\pgfusepath{stroke,fill}%
}%
\begin{pgfscope}%
\pgfsys@transformshift{0.588387in}{1.932747in}%
\pgfsys@useobject{currentmarker}{}%
\end{pgfscope}%
\end{pgfscope}%
\begin{pgfscope}%
\definecolor{textcolor}{rgb}{0.000000,0.000000,0.000000}%
\pgfsetstrokecolor{textcolor}%
\pgfsetfillcolor{textcolor}%
\pgftext[x=0.289968in, y=1.879986in, left, base]{\color{textcolor}{\rmfamily\fontsize{10.000000}{12.000000}\selectfont\catcode`\^=\active\def^{\ifmmode\sp\else\^{}\fi}\catcode`\%=\active\def%{\%}$\mathdefault{10^{5}}$}}%
\end{pgfscope}%
\begin{pgfscope}%
\pgfsetbuttcap%
\pgfsetroundjoin%
\definecolor{currentfill}{rgb}{0.000000,0.000000,0.000000}%
\pgfsetfillcolor{currentfill}%
\pgfsetlinewidth{0.803000pt}%
\definecolor{currentstroke}{rgb}{0.000000,0.000000,0.000000}%
\pgfsetstrokecolor{currentstroke}%
\pgfsetdash{}{0pt}%
\pgfsys@defobject{currentmarker}{\pgfqpoint{-0.048611in}{0.000000in}}{\pgfqpoint{-0.000000in}{0.000000in}}{%
\pgfpathmoveto{\pgfqpoint{-0.000000in}{0.000000in}}%
\pgfpathlineto{\pgfqpoint{-0.048611in}{0.000000in}}%
\pgfusepath{stroke,fill}%
}%
\begin{pgfscope}%
\pgfsys@transformshift{0.588387in}{2.460654in}%
\pgfsys@useobject{currentmarker}{}%
\end{pgfscope}%
\end{pgfscope}%
\begin{pgfscope}%
\definecolor{textcolor}{rgb}{0.000000,0.000000,0.000000}%
\pgfsetstrokecolor{textcolor}%
\pgfsetfillcolor{textcolor}%
\pgftext[x=0.289968in, y=2.407892in, left, base]{\color{textcolor}{\rmfamily\fontsize{10.000000}{12.000000}\selectfont\catcode`\^=\active\def^{\ifmmode\sp\else\^{}\fi}\catcode`\%=\active\def%{\%}$\mathdefault{10^{7}}$}}%
\end{pgfscope}%
\begin{pgfscope}%
\definecolor{textcolor}{rgb}{0.000000,0.000000,0.000000}%
\pgfsetstrokecolor{textcolor}%
\pgfsetfillcolor{textcolor}%
\pgftext[x=0.234413in,y=1.526746in,,bottom,rotate=90.000000]{\color{textcolor}{\rmfamily\fontsize{10.000000}{12.000000}\selectfont\catcode`\^=\active\def^{\ifmmode\sp\else\^{}\fi}\catcode`\%=\active\def%{\%}Checks [call]}}%
\end{pgfscope}%
\begin{pgfscope}%
\pgfpathrectangle{\pgfqpoint{0.588387in}{0.521603in}}{\pgfqpoint{4.669024in}{2.010285in}}%
\pgfusepath{clip}%
\pgfsetrectcap%
\pgfsetroundjoin%
\pgfsetlinewidth{1.505625pt}%
\pgfsetstrokecolor{currentstroke1}%
\pgfsetdash{}{0pt}%
\pgfpathmoveto{\pgfqpoint{0.800616in}{0.692438in}}%
\pgfpathlineto{\pgfqpoint{0.993550in}{0.771896in}}%
\pgfpathlineto{\pgfqpoint{1.186485in}{0.851354in}}%
\pgfpathlineto{\pgfqpoint{1.379420in}{0.930812in}}%
\pgfpathlineto{\pgfqpoint{1.572355in}{1.010269in}}%
\pgfpathlineto{\pgfqpoint{1.765290in}{1.089727in}}%
\pgfpathlineto{\pgfqpoint{1.958225in}{1.146115in}}%
\pgfpathlineto{\pgfqpoint{2.151160in}{1.191398in}}%
\pgfpathlineto{\pgfqpoint{2.344095in}{1.253427in}}%
\pgfpathlineto{\pgfqpoint{2.537029in}{1.298966in}}%
\pgfpathlineto{\pgfqpoint{2.729964in}{1.345900in}}%
\pgfpathlineto{\pgfqpoint{2.922899in}{1.396881in}}%
\pgfpathlineto{\pgfqpoint{3.115834in}{1.403146in}}%
\pgfpathlineto{\pgfqpoint{3.308769in}{1.514989in}}%
\pgfpathlineto{\pgfqpoint{3.501704in}{1.487782in}}%
\pgfpathlineto{\pgfqpoint{3.694639in}{1.574316in}}%
\pgfpathlineto{\pgfqpoint{3.887574in}{1.564214in}}%
\pgfpathlineto{\pgfqpoint{4.080508in}{1.592207in}}%
\pgfpathlineto{\pgfqpoint{4.273443in}{1.589336in}}%
\pgfpathlineto{\pgfqpoint{4.466378in}{1.627382in}}%
\pgfpathlineto{\pgfqpoint{4.659313in}{1.633780in}}%
\pgfpathlineto{\pgfqpoint{5.045183in}{1.669160in}}%
\pgfusepath{stroke}%
\end{pgfscope}%
\begin{pgfscope}%
\pgfpathrectangle{\pgfqpoint{0.588387in}{0.521603in}}{\pgfqpoint{4.669024in}{2.010285in}}%
\pgfusepath{clip}%
\pgfsetrectcap%
\pgfsetroundjoin%
\pgfsetlinewidth{1.505625pt}%
\pgfsetstrokecolor{currentstroke2}%
\pgfsetdash{}{0pt}%
\pgfpathmoveto{\pgfqpoint{0.800616in}{0.692438in}}%
\pgfpathlineto{\pgfqpoint{0.993550in}{0.771896in}}%
\pgfpathlineto{\pgfqpoint{1.186485in}{0.851354in}}%
\pgfpathlineto{\pgfqpoint{1.379420in}{0.930812in}}%
\pgfpathlineto{\pgfqpoint{1.572355in}{1.010269in}}%
\pgfpathlineto{\pgfqpoint{1.765290in}{1.089727in}}%
\pgfpathlineto{\pgfqpoint{1.958225in}{1.146115in}}%
\pgfpathlineto{\pgfqpoint{2.151160in}{1.191398in}}%
\pgfpathlineto{\pgfqpoint{2.344095in}{1.253427in}}%
\pgfpathlineto{\pgfqpoint{2.537029in}{1.298966in}}%
\pgfpathlineto{\pgfqpoint{2.729964in}{1.338701in}}%
\pgfpathlineto{\pgfqpoint{2.922899in}{1.355864in}}%
\pgfpathlineto{\pgfqpoint{3.115834in}{1.419323in}}%
\pgfpathlineto{\pgfqpoint{3.308769in}{1.454634in}}%
\pgfpathlineto{\pgfqpoint{3.501704in}{1.502715in}}%
\pgfpathlineto{\pgfqpoint{3.694639in}{1.556297in}}%
\pgfpathlineto{\pgfqpoint{3.887574in}{1.564214in}}%
\pgfpathlineto{\pgfqpoint{4.080508in}{1.621972in}}%
\pgfpathlineto{\pgfqpoint{4.273443in}{1.593655in}}%
\pgfpathlineto{\pgfqpoint{4.466378in}{1.628950in}}%
\pgfpathlineto{\pgfqpoint{4.659313in}{1.613995in}}%
\pgfusepath{stroke}%
\end{pgfscope}%
\begin{pgfscope}%
\pgfpathrectangle{\pgfqpoint{0.588387in}{0.521603in}}{\pgfqpoint{4.669024in}{2.010285in}}%
\pgfusepath{clip}%
\pgfsetrectcap%
\pgfsetroundjoin%
\pgfsetlinewidth{1.505625pt}%
\pgfsetstrokecolor{currentstroke3}%
\pgfsetdash{}{0pt}%
\pgfpathmoveto{\pgfqpoint{0.800616in}{0.612980in}}%
\pgfpathlineto{\pgfqpoint{0.993550in}{0.738918in}}%
\pgfpathlineto{\pgfqpoint{1.186485in}{0.836046in}}%
\pgfpathlineto{\pgfqpoint{1.379420in}{0.923413in}}%
\pgfpathlineto{\pgfqpoint{1.572355in}{1.006630in}}%
\pgfpathlineto{\pgfqpoint{1.765290in}{1.087922in}}%
\pgfpathlineto{\pgfqpoint{1.958225in}{1.168286in}}%
\pgfpathlineto{\pgfqpoint{2.151160in}{1.248194in}}%
\pgfpathlineto{\pgfqpoint{2.344095in}{1.327877in}}%
\pgfpathlineto{\pgfqpoint{2.537029in}{1.407447in}}%
\pgfpathlineto{\pgfqpoint{2.729964in}{1.486961in}}%
\pgfpathlineto{\pgfqpoint{2.922899in}{1.566447in}}%
\pgfpathlineto{\pgfqpoint{3.115834in}{1.645919in}}%
\pgfpathlineto{\pgfqpoint{3.308769in}{1.725384in}}%
\pgfpathlineto{\pgfqpoint{3.501704in}{1.804845in}}%
\pgfpathlineto{\pgfqpoint{3.694639in}{1.884305in}}%
\pgfpathlineto{\pgfqpoint{3.887574in}{1.963763in}}%
\pgfpathlineto{\pgfqpoint{4.080508in}{2.043222in}}%
\pgfpathlineto{\pgfqpoint{4.273443in}{2.122680in}}%
\pgfpathlineto{\pgfqpoint{4.466378in}{2.202138in}}%
\pgfpathlineto{\pgfqpoint{4.659313in}{2.281596in}}%
\pgfpathlineto{\pgfqpoint{5.045183in}{2.440512in}}%
\pgfusepath{stroke}%
\end{pgfscope}%
\begin{pgfscope}%
\pgfpathrectangle{\pgfqpoint{0.588387in}{0.521603in}}{\pgfqpoint{4.669024in}{2.010285in}}%
\pgfusepath{clip}%
\pgfsetrectcap%
\pgfsetroundjoin%
\pgfsetlinewidth{1.505625pt}%
\pgfsetstrokecolor{currentstroke4}%
\pgfsetdash{}{0pt}%
\pgfpathmoveto{\pgfqpoint{0.800616in}{0.692438in}}%
\pgfpathlineto{\pgfqpoint{0.993550in}{0.771896in}}%
\pgfpathlineto{\pgfqpoint{1.186485in}{0.851354in}}%
\pgfpathlineto{\pgfqpoint{1.379420in}{0.930812in}}%
\pgfpathlineto{\pgfqpoint{1.572355in}{1.010269in}}%
\pgfpathlineto{\pgfqpoint{1.765290in}{1.089727in}}%
\pgfpathlineto{\pgfqpoint{1.958225in}{1.148596in}}%
\pgfpathlineto{\pgfqpoint{2.151160in}{1.225672in}}%
\pgfpathlineto{\pgfqpoint{2.344095in}{1.296668in}}%
\pgfpathlineto{\pgfqpoint{2.537029in}{1.368449in}}%
\pgfpathlineto{\pgfqpoint{2.729964in}{1.343508in}}%
\pgfpathlineto{\pgfqpoint{2.922899in}{1.411153in}}%
\pgfpathlineto{\pgfqpoint{3.115834in}{1.453140in}}%
\pgfpathlineto{\pgfqpoint{3.308769in}{1.503213in}}%
\pgfpathlineto{\pgfqpoint{3.501704in}{1.508375in}}%
\pgfpathlineto{\pgfqpoint{3.694639in}{1.600387in}}%
\pgfpathlineto{\pgfqpoint{3.887574in}{1.629726in}}%
\pgfpathlineto{\pgfqpoint{4.080508in}{1.633394in}}%
\pgfpathlineto{\pgfqpoint{4.273443in}{1.681717in}}%
\pgfpathlineto{\pgfqpoint{4.466378in}{1.677764in}}%
\pgfpathlineto{\pgfqpoint{4.659313in}{1.694080in}}%
\pgfpathlineto{\pgfqpoint{5.045183in}{1.573002in}}%
\pgfusepath{stroke}%
\end{pgfscope}%
\begin{pgfscope}%
\pgfpathrectangle{\pgfqpoint{0.588387in}{0.521603in}}{\pgfqpoint{4.669024in}{2.010285in}}%
\pgfusepath{clip}%
\pgfsetrectcap%
\pgfsetroundjoin%
\pgfsetlinewidth{1.505625pt}%
\pgfsetstrokecolor{currentstroke5}%
\pgfsetdash{}{0pt}%
\pgfpathmoveto{\pgfqpoint{0.800616in}{0.692438in}}%
\pgfpathlineto{\pgfqpoint{0.993550in}{0.771896in}}%
\pgfpathlineto{\pgfqpoint{1.186485in}{0.851354in}}%
\pgfpathlineto{\pgfqpoint{1.379420in}{0.930812in}}%
\pgfpathlineto{\pgfqpoint{1.572355in}{1.010269in}}%
\pgfpathlineto{\pgfqpoint{1.765290in}{1.089727in}}%
\pgfpathlineto{\pgfqpoint{1.958225in}{1.149075in}}%
\pgfpathlineto{\pgfqpoint{2.151160in}{1.224209in}}%
\pgfpathlineto{\pgfqpoint{2.344095in}{1.295803in}}%
\pgfpathlineto{\pgfqpoint{2.537029in}{1.368449in}}%
\pgfpathlineto{\pgfqpoint{2.729964in}{1.330242in}}%
\pgfpathlineto{\pgfqpoint{2.922899in}{1.406242in}}%
\pgfpathlineto{\pgfqpoint{3.115834in}{1.447592in}}%
\pgfpathlineto{\pgfqpoint{3.308769in}{1.506528in}}%
\pgfpathlineto{\pgfqpoint{3.501704in}{1.502938in}}%
\pgfpathlineto{\pgfqpoint{3.694639in}{1.598497in}}%
\pgfpathlineto{\pgfqpoint{3.887574in}{1.564897in}}%
\pgfpathlineto{\pgfqpoint{4.080508in}{1.573235in}}%
\pgfpathlineto{\pgfqpoint{4.273443in}{1.663171in}}%
\pgfpathlineto{\pgfqpoint{4.466378in}{1.667503in}}%
\pgfpathlineto{\pgfqpoint{4.659313in}{1.633842in}}%
\pgfpathlineto{\pgfqpoint{5.045183in}{1.735002in}}%
\pgfusepath{stroke}%
\end{pgfscope}%
\begin{pgfscope}%
\pgfpathrectangle{\pgfqpoint{0.588387in}{0.521603in}}{\pgfqpoint{4.669024in}{2.010285in}}%
\pgfusepath{clip}%
\pgfsetrectcap%
\pgfsetroundjoin%
\pgfsetlinewidth{1.505625pt}%
\pgfsetstrokecolor{currentstroke6}%
\pgfsetdash{}{0pt}%
\pgfpathmoveto{\pgfqpoint{0.800616in}{0.692438in}}%
\pgfpathlineto{\pgfqpoint{0.993550in}{0.771896in}}%
\pgfpathlineto{\pgfqpoint{1.186485in}{0.851354in}}%
\pgfpathlineto{\pgfqpoint{1.379420in}{0.930812in}}%
\pgfpathlineto{\pgfqpoint{1.572355in}{1.010269in}}%
\pgfpathlineto{\pgfqpoint{1.765290in}{1.089727in}}%
\pgfpathlineto{\pgfqpoint{1.958225in}{1.166027in}}%
\pgfpathlineto{\pgfqpoint{2.151160in}{1.236224in}}%
\pgfpathlineto{\pgfqpoint{2.344095in}{1.311742in}}%
\pgfpathlineto{\pgfqpoint{2.537029in}{1.383861in}}%
\pgfpathlineto{\pgfqpoint{2.729964in}{1.385126in}}%
\pgfpathlineto{\pgfqpoint{2.922899in}{1.440140in}}%
\pgfpathlineto{\pgfqpoint{3.115834in}{1.500087in}}%
\pgfpathlineto{\pgfqpoint{3.308769in}{1.582096in}}%
\pgfpathlineto{\pgfqpoint{3.501704in}{1.564962in}}%
\pgfpathlineto{\pgfqpoint{3.694639in}{1.700810in}}%
\pgfpathlineto{\pgfqpoint{3.887574in}{1.623364in}}%
\pgfpathlineto{\pgfqpoint{4.080508in}{1.650788in}}%
\pgfpathlineto{\pgfqpoint{4.273443in}{1.761168in}}%
\pgfpathlineto{\pgfqpoint{4.466378in}{1.779206in}}%
\pgfpathlineto{\pgfqpoint{4.659313in}{1.646769in}}%
\pgfpathlineto{\pgfqpoint{5.045183in}{1.693969in}}%
\pgfusepath{stroke}%
\end{pgfscope}%
\begin{pgfscope}%
\pgfpathrectangle{\pgfqpoint{0.588387in}{0.521603in}}{\pgfqpoint{4.669024in}{2.010285in}}%
\pgfusepath{clip}%
\pgfsetrectcap%
\pgfsetroundjoin%
\pgfsetlinewidth{1.505625pt}%
\pgfsetstrokecolor{currentstroke7}%
\pgfsetdash{}{0pt}%
\pgfpathmoveto{\pgfqpoint{0.800616in}{0.692438in}}%
\pgfpathlineto{\pgfqpoint{0.993550in}{0.771896in}}%
\pgfpathlineto{\pgfqpoint{1.186485in}{0.851354in}}%
\pgfpathlineto{\pgfqpoint{1.379420in}{0.930812in}}%
\pgfpathlineto{\pgfqpoint{1.572355in}{1.010269in}}%
\pgfpathlineto{\pgfqpoint{1.765290in}{1.089727in}}%
\pgfpathlineto{\pgfqpoint{1.958225in}{1.166027in}}%
\pgfpathlineto{\pgfqpoint{2.151160in}{1.235426in}}%
\pgfpathlineto{\pgfqpoint{2.344095in}{1.311322in}}%
\pgfpathlineto{\pgfqpoint{2.537029in}{1.385052in}}%
\pgfpathlineto{\pgfqpoint{2.729964in}{1.378729in}}%
\pgfpathlineto{\pgfqpoint{2.922899in}{1.439046in}}%
\pgfpathlineto{\pgfqpoint{3.115834in}{1.504014in}}%
\pgfpathlineto{\pgfqpoint{3.308769in}{1.575512in}}%
\pgfpathlineto{\pgfqpoint{3.501704in}{1.558152in}}%
\pgfpathlineto{\pgfqpoint{3.694639in}{1.697074in}}%
\pgfpathlineto{\pgfqpoint{3.887574in}{1.597306in}}%
\pgfpathlineto{\pgfqpoint{4.080508in}{1.650228in}}%
\pgfpathlineto{\pgfqpoint{4.273443in}{1.753546in}}%
\pgfpathlineto{\pgfqpoint{4.466378in}{1.768038in}}%
\pgfpathlineto{\pgfqpoint{4.659313in}{1.679092in}}%
\pgfpathlineto{\pgfqpoint{5.045183in}{1.792914in}}%
\pgfusepath{stroke}%
\end{pgfscope}%
\begin{pgfscope}%
\pgfsetrectcap%
\pgfsetmiterjoin%
\pgfsetlinewidth{0.803000pt}%
\definecolor{currentstroke}{rgb}{0.000000,0.000000,0.000000}%
\pgfsetstrokecolor{currentstroke}%
\pgfsetdash{}{0pt}%
\pgfpathmoveto{\pgfqpoint{0.588387in}{0.521603in}}%
\pgfpathlineto{\pgfqpoint{0.588387in}{2.531888in}}%
\pgfusepath{stroke}%
\end{pgfscope}%
\begin{pgfscope}%
\pgfsetrectcap%
\pgfsetmiterjoin%
\pgfsetlinewidth{0.803000pt}%
\definecolor{currentstroke}{rgb}{0.000000,0.000000,0.000000}%
\pgfsetstrokecolor{currentstroke}%
\pgfsetdash{}{0pt}%
\pgfpathmoveto{\pgfqpoint{5.257411in}{0.521603in}}%
\pgfpathlineto{\pgfqpoint{5.257411in}{2.531888in}}%
\pgfusepath{stroke}%
\end{pgfscope}%
\begin{pgfscope}%
\pgfsetrectcap%
\pgfsetmiterjoin%
\pgfsetlinewidth{0.803000pt}%
\definecolor{currentstroke}{rgb}{0.000000,0.000000,0.000000}%
\pgfsetstrokecolor{currentstroke}%
\pgfsetdash{}{0pt}%
\pgfpathmoveto{\pgfqpoint{0.588387in}{0.521603in}}%
\pgfpathlineto{\pgfqpoint{5.257411in}{0.521603in}}%
\pgfusepath{stroke}%
\end{pgfscope}%
\begin{pgfscope}%
\pgfsetrectcap%
\pgfsetmiterjoin%
\pgfsetlinewidth{0.803000pt}%
\definecolor{currentstroke}{rgb}{0.000000,0.000000,0.000000}%
\pgfsetstrokecolor{currentstroke}%
\pgfsetdash{}{0pt}%
\pgfpathmoveto{\pgfqpoint{0.588387in}{2.531888in}}%
\pgfpathlineto{\pgfqpoint{5.257411in}{2.531888in}}%
\pgfusepath{stroke}%
\end{pgfscope}%
\begin{pgfscope}%
\definecolor{textcolor}{rgb}{0.000000,0.000000,0.000000}%
\pgfsetstrokecolor{textcolor}%
\pgfsetfillcolor{textcolor}%
\pgftext[x=2.922899in,y=2.615222in,,base]{\color{textcolor}{\rmfamily\fontsize{12.000000}{14.400000}\selectfont\catcode`\^=\active\def^{\ifmmode\sp\else\^{}\fi}\catcode`\%=\active\def%{\%}Mean}}%
\end{pgfscope}%
\begin{pgfscope}%
\pgfsetbuttcap%
\pgfsetmiterjoin%
\definecolor{currentfill}{rgb}{1.000000,1.000000,1.000000}%
\pgfsetfillcolor{currentfill}%
\pgfsetfillopacity{0.800000}%
\pgfsetlinewidth{1.003750pt}%
\definecolor{currentstroke}{rgb}{0.800000,0.800000,0.800000}%
\pgfsetstrokecolor{currentstroke}%
\pgfsetstrokeopacity{0.800000}%
\pgfsetdash{}{0pt}%
\pgfpathmoveto{\pgfqpoint{5.344911in}{1.133672in}}%
\pgfpathlineto{\pgfqpoint{8.259376in}{1.133672in}}%
\pgfpathquadraticcurveto{\pgfqpoint{8.284376in}{1.133672in}}{\pgfqpoint{8.284376in}{1.158672in}}%
\pgfpathlineto{\pgfqpoint{8.284376in}{2.444388in}}%
\pgfpathquadraticcurveto{\pgfqpoint{8.284376in}{2.469388in}}{\pgfqpoint{8.259376in}{2.469388in}}%
\pgfpathlineto{\pgfqpoint{5.344911in}{2.469388in}}%
\pgfpathquadraticcurveto{\pgfqpoint{5.319911in}{2.469388in}}{\pgfqpoint{5.319911in}{2.444388in}}%
\pgfpathlineto{\pgfqpoint{5.319911in}{1.158672in}}%
\pgfpathquadraticcurveto{\pgfqpoint{5.319911in}{1.133672in}}{\pgfqpoint{5.344911in}{1.133672in}}%
\pgfpathlineto{\pgfqpoint{5.344911in}{1.133672in}}%
\pgfpathclose%
\pgfusepath{stroke,fill}%
\end{pgfscope}%
\begin{pgfscope}%
\pgfsetrectcap%
\pgfsetroundjoin%
\pgfsetlinewidth{1.505625pt}%
\pgfsetstrokecolor{currentstroke3}%
\pgfsetdash{}{0pt}%
\pgfpathmoveto{\pgfqpoint{5.369911in}{2.368168in}}%
\pgfpathlineto{\pgfqpoint{5.494911in}{2.368168in}}%
\pgfpathlineto{\pgfqpoint{5.619911in}{2.368168in}}%
\pgfusepath{stroke}%
\end{pgfscope}%
\begin{pgfscope}%
\definecolor{textcolor}{rgb}{0.000000,0.000000,0.000000}%
\pgfsetstrokecolor{textcolor}%
\pgfsetfillcolor{textcolor}%
\pgftext[x=5.719911in,y=2.324418in,left,base]{\color{textcolor}{\rmfamily\fontsize{9.000000}{10.800000}\selectfont\catcode`\^=\active\def^{\ifmmode\sp\else\^{}\fi}\catcode`\%=\active\def%{\%}\NaiveCycles{}}}%
\end{pgfscope}%
\begin{pgfscope}%
\pgfsetrectcap%
\pgfsetroundjoin%
\pgfsetlinewidth{1.505625pt}%
\pgfsetstrokecolor{currentstroke1}%
\pgfsetdash{}{0pt}%
\pgfpathmoveto{\pgfqpoint{5.369911in}{2.184696in}}%
\pgfpathlineto{\pgfqpoint{5.494911in}{2.184696in}}%
\pgfpathlineto{\pgfqpoint{5.619911in}{2.184696in}}%
\pgfusepath{stroke}%
\end{pgfscope}%
\begin{pgfscope}%
\definecolor{textcolor}{rgb}{0.000000,0.000000,0.000000}%
\pgfsetstrokecolor{textcolor}%
\pgfsetfillcolor{textcolor}%
\pgftext[x=5.719911in,y=2.140946in,left,base]{\color{textcolor}{\rmfamily\fontsize{9.000000}{10.800000}\selectfont\catcode`\^=\active\def^{\ifmmode\sp\else\^{}\fi}\catcode`\%=\active\def%{\%}\CyclesMatchChunks{} \& \MergeLinear{}}}%
\end{pgfscope}%
\begin{pgfscope}%
\pgfsetrectcap%
\pgfsetroundjoin%
\pgfsetlinewidth{1.505625pt}%
\pgfsetstrokecolor{currentstroke2}%
\pgfsetdash{}{0pt}%
\pgfpathmoveto{\pgfqpoint{5.369911in}{1.997746in}}%
\pgfpathlineto{\pgfqpoint{5.494911in}{1.997746in}}%
\pgfpathlineto{\pgfqpoint{5.619911in}{1.997746in}}%
\pgfusepath{stroke}%
\end{pgfscope}%
\begin{pgfscope}%
\definecolor{textcolor}{rgb}{0.000000,0.000000,0.000000}%
\pgfsetstrokecolor{textcolor}%
\pgfsetfillcolor{textcolor}%
\pgftext[x=5.719911in,y=1.953996in,left,base]{\color{textcolor}{\rmfamily\fontsize{9.000000}{10.800000}\selectfont\catcode`\^=\active\def^{\ifmmode\sp\else\^{}\fi}\catcode`\%=\active\def%{\%}\CyclesMatchChunks{} \& \SharedVertices{}}}%
\end{pgfscope}%
\begin{pgfscope}%
\pgfsetrectcap%
\pgfsetroundjoin%
\pgfsetlinewidth{1.505625pt}%
\pgfsetstrokecolor{currentstroke4}%
\pgfsetdash{}{0pt}%
\pgfpathmoveto{\pgfqpoint{5.369911in}{1.810795in}}%
\pgfpathlineto{\pgfqpoint{5.494911in}{1.810795in}}%
\pgfpathlineto{\pgfqpoint{5.619911in}{1.810795in}}%
\pgfusepath{stroke}%
\end{pgfscope}%
\begin{pgfscope}%
\definecolor{textcolor}{rgb}{0.000000,0.000000,0.000000}%
\pgfsetstrokecolor{textcolor}%
\pgfsetfillcolor{textcolor}%
\pgftext[x=5.719911in,y=1.767045in,left,base]{\color{textcolor}{\rmfamily\fontsize{9.000000}{10.800000}\selectfont\catcode`\^=\active\def^{\ifmmode\sp\else\^{}\fi}\catcode`\%=\active\def%{\%}\Neighbors{} \& \MergeLinear{}}}%
\end{pgfscope}%
\begin{pgfscope}%
\pgfsetrectcap%
\pgfsetroundjoin%
\pgfsetlinewidth{1.505625pt}%
\pgfsetstrokecolor{currentstroke5}%
\pgfsetdash{}{0pt}%
\pgfpathmoveto{\pgfqpoint{5.369911in}{1.627324in}}%
\pgfpathlineto{\pgfqpoint{5.494911in}{1.627324in}}%
\pgfpathlineto{\pgfqpoint{5.619911in}{1.627324in}}%
\pgfusepath{stroke}%
\end{pgfscope}%
\begin{pgfscope}%
\definecolor{textcolor}{rgb}{0.000000,0.000000,0.000000}%
\pgfsetstrokecolor{textcolor}%
\pgfsetfillcolor{textcolor}%
\pgftext[x=5.719911in,y=1.583574in,left,base]{\color{textcolor}{\rmfamily\fontsize{9.000000}{10.800000}\selectfont\catcode`\^=\active\def^{\ifmmode\sp\else\^{}\fi}\catcode`\%=\active\def%{\%}\Neighbors{} \& \SharedVertices{}}}%
\end{pgfscope}%
\begin{pgfscope}%
\pgfsetrectcap%
\pgfsetroundjoin%
\pgfsetlinewidth{1.505625pt}%
\pgfsetstrokecolor{currentstroke6}%
\pgfsetdash{}{0pt}%
\pgfpathmoveto{\pgfqpoint{5.369911in}{1.440373in}}%
\pgfpathlineto{\pgfqpoint{5.494911in}{1.440373in}}%
\pgfpathlineto{\pgfqpoint{5.619911in}{1.440373in}}%
\pgfusepath{stroke}%
\end{pgfscope}%
\begin{pgfscope}%
\definecolor{textcolor}{rgb}{0.000000,0.000000,0.000000}%
\pgfsetstrokecolor{textcolor}%
\pgfsetfillcolor{textcolor}%
\pgftext[x=5.719911in,y=1.396623in,left,base]{\color{textcolor}{\rmfamily\fontsize{9.000000}{10.800000}\selectfont\catcode`\^=\active\def^{\ifmmode\sp\else\^{}\fi}\catcode`\%=\active\def%{\%}\None{} \& \MergeLinear{}}}%
\end{pgfscope}%
\begin{pgfscope}%
\pgfsetrectcap%
\pgfsetroundjoin%
\pgfsetlinewidth{1.505625pt}%
\pgfsetstrokecolor{currentstroke7}%
\pgfsetdash{}{0pt}%
\pgfpathmoveto{\pgfqpoint{5.369911in}{1.256902in}}%
\pgfpathlineto{\pgfqpoint{5.494911in}{1.256902in}}%
\pgfpathlineto{\pgfqpoint{5.619911in}{1.256902in}}%
\pgfusepath{stroke}%
\end{pgfscope}%
\begin{pgfscope}%
\definecolor{textcolor}{rgb}{0.000000,0.000000,0.000000}%
\pgfsetstrokecolor{textcolor}%
\pgfsetfillcolor{textcolor}%
\pgftext[x=5.719911in,y=1.213152in,left,base]{\color{textcolor}{\rmfamily\fontsize{9.000000}{10.800000}\selectfont\catcode`\^=\active\def^{\ifmmode\sp\else\^{}\fi}\catcode`\%=\active\def%{\%}\None{} \& \SharedVertices{}}}%
\end{pgfscope}%
\end{pgfpicture}%
\makeatother%
\endgroup%
}
	\caption[Checks performed for globally rigid graphs (all).]{
		The number of checks performed to find all NAC-colorings for globally rigid graphs.}%
	\label{fig:graph_globally_rigid_all_checks}
\end{figure}

To summarize this section, we mostly tested graphs having many NAC-colorings
or trivially having none.
Overall, if only a single NAC-coloring is requested
for these graph classes, you can notice that the complexity
is not growing fast neither for the \NaiveCycles{} nor for \Subgraphs{}.
We tested only graphs with up to one hundred vertices
as it is computationally hard to find larger graphs in these classes
and to run proper benchmarks for them.
For such graphs a NAC-coloring can be found in hundreds of milliseconds.
From graphs, we can see that both algorithms should scale well for larger graphs.
For finding a single NAC-coloring, \Subgraphs{} are outperformed by \NaiveCycles{},
for listing all NAC-colorings, the \NaiveCycles{} algorithm
is outperformed quickly even for small graphs.

It can be also seen that \MergeLinear\ is the most reliable one
while the \SharedVertices\ sometimes performs slightly better,
namely on globally rigid graphs.
%
Splitting strategies \None{}, \CycleMask{}, \Neighbors{} and \NeighborsDegree{}
performs similarly.

\subsection{Performance on graphs with no NAC-colorings}

In the previous section, the \Subgraphs{} algorithm performed worse considering runtime
than doing no subgraphs splitting heuristics,
but better considering the number of checks called.
As explained, it is caused by additional overhead and problem simplicity.

For many NP-complete problems, interesting instances are usually
those where there are only few or no solutions.
In this section, we focus on graphs with no NAC-colorings.
We searched for random graphs where \( |E| \ge 2|V(G)| - 2 \) that have
multiple monochromatic classes, but no NAC-coloring.
As this search was slow and unsuccessful, we searched only for
graphs with more than \( 2\sqrt{|V(G)|} \) \trcon{} components.
This once again shows how effective monochromatic classes are
in comparison with \trcon{} components.
We generated ten thousand of such graphs from 40 to 140 vertices in size.
Less than 30 of them had more than one monochromatic class.
The following benchmarks are run with monochromatic classes disabled.

For these graphs, \NaiveCycles{} algorithm needs to traverse all \( 2^{t-1} \)
where \( t \) is the number of \trcon{} components. It can be clearly seen that
this is not suitable for graphs as large as we use in this benchmark,
therefore, they are not present as they did not finish in reasonable time.
It can be seen from \Cref{fig:graph_no_nac_coloring_first_runtime}
that \SharedVertices{} is faster than \MergeLinear{},
for runtime and also for the number of checks performed
shown in \Cref{fig:graph_no_nac_coloring_first_checks}.
It can be also seen that \NeighborsDegree{} strategy is
faster than the other strategies, and it holds for both merging strategies.
Also notice that in contrast with the previous section,
runtime grows strictly exponentially. This was not the case for search for a single NAC-coloring.

\begin{figure}[p]
	\centering
	\scalebox{0.5}{%% Creator: Matplotlib, PGF backend
%%
%% To include the figure in your LaTeX document, write
%%   \input{<filename>.pgf}
%%
%% Make sure the required packages are loaded in your preamble
%%   \usepackage{pgf}
%%
%% Also ensure that all the required font packages are loaded; for instance,
%% the lmodern package is sometimes necessary when using math font.
%%   \usepackage{lmodern}
%%
%% Figures using additional raster images can only be included by \input if
%% they are in the same directory as the main LaTeX file. For loading figures
%% from other directories you can use the `import` package
%%   \usepackage{import}
%%
%% and then include the figures with
%%   \import{<path to file>}{<filename>.pgf}
%%
%% Matplotlib used the following preamble
%%   \def\mathdefault#1{#1}
%%   \everymath=\expandafter{\the\everymath\displaystyle}
%%   \IfFileExists{scrextend.sty}{
%%     \usepackage[fontsize=10.000000pt]{scrextend}
%%   }{
%%     \renewcommand{\normalsize}{\fontsize{10.000000}{12.000000}\selectfont}
%%     \normalsize
%%   }
%%   
%%   \ifdefined\pdftexversion\else  % non-pdftex case.
%%     \usepackage{fontspec}
%%     \setmainfont{DejaVuSans.ttf}[Path=\detokenize{/home/petr/Projects/PyRigi/.venv/lib/python3.12/site-packages/matplotlib/mpl-data/fonts/ttf/}]
%%     \setsansfont{DejaVuSans.ttf}[Path=\detokenize{/home/petr/Projects/PyRigi/.venv/lib/python3.12/site-packages/matplotlib/mpl-data/fonts/ttf/}]
%%     \setmonofont{DejaVuSansMono.ttf}[Path=\detokenize{/home/petr/Projects/PyRigi/.venv/lib/python3.12/site-packages/matplotlib/mpl-data/fonts/ttf/}]
%%   \fi
%%   \makeatletter\@ifpackageloaded{underscore}{}{\usepackage[strings]{underscore}}\makeatother
%%
\begingroup%
\makeatletter%
\begin{pgfpicture}%
\pgfpathrectangle{\pgfpointorigin}{\pgfqpoint{8.384376in}{2.841849in}}%
\pgfusepath{use as bounding box, clip}%
\begin{pgfscope}%
\pgfsetbuttcap%
\pgfsetmiterjoin%
\definecolor{currentfill}{rgb}{1.000000,1.000000,1.000000}%
\pgfsetfillcolor{currentfill}%
\pgfsetlinewidth{0.000000pt}%
\definecolor{currentstroke}{rgb}{1.000000,1.000000,1.000000}%
\pgfsetstrokecolor{currentstroke}%
\pgfsetdash{}{0pt}%
\pgfpathmoveto{\pgfqpoint{0.000000in}{0.000000in}}%
\pgfpathlineto{\pgfqpoint{8.384376in}{0.000000in}}%
\pgfpathlineto{\pgfqpoint{8.384376in}{2.841849in}}%
\pgfpathlineto{\pgfqpoint{0.000000in}{2.841849in}}%
\pgfpathlineto{\pgfqpoint{0.000000in}{0.000000in}}%
\pgfpathclose%
\pgfusepath{fill}%
\end{pgfscope}%
\begin{pgfscope}%
\pgfsetbuttcap%
\pgfsetmiterjoin%
\definecolor{currentfill}{rgb}{1.000000,1.000000,1.000000}%
\pgfsetfillcolor{currentfill}%
\pgfsetlinewidth{0.000000pt}%
\definecolor{currentstroke}{rgb}{0.000000,0.000000,0.000000}%
\pgfsetstrokecolor{currentstroke}%
\pgfsetstrokeopacity{0.000000}%
\pgfsetdash{}{0pt}%
\pgfpathmoveto{\pgfqpoint{0.588387in}{0.521603in}}%
\pgfpathlineto{\pgfqpoint{5.257411in}{0.521603in}}%
\pgfpathlineto{\pgfqpoint{5.257411in}{2.531888in}}%
\pgfpathlineto{\pgfqpoint{0.588387in}{2.531888in}}%
\pgfpathlineto{\pgfqpoint{0.588387in}{0.521603in}}%
\pgfpathclose%
\pgfusepath{fill}%
\end{pgfscope}%
\begin{pgfscope}%
\pgfsetbuttcap%
\pgfsetroundjoin%
\definecolor{currentfill}{rgb}{0.000000,0.000000,0.000000}%
\pgfsetfillcolor{currentfill}%
\pgfsetlinewidth{0.803000pt}%
\definecolor{currentstroke}{rgb}{0.000000,0.000000,0.000000}%
\pgfsetstrokecolor{currentstroke}%
\pgfsetdash{}{0pt}%
\pgfsys@defobject{currentmarker}{\pgfqpoint{0.000000in}{-0.048611in}}{\pgfqpoint{0.000000in}{0.000000in}}{%
\pgfpathmoveto{\pgfqpoint{0.000000in}{0.000000in}}%
\pgfpathlineto{\pgfqpoint{0.000000in}{-0.048611in}}%
\pgfusepath{stroke,fill}%
}%
\begin{pgfscope}%
\pgfsys@transformshift{1.017491in}{0.521603in}%
\pgfsys@useobject{currentmarker}{}%
\end{pgfscope}%
\end{pgfscope}%
\begin{pgfscope}%
\definecolor{textcolor}{rgb}{0.000000,0.000000,0.000000}%
\pgfsetstrokecolor{textcolor}%
\pgfsetfillcolor{textcolor}%
\pgftext[x=1.017491in,y=0.424381in,,top]{\color{textcolor}{\rmfamily\fontsize{10.000000}{12.000000}\selectfont\catcode`\^=\active\def^{\ifmmode\sp\else\^{}\fi}\catcode`\%=\active\def%{\%}$\mathdefault{20}$}}%
\end{pgfscope}%
\begin{pgfscope}%
\pgfsetbuttcap%
\pgfsetroundjoin%
\definecolor{currentfill}{rgb}{0.000000,0.000000,0.000000}%
\pgfsetfillcolor{currentfill}%
\pgfsetlinewidth{0.803000pt}%
\definecolor{currentstroke}{rgb}{0.000000,0.000000,0.000000}%
\pgfsetstrokecolor{currentstroke}%
\pgfsetdash{}{0pt}%
\pgfsys@defobject{currentmarker}{\pgfqpoint{0.000000in}{-0.048611in}}{\pgfqpoint{0.000000in}{0.000000in}}{%
\pgfpathmoveto{\pgfqpoint{0.000000in}{0.000000in}}%
\pgfpathlineto{\pgfqpoint{0.000000in}{-0.048611in}}%
\pgfusepath{stroke,fill}%
}%
\begin{pgfscope}%
\pgfsys@transformshift{1.637136in}{0.521603in}%
\pgfsys@useobject{currentmarker}{}%
\end{pgfscope}%
\end{pgfscope}%
\begin{pgfscope}%
\definecolor{textcolor}{rgb}{0.000000,0.000000,0.000000}%
\pgfsetstrokecolor{textcolor}%
\pgfsetfillcolor{textcolor}%
\pgftext[x=1.637136in,y=0.424381in,,top]{\color{textcolor}{\rmfamily\fontsize{10.000000}{12.000000}\selectfont\catcode`\^=\active\def^{\ifmmode\sp\else\^{}\fi}\catcode`\%=\active\def%{\%}$\mathdefault{40}$}}%
\end{pgfscope}%
\begin{pgfscope}%
\pgfsetbuttcap%
\pgfsetroundjoin%
\definecolor{currentfill}{rgb}{0.000000,0.000000,0.000000}%
\pgfsetfillcolor{currentfill}%
\pgfsetlinewidth{0.803000pt}%
\definecolor{currentstroke}{rgb}{0.000000,0.000000,0.000000}%
\pgfsetstrokecolor{currentstroke}%
\pgfsetdash{}{0pt}%
\pgfsys@defobject{currentmarker}{\pgfqpoint{0.000000in}{-0.048611in}}{\pgfqpoint{0.000000in}{0.000000in}}{%
\pgfpathmoveto{\pgfqpoint{0.000000in}{0.000000in}}%
\pgfpathlineto{\pgfqpoint{0.000000in}{-0.048611in}}%
\pgfusepath{stroke,fill}%
}%
\begin{pgfscope}%
\pgfsys@transformshift{2.256781in}{0.521603in}%
\pgfsys@useobject{currentmarker}{}%
\end{pgfscope}%
\end{pgfscope}%
\begin{pgfscope}%
\definecolor{textcolor}{rgb}{0.000000,0.000000,0.000000}%
\pgfsetstrokecolor{textcolor}%
\pgfsetfillcolor{textcolor}%
\pgftext[x=2.256781in,y=0.424381in,,top]{\color{textcolor}{\rmfamily\fontsize{10.000000}{12.000000}\selectfont\catcode`\^=\active\def^{\ifmmode\sp\else\^{}\fi}\catcode`\%=\active\def%{\%}$\mathdefault{60}$}}%
\end{pgfscope}%
\begin{pgfscope}%
\pgfsetbuttcap%
\pgfsetroundjoin%
\definecolor{currentfill}{rgb}{0.000000,0.000000,0.000000}%
\pgfsetfillcolor{currentfill}%
\pgfsetlinewidth{0.803000pt}%
\definecolor{currentstroke}{rgb}{0.000000,0.000000,0.000000}%
\pgfsetstrokecolor{currentstroke}%
\pgfsetdash{}{0pt}%
\pgfsys@defobject{currentmarker}{\pgfqpoint{0.000000in}{-0.048611in}}{\pgfqpoint{0.000000in}{0.000000in}}{%
\pgfpathmoveto{\pgfqpoint{0.000000in}{0.000000in}}%
\pgfpathlineto{\pgfqpoint{0.000000in}{-0.048611in}}%
\pgfusepath{stroke,fill}%
}%
\begin{pgfscope}%
\pgfsys@transformshift{2.876426in}{0.521603in}%
\pgfsys@useobject{currentmarker}{}%
\end{pgfscope}%
\end{pgfscope}%
\begin{pgfscope}%
\definecolor{textcolor}{rgb}{0.000000,0.000000,0.000000}%
\pgfsetstrokecolor{textcolor}%
\pgfsetfillcolor{textcolor}%
\pgftext[x=2.876426in,y=0.424381in,,top]{\color{textcolor}{\rmfamily\fontsize{10.000000}{12.000000}\selectfont\catcode`\^=\active\def^{\ifmmode\sp\else\^{}\fi}\catcode`\%=\active\def%{\%}$\mathdefault{80}$}}%
\end{pgfscope}%
\begin{pgfscope}%
\pgfsetbuttcap%
\pgfsetroundjoin%
\definecolor{currentfill}{rgb}{0.000000,0.000000,0.000000}%
\pgfsetfillcolor{currentfill}%
\pgfsetlinewidth{0.803000pt}%
\definecolor{currentstroke}{rgb}{0.000000,0.000000,0.000000}%
\pgfsetstrokecolor{currentstroke}%
\pgfsetdash{}{0pt}%
\pgfsys@defobject{currentmarker}{\pgfqpoint{0.000000in}{-0.048611in}}{\pgfqpoint{0.000000in}{0.000000in}}{%
\pgfpathmoveto{\pgfqpoint{0.000000in}{0.000000in}}%
\pgfpathlineto{\pgfqpoint{0.000000in}{-0.048611in}}%
\pgfusepath{stroke,fill}%
}%
\begin{pgfscope}%
\pgfsys@transformshift{3.496071in}{0.521603in}%
\pgfsys@useobject{currentmarker}{}%
\end{pgfscope}%
\end{pgfscope}%
\begin{pgfscope}%
\definecolor{textcolor}{rgb}{0.000000,0.000000,0.000000}%
\pgfsetstrokecolor{textcolor}%
\pgfsetfillcolor{textcolor}%
\pgftext[x=3.496071in,y=0.424381in,,top]{\color{textcolor}{\rmfamily\fontsize{10.000000}{12.000000}\selectfont\catcode`\^=\active\def^{\ifmmode\sp\else\^{}\fi}\catcode`\%=\active\def%{\%}$\mathdefault{100}$}}%
\end{pgfscope}%
\begin{pgfscope}%
\pgfsetbuttcap%
\pgfsetroundjoin%
\definecolor{currentfill}{rgb}{0.000000,0.000000,0.000000}%
\pgfsetfillcolor{currentfill}%
\pgfsetlinewidth{0.803000pt}%
\definecolor{currentstroke}{rgb}{0.000000,0.000000,0.000000}%
\pgfsetstrokecolor{currentstroke}%
\pgfsetdash{}{0pt}%
\pgfsys@defobject{currentmarker}{\pgfqpoint{0.000000in}{-0.048611in}}{\pgfqpoint{0.000000in}{0.000000in}}{%
\pgfpathmoveto{\pgfqpoint{0.000000in}{0.000000in}}%
\pgfpathlineto{\pgfqpoint{0.000000in}{-0.048611in}}%
\pgfusepath{stroke,fill}%
}%
\begin{pgfscope}%
\pgfsys@transformshift{4.115716in}{0.521603in}%
\pgfsys@useobject{currentmarker}{}%
\end{pgfscope}%
\end{pgfscope}%
\begin{pgfscope}%
\definecolor{textcolor}{rgb}{0.000000,0.000000,0.000000}%
\pgfsetstrokecolor{textcolor}%
\pgfsetfillcolor{textcolor}%
\pgftext[x=4.115716in,y=0.424381in,,top]{\color{textcolor}{\rmfamily\fontsize{10.000000}{12.000000}\selectfont\catcode`\^=\active\def^{\ifmmode\sp\else\^{}\fi}\catcode`\%=\active\def%{\%}$\mathdefault{120}$}}%
\end{pgfscope}%
\begin{pgfscope}%
\pgfsetbuttcap%
\pgfsetroundjoin%
\definecolor{currentfill}{rgb}{0.000000,0.000000,0.000000}%
\pgfsetfillcolor{currentfill}%
\pgfsetlinewidth{0.803000pt}%
\definecolor{currentstroke}{rgb}{0.000000,0.000000,0.000000}%
\pgfsetstrokecolor{currentstroke}%
\pgfsetdash{}{0pt}%
\pgfsys@defobject{currentmarker}{\pgfqpoint{0.000000in}{-0.048611in}}{\pgfqpoint{0.000000in}{0.000000in}}{%
\pgfpathmoveto{\pgfqpoint{0.000000in}{0.000000in}}%
\pgfpathlineto{\pgfqpoint{0.000000in}{-0.048611in}}%
\pgfusepath{stroke,fill}%
}%
\begin{pgfscope}%
\pgfsys@transformshift{4.735360in}{0.521603in}%
\pgfsys@useobject{currentmarker}{}%
\end{pgfscope}%
\end{pgfscope}%
\begin{pgfscope}%
\definecolor{textcolor}{rgb}{0.000000,0.000000,0.000000}%
\pgfsetstrokecolor{textcolor}%
\pgfsetfillcolor{textcolor}%
\pgftext[x=4.735360in,y=0.424381in,,top]{\color{textcolor}{\rmfamily\fontsize{10.000000}{12.000000}\selectfont\catcode`\^=\active\def^{\ifmmode\sp\else\^{}\fi}\catcode`\%=\active\def%{\%}$\mathdefault{140}$}}%
\end{pgfscope}%
\begin{pgfscope}%
\definecolor{textcolor}{rgb}{0.000000,0.000000,0.000000}%
\pgfsetstrokecolor{textcolor}%
\pgfsetfillcolor{textcolor}%
\pgftext[x=2.922899in,y=0.234413in,,top]{\color{textcolor}{\rmfamily\fontsize{10.000000}{12.000000}\selectfont\catcode`\^=\active\def^{\ifmmode\sp\else\^{}\fi}\catcode`\%=\active\def%{\%}Triangle components}}%
\end{pgfscope}%
\begin{pgfscope}%
\pgfsetbuttcap%
\pgfsetroundjoin%
\definecolor{currentfill}{rgb}{0.000000,0.000000,0.000000}%
\pgfsetfillcolor{currentfill}%
\pgfsetlinewidth{0.803000pt}%
\definecolor{currentstroke}{rgb}{0.000000,0.000000,0.000000}%
\pgfsetstrokecolor{currentstroke}%
\pgfsetdash{}{0pt}%
\pgfsys@defobject{currentmarker}{\pgfqpoint{-0.048611in}{0.000000in}}{\pgfqpoint{-0.000000in}{0.000000in}}{%
\pgfpathmoveto{\pgfqpoint{-0.000000in}{0.000000in}}%
\pgfpathlineto{\pgfqpoint{-0.048611in}{0.000000in}}%
\pgfusepath{stroke,fill}%
}%
\begin{pgfscope}%
\pgfsys@transformshift{0.588387in}{1.696092in}%
\pgfsys@useobject{currentmarker}{}%
\end{pgfscope}%
\end{pgfscope}%
\begin{pgfscope}%
\definecolor{textcolor}{rgb}{0.000000,0.000000,0.000000}%
\pgfsetstrokecolor{textcolor}%
\pgfsetfillcolor{textcolor}%
\pgftext[x=0.289968in, y=1.643330in, left, base]{\color{textcolor}{\rmfamily\fontsize{10.000000}{12.000000}\selectfont\catcode`\^=\active\def^{\ifmmode\sp\else\^{}\fi}\catcode`\%=\active\def%{\%}$\mathdefault{10^{3}}$}}%
\end{pgfscope}%
\begin{pgfscope}%
\pgfsetbuttcap%
\pgfsetroundjoin%
\definecolor{currentfill}{rgb}{0.000000,0.000000,0.000000}%
\pgfsetfillcolor{currentfill}%
\pgfsetlinewidth{0.602250pt}%
\definecolor{currentstroke}{rgb}{0.000000,0.000000,0.000000}%
\pgfsetstrokecolor{currentstroke}%
\pgfsetdash{}{0pt}%
\pgfsys@defobject{currentmarker}{\pgfqpoint{-0.027778in}{0.000000in}}{\pgfqpoint{-0.000000in}{0.000000in}}{%
\pgfpathmoveto{\pgfqpoint{-0.000000in}{0.000000in}}%
\pgfpathlineto{\pgfqpoint{-0.027778in}{0.000000in}}%
\pgfusepath{stroke,fill}%
}%
\begin{pgfscope}%
\pgfsys@transformshift{0.588387in}{0.835560in}%
\pgfsys@useobject{currentmarker}{}%
\end{pgfscope}%
\end{pgfscope}%
\begin{pgfscope}%
\pgfsetbuttcap%
\pgfsetroundjoin%
\definecolor{currentfill}{rgb}{0.000000,0.000000,0.000000}%
\pgfsetfillcolor{currentfill}%
\pgfsetlinewidth{0.602250pt}%
\definecolor{currentstroke}{rgb}{0.000000,0.000000,0.000000}%
\pgfsetstrokecolor{currentstroke}%
\pgfsetdash{}{0pt}%
\pgfsys@defobject{currentmarker}{\pgfqpoint{-0.027778in}{0.000000in}}{\pgfqpoint{-0.000000in}{0.000000in}}{%
\pgfpathmoveto{\pgfqpoint{-0.000000in}{0.000000in}}%
\pgfpathlineto{\pgfqpoint{-0.027778in}{0.000000in}}%
\pgfusepath{stroke,fill}%
}%
\begin{pgfscope}%
\pgfsys@transformshift{0.588387in}{1.052354in}%
\pgfsys@useobject{currentmarker}{}%
\end{pgfscope}%
\end{pgfscope}%
\begin{pgfscope}%
\pgfsetbuttcap%
\pgfsetroundjoin%
\definecolor{currentfill}{rgb}{0.000000,0.000000,0.000000}%
\pgfsetfillcolor{currentfill}%
\pgfsetlinewidth{0.602250pt}%
\definecolor{currentstroke}{rgb}{0.000000,0.000000,0.000000}%
\pgfsetstrokecolor{currentstroke}%
\pgfsetdash{}{0pt}%
\pgfsys@defobject{currentmarker}{\pgfqpoint{-0.027778in}{0.000000in}}{\pgfqpoint{-0.000000in}{0.000000in}}{%
\pgfpathmoveto{\pgfqpoint{-0.000000in}{0.000000in}}%
\pgfpathlineto{\pgfqpoint{-0.027778in}{0.000000in}}%
\pgfusepath{stroke,fill}%
}%
\begin{pgfscope}%
\pgfsys@transformshift{0.588387in}{1.206171in}%
\pgfsys@useobject{currentmarker}{}%
\end{pgfscope}%
\end{pgfscope}%
\begin{pgfscope}%
\pgfsetbuttcap%
\pgfsetroundjoin%
\definecolor{currentfill}{rgb}{0.000000,0.000000,0.000000}%
\pgfsetfillcolor{currentfill}%
\pgfsetlinewidth{0.602250pt}%
\definecolor{currentstroke}{rgb}{0.000000,0.000000,0.000000}%
\pgfsetstrokecolor{currentstroke}%
\pgfsetdash{}{0pt}%
\pgfsys@defobject{currentmarker}{\pgfqpoint{-0.027778in}{0.000000in}}{\pgfqpoint{-0.000000in}{0.000000in}}{%
\pgfpathmoveto{\pgfqpoint{-0.000000in}{0.000000in}}%
\pgfpathlineto{\pgfqpoint{-0.027778in}{0.000000in}}%
\pgfusepath{stroke,fill}%
}%
\begin{pgfscope}%
\pgfsys@transformshift{0.588387in}{1.325481in}%
\pgfsys@useobject{currentmarker}{}%
\end{pgfscope}%
\end{pgfscope}%
\begin{pgfscope}%
\pgfsetbuttcap%
\pgfsetroundjoin%
\definecolor{currentfill}{rgb}{0.000000,0.000000,0.000000}%
\pgfsetfillcolor{currentfill}%
\pgfsetlinewidth{0.602250pt}%
\definecolor{currentstroke}{rgb}{0.000000,0.000000,0.000000}%
\pgfsetstrokecolor{currentstroke}%
\pgfsetdash{}{0pt}%
\pgfsys@defobject{currentmarker}{\pgfqpoint{-0.027778in}{0.000000in}}{\pgfqpoint{-0.000000in}{0.000000in}}{%
\pgfpathmoveto{\pgfqpoint{-0.000000in}{0.000000in}}%
\pgfpathlineto{\pgfqpoint{-0.027778in}{0.000000in}}%
\pgfusepath{stroke,fill}%
}%
\begin{pgfscope}%
\pgfsys@transformshift{0.588387in}{1.422964in}%
\pgfsys@useobject{currentmarker}{}%
\end{pgfscope}%
\end{pgfscope}%
\begin{pgfscope}%
\pgfsetbuttcap%
\pgfsetroundjoin%
\definecolor{currentfill}{rgb}{0.000000,0.000000,0.000000}%
\pgfsetfillcolor{currentfill}%
\pgfsetlinewidth{0.602250pt}%
\definecolor{currentstroke}{rgb}{0.000000,0.000000,0.000000}%
\pgfsetstrokecolor{currentstroke}%
\pgfsetdash{}{0pt}%
\pgfsys@defobject{currentmarker}{\pgfqpoint{-0.027778in}{0.000000in}}{\pgfqpoint{-0.000000in}{0.000000in}}{%
\pgfpathmoveto{\pgfqpoint{-0.000000in}{0.000000in}}%
\pgfpathlineto{\pgfqpoint{-0.027778in}{0.000000in}}%
\pgfusepath{stroke,fill}%
}%
\begin{pgfscope}%
\pgfsys@transformshift{0.588387in}{1.505385in}%
\pgfsys@useobject{currentmarker}{}%
\end{pgfscope}%
\end{pgfscope}%
\begin{pgfscope}%
\pgfsetbuttcap%
\pgfsetroundjoin%
\definecolor{currentfill}{rgb}{0.000000,0.000000,0.000000}%
\pgfsetfillcolor{currentfill}%
\pgfsetlinewidth{0.602250pt}%
\definecolor{currentstroke}{rgb}{0.000000,0.000000,0.000000}%
\pgfsetstrokecolor{currentstroke}%
\pgfsetdash{}{0pt}%
\pgfsys@defobject{currentmarker}{\pgfqpoint{-0.027778in}{0.000000in}}{\pgfqpoint{-0.000000in}{0.000000in}}{%
\pgfpathmoveto{\pgfqpoint{-0.000000in}{0.000000in}}%
\pgfpathlineto{\pgfqpoint{-0.027778in}{0.000000in}}%
\pgfusepath{stroke,fill}%
}%
\begin{pgfscope}%
\pgfsys@transformshift{0.588387in}{1.576782in}%
\pgfsys@useobject{currentmarker}{}%
\end{pgfscope}%
\end{pgfscope}%
\begin{pgfscope}%
\pgfsetbuttcap%
\pgfsetroundjoin%
\definecolor{currentfill}{rgb}{0.000000,0.000000,0.000000}%
\pgfsetfillcolor{currentfill}%
\pgfsetlinewidth{0.602250pt}%
\definecolor{currentstroke}{rgb}{0.000000,0.000000,0.000000}%
\pgfsetstrokecolor{currentstroke}%
\pgfsetdash{}{0pt}%
\pgfsys@defobject{currentmarker}{\pgfqpoint{-0.027778in}{0.000000in}}{\pgfqpoint{-0.000000in}{0.000000in}}{%
\pgfpathmoveto{\pgfqpoint{-0.000000in}{0.000000in}}%
\pgfpathlineto{\pgfqpoint{-0.027778in}{0.000000in}}%
\pgfusepath{stroke,fill}%
}%
\begin{pgfscope}%
\pgfsys@transformshift{0.588387in}{1.639758in}%
\pgfsys@useobject{currentmarker}{}%
\end{pgfscope}%
\end{pgfscope}%
\begin{pgfscope}%
\pgfsetbuttcap%
\pgfsetroundjoin%
\definecolor{currentfill}{rgb}{0.000000,0.000000,0.000000}%
\pgfsetfillcolor{currentfill}%
\pgfsetlinewidth{0.602250pt}%
\definecolor{currentstroke}{rgb}{0.000000,0.000000,0.000000}%
\pgfsetstrokecolor{currentstroke}%
\pgfsetdash{}{0pt}%
\pgfsys@defobject{currentmarker}{\pgfqpoint{-0.027778in}{0.000000in}}{\pgfqpoint{-0.000000in}{0.000000in}}{%
\pgfpathmoveto{\pgfqpoint{-0.000000in}{0.000000in}}%
\pgfpathlineto{\pgfqpoint{-0.027778in}{0.000000in}}%
\pgfusepath{stroke,fill}%
}%
\begin{pgfscope}%
\pgfsys@transformshift{0.588387in}{2.066702in}%
\pgfsys@useobject{currentmarker}{}%
\end{pgfscope}%
\end{pgfscope}%
\begin{pgfscope}%
\pgfsetbuttcap%
\pgfsetroundjoin%
\definecolor{currentfill}{rgb}{0.000000,0.000000,0.000000}%
\pgfsetfillcolor{currentfill}%
\pgfsetlinewidth{0.602250pt}%
\definecolor{currentstroke}{rgb}{0.000000,0.000000,0.000000}%
\pgfsetstrokecolor{currentstroke}%
\pgfsetdash{}{0pt}%
\pgfsys@defobject{currentmarker}{\pgfqpoint{-0.027778in}{0.000000in}}{\pgfqpoint{-0.000000in}{0.000000in}}{%
\pgfpathmoveto{\pgfqpoint{-0.000000in}{0.000000in}}%
\pgfpathlineto{\pgfqpoint{-0.027778in}{0.000000in}}%
\pgfusepath{stroke,fill}%
}%
\begin{pgfscope}%
\pgfsys@transformshift{0.588387in}{2.283496in}%
\pgfsys@useobject{currentmarker}{}%
\end{pgfscope}%
\end{pgfscope}%
\begin{pgfscope}%
\pgfsetbuttcap%
\pgfsetroundjoin%
\definecolor{currentfill}{rgb}{0.000000,0.000000,0.000000}%
\pgfsetfillcolor{currentfill}%
\pgfsetlinewidth{0.602250pt}%
\definecolor{currentstroke}{rgb}{0.000000,0.000000,0.000000}%
\pgfsetstrokecolor{currentstroke}%
\pgfsetdash{}{0pt}%
\pgfsys@defobject{currentmarker}{\pgfqpoint{-0.027778in}{0.000000in}}{\pgfqpoint{-0.000000in}{0.000000in}}{%
\pgfpathmoveto{\pgfqpoint{-0.000000in}{0.000000in}}%
\pgfpathlineto{\pgfqpoint{-0.027778in}{0.000000in}}%
\pgfusepath{stroke,fill}%
}%
\begin{pgfscope}%
\pgfsys@transformshift{0.588387in}{2.437313in}%
\pgfsys@useobject{currentmarker}{}%
\end{pgfscope}%
\end{pgfscope}%
\begin{pgfscope}%
\definecolor{textcolor}{rgb}{0.000000,0.000000,0.000000}%
\pgfsetstrokecolor{textcolor}%
\pgfsetfillcolor{textcolor}%
\pgftext[x=0.234413in,y=1.526746in,,bottom,rotate=90.000000]{\color{textcolor}{\rmfamily\fontsize{10.000000}{12.000000}\selectfont\catcode`\^=\active\def^{\ifmmode\sp\else\^{}\fi}\catcode`\%=\active\def%{\%}Time [ms]}}%
\end{pgfscope}%
\begin{pgfscope}%
\pgfpathrectangle{\pgfqpoint{0.588387in}{0.521603in}}{\pgfqpoint{4.669024in}{2.010285in}}%
\pgfusepath{clip}%
\pgfsetrectcap%
\pgfsetroundjoin%
\pgfsetlinewidth{1.505625pt}%
\pgfsetstrokecolor{currentstroke1}%
\pgfsetdash{}{0pt}%
\pgfpathmoveto{\pgfqpoint{0.800616in}{1.014144in}}%
\pgfpathlineto{\pgfqpoint{0.831598in}{0.632380in}}%
\pgfpathlineto{\pgfqpoint{0.862580in}{0.700493in}}%
\pgfpathlineto{\pgfqpoint{0.893562in}{0.844372in}}%
\pgfpathlineto{\pgfqpoint{0.924545in}{0.922774in}}%
\pgfpathlineto{\pgfqpoint{0.955527in}{0.964352in}}%
\pgfpathlineto{\pgfqpoint{0.986509in}{0.948393in}}%
\pgfpathlineto{\pgfqpoint{1.017491in}{1.044323in}}%
\pgfpathlineto{\pgfqpoint{1.048474in}{1.077766in}}%
\pgfpathlineto{\pgfqpoint{1.079456in}{0.990339in}}%
\pgfpathlineto{\pgfqpoint{1.110438in}{1.024191in}}%
\pgfpathlineto{\pgfqpoint{1.141420in}{0.957825in}}%
\pgfpathlineto{\pgfqpoint{1.172402in}{1.016072in}}%
\pgfpathlineto{\pgfqpoint{1.203385in}{0.978019in}}%
\pgfpathlineto{\pgfqpoint{1.234367in}{1.061563in}}%
\pgfpathlineto{\pgfqpoint{1.265349in}{1.072881in}}%
\pgfpathlineto{\pgfqpoint{1.296331in}{1.087597in}}%
\pgfpathlineto{\pgfqpoint{1.327314in}{1.058847in}}%
\pgfpathlineto{\pgfqpoint{1.358296in}{1.055025in}}%
\pgfpathlineto{\pgfqpoint{1.389278in}{1.055178in}}%
\pgfpathlineto{\pgfqpoint{1.420260in}{1.135704in}}%
\pgfpathlineto{\pgfqpoint{1.451243in}{1.159891in}}%
\pgfpathlineto{\pgfqpoint{1.482225in}{1.083863in}}%
\pgfpathlineto{\pgfqpoint{1.513207in}{1.146726in}}%
\pgfpathlineto{\pgfqpoint{1.544189in}{1.130002in}}%
\pgfpathlineto{\pgfqpoint{1.575172in}{1.202030in}}%
\pgfpathlineto{\pgfqpoint{1.606154in}{1.183083in}}%
\pgfpathlineto{\pgfqpoint{1.637136in}{1.207661in}}%
\pgfpathlineto{\pgfqpoint{1.668118in}{1.194878in}}%
\pgfpathlineto{\pgfqpoint{1.699101in}{1.292110in}}%
\pgfpathlineto{\pgfqpoint{1.730083in}{1.268411in}}%
\pgfpathlineto{\pgfqpoint{1.761065in}{1.289792in}}%
\pgfpathlineto{\pgfqpoint{1.792047in}{1.317988in}}%
\pgfpathlineto{\pgfqpoint{1.823030in}{1.278435in}}%
\pgfpathlineto{\pgfqpoint{1.854012in}{1.340789in}}%
\pgfpathlineto{\pgfqpoint{1.884994in}{1.323185in}}%
\pgfpathlineto{\pgfqpoint{1.915976in}{1.329756in}}%
\pgfpathlineto{\pgfqpoint{1.946959in}{1.329308in}}%
\pgfpathlineto{\pgfqpoint{1.977941in}{1.368281in}}%
\pgfpathlineto{\pgfqpoint{2.008923in}{1.387997in}}%
\pgfpathlineto{\pgfqpoint{2.039905in}{1.433115in}}%
\pgfpathlineto{\pgfqpoint{2.070888in}{1.392169in}}%
\pgfpathlineto{\pgfqpoint{2.101870in}{1.455025in}}%
\pgfpathlineto{\pgfqpoint{2.132852in}{1.421362in}}%
\pgfpathlineto{\pgfqpoint{2.163834in}{1.490891in}}%
\pgfpathlineto{\pgfqpoint{2.194817in}{1.488755in}}%
\pgfpathlineto{\pgfqpoint{2.225799in}{1.494857in}}%
\pgfpathlineto{\pgfqpoint{2.256781in}{1.530761in}}%
\pgfpathlineto{\pgfqpoint{2.287763in}{1.520190in}}%
\pgfpathlineto{\pgfqpoint{2.318745in}{1.595894in}}%
\pgfpathlineto{\pgfqpoint{2.349728in}{1.531836in}}%
\pgfpathlineto{\pgfqpoint{2.380710in}{1.624395in}}%
\pgfpathlineto{\pgfqpoint{2.411692in}{1.618629in}}%
\pgfpathlineto{\pgfqpoint{2.442674in}{1.592766in}}%
\pgfpathlineto{\pgfqpoint{2.473657in}{1.649919in}}%
\pgfpathlineto{\pgfqpoint{2.504639in}{1.652613in}}%
\pgfpathlineto{\pgfqpoint{2.535621in}{1.648585in}}%
\pgfpathlineto{\pgfqpoint{2.566603in}{1.645100in}}%
\pgfpathlineto{\pgfqpoint{2.597586in}{1.703540in}}%
\pgfpathlineto{\pgfqpoint{2.628568in}{1.719959in}}%
\pgfpathlineto{\pgfqpoint{2.659550in}{1.685012in}}%
\pgfpathlineto{\pgfqpoint{2.690532in}{1.713727in}}%
\pgfpathlineto{\pgfqpoint{2.721515in}{1.696009in}}%
\pgfpathlineto{\pgfqpoint{2.752497in}{1.762318in}}%
\pgfpathlineto{\pgfqpoint{2.783479in}{1.823581in}}%
\pgfpathlineto{\pgfqpoint{2.814461in}{1.810349in}}%
\pgfpathlineto{\pgfqpoint{2.845444in}{1.742087in}}%
\pgfpathlineto{\pgfqpoint{2.876426in}{1.805067in}}%
\pgfpathlineto{\pgfqpoint{2.907408in}{1.835267in}}%
\pgfpathlineto{\pgfqpoint{2.938390in}{1.845268in}}%
\pgfpathlineto{\pgfqpoint{2.969373in}{1.857752in}}%
\pgfpathlineto{\pgfqpoint{3.000355in}{1.865305in}}%
\pgfpathlineto{\pgfqpoint{3.031337in}{1.863926in}}%
\pgfpathlineto{\pgfqpoint{3.062319in}{1.872560in}}%
\pgfpathlineto{\pgfqpoint{3.093302in}{1.961099in}}%
\pgfpathlineto{\pgfqpoint{3.124284in}{1.957375in}}%
\pgfpathlineto{\pgfqpoint{3.155266in}{1.947671in}}%
\pgfpathlineto{\pgfqpoint{3.186248in}{1.962548in}}%
\pgfpathlineto{\pgfqpoint{3.217231in}{1.986825in}}%
\pgfpathlineto{\pgfqpoint{3.248213in}{2.070508in}}%
\pgfpathlineto{\pgfqpoint{3.279195in}{1.983292in}}%
\pgfpathlineto{\pgfqpoint{3.310177in}{2.031472in}}%
\pgfpathlineto{\pgfqpoint{3.341159in}{2.109397in}}%
\pgfpathlineto{\pgfqpoint{3.372142in}{2.036838in}}%
\pgfpathlineto{\pgfqpoint{3.403124in}{2.052212in}}%
\pgfpathlineto{\pgfqpoint{3.434106in}{2.082413in}}%
\pgfpathlineto{\pgfqpoint{3.465088in}{2.136925in}}%
\pgfpathlineto{\pgfqpoint{3.496071in}{2.162624in}}%
\pgfpathlineto{\pgfqpoint{3.527053in}{2.122487in}}%
\pgfpathlineto{\pgfqpoint{3.558035in}{2.191809in}}%
\pgfpathlineto{\pgfqpoint{3.589017in}{2.166720in}}%
\pgfpathlineto{\pgfqpoint{3.620000in}{2.164896in}}%
\pgfpathlineto{\pgfqpoint{3.650982in}{2.197497in}}%
\pgfpathlineto{\pgfqpoint{3.681964in}{2.158708in}}%
\pgfpathlineto{\pgfqpoint{3.712946in}{2.208855in}}%
\pgfpathlineto{\pgfqpoint{3.743929in}{2.257054in}}%
\pgfpathlineto{\pgfqpoint{3.774911in}{2.219801in}}%
\pgfpathlineto{\pgfqpoint{3.805893in}{2.248437in}}%
\pgfpathlineto{\pgfqpoint{3.836875in}{2.254358in}}%
\pgfpathlineto{\pgfqpoint{3.867858in}{2.258893in}}%
\pgfpathlineto{\pgfqpoint{3.898840in}{2.310465in}}%
\pgfpathlineto{\pgfqpoint{3.929822in}{2.266198in}}%
\pgfpathlineto{\pgfqpoint{3.960804in}{2.257756in}}%
\pgfpathlineto{\pgfqpoint{3.991787in}{2.266106in}}%
\pgfpathlineto{\pgfqpoint{4.022769in}{2.316609in}}%
\pgfpathlineto{\pgfqpoint{4.053751in}{2.309847in}}%
\pgfpathlineto{\pgfqpoint{4.084733in}{2.320004in}}%
\pgfpathlineto{\pgfqpoint{4.115716in}{2.313641in}}%
\pgfpathlineto{\pgfqpoint{4.146698in}{2.344408in}}%
\pgfpathlineto{\pgfqpoint{4.177680in}{2.376917in}}%
\pgfpathlineto{\pgfqpoint{4.208662in}{2.374254in}}%
\pgfpathlineto{\pgfqpoint{4.239645in}{2.396712in}}%
\pgfpathlineto{\pgfqpoint{4.270627in}{2.355115in}}%
\pgfpathlineto{\pgfqpoint{4.332591in}{2.388501in}}%
\pgfpathlineto{\pgfqpoint{4.363573in}{2.335266in}}%
\pgfpathlineto{\pgfqpoint{4.394556in}{2.427873in}}%
\pgfpathlineto{\pgfqpoint{4.425538in}{2.428689in}}%
\pgfpathlineto{\pgfqpoint{4.456520in}{2.387181in}}%
\pgfpathlineto{\pgfqpoint{4.549467in}{2.432614in}}%
\pgfpathlineto{\pgfqpoint{4.704378in}{2.416599in}}%
\pgfusepath{stroke}%
\end{pgfscope}%
\begin{pgfscope}%
\pgfpathrectangle{\pgfqpoint{0.588387in}{0.521603in}}{\pgfqpoint{4.669024in}{2.010285in}}%
\pgfusepath{clip}%
\pgfsetrectcap%
\pgfsetroundjoin%
\pgfsetlinewidth{1.505625pt}%
\pgfsetstrokecolor{currentstroke2}%
\pgfsetdash{}{0pt}%
\pgfpathmoveto{\pgfqpoint{0.800616in}{1.034808in}}%
\pgfpathlineto{\pgfqpoint{0.831598in}{0.631309in}}%
\pgfpathlineto{\pgfqpoint{0.862580in}{0.684641in}}%
\pgfpathlineto{\pgfqpoint{0.893562in}{0.806969in}}%
\pgfpathlineto{\pgfqpoint{0.924545in}{0.885302in}}%
\pgfpathlineto{\pgfqpoint{0.955527in}{0.957722in}}%
\pgfpathlineto{\pgfqpoint{0.986509in}{0.952576in}}%
\pgfpathlineto{\pgfqpoint{1.017491in}{1.042919in}}%
\pgfpathlineto{\pgfqpoint{1.048474in}{1.086571in}}%
\pgfpathlineto{\pgfqpoint{1.079456in}{0.974285in}}%
\pgfpathlineto{\pgfqpoint{1.110438in}{1.016670in}}%
\pgfpathlineto{\pgfqpoint{1.141420in}{0.969305in}}%
\pgfpathlineto{\pgfqpoint{1.172402in}{1.018907in}}%
\pgfpathlineto{\pgfqpoint{1.203385in}{0.968021in}}%
\pgfpathlineto{\pgfqpoint{1.234367in}{1.064339in}}%
\pgfpathlineto{\pgfqpoint{1.265349in}{1.075928in}}%
\pgfpathlineto{\pgfqpoint{1.296331in}{1.082347in}}%
\pgfpathlineto{\pgfqpoint{1.327314in}{1.054081in}}%
\pgfpathlineto{\pgfqpoint{1.358296in}{1.050704in}}%
\pgfpathlineto{\pgfqpoint{1.389278in}{1.050634in}}%
\pgfpathlineto{\pgfqpoint{1.420260in}{1.136811in}}%
\pgfpathlineto{\pgfqpoint{1.451243in}{1.158636in}}%
\pgfpathlineto{\pgfqpoint{1.482225in}{1.085746in}}%
\pgfpathlineto{\pgfqpoint{1.513207in}{1.150818in}}%
\pgfpathlineto{\pgfqpoint{1.544189in}{1.113517in}}%
\pgfpathlineto{\pgfqpoint{1.575172in}{1.202419in}}%
\pgfpathlineto{\pgfqpoint{1.637136in}{1.192155in}}%
\pgfpathlineto{\pgfqpoint{1.668118in}{1.193903in}}%
\pgfpathlineto{\pgfqpoint{1.699101in}{1.315390in}}%
\pgfpathlineto{\pgfqpoint{1.730083in}{1.262641in}}%
\pgfpathlineto{\pgfqpoint{1.761065in}{1.279848in}}%
\pgfpathlineto{\pgfqpoint{1.792047in}{1.314284in}}%
\pgfpathlineto{\pgfqpoint{1.823030in}{1.270299in}}%
\pgfpathlineto{\pgfqpoint{1.854012in}{1.334154in}}%
\pgfpathlineto{\pgfqpoint{1.884994in}{1.307467in}}%
\pgfpathlineto{\pgfqpoint{1.915976in}{1.312418in}}%
\pgfpathlineto{\pgfqpoint{1.946959in}{1.309846in}}%
\pgfpathlineto{\pgfqpoint{1.977941in}{1.360739in}}%
\pgfpathlineto{\pgfqpoint{2.008923in}{1.373063in}}%
\pgfpathlineto{\pgfqpoint{2.039905in}{1.420146in}}%
\pgfpathlineto{\pgfqpoint{2.070888in}{1.396541in}}%
\pgfpathlineto{\pgfqpoint{2.101870in}{1.424801in}}%
\pgfpathlineto{\pgfqpoint{2.132852in}{1.418199in}}%
\pgfpathlineto{\pgfqpoint{2.163834in}{1.474669in}}%
\pgfpathlineto{\pgfqpoint{2.194817in}{1.436359in}}%
\pgfpathlineto{\pgfqpoint{2.225799in}{1.481344in}}%
\pgfpathlineto{\pgfqpoint{2.256781in}{1.508885in}}%
\pgfpathlineto{\pgfqpoint{2.287763in}{1.485618in}}%
\pgfpathlineto{\pgfqpoint{2.318745in}{1.555825in}}%
\pgfpathlineto{\pgfqpoint{2.349728in}{1.499143in}}%
\pgfpathlineto{\pgfqpoint{2.380710in}{1.587571in}}%
\pgfpathlineto{\pgfqpoint{2.411692in}{1.582751in}}%
\pgfpathlineto{\pgfqpoint{2.442674in}{1.541676in}}%
\pgfpathlineto{\pgfqpoint{2.473657in}{1.616038in}}%
\pgfpathlineto{\pgfqpoint{2.504639in}{1.607470in}}%
\pgfpathlineto{\pgfqpoint{2.535621in}{1.639187in}}%
\pgfpathlineto{\pgfqpoint{2.566603in}{1.605899in}}%
\pgfpathlineto{\pgfqpoint{2.597586in}{1.668997in}}%
\pgfpathlineto{\pgfqpoint{2.628568in}{1.668119in}}%
\pgfpathlineto{\pgfqpoint{2.659550in}{1.648078in}}%
\pgfpathlineto{\pgfqpoint{2.690532in}{1.663509in}}%
\pgfpathlineto{\pgfqpoint{2.721515in}{1.645820in}}%
\pgfpathlineto{\pgfqpoint{2.752497in}{1.715714in}}%
\pgfpathlineto{\pgfqpoint{2.783479in}{1.753082in}}%
\pgfpathlineto{\pgfqpoint{2.814461in}{1.752565in}}%
\pgfpathlineto{\pgfqpoint{2.845444in}{1.694887in}}%
\pgfpathlineto{\pgfqpoint{2.876426in}{1.746023in}}%
\pgfpathlineto{\pgfqpoint{2.907408in}{1.799948in}}%
\pgfpathlineto{\pgfqpoint{2.938390in}{1.788754in}}%
\pgfpathlineto{\pgfqpoint{2.969373in}{1.791757in}}%
\pgfpathlineto{\pgfqpoint{3.000355in}{1.817962in}}%
\pgfpathlineto{\pgfqpoint{3.031337in}{1.804319in}}%
\pgfpathlineto{\pgfqpoint{3.062319in}{1.811666in}}%
\pgfpathlineto{\pgfqpoint{3.093302in}{1.915197in}}%
\pgfpathlineto{\pgfqpoint{3.124284in}{1.877643in}}%
\pgfpathlineto{\pgfqpoint{3.155266in}{1.893472in}}%
\pgfpathlineto{\pgfqpoint{3.186248in}{1.897835in}}%
\pgfpathlineto{\pgfqpoint{3.217231in}{1.927026in}}%
\pgfpathlineto{\pgfqpoint{3.248213in}{2.011270in}}%
\pgfpathlineto{\pgfqpoint{3.279195in}{1.933613in}}%
\pgfpathlineto{\pgfqpoint{3.310177in}{1.937638in}}%
\pgfpathlineto{\pgfqpoint{3.341159in}{1.995561in}}%
\pgfpathlineto{\pgfqpoint{3.372142in}{1.961464in}}%
\pgfpathlineto{\pgfqpoint{3.403124in}{1.990032in}}%
\pgfpathlineto{\pgfqpoint{3.434106in}{1.995120in}}%
\pgfpathlineto{\pgfqpoint{3.465088in}{2.063400in}}%
\pgfpathlineto{\pgfqpoint{3.496071in}{2.055733in}}%
\pgfpathlineto{\pgfqpoint{3.527053in}{2.025442in}}%
\pgfpathlineto{\pgfqpoint{3.558035in}{2.090365in}}%
\pgfpathlineto{\pgfqpoint{3.589017in}{2.092535in}}%
\pgfpathlineto{\pgfqpoint{3.620000in}{2.104506in}}%
\pgfpathlineto{\pgfqpoint{3.650982in}{2.129986in}}%
\pgfpathlineto{\pgfqpoint{3.681964in}{2.134107in}}%
\pgfpathlineto{\pgfqpoint{3.712946in}{2.127849in}}%
\pgfpathlineto{\pgfqpoint{3.743929in}{2.175865in}}%
\pgfpathlineto{\pgfqpoint{3.774911in}{2.162940in}}%
\pgfpathlineto{\pgfqpoint{3.805893in}{2.227756in}}%
\pgfpathlineto{\pgfqpoint{3.836875in}{2.188339in}}%
\pgfpathlineto{\pgfqpoint{3.867858in}{2.166066in}}%
\pgfpathlineto{\pgfqpoint{3.898840in}{2.246173in}}%
\pgfpathlineto{\pgfqpoint{3.929822in}{2.181583in}}%
\pgfpathlineto{\pgfqpoint{3.991787in}{2.217329in}}%
\pgfpathlineto{\pgfqpoint{4.022769in}{2.241330in}}%
\pgfpathlineto{\pgfqpoint{4.053751in}{2.243329in}}%
\pgfpathlineto{\pgfqpoint{4.084733in}{2.240075in}}%
\pgfpathlineto{\pgfqpoint{4.115716in}{2.218075in}}%
\pgfpathlineto{\pgfqpoint{4.146698in}{2.277762in}}%
\pgfpathlineto{\pgfqpoint{4.177680in}{2.320138in}}%
\pgfpathlineto{\pgfqpoint{4.208662in}{2.310113in}}%
\pgfpathlineto{\pgfqpoint{4.239645in}{2.315961in}}%
\pgfpathlineto{\pgfqpoint{4.270627in}{2.371426in}}%
\pgfpathlineto{\pgfqpoint{4.301609in}{2.299127in}}%
\pgfpathlineto{\pgfqpoint{4.332591in}{2.376542in}}%
\pgfpathlineto{\pgfqpoint{4.363573in}{2.250602in}}%
\pgfpathlineto{\pgfqpoint{4.394556in}{2.376388in}}%
\pgfpathlineto{\pgfqpoint{4.425538in}{2.376130in}}%
\pgfpathlineto{\pgfqpoint{4.456520in}{2.366426in}}%
\pgfpathlineto{\pgfqpoint{4.487502in}{2.371161in}}%
\pgfpathlineto{\pgfqpoint{4.518485in}{2.431838in}}%
\pgfpathlineto{\pgfqpoint{4.549467in}{2.371741in}}%
\pgfpathlineto{\pgfqpoint{4.580449in}{2.409325in}}%
\pgfpathlineto{\pgfqpoint{4.611431in}{2.381573in}}%
\pgfpathlineto{\pgfqpoint{4.642414in}{2.417016in}}%
\pgfpathlineto{\pgfqpoint{4.673396in}{2.400376in}}%
\pgfpathlineto{\pgfqpoint{4.704378in}{2.340257in}}%
\pgfpathlineto{\pgfqpoint{4.797325in}{2.422816in}}%
\pgfpathlineto{\pgfqpoint{4.797325in}{2.422816in}}%
\pgfusepath{stroke}%
\end{pgfscope}%
\begin{pgfscope}%
\pgfpathrectangle{\pgfqpoint{0.588387in}{0.521603in}}{\pgfqpoint{4.669024in}{2.010285in}}%
\pgfusepath{clip}%
\pgfsetrectcap%
\pgfsetroundjoin%
\pgfsetlinewidth{1.505625pt}%
\pgfsetstrokecolor{currentstroke3}%
\pgfsetdash{}{0pt}%
\pgfpathmoveto{\pgfqpoint{0.800616in}{0.956326in}}%
\pgfpathlineto{\pgfqpoint{0.831598in}{0.614711in}}%
\pgfpathlineto{\pgfqpoint{0.862580in}{0.708424in}}%
\pgfpathlineto{\pgfqpoint{0.893562in}{0.811535in}}%
\pgfpathlineto{\pgfqpoint{0.924545in}{0.922019in}}%
\pgfpathlineto{\pgfqpoint{0.955527in}{0.960988in}}%
\pgfpathlineto{\pgfqpoint{0.986509in}{0.971209in}}%
\pgfpathlineto{\pgfqpoint{1.017491in}{1.075728in}}%
\pgfpathlineto{\pgfqpoint{1.048474in}{1.139752in}}%
\pgfpathlineto{\pgfqpoint{1.079456in}{1.045616in}}%
\pgfpathlineto{\pgfqpoint{1.110438in}{1.087972in}}%
\pgfpathlineto{\pgfqpoint{1.141420in}{1.070775in}}%
\pgfpathlineto{\pgfqpoint{1.172402in}{1.093147in}}%
\pgfpathlineto{\pgfqpoint{1.203385in}{1.051807in}}%
\pgfpathlineto{\pgfqpoint{1.234367in}{1.133226in}}%
\pgfpathlineto{\pgfqpoint{1.265349in}{1.150074in}}%
\pgfpathlineto{\pgfqpoint{1.296331in}{1.157987in}}%
\pgfpathlineto{\pgfqpoint{1.327314in}{1.133766in}}%
\pgfpathlineto{\pgfqpoint{1.358296in}{1.105464in}}%
\pgfpathlineto{\pgfqpoint{1.389278in}{1.132596in}}%
\pgfpathlineto{\pgfqpoint{1.420260in}{1.282193in}}%
\pgfpathlineto{\pgfqpoint{1.451243in}{1.237701in}}%
\pgfpathlineto{\pgfqpoint{1.482225in}{1.150634in}}%
\pgfpathlineto{\pgfqpoint{1.513207in}{1.225774in}}%
\pgfpathlineto{\pgfqpoint{1.544189in}{1.202752in}}%
\pgfpathlineto{\pgfqpoint{1.575172in}{1.332878in}}%
\pgfpathlineto{\pgfqpoint{1.606154in}{1.284402in}}%
\pgfpathlineto{\pgfqpoint{1.637136in}{1.302891in}}%
\pgfpathlineto{\pgfqpoint{1.668118in}{1.265782in}}%
\pgfpathlineto{\pgfqpoint{1.699101in}{1.388853in}}%
\pgfpathlineto{\pgfqpoint{1.730083in}{1.348678in}}%
\pgfpathlineto{\pgfqpoint{1.792047in}{1.404158in}}%
\pgfpathlineto{\pgfqpoint{1.823030in}{1.378160in}}%
\pgfpathlineto{\pgfqpoint{1.854012in}{1.440838in}}%
\pgfpathlineto{\pgfqpoint{1.884994in}{1.374933in}}%
\pgfpathlineto{\pgfqpoint{1.915976in}{1.384245in}}%
\pgfpathlineto{\pgfqpoint{1.946959in}{1.410603in}}%
\pgfpathlineto{\pgfqpoint{1.977941in}{1.444081in}}%
\pgfpathlineto{\pgfqpoint{2.008923in}{1.479431in}}%
\pgfpathlineto{\pgfqpoint{2.039905in}{1.516979in}}%
\pgfpathlineto{\pgfqpoint{2.070888in}{1.477337in}}%
\pgfpathlineto{\pgfqpoint{2.101870in}{1.531538in}}%
\pgfpathlineto{\pgfqpoint{2.132852in}{1.514954in}}%
\pgfpathlineto{\pgfqpoint{2.163834in}{1.596954in}}%
\pgfpathlineto{\pgfqpoint{2.194817in}{1.517516in}}%
\pgfpathlineto{\pgfqpoint{2.225799in}{1.567792in}}%
\pgfpathlineto{\pgfqpoint{2.256781in}{1.573434in}}%
\pgfpathlineto{\pgfqpoint{2.287763in}{1.555059in}}%
\pgfpathlineto{\pgfqpoint{2.318745in}{1.672444in}}%
\pgfpathlineto{\pgfqpoint{2.349728in}{1.593445in}}%
\pgfpathlineto{\pgfqpoint{2.380710in}{1.646499in}}%
\pgfpathlineto{\pgfqpoint{2.411692in}{1.653124in}}%
\pgfpathlineto{\pgfqpoint{2.442674in}{1.621053in}}%
\pgfpathlineto{\pgfqpoint{2.473657in}{1.690109in}}%
\pgfpathlineto{\pgfqpoint{2.504639in}{1.676702in}}%
\pgfpathlineto{\pgfqpoint{2.535621in}{1.657129in}}%
\pgfpathlineto{\pgfqpoint{2.566603in}{1.680399in}}%
\pgfpathlineto{\pgfqpoint{2.597586in}{1.759926in}}%
\pgfpathlineto{\pgfqpoint{2.628568in}{1.729608in}}%
\pgfpathlineto{\pgfqpoint{2.659550in}{1.720094in}}%
\pgfpathlineto{\pgfqpoint{2.690532in}{1.765607in}}%
\pgfpathlineto{\pgfqpoint{2.721515in}{1.688406in}}%
\pgfpathlineto{\pgfqpoint{2.752497in}{1.758024in}}%
\pgfpathlineto{\pgfqpoint{2.783479in}{1.853765in}}%
\pgfpathlineto{\pgfqpoint{2.814461in}{1.811297in}}%
\pgfpathlineto{\pgfqpoint{2.845444in}{1.754563in}}%
\pgfpathlineto{\pgfqpoint{2.876426in}{1.801915in}}%
\pgfpathlineto{\pgfqpoint{2.907408in}{1.877918in}}%
\pgfpathlineto{\pgfqpoint{2.938390in}{1.820685in}}%
\pgfpathlineto{\pgfqpoint{2.969373in}{1.870225in}}%
\pgfpathlineto{\pgfqpoint{3.000355in}{1.876714in}}%
\pgfpathlineto{\pgfqpoint{3.031337in}{1.898651in}}%
\pgfpathlineto{\pgfqpoint{3.062319in}{1.901168in}}%
\pgfpathlineto{\pgfqpoint{3.093302in}{1.920729in}}%
\pgfpathlineto{\pgfqpoint{3.124284in}{1.928749in}}%
\pgfpathlineto{\pgfqpoint{3.155266in}{1.917962in}}%
\pgfpathlineto{\pgfqpoint{3.186248in}{1.932897in}}%
\pgfpathlineto{\pgfqpoint{3.217231in}{1.950365in}}%
\pgfpathlineto{\pgfqpoint{3.248213in}{2.027114in}}%
\pgfpathlineto{\pgfqpoint{3.279195in}{1.958314in}}%
\pgfpathlineto{\pgfqpoint{3.310177in}{2.004089in}}%
\pgfpathlineto{\pgfqpoint{3.341159in}{2.067937in}}%
\pgfpathlineto{\pgfqpoint{3.372142in}{1.989417in}}%
\pgfpathlineto{\pgfqpoint{3.403124in}{2.028429in}}%
\pgfpathlineto{\pgfqpoint{3.434106in}{2.031285in}}%
\pgfpathlineto{\pgfqpoint{3.465088in}{2.061408in}}%
\pgfpathlineto{\pgfqpoint{3.496071in}{2.073800in}}%
\pgfpathlineto{\pgfqpoint{3.527053in}{2.094742in}}%
\pgfpathlineto{\pgfqpoint{3.558035in}{2.105238in}}%
\pgfpathlineto{\pgfqpoint{3.589017in}{2.117203in}}%
\pgfpathlineto{\pgfqpoint{3.620000in}{2.124160in}}%
\pgfpathlineto{\pgfqpoint{3.650982in}{2.114577in}}%
\pgfpathlineto{\pgfqpoint{3.681964in}{2.150342in}}%
\pgfpathlineto{\pgfqpoint{3.712946in}{2.142213in}}%
\pgfpathlineto{\pgfqpoint{3.743929in}{2.156105in}}%
\pgfpathlineto{\pgfqpoint{3.774911in}{2.149741in}}%
\pgfpathlineto{\pgfqpoint{3.805893in}{2.196163in}}%
\pgfpathlineto{\pgfqpoint{3.836875in}{2.199808in}}%
\pgfpathlineto{\pgfqpoint{3.867858in}{2.199815in}}%
\pgfpathlineto{\pgfqpoint{3.898840in}{2.234411in}}%
\pgfpathlineto{\pgfqpoint{3.929822in}{2.198118in}}%
\pgfpathlineto{\pgfqpoint{3.960804in}{2.201183in}}%
\pgfpathlineto{\pgfqpoint{3.991787in}{2.227991in}}%
\pgfpathlineto{\pgfqpoint{4.022769in}{2.266886in}}%
\pgfpathlineto{\pgfqpoint{4.053751in}{2.256867in}}%
\pgfpathlineto{\pgfqpoint{4.084733in}{2.239540in}}%
\pgfpathlineto{\pgfqpoint{4.115716in}{2.275355in}}%
\pgfpathlineto{\pgfqpoint{4.146698in}{2.272008in}}%
\pgfpathlineto{\pgfqpoint{4.177680in}{2.307553in}}%
\pgfpathlineto{\pgfqpoint{4.208662in}{2.304112in}}%
\pgfpathlineto{\pgfqpoint{4.239645in}{2.327211in}}%
\pgfpathlineto{\pgfqpoint{4.270627in}{2.330477in}}%
\pgfpathlineto{\pgfqpoint{4.301609in}{2.296999in}}%
\pgfpathlineto{\pgfqpoint{4.332591in}{2.363548in}}%
\pgfpathlineto{\pgfqpoint{4.363573in}{2.340578in}}%
\pgfpathlineto{\pgfqpoint{4.394556in}{2.370427in}}%
\pgfpathlineto{\pgfqpoint{4.425538in}{2.348825in}}%
\pgfpathlineto{\pgfqpoint{4.456520in}{2.366022in}}%
\pgfpathlineto{\pgfqpoint{4.487502in}{2.358785in}}%
\pgfpathlineto{\pgfqpoint{4.518485in}{2.380775in}}%
\pgfpathlineto{\pgfqpoint{4.549467in}{2.390802in}}%
\pgfpathlineto{\pgfqpoint{4.580449in}{2.374015in}}%
\pgfpathlineto{\pgfqpoint{4.611431in}{2.386299in}}%
\pgfpathlineto{\pgfqpoint{4.642414in}{2.390182in}}%
\pgfpathlineto{\pgfqpoint{4.673396in}{2.409958in}}%
\pgfpathlineto{\pgfqpoint{4.704378in}{2.396663in}}%
\pgfpathlineto{\pgfqpoint{4.735360in}{2.394954in}}%
\pgfpathlineto{\pgfqpoint{4.766343in}{2.434095in}}%
\pgfpathlineto{\pgfqpoint{4.797325in}{2.408055in}}%
\pgfpathlineto{\pgfqpoint{4.828307in}{2.394472in}}%
\pgfpathlineto{\pgfqpoint{4.859289in}{2.407702in}}%
\pgfpathlineto{\pgfqpoint{4.921254in}{2.364693in}}%
\pgfpathlineto{\pgfqpoint{4.952236in}{2.420752in}}%
\pgfpathlineto{\pgfqpoint{4.952236in}{2.420752in}}%
\pgfusepath{stroke}%
\end{pgfscope}%
\begin{pgfscope}%
\pgfpathrectangle{\pgfqpoint{0.588387in}{0.521603in}}{\pgfqpoint{4.669024in}{2.010285in}}%
\pgfusepath{clip}%
\pgfsetrectcap%
\pgfsetroundjoin%
\pgfsetlinewidth{1.505625pt}%
\pgfsetstrokecolor{currentstroke4}%
\pgfsetdash{}{0pt}%
\pgfpathmoveto{\pgfqpoint{0.800616in}{1.005790in}}%
\pgfpathlineto{\pgfqpoint{0.831598in}{0.618596in}}%
\pgfpathlineto{\pgfqpoint{0.862580in}{0.695434in}}%
\pgfpathlineto{\pgfqpoint{0.893562in}{0.796368in}}%
\pgfpathlineto{\pgfqpoint{0.924545in}{0.898074in}}%
\pgfpathlineto{\pgfqpoint{0.955527in}{0.954664in}}%
\pgfpathlineto{\pgfqpoint{0.986509in}{0.962348in}}%
\pgfpathlineto{\pgfqpoint{1.017491in}{1.044394in}}%
\pgfpathlineto{\pgfqpoint{1.048474in}{1.093320in}}%
\pgfpathlineto{\pgfqpoint{1.079456in}{0.998984in}}%
\pgfpathlineto{\pgfqpoint{1.110438in}{1.029194in}}%
\pgfpathlineto{\pgfqpoint{1.141420in}{0.976001in}}%
\pgfpathlineto{\pgfqpoint{1.172402in}{1.045172in}}%
\pgfpathlineto{\pgfqpoint{1.203385in}{0.983592in}}%
\pgfpathlineto{\pgfqpoint{1.234367in}{1.073233in}}%
\pgfpathlineto{\pgfqpoint{1.265349in}{1.081050in}}%
\pgfpathlineto{\pgfqpoint{1.296331in}{1.105044in}}%
\pgfpathlineto{\pgfqpoint{1.327314in}{1.067900in}}%
\pgfpathlineto{\pgfqpoint{1.358296in}{1.060192in}}%
\pgfpathlineto{\pgfqpoint{1.389278in}{1.057755in}}%
\pgfpathlineto{\pgfqpoint{1.420260in}{1.149015in}}%
\pgfpathlineto{\pgfqpoint{1.451243in}{1.168613in}}%
\pgfpathlineto{\pgfqpoint{1.482225in}{1.096912in}}%
\pgfpathlineto{\pgfqpoint{1.513207in}{1.162667in}}%
\pgfpathlineto{\pgfqpoint{1.544189in}{1.121673in}}%
\pgfpathlineto{\pgfqpoint{1.575172in}{1.216879in}}%
\pgfpathlineto{\pgfqpoint{1.606154in}{1.203302in}}%
\pgfpathlineto{\pgfqpoint{1.637136in}{1.196818in}}%
\pgfpathlineto{\pgfqpoint{1.668118in}{1.188952in}}%
\pgfpathlineto{\pgfqpoint{1.699101in}{1.324339in}}%
\pgfpathlineto{\pgfqpoint{1.730083in}{1.269800in}}%
\pgfpathlineto{\pgfqpoint{1.761065in}{1.285518in}}%
\pgfpathlineto{\pgfqpoint{1.792047in}{1.328185in}}%
\pgfpathlineto{\pgfqpoint{1.823030in}{1.275107in}}%
\pgfpathlineto{\pgfqpoint{1.854012in}{1.343024in}}%
\pgfpathlineto{\pgfqpoint{1.884994in}{1.312350in}}%
\pgfpathlineto{\pgfqpoint{1.915976in}{1.322354in}}%
\pgfpathlineto{\pgfqpoint{1.946959in}{1.308894in}}%
\pgfpathlineto{\pgfqpoint{1.977941in}{1.371835in}}%
\pgfpathlineto{\pgfqpoint{2.008923in}{1.376511in}}%
\pgfpathlineto{\pgfqpoint{2.039905in}{1.419812in}}%
\pgfpathlineto{\pgfqpoint{2.070888in}{1.395636in}}%
\pgfpathlineto{\pgfqpoint{2.101870in}{1.430245in}}%
\pgfpathlineto{\pgfqpoint{2.132852in}{1.409183in}}%
\pgfpathlineto{\pgfqpoint{2.163834in}{1.477654in}}%
\pgfpathlineto{\pgfqpoint{2.194817in}{1.432128in}}%
\pgfpathlineto{\pgfqpoint{2.225799in}{1.484316in}}%
\pgfpathlineto{\pgfqpoint{2.256781in}{1.510185in}}%
\pgfpathlineto{\pgfqpoint{2.287763in}{1.493986in}}%
\pgfpathlineto{\pgfqpoint{2.318745in}{1.573766in}}%
\pgfpathlineto{\pgfqpoint{2.349728in}{1.500744in}}%
\pgfpathlineto{\pgfqpoint{2.380710in}{1.590172in}}%
\pgfpathlineto{\pgfqpoint{2.411692in}{1.588574in}}%
\pgfpathlineto{\pgfqpoint{2.442674in}{1.555527in}}%
\pgfpathlineto{\pgfqpoint{2.473657in}{1.622760in}}%
\pgfpathlineto{\pgfqpoint{2.504639in}{1.604986in}}%
\pgfpathlineto{\pgfqpoint{2.535621in}{1.609687in}}%
\pgfpathlineto{\pgfqpoint{2.566603in}{1.596665in}}%
\pgfpathlineto{\pgfqpoint{2.597586in}{1.670138in}}%
\pgfpathlineto{\pgfqpoint{2.628568in}{1.668859in}}%
\pgfpathlineto{\pgfqpoint{2.659550in}{1.643855in}}%
\pgfpathlineto{\pgfqpoint{2.690532in}{1.673362in}}%
\pgfpathlineto{\pgfqpoint{2.721515in}{1.658589in}}%
\pgfpathlineto{\pgfqpoint{2.752497in}{1.705630in}}%
\pgfpathlineto{\pgfqpoint{2.783479in}{1.734191in}}%
\pgfpathlineto{\pgfqpoint{2.814461in}{1.759244in}}%
\pgfpathlineto{\pgfqpoint{2.845444in}{1.697377in}}%
\pgfpathlineto{\pgfqpoint{2.876426in}{1.750388in}}%
\pgfpathlineto{\pgfqpoint{2.907408in}{1.779174in}}%
\pgfpathlineto{\pgfqpoint{2.938390in}{1.799147in}}%
\pgfpathlineto{\pgfqpoint{2.969373in}{1.805843in}}%
\pgfpathlineto{\pgfqpoint{3.000355in}{1.783682in}}%
\pgfpathlineto{\pgfqpoint{3.031337in}{1.800500in}}%
\pgfpathlineto{\pgfqpoint{3.062319in}{1.800119in}}%
\pgfpathlineto{\pgfqpoint{3.093302in}{1.912493in}}%
\pgfpathlineto{\pgfqpoint{3.124284in}{1.886240in}}%
\pgfpathlineto{\pgfqpoint{3.155266in}{1.879652in}}%
\pgfpathlineto{\pgfqpoint{3.186248in}{1.894156in}}%
\pgfpathlineto{\pgfqpoint{3.217231in}{1.923313in}}%
\pgfpathlineto{\pgfqpoint{3.248213in}{1.991048in}}%
\pgfpathlineto{\pgfqpoint{3.279195in}{1.912410in}}%
\pgfpathlineto{\pgfqpoint{3.310177in}{1.941681in}}%
\pgfpathlineto{\pgfqpoint{3.341159in}{2.018573in}}%
\pgfpathlineto{\pgfqpoint{3.372142in}{1.957215in}}%
\pgfpathlineto{\pgfqpoint{3.403124in}{1.992278in}}%
\pgfpathlineto{\pgfqpoint{3.434106in}{1.993416in}}%
\pgfpathlineto{\pgfqpoint{3.465088in}{2.049153in}}%
\pgfpathlineto{\pgfqpoint{3.496071in}{2.060018in}}%
\pgfpathlineto{\pgfqpoint{3.527053in}{2.033505in}}%
\pgfpathlineto{\pgfqpoint{3.558035in}{2.128052in}}%
\pgfpathlineto{\pgfqpoint{3.589017in}{2.074241in}}%
\pgfpathlineto{\pgfqpoint{3.620000in}{2.101114in}}%
\pgfpathlineto{\pgfqpoint{3.650982in}{2.129259in}}%
\pgfpathlineto{\pgfqpoint{3.681964in}{2.127229in}}%
\pgfpathlineto{\pgfqpoint{3.712946in}{2.107945in}}%
\pgfpathlineto{\pgfqpoint{3.743929in}{2.120329in}}%
\pgfpathlineto{\pgfqpoint{3.774911in}{2.163294in}}%
\pgfpathlineto{\pgfqpoint{3.805893in}{2.210195in}}%
\pgfpathlineto{\pgfqpoint{3.836875in}{2.187593in}}%
\pgfpathlineto{\pgfqpoint{3.867858in}{2.203769in}}%
\pgfpathlineto{\pgfqpoint{3.898840in}{2.245686in}}%
\pgfpathlineto{\pgfqpoint{3.929822in}{2.193235in}}%
\pgfpathlineto{\pgfqpoint{3.960804in}{2.173724in}}%
\pgfpathlineto{\pgfqpoint{3.991787in}{2.211325in}}%
\pgfpathlineto{\pgfqpoint{4.022769in}{2.244933in}}%
\pgfpathlineto{\pgfqpoint{4.053751in}{2.250791in}}%
\pgfpathlineto{\pgfqpoint{4.084733in}{2.236233in}}%
\pgfpathlineto{\pgfqpoint{4.115716in}{2.210161in}}%
\pgfpathlineto{\pgfqpoint{4.146698in}{2.251107in}}%
\pgfpathlineto{\pgfqpoint{4.177680in}{2.313877in}}%
\pgfpathlineto{\pgfqpoint{4.208662in}{2.292584in}}%
\pgfpathlineto{\pgfqpoint{4.239645in}{2.331722in}}%
\pgfpathlineto{\pgfqpoint{4.270627in}{2.368698in}}%
\pgfpathlineto{\pgfqpoint{4.301609in}{2.361160in}}%
\pgfpathlineto{\pgfqpoint{4.332591in}{2.349683in}}%
\pgfpathlineto{\pgfqpoint{4.363573in}{2.340017in}}%
\pgfpathlineto{\pgfqpoint{4.394556in}{2.392274in}}%
\pgfpathlineto{\pgfqpoint{4.425538in}{2.362648in}}%
\pgfpathlineto{\pgfqpoint{4.456520in}{2.351361in}}%
\pgfpathlineto{\pgfqpoint{4.487502in}{2.349001in}}%
\pgfpathlineto{\pgfqpoint{4.518485in}{2.422037in}}%
\pgfpathlineto{\pgfqpoint{4.549467in}{2.403483in}}%
\pgfpathlineto{\pgfqpoint{4.580449in}{2.401854in}}%
\pgfpathlineto{\pgfqpoint{4.642414in}{2.419855in}}%
\pgfpathlineto{\pgfqpoint{4.704378in}{2.342975in}}%
\pgfusepath{stroke}%
\end{pgfscope}%
\begin{pgfscope}%
\pgfpathrectangle{\pgfqpoint{0.588387in}{0.521603in}}{\pgfqpoint{4.669024in}{2.010285in}}%
\pgfusepath{clip}%
\pgfsetrectcap%
\pgfsetroundjoin%
\pgfsetlinewidth{1.505625pt}%
\pgfsetstrokecolor{currentstroke5}%
\pgfsetdash{}{0pt}%
\pgfpathmoveto{\pgfqpoint{0.800616in}{0.998890in}}%
\pgfpathlineto{\pgfqpoint{0.831598in}{0.616571in}}%
\pgfpathlineto{\pgfqpoint{0.862580in}{0.710808in}}%
\pgfpathlineto{\pgfqpoint{0.893562in}{0.811516in}}%
\pgfpathlineto{\pgfqpoint{0.924545in}{0.935744in}}%
\pgfpathlineto{\pgfqpoint{0.955527in}{0.959563in}}%
\pgfpathlineto{\pgfqpoint{0.986509in}{0.944193in}}%
\pgfpathlineto{\pgfqpoint{1.017491in}{1.025016in}}%
\pgfpathlineto{\pgfqpoint{1.048474in}{1.024726in}}%
\pgfpathlineto{\pgfqpoint{1.079456in}{0.977980in}}%
\pgfpathlineto{\pgfqpoint{1.110438in}{0.989550in}}%
\pgfpathlineto{\pgfqpoint{1.141420in}{0.880082in}}%
\pgfpathlineto{\pgfqpoint{1.172402in}{0.981910in}}%
\pgfpathlineto{\pgfqpoint{1.203385in}{0.956082in}}%
\pgfpathlineto{\pgfqpoint{1.234367in}{1.027250in}}%
\pgfpathlineto{\pgfqpoint{1.265349in}{1.023190in}}%
\pgfpathlineto{\pgfqpoint{1.296331in}{1.072331in}}%
\pgfpathlineto{\pgfqpoint{1.327314in}{1.016227in}}%
\pgfpathlineto{\pgfqpoint{1.358296in}{1.017771in}}%
\pgfpathlineto{\pgfqpoint{1.389278in}{1.018215in}}%
\pgfpathlineto{\pgfqpoint{1.420260in}{1.058925in}}%
\pgfpathlineto{\pgfqpoint{1.451243in}{1.117386in}}%
\pgfpathlineto{\pgfqpoint{1.482225in}{1.032505in}}%
\pgfpathlineto{\pgfqpoint{1.513207in}{1.087020in}}%
\pgfpathlineto{\pgfqpoint{1.544189in}{1.088697in}}%
\pgfpathlineto{\pgfqpoint{1.575172in}{1.136993in}}%
\pgfpathlineto{\pgfqpoint{1.606154in}{1.114602in}}%
\pgfpathlineto{\pgfqpoint{1.637136in}{1.126095in}}%
\pgfpathlineto{\pgfqpoint{1.668118in}{1.147325in}}%
\pgfpathlineto{\pgfqpoint{1.699101in}{1.227297in}}%
\pgfpathlineto{\pgfqpoint{1.730083in}{1.202857in}}%
\pgfpathlineto{\pgfqpoint{1.761065in}{1.236936in}}%
\pgfpathlineto{\pgfqpoint{1.792047in}{1.272486in}}%
\pgfpathlineto{\pgfqpoint{1.823030in}{1.198071in}}%
\pgfpathlineto{\pgfqpoint{1.854012in}{1.283791in}}%
\pgfpathlineto{\pgfqpoint{1.884994in}{1.267310in}}%
\pgfpathlineto{\pgfqpoint{1.915976in}{1.280881in}}%
\pgfpathlineto{\pgfqpoint{1.946959in}{1.216213in}}%
\pgfpathlineto{\pgfqpoint{1.977941in}{1.313817in}}%
\pgfpathlineto{\pgfqpoint{2.008923in}{1.318386in}}%
\pgfpathlineto{\pgfqpoint{2.039905in}{1.369209in}}%
\pgfpathlineto{\pgfqpoint{2.070888in}{1.331319in}}%
\pgfpathlineto{\pgfqpoint{2.101870in}{1.355959in}}%
\pgfpathlineto{\pgfqpoint{2.132852in}{1.340320in}}%
\pgfpathlineto{\pgfqpoint{2.163834in}{1.399411in}}%
\pgfpathlineto{\pgfqpoint{2.194817in}{1.413594in}}%
\pgfpathlineto{\pgfqpoint{2.225799in}{1.403841in}}%
\pgfpathlineto{\pgfqpoint{2.256781in}{1.460769in}}%
\pgfpathlineto{\pgfqpoint{2.287763in}{1.456032in}}%
\pgfpathlineto{\pgfqpoint{2.318745in}{1.466611in}}%
\pgfpathlineto{\pgfqpoint{2.349728in}{1.446904in}}%
\pgfpathlineto{\pgfqpoint{2.380710in}{1.551837in}}%
\pgfpathlineto{\pgfqpoint{2.411692in}{1.536442in}}%
\pgfpathlineto{\pgfqpoint{2.442674in}{1.509191in}}%
\pgfpathlineto{\pgfqpoint{2.473657in}{1.603056in}}%
\pgfpathlineto{\pgfqpoint{2.504639in}{1.576744in}}%
\pgfpathlineto{\pgfqpoint{2.535621in}{1.593372in}}%
\pgfpathlineto{\pgfqpoint{2.566603in}{1.571567in}}%
\pgfpathlineto{\pgfqpoint{2.597586in}{1.601928in}}%
\pgfpathlineto{\pgfqpoint{2.628568in}{1.648495in}}%
\pgfpathlineto{\pgfqpoint{2.659550in}{1.609402in}}%
\pgfpathlineto{\pgfqpoint{2.690532in}{1.651893in}}%
\pgfpathlineto{\pgfqpoint{2.721515in}{1.618835in}}%
\pgfpathlineto{\pgfqpoint{2.752497in}{1.699210in}}%
\pgfpathlineto{\pgfqpoint{2.783479in}{1.754568in}}%
\pgfpathlineto{\pgfqpoint{2.814461in}{1.728787in}}%
\pgfpathlineto{\pgfqpoint{2.845444in}{1.655294in}}%
\pgfpathlineto{\pgfqpoint{2.876426in}{1.693116in}}%
\pgfpathlineto{\pgfqpoint{2.907408in}{1.751705in}}%
\pgfpathlineto{\pgfqpoint{2.938390in}{1.783621in}}%
\pgfpathlineto{\pgfqpoint{2.969373in}{1.752960in}}%
\pgfpathlineto{\pgfqpoint{3.000355in}{1.748885in}}%
\pgfpathlineto{\pgfqpoint{3.031337in}{1.714640in}}%
\pgfpathlineto{\pgfqpoint{3.062319in}{1.783719in}}%
\pgfpathlineto{\pgfqpoint{3.093302in}{1.898946in}}%
\pgfpathlineto{\pgfqpoint{3.124284in}{1.898477in}}%
\pgfpathlineto{\pgfqpoint{3.155266in}{1.898162in}}%
\pgfpathlineto{\pgfqpoint{3.186248in}{1.875056in}}%
\pgfpathlineto{\pgfqpoint{3.217231in}{1.921723in}}%
\pgfpathlineto{\pgfqpoint{3.248213in}{2.013579in}}%
\pgfpathlineto{\pgfqpoint{3.279195in}{1.928690in}}%
\pgfpathlineto{\pgfqpoint{3.310177in}{1.938455in}}%
\pgfpathlineto{\pgfqpoint{3.341159in}{2.041313in}}%
\pgfpathlineto{\pgfqpoint{3.372142in}{1.945819in}}%
\pgfpathlineto{\pgfqpoint{3.403124in}{1.987272in}}%
\pgfpathlineto{\pgfqpoint{3.434106in}{2.000886in}}%
\pgfpathlineto{\pgfqpoint{3.465088in}{2.054493in}}%
\pgfpathlineto{\pgfqpoint{3.496071in}{2.112473in}}%
\pgfpathlineto{\pgfqpoint{3.527053in}{2.029991in}}%
\pgfpathlineto{\pgfqpoint{3.558035in}{2.144582in}}%
\pgfpathlineto{\pgfqpoint{3.589017in}{2.031125in}}%
\pgfpathlineto{\pgfqpoint{3.620000in}{2.123629in}}%
\pgfpathlineto{\pgfqpoint{3.650982in}{2.168778in}}%
\pgfpathlineto{\pgfqpoint{3.681964in}{2.088643in}}%
\pgfpathlineto{\pgfqpoint{3.712946in}{2.147281in}}%
\pgfpathlineto{\pgfqpoint{3.774911in}{2.177823in}}%
\pgfpathlineto{\pgfqpoint{3.805893in}{2.359075in}}%
\pgfpathlineto{\pgfqpoint{3.836875in}{2.210978in}}%
\pgfpathlineto{\pgfqpoint{3.867858in}{2.234027in}}%
\pgfpathlineto{\pgfqpoint{3.898840in}{2.259117in}}%
\pgfpathlineto{\pgfqpoint{4.053751in}{2.266566in}}%
\pgfpathlineto{\pgfqpoint{4.177680in}{2.334780in}}%
\pgfpathlineto{\pgfqpoint{4.394556in}{2.405366in}}%
\pgfpathlineto{\pgfqpoint{4.518485in}{2.409888in}}%
\pgfusepath{stroke}%
\end{pgfscope}%
\begin{pgfscope}%
\pgfpathrectangle{\pgfqpoint{0.588387in}{0.521603in}}{\pgfqpoint{4.669024in}{2.010285in}}%
\pgfusepath{clip}%
\pgfsetrectcap%
\pgfsetroundjoin%
\pgfsetlinewidth{1.505625pt}%
\pgfsetstrokecolor{currentstroke6}%
\pgfsetdash{}{0pt}%
\pgfpathmoveto{\pgfqpoint{0.800616in}{1.001144in}}%
\pgfpathlineto{\pgfqpoint{0.831598in}{0.615078in}}%
\pgfpathlineto{\pgfqpoint{0.862580in}{0.683389in}}%
\pgfpathlineto{\pgfqpoint{0.893562in}{0.782188in}}%
\pgfpathlineto{\pgfqpoint{0.924545in}{0.895010in}}%
\pgfpathlineto{\pgfqpoint{0.955527in}{0.944540in}}%
\pgfpathlineto{\pgfqpoint{0.986509in}{0.946691in}}%
\pgfpathlineto{\pgfqpoint{1.017491in}{1.016260in}}%
\pgfpathlineto{\pgfqpoint{1.048474in}{1.032133in}}%
\pgfpathlineto{\pgfqpoint{1.079456in}{0.959323in}}%
\pgfpathlineto{\pgfqpoint{1.110438in}{0.963747in}}%
\pgfpathlineto{\pgfqpoint{1.141420in}{0.881193in}}%
\pgfpathlineto{\pgfqpoint{1.172402in}{0.975399in}}%
\pgfpathlineto{\pgfqpoint{1.203385in}{0.937430in}}%
\pgfpathlineto{\pgfqpoint{1.234367in}{1.013958in}}%
\pgfpathlineto{\pgfqpoint{1.265349in}{1.021905in}}%
\pgfpathlineto{\pgfqpoint{1.296331in}{1.074638in}}%
\pgfpathlineto{\pgfqpoint{1.327314in}{1.027643in}}%
\pgfpathlineto{\pgfqpoint{1.358296in}{1.017132in}}%
\pgfpathlineto{\pgfqpoint{1.389278in}{1.006528in}}%
\pgfpathlineto{\pgfqpoint{1.420260in}{1.049441in}}%
\pgfpathlineto{\pgfqpoint{1.451243in}{1.116492in}}%
\pgfpathlineto{\pgfqpoint{1.482225in}{1.027703in}}%
\pgfpathlineto{\pgfqpoint{1.513207in}{1.098460in}}%
\pgfpathlineto{\pgfqpoint{1.544189in}{1.082261in}}%
\pgfpathlineto{\pgfqpoint{1.575172in}{1.135898in}}%
\pgfpathlineto{\pgfqpoint{1.606154in}{1.113073in}}%
\pgfpathlineto{\pgfqpoint{1.637136in}{1.124222in}}%
\pgfpathlineto{\pgfqpoint{1.668118in}{1.138595in}}%
\pgfpathlineto{\pgfqpoint{1.699101in}{1.247857in}}%
\pgfpathlineto{\pgfqpoint{1.730083in}{1.201635in}}%
\pgfpathlineto{\pgfqpoint{1.761065in}{1.233816in}}%
\pgfpathlineto{\pgfqpoint{1.792047in}{1.281715in}}%
\pgfpathlineto{\pgfqpoint{1.823030in}{1.183476in}}%
\pgfpathlineto{\pgfqpoint{1.854012in}{1.290661in}}%
\pgfpathlineto{\pgfqpoint{1.884994in}{1.278100in}}%
\pgfpathlineto{\pgfqpoint{1.915976in}{1.267305in}}%
\pgfpathlineto{\pgfqpoint{1.946959in}{1.213423in}}%
\pgfpathlineto{\pgfqpoint{1.977941in}{1.294652in}}%
\pgfpathlineto{\pgfqpoint{2.008923in}{1.323185in}}%
\pgfpathlineto{\pgfqpoint{2.039905in}{1.372877in}}%
\pgfpathlineto{\pgfqpoint{2.070888in}{1.343589in}}%
\pgfpathlineto{\pgfqpoint{2.101870in}{1.345707in}}%
\pgfpathlineto{\pgfqpoint{2.132852in}{1.335617in}}%
\pgfpathlineto{\pgfqpoint{2.163834in}{1.383001in}}%
\pgfpathlineto{\pgfqpoint{2.194817in}{1.400320in}}%
\pgfpathlineto{\pgfqpoint{2.225799in}{1.394698in}}%
\pgfpathlineto{\pgfqpoint{2.256781in}{1.462234in}}%
\pgfpathlineto{\pgfqpoint{2.287763in}{1.450758in}}%
\pgfpathlineto{\pgfqpoint{2.318745in}{1.449541in}}%
\pgfpathlineto{\pgfqpoint{2.349728in}{1.440065in}}%
\pgfpathlineto{\pgfqpoint{2.380710in}{1.565046in}}%
\pgfpathlineto{\pgfqpoint{2.411692in}{1.527321in}}%
\pgfpathlineto{\pgfqpoint{2.442674in}{1.494996in}}%
\pgfpathlineto{\pgfqpoint{2.473657in}{1.601658in}}%
\pgfpathlineto{\pgfqpoint{2.504639in}{1.576578in}}%
\pgfpathlineto{\pgfqpoint{2.535621in}{1.579498in}}%
\pgfpathlineto{\pgfqpoint{2.566603in}{1.571125in}}%
\pgfpathlineto{\pgfqpoint{2.597586in}{1.584742in}}%
\pgfpathlineto{\pgfqpoint{2.628568in}{1.641162in}}%
\pgfpathlineto{\pgfqpoint{2.659550in}{1.596256in}}%
\pgfpathlineto{\pgfqpoint{2.690532in}{1.659770in}}%
\pgfpathlineto{\pgfqpoint{2.721515in}{1.606101in}}%
\pgfpathlineto{\pgfqpoint{2.752497in}{1.681371in}}%
\pgfpathlineto{\pgfqpoint{2.783479in}{1.717801in}}%
\pgfpathlineto{\pgfqpoint{2.814461in}{1.705555in}}%
\pgfpathlineto{\pgfqpoint{2.845444in}{1.645519in}}%
\pgfpathlineto{\pgfqpoint{2.876426in}{1.697338in}}%
\pgfpathlineto{\pgfqpoint{2.907408in}{1.737406in}}%
\pgfpathlineto{\pgfqpoint{2.938390in}{1.780234in}}%
\pgfpathlineto{\pgfqpoint{2.969373in}{1.744555in}}%
\pgfpathlineto{\pgfqpoint{3.000355in}{1.748811in}}%
\pgfpathlineto{\pgfqpoint{3.031337in}{1.707204in}}%
\pgfpathlineto{\pgfqpoint{3.062319in}{1.787346in}}%
\pgfpathlineto{\pgfqpoint{3.093302in}{1.880749in}}%
\pgfpathlineto{\pgfqpoint{3.124284in}{1.885952in}}%
\pgfpathlineto{\pgfqpoint{3.155266in}{1.879178in}}%
\pgfpathlineto{\pgfqpoint{3.186248in}{1.863809in}}%
\pgfpathlineto{\pgfqpoint{3.217231in}{1.922424in}}%
\pgfpathlineto{\pgfqpoint{3.248213in}{2.005463in}}%
\pgfpathlineto{\pgfqpoint{3.279195in}{1.932912in}}%
\pgfpathlineto{\pgfqpoint{3.310177in}{1.923263in}}%
\pgfpathlineto{\pgfqpoint{3.341159in}{2.017487in}}%
\pgfpathlineto{\pgfqpoint{3.372142in}{1.944922in}}%
\pgfpathlineto{\pgfqpoint{3.403124in}{1.989276in}}%
\pgfpathlineto{\pgfqpoint{3.434106in}{1.970661in}}%
\pgfpathlineto{\pgfqpoint{3.465088in}{2.039441in}}%
\pgfpathlineto{\pgfqpoint{3.496071in}{2.066468in}}%
\pgfpathlineto{\pgfqpoint{3.527053in}{2.046336in}}%
\pgfpathlineto{\pgfqpoint{3.558035in}{2.121054in}}%
\pgfpathlineto{\pgfqpoint{3.589017in}{2.029336in}}%
\pgfpathlineto{\pgfqpoint{3.620000in}{2.116537in}}%
\pgfpathlineto{\pgfqpoint{3.650982in}{2.198197in}}%
\pgfpathlineto{\pgfqpoint{3.681964in}{2.089341in}}%
\pgfpathlineto{\pgfqpoint{3.712946in}{2.119523in}}%
\pgfpathlineto{\pgfqpoint{3.774911in}{2.187827in}}%
\pgfpathlineto{\pgfqpoint{3.805893in}{2.331532in}}%
\pgfpathlineto{\pgfqpoint{3.836875in}{2.193130in}}%
\pgfpathlineto{\pgfqpoint{3.867858in}{2.227409in}}%
\pgfpathlineto{\pgfqpoint{3.898840in}{2.269959in}}%
\pgfpathlineto{\pgfqpoint{4.053751in}{2.295505in}}%
\pgfpathlineto{\pgfqpoint{4.177680in}{2.320947in}}%
\pgfpathlineto{\pgfqpoint{4.394556in}{2.362469in}}%
\pgfpathlineto{\pgfqpoint{4.518485in}{2.440512in}}%
\pgfusepath{stroke}%
\end{pgfscope}%
\begin{pgfscope}%
\pgfpathrectangle{\pgfqpoint{0.588387in}{0.521603in}}{\pgfqpoint{4.669024in}{2.010285in}}%
\pgfusepath{clip}%
\pgfsetrectcap%
\pgfsetroundjoin%
\pgfsetlinewidth{1.505625pt}%
\pgfsetstrokecolor{currentstroke7}%
\pgfsetdash{}{0pt}%
\pgfpathmoveto{\pgfqpoint{0.800616in}{0.990713in}}%
\pgfpathlineto{\pgfqpoint{0.831598in}{0.613777in}}%
\pgfpathlineto{\pgfqpoint{0.862580in}{0.703430in}}%
\pgfpathlineto{\pgfqpoint{0.893562in}{0.797352in}}%
\pgfpathlineto{\pgfqpoint{0.924545in}{0.913090in}}%
\pgfpathlineto{\pgfqpoint{0.955527in}{0.950544in}}%
\pgfpathlineto{\pgfqpoint{0.986509in}{0.959270in}}%
\pgfpathlineto{\pgfqpoint{1.017491in}{1.063315in}}%
\pgfpathlineto{\pgfqpoint{1.048474in}{1.136126in}}%
\pgfpathlineto{\pgfqpoint{1.079456in}{1.048495in}}%
\pgfpathlineto{\pgfqpoint{1.110438in}{1.087448in}}%
\pgfpathlineto{\pgfqpoint{1.141420in}{1.061110in}}%
\pgfpathlineto{\pgfqpoint{1.172402in}{1.083055in}}%
\pgfpathlineto{\pgfqpoint{1.203385in}{1.045211in}}%
\pgfpathlineto{\pgfqpoint{1.234367in}{1.120849in}}%
\pgfpathlineto{\pgfqpoint{1.265349in}{1.143015in}}%
\pgfpathlineto{\pgfqpoint{1.296331in}{1.149681in}}%
\pgfpathlineto{\pgfqpoint{1.327314in}{1.118492in}}%
\pgfpathlineto{\pgfqpoint{1.358296in}{1.099833in}}%
\pgfpathlineto{\pgfqpoint{1.389278in}{1.131454in}}%
\pgfpathlineto{\pgfqpoint{1.420260in}{1.275669in}}%
\pgfpathlineto{\pgfqpoint{1.451243in}{1.238484in}}%
\pgfpathlineto{\pgfqpoint{1.482225in}{1.145275in}}%
\pgfpathlineto{\pgfqpoint{1.513207in}{1.211045in}}%
\pgfpathlineto{\pgfqpoint{1.544189in}{1.192888in}}%
\pgfpathlineto{\pgfqpoint{1.575172in}{1.318228in}}%
\pgfpathlineto{\pgfqpoint{1.606154in}{1.283613in}}%
\pgfpathlineto{\pgfqpoint{1.637136in}{1.285466in}}%
\pgfpathlineto{\pgfqpoint{1.668118in}{1.262872in}}%
\pgfpathlineto{\pgfqpoint{1.699101in}{1.381557in}}%
\pgfpathlineto{\pgfqpoint{1.730083in}{1.335564in}}%
\pgfpathlineto{\pgfqpoint{1.761065in}{1.362004in}}%
\pgfpathlineto{\pgfqpoint{1.792047in}{1.383488in}}%
\pgfpathlineto{\pgfqpoint{1.823030in}{1.371620in}}%
\pgfpathlineto{\pgfqpoint{1.854012in}{1.420443in}}%
\pgfpathlineto{\pgfqpoint{1.884994in}{1.366014in}}%
\pgfpathlineto{\pgfqpoint{1.915976in}{1.371167in}}%
\pgfpathlineto{\pgfqpoint{1.946959in}{1.394952in}}%
\pgfpathlineto{\pgfqpoint{1.977941in}{1.423327in}}%
\pgfpathlineto{\pgfqpoint{2.008923in}{1.473343in}}%
\pgfpathlineto{\pgfqpoint{2.039905in}{1.499874in}}%
\pgfpathlineto{\pgfqpoint{2.070888in}{1.457305in}}%
\pgfpathlineto{\pgfqpoint{2.101870in}{1.514059in}}%
\pgfpathlineto{\pgfqpoint{2.132852in}{1.493101in}}%
\pgfpathlineto{\pgfqpoint{2.163834in}{1.578191in}}%
\pgfpathlineto{\pgfqpoint{2.194817in}{1.491846in}}%
\pgfpathlineto{\pgfqpoint{2.225799in}{1.552228in}}%
\pgfpathlineto{\pgfqpoint{2.256781in}{1.562068in}}%
\pgfpathlineto{\pgfqpoint{2.287763in}{1.538949in}}%
\pgfpathlineto{\pgfqpoint{2.318745in}{1.657504in}}%
\pgfpathlineto{\pgfqpoint{2.349728in}{1.586534in}}%
\pgfpathlineto{\pgfqpoint{2.380710in}{1.625885in}}%
\pgfpathlineto{\pgfqpoint{2.411692in}{1.633020in}}%
\pgfpathlineto{\pgfqpoint{2.442674in}{1.615568in}}%
\pgfpathlineto{\pgfqpoint{2.473657in}{1.646410in}}%
\pgfpathlineto{\pgfqpoint{2.504639in}{1.653763in}}%
\pgfpathlineto{\pgfqpoint{2.535621in}{1.639374in}}%
\pgfpathlineto{\pgfqpoint{2.566603in}{1.663457in}}%
\pgfpathlineto{\pgfqpoint{2.597586in}{1.745915in}}%
\pgfpathlineto{\pgfqpoint{2.628568in}{1.701155in}}%
\pgfpathlineto{\pgfqpoint{2.659550in}{1.708036in}}%
\pgfpathlineto{\pgfqpoint{2.690532in}{1.744540in}}%
\pgfpathlineto{\pgfqpoint{2.721515in}{1.665656in}}%
\pgfpathlineto{\pgfqpoint{2.752497in}{1.729535in}}%
\pgfpathlineto{\pgfqpoint{2.783479in}{1.831178in}}%
\pgfpathlineto{\pgfqpoint{2.814461in}{1.799783in}}%
\pgfpathlineto{\pgfqpoint{2.845444in}{1.733523in}}%
\pgfpathlineto{\pgfqpoint{2.876426in}{1.773486in}}%
\pgfpathlineto{\pgfqpoint{2.907408in}{1.853323in}}%
\pgfpathlineto{\pgfqpoint{2.938390in}{1.803296in}}%
\pgfpathlineto{\pgfqpoint{2.969373in}{1.852778in}}%
\pgfpathlineto{\pgfqpoint{3.000355in}{1.834251in}}%
\pgfpathlineto{\pgfqpoint{3.031337in}{1.873727in}}%
\pgfpathlineto{\pgfqpoint{3.062319in}{1.865046in}}%
\pgfpathlineto{\pgfqpoint{3.093302in}{1.895213in}}%
\pgfpathlineto{\pgfqpoint{3.155266in}{1.894232in}}%
\pgfpathlineto{\pgfqpoint{3.186248in}{1.898904in}}%
\pgfpathlineto{\pgfqpoint{3.217231in}{1.927081in}}%
\pgfpathlineto{\pgfqpoint{3.248213in}{1.997704in}}%
\pgfpathlineto{\pgfqpoint{3.279195in}{1.935664in}}%
\pgfpathlineto{\pgfqpoint{3.310177in}{1.974762in}}%
\pgfpathlineto{\pgfqpoint{3.341159in}{2.030320in}}%
\pgfpathlineto{\pgfqpoint{3.372142in}{1.958915in}}%
\pgfpathlineto{\pgfqpoint{3.403124in}{1.986253in}}%
\pgfpathlineto{\pgfqpoint{3.434106in}{2.003072in}}%
\pgfpathlineto{\pgfqpoint{3.465088in}{2.043295in}}%
\pgfpathlineto{\pgfqpoint{3.496071in}{2.032903in}}%
\pgfpathlineto{\pgfqpoint{3.527053in}{2.043305in}}%
\pgfpathlineto{\pgfqpoint{3.558035in}{2.056437in}}%
\pgfpathlineto{\pgfqpoint{3.589017in}{2.078702in}}%
\pgfpathlineto{\pgfqpoint{3.620000in}{2.064780in}}%
\pgfpathlineto{\pgfqpoint{3.650982in}{2.094674in}}%
\pgfpathlineto{\pgfqpoint{3.681964in}{2.129033in}}%
\pgfpathlineto{\pgfqpoint{3.712946in}{2.104539in}}%
\pgfpathlineto{\pgfqpoint{3.743929in}{2.142837in}}%
\pgfpathlineto{\pgfqpoint{3.774911in}{2.128237in}}%
\pgfpathlineto{\pgfqpoint{3.805893in}{2.170888in}}%
\pgfpathlineto{\pgfqpoint{3.836875in}{2.155363in}}%
\pgfpathlineto{\pgfqpoint{3.867858in}{2.164807in}}%
\pgfpathlineto{\pgfqpoint{3.898840in}{2.225992in}}%
\pgfpathlineto{\pgfqpoint{3.929822in}{2.188299in}}%
\pgfpathlineto{\pgfqpoint{3.960804in}{2.188068in}}%
\pgfpathlineto{\pgfqpoint{3.991787in}{2.203032in}}%
\pgfpathlineto{\pgfqpoint{4.022769in}{2.224216in}}%
\pgfpathlineto{\pgfqpoint{4.053751in}{2.240102in}}%
\pgfpathlineto{\pgfqpoint{4.084733in}{2.214741in}}%
\pgfpathlineto{\pgfqpoint{4.115716in}{2.267519in}}%
\pgfpathlineto{\pgfqpoint{4.146698in}{2.254813in}}%
\pgfpathlineto{\pgfqpoint{4.177680in}{2.279596in}}%
\pgfpathlineto{\pgfqpoint{4.208662in}{2.272439in}}%
\pgfpathlineto{\pgfqpoint{4.239645in}{2.293938in}}%
\pgfpathlineto{\pgfqpoint{4.270627in}{2.309381in}}%
\pgfpathlineto{\pgfqpoint{4.301609in}{2.258224in}}%
\pgfpathlineto{\pgfqpoint{4.332591in}{2.345203in}}%
\pgfpathlineto{\pgfqpoint{4.363573in}{2.333838in}}%
\pgfpathlineto{\pgfqpoint{4.394556in}{2.347607in}}%
\pgfpathlineto{\pgfqpoint{4.425538in}{2.358505in}}%
\pgfpathlineto{\pgfqpoint{4.456520in}{2.331622in}}%
\pgfpathlineto{\pgfqpoint{4.487502in}{2.324609in}}%
\pgfpathlineto{\pgfqpoint{4.518485in}{2.367176in}}%
\pgfpathlineto{\pgfqpoint{4.549467in}{2.381160in}}%
\pgfpathlineto{\pgfqpoint{4.580449in}{2.367688in}}%
\pgfpathlineto{\pgfqpoint{4.611431in}{2.369039in}}%
\pgfpathlineto{\pgfqpoint{4.642414in}{2.391786in}}%
\pgfpathlineto{\pgfqpoint{4.673396in}{2.391334in}}%
\pgfpathlineto{\pgfqpoint{4.704378in}{2.395990in}}%
\pgfpathlineto{\pgfqpoint{4.735360in}{2.413580in}}%
\pgfpathlineto{\pgfqpoint{4.766343in}{2.400878in}}%
\pgfpathlineto{\pgfqpoint{4.797325in}{2.413757in}}%
\pgfpathlineto{\pgfqpoint{4.828307in}{2.407302in}}%
\pgfpathlineto{\pgfqpoint{4.859289in}{2.387328in}}%
\pgfpathlineto{\pgfqpoint{4.890272in}{2.397144in}}%
\pgfpathlineto{\pgfqpoint{4.921254in}{2.369039in}}%
\pgfpathlineto{\pgfqpoint{4.952236in}{2.419993in}}%
\pgfpathlineto{\pgfqpoint{4.983218in}{2.437447in}}%
\pgfpathlineto{\pgfqpoint{5.014201in}{2.435036in}}%
\pgfpathlineto{\pgfqpoint{5.045183in}{2.430384in}}%
\pgfpathlineto{\pgfqpoint{5.045183in}{2.430384in}}%
\pgfusepath{stroke}%
\end{pgfscope}%
\begin{pgfscope}%
\pgfpathrectangle{\pgfqpoint{0.588387in}{0.521603in}}{\pgfqpoint{4.669024in}{2.010285in}}%
\pgfusepath{clip}%
\pgfsetrectcap%
\pgfsetroundjoin%
\pgfsetlinewidth{1.505625pt}%
\definecolor{currentstroke}{rgb}{0.498039,0.498039,0.498039}%
\pgfsetstrokecolor{currentstroke}%
\pgfsetdash{}{0pt}%
\pgfpathmoveto{\pgfqpoint{0.800616in}{1.028687in}}%
\pgfpathlineto{\pgfqpoint{0.831598in}{0.612980in}}%
\pgfpathlineto{\pgfqpoint{0.862580in}{0.681696in}}%
\pgfpathlineto{\pgfqpoint{0.893562in}{0.778601in}}%
\pgfpathlineto{\pgfqpoint{0.924545in}{0.896793in}}%
\pgfpathlineto{\pgfqpoint{0.955527in}{0.944618in}}%
\pgfpathlineto{\pgfqpoint{0.986509in}{0.953604in}}%
\pgfpathlineto{\pgfqpoint{1.017491in}{1.040022in}}%
\pgfpathlineto{\pgfqpoint{1.048474in}{1.093796in}}%
\pgfpathlineto{\pgfqpoint{1.079456in}{1.000333in}}%
\pgfpathlineto{\pgfqpoint{1.110438in}{1.025971in}}%
\pgfpathlineto{\pgfqpoint{1.141420in}{0.970816in}}%
\pgfpathlineto{\pgfqpoint{1.172402in}{1.021619in}}%
\pgfpathlineto{\pgfqpoint{1.203385in}{0.977769in}}%
\pgfpathlineto{\pgfqpoint{1.234367in}{1.074187in}}%
\pgfpathlineto{\pgfqpoint{1.265349in}{1.078773in}}%
\pgfpathlineto{\pgfqpoint{1.296331in}{1.101797in}}%
\pgfpathlineto{\pgfqpoint{1.327314in}{1.059651in}}%
\pgfpathlineto{\pgfqpoint{1.358296in}{1.057031in}}%
\pgfpathlineto{\pgfqpoint{1.389278in}{1.058027in}}%
\pgfpathlineto{\pgfqpoint{1.420260in}{1.138955in}}%
\pgfpathlineto{\pgfqpoint{1.451243in}{1.169979in}}%
\pgfpathlineto{\pgfqpoint{1.482225in}{1.079795in}}%
\pgfpathlineto{\pgfqpoint{1.513207in}{1.144035in}}%
\pgfpathlineto{\pgfqpoint{1.544189in}{1.113722in}}%
\pgfpathlineto{\pgfqpoint{1.575172in}{1.198482in}}%
\pgfpathlineto{\pgfqpoint{1.606154in}{1.193783in}}%
\pgfpathlineto{\pgfqpoint{1.637136in}{1.179400in}}%
\pgfpathlineto{\pgfqpoint{1.668118in}{1.184820in}}%
\pgfpathlineto{\pgfqpoint{1.699101in}{1.301231in}}%
\pgfpathlineto{\pgfqpoint{1.730083in}{1.258039in}}%
\pgfpathlineto{\pgfqpoint{1.761065in}{1.276716in}}%
\pgfpathlineto{\pgfqpoint{1.792047in}{1.314627in}}%
\pgfpathlineto{\pgfqpoint{1.823030in}{1.264487in}}%
\pgfpathlineto{\pgfqpoint{1.854012in}{1.351472in}}%
\pgfpathlineto{\pgfqpoint{1.884994in}{1.286117in}}%
\pgfpathlineto{\pgfqpoint{1.915976in}{1.307831in}}%
\pgfpathlineto{\pgfqpoint{1.946959in}{1.302996in}}%
\pgfpathlineto{\pgfqpoint{1.977941in}{1.357774in}}%
\pgfpathlineto{\pgfqpoint{2.008923in}{1.367538in}}%
\pgfpathlineto{\pgfqpoint{2.039905in}{1.402391in}}%
\pgfpathlineto{\pgfqpoint{2.070888in}{1.379683in}}%
\pgfpathlineto{\pgfqpoint{2.101870in}{1.415884in}}%
\pgfpathlineto{\pgfqpoint{2.132852in}{1.383784in}}%
\pgfpathlineto{\pgfqpoint{2.163834in}{1.461848in}}%
\pgfpathlineto{\pgfqpoint{2.194817in}{1.428131in}}%
\pgfpathlineto{\pgfqpoint{2.225799in}{1.464192in}}%
\pgfpathlineto{\pgfqpoint{2.256781in}{1.495554in}}%
\pgfpathlineto{\pgfqpoint{2.287763in}{1.461346in}}%
\pgfpathlineto{\pgfqpoint{2.318745in}{1.551095in}}%
\pgfpathlineto{\pgfqpoint{2.349728in}{1.497595in}}%
\pgfpathlineto{\pgfqpoint{2.380710in}{1.574966in}}%
\pgfpathlineto{\pgfqpoint{2.411692in}{1.558882in}}%
\pgfpathlineto{\pgfqpoint{2.442674in}{1.524802in}}%
\pgfpathlineto{\pgfqpoint{2.473657in}{1.579790in}}%
\pgfpathlineto{\pgfqpoint{2.504639in}{1.585205in}}%
\pgfpathlineto{\pgfqpoint{2.535621in}{1.586819in}}%
\pgfpathlineto{\pgfqpoint{2.566603in}{1.560234in}}%
\pgfpathlineto{\pgfqpoint{2.597586in}{1.644853in}}%
\pgfpathlineto{\pgfqpoint{2.628568in}{1.639282in}}%
\pgfpathlineto{\pgfqpoint{2.659550in}{1.607609in}}%
\pgfpathlineto{\pgfqpoint{2.690532in}{1.643841in}}%
\pgfpathlineto{\pgfqpoint{2.721515in}{1.620158in}}%
\pgfpathlineto{\pgfqpoint{2.752497in}{1.669537in}}%
\pgfpathlineto{\pgfqpoint{2.783479in}{1.706205in}}%
\pgfpathlineto{\pgfqpoint{2.814461in}{1.731253in}}%
\pgfpathlineto{\pgfqpoint{2.845444in}{1.671613in}}%
\pgfpathlineto{\pgfqpoint{2.876426in}{1.721963in}}%
\pgfpathlineto{\pgfqpoint{2.907408in}{1.756394in}}%
\pgfpathlineto{\pgfqpoint{2.938390in}{1.767073in}}%
\pgfpathlineto{\pgfqpoint{2.969373in}{1.782419in}}%
\pgfpathlineto{\pgfqpoint{3.000355in}{1.789619in}}%
\pgfpathlineto{\pgfqpoint{3.031337in}{1.778894in}}%
\pgfpathlineto{\pgfqpoint{3.062319in}{1.773396in}}%
\pgfpathlineto{\pgfqpoint{3.093302in}{1.867411in}}%
\pgfpathlineto{\pgfqpoint{3.124284in}{1.836685in}}%
\pgfpathlineto{\pgfqpoint{3.155266in}{1.855891in}}%
\pgfpathlineto{\pgfqpoint{3.186248in}{1.866349in}}%
\pgfpathlineto{\pgfqpoint{3.217231in}{1.899500in}}%
\pgfpathlineto{\pgfqpoint{3.248213in}{1.960457in}}%
\pgfpathlineto{\pgfqpoint{3.279195in}{1.881033in}}%
\pgfpathlineto{\pgfqpoint{3.310177in}{1.910875in}}%
\pgfpathlineto{\pgfqpoint{3.341159in}{1.967183in}}%
\pgfpathlineto{\pgfqpoint{3.372142in}{1.932918in}}%
\pgfpathlineto{\pgfqpoint{3.403124in}{1.947687in}}%
\pgfpathlineto{\pgfqpoint{3.434106in}{1.992095in}}%
\pgfpathlineto{\pgfqpoint{3.465088in}{2.009320in}}%
\pgfpathlineto{\pgfqpoint{3.496071in}{2.037629in}}%
\pgfpathlineto{\pgfqpoint{3.527053in}{1.979388in}}%
\pgfpathlineto{\pgfqpoint{3.558035in}{2.060856in}}%
\pgfpathlineto{\pgfqpoint{3.589017in}{2.035020in}}%
\pgfpathlineto{\pgfqpoint{3.620000in}{2.067796in}}%
\pgfpathlineto{\pgfqpoint{3.650982in}{2.105999in}}%
\pgfpathlineto{\pgfqpoint{3.681964in}{2.087746in}}%
\pgfpathlineto{\pgfqpoint{3.712946in}{2.085177in}}%
\pgfpathlineto{\pgfqpoint{3.743929in}{2.104188in}}%
\pgfpathlineto{\pgfqpoint{3.774911in}{2.129282in}}%
\pgfpathlineto{\pgfqpoint{3.805893in}{2.184666in}}%
\pgfpathlineto{\pgfqpoint{3.836875in}{2.160340in}}%
\pgfpathlineto{\pgfqpoint{3.867858in}{2.165039in}}%
\pgfpathlineto{\pgfqpoint{3.898840in}{2.215720in}}%
\pgfpathlineto{\pgfqpoint{3.929822in}{2.157629in}}%
\pgfpathlineto{\pgfqpoint{3.960804in}{2.144212in}}%
\pgfpathlineto{\pgfqpoint{3.991787in}{2.151927in}}%
\pgfpathlineto{\pgfqpoint{4.022769in}{2.209905in}}%
\pgfpathlineto{\pgfqpoint{4.053751in}{2.229119in}}%
\pgfpathlineto{\pgfqpoint{4.084733in}{2.205459in}}%
\pgfpathlineto{\pgfqpoint{4.115716in}{2.272936in}}%
\pgfpathlineto{\pgfqpoint{4.146698in}{2.233788in}}%
\pgfpathlineto{\pgfqpoint{4.177680in}{2.269886in}}%
\pgfpathlineto{\pgfqpoint{4.208662in}{2.269982in}}%
\pgfpathlineto{\pgfqpoint{4.239645in}{2.303380in}}%
\pgfpathlineto{\pgfqpoint{4.270627in}{2.318212in}}%
\pgfpathlineto{\pgfqpoint{4.301609in}{2.255977in}}%
\pgfpathlineto{\pgfqpoint{4.332591in}{2.315938in}}%
\pgfpathlineto{\pgfqpoint{4.363573in}{2.315407in}}%
\pgfpathlineto{\pgfqpoint{4.394556in}{2.350523in}}%
\pgfpathlineto{\pgfqpoint{4.425538in}{2.325593in}}%
\pgfpathlineto{\pgfqpoint{4.456520in}{2.321749in}}%
\pgfpathlineto{\pgfqpoint{4.487502in}{2.366070in}}%
\pgfpathlineto{\pgfqpoint{4.518485in}{2.359501in}}%
\pgfpathlineto{\pgfqpoint{4.549467in}{2.386483in}}%
\pgfpathlineto{\pgfqpoint{4.580449in}{2.364004in}}%
\pgfpathlineto{\pgfqpoint{4.611431in}{2.390474in}}%
\pgfpathlineto{\pgfqpoint{4.642414in}{2.367798in}}%
\pgfpathlineto{\pgfqpoint{4.673396in}{2.373652in}}%
\pgfpathlineto{\pgfqpoint{4.704378in}{2.382907in}}%
\pgfpathlineto{\pgfqpoint{4.735360in}{2.430655in}}%
\pgfpathlineto{\pgfqpoint{4.797325in}{2.407349in}}%
\pgfpathlineto{\pgfqpoint{4.859289in}{2.388793in}}%
\pgfpathlineto{\pgfqpoint{4.952236in}{2.436109in}}%
\pgfpathlineto{\pgfqpoint{4.983218in}{2.437714in}}%
\pgfpathlineto{\pgfqpoint{4.983218in}{2.437714in}}%
\pgfusepath{stroke}%
\end{pgfscope}%
\begin{pgfscope}%
\pgfsetrectcap%
\pgfsetmiterjoin%
\pgfsetlinewidth{0.803000pt}%
\definecolor{currentstroke}{rgb}{0.000000,0.000000,0.000000}%
\pgfsetstrokecolor{currentstroke}%
\pgfsetdash{}{0pt}%
\pgfpathmoveto{\pgfqpoint{0.588387in}{0.521603in}}%
\pgfpathlineto{\pgfqpoint{0.588387in}{2.531888in}}%
\pgfusepath{stroke}%
\end{pgfscope}%
\begin{pgfscope}%
\pgfsetrectcap%
\pgfsetmiterjoin%
\pgfsetlinewidth{0.803000pt}%
\definecolor{currentstroke}{rgb}{0.000000,0.000000,0.000000}%
\pgfsetstrokecolor{currentstroke}%
\pgfsetdash{}{0pt}%
\pgfpathmoveto{\pgfqpoint{5.257411in}{0.521603in}}%
\pgfpathlineto{\pgfqpoint{5.257411in}{2.531888in}}%
\pgfusepath{stroke}%
\end{pgfscope}%
\begin{pgfscope}%
\pgfsetrectcap%
\pgfsetmiterjoin%
\pgfsetlinewidth{0.803000pt}%
\definecolor{currentstroke}{rgb}{0.000000,0.000000,0.000000}%
\pgfsetstrokecolor{currentstroke}%
\pgfsetdash{}{0pt}%
\pgfpathmoveto{\pgfqpoint{0.588387in}{0.521603in}}%
\pgfpathlineto{\pgfqpoint{5.257411in}{0.521603in}}%
\pgfusepath{stroke}%
\end{pgfscope}%
\begin{pgfscope}%
\pgfsetrectcap%
\pgfsetmiterjoin%
\pgfsetlinewidth{0.803000pt}%
\definecolor{currentstroke}{rgb}{0.000000,0.000000,0.000000}%
\pgfsetstrokecolor{currentstroke}%
\pgfsetdash{}{0pt}%
\pgfpathmoveto{\pgfqpoint{0.588387in}{2.531888in}}%
\pgfpathlineto{\pgfqpoint{5.257411in}{2.531888in}}%
\pgfusepath{stroke}%
\end{pgfscope}%
\begin{pgfscope}%
\definecolor{textcolor}{rgb}{0.000000,0.000000,0.000000}%
\pgfsetstrokecolor{textcolor}%
\pgfsetfillcolor{textcolor}%
\pgftext[x=2.922899in,y=2.615222in,,base]{\color{textcolor}{\rmfamily\fontsize{12.000000}{14.400000}\selectfont\catcode`\^=\active\def^{\ifmmode\sp\else\^{}\fi}\catcode`\%=\active\def%{\%}Mean}}%
\end{pgfscope}%
\begin{pgfscope}%
\pgfsetbuttcap%
\pgfsetmiterjoin%
\definecolor{currentfill}{rgb}{1.000000,1.000000,1.000000}%
\pgfsetfillcolor{currentfill}%
\pgfsetfillopacity{0.800000}%
\pgfsetlinewidth{1.003750pt}%
\definecolor{currentstroke}{rgb}{0.800000,0.800000,0.800000}%
\pgfsetstrokecolor{currentstroke}%
\pgfsetstrokeopacity{0.800000}%
\pgfsetdash{}{0pt}%
\pgfpathmoveto{\pgfqpoint{5.344911in}{0.943243in}}%
\pgfpathlineto{\pgfqpoint{8.259376in}{0.943243in}}%
\pgfpathquadraticcurveto{\pgfqpoint{8.284376in}{0.943243in}}{\pgfqpoint{8.284376in}{0.968243in}}%
\pgfpathlineto{\pgfqpoint{8.284376in}{2.444388in}}%
\pgfpathquadraticcurveto{\pgfqpoint{8.284376in}{2.469388in}}{\pgfqpoint{8.259376in}{2.469388in}}%
\pgfpathlineto{\pgfqpoint{5.344911in}{2.469388in}}%
\pgfpathquadraticcurveto{\pgfqpoint{5.319911in}{2.469388in}}{\pgfqpoint{5.319911in}{2.444388in}}%
\pgfpathlineto{\pgfqpoint{5.319911in}{0.968243in}}%
\pgfpathquadraticcurveto{\pgfqpoint{5.319911in}{0.943243in}}{\pgfqpoint{5.344911in}{0.943243in}}%
\pgfpathlineto{\pgfqpoint{5.344911in}{0.943243in}}%
\pgfpathclose%
\pgfusepath{stroke,fill}%
\end{pgfscope}%
\begin{pgfscope}%
\pgfsetrectcap%
\pgfsetroundjoin%
\pgfsetlinewidth{1.505625pt}%
\pgfsetstrokecolor{currentstroke1}%
\pgfsetdash{}{0pt}%
\pgfpathmoveto{\pgfqpoint{5.369911in}{2.368168in}}%
\pgfpathlineto{\pgfqpoint{5.494911in}{2.368168in}}%
\pgfpathlineto{\pgfqpoint{5.619911in}{2.368168in}}%
\pgfusepath{stroke}%
\end{pgfscope}%
\begin{pgfscope}%
\definecolor{textcolor}{rgb}{0.000000,0.000000,0.000000}%
\pgfsetstrokecolor{textcolor}%
\pgfsetfillcolor{textcolor}%
\pgftext[x=5.719911in,y=2.324418in,left,base]{\color{textcolor}{\rmfamily\fontsize{9.000000}{10.800000}\selectfont\catcode`\^=\active\def^{\ifmmode\sp\else\^{}\fi}\catcode`\%=\active\def%{\%}\CyclesMatchChunks{} \& \MergeLinear{}}}%
\end{pgfscope}%
\begin{pgfscope}%
\pgfsetrectcap%
\pgfsetroundjoin%
\pgfsetlinewidth{1.505625pt}%
\pgfsetstrokecolor{currentstroke2}%
\pgfsetdash{}{0pt}%
\pgfpathmoveto{\pgfqpoint{5.369911in}{2.181217in}}%
\pgfpathlineto{\pgfqpoint{5.494911in}{2.181217in}}%
\pgfpathlineto{\pgfqpoint{5.619911in}{2.181217in}}%
\pgfusepath{stroke}%
\end{pgfscope}%
\begin{pgfscope}%
\definecolor{textcolor}{rgb}{0.000000,0.000000,0.000000}%
\pgfsetstrokecolor{textcolor}%
\pgfsetfillcolor{textcolor}%
\pgftext[x=5.719911in,y=2.137467in,left,base]{\color{textcolor}{\rmfamily\fontsize{9.000000}{10.800000}\selectfont\catcode`\^=\active\def^{\ifmmode\sp\else\^{}\fi}\catcode`\%=\active\def%{\%}\CyclesMatchChunks{} \& \SharedVertices{}}}%
\end{pgfscope}%
\begin{pgfscope}%
\pgfsetrectcap%
\pgfsetroundjoin%
\pgfsetlinewidth{1.505625pt}%
\pgfsetstrokecolor{currentstroke3}%
\pgfsetdash{}{0pt}%
\pgfpathmoveto{\pgfqpoint{5.369911in}{1.994267in}}%
\pgfpathlineto{\pgfqpoint{5.494911in}{1.994267in}}%
\pgfpathlineto{\pgfqpoint{5.619911in}{1.994267in}}%
\pgfusepath{stroke}%
\end{pgfscope}%
\begin{pgfscope}%
\definecolor{textcolor}{rgb}{0.000000,0.000000,0.000000}%
\pgfsetstrokecolor{textcolor}%
\pgfsetfillcolor{textcolor}%
\pgftext[x=5.719911in,y=1.950517in,left,base]{\color{textcolor}{\rmfamily\fontsize{9.000000}{10.800000}\selectfont\catcode`\^=\active\def^{\ifmmode\sp\else\^{}\fi}\catcode`\%=\active\def%{\%}\Neighbors{} \& \MergeLinear{}}}%
\end{pgfscope}%
\begin{pgfscope}%
\pgfsetrectcap%
\pgfsetroundjoin%
\pgfsetlinewidth{1.505625pt}%
\pgfsetstrokecolor{currentstroke4}%
\pgfsetdash{}{0pt}%
\pgfpathmoveto{\pgfqpoint{5.369911in}{1.810795in}}%
\pgfpathlineto{\pgfqpoint{5.494911in}{1.810795in}}%
\pgfpathlineto{\pgfqpoint{5.619911in}{1.810795in}}%
\pgfusepath{stroke}%
\end{pgfscope}%
\begin{pgfscope}%
\definecolor{textcolor}{rgb}{0.000000,0.000000,0.000000}%
\pgfsetstrokecolor{textcolor}%
\pgfsetfillcolor{textcolor}%
\pgftext[x=5.719911in,y=1.767045in,left,base]{\color{textcolor}{\rmfamily\fontsize{9.000000}{10.800000}\selectfont\catcode`\^=\active\def^{\ifmmode\sp\else\^{}\fi}\catcode`\%=\active\def%{\%}\Neighbors{} \& \SharedVertices{}}}%
\end{pgfscope}%
\begin{pgfscope}%
\pgfsetrectcap%
\pgfsetroundjoin%
\pgfsetlinewidth{1.505625pt}%
\pgfsetstrokecolor{currentstroke5}%
\pgfsetdash{}{0pt}%
\pgfpathmoveto{\pgfqpoint{5.369911in}{1.623845in}}%
\pgfpathlineto{\pgfqpoint{5.494911in}{1.623845in}}%
\pgfpathlineto{\pgfqpoint{5.619911in}{1.623845in}}%
\pgfusepath{stroke}%
\end{pgfscope}%
\begin{pgfscope}%
\definecolor{textcolor}{rgb}{0.000000,0.000000,0.000000}%
\pgfsetstrokecolor{textcolor}%
\pgfsetfillcolor{textcolor}%
\pgftext[x=5.719911in,y=1.580095in,left,base]{\color{textcolor}{\rmfamily\fontsize{9.000000}{10.800000}\selectfont\catcode`\^=\active\def^{\ifmmode\sp\else\^{}\fi}\catcode`\%=\active\def%{\%}\NeighborsDegree{} \& \MergeLinear{}}}%
\end{pgfscope}%
\begin{pgfscope}%
\pgfsetrectcap%
\pgfsetroundjoin%
\pgfsetlinewidth{1.505625pt}%
\pgfsetstrokecolor{currentstroke6}%
\pgfsetdash{}{0pt}%
\pgfpathmoveto{\pgfqpoint{5.369911in}{1.436894in}}%
\pgfpathlineto{\pgfqpoint{5.494911in}{1.436894in}}%
\pgfpathlineto{\pgfqpoint{5.619911in}{1.436894in}}%
\pgfusepath{stroke}%
\end{pgfscope}%
\begin{pgfscope}%
\definecolor{textcolor}{rgb}{0.000000,0.000000,0.000000}%
\pgfsetstrokecolor{textcolor}%
\pgfsetfillcolor{textcolor}%
\pgftext[x=5.719911in,y=1.393144in,left,base]{\color{textcolor}{\rmfamily\fontsize{9.000000}{10.800000}\selectfont\catcode`\^=\active\def^{\ifmmode\sp\else\^{}\fi}\catcode`\%=\active\def%{\%}\NeighborsDegree{} \& \SharedVertices{}}}%
\end{pgfscope}%
\begin{pgfscope}%
\pgfsetrectcap%
\pgfsetroundjoin%
\pgfsetlinewidth{1.505625pt}%
\pgfsetstrokecolor{currentstroke7}%
\pgfsetdash{}{0pt}%
\pgfpathmoveto{\pgfqpoint{5.369911in}{1.249944in}}%
\pgfpathlineto{\pgfqpoint{5.494911in}{1.249944in}}%
\pgfpathlineto{\pgfqpoint{5.619911in}{1.249944in}}%
\pgfusepath{stroke}%
\end{pgfscope}%
\begin{pgfscope}%
\definecolor{textcolor}{rgb}{0.000000,0.000000,0.000000}%
\pgfsetstrokecolor{textcolor}%
\pgfsetfillcolor{textcolor}%
\pgftext[x=5.719911in,y=1.206194in,left,base]{\color{textcolor}{\rmfamily\fontsize{9.000000}{10.800000}\selectfont\catcode`\^=\active\def^{\ifmmode\sp\else\^{}\fi}\catcode`\%=\active\def%{\%}\None{} \& \MergeLinear{}}}%
\end{pgfscope}%
\begin{pgfscope}%
\pgfsetrectcap%
\pgfsetroundjoin%
\pgfsetlinewidth{1.505625pt}%
\definecolor{currentstroke}{rgb}{0.498039,0.498039,0.498039}%
\pgfsetstrokecolor{currentstroke}%
\pgfsetdash{}{0pt}%
\pgfpathmoveto{\pgfqpoint{5.369911in}{1.066472in}}%
\pgfpathlineto{\pgfqpoint{5.494911in}{1.066472in}}%
\pgfpathlineto{\pgfqpoint{5.619911in}{1.066472in}}%
\pgfusepath{stroke}%
\end{pgfscope}%
\begin{pgfscope}%
\definecolor{textcolor}{rgb}{0.000000,0.000000,0.000000}%
\pgfsetstrokecolor{textcolor}%
\pgfsetfillcolor{textcolor}%
\pgftext[x=5.719911in,y=1.022722in,left,base]{\color{textcolor}{\rmfamily\fontsize{9.000000}{10.800000}\selectfont\catcode`\^=\active\def^{\ifmmode\sp\else\^{}\fi}\catcode`\%=\active\def%{\%}\None{} \& \SharedVertices{}}}%
\end{pgfscope}%
\end{pgfpicture}%
\makeatother%
\endgroup%
}
	\caption[Mean runtime for graphs with no NAC-coloring (some).]{
		Mean running time (ms) to find all NAC-colorings for graphs with no NAC-coloring.}%
	\label{fig:graph_no_nac_coloring_first_runtime}
\end{figure}
\begin{figure}[p]
	\centering
	\scalebox{0.5}{%% Creator: Matplotlib, PGF backend
%%
%% To include the figure in your LaTeX document, write
%%   \input{<filename>.pgf}
%%
%% Make sure the required packages are loaded in your preamble
%%   \usepackage{pgf}
%%
%% Also ensure that all the required font packages are loaded; for instance,
%% the lmodern package is sometimes necessary when using math font.
%%   \usepackage{lmodern}
%%
%% Figures using additional raster images can only be included by \input if
%% they are in the same directory as the main LaTeX file. For loading figures
%% from other directories you can use the `import` package
%%   \usepackage{import}
%%
%% and then include the figures with
%%   \import{<path to file>}{<filename>.pgf}
%%
%% Matplotlib used the following preamble
%%   \def\mathdefault#1{#1}
%%   \everymath=\expandafter{\the\everymath\displaystyle}
%%   \IfFileExists{scrextend.sty}{
%%     \usepackage[fontsize=10.000000pt]{scrextend}
%%   }{
%%     \renewcommand{\normalsize}{\fontsize{10.000000}{12.000000}\selectfont}
%%     \normalsize
%%   }
%%   
%%   \ifdefined\pdftexversion\else  % non-pdftex case.
%%     \usepackage{fontspec}
%%     \setmainfont{DejaVuSans.ttf}[Path=\detokenize{/home/petr/Projects/PyRigi/.venv/lib/python3.12/site-packages/matplotlib/mpl-data/fonts/ttf/}]
%%     \setsansfont{DejaVuSans.ttf}[Path=\detokenize{/home/petr/Projects/PyRigi/.venv/lib/python3.12/site-packages/matplotlib/mpl-data/fonts/ttf/}]
%%     \setmonofont{DejaVuSansMono.ttf}[Path=\detokenize{/home/petr/Projects/PyRigi/.venv/lib/python3.12/site-packages/matplotlib/mpl-data/fonts/ttf/}]
%%   \fi
%%   \makeatletter\@ifpackageloaded{under\Score{}}{}{\usepackage[strings]{under\Score{}}}\makeatother
%%
\begingroup%
\makeatletter%
\begin{pgfpicture}%
\pgfpathrectangle{\pgfpointorigin}{\pgfqpoint{8.384376in}{2.841849in}}%
\pgfusepath{use as bounding box, clip}%
\begin{pgfscope}%
\pgfsetbuttcap%
\pgfsetmiterjoin%
\definecolor{currentfill}{rgb}{1.000000,1.000000,1.000000}%
\pgfsetfillcolor{currentfill}%
\pgfsetlinewidth{0.000000pt}%
\definecolor{currentstroke}{rgb}{1.000000,1.000000,1.000000}%
\pgfsetstrokecolor{currentstroke}%
\pgfsetdash{}{0pt}%
\pgfpathmoveto{\pgfqpoint{0.000000in}{0.000000in}}%
\pgfpathlineto{\pgfqpoint{8.384376in}{0.000000in}}%
\pgfpathlineto{\pgfqpoint{8.384376in}{2.841849in}}%
\pgfpathlineto{\pgfqpoint{0.000000in}{2.841849in}}%
\pgfpathlineto{\pgfqpoint{0.000000in}{0.000000in}}%
\pgfpathclose%
\pgfusepath{fill}%
\end{pgfscope}%
\begin{pgfscope}%
\pgfsetbuttcap%
\pgfsetmiterjoin%
\definecolor{currentfill}{rgb}{1.000000,1.000000,1.000000}%
\pgfsetfillcolor{currentfill}%
\pgfsetlinewidth{0.000000pt}%
\definecolor{currentstroke}{rgb}{0.000000,0.000000,0.000000}%
\pgfsetstrokecolor{currentstroke}%
\pgfsetstrokeopacity{0.000000}%
\pgfsetdash{}{0pt}%
\pgfpathmoveto{\pgfqpoint{0.588387in}{0.521603in}}%
\pgfpathlineto{\pgfqpoint{4.248423in}{0.521603in}}%
\pgfpathlineto{\pgfqpoint{4.248423in}{2.741849in}}%
\pgfpathlineto{\pgfqpoint{0.588387in}{2.741849in}}%
\pgfpathlineto{\pgfqpoint{0.588387in}{0.521603in}}%
\pgfpathclose%
\pgfusepath{fill}%
\end{pgfscope}%
\begin{pgfscope}%
\pgfsetbuttcap%
\pgfsetroundjoin%
\definecolor{currentfill}{rgb}{0.000000,0.000000,0.000000}%
\pgfsetfillcolor{currentfill}%
\pgfsetlinewidth{0.803000pt}%
\definecolor{currentstroke}{rgb}{0.000000,0.000000,0.000000}%
\pgfsetstrokecolor{currentstroke}%
\pgfsetdash{}{0pt}%
\pgfsys@defobject{currentmarker}{\pgfqpoint{0.000000in}{-0.048611in}}{\pgfqpoint{0.000000in}{0.000000in}}{%
\pgfpathmoveto{\pgfqpoint{0.000000in}{0.000000in}}%
\pgfpathlineto{\pgfqpoint{0.000000in}{-0.048611in}}%
\pgfusepath{stroke,fill}%
}%
\begin{pgfscope}%
\pgfsys@transformshift{0.918775in}{0.521603in}%
\pgfsys@useobject{currentmarker}{}%
\end{pgfscope}%
\end{pgfscope}%
\begin{pgfscope}%
\definecolor{textcolor}{rgb}{0.000000,0.000000,0.000000}%
\pgfsetstrokecolor{textcolor}%
\pgfsetfillcolor{textcolor}%
\pgftext[x=0.918775in,y=0.424381in,,top]{\color{textcolor}{\rmfamily\fontsize{10.000000}{12.000000}\selectfont\catcode`\^=\active\def^{\ifmmode\sp\else\^{}\fi}\catcode`\%=\active\def%{\%}$\mathdefault{20}$}}%
\end{pgfscope}%
\begin{pgfscope}%
\pgfsetbuttcap%
\pgfsetroundjoin%
\definecolor{currentfill}{rgb}{0.000000,0.000000,0.000000}%
\pgfsetfillcolor{currentfill}%
\pgfsetlinewidth{0.803000pt}%
\definecolor{currentstroke}{rgb}{0.000000,0.000000,0.000000}%
\pgfsetstrokecolor{currentstroke}%
\pgfsetdash{}{0pt}%
\pgfsys@defobject{currentmarker}{\pgfqpoint{0.000000in}{-0.048611in}}{\pgfqpoint{0.000000in}{0.000000in}}{%
\pgfpathmoveto{\pgfqpoint{0.000000in}{0.000000in}}%
\pgfpathlineto{\pgfqpoint{0.000000in}{-0.048611in}}%
\pgfusepath{stroke,fill}%
}%
\begin{pgfscope}%
\pgfsys@transformshift{1.387409in}{0.521603in}%
\pgfsys@useobject{currentmarker}{}%
\end{pgfscope}%
\end{pgfscope}%
\begin{pgfscope}%
\definecolor{textcolor}{rgb}{0.000000,0.000000,0.000000}%
\pgfsetstrokecolor{textcolor}%
\pgfsetfillcolor{textcolor}%
\pgftext[x=1.387409in,y=0.424381in,,top]{\color{textcolor}{\rmfamily\fontsize{10.000000}{12.000000}\selectfont\catcode`\^=\active\def^{\ifmmode\sp\else\^{}\fi}\catcode`\%=\active\def%{\%}$\mathdefault{40}$}}%
\end{pgfscope}%
\begin{pgfscope}%
\pgfsetbuttcap%
\pgfsetroundjoin%
\definecolor{currentfill}{rgb}{0.000000,0.000000,0.000000}%
\pgfsetfillcolor{currentfill}%
\pgfsetlinewidth{0.803000pt}%
\definecolor{currentstroke}{rgb}{0.000000,0.000000,0.000000}%
\pgfsetstrokecolor{currentstroke}%
\pgfsetdash{}{0pt}%
\pgfsys@defobject{currentmarker}{\pgfqpoint{0.000000in}{-0.048611in}}{\pgfqpoint{0.000000in}{0.000000in}}{%
\pgfpathmoveto{\pgfqpoint{0.000000in}{0.000000in}}%
\pgfpathlineto{\pgfqpoint{0.000000in}{-0.048611in}}%
\pgfusepath{stroke,fill}%
}%
\begin{pgfscope}%
\pgfsys@transformshift{1.856044in}{0.521603in}%
\pgfsys@useobject{currentmarker}{}%
\end{pgfscope}%
\end{pgfscope}%
\begin{pgfscope}%
\definecolor{textcolor}{rgb}{0.000000,0.000000,0.000000}%
\pgfsetstrokecolor{textcolor}%
\pgfsetfillcolor{textcolor}%
\pgftext[x=1.856044in,y=0.424381in,,top]{\color{textcolor}{\rmfamily\fontsize{10.000000}{12.000000}\selectfont\catcode`\^=\active\def^{\ifmmode\sp\else\^{}\fi}\catcode`\%=\active\def%{\%}$\mathdefault{60}$}}%
\end{pgfscope}%
\begin{pgfscope}%
\pgfsetbuttcap%
\pgfsetroundjoin%
\definecolor{currentfill}{rgb}{0.000000,0.000000,0.000000}%
\pgfsetfillcolor{currentfill}%
\pgfsetlinewidth{0.803000pt}%
\definecolor{currentstroke}{rgb}{0.000000,0.000000,0.000000}%
\pgfsetstrokecolor{currentstroke}%
\pgfsetdash{}{0pt}%
\pgfsys@defobject{currentmarker}{\pgfqpoint{0.000000in}{-0.048611in}}{\pgfqpoint{0.000000in}{0.000000in}}{%
\pgfpathmoveto{\pgfqpoint{0.000000in}{0.000000in}}%
\pgfpathlineto{\pgfqpoint{0.000000in}{-0.048611in}}%
\pgfusepath{stroke,fill}%
}%
\begin{pgfscope}%
\pgfsys@transformshift{2.324678in}{0.521603in}%
\pgfsys@useobject{currentmarker}{}%
\end{pgfscope}%
\end{pgfscope}%
\begin{pgfscope}%
\definecolor{textcolor}{rgb}{0.000000,0.000000,0.000000}%
\pgfsetstrokecolor{textcolor}%
\pgfsetfillcolor{textcolor}%
\pgftext[x=2.324678in,y=0.424381in,,top]{\color{textcolor}{\rmfamily\fontsize{10.000000}{12.000000}\selectfont\catcode`\^=\active\def^{\ifmmode\sp\else\^{}\fi}\catcode`\%=\active\def%{\%}$\mathdefault{80}$}}%
\end{pgfscope}%
\begin{pgfscope}%
\pgfsetbuttcap%
\pgfsetroundjoin%
\definecolor{currentfill}{rgb}{0.000000,0.000000,0.000000}%
\pgfsetfillcolor{currentfill}%
\pgfsetlinewidth{0.803000pt}%
\definecolor{currentstroke}{rgb}{0.000000,0.000000,0.000000}%
\pgfsetstrokecolor{currentstroke}%
\pgfsetdash{}{0pt}%
\pgfsys@defobject{currentmarker}{\pgfqpoint{0.000000in}{-0.048611in}}{\pgfqpoint{0.000000in}{0.000000in}}{%
\pgfpathmoveto{\pgfqpoint{0.000000in}{0.000000in}}%
\pgfpathlineto{\pgfqpoint{0.000000in}{-0.048611in}}%
\pgfusepath{stroke,fill}%
}%
\begin{pgfscope}%
\pgfsys@transformshift{2.793313in}{0.521603in}%
\pgfsys@useobject{currentmarker}{}%
\end{pgfscope}%
\end{pgfscope}%
\begin{pgfscope}%
\definecolor{textcolor}{rgb}{0.000000,0.000000,0.000000}%
\pgfsetstrokecolor{textcolor}%
\pgfsetfillcolor{textcolor}%
\pgftext[x=2.793313in,y=0.424381in,,top]{\color{textcolor}{\rmfamily\fontsize{10.000000}{12.000000}\selectfont\catcode`\^=\active\def^{\ifmmode\sp\else\^{}\fi}\catcode`\%=\active\def%{\%}$\mathdefault{100}$}}%
\end{pgfscope}%
\begin{pgfscope}%
\pgfsetbuttcap%
\pgfsetroundjoin%
\definecolor{currentfill}{rgb}{0.000000,0.000000,0.000000}%
\pgfsetfillcolor{currentfill}%
\pgfsetlinewidth{0.803000pt}%
\definecolor{currentstroke}{rgb}{0.000000,0.000000,0.000000}%
\pgfsetstrokecolor{currentstroke}%
\pgfsetdash{}{0pt}%
\pgfsys@defobject{currentmarker}{\pgfqpoint{0.000000in}{-0.048611in}}{\pgfqpoint{0.000000in}{0.000000in}}{%
\pgfpathmoveto{\pgfqpoint{0.000000in}{0.000000in}}%
\pgfpathlineto{\pgfqpoint{0.000000in}{-0.048611in}}%
\pgfusepath{stroke,fill}%
}%
\begin{pgfscope}%
\pgfsys@transformshift{3.261947in}{0.521603in}%
\pgfsys@useobject{currentmarker}{}%
\end{pgfscope}%
\end{pgfscope}%
\begin{pgfscope}%
\definecolor{textcolor}{rgb}{0.000000,0.000000,0.000000}%
\pgfsetstrokecolor{textcolor}%
\pgfsetfillcolor{textcolor}%
\pgftext[x=3.261947in,y=0.424381in,,top]{\color{textcolor}{\rmfamily\fontsize{10.000000}{12.000000}\selectfont\catcode`\^=\active\def^{\ifmmode\sp\else\^{}\fi}\catcode`\%=\active\def%{\%}$\mathdefault{120}$}}%
\end{pgfscope}%
\begin{pgfscope}%
\pgfsetbuttcap%
\pgfsetroundjoin%
\definecolor{currentfill}{rgb}{0.000000,0.000000,0.000000}%
\pgfsetfillcolor{currentfill}%
\pgfsetlinewidth{0.803000pt}%
\definecolor{currentstroke}{rgb}{0.000000,0.000000,0.000000}%
\pgfsetstrokecolor{currentstroke}%
\pgfsetdash{}{0pt}%
\pgfsys@defobject{currentmarker}{\pgfqpoint{0.000000in}{-0.048611in}}{\pgfqpoint{0.000000in}{0.000000in}}{%
\pgfpathmoveto{\pgfqpoint{0.000000in}{0.000000in}}%
\pgfpathlineto{\pgfqpoint{0.000000in}{-0.048611in}}%
\pgfusepath{stroke,fill}%
}%
\begin{pgfscope}%
\pgfsys@transformshift{3.730582in}{0.521603in}%
\pgfsys@useobject{currentmarker}{}%
\end{pgfscope}%
\end{pgfscope}%
\begin{pgfscope}%
\definecolor{textcolor}{rgb}{0.000000,0.000000,0.000000}%
\pgfsetstrokecolor{textcolor}%
\pgfsetfillcolor{textcolor}%
\pgftext[x=3.730582in,y=0.424381in,,top]{\color{textcolor}{\rmfamily\fontsize{10.000000}{12.000000}\selectfont\catcode`\^=\active\def^{\ifmmode\sp\else\^{}\fi}\catcode`\%=\active\def%{\%}$\mathdefault{140}$}}%
\end{pgfscope}%
\begin{pgfscope}%
\pgfsetbuttcap%
\pgfsetroundjoin%
\definecolor{currentfill}{rgb}{0.000000,0.000000,0.000000}%
\pgfsetfillcolor{currentfill}%
\pgfsetlinewidth{0.803000pt}%
\definecolor{currentstroke}{rgb}{0.000000,0.000000,0.000000}%
\pgfsetstrokecolor{currentstroke}%
\pgfsetdash{}{0pt}%
\pgfsys@defobject{currentmarker}{\pgfqpoint{0.000000in}{-0.048611in}}{\pgfqpoint{0.000000in}{0.000000in}}{%
\pgfpathmoveto{\pgfqpoint{0.000000in}{0.000000in}}%
\pgfpathlineto{\pgfqpoint{0.000000in}{-0.048611in}}%
\pgfusepath{stroke,fill}%
}%
\begin{pgfscope}%
\pgfsys@transformshift{4.199216in}{0.521603in}%
\pgfsys@useobject{currentmarker}{}%
\end{pgfscope}%
\end{pgfscope}%
\begin{pgfscope}%
\definecolor{textcolor}{rgb}{0.000000,0.000000,0.000000}%
\pgfsetstrokecolor{textcolor}%
\pgfsetfillcolor{textcolor}%
\pgftext[x=4.199216in,y=0.424381in,,top]{\color{textcolor}{\rmfamily\fontsize{10.000000}{12.000000}\selectfont\catcode`\^=\active\def^{\ifmmode\sp\else\^{}\fi}\catcode`\%=\active\def%{\%}$\mathdefault{160}$}}%
\end{pgfscope}%
\begin{pgfscope}%
\definecolor{textcolor}{rgb}{0.000000,0.000000,0.000000}%
\pgfsetstrokecolor{textcolor}%
\pgfsetfillcolor{textcolor}%
\pgftext[x=2.418405in,y=0.234413in,,top]{\color{textcolor}{\rmfamily\fontsize{10.000000}{12.000000}\selectfont\catcode`\^=\active\def^{\ifmmode\sp\else\^{}\fi}\catcode`\%=\active\def%{\%}$\triangle$-connected components}}%
\end{pgfscope}%
\begin{pgfscope}%
\pgfsetbuttcap%
\pgfsetroundjoin%
\definecolor{currentfill}{rgb}{0.000000,0.000000,0.000000}%
\pgfsetfillcolor{currentfill}%
\pgfsetlinewidth{0.803000pt}%
\definecolor{currentstroke}{rgb}{0.000000,0.000000,0.000000}%
\pgfsetstrokecolor{currentstroke}%
\pgfsetdash{}{0pt}%
\pgfsys@defobject{currentmarker}{\pgfqpoint{-0.048611in}{0.000000in}}{\pgfqpoint{-0.000000in}{0.000000in}}{%
\pgfpathmoveto{\pgfqpoint{-0.000000in}{0.000000in}}%
\pgfpathlineto{\pgfqpoint{-0.048611in}{0.000000in}}%
\pgfusepath{stroke,fill}%
}%
\begin{pgfscope}%
\pgfsys@transformshift{0.588387in}{0.617054in}%
\pgfsys@useobject{currentmarker}{}%
\end{pgfscope}%
\end{pgfscope}%
\begin{pgfscope}%
\definecolor{textcolor}{rgb}{0.000000,0.000000,0.000000}%
\pgfsetstrokecolor{textcolor}%
\pgfsetfillcolor{textcolor}%
\pgftext[x=0.289968in, y=0.564293in, left, base]{\color{textcolor}{\rmfamily\fontsize{10.000000}{12.000000}\selectfont\catcode`\^=\active\def^{\ifmmode\sp\else\^{}\fi}\catcode`\%=\active\def%{\%}$\mathdefault{10^{2}}$}}%
\end{pgfscope}%
\begin{pgfscope}%
\pgfsetbuttcap%
\pgfsetroundjoin%
\definecolor{currentfill}{rgb}{0.000000,0.000000,0.000000}%
\pgfsetfillcolor{currentfill}%
\pgfsetlinewidth{0.803000pt}%
\definecolor{currentstroke}{rgb}{0.000000,0.000000,0.000000}%
\pgfsetstrokecolor{currentstroke}%
\pgfsetdash{}{0pt}%
\pgfsys@defobject{currentmarker}{\pgfqpoint{-0.048611in}{0.000000in}}{\pgfqpoint{-0.000000in}{0.000000in}}{%
\pgfpathmoveto{\pgfqpoint{-0.000000in}{0.000000in}}%
\pgfpathlineto{\pgfqpoint{-0.048611in}{0.000000in}}%
\pgfusepath{stroke,fill}%
}%
\begin{pgfscope}%
\pgfsys@transformshift{0.588387in}{2.274401in}%
\pgfsys@useobject{currentmarker}{}%
\end{pgfscope}%
\end{pgfscope}%
\begin{pgfscope}%
\definecolor{textcolor}{rgb}{0.000000,0.000000,0.000000}%
\pgfsetstrokecolor{textcolor}%
\pgfsetfillcolor{textcolor}%
\pgftext[x=0.289968in, y=2.221640in, left, base]{\color{textcolor}{\rmfamily\fontsize{10.000000}{12.000000}\selectfont\catcode`\^=\active\def^{\ifmmode\sp\else\^{}\fi}\catcode`\%=\active\def%{\%}$\mathdefault{10^{3}}$}}%
\end{pgfscope}%
\begin{pgfscope}%
\pgfsetbuttcap%
\pgfsetroundjoin%
\definecolor{currentfill}{rgb}{0.000000,0.000000,0.000000}%
\pgfsetfillcolor{currentfill}%
\pgfsetlinewidth{0.602250pt}%
\definecolor{currentstroke}{rgb}{0.000000,0.000000,0.000000}%
\pgfsetstrokecolor{currentstroke}%
\pgfsetdash{}{0pt}%
\pgfsys@defobject{currentmarker}{\pgfqpoint{-0.027778in}{0.000000in}}{\pgfqpoint{-0.000000in}{0.000000in}}{%
\pgfpathmoveto{\pgfqpoint{-0.000000in}{0.000000in}}%
\pgfpathlineto{\pgfqpoint{-0.027778in}{0.000000in}}%
\pgfusepath{stroke,fill}%
}%
\begin{pgfscope}%
\pgfsys@transformshift{0.588387in}{0.541218in}%
\pgfsys@useobject{currentmarker}{}%
\end{pgfscope}%
\end{pgfscope}%
\begin{pgfscope}%
\pgfsetbuttcap%
\pgfsetroundjoin%
\definecolor{currentfill}{rgb}{0.000000,0.000000,0.000000}%
\pgfsetfillcolor{currentfill}%
\pgfsetlinewidth{0.602250pt}%
\definecolor{currentstroke}{rgb}{0.000000,0.000000,0.000000}%
\pgfsetstrokecolor{currentstroke}%
\pgfsetdash{}{0pt}%
\pgfsys@defobject{currentmarker}{\pgfqpoint{-0.027778in}{0.000000in}}{\pgfqpoint{-0.000000in}{0.000000in}}{%
\pgfpathmoveto{\pgfqpoint{-0.000000in}{0.000000in}}%
\pgfpathlineto{\pgfqpoint{-0.027778in}{0.000000in}}%
\pgfusepath{stroke,fill}%
}%
\begin{pgfscope}%
\pgfsys@transformshift{0.588387in}{1.115965in}%
\pgfsys@useobject{currentmarker}{}%
\end{pgfscope}%
\end{pgfscope}%
\begin{pgfscope}%
\pgfsetbuttcap%
\pgfsetroundjoin%
\definecolor{currentfill}{rgb}{0.000000,0.000000,0.000000}%
\pgfsetfillcolor{currentfill}%
\pgfsetlinewidth{0.602250pt}%
\definecolor{currentstroke}{rgb}{0.000000,0.000000,0.000000}%
\pgfsetstrokecolor{currentstroke}%
\pgfsetdash{}{0pt}%
\pgfsys@defobject{currentmarker}{\pgfqpoint{-0.027778in}{0.000000in}}{\pgfqpoint{-0.000000in}{0.000000in}}{%
\pgfpathmoveto{\pgfqpoint{-0.000000in}{0.000000in}}%
\pgfpathlineto{\pgfqpoint{-0.027778in}{0.000000in}}%
\pgfusepath{stroke,fill}%
}%
\begin{pgfscope}%
\pgfsys@transformshift{0.588387in}{1.407810in}%
\pgfsys@useobject{currentmarker}{}%
\end{pgfscope}%
\end{pgfscope}%
\begin{pgfscope}%
\pgfsetbuttcap%
\pgfsetroundjoin%
\definecolor{currentfill}{rgb}{0.000000,0.000000,0.000000}%
\pgfsetfillcolor{currentfill}%
\pgfsetlinewidth{0.602250pt}%
\definecolor{currentstroke}{rgb}{0.000000,0.000000,0.000000}%
\pgfsetstrokecolor{currentstroke}%
\pgfsetdash{}{0pt}%
\pgfsys@defobject{currentmarker}{\pgfqpoint{-0.027778in}{0.000000in}}{\pgfqpoint{-0.000000in}{0.000000in}}{%
\pgfpathmoveto{\pgfqpoint{-0.000000in}{0.000000in}}%
\pgfpathlineto{\pgfqpoint{-0.027778in}{0.000000in}}%
\pgfusepath{stroke,fill}%
}%
\begin{pgfscope}%
\pgfsys@transformshift{0.588387in}{1.614876in}%
\pgfsys@useobject{currentmarker}{}%
\end{pgfscope}%
\end{pgfscope}%
\begin{pgfscope}%
\pgfsetbuttcap%
\pgfsetroundjoin%
\definecolor{currentfill}{rgb}{0.000000,0.000000,0.000000}%
\pgfsetfillcolor{currentfill}%
\pgfsetlinewidth{0.602250pt}%
\definecolor{currentstroke}{rgb}{0.000000,0.000000,0.000000}%
\pgfsetstrokecolor{currentstroke}%
\pgfsetdash{}{0pt}%
\pgfsys@defobject{currentmarker}{\pgfqpoint{-0.027778in}{0.000000in}}{\pgfqpoint{-0.000000in}{0.000000in}}{%
\pgfpathmoveto{\pgfqpoint{-0.000000in}{0.000000in}}%
\pgfpathlineto{\pgfqpoint{-0.027778in}{0.000000in}}%
\pgfusepath{stroke,fill}%
}%
\begin{pgfscope}%
\pgfsys@transformshift{0.588387in}{1.775490in}%
\pgfsys@useobject{currentmarker}{}%
\end{pgfscope}%
\end{pgfscope}%
\begin{pgfscope}%
\pgfsetbuttcap%
\pgfsetroundjoin%
\definecolor{currentfill}{rgb}{0.000000,0.000000,0.000000}%
\pgfsetfillcolor{currentfill}%
\pgfsetlinewidth{0.602250pt}%
\definecolor{currentstroke}{rgb}{0.000000,0.000000,0.000000}%
\pgfsetstrokecolor{currentstroke}%
\pgfsetdash{}{0pt}%
\pgfsys@defobject{currentmarker}{\pgfqpoint{-0.027778in}{0.000000in}}{\pgfqpoint{-0.000000in}{0.000000in}}{%
\pgfpathmoveto{\pgfqpoint{-0.000000in}{0.000000in}}%
\pgfpathlineto{\pgfqpoint{-0.027778in}{0.000000in}}%
\pgfusepath{stroke,fill}%
}%
\begin{pgfscope}%
\pgfsys@transformshift{0.588387in}{1.906721in}%
\pgfsys@useobject{currentmarker}{}%
\end{pgfscope}%
\end{pgfscope}%
\begin{pgfscope}%
\pgfsetbuttcap%
\pgfsetroundjoin%
\definecolor{currentfill}{rgb}{0.000000,0.000000,0.000000}%
\pgfsetfillcolor{currentfill}%
\pgfsetlinewidth{0.602250pt}%
\definecolor{currentstroke}{rgb}{0.000000,0.000000,0.000000}%
\pgfsetstrokecolor{currentstroke}%
\pgfsetdash{}{0pt}%
\pgfsys@defobject{currentmarker}{\pgfqpoint{-0.027778in}{0.000000in}}{\pgfqpoint{-0.000000in}{0.000000in}}{%
\pgfpathmoveto{\pgfqpoint{-0.000000in}{0.000000in}}%
\pgfpathlineto{\pgfqpoint{-0.027778in}{0.000000in}}%
\pgfusepath{stroke,fill}%
}%
\begin{pgfscope}%
\pgfsys@transformshift{0.588387in}{2.017675in}%
\pgfsys@useobject{currentmarker}{}%
\end{pgfscope}%
\end{pgfscope}%
\begin{pgfscope}%
\pgfsetbuttcap%
\pgfsetroundjoin%
\definecolor{currentfill}{rgb}{0.000000,0.000000,0.000000}%
\pgfsetfillcolor{currentfill}%
\pgfsetlinewidth{0.602250pt}%
\definecolor{currentstroke}{rgb}{0.000000,0.000000,0.000000}%
\pgfsetstrokecolor{currentstroke}%
\pgfsetdash{}{0pt}%
\pgfsys@defobject{currentmarker}{\pgfqpoint{-0.027778in}{0.000000in}}{\pgfqpoint{-0.000000in}{0.000000in}}{%
\pgfpathmoveto{\pgfqpoint{-0.000000in}{0.000000in}}%
\pgfpathlineto{\pgfqpoint{-0.027778in}{0.000000in}}%
\pgfusepath{stroke,fill}%
}%
\begin{pgfscope}%
\pgfsys@transformshift{0.588387in}{2.113788in}%
\pgfsys@useobject{currentmarker}{}%
\end{pgfscope}%
\end{pgfscope}%
\begin{pgfscope}%
\pgfsetbuttcap%
\pgfsetroundjoin%
\definecolor{currentfill}{rgb}{0.000000,0.000000,0.000000}%
\pgfsetfillcolor{currentfill}%
\pgfsetlinewidth{0.602250pt}%
\definecolor{currentstroke}{rgb}{0.000000,0.000000,0.000000}%
\pgfsetstrokecolor{currentstroke}%
\pgfsetdash{}{0pt}%
\pgfsys@defobject{currentmarker}{\pgfqpoint{-0.027778in}{0.000000in}}{\pgfqpoint{-0.000000in}{0.000000in}}{%
\pgfpathmoveto{\pgfqpoint{-0.000000in}{0.000000in}}%
\pgfpathlineto{\pgfqpoint{-0.027778in}{0.000000in}}%
\pgfusepath{stroke,fill}%
}%
\begin{pgfscope}%
\pgfsys@transformshift{0.588387in}{2.198565in}%
\pgfsys@useobject{currentmarker}{}%
\end{pgfscope}%
\end{pgfscope}%
\begin{pgfscope}%
\definecolor{textcolor}{rgb}{0.000000,0.000000,0.000000}%
\pgfsetstrokecolor{textcolor}%
\pgfsetfillcolor{textcolor}%
\pgftext[x=0.234413in,y=1.631726in,,bottom,rotate=90.000000]{\color{textcolor}{\rmfamily\fontsize{10.000000}{12.000000}\selectfont\catcode`\^=\active\def^{\ifmmode\sp\else\^{}\fi}\catcode`\%=\active\def%{\%}Checks [call]}}%
\end{pgfscope}%
\begin{pgfscope}%
\pgfpathrectangle{\pgfqpoint{0.588387in}{0.521603in}}{\pgfqpoint{3.660036in}{2.220246in}}%
\pgfusepath{clip}%
\pgfsetrectcap%
\pgfsetroundjoin%
\pgfsetlinewidth{1.505625pt}%
\pgfsetstrokecolor{currentstroke1}%
\pgfsetdash{}{0pt}%
\pgfpathmoveto{\pgfqpoint{0.754752in}{0.830864in}}%
\pgfpathlineto{\pgfqpoint{0.778184in}{1.131821in}}%
\pgfpathlineto{\pgfqpoint{0.801616in}{0.696192in}}%
\pgfpathlineto{\pgfqpoint{0.825048in}{0.680605in}}%
\pgfpathlineto{\pgfqpoint{0.848479in}{0.847420in}}%
\pgfpathlineto{\pgfqpoint{0.871911in}{1.015654in}}%
\pgfpathlineto{\pgfqpoint{0.895343in}{1.123583in}}%
\pgfpathlineto{\pgfqpoint{0.918775in}{0.826297in}}%
\pgfpathlineto{\pgfqpoint{0.942206in}{0.864990in}}%
\pgfpathlineto{\pgfqpoint{0.965638in}{1.044881in}}%
\pgfpathlineto{\pgfqpoint{0.989070in}{1.171189in}}%
\pgfpathlineto{\pgfqpoint{1.012501in}{1.191170in}}%
\pgfpathlineto{\pgfqpoint{1.035933in}{1.009794in}}%
\pgfpathlineto{\pgfqpoint{1.059365in}{1.091039in}}%
\pgfpathlineto{\pgfqpoint{1.082797in}{1.208475in}}%
\pgfpathlineto{\pgfqpoint{1.106228in}{1.225004in}}%
\pgfpathlineto{\pgfqpoint{1.129660in}{1.305678in}}%
\pgfpathlineto{\pgfqpoint{1.153092in}{1.182593in}}%
\pgfpathlineto{\pgfqpoint{1.176524in}{1.249649in}}%
\pgfpathlineto{\pgfqpoint{1.199955in}{1.264848in}}%
\pgfpathlineto{\pgfqpoint{1.223387in}{1.335519in}}%
\pgfpathlineto{\pgfqpoint{1.246819in}{1.421759in}}%
\pgfpathlineto{\pgfqpoint{1.270250in}{1.323882in}}%
\pgfpathlineto{\pgfqpoint{1.293682in}{1.293649in}}%
\pgfpathlineto{\pgfqpoint{1.317114in}{1.381994in}}%
\pgfpathlineto{\pgfqpoint{1.340546in}{1.430670in}}%
\pgfpathlineto{\pgfqpoint{1.363977in}{1.517966in}}%
\pgfpathlineto{\pgfqpoint{1.387409in}{1.370780in}}%
\pgfpathlineto{\pgfqpoint{1.410841in}{1.404603in}}%
\pgfpathlineto{\pgfqpoint{1.434273in}{1.438987in}}%
\pgfpathlineto{\pgfqpoint{1.457704in}{1.535883in}}%
\pgfpathlineto{\pgfqpoint{1.481136in}{1.548900in}}%
\pgfpathlineto{\pgfqpoint{1.504568in}{1.451033in}}%
\pgfpathlineto{\pgfqpoint{1.527999in}{1.521119in}}%
\pgfpathlineto{\pgfqpoint{1.551431in}{1.591205in}}%
\pgfpathlineto{\pgfqpoint{1.574863in}{1.592010in}}%
\pgfpathlineto{\pgfqpoint{1.598295in}{1.598000in}}%
\pgfpathlineto{\pgfqpoint{1.621726in}{1.543439in}}%
\pgfpathlineto{\pgfqpoint{1.645158in}{1.574563in}}%
\pgfpathlineto{\pgfqpoint{1.668590in}{1.567735in}}%
\pgfpathlineto{\pgfqpoint{1.692021in}{1.608924in}}%
\pgfpathlineto{\pgfqpoint{1.715453in}{1.737117in}}%
\pgfpathlineto{\pgfqpoint{1.738885in}{1.618256in}}%
\pgfpathlineto{\pgfqpoint{1.762317in}{1.586262in}}%
\pgfpathlineto{\pgfqpoint{1.785748in}{1.620612in}}%
\pgfpathlineto{\pgfqpoint{1.809180in}{1.704966in}}%
\pgfpathlineto{\pgfqpoint{1.832612in}{1.719475in}}%
\pgfpathlineto{\pgfqpoint{1.856044in}{1.638168in}}%
\pgfpathlineto{\pgfqpoint{1.879475in}{1.697834in}}%
\pgfpathlineto{\pgfqpoint{1.902907in}{1.673851in}}%
\pgfpathlineto{\pgfqpoint{1.926339in}{1.750095in}}%
\pgfpathlineto{\pgfqpoint{1.949770in}{1.715572in}}%
\pgfpathlineto{\pgfqpoint{1.973202in}{1.663576in}}%
\pgfpathlineto{\pgfqpoint{1.996634in}{1.729289in}}%
\pgfpathlineto{\pgfqpoint{2.020066in}{1.803720in}}%
\pgfpathlineto{\pgfqpoint{2.043497in}{1.783127in}}%
\pgfpathlineto{\pgfqpoint{2.066929in}{1.794433in}}%
\pgfpathlineto{\pgfqpoint{2.090361in}{1.794001in}}%
\pgfpathlineto{\pgfqpoint{2.113793in}{1.761927in}}%
\pgfpathlineto{\pgfqpoint{2.137224in}{1.816143in}}%
\pgfpathlineto{\pgfqpoint{2.160656in}{1.816241in}}%
\pgfpathlineto{\pgfqpoint{2.184088in}{1.864563in}}%
\pgfpathlineto{\pgfqpoint{2.207519in}{1.861720in}}%
\pgfpathlineto{\pgfqpoint{2.230951in}{1.825946in}}%
\pgfpathlineto{\pgfqpoint{2.254383in}{1.832932in}}%
\pgfpathlineto{\pgfqpoint{2.277815in}{1.898101in}}%
\pgfpathlineto{\pgfqpoint{2.301246in}{1.905760in}}%
\pgfpathlineto{\pgfqpoint{2.324678in}{1.871693in}}%
\pgfpathlineto{\pgfqpoint{2.348110in}{1.833546in}}%
\pgfpathlineto{\pgfqpoint{2.371542in}{1.953281in}}%
\pgfpathlineto{\pgfqpoint{2.394973in}{1.895522in}}%
\pgfpathlineto{\pgfqpoint{2.418405in}{1.878448in}}%
\pgfpathlineto{\pgfqpoint{2.441837in}{1.844615in}}%
\pgfpathlineto{\pgfqpoint{2.465268in}{1.922357in}}%
\pgfpathlineto{\pgfqpoint{2.488700in}{1.882009in}}%
\pgfpathlineto{\pgfqpoint{2.512132in}{1.912624in}}%
\pgfpathlineto{\pgfqpoint{2.535564in}{2.030973in}}%
\pgfpathlineto{\pgfqpoint{2.558995in}{1.990858in}}%
\pgfpathlineto{\pgfqpoint{2.582427in}{2.011389in}}%
\pgfpathlineto{\pgfqpoint{2.605859in}{1.947151in}}%
\pgfpathlineto{\pgfqpoint{2.629291in}{1.953736in}}%
\pgfpathlineto{\pgfqpoint{2.652722in}{2.024657in}}%
\pgfpathlineto{\pgfqpoint{2.676154in}{1.973138in}}%
\pgfpathlineto{\pgfqpoint{2.699586in}{1.944120in}}%
\pgfpathlineto{\pgfqpoint{2.723017in}{1.939197in}}%
\pgfpathlineto{\pgfqpoint{2.746449in}{2.009646in}}%
\pgfpathlineto{\pgfqpoint{2.769881in}{2.048335in}}%
\pgfpathlineto{\pgfqpoint{2.793313in}{1.978708in}}%
\pgfpathlineto{\pgfqpoint{2.816744in}{2.094333in}}%
\pgfpathlineto{\pgfqpoint{2.840176in}{1.966545in}}%
\pgfpathlineto{\pgfqpoint{2.863608in}{1.986147in}}%
\pgfpathlineto{\pgfqpoint{2.887039in}{2.073655in}}%
\pgfpathlineto{\pgfqpoint{2.910471in}{1.987633in}}%
\pgfpathlineto{\pgfqpoint{2.933903in}{2.050439in}}%
\pgfpathlineto{\pgfqpoint{2.957335in}{2.085154in}}%
\pgfpathlineto{\pgfqpoint{2.980766in}{1.944120in}}%
\pgfpathlineto{\pgfqpoint{3.004198in}{2.085154in}}%
\pgfpathlineto{\pgfqpoint{3.027630in}{1.979672in}}%
\pgfpathlineto{\pgfqpoint{3.051062in}{2.047880in}}%
\pgfpathlineto{\pgfqpoint{3.074493in}{2.117378in}}%
\pgfpathlineto{\pgfqpoint{3.097925in}{2.125010in}}%
\pgfpathlineto{\pgfqpoint{3.121357in}{2.007318in}}%
\pgfpathlineto{\pgfqpoint{3.144788in}{2.023410in}}%
\pgfpathlineto{\pgfqpoint{3.168220in}{1.991711in}}%
\pgfpathlineto{\pgfqpoint{3.191652in}{2.008013in}}%
\pgfpathlineto{\pgfqpoint{3.215084in}{2.309976in}}%
\pgfpathlineto{\pgfqpoint{3.238515in}{2.050635in}}%
\pgfpathlineto{\pgfqpoint{3.261947in}{2.021776in}}%
\pgfpathlineto{\pgfqpoint{3.285379in}{2.029909in}}%
\pgfpathlineto{\pgfqpoint{3.308811in}{2.173612in}}%
\pgfpathlineto{\pgfqpoint{3.332242in}{2.074157in}}%
\pgfpathlineto{\pgfqpoint{3.355674in}{2.049062in}}%
\pgfpathlineto{\pgfqpoint{3.379106in}{2.053771in}}%
\pgfpathlineto{\pgfqpoint{3.402537in}{2.084405in}}%
\pgfpathlineto{\pgfqpoint{3.425969in}{2.096794in}}%
\pgfpathlineto{\pgfqpoint{3.449401in}{2.069251in}}%
\pgfpathlineto{\pgfqpoint{3.472833in}{2.073070in}}%
\pgfpathlineto{\pgfqpoint{3.496264in}{2.099246in}}%
\pgfpathlineto{\pgfqpoint{3.519696in}{2.117378in}}%
\pgfpathlineto{\pgfqpoint{3.589991in}{2.238995in}}%
\pgfpathlineto{\pgfqpoint{3.707150in}{2.162487in}}%
\pgfusepath{stroke}%
\end{pgfscope}%
\begin{pgfscope}%
\pgfpathrectangle{\pgfqpoint{0.588387in}{0.521603in}}{\pgfqpoint{3.660036in}{2.220246in}}%
\pgfusepath{clip}%
\pgfsetrectcap%
\pgfsetroundjoin%
\pgfsetlinewidth{1.505625pt}%
\pgfsetstrokecolor{currentstroke2}%
\pgfsetdash{}{0pt}%
\pgfpathmoveto{\pgfqpoint{0.754752in}{0.842213in}}%
\pgfpathlineto{\pgfqpoint{0.778184in}{1.131821in}}%
\pgfpathlineto{\pgfqpoint{0.801616in}{0.696320in}}%
\pgfpathlineto{\pgfqpoint{0.825048in}{0.685490in}}%
\pgfpathlineto{\pgfqpoint{0.871911in}{1.015911in}}%
\pgfpathlineto{\pgfqpoint{0.895343in}{1.121475in}}%
\pgfpathlineto{\pgfqpoint{0.918775in}{0.826393in}}%
\pgfpathlineto{\pgfqpoint{0.942206in}{0.866935in}}%
\pgfpathlineto{\pgfqpoint{0.965638in}{1.047363in}}%
\pgfpathlineto{\pgfqpoint{0.989070in}{1.175280in}}%
\pgfpathlineto{\pgfqpoint{1.012501in}{1.198592in}}%
\pgfpathlineto{\pgfqpoint{1.035933in}{1.013580in}}%
\pgfpathlineto{\pgfqpoint{1.059365in}{1.092519in}}%
\pgfpathlineto{\pgfqpoint{1.082797in}{1.208475in}}%
\pgfpathlineto{\pgfqpoint{1.106228in}{1.222794in}}%
\pgfpathlineto{\pgfqpoint{1.129660in}{1.308221in}}%
\pgfpathlineto{\pgfqpoint{1.153092in}{1.184274in}}%
\pgfpathlineto{\pgfqpoint{1.176524in}{1.250645in}}%
\pgfpathlineto{\pgfqpoint{1.199955in}{1.268391in}}%
\pgfpathlineto{\pgfqpoint{1.223387in}{1.337033in}}%
\pgfpathlineto{\pgfqpoint{1.246819in}{1.419707in}}%
\pgfpathlineto{\pgfqpoint{1.270250in}{1.339016in}}%
\pgfpathlineto{\pgfqpoint{1.293682in}{1.289086in}}%
\pgfpathlineto{\pgfqpoint{1.317114in}{1.394183in}}%
\pgfpathlineto{\pgfqpoint{1.340546in}{1.436776in}}%
\pgfpathlineto{\pgfqpoint{1.363977in}{1.515946in}}%
\pgfpathlineto{\pgfqpoint{1.387409in}{1.373849in}}%
\pgfpathlineto{\pgfqpoint{1.434273in}{1.438987in}}%
\pgfpathlineto{\pgfqpoint{1.457704in}{1.545194in}}%
\pgfpathlineto{\pgfqpoint{1.481136in}{1.557822in}}%
\pgfpathlineto{\pgfqpoint{1.504568in}{1.451033in}}%
\pgfpathlineto{\pgfqpoint{1.551431in}{1.598172in}}%
\pgfpathlineto{\pgfqpoint{1.574863in}{1.586326in}}%
\pgfpathlineto{\pgfqpoint{1.598295in}{1.603128in}}%
\pgfpathlineto{\pgfqpoint{1.621726in}{1.560771in}}%
\pgfpathlineto{\pgfqpoint{1.645158in}{1.584855in}}%
\pgfpathlineto{\pgfqpoint{1.668590in}{1.581237in}}%
\pgfpathlineto{\pgfqpoint{1.692021in}{1.608355in}}%
\pgfpathlineto{\pgfqpoint{1.715453in}{1.727435in}}%
\pgfpathlineto{\pgfqpoint{1.738885in}{1.614876in}}%
\pgfpathlineto{\pgfqpoint{1.762317in}{1.595801in}}%
\pgfpathlineto{\pgfqpoint{1.785748in}{1.611027in}}%
\pgfpathlineto{\pgfqpoint{1.809180in}{1.704401in}}%
\pgfpathlineto{\pgfqpoint{1.832612in}{1.724722in}}%
\pgfpathlineto{\pgfqpoint{1.856044in}{1.650527in}}%
\pgfpathlineto{\pgfqpoint{1.879475in}{1.695717in}}%
\pgfpathlineto{\pgfqpoint{1.902907in}{1.686492in}}%
\pgfpathlineto{\pgfqpoint{1.926339in}{1.750095in}}%
\pgfpathlineto{\pgfqpoint{1.949770in}{1.721706in}}%
\pgfpathlineto{\pgfqpoint{1.973202in}{1.663576in}}%
\pgfpathlineto{\pgfqpoint{1.996634in}{1.729990in}}%
\pgfpathlineto{\pgfqpoint{2.020066in}{1.780681in}}%
\pgfpathlineto{\pgfqpoint{2.043497in}{1.797418in}}%
\pgfpathlineto{\pgfqpoint{2.066929in}{1.818697in}}%
\pgfpathlineto{\pgfqpoint{2.090361in}{1.820558in}}%
\pgfpathlineto{\pgfqpoint{2.113793in}{1.763880in}}%
\pgfpathlineto{\pgfqpoint{2.137224in}{1.802262in}}%
\pgfpathlineto{\pgfqpoint{2.160656in}{1.834209in}}%
\pgfpathlineto{\pgfqpoint{2.184088in}{1.869969in}}%
\pgfpathlineto{\pgfqpoint{2.207519in}{1.857061in}}%
\pgfpathlineto{\pgfqpoint{2.230951in}{1.815548in}}%
\pgfpathlineto{\pgfqpoint{2.254383in}{1.850706in}}%
\pgfpathlineto{\pgfqpoint{2.277815in}{1.898216in}}%
\pgfpathlineto{\pgfqpoint{2.301246in}{1.905760in}}%
\pgfpathlineto{\pgfqpoint{2.324678in}{1.881698in}}%
\pgfpathlineto{\pgfqpoint{2.348110in}{1.849308in}}%
\pgfpathlineto{\pgfqpoint{2.371542in}{1.961798in}}%
\pgfpathlineto{\pgfqpoint{2.394973in}{1.889339in}}%
\pgfpathlineto{\pgfqpoint{2.418405in}{1.895978in}}%
\pgfpathlineto{\pgfqpoint{2.441837in}{1.857061in}}%
\pgfpathlineto{\pgfqpoint{2.465268in}{1.921251in}}%
\pgfpathlineto{\pgfqpoint{2.488700in}{1.882009in}}%
\pgfpathlineto{\pgfqpoint{2.512132in}{1.915976in}}%
\pgfpathlineto{\pgfqpoint{2.535564in}{2.014638in}}%
\pgfpathlineto{\pgfqpoint{2.558995in}{1.940466in}}%
\pgfpathlineto{\pgfqpoint{2.582427in}{1.996074in}}%
\pgfpathlineto{\pgfqpoint{2.605859in}{1.931611in}}%
\pgfpathlineto{\pgfqpoint{2.652722in}{2.025615in}}%
\pgfpathlineto{\pgfqpoint{2.676154in}{2.011479in}}%
\pgfpathlineto{\pgfqpoint{2.699586in}{1.942462in}}%
\pgfpathlineto{\pgfqpoint{2.723017in}{1.953866in}}%
\pgfpathlineto{\pgfqpoint{2.746449in}{2.019366in}}%
\pgfpathlineto{\pgfqpoint{2.769881in}{2.041021in}}%
\pgfpathlineto{\pgfqpoint{2.793313in}{2.034388in}}%
\pgfpathlineto{\pgfqpoint{2.816744in}{2.091039in}}%
\pgfpathlineto{\pgfqpoint{2.840176in}{1.968309in}}%
\pgfpathlineto{\pgfqpoint{2.863608in}{1.986147in}}%
\pgfpathlineto{\pgfqpoint{2.887039in}{2.094996in}}%
\pgfpathlineto{\pgfqpoint{2.910471in}{1.992891in}}%
\pgfpathlineto{\pgfqpoint{2.933903in}{2.062906in}}%
\pgfpathlineto{\pgfqpoint{2.957335in}{2.085154in}}%
\pgfpathlineto{\pgfqpoint{2.980766in}{1.944120in}}%
\pgfpathlineto{\pgfqpoint{3.004198in}{2.024226in}}%
\pgfpathlineto{\pgfqpoint{3.027630in}{2.177136in}}%
\pgfpathlineto{\pgfqpoint{3.051062in}{2.045905in}}%
\pgfpathlineto{\pgfqpoint{3.074493in}{2.113788in}}%
\pgfpathlineto{\pgfqpoint{3.097925in}{2.135936in}}%
\pgfpathlineto{\pgfqpoint{3.121357in}{2.047880in}}%
\pgfpathlineto{\pgfqpoint{3.144788in}{2.023410in}}%
\pgfpathlineto{\pgfqpoint{3.168220in}{1.991711in}}%
\pgfpathlineto{\pgfqpoint{3.191652in}{2.011479in}}%
\pgfpathlineto{\pgfqpoint{3.215084in}{2.443682in}}%
\pgfpathlineto{\pgfqpoint{3.238515in}{2.050635in}}%
\pgfpathlineto{\pgfqpoint{3.261947in}{2.021776in}}%
\pgfpathlineto{\pgfqpoint{3.285379in}{2.029909in}}%
\pgfpathlineto{\pgfqpoint{3.308811in}{2.278707in}}%
\pgfpathlineto{\pgfqpoint{3.332242in}{2.078287in}}%
\pgfpathlineto{\pgfqpoint{3.355674in}{2.041939in}}%
\pgfpathlineto{\pgfqpoint{3.379106in}{2.095564in}}%
\pgfpathlineto{\pgfqpoint{3.402537in}{2.099246in}}%
\pgfpathlineto{\pgfqpoint{3.425969in}{2.106554in}}%
\pgfpathlineto{\pgfqpoint{3.449401in}{2.076868in}}%
\pgfpathlineto{\pgfqpoint{3.472833in}{2.130682in}}%
\pgfpathlineto{\pgfqpoint{3.496264in}{2.094641in}}%
\pgfpathlineto{\pgfqpoint{3.519696in}{2.118570in}}%
\pgfpathlineto{\pgfqpoint{3.543128in}{2.128041in}}%
\pgfpathlineto{\pgfqpoint{3.566560in}{2.477498in}}%
\pgfpathlineto{\pgfqpoint{3.589991in}{2.135063in}}%
\pgfpathlineto{\pgfqpoint{3.613423in}{2.142018in}}%
\pgfpathlineto{\pgfqpoint{3.636855in}{2.106554in}}%
\pgfpathlineto{\pgfqpoint{3.660286in}{2.120950in}}%
\pgfpathlineto{\pgfqpoint{3.683718in}{2.169182in}}%
\pgfpathlineto{\pgfqpoint{3.707150in}{2.162487in}}%
\pgfpathlineto{\pgfqpoint{3.777445in}{2.148906in}}%
\pgfpathlineto{\pgfqpoint{3.824309in}{2.155728in}}%
\pgfpathlineto{\pgfqpoint{3.824309in}{2.155728in}}%
\pgfusepath{stroke}%
\end{pgfscope}%
\begin{pgfscope}%
\pgfpathrectangle{\pgfqpoint{0.588387in}{0.521603in}}{\pgfqpoint{3.660036in}{2.220246in}}%
\pgfusepath{clip}%
\pgfsetrectcap%
\pgfsetroundjoin%
\pgfsetlinewidth{1.505625pt}%
\pgfsetstrokecolor{currentstroke3}%
\pgfsetdash{}{0pt}%
\pgfpathmoveto{\pgfqpoint{0.754752in}{0.722227in}}%
\pgfpathlineto{\pgfqpoint{0.778184in}{1.031008in}}%
\pgfpathlineto{\pgfqpoint{0.801616in}{0.705776in}}%
\pgfpathlineto{\pgfqpoint{0.825048in}{0.622893in}}%
\pgfpathlineto{\pgfqpoint{0.848479in}{0.759775in}}%
\pgfpathlineto{\pgfqpoint{0.871911in}{0.929663in}}%
\pgfpathlineto{\pgfqpoint{0.895343in}{1.055849in}}%
\pgfpathlineto{\pgfqpoint{0.918775in}{0.776936in}}%
\pgfpathlineto{\pgfqpoint{0.942206in}{0.833505in}}%
\pgfpathlineto{\pgfqpoint{0.965638in}{0.993930in}}%
\pgfpathlineto{\pgfqpoint{0.989070in}{1.131226in}}%
\pgfpathlineto{\pgfqpoint{1.012501in}{1.124064in}}%
\pgfpathlineto{\pgfqpoint{1.035933in}{0.971227in}}%
\pgfpathlineto{\pgfqpoint{1.059365in}{1.056521in}}%
\pgfpathlineto{\pgfqpoint{1.082797in}{1.173921in}}%
\pgfpathlineto{\pgfqpoint{1.106228in}{1.160461in}}%
\pgfpathlineto{\pgfqpoint{1.129660in}{1.233253in}}%
\pgfpathlineto{\pgfqpoint{1.153092in}{1.153001in}}%
\pgfpathlineto{\pgfqpoint{1.176524in}{1.218926in}}%
\pgfpathlineto{\pgfqpoint{1.199955in}{1.213112in}}%
\pgfpathlineto{\pgfqpoint{1.223387in}{1.301445in}}%
\pgfpathlineto{\pgfqpoint{1.246819in}{1.362667in}}%
\pgfpathlineto{\pgfqpoint{1.270250in}{1.300674in}}%
\pgfpathlineto{\pgfqpoint{1.293682in}{1.257469in}}%
\pgfpathlineto{\pgfqpoint{1.340546in}{1.411374in}}%
\pgfpathlineto{\pgfqpoint{1.363977in}{1.479998in}}%
\pgfpathlineto{\pgfqpoint{1.387409in}{1.338690in}}%
\pgfpathlineto{\pgfqpoint{1.410841in}{1.364828in}}%
\pgfpathlineto{\pgfqpoint{1.434273in}{1.431987in}}%
\pgfpathlineto{\pgfqpoint{1.457704in}{1.517894in}}%
\pgfpathlineto{\pgfqpoint{1.481136in}{1.508426in}}%
\pgfpathlineto{\pgfqpoint{1.504568in}{1.436955in}}%
\pgfpathlineto{\pgfqpoint{1.527999in}{1.478509in}}%
\pgfpathlineto{\pgfqpoint{1.551431in}{1.543487in}}%
\pgfpathlineto{\pgfqpoint{1.574863in}{1.529303in}}%
\pgfpathlineto{\pgfqpoint{1.598295in}{1.561953in}}%
\pgfpathlineto{\pgfqpoint{1.621726in}{1.537673in}}%
\pgfpathlineto{\pgfqpoint{1.645158in}{1.567810in}}%
\pgfpathlineto{\pgfqpoint{1.668590in}{1.564476in}}%
\pgfpathlineto{\pgfqpoint{1.692021in}{1.596827in}}%
\pgfpathlineto{\pgfqpoint{1.715453in}{1.664433in}}%
\pgfpathlineto{\pgfqpoint{1.738885in}{1.622476in}}%
\pgfpathlineto{\pgfqpoint{1.762317in}{1.594474in}}%
\pgfpathlineto{\pgfqpoint{1.785748in}{1.631737in}}%
\pgfpathlineto{\pgfqpoint{1.809180in}{1.694839in}}%
\pgfpathlineto{\pgfqpoint{1.832612in}{1.745553in}}%
\pgfpathlineto{\pgfqpoint{1.856044in}{1.643107in}}%
\pgfpathlineto{\pgfqpoint{1.879475in}{1.670196in}}%
\pgfpathlineto{\pgfqpoint{1.926339in}{1.757689in}}%
\pgfpathlineto{\pgfqpoint{1.949770in}{1.750490in}}%
\pgfpathlineto{\pgfqpoint{1.973202in}{1.698287in}}%
\pgfpathlineto{\pgfqpoint{1.996634in}{1.751629in}}%
\pgfpathlineto{\pgfqpoint{2.020066in}{1.810850in}}%
\pgfpathlineto{\pgfqpoint{2.043497in}{1.779025in}}%
\pgfpathlineto{\pgfqpoint{2.066929in}{1.804977in}}%
\pgfpathlineto{\pgfqpoint{2.090361in}{1.782221in}}%
\pgfpathlineto{\pgfqpoint{2.113793in}{1.802373in}}%
\pgfpathlineto{\pgfqpoint{2.137224in}{1.799941in}}%
\pgfpathlineto{\pgfqpoint{2.160656in}{1.835012in}}%
\pgfpathlineto{\pgfqpoint{2.184088in}{1.887952in}}%
\pgfpathlineto{\pgfqpoint{2.207519in}{1.845456in}}%
\pgfpathlineto{\pgfqpoint{2.230951in}{1.825029in}}%
\pgfpathlineto{\pgfqpoint{2.254383in}{1.867595in}}%
\pgfpathlineto{\pgfqpoint{2.277815in}{1.900294in}}%
\pgfpathlineto{\pgfqpoint{2.301246in}{1.924447in}}%
\pgfpathlineto{\pgfqpoint{2.324678in}{1.881761in}}%
\pgfpathlineto{\pgfqpoint{2.348110in}{1.881229in}}%
\pgfpathlineto{\pgfqpoint{2.371542in}{1.915013in}}%
\pgfpathlineto{\pgfqpoint{2.394973in}{1.951265in}}%
\pgfpathlineto{\pgfqpoint{2.418405in}{1.963646in}}%
\pgfpathlineto{\pgfqpoint{2.441837in}{1.916517in}}%
\pgfpathlineto{\pgfqpoint{2.465268in}{1.933841in}}%
\pgfpathlineto{\pgfqpoint{2.488700in}{1.978788in}}%
\pgfpathlineto{\pgfqpoint{2.512132in}{1.976742in}}%
\pgfpathlineto{\pgfqpoint{2.535564in}{1.996281in}}%
\pgfpathlineto{\pgfqpoint{2.558995in}{1.975234in}}%
\pgfpathlineto{\pgfqpoint{2.582427in}{2.024647in}}%
\pgfpathlineto{\pgfqpoint{2.605859in}{1.983379in}}%
\pgfpathlineto{\pgfqpoint{2.629291in}{2.007454in}}%
\pgfpathlineto{\pgfqpoint{2.652722in}{2.070391in}}%
\pgfpathlineto{\pgfqpoint{2.676154in}{2.060654in}}%
\pgfpathlineto{\pgfqpoint{2.699586in}{1.995175in}}%
\pgfpathlineto{\pgfqpoint{2.723017in}{2.026362in}}%
\pgfpathlineto{\pgfqpoint{2.746449in}{2.062249in}}%
\pgfpathlineto{\pgfqpoint{2.769881in}{2.079704in}}%
\pgfpathlineto{\pgfqpoint{2.793313in}{2.080256in}}%
\pgfpathlineto{\pgfqpoint{2.816744in}{2.087352in}}%
\pgfpathlineto{\pgfqpoint{2.840176in}{2.060979in}}%
\pgfpathlineto{\pgfqpoint{2.863608in}{2.107256in}}%
\pgfpathlineto{\pgfqpoint{2.887039in}{2.127643in}}%
\pgfpathlineto{\pgfqpoint{2.910471in}{2.057339in}}%
\pgfpathlineto{\pgfqpoint{2.933903in}{2.083842in}}%
\pgfpathlineto{\pgfqpoint{2.957335in}{2.098853in}}%
\pgfpathlineto{\pgfqpoint{2.980766in}{2.099678in}}%
\pgfpathlineto{\pgfqpoint{3.004198in}{2.122844in}}%
\pgfpathlineto{\pgfqpoint{3.027630in}{2.129048in}}%
\pgfpathlineto{\pgfqpoint{3.051062in}{2.148398in}}%
\pgfpathlineto{\pgfqpoint{3.074493in}{2.157787in}}%
\pgfpathlineto{\pgfqpoint{3.097925in}{2.153688in}}%
\pgfpathlineto{\pgfqpoint{3.121357in}{2.206857in}}%
\pgfpathlineto{\pgfqpoint{3.144788in}{2.145832in}}%
\pgfpathlineto{\pgfqpoint{3.168220in}{2.137057in}}%
\pgfpathlineto{\pgfqpoint{3.215084in}{2.290909in}}%
\pgfpathlineto{\pgfqpoint{3.238515in}{2.208925in}}%
\pgfpathlineto{\pgfqpoint{3.261947in}{2.175556in}}%
\pgfpathlineto{\pgfqpoint{3.285379in}{2.168670in}}%
\pgfpathlineto{\pgfqpoint{3.308811in}{2.203030in}}%
\pgfpathlineto{\pgfqpoint{3.332242in}{2.222915in}}%
\pgfpathlineto{\pgfqpoint{3.355674in}{2.245129in}}%
\pgfpathlineto{\pgfqpoint{3.402537in}{2.219970in}}%
\pgfpathlineto{\pgfqpoint{3.425969in}{2.268837in}}%
\pgfpathlineto{\pgfqpoint{3.449401in}{2.268438in}}%
\pgfpathlineto{\pgfqpoint{3.472833in}{2.247157in}}%
\pgfpathlineto{\pgfqpoint{3.496264in}{2.234749in}}%
\pgfpathlineto{\pgfqpoint{3.519696in}{2.261327in}}%
\pgfpathlineto{\pgfqpoint{3.543128in}{2.298776in}}%
\pgfpathlineto{\pgfqpoint{3.566560in}{2.351244in}}%
\pgfpathlineto{\pgfqpoint{3.589991in}{2.400656in}}%
\pgfpathlineto{\pgfqpoint{3.613423in}{2.280672in}}%
\pgfpathlineto{\pgfqpoint{3.636855in}{2.262792in}}%
\pgfpathlineto{\pgfqpoint{3.660286in}{2.310718in}}%
\pgfpathlineto{\pgfqpoint{3.683718in}{2.402476in}}%
\pgfpathlineto{\pgfqpoint{3.707150in}{2.396882in}}%
\pgfpathlineto{\pgfqpoint{3.730582in}{2.353397in}}%
\pgfpathlineto{\pgfqpoint{3.754013in}{2.465456in}}%
\pgfpathlineto{\pgfqpoint{3.777445in}{2.428071in}}%
\pgfpathlineto{\pgfqpoint{3.800877in}{2.511426in}}%
\pgfpathlineto{\pgfqpoint{3.824309in}{2.477136in}}%
\pgfpathlineto{\pgfqpoint{3.847740in}{2.480207in}}%
\pgfpathlineto{\pgfqpoint{3.871172in}{2.345616in}}%
\pgfpathlineto{\pgfqpoint{3.894604in}{2.524765in}}%
\pgfpathlineto{\pgfqpoint{3.918035in}{2.504140in}}%
\pgfpathlineto{\pgfqpoint{3.964899in}{2.497307in}}%
\pgfpathlineto{\pgfqpoint{4.011762in}{2.532853in}}%
\pgfpathlineto{\pgfqpoint{4.011762in}{2.532853in}}%
\pgfusepath{stroke}%
\end{pgfscope}%
\begin{pgfscope}%
\pgfpathrectangle{\pgfqpoint{0.588387in}{0.521603in}}{\pgfqpoint{3.660036in}{2.220246in}}%
\pgfusepath{clip}%
\pgfsetrectcap%
\pgfsetroundjoin%
\pgfsetlinewidth{1.505625pt}%
\pgfsetstrokecolor{currentstroke4}%
\pgfsetdash{}{0pt}%
\pgfpathmoveto{\pgfqpoint{0.754752in}{0.830635in}}%
\pgfpathlineto{\pgfqpoint{0.778184in}{1.139005in}}%
\pgfpathlineto{\pgfqpoint{0.801616in}{0.701688in}}%
\pgfpathlineto{\pgfqpoint{0.825048in}{0.677761in}}%
\pgfpathlineto{\pgfqpoint{0.848479in}{0.841081in}}%
\pgfpathlineto{\pgfqpoint{0.871911in}{1.013748in}}%
\pgfpathlineto{\pgfqpoint{0.895343in}{1.127363in}}%
\pgfpathlineto{\pgfqpoint{0.918775in}{0.821442in}}%
\pgfpathlineto{\pgfqpoint{0.942206in}{0.862287in}}%
\pgfpathlineto{\pgfqpoint{0.965638in}{1.043426in}}%
\pgfpathlineto{\pgfqpoint{0.989070in}{1.177447in}}%
\pgfpathlineto{\pgfqpoint{1.012501in}{1.187144in}}%
\pgfpathlineto{\pgfqpoint{1.035933in}{1.005888in}}%
\pgfpathlineto{\pgfqpoint{1.059365in}{1.085790in}}%
\pgfpathlineto{\pgfqpoint{1.082797in}{1.205720in}}%
\pgfpathlineto{\pgfqpoint{1.106228in}{1.220340in}}%
\pgfpathlineto{\pgfqpoint{1.129660in}{1.304635in}}%
\pgfpathlineto{\pgfqpoint{1.153092in}{1.180740in}}%
\pgfpathlineto{\pgfqpoint{1.176524in}{1.245570in}}%
\pgfpathlineto{\pgfqpoint{1.199955in}{1.262503in}}%
\pgfpathlineto{\pgfqpoint{1.223387in}{1.341682in}}%
\pgfpathlineto{\pgfqpoint{1.246819in}{1.404435in}}%
\pgfpathlineto{\pgfqpoint{1.270250in}{1.324470in}}%
\pgfpathlineto{\pgfqpoint{1.293682in}{1.286793in}}%
\pgfpathlineto{\pgfqpoint{1.317114in}{1.388275in}}%
\pgfpathlineto{\pgfqpoint{1.340546in}{1.438000in}}%
\pgfpathlineto{\pgfqpoint{1.363977in}{1.505035in}}%
\pgfpathlineto{\pgfqpoint{1.387409in}{1.359434in}}%
\pgfpathlineto{\pgfqpoint{1.410841in}{1.392137in}}%
\pgfpathlineto{\pgfqpoint{1.434273in}{1.436163in}}%
\pgfpathlineto{\pgfqpoint{1.457704in}{1.544653in}}%
\pgfpathlineto{\pgfqpoint{1.481136in}{1.541655in}}%
\pgfpathlineto{\pgfqpoint{1.504568in}{1.448533in}}%
\pgfpathlineto{\pgfqpoint{1.527999in}{1.490891in}}%
\pgfpathlineto{\pgfqpoint{1.551431in}{1.558227in}}%
\pgfpathlineto{\pgfqpoint{1.574863in}{1.574279in}}%
\pgfpathlineto{\pgfqpoint{1.598295in}{1.588048in}}%
\pgfpathlineto{\pgfqpoint{1.621726in}{1.540349in}}%
\pgfpathlineto{\pgfqpoint{1.645158in}{1.568872in}}%
\pgfpathlineto{\pgfqpoint{1.668590in}{1.580383in}}%
\pgfpathlineto{\pgfqpoint{1.692021in}{1.602636in}}%
\pgfpathlineto{\pgfqpoint{1.715453in}{1.686336in}}%
\pgfpathlineto{\pgfqpoint{1.738885in}{1.610204in}}%
\pgfpathlineto{\pgfqpoint{1.762317in}{1.582121in}}%
\pgfpathlineto{\pgfqpoint{1.785748in}{1.604242in}}%
\pgfpathlineto{\pgfqpoint{1.809180in}{1.712477in}}%
\pgfpathlineto{\pgfqpoint{1.832612in}{1.731738in}}%
\pgfpathlineto{\pgfqpoint{1.856044in}{1.643107in}}%
\pgfpathlineto{\pgfqpoint{1.879475in}{1.680647in}}%
\pgfpathlineto{\pgfqpoint{1.902907in}{1.686242in}}%
\pgfpathlineto{\pgfqpoint{1.926339in}{1.747106in}}%
\pgfpathlineto{\pgfqpoint{1.949770in}{1.732770in}}%
\pgfpathlineto{\pgfqpoint{1.973202in}{1.661705in}}%
\pgfpathlineto{\pgfqpoint{1.996634in}{1.748332in}}%
\pgfpathlineto{\pgfqpoint{2.020066in}{1.793631in}}%
\pgfpathlineto{\pgfqpoint{2.043497in}{1.779891in}}%
\pgfpathlineto{\pgfqpoint{2.066929in}{1.790684in}}%
\pgfpathlineto{\pgfqpoint{2.090361in}{1.793857in}}%
\pgfpathlineto{\pgfqpoint{2.113793in}{1.754308in}}%
\pgfpathlineto{\pgfqpoint{2.137224in}{1.799629in}}%
\pgfpathlineto{\pgfqpoint{2.160656in}{1.815561in}}%
\pgfpathlineto{\pgfqpoint{2.184088in}{1.866764in}}%
\pgfpathlineto{\pgfqpoint{2.207519in}{1.847060in}}%
\pgfpathlineto{\pgfqpoint{2.230951in}{1.805846in}}%
\pgfpathlineto{\pgfqpoint{2.254383in}{1.828831in}}%
\pgfpathlineto{\pgfqpoint{2.277815in}{1.892412in}}%
\pgfpathlineto{\pgfqpoint{2.301246in}{1.897059in}}%
\pgfpathlineto{\pgfqpoint{2.324678in}{1.840155in}}%
\pgfpathlineto{\pgfqpoint{2.348110in}{1.829105in}}%
\pgfpathlineto{\pgfqpoint{2.371542in}{1.906264in}}%
\pgfpathlineto{\pgfqpoint{2.394973in}{1.889339in}}%
\pgfpathlineto{\pgfqpoint{2.418405in}{1.881767in}}%
\pgfpathlineto{\pgfqpoint{2.441837in}{1.850865in}}%
\pgfpathlineto{\pgfqpoint{2.465268in}{1.919589in}}%
\pgfpathlineto{\pgfqpoint{2.488700in}{1.898881in}}%
\pgfpathlineto{\pgfqpoint{2.512132in}{1.937930in}}%
\pgfpathlineto{\pgfqpoint{2.535564in}{1.983995in}}%
\pgfpathlineto{\pgfqpoint{2.558995in}{1.934028in}}%
\pgfpathlineto{\pgfqpoint{2.582427in}{1.987174in}}%
\pgfpathlineto{\pgfqpoint{2.605859in}{1.926960in}}%
\pgfpathlineto{\pgfqpoint{2.629291in}{1.955420in}}%
\pgfpathlineto{\pgfqpoint{2.652722in}{2.066657in}}%
\pgfpathlineto{\pgfqpoint{2.676154in}{2.091864in}}%
\pgfpathlineto{\pgfqpoint{2.699586in}{1.915393in}}%
\pgfpathlineto{\pgfqpoint{2.723017in}{1.928355in}}%
\pgfpathlineto{\pgfqpoint{2.746449in}{2.059273in}}%
\pgfpathlineto{\pgfqpoint{2.769881in}{2.038566in}}%
\pgfpathlineto{\pgfqpoint{2.793313in}{2.000564in}}%
\pgfpathlineto{\pgfqpoint{2.816744in}{2.109778in}}%
\pgfpathlineto{\pgfqpoint{2.840176in}{1.961226in}}%
\pgfpathlineto{\pgfqpoint{2.863608in}{1.990267in}}%
\pgfpathlineto{\pgfqpoint{2.887039in}{2.072190in}}%
\pgfpathlineto{\pgfqpoint{2.910471in}{1.979005in}}%
\pgfpathlineto{\pgfqpoint{2.957335in}{2.068293in}}%
\pgfpathlineto{\pgfqpoint{2.980766in}{1.927996in}}%
\pgfpathlineto{\pgfqpoint{3.004198in}{2.001032in}}%
\pgfpathlineto{\pgfqpoint{3.051062in}{2.055725in}}%
\pgfpathlineto{\pgfqpoint{3.074493in}{2.099017in}}%
\pgfpathlineto{\pgfqpoint{3.097925in}{2.106554in}}%
\pgfpathlineto{\pgfqpoint{3.121357in}{2.020753in}}%
\pgfpathlineto{\pgfqpoint{3.144788in}{2.023410in}}%
\pgfpathlineto{\pgfqpoint{3.168220in}{1.990004in}}%
\pgfpathlineto{\pgfqpoint{3.191652in}{2.015615in}}%
\pgfpathlineto{\pgfqpoint{3.215084in}{2.341131in}}%
\pgfpathlineto{\pgfqpoint{3.238515in}{2.049062in}}%
\pgfpathlineto{\pgfqpoint{3.261947in}{2.009401in}}%
\pgfpathlineto{\pgfqpoint{3.285379in}{2.016303in}}%
\pgfpathlineto{\pgfqpoint{3.308811in}{2.148906in}}%
\pgfpathlineto{\pgfqpoint{3.332242in}{2.079029in}}%
\pgfpathlineto{\pgfqpoint{3.355674in}{2.054051in}}%
\pgfpathlineto{\pgfqpoint{3.379106in}{2.049849in}}%
\pgfpathlineto{\pgfqpoint{3.402537in}{2.069251in}}%
\pgfpathlineto{\pgfqpoint{3.425969in}{2.086900in}}%
\pgfpathlineto{\pgfqpoint{3.449401in}{2.057673in}}%
\pgfpathlineto{\pgfqpoint{3.472833in}{2.150047in}}%
\pgfpathlineto{\pgfqpoint{3.496264in}{2.091864in}}%
\pgfpathlineto{\pgfqpoint{3.519696in}{2.108369in}}%
\pgfpathlineto{\pgfqpoint{3.543128in}{2.106554in}}%
\pgfpathlineto{\pgfqpoint{3.566560in}{2.102909in}}%
\pgfpathlineto{\pgfqpoint{3.589991in}{2.184024in}}%
\pgfpathlineto{\pgfqpoint{3.613423in}{2.135063in}}%
\pgfpathlineto{\pgfqpoint{3.636855in}{2.110180in}}%
\pgfpathlineto{\pgfqpoint{3.660286in}{2.116183in}}%
\pgfpathlineto{\pgfqpoint{3.707150in}{2.152325in}}%
\pgfpathlineto{\pgfqpoint{3.730582in}{2.125685in}}%
\pgfpathlineto{\pgfqpoint{3.777445in}{2.162487in}}%
\pgfpathlineto{\pgfqpoint{3.824309in}{2.155728in}}%
\pgfpathlineto{\pgfqpoint{3.918035in}{2.162487in}}%
\pgfpathlineto{\pgfqpoint{3.918035in}{2.162487in}}%
\pgfusepath{stroke}%
\end{pgfscope}%
\begin{pgfscope}%
\pgfpathrectangle{\pgfqpoint{0.588387in}{0.521603in}}{\pgfqpoint{3.660036in}{2.220246in}}%
\pgfusepath{clip}%
\pgfsetrectcap%
\pgfsetroundjoin%
\pgfsetlinewidth{1.505625pt}%
\pgfsetstrokecolor{currentstroke5}%
\pgfsetdash{}{0pt}%
\pgfpathmoveto{\pgfqpoint{0.754752in}{0.819448in}}%
\pgfpathlineto{\pgfqpoint{0.778184in}{1.122976in}}%
\pgfpathlineto{\pgfqpoint{0.801616in}{0.697089in}}%
\pgfpathlineto{\pgfqpoint{0.825048in}{0.682630in}}%
\pgfpathlineto{\pgfqpoint{0.848479in}{0.851381in}}%
\pgfpathlineto{\pgfqpoint{0.871911in}{1.014335in}}%
\pgfpathlineto{\pgfqpoint{0.895343in}{1.135282in}}%
\pgfpathlineto{\pgfqpoint{0.918775in}{0.830392in}}%
\pgfpathlineto{\pgfqpoint{0.942206in}{0.898188in}}%
\pgfpathlineto{\pgfqpoint{0.965638in}{1.059232in}}%
\pgfpathlineto{\pgfqpoint{0.989070in}{1.199583in}}%
\pgfpathlineto{\pgfqpoint{1.012501in}{1.243153in}}%
\pgfpathlineto{\pgfqpoint{1.035933in}{1.023141in}}%
\pgfpathlineto{\pgfqpoint{1.059365in}{1.100215in}}%
\pgfpathlineto{\pgfqpoint{1.082797in}{1.221780in}}%
\pgfpathlineto{\pgfqpoint{1.106228in}{1.251357in}}%
\pgfpathlineto{\pgfqpoint{1.129660in}{1.347915in}}%
\pgfpathlineto{\pgfqpoint{1.153092in}{1.194363in}}%
\pgfpathlineto{\pgfqpoint{1.176524in}{1.252942in}}%
\pgfpathlineto{\pgfqpoint{1.199955in}{1.283686in}}%
\pgfpathlineto{\pgfqpoint{1.223387in}{1.378794in}}%
\pgfpathlineto{\pgfqpoint{1.246819in}{1.451559in}}%
\pgfpathlineto{\pgfqpoint{1.270250in}{1.331204in}}%
\pgfpathlineto{\pgfqpoint{1.293682in}{1.315216in}}%
\pgfpathlineto{\pgfqpoint{1.317114in}{1.402516in}}%
\pgfpathlineto{\pgfqpoint{1.340546in}{1.482262in}}%
\pgfpathlineto{\pgfqpoint{1.363977in}{1.558559in}}%
\pgfpathlineto{\pgfqpoint{1.387409in}{1.403934in}}%
\pgfpathlineto{\pgfqpoint{1.410841in}{1.421556in}}%
\pgfpathlineto{\pgfqpoint{1.434273in}{1.491641in}}%
\pgfpathlineto{\pgfqpoint{1.457704in}{1.582400in}}%
\pgfpathlineto{\pgfqpoint{1.481136in}{1.612422in}}%
\pgfpathlineto{\pgfqpoint{1.504568in}{1.487362in}}%
\pgfpathlineto{\pgfqpoint{1.527999in}{1.524253in}}%
\pgfpathlineto{\pgfqpoint{1.551431in}{1.587797in}}%
\pgfpathlineto{\pgfqpoint{1.574863in}{1.603186in}}%
\pgfpathlineto{\pgfqpoint{1.598295in}{1.650680in}}%
\pgfpathlineto{\pgfqpoint{1.621726in}{1.581091in}}%
\pgfpathlineto{\pgfqpoint{1.645158in}{1.644327in}}%
\pgfpathlineto{\pgfqpoint{1.668590in}{1.624351in}}%
\pgfpathlineto{\pgfqpoint{1.692021in}{1.670972in}}%
\pgfpathlineto{\pgfqpoint{1.715453in}{1.759120in}}%
\pgfpathlineto{\pgfqpoint{1.738885in}{1.664564in}}%
\pgfpathlineto{\pgfqpoint{1.762317in}{1.637495in}}%
\pgfpathlineto{\pgfqpoint{1.785748in}{1.690712in}}%
\pgfpathlineto{\pgfqpoint{1.809180in}{1.766372in}}%
\pgfpathlineto{\pgfqpoint{1.832612in}{1.827321in}}%
\pgfpathlineto{\pgfqpoint{1.856044in}{1.706658in}}%
\pgfpathlineto{\pgfqpoint{1.879475in}{1.730341in}}%
\pgfpathlineto{\pgfqpoint{1.902907in}{1.791623in}}%
\pgfpathlineto{\pgfqpoint{1.926339in}{1.807344in}}%
\pgfpathlineto{\pgfqpoint{1.949770in}{1.842199in}}%
\pgfpathlineto{\pgfqpoint{1.973202in}{1.744606in}}%
\pgfpathlineto{\pgfqpoint{1.996634in}{1.794316in}}%
\pgfpathlineto{\pgfqpoint{2.020066in}{1.932640in}}%
\pgfpathlineto{\pgfqpoint{2.043497in}{1.849072in}}%
\pgfpathlineto{\pgfqpoint{2.066929in}{1.858987in}}%
\pgfpathlineto{\pgfqpoint{2.090361in}{1.841733in}}%
\pgfpathlineto{\pgfqpoint{2.113793in}{1.840418in}}%
\pgfpathlineto{\pgfqpoint{2.137224in}{1.840531in}}%
\pgfpathlineto{\pgfqpoint{2.160656in}{1.914714in}}%
\pgfpathlineto{\pgfqpoint{2.184088in}{1.918146in}}%
\pgfpathlineto{\pgfqpoint{2.207519in}{1.866954in}}%
\pgfpathlineto{\pgfqpoint{2.230951in}{1.896451in}}%
\pgfpathlineto{\pgfqpoint{2.254383in}{1.912694in}}%
\pgfpathlineto{\pgfqpoint{2.277815in}{1.984303in}}%
\pgfpathlineto{\pgfqpoint{2.301246in}{1.979491in}}%
\pgfpathlineto{\pgfqpoint{2.324678in}{1.944689in}}%
\pgfpathlineto{\pgfqpoint{2.348110in}{1.914674in}}%
\pgfpathlineto{\pgfqpoint{2.371542in}{2.054330in}}%
\pgfpathlineto{\pgfqpoint{2.394973in}{2.016903in}}%
\pgfpathlineto{\pgfqpoint{2.418405in}{1.986147in}}%
\pgfpathlineto{\pgfqpoint{2.441837in}{1.932061in}}%
\pgfpathlineto{\pgfqpoint{2.465268in}{2.016028in}}%
\pgfpathlineto{\pgfqpoint{2.488700in}{1.979672in}}%
\pgfpathlineto{\pgfqpoint{2.512132in}{2.112887in}}%
\pgfpathlineto{\pgfqpoint{2.535564in}{2.037523in}}%
\pgfpathlineto{\pgfqpoint{2.558995in}{2.006274in}}%
\pgfpathlineto{\pgfqpoint{2.582427in}{2.140863in}}%
\pgfpathlineto{\pgfqpoint{2.605859in}{2.062520in}}%
\pgfpathlineto{\pgfqpoint{2.629291in}{2.102909in}}%
\pgfpathlineto{\pgfqpoint{2.652722in}{2.202075in}}%
\pgfpathlineto{\pgfqpoint{2.676154in}{2.185654in}}%
\pgfpathlineto{\pgfqpoint{2.699586in}{2.049357in}}%
\pgfpathlineto{\pgfqpoint{2.723017in}{2.063485in}}%
\pgfpathlineto{\pgfqpoint{2.746449in}{2.252972in}}%
\pgfpathlineto{\pgfqpoint{2.769881in}{2.200960in}}%
\pgfpathlineto{\pgfqpoint{2.793313in}{2.065412in}}%
\pgfpathlineto{\pgfqpoint{2.816744in}{2.186739in}}%
\pgfpathlineto{\pgfqpoint{2.840176in}{2.073070in}}%
\pgfpathlineto{\pgfqpoint{2.863608in}{2.131561in}}%
\pgfpathlineto{\pgfqpoint{2.887039in}{2.200362in}}%
\pgfpathlineto{\pgfqpoint{2.910471in}{2.104278in}}%
\pgfpathlineto{\pgfqpoint{2.933903in}{2.113788in}}%
\pgfpathlineto{\pgfqpoint{2.957335in}{2.196162in}}%
\pgfpathlineto{\pgfqpoint{3.004198in}{2.388638in}}%
\pgfpathlineto{\pgfqpoint{3.027630in}{2.598157in}}%
\pgfpathlineto{\pgfqpoint{3.051062in}{2.210463in}}%
\pgfpathlineto{\pgfqpoint{3.074493in}{2.329796in}}%
\pgfpathlineto{\pgfqpoint{3.097925in}{2.309519in}}%
\pgfpathlineto{\pgfqpoint{3.215084in}{2.571981in}}%
\pgfpathlineto{\pgfqpoint{3.308811in}{2.242013in}}%
\pgfpathlineto{\pgfqpoint{3.472833in}{2.424575in}}%
\pgfpathlineto{\pgfqpoint{3.566560in}{2.452085in}}%
\pgfusepath{stroke}%
\end{pgfscope}%
\begin{pgfscope}%
\pgfpathrectangle{\pgfqpoint{0.588387in}{0.521603in}}{\pgfqpoint{3.660036in}{2.220246in}}%
\pgfusepath{clip}%
\pgfsetrectcap%
\pgfsetroundjoin%
\pgfsetlinewidth{1.505625pt}%
\pgfsetstrokecolor{currentstroke6}%
\pgfsetdash{}{0pt}%
\pgfpathmoveto{\pgfqpoint{0.754752in}{0.808740in}}%
\pgfpathlineto{\pgfqpoint{0.778184in}{1.131751in}}%
\pgfpathlineto{\pgfqpoint{0.801616in}{0.697066in}}%
\pgfpathlineto{\pgfqpoint{0.825048in}{0.682834in}}%
\pgfpathlineto{\pgfqpoint{0.848479in}{0.851938in}}%
\pgfpathlineto{\pgfqpoint{0.871911in}{1.014335in}}%
\pgfpathlineto{\pgfqpoint{0.895343in}{1.134721in}}%
\pgfpathlineto{\pgfqpoint{0.918775in}{0.832950in}}%
\pgfpathlineto{\pgfqpoint{0.942206in}{0.898542in}}%
\pgfpathlineto{\pgfqpoint{0.965638in}{1.062293in}}%
\pgfpathlineto{\pgfqpoint{0.989070in}{1.202842in}}%
\pgfpathlineto{\pgfqpoint{1.012501in}{1.243549in}}%
\pgfpathlineto{\pgfqpoint{1.035933in}{1.024069in}}%
\pgfpathlineto{\pgfqpoint{1.059365in}{1.100215in}}%
\pgfpathlineto{\pgfqpoint{1.082797in}{1.220955in}}%
\pgfpathlineto{\pgfqpoint{1.106228in}{1.253083in}}%
\pgfpathlineto{\pgfqpoint{1.129660in}{1.345961in}}%
\pgfpathlineto{\pgfqpoint{1.153092in}{1.194363in}}%
\pgfpathlineto{\pgfqpoint{1.176524in}{1.254757in}}%
\pgfpathlineto{\pgfqpoint{1.199955in}{1.285980in}}%
\pgfpathlineto{\pgfqpoint{1.223387in}{1.375259in}}%
\pgfpathlineto{\pgfqpoint{1.246819in}{1.451903in}}%
\pgfpathlineto{\pgfqpoint{1.270250in}{1.336467in}}%
\pgfpathlineto{\pgfqpoint{1.293682in}{1.316379in}}%
\pgfpathlineto{\pgfqpoint{1.317114in}{1.402168in}}%
\pgfpathlineto{\pgfqpoint{1.340546in}{1.480182in}}%
\pgfpathlineto{\pgfqpoint{1.363977in}{1.548379in}}%
\pgfpathlineto{\pgfqpoint{1.387409in}{1.398823in}}%
\pgfpathlineto{\pgfqpoint{1.410841in}{1.421339in}}%
\pgfpathlineto{\pgfqpoint{1.434273in}{1.491641in}}%
\pgfpathlineto{\pgfqpoint{1.457704in}{1.579441in}}%
\pgfpathlineto{\pgfqpoint{1.481136in}{1.597673in}}%
\pgfpathlineto{\pgfqpoint{1.504568in}{1.493068in}}%
\pgfpathlineto{\pgfqpoint{1.527999in}{1.523019in}}%
\pgfpathlineto{\pgfqpoint{1.551431in}{1.586262in}}%
\pgfpathlineto{\pgfqpoint{1.574863in}{1.608046in}}%
\pgfpathlineto{\pgfqpoint{1.598295in}{1.646283in}}%
\pgfpathlineto{\pgfqpoint{1.621726in}{1.574808in}}%
\pgfpathlineto{\pgfqpoint{1.645158in}{1.612756in}}%
\pgfpathlineto{\pgfqpoint{1.668590in}{1.623996in}}%
\pgfpathlineto{\pgfqpoint{1.692021in}{1.669569in}}%
\pgfpathlineto{\pgfqpoint{1.715453in}{1.739757in}}%
\pgfpathlineto{\pgfqpoint{1.738885in}{1.651002in}}%
\pgfpathlineto{\pgfqpoint{1.762317in}{1.638567in}}%
\pgfpathlineto{\pgfqpoint{1.785748in}{1.690712in}}%
\pgfpathlineto{\pgfqpoint{1.809180in}{1.771839in}}%
\pgfpathlineto{\pgfqpoint{1.832612in}{1.813117in}}%
\pgfpathlineto{\pgfqpoint{1.856044in}{1.704542in}}%
\pgfpathlineto{\pgfqpoint{1.879475in}{1.718686in}}%
\pgfpathlineto{\pgfqpoint{1.902907in}{1.785024in}}%
\pgfpathlineto{\pgfqpoint{1.926339in}{1.807344in}}%
\pgfpathlineto{\pgfqpoint{1.949770in}{1.828065in}}%
\pgfpathlineto{\pgfqpoint{1.973202in}{1.744606in}}%
\pgfpathlineto{\pgfqpoint{1.996634in}{1.785852in}}%
\pgfpathlineto{\pgfqpoint{2.020066in}{1.887090in}}%
\pgfpathlineto{\pgfqpoint{2.043497in}{1.842901in}}%
\pgfpathlineto{\pgfqpoint{2.066929in}{1.858987in}}%
\pgfpathlineto{\pgfqpoint{2.090361in}{1.836109in}}%
\pgfpathlineto{\pgfqpoint{2.113793in}{1.840418in}}%
\pgfpathlineto{\pgfqpoint{2.137224in}{1.840531in}}%
\pgfpathlineto{\pgfqpoint{2.160656in}{1.900577in}}%
\pgfpathlineto{\pgfqpoint{2.184088in}{1.920033in}}%
\pgfpathlineto{\pgfqpoint{2.207519in}{1.864414in}}%
\pgfpathlineto{\pgfqpoint{2.230951in}{1.863459in}}%
\pgfpathlineto{\pgfqpoint{2.254383in}{1.893402in}}%
\pgfpathlineto{\pgfqpoint{2.277815in}{1.968907in}}%
\pgfpathlineto{\pgfqpoint{2.301246in}{1.975141in}}%
\pgfpathlineto{\pgfqpoint{2.324678in}{1.915662in}}%
\pgfpathlineto{\pgfqpoint{2.348110in}{1.914674in}}%
\pgfpathlineto{\pgfqpoint{2.371542in}{2.002834in}}%
\pgfpathlineto{\pgfqpoint{2.394973in}{2.008621in}}%
\pgfpathlineto{\pgfqpoint{2.418405in}{1.986147in}}%
\pgfpathlineto{\pgfqpoint{2.441837in}{1.932061in}}%
\pgfpathlineto{\pgfqpoint{2.465268in}{1.995963in}}%
\pgfpathlineto{\pgfqpoint{2.488700in}{1.979672in}}%
\pgfpathlineto{\pgfqpoint{2.512132in}{2.100164in}}%
\pgfpathlineto{\pgfqpoint{2.535564in}{2.030631in}}%
\pgfpathlineto{\pgfqpoint{2.558995in}{1.991072in}}%
\pgfpathlineto{\pgfqpoint{2.582427in}{2.142018in}}%
\pgfpathlineto{\pgfqpoint{2.605859in}{2.024837in}}%
\pgfpathlineto{\pgfqpoint{2.629291in}{2.078760in}}%
\pgfpathlineto{\pgfqpoint{2.652722in}{2.200322in}}%
\pgfpathlineto{\pgfqpoint{2.676154in}{2.214385in}}%
\pgfpathlineto{\pgfqpoint{2.699586in}{2.031424in}}%
\pgfpathlineto{\pgfqpoint{2.723017in}{2.063485in}}%
\pgfpathlineto{\pgfqpoint{2.746449in}{2.172783in}}%
\pgfpathlineto{\pgfqpoint{2.769881in}{2.179384in}}%
\pgfpathlineto{\pgfqpoint{2.793313in}{2.088144in}}%
\pgfpathlineto{\pgfqpoint{2.816744in}{2.191062in}}%
\pgfpathlineto{\pgfqpoint{2.840176in}{2.073070in}}%
\pgfpathlineto{\pgfqpoint{2.863608in}{2.131561in}}%
\pgfpathlineto{\pgfqpoint{2.887039in}{2.212034in}}%
\pgfpathlineto{\pgfqpoint{2.910471in}{2.132876in}}%
\pgfpathlineto{\pgfqpoint{2.933903in}{2.113788in}}%
\pgfpathlineto{\pgfqpoint{2.957335in}{2.158834in}}%
\pgfpathlineto{\pgfqpoint{3.004198in}{2.175816in}}%
\pgfpathlineto{\pgfqpoint{3.027630in}{2.271516in}}%
\pgfpathlineto{\pgfqpoint{3.051062in}{2.165005in}}%
\pgfpathlineto{\pgfqpoint{3.074493in}{2.264253in}}%
\pgfpathlineto{\pgfqpoint{3.097925in}{2.248011in}}%
\pgfpathlineto{\pgfqpoint{3.215084in}{2.346919in}}%
\pgfpathlineto{\pgfqpoint{3.308811in}{2.242013in}}%
\pgfpathlineto{\pgfqpoint{3.472833in}{2.271516in}}%
\pgfusepath{stroke}%
\end{pgfscope}%
\begin{pgfscope}%
\pgfpathrectangle{\pgfqpoint{0.588387in}{0.521603in}}{\pgfqpoint{3.660036in}{2.220246in}}%
\pgfusepath{clip}%
\pgfsetrectcap%
\pgfsetroundjoin%
\pgfsetlinewidth{1.505625pt}%
\pgfsetstrokecolor{currentstroke7}%
\pgfsetdash{}{0pt}%
\pgfpathmoveto{\pgfqpoint{0.754752in}{0.740045in}}%
\pgfpathlineto{\pgfqpoint{0.778184in}{1.036119in}}%
\pgfpathlineto{\pgfqpoint{0.801616in}{0.707609in}}%
\pgfpathlineto{\pgfqpoint{0.825048in}{0.622524in}}%
\pgfpathlineto{\pgfqpoint{0.848479in}{0.769049in}}%
\pgfpathlineto{\pgfqpoint{0.871911in}{0.931633in}}%
\pgfpathlineto{\pgfqpoint{0.895343in}{1.055643in}}%
\pgfpathlineto{\pgfqpoint{0.918775in}{0.780636in}}%
\pgfpathlineto{\pgfqpoint{0.942206in}{0.834470in}}%
\pgfpathlineto{\pgfqpoint{0.965638in}{0.998423in}}%
\pgfpathlineto{\pgfqpoint{0.989070in}{1.137967in}}%
\pgfpathlineto{\pgfqpoint{1.012501in}{1.140643in}}%
\pgfpathlineto{\pgfqpoint{1.035933in}{0.978029in}}%
\pgfpathlineto{\pgfqpoint{1.059365in}{1.063999in}}%
\pgfpathlineto{\pgfqpoint{1.082797in}{1.179370in}}%
\pgfpathlineto{\pgfqpoint{1.106228in}{1.164664in}}%
\pgfpathlineto{\pgfqpoint{1.129660in}{1.237096in}}%
\pgfpathlineto{\pgfqpoint{1.153092in}{1.158592in}}%
\pgfpathlineto{\pgfqpoint{1.176524in}{1.223873in}}%
\pgfpathlineto{\pgfqpoint{1.199955in}{1.228768in}}%
\pgfpathlineto{\pgfqpoint{1.223387in}{1.291602in}}%
\pgfpathlineto{\pgfqpoint{1.246819in}{1.387499in}}%
\pgfpathlineto{\pgfqpoint{1.270250in}{1.320770in}}%
\pgfpathlineto{\pgfqpoint{1.293682in}{1.265554in}}%
\pgfpathlineto{\pgfqpoint{1.317114in}{1.370072in}}%
\pgfpathlineto{\pgfqpoint{1.340546in}{1.411763in}}%
\pgfpathlineto{\pgfqpoint{1.363977in}{1.488439in}}%
\pgfpathlineto{\pgfqpoint{1.387409in}{1.386514in}}%
\pgfpathlineto{\pgfqpoint{1.410841in}{1.376012in}}%
\pgfpathlineto{\pgfqpoint{1.434273in}{1.434712in}}%
\pgfpathlineto{\pgfqpoint{1.457704in}{1.534736in}}%
\pgfpathlineto{\pgfqpoint{1.481136in}{1.529028in}}%
\pgfpathlineto{\pgfqpoint{1.504568in}{1.480250in}}%
\pgfpathlineto{\pgfqpoint{1.527999in}{1.533211in}}%
\pgfpathlineto{\pgfqpoint{1.551431in}{1.541497in}}%
\pgfpathlineto{\pgfqpoint{1.574863in}{1.563369in}}%
\pgfpathlineto{\pgfqpoint{1.598295in}{1.568260in}}%
\pgfpathlineto{\pgfqpoint{1.621726in}{1.564188in}}%
\pgfpathlineto{\pgfqpoint{1.645158in}{1.595317in}}%
\pgfpathlineto{\pgfqpoint{1.668590in}{1.582948in}}%
\pgfpathlineto{\pgfqpoint{1.692021in}{1.604205in}}%
\pgfpathlineto{\pgfqpoint{1.715453in}{1.693688in}}%
\pgfpathlineto{\pgfqpoint{1.738885in}{1.625904in}}%
\pgfpathlineto{\pgfqpoint{1.762317in}{1.614573in}}%
\pgfpathlineto{\pgfqpoint{1.785748in}{1.636787in}}%
\pgfpathlineto{\pgfqpoint{1.809180in}{1.713942in}}%
\pgfpathlineto{\pgfqpoint{1.832612in}{1.751304in}}%
\pgfpathlineto{\pgfqpoint{1.856044in}{1.671736in}}%
\pgfpathlineto{\pgfqpoint{1.879475in}{1.680089in}}%
\pgfpathlineto{\pgfqpoint{1.902907in}{1.750204in}}%
\pgfpathlineto{\pgfqpoint{1.926339in}{1.768048in}}%
\pgfpathlineto{\pgfqpoint{1.949770in}{1.755900in}}%
\pgfpathlineto{\pgfqpoint{1.973202in}{1.713008in}}%
\pgfpathlineto{\pgfqpoint{1.996634in}{1.759049in}}%
\pgfpathlineto{\pgfqpoint{2.020066in}{1.855244in}}%
\pgfpathlineto{\pgfqpoint{2.043497in}{1.806950in}}%
\pgfpathlineto{\pgfqpoint{2.066929in}{1.820633in}}%
\pgfpathlineto{\pgfqpoint{2.090361in}{1.803591in}}%
\pgfpathlineto{\pgfqpoint{2.113793in}{1.808014in}}%
\pgfpathlineto{\pgfqpoint{2.137224in}{1.824389in}}%
\pgfpathlineto{\pgfqpoint{2.160656in}{1.859627in}}%
\pgfpathlineto{\pgfqpoint{2.184088in}{1.906599in}}%
\pgfpathlineto{\pgfqpoint{2.207519in}{1.877773in}}%
\pgfpathlineto{\pgfqpoint{2.230951in}{1.881854in}}%
\pgfpathlineto{\pgfqpoint{2.254383in}{1.872643in}}%
\pgfpathlineto{\pgfqpoint{2.277815in}{1.921170in}}%
\pgfpathlineto{\pgfqpoint{2.301246in}{1.943756in}}%
\pgfpathlineto{\pgfqpoint{2.324678in}{1.896086in}}%
\pgfpathlineto{\pgfqpoint{2.348110in}{1.895397in}}%
\pgfpathlineto{\pgfqpoint{2.371542in}{2.157666in}}%
\pgfpathlineto{\pgfqpoint{2.394973in}{1.963019in}}%
\pgfpathlineto{\pgfqpoint{2.418405in}{1.981762in}}%
\pgfpathlineto{\pgfqpoint{2.441837in}{1.948158in}}%
\pgfpathlineto{\pgfqpoint{2.465268in}{1.946045in}}%
\pgfpathlineto{\pgfqpoint{2.488700in}{1.990194in}}%
\pgfpathlineto{\pgfqpoint{2.512132in}{2.007173in}}%
\pgfpathlineto{\pgfqpoint{2.535564in}{2.026224in}}%
\pgfpathlineto{\pgfqpoint{2.558995in}{2.010335in}}%
\pgfpathlineto{\pgfqpoint{2.582427in}{2.022884in}}%
\pgfpathlineto{\pgfqpoint{2.605859in}{1.981263in}}%
\pgfpathlineto{\pgfqpoint{2.629291in}{2.020708in}}%
\pgfpathlineto{\pgfqpoint{2.652722in}{2.086635in}}%
\pgfpathlineto{\pgfqpoint{2.676154in}{2.102387in}}%
\pgfpathlineto{\pgfqpoint{2.699586in}{2.051533in}}%
\pgfpathlineto{\pgfqpoint{2.723017in}{2.077556in}}%
\pgfpathlineto{\pgfqpoint{2.746449in}{2.083805in}}%
\pgfpathlineto{\pgfqpoint{2.769881in}{2.087805in}}%
\pgfpathlineto{\pgfqpoint{2.793313in}{2.106303in}}%
\pgfpathlineto{\pgfqpoint{2.816744in}{2.117703in}}%
\pgfpathlineto{\pgfqpoint{2.840176in}{2.058968in}}%
\pgfpathlineto{\pgfqpoint{2.863608in}{2.112478in}}%
\pgfpathlineto{\pgfqpoint{2.887039in}{2.129406in}}%
\pgfpathlineto{\pgfqpoint{2.910471in}{2.117684in}}%
\pgfpathlineto{\pgfqpoint{2.933903in}{2.108006in}}%
\pgfpathlineto{\pgfqpoint{2.957335in}{2.133064in}}%
\pgfpathlineto{\pgfqpoint{2.980766in}{2.107408in}}%
\pgfpathlineto{\pgfqpoint{3.004198in}{2.196602in}}%
\pgfpathlineto{\pgfqpoint{3.027630in}{2.194134in}}%
\pgfpathlineto{\pgfqpoint{3.051062in}{2.206226in}}%
\pgfpathlineto{\pgfqpoint{3.074493in}{2.198108in}}%
\pgfpathlineto{\pgfqpoint{3.097925in}{2.184400in}}%
\pgfpathlineto{\pgfqpoint{3.121357in}{2.174400in}}%
\pgfpathlineto{\pgfqpoint{3.144788in}{2.168480in}}%
\pgfpathlineto{\pgfqpoint{3.168220in}{2.144978in}}%
\pgfpathlineto{\pgfqpoint{3.191652in}{2.239667in}}%
\pgfpathlineto{\pgfqpoint{3.215084in}{2.497511in}}%
\pgfpathlineto{\pgfqpoint{3.238515in}{2.218249in}}%
\pgfpathlineto{\pgfqpoint{3.261947in}{2.207434in}}%
\pgfpathlineto{\pgfqpoint{3.285379in}{2.178352in}}%
\pgfpathlineto{\pgfqpoint{3.308811in}{2.238131in}}%
\pgfpathlineto{\pgfqpoint{3.332242in}{2.238188in}}%
\pgfpathlineto{\pgfqpoint{3.355674in}{2.353486in}}%
\pgfpathlineto{\pgfqpoint{3.379106in}{2.196964in}}%
\pgfpathlineto{\pgfqpoint{3.402537in}{2.227308in}}%
\pgfpathlineto{\pgfqpoint{3.425969in}{2.254229in}}%
\pgfpathlineto{\pgfqpoint{3.449401in}{2.262060in}}%
\pgfpathlineto{\pgfqpoint{3.472833in}{2.267564in}}%
\pgfpathlineto{\pgfqpoint{3.496264in}{2.263766in}}%
\pgfpathlineto{\pgfqpoint{3.519696in}{2.277911in}}%
\pgfpathlineto{\pgfqpoint{3.543128in}{2.298593in}}%
\pgfpathlineto{\pgfqpoint{3.566560in}{2.424575in}}%
\pgfpathlineto{\pgfqpoint{3.589991in}{2.370937in}}%
\pgfpathlineto{\pgfqpoint{3.613423in}{2.277848in}}%
\pgfpathlineto{\pgfqpoint{3.636855in}{2.287370in}}%
\pgfpathlineto{\pgfqpoint{3.660286in}{2.280656in}}%
\pgfpathlineto{\pgfqpoint{3.683718in}{2.410415in}}%
\pgfpathlineto{\pgfqpoint{3.707150in}{2.412200in}}%
\pgfpathlineto{\pgfqpoint{3.730582in}{2.392558in}}%
\pgfpathlineto{\pgfqpoint{3.754013in}{2.408027in}}%
\pgfpathlineto{\pgfqpoint{3.777445in}{2.353397in}}%
\pgfpathlineto{\pgfqpoint{3.800877in}{2.484341in}}%
\pgfpathlineto{\pgfqpoint{3.824309in}{2.436165in}}%
\pgfpathlineto{\pgfqpoint{3.847740in}{2.400817in}}%
\pgfpathlineto{\pgfqpoint{3.871172in}{2.361095in}}%
\pgfpathlineto{\pgfqpoint{3.894604in}{2.435398in}}%
\pgfpathlineto{\pgfqpoint{3.918035in}{2.552683in}}%
\pgfpathlineto{\pgfqpoint{3.941467in}{2.575779in}}%
\pgfpathlineto{\pgfqpoint{3.964899in}{2.532853in}}%
\pgfpathlineto{\pgfqpoint{3.988331in}{2.528820in}}%
\pgfpathlineto{\pgfqpoint{4.011762in}{2.640929in}}%
\pgfpathlineto{\pgfqpoint{4.035194in}{2.575779in}}%
\pgfpathlineto{\pgfqpoint{4.082057in}{2.544816in}}%
\pgfpathlineto{\pgfqpoint{4.082057in}{2.544816in}}%
\pgfusepath{stroke}%
\end{pgfscope}%
\begin{pgfscope}%
\pgfpathrectangle{\pgfqpoint{0.588387in}{0.521603in}}{\pgfqpoint{3.660036in}{2.220246in}}%
\pgfusepath{clip}%
\pgfsetrectcap%
\pgfsetroundjoin%
\pgfsetlinewidth{1.505625pt}%
\definecolor{currentstroke}{rgb}{0.498039,0.498039,0.498039}%
\pgfsetstrokecolor{currentstroke}%
\pgfsetdash{}{0pt}%
\pgfpathmoveto{\pgfqpoint{0.754752in}{0.827711in}}%
\pgfpathlineto{\pgfqpoint{0.778184in}{1.131304in}}%
\pgfpathlineto{\pgfqpoint{0.801616in}{0.707708in}}%
\pgfpathlineto{\pgfqpoint{0.825048in}{0.681736in}}%
\pgfpathlineto{\pgfqpoint{0.848479in}{0.859341in}}%
\pgfpathlineto{\pgfqpoint{0.871911in}{1.015508in}}%
\pgfpathlineto{\pgfqpoint{0.895343in}{1.129132in}}%
\pgfpathlineto{\pgfqpoint{0.918775in}{0.824744in}}%
\pgfpathlineto{\pgfqpoint{0.942206in}{0.864311in}}%
\pgfpathlineto{\pgfqpoint{0.965638in}{1.046801in}}%
\pgfpathlineto{\pgfqpoint{0.989070in}{1.188128in}}%
\pgfpathlineto{\pgfqpoint{1.012501in}{1.205349in}}%
\pgfpathlineto{\pgfqpoint{1.035933in}{1.009379in}}%
\pgfpathlineto{\pgfqpoint{1.059365in}{1.095469in}}%
\pgfpathlineto{\pgfqpoint{1.082797in}{1.210423in}}%
\pgfpathlineto{\pgfqpoint{1.106228in}{1.226421in}}%
\pgfpathlineto{\pgfqpoint{1.129660in}{1.311560in}}%
\pgfpathlineto{\pgfqpoint{1.153092in}{1.187748in}}%
\pgfpathlineto{\pgfqpoint{1.176524in}{1.249857in}}%
\pgfpathlineto{\pgfqpoint{1.199955in}{1.276857in}}%
\pgfpathlineto{\pgfqpoint{1.223387in}{1.327389in}}%
\pgfpathlineto{\pgfqpoint{1.246819in}{1.430061in}}%
\pgfpathlineto{\pgfqpoint{1.270250in}{1.341316in}}%
\pgfpathlineto{\pgfqpoint{1.293682in}{1.303937in}}%
\pgfpathlineto{\pgfqpoint{1.317114in}{1.408979in}}%
\pgfpathlineto{\pgfqpoint{1.340546in}{1.441711in}}%
\pgfpathlineto{\pgfqpoint{1.363977in}{1.512779in}}%
\pgfpathlineto{\pgfqpoint{1.387409in}{1.379730in}}%
\pgfpathlineto{\pgfqpoint{1.410841in}{1.406825in}}%
\pgfpathlineto{\pgfqpoint{1.434273in}{1.439721in}}%
\pgfpathlineto{\pgfqpoint{1.457704in}{1.554623in}}%
\pgfpathlineto{\pgfqpoint{1.481136in}{1.539867in}}%
\pgfpathlineto{\pgfqpoint{1.504568in}{1.453726in}}%
\pgfpathlineto{\pgfqpoint{1.527999in}{1.529093in}}%
\pgfpathlineto{\pgfqpoint{1.551431in}{1.556699in}}%
\pgfpathlineto{\pgfqpoint{1.574863in}{1.597297in}}%
\pgfpathlineto{\pgfqpoint{1.598295in}{1.599168in}}%
\pgfpathlineto{\pgfqpoint{1.621726in}{1.554321in}}%
\pgfpathlineto{\pgfqpoint{1.645158in}{1.586769in}}%
\pgfpathlineto{\pgfqpoint{1.668590in}{1.613517in}}%
\pgfpathlineto{\pgfqpoint{1.692021in}{1.616286in}}%
\pgfpathlineto{\pgfqpoint{1.715453in}{1.716777in}}%
\pgfpathlineto{\pgfqpoint{1.738885in}{1.616568in}}%
\pgfpathlineto{\pgfqpoint{1.762317in}{1.599204in}}%
\pgfpathlineto{\pgfqpoint{1.785748in}{1.610545in}}%
\pgfpathlineto{\pgfqpoint{1.809180in}{1.707783in}}%
\pgfpathlineto{\pgfqpoint{1.832612in}{1.751317in}}%
\pgfpathlineto{\pgfqpoint{1.856044in}{1.685511in}}%
\pgfpathlineto{\pgfqpoint{1.879475in}{1.694107in}}%
\pgfpathlineto{\pgfqpoint{1.902907in}{1.693476in}}%
\pgfpathlineto{\pgfqpoint{1.926339in}{1.759969in}}%
\pgfpathlineto{\pgfqpoint{1.949770in}{1.745357in}}%
\pgfpathlineto{\pgfqpoint{1.973202in}{1.672120in}}%
\pgfpathlineto{\pgfqpoint{1.996634in}{1.756317in}}%
\pgfpathlineto{\pgfqpoint{2.020066in}{1.847201in}}%
\pgfpathlineto{\pgfqpoint{2.043497in}{1.803720in}}%
\pgfpathlineto{\pgfqpoint{2.066929in}{1.809602in}}%
\pgfpathlineto{\pgfqpoint{2.090361in}{1.802868in}}%
\pgfpathlineto{\pgfqpoint{2.113793in}{1.762904in}}%
\pgfpathlineto{\pgfqpoint{2.137224in}{1.807788in}}%
\pgfpathlineto{\pgfqpoint{2.160656in}{1.822842in}}%
\pgfpathlineto{\pgfqpoint{2.184088in}{1.908638in}}%
\pgfpathlineto{\pgfqpoint{2.207519in}{1.860326in}}%
\pgfpathlineto{\pgfqpoint{2.230951in}{1.823050in}}%
\pgfpathlineto{\pgfqpoint{2.254383in}{1.836197in}}%
\pgfpathlineto{\pgfqpoint{2.277815in}{1.929658in}}%
\pgfpathlineto{\pgfqpoint{2.301246in}{1.918146in}}%
\pgfpathlineto{\pgfqpoint{2.324678in}{1.862184in}}%
\pgfpathlineto{\pgfqpoint{2.348110in}{1.835314in}}%
\pgfpathlineto{\pgfqpoint{2.371542in}{1.944554in}}%
\pgfpathlineto{\pgfqpoint{2.418405in}{1.891635in}}%
\pgfpathlineto{\pgfqpoint{2.441837in}{1.850865in}}%
\pgfpathlineto{\pgfqpoint{2.465268in}{1.931141in}}%
\pgfpathlineto{\pgfqpoint{2.488700in}{1.894623in}}%
\pgfpathlineto{\pgfqpoint{2.512132in}{1.951054in}}%
\pgfpathlineto{\pgfqpoint{2.535564in}{1.999035in}}%
\pgfpathlineto{\pgfqpoint{2.558995in}{1.958552in}}%
\pgfpathlineto{\pgfqpoint{2.582427in}{1.988665in}}%
\pgfpathlineto{\pgfqpoint{2.605859in}{1.926701in}}%
\pgfpathlineto{\pgfqpoint{2.629291in}{1.968749in}}%
\pgfpathlineto{\pgfqpoint{2.652722in}{2.032284in}}%
\pgfpathlineto{\pgfqpoint{2.676154in}{2.078760in}}%
\pgfpathlineto{\pgfqpoint{2.699586in}{1.974132in}}%
\pgfpathlineto{\pgfqpoint{2.723017in}{1.946918in}}%
\pgfpathlineto{\pgfqpoint{2.746449in}{2.032284in}}%
\pgfpathlineto{\pgfqpoint{2.769881in}{2.036720in}}%
\pgfpathlineto{\pgfqpoint{2.793313in}{2.022684in}}%
\pgfpathlineto{\pgfqpoint{2.816744in}{2.120157in}}%
\pgfpathlineto{\pgfqpoint{2.840176in}{1.968309in}}%
\pgfpathlineto{\pgfqpoint{2.863608in}{1.997461in}}%
\pgfpathlineto{\pgfqpoint{2.887039in}{2.100375in}}%
\pgfpathlineto{\pgfqpoint{2.910471in}{2.034869in}}%
\pgfpathlineto{\pgfqpoint{2.933903in}{2.052205in}}%
\pgfpathlineto{\pgfqpoint{2.957335in}{2.104369in}}%
\pgfpathlineto{\pgfqpoint{2.980766in}{1.944120in}}%
\pgfpathlineto{\pgfqpoint{3.004198in}{2.018497in}}%
\pgfpathlineto{\pgfqpoint{3.027630in}{2.062842in}}%
\pgfpathlineto{\pgfqpoint{3.074493in}{2.102909in}}%
\pgfpathlineto{\pgfqpoint{3.097925in}{2.124504in}}%
\pgfpathlineto{\pgfqpoint{3.121357in}{2.007318in}}%
\pgfpathlineto{\pgfqpoint{3.144788in}{2.029909in}}%
\pgfpathlineto{\pgfqpoint{3.168220in}{1.996810in}}%
\pgfpathlineto{\pgfqpoint{3.191652in}{2.029909in}}%
\pgfpathlineto{\pgfqpoint{3.215084in}{2.430559in}}%
\pgfpathlineto{\pgfqpoint{3.238515in}{2.053771in}}%
\pgfpathlineto{\pgfqpoint{3.261947in}{2.021776in}}%
\pgfpathlineto{\pgfqpoint{3.285379in}{2.029909in}}%
\pgfpathlineto{\pgfqpoint{3.308811in}{2.162487in}}%
\pgfpathlineto{\pgfqpoint{3.332242in}{2.080646in}}%
\pgfpathlineto{\pgfqpoint{3.355674in}{2.108973in}}%
\pgfpathlineto{\pgfqpoint{3.379106in}{2.053771in}}%
\pgfpathlineto{\pgfqpoint{3.425969in}{2.099246in}}%
\pgfpathlineto{\pgfqpoint{3.449401in}{2.069251in}}%
\pgfpathlineto{\pgfqpoint{3.472833in}{2.151472in}}%
\pgfpathlineto{\pgfqpoint{3.496264in}{2.102909in}}%
\pgfpathlineto{\pgfqpoint{3.519696in}{2.120950in}}%
\pgfpathlineto{\pgfqpoint{3.543128in}{2.099246in}}%
\pgfpathlineto{\pgfqpoint{3.566560in}{2.331571in}}%
\pgfpathlineto{\pgfqpoint{3.589991in}{2.165842in}}%
\pgfpathlineto{\pgfqpoint{3.613423in}{2.142018in}}%
\pgfpathlineto{\pgfqpoint{3.636855in}{2.113788in}}%
\pgfpathlineto{\pgfqpoint{3.660286in}{2.120950in}}%
\pgfpathlineto{\pgfqpoint{3.707150in}{2.162487in}}%
\pgfpathlineto{\pgfqpoint{3.730582in}{2.135063in}}%
\pgfpathlineto{\pgfqpoint{3.777445in}{2.162487in}}%
\pgfpathlineto{\pgfqpoint{3.824309in}{2.155728in}}%
\pgfpathlineto{\pgfqpoint{3.894604in}{2.201757in}}%
\pgfpathlineto{\pgfqpoint{3.918035in}{2.175816in}}%
\pgfpathlineto{\pgfqpoint{3.918035in}{2.175816in}}%
\pgfusepath{stroke}%
\end{pgfscope}%
\begin{pgfscope}%
\pgfsetrectcap%
\pgfsetmiterjoin%
\pgfsetlinewidth{0.803000pt}%
\definecolor{currentstroke}{rgb}{0.000000,0.000000,0.000000}%
\pgfsetstrokecolor{currentstroke}%
\pgfsetdash{}{0pt}%
\pgfpathmoveto{\pgfqpoint{0.588387in}{0.521603in}}%
\pgfpathlineto{\pgfqpoint{0.588387in}{2.741849in}}%
\pgfusepath{stroke}%
\end{pgfscope}%
\begin{pgfscope}%
\pgfsetrectcap%
\pgfsetmiterjoin%
\pgfsetlinewidth{0.803000pt}%
\definecolor{currentstroke}{rgb}{0.000000,0.000000,0.000000}%
\pgfsetstrokecolor{currentstroke}%
\pgfsetdash{}{0pt}%
\pgfpathmoveto{\pgfqpoint{4.248423in}{0.521603in}}%
\pgfpathlineto{\pgfqpoint{4.248423in}{2.741849in}}%
\pgfusepath{stroke}%
\end{pgfscope}%
\begin{pgfscope}%
\pgfsetrectcap%
\pgfsetmiterjoin%
\pgfsetlinewidth{0.803000pt}%
\definecolor{currentstroke}{rgb}{0.000000,0.000000,0.000000}%
\pgfsetstrokecolor{currentstroke}%
\pgfsetdash{}{0pt}%
\pgfpathmoveto{\pgfqpoint{0.588387in}{0.521603in}}%
\pgfpathlineto{\pgfqpoint{4.248423in}{0.521603in}}%
\pgfusepath{stroke}%
\end{pgfscope}%
\begin{pgfscope}%
\pgfsetrectcap%
\pgfsetmiterjoin%
\pgfsetlinewidth{0.803000pt}%
\definecolor{currentstroke}{rgb}{0.000000,0.000000,0.000000}%
\pgfsetstrokecolor{currentstroke}%
\pgfsetdash{}{0pt}%
\pgfpathmoveto{\pgfqpoint{0.588387in}{2.741849in}}%
\pgfpathlineto{\pgfqpoint{4.248423in}{2.741849in}}%
\pgfusepath{stroke}%
\end{pgfscope}%
\begin{pgfscope}%
\pgfsetbuttcap%
\pgfsetmiterjoin%
\definecolor{currentfill}{rgb}{1.000000,1.000000,1.000000}%
\pgfsetfillcolor{currentfill}%
\pgfsetfillopacity{0.800000}%
\pgfsetlinewidth{1.003750pt}%
\definecolor{currentstroke}{rgb}{0.800000,0.800000,0.800000}%
\pgfsetstrokecolor{currentstroke}%
\pgfsetstrokeopacity{0.800000}%
\pgfsetdash{}{0pt}%
\pgfpathmoveto{\pgfqpoint{4.365089in}{0.623654in}}%
\pgfpathlineto{\pgfqpoint{8.251043in}{0.623654in}}%
\pgfpathquadraticcurveto{\pgfqpoint{8.284376in}{0.623654in}}{\pgfqpoint{8.284376in}{0.656988in}}%
\pgfpathlineto{\pgfqpoint{8.284376in}{2.625183in}}%
\pgfpathquadraticcurveto{\pgfqpoint{8.284376in}{2.658516in}}{\pgfqpoint{8.251043in}{2.658516in}}%
\pgfpathlineto{\pgfqpoint{4.365089in}{2.658516in}}%
\pgfpathquadraticcurveto{\pgfqpoint{4.331756in}{2.658516in}}{\pgfqpoint{4.331756in}{2.625183in}}%
\pgfpathlineto{\pgfqpoint{4.331756in}{0.656988in}}%
\pgfpathquadraticcurveto{\pgfqpoint{4.331756in}{0.623654in}}{\pgfqpoint{4.365089in}{0.623654in}}%
\pgfpathlineto{\pgfqpoint{4.365089in}{0.623654in}}%
\pgfpathclose%
\pgfusepath{stroke,fill}%
\end{pgfscope}%
\begin{pgfscope}%
\pgfsetrectcap%
\pgfsetroundjoin%
\pgfsetlinewidth{1.505625pt}%
\pgfsetstrokecolor{currentstroke1}%
\pgfsetdash{}{0pt}%
\pgfpathmoveto{\pgfqpoint{4.398423in}{2.523555in}}%
\pgfpathlineto{\pgfqpoint{4.565089in}{2.523555in}}%
\pgfpathlineto{\pgfqpoint{4.731756in}{2.523555in}}%
\pgfusepath{stroke}%
\end{pgfscope}%
\begin{pgfscope}%
\definecolor{textcolor}{rgb}{0.000000,0.000000,0.000000}%
\pgfsetstrokecolor{textcolor}%
\pgfsetfillcolor{textcolor}%
\pgftext[x=4.865089in,y=2.465222in,left,base]{\color{textcolor}{\rmfamily\fontsize{12.000000}{14.400000}\selectfont\catcode`\^=\active\def^{\ifmmode\sp\else\^{}\fi}\catcode`\%=\active\def%{\%}\CyclesMatchChunks{} \& \MergeLinear{}}}%
\end{pgfscope}%
\begin{pgfscope}%
\pgfsetrectcap%
\pgfsetroundjoin%
\pgfsetlinewidth{1.505625pt}%
\pgfsetstrokecolor{currentstroke2}%
\pgfsetdash{}{0pt}%
\pgfpathmoveto{\pgfqpoint{4.398423in}{2.274288in}}%
\pgfpathlineto{\pgfqpoint{4.565089in}{2.274288in}}%
\pgfpathlineto{\pgfqpoint{4.731756in}{2.274288in}}%
\pgfusepath{stroke}%
\end{pgfscope}%
\begin{pgfscope}%
\definecolor{textcolor}{rgb}{0.000000,0.000000,0.000000}%
\pgfsetstrokecolor{textcolor}%
\pgfsetfillcolor{textcolor}%
\pgftext[x=4.865089in,y=2.215954in,left,base]{\color{textcolor}{\rmfamily\fontsize{12.000000}{14.400000}\selectfont\catcode`\^=\active\def^{\ifmmode\sp\else\^{}\fi}\catcode`\%=\active\def%{\%}\CyclesMatchChunks{} \& \SharedVertices{}}}%
\end{pgfscope}%
\begin{pgfscope}%
\pgfsetrectcap%
\pgfsetroundjoin%
\pgfsetlinewidth{1.505625pt}%
\pgfsetstrokecolor{currentstroke3}%
\pgfsetdash{}{0pt}%
\pgfpathmoveto{\pgfqpoint{4.398423in}{2.025020in}}%
\pgfpathlineto{\pgfqpoint{4.565089in}{2.025020in}}%
\pgfpathlineto{\pgfqpoint{4.731756in}{2.025020in}}%
\pgfusepath{stroke}%
\end{pgfscope}%
\begin{pgfscope}%
\definecolor{textcolor}{rgb}{0.000000,0.000000,0.000000}%
\pgfsetstrokecolor{textcolor}%
\pgfsetfillcolor{textcolor}%
\pgftext[x=4.865089in,y=1.966687in,left,base]{\color{textcolor}{\rmfamily\fontsize{12.000000}{14.400000}\selectfont\catcode`\^=\active\def^{\ifmmode\sp\else\^{}\fi}\catcode`\%=\active\def%{\%}\Neighbors{} \& \MergeLinear{}}}%
\end{pgfscope}%
\begin{pgfscope}%
\pgfsetrectcap%
\pgfsetroundjoin%
\pgfsetlinewidth{1.505625pt}%
\pgfsetstrokecolor{currentstroke4}%
\pgfsetdash{}{0pt}%
\pgfpathmoveto{\pgfqpoint{4.398423in}{1.780391in}}%
\pgfpathlineto{\pgfqpoint{4.565089in}{1.780391in}}%
\pgfpathlineto{\pgfqpoint{4.731756in}{1.780391in}}%
\pgfusepath{stroke}%
\end{pgfscope}%
\begin{pgfscope}%
\definecolor{textcolor}{rgb}{0.000000,0.000000,0.000000}%
\pgfsetstrokecolor{textcolor}%
\pgfsetfillcolor{textcolor}%
\pgftext[x=4.865089in,y=1.722058in,left,base]{\color{textcolor}{\rmfamily\fontsize{12.000000}{14.400000}\selectfont\catcode`\^=\active\def^{\ifmmode\sp\else\^{}\fi}\catcode`\%=\active\def%{\%}\Neighbors{} \& \SharedVertices{}}}%
\end{pgfscope}%
\begin{pgfscope}%
\pgfsetrectcap%
\pgfsetroundjoin%
\pgfsetlinewidth{1.505625pt}%
\pgfsetstrokecolor{currentstroke5}%
\pgfsetdash{}{0pt}%
\pgfpathmoveto{\pgfqpoint{4.398423in}{1.531124in}}%
\pgfpathlineto{\pgfqpoint{4.565089in}{1.531124in}}%
\pgfpathlineto{\pgfqpoint{4.731756in}{1.531124in}}%
\pgfusepath{stroke}%
\end{pgfscope}%
\begin{pgfscope}%
\definecolor{textcolor}{rgb}{0.000000,0.000000,0.000000}%
\pgfsetstrokecolor{textcolor}%
\pgfsetfillcolor{textcolor}%
\pgftext[x=4.865089in,y=1.472791in,left,base]{\color{textcolor}{\rmfamily\fontsize{12.000000}{14.400000}\selectfont\catcode`\^=\active\def^{\ifmmode\sp\else\^{}\fi}\catcode`\%=\active\def%{\%}\NeighborsDegree{} \& \MergeLinear{}}}%
\end{pgfscope}%
\begin{pgfscope}%
\pgfsetrectcap%
\pgfsetroundjoin%
\pgfsetlinewidth{1.505625pt}%
\pgfsetstrokecolor{currentstroke6}%
\pgfsetdash{}{0pt}%
\pgfpathmoveto{\pgfqpoint{4.398423in}{1.281857in}}%
\pgfpathlineto{\pgfqpoint{4.565089in}{1.281857in}}%
\pgfpathlineto{\pgfqpoint{4.731756in}{1.281857in}}%
\pgfusepath{stroke}%
\end{pgfscope}%
\begin{pgfscope}%
\definecolor{textcolor}{rgb}{0.000000,0.000000,0.000000}%
\pgfsetstrokecolor{textcolor}%
\pgfsetfillcolor{textcolor}%
\pgftext[x=4.865089in,y=1.223523in,left,base]{\color{textcolor}{\rmfamily\fontsize{12.000000}{14.400000}\selectfont\catcode`\^=\active\def^{\ifmmode\sp\else\^{}\fi}\catcode`\%=\active\def%{\%}\NeighborsDegree{} \& \SharedVertices{}}}%
\end{pgfscope}%
\begin{pgfscope}%
\pgfsetrectcap%
\pgfsetroundjoin%
\pgfsetlinewidth{1.505625pt}%
\pgfsetstrokecolor{currentstroke7}%
\pgfsetdash{}{0pt}%
\pgfpathmoveto{\pgfqpoint{4.398423in}{1.032589in}}%
\pgfpathlineto{\pgfqpoint{4.565089in}{1.032589in}}%
\pgfpathlineto{\pgfqpoint{4.731756in}{1.032589in}}%
\pgfusepath{stroke}%
\end{pgfscope}%
\begin{pgfscope}%
\definecolor{textcolor}{rgb}{0.000000,0.000000,0.000000}%
\pgfsetstrokecolor{textcolor}%
\pgfsetfillcolor{textcolor}%
\pgftext[x=4.865089in,y=0.974256in,left,base]{\color{textcolor}{\rmfamily\fontsize{12.000000}{14.400000}\selectfont\catcode`\^=\active\def^{\ifmmode\sp\else\^{}\fi}\catcode`\%=\active\def%{\%}\None{} \& \MergeLinear{}}}%
\end{pgfscope}%
\begin{pgfscope}%
\pgfsetrectcap%
\pgfsetroundjoin%
\pgfsetlinewidth{1.505625pt}%
\definecolor{currentstroke}{rgb}{0.498039,0.498039,0.498039}%
\pgfsetstrokecolor{currentstroke}%
\pgfsetdash{}{0pt}%
\pgfpathmoveto{\pgfqpoint{4.398423in}{0.787961in}}%
\pgfpathlineto{\pgfqpoint{4.565089in}{0.787961in}}%
\pgfpathlineto{\pgfqpoint{4.731756in}{0.787961in}}%
\pgfusepath{stroke}%
\end{pgfscope}%
\begin{pgfscope}%
\definecolor{textcolor}{rgb}{0.000000,0.000000,0.000000}%
\pgfsetstrokecolor{textcolor}%
\pgfsetfillcolor{textcolor}%
\pgftext[x=4.865089in,y=0.729627in,left,base]{\color{textcolor}{\rmfamily\fontsize{12.000000}{14.400000}\selectfont\catcode`\^=\active\def^{\ifmmode\sp\else\^{}\fi}\catcode`\%=\active\def%{\%}\None{} \& \SharedVertices{}}}%
\end{pgfscope}%
\end{pgfpicture}%
\makeatother%
\endgroup%
}
	\caption[Checks performed for graphs with no NAC-coloring (some).]{
		The number of checks performed to find all NAC-colorings for graphs with no NAC-coloring.}%
	\label{fig:graph_no_nac_coloring_first_checks}
\end{figure}

In \Subgraphs{} algorithm description, an important parameter was the size of subgraphs \( k \).,
almost all the benchmarks in the previous section were run with \( k = 4 \).
Now we show the impact of	\( k \) on runtime and number of checks.
Note that you see averages over all the strategies used for benchmarking
graphs with no NAC-colorings.
From graphs in \Cref{fig:graph_no_nac_coloring_first_runtime_subgraph_size,fig:graph_no_nac_coloring_first_checks_subgraph_size},
it can be seen that the algorithm benefits from smaller \( k \) significantly.
This is also one of the reasons why most of
the strategies for \Subgraphs{} perform so similarly.

\begin{figure}[p]
	\centering
	\scalebox{0.5}{%% Creator: Matplotlib, PGF backend
%%
%% To include the figure in your LaTeX document, write
%%   \input{<filename>.pgf}
%%
%% Make sure the required packages are loaded in your preamble
%%   \usepackage{pgf}
%%
%% Also ensure that all the required font packages are loaded; for instance,
%% the lmodern package is sometimes necessary when using math font.
%%   \usepackage{lmodern}
%%
%% Figures using additional raster images can only be included by \input if
%% they are in the same directory as the main LaTeX file. For loading figures
%% from other directories you can use the `import` package
%%   \usepackage{import}
%%
%% and then include the figures with
%%   \import{<path to file>}{<filename>.pgf}
%%
%% Matplotlib used the following preamble
%%   \def\mathdefault#1{#1}
%%   \everymath=\expandafter{\the\everymath\displaystyle}
%%   \IfFileExists{scrextend.sty}{
%%     \usepackage[fontsize=10.000000pt]{scrextend}
%%   }{
%%     \renewcommand{\normalsize}{\fontsize{10.000000}{12.000000}\selectfont}
%%     \normalsize
%%   }
%%   
%%   \ifdefined\pdftexversion\else  % non-pdftex case.
%%     \usepackage{fontspec}
%%     \setmainfont{DejaVuSans.ttf}[Path=\detokenize{/home/petr/Projects/PyRigi/.venv/lib/python3.12/site-packages/matplotlib/mpl-data/fonts/ttf/}]
%%     \setsansfont{DejaVuSans.ttf}[Path=\detokenize{/home/petr/Projects/PyRigi/.venv/lib/python3.12/site-packages/matplotlib/mpl-data/fonts/ttf/}]
%%     \setmonofont{DejaVuSansMono.ttf}[Path=\detokenize{/home/petr/Projects/PyRigi/.venv/lib/python3.12/site-packages/matplotlib/mpl-data/fonts/ttf/}]
%%   \fi
%%   \makeatletter\@ifpackageloaded{under\Score{}}{}{\usepackage[strings]{under\Score{}}}\makeatother
%%
\begingroup%
\makeatletter%
\begin{pgfpicture}%
\pgfpathrectangle{\pgfpointorigin}{\pgfqpoint{8.384376in}{2.841849in}}%
\pgfusepath{use as bounding box, clip}%
\begin{pgfscope}%
\pgfsetbuttcap%
\pgfsetmiterjoin%
\definecolor{currentfill}{rgb}{1.000000,1.000000,1.000000}%
\pgfsetfillcolor{currentfill}%
\pgfsetlinewidth{0.000000pt}%
\definecolor{currentstroke}{rgb}{1.000000,1.000000,1.000000}%
\pgfsetstrokecolor{currentstroke}%
\pgfsetdash{}{0pt}%
\pgfpathmoveto{\pgfqpoint{0.000000in}{0.000000in}}%
\pgfpathlineto{\pgfqpoint{8.384376in}{0.000000in}}%
\pgfpathlineto{\pgfqpoint{8.384376in}{2.841849in}}%
\pgfpathlineto{\pgfqpoint{0.000000in}{2.841849in}}%
\pgfpathlineto{\pgfqpoint{0.000000in}{0.000000in}}%
\pgfpathclose%
\pgfusepath{fill}%
\end{pgfscope}%
\begin{pgfscope}%
\pgfsetbuttcap%
\pgfsetmiterjoin%
\definecolor{currentfill}{rgb}{1.000000,1.000000,1.000000}%
\pgfsetfillcolor{currentfill}%
\pgfsetlinewidth{0.000000pt}%
\definecolor{currentstroke}{rgb}{0.000000,0.000000,0.000000}%
\pgfsetstrokecolor{currentstroke}%
\pgfsetstrokeopacity{0.000000}%
\pgfsetdash{}{0pt}%
\pgfpathmoveto{\pgfqpoint{0.588387in}{0.521603in}}%
\pgfpathlineto{\pgfqpoint{7.692348in}{0.521603in}}%
\pgfpathlineto{\pgfqpoint{7.692348in}{2.741849in}}%
\pgfpathlineto{\pgfqpoint{0.588387in}{2.741849in}}%
\pgfpathlineto{\pgfqpoint{0.588387in}{0.521603in}}%
\pgfpathclose%
\pgfusepath{fill}%
\end{pgfscope}%
\begin{pgfscope}%
\pgfsetbuttcap%
\pgfsetroundjoin%
\definecolor{currentfill}{rgb}{0.000000,0.000000,0.000000}%
\pgfsetfillcolor{currentfill}%
\pgfsetlinewidth{0.803000pt}%
\definecolor{currentstroke}{rgb}{0.000000,0.000000,0.000000}%
\pgfsetstrokecolor{currentstroke}%
\pgfsetdash{}{0pt}%
\pgfsys@defobject{currentmarker}{\pgfqpoint{0.000000in}{-0.048611in}}{\pgfqpoint{0.000000in}{0.000000in}}{%
\pgfpathmoveto{\pgfqpoint{0.000000in}{0.000000in}}%
\pgfpathlineto{\pgfqpoint{0.000000in}{-0.048611in}}%
\pgfusepath{stroke,fill}%
}%
\begin{pgfscope}%
\pgfsys@transformshift{1.200465in}{0.521603in}%
\pgfsys@useobject{currentmarker}{}%
\end{pgfscope}%
\end{pgfscope}%
\begin{pgfscope}%
\definecolor{textcolor}{rgb}{0.000000,0.000000,0.000000}%
\pgfsetstrokecolor{textcolor}%
\pgfsetfillcolor{textcolor}%
\pgftext[x=1.200465in,y=0.424381in,,top]{\color{textcolor}{\rmfamily\fontsize{10.000000}{12.000000}\selectfont\catcode`\^=\active\def^{\ifmmode\sp\else\^{}\fi}\catcode`\%=\active\def%{\%}$\mathdefault{16}$}}%
\end{pgfscope}%
\begin{pgfscope}%
\pgfsetbuttcap%
\pgfsetroundjoin%
\definecolor{currentfill}{rgb}{0.000000,0.000000,0.000000}%
\pgfsetfillcolor{currentfill}%
\pgfsetlinewidth{0.803000pt}%
\definecolor{currentstroke}{rgb}{0.000000,0.000000,0.000000}%
\pgfsetstrokecolor{currentstroke}%
\pgfsetdash{}{0pt}%
\pgfsys@defobject{currentmarker}{\pgfqpoint{0.000000in}{-0.048611in}}{\pgfqpoint{0.000000in}{0.000000in}}{%
\pgfpathmoveto{\pgfqpoint{0.000000in}{0.000000in}}%
\pgfpathlineto{\pgfqpoint{0.000000in}{-0.048611in}}%
\pgfusepath{stroke,fill}%
}%
\begin{pgfscope}%
\pgfsys@transformshift{1.971587in}{0.521603in}%
\pgfsys@useobject{currentmarker}{}%
\end{pgfscope}%
\end{pgfscope}%
\begin{pgfscope}%
\definecolor{textcolor}{rgb}{0.000000,0.000000,0.000000}%
\pgfsetstrokecolor{textcolor}%
\pgfsetfillcolor{textcolor}%
\pgftext[x=1.971587in,y=0.424381in,,top]{\color{textcolor}{\rmfamily\fontsize{10.000000}{12.000000}\selectfont\catcode`\^=\active\def^{\ifmmode\sp\else\^{}\fi}\catcode`\%=\active\def%{\%}$\mathdefault{24}$}}%
\end{pgfscope}%
\begin{pgfscope}%
\pgfsetbuttcap%
\pgfsetroundjoin%
\definecolor{currentfill}{rgb}{0.000000,0.000000,0.000000}%
\pgfsetfillcolor{currentfill}%
\pgfsetlinewidth{0.803000pt}%
\definecolor{currentstroke}{rgb}{0.000000,0.000000,0.000000}%
\pgfsetstrokecolor{currentstroke}%
\pgfsetdash{}{0pt}%
\pgfsys@defobject{currentmarker}{\pgfqpoint{0.000000in}{-0.048611in}}{\pgfqpoint{0.000000in}{0.000000in}}{%
\pgfpathmoveto{\pgfqpoint{0.000000in}{0.000000in}}%
\pgfpathlineto{\pgfqpoint{0.000000in}{-0.048611in}}%
\pgfusepath{stroke,fill}%
}%
\begin{pgfscope}%
\pgfsys@transformshift{2.742709in}{0.521603in}%
\pgfsys@useobject{currentmarker}{}%
\end{pgfscope}%
\end{pgfscope}%
\begin{pgfscope}%
\definecolor{textcolor}{rgb}{0.000000,0.000000,0.000000}%
\pgfsetstrokecolor{textcolor}%
\pgfsetfillcolor{textcolor}%
\pgftext[x=2.742709in,y=0.424381in,,top]{\color{textcolor}{\rmfamily\fontsize{10.000000}{12.000000}\selectfont\catcode`\^=\active\def^{\ifmmode\sp\else\^{}\fi}\catcode`\%=\active\def%{\%}$\mathdefault{32}$}}%
\end{pgfscope}%
\begin{pgfscope}%
\pgfsetbuttcap%
\pgfsetroundjoin%
\definecolor{currentfill}{rgb}{0.000000,0.000000,0.000000}%
\pgfsetfillcolor{currentfill}%
\pgfsetlinewidth{0.803000pt}%
\definecolor{currentstroke}{rgb}{0.000000,0.000000,0.000000}%
\pgfsetstrokecolor{currentstroke}%
\pgfsetdash{}{0pt}%
\pgfsys@defobject{currentmarker}{\pgfqpoint{0.000000in}{-0.048611in}}{\pgfqpoint{0.000000in}{0.000000in}}{%
\pgfpathmoveto{\pgfqpoint{0.000000in}{0.000000in}}%
\pgfpathlineto{\pgfqpoint{0.000000in}{-0.048611in}}%
\pgfusepath{stroke,fill}%
}%
\begin{pgfscope}%
\pgfsys@transformshift{3.513831in}{0.521603in}%
\pgfsys@useobject{currentmarker}{}%
\end{pgfscope}%
\end{pgfscope}%
\begin{pgfscope}%
\definecolor{textcolor}{rgb}{0.000000,0.000000,0.000000}%
\pgfsetstrokecolor{textcolor}%
\pgfsetfillcolor{textcolor}%
\pgftext[x=3.513831in,y=0.424381in,,top]{\color{textcolor}{\rmfamily\fontsize{10.000000}{12.000000}\selectfont\catcode`\^=\active\def^{\ifmmode\sp\else\^{}\fi}\catcode`\%=\active\def%{\%}$\mathdefault{40}$}}%
\end{pgfscope}%
\begin{pgfscope}%
\pgfsetbuttcap%
\pgfsetroundjoin%
\definecolor{currentfill}{rgb}{0.000000,0.000000,0.000000}%
\pgfsetfillcolor{currentfill}%
\pgfsetlinewidth{0.803000pt}%
\definecolor{currentstroke}{rgb}{0.000000,0.000000,0.000000}%
\pgfsetstrokecolor{currentstroke}%
\pgfsetdash{}{0pt}%
\pgfsys@defobject{currentmarker}{\pgfqpoint{0.000000in}{-0.048611in}}{\pgfqpoint{0.000000in}{0.000000in}}{%
\pgfpathmoveto{\pgfqpoint{0.000000in}{0.000000in}}%
\pgfpathlineto{\pgfqpoint{0.000000in}{-0.048611in}}%
\pgfusepath{stroke,fill}%
}%
\begin{pgfscope}%
\pgfsys@transformshift{4.284953in}{0.521603in}%
\pgfsys@useobject{currentmarker}{}%
\end{pgfscope}%
\end{pgfscope}%
\begin{pgfscope}%
\definecolor{textcolor}{rgb}{0.000000,0.000000,0.000000}%
\pgfsetstrokecolor{textcolor}%
\pgfsetfillcolor{textcolor}%
\pgftext[x=4.284953in,y=0.424381in,,top]{\color{textcolor}{\rmfamily\fontsize{10.000000}{12.000000}\selectfont\catcode`\^=\active\def^{\ifmmode\sp\else\^{}\fi}\catcode`\%=\active\def%{\%}$\mathdefault{48}$}}%
\end{pgfscope}%
\begin{pgfscope}%
\pgfsetbuttcap%
\pgfsetroundjoin%
\definecolor{currentfill}{rgb}{0.000000,0.000000,0.000000}%
\pgfsetfillcolor{currentfill}%
\pgfsetlinewidth{0.803000pt}%
\definecolor{currentstroke}{rgb}{0.000000,0.000000,0.000000}%
\pgfsetstrokecolor{currentstroke}%
\pgfsetdash{}{0pt}%
\pgfsys@defobject{currentmarker}{\pgfqpoint{0.000000in}{-0.048611in}}{\pgfqpoint{0.000000in}{0.000000in}}{%
\pgfpathmoveto{\pgfqpoint{0.000000in}{0.000000in}}%
\pgfpathlineto{\pgfqpoint{0.000000in}{-0.048611in}}%
\pgfusepath{stroke,fill}%
}%
\begin{pgfscope}%
\pgfsys@transformshift{5.056075in}{0.521603in}%
\pgfsys@useobject{currentmarker}{}%
\end{pgfscope}%
\end{pgfscope}%
\begin{pgfscope}%
\definecolor{textcolor}{rgb}{0.000000,0.000000,0.000000}%
\pgfsetstrokecolor{textcolor}%
\pgfsetfillcolor{textcolor}%
\pgftext[x=5.056075in,y=0.424381in,,top]{\color{textcolor}{\rmfamily\fontsize{10.000000}{12.000000}\selectfont\catcode`\^=\active\def^{\ifmmode\sp\else\^{}\fi}\catcode`\%=\active\def%{\%}$\mathdefault{56}$}}%
\end{pgfscope}%
\begin{pgfscope}%
\pgfsetbuttcap%
\pgfsetroundjoin%
\definecolor{currentfill}{rgb}{0.000000,0.000000,0.000000}%
\pgfsetfillcolor{currentfill}%
\pgfsetlinewidth{0.803000pt}%
\definecolor{currentstroke}{rgb}{0.000000,0.000000,0.000000}%
\pgfsetstrokecolor{currentstroke}%
\pgfsetdash{}{0pt}%
\pgfsys@defobject{currentmarker}{\pgfqpoint{0.000000in}{-0.048611in}}{\pgfqpoint{0.000000in}{0.000000in}}{%
\pgfpathmoveto{\pgfqpoint{0.000000in}{0.000000in}}%
\pgfpathlineto{\pgfqpoint{0.000000in}{-0.048611in}}%
\pgfusepath{stroke,fill}%
}%
\begin{pgfscope}%
\pgfsys@transformshift{5.827197in}{0.521603in}%
\pgfsys@useobject{currentmarker}{}%
\end{pgfscope}%
\end{pgfscope}%
\begin{pgfscope}%
\definecolor{textcolor}{rgb}{0.000000,0.000000,0.000000}%
\pgfsetstrokecolor{textcolor}%
\pgfsetfillcolor{textcolor}%
\pgftext[x=5.827197in,y=0.424381in,,top]{\color{textcolor}{\rmfamily\fontsize{10.000000}{12.000000}\selectfont\catcode`\^=\active\def^{\ifmmode\sp\else\^{}\fi}\catcode`\%=\active\def%{\%}$\mathdefault{64}$}}%
\end{pgfscope}%
\begin{pgfscope}%
\pgfsetbuttcap%
\pgfsetroundjoin%
\definecolor{currentfill}{rgb}{0.000000,0.000000,0.000000}%
\pgfsetfillcolor{currentfill}%
\pgfsetlinewidth{0.803000pt}%
\definecolor{currentstroke}{rgb}{0.000000,0.000000,0.000000}%
\pgfsetstrokecolor{currentstroke}%
\pgfsetdash{}{0pt}%
\pgfsys@defobject{currentmarker}{\pgfqpoint{0.000000in}{-0.048611in}}{\pgfqpoint{0.000000in}{0.000000in}}{%
\pgfpathmoveto{\pgfqpoint{0.000000in}{0.000000in}}%
\pgfpathlineto{\pgfqpoint{0.000000in}{-0.048611in}}%
\pgfusepath{stroke,fill}%
}%
\begin{pgfscope}%
\pgfsys@transformshift{6.598319in}{0.521603in}%
\pgfsys@useobject{currentmarker}{}%
\end{pgfscope}%
\end{pgfscope}%
\begin{pgfscope}%
\definecolor{textcolor}{rgb}{0.000000,0.000000,0.000000}%
\pgfsetstrokecolor{textcolor}%
\pgfsetfillcolor{textcolor}%
\pgftext[x=6.598319in,y=0.424381in,,top]{\color{textcolor}{\rmfamily\fontsize{10.000000}{12.000000}\selectfont\catcode`\^=\active\def^{\ifmmode\sp\else\^{}\fi}\catcode`\%=\active\def%{\%}$\mathdefault{72}$}}%
\end{pgfscope}%
\begin{pgfscope}%
\pgfsetbuttcap%
\pgfsetroundjoin%
\definecolor{currentfill}{rgb}{0.000000,0.000000,0.000000}%
\pgfsetfillcolor{currentfill}%
\pgfsetlinewidth{0.803000pt}%
\definecolor{currentstroke}{rgb}{0.000000,0.000000,0.000000}%
\pgfsetstrokecolor{currentstroke}%
\pgfsetdash{}{0pt}%
\pgfsys@defobject{currentmarker}{\pgfqpoint{0.000000in}{-0.048611in}}{\pgfqpoint{0.000000in}{0.000000in}}{%
\pgfpathmoveto{\pgfqpoint{0.000000in}{0.000000in}}%
\pgfpathlineto{\pgfqpoint{0.000000in}{-0.048611in}}%
\pgfusepath{stroke,fill}%
}%
\begin{pgfscope}%
\pgfsys@transformshift{7.369440in}{0.521603in}%
\pgfsys@useobject{currentmarker}{}%
\end{pgfscope}%
\end{pgfscope}%
\begin{pgfscope}%
\definecolor{textcolor}{rgb}{0.000000,0.000000,0.000000}%
\pgfsetstrokecolor{textcolor}%
\pgfsetfillcolor{textcolor}%
\pgftext[x=7.369440in,y=0.424381in,,top]{\color{textcolor}{\rmfamily\fontsize{10.000000}{12.000000}\selectfont\catcode`\^=\active\def^{\ifmmode\sp\else\^{}\fi}\catcode`\%=\active\def%{\%}$\mathdefault{80}$}}%
\end{pgfscope}%
\begin{pgfscope}%
\definecolor{textcolor}{rgb}{0.000000,0.000000,0.000000}%
\pgfsetstrokecolor{textcolor}%
\pgfsetfillcolor{textcolor}%
\pgftext[x=4.140367in,y=0.234413in,,top]{\color{textcolor}{\rmfamily\fontsize{10.000000}{12.000000}\selectfont\catcode`\^=\active\def^{\ifmmode\sp\else\^{}\fi}\catcode`\%=\active\def%{\%}Triangle components}}%
\end{pgfscope}%
\begin{pgfscope}%
\pgfsetbuttcap%
\pgfsetroundjoin%
\definecolor{currentfill}{rgb}{0.000000,0.000000,0.000000}%
\pgfsetfillcolor{currentfill}%
\pgfsetlinewidth{0.803000pt}%
\definecolor{currentstroke}{rgb}{0.000000,0.000000,0.000000}%
\pgfsetstrokecolor{currentstroke}%
\pgfsetdash{}{0pt}%
\pgfsys@defobject{currentmarker}{\pgfqpoint{-0.048611in}{0.000000in}}{\pgfqpoint{-0.000000in}{0.000000in}}{%
\pgfpathmoveto{\pgfqpoint{-0.000000in}{0.000000in}}%
\pgfpathlineto{\pgfqpoint{-0.048611in}{0.000000in}}%
\pgfusepath{stroke,fill}%
}%
\begin{pgfscope}%
\pgfsys@transformshift{0.588387in}{0.566055in}%
\pgfsys@useobject{currentmarker}{}%
\end{pgfscope}%
\end{pgfscope}%
\begin{pgfscope}%
\definecolor{textcolor}{rgb}{0.000000,0.000000,0.000000}%
\pgfsetstrokecolor{textcolor}%
\pgfsetfillcolor{textcolor}%
\pgftext[x=0.289968in, y=0.513294in, left, base]{\color{textcolor}{\rmfamily\fontsize{10.000000}{12.000000}\selectfont\catcode`\^=\active\def^{\ifmmode\sp\else\^{}\fi}\catcode`\%=\active\def%{\%}$\mathdefault{10^{2}}$}}%
\end{pgfscope}%
\begin{pgfscope}%
\pgfsetbuttcap%
\pgfsetroundjoin%
\definecolor{currentfill}{rgb}{0.000000,0.000000,0.000000}%
\pgfsetfillcolor{currentfill}%
\pgfsetlinewidth{0.803000pt}%
\definecolor{currentstroke}{rgb}{0.000000,0.000000,0.000000}%
\pgfsetstrokecolor{currentstroke}%
\pgfsetdash{}{0pt}%
\pgfsys@defobject{currentmarker}{\pgfqpoint{-0.048611in}{0.000000in}}{\pgfqpoint{-0.000000in}{0.000000in}}{%
\pgfpathmoveto{\pgfqpoint{-0.000000in}{0.000000in}}%
\pgfpathlineto{\pgfqpoint{-0.048611in}{0.000000in}}%
\pgfusepath{stroke,fill}%
}%
\begin{pgfscope}%
\pgfsys@transformshift{0.588387in}{2.285770in}%
\pgfsys@useobject{currentmarker}{}%
\end{pgfscope}%
\end{pgfscope}%
\begin{pgfscope}%
\definecolor{textcolor}{rgb}{0.000000,0.000000,0.000000}%
\pgfsetstrokecolor{textcolor}%
\pgfsetfillcolor{textcolor}%
\pgftext[x=0.289968in, y=2.233008in, left, base]{\color{textcolor}{\rmfamily\fontsize{10.000000}{12.000000}\selectfont\catcode`\^=\active\def^{\ifmmode\sp\else\^{}\fi}\catcode`\%=\active\def%{\%}$\mathdefault{10^{3}}$}}%
\end{pgfscope}%
\begin{pgfscope}%
\pgfsetbuttcap%
\pgfsetroundjoin%
\definecolor{currentfill}{rgb}{0.000000,0.000000,0.000000}%
\pgfsetfillcolor{currentfill}%
\pgfsetlinewidth{0.602250pt}%
\definecolor{currentstroke}{rgb}{0.000000,0.000000,0.000000}%
\pgfsetstrokecolor{currentstroke}%
\pgfsetdash{}{0pt}%
\pgfsys@defobject{currentmarker}{\pgfqpoint{-0.027778in}{0.000000in}}{\pgfqpoint{-0.000000in}{0.000000in}}{%
\pgfpathmoveto{\pgfqpoint{-0.000000in}{0.000000in}}%
\pgfpathlineto{\pgfqpoint{-0.027778in}{0.000000in}}%
\pgfusepath{stroke,fill}%
}%
\begin{pgfscope}%
\pgfsys@transformshift{0.588387in}{1.083741in}%
\pgfsys@useobject{currentmarker}{}%
\end{pgfscope}%
\end{pgfscope}%
\begin{pgfscope}%
\pgfsetbuttcap%
\pgfsetroundjoin%
\definecolor{currentfill}{rgb}{0.000000,0.000000,0.000000}%
\pgfsetfillcolor{currentfill}%
\pgfsetlinewidth{0.602250pt}%
\definecolor{currentstroke}{rgb}{0.000000,0.000000,0.000000}%
\pgfsetstrokecolor{currentstroke}%
\pgfsetdash{}{0pt}%
\pgfsys@defobject{currentmarker}{\pgfqpoint{-0.027778in}{0.000000in}}{\pgfqpoint{-0.000000in}{0.000000in}}{%
\pgfpathmoveto{\pgfqpoint{-0.000000in}{0.000000in}}%
\pgfpathlineto{\pgfqpoint{-0.027778in}{0.000000in}}%
\pgfusepath{stroke,fill}%
}%
\begin{pgfscope}%
\pgfsys@transformshift{0.588387in}{1.386567in}%
\pgfsys@useobject{currentmarker}{}%
\end{pgfscope}%
\end{pgfscope}%
\begin{pgfscope}%
\pgfsetbuttcap%
\pgfsetroundjoin%
\definecolor{currentfill}{rgb}{0.000000,0.000000,0.000000}%
\pgfsetfillcolor{currentfill}%
\pgfsetlinewidth{0.602250pt}%
\definecolor{currentstroke}{rgb}{0.000000,0.000000,0.000000}%
\pgfsetstrokecolor{currentstroke}%
\pgfsetdash{}{0pt}%
\pgfsys@defobject{currentmarker}{\pgfqpoint{-0.027778in}{0.000000in}}{\pgfqpoint{-0.000000in}{0.000000in}}{%
\pgfpathmoveto{\pgfqpoint{-0.000000in}{0.000000in}}%
\pgfpathlineto{\pgfqpoint{-0.027778in}{0.000000in}}%
\pgfusepath{stroke,fill}%
}%
\begin{pgfscope}%
\pgfsys@transformshift{0.588387in}{1.601426in}%
\pgfsys@useobject{currentmarker}{}%
\end{pgfscope}%
\end{pgfscope}%
\begin{pgfscope}%
\pgfsetbuttcap%
\pgfsetroundjoin%
\definecolor{currentfill}{rgb}{0.000000,0.000000,0.000000}%
\pgfsetfillcolor{currentfill}%
\pgfsetlinewidth{0.602250pt}%
\definecolor{currentstroke}{rgb}{0.000000,0.000000,0.000000}%
\pgfsetstrokecolor{currentstroke}%
\pgfsetdash{}{0pt}%
\pgfsys@defobject{currentmarker}{\pgfqpoint{-0.027778in}{0.000000in}}{\pgfqpoint{-0.000000in}{0.000000in}}{%
\pgfpathmoveto{\pgfqpoint{-0.000000in}{0.000000in}}%
\pgfpathlineto{\pgfqpoint{-0.027778in}{0.000000in}}%
\pgfusepath{stroke,fill}%
}%
\begin{pgfscope}%
\pgfsys@transformshift{0.588387in}{1.768084in}%
\pgfsys@useobject{currentmarker}{}%
\end{pgfscope}%
\end{pgfscope}%
\begin{pgfscope}%
\pgfsetbuttcap%
\pgfsetroundjoin%
\definecolor{currentfill}{rgb}{0.000000,0.000000,0.000000}%
\pgfsetfillcolor{currentfill}%
\pgfsetlinewidth{0.602250pt}%
\definecolor{currentstroke}{rgb}{0.000000,0.000000,0.000000}%
\pgfsetstrokecolor{currentstroke}%
\pgfsetdash{}{0pt}%
\pgfsys@defobject{currentmarker}{\pgfqpoint{-0.027778in}{0.000000in}}{\pgfqpoint{-0.000000in}{0.000000in}}{%
\pgfpathmoveto{\pgfqpoint{-0.000000in}{0.000000in}}%
\pgfpathlineto{\pgfqpoint{-0.027778in}{0.000000in}}%
\pgfusepath{stroke,fill}%
}%
\begin{pgfscope}%
\pgfsys@transformshift{0.588387in}{1.904253in}%
\pgfsys@useobject{currentmarker}{}%
\end{pgfscope}%
\end{pgfscope}%
\begin{pgfscope}%
\pgfsetbuttcap%
\pgfsetroundjoin%
\definecolor{currentfill}{rgb}{0.000000,0.000000,0.000000}%
\pgfsetfillcolor{currentfill}%
\pgfsetlinewidth{0.602250pt}%
\definecolor{currentstroke}{rgb}{0.000000,0.000000,0.000000}%
\pgfsetstrokecolor{currentstroke}%
\pgfsetdash{}{0pt}%
\pgfsys@defobject{currentmarker}{\pgfqpoint{-0.027778in}{0.000000in}}{\pgfqpoint{-0.000000in}{0.000000in}}{%
\pgfpathmoveto{\pgfqpoint{-0.000000in}{0.000000in}}%
\pgfpathlineto{\pgfqpoint{-0.027778in}{0.000000in}}%
\pgfusepath{stroke,fill}%
}%
\begin{pgfscope}%
\pgfsys@transformshift{0.588387in}{2.019382in}%
\pgfsys@useobject{currentmarker}{}%
\end{pgfscope}%
\end{pgfscope}%
\begin{pgfscope}%
\pgfsetbuttcap%
\pgfsetroundjoin%
\definecolor{currentfill}{rgb}{0.000000,0.000000,0.000000}%
\pgfsetfillcolor{currentfill}%
\pgfsetlinewidth{0.602250pt}%
\definecolor{currentstroke}{rgb}{0.000000,0.000000,0.000000}%
\pgfsetstrokecolor{currentstroke}%
\pgfsetdash{}{0pt}%
\pgfsys@defobject{currentmarker}{\pgfqpoint{-0.027778in}{0.000000in}}{\pgfqpoint{-0.000000in}{0.000000in}}{%
\pgfpathmoveto{\pgfqpoint{-0.000000in}{0.000000in}}%
\pgfpathlineto{\pgfqpoint{-0.027778in}{0.000000in}}%
\pgfusepath{stroke,fill}%
}%
\begin{pgfscope}%
\pgfsys@transformshift{0.588387in}{2.119112in}%
\pgfsys@useobject{currentmarker}{}%
\end{pgfscope}%
\end{pgfscope}%
\begin{pgfscope}%
\pgfsetbuttcap%
\pgfsetroundjoin%
\definecolor{currentfill}{rgb}{0.000000,0.000000,0.000000}%
\pgfsetfillcolor{currentfill}%
\pgfsetlinewidth{0.602250pt}%
\definecolor{currentstroke}{rgb}{0.000000,0.000000,0.000000}%
\pgfsetstrokecolor{currentstroke}%
\pgfsetdash{}{0pt}%
\pgfsys@defobject{currentmarker}{\pgfqpoint{-0.027778in}{0.000000in}}{\pgfqpoint{-0.000000in}{0.000000in}}{%
\pgfpathmoveto{\pgfqpoint{-0.000000in}{0.000000in}}%
\pgfpathlineto{\pgfqpoint{-0.027778in}{0.000000in}}%
\pgfusepath{stroke,fill}%
}%
\begin{pgfscope}%
\pgfsys@transformshift{0.588387in}{2.207080in}%
\pgfsys@useobject{currentmarker}{}%
\end{pgfscope}%
\end{pgfscope}%
\begin{pgfscope}%
\definecolor{textcolor}{rgb}{0.000000,0.000000,0.000000}%
\pgfsetstrokecolor{textcolor}%
\pgfsetfillcolor{textcolor}%
\pgftext[x=0.234413in,y=1.631726in,,bottom,rotate=90.000000]{\color{textcolor}{\rmfamily\fontsize{10.000000}{12.000000}\selectfont\catcode`\^=\active\def^{\ifmmode\sp\else\^{}\fi}\catcode`\%=\active\def%{\%}Time [ms]}}%
\end{pgfscope}%
\begin{pgfscope}%
\pgfpathrectangle{\pgfqpoint{0.588387in}{0.521603in}}{\pgfqpoint{7.103961in}{2.220246in}}%
\pgfusepath{clip}%
\pgfsetrectcap%
\pgfsetroundjoin%
\pgfsetlinewidth{1.505625pt}%
\pgfsetstrokecolor{currentstroke1}%
\pgfsetdash{}{0pt}%
\pgfpathmoveto{\pgfqpoint{0.911295in}{0.627672in}}%
\pgfpathlineto{\pgfqpoint{1.007685in}{0.716963in}}%
\pgfpathlineto{\pgfqpoint{1.104075in}{0.779647in}}%
\pgfpathlineto{\pgfqpoint{1.200465in}{0.862644in}}%
\pgfpathlineto{\pgfqpoint{1.296855in}{0.897036in}}%
\pgfpathlineto{\pgfqpoint{1.393246in}{0.925354in}}%
\pgfpathlineto{\pgfqpoint{1.489636in}{0.896776in}}%
\pgfpathlineto{\pgfqpoint{1.586026in}{0.914595in}}%
\pgfpathlineto{\pgfqpoint{1.682416in}{0.907565in}}%
\pgfpathlineto{\pgfqpoint{1.778807in}{0.866470in}}%
\pgfpathlineto{\pgfqpoint{1.875197in}{0.887986in}}%
\pgfpathlineto{\pgfqpoint{1.971587in}{0.905565in}}%
\pgfpathlineto{\pgfqpoint{2.067977in}{0.873759in}}%
\pgfpathlineto{\pgfqpoint{2.164368in}{0.930435in}}%
\pgfpathlineto{\pgfqpoint{2.260758in}{0.942224in}}%
\pgfpathlineto{\pgfqpoint{2.357148in}{1.001527in}}%
\pgfpathlineto{\pgfqpoint{2.453538in}{1.004364in}}%
\pgfpathlineto{\pgfqpoint{2.549929in}{1.036659in}}%
\pgfpathlineto{\pgfqpoint{2.646319in}{1.067083in}}%
\pgfpathlineto{\pgfqpoint{2.742709in}{1.071488in}}%
\pgfpathlineto{\pgfqpoint{2.839099in}{1.087297in}}%
\pgfpathlineto{\pgfqpoint{2.935490in}{1.148348in}}%
\pgfpathlineto{\pgfqpoint{3.031880in}{1.198726in}}%
\pgfpathlineto{\pgfqpoint{3.128270in}{1.219945in}}%
\pgfpathlineto{\pgfqpoint{3.224660in}{1.230570in}}%
\pgfpathlineto{\pgfqpoint{3.321050in}{1.333140in}}%
\pgfpathlineto{\pgfqpoint{3.417441in}{1.303696in}}%
\pgfpathlineto{\pgfqpoint{3.513831in}{1.364559in}}%
\pgfpathlineto{\pgfqpoint{3.610221in}{1.351608in}}%
\pgfpathlineto{\pgfqpoint{3.706611in}{1.373302in}}%
\pgfpathlineto{\pgfqpoint{3.803002in}{1.392192in}}%
\pgfpathlineto{\pgfqpoint{3.899392in}{1.478673in}}%
\pgfpathlineto{\pgfqpoint{3.995782in}{1.488131in}}%
\pgfpathlineto{\pgfqpoint{4.092172in}{1.494595in}}%
\pgfpathlineto{\pgfqpoint{4.188563in}{1.572961in}}%
\pgfpathlineto{\pgfqpoint{4.284953in}{1.569559in}}%
\pgfpathlineto{\pgfqpoint{4.381343in}{1.483061in}}%
\pgfpathlineto{\pgfqpoint{4.477733in}{1.658581in}}%
\pgfpathlineto{\pgfqpoint{4.574124in}{1.664235in}}%
\pgfpathlineto{\pgfqpoint{4.670514in}{1.849941in}}%
\pgfpathlineto{\pgfqpoint{4.766904in}{1.651303in}}%
\pgfpathlineto{\pgfqpoint{4.863294in}{2.057725in}}%
\pgfpathlineto{\pgfqpoint{4.959684in}{1.842148in}}%
\pgfpathlineto{\pgfqpoint{5.056075in}{1.902955in}}%
\pgfpathlineto{\pgfqpoint{5.152465in}{1.933546in}}%
\pgfpathlineto{\pgfqpoint{5.248855in}{1.873805in}}%
\pgfpathlineto{\pgfqpoint{5.441636in}{2.147282in}}%
\pgfpathlineto{\pgfqpoint{5.538026in}{1.883267in}}%
\pgfpathlineto{\pgfqpoint{5.634416in}{2.201980in}}%
\pgfpathlineto{\pgfqpoint{5.827197in}{2.411440in}}%
\pgfpathlineto{\pgfqpoint{5.923587in}{2.396780in}}%
\pgfpathlineto{\pgfqpoint{6.019977in}{2.146326in}}%
\pgfpathlineto{\pgfqpoint{6.116367in}{2.492769in}}%
\pgfpathlineto{\pgfqpoint{6.212758in}{2.179933in}}%
\pgfpathlineto{\pgfqpoint{6.405538in}{2.266142in}}%
\pgfpathlineto{\pgfqpoint{6.598319in}{2.637381in}}%
\pgfpathlineto{\pgfqpoint{6.887489in}{2.378535in}}%
\pgfpathlineto{\pgfqpoint{7.369440in}{2.640929in}}%
\pgfusepath{stroke}%
\end{pgfscope}%
\begin{pgfscope}%
\pgfpathrectangle{\pgfqpoint{0.588387in}{0.521603in}}{\pgfqpoint{7.103961in}{2.220246in}}%
\pgfusepath{clip}%
\pgfsetrectcap%
\pgfsetroundjoin%
\pgfsetlinewidth{1.505625pt}%
\pgfsetstrokecolor{currentstroke2}%
\pgfsetdash{}{0pt}%
\pgfpathmoveto{\pgfqpoint{0.911295in}{0.622524in}}%
\pgfpathlineto{\pgfqpoint{1.007685in}{0.748635in}}%
\pgfpathlineto{\pgfqpoint{1.104075in}{0.821323in}}%
\pgfpathlineto{\pgfqpoint{1.200465in}{0.862482in}}%
\pgfpathlineto{\pgfqpoint{1.296855in}{0.890851in}}%
\pgfpathlineto{\pgfqpoint{1.393246in}{0.950999in}}%
\pgfpathlineto{\pgfqpoint{1.489636in}{0.918876in}}%
\pgfpathlineto{\pgfqpoint{1.586026in}{0.906775in}}%
\pgfpathlineto{\pgfqpoint{1.682416in}{0.912984in}}%
\pgfpathlineto{\pgfqpoint{1.778807in}{0.871058in}}%
\pgfpathlineto{\pgfqpoint{1.875197in}{0.901326in}}%
\pgfpathlineto{\pgfqpoint{1.971587in}{0.897730in}}%
\pgfpathlineto{\pgfqpoint{2.067977in}{0.855432in}}%
\pgfpathlineto{\pgfqpoint{2.164368in}{0.886323in}}%
\pgfpathlineto{\pgfqpoint{2.260758in}{0.925999in}}%
\pgfpathlineto{\pgfqpoint{2.357148in}{0.939629in}}%
\pgfpathlineto{\pgfqpoint{2.453538in}{0.961896in}}%
\pgfpathlineto{\pgfqpoint{2.549929in}{0.969303in}}%
\pgfpathlineto{\pgfqpoint{2.646319in}{1.034593in}}%
\pgfpathlineto{\pgfqpoint{2.742709in}{1.024894in}}%
\pgfpathlineto{\pgfqpoint{2.839099in}{1.022830in}}%
\pgfpathlineto{\pgfqpoint{2.935490in}{1.105264in}}%
\pgfpathlineto{\pgfqpoint{3.031880in}{1.138222in}}%
\pgfpathlineto{\pgfqpoint{3.128270in}{1.131220in}}%
\pgfpathlineto{\pgfqpoint{3.224660in}{1.142878in}}%
\pgfpathlineto{\pgfqpoint{3.321050in}{1.293424in}}%
\pgfpathlineto{\pgfqpoint{3.417441in}{1.236528in}}%
\pgfpathlineto{\pgfqpoint{3.513831in}{1.251891in}}%
\pgfpathlineto{\pgfqpoint{3.610221in}{1.257411in}}%
\pgfpathlineto{\pgfqpoint{3.706611in}{1.311465in}}%
\pgfpathlineto{\pgfqpoint{3.803002in}{1.345520in}}%
\pgfpathlineto{\pgfqpoint{3.899392in}{1.339361in}}%
\pgfpathlineto{\pgfqpoint{3.995782in}{1.371682in}}%
\pgfpathlineto{\pgfqpoint{4.092172in}{1.398055in}}%
\pgfpathlineto{\pgfqpoint{4.188563in}{1.506518in}}%
\pgfpathlineto{\pgfqpoint{4.284953in}{1.457625in}}%
\pgfpathlineto{\pgfqpoint{4.381343in}{1.403713in}}%
\pgfpathlineto{\pgfqpoint{4.477733in}{1.534946in}}%
\pgfpathlineto{\pgfqpoint{4.574124in}{1.555164in}}%
\pgfpathlineto{\pgfqpoint{4.670514in}{1.724726in}}%
\pgfpathlineto{\pgfqpoint{4.766904in}{1.605151in}}%
\pgfpathlineto{\pgfqpoint{4.863294in}{1.848713in}}%
\pgfpathlineto{\pgfqpoint{4.959684in}{1.725169in}}%
\pgfpathlineto{\pgfqpoint{5.056075in}{1.759823in}}%
\pgfpathlineto{\pgfqpoint{5.152465in}{1.808638in}}%
\pgfpathlineto{\pgfqpoint{5.248855in}{1.740546in}}%
\pgfpathlineto{\pgfqpoint{5.441636in}{2.029240in}}%
\pgfpathlineto{\pgfqpoint{5.538026in}{1.743330in}}%
\pgfpathlineto{\pgfqpoint{5.634416in}{2.009718in}}%
\pgfpathlineto{\pgfqpoint{5.827197in}{2.250677in}}%
\pgfpathlineto{\pgfqpoint{5.923587in}{2.227462in}}%
\pgfpathlineto{\pgfqpoint{6.019977in}{2.000856in}}%
\pgfpathlineto{\pgfqpoint{6.116367in}{2.298506in}}%
\pgfpathlineto{\pgfqpoint{6.212758in}{2.026920in}}%
\pgfpathlineto{\pgfqpoint{6.405538in}{2.074239in}}%
\pgfpathlineto{\pgfqpoint{6.598319in}{2.480339in}}%
\pgfpathlineto{\pgfqpoint{6.887489in}{2.212866in}}%
\pgfpathlineto{\pgfqpoint{7.369440in}{2.473400in}}%
\pgfusepath{stroke}%
\end{pgfscope}%
\begin{pgfscope}%
\pgfpathrectangle{\pgfqpoint{0.588387in}{0.521603in}}{\pgfqpoint{7.103961in}{2.220246in}}%
\pgfusepath{clip}%
\pgfsetrectcap%
\pgfsetroundjoin%
\pgfsetlinewidth{1.505625pt}%
\pgfsetstrokecolor{currentstroke3}%
\pgfsetdash{}{0pt}%
\pgfpathmoveto{\pgfqpoint{0.911295in}{0.689895in}}%
\pgfpathlineto{\pgfqpoint{1.007685in}{0.796328in}}%
\pgfpathlineto{\pgfqpoint{1.104075in}{0.772888in}}%
\pgfpathlineto{\pgfqpoint{1.200465in}{0.841509in}}%
\pgfpathlineto{\pgfqpoint{1.296855in}{0.877182in}}%
\pgfpathlineto{\pgfqpoint{1.393246in}{0.959394in}}%
\pgfpathlineto{\pgfqpoint{1.489636in}{0.916184in}}%
\pgfpathlineto{\pgfqpoint{1.586026in}{0.869314in}}%
\pgfpathlineto{\pgfqpoint{1.682416in}{0.885536in}}%
\pgfpathlineto{\pgfqpoint{1.778807in}{0.856569in}}%
\pgfpathlineto{\pgfqpoint{1.875197in}{0.866021in}}%
\pgfpathlineto{\pgfqpoint{1.971587in}{0.918266in}}%
\pgfpathlineto{\pgfqpoint{2.067977in}{0.821436in}}%
\pgfpathlineto{\pgfqpoint{2.164368in}{0.841036in}}%
\pgfpathlineto{\pgfqpoint{2.260758in}{0.891687in}}%
\pgfpathlineto{\pgfqpoint{2.357148in}{0.933552in}}%
\pgfpathlineto{\pgfqpoint{2.453538in}{0.966341in}}%
\pgfpathlineto{\pgfqpoint{2.549929in}{0.917841in}}%
\pgfpathlineto{\pgfqpoint{2.646319in}{0.976135in}}%
\pgfpathlineto{\pgfqpoint{2.742709in}{1.002773in}}%
\pgfpathlineto{\pgfqpoint{2.839099in}{0.993212in}}%
\pgfpathlineto{\pgfqpoint{2.935490in}{1.079109in}}%
\pgfpathlineto{\pgfqpoint{3.031880in}{1.089590in}}%
\pgfpathlineto{\pgfqpoint{3.128270in}{1.098874in}}%
\pgfpathlineto{\pgfqpoint{3.224660in}{1.106548in}}%
\pgfpathlineto{\pgfqpoint{3.321050in}{1.249179in}}%
\pgfpathlineto{\pgfqpoint{3.417441in}{1.207213in}}%
\pgfpathlineto{\pgfqpoint{3.513831in}{1.182548in}}%
\pgfpathlineto{\pgfqpoint{3.610221in}{1.210200in}}%
\pgfpathlineto{\pgfqpoint{3.706611in}{1.285556in}}%
\pgfpathlineto{\pgfqpoint{3.803002in}{1.270587in}}%
\pgfpathlineto{\pgfqpoint{3.899392in}{1.331094in}}%
\pgfpathlineto{\pgfqpoint{3.995782in}{1.342944in}}%
\pgfpathlineto{\pgfqpoint{4.092172in}{1.344925in}}%
\pgfpathlineto{\pgfqpoint{4.188563in}{1.493314in}}%
\pgfpathlineto{\pgfqpoint{4.284953in}{1.411905in}}%
\pgfpathlineto{\pgfqpoint{4.381343in}{1.390396in}}%
\pgfpathlineto{\pgfqpoint{4.477733in}{1.473428in}}%
\pgfpathlineto{\pgfqpoint{4.574124in}{1.496343in}}%
\pgfpathlineto{\pgfqpoint{4.670514in}{1.695249in}}%
\pgfpathlineto{\pgfqpoint{4.766904in}{1.503029in}}%
\pgfpathlineto{\pgfqpoint{4.863294in}{1.818790in}}%
\pgfpathlineto{\pgfqpoint{4.959684in}{1.637777in}}%
\pgfpathlineto{\pgfqpoint{5.056075in}{1.667643in}}%
\pgfpathlineto{\pgfqpoint{5.152465in}{1.773294in}}%
\pgfpathlineto{\pgfqpoint{5.248855in}{1.687161in}}%
\pgfpathlineto{\pgfqpoint{5.441636in}{1.945125in}}%
\pgfpathlineto{\pgfqpoint{5.538026in}{1.631952in}}%
\pgfpathlineto{\pgfqpoint{5.634416in}{1.911839in}}%
\pgfpathlineto{\pgfqpoint{5.827197in}{2.121442in}}%
\pgfpathlineto{\pgfqpoint{5.923587in}{2.148180in}}%
\pgfpathlineto{\pgfqpoint{6.019977in}{1.909602in}}%
\pgfpathlineto{\pgfqpoint{6.116367in}{2.192203in}}%
\pgfpathlineto{\pgfqpoint{6.212758in}{1.946008in}}%
\pgfpathlineto{\pgfqpoint{6.405538in}{1.991667in}}%
\pgfpathlineto{\pgfqpoint{6.598319in}{2.392098in}}%
\pgfpathlineto{\pgfqpoint{6.887489in}{2.104737in}}%
\pgfpathlineto{\pgfqpoint{7.369440in}{2.351673in}}%
\pgfusepath{stroke}%
\end{pgfscope}%
\begin{pgfscope}%
\pgfpathrectangle{\pgfqpoint{0.588387in}{0.521603in}}{\pgfqpoint{7.103961in}{2.220246in}}%
\pgfusepath{clip}%
\pgfsetrectcap%
\pgfsetroundjoin%
\pgfsetlinewidth{1.505625pt}%
\pgfsetstrokecolor{currentstroke4}%
\pgfsetdash{}{0pt}%
\pgfpathmoveto{\pgfqpoint{0.911295in}{0.660977in}}%
\pgfpathlineto{\pgfqpoint{1.007685in}{0.803543in}}%
\pgfpathlineto{\pgfqpoint{1.104075in}{1.001028in}}%
\pgfpathlineto{\pgfqpoint{1.200465in}{1.160946in}}%
\pgfpathlineto{\pgfqpoint{1.296855in}{1.223041in}}%
\pgfpathlineto{\pgfqpoint{1.393246in}{1.006619in}}%
\pgfpathlineto{\pgfqpoint{1.489636in}{0.978099in}}%
\pgfpathlineto{\pgfqpoint{1.586026in}{1.031442in}}%
\pgfpathlineto{\pgfqpoint{1.682416in}{1.061200in}}%
\pgfpathlineto{\pgfqpoint{1.778807in}{1.062754in}}%
\pgfpathlineto{\pgfqpoint{1.875197in}{1.142241in}}%
\pgfpathlineto{\pgfqpoint{1.971587in}{0.960193in}}%
\pgfpathlineto{\pgfqpoint{2.067977in}{0.957976in}}%
\pgfpathlineto{\pgfqpoint{2.164368in}{1.001218in}}%
\pgfpathlineto{\pgfqpoint{2.260758in}{1.076375in}}%
\pgfpathlineto{\pgfqpoint{2.357148in}{1.105351in}}%
\pgfpathlineto{\pgfqpoint{2.453538in}{1.146619in}}%
\pgfpathlineto{\pgfqpoint{2.549929in}{1.021262in}}%
\pgfpathlineto{\pgfqpoint{2.646319in}{1.086967in}}%
\pgfpathlineto{\pgfqpoint{2.742709in}{1.117274in}}%
\pgfpathlineto{\pgfqpoint{2.839099in}{1.098515in}}%
\pgfpathlineto{\pgfqpoint{2.935490in}{1.207039in}}%
\pgfpathlineto{\pgfqpoint{3.031880in}{1.240931in}}%
\pgfpathlineto{\pgfqpoint{3.128270in}{1.155433in}}%
\pgfpathlineto{\pgfqpoint{3.224660in}{1.175097in}}%
\pgfpathlineto{\pgfqpoint{3.321050in}{1.300185in}}%
\pgfpathlineto{\pgfqpoint{3.417441in}{1.272163in}}%
\pgfpathlineto{\pgfqpoint{3.513831in}{1.275626in}}%
\pgfpathlineto{\pgfqpoint{3.610221in}{1.304213in}}%
\pgfpathlineto{\pgfqpoint{3.706611in}{1.329835in}}%
\pgfpathlineto{\pgfqpoint{3.803002in}{1.295325in}}%
\pgfpathlineto{\pgfqpoint{3.899392in}{1.362427in}}%
\pgfpathlineto{\pgfqpoint{3.995782in}{1.387213in}}%
\pgfpathlineto{\pgfqpoint{4.092172in}{1.389215in}}%
\pgfpathlineto{\pgfqpoint{4.188563in}{1.637807in}}%
\pgfpathlineto{\pgfqpoint{4.284953in}{1.431059in}}%
\pgfpathlineto{\pgfqpoint{4.381343in}{1.427808in}}%
\pgfpathlineto{\pgfqpoint{4.477733in}{1.485761in}}%
\pgfpathlineto{\pgfqpoint{4.574124in}{1.575180in}}%
\pgfpathlineto{\pgfqpoint{4.670514in}{1.649654in}}%
\pgfpathlineto{\pgfqpoint{4.766904in}{1.602626in}}%
\pgfpathlineto{\pgfqpoint{4.863294in}{1.619367in}}%
\pgfpathlineto{\pgfqpoint{4.959684in}{1.648335in}}%
\pgfpathlineto{\pgfqpoint{5.056075in}{1.668354in}}%
\pgfpathlineto{\pgfqpoint{5.152465in}{1.744344in}}%
\pgfpathlineto{\pgfqpoint{5.248855in}{1.715666in}}%
\pgfpathlineto{\pgfqpoint{5.345245in}{1.729971in}}%
\pgfpathlineto{\pgfqpoint{5.441636in}{1.914672in}}%
\pgfpathlineto{\pgfqpoint{5.538026in}{1.603198in}}%
\pgfpathlineto{\pgfqpoint{5.634416in}{1.900822in}}%
\pgfpathlineto{\pgfqpoint{5.730806in}{1.891700in}}%
\pgfpathlineto{\pgfqpoint{5.827197in}{2.015210in}}%
\pgfpathlineto{\pgfqpoint{5.923587in}{2.064849in}}%
\pgfpathlineto{\pgfqpoint{6.019977in}{1.898874in}}%
\pgfpathlineto{\pgfqpoint{6.116367in}{2.122093in}}%
\pgfpathlineto{\pgfqpoint{6.212758in}{1.912224in}}%
\pgfpathlineto{\pgfqpoint{6.309148in}{1.960579in}}%
\pgfpathlineto{\pgfqpoint{6.405538in}{1.991588in}}%
\pgfpathlineto{\pgfqpoint{6.598319in}{2.288639in}}%
\pgfpathlineto{\pgfqpoint{6.791099in}{2.162411in}}%
\pgfpathlineto{\pgfqpoint{6.887489in}{2.094757in}}%
\pgfpathlineto{\pgfqpoint{7.080270in}{2.445675in}}%
\pgfpathlineto{\pgfqpoint{7.369440in}{2.343292in}}%
\pgfusepath{stroke}%
\end{pgfscope}%
\begin{pgfscope}%
\pgfpathrectangle{\pgfqpoint{0.588387in}{0.521603in}}{\pgfqpoint{7.103961in}{2.220246in}}%
\pgfusepath{clip}%
\pgfsetrectcap%
\pgfsetroundjoin%
\pgfsetlinewidth{1.505625pt}%
\pgfsetstrokecolor{currentstroke5}%
\pgfsetdash{}{0pt}%
\pgfpathmoveto{\pgfqpoint{0.911295in}{2.039269in}}%
\pgfpathlineto{\pgfqpoint{1.007685in}{0.801865in}}%
\pgfpathlineto{\pgfqpoint{1.104075in}{0.924122in}}%
\pgfpathlineto{\pgfqpoint{1.200465in}{1.033001in}}%
\pgfpathlineto{\pgfqpoint{1.296855in}{1.121377in}}%
\pgfpathlineto{\pgfqpoint{1.393246in}{1.298890in}}%
\pgfpathlineto{\pgfqpoint{1.489636in}{1.369344in}}%
\pgfpathlineto{\pgfqpoint{1.586026in}{1.524834in}}%
\pgfpathlineto{\pgfqpoint{1.682416in}{0.974821in}}%
\pgfpathlineto{\pgfqpoint{1.778807in}{0.943209in}}%
\pgfpathlineto{\pgfqpoint{1.875197in}{1.013391in}}%
\pgfpathlineto{\pgfqpoint{1.971587in}{1.125246in}}%
\pgfpathlineto{\pgfqpoint{2.067977in}{1.135859in}}%
\pgfpathlineto{\pgfqpoint{2.164368in}{1.199592in}}%
\pgfpathlineto{\pgfqpoint{2.260758in}{1.301292in}}%
\pgfpathlineto{\pgfqpoint{2.357148in}{1.022842in}}%
\pgfpathlineto{\pgfqpoint{2.453538in}{1.063503in}}%
\pgfpathlineto{\pgfqpoint{2.549929in}{1.112539in}}%
\pgfpathlineto{\pgfqpoint{2.646319in}{1.163211in}}%
\pgfpathlineto{\pgfqpoint{2.742709in}{1.233768in}}%
\pgfpathlineto{\pgfqpoint{2.839099in}{1.137694in}}%
\pgfpathlineto{\pgfqpoint{2.935490in}{1.325466in}}%
\pgfpathlineto{\pgfqpoint{3.031880in}{1.184656in}}%
\pgfpathlineto{\pgfqpoint{3.128270in}{1.180163in}}%
\pgfpathlineto{\pgfqpoint{3.224660in}{1.198583in}}%
\pgfpathlineto{\pgfqpoint{3.321050in}{1.378312in}}%
\pgfpathlineto{\pgfqpoint{3.417441in}{1.328770in}}%
\pgfpathlineto{\pgfqpoint{3.513831in}{1.324088in}}%
\pgfpathlineto{\pgfqpoint{3.610221in}{1.377993in}}%
\pgfpathlineto{\pgfqpoint{3.706611in}{1.322980in}}%
\pgfpathlineto{\pgfqpoint{3.803002in}{1.333738in}}%
\pgfpathlineto{\pgfqpoint{3.899392in}{1.359184in}}%
\pgfpathlineto{\pgfqpoint{3.995782in}{1.437383in}}%
\pgfpathlineto{\pgfqpoint{4.092172in}{1.393850in}}%
\pgfpathlineto{\pgfqpoint{4.188563in}{1.634078in}}%
\pgfpathlineto{\pgfqpoint{4.284953in}{1.527674in}}%
\pgfpathlineto{\pgfqpoint{4.381343in}{1.455201in}}%
\pgfpathlineto{\pgfqpoint{4.477733in}{1.498861in}}%
\pgfpathlineto{\pgfqpoint{4.574124in}{1.563881in}}%
\pgfpathlineto{\pgfqpoint{4.670514in}{1.720281in}}%
\pgfpathlineto{\pgfqpoint{4.766904in}{1.684398in}}%
\pgfpathlineto{\pgfqpoint{4.863294in}{1.753947in}}%
\pgfpathlineto{\pgfqpoint{4.959684in}{1.708801in}}%
\pgfpathlineto{\pgfqpoint{5.056075in}{1.675786in}}%
\pgfpathlineto{\pgfqpoint{5.152465in}{1.791247in}}%
\pgfpathlineto{\pgfqpoint{5.248855in}{1.686734in}}%
\pgfpathlineto{\pgfqpoint{5.441636in}{1.997458in}}%
\pgfpathlineto{\pgfqpoint{5.538026in}{1.632960in}}%
\pgfpathlineto{\pgfqpoint{5.634416in}{1.879577in}}%
\pgfpathlineto{\pgfqpoint{5.827197in}{2.015907in}}%
\pgfpathlineto{\pgfqpoint{5.923587in}{2.060002in}}%
\pgfpathlineto{\pgfqpoint{6.019977in}{1.848713in}}%
\pgfpathlineto{\pgfqpoint{6.116367in}{2.107587in}}%
\pgfpathlineto{\pgfqpoint{6.212758in}{1.968188in}}%
\pgfpathlineto{\pgfqpoint{6.405538in}{1.956721in}}%
\pgfpathlineto{\pgfqpoint{6.598319in}{2.219323in}}%
\pgfpathlineto{\pgfqpoint{6.887489in}{1.982196in}}%
\pgfpathlineto{\pgfqpoint{7.369440in}{2.321408in}}%
\pgfusepath{stroke}%
\end{pgfscope}%
\begin{pgfscope}%
\pgfpathrectangle{\pgfqpoint{0.588387in}{0.521603in}}{\pgfqpoint{7.103961in}{2.220246in}}%
\pgfusepath{clip}%
\pgfsetrectcap%
\pgfsetroundjoin%
\pgfsetlinewidth{1.505625pt}%
\pgfsetstrokecolor{currentstroke6}%
\pgfsetdash{}{0pt}%
\pgfpathmoveto{\pgfqpoint{0.911295in}{2.074655in}}%
\pgfpathlineto{\pgfqpoint{1.007685in}{2.237534in}}%
\pgfpathlineto{\pgfqpoint{1.104075in}{2.507991in}}%
\pgfpathlineto{\pgfqpoint{1.200465in}{1.155017in}}%
\pgfpathlineto{\pgfqpoint{1.296855in}{1.254459in}}%
\pgfpathlineto{\pgfqpoint{1.393246in}{1.428645in}}%
\pgfpathlineto{\pgfqpoint{1.489636in}{1.487286in}}%
\pgfpathlineto{\pgfqpoint{1.586026in}{1.642403in}}%
\pgfpathlineto{\pgfqpoint{1.682416in}{1.809143in}}%
\pgfpathlineto{\pgfqpoint{1.778807in}{1.853991in}}%
\pgfpathlineto{\pgfqpoint{1.875197in}{2.108712in}}%
\pgfpathlineto{\pgfqpoint{1.971587in}{1.201064in}}%
\pgfpathlineto{\pgfqpoint{2.067977in}{1.246913in}}%
\pgfpathlineto{\pgfqpoint{2.164368in}{1.301647in}}%
\pgfpathlineto{\pgfqpoint{2.260758in}{1.428830in}}%
\pgfpathlineto{\pgfqpoint{2.357148in}{1.569217in}}%
\pgfpathlineto{\pgfqpoint{2.453538in}{1.670172in}}%
\pgfpathlineto{\pgfqpoint{2.549929in}{1.698689in}}%
\pgfpathlineto{\pgfqpoint{2.646319in}{1.879924in}}%
\pgfpathlineto{\pgfqpoint{2.742709in}{1.315012in}}%
\pgfpathlineto{\pgfqpoint{2.839099in}{1.300232in}}%
\pgfpathlineto{\pgfqpoint{2.935490in}{1.430300in}}%
\pgfpathlineto{\pgfqpoint{3.031880in}{1.524112in}}%
\pgfpathlineto{\pgfqpoint{3.128270in}{1.573446in}}%
\pgfpathlineto{\pgfqpoint{3.224660in}{1.586290in}}%
\pgfpathlineto{\pgfqpoint{3.321050in}{1.792897in}}%
\pgfpathlineto{\pgfqpoint{3.417441in}{1.790603in}}%
\pgfpathlineto{\pgfqpoint{3.513831in}{1.412769in}}%
\pgfpathlineto{\pgfqpoint{3.610221in}{1.478673in}}%
\pgfpathlineto{\pgfqpoint{3.706611in}{1.662140in}}%
\pgfpathlineto{\pgfqpoint{3.803002in}{1.590883in}}%
\pgfpathlineto{\pgfqpoint{3.899392in}{1.549731in}}%
\pgfpathlineto{\pgfqpoint{3.995782in}{1.667073in}}%
\pgfpathlineto{\pgfqpoint{4.092172in}{1.719883in}}%
\pgfpathlineto{\pgfqpoint{4.188563in}{1.983223in}}%
\pgfpathlineto{\pgfqpoint{4.284953in}{1.594706in}}%
\pgfpathlineto{\pgfqpoint{4.381343in}{1.636679in}}%
\pgfpathlineto{\pgfqpoint{4.477733in}{1.657175in}}%
\pgfpathlineto{\pgfqpoint{4.574124in}{1.727610in}}%
\pgfpathlineto{\pgfqpoint{4.670514in}{1.862113in}}%
\pgfpathlineto{\pgfqpoint{4.766904in}{1.895803in}}%
\pgfpathlineto{\pgfqpoint{4.863294in}{1.798811in}}%
\pgfpathlineto{\pgfqpoint{4.959684in}{1.888020in}}%
\pgfpathlineto{\pgfqpoint{5.056075in}{1.807954in}}%
\pgfpathlineto{\pgfqpoint{5.152465in}{1.872792in}}%
\pgfpathlineto{\pgfqpoint{5.248855in}{1.860272in}}%
\pgfpathlineto{\pgfqpoint{5.441636in}{2.253724in}}%
\pgfpathlineto{\pgfqpoint{5.538026in}{1.769576in}}%
\pgfpathlineto{\pgfqpoint{5.634416in}{2.100682in}}%
\pgfpathlineto{\pgfqpoint{5.827197in}{2.069415in}}%
\pgfpathlineto{\pgfqpoint{5.923587in}{2.023373in}}%
\pgfpathlineto{\pgfqpoint{6.019977in}{1.961588in}}%
\pgfpathlineto{\pgfqpoint{6.116367in}{2.170734in}}%
\pgfpathlineto{\pgfqpoint{6.212758in}{2.095159in}}%
\pgfpathlineto{\pgfqpoint{6.405538in}{2.121326in}}%
\pgfpathlineto{\pgfqpoint{6.598319in}{2.257224in}}%
\pgfpathlineto{\pgfqpoint{6.887489in}{2.150869in}}%
\pgfpathlineto{\pgfqpoint{7.369440in}{2.359833in}}%
\pgfusepath{stroke}%
\end{pgfscope}%
\begin{pgfscope}%
\pgfsetrectcap%
\pgfsetmiterjoin%
\pgfsetlinewidth{0.803000pt}%
\definecolor{currentstroke}{rgb}{0.000000,0.000000,0.000000}%
\pgfsetstrokecolor{currentstroke}%
\pgfsetdash{}{0pt}%
\pgfpathmoveto{\pgfqpoint{0.588387in}{0.521603in}}%
\pgfpathlineto{\pgfqpoint{0.588387in}{2.741849in}}%
\pgfusepath{stroke}%
\end{pgfscope}%
\begin{pgfscope}%
\pgfsetrectcap%
\pgfsetmiterjoin%
\pgfsetlinewidth{0.803000pt}%
\definecolor{currentstroke}{rgb}{0.000000,0.000000,0.000000}%
\pgfsetstrokecolor{currentstroke}%
\pgfsetdash{}{0pt}%
\pgfpathmoveto{\pgfqpoint{7.692348in}{0.521603in}}%
\pgfpathlineto{\pgfqpoint{7.692348in}{2.741849in}}%
\pgfusepath{stroke}%
\end{pgfscope}%
\begin{pgfscope}%
\pgfsetrectcap%
\pgfsetmiterjoin%
\pgfsetlinewidth{0.803000pt}%
\definecolor{currentstroke}{rgb}{0.000000,0.000000,0.000000}%
\pgfsetstrokecolor{currentstroke}%
\pgfsetdash{}{0pt}%
\pgfpathmoveto{\pgfqpoint{0.588387in}{0.521603in}}%
\pgfpathlineto{\pgfqpoint{7.692348in}{0.521603in}}%
\pgfusepath{stroke}%
\end{pgfscope}%
\begin{pgfscope}%
\pgfsetrectcap%
\pgfsetmiterjoin%
\pgfsetlinewidth{0.803000pt}%
\definecolor{currentstroke}{rgb}{0.000000,0.000000,0.000000}%
\pgfsetstrokecolor{currentstroke}%
\pgfsetdash{}{0pt}%
\pgfpathmoveto{\pgfqpoint{0.588387in}{2.741849in}}%
\pgfpathlineto{\pgfqpoint{7.692348in}{2.741849in}}%
\pgfusepath{stroke}%
\end{pgfscope}%
\begin{pgfscope}%
\pgfsetbuttcap%
\pgfsetmiterjoin%
\definecolor{currentfill}{rgb}{1.000000,1.000000,1.000000}%
\pgfsetfillcolor{currentfill}%
\pgfsetfillopacity{0.800000}%
\pgfsetlinewidth{1.003750pt}%
\definecolor{currentstroke}{rgb}{0.800000,0.800000,0.800000}%
\pgfsetstrokecolor{currentstroke}%
\pgfsetstrokeopacity{0.800000}%
\pgfsetdash{}{0pt}%
\pgfpathmoveto{\pgfqpoint{7.779848in}{1.541020in}}%
\pgfpathlineto{\pgfqpoint{8.259376in}{1.541020in}}%
\pgfpathquadraticcurveto{\pgfqpoint{8.284376in}{1.541020in}}{\pgfqpoint{8.284376in}{1.566020in}}%
\pgfpathlineto{\pgfqpoint{8.284376in}{2.654349in}}%
\pgfpathquadraticcurveto{\pgfqpoint{8.284376in}{2.679349in}}{\pgfqpoint{8.259376in}{2.679349in}}%
\pgfpathlineto{\pgfqpoint{7.779848in}{2.679349in}}%
\pgfpathquadraticcurveto{\pgfqpoint{7.754848in}{2.679349in}}{\pgfqpoint{7.754848in}{2.654349in}}%
\pgfpathlineto{\pgfqpoint{7.754848in}{1.566020in}}%
\pgfpathquadraticcurveto{\pgfqpoint{7.754848in}{1.541020in}}{\pgfqpoint{7.779848in}{1.541020in}}%
\pgfpathlineto{\pgfqpoint{7.779848in}{1.541020in}}%
\pgfpathclose%
\pgfusepath{stroke,fill}%
\end{pgfscope}%
\begin{pgfscope}%
\pgfsetrectcap%
\pgfsetroundjoin%
\pgfsetlinewidth{1.505625pt}%
\pgfsetstrokecolor{currentstroke1}%
\pgfsetdash{}{0pt}%
\pgfpathmoveto{\pgfqpoint{7.804848in}{2.578129in}}%
\pgfpathlineto{\pgfqpoint{7.929848in}{2.578129in}}%
\pgfpathlineto{\pgfqpoint{8.054848in}{2.578129in}}%
\pgfusepath{stroke}%
\end{pgfscope}%
\begin{pgfscope}%
\definecolor{textcolor}{rgb}{0.000000,0.000000,0.000000}%
\pgfsetstrokecolor{textcolor}%
\pgfsetfillcolor{textcolor}%
\pgftext[x=8.154848in,y=2.534379in,left,base]{\color{textcolor}{\rmfamily\fontsize{9.000000}{10.800000}\selectfont\catcode`\^=\active\def^{\ifmmode\sp\else\^{}\fi}\catcode`\%=\active\def%{\%}3}}%
\end{pgfscope}%
\begin{pgfscope}%
\pgfsetrectcap%
\pgfsetroundjoin%
\pgfsetlinewidth{1.505625pt}%
\pgfsetstrokecolor{currentstroke2}%
\pgfsetdash{}{0pt}%
\pgfpathmoveto{\pgfqpoint{7.804848in}{2.394657in}}%
\pgfpathlineto{\pgfqpoint{7.929848in}{2.394657in}}%
\pgfpathlineto{\pgfqpoint{8.054848in}{2.394657in}}%
\pgfusepath{stroke}%
\end{pgfscope}%
\begin{pgfscope}%
\definecolor{textcolor}{rgb}{0.000000,0.000000,0.000000}%
\pgfsetstrokecolor{textcolor}%
\pgfsetfillcolor{textcolor}%
\pgftext[x=8.154848in,y=2.350907in,left,base]{\color{textcolor}{\rmfamily\fontsize{9.000000}{10.800000}\selectfont\catcode`\^=\active\def^{\ifmmode\sp\else\^{}\fi}\catcode`\%=\active\def%{\%}4}}%
\end{pgfscope}%
\begin{pgfscope}%
\pgfsetrectcap%
\pgfsetroundjoin%
\pgfsetlinewidth{1.505625pt}%
\pgfsetstrokecolor{currentstroke3}%
\pgfsetdash{}{0pt}%
\pgfpathmoveto{\pgfqpoint{7.804848in}{2.211185in}}%
\pgfpathlineto{\pgfqpoint{7.929848in}{2.211185in}}%
\pgfpathlineto{\pgfqpoint{8.054848in}{2.211185in}}%
\pgfusepath{stroke}%
\end{pgfscope}%
\begin{pgfscope}%
\definecolor{textcolor}{rgb}{0.000000,0.000000,0.000000}%
\pgfsetstrokecolor{textcolor}%
\pgfsetfillcolor{textcolor}%
\pgftext[x=8.154848in,y=2.167435in,left,base]{\color{textcolor}{\rmfamily\fontsize{9.000000}{10.800000}\selectfont\catcode`\^=\active\def^{\ifmmode\sp\else\^{}\fi}\catcode`\%=\active\def%{\%}5}}%
\end{pgfscope}%
\begin{pgfscope}%
\pgfsetrectcap%
\pgfsetroundjoin%
\pgfsetlinewidth{1.505625pt}%
\pgfsetstrokecolor{currentstroke4}%
\pgfsetdash{}{0pt}%
\pgfpathmoveto{\pgfqpoint{7.804848in}{2.027714in}}%
\pgfpathlineto{\pgfqpoint{7.929848in}{2.027714in}}%
\pgfpathlineto{\pgfqpoint{8.054848in}{2.027714in}}%
\pgfusepath{stroke}%
\end{pgfscope}%
\begin{pgfscope}%
\definecolor{textcolor}{rgb}{0.000000,0.000000,0.000000}%
\pgfsetstrokecolor{textcolor}%
\pgfsetfillcolor{textcolor}%
\pgftext[x=8.154848in,y=1.983964in,left,base]{\color{textcolor}{\rmfamily\fontsize{9.000000}{10.800000}\selectfont\catcode`\^=\active\def^{\ifmmode\sp\else\^{}\fi}\catcode`\%=\active\def%{\%}6}}%
\end{pgfscope}%
\begin{pgfscope}%
\pgfsetrectcap%
\pgfsetroundjoin%
\pgfsetlinewidth{1.505625pt}%
\pgfsetstrokecolor{currentstroke5}%
\pgfsetdash{}{0pt}%
\pgfpathmoveto{\pgfqpoint{7.804848in}{1.844242in}}%
\pgfpathlineto{\pgfqpoint{7.929848in}{1.844242in}}%
\pgfpathlineto{\pgfqpoint{8.054848in}{1.844242in}}%
\pgfusepath{stroke}%
\end{pgfscope}%
\begin{pgfscope}%
\definecolor{textcolor}{rgb}{0.000000,0.000000,0.000000}%
\pgfsetstrokecolor{textcolor}%
\pgfsetfillcolor{textcolor}%
\pgftext[x=8.154848in,y=1.800492in,left,base]{\color{textcolor}{\rmfamily\fontsize{9.000000}{10.800000}\selectfont\catcode`\^=\active\def^{\ifmmode\sp\else\^{}\fi}\catcode`\%=\active\def%{\%}7}}%
\end{pgfscope}%
\begin{pgfscope}%
\pgfsetrectcap%
\pgfsetroundjoin%
\pgfsetlinewidth{1.505625pt}%
\pgfsetstrokecolor{currentstroke6}%
\pgfsetdash{}{0pt}%
\pgfpathmoveto{\pgfqpoint{7.804848in}{1.660771in}}%
\pgfpathlineto{\pgfqpoint{7.929848in}{1.660771in}}%
\pgfpathlineto{\pgfqpoint{8.054848in}{1.660771in}}%
\pgfusepath{stroke}%
\end{pgfscope}%
\begin{pgfscope}%
\definecolor{textcolor}{rgb}{0.000000,0.000000,0.000000}%
\pgfsetstrokecolor{textcolor}%
\pgfsetfillcolor{textcolor}%
\pgftext[x=8.154848in,y=1.617021in,left,base]{\color{textcolor}{\rmfamily\fontsize{9.000000}{10.800000}\selectfont\catcode`\^=\active\def^{\ifmmode\sp\else\^{}\fi}\catcode`\%=\active\def%{\%}8}}%
\end{pgfscope}%
\end{pgfpicture}%
\makeatother%
\endgroup%
}
	\caption[Mean runtime for graphs with no NAC-coloring (some).]{
		Mean running time (ms) to find all NAC-colorings for graphs with no NAC-coloring for different subgraph sizes \( k \).}%
	\label{fig:graph_no_nac_coloring_first_runtime_subgraph_size}
\end{figure}
\begin{figure}[p]
	\centering
	\scalebox{0.5}{%% Creator: Matplotlib, PGF backend
%%
%% To include the figure in your LaTeX document, write
%%   \input{<filename>.pgf}
%%
%% Make sure the required packages are loaded in your preamble
%%   \usepackage{pgf}
%%
%% Also ensure that all the required font packages are loaded; for instance,
%% the lmodern package is sometimes necessary when using math font.
%%   \usepackage{lmodern}
%%
%% Figures using additional raster images can only be included by \input if
%% they are in the same directory as the main LaTeX file. For loading figures
%% from other directories you can use the `import` package
%%   \usepackage{import}
%%
%% and then include the figures with
%%   \import{<path to file>}{<filename>.pgf}
%%
%% Matplotlib used the following preamble
%%   \def\mathdefault#1{#1}
%%   \everymath=\expandafter{\the\everymath\displaystyle}
%%   \IfFileExists{scrextend.sty}{
%%     \usepackage[fontsize=10.000000pt]{scrextend}
%%   }{
%%     \renewcommand{\normalsize}{\fontsize{10.000000}{12.000000}\selectfont}
%%     \normalsize
%%   }
%%   
%%   \ifdefined\pdftexversion\else  % non-pdftex case.
%%     \usepackage{fontspec}
%%     \setmainfont{DejaVuSans.ttf}[Path=\detokenize{/home/petr/Projects/PyRigi/.venv/lib/python3.12/site-packages/matplotlib/mpl-data/fonts/ttf/}]
%%     \setsansfont{DejaVuSans.ttf}[Path=\detokenize{/home/petr/Projects/PyRigi/.venv/lib/python3.12/site-packages/matplotlib/mpl-data/fonts/ttf/}]
%%     \setmonofont{DejaVuSansMono.ttf}[Path=\detokenize{/home/petr/Projects/PyRigi/.venv/lib/python3.12/site-packages/matplotlib/mpl-data/fonts/ttf/}]
%%   \fi
%%   \makeatletter\@ifpackageloaded{underscore}{}{\usepackage[strings]{underscore}}\makeatother
%%
\begingroup%
\makeatletter%
\begin{pgfpicture}%
\pgfpathrectangle{\pgfpointorigin}{\pgfqpoint{8.384376in}{2.841849in}}%
\pgfusepath{use as bounding box, clip}%
\begin{pgfscope}%
\pgfsetbuttcap%
\pgfsetmiterjoin%
\definecolor{currentfill}{rgb}{1.000000,1.000000,1.000000}%
\pgfsetfillcolor{currentfill}%
\pgfsetlinewidth{0.000000pt}%
\definecolor{currentstroke}{rgb}{1.000000,1.000000,1.000000}%
\pgfsetstrokecolor{currentstroke}%
\pgfsetdash{}{0pt}%
\pgfpathmoveto{\pgfqpoint{0.000000in}{0.000000in}}%
\pgfpathlineto{\pgfqpoint{8.384376in}{0.000000in}}%
\pgfpathlineto{\pgfqpoint{8.384376in}{2.841849in}}%
\pgfpathlineto{\pgfqpoint{0.000000in}{2.841849in}}%
\pgfpathlineto{\pgfqpoint{0.000000in}{0.000000in}}%
\pgfpathclose%
\pgfusepath{fill}%
\end{pgfscope}%
\begin{pgfscope}%
\pgfsetbuttcap%
\pgfsetmiterjoin%
\definecolor{currentfill}{rgb}{1.000000,1.000000,1.000000}%
\pgfsetfillcolor{currentfill}%
\pgfsetlinewidth{0.000000pt}%
\definecolor{currentstroke}{rgb}{0.000000,0.000000,0.000000}%
\pgfsetstrokecolor{currentstroke}%
\pgfsetstrokeopacity{0.000000}%
\pgfsetdash{}{0pt}%
\pgfpathmoveto{\pgfqpoint{0.588387in}{0.521603in}}%
\pgfpathlineto{\pgfqpoint{7.692348in}{0.521603in}}%
\pgfpathlineto{\pgfqpoint{7.692348in}{2.531888in}}%
\pgfpathlineto{\pgfqpoint{0.588387in}{2.531888in}}%
\pgfpathlineto{\pgfqpoint{0.588387in}{0.521603in}}%
\pgfpathclose%
\pgfusepath{fill}%
\end{pgfscope}%
\begin{pgfscope}%
\pgfsetbuttcap%
\pgfsetroundjoin%
\definecolor{currentfill}{rgb}{0.000000,0.000000,0.000000}%
\pgfsetfillcolor{currentfill}%
\pgfsetlinewidth{0.803000pt}%
\definecolor{currentstroke}{rgb}{0.000000,0.000000,0.000000}%
\pgfsetstrokecolor{currentstroke}%
\pgfsetdash{}{0pt}%
\pgfsys@defobject{currentmarker}{\pgfqpoint{0.000000in}{-0.048611in}}{\pgfqpoint{0.000000in}{0.000000in}}{%
\pgfpathmoveto{\pgfqpoint{0.000000in}{0.000000in}}%
\pgfpathlineto{\pgfqpoint{0.000000in}{-0.048611in}}%
\pgfusepath{stroke,fill}%
}%
\begin{pgfscope}%
\pgfsys@transformshift{1.241273in}{0.521603in}%
\pgfsys@useobject{currentmarker}{}%
\end{pgfscope}%
\end{pgfscope}%
\begin{pgfscope}%
\definecolor{textcolor}{rgb}{0.000000,0.000000,0.000000}%
\pgfsetstrokecolor{textcolor}%
\pgfsetfillcolor{textcolor}%
\pgftext[x=1.241273in,y=0.424381in,,top]{\color{textcolor}{\rmfamily\fontsize{10.000000}{12.000000}\selectfont\catcode`\^=\active\def^{\ifmmode\sp\else\^{}\fi}\catcode`\%=\active\def%{\%}$\mathdefault{20}$}}%
\end{pgfscope}%
\begin{pgfscope}%
\pgfsetbuttcap%
\pgfsetroundjoin%
\definecolor{currentfill}{rgb}{0.000000,0.000000,0.000000}%
\pgfsetfillcolor{currentfill}%
\pgfsetlinewidth{0.803000pt}%
\definecolor{currentstroke}{rgb}{0.000000,0.000000,0.000000}%
\pgfsetstrokecolor{currentstroke}%
\pgfsetdash{}{0pt}%
\pgfsys@defobject{currentmarker}{\pgfqpoint{0.000000in}{-0.048611in}}{\pgfqpoint{0.000000in}{0.000000in}}{%
\pgfpathmoveto{\pgfqpoint{0.000000in}{0.000000in}}%
\pgfpathlineto{\pgfqpoint{0.000000in}{-0.048611in}}%
\pgfusepath{stroke,fill}%
}%
\begin{pgfscope}%
\pgfsys@transformshift{2.184068in}{0.521603in}%
\pgfsys@useobject{currentmarker}{}%
\end{pgfscope}%
\end{pgfscope}%
\begin{pgfscope}%
\definecolor{textcolor}{rgb}{0.000000,0.000000,0.000000}%
\pgfsetstrokecolor{textcolor}%
\pgfsetfillcolor{textcolor}%
\pgftext[x=2.184068in,y=0.424381in,,top]{\color{textcolor}{\rmfamily\fontsize{10.000000}{12.000000}\selectfont\catcode`\^=\active\def^{\ifmmode\sp\else\^{}\fi}\catcode`\%=\active\def%{\%}$\mathdefault{40}$}}%
\end{pgfscope}%
\begin{pgfscope}%
\pgfsetbuttcap%
\pgfsetroundjoin%
\definecolor{currentfill}{rgb}{0.000000,0.000000,0.000000}%
\pgfsetfillcolor{currentfill}%
\pgfsetlinewidth{0.803000pt}%
\definecolor{currentstroke}{rgb}{0.000000,0.000000,0.000000}%
\pgfsetstrokecolor{currentstroke}%
\pgfsetdash{}{0pt}%
\pgfsys@defobject{currentmarker}{\pgfqpoint{0.000000in}{-0.048611in}}{\pgfqpoint{0.000000in}{0.000000in}}{%
\pgfpathmoveto{\pgfqpoint{0.000000in}{0.000000in}}%
\pgfpathlineto{\pgfqpoint{0.000000in}{-0.048611in}}%
\pgfusepath{stroke,fill}%
}%
\begin{pgfscope}%
\pgfsys@transformshift{3.126863in}{0.521603in}%
\pgfsys@useobject{currentmarker}{}%
\end{pgfscope}%
\end{pgfscope}%
\begin{pgfscope}%
\definecolor{textcolor}{rgb}{0.000000,0.000000,0.000000}%
\pgfsetstrokecolor{textcolor}%
\pgfsetfillcolor{textcolor}%
\pgftext[x=3.126863in,y=0.424381in,,top]{\color{textcolor}{\rmfamily\fontsize{10.000000}{12.000000}\selectfont\catcode`\^=\active\def^{\ifmmode\sp\else\^{}\fi}\catcode`\%=\active\def%{\%}$\mathdefault{60}$}}%
\end{pgfscope}%
\begin{pgfscope}%
\pgfsetbuttcap%
\pgfsetroundjoin%
\definecolor{currentfill}{rgb}{0.000000,0.000000,0.000000}%
\pgfsetfillcolor{currentfill}%
\pgfsetlinewidth{0.803000pt}%
\definecolor{currentstroke}{rgb}{0.000000,0.000000,0.000000}%
\pgfsetstrokecolor{currentstroke}%
\pgfsetdash{}{0pt}%
\pgfsys@defobject{currentmarker}{\pgfqpoint{0.000000in}{-0.048611in}}{\pgfqpoint{0.000000in}{0.000000in}}{%
\pgfpathmoveto{\pgfqpoint{0.000000in}{0.000000in}}%
\pgfpathlineto{\pgfqpoint{0.000000in}{-0.048611in}}%
\pgfusepath{stroke,fill}%
}%
\begin{pgfscope}%
\pgfsys@transformshift{4.069658in}{0.521603in}%
\pgfsys@useobject{currentmarker}{}%
\end{pgfscope}%
\end{pgfscope}%
\begin{pgfscope}%
\definecolor{textcolor}{rgb}{0.000000,0.000000,0.000000}%
\pgfsetstrokecolor{textcolor}%
\pgfsetfillcolor{textcolor}%
\pgftext[x=4.069658in,y=0.424381in,,top]{\color{textcolor}{\rmfamily\fontsize{10.000000}{12.000000}\selectfont\catcode`\^=\active\def^{\ifmmode\sp\else\^{}\fi}\catcode`\%=\active\def%{\%}$\mathdefault{80}$}}%
\end{pgfscope}%
\begin{pgfscope}%
\pgfsetbuttcap%
\pgfsetroundjoin%
\definecolor{currentfill}{rgb}{0.000000,0.000000,0.000000}%
\pgfsetfillcolor{currentfill}%
\pgfsetlinewidth{0.803000pt}%
\definecolor{currentstroke}{rgb}{0.000000,0.000000,0.000000}%
\pgfsetstrokecolor{currentstroke}%
\pgfsetdash{}{0pt}%
\pgfsys@defobject{currentmarker}{\pgfqpoint{0.000000in}{-0.048611in}}{\pgfqpoint{0.000000in}{0.000000in}}{%
\pgfpathmoveto{\pgfqpoint{0.000000in}{0.000000in}}%
\pgfpathlineto{\pgfqpoint{0.000000in}{-0.048611in}}%
\pgfusepath{stroke,fill}%
}%
\begin{pgfscope}%
\pgfsys@transformshift{5.012453in}{0.521603in}%
\pgfsys@useobject{currentmarker}{}%
\end{pgfscope}%
\end{pgfscope}%
\begin{pgfscope}%
\definecolor{textcolor}{rgb}{0.000000,0.000000,0.000000}%
\pgfsetstrokecolor{textcolor}%
\pgfsetfillcolor{textcolor}%
\pgftext[x=5.012453in,y=0.424381in,,top]{\color{textcolor}{\rmfamily\fontsize{10.000000}{12.000000}\selectfont\catcode`\^=\active\def^{\ifmmode\sp\else\^{}\fi}\catcode`\%=\active\def%{\%}$\mathdefault{100}$}}%
\end{pgfscope}%
\begin{pgfscope}%
\pgfsetbuttcap%
\pgfsetroundjoin%
\definecolor{currentfill}{rgb}{0.000000,0.000000,0.000000}%
\pgfsetfillcolor{currentfill}%
\pgfsetlinewidth{0.803000pt}%
\definecolor{currentstroke}{rgb}{0.000000,0.000000,0.000000}%
\pgfsetstrokecolor{currentstroke}%
\pgfsetdash{}{0pt}%
\pgfsys@defobject{currentmarker}{\pgfqpoint{0.000000in}{-0.048611in}}{\pgfqpoint{0.000000in}{0.000000in}}{%
\pgfpathmoveto{\pgfqpoint{0.000000in}{0.000000in}}%
\pgfpathlineto{\pgfqpoint{0.000000in}{-0.048611in}}%
\pgfusepath{stroke,fill}%
}%
\begin{pgfscope}%
\pgfsys@transformshift{5.955248in}{0.521603in}%
\pgfsys@useobject{currentmarker}{}%
\end{pgfscope}%
\end{pgfscope}%
\begin{pgfscope}%
\definecolor{textcolor}{rgb}{0.000000,0.000000,0.000000}%
\pgfsetstrokecolor{textcolor}%
\pgfsetfillcolor{textcolor}%
\pgftext[x=5.955248in,y=0.424381in,,top]{\color{textcolor}{\rmfamily\fontsize{10.000000}{12.000000}\selectfont\catcode`\^=\active\def^{\ifmmode\sp\else\^{}\fi}\catcode`\%=\active\def%{\%}$\mathdefault{120}$}}%
\end{pgfscope}%
\begin{pgfscope}%
\pgfsetbuttcap%
\pgfsetroundjoin%
\definecolor{currentfill}{rgb}{0.000000,0.000000,0.000000}%
\pgfsetfillcolor{currentfill}%
\pgfsetlinewidth{0.803000pt}%
\definecolor{currentstroke}{rgb}{0.000000,0.000000,0.000000}%
\pgfsetstrokecolor{currentstroke}%
\pgfsetdash{}{0pt}%
\pgfsys@defobject{currentmarker}{\pgfqpoint{0.000000in}{-0.048611in}}{\pgfqpoint{0.000000in}{0.000000in}}{%
\pgfpathmoveto{\pgfqpoint{0.000000in}{0.000000in}}%
\pgfpathlineto{\pgfqpoint{0.000000in}{-0.048611in}}%
\pgfusepath{stroke,fill}%
}%
\begin{pgfscope}%
\pgfsys@transformshift{6.898043in}{0.521603in}%
\pgfsys@useobject{currentmarker}{}%
\end{pgfscope}%
\end{pgfscope}%
\begin{pgfscope}%
\definecolor{textcolor}{rgb}{0.000000,0.000000,0.000000}%
\pgfsetstrokecolor{textcolor}%
\pgfsetfillcolor{textcolor}%
\pgftext[x=6.898043in,y=0.424381in,,top]{\color{textcolor}{\rmfamily\fontsize{10.000000}{12.000000}\selectfont\catcode`\^=\active\def^{\ifmmode\sp\else\^{}\fi}\catcode`\%=\active\def%{\%}$\mathdefault{140}$}}%
\end{pgfscope}%
\begin{pgfscope}%
\definecolor{textcolor}{rgb}{0.000000,0.000000,0.000000}%
\pgfsetstrokecolor{textcolor}%
\pgfsetfillcolor{textcolor}%
\pgftext[x=4.140367in,y=0.234413in,,top]{\color{textcolor}{\rmfamily\fontsize{10.000000}{12.000000}\selectfont\catcode`\^=\active\def^{\ifmmode\sp\else\^{}\fi}\catcode`\%=\active\def%{\%}Triangle components}}%
\end{pgfscope}%
\begin{pgfscope}%
\pgfsetbuttcap%
\pgfsetroundjoin%
\definecolor{currentfill}{rgb}{0.000000,0.000000,0.000000}%
\pgfsetfillcolor{currentfill}%
\pgfsetlinewidth{0.803000pt}%
\definecolor{currentstroke}{rgb}{0.000000,0.000000,0.000000}%
\pgfsetstrokecolor{currentstroke}%
\pgfsetdash{}{0pt}%
\pgfsys@defobject{currentmarker}{\pgfqpoint{-0.048611in}{0.000000in}}{\pgfqpoint{-0.000000in}{0.000000in}}{%
\pgfpathmoveto{\pgfqpoint{-0.000000in}{0.000000in}}%
\pgfpathlineto{\pgfqpoint{-0.048611in}{0.000000in}}%
\pgfusepath{stroke,fill}%
}%
\begin{pgfscope}%
\pgfsys@transformshift{0.588387in}{0.943926in}%
\pgfsys@useobject{currentmarker}{}%
\end{pgfscope}%
\end{pgfscope}%
\begin{pgfscope}%
\definecolor{textcolor}{rgb}{0.000000,0.000000,0.000000}%
\pgfsetstrokecolor{textcolor}%
\pgfsetfillcolor{textcolor}%
\pgftext[x=0.289968in, y=0.891164in, left, base]{\color{textcolor}{\rmfamily\fontsize{10.000000}{12.000000}\selectfont\catcode`\^=\active\def^{\ifmmode\sp\else\^{}\fi}\catcode`\%=\active\def%{\%}$\mathdefault{10^{2}}$}}%
\end{pgfscope}%
\begin{pgfscope}%
\pgfsetbuttcap%
\pgfsetroundjoin%
\definecolor{currentfill}{rgb}{0.000000,0.000000,0.000000}%
\pgfsetfillcolor{currentfill}%
\pgfsetlinewidth{0.803000pt}%
\definecolor{currentstroke}{rgb}{0.000000,0.000000,0.000000}%
\pgfsetstrokecolor{currentstroke}%
\pgfsetdash{}{0pt}%
\pgfsys@defobject{currentmarker}{\pgfqpoint{-0.048611in}{0.000000in}}{\pgfqpoint{-0.000000in}{0.000000in}}{%
\pgfpathmoveto{\pgfqpoint{-0.000000in}{0.000000in}}%
\pgfpathlineto{\pgfqpoint{-0.048611in}{0.000000in}}%
\pgfusepath{stroke,fill}%
}%
\begin{pgfscope}%
\pgfsys@transformshift{0.588387in}{1.872122in}%
\pgfsys@useobject{currentmarker}{}%
\end{pgfscope}%
\end{pgfscope}%
\begin{pgfscope}%
\definecolor{textcolor}{rgb}{0.000000,0.000000,0.000000}%
\pgfsetstrokecolor{textcolor}%
\pgfsetfillcolor{textcolor}%
\pgftext[x=0.289968in, y=1.819360in, left, base]{\color{textcolor}{\rmfamily\fontsize{10.000000}{12.000000}\selectfont\catcode`\^=\active\def^{\ifmmode\sp\else\^{}\fi}\catcode`\%=\active\def%{\%}$\mathdefault{10^{3}}$}}%
\end{pgfscope}%
\begin{pgfscope}%
\pgfsetbuttcap%
\pgfsetroundjoin%
\definecolor{currentfill}{rgb}{0.000000,0.000000,0.000000}%
\pgfsetfillcolor{currentfill}%
\pgfsetlinewidth{0.602250pt}%
\definecolor{currentstroke}{rgb}{0.000000,0.000000,0.000000}%
\pgfsetstrokecolor{currentstroke}%
\pgfsetdash{}{0pt}%
\pgfsys@defobject{currentmarker}{\pgfqpoint{-0.027778in}{0.000000in}}{\pgfqpoint{-0.000000in}{0.000000in}}{%
\pgfpathmoveto{\pgfqpoint{-0.000000in}{0.000000in}}%
\pgfpathlineto{\pgfqpoint{-0.027778in}{0.000000in}}%
\pgfusepath{stroke,fill}%
}%
\begin{pgfscope}%
\pgfsys@transformshift{0.588387in}{0.574559in}%
\pgfsys@useobject{currentmarker}{}%
\end{pgfscope}%
\end{pgfscope}%
\begin{pgfscope}%
\pgfsetbuttcap%
\pgfsetroundjoin%
\definecolor{currentfill}{rgb}{0.000000,0.000000,0.000000}%
\pgfsetfillcolor{currentfill}%
\pgfsetlinewidth{0.602250pt}%
\definecolor{currentstroke}{rgb}{0.000000,0.000000,0.000000}%
\pgfsetstrokecolor{currentstroke}%
\pgfsetdash{}{0pt}%
\pgfsys@defobject{currentmarker}{\pgfqpoint{-0.027778in}{0.000000in}}{\pgfqpoint{-0.000000in}{0.000000in}}{%
\pgfpathmoveto{\pgfqpoint{-0.000000in}{0.000000in}}%
\pgfpathlineto{\pgfqpoint{-0.027778in}{0.000000in}}%
\pgfusepath{stroke,fill}%
}%
\begin{pgfscope}%
\pgfsys@transformshift{0.588387in}{0.664511in}%
\pgfsys@useobject{currentmarker}{}%
\end{pgfscope}%
\end{pgfscope}%
\begin{pgfscope}%
\pgfsetbuttcap%
\pgfsetroundjoin%
\definecolor{currentfill}{rgb}{0.000000,0.000000,0.000000}%
\pgfsetfillcolor{currentfill}%
\pgfsetlinewidth{0.602250pt}%
\definecolor{currentstroke}{rgb}{0.000000,0.000000,0.000000}%
\pgfsetstrokecolor{currentstroke}%
\pgfsetdash{}{0pt}%
\pgfsys@defobject{currentmarker}{\pgfqpoint{-0.027778in}{0.000000in}}{\pgfqpoint{-0.000000in}{0.000000in}}{%
\pgfpathmoveto{\pgfqpoint{-0.000000in}{0.000000in}}%
\pgfpathlineto{\pgfqpoint{-0.027778in}{0.000000in}}%
\pgfusepath{stroke,fill}%
}%
\begin{pgfscope}%
\pgfsys@transformshift{0.588387in}{0.738007in}%
\pgfsys@useobject{currentmarker}{}%
\end{pgfscope}%
\end{pgfscope}%
\begin{pgfscope}%
\pgfsetbuttcap%
\pgfsetroundjoin%
\definecolor{currentfill}{rgb}{0.000000,0.000000,0.000000}%
\pgfsetfillcolor{currentfill}%
\pgfsetlinewidth{0.602250pt}%
\definecolor{currentstroke}{rgb}{0.000000,0.000000,0.000000}%
\pgfsetstrokecolor{currentstroke}%
\pgfsetdash{}{0pt}%
\pgfsys@defobject{currentmarker}{\pgfqpoint{-0.027778in}{0.000000in}}{\pgfqpoint{-0.000000in}{0.000000in}}{%
\pgfpathmoveto{\pgfqpoint{-0.000000in}{0.000000in}}%
\pgfpathlineto{\pgfqpoint{-0.027778in}{0.000000in}}%
\pgfusepath{stroke,fill}%
}%
\begin{pgfscope}%
\pgfsys@transformshift{0.588387in}{0.800146in}%
\pgfsys@useobject{currentmarker}{}%
\end{pgfscope}%
\end{pgfscope}%
\begin{pgfscope}%
\pgfsetbuttcap%
\pgfsetroundjoin%
\definecolor{currentfill}{rgb}{0.000000,0.000000,0.000000}%
\pgfsetfillcolor{currentfill}%
\pgfsetlinewidth{0.602250pt}%
\definecolor{currentstroke}{rgb}{0.000000,0.000000,0.000000}%
\pgfsetstrokecolor{currentstroke}%
\pgfsetdash{}{0pt}%
\pgfsys@defobject{currentmarker}{\pgfqpoint{-0.027778in}{0.000000in}}{\pgfqpoint{-0.000000in}{0.000000in}}{%
\pgfpathmoveto{\pgfqpoint{-0.000000in}{0.000000in}}%
\pgfpathlineto{\pgfqpoint{-0.027778in}{0.000000in}}%
\pgfusepath{stroke,fill}%
}%
\begin{pgfscope}%
\pgfsys@transformshift{0.588387in}{0.853974in}%
\pgfsys@useobject{currentmarker}{}%
\end{pgfscope}%
\end{pgfscope}%
\begin{pgfscope}%
\pgfsetbuttcap%
\pgfsetroundjoin%
\definecolor{currentfill}{rgb}{0.000000,0.000000,0.000000}%
\pgfsetfillcolor{currentfill}%
\pgfsetlinewidth{0.602250pt}%
\definecolor{currentstroke}{rgb}{0.000000,0.000000,0.000000}%
\pgfsetstrokecolor{currentstroke}%
\pgfsetdash{}{0pt}%
\pgfsys@defobject{currentmarker}{\pgfqpoint{-0.027778in}{0.000000in}}{\pgfqpoint{-0.000000in}{0.000000in}}{%
\pgfpathmoveto{\pgfqpoint{-0.000000in}{0.000000in}}%
\pgfpathlineto{\pgfqpoint{-0.027778in}{0.000000in}}%
\pgfusepath{stroke,fill}%
}%
\begin{pgfscope}%
\pgfsys@transformshift{0.588387in}{0.901454in}%
\pgfsys@useobject{currentmarker}{}%
\end{pgfscope}%
\end{pgfscope}%
\begin{pgfscope}%
\pgfsetbuttcap%
\pgfsetroundjoin%
\definecolor{currentfill}{rgb}{0.000000,0.000000,0.000000}%
\pgfsetfillcolor{currentfill}%
\pgfsetlinewidth{0.602250pt}%
\definecolor{currentstroke}{rgb}{0.000000,0.000000,0.000000}%
\pgfsetstrokecolor{currentstroke}%
\pgfsetdash{}{0pt}%
\pgfsys@defobject{currentmarker}{\pgfqpoint{-0.027778in}{0.000000in}}{\pgfqpoint{-0.000000in}{0.000000in}}{%
\pgfpathmoveto{\pgfqpoint{-0.000000in}{0.000000in}}%
\pgfpathlineto{\pgfqpoint{-0.027778in}{0.000000in}}%
\pgfusepath{stroke,fill}%
}%
\begin{pgfscope}%
\pgfsys@transformshift{0.588387in}{1.223341in}%
\pgfsys@useobject{currentmarker}{}%
\end{pgfscope}%
\end{pgfscope}%
\begin{pgfscope}%
\pgfsetbuttcap%
\pgfsetroundjoin%
\definecolor{currentfill}{rgb}{0.000000,0.000000,0.000000}%
\pgfsetfillcolor{currentfill}%
\pgfsetlinewidth{0.602250pt}%
\definecolor{currentstroke}{rgb}{0.000000,0.000000,0.000000}%
\pgfsetstrokecolor{currentstroke}%
\pgfsetdash{}{0pt}%
\pgfsys@defobject{currentmarker}{\pgfqpoint{-0.027778in}{0.000000in}}{\pgfqpoint{-0.000000in}{0.000000in}}{%
\pgfpathmoveto{\pgfqpoint{-0.000000in}{0.000000in}}%
\pgfpathlineto{\pgfqpoint{-0.027778in}{0.000000in}}%
\pgfusepath{stroke,fill}%
}%
\begin{pgfscope}%
\pgfsys@transformshift{0.588387in}{1.386788in}%
\pgfsys@useobject{currentmarker}{}%
\end{pgfscope}%
\end{pgfscope}%
\begin{pgfscope}%
\pgfsetbuttcap%
\pgfsetroundjoin%
\definecolor{currentfill}{rgb}{0.000000,0.000000,0.000000}%
\pgfsetfillcolor{currentfill}%
\pgfsetlinewidth{0.602250pt}%
\definecolor{currentstroke}{rgb}{0.000000,0.000000,0.000000}%
\pgfsetstrokecolor{currentstroke}%
\pgfsetdash{}{0pt}%
\pgfsys@defobject{currentmarker}{\pgfqpoint{-0.027778in}{0.000000in}}{\pgfqpoint{-0.000000in}{0.000000in}}{%
\pgfpathmoveto{\pgfqpoint{-0.000000in}{0.000000in}}%
\pgfpathlineto{\pgfqpoint{-0.027778in}{0.000000in}}%
\pgfusepath{stroke,fill}%
}%
\begin{pgfscope}%
\pgfsys@transformshift{0.588387in}{1.502755in}%
\pgfsys@useobject{currentmarker}{}%
\end{pgfscope}%
\end{pgfscope}%
\begin{pgfscope}%
\pgfsetbuttcap%
\pgfsetroundjoin%
\definecolor{currentfill}{rgb}{0.000000,0.000000,0.000000}%
\pgfsetfillcolor{currentfill}%
\pgfsetlinewidth{0.602250pt}%
\definecolor{currentstroke}{rgb}{0.000000,0.000000,0.000000}%
\pgfsetstrokecolor{currentstroke}%
\pgfsetdash{}{0pt}%
\pgfsys@defobject{currentmarker}{\pgfqpoint{-0.027778in}{0.000000in}}{\pgfqpoint{-0.000000in}{0.000000in}}{%
\pgfpathmoveto{\pgfqpoint{-0.000000in}{0.000000in}}%
\pgfpathlineto{\pgfqpoint{-0.027778in}{0.000000in}}%
\pgfusepath{stroke,fill}%
}%
\begin{pgfscope}%
\pgfsys@transformshift{0.588387in}{1.592707in}%
\pgfsys@useobject{currentmarker}{}%
\end{pgfscope}%
\end{pgfscope}%
\begin{pgfscope}%
\pgfsetbuttcap%
\pgfsetroundjoin%
\definecolor{currentfill}{rgb}{0.000000,0.000000,0.000000}%
\pgfsetfillcolor{currentfill}%
\pgfsetlinewidth{0.602250pt}%
\definecolor{currentstroke}{rgb}{0.000000,0.000000,0.000000}%
\pgfsetstrokecolor{currentstroke}%
\pgfsetdash{}{0pt}%
\pgfsys@defobject{currentmarker}{\pgfqpoint{-0.027778in}{0.000000in}}{\pgfqpoint{-0.000000in}{0.000000in}}{%
\pgfpathmoveto{\pgfqpoint{-0.000000in}{0.000000in}}%
\pgfpathlineto{\pgfqpoint{-0.027778in}{0.000000in}}%
\pgfusepath{stroke,fill}%
}%
\begin{pgfscope}%
\pgfsys@transformshift{0.588387in}{1.666203in}%
\pgfsys@useobject{currentmarker}{}%
\end{pgfscope}%
\end{pgfscope}%
\begin{pgfscope}%
\pgfsetbuttcap%
\pgfsetroundjoin%
\definecolor{currentfill}{rgb}{0.000000,0.000000,0.000000}%
\pgfsetfillcolor{currentfill}%
\pgfsetlinewidth{0.602250pt}%
\definecolor{currentstroke}{rgb}{0.000000,0.000000,0.000000}%
\pgfsetstrokecolor{currentstroke}%
\pgfsetdash{}{0pt}%
\pgfsys@defobject{currentmarker}{\pgfqpoint{-0.027778in}{0.000000in}}{\pgfqpoint{-0.000000in}{0.000000in}}{%
\pgfpathmoveto{\pgfqpoint{-0.000000in}{0.000000in}}%
\pgfpathlineto{\pgfqpoint{-0.027778in}{0.000000in}}%
\pgfusepath{stroke,fill}%
}%
\begin{pgfscope}%
\pgfsys@transformshift{0.588387in}{1.728342in}%
\pgfsys@useobject{currentmarker}{}%
\end{pgfscope}%
\end{pgfscope}%
\begin{pgfscope}%
\pgfsetbuttcap%
\pgfsetroundjoin%
\definecolor{currentfill}{rgb}{0.000000,0.000000,0.000000}%
\pgfsetfillcolor{currentfill}%
\pgfsetlinewidth{0.602250pt}%
\definecolor{currentstroke}{rgb}{0.000000,0.000000,0.000000}%
\pgfsetstrokecolor{currentstroke}%
\pgfsetdash{}{0pt}%
\pgfsys@defobject{currentmarker}{\pgfqpoint{-0.027778in}{0.000000in}}{\pgfqpoint{-0.000000in}{0.000000in}}{%
\pgfpathmoveto{\pgfqpoint{-0.000000in}{0.000000in}}%
\pgfpathlineto{\pgfqpoint{-0.027778in}{0.000000in}}%
\pgfusepath{stroke,fill}%
}%
\begin{pgfscope}%
\pgfsys@transformshift{0.588387in}{1.782170in}%
\pgfsys@useobject{currentmarker}{}%
\end{pgfscope}%
\end{pgfscope}%
\begin{pgfscope}%
\pgfsetbuttcap%
\pgfsetroundjoin%
\definecolor{currentfill}{rgb}{0.000000,0.000000,0.000000}%
\pgfsetfillcolor{currentfill}%
\pgfsetlinewidth{0.602250pt}%
\definecolor{currentstroke}{rgb}{0.000000,0.000000,0.000000}%
\pgfsetstrokecolor{currentstroke}%
\pgfsetdash{}{0pt}%
\pgfsys@defobject{currentmarker}{\pgfqpoint{-0.027778in}{0.000000in}}{\pgfqpoint{-0.000000in}{0.000000in}}{%
\pgfpathmoveto{\pgfqpoint{-0.000000in}{0.000000in}}%
\pgfpathlineto{\pgfqpoint{-0.027778in}{0.000000in}}%
\pgfusepath{stroke,fill}%
}%
\begin{pgfscope}%
\pgfsys@transformshift{0.588387in}{1.829650in}%
\pgfsys@useobject{currentmarker}{}%
\end{pgfscope}%
\end{pgfscope}%
\begin{pgfscope}%
\pgfsetbuttcap%
\pgfsetroundjoin%
\definecolor{currentfill}{rgb}{0.000000,0.000000,0.000000}%
\pgfsetfillcolor{currentfill}%
\pgfsetlinewidth{0.602250pt}%
\definecolor{currentstroke}{rgb}{0.000000,0.000000,0.000000}%
\pgfsetstrokecolor{currentstroke}%
\pgfsetdash{}{0pt}%
\pgfsys@defobject{currentmarker}{\pgfqpoint{-0.027778in}{0.000000in}}{\pgfqpoint{-0.000000in}{0.000000in}}{%
\pgfpathmoveto{\pgfqpoint{-0.000000in}{0.000000in}}%
\pgfpathlineto{\pgfqpoint{-0.027778in}{0.000000in}}%
\pgfusepath{stroke,fill}%
}%
\begin{pgfscope}%
\pgfsys@transformshift{0.588387in}{2.151536in}%
\pgfsys@useobject{currentmarker}{}%
\end{pgfscope}%
\end{pgfscope}%
\begin{pgfscope}%
\pgfsetbuttcap%
\pgfsetroundjoin%
\definecolor{currentfill}{rgb}{0.000000,0.000000,0.000000}%
\pgfsetfillcolor{currentfill}%
\pgfsetlinewidth{0.602250pt}%
\definecolor{currentstroke}{rgb}{0.000000,0.000000,0.000000}%
\pgfsetstrokecolor{currentstroke}%
\pgfsetdash{}{0pt}%
\pgfsys@defobject{currentmarker}{\pgfqpoint{-0.027778in}{0.000000in}}{\pgfqpoint{-0.000000in}{0.000000in}}{%
\pgfpathmoveto{\pgfqpoint{-0.000000in}{0.000000in}}%
\pgfpathlineto{\pgfqpoint{-0.027778in}{0.000000in}}%
\pgfusepath{stroke,fill}%
}%
\begin{pgfscope}%
\pgfsys@transformshift{0.588387in}{2.314984in}%
\pgfsys@useobject{currentmarker}{}%
\end{pgfscope}%
\end{pgfscope}%
\begin{pgfscope}%
\pgfsetbuttcap%
\pgfsetroundjoin%
\definecolor{currentfill}{rgb}{0.000000,0.000000,0.000000}%
\pgfsetfillcolor{currentfill}%
\pgfsetlinewidth{0.602250pt}%
\definecolor{currentstroke}{rgb}{0.000000,0.000000,0.000000}%
\pgfsetstrokecolor{currentstroke}%
\pgfsetdash{}{0pt}%
\pgfsys@defobject{currentmarker}{\pgfqpoint{-0.027778in}{0.000000in}}{\pgfqpoint{-0.000000in}{0.000000in}}{%
\pgfpathmoveto{\pgfqpoint{-0.000000in}{0.000000in}}%
\pgfpathlineto{\pgfqpoint{-0.027778in}{0.000000in}}%
\pgfusepath{stroke,fill}%
}%
\begin{pgfscope}%
\pgfsys@transformshift{0.588387in}{2.430951in}%
\pgfsys@useobject{currentmarker}{}%
\end{pgfscope}%
\end{pgfscope}%
\begin{pgfscope}%
\pgfsetbuttcap%
\pgfsetroundjoin%
\definecolor{currentfill}{rgb}{0.000000,0.000000,0.000000}%
\pgfsetfillcolor{currentfill}%
\pgfsetlinewidth{0.602250pt}%
\definecolor{currentstroke}{rgb}{0.000000,0.000000,0.000000}%
\pgfsetstrokecolor{currentstroke}%
\pgfsetdash{}{0pt}%
\pgfsys@defobject{currentmarker}{\pgfqpoint{-0.027778in}{0.000000in}}{\pgfqpoint{-0.000000in}{0.000000in}}{%
\pgfpathmoveto{\pgfqpoint{-0.000000in}{0.000000in}}%
\pgfpathlineto{\pgfqpoint{-0.027778in}{0.000000in}}%
\pgfusepath{stroke,fill}%
}%
\begin{pgfscope}%
\pgfsys@transformshift{0.588387in}{2.520903in}%
\pgfsys@useobject{currentmarker}{}%
\end{pgfscope}%
\end{pgfscope}%
\begin{pgfscope}%
\definecolor{textcolor}{rgb}{0.000000,0.000000,0.000000}%
\pgfsetstrokecolor{textcolor}%
\pgfsetfillcolor{textcolor}%
\pgftext[x=0.234413in,y=1.526746in,,bottom,rotate=90.000000]{\color{textcolor}{\rmfamily\fontsize{10.000000}{12.000000}\selectfont\catcode`\^=\active\def^{\ifmmode\sp\else\^{}\fi}\catcode`\%=\active\def%{\%}Checks [call]}}%
\end{pgfscope}%
\begin{pgfscope}%
\pgfpathrectangle{\pgfqpoint{0.588387in}{0.521603in}}{\pgfqpoint{7.103961in}{2.010285in}}%
\pgfusepath{clip}%
\pgfsetrectcap%
\pgfsetroundjoin%
\pgfsetlinewidth{1.505625pt}%
\pgfsetstrokecolor{currentstroke1}%
\pgfsetdash{}{0pt}%
\pgfpathmoveto{\pgfqpoint{0.911295in}{0.612980in}}%
\pgfpathlineto{\pgfqpoint{0.958434in}{0.710195in}}%
\pgfpathlineto{\pgfqpoint{1.005574in}{0.683015in}}%
\pgfpathlineto{\pgfqpoint{1.052714in}{0.711936in}}%
\pgfpathlineto{\pgfqpoint{1.099854in}{0.788257in}}%
\pgfpathlineto{\pgfqpoint{1.146993in}{0.764023in}}%
\pgfpathlineto{\pgfqpoint{1.194133in}{0.789079in}}%
\pgfpathlineto{\pgfqpoint{1.241273in}{0.854868in}}%
\pgfpathlineto{\pgfqpoint{1.288413in}{0.834403in}}%
\pgfpathlineto{\pgfqpoint{1.335552in}{0.854741in}}%
\pgfpathlineto{\pgfqpoint{1.382692in}{0.913934in}}%
\pgfpathlineto{\pgfqpoint{1.429832in}{0.896514in}}%
\pgfpathlineto{\pgfqpoint{1.476972in}{0.911891in}}%
\pgfpathlineto{\pgfqpoint{1.524111in}{0.961087in}}%
\pgfpathlineto{\pgfqpoint{1.571251in}{0.946997in}}%
\pgfpathlineto{\pgfqpoint{1.618391in}{0.961271in}}%
\pgfpathlineto{\pgfqpoint{1.665531in}{1.006685in}}%
\pgfpathlineto{\pgfqpoint{1.712670in}{0.990081in}}%
\pgfpathlineto{\pgfqpoint{1.759810in}{1.005201in}}%
\pgfpathlineto{\pgfqpoint{1.806950in}{1.047633in}}%
\pgfpathlineto{\pgfqpoint{1.854090in}{1.040583in}}%
\pgfpathlineto{\pgfqpoint{1.901229in}{1.047403in}}%
\pgfpathlineto{\pgfqpoint{1.948369in}{1.085711in}}%
\pgfpathlineto{\pgfqpoint{1.995509in}{1.072256in}}%
\pgfpathlineto{\pgfqpoint{2.042649in}{1.089346in}}%
\pgfpathlineto{\pgfqpoint{2.089788in}{1.120663in}}%
\pgfpathlineto{\pgfqpoint{2.136928in}{1.104752in}}%
\pgfpathlineto{\pgfqpoint{2.184068in}{1.125681in}}%
\pgfpathlineto{\pgfqpoint{2.231208in}{1.150650in}}%
\pgfpathlineto{\pgfqpoint{2.278347in}{1.133389in}}%
\pgfpathlineto{\pgfqpoint{2.325487in}{1.152457in}}%
\pgfpathlineto{\pgfqpoint{2.372627in}{1.182714in}}%
\pgfpathlineto{\pgfqpoint{2.419767in}{1.175458in}}%
\pgfpathlineto{\pgfqpoint{2.466906in}{1.183992in}}%
\pgfpathlineto{\pgfqpoint{2.514046in}{1.203333in}}%
\pgfpathlineto{\pgfqpoint{2.561186in}{1.201929in}}%
\pgfpathlineto{\pgfqpoint{2.608326in}{1.199597in}}%
\pgfpathlineto{\pgfqpoint{2.655465in}{1.238763in}}%
\pgfpathlineto{\pgfqpoint{2.702605in}{1.227234in}}%
\pgfpathlineto{\pgfqpoint{2.749745in}{1.236038in}}%
\pgfpathlineto{\pgfqpoint{2.796885in}{1.246729in}}%
\pgfpathlineto{\pgfqpoint{2.844024in}{1.258285in}}%
\pgfpathlineto{\pgfqpoint{2.891164in}{1.255570in}}%
\pgfpathlineto{\pgfqpoint{2.938304in}{1.277110in}}%
\pgfpathlineto{\pgfqpoint{2.985444in}{1.265812in}}%
\pgfpathlineto{\pgfqpoint{3.032583in}{1.278338in}}%
\pgfpathlineto{\pgfqpoint{3.079723in}{1.305148in}}%
\pgfpathlineto{\pgfqpoint{3.126863in}{1.294816in}}%
\pgfpathlineto{\pgfqpoint{3.174003in}{1.300867in}}%
\pgfpathlineto{\pgfqpoint{3.221142in}{1.327030in}}%
\pgfpathlineto{\pgfqpoint{3.268282in}{1.304325in}}%
\pgfpathlineto{\pgfqpoint{3.315422in}{1.328413in}}%
\pgfpathlineto{\pgfqpoint{3.362562in}{1.340650in}}%
\pgfpathlineto{\pgfqpoint{3.409701in}{1.332191in}}%
\pgfpathlineto{\pgfqpoint{3.456841in}{1.367724in}}%
\pgfpathlineto{\pgfqpoint{3.503981in}{1.358911in}}%
\pgfpathlineto{\pgfqpoint{3.551121in}{1.345657in}}%
\pgfpathlineto{\pgfqpoint{3.598260in}{1.366865in}}%
\pgfpathlineto{\pgfqpoint{3.645400in}{1.364126in}}%
\pgfpathlineto{\pgfqpoint{3.692540in}{1.383705in}}%
\pgfpathlineto{\pgfqpoint{3.739680in}{1.383821in}}%
\pgfpathlineto{\pgfqpoint{3.786819in}{1.395297in}}%
\pgfpathlineto{\pgfqpoint{3.833959in}{1.387459in}}%
\pgfpathlineto{\pgfqpoint{3.881099in}{1.402598in}}%
\pgfpathlineto{\pgfqpoint{3.928239in}{1.407414in}}%
\pgfpathlineto{\pgfqpoint{3.975378in}{1.411542in}}%
\pgfpathlineto{\pgfqpoint{4.022518in}{1.402383in}}%
\pgfpathlineto{\pgfqpoint{4.069658in}{1.436352in}}%
\pgfpathlineto{\pgfqpoint{4.116798in}{1.418226in}}%
\pgfpathlineto{\pgfqpoint{4.163937in}{1.481472in}}%
\pgfpathlineto{\pgfqpoint{4.211077in}{1.441664in}}%
\pgfpathlineto{\pgfqpoint{4.258217in}{1.423985in}}%
\pgfpathlineto{\pgfqpoint{4.305357in}{1.460283in}}%
\pgfpathlineto{\pgfqpoint{4.352496in}{1.454872in}}%
\pgfpathlineto{\pgfqpoint{4.399636in}{1.434864in}}%
\pgfpathlineto{\pgfqpoint{4.446776in}{1.488132in}}%
\pgfpathlineto{\pgfqpoint{4.493916in}{1.468046in}}%
\pgfpathlineto{\pgfqpoint{4.541055in}{1.484195in}}%
\pgfpathlineto{\pgfqpoint{4.588195in}{1.540412in}}%
\pgfpathlineto{\pgfqpoint{4.635335in}{1.505967in}}%
\pgfpathlineto{\pgfqpoint{4.682475in}{1.487348in}}%
\pgfpathlineto{\pgfqpoint{4.729614in}{1.505367in}}%
\pgfpathlineto{\pgfqpoint{4.776754in}{1.571605in}}%
\pgfpathlineto{\pgfqpoint{4.823894in}{1.534245in}}%
\pgfpathlineto{\pgfqpoint{4.871034in}{1.497685in}}%
\pgfpathlineto{\pgfqpoint{4.918173in}{1.532533in}}%
\pgfpathlineto{\pgfqpoint{4.965313in}{1.525292in}}%
\pgfpathlineto{\pgfqpoint{5.012453in}{1.552468in}}%
\pgfpathlineto{\pgfqpoint{5.059593in}{1.558510in}}%
\pgfpathlineto{\pgfqpoint{5.106732in}{1.502755in}}%
\pgfpathlineto{\pgfqpoint{5.153872in}{1.506766in}}%
\pgfpathlineto{\pgfqpoint{5.201012in}{1.543308in}}%
\pgfpathlineto{\pgfqpoint{5.248152in}{1.561889in}}%
\pgfpathlineto{\pgfqpoint{5.295291in}{1.541176in}}%
\pgfpathlineto{\pgfqpoint{5.342431in}{1.541176in}}%
\pgfpathlineto{\pgfqpoint{5.436711in}{1.595462in}}%
\pgfpathlineto{\pgfqpoint{5.530990in}{1.588073in}}%
\pgfpathlineto{\pgfqpoint{5.578130in}{1.573361in}}%
\pgfpathlineto{\pgfqpoint{5.625270in}{1.575579in}}%
\pgfpathlineto{\pgfqpoint{5.860968in}{1.626706in}}%
\pgfusepath{stroke}%
\end{pgfscope}%
\begin{pgfscope}%
\pgfpathrectangle{\pgfqpoint{0.588387in}{0.521603in}}{\pgfqpoint{7.103961in}{2.010285in}}%
\pgfusepath{clip}%
\pgfsetrectcap%
\pgfsetroundjoin%
\pgfsetlinewidth{1.505625pt}%
\pgfsetstrokecolor{currentstroke2}%
\pgfsetdash{}{0pt}%
\pgfpathmoveto{\pgfqpoint{0.911295in}{0.761873in}}%
\pgfpathlineto{\pgfqpoint{0.958434in}{0.891873in}}%
\pgfpathlineto{\pgfqpoint{1.005574in}{0.991297in}}%
\pgfpathlineto{\pgfqpoint{1.052714in}{0.855959in}}%
\pgfpathlineto{\pgfqpoint{1.099854in}{0.892437in}}%
\pgfpathlineto{\pgfqpoint{1.146993in}{0.989666in}}%
\pgfpathlineto{\pgfqpoint{1.194133in}{1.068546in}}%
\pgfpathlineto{\pgfqpoint{1.241273in}{0.960744in}}%
\pgfpathlineto{\pgfqpoint{1.288413in}{0.990080in}}%
\pgfpathlineto{\pgfqpoint{1.335552in}{1.067422in}}%
\pgfpathlineto{\pgfqpoint{1.382692in}{1.136676in}}%
\pgfpathlineto{\pgfqpoint{1.429832in}{1.047201in}}%
\pgfpathlineto{\pgfqpoint{1.476972in}{1.068520in}}%
\pgfpathlineto{\pgfqpoint{1.524111in}{1.134055in}}%
\pgfpathlineto{\pgfqpoint{1.571251in}{1.189823in}}%
\pgfpathlineto{\pgfqpoint{1.618391in}{1.113199in}}%
\pgfpathlineto{\pgfqpoint{1.665531in}{1.134351in}}%
\pgfpathlineto{\pgfqpoint{1.712670in}{1.190087in}}%
\pgfpathlineto{\pgfqpoint{1.759810in}{1.239261in}}%
\pgfpathlineto{\pgfqpoint{1.806950in}{1.175397in}}%
\pgfpathlineto{\pgfqpoint{1.854090in}{1.192868in}}%
\pgfpathlineto{\pgfqpoint{1.901229in}{1.242311in}}%
\pgfpathlineto{\pgfqpoint{1.948369in}{1.286485in}}%
\pgfpathlineto{\pgfqpoint{1.995509in}{1.225837in}}%
\pgfpathlineto{\pgfqpoint{2.042649in}{1.247206in}}%
\pgfpathlineto{\pgfqpoint{2.089788in}{1.287308in}}%
\pgfpathlineto{\pgfqpoint{2.136928in}{1.325580in}}%
\pgfpathlineto{\pgfqpoint{2.184068in}{1.279135in}}%
\pgfpathlineto{\pgfqpoint{2.231208in}{1.288398in}}%
\pgfpathlineto{\pgfqpoint{2.278347in}{1.322512in}}%
\pgfpathlineto{\pgfqpoint{2.325487in}{1.363554in}}%
\pgfpathlineto{\pgfqpoint{2.372627in}{1.318462in}}%
\pgfpathlineto{\pgfqpoint{2.419767in}{1.330689in}}%
\pgfpathlineto{\pgfqpoint{2.466906in}{1.368658in}}%
\pgfpathlineto{\pgfqpoint{2.514046in}{1.394485in}}%
\pgfpathlineto{\pgfqpoint{2.561186in}{1.355832in}}%
\pgfpathlineto{\pgfqpoint{2.608326in}{1.359394in}}%
\pgfpathlineto{\pgfqpoint{2.655465in}{1.400791in}}%
\pgfpathlineto{\pgfqpoint{2.702605in}{1.428290in}}%
\pgfpathlineto{\pgfqpoint{2.749745in}{1.387980in}}%
\pgfpathlineto{\pgfqpoint{2.796885in}{1.392476in}}%
\pgfpathlineto{\pgfqpoint{2.844024in}{1.427871in}}%
\pgfpathlineto{\pgfqpoint{2.891164in}{1.452720in}}%
\pgfpathlineto{\pgfqpoint{2.938304in}{1.417713in}}%
\pgfpathlineto{\pgfqpoint{2.985444in}{1.423985in}}%
\pgfpathlineto{\pgfqpoint{3.032583in}{1.461672in}}%
\pgfpathlineto{\pgfqpoint{3.079723in}{1.485649in}}%
\pgfpathlineto{\pgfqpoint{3.126863in}{1.451306in}}%
\pgfpathlineto{\pgfqpoint{3.174003in}{1.457841in}}%
\pgfpathlineto{\pgfqpoint{3.221142in}{1.487832in}}%
\pgfpathlineto{\pgfqpoint{3.268282in}{1.502570in}}%
\pgfpathlineto{\pgfqpoint{3.315422in}{1.474814in}}%
\pgfpathlineto{\pgfqpoint{3.362562in}{1.478963in}}%
\pgfpathlineto{\pgfqpoint{3.409701in}{1.509340in}}%
\pgfpathlineto{\pgfqpoint{3.456841in}{1.550551in}}%
\pgfpathlineto{\pgfqpoint{3.503981in}{1.501491in}}%
\pgfpathlineto{\pgfqpoint{3.551121in}{1.505993in}}%
\pgfpathlineto{\pgfqpoint{3.598260in}{1.535547in}}%
\pgfpathlineto{\pgfqpoint{3.645400in}{1.547123in}}%
\pgfpathlineto{\pgfqpoint{3.692540in}{1.521793in}}%
\pgfpathlineto{\pgfqpoint{3.739680in}{1.535298in}}%
\pgfpathlineto{\pgfqpoint{3.786819in}{1.556704in}}%
\pgfpathlineto{\pgfqpoint{3.833959in}{1.576734in}}%
\pgfpathlineto{\pgfqpoint{3.881099in}{1.553238in}}%
\pgfpathlineto{\pgfqpoint{3.928239in}{1.550736in}}%
\pgfpathlineto{\pgfqpoint{3.975378in}{1.580869in}}%
\pgfpathlineto{\pgfqpoint{4.022518in}{1.590355in}}%
\pgfpathlineto{\pgfqpoint{4.069658in}{1.584867in}}%
\pgfpathlineto{\pgfqpoint{4.116798in}{1.571845in}}%
\pgfpathlineto{\pgfqpoint{4.163937in}{1.638733in}}%
\pgfpathlineto{\pgfqpoint{4.211077in}{1.611281in}}%
\pgfpathlineto{\pgfqpoint{4.258217in}{1.591947in}}%
\pgfpathlineto{\pgfqpoint{4.305357in}{1.596319in}}%
\pgfpathlineto{\pgfqpoint{4.399636in}{1.634730in}}%
\pgfpathlineto{\pgfqpoint{4.446776in}{1.619851in}}%
\pgfpathlineto{\pgfqpoint{4.493916in}{1.618919in}}%
\pgfpathlineto{\pgfqpoint{4.588195in}{1.664960in}}%
\pgfpathlineto{\pgfqpoint{4.635335in}{1.627122in}}%
\pgfpathlineto{\pgfqpoint{4.682475in}{1.632744in}}%
\pgfpathlineto{\pgfqpoint{4.729614in}{1.663093in}}%
\pgfpathlineto{\pgfqpoint{4.776754in}{1.698988in}}%
\pgfpathlineto{\pgfqpoint{4.823894in}{1.647742in}}%
\pgfpathlineto{\pgfqpoint{4.871034in}{1.657214in}}%
\pgfpathlineto{\pgfqpoint{4.918173in}{1.675021in}}%
\pgfpathlineto{\pgfqpoint{4.965313in}{1.679471in}}%
\pgfpathlineto{\pgfqpoint{5.012453in}{1.681252in}}%
\pgfpathlineto{\pgfqpoint{5.059593in}{1.697653in}}%
\pgfpathlineto{\pgfqpoint{5.106732in}{1.677448in}}%
\pgfpathlineto{\pgfqpoint{5.153872in}{1.695368in}}%
\pgfpathlineto{\pgfqpoint{5.201012in}{1.687816in}}%
\pgfpathlineto{\pgfqpoint{5.248152in}{1.687582in}}%
\pgfpathlineto{\pgfqpoint{5.342431in}{1.709411in}}%
\pgfpathlineto{\pgfqpoint{5.389571in}{1.685129in}}%
\pgfpathlineto{\pgfqpoint{5.436711in}{1.705374in}}%
\pgfpathlineto{\pgfqpoint{5.483850in}{1.708832in}}%
\pgfpathlineto{\pgfqpoint{5.530990in}{1.743390in}}%
\pgfpathlineto{\pgfqpoint{5.578130in}{1.722829in}}%
\pgfpathlineto{\pgfqpoint{5.625270in}{1.717066in}}%
\pgfpathlineto{\pgfqpoint{5.672409in}{1.729290in}}%
\pgfpathlineto{\pgfqpoint{5.719549in}{1.735786in}}%
\pgfpathlineto{\pgfqpoint{5.766689in}{1.714383in}}%
\pgfpathlineto{\pgfqpoint{5.813829in}{1.740320in}}%
\pgfpathlineto{\pgfqpoint{5.860968in}{1.771361in}}%
\pgfpathlineto{\pgfqpoint{5.908108in}{1.747509in}}%
\pgfpathlineto{\pgfqpoint{5.955248in}{1.735647in}}%
\pgfpathlineto{\pgfqpoint{6.002388in}{1.733573in}}%
\pgfpathlineto{\pgfqpoint{6.049527in}{1.752916in}}%
\pgfpathlineto{\pgfqpoint{6.096667in}{1.762162in}}%
\pgfpathlineto{\pgfqpoint{6.143807in}{1.757495in}}%
\pgfpathlineto{\pgfqpoint{6.190947in}{1.755347in}}%
\pgfpathlineto{\pgfqpoint{6.238086in}{1.761270in}}%
\pgfpathlineto{\pgfqpoint{6.285226in}{1.773300in}}%
\pgfpathlineto{\pgfqpoint{6.332366in}{1.756252in}}%
\pgfpathlineto{\pgfqpoint{6.379506in}{1.763924in}}%
\pgfpathlineto{\pgfqpoint{6.426645in}{1.773634in}}%
\pgfpathlineto{\pgfqpoint{6.473785in}{1.789164in}}%
\pgfpathlineto{\pgfqpoint{6.520925in}{1.791993in}}%
\pgfpathlineto{\pgfqpoint{6.568065in}{1.806609in}}%
\pgfpathlineto{\pgfqpoint{6.615204in}{1.823366in}}%
\pgfpathlineto{\pgfqpoint{6.662344in}{1.796568in}}%
\pgfpathlineto{\pgfqpoint{6.709484in}{1.780928in}}%
\pgfpathlineto{\pgfqpoint{6.756624in}{1.788713in}}%
\pgfpathlineto{\pgfqpoint{6.850903in}{1.808312in}}%
\pgfpathlineto{\pgfqpoint{6.898043in}{1.794086in}}%
\pgfpathlineto{\pgfqpoint{6.992322in}{1.807934in}}%
\pgfpathlineto{\pgfqpoint{7.086602in}{1.805659in}}%
\pgfpathlineto{\pgfqpoint{7.133742in}{1.809444in}}%
\pgfpathlineto{\pgfqpoint{7.228021in}{1.831437in}}%
\pgfpathlineto{\pgfqpoint{7.275161in}{1.816909in}}%
\pgfpathlineto{\pgfqpoint{7.275161in}{1.816909in}}%
\pgfusepath{stroke}%
\end{pgfscope}%
\begin{pgfscope}%
\pgfpathrectangle{\pgfqpoint{0.588387in}{0.521603in}}{\pgfqpoint{7.103961in}{2.010285in}}%
\pgfusepath{clip}%
\pgfsetrectcap%
\pgfsetroundjoin%
\pgfsetlinewidth{1.505625pt}%
\pgfsetstrokecolor{currentstroke3}%
\pgfsetdash{}{0pt}%
\pgfpathmoveto{\pgfqpoint{0.911295in}{1.128642in}}%
\pgfpathlineto{\pgfqpoint{0.958434in}{1.320804in}}%
\pgfpathlineto{\pgfqpoint{1.005574in}{0.991288in}}%
\pgfpathlineto{\pgfqpoint{1.052714in}{1.045019in}}%
\pgfpathlineto{\pgfqpoint{1.099854in}{1.171100in}}%
\pgfpathlineto{\pgfqpoint{1.146993in}{1.268564in}}%
\pgfpathlineto{\pgfqpoint{1.194133in}{1.323932in}}%
\pgfpathlineto{\pgfqpoint{1.241273in}{1.134455in}}%
\pgfpathlineto{\pgfqpoint{1.288413in}{1.172942in}}%
\pgfpathlineto{\pgfqpoint{1.335552in}{1.268457in}}%
\pgfpathlineto{\pgfqpoint{1.382692in}{1.350604in}}%
\pgfpathlineto{\pgfqpoint{1.429832in}{1.415837in}}%
\pgfpathlineto{\pgfqpoint{1.476972in}{1.240020in}}%
\pgfpathlineto{\pgfqpoint{1.524111in}{1.268891in}}%
\pgfpathlineto{\pgfqpoint{1.571251in}{1.347142in}}%
\pgfpathlineto{\pgfqpoint{1.618391in}{1.411653in}}%
\pgfpathlineto{\pgfqpoint{1.665531in}{1.469355in}}%
\pgfpathlineto{\pgfqpoint{1.712670in}{1.322401in}}%
\pgfpathlineto{\pgfqpoint{1.759810in}{1.348566in}}%
\pgfpathlineto{\pgfqpoint{1.854090in}{1.479087in}}%
\pgfpathlineto{\pgfqpoint{1.901229in}{1.522307in}}%
\pgfpathlineto{\pgfqpoint{1.948369in}{1.400596in}}%
\pgfpathlineto{\pgfqpoint{1.995509in}{1.414724in}}%
\pgfpathlineto{\pgfqpoint{2.042649in}{1.479837in}}%
\pgfpathlineto{\pgfqpoint{2.089788in}{1.524427in}}%
\pgfpathlineto{\pgfqpoint{2.136928in}{1.567171in}}%
\pgfpathlineto{\pgfqpoint{2.184068in}{1.463824in}}%
\pgfpathlineto{\pgfqpoint{2.231208in}{1.477763in}}%
\pgfpathlineto{\pgfqpoint{2.278347in}{1.517494in}}%
\pgfpathlineto{\pgfqpoint{2.325487in}{1.575084in}}%
\pgfpathlineto{\pgfqpoint{2.372627in}{1.612907in}}%
\pgfpathlineto{\pgfqpoint{2.419767in}{1.521126in}}%
\pgfpathlineto{\pgfqpoint{2.466906in}{1.536060in}}%
\pgfpathlineto{\pgfqpoint{2.514046in}{1.571759in}}%
\pgfpathlineto{\pgfqpoint{2.561186in}{1.604672in}}%
\pgfpathlineto{\pgfqpoint{2.608326in}{1.635892in}}%
\pgfpathlineto{\pgfqpoint{2.655465in}{1.560511in}}%
\pgfpathlineto{\pgfqpoint{2.702605in}{1.570852in}}%
\pgfpathlineto{\pgfqpoint{2.749745in}{1.606475in}}%
\pgfpathlineto{\pgfqpoint{2.796885in}{1.635030in}}%
\pgfpathlineto{\pgfqpoint{2.844024in}{1.691664in}}%
\pgfpathlineto{\pgfqpoint{2.891164in}{1.590575in}}%
\pgfpathlineto{\pgfqpoint{2.938304in}{1.604290in}}%
\pgfpathlineto{\pgfqpoint{2.985444in}{1.636199in}}%
\pgfpathlineto{\pgfqpoint{3.032583in}{1.679706in}}%
\pgfpathlineto{\pgfqpoint{3.079723in}{1.709941in}}%
\pgfpathlineto{\pgfqpoint{3.126863in}{1.638454in}}%
\pgfpathlineto{\pgfqpoint{3.174003in}{1.646306in}}%
\pgfpathlineto{\pgfqpoint{3.221142in}{1.683244in}}%
\pgfpathlineto{\pgfqpoint{3.268282in}{1.700072in}}%
\pgfpathlineto{\pgfqpoint{3.315422in}{1.731871in}}%
\pgfpathlineto{\pgfqpoint{3.362562in}{1.660028in}}%
\pgfpathlineto{\pgfqpoint{3.409701in}{1.679664in}}%
\pgfpathlineto{\pgfqpoint{3.456841in}{1.729595in}}%
\pgfpathlineto{\pgfqpoint{3.503981in}{1.734567in}}%
\pgfpathlineto{\pgfqpoint{3.551121in}{1.755403in}}%
\pgfpathlineto{\pgfqpoint{3.598260in}{1.701347in}}%
\pgfpathlineto{\pgfqpoint{3.645400in}{1.700144in}}%
\pgfpathlineto{\pgfqpoint{3.692540in}{1.729099in}}%
\pgfpathlineto{\pgfqpoint{3.739680in}{1.763109in}}%
\pgfpathlineto{\pgfqpoint{3.786819in}{1.787633in}}%
\pgfpathlineto{\pgfqpoint{3.833959in}{1.726943in}}%
\pgfpathlineto{\pgfqpoint{3.928239in}{1.755529in}}%
\pgfpathlineto{\pgfqpoint{3.975378in}{1.794021in}}%
\pgfpathlineto{\pgfqpoint{4.022518in}{1.802637in}}%
\pgfpathlineto{\pgfqpoint{4.069658in}{1.766178in}}%
\pgfpathlineto{\pgfqpoint{4.116798in}{1.758255in}}%
\pgfpathlineto{\pgfqpoint{4.163937in}{1.832277in}}%
\pgfpathlineto{\pgfqpoint{4.211077in}{1.806525in}}%
\pgfpathlineto{\pgfqpoint{4.258217in}{1.829439in}}%
\pgfpathlineto{\pgfqpoint{4.305357in}{1.777048in}}%
\pgfpathlineto{\pgfqpoint{4.352496in}{1.788332in}}%
\pgfpathlineto{\pgfqpoint{4.399636in}{1.804450in}}%
\pgfpathlineto{\pgfqpoint{4.446776in}{1.847877in}}%
\pgfpathlineto{\pgfqpoint{4.493916in}{1.853150in}}%
\pgfpathlineto{\pgfqpoint{4.541055in}{1.815781in}}%
\pgfpathlineto{\pgfqpoint{4.588195in}{1.842211in}}%
\pgfpathlineto{\pgfqpoint{4.635335in}{1.828412in}}%
\pgfpathlineto{\pgfqpoint{4.682475in}{1.851723in}}%
\pgfpathlineto{\pgfqpoint{4.729614in}{1.898618in}}%
\pgfpathlineto{\pgfqpoint{4.776754in}{1.858104in}}%
\pgfpathlineto{\pgfqpoint{4.823894in}{1.837563in}}%
\pgfpathlineto{\pgfqpoint{4.871034in}{1.856848in}}%
\pgfpathlineto{\pgfqpoint{4.918173in}{1.899633in}}%
\pgfpathlineto{\pgfqpoint{4.965313in}{1.892250in}}%
\pgfpathlineto{\pgfqpoint{5.012453in}{1.869075in}}%
\pgfpathlineto{\pgfqpoint{5.059593in}{1.886944in}}%
\pgfpathlineto{\pgfqpoint{5.106732in}{1.863978in}}%
\pgfpathlineto{\pgfqpoint{5.201012in}{1.921513in}}%
\pgfpathlineto{\pgfqpoint{5.248152in}{1.877417in}}%
\pgfpathlineto{\pgfqpoint{5.295291in}{1.874651in}}%
\pgfpathlineto{\pgfqpoint{5.342431in}{1.887422in}}%
\pgfpathlineto{\pgfqpoint{5.389571in}{1.903643in}}%
\pgfpathlineto{\pgfqpoint{5.436711in}{1.950980in}}%
\pgfpathlineto{\pgfqpoint{5.483850in}{1.910142in}}%
\pgfpathlineto{\pgfqpoint{5.530990in}{1.901477in}}%
\pgfpathlineto{\pgfqpoint{5.578130in}{1.935553in}}%
\pgfpathlineto{\pgfqpoint{5.625270in}{1.939056in}}%
\pgfpathlineto{\pgfqpoint{5.672409in}{1.956422in}}%
\pgfpathlineto{\pgfqpoint{5.719549in}{1.901596in}}%
\pgfpathlineto{\pgfqpoint{5.766689in}{1.903891in}}%
\pgfpathlineto{\pgfqpoint{5.813829in}{1.957300in}}%
\pgfpathlineto{\pgfqpoint{5.860968in}{2.077993in}}%
\pgfpathlineto{\pgfqpoint{5.908108in}{1.951743in}}%
\pgfpathlineto{\pgfqpoint{5.955248in}{1.921790in}}%
\pgfpathlineto{\pgfqpoint{6.002388in}{1.919530in}}%
\pgfpathlineto{\pgfqpoint{6.049527in}{1.949776in}}%
\pgfpathlineto{\pgfqpoint{6.096667in}{1.951785in}}%
\pgfpathlineto{\pgfqpoint{6.143807in}{1.996636in}}%
\pgfpathlineto{\pgfqpoint{6.190947in}{1.948185in}}%
\pgfpathlineto{\pgfqpoint{6.238086in}{1.935480in}}%
\pgfpathlineto{\pgfqpoint{6.285226in}{1.960098in}}%
\pgfpathlineto{\pgfqpoint{6.332366in}{1.973340in}}%
\pgfpathlineto{\pgfqpoint{6.379506in}{1.982231in}}%
\pgfpathlineto{\pgfqpoint{6.426645in}{1.956807in}}%
\pgfpathlineto{\pgfqpoint{6.473785in}{1.959578in}}%
\pgfpathlineto{\pgfqpoint{6.520925in}{1.972499in}}%
\pgfpathlineto{\pgfqpoint{6.568065in}{2.044429in}}%
\pgfpathlineto{\pgfqpoint{6.615204in}{2.028562in}}%
\pgfpathlineto{\pgfqpoint{6.662344in}{1.963039in}}%
\pgfpathlineto{\pgfqpoint{6.709484in}{1.969388in}}%
\pgfpathlineto{\pgfqpoint{6.756624in}{1.984877in}}%
\pgfpathlineto{\pgfqpoint{6.803763in}{2.008989in}}%
\pgfpathlineto{\pgfqpoint{6.850903in}{2.009162in}}%
\pgfpathlineto{\pgfqpoint{6.898043in}{1.983766in}}%
\pgfpathlineto{\pgfqpoint{6.945183in}{1.984851in}}%
\pgfpathlineto{\pgfqpoint{7.039462in}{2.013049in}}%
\pgfpathlineto{\pgfqpoint{7.086602in}{2.040908in}}%
\pgfpathlineto{\pgfqpoint{7.133742in}{1.991301in}}%
\pgfpathlineto{\pgfqpoint{7.180881in}{1.993694in}}%
\pgfpathlineto{\pgfqpoint{7.228021in}{2.014609in}}%
\pgfpathlineto{\pgfqpoint{7.322301in}{2.040908in}}%
\pgfpathlineto{\pgfqpoint{7.369440in}{2.023567in}}%
\pgfpathlineto{\pgfqpoint{7.369440in}{2.023567in}}%
\pgfusepath{stroke}%
\end{pgfscope}%
\begin{pgfscope}%
\pgfpathrectangle{\pgfqpoint{0.588387in}{0.521603in}}{\pgfqpoint{7.103961in}{2.010285in}}%
\pgfusepath{clip}%
\pgfsetrectcap%
\pgfsetroundjoin%
\pgfsetlinewidth{1.505625pt}%
\pgfsetstrokecolor{currentstroke4}%
\pgfsetdash{}{0pt}%
\pgfpathmoveto{\pgfqpoint{0.911295in}{1.128864in}}%
\pgfpathlineto{\pgfqpoint{0.958434in}{1.321120in}}%
\pgfpathlineto{\pgfqpoint{1.005574in}{1.413162in}}%
\pgfpathlineto{\pgfqpoint{1.052714in}{1.602918in}}%
\pgfpathlineto{\pgfqpoint{1.099854in}{1.689937in}}%
\pgfpathlineto{\pgfqpoint{1.146993in}{1.268845in}}%
\pgfpathlineto{\pgfqpoint{1.194133in}{1.323318in}}%
\pgfpathlineto{\pgfqpoint{1.241273in}{1.451086in}}%
\pgfpathlineto{\pgfqpoint{1.288413in}{1.547738in}}%
\pgfpathlineto{\pgfqpoint{1.335552in}{1.600203in}}%
\pgfpathlineto{\pgfqpoint{1.382692in}{1.732210in}}%
\pgfpathlineto{\pgfqpoint{1.429832in}{1.414552in}}%
\pgfpathlineto{\pgfqpoint{1.476972in}{1.450873in}}%
\pgfpathlineto{\pgfqpoint{1.524111in}{1.548118in}}%
\pgfpathlineto{\pgfqpoint{1.571251in}{1.626706in}}%
\pgfpathlineto{\pgfqpoint{1.618391in}{1.692003in}}%
\pgfpathlineto{\pgfqpoint{1.665531in}{1.731222in}}%
\pgfpathlineto{\pgfqpoint{1.712670in}{1.518151in}}%
\pgfpathlineto{\pgfqpoint{1.759810in}{1.549012in}}%
\pgfpathlineto{\pgfqpoint{1.806950in}{1.627678in}}%
\pgfpathlineto{\pgfqpoint{1.854090in}{1.697867in}}%
\pgfpathlineto{\pgfqpoint{1.901229in}{1.749138in}}%
\pgfpathlineto{\pgfqpoint{1.948369in}{1.802442in}}%
\pgfpathlineto{\pgfqpoint{1.995509in}{1.602813in}}%
\pgfpathlineto{\pgfqpoint{2.042649in}{1.635440in}}%
\pgfpathlineto{\pgfqpoint{2.089788in}{1.699462in}}%
\pgfpathlineto{\pgfqpoint{2.136928in}{1.757616in}}%
\pgfpathlineto{\pgfqpoint{2.184068in}{1.807705in}}%
\pgfpathlineto{\pgfqpoint{2.231208in}{1.846072in}}%
\pgfpathlineto{\pgfqpoint{2.278347in}{1.670634in}}%
\pgfpathlineto{\pgfqpoint{2.325487in}{1.703934in}}%
\pgfpathlineto{\pgfqpoint{2.372627in}{1.765098in}}%
\pgfpathlineto{\pgfqpoint{2.419767in}{1.810817in}}%
\pgfpathlineto{\pgfqpoint{2.466906in}{1.854610in}}%
\pgfpathlineto{\pgfqpoint{2.514046in}{1.884210in}}%
\pgfpathlineto{\pgfqpoint{2.561186in}{1.738410in}}%
\pgfpathlineto{\pgfqpoint{2.608326in}{1.752096in}}%
\pgfpathlineto{\pgfqpoint{2.655465in}{1.812685in}}%
\pgfpathlineto{\pgfqpoint{2.702605in}{1.852421in}}%
\pgfpathlineto{\pgfqpoint{2.749745in}{1.890508in}}%
\pgfpathlineto{\pgfqpoint{2.796885in}{1.914668in}}%
\pgfpathlineto{\pgfqpoint{2.844024in}{1.801735in}}%
\pgfpathlineto{\pgfqpoint{2.891164in}{1.807868in}}%
\pgfpathlineto{\pgfqpoint{2.938304in}{1.848532in}}%
\pgfpathlineto{\pgfqpoint{2.985444in}{1.885298in}}%
\pgfpathlineto{\pgfqpoint{3.032583in}{1.931978in}}%
\pgfpathlineto{\pgfqpoint{3.079723in}{1.963877in}}%
\pgfpathlineto{\pgfqpoint{3.126863in}{1.870524in}}%
\pgfpathlineto{\pgfqpoint{3.174003in}{1.856991in}}%
\pgfpathlineto{\pgfqpoint{3.221142in}{1.917846in}}%
\pgfpathlineto{\pgfqpoint{3.268282in}{1.911975in}}%
\pgfpathlineto{\pgfqpoint{3.315422in}{1.958796in}}%
\pgfpathlineto{\pgfqpoint{3.362562in}{1.976725in}}%
\pgfpathlineto{\pgfqpoint{3.409701in}{1.879032in}}%
\pgfpathlineto{\pgfqpoint{3.456841in}{1.939133in}}%
\pgfpathlineto{\pgfqpoint{3.503981in}{1.925966in}}%
\pgfpathlineto{\pgfqpoint{3.551121in}{1.944309in}}%
\pgfpathlineto{\pgfqpoint{3.598260in}{1.994169in}}%
\pgfpathlineto{\pgfqpoint{3.645400in}{2.002143in}}%
\pgfpathlineto{\pgfqpoint{3.692540in}{1.914398in}}%
\pgfpathlineto{\pgfqpoint{3.739680in}{1.932777in}}%
\pgfpathlineto{\pgfqpoint{3.786819in}{1.963993in}}%
\pgfpathlineto{\pgfqpoint{3.833959in}{1.988054in}}%
\pgfpathlineto{\pgfqpoint{3.881099in}{2.031744in}}%
\pgfpathlineto{\pgfqpoint{3.928239in}{2.129945in}}%
\pgfpathlineto{\pgfqpoint{3.975378in}{1.959452in}}%
\pgfpathlineto{\pgfqpoint{4.022518in}{1.949108in}}%
\pgfpathlineto{\pgfqpoint{4.069658in}{2.013309in}}%
\pgfpathlineto{\pgfqpoint{4.116798in}{2.009645in}}%
\pgfpathlineto{\pgfqpoint{4.163937in}{2.145124in}}%
\pgfpathlineto{\pgfqpoint{4.211077in}{2.061081in}}%
\pgfpathlineto{\pgfqpoint{4.258217in}{1.957272in}}%
\pgfpathlineto{\pgfqpoint{4.305357in}{1.974562in}}%
\pgfpathlineto{\pgfqpoint{4.352496in}{2.027640in}}%
\pgfpathlineto{\pgfqpoint{4.399636in}{2.015552in}}%
\pgfpathlineto{\pgfqpoint{4.446776in}{2.085654in}}%
\pgfpathlineto{\pgfqpoint{4.493916in}{2.091551in}}%
\pgfpathlineto{\pgfqpoint{4.541055in}{2.027209in}}%
\pgfpathlineto{\pgfqpoint{4.588195in}{2.116099in}}%
\pgfpathlineto{\pgfqpoint{4.635335in}{2.060576in}}%
\pgfpathlineto{\pgfqpoint{4.682475in}{2.074722in}}%
\pgfpathlineto{\pgfqpoint{4.729614in}{2.113438in}}%
\pgfpathlineto{\pgfqpoint{4.776754in}{2.197400in}}%
\pgfpathlineto{\pgfqpoint{4.823894in}{2.062801in}}%
\pgfpathlineto{\pgfqpoint{4.871034in}{2.046178in}}%
\pgfpathlineto{\pgfqpoint{4.918173in}{2.111270in}}%
\pgfpathlineto{\pgfqpoint{4.965313in}{2.093906in}}%
\pgfpathlineto{\pgfqpoint{5.012453in}{2.179892in}}%
\pgfpathlineto{\pgfqpoint{5.059593in}{2.182163in}}%
\pgfpathlineto{\pgfqpoint{5.106732in}{2.047745in}}%
\pgfpathlineto{\pgfqpoint{5.153872in}{2.063596in}}%
\pgfpathlineto{\pgfqpoint{5.201012in}{2.103894in}}%
\pgfpathlineto{\pgfqpoint{5.248152in}{2.173628in}}%
\pgfpathlineto{\pgfqpoint{5.295291in}{2.151416in}}%
\pgfpathlineto{\pgfqpoint{5.342431in}{2.159251in}}%
\pgfpathlineto{\pgfqpoint{5.436711in}{2.187261in}}%
\pgfpathlineto{\pgfqpoint{5.483850in}{2.297070in}}%
\pgfpathlineto{\pgfqpoint{5.530990in}{2.137905in}}%
\pgfpathlineto{\pgfqpoint{5.578130in}{2.245244in}}%
\pgfpathlineto{\pgfqpoint{5.625270in}{2.199464in}}%
\pgfpathlineto{\pgfqpoint{5.860968in}{2.402469in}}%
\pgfpathlineto{\pgfqpoint{6.049527in}{2.184143in}}%
\pgfpathlineto{\pgfqpoint{6.379506in}{2.182187in}}%
\pgfpathlineto{\pgfqpoint{6.568065in}{2.307327in}}%
\pgfusepath{stroke}%
\end{pgfscope}%
\begin{pgfscope}%
\pgfpathrectangle{\pgfqpoint{0.588387in}{0.521603in}}{\pgfqpoint{7.103961in}{2.010285in}}%
\pgfusepath{clip}%
\pgfsetrectcap%
\pgfsetroundjoin%
\pgfsetlinewidth{1.505625pt}%
\pgfsetstrokecolor{currentstroke5}%
\pgfsetdash{}{0pt}%
\pgfpathmoveto{\pgfqpoint{0.911295in}{2.440512in}}%
\pgfpathlineto{\pgfqpoint{0.958434in}{1.320808in}}%
\pgfpathlineto{\pgfqpoint{1.005574in}{1.414116in}}%
\pgfpathlineto{\pgfqpoint{1.052714in}{1.603755in}}%
\pgfpathlineto{\pgfqpoint{1.099854in}{1.689318in}}%
\pgfpathlineto{\pgfqpoint{1.146993in}{1.878868in}}%
\pgfpathlineto{\pgfqpoint{1.194133in}{1.971825in}}%
\pgfpathlineto{\pgfqpoint{1.241273in}{2.157907in}}%
\pgfpathlineto{\pgfqpoint{1.288413in}{1.548677in}}%
\pgfpathlineto{\pgfqpoint{1.335552in}{1.599235in}}%
\pgfpathlineto{\pgfqpoint{1.382692in}{1.734239in}}%
\pgfpathlineto{\pgfqpoint{1.429832in}{1.824410in}}%
\pgfpathlineto{\pgfqpoint{1.476972in}{1.877562in}}%
\pgfpathlineto{\pgfqpoint{1.524111in}{2.000288in}}%
\pgfpathlineto{\pgfqpoint{1.571251in}{2.106634in}}%
\pgfpathlineto{\pgfqpoint{1.618391in}{1.691745in}}%
\pgfpathlineto{\pgfqpoint{1.665531in}{1.732501in}}%
\pgfpathlineto{\pgfqpoint{1.712670in}{1.822806in}}%
\pgfpathlineto{\pgfqpoint{1.759810in}{1.901890in}}%
\pgfpathlineto{\pgfqpoint{1.806950in}{1.968749in}}%
\pgfpathlineto{\pgfqpoint{1.854090in}{2.021250in}}%
\pgfpathlineto{\pgfqpoint{1.901229in}{2.113111in}}%
\pgfpathlineto{\pgfqpoint{1.948369in}{1.817114in}}%
\pgfpathlineto{\pgfqpoint{1.995509in}{1.828737in}}%
\pgfpathlineto{\pgfqpoint{2.042649in}{1.934651in}}%
\pgfpathlineto{\pgfqpoint{2.089788in}{1.979989in}}%
\pgfpathlineto{\pgfqpoint{2.136928in}{2.035226in}}%
\pgfpathlineto{\pgfqpoint{2.184068in}{2.113769in}}%
\pgfpathlineto{\pgfqpoint{2.231208in}{2.109217in}}%
\pgfpathlineto{\pgfqpoint{2.278347in}{1.874557in}}%
\pgfpathlineto{\pgfqpoint{2.325487in}{1.930995in}}%
\pgfpathlineto{\pgfqpoint{2.372627in}{2.023844in}}%
\pgfpathlineto{\pgfqpoint{2.419767in}{2.183506in}}%
\pgfpathlineto{\pgfqpoint{2.466906in}{2.119126in}}%
\pgfpathlineto{\pgfqpoint{2.514046in}{2.153045in}}%
\pgfpathlineto{\pgfqpoint{2.561186in}{2.149437in}}%
\pgfpathlineto{\pgfqpoint{2.608326in}{1.934952in}}%
\pgfpathlineto{\pgfqpoint{2.655465in}{2.000452in}}%
\pgfpathlineto{\pgfqpoint{2.702605in}{2.067145in}}%
\pgfpathlineto{\pgfqpoint{2.749745in}{2.120022in}}%
\pgfpathlineto{\pgfqpoint{2.796885in}{2.113962in}}%
\pgfpathlineto{\pgfqpoint{2.844024in}{2.307798in}}%
\pgfpathlineto{\pgfqpoint{2.891164in}{2.206904in}}%
\pgfpathlineto{\pgfqpoint{2.938304in}{2.027790in}}%
\pgfpathlineto{\pgfqpoint{2.985444in}{2.008907in}}%
\pgfpathlineto{\pgfqpoint{3.032583in}{2.087500in}}%
\pgfpathlineto{\pgfqpoint{3.126863in}{2.309845in}}%
\pgfpathlineto{\pgfqpoint{3.174003in}{2.196860in}}%
\pgfpathlineto{\pgfqpoint{3.221142in}{2.301318in}}%
\pgfpathlineto{\pgfqpoint{3.315422in}{2.160111in}}%
\pgfpathlineto{\pgfqpoint{3.362562in}{2.114404in}}%
\pgfpathlineto{\pgfqpoint{3.409701in}{2.140227in}}%
\pgfpathlineto{\pgfqpoint{3.456841in}{2.352915in}}%
\pgfpathlineto{\pgfqpoint{3.503981in}{2.213446in}}%
\pgfpathlineto{\pgfqpoint{3.598260in}{2.090530in}}%
\pgfpathlineto{\pgfqpoint{3.692540in}{2.146262in}}%
\pgfpathlineto{\pgfqpoint{3.833959in}{2.205767in}}%
\pgfpathlineto{\pgfqpoint{4.069658in}{2.271765in}}%
\pgfusepath{stroke}%
\end{pgfscope}%
\begin{pgfscope}%
\pgfsetrectcap%
\pgfsetmiterjoin%
\pgfsetlinewidth{0.803000pt}%
\definecolor{currentstroke}{rgb}{0.000000,0.000000,0.000000}%
\pgfsetstrokecolor{currentstroke}%
\pgfsetdash{}{0pt}%
\pgfpathmoveto{\pgfqpoint{0.588387in}{0.521603in}}%
\pgfpathlineto{\pgfqpoint{0.588387in}{2.531888in}}%
\pgfusepath{stroke}%
\end{pgfscope}%
\begin{pgfscope}%
\pgfsetrectcap%
\pgfsetmiterjoin%
\pgfsetlinewidth{0.803000pt}%
\definecolor{currentstroke}{rgb}{0.000000,0.000000,0.000000}%
\pgfsetstrokecolor{currentstroke}%
\pgfsetdash{}{0pt}%
\pgfpathmoveto{\pgfqpoint{7.692348in}{0.521603in}}%
\pgfpathlineto{\pgfqpoint{7.692348in}{2.531888in}}%
\pgfusepath{stroke}%
\end{pgfscope}%
\begin{pgfscope}%
\pgfsetrectcap%
\pgfsetmiterjoin%
\pgfsetlinewidth{0.803000pt}%
\definecolor{currentstroke}{rgb}{0.000000,0.000000,0.000000}%
\pgfsetstrokecolor{currentstroke}%
\pgfsetdash{}{0pt}%
\pgfpathmoveto{\pgfqpoint{0.588387in}{0.521603in}}%
\pgfpathlineto{\pgfqpoint{7.692348in}{0.521603in}}%
\pgfusepath{stroke}%
\end{pgfscope}%
\begin{pgfscope}%
\pgfsetrectcap%
\pgfsetmiterjoin%
\pgfsetlinewidth{0.803000pt}%
\definecolor{currentstroke}{rgb}{0.000000,0.000000,0.000000}%
\pgfsetstrokecolor{currentstroke}%
\pgfsetdash{}{0pt}%
\pgfpathmoveto{\pgfqpoint{0.588387in}{2.531888in}}%
\pgfpathlineto{\pgfqpoint{7.692348in}{2.531888in}}%
\pgfusepath{stroke}%
\end{pgfscope}%
\begin{pgfscope}%
\definecolor{textcolor}{rgb}{0.000000,0.000000,0.000000}%
\pgfsetstrokecolor{textcolor}%
\pgfsetfillcolor{textcolor}%
\pgftext[x=4.140367in,y=2.615222in,,base]{\color{textcolor}{\rmfamily\fontsize{12.000000}{14.400000}\selectfont\catcode`\^=\active\def^{\ifmmode\sp\else\^{}\fi}\catcode`\%=\active\def%{\%}Mean}}%
\end{pgfscope}%
\begin{pgfscope}%
\pgfsetbuttcap%
\pgfsetmiterjoin%
\definecolor{currentfill}{rgb}{1.000000,1.000000,1.000000}%
\pgfsetfillcolor{currentfill}%
\pgfsetfillopacity{0.800000}%
\pgfsetlinewidth{1.003750pt}%
\definecolor{currentstroke}{rgb}{0.800000,0.800000,0.800000}%
\pgfsetstrokecolor{currentstroke}%
\pgfsetstrokeopacity{0.800000}%
\pgfsetdash{}{0pt}%
\pgfpathmoveto{\pgfqpoint{7.779848in}{1.514531in}}%
\pgfpathlineto{\pgfqpoint{8.259376in}{1.514531in}}%
\pgfpathquadraticcurveto{\pgfqpoint{8.284376in}{1.514531in}}{\pgfqpoint{8.284376in}{1.539531in}}%
\pgfpathlineto{\pgfqpoint{8.284376in}{2.444388in}}%
\pgfpathquadraticcurveto{\pgfqpoint{8.284376in}{2.469388in}}{\pgfqpoint{8.259376in}{2.469388in}}%
\pgfpathlineto{\pgfqpoint{7.779848in}{2.469388in}}%
\pgfpathquadraticcurveto{\pgfqpoint{7.754848in}{2.469388in}}{\pgfqpoint{7.754848in}{2.444388in}}%
\pgfpathlineto{\pgfqpoint{7.754848in}{1.539531in}}%
\pgfpathquadraticcurveto{\pgfqpoint{7.754848in}{1.514531in}}{\pgfqpoint{7.779848in}{1.514531in}}%
\pgfpathlineto{\pgfqpoint{7.779848in}{1.514531in}}%
\pgfpathclose%
\pgfusepath{stroke,fill}%
\end{pgfscope}%
\begin{pgfscope}%
\pgfsetrectcap%
\pgfsetroundjoin%
\pgfsetlinewidth{1.505625pt}%
\pgfsetstrokecolor{currentstroke1}%
\pgfsetdash{}{0pt}%
\pgfpathmoveto{\pgfqpoint{7.804848in}{2.368168in}}%
\pgfpathlineto{\pgfqpoint{7.929848in}{2.368168in}}%
\pgfpathlineto{\pgfqpoint{8.054848in}{2.368168in}}%
\pgfusepath{stroke}%
\end{pgfscope}%
\begin{pgfscope}%
\definecolor{textcolor}{rgb}{0.000000,0.000000,0.000000}%
\pgfsetstrokecolor{textcolor}%
\pgfsetfillcolor{textcolor}%
\pgftext[x=8.154848in,y=2.324418in,left,base]{\color{textcolor}{\rmfamily\fontsize{9.000000}{10.800000}\selectfont\catcode`\^=\active\def^{\ifmmode\sp\else\^{}\fi}\catcode`\%=\active\def%{\%}3}}%
\end{pgfscope}%
\begin{pgfscope}%
\pgfsetrectcap%
\pgfsetroundjoin%
\pgfsetlinewidth{1.505625pt}%
\pgfsetstrokecolor{currentstroke2}%
\pgfsetdash{}{0pt}%
\pgfpathmoveto{\pgfqpoint{7.804848in}{2.184696in}}%
\pgfpathlineto{\pgfqpoint{7.929848in}{2.184696in}}%
\pgfpathlineto{\pgfqpoint{8.054848in}{2.184696in}}%
\pgfusepath{stroke}%
\end{pgfscope}%
\begin{pgfscope}%
\definecolor{textcolor}{rgb}{0.000000,0.000000,0.000000}%
\pgfsetstrokecolor{textcolor}%
\pgfsetfillcolor{textcolor}%
\pgftext[x=8.154848in,y=2.140946in,left,base]{\color{textcolor}{\rmfamily\fontsize{9.000000}{10.800000}\selectfont\catcode`\^=\active\def^{\ifmmode\sp\else\^{}\fi}\catcode`\%=\active\def%{\%}4}}%
\end{pgfscope}%
\begin{pgfscope}%
\pgfsetrectcap%
\pgfsetroundjoin%
\pgfsetlinewidth{1.505625pt}%
\pgfsetstrokecolor{currentstroke3}%
\pgfsetdash{}{0pt}%
\pgfpathmoveto{\pgfqpoint{7.804848in}{2.001225in}}%
\pgfpathlineto{\pgfqpoint{7.929848in}{2.001225in}}%
\pgfpathlineto{\pgfqpoint{8.054848in}{2.001225in}}%
\pgfusepath{stroke}%
\end{pgfscope}%
\begin{pgfscope}%
\definecolor{textcolor}{rgb}{0.000000,0.000000,0.000000}%
\pgfsetstrokecolor{textcolor}%
\pgfsetfillcolor{textcolor}%
\pgftext[x=8.154848in,y=1.957475in,left,base]{\color{textcolor}{\rmfamily\fontsize{9.000000}{10.800000}\selectfont\catcode`\^=\active\def^{\ifmmode\sp\else\^{}\fi}\catcode`\%=\active\def%{\%}5}}%
\end{pgfscope}%
\begin{pgfscope}%
\pgfsetrectcap%
\pgfsetroundjoin%
\pgfsetlinewidth{1.505625pt}%
\pgfsetstrokecolor{currentstroke4}%
\pgfsetdash{}{0pt}%
\pgfpathmoveto{\pgfqpoint{7.804848in}{1.817753in}}%
\pgfpathlineto{\pgfqpoint{7.929848in}{1.817753in}}%
\pgfpathlineto{\pgfqpoint{8.054848in}{1.817753in}}%
\pgfusepath{stroke}%
\end{pgfscope}%
\begin{pgfscope}%
\definecolor{textcolor}{rgb}{0.000000,0.000000,0.000000}%
\pgfsetstrokecolor{textcolor}%
\pgfsetfillcolor{textcolor}%
\pgftext[x=8.154848in,y=1.774003in,left,base]{\color{textcolor}{\rmfamily\fontsize{9.000000}{10.800000}\selectfont\catcode`\^=\active\def^{\ifmmode\sp\else\^{}\fi}\catcode`\%=\active\def%{\%}6}}%
\end{pgfscope}%
\begin{pgfscope}%
\pgfsetrectcap%
\pgfsetroundjoin%
\pgfsetlinewidth{1.505625pt}%
\pgfsetstrokecolor{currentstroke5}%
\pgfsetdash{}{0pt}%
\pgfpathmoveto{\pgfqpoint{7.804848in}{1.634281in}}%
\pgfpathlineto{\pgfqpoint{7.929848in}{1.634281in}}%
\pgfpathlineto{\pgfqpoint{8.054848in}{1.634281in}}%
\pgfusepath{stroke}%
\end{pgfscope}%
\begin{pgfscope}%
\definecolor{textcolor}{rgb}{0.000000,0.000000,0.000000}%
\pgfsetstrokecolor{textcolor}%
\pgfsetfillcolor{textcolor}%
\pgftext[x=8.154848in,y=1.590531in,left,base]{\color{textcolor}{\rmfamily\fontsize{9.000000}{10.800000}\selectfont\catcode`\^=\active\def^{\ifmmode\sp\else\^{}\fi}\catcode`\%=\active\def%{\%}7}}%
\end{pgfscope}%
\end{pgfpicture}%
\makeatother%
\endgroup%
}
	\caption[Checks performed for graphs with no NAC-coloring (some).]{
		The number of checks performed to find all NAC-colorings for graphs with no NAC-coloring for different subgraph sizes \( k \).}%
	\label{fig:graph_no_nac_coloring_first_checks_subgraph_size}
\end{figure}



\subsubsection{Failing strategies}%
\label{sec:failing_strategies}

In this section, we show the performance of other strategies described in \Cref{chapter:alg}.
We do not show these strategies in previous graphs as they would influence
the scale and would make graphs and legends unreadable.

Some of these strategies performed as well as our preferred strategies for some graph classes,
but fail for others and therefore are not universal enough.
First, we show in \Cref{fig:graph_mimimally_rigid_failing_merging_first_runtime, fig:graph_no_nac_coloring_generated_rigid_failing_merging_first_runtime}
how strategies like \Log{} and \PromisingCycles{} fails compared to the others.
We fixed a single strategy, but behavior is similar for other strategies.
We also show in \Cref{fig:graph_mimimally_rigid_failing_split_first_runtime}
that \KernighanLin{} and \Cuts{} are not suitable for general graphs.

\begin{figure}[p]
	\centering
	\scalebox{0.5}{%% Creator: Matplotlib, PGF backend
%%
%% To include the figure in your LaTeX document, write
%%   \input{<filename>.pgf}
%%
%% Make sure the required packages are loaded in your preamble
%%   \usepackage{pgf}
%%
%% Also ensure that all the required font packages are loaded; for instance,
%% the lmodern package is sometimes necessary when using math font.
%%   \usepackage{lmodern}
%%
%% Figures using additional raster images can only be included by \input if
%% they are in the same directory as the main LaTeX file. For loading figures
%% from other directories you can use the `import` package
%%   \usepackage{import}
%%
%% and then include the figures with
%%   \import{<path to file>}{<filename>.pgf}
%%
%% Matplotlib used the following preamble
%%   \def\mathdefault#1{#1}
%%   \everymath=\expandafter{\the\everymath\displaystyle}
%%   \IfFileExists{scrextend.sty}{
%%     \usepackage[fontsize=10.000000pt]{scrextend}
%%   }{
%%     \renewcommand{\normalsize}{\fontsize{10.000000}{12.000000}\selectfont}
%%     \normalsize
%%   }
%%   
%%   \ifdefined\pdftexversion\else  % non-pdftex case.
%%     \usepackage{fontspec}
%%     \setmainfont{DejaVuSans.ttf}[Path=\detokenize{/home/petr/Projects/PyRigi/.venv/lib/python3.12/site-packages/matplotlib/mpl-data/fonts/ttf/}]
%%     \setsansfont{DejaVuSans.ttf}[Path=\detokenize{/home/petr/Projects/PyRigi/.venv/lib/python3.12/site-packages/matplotlib/mpl-data/fonts/ttf/}]
%%     \setmonofont{DejaVuSansMono.ttf}[Path=\detokenize{/home/petr/Projects/PyRigi/.venv/lib/python3.12/site-packages/matplotlib/mpl-data/fonts/ttf/}]
%%   \fi
%%   \makeatletter\@ifpackageloaded{under\Score{}}{}{\usepackage[strings]{under\Score{}}}\makeatother
%%
\begingroup%
\makeatletter%
\begin{pgfpicture}%
\pgfpathrectangle{\pgfpointorigin}{\pgfqpoint{8.384376in}{2.841849in}}%
\pgfusepath{use as bounding box, clip}%
\begin{pgfscope}%
\pgfsetbuttcap%
\pgfsetmiterjoin%
\definecolor{currentfill}{rgb}{1.000000,1.000000,1.000000}%
\pgfsetfillcolor{currentfill}%
\pgfsetlinewidth{0.000000pt}%
\definecolor{currentstroke}{rgb}{1.000000,1.000000,1.000000}%
\pgfsetstrokecolor{currentstroke}%
\pgfsetdash{}{0pt}%
\pgfpathmoveto{\pgfqpoint{0.000000in}{0.000000in}}%
\pgfpathlineto{\pgfqpoint{8.384376in}{0.000000in}}%
\pgfpathlineto{\pgfqpoint{8.384376in}{2.841849in}}%
\pgfpathlineto{\pgfqpoint{0.000000in}{2.841849in}}%
\pgfpathlineto{\pgfqpoint{0.000000in}{0.000000in}}%
\pgfpathclose%
\pgfusepath{fill}%
\end{pgfscope}%
\begin{pgfscope}%
\pgfsetbuttcap%
\pgfsetmiterjoin%
\definecolor{currentfill}{rgb}{1.000000,1.000000,1.000000}%
\pgfsetfillcolor{currentfill}%
\pgfsetlinewidth{0.000000pt}%
\definecolor{currentstroke}{rgb}{0.000000,0.000000,0.000000}%
\pgfsetstrokecolor{currentstroke}%
\pgfsetstrokeopacity{0.000000}%
\pgfsetdash{}{0pt}%
\pgfpathmoveto{\pgfqpoint{0.588387in}{0.521603in}}%
\pgfpathlineto{\pgfqpoint{5.903102in}{0.521603in}}%
\pgfpathlineto{\pgfqpoint{5.903102in}{2.531888in}}%
\pgfpathlineto{\pgfqpoint{0.588387in}{2.531888in}}%
\pgfpathlineto{\pgfqpoint{0.588387in}{0.521603in}}%
\pgfpathclose%
\pgfusepath{fill}%
\end{pgfscope}%
\begin{pgfscope}%
\pgfsetbuttcap%
\pgfsetroundjoin%
\definecolor{currentfill}{rgb}{0.000000,0.000000,0.000000}%
\pgfsetfillcolor{currentfill}%
\pgfsetlinewidth{0.803000pt}%
\definecolor{currentstroke}{rgb}{0.000000,0.000000,0.000000}%
\pgfsetstrokecolor{currentstroke}%
\pgfsetdash{}{0pt}%
\pgfsys@defobject{currentmarker}{\pgfqpoint{0.000000in}{-0.048611in}}{\pgfqpoint{0.000000in}{0.000000in}}{%
\pgfpathmoveto{\pgfqpoint{0.000000in}{0.000000in}}%
\pgfpathlineto{\pgfqpoint{0.000000in}{-0.048611in}}%
\pgfusepath{stroke,fill}%
}%
\begin{pgfscope}%
\pgfsys@transformshift{1.027172in}{0.521603in}%
\pgfsys@useobject{currentmarker}{}%
\end{pgfscope}%
\end{pgfscope}%
\begin{pgfscope}%
\definecolor{textcolor}{rgb}{0.000000,0.000000,0.000000}%
\pgfsetstrokecolor{textcolor}%
\pgfsetfillcolor{textcolor}%
\pgftext[x=1.027172in,y=0.424381in,,top]{\color{textcolor}{\rmfamily\fontsize{10.000000}{12.000000}\selectfont\catcode`\^=\active\def^{\ifmmode\sp\else\^{}\fi}\catcode`\%=\active\def%{\%}$\mathdefault{12}$}}%
\end{pgfscope}%
\begin{pgfscope}%
\pgfsetbuttcap%
\pgfsetroundjoin%
\definecolor{currentfill}{rgb}{0.000000,0.000000,0.000000}%
\pgfsetfillcolor{currentfill}%
\pgfsetlinewidth{0.803000pt}%
\definecolor{currentstroke}{rgb}{0.000000,0.000000,0.000000}%
\pgfsetstrokecolor{currentstroke}%
\pgfsetdash{}{0pt}%
\pgfsys@defobject{currentmarker}{\pgfqpoint{0.000000in}{-0.048611in}}{\pgfqpoint{0.000000in}{0.000000in}}{%
\pgfpathmoveto{\pgfqpoint{0.000000in}{0.000000in}}%
\pgfpathlineto{\pgfqpoint{0.000000in}{-0.048611in}}%
\pgfusepath{stroke,fill}%
}%
\begin{pgfscope}%
\pgfsys@transformshift{1.618791in}{0.521603in}%
\pgfsys@useobject{currentmarker}{}%
\end{pgfscope}%
\end{pgfscope}%
\begin{pgfscope}%
\definecolor{textcolor}{rgb}{0.000000,0.000000,0.000000}%
\pgfsetstrokecolor{textcolor}%
\pgfsetfillcolor{textcolor}%
\pgftext[x=1.618791in,y=0.424381in,,top]{\color{textcolor}{\rmfamily\fontsize{10.000000}{12.000000}\selectfont\catcode`\^=\active\def^{\ifmmode\sp\else\^{}\fi}\catcode`\%=\active\def%{\%}$\mathdefault{18}$}}%
\end{pgfscope}%
\begin{pgfscope}%
\pgfsetbuttcap%
\pgfsetroundjoin%
\definecolor{currentfill}{rgb}{0.000000,0.000000,0.000000}%
\pgfsetfillcolor{currentfill}%
\pgfsetlinewidth{0.803000pt}%
\definecolor{currentstroke}{rgb}{0.000000,0.000000,0.000000}%
\pgfsetstrokecolor{currentstroke}%
\pgfsetdash{}{0pt}%
\pgfsys@defobject{currentmarker}{\pgfqpoint{0.000000in}{-0.048611in}}{\pgfqpoint{0.000000in}{0.000000in}}{%
\pgfpathmoveto{\pgfqpoint{0.000000in}{0.000000in}}%
\pgfpathlineto{\pgfqpoint{0.000000in}{-0.048611in}}%
\pgfusepath{stroke,fill}%
}%
\begin{pgfscope}%
\pgfsys@transformshift{2.210411in}{0.521603in}%
\pgfsys@useobject{currentmarker}{}%
\end{pgfscope}%
\end{pgfscope}%
\begin{pgfscope}%
\definecolor{textcolor}{rgb}{0.000000,0.000000,0.000000}%
\pgfsetstrokecolor{textcolor}%
\pgfsetfillcolor{textcolor}%
\pgftext[x=2.210411in,y=0.424381in,,top]{\color{textcolor}{\rmfamily\fontsize{10.000000}{12.000000}\selectfont\catcode`\^=\active\def^{\ifmmode\sp\else\^{}\fi}\catcode`\%=\active\def%{\%}$\mathdefault{24}$}}%
\end{pgfscope}%
\begin{pgfscope}%
\pgfsetbuttcap%
\pgfsetroundjoin%
\definecolor{currentfill}{rgb}{0.000000,0.000000,0.000000}%
\pgfsetfillcolor{currentfill}%
\pgfsetlinewidth{0.803000pt}%
\definecolor{currentstroke}{rgb}{0.000000,0.000000,0.000000}%
\pgfsetstrokecolor{currentstroke}%
\pgfsetdash{}{0pt}%
\pgfsys@defobject{currentmarker}{\pgfqpoint{0.000000in}{-0.048611in}}{\pgfqpoint{0.000000in}{0.000000in}}{%
\pgfpathmoveto{\pgfqpoint{0.000000in}{0.000000in}}%
\pgfpathlineto{\pgfqpoint{0.000000in}{-0.048611in}}%
\pgfusepath{stroke,fill}%
}%
\begin{pgfscope}%
\pgfsys@transformshift{2.802030in}{0.521603in}%
\pgfsys@useobject{currentmarker}{}%
\end{pgfscope}%
\end{pgfscope}%
\begin{pgfscope}%
\definecolor{textcolor}{rgb}{0.000000,0.000000,0.000000}%
\pgfsetstrokecolor{textcolor}%
\pgfsetfillcolor{textcolor}%
\pgftext[x=2.802030in,y=0.424381in,,top]{\color{textcolor}{\rmfamily\fontsize{10.000000}{12.000000}\selectfont\catcode`\^=\active\def^{\ifmmode\sp\else\^{}\fi}\catcode`\%=\active\def%{\%}$\mathdefault{30}$}}%
\end{pgfscope}%
\begin{pgfscope}%
\pgfsetbuttcap%
\pgfsetroundjoin%
\definecolor{currentfill}{rgb}{0.000000,0.000000,0.000000}%
\pgfsetfillcolor{currentfill}%
\pgfsetlinewidth{0.803000pt}%
\definecolor{currentstroke}{rgb}{0.000000,0.000000,0.000000}%
\pgfsetstrokecolor{currentstroke}%
\pgfsetdash{}{0pt}%
\pgfsys@defobject{currentmarker}{\pgfqpoint{0.000000in}{-0.048611in}}{\pgfqpoint{0.000000in}{0.000000in}}{%
\pgfpathmoveto{\pgfqpoint{0.000000in}{0.000000in}}%
\pgfpathlineto{\pgfqpoint{0.000000in}{-0.048611in}}%
\pgfusepath{stroke,fill}%
}%
\begin{pgfscope}%
\pgfsys@transformshift{3.393649in}{0.521603in}%
\pgfsys@useobject{currentmarker}{}%
\end{pgfscope}%
\end{pgfscope}%
\begin{pgfscope}%
\definecolor{textcolor}{rgb}{0.000000,0.000000,0.000000}%
\pgfsetstrokecolor{textcolor}%
\pgfsetfillcolor{textcolor}%
\pgftext[x=3.393649in,y=0.424381in,,top]{\color{textcolor}{\rmfamily\fontsize{10.000000}{12.000000}\selectfont\catcode`\^=\active\def^{\ifmmode\sp\else\^{}\fi}\catcode`\%=\active\def%{\%}$\mathdefault{36}$}}%
\end{pgfscope}%
\begin{pgfscope}%
\pgfsetbuttcap%
\pgfsetroundjoin%
\definecolor{currentfill}{rgb}{0.000000,0.000000,0.000000}%
\pgfsetfillcolor{currentfill}%
\pgfsetlinewidth{0.803000pt}%
\definecolor{currentstroke}{rgb}{0.000000,0.000000,0.000000}%
\pgfsetstrokecolor{currentstroke}%
\pgfsetdash{}{0pt}%
\pgfsys@defobject{currentmarker}{\pgfqpoint{0.000000in}{-0.048611in}}{\pgfqpoint{0.000000in}{0.000000in}}{%
\pgfpathmoveto{\pgfqpoint{0.000000in}{0.000000in}}%
\pgfpathlineto{\pgfqpoint{0.000000in}{-0.048611in}}%
\pgfusepath{stroke,fill}%
}%
\begin{pgfscope}%
\pgfsys@transformshift{3.985269in}{0.521603in}%
\pgfsys@useobject{currentmarker}{}%
\end{pgfscope}%
\end{pgfscope}%
\begin{pgfscope}%
\definecolor{textcolor}{rgb}{0.000000,0.000000,0.000000}%
\pgfsetstrokecolor{textcolor}%
\pgfsetfillcolor{textcolor}%
\pgftext[x=3.985269in,y=0.424381in,,top]{\color{textcolor}{\rmfamily\fontsize{10.000000}{12.000000}\selectfont\catcode`\^=\active\def^{\ifmmode\sp\else\^{}\fi}\catcode`\%=\active\def%{\%}$\mathdefault{42}$}}%
\end{pgfscope}%
\begin{pgfscope}%
\pgfsetbuttcap%
\pgfsetroundjoin%
\definecolor{currentfill}{rgb}{0.000000,0.000000,0.000000}%
\pgfsetfillcolor{currentfill}%
\pgfsetlinewidth{0.803000pt}%
\definecolor{currentstroke}{rgb}{0.000000,0.000000,0.000000}%
\pgfsetstrokecolor{currentstroke}%
\pgfsetdash{}{0pt}%
\pgfsys@defobject{currentmarker}{\pgfqpoint{0.000000in}{-0.048611in}}{\pgfqpoint{0.000000in}{0.000000in}}{%
\pgfpathmoveto{\pgfqpoint{0.000000in}{0.000000in}}%
\pgfpathlineto{\pgfqpoint{0.000000in}{-0.048611in}}%
\pgfusepath{stroke,fill}%
}%
\begin{pgfscope}%
\pgfsys@transformshift{4.576888in}{0.521603in}%
\pgfsys@useobject{currentmarker}{}%
\end{pgfscope}%
\end{pgfscope}%
\begin{pgfscope}%
\definecolor{textcolor}{rgb}{0.000000,0.000000,0.000000}%
\pgfsetstrokecolor{textcolor}%
\pgfsetfillcolor{textcolor}%
\pgftext[x=4.576888in,y=0.424381in,,top]{\color{textcolor}{\rmfamily\fontsize{10.000000}{12.000000}\selectfont\catcode`\^=\active\def^{\ifmmode\sp\else\^{}\fi}\catcode`\%=\active\def%{\%}$\mathdefault{48}$}}%
\end{pgfscope}%
\begin{pgfscope}%
\pgfsetbuttcap%
\pgfsetroundjoin%
\definecolor{currentfill}{rgb}{0.000000,0.000000,0.000000}%
\pgfsetfillcolor{currentfill}%
\pgfsetlinewidth{0.803000pt}%
\definecolor{currentstroke}{rgb}{0.000000,0.000000,0.000000}%
\pgfsetstrokecolor{currentstroke}%
\pgfsetdash{}{0pt}%
\pgfsys@defobject{currentmarker}{\pgfqpoint{0.000000in}{-0.048611in}}{\pgfqpoint{0.000000in}{0.000000in}}{%
\pgfpathmoveto{\pgfqpoint{0.000000in}{0.000000in}}%
\pgfpathlineto{\pgfqpoint{0.000000in}{-0.048611in}}%
\pgfusepath{stroke,fill}%
}%
\begin{pgfscope}%
\pgfsys@transformshift{5.168508in}{0.521603in}%
\pgfsys@useobject{currentmarker}{}%
\end{pgfscope}%
\end{pgfscope}%
\begin{pgfscope}%
\definecolor{textcolor}{rgb}{0.000000,0.000000,0.000000}%
\pgfsetstrokecolor{textcolor}%
\pgfsetfillcolor{textcolor}%
\pgftext[x=5.168508in,y=0.424381in,,top]{\color{textcolor}{\rmfamily\fontsize{10.000000}{12.000000}\selectfont\catcode`\^=\active\def^{\ifmmode\sp\else\^{}\fi}\catcode`\%=\active\def%{\%}$\mathdefault{54}$}}%
\end{pgfscope}%
\begin{pgfscope}%
\pgfsetbuttcap%
\pgfsetroundjoin%
\definecolor{currentfill}{rgb}{0.000000,0.000000,0.000000}%
\pgfsetfillcolor{currentfill}%
\pgfsetlinewidth{0.803000pt}%
\definecolor{currentstroke}{rgb}{0.000000,0.000000,0.000000}%
\pgfsetstrokecolor{currentstroke}%
\pgfsetdash{}{0pt}%
\pgfsys@defobject{currentmarker}{\pgfqpoint{0.000000in}{-0.048611in}}{\pgfqpoint{0.000000in}{0.000000in}}{%
\pgfpathmoveto{\pgfqpoint{0.000000in}{0.000000in}}%
\pgfpathlineto{\pgfqpoint{0.000000in}{-0.048611in}}%
\pgfusepath{stroke,fill}%
}%
\begin{pgfscope}%
\pgfsys@transformshift{5.760127in}{0.521603in}%
\pgfsys@useobject{currentmarker}{}%
\end{pgfscope}%
\end{pgfscope}%
\begin{pgfscope}%
\definecolor{textcolor}{rgb}{0.000000,0.000000,0.000000}%
\pgfsetstrokecolor{textcolor}%
\pgfsetfillcolor{textcolor}%
\pgftext[x=5.760127in,y=0.424381in,,top]{\color{textcolor}{\rmfamily\fontsize{10.000000}{12.000000}\selectfont\catcode`\^=\active\def^{\ifmmode\sp\else\^{}\fi}\catcode`\%=\active\def%{\%}$\mathdefault{60}$}}%
\end{pgfscope}%
\begin{pgfscope}%
\definecolor{textcolor}{rgb}{0.000000,0.000000,0.000000}%
\pgfsetstrokecolor{textcolor}%
\pgfsetfillcolor{textcolor}%
\pgftext[x=3.245745in,y=0.234413in,,top]{\color{textcolor}{\rmfamily\fontsize{10.000000}{12.000000}\selectfont\catcode`\^=\active\def^{\ifmmode\sp\else\^{}\fi}\catcode`\%=\active\def%{\%}Vertices}}%
\end{pgfscope}%
\begin{pgfscope}%
\pgfsetbuttcap%
\pgfsetroundjoin%
\definecolor{currentfill}{rgb}{0.000000,0.000000,0.000000}%
\pgfsetfillcolor{currentfill}%
\pgfsetlinewidth{0.803000pt}%
\definecolor{currentstroke}{rgb}{0.000000,0.000000,0.000000}%
\pgfsetstrokecolor{currentstroke}%
\pgfsetdash{}{0pt}%
\pgfsys@defobject{currentmarker}{\pgfqpoint{-0.048611in}{0.000000in}}{\pgfqpoint{-0.000000in}{0.000000in}}{%
\pgfpathmoveto{\pgfqpoint{-0.000000in}{0.000000in}}%
\pgfpathlineto{\pgfqpoint{-0.048611in}{0.000000in}}%
\pgfusepath{stroke,fill}%
}%
\begin{pgfscope}%
\pgfsys@transformshift{0.588387in}{0.617445in}%
\pgfsys@useobject{currentmarker}{}%
\end{pgfscope}%
\end{pgfscope}%
\begin{pgfscope}%
\definecolor{textcolor}{rgb}{0.000000,0.000000,0.000000}%
\pgfsetstrokecolor{textcolor}%
\pgfsetfillcolor{textcolor}%
\pgftext[x=0.289968in, y=0.564684in, left, base]{\color{textcolor}{\rmfamily\fontsize{10.000000}{12.000000}\selectfont\catcode`\^=\active\def^{\ifmmode\sp\else\^{}\fi}\catcode`\%=\active\def%{\%}$\mathdefault{10^{1}}$}}%
\end{pgfscope}%
\begin{pgfscope}%
\pgfsetbuttcap%
\pgfsetroundjoin%
\definecolor{currentfill}{rgb}{0.000000,0.000000,0.000000}%
\pgfsetfillcolor{currentfill}%
\pgfsetlinewidth{0.803000pt}%
\definecolor{currentstroke}{rgb}{0.000000,0.000000,0.000000}%
\pgfsetstrokecolor{currentstroke}%
\pgfsetdash{}{0pt}%
\pgfsys@defobject{currentmarker}{\pgfqpoint{-0.048611in}{0.000000in}}{\pgfqpoint{-0.000000in}{0.000000in}}{%
\pgfpathmoveto{\pgfqpoint{-0.000000in}{0.000000in}}%
\pgfpathlineto{\pgfqpoint{-0.048611in}{0.000000in}}%
\pgfusepath{stroke,fill}%
}%
\begin{pgfscope}%
\pgfsys@transformshift{0.588387in}{1.073413in}%
\pgfsys@useobject{currentmarker}{}%
\end{pgfscope}%
\end{pgfscope}%
\begin{pgfscope}%
\definecolor{textcolor}{rgb}{0.000000,0.000000,0.000000}%
\pgfsetstrokecolor{textcolor}%
\pgfsetfillcolor{textcolor}%
\pgftext[x=0.289968in, y=1.020651in, left, base]{\color{textcolor}{\rmfamily\fontsize{10.000000}{12.000000}\selectfont\catcode`\^=\active\def^{\ifmmode\sp\else\^{}\fi}\catcode`\%=\active\def%{\%}$\mathdefault{10^{2}}$}}%
\end{pgfscope}%
\begin{pgfscope}%
\pgfsetbuttcap%
\pgfsetroundjoin%
\definecolor{currentfill}{rgb}{0.000000,0.000000,0.000000}%
\pgfsetfillcolor{currentfill}%
\pgfsetlinewidth{0.803000pt}%
\definecolor{currentstroke}{rgb}{0.000000,0.000000,0.000000}%
\pgfsetstrokecolor{currentstroke}%
\pgfsetdash{}{0pt}%
\pgfsys@defobject{currentmarker}{\pgfqpoint{-0.048611in}{0.000000in}}{\pgfqpoint{-0.000000in}{0.000000in}}{%
\pgfpathmoveto{\pgfqpoint{-0.000000in}{0.000000in}}%
\pgfpathlineto{\pgfqpoint{-0.048611in}{0.000000in}}%
\pgfusepath{stroke,fill}%
}%
\begin{pgfscope}%
\pgfsys@transformshift{0.588387in}{1.529380in}%
\pgfsys@useobject{currentmarker}{}%
\end{pgfscope}%
\end{pgfscope}%
\begin{pgfscope}%
\definecolor{textcolor}{rgb}{0.000000,0.000000,0.000000}%
\pgfsetstrokecolor{textcolor}%
\pgfsetfillcolor{textcolor}%
\pgftext[x=0.289968in, y=1.476619in, left, base]{\color{textcolor}{\rmfamily\fontsize{10.000000}{12.000000}\selectfont\catcode`\^=\active\def^{\ifmmode\sp\else\^{}\fi}\catcode`\%=\active\def%{\%}$\mathdefault{10^{3}}$}}%
\end{pgfscope}%
\begin{pgfscope}%
\pgfsetbuttcap%
\pgfsetroundjoin%
\definecolor{currentfill}{rgb}{0.000000,0.000000,0.000000}%
\pgfsetfillcolor{currentfill}%
\pgfsetlinewidth{0.803000pt}%
\definecolor{currentstroke}{rgb}{0.000000,0.000000,0.000000}%
\pgfsetstrokecolor{currentstroke}%
\pgfsetdash{}{0pt}%
\pgfsys@defobject{currentmarker}{\pgfqpoint{-0.048611in}{0.000000in}}{\pgfqpoint{-0.000000in}{0.000000in}}{%
\pgfpathmoveto{\pgfqpoint{-0.000000in}{0.000000in}}%
\pgfpathlineto{\pgfqpoint{-0.048611in}{0.000000in}}%
\pgfusepath{stroke,fill}%
}%
\begin{pgfscope}%
\pgfsys@transformshift{0.588387in}{1.985347in}%
\pgfsys@useobject{currentmarker}{}%
\end{pgfscope}%
\end{pgfscope}%
\begin{pgfscope}%
\definecolor{textcolor}{rgb}{0.000000,0.000000,0.000000}%
\pgfsetstrokecolor{textcolor}%
\pgfsetfillcolor{textcolor}%
\pgftext[x=0.289968in, y=1.932586in, left, base]{\color{textcolor}{\rmfamily\fontsize{10.000000}{12.000000}\selectfont\catcode`\^=\active\def^{\ifmmode\sp\else\^{}\fi}\catcode`\%=\active\def%{\%}$\mathdefault{10^{4}}$}}%
\end{pgfscope}%
\begin{pgfscope}%
\pgfsetbuttcap%
\pgfsetroundjoin%
\definecolor{currentfill}{rgb}{0.000000,0.000000,0.000000}%
\pgfsetfillcolor{currentfill}%
\pgfsetlinewidth{0.803000pt}%
\definecolor{currentstroke}{rgb}{0.000000,0.000000,0.000000}%
\pgfsetstrokecolor{currentstroke}%
\pgfsetdash{}{0pt}%
\pgfsys@defobject{currentmarker}{\pgfqpoint{-0.048611in}{0.000000in}}{\pgfqpoint{-0.000000in}{0.000000in}}{%
\pgfpathmoveto{\pgfqpoint{-0.000000in}{0.000000in}}%
\pgfpathlineto{\pgfqpoint{-0.048611in}{0.000000in}}%
\pgfusepath{stroke,fill}%
}%
\begin{pgfscope}%
\pgfsys@transformshift{0.588387in}{2.441315in}%
\pgfsys@useobject{currentmarker}{}%
\end{pgfscope}%
\end{pgfscope}%
\begin{pgfscope}%
\definecolor{textcolor}{rgb}{0.000000,0.000000,0.000000}%
\pgfsetstrokecolor{textcolor}%
\pgfsetfillcolor{textcolor}%
\pgftext[x=0.289968in, y=2.388553in, left, base]{\color{textcolor}{\rmfamily\fontsize{10.000000}{12.000000}\selectfont\catcode`\^=\active\def^{\ifmmode\sp\else\^{}\fi}\catcode`\%=\active\def%{\%}$\mathdefault{10^{5}}$}}%
\end{pgfscope}%
\begin{pgfscope}%
\pgfsetbuttcap%
\pgfsetroundjoin%
\definecolor{currentfill}{rgb}{0.000000,0.000000,0.000000}%
\pgfsetfillcolor{currentfill}%
\pgfsetlinewidth{0.602250pt}%
\definecolor{currentstroke}{rgb}{0.000000,0.000000,0.000000}%
\pgfsetstrokecolor{currentstroke}%
\pgfsetdash{}{0pt}%
\pgfsys@defobject{currentmarker}{\pgfqpoint{-0.027778in}{0.000000in}}{\pgfqpoint{-0.000000in}{0.000000in}}{%
\pgfpathmoveto{\pgfqpoint{-0.000000in}{0.000000in}}%
\pgfpathlineto{\pgfqpoint{-0.027778in}{0.000000in}}%
\pgfusepath{stroke,fill}%
}%
\begin{pgfscope}%
\pgfsys@transformshift{0.588387in}{0.546815in}%
\pgfsys@useobject{currentmarker}{}%
\end{pgfscope}%
\end{pgfscope}%
\begin{pgfscope}%
\pgfsetbuttcap%
\pgfsetroundjoin%
\definecolor{currentfill}{rgb}{0.000000,0.000000,0.000000}%
\pgfsetfillcolor{currentfill}%
\pgfsetlinewidth{0.602250pt}%
\definecolor{currentstroke}{rgb}{0.000000,0.000000,0.000000}%
\pgfsetstrokecolor{currentstroke}%
\pgfsetdash{}{0pt}%
\pgfsys@defobject{currentmarker}{\pgfqpoint{-0.027778in}{0.000000in}}{\pgfqpoint{-0.000000in}{0.000000in}}{%
\pgfpathmoveto{\pgfqpoint{-0.000000in}{0.000000in}}%
\pgfpathlineto{\pgfqpoint{-0.027778in}{0.000000in}}%
\pgfusepath{stroke,fill}%
}%
\begin{pgfscope}%
\pgfsys@transformshift{0.588387in}{0.573257in}%
\pgfsys@useobject{currentmarker}{}%
\end{pgfscope}%
\end{pgfscope}%
\begin{pgfscope}%
\pgfsetbuttcap%
\pgfsetroundjoin%
\definecolor{currentfill}{rgb}{0.000000,0.000000,0.000000}%
\pgfsetfillcolor{currentfill}%
\pgfsetlinewidth{0.602250pt}%
\definecolor{currentstroke}{rgb}{0.000000,0.000000,0.000000}%
\pgfsetstrokecolor{currentstroke}%
\pgfsetdash{}{0pt}%
\pgfsys@defobject{currentmarker}{\pgfqpoint{-0.027778in}{0.000000in}}{\pgfqpoint{-0.000000in}{0.000000in}}{%
\pgfpathmoveto{\pgfqpoint{-0.000000in}{0.000000in}}%
\pgfpathlineto{\pgfqpoint{-0.027778in}{0.000000in}}%
\pgfusepath{stroke,fill}%
}%
\begin{pgfscope}%
\pgfsys@transformshift{0.588387in}{0.596581in}%
\pgfsys@useobject{currentmarker}{}%
\end{pgfscope}%
\end{pgfscope}%
\begin{pgfscope}%
\pgfsetbuttcap%
\pgfsetroundjoin%
\definecolor{currentfill}{rgb}{0.000000,0.000000,0.000000}%
\pgfsetfillcolor{currentfill}%
\pgfsetlinewidth{0.602250pt}%
\definecolor{currentstroke}{rgb}{0.000000,0.000000,0.000000}%
\pgfsetstrokecolor{currentstroke}%
\pgfsetdash{}{0pt}%
\pgfsys@defobject{currentmarker}{\pgfqpoint{-0.027778in}{0.000000in}}{\pgfqpoint{-0.000000in}{0.000000in}}{%
\pgfpathmoveto{\pgfqpoint{-0.000000in}{0.000000in}}%
\pgfpathlineto{\pgfqpoint{-0.027778in}{0.000000in}}%
\pgfusepath{stroke,fill}%
}%
\begin{pgfscope}%
\pgfsys@transformshift{0.588387in}{0.754705in}%
\pgfsys@useobject{currentmarker}{}%
\end{pgfscope}%
\end{pgfscope}%
\begin{pgfscope}%
\pgfsetbuttcap%
\pgfsetroundjoin%
\definecolor{currentfill}{rgb}{0.000000,0.000000,0.000000}%
\pgfsetfillcolor{currentfill}%
\pgfsetlinewidth{0.602250pt}%
\definecolor{currentstroke}{rgb}{0.000000,0.000000,0.000000}%
\pgfsetstrokecolor{currentstroke}%
\pgfsetdash{}{0pt}%
\pgfsys@defobject{currentmarker}{\pgfqpoint{-0.027778in}{0.000000in}}{\pgfqpoint{-0.000000in}{0.000000in}}{%
\pgfpathmoveto{\pgfqpoint{-0.000000in}{0.000000in}}%
\pgfpathlineto{\pgfqpoint{-0.027778in}{0.000000in}}%
\pgfusepath{stroke,fill}%
}%
\begin{pgfscope}%
\pgfsys@transformshift{0.588387in}{0.834997in}%
\pgfsys@useobject{currentmarker}{}%
\end{pgfscope}%
\end{pgfscope}%
\begin{pgfscope}%
\pgfsetbuttcap%
\pgfsetroundjoin%
\definecolor{currentfill}{rgb}{0.000000,0.000000,0.000000}%
\pgfsetfillcolor{currentfill}%
\pgfsetlinewidth{0.602250pt}%
\definecolor{currentstroke}{rgb}{0.000000,0.000000,0.000000}%
\pgfsetstrokecolor{currentstroke}%
\pgfsetdash{}{0pt}%
\pgfsys@defobject{currentmarker}{\pgfqpoint{-0.027778in}{0.000000in}}{\pgfqpoint{-0.000000in}{0.000000in}}{%
\pgfpathmoveto{\pgfqpoint{-0.000000in}{0.000000in}}%
\pgfpathlineto{\pgfqpoint{-0.027778in}{0.000000in}}%
\pgfusepath{stroke,fill}%
}%
\begin{pgfscope}%
\pgfsys@transformshift{0.588387in}{0.891965in}%
\pgfsys@useobject{currentmarker}{}%
\end{pgfscope}%
\end{pgfscope}%
\begin{pgfscope}%
\pgfsetbuttcap%
\pgfsetroundjoin%
\definecolor{currentfill}{rgb}{0.000000,0.000000,0.000000}%
\pgfsetfillcolor{currentfill}%
\pgfsetlinewidth{0.602250pt}%
\definecolor{currentstroke}{rgb}{0.000000,0.000000,0.000000}%
\pgfsetstrokecolor{currentstroke}%
\pgfsetdash{}{0pt}%
\pgfsys@defobject{currentmarker}{\pgfqpoint{-0.027778in}{0.000000in}}{\pgfqpoint{-0.000000in}{0.000000in}}{%
\pgfpathmoveto{\pgfqpoint{-0.000000in}{0.000000in}}%
\pgfpathlineto{\pgfqpoint{-0.027778in}{0.000000in}}%
\pgfusepath{stroke,fill}%
}%
\begin{pgfscope}%
\pgfsys@transformshift{0.588387in}{0.936153in}%
\pgfsys@useobject{currentmarker}{}%
\end{pgfscope}%
\end{pgfscope}%
\begin{pgfscope}%
\pgfsetbuttcap%
\pgfsetroundjoin%
\definecolor{currentfill}{rgb}{0.000000,0.000000,0.000000}%
\pgfsetfillcolor{currentfill}%
\pgfsetlinewidth{0.602250pt}%
\definecolor{currentstroke}{rgb}{0.000000,0.000000,0.000000}%
\pgfsetstrokecolor{currentstroke}%
\pgfsetdash{}{0pt}%
\pgfsys@defobject{currentmarker}{\pgfqpoint{-0.027778in}{0.000000in}}{\pgfqpoint{-0.000000in}{0.000000in}}{%
\pgfpathmoveto{\pgfqpoint{-0.000000in}{0.000000in}}%
\pgfpathlineto{\pgfqpoint{-0.027778in}{0.000000in}}%
\pgfusepath{stroke,fill}%
}%
\begin{pgfscope}%
\pgfsys@transformshift{0.588387in}{0.972257in}%
\pgfsys@useobject{currentmarker}{}%
\end{pgfscope}%
\end{pgfscope}%
\begin{pgfscope}%
\pgfsetbuttcap%
\pgfsetroundjoin%
\definecolor{currentfill}{rgb}{0.000000,0.000000,0.000000}%
\pgfsetfillcolor{currentfill}%
\pgfsetlinewidth{0.602250pt}%
\definecolor{currentstroke}{rgb}{0.000000,0.000000,0.000000}%
\pgfsetstrokecolor{currentstroke}%
\pgfsetdash{}{0pt}%
\pgfsys@defobject{currentmarker}{\pgfqpoint{-0.027778in}{0.000000in}}{\pgfqpoint{-0.000000in}{0.000000in}}{%
\pgfpathmoveto{\pgfqpoint{-0.000000in}{0.000000in}}%
\pgfpathlineto{\pgfqpoint{-0.027778in}{0.000000in}}%
\pgfusepath{stroke,fill}%
}%
\begin{pgfscope}%
\pgfsys@transformshift{0.588387in}{1.002782in}%
\pgfsys@useobject{currentmarker}{}%
\end{pgfscope}%
\end{pgfscope}%
\begin{pgfscope}%
\pgfsetbuttcap%
\pgfsetroundjoin%
\definecolor{currentfill}{rgb}{0.000000,0.000000,0.000000}%
\pgfsetfillcolor{currentfill}%
\pgfsetlinewidth{0.602250pt}%
\definecolor{currentstroke}{rgb}{0.000000,0.000000,0.000000}%
\pgfsetstrokecolor{currentstroke}%
\pgfsetdash{}{0pt}%
\pgfsys@defobject{currentmarker}{\pgfqpoint{-0.027778in}{0.000000in}}{\pgfqpoint{-0.000000in}{0.000000in}}{%
\pgfpathmoveto{\pgfqpoint{-0.000000in}{0.000000in}}%
\pgfpathlineto{\pgfqpoint{-0.027778in}{0.000000in}}%
\pgfusepath{stroke,fill}%
}%
\begin{pgfscope}%
\pgfsys@transformshift{0.588387in}{1.029225in}%
\pgfsys@useobject{currentmarker}{}%
\end{pgfscope}%
\end{pgfscope}%
\begin{pgfscope}%
\pgfsetbuttcap%
\pgfsetroundjoin%
\definecolor{currentfill}{rgb}{0.000000,0.000000,0.000000}%
\pgfsetfillcolor{currentfill}%
\pgfsetlinewidth{0.602250pt}%
\definecolor{currentstroke}{rgb}{0.000000,0.000000,0.000000}%
\pgfsetstrokecolor{currentstroke}%
\pgfsetdash{}{0pt}%
\pgfsys@defobject{currentmarker}{\pgfqpoint{-0.027778in}{0.000000in}}{\pgfqpoint{-0.000000in}{0.000000in}}{%
\pgfpathmoveto{\pgfqpoint{-0.000000in}{0.000000in}}%
\pgfpathlineto{\pgfqpoint{-0.027778in}{0.000000in}}%
\pgfusepath{stroke,fill}%
}%
\begin{pgfscope}%
\pgfsys@transformshift{0.588387in}{1.052549in}%
\pgfsys@useobject{currentmarker}{}%
\end{pgfscope}%
\end{pgfscope}%
\begin{pgfscope}%
\pgfsetbuttcap%
\pgfsetroundjoin%
\definecolor{currentfill}{rgb}{0.000000,0.000000,0.000000}%
\pgfsetfillcolor{currentfill}%
\pgfsetlinewidth{0.602250pt}%
\definecolor{currentstroke}{rgb}{0.000000,0.000000,0.000000}%
\pgfsetstrokecolor{currentstroke}%
\pgfsetdash{}{0pt}%
\pgfsys@defobject{currentmarker}{\pgfqpoint{-0.027778in}{0.000000in}}{\pgfqpoint{-0.000000in}{0.000000in}}{%
\pgfpathmoveto{\pgfqpoint{-0.000000in}{0.000000in}}%
\pgfpathlineto{\pgfqpoint{-0.027778in}{0.000000in}}%
\pgfusepath{stroke,fill}%
}%
\begin{pgfscope}%
\pgfsys@transformshift{0.588387in}{1.210673in}%
\pgfsys@useobject{currentmarker}{}%
\end{pgfscope}%
\end{pgfscope}%
\begin{pgfscope}%
\pgfsetbuttcap%
\pgfsetroundjoin%
\definecolor{currentfill}{rgb}{0.000000,0.000000,0.000000}%
\pgfsetfillcolor{currentfill}%
\pgfsetlinewidth{0.602250pt}%
\definecolor{currentstroke}{rgb}{0.000000,0.000000,0.000000}%
\pgfsetstrokecolor{currentstroke}%
\pgfsetdash{}{0pt}%
\pgfsys@defobject{currentmarker}{\pgfqpoint{-0.027778in}{0.000000in}}{\pgfqpoint{-0.000000in}{0.000000in}}{%
\pgfpathmoveto{\pgfqpoint{-0.000000in}{0.000000in}}%
\pgfpathlineto{\pgfqpoint{-0.027778in}{0.000000in}}%
\pgfusepath{stroke,fill}%
}%
\begin{pgfscope}%
\pgfsys@transformshift{0.588387in}{1.290964in}%
\pgfsys@useobject{currentmarker}{}%
\end{pgfscope}%
\end{pgfscope}%
\begin{pgfscope}%
\pgfsetbuttcap%
\pgfsetroundjoin%
\definecolor{currentfill}{rgb}{0.000000,0.000000,0.000000}%
\pgfsetfillcolor{currentfill}%
\pgfsetlinewidth{0.602250pt}%
\definecolor{currentstroke}{rgb}{0.000000,0.000000,0.000000}%
\pgfsetstrokecolor{currentstroke}%
\pgfsetdash{}{0pt}%
\pgfsys@defobject{currentmarker}{\pgfqpoint{-0.027778in}{0.000000in}}{\pgfqpoint{-0.000000in}{0.000000in}}{%
\pgfpathmoveto{\pgfqpoint{-0.000000in}{0.000000in}}%
\pgfpathlineto{\pgfqpoint{-0.027778in}{0.000000in}}%
\pgfusepath{stroke,fill}%
}%
\begin{pgfscope}%
\pgfsys@transformshift{0.588387in}{1.347932in}%
\pgfsys@useobject{currentmarker}{}%
\end{pgfscope}%
\end{pgfscope}%
\begin{pgfscope}%
\pgfsetbuttcap%
\pgfsetroundjoin%
\definecolor{currentfill}{rgb}{0.000000,0.000000,0.000000}%
\pgfsetfillcolor{currentfill}%
\pgfsetlinewidth{0.602250pt}%
\definecolor{currentstroke}{rgb}{0.000000,0.000000,0.000000}%
\pgfsetstrokecolor{currentstroke}%
\pgfsetdash{}{0pt}%
\pgfsys@defobject{currentmarker}{\pgfqpoint{-0.027778in}{0.000000in}}{\pgfqpoint{-0.000000in}{0.000000in}}{%
\pgfpathmoveto{\pgfqpoint{-0.000000in}{0.000000in}}%
\pgfpathlineto{\pgfqpoint{-0.027778in}{0.000000in}}%
\pgfusepath{stroke,fill}%
}%
\begin{pgfscope}%
\pgfsys@transformshift{0.588387in}{1.392120in}%
\pgfsys@useobject{currentmarker}{}%
\end{pgfscope}%
\end{pgfscope}%
\begin{pgfscope}%
\pgfsetbuttcap%
\pgfsetroundjoin%
\definecolor{currentfill}{rgb}{0.000000,0.000000,0.000000}%
\pgfsetfillcolor{currentfill}%
\pgfsetlinewidth{0.602250pt}%
\definecolor{currentstroke}{rgb}{0.000000,0.000000,0.000000}%
\pgfsetstrokecolor{currentstroke}%
\pgfsetdash{}{0pt}%
\pgfsys@defobject{currentmarker}{\pgfqpoint{-0.027778in}{0.000000in}}{\pgfqpoint{-0.000000in}{0.000000in}}{%
\pgfpathmoveto{\pgfqpoint{-0.000000in}{0.000000in}}%
\pgfpathlineto{\pgfqpoint{-0.027778in}{0.000000in}}%
\pgfusepath{stroke,fill}%
}%
\begin{pgfscope}%
\pgfsys@transformshift{0.588387in}{1.428224in}%
\pgfsys@useobject{currentmarker}{}%
\end{pgfscope}%
\end{pgfscope}%
\begin{pgfscope}%
\pgfsetbuttcap%
\pgfsetroundjoin%
\definecolor{currentfill}{rgb}{0.000000,0.000000,0.000000}%
\pgfsetfillcolor{currentfill}%
\pgfsetlinewidth{0.602250pt}%
\definecolor{currentstroke}{rgb}{0.000000,0.000000,0.000000}%
\pgfsetstrokecolor{currentstroke}%
\pgfsetdash{}{0pt}%
\pgfsys@defobject{currentmarker}{\pgfqpoint{-0.027778in}{0.000000in}}{\pgfqpoint{-0.000000in}{0.000000in}}{%
\pgfpathmoveto{\pgfqpoint{-0.000000in}{0.000000in}}%
\pgfpathlineto{\pgfqpoint{-0.027778in}{0.000000in}}%
\pgfusepath{stroke,fill}%
}%
\begin{pgfscope}%
\pgfsys@transformshift{0.588387in}{1.458750in}%
\pgfsys@useobject{currentmarker}{}%
\end{pgfscope}%
\end{pgfscope}%
\begin{pgfscope}%
\pgfsetbuttcap%
\pgfsetroundjoin%
\definecolor{currentfill}{rgb}{0.000000,0.000000,0.000000}%
\pgfsetfillcolor{currentfill}%
\pgfsetlinewidth{0.602250pt}%
\definecolor{currentstroke}{rgb}{0.000000,0.000000,0.000000}%
\pgfsetstrokecolor{currentstroke}%
\pgfsetdash{}{0pt}%
\pgfsys@defobject{currentmarker}{\pgfqpoint{-0.027778in}{0.000000in}}{\pgfqpoint{-0.000000in}{0.000000in}}{%
\pgfpathmoveto{\pgfqpoint{-0.000000in}{0.000000in}}%
\pgfpathlineto{\pgfqpoint{-0.027778in}{0.000000in}}%
\pgfusepath{stroke,fill}%
}%
\begin{pgfscope}%
\pgfsys@transformshift{0.588387in}{1.485192in}%
\pgfsys@useobject{currentmarker}{}%
\end{pgfscope}%
\end{pgfscope}%
\begin{pgfscope}%
\pgfsetbuttcap%
\pgfsetroundjoin%
\definecolor{currentfill}{rgb}{0.000000,0.000000,0.000000}%
\pgfsetfillcolor{currentfill}%
\pgfsetlinewidth{0.602250pt}%
\definecolor{currentstroke}{rgb}{0.000000,0.000000,0.000000}%
\pgfsetstrokecolor{currentstroke}%
\pgfsetdash{}{0pt}%
\pgfsys@defobject{currentmarker}{\pgfqpoint{-0.027778in}{0.000000in}}{\pgfqpoint{-0.000000in}{0.000000in}}{%
\pgfpathmoveto{\pgfqpoint{-0.000000in}{0.000000in}}%
\pgfpathlineto{\pgfqpoint{-0.027778in}{0.000000in}}%
\pgfusepath{stroke,fill}%
}%
\begin{pgfscope}%
\pgfsys@transformshift{0.588387in}{1.508516in}%
\pgfsys@useobject{currentmarker}{}%
\end{pgfscope}%
\end{pgfscope}%
\begin{pgfscope}%
\pgfsetbuttcap%
\pgfsetroundjoin%
\definecolor{currentfill}{rgb}{0.000000,0.000000,0.000000}%
\pgfsetfillcolor{currentfill}%
\pgfsetlinewidth{0.602250pt}%
\definecolor{currentstroke}{rgb}{0.000000,0.000000,0.000000}%
\pgfsetstrokecolor{currentstroke}%
\pgfsetdash{}{0pt}%
\pgfsys@defobject{currentmarker}{\pgfqpoint{-0.027778in}{0.000000in}}{\pgfqpoint{-0.000000in}{0.000000in}}{%
\pgfpathmoveto{\pgfqpoint{-0.000000in}{0.000000in}}%
\pgfpathlineto{\pgfqpoint{-0.027778in}{0.000000in}}%
\pgfusepath{stroke,fill}%
}%
\begin{pgfscope}%
\pgfsys@transformshift{0.588387in}{1.666640in}%
\pgfsys@useobject{currentmarker}{}%
\end{pgfscope}%
\end{pgfscope}%
\begin{pgfscope}%
\pgfsetbuttcap%
\pgfsetroundjoin%
\definecolor{currentfill}{rgb}{0.000000,0.000000,0.000000}%
\pgfsetfillcolor{currentfill}%
\pgfsetlinewidth{0.602250pt}%
\definecolor{currentstroke}{rgb}{0.000000,0.000000,0.000000}%
\pgfsetstrokecolor{currentstroke}%
\pgfsetdash{}{0pt}%
\pgfsys@defobject{currentmarker}{\pgfqpoint{-0.027778in}{0.000000in}}{\pgfqpoint{-0.000000in}{0.000000in}}{%
\pgfpathmoveto{\pgfqpoint{-0.000000in}{0.000000in}}%
\pgfpathlineto{\pgfqpoint{-0.027778in}{0.000000in}}%
\pgfusepath{stroke,fill}%
}%
\begin{pgfscope}%
\pgfsys@transformshift{0.588387in}{1.746932in}%
\pgfsys@useobject{currentmarker}{}%
\end{pgfscope}%
\end{pgfscope}%
\begin{pgfscope}%
\pgfsetbuttcap%
\pgfsetroundjoin%
\definecolor{currentfill}{rgb}{0.000000,0.000000,0.000000}%
\pgfsetfillcolor{currentfill}%
\pgfsetlinewidth{0.602250pt}%
\definecolor{currentstroke}{rgb}{0.000000,0.000000,0.000000}%
\pgfsetstrokecolor{currentstroke}%
\pgfsetdash{}{0pt}%
\pgfsys@defobject{currentmarker}{\pgfqpoint{-0.027778in}{0.000000in}}{\pgfqpoint{-0.000000in}{0.000000in}}{%
\pgfpathmoveto{\pgfqpoint{-0.000000in}{0.000000in}}%
\pgfpathlineto{\pgfqpoint{-0.027778in}{0.000000in}}%
\pgfusepath{stroke,fill}%
}%
\begin{pgfscope}%
\pgfsys@transformshift{0.588387in}{1.803900in}%
\pgfsys@useobject{currentmarker}{}%
\end{pgfscope}%
\end{pgfscope}%
\begin{pgfscope}%
\pgfsetbuttcap%
\pgfsetroundjoin%
\definecolor{currentfill}{rgb}{0.000000,0.000000,0.000000}%
\pgfsetfillcolor{currentfill}%
\pgfsetlinewidth{0.602250pt}%
\definecolor{currentstroke}{rgb}{0.000000,0.000000,0.000000}%
\pgfsetstrokecolor{currentstroke}%
\pgfsetdash{}{0pt}%
\pgfsys@defobject{currentmarker}{\pgfqpoint{-0.027778in}{0.000000in}}{\pgfqpoint{-0.000000in}{0.000000in}}{%
\pgfpathmoveto{\pgfqpoint{-0.000000in}{0.000000in}}%
\pgfpathlineto{\pgfqpoint{-0.027778in}{0.000000in}}%
\pgfusepath{stroke,fill}%
}%
\begin{pgfscope}%
\pgfsys@transformshift{0.588387in}{1.848088in}%
\pgfsys@useobject{currentmarker}{}%
\end{pgfscope}%
\end{pgfscope}%
\begin{pgfscope}%
\pgfsetbuttcap%
\pgfsetroundjoin%
\definecolor{currentfill}{rgb}{0.000000,0.000000,0.000000}%
\pgfsetfillcolor{currentfill}%
\pgfsetlinewidth{0.602250pt}%
\definecolor{currentstroke}{rgb}{0.000000,0.000000,0.000000}%
\pgfsetstrokecolor{currentstroke}%
\pgfsetdash{}{0pt}%
\pgfsys@defobject{currentmarker}{\pgfqpoint{-0.027778in}{0.000000in}}{\pgfqpoint{-0.000000in}{0.000000in}}{%
\pgfpathmoveto{\pgfqpoint{-0.000000in}{0.000000in}}%
\pgfpathlineto{\pgfqpoint{-0.027778in}{0.000000in}}%
\pgfusepath{stroke,fill}%
}%
\begin{pgfscope}%
\pgfsys@transformshift{0.588387in}{1.884192in}%
\pgfsys@useobject{currentmarker}{}%
\end{pgfscope}%
\end{pgfscope}%
\begin{pgfscope}%
\pgfsetbuttcap%
\pgfsetroundjoin%
\definecolor{currentfill}{rgb}{0.000000,0.000000,0.000000}%
\pgfsetfillcolor{currentfill}%
\pgfsetlinewidth{0.602250pt}%
\definecolor{currentstroke}{rgb}{0.000000,0.000000,0.000000}%
\pgfsetstrokecolor{currentstroke}%
\pgfsetdash{}{0pt}%
\pgfsys@defobject{currentmarker}{\pgfqpoint{-0.027778in}{0.000000in}}{\pgfqpoint{-0.000000in}{0.000000in}}{%
\pgfpathmoveto{\pgfqpoint{-0.000000in}{0.000000in}}%
\pgfpathlineto{\pgfqpoint{-0.027778in}{0.000000in}}%
\pgfusepath{stroke,fill}%
}%
\begin{pgfscope}%
\pgfsys@transformshift{0.588387in}{1.914717in}%
\pgfsys@useobject{currentmarker}{}%
\end{pgfscope}%
\end{pgfscope}%
\begin{pgfscope}%
\pgfsetbuttcap%
\pgfsetroundjoin%
\definecolor{currentfill}{rgb}{0.000000,0.000000,0.000000}%
\pgfsetfillcolor{currentfill}%
\pgfsetlinewidth{0.602250pt}%
\definecolor{currentstroke}{rgb}{0.000000,0.000000,0.000000}%
\pgfsetstrokecolor{currentstroke}%
\pgfsetdash{}{0pt}%
\pgfsys@defobject{currentmarker}{\pgfqpoint{-0.027778in}{0.000000in}}{\pgfqpoint{-0.000000in}{0.000000in}}{%
\pgfpathmoveto{\pgfqpoint{-0.000000in}{0.000000in}}%
\pgfpathlineto{\pgfqpoint{-0.027778in}{0.000000in}}%
\pgfusepath{stroke,fill}%
}%
\begin{pgfscope}%
\pgfsys@transformshift{0.588387in}{1.941160in}%
\pgfsys@useobject{currentmarker}{}%
\end{pgfscope}%
\end{pgfscope}%
\begin{pgfscope}%
\pgfsetbuttcap%
\pgfsetroundjoin%
\definecolor{currentfill}{rgb}{0.000000,0.000000,0.000000}%
\pgfsetfillcolor{currentfill}%
\pgfsetlinewidth{0.602250pt}%
\definecolor{currentstroke}{rgb}{0.000000,0.000000,0.000000}%
\pgfsetstrokecolor{currentstroke}%
\pgfsetdash{}{0pt}%
\pgfsys@defobject{currentmarker}{\pgfqpoint{-0.027778in}{0.000000in}}{\pgfqpoint{-0.000000in}{0.000000in}}{%
\pgfpathmoveto{\pgfqpoint{-0.000000in}{0.000000in}}%
\pgfpathlineto{\pgfqpoint{-0.027778in}{0.000000in}}%
\pgfusepath{stroke,fill}%
}%
\begin{pgfscope}%
\pgfsys@transformshift{0.588387in}{1.964483in}%
\pgfsys@useobject{currentmarker}{}%
\end{pgfscope}%
\end{pgfscope}%
\begin{pgfscope}%
\pgfsetbuttcap%
\pgfsetroundjoin%
\definecolor{currentfill}{rgb}{0.000000,0.000000,0.000000}%
\pgfsetfillcolor{currentfill}%
\pgfsetlinewidth{0.602250pt}%
\definecolor{currentstroke}{rgb}{0.000000,0.000000,0.000000}%
\pgfsetstrokecolor{currentstroke}%
\pgfsetdash{}{0pt}%
\pgfsys@defobject{currentmarker}{\pgfqpoint{-0.027778in}{0.000000in}}{\pgfqpoint{-0.000000in}{0.000000in}}{%
\pgfpathmoveto{\pgfqpoint{-0.000000in}{0.000000in}}%
\pgfpathlineto{\pgfqpoint{-0.027778in}{0.000000in}}%
\pgfusepath{stroke,fill}%
}%
\begin{pgfscope}%
\pgfsys@transformshift{0.588387in}{2.122607in}%
\pgfsys@useobject{currentmarker}{}%
\end{pgfscope}%
\end{pgfscope}%
\begin{pgfscope}%
\pgfsetbuttcap%
\pgfsetroundjoin%
\definecolor{currentfill}{rgb}{0.000000,0.000000,0.000000}%
\pgfsetfillcolor{currentfill}%
\pgfsetlinewidth{0.602250pt}%
\definecolor{currentstroke}{rgb}{0.000000,0.000000,0.000000}%
\pgfsetstrokecolor{currentstroke}%
\pgfsetdash{}{0pt}%
\pgfsys@defobject{currentmarker}{\pgfqpoint{-0.027778in}{0.000000in}}{\pgfqpoint{-0.000000in}{0.000000in}}{%
\pgfpathmoveto{\pgfqpoint{-0.000000in}{0.000000in}}%
\pgfpathlineto{\pgfqpoint{-0.027778in}{0.000000in}}%
\pgfusepath{stroke,fill}%
}%
\begin{pgfscope}%
\pgfsys@transformshift{0.588387in}{2.202899in}%
\pgfsys@useobject{currentmarker}{}%
\end{pgfscope}%
\end{pgfscope}%
\begin{pgfscope}%
\pgfsetbuttcap%
\pgfsetroundjoin%
\definecolor{currentfill}{rgb}{0.000000,0.000000,0.000000}%
\pgfsetfillcolor{currentfill}%
\pgfsetlinewidth{0.602250pt}%
\definecolor{currentstroke}{rgb}{0.000000,0.000000,0.000000}%
\pgfsetstrokecolor{currentstroke}%
\pgfsetdash{}{0pt}%
\pgfsys@defobject{currentmarker}{\pgfqpoint{-0.027778in}{0.000000in}}{\pgfqpoint{-0.000000in}{0.000000in}}{%
\pgfpathmoveto{\pgfqpoint{-0.000000in}{0.000000in}}%
\pgfpathlineto{\pgfqpoint{-0.027778in}{0.000000in}}%
\pgfusepath{stroke,fill}%
}%
\begin{pgfscope}%
\pgfsys@transformshift{0.588387in}{2.259867in}%
\pgfsys@useobject{currentmarker}{}%
\end{pgfscope}%
\end{pgfscope}%
\begin{pgfscope}%
\pgfsetbuttcap%
\pgfsetroundjoin%
\definecolor{currentfill}{rgb}{0.000000,0.000000,0.000000}%
\pgfsetfillcolor{currentfill}%
\pgfsetlinewidth{0.602250pt}%
\definecolor{currentstroke}{rgb}{0.000000,0.000000,0.000000}%
\pgfsetstrokecolor{currentstroke}%
\pgfsetdash{}{0pt}%
\pgfsys@defobject{currentmarker}{\pgfqpoint{-0.027778in}{0.000000in}}{\pgfqpoint{-0.000000in}{0.000000in}}{%
\pgfpathmoveto{\pgfqpoint{-0.000000in}{0.000000in}}%
\pgfpathlineto{\pgfqpoint{-0.027778in}{0.000000in}}%
\pgfusepath{stroke,fill}%
}%
\begin{pgfscope}%
\pgfsys@transformshift{0.588387in}{2.304055in}%
\pgfsys@useobject{currentmarker}{}%
\end{pgfscope}%
\end{pgfscope}%
\begin{pgfscope}%
\pgfsetbuttcap%
\pgfsetroundjoin%
\definecolor{currentfill}{rgb}{0.000000,0.000000,0.000000}%
\pgfsetfillcolor{currentfill}%
\pgfsetlinewidth{0.602250pt}%
\definecolor{currentstroke}{rgb}{0.000000,0.000000,0.000000}%
\pgfsetstrokecolor{currentstroke}%
\pgfsetdash{}{0pt}%
\pgfsys@defobject{currentmarker}{\pgfqpoint{-0.027778in}{0.000000in}}{\pgfqpoint{-0.000000in}{0.000000in}}{%
\pgfpathmoveto{\pgfqpoint{-0.000000in}{0.000000in}}%
\pgfpathlineto{\pgfqpoint{-0.027778in}{0.000000in}}%
\pgfusepath{stroke,fill}%
}%
\begin{pgfscope}%
\pgfsys@transformshift{0.588387in}{2.340159in}%
\pgfsys@useobject{currentmarker}{}%
\end{pgfscope}%
\end{pgfscope}%
\begin{pgfscope}%
\pgfsetbuttcap%
\pgfsetroundjoin%
\definecolor{currentfill}{rgb}{0.000000,0.000000,0.000000}%
\pgfsetfillcolor{currentfill}%
\pgfsetlinewidth{0.602250pt}%
\definecolor{currentstroke}{rgb}{0.000000,0.000000,0.000000}%
\pgfsetstrokecolor{currentstroke}%
\pgfsetdash{}{0pt}%
\pgfsys@defobject{currentmarker}{\pgfqpoint{-0.027778in}{0.000000in}}{\pgfqpoint{-0.000000in}{0.000000in}}{%
\pgfpathmoveto{\pgfqpoint{-0.000000in}{0.000000in}}%
\pgfpathlineto{\pgfqpoint{-0.027778in}{0.000000in}}%
\pgfusepath{stroke,fill}%
}%
\begin{pgfscope}%
\pgfsys@transformshift{0.588387in}{2.370685in}%
\pgfsys@useobject{currentmarker}{}%
\end{pgfscope}%
\end{pgfscope}%
\begin{pgfscope}%
\pgfsetbuttcap%
\pgfsetroundjoin%
\definecolor{currentfill}{rgb}{0.000000,0.000000,0.000000}%
\pgfsetfillcolor{currentfill}%
\pgfsetlinewidth{0.602250pt}%
\definecolor{currentstroke}{rgb}{0.000000,0.000000,0.000000}%
\pgfsetstrokecolor{currentstroke}%
\pgfsetdash{}{0pt}%
\pgfsys@defobject{currentmarker}{\pgfqpoint{-0.027778in}{0.000000in}}{\pgfqpoint{-0.000000in}{0.000000in}}{%
\pgfpathmoveto{\pgfqpoint{-0.000000in}{0.000000in}}%
\pgfpathlineto{\pgfqpoint{-0.027778in}{0.000000in}}%
\pgfusepath{stroke,fill}%
}%
\begin{pgfscope}%
\pgfsys@transformshift{0.588387in}{2.397127in}%
\pgfsys@useobject{currentmarker}{}%
\end{pgfscope}%
\end{pgfscope}%
\begin{pgfscope}%
\pgfsetbuttcap%
\pgfsetroundjoin%
\definecolor{currentfill}{rgb}{0.000000,0.000000,0.000000}%
\pgfsetfillcolor{currentfill}%
\pgfsetlinewidth{0.602250pt}%
\definecolor{currentstroke}{rgb}{0.000000,0.000000,0.000000}%
\pgfsetstrokecolor{currentstroke}%
\pgfsetdash{}{0pt}%
\pgfsys@defobject{currentmarker}{\pgfqpoint{-0.027778in}{0.000000in}}{\pgfqpoint{-0.000000in}{0.000000in}}{%
\pgfpathmoveto{\pgfqpoint{-0.000000in}{0.000000in}}%
\pgfpathlineto{\pgfqpoint{-0.027778in}{0.000000in}}%
\pgfusepath{stroke,fill}%
}%
\begin{pgfscope}%
\pgfsys@transformshift{0.588387in}{2.420451in}%
\pgfsys@useobject{currentmarker}{}%
\end{pgfscope}%
\end{pgfscope}%
\begin{pgfscope}%
\definecolor{textcolor}{rgb}{0.000000,0.000000,0.000000}%
\pgfsetstrokecolor{textcolor}%
\pgfsetfillcolor{textcolor}%
\pgftext[x=0.234413in,y=1.526746in,,bottom,rotate=90.000000]{\color{textcolor}{\rmfamily\fontsize{10.000000}{12.000000}\selectfont\catcode`\^=\active\def^{\ifmmode\sp\else\^{}\fi}\catcode`\%=\active\def%{\%}Checks [call]}}%
\end{pgfscope}%
\begin{pgfscope}%
\pgfpathrectangle{\pgfqpoint{0.588387in}{0.521603in}}{\pgfqpoint{5.314715in}{2.010285in}}%
\pgfusepath{clip}%
\pgfsetrectcap%
\pgfsetroundjoin%
\pgfsetlinewidth{1.505625pt}%
\pgfsetstrokecolor{currentstroke1}%
\pgfsetdash{}{0pt}%
\pgfpathmoveto{\pgfqpoint{0.829965in}{0.619261in}}%
\pgfpathlineto{\pgfqpoint{0.928568in}{0.728924in}}%
\pgfpathlineto{\pgfqpoint{1.027172in}{0.859337in}}%
\pgfpathlineto{\pgfqpoint{1.125775in}{0.918199in}}%
\pgfpathlineto{\pgfqpoint{1.224378in}{0.998866in}}%
\pgfpathlineto{\pgfqpoint{1.322981in}{1.026067in}}%
\pgfpathlineto{\pgfqpoint{1.421585in}{1.073639in}}%
\pgfpathlineto{\pgfqpoint{1.520188in}{1.104767in}}%
\pgfpathlineto{\pgfqpoint{1.618791in}{1.122282in}}%
\pgfpathlineto{\pgfqpoint{1.717394in}{1.140219in}}%
\pgfpathlineto{\pgfqpoint{1.815998in}{1.167906in}}%
\pgfpathlineto{\pgfqpoint{1.914601in}{1.176211in}}%
\pgfpathlineto{\pgfqpoint{2.013204in}{1.189322in}}%
\pgfpathlineto{\pgfqpoint{2.111807in}{1.207145in}}%
\pgfpathlineto{\pgfqpoint{2.210411in}{1.214503in}}%
\pgfpathlineto{\pgfqpoint{2.309014in}{1.220750in}}%
\pgfpathlineto{\pgfqpoint{2.407617in}{1.239676in}}%
\pgfpathlineto{\pgfqpoint{2.506220in}{1.245184in}}%
\pgfpathlineto{\pgfqpoint{2.604824in}{1.253273in}}%
\pgfpathlineto{\pgfqpoint{2.703427in}{1.259430in}}%
\pgfpathlineto{\pgfqpoint{2.802030in}{1.287048in}}%
\pgfpathlineto{\pgfqpoint{2.900633in}{1.276072in}}%
\pgfpathlineto{\pgfqpoint{2.999237in}{1.284064in}}%
\pgfpathlineto{\pgfqpoint{3.097840in}{1.299068in}}%
\pgfpathlineto{\pgfqpoint{3.196443in}{1.301347in}}%
\pgfpathlineto{\pgfqpoint{3.295046in}{1.316929in}}%
\pgfpathlineto{\pgfqpoint{3.393649in}{1.315886in}}%
\pgfpathlineto{\pgfqpoint{3.492253in}{1.323325in}}%
\pgfpathlineto{\pgfqpoint{3.590856in}{1.339857in}}%
\pgfpathlineto{\pgfqpoint{3.689459in}{1.330599in}}%
\pgfpathlineto{\pgfqpoint{3.788062in}{1.340235in}}%
\pgfpathlineto{\pgfqpoint{3.886666in}{1.347126in}}%
\pgfpathlineto{\pgfqpoint{3.985269in}{1.346691in}}%
\pgfpathlineto{\pgfqpoint{4.083872in}{1.360606in}}%
\pgfpathlineto{\pgfqpoint{4.182475in}{1.361973in}}%
\pgfpathlineto{\pgfqpoint{4.281079in}{1.365423in}}%
\pgfpathlineto{\pgfqpoint{4.379682in}{1.367185in}}%
\pgfpathlineto{\pgfqpoint{4.478285in}{1.379920in}}%
\pgfpathlineto{\pgfqpoint{4.576888in}{1.382957in}}%
\pgfpathlineto{\pgfqpoint{4.675492in}{1.383320in}}%
\pgfpathlineto{\pgfqpoint{4.774095in}{1.396260in}}%
\pgfpathlineto{\pgfqpoint{4.872698in}{1.383339in}}%
\pgfpathlineto{\pgfqpoint{4.971301in}{1.393784in}}%
\pgfpathlineto{\pgfqpoint{5.069905in}{1.407612in}}%
\pgfpathlineto{\pgfqpoint{5.168508in}{1.403343in}}%
\pgfpathlineto{\pgfqpoint{5.267111in}{1.405055in}}%
\pgfpathlineto{\pgfqpoint{5.365714in}{1.414761in}}%
\pgfpathlineto{\pgfqpoint{5.464318in}{1.418165in}}%
\pgfpathlineto{\pgfqpoint{5.562921in}{1.426203in}}%
\pgfpathlineto{\pgfqpoint{5.661524in}{1.421490in}}%
\pgfusepath{stroke}%
\end{pgfscope}%
\begin{pgfscope}%
\pgfpathrectangle{\pgfqpoint{0.588387in}{0.521603in}}{\pgfqpoint{5.314715in}{2.010285in}}%
\pgfusepath{clip}%
\pgfsetrectcap%
\pgfsetroundjoin%
\pgfsetlinewidth{1.505625pt}%
\pgfsetstrokecolor{currentstroke2}%
\pgfsetdash{}{0pt}%
\pgfpathmoveto{\pgfqpoint{0.829965in}{0.612980in}}%
\pgfpathlineto{\pgfqpoint{0.928568in}{0.702397in}}%
\pgfpathlineto{\pgfqpoint{1.027172in}{0.828491in}}%
\pgfpathlineto{\pgfqpoint{1.125775in}{0.929221in}}%
\pgfpathlineto{\pgfqpoint{1.224378in}{1.052469in}}%
\pgfpathlineto{\pgfqpoint{1.322981in}{1.098278in}}%
\pgfpathlineto{\pgfqpoint{1.421585in}{1.219878in}}%
\pgfpathlineto{\pgfqpoint{1.520188in}{1.419414in}}%
\pgfpathlineto{\pgfqpoint{1.618791in}{1.405465in}}%
\pgfpathlineto{\pgfqpoint{1.717394in}{1.617512in}}%
\pgfpathlineto{\pgfqpoint{1.815998in}{1.698072in}}%
\pgfpathlineto{\pgfqpoint{1.914601in}{1.760965in}}%
\pgfpathlineto{\pgfqpoint{2.013204in}{1.741550in}}%
\pgfpathlineto{\pgfqpoint{2.111807in}{1.778650in}}%
\pgfpathlineto{\pgfqpoint{2.210411in}{1.825986in}}%
\pgfpathlineto{\pgfqpoint{2.309014in}{1.802117in}}%
\pgfpathlineto{\pgfqpoint{2.407617in}{1.851393in}}%
\pgfpathlineto{\pgfqpoint{2.506220in}{1.798896in}}%
\pgfpathlineto{\pgfqpoint{2.604824in}{1.893284in}}%
\pgfpathlineto{\pgfqpoint{2.703427in}{1.931204in}}%
\pgfpathlineto{\pgfqpoint{2.802030in}{1.953077in}}%
\pgfpathlineto{\pgfqpoint{2.900633in}{2.044922in}}%
\pgfpathlineto{\pgfqpoint{2.999237in}{1.952866in}}%
\pgfusepath{stroke}%
\end{pgfscope}%
\begin{pgfscope}%
\pgfpathrectangle{\pgfqpoint{0.588387in}{0.521603in}}{\pgfqpoint{5.314715in}{2.010285in}}%
\pgfusepath{clip}%
\pgfsetrectcap%
\pgfsetroundjoin%
\pgfsetlinewidth{1.505625pt}%
\pgfsetstrokecolor{currentstroke3}%
\pgfsetdash{}{0pt}%
\pgfpathmoveto{\pgfqpoint{0.829965in}{0.613812in}}%
\pgfpathlineto{\pgfqpoint{0.928568in}{0.748618in}}%
\pgfpathlineto{\pgfqpoint{1.027172in}{0.916622in}}%
\pgfpathlineto{\pgfqpoint{1.125775in}{1.111988in}}%
\pgfpathlineto{\pgfqpoint{1.224378in}{1.392439in}}%
\pgfpathlineto{\pgfqpoint{1.322981in}{1.478612in}}%
\pgfpathlineto{\pgfqpoint{1.421585in}{1.651438in}}%
\pgfpathlineto{\pgfqpoint{1.520188in}{1.795454in}}%
\pgfpathlineto{\pgfqpoint{1.618791in}{1.985003in}}%
\pgfpathlineto{\pgfqpoint{1.717394in}{2.080963in}}%
\pgfpathlineto{\pgfqpoint{1.815998in}{2.168477in}}%
\pgfpathlineto{\pgfqpoint{1.914601in}{2.250575in}}%
\pgfpathlineto{\pgfqpoint{2.013204in}{2.243995in}}%
\pgfpathlineto{\pgfqpoint{2.111807in}{2.217028in}}%
\pgfpathlineto{\pgfqpoint{2.210411in}{2.206793in}}%
\pgfpathlineto{\pgfqpoint{2.309014in}{2.136006in}}%
\pgfpathlineto{\pgfqpoint{2.407617in}{2.163733in}}%
\pgfpathlineto{\pgfqpoint{2.506220in}{2.440512in}}%
\pgfpathlineto{\pgfqpoint{2.604824in}{2.150529in}}%
\pgfpathlineto{\pgfqpoint{2.802030in}{1.955515in}}%
\pgfusepath{stroke}%
\end{pgfscope}%
\begin{pgfscope}%
\pgfpathrectangle{\pgfqpoint{0.588387in}{0.521603in}}{\pgfqpoint{5.314715in}{2.010285in}}%
\pgfusepath{clip}%
\pgfsetrectcap%
\pgfsetroundjoin%
\pgfsetlinewidth{1.505625pt}%
\pgfsetstrokecolor{currentstroke4}%
\pgfsetdash{}{0pt}%
\pgfpathmoveto{\pgfqpoint{0.829965in}{0.629159in}}%
\pgfpathlineto{\pgfqpoint{0.928568in}{0.760198in}}%
\pgfpathlineto{\pgfqpoint{1.027172in}{0.896288in}}%
\pgfpathlineto{\pgfqpoint{1.125775in}{1.009047in}}%
\pgfpathlineto{\pgfqpoint{1.224378in}{1.142546in}}%
\pgfpathlineto{\pgfqpoint{1.322981in}{1.154322in}}%
\pgfpathlineto{\pgfqpoint{1.421585in}{1.333977in}}%
\pgfpathlineto{\pgfqpoint{1.520188in}{1.348843in}}%
\pgfpathlineto{\pgfqpoint{1.618791in}{1.461491in}}%
\pgfpathlineto{\pgfqpoint{1.717394in}{1.732142in}}%
\pgfpathlineto{\pgfqpoint{1.815998in}{1.609311in}}%
\pgfpathlineto{\pgfqpoint{1.914601in}{1.832239in}}%
\pgfpathlineto{\pgfqpoint{2.013204in}{1.779231in}}%
\pgfpathlineto{\pgfqpoint{2.111807in}{1.866823in}}%
\pgfpathlineto{\pgfqpoint{2.210411in}{1.544283in}}%
\pgfpathlineto{\pgfqpoint{2.309014in}{1.686554in}}%
\pgfpathlineto{\pgfqpoint{2.407617in}{1.871254in}}%
\pgfpathlineto{\pgfqpoint{2.506220in}{1.755189in}}%
\pgfpathlineto{\pgfqpoint{2.604824in}{1.684089in}}%
\pgfpathlineto{\pgfqpoint{2.703427in}{1.697330in}}%
\pgfpathlineto{\pgfqpoint{2.802030in}{1.770161in}}%
\pgfpathlineto{\pgfqpoint{2.900633in}{1.765294in}}%
\pgfpathlineto{\pgfqpoint{2.999237in}{1.784068in}}%
\pgfpathlineto{\pgfqpoint{3.097840in}{1.772148in}}%
\pgfpathlineto{\pgfqpoint{3.196443in}{1.758681in}}%
\pgfpathlineto{\pgfqpoint{3.295046in}{1.514994in}}%
\pgfpathlineto{\pgfqpoint{3.393649in}{1.663394in}}%
\pgfpathlineto{\pgfqpoint{3.492253in}{1.653838in}}%
\pgfpathlineto{\pgfqpoint{3.590856in}{1.511186in}}%
\pgfpathlineto{\pgfqpoint{3.689459in}{1.591162in}}%
\pgfpathlineto{\pgfqpoint{3.788062in}{1.653733in}}%
\pgfpathlineto{\pgfqpoint{3.886666in}{1.766809in}}%
\pgfpathlineto{\pgfqpoint{3.985269in}{1.482735in}}%
\pgfpathlineto{\pgfqpoint{4.083872in}{1.574385in}}%
\pgfpathlineto{\pgfqpoint{4.182475in}{1.589731in}}%
\pgfpathlineto{\pgfqpoint{4.281079in}{1.661478in}}%
\pgfpathlineto{\pgfqpoint{4.379682in}{1.633424in}}%
\pgfpathlineto{\pgfqpoint{4.478285in}{1.582188in}}%
\pgfpathlineto{\pgfqpoint{4.576888in}{1.687255in}}%
\pgfpathlineto{\pgfqpoint{4.675492in}{1.597032in}}%
\pgfpathlineto{\pgfqpoint{4.774095in}{1.678339in}}%
\pgfpathlineto{\pgfqpoint{4.872698in}{1.632820in}}%
\pgfpathlineto{\pgfqpoint{4.971301in}{1.707334in}}%
\pgfpathlineto{\pgfqpoint{5.069905in}{1.871999in}}%
\pgfpathlineto{\pgfqpoint{5.168508in}{2.001876in}}%
\pgfpathlineto{\pgfqpoint{5.267111in}{1.618578in}}%
\pgfpathlineto{\pgfqpoint{5.365714in}{1.655909in}}%
\pgfpathlineto{\pgfqpoint{5.464318in}{1.644451in}}%
\pgfpathlineto{\pgfqpoint{5.562921in}{1.576659in}}%
\pgfpathlineto{\pgfqpoint{5.661524in}{1.691175in}}%
\pgfusepath{stroke}%
\end{pgfscope}%
\begin{pgfscope}%
\pgfsetrectcap%
\pgfsetmiterjoin%
\pgfsetlinewidth{0.803000pt}%
\definecolor{currentstroke}{rgb}{0.000000,0.000000,0.000000}%
\pgfsetstrokecolor{currentstroke}%
\pgfsetdash{}{0pt}%
\pgfpathmoveto{\pgfqpoint{0.588387in}{0.521603in}}%
\pgfpathlineto{\pgfqpoint{0.588387in}{2.531888in}}%
\pgfusepath{stroke}%
\end{pgfscope}%
\begin{pgfscope}%
\pgfsetrectcap%
\pgfsetmiterjoin%
\pgfsetlinewidth{0.803000pt}%
\definecolor{currentstroke}{rgb}{0.000000,0.000000,0.000000}%
\pgfsetstrokecolor{currentstroke}%
\pgfsetdash{}{0pt}%
\pgfpathmoveto{\pgfqpoint{5.903102in}{0.521603in}}%
\pgfpathlineto{\pgfqpoint{5.903102in}{2.531888in}}%
\pgfusepath{stroke}%
\end{pgfscope}%
\begin{pgfscope}%
\pgfsetrectcap%
\pgfsetmiterjoin%
\pgfsetlinewidth{0.803000pt}%
\definecolor{currentstroke}{rgb}{0.000000,0.000000,0.000000}%
\pgfsetstrokecolor{currentstroke}%
\pgfsetdash{}{0pt}%
\pgfpathmoveto{\pgfqpoint{0.588387in}{0.521603in}}%
\pgfpathlineto{\pgfqpoint{5.903102in}{0.521603in}}%
\pgfusepath{stroke}%
\end{pgfscope}%
\begin{pgfscope}%
\pgfsetrectcap%
\pgfsetmiterjoin%
\pgfsetlinewidth{0.803000pt}%
\definecolor{currentstroke}{rgb}{0.000000,0.000000,0.000000}%
\pgfsetstrokecolor{currentstroke}%
\pgfsetdash{}{0pt}%
\pgfpathmoveto{\pgfqpoint{0.588387in}{2.531888in}}%
\pgfpathlineto{\pgfqpoint{5.903102in}{2.531888in}}%
\pgfusepath{stroke}%
\end{pgfscope}%
\begin{pgfscope}%
\definecolor{textcolor}{rgb}{0.000000,0.000000,0.000000}%
\pgfsetstrokecolor{textcolor}%
\pgfsetfillcolor{textcolor}%
\pgftext[x=3.245745in,y=2.615222in,,base]{\color{textcolor}{\rmfamily\fontsize{12.000000}{14.400000}\selectfont\catcode`\^=\active\def^{\ifmmode\sp\else\^{}\fi}\catcode`\%=\active\def%{\%}Mean}}%
\end{pgfscope}%
\begin{pgfscope}%
\pgfsetbuttcap%
\pgfsetmiterjoin%
\definecolor{currentfill}{rgb}{1.000000,1.000000,1.000000}%
\pgfsetfillcolor{currentfill}%
\pgfsetfillopacity{0.800000}%
\pgfsetlinewidth{1.003750pt}%
\definecolor{currentstroke}{rgb}{0.800000,0.800000,0.800000}%
\pgfsetstrokecolor{currentstroke}%
\pgfsetstrokeopacity{0.800000}%
\pgfsetdash{}{0pt}%
\pgfpathmoveto{\pgfqpoint{5.990602in}{1.691044in}}%
\pgfpathlineto{\pgfqpoint{8.259376in}{1.691044in}}%
\pgfpathquadraticcurveto{\pgfqpoint{8.284376in}{1.691044in}}{\pgfqpoint{8.284376in}{1.716044in}}%
\pgfpathlineto{\pgfqpoint{8.284376in}{2.444388in}}%
\pgfpathquadraticcurveto{\pgfqpoint{8.284376in}{2.469388in}}{\pgfqpoint{8.259376in}{2.469388in}}%
\pgfpathlineto{\pgfqpoint{5.990602in}{2.469388in}}%
\pgfpathquadraticcurveto{\pgfqpoint{5.965602in}{2.469388in}}{\pgfqpoint{5.965602in}{2.444388in}}%
\pgfpathlineto{\pgfqpoint{5.965602in}{1.716044in}}%
\pgfpathquadraticcurveto{\pgfqpoint{5.965602in}{1.691044in}}{\pgfqpoint{5.990602in}{1.691044in}}%
\pgfpathlineto{\pgfqpoint{5.990602in}{1.691044in}}%
\pgfpathclose%
\pgfusepath{stroke,fill}%
\end{pgfscope}%
\begin{pgfscope}%
\pgfsetrectcap%
\pgfsetroundjoin%
\pgfsetlinewidth{1.505625pt}%
\pgfsetstrokecolor{currentstroke1}%
\pgfsetdash{}{0pt}%
\pgfpathmoveto{\pgfqpoint{6.015602in}{2.368168in}}%
\pgfpathlineto{\pgfqpoint{6.140602in}{2.368168in}}%
\pgfpathlineto{\pgfqpoint{6.265602in}{2.368168in}}%
\pgfusepath{stroke}%
\end{pgfscope}%
\begin{pgfscope}%
\definecolor{textcolor}{rgb}{0.000000,0.000000,0.000000}%
\pgfsetstrokecolor{textcolor}%
\pgfsetfillcolor{textcolor}%
\pgftext[x=6.365602in,y=2.324418in,left,base]{\color{textcolor}{\rmfamily\fontsize{9.000000}{10.800000}\selectfont\catcode`\^=\active\def^{\ifmmode\sp\else\^{}\fi}\catcode`\%=\active\def%{\%}\Neighbors{} \& \MergeLinear{}}}%
\end{pgfscope}%
\begin{pgfscope}%
\pgfsetrectcap%
\pgfsetroundjoin%
\pgfsetlinewidth{1.505625pt}%
\pgfsetstrokecolor{currentstroke2}%
\pgfsetdash{}{0pt}%
\pgfpathmoveto{\pgfqpoint{6.015602in}{2.184696in}}%
\pgfpathlineto{\pgfqpoint{6.140602in}{2.184696in}}%
\pgfpathlineto{\pgfqpoint{6.265602in}{2.184696in}}%
\pgfusepath{stroke}%
\end{pgfscope}%
\begin{pgfscope}%
\definecolor{textcolor}{rgb}{0.000000,0.000000,0.000000}%
\pgfsetstrokecolor{textcolor}%
\pgfsetfillcolor{textcolor}%
\pgftext[x=6.365602in,y=2.140946in,left,base]{\color{textcolor}{\rmfamily\fontsize{9.000000}{10.800000}\selectfont\catcode`\^=\active\def^{\ifmmode\sp\else\^{}\fi}\catcode`\%=\active\def%{\%}\Neighbors{} \& \Log{}}}%
\end{pgfscope}%
\begin{pgfscope}%
\pgfsetrectcap%
\pgfsetroundjoin%
\pgfsetlinewidth{1.505625pt}%
\pgfsetstrokecolor{currentstroke3}%
\pgfsetdash{}{0pt}%
\pgfpathmoveto{\pgfqpoint{6.015602in}{2.001225in}}%
\pgfpathlineto{\pgfqpoint{6.140602in}{2.001225in}}%
\pgfpathlineto{\pgfqpoint{6.265602in}{2.001225in}}%
\pgfusepath{stroke}%
\end{pgfscope}%
\begin{pgfscope}%
\definecolor{textcolor}{rgb}{0.000000,0.000000,0.000000}%
\pgfsetstrokecolor{textcolor}%
\pgfsetfillcolor{textcolor}%
\pgftext[x=6.365602in,y=1.957475in,left,base]{\color{textcolor}{\rmfamily\fontsize{9.000000}{10.800000}\selectfont\catcode`\^=\active\def^{\ifmmode\sp\else\^{}\fi}\catcode`\%=\active\def%{\%}\Neighbors{} \& \PromisingCycles{}}}%
\end{pgfscope}%
\begin{pgfscope}%
\pgfsetrectcap%
\pgfsetroundjoin%
\pgfsetlinewidth{1.505625pt}%
\pgfsetstrokecolor{currentstroke4}%
\pgfsetdash{}{0pt}%
\pgfpathmoveto{\pgfqpoint{6.015602in}{1.814274in}}%
\pgfpathlineto{\pgfqpoint{6.140602in}{1.814274in}}%
\pgfpathlineto{\pgfqpoint{6.265602in}{1.814274in}}%
\pgfusepath{stroke}%
\end{pgfscope}%
\begin{pgfscope}%
\definecolor{textcolor}{rgb}{0.000000,0.000000,0.000000}%
\pgfsetstrokecolor{textcolor}%
\pgfsetfillcolor{textcolor}%
\pgftext[x=6.365602in,y=1.770524in,left,base]{\color{textcolor}{\rmfamily\fontsize{9.000000}{10.800000}\selectfont\catcode`\^=\active\def^{\ifmmode\sp\else\^{}\fi}\catcode`\%=\active\def%{\%}\Neighbors{} \& \SharedVertices{}}}%
\end{pgfscope}%
\end{pgfpicture}%
\makeatother%
\endgroup%
}
	\caption[Mean runtime for minimally rigid graphs (some).]{
		Mean running time (ms) to find all NAC-colorings for minimally rigid graphs with failing merging strategies.}%
	\label{fig:graph_mimimally_rigid_failing_merging_first_runtime}
\end{figure}
\begin{figure}[p]
	\centering
	\scalebox{0.5}{%% Creator: Matplotlib, PGF backend
%%
%% To include the figure in your LaTeX document, write
%%   \input{<filename>.pgf}
%%
%% Make sure the required packages are loaded in your preamble
%%   \usepackage{pgf}
%%
%% Also ensure that all the required font packages are loaded; for instance,
%% the lmodern package is sometimes necessary when using math font.
%%   \usepackage{lmodern}
%%
%% Figures using additional raster images can only be included by \input if
%% they are in the same directory as the main LaTeX file. For loading figures
%% from other directories you can use the `import` package
%%   \usepackage{import}
%%
%% and then include the figures with
%%   \import{<path to file>}{<filename>.pgf}
%%
%% Matplotlib used the following preamble
%%   \def\mathdefault#1{#1}
%%   \everymath=\expandafter{\the\everymath\displaystyle}
%%   \IfFileExists{scrextend.sty}{
%%     \usepackage[fontsize=10.000000pt]{scrextend}
%%   }{
%%     \renewcommand{\normalsize}{\fontsize{10.000000}{12.000000}\selectfont}
%%     \normalsize
%%   }
%%   
%%   \ifdefined\pdftexversion\else  % non-pdftex case.
%%     \usepackage{fontspec}
%%     \setmainfont{DejaVuSans.ttf}[Path=\detokenize{/home/petr/Projects/PyRigi/.venv/lib/python3.12/site-packages/matplotlib/mpl-data/fonts/ttf/}]
%%     \setsansfont{DejaVuSans.ttf}[Path=\detokenize{/home/petr/Projects/PyRigi/.venv/lib/python3.12/site-packages/matplotlib/mpl-data/fonts/ttf/}]
%%     \setmonofont{DejaVuSansMono.ttf}[Path=\detokenize{/home/petr/Projects/PyRigi/.venv/lib/python3.12/site-packages/matplotlib/mpl-data/fonts/ttf/}]
%%   \fi
%%   \makeatletter\@ifpackageloaded{underscore}{}{\usepackage[strings]{underscore}}\makeatother
%%
\begingroup%
\makeatletter%
\begin{pgfpicture}%
\pgfpathrectangle{\pgfpointorigin}{\pgfqpoint{8.384376in}{2.841849in}}%
\pgfusepath{use as bounding box, clip}%
\begin{pgfscope}%
\pgfsetbuttcap%
\pgfsetmiterjoin%
\definecolor{currentfill}{rgb}{1.000000,1.000000,1.000000}%
\pgfsetfillcolor{currentfill}%
\pgfsetlinewidth{0.000000pt}%
\definecolor{currentstroke}{rgb}{1.000000,1.000000,1.000000}%
\pgfsetstrokecolor{currentstroke}%
\pgfsetdash{}{0pt}%
\pgfpathmoveto{\pgfqpoint{0.000000in}{0.000000in}}%
\pgfpathlineto{\pgfqpoint{8.384376in}{0.000000in}}%
\pgfpathlineto{\pgfqpoint{8.384376in}{2.841849in}}%
\pgfpathlineto{\pgfqpoint{0.000000in}{2.841849in}}%
\pgfpathlineto{\pgfqpoint{0.000000in}{0.000000in}}%
\pgfpathclose%
\pgfusepath{fill}%
\end{pgfscope}%
\begin{pgfscope}%
\pgfsetbuttcap%
\pgfsetmiterjoin%
\definecolor{currentfill}{rgb}{1.000000,1.000000,1.000000}%
\pgfsetfillcolor{currentfill}%
\pgfsetlinewidth{0.000000pt}%
\definecolor{currentstroke}{rgb}{0.000000,0.000000,0.000000}%
\pgfsetstrokecolor{currentstroke}%
\pgfsetstrokeopacity{0.000000}%
\pgfsetdash{}{0pt}%
\pgfpathmoveto{\pgfqpoint{0.588387in}{0.521603in}}%
\pgfpathlineto{\pgfqpoint{6.119045in}{0.521603in}}%
\pgfpathlineto{\pgfqpoint{6.119045in}{2.531888in}}%
\pgfpathlineto{\pgfqpoint{0.588387in}{2.531888in}}%
\pgfpathlineto{\pgfqpoint{0.588387in}{0.521603in}}%
\pgfpathclose%
\pgfusepath{fill}%
\end{pgfscope}%
\begin{pgfscope}%
\pgfsetbuttcap%
\pgfsetroundjoin%
\definecolor{currentfill}{rgb}{0.000000,0.000000,0.000000}%
\pgfsetfillcolor{currentfill}%
\pgfsetlinewidth{0.803000pt}%
\definecolor{currentstroke}{rgb}{0.000000,0.000000,0.000000}%
\pgfsetstrokecolor{currentstroke}%
\pgfsetdash{}{0pt}%
\pgfsys@defobject{currentmarker}{\pgfqpoint{0.000000in}{-0.048611in}}{\pgfqpoint{0.000000in}{0.000000in}}{%
\pgfpathmoveto{\pgfqpoint{0.000000in}{0.000000in}}%
\pgfpathlineto{\pgfqpoint{0.000000in}{-0.048611in}}%
\pgfusepath{stroke,fill}%
}%
\begin{pgfscope}%
\pgfsys@transformshift{1.045000in}{0.521603in}%
\pgfsys@useobject{currentmarker}{}%
\end{pgfscope}%
\end{pgfscope}%
\begin{pgfscope}%
\definecolor{textcolor}{rgb}{0.000000,0.000000,0.000000}%
\pgfsetstrokecolor{textcolor}%
\pgfsetfillcolor{textcolor}%
\pgftext[x=1.045000in,y=0.424381in,,top]{\color{textcolor}{\rmfamily\fontsize{10.000000}{12.000000}\selectfont\catcode`\^=\active\def^{\ifmmode\sp\else\^{}\fi}\catcode`\%=\active\def%{\%}$\mathdefault{12}$}}%
\end{pgfscope}%
\begin{pgfscope}%
\pgfsetbuttcap%
\pgfsetroundjoin%
\definecolor{currentfill}{rgb}{0.000000,0.000000,0.000000}%
\pgfsetfillcolor{currentfill}%
\pgfsetlinewidth{0.803000pt}%
\definecolor{currentstroke}{rgb}{0.000000,0.000000,0.000000}%
\pgfsetstrokecolor{currentstroke}%
\pgfsetdash{}{0pt}%
\pgfsys@defobject{currentmarker}{\pgfqpoint{0.000000in}{-0.048611in}}{\pgfqpoint{0.000000in}{0.000000in}}{%
\pgfpathmoveto{\pgfqpoint{0.000000in}{0.000000in}}%
\pgfpathlineto{\pgfqpoint{0.000000in}{-0.048611in}}%
\pgfusepath{stroke,fill}%
}%
\begin{pgfscope}%
\pgfsys@transformshift{1.660658in}{0.521603in}%
\pgfsys@useobject{currentmarker}{}%
\end{pgfscope}%
\end{pgfscope}%
\begin{pgfscope}%
\definecolor{textcolor}{rgb}{0.000000,0.000000,0.000000}%
\pgfsetstrokecolor{textcolor}%
\pgfsetfillcolor{textcolor}%
\pgftext[x=1.660658in,y=0.424381in,,top]{\color{textcolor}{\rmfamily\fontsize{10.000000}{12.000000}\selectfont\catcode`\^=\active\def^{\ifmmode\sp\else\^{}\fi}\catcode`\%=\active\def%{\%}$\mathdefault{18}$}}%
\end{pgfscope}%
\begin{pgfscope}%
\pgfsetbuttcap%
\pgfsetroundjoin%
\definecolor{currentfill}{rgb}{0.000000,0.000000,0.000000}%
\pgfsetfillcolor{currentfill}%
\pgfsetlinewidth{0.803000pt}%
\definecolor{currentstroke}{rgb}{0.000000,0.000000,0.000000}%
\pgfsetstrokecolor{currentstroke}%
\pgfsetdash{}{0pt}%
\pgfsys@defobject{currentmarker}{\pgfqpoint{0.000000in}{-0.048611in}}{\pgfqpoint{0.000000in}{0.000000in}}{%
\pgfpathmoveto{\pgfqpoint{0.000000in}{0.000000in}}%
\pgfpathlineto{\pgfqpoint{0.000000in}{-0.048611in}}%
\pgfusepath{stroke,fill}%
}%
\begin{pgfscope}%
\pgfsys@transformshift{2.276315in}{0.521603in}%
\pgfsys@useobject{currentmarker}{}%
\end{pgfscope}%
\end{pgfscope}%
\begin{pgfscope}%
\definecolor{textcolor}{rgb}{0.000000,0.000000,0.000000}%
\pgfsetstrokecolor{textcolor}%
\pgfsetfillcolor{textcolor}%
\pgftext[x=2.276315in,y=0.424381in,,top]{\color{textcolor}{\rmfamily\fontsize{10.000000}{12.000000}\selectfont\catcode`\^=\active\def^{\ifmmode\sp\else\^{}\fi}\catcode`\%=\active\def%{\%}$\mathdefault{24}$}}%
\end{pgfscope}%
\begin{pgfscope}%
\pgfsetbuttcap%
\pgfsetroundjoin%
\definecolor{currentfill}{rgb}{0.000000,0.000000,0.000000}%
\pgfsetfillcolor{currentfill}%
\pgfsetlinewidth{0.803000pt}%
\definecolor{currentstroke}{rgb}{0.000000,0.000000,0.000000}%
\pgfsetstrokecolor{currentstroke}%
\pgfsetdash{}{0pt}%
\pgfsys@defobject{currentmarker}{\pgfqpoint{0.000000in}{-0.048611in}}{\pgfqpoint{0.000000in}{0.000000in}}{%
\pgfpathmoveto{\pgfqpoint{0.000000in}{0.000000in}}%
\pgfpathlineto{\pgfqpoint{0.000000in}{-0.048611in}}%
\pgfusepath{stroke,fill}%
}%
\begin{pgfscope}%
\pgfsys@transformshift{2.891973in}{0.521603in}%
\pgfsys@useobject{currentmarker}{}%
\end{pgfscope}%
\end{pgfscope}%
\begin{pgfscope}%
\definecolor{textcolor}{rgb}{0.000000,0.000000,0.000000}%
\pgfsetstrokecolor{textcolor}%
\pgfsetfillcolor{textcolor}%
\pgftext[x=2.891973in,y=0.424381in,,top]{\color{textcolor}{\rmfamily\fontsize{10.000000}{12.000000}\selectfont\catcode`\^=\active\def^{\ifmmode\sp\else\^{}\fi}\catcode`\%=\active\def%{\%}$\mathdefault{30}$}}%
\end{pgfscope}%
\begin{pgfscope}%
\pgfsetbuttcap%
\pgfsetroundjoin%
\definecolor{currentfill}{rgb}{0.000000,0.000000,0.000000}%
\pgfsetfillcolor{currentfill}%
\pgfsetlinewidth{0.803000pt}%
\definecolor{currentstroke}{rgb}{0.000000,0.000000,0.000000}%
\pgfsetstrokecolor{currentstroke}%
\pgfsetdash{}{0pt}%
\pgfsys@defobject{currentmarker}{\pgfqpoint{0.000000in}{-0.048611in}}{\pgfqpoint{0.000000in}{0.000000in}}{%
\pgfpathmoveto{\pgfqpoint{0.000000in}{0.000000in}}%
\pgfpathlineto{\pgfqpoint{0.000000in}{-0.048611in}}%
\pgfusepath{stroke,fill}%
}%
\begin{pgfscope}%
\pgfsys@transformshift{3.507630in}{0.521603in}%
\pgfsys@useobject{currentmarker}{}%
\end{pgfscope}%
\end{pgfscope}%
\begin{pgfscope}%
\definecolor{textcolor}{rgb}{0.000000,0.000000,0.000000}%
\pgfsetstrokecolor{textcolor}%
\pgfsetfillcolor{textcolor}%
\pgftext[x=3.507630in,y=0.424381in,,top]{\color{textcolor}{\rmfamily\fontsize{10.000000}{12.000000}\selectfont\catcode`\^=\active\def^{\ifmmode\sp\else\^{}\fi}\catcode`\%=\active\def%{\%}$\mathdefault{36}$}}%
\end{pgfscope}%
\begin{pgfscope}%
\pgfsetbuttcap%
\pgfsetroundjoin%
\definecolor{currentfill}{rgb}{0.000000,0.000000,0.000000}%
\pgfsetfillcolor{currentfill}%
\pgfsetlinewidth{0.803000pt}%
\definecolor{currentstroke}{rgb}{0.000000,0.000000,0.000000}%
\pgfsetstrokecolor{currentstroke}%
\pgfsetdash{}{0pt}%
\pgfsys@defobject{currentmarker}{\pgfqpoint{0.000000in}{-0.048611in}}{\pgfqpoint{0.000000in}{0.000000in}}{%
\pgfpathmoveto{\pgfqpoint{0.000000in}{0.000000in}}%
\pgfpathlineto{\pgfqpoint{0.000000in}{-0.048611in}}%
\pgfusepath{stroke,fill}%
}%
\begin{pgfscope}%
\pgfsys@transformshift{4.123288in}{0.521603in}%
\pgfsys@useobject{currentmarker}{}%
\end{pgfscope}%
\end{pgfscope}%
\begin{pgfscope}%
\definecolor{textcolor}{rgb}{0.000000,0.000000,0.000000}%
\pgfsetstrokecolor{textcolor}%
\pgfsetfillcolor{textcolor}%
\pgftext[x=4.123288in,y=0.424381in,,top]{\color{textcolor}{\rmfamily\fontsize{10.000000}{12.000000}\selectfont\catcode`\^=\active\def^{\ifmmode\sp\else\^{}\fi}\catcode`\%=\active\def%{\%}$\mathdefault{42}$}}%
\end{pgfscope}%
\begin{pgfscope}%
\pgfsetbuttcap%
\pgfsetroundjoin%
\definecolor{currentfill}{rgb}{0.000000,0.000000,0.000000}%
\pgfsetfillcolor{currentfill}%
\pgfsetlinewidth{0.803000pt}%
\definecolor{currentstroke}{rgb}{0.000000,0.000000,0.000000}%
\pgfsetstrokecolor{currentstroke}%
\pgfsetdash{}{0pt}%
\pgfsys@defobject{currentmarker}{\pgfqpoint{0.000000in}{-0.048611in}}{\pgfqpoint{0.000000in}{0.000000in}}{%
\pgfpathmoveto{\pgfqpoint{0.000000in}{0.000000in}}%
\pgfpathlineto{\pgfqpoint{0.000000in}{-0.048611in}}%
\pgfusepath{stroke,fill}%
}%
\begin{pgfscope}%
\pgfsys@transformshift{4.738945in}{0.521603in}%
\pgfsys@useobject{currentmarker}{}%
\end{pgfscope}%
\end{pgfscope}%
\begin{pgfscope}%
\definecolor{textcolor}{rgb}{0.000000,0.000000,0.000000}%
\pgfsetstrokecolor{textcolor}%
\pgfsetfillcolor{textcolor}%
\pgftext[x=4.738945in,y=0.424381in,,top]{\color{textcolor}{\rmfamily\fontsize{10.000000}{12.000000}\selectfont\catcode`\^=\active\def^{\ifmmode\sp\else\^{}\fi}\catcode`\%=\active\def%{\%}$\mathdefault{48}$}}%
\end{pgfscope}%
\begin{pgfscope}%
\pgfsetbuttcap%
\pgfsetroundjoin%
\definecolor{currentfill}{rgb}{0.000000,0.000000,0.000000}%
\pgfsetfillcolor{currentfill}%
\pgfsetlinewidth{0.803000pt}%
\definecolor{currentstroke}{rgb}{0.000000,0.000000,0.000000}%
\pgfsetstrokecolor{currentstroke}%
\pgfsetdash{}{0pt}%
\pgfsys@defobject{currentmarker}{\pgfqpoint{0.000000in}{-0.048611in}}{\pgfqpoint{0.000000in}{0.000000in}}{%
\pgfpathmoveto{\pgfqpoint{0.000000in}{0.000000in}}%
\pgfpathlineto{\pgfqpoint{0.000000in}{-0.048611in}}%
\pgfusepath{stroke,fill}%
}%
\begin{pgfscope}%
\pgfsys@transformshift{5.354603in}{0.521603in}%
\pgfsys@useobject{currentmarker}{}%
\end{pgfscope}%
\end{pgfscope}%
\begin{pgfscope}%
\definecolor{textcolor}{rgb}{0.000000,0.000000,0.000000}%
\pgfsetstrokecolor{textcolor}%
\pgfsetfillcolor{textcolor}%
\pgftext[x=5.354603in,y=0.424381in,,top]{\color{textcolor}{\rmfamily\fontsize{10.000000}{12.000000}\selectfont\catcode`\^=\active\def^{\ifmmode\sp\else\^{}\fi}\catcode`\%=\active\def%{\%}$\mathdefault{54}$}}%
\end{pgfscope}%
\begin{pgfscope}%
\pgfsetbuttcap%
\pgfsetroundjoin%
\definecolor{currentfill}{rgb}{0.000000,0.000000,0.000000}%
\pgfsetfillcolor{currentfill}%
\pgfsetlinewidth{0.803000pt}%
\definecolor{currentstroke}{rgb}{0.000000,0.000000,0.000000}%
\pgfsetstrokecolor{currentstroke}%
\pgfsetdash{}{0pt}%
\pgfsys@defobject{currentmarker}{\pgfqpoint{0.000000in}{-0.048611in}}{\pgfqpoint{0.000000in}{0.000000in}}{%
\pgfpathmoveto{\pgfqpoint{0.000000in}{0.000000in}}%
\pgfpathlineto{\pgfqpoint{0.000000in}{-0.048611in}}%
\pgfusepath{stroke,fill}%
}%
\begin{pgfscope}%
\pgfsys@transformshift{5.970261in}{0.521603in}%
\pgfsys@useobject{currentmarker}{}%
\end{pgfscope}%
\end{pgfscope}%
\begin{pgfscope}%
\definecolor{textcolor}{rgb}{0.000000,0.000000,0.000000}%
\pgfsetstrokecolor{textcolor}%
\pgfsetfillcolor{textcolor}%
\pgftext[x=5.970261in,y=0.424381in,,top]{\color{textcolor}{\rmfamily\fontsize{10.000000}{12.000000}\selectfont\catcode`\^=\active\def^{\ifmmode\sp\else\^{}\fi}\catcode`\%=\active\def%{\%}$\mathdefault{60}$}}%
\end{pgfscope}%
\begin{pgfscope}%
\definecolor{textcolor}{rgb}{0.000000,0.000000,0.000000}%
\pgfsetstrokecolor{textcolor}%
\pgfsetfillcolor{textcolor}%
\pgftext[x=3.353716in,y=0.234413in,,top]{\color{textcolor}{\rmfamily\fontsize{10.000000}{12.000000}\selectfont\catcode`\^=\active\def^{\ifmmode\sp\else\^{}\fi}\catcode`\%=\active\def%{\%}Vertices}}%
\end{pgfscope}%
\begin{pgfscope}%
\pgfsetbuttcap%
\pgfsetroundjoin%
\definecolor{currentfill}{rgb}{0.000000,0.000000,0.000000}%
\pgfsetfillcolor{currentfill}%
\pgfsetlinewidth{0.803000pt}%
\definecolor{currentstroke}{rgb}{0.000000,0.000000,0.000000}%
\pgfsetstrokecolor{currentstroke}%
\pgfsetdash{}{0pt}%
\pgfsys@defobject{currentmarker}{\pgfqpoint{-0.048611in}{0.000000in}}{\pgfqpoint{-0.000000in}{0.000000in}}{%
\pgfpathmoveto{\pgfqpoint{-0.000000in}{0.000000in}}%
\pgfpathlineto{\pgfqpoint{-0.048611in}{0.000000in}}%
\pgfusepath{stroke,fill}%
}%
\begin{pgfscope}%
\pgfsys@transformshift{0.588387in}{0.803638in}%
\pgfsys@useobject{currentmarker}{}%
\end{pgfscope}%
\end{pgfscope}%
\begin{pgfscope}%
\definecolor{textcolor}{rgb}{0.000000,0.000000,0.000000}%
\pgfsetstrokecolor{textcolor}%
\pgfsetfillcolor{textcolor}%
\pgftext[x=0.289968in, y=0.750876in, left, base]{\color{textcolor}{\rmfamily\fontsize{10.000000}{12.000000}\selectfont\catcode`\^=\active\def^{\ifmmode\sp\else\^{}\fi}\catcode`\%=\active\def%{\%}$\mathdefault{10^{1}}$}}%
\end{pgfscope}%
\begin{pgfscope}%
\pgfsetbuttcap%
\pgfsetroundjoin%
\definecolor{currentfill}{rgb}{0.000000,0.000000,0.000000}%
\pgfsetfillcolor{currentfill}%
\pgfsetlinewidth{0.803000pt}%
\definecolor{currentstroke}{rgb}{0.000000,0.000000,0.000000}%
\pgfsetstrokecolor{currentstroke}%
\pgfsetdash{}{0pt}%
\pgfsys@defobject{currentmarker}{\pgfqpoint{-0.048611in}{0.000000in}}{\pgfqpoint{-0.000000in}{0.000000in}}{%
\pgfpathmoveto{\pgfqpoint{-0.000000in}{0.000000in}}%
\pgfpathlineto{\pgfqpoint{-0.048611in}{0.000000in}}%
\pgfusepath{stroke,fill}%
}%
\begin{pgfscope}%
\pgfsys@transformshift{0.588387in}{1.353421in}%
\pgfsys@useobject{currentmarker}{}%
\end{pgfscope}%
\end{pgfscope}%
\begin{pgfscope}%
\definecolor{textcolor}{rgb}{0.000000,0.000000,0.000000}%
\pgfsetstrokecolor{textcolor}%
\pgfsetfillcolor{textcolor}%
\pgftext[x=0.289968in, y=1.300660in, left, base]{\color{textcolor}{\rmfamily\fontsize{10.000000}{12.000000}\selectfont\catcode`\^=\active\def^{\ifmmode\sp\else\^{}\fi}\catcode`\%=\active\def%{\%}$\mathdefault{10^{2}}$}}%
\end{pgfscope}%
\begin{pgfscope}%
\pgfsetbuttcap%
\pgfsetroundjoin%
\definecolor{currentfill}{rgb}{0.000000,0.000000,0.000000}%
\pgfsetfillcolor{currentfill}%
\pgfsetlinewidth{0.803000pt}%
\definecolor{currentstroke}{rgb}{0.000000,0.000000,0.000000}%
\pgfsetstrokecolor{currentstroke}%
\pgfsetdash{}{0pt}%
\pgfsys@defobject{currentmarker}{\pgfqpoint{-0.048611in}{0.000000in}}{\pgfqpoint{-0.000000in}{0.000000in}}{%
\pgfpathmoveto{\pgfqpoint{-0.000000in}{0.000000in}}%
\pgfpathlineto{\pgfqpoint{-0.048611in}{0.000000in}}%
\pgfusepath{stroke,fill}%
}%
\begin{pgfscope}%
\pgfsys@transformshift{0.588387in}{1.903204in}%
\pgfsys@useobject{currentmarker}{}%
\end{pgfscope}%
\end{pgfscope}%
\begin{pgfscope}%
\definecolor{textcolor}{rgb}{0.000000,0.000000,0.000000}%
\pgfsetstrokecolor{textcolor}%
\pgfsetfillcolor{textcolor}%
\pgftext[x=0.289968in, y=1.850443in, left, base]{\color{textcolor}{\rmfamily\fontsize{10.000000}{12.000000}\selectfont\catcode`\^=\active\def^{\ifmmode\sp\else\^{}\fi}\catcode`\%=\active\def%{\%}$\mathdefault{10^{3}}$}}%
\end{pgfscope}%
\begin{pgfscope}%
\pgfsetbuttcap%
\pgfsetroundjoin%
\definecolor{currentfill}{rgb}{0.000000,0.000000,0.000000}%
\pgfsetfillcolor{currentfill}%
\pgfsetlinewidth{0.803000pt}%
\definecolor{currentstroke}{rgb}{0.000000,0.000000,0.000000}%
\pgfsetstrokecolor{currentstroke}%
\pgfsetdash{}{0pt}%
\pgfsys@defobject{currentmarker}{\pgfqpoint{-0.048611in}{0.000000in}}{\pgfqpoint{-0.000000in}{0.000000in}}{%
\pgfpathmoveto{\pgfqpoint{-0.000000in}{0.000000in}}%
\pgfpathlineto{\pgfqpoint{-0.048611in}{0.000000in}}%
\pgfusepath{stroke,fill}%
}%
\begin{pgfscope}%
\pgfsys@transformshift{0.588387in}{2.452988in}%
\pgfsys@useobject{currentmarker}{}%
\end{pgfscope}%
\end{pgfscope}%
\begin{pgfscope}%
\definecolor{textcolor}{rgb}{0.000000,0.000000,0.000000}%
\pgfsetstrokecolor{textcolor}%
\pgfsetfillcolor{textcolor}%
\pgftext[x=0.289968in, y=2.400226in, left, base]{\color{textcolor}{\rmfamily\fontsize{10.000000}{12.000000}\selectfont\catcode`\^=\active\def^{\ifmmode\sp\else\^{}\fi}\catcode`\%=\active\def%{\%}$\mathdefault{10^{4}}$}}%
\end{pgfscope}%
\begin{pgfscope}%
\pgfsetbuttcap%
\pgfsetroundjoin%
\definecolor{currentfill}{rgb}{0.000000,0.000000,0.000000}%
\pgfsetfillcolor{currentfill}%
\pgfsetlinewidth{0.602250pt}%
\definecolor{currentstroke}{rgb}{0.000000,0.000000,0.000000}%
\pgfsetstrokecolor{currentstroke}%
\pgfsetdash{}{0pt}%
\pgfsys@defobject{currentmarker}{\pgfqpoint{-0.027778in}{0.000000in}}{\pgfqpoint{-0.000000in}{0.000000in}}{%
\pgfpathmoveto{\pgfqpoint{-0.000000in}{0.000000in}}%
\pgfpathlineto{\pgfqpoint{-0.027778in}{0.000000in}}%
\pgfusepath{stroke,fill}%
}%
\begin{pgfscope}%
\pgfsys@transformshift{0.588387in}{0.584857in}%
\pgfsys@useobject{currentmarker}{}%
\end{pgfscope}%
\end{pgfscope}%
\begin{pgfscope}%
\pgfsetbuttcap%
\pgfsetroundjoin%
\definecolor{currentfill}{rgb}{0.000000,0.000000,0.000000}%
\pgfsetfillcolor{currentfill}%
\pgfsetlinewidth{0.602250pt}%
\definecolor{currentstroke}{rgb}{0.000000,0.000000,0.000000}%
\pgfsetstrokecolor{currentstroke}%
\pgfsetdash{}{0pt}%
\pgfsys@defobject{currentmarker}{\pgfqpoint{-0.027778in}{0.000000in}}{\pgfqpoint{-0.000000in}{0.000000in}}{%
\pgfpathmoveto{\pgfqpoint{-0.000000in}{0.000000in}}%
\pgfpathlineto{\pgfqpoint{-0.027778in}{0.000000in}}%
\pgfusepath{stroke,fill}%
}%
\begin{pgfscope}%
\pgfsys@transformshift{0.588387in}{0.638137in}%
\pgfsys@useobject{currentmarker}{}%
\end{pgfscope}%
\end{pgfscope}%
\begin{pgfscope}%
\pgfsetbuttcap%
\pgfsetroundjoin%
\definecolor{currentfill}{rgb}{0.000000,0.000000,0.000000}%
\pgfsetfillcolor{currentfill}%
\pgfsetlinewidth{0.602250pt}%
\definecolor{currentstroke}{rgb}{0.000000,0.000000,0.000000}%
\pgfsetstrokecolor{currentstroke}%
\pgfsetdash{}{0pt}%
\pgfsys@defobject{currentmarker}{\pgfqpoint{-0.027778in}{0.000000in}}{\pgfqpoint{-0.000000in}{0.000000in}}{%
\pgfpathmoveto{\pgfqpoint{-0.000000in}{0.000000in}}%
\pgfpathlineto{\pgfqpoint{-0.027778in}{0.000000in}}%
\pgfusepath{stroke,fill}%
}%
\begin{pgfscope}%
\pgfsys@transformshift{0.588387in}{0.681669in}%
\pgfsys@useobject{currentmarker}{}%
\end{pgfscope}%
\end{pgfscope}%
\begin{pgfscope}%
\pgfsetbuttcap%
\pgfsetroundjoin%
\definecolor{currentfill}{rgb}{0.000000,0.000000,0.000000}%
\pgfsetfillcolor{currentfill}%
\pgfsetlinewidth{0.602250pt}%
\definecolor{currentstroke}{rgb}{0.000000,0.000000,0.000000}%
\pgfsetstrokecolor{currentstroke}%
\pgfsetdash{}{0pt}%
\pgfsys@defobject{currentmarker}{\pgfqpoint{-0.027778in}{0.000000in}}{\pgfqpoint{-0.000000in}{0.000000in}}{%
\pgfpathmoveto{\pgfqpoint{-0.000000in}{0.000000in}}%
\pgfpathlineto{\pgfqpoint{-0.027778in}{0.000000in}}%
\pgfusepath{stroke,fill}%
}%
\begin{pgfscope}%
\pgfsys@transformshift{0.588387in}{0.718475in}%
\pgfsys@useobject{currentmarker}{}%
\end{pgfscope}%
\end{pgfscope}%
\begin{pgfscope}%
\pgfsetbuttcap%
\pgfsetroundjoin%
\definecolor{currentfill}{rgb}{0.000000,0.000000,0.000000}%
\pgfsetfillcolor{currentfill}%
\pgfsetlinewidth{0.602250pt}%
\definecolor{currentstroke}{rgb}{0.000000,0.000000,0.000000}%
\pgfsetstrokecolor{currentstroke}%
\pgfsetdash{}{0pt}%
\pgfsys@defobject{currentmarker}{\pgfqpoint{-0.027778in}{0.000000in}}{\pgfqpoint{-0.000000in}{0.000000in}}{%
\pgfpathmoveto{\pgfqpoint{-0.000000in}{0.000000in}}%
\pgfpathlineto{\pgfqpoint{-0.027778in}{0.000000in}}%
\pgfusepath{stroke,fill}%
}%
\begin{pgfscope}%
\pgfsys@transformshift{0.588387in}{0.750358in}%
\pgfsys@useobject{currentmarker}{}%
\end{pgfscope}%
\end{pgfscope}%
\begin{pgfscope}%
\pgfsetbuttcap%
\pgfsetroundjoin%
\definecolor{currentfill}{rgb}{0.000000,0.000000,0.000000}%
\pgfsetfillcolor{currentfill}%
\pgfsetlinewidth{0.602250pt}%
\definecolor{currentstroke}{rgb}{0.000000,0.000000,0.000000}%
\pgfsetstrokecolor{currentstroke}%
\pgfsetdash{}{0pt}%
\pgfsys@defobject{currentmarker}{\pgfqpoint{-0.027778in}{0.000000in}}{\pgfqpoint{-0.000000in}{0.000000in}}{%
\pgfpathmoveto{\pgfqpoint{-0.000000in}{0.000000in}}%
\pgfpathlineto{\pgfqpoint{-0.027778in}{0.000000in}}%
\pgfusepath{stroke,fill}%
}%
\begin{pgfscope}%
\pgfsys@transformshift{0.588387in}{0.778481in}%
\pgfsys@useobject{currentmarker}{}%
\end{pgfscope}%
\end{pgfscope}%
\begin{pgfscope}%
\pgfsetbuttcap%
\pgfsetroundjoin%
\definecolor{currentfill}{rgb}{0.000000,0.000000,0.000000}%
\pgfsetfillcolor{currentfill}%
\pgfsetlinewidth{0.602250pt}%
\definecolor{currentstroke}{rgb}{0.000000,0.000000,0.000000}%
\pgfsetstrokecolor{currentstroke}%
\pgfsetdash{}{0pt}%
\pgfsys@defobject{currentmarker}{\pgfqpoint{-0.027778in}{0.000000in}}{\pgfqpoint{-0.000000in}{0.000000in}}{%
\pgfpathmoveto{\pgfqpoint{-0.000000in}{0.000000in}}%
\pgfpathlineto{\pgfqpoint{-0.027778in}{0.000000in}}%
\pgfusepath{stroke,fill}%
}%
\begin{pgfscope}%
\pgfsys@transformshift{0.588387in}{0.969139in}%
\pgfsys@useobject{currentmarker}{}%
\end{pgfscope}%
\end{pgfscope}%
\begin{pgfscope}%
\pgfsetbuttcap%
\pgfsetroundjoin%
\definecolor{currentfill}{rgb}{0.000000,0.000000,0.000000}%
\pgfsetfillcolor{currentfill}%
\pgfsetlinewidth{0.602250pt}%
\definecolor{currentstroke}{rgb}{0.000000,0.000000,0.000000}%
\pgfsetstrokecolor{currentstroke}%
\pgfsetdash{}{0pt}%
\pgfsys@defobject{currentmarker}{\pgfqpoint{-0.027778in}{0.000000in}}{\pgfqpoint{-0.000000in}{0.000000in}}{%
\pgfpathmoveto{\pgfqpoint{-0.000000in}{0.000000in}}%
\pgfpathlineto{\pgfqpoint{-0.027778in}{0.000000in}}%
\pgfusepath{stroke,fill}%
}%
\begin{pgfscope}%
\pgfsys@transformshift{0.588387in}{1.065951in}%
\pgfsys@useobject{currentmarker}{}%
\end{pgfscope}%
\end{pgfscope}%
\begin{pgfscope}%
\pgfsetbuttcap%
\pgfsetroundjoin%
\definecolor{currentfill}{rgb}{0.000000,0.000000,0.000000}%
\pgfsetfillcolor{currentfill}%
\pgfsetlinewidth{0.602250pt}%
\definecolor{currentstroke}{rgb}{0.000000,0.000000,0.000000}%
\pgfsetstrokecolor{currentstroke}%
\pgfsetdash{}{0pt}%
\pgfsys@defobject{currentmarker}{\pgfqpoint{-0.027778in}{0.000000in}}{\pgfqpoint{-0.000000in}{0.000000in}}{%
\pgfpathmoveto{\pgfqpoint{-0.000000in}{0.000000in}}%
\pgfpathlineto{\pgfqpoint{-0.027778in}{0.000000in}}%
\pgfusepath{stroke,fill}%
}%
\begin{pgfscope}%
\pgfsys@transformshift{0.588387in}{1.134640in}%
\pgfsys@useobject{currentmarker}{}%
\end{pgfscope}%
\end{pgfscope}%
\begin{pgfscope}%
\pgfsetbuttcap%
\pgfsetroundjoin%
\definecolor{currentfill}{rgb}{0.000000,0.000000,0.000000}%
\pgfsetfillcolor{currentfill}%
\pgfsetlinewidth{0.602250pt}%
\definecolor{currentstroke}{rgb}{0.000000,0.000000,0.000000}%
\pgfsetstrokecolor{currentstroke}%
\pgfsetdash{}{0pt}%
\pgfsys@defobject{currentmarker}{\pgfqpoint{-0.027778in}{0.000000in}}{\pgfqpoint{-0.000000in}{0.000000in}}{%
\pgfpathmoveto{\pgfqpoint{-0.000000in}{0.000000in}}%
\pgfpathlineto{\pgfqpoint{-0.027778in}{0.000000in}}%
\pgfusepath{stroke,fill}%
}%
\begin{pgfscope}%
\pgfsys@transformshift{0.588387in}{1.187920in}%
\pgfsys@useobject{currentmarker}{}%
\end{pgfscope}%
\end{pgfscope}%
\begin{pgfscope}%
\pgfsetbuttcap%
\pgfsetroundjoin%
\definecolor{currentfill}{rgb}{0.000000,0.000000,0.000000}%
\pgfsetfillcolor{currentfill}%
\pgfsetlinewidth{0.602250pt}%
\definecolor{currentstroke}{rgb}{0.000000,0.000000,0.000000}%
\pgfsetstrokecolor{currentstroke}%
\pgfsetdash{}{0pt}%
\pgfsys@defobject{currentmarker}{\pgfqpoint{-0.027778in}{0.000000in}}{\pgfqpoint{-0.000000in}{0.000000in}}{%
\pgfpathmoveto{\pgfqpoint{-0.000000in}{0.000000in}}%
\pgfpathlineto{\pgfqpoint{-0.027778in}{0.000000in}}%
\pgfusepath{stroke,fill}%
}%
\begin{pgfscope}%
\pgfsys@transformshift{0.588387in}{1.231452in}%
\pgfsys@useobject{currentmarker}{}%
\end{pgfscope}%
\end{pgfscope}%
\begin{pgfscope}%
\pgfsetbuttcap%
\pgfsetroundjoin%
\definecolor{currentfill}{rgb}{0.000000,0.000000,0.000000}%
\pgfsetfillcolor{currentfill}%
\pgfsetlinewidth{0.602250pt}%
\definecolor{currentstroke}{rgb}{0.000000,0.000000,0.000000}%
\pgfsetstrokecolor{currentstroke}%
\pgfsetdash{}{0pt}%
\pgfsys@defobject{currentmarker}{\pgfqpoint{-0.027778in}{0.000000in}}{\pgfqpoint{-0.000000in}{0.000000in}}{%
\pgfpathmoveto{\pgfqpoint{-0.000000in}{0.000000in}}%
\pgfpathlineto{\pgfqpoint{-0.027778in}{0.000000in}}%
\pgfusepath{stroke,fill}%
}%
\begin{pgfscope}%
\pgfsys@transformshift{0.588387in}{1.268259in}%
\pgfsys@useobject{currentmarker}{}%
\end{pgfscope}%
\end{pgfscope}%
\begin{pgfscope}%
\pgfsetbuttcap%
\pgfsetroundjoin%
\definecolor{currentfill}{rgb}{0.000000,0.000000,0.000000}%
\pgfsetfillcolor{currentfill}%
\pgfsetlinewidth{0.602250pt}%
\definecolor{currentstroke}{rgb}{0.000000,0.000000,0.000000}%
\pgfsetstrokecolor{currentstroke}%
\pgfsetdash{}{0pt}%
\pgfsys@defobject{currentmarker}{\pgfqpoint{-0.027778in}{0.000000in}}{\pgfqpoint{-0.000000in}{0.000000in}}{%
\pgfpathmoveto{\pgfqpoint{-0.000000in}{0.000000in}}%
\pgfpathlineto{\pgfqpoint{-0.027778in}{0.000000in}}%
\pgfusepath{stroke,fill}%
}%
\begin{pgfscope}%
\pgfsys@transformshift{0.588387in}{1.300142in}%
\pgfsys@useobject{currentmarker}{}%
\end{pgfscope}%
\end{pgfscope}%
\begin{pgfscope}%
\pgfsetbuttcap%
\pgfsetroundjoin%
\definecolor{currentfill}{rgb}{0.000000,0.000000,0.000000}%
\pgfsetfillcolor{currentfill}%
\pgfsetlinewidth{0.602250pt}%
\definecolor{currentstroke}{rgb}{0.000000,0.000000,0.000000}%
\pgfsetstrokecolor{currentstroke}%
\pgfsetdash{}{0pt}%
\pgfsys@defobject{currentmarker}{\pgfqpoint{-0.027778in}{0.000000in}}{\pgfqpoint{-0.000000in}{0.000000in}}{%
\pgfpathmoveto{\pgfqpoint{-0.000000in}{0.000000in}}%
\pgfpathlineto{\pgfqpoint{-0.027778in}{0.000000in}}%
\pgfusepath{stroke,fill}%
}%
\begin{pgfscope}%
\pgfsys@transformshift{0.588387in}{1.328264in}%
\pgfsys@useobject{currentmarker}{}%
\end{pgfscope}%
\end{pgfscope}%
\begin{pgfscope}%
\pgfsetbuttcap%
\pgfsetroundjoin%
\definecolor{currentfill}{rgb}{0.000000,0.000000,0.000000}%
\pgfsetfillcolor{currentfill}%
\pgfsetlinewidth{0.602250pt}%
\definecolor{currentstroke}{rgb}{0.000000,0.000000,0.000000}%
\pgfsetstrokecolor{currentstroke}%
\pgfsetdash{}{0pt}%
\pgfsys@defobject{currentmarker}{\pgfqpoint{-0.027778in}{0.000000in}}{\pgfqpoint{-0.000000in}{0.000000in}}{%
\pgfpathmoveto{\pgfqpoint{-0.000000in}{0.000000in}}%
\pgfpathlineto{\pgfqpoint{-0.027778in}{0.000000in}}%
\pgfusepath{stroke,fill}%
}%
\begin{pgfscope}%
\pgfsys@transformshift{0.588387in}{1.518922in}%
\pgfsys@useobject{currentmarker}{}%
\end{pgfscope}%
\end{pgfscope}%
\begin{pgfscope}%
\pgfsetbuttcap%
\pgfsetroundjoin%
\definecolor{currentfill}{rgb}{0.000000,0.000000,0.000000}%
\pgfsetfillcolor{currentfill}%
\pgfsetlinewidth{0.602250pt}%
\definecolor{currentstroke}{rgb}{0.000000,0.000000,0.000000}%
\pgfsetstrokecolor{currentstroke}%
\pgfsetdash{}{0pt}%
\pgfsys@defobject{currentmarker}{\pgfqpoint{-0.027778in}{0.000000in}}{\pgfqpoint{-0.000000in}{0.000000in}}{%
\pgfpathmoveto{\pgfqpoint{-0.000000in}{0.000000in}}%
\pgfpathlineto{\pgfqpoint{-0.027778in}{0.000000in}}%
\pgfusepath{stroke,fill}%
}%
\begin{pgfscope}%
\pgfsys@transformshift{0.588387in}{1.615734in}%
\pgfsys@useobject{currentmarker}{}%
\end{pgfscope}%
\end{pgfscope}%
\begin{pgfscope}%
\pgfsetbuttcap%
\pgfsetroundjoin%
\definecolor{currentfill}{rgb}{0.000000,0.000000,0.000000}%
\pgfsetfillcolor{currentfill}%
\pgfsetlinewidth{0.602250pt}%
\definecolor{currentstroke}{rgb}{0.000000,0.000000,0.000000}%
\pgfsetstrokecolor{currentstroke}%
\pgfsetdash{}{0pt}%
\pgfsys@defobject{currentmarker}{\pgfqpoint{-0.027778in}{0.000000in}}{\pgfqpoint{-0.000000in}{0.000000in}}{%
\pgfpathmoveto{\pgfqpoint{-0.000000in}{0.000000in}}%
\pgfpathlineto{\pgfqpoint{-0.027778in}{0.000000in}}%
\pgfusepath{stroke,fill}%
}%
\begin{pgfscope}%
\pgfsys@transformshift{0.588387in}{1.684424in}%
\pgfsys@useobject{currentmarker}{}%
\end{pgfscope}%
\end{pgfscope}%
\begin{pgfscope}%
\pgfsetbuttcap%
\pgfsetroundjoin%
\definecolor{currentfill}{rgb}{0.000000,0.000000,0.000000}%
\pgfsetfillcolor{currentfill}%
\pgfsetlinewidth{0.602250pt}%
\definecolor{currentstroke}{rgb}{0.000000,0.000000,0.000000}%
\pgfsetstrokecolor{currentstroke}%
\pgfsetdash{}{0pt}%
\pgfsys@defobject{currentmarker}{\pgfqpoint{-0.027778in}{0.000000in}}{\pgfqpoint{-0.000000in}{0.000000in}}{%
\pgfpathmoveto{\pgfqpoint{-0.000000in}{0.000000in}}%
\pgfpathlineto{\pgfqpoint{-0.027778in}{0.000000in}}%
\pgfusepath{stroke,fill}%
}%
\begin{pgfscope}%
\pgfsys@transformshift{0.588387in}{1.737703in}%
\pgfsys@useobject{currentmarker}{}%
\end{pgfscope}%
\end{pgfscope}%
\begin{pgfscope}%
\pgfsetbuttcap%
\pgfsetroundjoin%
\definecolor{currentfill}{rgb}{0.000000,0.000000,0.000000}%
\pgfsetfillcolor{currentfill}%
\pgfsetlinewidth{0.602250pt}%
\definecolor{currentstroke}{rgb}{0.000000,0.000000,0.000000}%
\pgfsetstrokecolor{currentstroke}%
\pgfsetdash{}{0pt}%
\pgfsys@defobject{currentmarker}{\pgfqpoint{-0.027778in}{0.000000in}}{\pgfqpoint{-0.000000in}{0.000000in}}{%
\pgfpathmoveto{\pgfqpoint{-0.000000in}{0.000000in}}%
\pgfpathlineto{\pgfqpoint{-0.027778in}{0.000000in}}%
\pgfusepath{stroke,fill}%
}%
\begin{pgfscope}%
\pgfsys@transformshift{0.588387in}{1.781236in}%
\pgfsys@useobject{currentmarker}{}%
\end{pgfscope}%
\end{pgfscope}%
\begin{pgfscope}%
\pgfsetbuttcap%
\pgfsetroundjoin%
\definecolor{currentfill}{rgb}{0.000000,0.000000,0.000000}%
\pgfsetfillcolor{currentfill}%
\pgfsetlinewidth{0.602250pt}%
\definecolor{currentstroke}{rgb}{0.000000,0.000000,0.000000}%
\pgfsetstrokecolor{currentstroke}%
\pgfsetdash{}{0pt}%
\pgfsys@defobject{currentmarker}{\pgfqpoint{-0.027778in}{0.000000in}}{\pgfqpoint{-0.000000in}{0.000000in}}{%
\pgfpathmoveto{\pgfqpoint{-0.000000in}{0.000000in}}%
\pgfpathlineto{\pgfqpoint{-0.027778in}{0.000000in}}%
\pgfusepath{stroke,fill}%
}%
\begin{pgfscope}%
\pgfsys@transformshift{0.588387in}{1.818042in}%
\pgfsys@useobject{currentmarker}{}%
\end{pgfscope}%
\end{pgfscope}%
\begin{pgfscope}%
\pgfsetbuttcap%
\pgfsetroundjoin%
\definecolor{currentfill}{rgb}{0.000000,0.000000,0.000000}%
\pgfsetfillcolor{currentfill}%
\pgfsetlinewidth{0.602250pt}%
\definecolor{currentstroke}{rgb}{0.000000,0.000000,0.000000}%
\pgfsetstrokecolor{currentstroke}%
\pgfsetdash{}{0pt}%
\pgfsys@defobject{currentmarker}{\pgfqpoint{-0.027778in}{0.000000in}}{\pgfqpoint{-0.000000in}{0.000000in}}{%
\pgfpathmoveto{\pgfqpoint{-0.000000in}{0.000000in}}%
\pgfpathlineto{\pgfqpoint{-0.027778in}{0.000000in}}%
\pgfusepath{stroke,fill}%
}%
\begin{pgfscope}%
\pgfsys@transformshift{0.588387in}{1.849925in}%
\pgfsys@useobject{currentmarker}{}%
\end{pgfscope}%
\end{pgfscope}%
\begin{pgfscope}%
\pgfsetbuttcap%
\pgfsetroundjoin%
\definecolor{currentfill}{rgb}{0.000000,0.000000,0.000000}%
\pgfsetfillcolor{currentfill}%
\pgfsetlinewidth{0.602250pt}%
\definecolor{currentstroke}{rgb}{0.000000,0.000000,0.000000}%
\pgfsetstrokecolor{currentstroke}%
\pgfsetdash{}{0pt}%
\pgfsys@defobject{currentmarker}{\pgfqpoint{-0.027778in}{0.000000in}}{\pgfqpoint{-0.000000in}{0.000000in}}{%
\pgfpathmoveto{\pgfqpoint{-0.000000in}{0.000000in}}%
\pgfpathlineto{\pgfqpoint{-0.027778in}{0.000000in}}%
\pgfusepath{stroke,fill}%
}%
\begin{pgfscope}%
\pgfsys@transformshift{0.588387in}{1.878048in}%
\pgfsys@useobject{currentmarker}{}%
\end{pgfscope}%
\end{pgfscope}%
\begin{pgfscope}%
\pgfsetbuttcap%
\pgfsetroundjoin%
\definecolor{currentfill}{rgb}{0.000000,0.000000,0.000000}%
\pgfsetfillcolor{currentfill}%
\pgfsetlinewidth{0.602250pt}%
\definecolor{currentstroke}{rgb}{0.000000,0.000000,0.000000}%
\pgfsetstrokecolor{currentstroke}%
\pgfsetdash{}{0pt}%
\pgfsys@defobject{currentmarker}{\pgfqpoint{-0.027778in}{0.000000in}}{\pgfqpoint{-0.000000in}{0.000000in}}{%
\pgfpathmoveto{\pgfqpoint{-0.000000in}{0.000000in}}%
\pgfpathlineto{\pgfqpoint{-0.027778in}{0.000000in}}%
\pgfusepath{stroke,fill}%
}%
\begin{pgfscope}%
\pgfsys@transformshift{0.588387in}{2.068706in}%
\pgfsys@useobject{currentmarker}{}%
\end{pgfscope}%
\end{pgfscope}%
\begin{pgfscope}%
\pgfsetbuttcap%
\pgfsetroundjoin%
\definecolor{currentfill}{rgb}{0.000000,0.000000,0.000000}%
\pgfsetfillcolor{currentfill}%
\pgfsetlinewidth{0.602250pt}%
\definecolor{currentstroke}{rgb}{0.000000,0.000000,0.000000}%
\pgfsetstrokecolor{currentstroke}%
\pgfsetdash{}{0pt}%
\pgfsys@defobject{currentmarker}{\pgfqpoint{-0.027778in}{0.000000in}}{\pgfqpoint{-0.000000in}{0.000000in}}{%
\pgfpathmoveto{\pgfqpoint{-0.000000in}{0.000000in}}%
\pgfpathlineto{\pgfqpoint{-0.027778in}{0.000000in}}%
\pgfusepath{stroke,fill}%
}%
\begin{pgfscope}%
\pgfsys@transformshift{0.588387in}{2.165518in}%
\pgfsys@useobject{currentmarker}{}%
\end{pgfscope}%
\end{pgfscope}%
\begin{pgfscope}%
\pgfsetbuttcap%
\pgfsetroundjoin%
\definecolor{currentfill}{rgb}{0.000000,0.000000,0.000000}%
\pgfsetfillcolor{currentfill}%
\pgfsetlinewidth{0.602250pt}%
\definecolor{currentstroke}{rgb}{0.000000,0.000000,0.000000}%
\pgfsetstrokecolor{currentstroke}%
\pgfsetdash{}{0pt}%
\pgfsys@defobject{currentmarker}{\pgfqpoint{-0.027778in}{0.000000in}}{\pgfqpoint{-0.000000in}{0.000000in}}{%
\pgfpathmoveto{\pgfqpoint{-0.000000in}{0.000000in}}%
\pgfpathlineto{\pgfqpoint{-0.027778in}{0.000000in}}%
\pgfusepath{stroke,fill}%
}%
\begin{pgfscope}%
\pgfsys@transformshift{0.588387in}{2.234207in}%
\pgfsys@useobject{currentmarker}{}%
\end{pgfscope}%
\end{pgfscope}%
\begin{pgfscope}%
\pgfsetbuttcap%
\pgfsetroundjoin%
\definecolor{currentfill}{rgb}{0.000000,0.000000,0.000000}%
\pgfsetfillcolor{currentfill}%
\pgfsetlinewidth{0.602250pt}%
\definecolor{currentstroke}{rgb}{0.000000,0.000000,0.000000}%
\pgfsetstrokecolor{currentstroke}%
\pgfsetdash{}{0pt}%
\pgfsys@defobject{currentmarker}{\pgfqpoint{-0.027778in}{0.000000in}}{\pgfqpoint{-0.000000in}{0.000000in}}{%
\pgfpathmoveto{\pgfqpoint{-0.000000in}{0.000000in}}%
\pgfpathlineto{\pgfqpoint{-0.027778in}{0.000000in}}%
\pgfusepath{stroke,fill}%
}%
\begin{pgfscope}%
\pgfsys@transformshift{0.588387in}{2.287486in}%
\pgfsys@useobject{currentmarker}{}%
\end{pgfscope}%
\end{pgfscope}%
\begin{pgfscope}%
\pgfsetbuttcap%
\pgfsetroundjoin%
\definecolor{currentfill}{rgb}{0.000000,0.000000,0.000000}%
\pgfsetfillcolor{currentfill}%
\pgfsetlinewidth{0.602250pt}%
\definecolor{currentstroke}{rgb}{0.000000,0.000000,0.000000}%
\pgfsetstrokecolor{currentstroke}%
\pgfsetdash{}{0pt}%
\pgfsys@defobject{currentmarker}{\pgfqpoint{-0.027778in}{0.000000in}}{\pgfqpoint{-0.000000in}{0.000000in}}{%
\pgfpathmoveto{\pgfqpoint{-0.000000in}{0.000000in}}%
\pgfpathlineto{\pgfqpoint{-0.027778in}{0.000000in}}%
\pgfusepath{stroke,fill}%
}%
\begin{pgfscope}%
\pgfsys@transformshift{0.588387in}{2.331019in}%
\pgfsys@useobject{currentmarker}{}%
\end{pgfscope}%
\end{pgfscope}%
\begin{pgfscope}%
\pgfsetbuttcap%
\pgfsetroundjoin%
\definecolor{currentfill}{rgb}{0.000000,0.000000,0.000000}%
\pgfsetfillcolor{currentfill}%
\pgfsetlinewidth{0.602250pt}%
\definecolor{currentstroke}{rgb}{0.000000,0.000000,0.000000}%
\pgfsetstrokecolor{currentstroke}%
\pgfsetdash{}{0pt}%
\pgfsys@defobject{currentmarker}{\pgfqpoint{-0.027778in}{0.000000in}}{\pgfqpoint{-0.000000in}{0.000000in}}{%
\pgfpathmoveto{\pgfqpoint{-0.000000in}{0.000000in}}%
\pgfpathlineto{\pgfqpoint{-0.027778in}{0.000000in}}%
\pgfusepath{stroke,fill}%
}%
\begin{pgfscope}%
\pgfsys@transformshift{0.588387in}{2.367825in}%
\pgfsys@useobject{currentmarker}{}%
\end{pgfscope}%
\end{pgfscope}%
\begin{pgfscope}%
\pgfsetbuttcap%
\pgfsetroundjoin%
\definecolor{currentfill}{rgb}{0.000000,0.000000,0.000000}%
\pgfsetfillcolor{currentfill}%
\pgfsetlinewidth{0.602250pt}%
\definecolor{currentstroke}{rgb}{0.000000,0.000000,0.000000}%
\pgfsetstrokecolor{currentstroke}%
\pgfsetdash{}{0pt}%
\pgfsys@defobject{currentmarker}{\pgfqpoint{-0.027778in}{0.000000in}}{\pgfqpoint{-0.000000in}{0.000000in}}{%
\pgfpathmoveto{\pgfqpoint{-0.000000in}{0.000000in}}%
\pgfpathlineto{\pgfqpoint{-0.027778in}{0.000000in}}%
\pgfusepath{stroke,fill}%
}%
\begin{pgfscope}%
\pgfsys@transformshift{0.588387in}{2.399708in}%
\pgfsys@useobject{currentmarker}{}%
\end{pgfscope}%
\end{pgfscope}%
\begin{pgfscope}%
\pgfsetbuttcap%
\pgfsetroundjoin%
\definecolor{currentfill}{rgb}{0.000000,0.000000,0.000000}%
\pgfsetfillcolor{currentfill}%
\pgfsetlinewidth{0.602250pt}%
\definecolor{currentstroke}{rgb}{0.000000,0.000000,0.000000}%
\pgfsetstrokecolor{currentstroke}%
\pgfsetdash{}{0pt}%
\pgfsys@defobject{currentmarker}{\pgfqpoint{-0.027778in}{0.000000in}}{\pgfqpoint{-0.000000in}{0.000000in}}{%
\pgfpathmoveto{\pgfqpoint{-0.000000in}{0.000000in}}%
\pgfpathlineto{\pgfqpoint{-0.027778in}{0.000000in}}%
\pgfusepath{stroke,fill}%
}%
\begin{pgfscope}%
\pgfsys@transformshift{0.588387in}{2.427831in}%
\pgfsys@useobject{currentmarker}{}%
\end{pgfscope}%
\end{pgfscope}%
\begin{pgfscope}%
\definecolor{textcolor}{rgb}{0.000000,0.000000,0.000000}%
\pgfsetstrokecolor{textcolor}%
\pgfsetfillcolor{textcolor}%
\pgftext[x=0.234413in,y=1.526746in,,bottom,rotate=90.000000]{\color{textcolor}{\rmfamily\fontsize{10.000000}{12.000000}\selectfont\catcode`\^=\active\def^{\ifmmode\sp\else\^{}\fi}\catcode`\%=\active\def%{\%}Checks [call]}}%
\end{pgfscope}%
\begin{pgfscope}%
\pgfpathrectangle{\pgfqpoint{0.588387in}{0.521603in}}{\pgfqpoint{5.530657in}{2.010285in}}%
\pgfusepath{clip}%
\pgfsetrectcap%
\pgfsetroundjoin%
\pgfsetlinewidth{1.505625pt}%
\pgfsetstrokecolor{currentstroke1}%
\pgfsetdash{}{0pt}%
\pgfpathmoveto{\pgfqpoint{0.839781in}{0.803638in}}%
\pgfpathlineto{\pgfqpoint{0.942390in}{0.612980in}}%
\pgfpathlineto{\pgfqpoint{1.147610in}{1.153752in}}%
\pgfpathlineto{\pgfqpoint{1.250219in}{1.477292in}}%
\pgfpathlineto{\pgfqpoint{1.352829in}{1.499877in}}%
\pgfpathlineto{\pgfqpoint{1.455438in}{1.497712in}}%
\pgfpathlineto{\pgfqpoint{1.660658in}{1.478709in}}%
\pgfpathlineto{\pgfqpoint{1.763267in}{1.542762in}}%
\pgfpathlineto{\pgfqpoint{2.173706in}{1.206296in}}%
\pgfusepath{stroke}%
\end{pgfscope}%
\begin{pgfscope}%
\pgfpathrectangle{\pgfqpoint{0.588387in}{0.521603in}}{\pgfqpoint{5.530657in}{2.010285in}}%
\pgfusepath{clip}%
\pgfsetrectcap%
\pgfsetroundjoin%
\pgfsetlinewidth{1.505625pt}%
\pgfsetstrokecolor{currentstroke2}%
\pgfsetdash{}{0pt}%
\pgfpathmoveto{\pgfqpoint{0.839781in}{0.883293in}}%
\pgfpathlineto{\pgfqpoint{0.942390in}{0.992452in}}%
\pgfpathlineto{\pgfqpoint{1.045000in}{1.068964in}}%
\pgfpathlineto{\pgfqpoint{1.147610in}{1.298556in}}%
\pgfpathlineto{\pgfqpoint{1.250219in}{1.477460in}}%
\pgfpathlineto{\pgfqpoint{1.352829in}{1.501519in}}%
\pgfpathlineto{\pgfqpoint{1.455438in}{1.642319in}}%
\pgfpathlineto{\pgfqpoint{1.558048in}{1.593980in}}%
\pgfpathlineto{\pgfqpoint{1.660658in}{1.788508in}}%
\pgfpathlineto{\pgfqpoint{1.763267in}{2.063882in}}%
\pgfpathlineto{\pgfqpoint{1.865877in}{2.032444in}}%
\pgfpathlineto{\pgfqpoint{1.968486in}{2.165374in}}%
\pgfpathlineto{\pgfqpoint{2.071096in}{2.119205in}}%
\pgfpathlineto{\pgfqpoint{2.173706in}{1.824171in}}%
\pgfpathlineto{\pgfqpoint{2.276315in}{2.440512in}}%
\pgfpathlineto{\pgfqpoint{2.378925in}{1.927530in}}%
\pgfpathlineto{\pgfqpoint{2.481534in}{1.917642in}}%
\pgfpathlineto{\pgfqpoint{2.584144in}{1.622792in}}%
\pgfpathlineto{\pgfqpoint{2.686753in}{1.853112in}}%
\pgfpathlineto{\pgfqpoint{2.789363in}{2.137544in}}%
\pgfpathlineto{\pgfqpoint{2.891973in}{2.330740in}}%
\pgfpathlineto{\pgfqpoint{2.994582in}{2.266198in}}%
\pgfpathlineto{\pgfqpoint{3.097192in}{2.076958in}}%
\pgfpathlineto{\pgfqpoint{3.302411in}{1.491098in}}%
\pgfpathlineto{\pgfqpoint{3.405021in}{2.050692in}}%
\pgfpathlineto{\pgfqpoint{3.610240in}{2.101972in}}%
\pgfpathlineto{\pgfqpoint{3.712849in}{1.875648in}}%
\pgfpathlineto{\pgfqpoint{4.020678in}{1.933861in}}%
\pgfpathlineto{\pgfqpoint{4.431117in}{1.734821in}}%
\pgfusepath{stroke}%
\end{pgfscope}%
\begin{pgfscope}%
\pgfpathrectangle{\pgfqpoint{0.588387in}{0.521603in}}{\pgfqpoint{5.530657in}{2.010285in}}%
\pgfusepath{clip}%
\pgfsetrectcap%
\pgfsetroundjoin%
\pgfsetlinewidth{1.505625pt}%
\pgfsetstrokecolor{currentstroke3}%
\pgfsetdash{}{0pt}%
\pgfpathmoveto{\pgfqpoint{0.839781in}{0.810759in}}%
\pgfpathlineto{\pgfqpoint{0.942390in}{0.938362in}}%
\pgfpathlineto{\pgfqpoint{1.045000in}{1.095299in}}%
\pgfpathlineto{\pgfqpoint{1.147610in}{1.166272in}}%
\pgfpathlineto{\pgfqpoint{1.250219in}{1.263537in}}%
\pgfpathlineto{\pgfqpoint{1.352829in}{1.296334in}}%
\pgfpathlineto{\pgfqpoint{1.455438in}{1.353694in}}%
\pgfpathlineto{\pgfqpoint{1.558048in}{1.391227in}}%
\pgfpathlineto{\pgfqpoint{1.660658in}{1.412345in}}%
\pgfpathlineto{\pgfqpoint{1.763267in}{1.433973in}}%
\pgfpathlineto{\pgfqpoint{1.865877in}{1.467356in}}%
\pgfpathlineto{\pgfqpoint{1.968486in}{1.477370in}}%
\pgfpathlineto{\pgfqpoint{2.071096in}{1.493179in}}%
\pgfpathlineto{\pgfqpoint{2.173706in}{1.514669in}}%
\pgfpathlineto{\pgfqpoint{2.276315in}{1.523540in}}%
\pgfpathlineto{\pgfqpoint{2.378925in}{1.531073in}}%
\pgfpathlineto{\pgfqpoint{2.481534in}{1.553894in}}%
\pgfpathlineto{\pgfqpoint{2.584144in}{1.560535in}}%
\pgfpathlineto{\pgfqpoint{2.686753in}{1.570288in}}%
\pgfpathlineto{\pgfqpoint{2.789363in}{1.577712in}}%
\pgfpathlineto{\pgfqpoint{2.891973in}{1.611012in}}%
\pgfpathlineto{\pgfqpoint{2.994582in}{1.597778in}}%
\pgfpathlineto{\pgfqpoint{3.097192in}{1.607415in}}%
\pgfpathlineto{\pgfqpoint{3.199801in}{1.625505in}}%
\pgfpathlineto{\pgfqpoint{3.302411in}{1.628253in}}%
\pgfpathlineto{\pgfqpoint{3.405021in}{1.647042in}}%
\pgfpathlineto{\pgfqpoint{3.507630in}{1.645784in}}%
\pgfpathlineto{\pgfqpoint{3.610240in}{1.654753in}}%
\pgfpathlineto{\pgfqpoint{3.712849in}{1.674779in}}%
\pgfpathlineto{\pgfqpoint{3.815459in}{1.663524in}}%
\pgfpathlineto{\pgfqpoint{3.918069in}{1.675143in}}%
\pgfpathlineto{\pgfqpoint{4.020678in}{1.683452in}}%
\pgfpathlineto{\pgfqpoint{4.123288in}{1.682927in}}%
\pgfpathlineto{\pgfqpoint{4.225897in}{1.699705in}}%
\pgfpathlineto{\pgfqpoint{4.328507in}{1.701353in}}%
\pgfpathlineto{\pgfqpoint{4.431117in}{1.705513in}}%
\pgfpathlineto{\pgfqpoint{4.533726in}{1.707638in}}%
\pgfpathlineto{\pgfqpoint{4.636336in}{1.722502in}}%
\pgfpathlineto{\pgfqpoint{4.738945in}{1.726253in}}%
\pgfpathlineto{\pgfqpoint{4.841555in}{1.727092in}}%
\pgfpathlineto{\pgfqpoint{4.944165in}{1.742695in}}%
\pgfpathlineto{\pgfqpoint{5.046774in}{1.727115in}}%
\pgfpathlineto{\pgfqpoint{5.149384in}{1.739764in}}%
\pgfpathlineto{\pgfqpoint{5.251993in}{1.755357in}}%
\pgfpathlineto{\pgfqpoint{5.354603in}{1.747103in}}%
\pgfpathlineto{\pgfqpoint{5.457213in}{1.751421in}}%
\pgfpathlineto{\pgfqpoint{5.559822in}{1.761663in}}%
\pgfpathlineto{\pgfqpoint{5.662432in}{1.764929in}}%
\pgfpathlineto{\pgfqpoint{5.765041in}{1.775858in}}%
\pgfpathlineto{\pgfqpoint{5.867651in}{1.773748in}}%
\pgfusepath{stroke}%
\end{pgfscope}%
\begin{pgfscope}%
\pgfpathrectangle{\pgfqpoint{0.588387in}{0.521603in}}{\pgfqpoint{5.530657in}{2.010285in}}%
\pgfusepath{clip}%
\pgfsetrectcap%
\pgfsetroundjoin%
\pgfsetlinewidth{1.505625pt}%
\pgfsetstrokecolor{currentstroke4}%
\pgfsetdash{}{0pt}%
\pgfpathmoveto{\pgfqpoint{0.839781in}{0.810540in}}%
\pgfpathlineto{\pgfqpoint{0.942390in}{0.956638in}}%
\pgfpathlineto{\pgfqpoint{1.045000in}{1.094315in}}%
\pgfpathlineto{\pgfqpoint{1.147610in}{1.183282in}}%
\pgfpathlineto{\pgfqpoint{1.250219in}{1.272154in}}%
\pgfpathlineto{\pgfqpoint{1.352829in}{1.326006in}}%
\pgfpathlineto{\pgfqpoint{1.455438in}{1.362196in}}%
\pgfpathlineto{\pgfqpoint{1.558048in}{1.403950in}}%
\pgfpathlineto{\pgfqpoint{1.660658in}{1.430568in}}%
\pgfpathlineto{\pgfqpoint{1.763267in}{1.445628in}}%
\pgfpathlineto{\pgfqpoint{1.865877in}{1.487402in}}%
\pgfpathlineto{\pgfqpoint{1.968486in}{1.499282in}}%
\pgfpathlineto{\pgfqpoint{2.071096in}{1.508031in}}%
\pgfpathlineto{\pgfqpoint{2.173706in}{1.536086in}}%
\pgfpathlineto{\pgfqpoint{2.276315in}{1.541989in}}%
\pgfpathlineto{\pgfqpoint{2.378925in}{1.561813in}}%
\pgfpathlineto{\pgfqpoint{2.481534in}{1.573860in}}%
\pgfpathlineto{\pgfqpoint{2.584144in}{1.577828in}}%
\pgfpathlineto{\pgfqpoint{2.686753in}{1.590357in}}%
\pgfpathlineto{\pgfqpoint{2.789363in}{1.598731in}}%
\pgfpathlineto{\pgfqpoint{2.891973in}{1.631325in}}%
\pgfpathlineto{\pgfqpoint{2.994582in}{1.609051in}}%
\pgfpathlineto{\pgfqpoint{3.097192in}{1.630022in}}%
\pgfpathlineto{\pgfqpoint{3.199801in}{1.643614in}}%
\pgfpathlineto{\pgfqpoint{3.302411in}{1.646860in}}%
\pgfpathlineto{\pgfqpoint{3.405021in}{1.666715in}}%
\pgfpathlineto{\pgfqpoint{3.507630in}{1.666639in}}%
\pgfpathlineto{\pgfqpoint{3.610240in}{1.669551in}}%
\pgfpathlineto{\pgfqpoint{3.712849in}{1.690447in}}%
\pgfpathlineto{\pgfqpoint{3.815459in}{1.687877in}}%
\pgfpathlineto{\pgfqpoint{3.918069in}{1.701466in}}%
\pgfpathlineto{\pgfqpoint{4.020678in}{1.702334in}}%
\pgfpathlineto{\pgfqpoint{4.123288in}{1.697905in}}%
\pgfpathlineto{\pgfqpoint{4.225897in}{1.717169in}}%
\pgfpathlineto{\pgfqpoint{4.328507in}{1.715668in}}%
\pgfpathlineto{\pgfqpoint{4.431117in}{1.723292in}}%
\pgfpathlineto{\pgfqpoint{4.533726in}{1.722126in}}%
\pgfpathlineto{\pgfqpoint{4.636336in}{1.736918in}}%
\pgfpathlineto{\pgfqpoint{4.738945in}{1.741332in}}%
\pgfpathlineto{\pgfqpoint{4.841555in}{1.742109in}}%
\pgfpathlineto{\pgfqpoint{4.944165in}{1.756645in}}%
\pgfpathlineto{\pgfqpoint{5.046774in}{1.747727in}}%
\pgfpathlineto{\pgfqpoint{5.149384in}{1.755799in}}%
\pgfpathlineto{\pgfqpoint{5.251993in}{1.772137in}}%
\pgfpathlineto{\pgfqpoint{5.354603in}{1.763448in}}%
\pgfpathlineto{\pgfqpoint{5.457213in}{1.762161in}}%
\pgfpathlineto{\pgfqpoint{5.559822in}{1.774909in}}%
\pgfpathlineto{\pgfqpoint{5.662432in}{1.781404in}}%
\pgfpathlineto{\pgfqpoint{5.765041in}{1.788695in}}%
\pgfpathlineto{\pgfqpoint{5.867651in}{1.788139in}}%
\pgfusepath{stroke}%
\end{pgfscope}%
\begin{pgfscope}%
\pgfpathrectangle{\pgfqpoint{0.588387in}{0.521603in}}{\pgfqpoint{5.530657in}{2.010285in}}%
\pgfusepath{clip}%
\pgfsetrectcap%
\pgfsetroundjoin%
\pgfsetlinewidth{1.505625pt}%
\pgfsetstrokecolor{currentstroke5}%
\pgfsetdash{}{0pt}%
\pgfpathmoveto{\pgfqpoint{0.839781in}{0.866681in}}%
\pgfpathlineto{\pgfqpoint{0.942390in}{1.012243in}}%
\pgfpathlineto{\pgfqpoint{1.045000in}{1.132909in}}%
\pgfpathlineto{\pgfqpoint{1.147610in}{1.252539in}}%
\pgfpathlineto{\pgfqpoint{1.250219in}{1.352371in}}%
\pgfpathlineto{\pgfqpoint{1.352829in}{1.375275in}}%
\pgfpathlineto{\pgfqpoint{1.455438in}{1.416296in}}%
\pgfpathlineto{\pgfqpoint{1.558048in}{1.464166in}}%
\pgfpathlineto{\pgfqpoint{1.660658in}{1.480882in}}%
\pgfpathlineto{\pgfqpoint{1.763267in}{1.491653in}}%
\pgfpathlineto{\pgfqpoint{1.865877in}{1.525055in}}%
\pgfpathlineto{\pgfqpoint{1.968486in}{1.535060in}}%
\pgfpathlineto{\pgfqpoint{2.071096in}{1.545498in}}%
\pgfpathlineto{\pgfqpoint{2.173706in}{1.551806in}}%
\pgfpathlineto{\pgfqpoint{2.276315in}{1.579781in}}%
\pgfpathlineto{\pgfqpoint{2.378925in}{1.579231in}}%
\pgfpathlineto{\pgfqpoint{2.481534in}{1.588942in}}%
\pgfpathlineto{\pgfqpoint{2.584144in}{1.622671in}}%
\pgfpathlineto{\pgfqpoint{2.686753in}{1.620828in}}%
\pgfpathlineto{\pgfqpoint{2.789363in}{1.627277in}}%
\pgfpathlineto{\pgfqpoint{2.891973in}{1.645926in}}%
\pgfpathlineto{\pgfqpoint{2.994582in}{1.641926in}}%
\pgfpathlineto{\pgfqpoint{3.097192in}{1.636972in}}%
\pgfpathlineto{\pgfqpoint{3.199801in}{1.659516in}}%
\pgfpathlineto{\pgfqpoint{3.302411in}{1.661987in}}%
\pgfpathlineto{\pgfqpoint{3.405021in}{1.684088in}}%
\pgfpathlineto{\pgfqpoint{3.507630in}{1.693627in}}%
\pgfpathlineto{\pgfqpoint{3.610240in}{1.679771in}}%
\pgfpathlineto{\pgfqpoint{3.712849in}{1.696162in}}%
\pgfpathlineto{\pgfqpoint{3.815459in}{1.704880in}}%
\pgfpathlineto{\pgfqpoint{3.918069in}{1.703420in}}%
\pgfpathlineto{\pgfqpoint{4.020678in}{1.716199in}}%
\pgfpathlineto{\pgfqpoint{4.123288in}{1.716721in}}%
\pgfpathlineto{\pgfqpoint{4.225897in}{1.730523in}}%
\pgfpathlineto{\pgfqpoint{4.328507in}{1.742870in}}%
\pgfpathlineto{\pgfqpoint{4.431117in}{1.748289in}}%
\pgfpathlineto{\pgfqpoint{4.533726in}{1.748099in}}%
\pgfpathlineto{\pgfqpoint{4.636336in}{1.759076in}}%
\pgfpathlineto{\pgfqpoint{4.738945in}{1.746694in}}%
\pgfpathlineto{\pgfqpoint{4.841555in}{1.761732in}}%
\pgfpathlineto{\pgfqpoint{4.944165in}{1.784955in}}%
\pgfpathlineto{\pgfqpoint{5.046774in}{1.754582in}}%
\pgfpathlineto{\pgfqpoint{5.149384in}{1.763587in}}%
\pgfpathlineto{\pgfqpoint{5.251993in}{1.796429in}}%
\pgfpathlineto{\pgfqpoint{5.354603in}{1.783319in}}%
\pgfpathlineto{\pgfqpoint{5.457213in}{1.809358in}}%
\pgfpathlineto{\pgfqpoint{5.559822in}{1.789979in}}%
\pgfpathlineto{\pgfqpoint{5.662432in}{1.796858in}}%
\pgfpathlineto{\pgfqpoint{5.765041in}{1.801615in}}%
\pgfpathlineto{\pgfqpoint{5.867651in}{1.801825in}}%
\pgfusepath{stroke}%
\end{pgfscope}%
\begin{pgfscope}%
\pgfsetrectcap%
\pgfsetmiterjoin%
\pgfsetlinewidth{0.803000pt}%
\definecolor{currentstroke}{rgb}{0.000000,0.000000,0.000000}%
\pgfsetstrokecolor{currentstroke}%
\pgfsetdash{}{0pt}%
\pgfpathmoveto{\pgfqpoint{0.588387in}{0.521603in}}%
\pgfpathlineto{\pgfqpoint{0.588387in}{2.531888in}}%
\pgfusepath{stroke}%
\end{pgfscope}%
\begin{pgfscope}%
\pgfsetrectcap%
\pgfsetmiterjoin%
\pgfsetlinewidth{0.803000pt}%
\definecolor{currentstroke}{rgb}{0.000000,0.000000,0.000000}%
\pgfsetstrokecolor{currentstroke}%
\pgfsetdash{}{0pt}%
\pgfpathmoveto{\pgfqpoint{6.119045in}{0.521603in}}%
\pgfpathlineto{\pgfqpoint{6.119045in}{2.531888in}}%
\pgfusepath{stroke}%
\end{pgfscope}%
\begin{pgfscope}%
\pgfsetrectcap%
\pgfsetmiterjoin%
\pgfsetlinewidth{0.803000pt}%
\definecolor{currentstroke}{rgb}{0.000000,0.000000,0.000000}%
\pgfsetstrokecolor{currentstroke}%
\pgfsetdash{}{0pt}%
\pgfpathmoveto{\pgfqpoint{0.588387in}{0.521603in}}%
\pgfpathlineto{\pgfqpoint{6.119045in}{0.521603in}}%
\pgfusepath{stroke}%
\end{pgfscope}%
\begin{pgfscope}%
\pgfsetrectcap%
\pgfsetmiterjoin%
\pgfsetlinewidth{0.803000pt}%
\definecolor{currentstroke}{rgb}{0.000000,0.000000,0.000000}%
\pgfsetstrokecolor{currentstroke}%
\pgfsetdash{}{0pt}%
\pgfpathmoveto{\pgfqpoint{0.588387in}{2.531888in}}%
\pgfpathlineto{\pgfqpoint{6.119045in}{2.531888in}}%
\pgfusepath{stroke}%
\end{pgfscope}%
\begin{pgfscope}%
\definecolor{textcolor}{rgb}{0.000000,0.000000,0.000000}%
\pgfsetstrokecolor{textcolor}%
\pgfsetfillcolor{textcolor}%
\pgftext[x=3.353716in,y=2.615222in,,base]{\color{textcolor}{\rmfamily\fontsize{12.000000}{14.400000}\selectfont\catcode`\^=\active\def^{\ifmmode\sp\else\^{}\fi}\catcode`\%=\active\def%{\%}Mean}}%
\end{pgfscope}%
\begin{pgfscope}%
\pgfsetbuttcap%
\pgfsetmiterjoin%
\definecolor{currentfill}{rgb}{1.000000,1.000000,1.000000}%
\pgfsetfillcolor{currentfill}%
\pgfsetfillopacity{0.800000}%
\pgfsetlinewidth{1.003750pt}%
\definecolor{currentstroke}{rgb}{0.800000,0.800000,0.800000}%
\pgfsetstrokecolor{currentstroke}%
\pgfsetstrokeopacity{0.800000}%
\pgfsetdash{}{0pt}%
\pgfpathmoveto{\pgfqpoint{6.206545in}{1.507573in}}%
\pgfpathlineto{\pgfqpoint{8.259376in}{1.507573in}}%
\pgfpathquadraticcurveto{\pgfqpoint{8.284376in}{1.507573in}}{\pgfqpoint{8.284376in}{1.532573in}}%
\pgfpathlineto{\pgfqpoint{8.284376in}{2.444388in}}%
\pgfpathquadraticcurveto{\pgfqpoint{8.284376in}{2.469388in}}{\pgfqpoint{8.259376in}{2.469388in}}%
\pgfpathlineto{\pgfqpoint{6.206545in}{2.469388in}}%
\pgfpathquadraticcurveto{\pgfqpoint{6.181545in}{2.469388in}}{\pgfqpoint{6.181545in}{2.444388in}}%
\pgfpathlineto{\pgfqpoint{6.181545in}{1.532573in}}%
\pgfpathquadraticcurveto{\pgfqpoint{6.181545in}{1.507573in}}{\pgfqpoint{6.206545in}{1.507573in}}%
\pgfpathlineto{\pgfqpoint{6.206545in}{1.507573in}}%
\pgfpathclose%
\pgfusepath{stroke,fill}%
\end{pgfscope}%
\begin{pgfscope}%
\pgfsetrectcap%
\pgfsetroundjoin%
\pgfsetlinewidth{1.505625pt}%
\pgfsetstrokecolor{currentstroke3}%
\pgfsetdash{}{0pt}%
\pgfpathmoveto{\pgfqpoint{6.231545in}{2.368168in}}%
\pgfpathlineto{\pgfqpoint{6.356545in}{2.368168in}}%
\pgfpathlineto{\pgfqpoint{6.481545in}{2.368168in}}%
\pgfusepath{stroke}%
\end{pgfscope}%
\begin{pgfscope}%
\definecolor{textcolor}{rgb}{0.000000,0.000000,0.000000}%
\pgfsetstrokecolor{textcolor}%
\pgfsetfillcolor{textcolor}%
\pgftext[x=6.581545in,y=2.324418in,left,base]{\color{textcolor}{\rmfamily\fontsize{9.000000}{10.800000}\selectfont\catcode`\^=\active\def^{\ifmmode\sp\else\^{}\fi}\catcode`\%=\active\def%{\%}\Neighbors{} \& \MergeLinear{}}}%
\end{pgfscope}%
\begin{pgfscope}%
\pgfsetrectcap%
\pgfsetroundjoin%
\pgfsetlinewidth{1.505625pt}%
\pgfsetstrokecolor{currentstroke4}%
\pgfsetdash{}{0pt}%
\pgfpathmoveto{\pgfqpoint{6.231545in}{2.184696in}}%
\pgfpathlineto{\pgfqpoint{6.356545in}{2.184696in}}%
\pgfpathlineto{\pgfqpoint{6.481545in}{2.184696in}}%
\pgfusepath{stroke}%
\end{pgfscope}%
\begin{pgfscope}%
\definecolor{textcolor}{rgb}{0.000000,0.000000,0.000000}%
\pgfsetstrokecolor{textcolor}%
\pgfsetfillcolor{textcolor}%
\pgftext[x=6.581545in,y=2.140946in,left,base]{\color{textcolor}{\rmfamily\fontsize{9.000000}{10.800000}\selectfont\catcode`\^=\active\def^{\ifmmode\sp\else\^{}\fi}\catcode`\%=\active\def%{\%}\NeighborsDegree{} \& \MergeLinear{}}}%
\end{pgfscope}%
\begin{pgfscope}%
\pgfsetrectcap%
\pgfsetroundjoin%
\pgfsetlinewidth{1.505625pt}%
\pgfsetstrokecolor{currentstroke5}%
\pgfsetdash{}{0pt}%
\pgfpathmoveto{\pgfqpoint{6.231545in}{1.997746in}}%
\pgfpathlineto{\pgfqpoint{6.356545in}{1.997746in}}%
\pgfpathlineto{\pgfqpoint{6.481545in}{1.997746in}}%
\pgfusepath{stroke}%
\end{pgfscope}%
\begin{pgfscope}%
\definecolor{textcolor}{rgb}{0.000000,0.000000,0.000000}%
\pgfsetstrokecolor{textcolor}%
\pgfsetfillcolor{textcolor}%
\pgftext[x=6.581545in,y=1.953996in,left,base]{\color{textcolor}{\rmfamily\fontsize{9.000000}{10.800000}\selectfont\catcode`\^=\active\def^{\ifmmode\sp\else\^{}\fi}\catcode`\%=\active\def%{\%}\None{} \& \MergeLinear{}}}%
\end{pgfscope}%
\begin{pgfscope}%
\pgfsetrectcap%
\pgfsetroundjoin%
\pgfsetlinewidth{1.505625pt}%
\pgfsetstrokecolor{currentstroke1}%
\pgfsetdash{}{0pt}%
\pgfpathmoveto{\pgfqpoint{6.231545in}{1.814274in}}%
\pgfpathlineto{\pgfqpoint{6.356545in}{1.814274in}}%
\pgfpathlineto{\pgfqpoint{6.481545in}{1.814274in}}%
\pgfusepath{stroke}%
\end{pgfscope}%
\begin{pgfscope}%
\definecolor{textcolor}{rgb}{0.000000,0.000000,0.000000}%
\pgfsetstrokecolor{textcolor}%
\pgfsetfillcolor{textcolor}%
\pgftext[x=6.581545in,y=1.770524in,left,base]{\color{textcolor}{\rmfamily\fontsize{9.000000}{10.800000}\selectfont\catcode`\^=\active\def^{\ifmmode\sp\else\^{}\fi}\catcode`\%=\active\def%{\%}\Cuts{} \& \MergeLinear{}}}%
\end{pgfscope}%
\begin{pgfscope}%
\pgfsetrectcap%
\pgfsetroundjoin%
\pgfsetlinewidth{1.505625pt}%
\pgfsetstrokecolor{currentstroke2}%
\pgfsetdash{}{0pt}%
\pgfpathmoveto{\pgfqpoint{6.231545in}{1.630803in}}%
\pgfpathlineto{\pgfqpoint{6.356545in}{1.630803in}}%
\pgfpathlineto{\pgfqpoint{6.481545in}{1.630803in}}%
\pgfusepath{stroke}%
\end{pgfscope}%
\begin{pgfscope}%
\definecolor{textcolor}{rgb}{0.000000,0.000000,0.000000}%
\pgfsetstrokecolor{textcolor}%
\pgfsetfillcolor{textcolor}%
\pgftext[x=6.581545in,y=1.587053in,left,base]{\color{textcolor}{\rmfamily\fontsize{9.000000}{10.800000}\selectfont\catcode`\^=\active\def^{\ifmmode\sp\else\^{}\fi}\catcode`\%=\active\def%{\%}\KernighanLin{} \& \MergeLinear{}}}%
\end{pgfscope}%
\end{pgfpicture}%
\makeatother%
\endgroup%
}
	\caption[Mean runtime for minimally rigid graphs (some).]{
		Mean running time (ms) to find all NAC-colorings for minimally rigid graphs with failing splitting strategies.}%
	\label{fig:graph_mimimally_rigid_failing_split_first_runtime}
\end{figure}
\begin{figure}[p]
	\centering
	\scalebox{0.5}{%% Creator: Matplotlib, PGF backend
%%
%% To include the figure in your LaTeX document, write
%%   \input{<filename>.pgf}
%%
%% Make sure the required packages are loaded in your preamble
%%   \usepackage{pgf}
%%
%% Also ensure that all the required font packages are loaded; for instance,
%% the lmodern package is sometimes necessary when using math font.
%%   \usepackage{lmodern}
%%
%% Figures using additional raster images can only be included by \input if
%% they are in the same directory as the main LaTeX file. For loading figures
%% from other directories you can use the `import` package
%%   \usepackage{import}
%%
%% and then include the figures with
%%   \import{<path to file>}{<filename>.pgf}
%%
%% Matplotlib used the following preamble
%%   \def\mathdefault#1{#1}
%%   \everymath=\expandafter{\the\everymath\displaystyle}
%%   \IfFileExists{scrextend.sty}{
%%     \usepackage[fontsize=10.000000pt]{scrextend}
%%   }{
%%     \renewcommand{\normalsize}{\fontsize{10.000000}{12.000000}\selectfont}
%%     \normalsize
%%   }
%%   
%%   \ifdefined\pdftexversion\else  % non-pdftex case.
%%     \usepackage{fontspec}
%%     \setmainfont{DejaVuSans.ttf}[Path=\detokenize{/home/petr/Projects/PyRigi/.venv/lib/python3.12/site-packages/matplotlib/mpl-data/fonts/ttf/}]
%%     \setsansfont{DejaVuSans.ttf}[Path=\detokenize{/home/petr/Projects/PyRigi/.venv/lib/python3.12/site-packages/matplotlib/mpl-data/fonts/ttf/}]
%%     \setmonofont{DejaVuSansMono.ttf}[Path=\detokenize{/home/petr/Projects/PyRigi/.venv/lib/python3.12/site-packages/matplotlib/mpl-data/fonts/ttf/}]
%%   \fi
%%   \makeatletter\@ifpackageloaded{under\Score{}}{}{\usepackage[strings]{under\Score{}}}\makeatother
%%
\begingroup%
\makeatletter%
\begin{pgfpicture}%
\pgfpathrectangle{\pgfpointorigin}{\pgfqpoint{8.384376in}{2.841849in}}%
\pgfusepath{use as bounding box, clip}%
\begin{pgfscope}%
\pgfsetbuttcap%
\pgfsetmiterjoin%
\definecolor{currentfill}{rgb}{1.000000,1.000000,1.000000}%
\pgfsetfillcolor{currentfill}%
\pgfsetlinewidth{0.000000pt}%
\definecolor{currentstroke}{rgb}{1.000000,1.000000,1.000000}%
\pgfsetstrokecolor{currentstroke}%
\pgfsetdash{}{0pt}%
\pgfpathmoveto{\pgfqpoint{0.000000in}{0.000000in}}%
\pgfpathlineto{\pgfqpoint{8.384376in}{0.000000in}}%
\pgfpathlineto{\pgfqpoint{8.384376in}{2.841849in}}%
\pgfpathlineto{\pgfqpoint{0.000000in}{2.841849in}}%
\pgfpathlineto{\pgfqpoint{0.000000in}{0.000000in}}%
\pgfpathclose%
\pgfusepath{fill}%
\end{pgfscope}%
\begin{pgfscope}%
\pgfsetbuttcap%
\pgfsetmiterjoin%
\definecolor{currentfill}{rgb}{1.000000,1.000000,1.000000}%
\pgfsetfillcolor{currentfill}%
\pgfsetlinewidth{0.000000pt}%
\definecolor{currentstroke}{rgb}{0.000000,0.000000,0.000000}%
\pgfsetstrokecolor{currentstroke}%
\pgfsetstrokeopacity{0.000000}%
\pgfsetdash{}{0pt}%
\pgfpathmoveto{\pgfqpoint{0.588387in}{0.521603in}}%
\pgfpathlineto{\pgfqpoint{5.903102in}{0.521603in}}%
\pgfpathlineto{\pgfqpoint{5.903102in}{2.531888in}}%
\pgfpathlineto{\pgfqpoint{0.588387in}{2.531888in}}%
\pgfpathlineto{\pgfqpoint{0.588387in}{0.521603in}}%
\pgfpathclose%
\pgfusepath{fill}%
\end{pgfscope}%
\begin{pgfscope}%
\pgfsetbuttcap%
\pgfsetroundjoin%
\definecolor{currentfill}{rgb}{0.000000,0.000000,0.000000}%
\pgfsetfillcolor{currentfill}%
\pgfsetlinewidth{0.803000pt}%
\definecolor{currentstroke}{rgb}{0.000000,0.000000,0.000000}%
\pgfsetstrokecolor{currentstroke}%
\pgfsetdash{}{0pt}%
\pgfsys@defobject{currentmarker}{\pgfqpoint{0.000000in}{-0.048611in}}{\pgfqpoint{0.000000in}{0.000000in}}{%
\pgfpathmoveto{\pgfqpoint{0.000000in}{0.000000in}}%
\pgfpathlineto{\pgfqpoint{0.000000in}{-0.048611in}}%
\pgfusepath{stroke,fill}%
}%
\begin{pgfscope}%
\pgfsys@transformshift{1.027172in}{0.521603in}%
\pgfsys@useobject{currentmarker}{}%
\end{pgfscope}%
\end{pgfscope}%
\begin{pgfscope}%
\definecolor{textcolor}{rgb}{0.000000,0.000000,0.000000}%
\pgfsetstrokecolor{textcolor}%
\pgfsetfillcolor{textcolor}%
\pgftext[x=1.027172in,y=0.424381in,,top]{\color{textcolor}{\rmfamily\fontsize{10.000000}{12.000000}\selectfont\catcode`\^=\active\def^{\ifmmode\sp\else\^{}\fi}\catcode`\%=\active\def%{\%}$\mathdefault{12}$}}%
\end{pgfscope}%
\begin{pgfscope}%
\pgfsetbuttcap%
\pgfsetroundjoin%
\definecolor{currentfill}{rgb}{0.000000,0.000000,0.000000}%
\pgfsetfillcolor{currentfill}%
\pgfsetlinewidth{0.803000pt}%
\definecolor{currentstroke}{rgb}{0.000000,0.000000,0.000000}%
\pgfsetstrokecolor{currentstroke}%
\pgfsetdash{}{0pt}%
\pgfsys@defobject{currentmarker}{\pgfqpoint{0.000000in}{-0.048611in}}{\pgfqpoint{0.000000in}{0.000000in}}{%
\pgfpathmoveto{\pgfqpoint{0.000000in}{0.000000in}}%
\pgfpathlineto{\pgfqpoint{0.000000in}{-0.048611in}}%
\pgfusepath{stroke,fill}%
}%
\begin{pgfscope}%
\pgfsys@transformshift{1.618791in}{0.521603in}%
\pgfsys@useobject{currentmarker}{}%
\end{pgfscope}%
\end{pgfscope}%
\begin{pgfscope}%
\definecolor{textcolor}{rgb}{0.000000,0.000000,0.000000}%
\pgfsetstrokecolor{textcolor}%
\pgfsetfillcolor{textcolor}%
\pgftext[x=1.618791in,y=0.424381in,,top]{\color{textcolor}{\rmfamily\fontsize{10.000000}{12.000000}\selectfont\catcode`\^=\active\def^{\ifmmode\sp\else\^{}\fi}\catcode`\%=\active\def%{\%}$\mathdefault{18}$}}%
\end{pgfscope}%
\begin{pgfscope}%
\pgfsetbuttcap%
\pgfsetroundjoin%
\definecolor{currentfill}{rgb}{0.000000,0.000000,0.000000}%
\pgfsetfillcolor{currentfill}%
\pgfsetlinewidth{0.803000pt}%
\definecolor{currentstroke}{rgb}{0.000000,0.000000,0.000000}%
\pgfsetstrokecolor{currentstroke}%
\pgfsetdash{}{0pt}%
\pgfsys@defobject{currentmarker}{\pgfqpoint{0.000000in}{-0.048611in}}{\pgfqpoint{0.000000in}{0.000000in}}{%
\pgfpathmoveto{\pgfqpoint{0.000000in}{0.000000in}}%
\pgfpathlineto{\pgfqpoint{0.000000in}{-0.048611in}}%
\pgfusepath{stroke,fill}%
}%
\begin{pgfscope}%
\pgfsys@transformshift{2.210411in}{0.521603in}%
\pgfsys@useobject{currentmarker}{}%
\end{pgfscope}%
\end{pgfscope}%
\begin{pgfscope}%
\definecolor{textcolor}{rgb}{0.000000,0.000000,0.000000}%
\pgfsetstrokecolor{textcolor}%
\pgfsetfillcolor{textcolor}%
\pgftext[x=2.210411in,y=0.424381in,,top]{\color{textcolor}{\rmfamily\fontsize{10.000000}{12.000000}\selectfont\catcode`\^=\active\def^{\ifmmode\sp\else\^{}\fi}\catcode`\%=\active\def%{\%}$\mathdefault{24}$}}%
\end{pgfscope}%
\begin{pgfscope}%
\pgfsetbuttcap%
\pgfsetroundjoin%
\definecolor{currentfill}{rgb}{0.000000,0.000000,0.000000}%
\pgfsetfillcolor{currentfill}%
\pgfsetlinewidth{0.803000pt}%
\definecolor{currentstroke}{rgb}{0.000000,0.000000,0.000000}%
\pgfsetstrokecolor{currentstroke}%
\pgfsetdash{}{0pt}%
\pgfsys@defobject{currentmarker}{\pgfqpoint{0.000000in}{-0.048611in}}{\pgfqpoint{0.000000in}{0.000000in}}{%
\pgfpathmoveto{\pgfqpoint{0.000000in}{0.000000in}}%
\pgfpathlineto{\pgfqpoint{0.000000in}{-0.048611in}}%
\pgfusepath{stroke,fill}%
}%
\begin{pgfscope}%
\pgfsys@transformshift{2.802030in}{0.521603in}%
\pgfsys@useobject{currentmarker}{}%
\end{pgfscope}%
\end{pgfscope}%
\begin{pgfscope}%
\definecolor{textcolor}{rgb}{0.000000,0.000000,0.000000}%
\pgfsetstrokecolor{textcolor}%
\pgfsetfillcolor{textcolor}%
\pgftext[x=2.802030in,y=0.424381in,,top]{\color{textcolor}{\rmfamily\fontsize{10.000000}{12.000000}\selectfont\catcode`\^=\active\def^{\ifmmode\sp\else\^{}\fi}\catcode`\%=\active\def%{\%}$\mathdefault{30}$}}%
\end{pgfscope}%
\begin{pgfscope}%
\pgfsetbuttcap%
\pgfsetroundjoin%
\definecolor{currentfill}{rgb}{0.000000,0.000000,0.000000}%
\pgfsetfillcolor{currentfill}%
\pgfsetlinewidth{0.803000pt}%
\definecolor{currentstroke}{rgb}{0.000000,0.000000,0.000000}%
\pgfsetstrokecolor{currentstroke}%
\pgfsetdash{}{0pt}%
\pgfsys@defobject{currentmarker}{\pgfqpoint{0.000000in}{-0.048611in}}{\pgfqpoint{0.000000in}{0.000000in}}{%
\pgfpathmoveto{\pgfqpoint{0.000000in}{0.000000in}}%
\pgfpathlineto{\pgfqpoint{0.000000in}{-0.048611in}}%
\pgfusepath{stroke,fill}%
}%
\begin{pgfscope}%
\pgfsys@transformshift{3.393649in}{0.521603in}%
\pgfsys@useobject{currentmarker}{}%
\end{pgfscope}%
\end{pgfscope}%
\begin{pgfscope}%
\definecolor{textcolor}{rgb}{0.000000,0.000000,0.000000}%
\pgfsetstrokecolor{textcolor}%
\pgfsetfillcolor{textcolor}%
\pgftext[x=3.393649in,y=0.424381in,,top]{\color{textcolor}{\rmfamily\fontsize{10.000000}{12.000000}\selectfont\catcode`\^=\active\def^{\ifmmode\sp\else\^{}\fi}\catcode`\%=\active\def%{\%}$\mathdefault{36}$}}%
\end{pgfscope}%
\begin{pgfscope}%
\pgfsetbuttcap%
\pgfsetroundjoin%
\definecolor{currentfill}{rgb}{0.000000,0.000000,0.000000}%
\pgfsetfillcolor{currentfill}%
\pgfsetlinewidth{0.803000pt}%
\definecolor{currentstroke}{rgb}{0.000000,0.000000,0.000000}%
\pgfsetstrokecolor{currentstroke}%
\pgfsetdash{}{0pt}%
\pgfsys@defobject{currentmarker}{\pgfqpoint{0.000000in}{-0.048611in}}{\pgfqpoint{0.000000in}{0.000000in}}{%
\pgfpathmoveto{\pgfqpoint{0.000000in}{0.000000in}}%
\pgfpathlineto{\pgfqpoint{0.000000in}{-0.048611in}}%
\pgfusepath{stroke,fill}%
}%
\begin{pgfscope}%
\pgfsys@transformshift{3.985269in}{0.521603in}%
\pgfsys@useobject{currentmarker}{}%
\end{pgfscope}%
\end{pgfscope}%
\begin{pgfscope}%
\definecolor{textcolor}{rgb}{0.000000,0.000000,0.000000}%
\pgfsetstrokecolor{textcolor}%
\pgfsetfillcolor{textcolor}%
\pgftext[x=3.985269in,y=0.424381in,,top]{\color{textcolor}{\rmfamily\fontsize{10.000000}{12.000000}\selectfont\catcode`\^=\active\def^{\ifmmode\sp\else\^{}\fi}\catcode`\%=\active\def%{\%}$\mathdefault{42}$}}%
\end{pgfscope}%
\begin{pgfscope}%
\pgfsetbuttcap%
\pgfsetroundjoin%
\definecolor{currentfill}{rgb}{0.000000,0.000000,0.000000}%
\pgfsetfillcolor{currentfill}%
\pgfsetlinewidth{0.803000pt}%
\definecolor{currentstroke}{rgb}{0.000000,0.000000,0.000000}%
\pgfsetstrokecolor{currentstroke}%
\pgfsetdash{}{0pt}%
\pgfsys@defobject{currentmarker}{\pgfqpoint{0.000000in}{-0.048611in}}{\pgfqpoint{0.000000in}{0.000000in}}{%
\pgfpathmoveto{\pgfqpoint{0.000000in}{0.000000in}}%
\pgfpathlineto{\pgfqpoint{0.000000in}{-0.048611in}}%
\pgfusepath{stroke,fill}%
}%
\begin{pgfscope}%
\pgfsys@transformshift{4.576888in}{0.521603in}%
\pgfsys@useobject{currentmarker}{}%
\end{pgfscope}%
\end{pgfscope}%
\begin{pgfscope}%
\definecolor{textcolor}{rgb}{0.000000,0.000000,0.000000}%
\pgfsetstrokecolor{textcolor}%
\pgfsetfillcolor{textcolor}%
\pgftext[x=4.576888in,y=0.424381in,,top]{\color{textcolor}{\rmfamily\fontsize{10.000000}{12.000000}\selectfont\catcode`\^=\active\def^{\ifmmode\sp\else\^{}\fi}\catcode`\%=\active\def%{\%}$\mathdefault{48}$}}%
\end{pgfscope}%
\begin{pgfscope}%
\pgfsetbuttcap%
\pgfsetroundjoin%
\definecolor{currentfill}{rgb}{0.000000,0.000000,0.000000}%
\pgfsetfillcolor{currentfill}%
\pgfsetlinewidth{0.803000pt}%
\definecolor{currentstroke}{rgb}{0.000000,0.000000,0.000000}%
\pgfsetstrokecolor{currentstroke}%
\pgfsetdash{}{0pt}%
\pgfsys@defobject{currentmarker}{\pgfqpoint{0.000000in}{-0.048611in}}{\pgfqpoint{0.000000in}{0.000000in}}{%
\pgfpathmoveto{\pgfqpoint{0.000000in}{0.000000in}}%
\pgfpathlineto{\pgfqpoint{0.000000in}{-0.048611in}}%
\pgfusepath{stroke,fill}%
}%
\begin{pgfscope}%
\pgfsys@transformshift{5.168508in}{0.521603in}%
\pgfsys@useobject{currentmarker}{}%
\end{pgfscope}%
\end{pgfscope}%
\begin{pgfscope}%
\definecolor{textcolor}{rgb}{0.000000,0.000000,0.000000}%
\pgfsetstrokecolor{textcolor}%
\pgfsetfillcolor{textcolor}%
\pgftext[x=5.168508in,y=0.424381in,,top]{\color{textcolor}{\rmfamily\fontsize{10.000000}{12.000000}\selectfont\catcode`\^=\active\def^{\ifmmode\sp\else\^{}\fi}\catcode`\%=\active\def%{\%}$\mathdefault{54}$}}%
\end{pgfscope}%
\begin{pgfscope}%
\pgfsetbuttcap%
\pgfsetroundjoin%
\definecolor{currentfill}{rgb}{0.000000,0.000000,0.000000}%
\pgfsetfillcolor{currentfill}%
\pgfsetlinewidth{0.803000pt}%
\definecolor{currentstroke}{rgb}{0.000000,0.000000,0.000000}%
\pgfsetstrokecolor{currentstroke}%
\pgfsetdash{}{0pt}%
\pgfsys@defobject{currentmarker}{\pgfqpoint{0.000000in}{-0.048611in}}{\pgfqpoint{0.000000in}{0.000000in}}{%
\pgfpathmoveto{\pgfqpoint{0.000000in}{0.000000in}}%
\pgfpathlineto{\pgfqpoint{0.000000in}{-0.048611in}}%
\pgfusepath{stroke,fill}%
}%
\begin{pgfscope}%
\pgfsys@transformshift{5.760127in}{0.521603in}%
\pgfsys@useobject{currentmarker}{}%
\end{pgfscope}%
\end{pgfscope}%
\begin{pgfscope}%
\definecolor{textcolor}{rgb}{0.000000,0.000000,0.000000}%
\pgfsetstrokecolor{textcolor}%
\pgfsetfillcolor{textcolor}%
\pgftext[x=5.760127in,y=0.424381in,,top]{\color{textcolor}{\rmfamily\fontsize{10.000000}{12.000000}\selectfont\catcode`\^=\active\def^{\ifmmode\sp\else\^{}\fi}\catcode`\%=\active\def%{\%}$\mathdefault{60}$}}%
\end{pgfscope}%
\begin{pgfscope}%
\definecolor{textcolor}{rgb}{0.000000,0.000000,0.000000}%
\pgfsetstrokecolor{textcolor}%
\pgfsetfillcolor{textcolor}%
\pgftext[x=3.245745in,y=0.234413in,,top]{\color{textcolor}{\rmfamily\fontsize{10.000000}{12.000000}\selectfont\catcode`\^=\active\def^{\ifmmode\sp\else\^{}\fi}\catcode`\%=\active\def%{\%}Vertices}}%
\end{pgfscope}%
\begin{pgfscope}%
\pgfsetbuttcap%
\pgfsetroundjoin%
\definecolor{currentfill}{rgb}{0.000000,0.000000,0.000000}%
\pgfsetfillcolor{currentfill}%
\pgfsetlinewidth{0.803000pt}%
\definecolor{currentstroke}{rgb}{0.000000,0.000000,0.000000}%
\pgfsetstrokecolor{currentstroke}%
\pgfsetdash{}{0pt}%
\pgfsys@defobject{currentmarker}{\pgfqpoint{-0.048611in}{0.000000in}}{\pgfqpoint{-0.000000in}{0.000000in}}{%
\pgfpathmoveto{\pgfqpoint{-0.000000in}{0.000000in}}%
\pgfpathlineto{\pgfqpoint{-0.048611in}{0.000000in}}%
\pgfusepath{stroke,fill}%
}%
\begin{pgfscope}%
\pgfsys@transformshift{0.588387in}{0.617445in}%
\pgfsys@useobject{currentmarker}{}%
\end{pgfscope}%
\end{pgfscope}%
\begin{pgfscope}%
\definecolor{textcolor}{rgb}{0.000000,0.000000,0.000000}%
\pgfsetstrokecolor{textcolor}%
\pgfsetfillcolor{textcolor}%
\pgftext[x=0.289968in, y=0.564684in, left, base]{\color{textcolor}{\rmfamily\fontsize{10.000000}{12.000000}\selectfont\catcode`\^=\active\def^{\ifmmode\sp\else\^{}\fi}\catcode`\%=\active\def%{\%}$\mathdefault{10^{1}}$}}%
\end{pgfscope}%
\begin{pgfscope}%
\pgfsetbuttcap%
\pgfsetroundjoin%
\definecolor{currentfill}{rgb}{0.000000,0.000000,0.000000}%
\pgfsetfillcolor{currentfill}%
\pgfsetlinewidth{0.803000pt}%
\definecolor{currentstroke}{rgb}{0.000000,0.000000,0.000000}%
\pgfsetstrokecolor{currentstroke}%
\pgfsetdash{}{0pt}%
\pgfsys@defobject{currentmarker}{\pgfqpoint{-0.048611in}{0.000000in}}{\pgfqpoint{-0.000000in}{0.000000in}}{%
\pgfpathmoveto{\pgfqpoint{-0.000000in}{0.000000in}}%
\pgfpathlineto{\pgfqpoint{-0.048611in}{0.000000in}}%
\pgfusepath{stroke,fill}%
}%
\begin{pgfscope}%
\pgfsys@transformshift{0.588387in}{1.073413in}%
\pgfsys@useobject{currentmarker}{}%
\end{pgfscope}%
\end{pgfscope}%
\begin{pgfscope}%
\definecolor{textcolor}{rgb}{0.000000,0.000000,0.000000}%
\pgfsetstrokecolor{textcolor}%
\pgfsetfillcolor{textcolor}%
\pgftext[x=0.289968in, y=1.020651in, left, base]{\color{textcolor}{\rmfamily\fontsize{10.000000}{12.000000}\selectfont\catcode`\^=\active\def^{\ifmmode\sp\else\^{}\fi}\catcode`\%=\active\def%{\%}$\mathdefault{10^{2}}$}}%
\end{pgfscope}%
\begin{pgfscope}%
\pgfsetbuttcap%
\pgfsetroundjoin%
\definecolor{currentfill}{rgb}{0.000000,0.000000,0.000000}%
\pgfsetfillcolor{currentfill}%
\pgfsetlinewidth{0.803000pt}%
\definecolor{currentstroke}{rgb}{0.000000,0.000000,0.000000}%
\pgfsetstrokecolor{currentstroke}%
\pgfsetdash{}{0pt}%
\pgfsys@defobject{currentmarker}{\pgfqpoint{-0.048611in}{0.000000in}}{\pgfqpoint{-0.000000in}{0.000000in}}{%
\pgfpathmoveto{\pgfqpoint{-0.000000in}{0.000000in}}%
\pgfpathlineto{\pgfqpoint{-0.048611in}{0.000000in}}%
\pgfusepath{stroke,fill}%
}%
\begin{pgfscope}%
\pgfsys@transformshift{0.588387in}{1.529380in}%
\pgfsys@useobject{currentmarker}{}%
\end{pgfscope}%
\end{pgfscope}%
\begin{pgfscope}%
\definecolor{textcolor}{rgb}{0.000000,0.000000,0.000000}%
\pgfsetstrokecolor{textcolor}%
\pgfsetfillcolor{textcolor}%
\pgftext[x=0.289968in, y=1.476619in, left, base]{\color{textcolor}{\rmfamily\fontsize{10.000000}{12.000000}\selectfont\catcode`\^=\active\def^{\ifmmode\sp\else\^{}\fi}\catcode`\%=\active\def%{\%}$\mathdefault{10^{3}}$}}%
\end{pgfscope}%
\begin{pgfscope}%
\pgfsetbuttcap%
\pgfsetroundjoin%
\definecolor{currentfill}{rgb}{0.000000,0.000000,0.000000}%
\pgfsetfillcolor{currentfill}%
\pgfsetlinewidth{0.803000pt}%
\definecolor{currentstroke}{rgb}{0.000000,0.000000,0.000000}%
\pgfsetstrokecolor{currentstroke}%
\pgfsetdash{}{0pt}%
\pgfsys@defobject{currentmarker}{\pgfqpoint{-0.048611in}{0.000000in}}{\pgfqpoint{-0.000000in}{0.000000in}}{%
\pgfpathmoveto{\pgfqpoint{-0.000000in}{0.000000in}}%
\pgfpathlineto{\pgfqpoint{-0.048611in}{0.000000in}}%
\pgfusepath{stroke,fill}%
}%
\begin{pgfscope}%
\pgfsys@transformshift{0.588387in}{1.985347in}%
\pgfsys@useobject{currentmarker}{}%
\end{pgfscope}%
\end{pgfscope}%
\begin{pgfscope}%
\definecolor{textcolor}{rgb}{0.000000,0.000000,0.000000}%
\pgfsetstrokecolor{textcolor}%
\pgfsetfillcolor{textcolor}%
\pgftext[x=0.289968in, y=1.932586in, left, base]{\color{textcolor}{\rmfamily\fontsize{10.000000}{12.000000}\selectfont\catcode`\^=\active\def^{\ifmmode\sp\else\^{}\fi}\catcode`\%=\active\def%{\%}$\mathdefault{10^{4}}$}}%
\end{pgfscope}%
\begin{pgfscope}%
\pgfsetbuttcap%
\pgfsetroundjoin%
\definecolor{currentfill}{rgb}{0.000000,0.000000,0.000000}%
\pgfsetfillcolor{currentfill}%
\pgfsetlinewidth{0.803000pt}%
\definecolor{currentstroke}{rgb}{0.000000,0.000000,0.000000}%
\pgfsetstrokecolor{currentstroke}%
\pgfsetdash{}{0pt}%
\pgfsys@defobject{currentmarker}{\pgfqpoint{-0.048611in}{0.000000in}}{\pgfqpoint{-0.000000in}{0.000000in}}{%
\pgfpathmoveto{\pgfqpoint{-0.000000in}{0.000000in}}%
\pgfpathlineto{\pgfqpoint{-0.048611in}{0.000000in}}%
\pgfusepath{stroke,fill}%
}%
\begin{pgfscope}%
\pgfsys@transformshift{0.588387in}{2.441315in}%
\pgfsys@useobject{currentmarker}{}%
\end{pgfscope}%
\end{pgfscope}%
\begin{pgfscope}%
\definecolor{textcolor}{rgb}{0.000000,0.000000,0.000000}%
\pgfsetstrokecolor{textcolor}%
\pgfsetfillcolor{textcolor}%
\pgftext[x=0.289968in, y=2.388553in, left, base]{\color{textcolor}{\rmfamily\fontsize{10.000000}{12.000000}\selectfont\catcode`\^=\active\def^{\ifmmode\sp\else\^{}\fi}\catcode`\%=\active\def%{\%}$\mathdefault{10^{5}}$}}%
\end{pgfscope}%
\begin{pgfscope}%
\pgfsetbuttcap%
\pgfsetroundjoin%
\definecolor{currentfill}{rgb}{0.000000,0.000000,0.000000}%
\pgfsetfillcolor{currentfill}%
\pgfsetlinewidth{0.602250pt}%
\definecolor{currentstroke}{rgb}{0.000000,0.000000,0.000000}%
\pgfsetstrokecolor{currentstroke}%
\pgfsetdash{}{0pt}%
\pgfsys@defobject{currentmarker}{\pgfqpoint{-0.027778in}{0.000000in}}{\pgfqpoint{-0.000000in}{0.000000in}}{%
\pgfpathmoveto{\pgfqpoint{-0.000000in}{0.000000in}}%
\pgfpathlineto{\pgfqpoint{-0.027778in}{0.000000in}}%
\pgfusepath{stroke,fill}%
}%
\begin{pgfscope}%
\pgfsys@transformshift{0.588387in}{0.546815in}%
\pgfsys@useobject{currentmarker}{}%
\end{pgfscope}%
\end{pgfscope}%
\begin{pgfscope}%
\pgfsetbuttcap%
\pgfsetroundjoin%
\definecolor{currentfill}{rgb}{0.000000,0.000000,0.000000}%
\pgfsetfillcolor{currentfill}%
\pgfsetlinewidth{0.602250pt}%
\definecolor{currentstroke}{rgb}{0.000000,0.000000,0.000000}%
\pgfsetstrokecolor{currentstroke}%
\pgfsetdash{}{0pt}%
\pgfsys@defobject{currentmarker}{\pgfqpoint{-0.027778in}{0.000000in}}{\pgfqpoint{-0.000000in}{0.000000in}}{%
\pgfpathmoveto{\pgfqpoint{-0.000000in}{0.000000in}}%
\pgfpathlineto{\pgfqpoint{-0.027778in}{0.000000in}}%
\pgfusepath{stroke,fill}%
}%
\begin{pgfscope}%
\pgfsys@transformshift{0.588387in}{0.573257in}%
\pgfsys@useobject{currentmarker}{}%
\end{pgfscope}%
\end{pgfscope}%
\begin{pgfscope}%
\pgfsetbuttcap%
\pgfsetroundjoin%
\definecolor{currentfill}{rgb}{0.000000,0.000000,0.000000}%
\pgfsetfillcolor{currentfill}%
\pgfsetlinewidth{0.602250pt}%
\definecolor{currentstroke}{rgb}{0.000000,0.000000,0.000000}%
\pgfsetstrokecolor{currentstroke}%
\pgfsetdash{}{0pt}%
\pgfsys@defobject{currentmarker}{\pgfqpoint{-0.027778in}{0.000000in}}{\pgfqpoint{-0.000000in}{0.000000in}}{%
\pgfpathmoveto{\pgfqpoint{-0.000000in}{0.000000in}}%
\pgfpathlineto{\pgfqpoint{-0.027778in}{0.000000in}}%
\pgfusepath{stroke,fill}%
}%
\begin{pgfscope}%
\pgfsys@transformshift{0.588387in}{0.596581in}%
\pgfsys@useobject{currentmarker}{}%
\end{pgfscope}%
\end{pgfscope}%
\begin{pgfscope}%
\pgfsetbuttcap%
\pgfsetroundjoin%
\definecolor{currentfill}{rgb}{0.000000,0.000000,0.000000}%
\pgfsetfillcolor{currentfill}%
\pgfsetlinewidth{0.602250pt}%
\definecolor{currentstroke}{rgb}{0.000000,0.000000,0.000000}%
\pgfsetstrokecolor{currentstroke}%
\pgfsetdash{}{0pt}%
\pgfsys@defobject{currentmarker}{\pgfqpoint{-0.027778in}{0.000000in}}{\pgfqpoint{-0.000000in}{0.000000in}}{%
\pgfpathmoveto{\pgfqpoint{-0.000000in}{0.000000in}}%
\pgfpathlineto{\pgfqpoint{-0.027778in}{0.000000in}}%
\pgfusepath{stroke,fill}%
}%
\begin{pgfscope}%
\pgfsys@transformshift{0.588387in}{0.754705in}%
\pgfsys@useobject{currentmarker}{}%
\end{pgfscope}%
\end{pgfscope}%
\begin{pgfscope}%
\pgfsetbuttcap%
\pgfsetroundjoin%
\definecolor{currentfill}{rgb}{0.000000,0.000000,0.000000}%
\pgfsetfillcolor{currentfill}%
\pgfsetlinewidth{0.602250pt}%
\definecolor{currentstroke}{rgb}{0.000000,0.000000,0.000000}%
\pgfsetstrokecolor{currentstroke}%
\pgfsetdash{}{0pt}%
\pgfsys@defobject{currentmarker}{\pgfqpoint{-0.027778in}{0.000000in}}{\pgfqpoint{-0.000000in}{0.000000in}}{%
\pgfpathmoveto{\pgfqpoint{-0.000000in}{0.000000in}}%
\pgfpathlineto{\pgfqpoint{-0.027778in}{0.000000in}}%
\pgfusepath{stroke,fill}%
}%
\begin{pgfscope}%
\pgfsys@transformshift{0.588387in}{0.834997in}%
\pgfsys@useobject{currentmarker}{}%
\end{pgfscope}%
\end{pgfscope}%
\begin{pgfscope}%
\pgfsetbuttcap%
\pgfsetroundjoin%
\definecolor{currentfill}{rgb}{0.000000,0.000000,0.000000}%
\pgfsetfillcolor{currentfill}%
\pgfsetlinewidth{0.602250pt}%
\definecolor{currentstroke}{rgb}{0.000000,0.000000,0.000000}%
\pgfsetstrokecolor{currentstroke}%
\pgfsetdash{}{0pt}%
\pgfsys@defobject{currentmarker}{\pgfqpoint{-0.027778in}{0.000000in}}{\pgfqpoint{-0.000000in}{0.000000in}}{%
\pgfpathmoveto{\pgfqpoint{-0.000000in}{0.000000in}}%
\pgfpathlineto{\pgfqpoint{-0.027778in}{0.000000in}}%
\pgfusepath{stroke,fill}%
}%
\begin{pgfscope}%
\pgfsys@transformshift{0.588387in}{0.891965in}%
\pgfsys@useobject{currentmarker}{}%
\end{pgfscope}%
\end{pgfscope}%
\begin{pgfscope}%
\pgfsetbuttcap%
\pgfsetroundjoin%
\definecolor{currentfill}{rgb}{0.000000,0.000000,0.000000}%
\pgfsetfillcolor{currentfill}%
\pgfsetlinewidth{0.602250pt}%
\definecolor{currentstroke}{rgb}{0.000000,0.000000,0.000000}%
\pgfsetstrokecolor{currentstroke}%
\pgfsetdash{}{0pt}%
\pgfsys@defobject{currentmarker}{\pgfqpoint{-0.027778in}{0.000000in}}{\pgfqpoint{-0.000000in}{0.000000in}}{%
\pgfpathmoveto{\pgfqpoint{-0.000000in}{0.000000in}}%
\pgfpathlineto{\pgfqpoint{-0.027778in}{0.000000in}}%
\pgfusepath{stroke,fill}%
}%
\begin{pgfscope}%
\pgfsys@transformshift{0.588387in}{0.936153in}%
\pgfsys@useobject{currentmarker}{}%
\end{pgfscope}%
\end{pgfscope}%
\begin{pgfscope}%
\pgfsetbuttcap%
\pgfsetroundjoin%
\definecolor{currentfill}{rgb}{0.000000,0.000000,0.000000}%
\pgfsetfillcolor{currentfill}%
\pgfsetlinewidth{0.602250pt}%
\definecolor{currentstroke}{rgb}{0.000000,0.000000,0.000000}%
\pgfsetstrokecolor{currentstroke}%
\pgfsetdash{}{0pt}%
\pgfsys@defobject{currentmarker}{\pgfqpoint{-0.027778in}{0.000000in}}{\pgfqpoint{-0.000000in}{0.000000in}}{%
\pgfpathmoveto{\pgfqpoint{-0.000000in}{0.000000in}}%
\pgfpathlineto{\pgfqpoint{-0.027778in}{0.000000in}}%
\pgfusepath{stroke,fill}%
}%
\begin{pgfscope}%
\pgfsys@transformshift{0.588387in}{0.972257in}%
\pgfsys@useobject{currentmarker}{}%
\end{pgfscope}%
\end{pgfscope}%
\begin{pgfscope}%
\pgfsetbuttcap%
\pgfsetroundjoin%
\definecolor{currentfill}{rgb}{0.000000,0.000000,0.000000}%
\pgfsetfillcolor{currentfill}%
\pgfsetlinewidth{0.602250pt}%
\definecolor{currentstroke}{rgb}{0.000000,0.000000,0.000000}%
\pgfsetstrokecolor{currentstroke}%
\pgfsetdash{}{0pt}%
\pgfsys@defobject{currentmarker}{\pgfqpoint{-0.027778in}{0.000000in}}{\pgfqpoint{-0.000000in}{0.000000in}}{%
\pgfpathmoveto{\pgfqpoint{-0.000000in}{0.000000in}}%
\pgfpathlineto{\pgfqpoint{-0.027778in}{0.000000in}}%
\pgfusepath{stroke,fill}%
}%
\begin{pgfscope}%
\pgfsys@transformshift{0.588387in}{1.002782in}%
\pgfsys@useobject{currentmarker}{}%
\end{pgfscope}%
\end{pgfscope}%
\begin{pgfscope}%
\pgfsetbuttcap%
\pgfsetroundjoin%
\definecolor{currentfill}{rgb}{0.000000,0.000000,0.000000}%
\pgfsetfillcolor{currentfill}%
\pgfsetlinewidth{0.602250pt}%
\definecolor{currentstroke}{rgb}{0.000000,0.000000,0.000000}%
\pgfsetstrokecolor{currentstroke}%
\pgfsetdash{}{0pt}%
\pgfsys@defobject{currentmarker}{\pgfqpoint{-0.027778in}{0.000000in}}{\pgfqpoint{-0.000000in}{0.000000in}}{%
\pgfpathmoveto{\pgfqpoint{-0.000000in}{0.000000in}}%
\pgfpathlineto{\pgfqpoint{-0.027778in}{0.000000in}}%
\pgfusepath{stroke,fill}%
}%
\begin{pgfscope}%
\pgfsys@transformshift{0.588387in}{1.029225in}%
\pgfsys@useobject{currentmarker}{}%
\end{pgfscope}%
\end{pgfscope}%
\begin{pgfscope}%
\pgfsetbuttcap%
\pgfsetroundjoin%
\definecolor{currentfill}{rgb}{0.000000,0.000000,0.000000}%
\pgfsetfillcolor{currentfill}%
\pgfsetlinewidth{0.602250pt}%
\definecolor{currentstroke}{rgb}{0.000000,0.000000,0.000000}%
\pgfsetstrokecolor{currentstroke}%
\pgfsetdash{}{0pt}%
\pgfsys@defobject{currentmarker}{\pgfqpoint{-0.027778in}{0.000000in}}{\pgfqpoint{-0.000000in}{0.000000in}}{%
\pgfpathmoveto{\pgfqpoint{-0.000000in}{0.000000in}}%
\pgfpathlineto{\pgfqpoint{-0.027778in}{0.000000in}}%
\pgfusepath{stroke,fill}%
}%
\begin{pgfscope}%
\pgfsys@transformshift{0.588387in}{1.052549in}%
\pgfsys@useobject{currentmarker}{}%
\end{pgfscope}%
\end{pgfscope}%
\begin{pgfscope}%
\pgfsetbuttcap%
\pgfsetroundjoin%
\definecolor{currentfill}{rgb}{0.000000,0.000000,0.000000}%
\pgfsetfillcolor{currentfill}%
\pgfsetlinewidth{0.602250pt}%
\definecolor{currentstroke}{rgb}{0.000000,0.000000,0.000000}%
\pgfsetstrokecolor{currentstroke}%
\pgfsetdash{}{0pt}%
\pgfsys@defobject{currentmarker}{\pgfqpoint{-0.027778in}{0.000000in}}{\pgfqpoint{-0.000000in}{0.000000in}}{%
\pgfpathmoveto{\pgfqpoint{-0.000000in}{0.000000in}}%
\pgfpathlineto{\pgfqpoint{-0.027778in}{0.000000in}}%
\pgfusepath{stroke,fill}%
}%
\begin{pgfscope}%
\pgfsys@transformshift{0.588387in}{1.210673in}%
\pgfsys@useobject{currentmarker}{}%
\end{pgfscope}%
\end{pgfscope}%
\begin{pgfscope}%
\pgfsetbuttcap%
\pgfsetroundjoin%
\definecolor{currentfill}{rgb}{0.000000,0.000000,0.000000}%
\pgfsetfillcolor{currentfill}%
\pgfsetlinewidth{0.602250pt}%
\definecolor{currentstroke}{rgb}{0.000000,0.000000,0.000000}%
\pgfsetstrokecolor{currentstroke}%
\pgfsetdash{}{0pt}%
\pgfsys@defobject{currentmarker}{\pgfqpoint{-0.027778in}{0.000000in}}{\pgfqpoint{-0.000000in}{0.000000in}}{%
\pgfpathmoveto{\pgfqpoint{-0.000000in}{0.000000in}}%
\pgfpathlineto{\pgfqpoint{-0.027778in}{0.000000in}}%
\pgfusepath{stroke,fill}%
}%
\begin{pgfscope}%
\pgfsys@transformshift{0.588387in}{1.290964in}%
\pgfsys@useobject{currentmarker}{}%
\end{pgfscope}%
\end{pgfscope}%
\begin{pgfscope}%
\pgfsetbuttcap%
\pgfsetroundjoin%
\definecolor{currentfill}{rgb}{0.000000,0.000000,0.000000}%
\pgfsetfillcolor{currentfill}%
\pgfsetlinewidth{0.602250pt}%
\definecolor{currentstroke}{rgb}{0.000000,0.000000,0.000000}%
\pgfsetstrokecolor{currentstroke}%
\pgfsetdash{}{0pt}%
\pgfsys@defobject{currentmarker}{\pgfqpoint{-0.027778in}{0.000000in}}{\pgfqpoint{-0.000000in}{0.000000in}}{%
\pgfpathmoveto{\pgfqpoint{-0.000000in}{0.000000in}}%
\pgfpathlineto{\pgfqpoint{-0.027778in}{0.000000in}}%
\pgfusepath{stroke,fill}%
}%
\begin{pgfscope}%
\pgfsys@transformshift{0.588387in}{1.347932in}%
\pgfsys@useobject{currentmarker}{}%
\end{pgfscope}%
\end{pgfscope}%
\begin{pgfscope}%
\pgfsetbuttcap%
\pgfsetroundjoin%
\definecolor{currentfill}{rgb}{0.000000,0.000000,0.000000}%
\pgfsetfillcolor{currentfill}%
\pgfsetlinewidth{0.602250pt}%
\definecolor{currentstroke}{rgb}{0.000000,0.000000,0.000000}%
\pgfsetstrokecolor{currentstroke}%
\pgfsetdash{}{0pt}%
\pgfsys@defobject{currentmarker}{\pgfqpoint{-0.027778in}{0.000000in}}{\pgfqpoint{-0.000000in}{0.000000in}}{%
\pgfpathmoveto{\pgfqpoint{-0.000000in}{0.000000in}}%
\pgfpathlineto{\pgfqpoint{-0.027778in}{0.000000in}}%
\pgfusepath{stroke,fill}%
}%
\begin{pgfscope}%
\pgfsys@transformshift{0.588387in}{1.392120in}%
\pgfsys@useobject{currentmarker}{}%
\end{pgfscope}%
\end{pgfscope}%
\begin{pgfscope}%
\pgfsetbuttcap%
\pgfsetroundjoin%
\definecolor{currentfill}{rgb}{0.000000,0.000000,0.000000}%
\pgfsetfillcolor{currentfill}%
\pgfsetlinewidth{0.602250pt}%
\definecolor{currentstroke}{rgb}{0.000000,0.000000,0.000000}%
\pgfsetstrokecolor{currentstroke}%
\pgfsetdash{}{0pt}%
\pgfsys@defobject{currentmarker}{\pgfqpoint{-0.027778in}{0.000000in}}{\pgfqpoint{-0.000000in}{0.000000in}}{%
\pgfpathmoveto{\pgfqpoint{-0.000000in}{0.000000in}}%
\pgfpathlineto{\pgfqpoint{-0.027778in}{0.000000in}}%
\pgfusepath{stroke,fill}%
}%
\begin{pgfscope}%
\pgfsys@transformshift{0.588387in}{1.428224in}%
\pgfsys@useobject{currentmarker}{}%
\end{pgfscope}%
\end{pgfscope}%
\begin{pgfscope}%
\pgfsetbuttcap%
\pgfsetroundjoin%
\definecolor{currentfill}{rgb}{0.000000,0.000000,0.000000}%
\pgfsetfillcolor{currentfill}%
\pgfsetlinewidth{0.602250pt}%
\definecolor{currentstroke}{rgb}{0.000000,0.000000,0.000000}%
\pgfsetstrokecolor{currentstroke}%
\pgfsetdash{}{0pt}%
\pgfsys@defobject{currentmarker}{\pgfqpoint{-0.027778in}{0.000000in}}{\pgfqpoint{-0.000000in}{0.000000in}}{%
\pgfpathmoveto{\pgfqpoint{-0.000000in}{0.000000in}}%
\pgfpathlineto{\pgfqpoint{-0.027778in}{0.000000in}}%
\pgfusepath{stroke,fill}%
}%
\begin{pgfscope}%
\pgfsys@transformshift{0.588387in}{1.458750in}%
\pgfsys@useobject{currentmarker}{}%
\end{pgfscope}%
\end{pgfscope}%
\begin{pgfscope}%
\pgfsetbuttcap%
\pgfsetroundjoin%
\definecolor{currentfill}{rgb}{0.000000,0.000000,0.000000}%
\pgfsetfillcolor{currentfill}%
\pgfsetlinewidth{0.602250pt}%
\definecolor{currentstroke}{rgb}{0.000000,0.000000,0.000000}%
\pgfsetstrokecolor{currentstroke}%
\pgfsetdash{}{0pt}%
\pgfsys@defobject{currentmarker}{\pgfqpoint{-0.027778in}{0.000000in}}{\pgfqpoint{-0.000000in}{0.000000in}}{%
\pgfpathmoveto{\pgfqpoint{-0.000000in}{0.000000in}}%
\pgfpathlineto{\pgfqpoint{-0.027778in}{0.000000in}}%
\pgfusepath{stroke,fill}%
}%
\begin{pgfscope}%
\pgfsys@transformshift{0.588387in}{1.485192in}%
\pgfsys@useobject{currentmarker}{}%
\end{pgfscope}%
\end{pgfscope}%
\begin{pgfscope}%
\pgfsetbuttcap%
\pgfsetroundjoin%
\definecolor{currentfill}{rgb}{0.000000,0.000000,0.000000}%
\pgfsetfillcolor{currentfill}%
\pgfsetlinewidth{0.602250pt}%
\definecolor{currentstroke}{rgb}{0.000000,0.000000,0.000000}%
\pgfsetstrokecolor{currentstroke}%
\pgfsetdash{}{0pt}%
\pgfsys@defobject{currentmarker}{\pgfqpoint{-0.027778in}{0.000000in}}{\pgfqpoint{-0.000000in}{0.000000in}}{%
\pgfpathmoveto{\pgfqpoint{-0.000000in}{0.000000in}}%
\pgfpathlineto{\pgfqpoint{-0.027778in}{0.000000in}}%
\pgfusepath{stroke,fill}%
}%
\begin{pgfscope}%
\pgfsys@transformshift{0.588387in}{1.508516in}%
\pgfsys@useobject{currentmarker}{}%
\end{pgfscope}%
\end{pgfscope}%
\begin{pgfscope}%
\pgfsetbuttcap%
\pgfsetroundjoin%
\definecolor{currentfill}{rgb}{0.000000,0.000000,0.000000}%
\pgfsetfillcolor{currentfill}%
\pgfsetlinewidth{0.602250pt}%
\definecolor{currentstroke}{rgb}{0.000000,0.000000,0.000000}%
\pgfsetstrokecolor{currentstroke}%
\pgfsetdash{}{0pt}%
\pgfsys@defobject{currentmarker}{\pgfqpoint{-0.027778in}{0.000000in}}{\pgfqpoint{-0.000000in}{0.000000in}}{%
\pgfpathmoveto{\pgfqpoint{-0.000000in}{0.000000in}}%
\pgfpathlineto{\pgfqpoint{-0.027778in}{0.000000in}}%
\pgfusepath{stroke,fill}%
}%
\begin{pgfscope}%
\pgfsys@transformshift{0.588387in}{1.666640in}%
\pgfsys@useobject{currentmarker}{}%
\end{pgfscope}%
\end{pgfscope}%
\begin{pgfscope}%
\pgfsetbuttcap%
\pgfsetroundjoin%
\definecolor{currentfill}{rgb}{0.000000,0.000000,0.000000}%
\pgfsetfillcolor{currentfill}%
\pgfsetlinewidth{0.602250pt}%
\definecolor{currentstroke}{rgb}{0.000000,0.000000,0.000000}%
\pgfsetstrokecolor{currentstroke}%
\pgfsetdash{}{0pt}%
\pgfsys@defobject{currentmarker}{\pgfqpoint{-0.027778in}{0.000000in}}{\pgfqpoint{-0.000000in}{0.000000in}}{%
\pgfpathmoveto{\pgfqpoint{-0.000000in}{0.000000in}}%
\pgfpathlineto{\pgfqpoint{-0.027778in}{0.000000in}}%
\pgfusepath{stroke,fill}%
}%
\begin{pgfscope}%
\pgfsys@transformshift{0.588387in}{1.746932in}%
\pgfsys@useobject{currentmarker}{}%
\end{pgfscope}%
\end{pgfscope}%
\begin{pgfscope}%
\pgfsetbuttcap%
\pgfsetroundjoin%
\definecolor{currentfill}{rgb}{0.000000,0.000000,0.000000}%
\pgfsetfillcolor{currentfill}%
\pgfsetlinewidth{0.602250pt}%
\definecolor{currentstroke}{rgb}{0.000000,0.000000,0.000000}%
\pgfsetstrokecolor{currentstroke}%
\pgfsetdash{}{0pt}%
\pgfsys@defobject{currentmarker}{\pgfqpoint{-0.027778in}{0.000000in}}{\pgfqpoint{-0.000000in}{0.000000in}}{%
\pgfpathmoveto{\pgfqpoint{-0.000000in}{0.000000in}}%
\pgfpathlineto{\pgfqpoint{-0.027778in}{0.000000in}}%
\pgfusepath{stroke,fill}%
}%
\begin{pgfscope}%
\pgfsys@transformshift{0.588387in}{1.803900in}%
\pgfsys@useobject{currentmarker}{}%
\end{pgfscope}%
\end{pgfscope}%
\begin{pgfscope}%
\pgfsetbuttcap%
\pgfsetroundjoin%
\definecolor{currentfill}{rgb}{0.000000,0.000000,0.000000}%
\pgfsetfillcolor{currentfill}%
\pgfsetlinewidth{0.602250pt}%
\definecolor{currentstroke}{rgb}{0.000000,0.000000,0.000000}%
\pgfsetstrokecolor{currentstroke}%
\pgfsetdash{}{0pt}%
\pgfsys@defobject{currentmarker}{\pgfqpoint{-0.027778in}{0.000000in}}{\pgfqpoint{-0.000000in}{0.000000in}}{%
\pgfpathmoveto{\pgfqpoint{-0.000000in}{0.000000in}}%
\pgfpathlineto{\pgfqpoint{-0.027778in}{0.000000in}}%
\pgfusepath{stroke,fill}%
}%
\begin{pgfscope}%
\pgfsys@transformshift{0.588387in}{1.848088in}%
\pgfsys@useobject{currentmarker}{}%
\end{pgfscope}%
\end{pgfscope}%
\begin{pgfscope}%
\pgfsetbuttcap%
\pgfsetroundjoin%
\definecolor{currentfill}{rgb}{0.000000,0.000000,0.000000}%
\pgfsetfillcolor{currentfill}%
\pgfsetlinewidth{0.602250pt}%
\definecolor{currentstroke}{rgb}{0.000000,0.000000,0.000000}%
\pgfsetstrokecolor{currentstroke}%
\pgfsetdash{}{0pt}%
\pgfsys@defobject{currentmarker}{\pgfqpoint{-0.027778in}{0.000000in}}{\pgfqpoint{-0.000000in}{0.000000in}}{%
\pgfpathmoveto{\pgfqpoint{-0.000000in}{0.000000in}}%
\pgfpathlineto{\pgfqpoint{-0.027778in}{0.000000in}}%
\pgfusepath{stroke,fill}%
}%
\begin{pgfscope}%
\pgfsys@transformshift{0.588387in}{1.884192in}%
\pgfsys@useobject{currentmarker}{}%
\end{pgfscope}%
\end{pgfscope}%
\begin{pgfscope}%
\pgfsetbuttcap%
\pgfsetroundjoin%
\definecolor{currentfill}{rgb}{0.000000,0.000000,0.000000}%
\pgfsetfillcolor{currentfill}%
\pgfsetlinewidth{0.602250pt}%
\definecolor{currentstroke}{rgb}{0.000000,0.000000,0.000000}%
\pgfsetstrokecolor{currentstroke}%
\pgfsetdash{}{0pt}%
\pgfsys@defobject{currentmarker}{\pgfqpoint{-0.027778in}{0.000000in}}{\pgfqpoint{-0.000000in}{0.000000in}}{%
\pgfpathmoveto{\pgfqpoint{-0.000000in}{0.000000in}}%
\pgfpathlineto{\pgfqpoint{-0.027778in}{0.000000in}}%
\pgfusepath{stroke,fill}%
}%
\begin{pgfscope}%
\pgfsys@transformshift{0.588387in}{1.914717in}%
\pgfsys@useobject{currentmarker}{}%
\end{pgfscope}%
\end{pgfscope}%
\begin{pgfscope}%
\pgfsetbuttcap%
\pgfsetroundjoin%
\definecolor{currentfill}{rgb}{0.000000,0.000000,0.000000}%
\pgfsetfillcolor{currentfill}%
\pgfsetlinewidth{0.602250pt}%
\definecolor{currentstroke}{rgb}{0.000000,0.000000,0.000000}%
\pgfsetstrokecolor{currentstroke}%
\pgfsetdash{}{0pt}%
\pgfsys@defobject{currentmarker}{\pgfqpoint{-0.027778in}{0.000000in}}{\pgfqpoint{-0.000000in}{0.000000in}}{%
\pgfpathmoveto{\pgfqpoint{-0.000000in}{0.000000in}}%
\pgfpathlineto{\pgfqpoint{-0.027778in}{0.000000in}}%
\pgfusepath{stroke,fill}%
}%
\begin{pgfscope}%
\pgfsys@transformshift{0.588387in}{1.941160in}%
\pgfsys@useobject{currentmarker}{}%
\end{pgfscope}%
\end{pgfscope}%
\begin{pgfscope}%
\pgfsetbuttcap%
\pgfsetroundjoin%
\definecolor{currentfill}{rgb}{0.000000,0.000000,0.000000}%
\pgfsetfillcolor{currentfill}%
\pgfsetlinewidth{0.602250pt}%
\definecolor{currentstroke}{rgb}{0.000000,0.000000,0.000000}%
\pgfsetstrokecolor{currentstroke}%
\pgfsetdash{}{0pt}%
\pgfsys@defobject{currentmarker}{\pgfqpoint{-0.027778in}{0.000000in}}{\pgfqpoint{-0.000000in}{0.000000in}}{%
\pgfpathmoveto{\pgfqpoint{-0.000000in}{0.000000in}}%
\pgfpathlineto{\pgfqpoint{-0.027778in}{0.000000in}}%
\pgfusepath{stroke,fill}%
}%
\begin{pgfscope}%
\pgfsys@transformshift{0.588387in}{1.964483in}%
\pgfsys@useobject{currentmarker}{}%
\end{pgfscope}%
\end{pgfscope}%
\begin{pgfscope}%
\pgfsetbuttcap%
\pgfsetroundjoin%
\definecolor{currentfill}{rgb}{0.000000,0.000000,0.000000}%
\pgfsetfillcolor{currentfill}%
\pgfsetlinewidth{0.602250pt}%
\definecolor{currentstroke}{rgb}{0.000000,0.000000,0.000000}%
\pgfsetstrokecolor{currentstroke}%
\pgfsetdash{}{0pt}%
\pgfsys@defobject{currentmarker}{\pgfqpoint{-0.027778in}{0.000000in}}{\pgfqpoint{-0.000000in}{0.000000in}}{%
\pgfpathmoveto{\pgfqpoint{-0.000000in}{0.000000in}}%
\pgfpathlineto{\pgfqpoint{-0.027778in}{0.000000in}}%
\pgfusepath{stroke,fill}%
}%
\begin{pgfscope}%
\pgfsys@transformshift{0.588387in}{2.122607in}%
\pgfsys@useobject{currentmarker}{}%
\end{pgfscope}%
\end{pgfscope}%
\begin{pgfscope}%
\pgfsetbuttcap%
\pgfsetroundjoin%
\definecolor{currentfill}{rgb}{0.000000,0.000000,0.000000}%
\pgfsetfillcolor{currentfill}%
\pgfsetlinewidth{0.602250pt}%
\definecolor{currentstroke}{rgb}{0.000000,0.000000,0.000000}%
\pgfsetstrokecolor{currentstroke}%
\pgfsetdash{}{0pt}%
\pgfsys@defobject{currentmarker}{\pgfqpoint{-0.027778in}{0.000000in}}{\pgfqpoint{-0.000000in}{0.000000in}}{%
\pgfpathmoveto{\pgfqpoint{-0.000000in}{0.000000in}}%
\pgfpathlineto{\pgfqpoint{-0.027778in}{0.000000in}}%
\pgfusepath{stroke,fill}%
}%
\begin{pgfscope}%
\pgfsys@transformshift{0.588387in}{2.202899in}%
\pgfsys@useobject{currentmarker}{}%
\end{pgfscope}%
\end{pgfscope}%
\begin{pgfscope}%
\pgfsetbuttcap%
\pgfsetroundjoin%
\definecolor{currentfill}{rgb}{0.000000,0.000000,0.000000}%
\pgfsetfillcolor{currentfill}%
\pgfsetlinewidth{0.602250pt}%
\definecolor{currentstroke}{rgb}{0.000000,0.000000,0.000000}%
\pgfsetstrokecolor{currentstroke}%
\pgfsetdash{}{0pt}%
\pgfsys@defobject{currentmarker}{\pgfqpoint{-0.027778in}{0.000000in}}{\pgfqpoint{-0.000000in}{0.000000in}}{%
\pgfpathmoveto{\pgfqpoint{-0.000000in}{0.000000in}}%
\pgfpathlineto{\pgfqpoint{-0.027778in}{0.000000in}}%
\pgfusepath{stroke,fill}%
}%
\begin{pgfscope}%
\pgfsys@transformshift{0.588387in}{2.259867in}%
\pgfsys@useobject{currentmarker}{}%
\end{pgfscope}%
\end{pgfscope}%
\begin{pgfscope}%
\pgfsetbuttcap%
\pgfsetroundjoin%
\definecolor{currentfill}{rgb}{0.000000,0.000000,0.000000}%
\pgfsetfillcolor{currentfill}%
\pgfsetlinewidth{0.602250pt}%
\definecolor{currentstroke}{rgb}{0.000000,0.000000,0.000000}%
\pgfsetstrokecolor{currentstroke}%
\pgfsetdash{}{0pt}%
\pgfsys@defobject{currentmarker}{\pgfqpoint{-0.027778in}{0.000000in}}{\pgfqpoint{-0.000000in}{0.000000in}}{%
\pgfpathmoveto{\pgfqpoint{-0.000000in}{0.000000in}}%
\pgfpathlineto{\pgfqpoint{-0.027778in}{0.000000in}}%
\pgfusepath{stroke,fill}%
}%
\begin{pgfscope}%
\pgfsys@transformshift{0.588387in}{2.304055in}%
\pgfsys@useobject{currentmarker}{}%
\end{pgfscope}%
\end{pgfscope}%
\begin{pgfscope}%
\pgfsetbuttcap%
\pgfsetroundjoin%
\definecolor{currentfill}{rgb}{0.000000,0.000000,0.000000}%
\pgfsetfillcolor{currentfill}%
\pgfsetlinewidth{0.602250pt}%
\definecolor{currentstroke}{rgb}{0.000000,0.000000,0.000000}%
\pgfsetstrokecolor{currentstroke}%
\pgfsetdash{}{0pt}%
\pgfsys@defobject{currentmarker}{\pgfqpoint{-0.027778in}{0.000000in}}{\pgfqpoint{-0.000000in}{0.000000in}}{%
\pgfpathmoveto{\pgfqpoint{-0.000000in}{0.000000in}}%
\pgfpathlineto{\pgfqpoint{-0.027778in}{0.000000in}}%
\pgfusepath{stroke,fill}%
}%
\begin{pgfscope}%
\pgfsys@transformshift{0.588387in}{2.340159in}%
\pgfsys@useobject{currentmarker}{}%
\end{pgfscope}%
\end{pgfscope}%
\begin{pgfscope}%
\pgfsetbuttcap%
\pgfsetroundjoin%
\definecolor{currentfill}{rgb}{0.000000,0.000000,0.000000}%
\pgfsetfillcolor{currentfill}%
\pgfsetlinewidth{0.602250pt}%
\definecolor{currentstroke}{rgb}{0.000000,0.000000,0.000000}%
\pgfsetstrokecolor{currentstroke}%
\pgfsetdash{}{0pt}%
\pgfsys@defobject{currentmarker}{\pgfqpoint{-0.027778in}{0.000000in}}{\pgfqpoint{-0.000000in}{0.000000in}}{%
\pgfpathmoveto{\pgfqpoint{-0.000000in}{0.000000in}}%
\pgfpathlineto{\pgfqpoint{-0.027778in}{0.000000in}}%
\pgfusepath{stroke,fill}%
}%
\begin{pgfscope}%
\pgfsys@transformshift{0.588387in}{2.370685in}%
\pgfsys@useobject{currentmarker}{}%
\end{pgfscope}%
\end{pgfscope}%
\begin{pgfscope}%
\pgfsetbuttcap%
\pgfsetroundjoin%
\definecolor{currentfill}{rgb}{0.000000,0.000000,0.000000}%
\pgfsetfillcolor{currentfill}%
\pgfsetlinewidth{0.602250pt}%
\definecolor{currentstroke}{rgb}{0.000000,0.000000,0.000000}%
\pgfsetstrokecolor{currentstroke}%
\pgfsetdash{}{0pt}%
\pgfsys@defobject{currentmarker}{\pgfqpoint{-0.027778in}{0.000000in}}{\pgfqpoint{-0.000000in}{0.000000in}}{%
\pgfpathmoveto{\pgfqpoint{-0.000000in}{0.000000in}}%
\pgfpathlineto{\pgfqpoint{-0.027778in}{0.000000in}}%
\pgfusepath{stroke,fill}%
}%
\begin{pgfscope}%
\pgfsys@transformshift{0.588387in}{2.397127in}%
\pgfsys@useobject{currentmarker}{}%
\end{pgfscope}%
\end{pgfscope}%
\begin{pgfscope}%
\pgfsetbuttcap%
\pgfsetroundjoin%
\definecolor{currentfill}{rgb}{0.000000,0.000000,0.000000}%
\pgfsetfillcolor{currentfill}%
\pgfsetlinewidth{0.602250pt}%
\definecolor{currentstroke}{rgb}{0.000000,0.000000,0.000000}%
\pgfsetstrokecolor{currentstroke}%
\pgfsetdash{}{0pt}%
\pgfsys@defobject{currentmarker}{\pgfqpoint{-0.027778in}{0.000000in}}{\pgfqpoint{-0.000000in}{0.000000in}}{%
\pgfpathmoveto{\pgfqpoint{-0.000000in}{0.000000in}}%
\pgfpathlineto{\pgfqpoint{-0.027778in}{0.000000in}}%
\pgfusepath{stroke,fill}%
}%
\begin{pgfscope}%
\pgfsys@transformshift{0.588387in}{2.420451in}%
\pgfsys@useobject{currentmarker}{}%
\end{pgfscope}%
\end{pgfscope}%
\begin{pgfscope}%
\definecolor{textcolor}{rgb}{0.000000,0.000000,0.000000}%
\pgfsetstrokecolor{textcolor}%
\pgfsetfillcolor{textcolor}%
\pgftext[x=0.234413in,y=1.526746in,,bottom,rotate=90.000000]{\color{textcolor}{\rmfamily\fontsize{10.000000}{12.000000}\selectfont\catcode`\^=\active\def^{\ifmmode\sp\else\^{}\fi}\catcode`\%=\active\def%{\%}Checks [call]}}%
\end{pgfscope}%
\begin{pgfscope}%
\pgfpathrectangle{\pgfqpoint{0.588387in}{0.521603in}}{\pgfqpoint{5.314715in}{2.010285in}}%
\pgfusepath{clip}%
\pgfsetrectcap%
\pgfsetroundjoin%
\pgfsetlinewidth{1.505625pt}%
\pgfsetstrokecolor{currentstroke1}%
\pgfsetdash{}{0pt}%
\pgfpathmoveto{\pgfqpoint{0.829965in}{0.619261in}}%
\pgfpathlineto{\pgfqpoint{0.928568in}{0.728924in}}%
\pgfpathlineto{\pgfqpoint{1.027172in}{0.859337in}}%
\pgfpathlineto{\pgfqpoint{1.125775in}{0.918199in}}%
\pgfpathlineto{\pgfqpoint{1.224378in}{0.998866in}}%
\pgfpathlineto{\pgfqpoint{1.322981in}{1.026067in}}%
\pgfpathlineto{\pgfqpoint{1.421585in}{1.073639in}}%
\pgfpathlineto{\pgfqpoint{1.520188in}{1.104767in}}%
\pgfpathlineto{\pgfqpoint{1.618791in}{1.122282in}}%
\pgfpathlineto{\pgfqpoint{1.717394in}{1.140219in}}%
\pgfpathlineto{\pgfqpoint{1.815998in}{1.167906in}}%
\pgfpathlineto{\pgfqpoint{1.914601in}{1.176211in}}%
\pgfpathlineto{\pgfqpoint{2.013204in}{1.189322in}}%
\pgfpathlineto{\pgfqpoint{2.111807in}{1.207145in}}%
\pgfpathlineto{\pgfqpoint{2.210411in}{1.214503in}}%
\pgfpathlineto{\pgfqpoint{2.309014in}{1.220750in}}%
\pgfpathlineto{\pgfqpoint{2.407617in}{1.239676in}}%
\pgfpathlineto{\pgfqpoint{2.506220in}{1.245184in}}%
\pgfpathlineto{\pgfqpoint{2.604824in}{1.253273in}}%
\pgfpathlineto{\pgfqpoint{2.703427in}{1.259430in}}%
\pgfpathlineto{\pgfqpoint{2.802030in}{1.287048in}}%
\pgfpathlineto{\pgfqpoint{2.900633in}{1.276072in}}%
\pgfpathlineto{\pgfqpoint{2.999237in}{1.284064in}}%
\pgfpathlineto{\pgfqpoint{3.097840in}{1.299068in}}%
\pgfpathlineto{\pgfqpoint{3.196443in}{1.301347in}}%
\pgfpathlineto{\pgfqpoint{3.295046in}{1.316929in}}%
\pgfpathlineto{\pgfqpoint{3.393649in}{1.315886in}}%
\pgfpathlineto{\pgfqpoint{3.492253in}{1.323325in}}%
\pgfpathlineto{\pgfqpoint{3.590856in}{1.339857in}}%
\pgfpathlineto{\pgfqpoint{3.689459in}{1.330599in}}%
\pgfpathlineto{\pgfqpoint{3.788062in}{1.340235in}}%
\pgfpathlineto{\pgfqpoint{3.886666in}{1.347126in}}%
\pgfpathlineto{\pgfqpoint{3.985269in}{1.346691in}}%
\pgfpathlineto{\pgfqpoint{4.083872in}{1.360606in}}%
\pgfpathlineto{\pgfqpoint{4.182475in}{1.361973in}}%
\pgfpathlineto{\pgfqpoint{4.281079in}{1.365423in}}%
\pgfpathlineto{\pgfqpoint{4.379682in}{1.367185in}}%
\pgfpathlineto{\pgfqpoint{4.478285in}{1.379920in}}%
\pgfpathlineto{\pgfqpoint{4.576888in}{1.382957in}}%
\pgfpathlineto{\pgfqpoint{4.675492in}{1.383320in}}%
\pgfpathlineto{\pgfqpoint{4.774095in}{1.396260in}}%
\pgfpathlineto{\pgfqpoint{4.872698in}{1.383339in}}%
\pgfpathlineto{\pgfqpoint{4.971301in}{1.393784in}}%
\pgfpathlineto{\pgfqpoint{5.069905in}{1.407612in}}%
\pgfpathlineto{\pgfqpoint{5.168508in}{1.403343in}}%
\pgfpathlineto{\pgfqpoint{5.267111in}{1.405055in}}%
\pgfpathlineto{\pgfqpoint{5.365714in}{1.414761in}}%
\pgfpathlineto{\pgfqpoint{5.464318in}{1.418165in}}%
\pgfpathlineto{\pgfqpoint{5.562921in}{1.426203in}}%
\pgfpathlineto{\pgfqpoint{5.661524in}{1.421490in}}%
\pgfusepath{stroke}%
\end{pgfscope}%
\begin{pgfscope}%
\pgfpathrectangle{\pgfqpoint{0.588387in}{0.521603in}}{\pgfqpoint{5.314715in}{2.010285in}}%
\pgfusepath{clip}%
\pgfsetrectcap%
\pgfsetroundjoin%
\pgfsetlinewidth{1.505625pt}%
\pgfsetstrokecolor{currentstroke2}%
\pgfsetdash{}{0pt}%
\pgfpathmoveto{\pgfqpoint{0.829965in}{0.612980in}}%
\pgfpathlineto{\pgfqpoint{0.928568in}{0.702397in}}%
\pgfpathlineto{\pgfqpoint{1.027172in}{0.828491in}}%
\pgfpathlineto{\pgfqpoint{1.125775in}{0.929221in}}%
\pgfpathlineto{\pgfqpoint{1.224378in}{1.052469in}}%
\pgfpathlineto{\pgfqpoint{1.322981in}{1.098278in}}%
\pgfpathlineto{\pgfqpoint{1.421585in}{1.219878in}}%
\pgfpathlineto{\pgfqpoint{1.520188in}{1.419414in}}%
\pgfpathlineto{\pgfqpoint{1.618791in}{1.405465in}}%
\pgfpathlineto{\pgfqpoint{1.717394in}{1.617512in}}%
\pgfpathlineto{\pgfqpoint{1.815998in}{1.698072in}}%
\pgfpathlineto{\pgfqpoint{1.914601in}{1.760965in}}%
\pgfpathlineto{\pgfqpoint{2.013204in}{1.741550in}}%
\pgfpathlineto{\pgfqpoint{2.111807in}{1.778650in}}%
\pgfpathlineto{\pgfqpoint{2.210411in}{1.825986in}}%
\pgfpathlineto{\pgfqpoint{2.309014in}{1.802117in}}%
\pgfpathlineto{\pgfqpoint{2.407617in}{1.851393in}}%
\pgfpathlineto{\pgfqpoint{2.506220in}{1.798896in}}%
\pgfpathlineto{\pgfqpoint{2.604824in}{1.893284in}}%
\pgfpathlineto{\pgfqpoint{2.703427in}{1.931204in}}%
\pgfpathlineto{\pgfqpoint{2.802030in}{1.953077in}}%
\pgfpathlineto{\pgfqpoint{2.900633in}{2.044922in}}%
\pgfpathlineto{\pgfqpoint{2.999237in}{1.952866in}}%
\pgfusepath{stroke}%
\end{pgfscope}%
\begin{pgfscope}%
\pgfpathrectangle{\pgfqpoint{0.588387in}{0.521603in}}{\pgfqpoint{5.314715in}{2.010285in}}%
\pgfusepath{clip}%
\pgfsetrectcap%
\pgfsetroundjoin%
\pgfsetlinewidth{1.505625pt}%
\pgfsetstrokecolor{currentstroke3}%
\pgfsetdash{}{0pt}%
\pgfpathmoveto{\pgfqpoint{0.829965in}{0.613812in}}%
\pgfpathlineto{\pgfqpoint{0.928568in}{0.748618in}}%
\pgfpathlineto{\pgfqpoint{1.027172in}{0.916622in}}%
\pgfpathlineto{\pgfqpoint{1.125775in}{1.111988in}}%
\pgfpathlineto{\pgfqpoint{1.224378in}{1.392439in}}%
\pgfpathlineto{\pgfqpoint{1.322981in}{1.478612in}}%
\pgfpathlineto{\pgfqpoint{1.421585in}{1.651438in}}%
\pgfpathlineto{\pgfqpoint{1.520188in}{1.795454in}}%
\pgfpathlineto{\pgfqpoint{1.618791in}{1.985003in}}%
\pgfpathlineto{\pgfqpoint{1.717394in}{2.080963in}}%
\pgfpathlineto{\pgfqpoint{1.815998in}{2.168477in}}%
\pgfpathlineto{\pgfqpoint{1.914601in}{2.250575in}}%
\pgfpathlineto{\pgfqpoint{2.013204in}{2.243995in}}%
\pgfpathlineto{\pgfqpoint{2.111807in}{2.217028in}}%
\pgfpathlineto{\pgfqpoint{2.210411in}{2.206793in}}%
\pgfpathlineto{\pgfqpoint{2.309014in}{2.136006in}}%
\pgfpathlineto{\pgfqpoint{2.407617in}{2.163733in}}%
\pgfpathlineto{\pgfqpoint{2.506220in}{2.440512in}}%
\pgfpathlineto{\pgfqpoint{2.604824in}{2.150529in}}%
\pgfpathlineto{\pgfqpoint{2.802030in}{1.955515in}}%
\pgfusepath{stroke}%
\end{pgfscope}%
\begin{pgfscope}%
\pgfpathrectangle{\pgfqpoint{0.588387in}{0.521603in}}{\pgfqpoint{5.314715in}{2.010285in}}%
\pgfusepath{clip}%
\pgfsetrectcap%
\pgfsetroundjoin%
\pgfsetlinewidth{1.505625pt}%
\pgfsetstrokecolor{currentstroke4}%
\pgfsetdash{}{0pt}%
\pgfpathmoveto{\pgfqpoint{0.829965in}{0.629159in}}%
\pgfpathlineto{\pgfqpoint{0.928568in}{0.760198in}}%
\pgfpathlineto{\pgfqpoint{1.027172in}{0.896288in}}%
\pgfpathlineto{\pgfqpoint{1.125775in}{1.009047in}}%
\pgfpathlineto{\pgfqpoint{1.224378in}{1.142546in}}%
\pgfpathlineto{\pgfqpoint{1.322981in}{1.154322in}}%
\pgfpathlineto{\pgfqpoint{1.421585in}{1.333977in}}%
\pgfpathlineto{\pgfqpoint{1.520188in}{1.348843in}}%
\pgfpathlineto{\pgfqpoint{1.618791in}{1.461491in}}%
\pgfpathlineto{\pgfqpoint{1.717394in}{1.732142in}}%
\pgfpathlineto{\pgfqpoint{1.815998in}{1.609311in}}%
\pgfpathlineto{\pgfqpoint{1.914601in}{1.832239in}}%
\pgfpathlineto{\pgfqpoint{2.013204in}{1.779231in}}%
\pgfpathlineto{\pgfqpoint{2.111807in}{1.866823in}}%
\pgfpathlineto{\pgfqpoint{2.210411in}{1.544283in}}%
\pgfpathlineto{\pgfqpoint{2.309014in}{1.686554in}}%
\pgfpathlineto{\pgfqpoint{2.407617in}{1.871254in}}%
\pgfpathlineto{\pgfqpoint{2.506220in}{1.755189in}}%
\pgfpathlineto{\pgfqpoint{2.604824in}{1.684089in}}%
\pgfpathlineto{\pgfqpoint{2.703427in}{1.697330in}}%
\pgfpathlineto{\pgfqpoint{2.802030in}{1.770161in}}%
\pgfpathlineto{\pgfqpoint{2.900633in}{1.765294in}}%
\pgfpathlineto{\pgfqpoint{2.999237in}{1.784068in}}%
\pgfpathlineto{\pgfqpoint{3.097840in}{1.772148in}}%
\pgfpathlineto{\pgfqpoint{3.196443in}{1.758681in}}%
\pgfpathlineto{\pgfqpoint{3.295046in}{1.514994in}}%
\pgfpathlineto{\pgfqpoint{3.393649in}{1.663394in}}%
\pgfpathlineto{\pgfqpoint{3.492253in}{1.653838in}}%
\pgfpathlineto{\pgfqpoint{3.590856in}{1.511186in}}%
\pgfpathlineto{\pgfqpoint{3.689459in}{1.591162in}}%
\pgfpathlineto{\pgfqpoint{3.788062in}{1.653733in}}%
\pgfpathlineto{\pgfqpoint{3.886666in}{1.766809in}}%
\pgfpathlineto{\pgfqpoint{3.985269in}{1.482735in}}%
\pgfpathlineto{\pgfqpoint{4.083872in}{1.574385in}}%
\pgfpathlineto{\pgfqpoint{4.182475in}{1.589731in}}%
\pgfpathlineto{\pgfqpoint{4.281079in}{1.661478in}}%
\pgfpathlineto{\pgfqpoint{4.379682in}{1.633424in}}%
\pgfpathlineto{\pgfqpoint{4.478285in}{1.582188in}}%
\pgfpathlineto{\pgfqpoint{4.576888in}{1.687255in}}%
\pgfpathlineto{\pgfqpoint{4.675492in}{1.597032in}}%
\pgfpathlineto{\pgfqpoint{4.774095in}{1.678339in}}%
\pgfpathlineto{\pgfqpoint{4.872698in}{1.632820in}}%
\pgfpathlineto{\pgfqpoint{4.971301in}{1.707334in}}%
\pgfpathlineto{\pgfqpoint{5.069905in}{1.871999in}}%
\pgfpathlineto{\pgfqpoint{5.168508in}{2.001876in}}%
\pgfpathlineto{\pgfqpoint{5.267111in}{1.618578in}}%
\pgfpathlineto{\pgfqpoint{5.365714in}{1.655909in}}%
\pgfpathlineto{\pgfqpoint{5.464318in}{1.644451in}}%
\pgfpathlineto{\pgfqpoint{5.562921in}{1.576659in}}%
\pgfpathlineto{\pgfqpoint{5.661524in}{1.691175in}}%
\pgfusepath{stroke}%
\end{pgfscope}%
\begin{pgfscope}%
\pgfsetrectcap%
\pgfsetmiterjoin%
\pgfsetlinewidth{0.803000pt}%
\definecolor{currentstroke}{rgb}{0.000000,0.000000,0.000000}%
\pgfsetstrokecolor{currentstroke}%
\pgfsetdash{}{0pt}%
\pgfpathmoveto{\pgfqpoint{0.588387in}{0.521603in}}%
\pgfpathlineto{\pgfqpoint{0.588387in}{2.531888in}}%
\pgfusepath{stroke}%
\end{pgfscope}%
\begin{pgfscope}%
\pgfsetrectcap%
\pgfsetmiterjoin%
\pgfsetlinewidth{0.803000pt}%
\definecolor{currentstroke}{rgb}{0.000000,0.000000,0.000000}%
\pgfsetstrokecolor{currentstroke}%
\pgfsetdash{}{0pt}%
\pgfpathmoveto{\pgfqpoint{5.903102in}{0.521603in}}%
\pgfpathlineto{\pgfqpoint{5.903102in}{2.531888in}}%
\pgfusepath{stroke}%
\end{pgfscope}%
\begin{pgfscope}%
\pgfsetrectcap%
\pgfsetmiterjoin%
\pgfsetlinewidth{0.803000pt}%
\definecolor{currentstroke}{rgb}{0.000000,0.000000,0.000000}%
\pgfsetstrokecolor{currentstroke}%
\pgfsetdash{}{0pt}%
\pgfpathmoveto{\pgfqpoint{0.588387in}{0.521603in}}%
\pgfpathlineto{\pgfqpoint{5.903102in}{0.521603in}}%
\pgfusepath{stroke}%
\end{pgfscope}%
\begin{pgfscope}%
\pgfsetrectcap%
\pgfsetmiterjoin%
\pgfsetlinewidth{0.803000pt}%
\definecolor{currentstroke}{rgb}{0.000000,0.000000,0.000000}%
\pgfsetstrokecolor{currentstroke}%
\pgfsetdash{}{0pt}%
\pgfpathmoveto{\pgfqpoint{0.588387in}{2.531888in}}%
\pgfpathlineto{\pgfqpoint{5.903102in}{2.531888in}}%
\pgfusepath{stroke}%
\end{pgfscope}%
\begin{pgfscope}%
\definecolor{textcolor}{rgb}{0.000000,0.000000,0.000000}%
\pgfsetstrokecolor{textcolor}%
\pgfsetfillcolor{textcolor}%
\pgftext[x=3.245745in,y=2.615222in,,base]{\color{textcolor}{\rmfamily\fontsize{12.000000}{14.400000}\selectfont\catcode`\^=\active\def^{\ifmmode\sp\else\^{}\fi}\catcode`\%=\active\def%{\%}Mean}}%
\end{pgfscope}%
\begin{pgfscope}%
\pgfsetbuttcap%
\pgfsetmiterjoin%
\definecolor{currentfill}{rgb}{1.000000,1.000000,1.000000}%
\pgfsetfillcolor{currentfill}%
\pgfsetfillopacity{0.800000}%
\pgfsetlinewidth{1.003750pt}%
\definecolor{currentstroke}{rgb}{0.800000,0.800000,0.800000}%
\pgfsetstrokecolor{currentstroke}%
\pgfsetstrokeopacity{0.800000}%
\pgfsetdash{}{0pt}%
\pgfpathmoveto{\pgfqpoint{5.990602in}{1.691044in}}%
\pgfpathlineto{\pgfqpoint{8.259376in}{1.691044in}}%
\pgfpathquadraticcurveto{\pgfqpoint{8.284376in}{1.691044in}}{\pgfqpoint{8.284376in}{1.716044in}}%
\pgfpathlineto{\pgfqpoint{8.284376in}{2.444388in}}%
\pgfpathquadraticcurveto{\pgfqpoint{8.284376in}{2.469388in}}{\pgfqpoint{8.259376in}{2.469388in}}%
\pgfpathlineto{\pgfqpoint{5.990602in}{2.469388in}}%
\pgfpathquadraticcurveto{\pgfqpoint{5.965602in}{2.469388in}}{\pgfqpoint{5.965602in}{2.444388in}}%
\pgfpathlineto{\pgfqpoint{5.965602in}{1.716044in}}%
\pgfpathquadraticcurveto{\pgfqpoint{5.965602in}{1.691044in}}{\pgfqpoint{5.990602in}{1.691044in}}%
\pgfpathlineto{\pgfqpoint{5.990602in}{1.691044in}}%
\pgfpathclose%
\pgfusepath{stroke,fill}%
\end{pgfscope}%
\begin{pgfscope}%
\pgfsetrectcap%
\pgfsetroundjoin%
\pgfsetlinewidth{1.505625pt}%
\pgfsetstrokecolor{currentstroke1}%
\pgfsetdash{}{0pt}%
\pgfpathmoveto{\pgfqpoint{6.015602in}{2.368168in}}%
\pgfpathlineto{\pgfqpoint{6.140602in}{2.368168in}}%
\pgfpathlineto{\pgfqpoint{6.265602in}{2.368168in}}%
\pgfusepath{stroke}%
\end{pgfscope}%
\begin{pgfscope}%
\definecolor{textcolor}{rgb}{0.000000,0.000000,0.000000}%
\pgfsetstrokecolor{textcolor}%
\pgfsetfillcolor{textcolor}%
\pgftext[x=6.365602in,y=2.324418in,left,base]{\color{textcolor}{\rmfamily\fontsize{9.000000}{10.800000}\selectfont\catcode`\^=\active\def^{\ifmmode\sp\else\^{}\fi}\catcode`\%=\active\def%{\%}\Neighbors{} \& \MergeLinear{}}}%
\end{pgfscope}%
\begin{pgfscope}%
\pgfsetrectcap%
\pgfsetroundjoin%
\pgfsetlinewidth{1.505625pt}%
\pgfsetstrokecolor{currentstroke2}%
\pgfsetdash{}{0pt}%
\pgfpathmoveto{\pgfqpoint{6.015602in}{2.184696in}}%
\pgfpathlineto{\pgfqpoint{6.140602in}{2.184696in}}%
\pgfpathlineto{\pgfqpoint{6.265602in}{2.184696in}}%
\pgfusepath{stroke}%
\end{pgfscope}%
\begin{pgfscope}%
\definecolor{textcolor}{rgb}{0.000000,0.000000,0.000000}%
\pgfsetstrokecolor{textcolor}%
\pgfsetfillcolor{textcolor}%
\pgftext[x=6.365602in,y=2.140946in,left,base]{\color{textcolor}{\rmfamily\fontsize{9.000000}{10.800000}\selectfont\catcode`\^=\active\def^{\ifmmode\sp\else\^{}\fi}\catcode`\%=\active\def%{\%}\Neighbors{} \& \Log{}}}%
\end{pgfscope}%
\begin{pgfscope}%
\pgfsetrectcap%
\pgfsetroundjoin%
\pgfsetlinewidth{1.505625pt}%
\pgfsetstrokecolor{currentstroke3}%
\pgfsetdash{}{0pt}%
\pgfpathmoveto{\pgfqpoint{6.015602in}{2.001225in}}%
\pgfpathlineto{\pgfqpoint{6.140602in}{2.001225in}}%
\pgfpathlineto{\pgfqpoint{6.265602in}{2.001225in}}%
\pgfusepath{stroke}%
\end{pgfscope}%
\begin{pgfscope}%
\definecolor{textcolor}{rgb}{0.000000,0.000000,0.000000}%
\pgfsetstrokecolor{textcolor}%
\pgfsetfillcolor{textcolor}%
\pgftext[x=6.365602in,y=1.957475in,left,base]{\color{textcolor}{\rmfamily\fontsize{9.000000}{10.800000}\selectfont\catcode`\^=\active\def^{\ifmmode\sp\else\^{}\fi}\catcode`\%=\active\def%{\%}\Neighbors{} \& \PromisingCycles{}}}%
\end{pgfscope}%
\begin{pgfscope}%
\pgfsetrectcap%
\pgfsetroundjoin%
\pgfsetlinewidth{1.505625pt}%
\pgfsetstrokecolor{currentstroke4}%
\pgfsetdash{}{0pt}%
\pgfpathmoveto{\pgfqpoint{6.015602in}{1.814274in}}%
\pgfpathlineto{\pgfqpoint{6.140602in}{1.814274in}}%
\pgfpathlineto{\pgfqpoint{6.265602in}{1.814274in}}%
\pgfusepath{stroke}%
\end{pgfscope}%
\begin{pgfscope}%
\definecolor{textcolor}{rgb}{0.000000,0.000000,0.000000}%
\pgfsetstrokecolor{textcolor}%
\pgfsetfillcolor{textcolor}%
\pgftext[x=6.365602in,y=1.770524in,left,base]{\color{textcolor}{\rmfamily\fontsize{9.000000}{10.800000}\selectfont\catcode`\^=\active\def^{\ifmmode\sp\else\^{}\fi}\catcode`\%=\active\def%{\%}\Neighbors{} \& \SharedVertices{}}}%
\end{pgfscope}%
\end{pgfpicture}%
\makeatother%
\endgroup%
}
	\caption[Mean runtime for minimally rigid graphs (some).]{
		Mean running time (ms) to find all NAC-colorings for graphs with no NAC-coloring for different subgraph sizes \( k \).}%
	\label{fig:graph_no_nac_coloring_generated_rigid_failing_merging_first_runtime}
\end{figure}

First smart split described in \Cref{sec:smart_split}
did not improve the runtime.
We expected minor performance hit for smaller graphs because heuristic is run
multiple times, but gains for larger graphs where subgraphs merging order
should join subgraphs near to each other together. This is not the case.

% Smart split
\todo[inline]{Run smart split with log}

\subsubsection{Final comparison}

Based on our benchmarks presented in the previous sections,
we choose algorithms that should be preserved and merged into PyRigi.
For graph classes with a lot of NAC-colorings,
\NaiveCycles{} is usually the best choice
when we search for a single NAC-coloring.
The user of the library has to pay attention while using this strategy
as if there is no NAC-coloring and the graph does not trivially collapse
into a few monochromatic classes, the runtime will be huge.
For these cases and cases where we search for all NAC-colorings,
we preserve the \NeighborsDegree{} strategy and both the merging strategies.
\MergeLinear{} performs better in general case and
\SharedVertices{} performed better for graphs with no NAC-colorings.


