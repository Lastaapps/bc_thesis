\chapter{Implementation \& Benchmarks}%
\label{chapter:benchmarks}

\begin{chapterabstract}

	In this chapter we first describe the structure of the project
	and discuss some of the design choices.
	After that, we evaluate performance of the algorithms
	proposed in \Cref{chapter:alg}.
	First, we compare the approaches with previous approaches,
	and then we compare heuristics with each other
	for different use-cases.
	We show reduction both in runtime and in the number
	of \IsNACColoring{} checks performed.

\end{chapterabstract}

\section{Implementation}

\todo[inline]{Popsat, jak/kde je kód přiložen}

In this section we first describe the structure of the project containing
the code of the algorithm.
Next, we mention libraries, relation to PyRigi and
some worth mentioning implementation details.

The code is written in Python, minimal supported version is Python 3.12.
To set up the project, create a virtual environment and install packages
from \texttt{requirements.txt}. On NixOS, you can use \texttt{shell.nix}.
See \texttt{README.md} for additional instructions.
We go through the main folders and files of the project.

In \texttt{graphs\_store} we store datasets used for benchmarking.
Graphs are either obtained from~\cite{extremal_graphs},
generated using Nauty~\cite{nauty} with a plugin~\cite{nauty_plugin}
or generated using NetworkX~\cite{networkx} and checks from PyRigi~\cite{pyrigi}.
Graph are usually stored in Graph6 format.
Code for reading graphs from the store can be found in \texttt{benchmarks/dataset.py},
code for generating some graph classes can be found in  \texttt{benchmarks/generators.py},
In the \texttt{benchmark} directory, there are all result of the benchmarks that we run
during the project development. The most relevant for a reader is the CSV file
stored in \texttt{precomputed} directory containing individual benchmark results.
The base directory contains tooling for running, visualizing and exporting benchmarks.
File \texttt{NAC\_playground.ipynb} in the root directory presents a simple case
to visualize how the Python API can be used.
File \texttt{NAC\_presentation.ipynb} shows how the basic benchmarks can be run.

The code of the algorithm described in~\Cref{sec:stable_cuts_implementation}
with additional helper functions is implemented in directory \texttt{stablecut}.
Note that some changes were done when the code was merged into PyRigi.
The code of the algorithm described in~\Cref{chapter:alg}
is stored in directory \texttt{nac}.
Directory \texttt{nac/util} stores helper functions and classes
like an implementation of the \textsc{UnionFind} data structure.
File \texttt{check.py} implements \IsNACColoring{} check.
File \texttt{monochromatic\_classes.py} is used to find \trcon{} components
and monochromatic classes in a graph. With this, we can compare performance
between using monochromatic classes, \trcon{} components or just edges.
File \texttt{cycle\_detection.py} holds algorithms for finding cycles
used in \Cref{sec:small_cycles}
and related heuristics as described in \Cref{chapter:alg}.
In \Cref{sec:polynomial_optimizations}
we presented checks that can sometimes find
a NAC-coloring or determine that there is none in polynomial time.
These checks are implemented in \texttt{existence.py} and
used mostly from \texttt{single.py} that is the entry-point for finding a single NAC-coloring of a graph.
General NAC-coloring searching is implemented in \texttt{search.py}
along with parameter processing, graph vertices relabeling,
optimizations like search for articulation vertices are performed.
After that, the correct algorithm (\Naive{}, \NaiveCycles{} or \Subgraphs{})
is chosen and called.
These algorithms are implemented in \texttt{algorithms.py} alongside many helper functions.
Heuristics for \Subgraphs{} algorithm are stored in \texttt{strategies.py}.
Tests of the whole code are stored in directory \texttt{test}.

Common function parameters are:
\textsc{graph} repressing the (sub)graph where NAC-colorings should be found,
\textsc{comp\_graph} where vertices are some integer IDs of monochromatic classes
and edges exists if the classes are neighboring.
An ID of a monochromatic class also serves as index into \textsc{component\_to\_edges}
that maps an ID of a monochromatic class to its edges.
NAC-colorings are represented as bit-masks where bit's offset correspond to a component ID.

As \IsNACColoring{} is a core component of all our algorithms, we optimize it as much as we could.
In the implementation of \IsNACColoring{}, subgraphs from \( \red \) and \( \blue \) edges are created.
To create such subgraphs in code, edges can be added to an empty graph
using NetworkX's function \textsc{add\_edges\_from}.
This is rather slow as creating new vertices in the empty graph causes noticeable overhead.
Therefore, we create a graph with no edges and the same vertices as the original graph,
cache it and reuse it for the checks. First edges are added, the check is run, and the edges are cleared.
By doing this, the performance of \IsNACColoring{} is increased by roughly forty percent.
Another way how the performance could be increased is by reserving space in lists
when the final size is known. This is unfortunately not possible in Python.

The code uses \textsc{Graph} class and related algorithms from NetworkX~\cite{networkx}
as the base of all operations. We use some utility functions from PyRigi~\cite{pyrigi}
related to rigidity tests and rigidity components search.
Otherwise, the code is not dependent on PyRigi.
Pytest is used for testing.

\section{Benchmarks}

In this section we first set meaningful targets for our benchmarks,
then we compare the performance of our algorithm with the previous implementations
and show running time and internal search optimizations for various graph classes.

The main question regarding NAC-coloring search is whether a graph has a NAC-coloring.
We usually ask the algorithm to not only answer yes, but to also provide a certificate.
For flexible graphs, it is usually algorithmically quite simple to find a NAC-coloring,
so this question is more interesting for rigid graphs.
For flexible graphs, it is more interesting to ask for the number of NAC-colorings
of a graph.
Note, that for larger flexible graphs with around thirty vertices
the number of NAC-colorings is huge as it often grows exponentially.
This slows our algorithm down as just materializing exponential
number of NAC-colorings takes exponential time.
For such cases, the FPT algorithm described in \Cref{chapter:fpt}
is a better fit as it does not materialize all the NAC-coloring on subgraphs.

The benchmarks comparing our algorithm with the previous implementations
were run on Linux on a laptop with Intel i7 of the 11th generation
with CPython 3.12 and SageMath 10.4.
The remaining benchmarks were run a laptop with Intel i5 of the 6th generation
using CPython 3.12.
On modern hardware, the times can be easily cut in half.

\subsection{Improvement over previous solutions}

\Cref{tab:all_min_rigid}
shows the time required for finding all the NAC-colorings
of all minimally rigid graphs with given size (generated using Nauty~\cite{nauty}
with a corresponding plugin~\cite{nauty_plugin}).
We show results of the implementation in \flexrilog{}~\cite{flexrilog} run in SageMath
and compare them to our implementation of the same \Naive{} algorithm
using $\triangle$-connected components
and monochromatic classes as described in \Cref{sec:NACvalid}.
Next column shows \NaiveCycles{} from \Cref{sec:small_cycles}
using monochromatic classes.
The last column is for the \NeighborsDegree{} (each initial subgraph has $k=4$ monochromatic classes)
with \MergeLinear{} merging strategy.
For twelve vertices, \Neighbors{} algorithm took around four hours for over 800k minimally rigid graphs.
In every case, our algorithms are significantly faster than implementation in \flexrilog{}~\cite{flexrilog}.
Notice also huge advantage gained by using monochromatic classes instead of \trcon{} components,
that are also used by \flexrilog{}.
%
\begin{table}[ht]
	\caption[Running times on minimally rigid graphs.]{
		The time (in seconds) needed to find all NAC-colorings for all minimally rigid graphs with a given size. Run by us.
		\textsc{FRLG} stands for \flexrilog{}, \textsc{ND} for \NeighborsDegree{}.}%
	\label{tab:all_min_rigid}
	\vspace{0.3cm}
	\centering
	\begin{tabular}{ccccccc}
		\hline
		\,$|V(G)|$\, & \,\#graphs\, & \,FRLG\, & \,$\triangle$-comps.\, & \,monochr.\, & \,cycles\, & \,\textsc{ND}\, \\
		\hline
		% 5        & 3           & 0.007 s      & 0.002 s            & 0.001 s       & 0.001 s & 0.002 s          \\
		% 6        & 13          & 0.063 s      & 0.030 s            & 0.010 s       & 0.005 s & 0.007 s          \\
		% 7        & 70          & 0.57 s       & 0.052 s            & 0.047 s       & 0.029 s & 0.041 s          \\
		8            & 608          & 14               & 1.09                   & 0.97         & 0.36       & 0.49                   \\
		9            & 7\,222       & 509              & 34                     & 29           & 5.8        & 8.6                    \\
		10           & 110\,132     & 27k              & 1\,725                 & 1\,446       & 151        & 213                    \\
		11           & 2\,039\,273  & -                & -                      & -            & 5\,440     & 6\,650                 \\
		\hline
	\end{tabular}
\end{table}

\Cref{fig:graph_time_minimally_rigid}
shows timings to compute all NAC-colorings of minimally rigid graphs
depending on the strategy used.
We did not list all NAC-coloring for minimally rigid graphs with more than twelve vertices
as there is too many such graphs.

The following dataset has been randomly generated
using NetworkX~\cite{networkx} and PyRigi~\cite{pyrigi}.
%
You can see that for graphs up to around fourteen vertices the \NaiveCycles{} algorithm
is still faster than \Subgraphs{}.
For graphs with more than eighteen vertices,
the growing advantage of \Subgraphs{} is already significant.

\begin{figure}[ht]
	\centering
	\scalebox{0.5}{%% Creator: Matplotlib, PGF backend
%%
%% To include the figure in your LaTeX document, write
%%   \input{<filename>.pgf}
%%
%% Make sure the required packages are loaded in your preamble
%%   \usepackage{pgf}
%%
%% Also ensure that all the required font packages are loaded; for instance,
%% the lmodern package is sometimes necessary when using math font.
%%   \usepackage{lmodern}
%%
%% Figures using additional raster images can only be included by \input if
%% they are in the same directory as the main LaTeX file. For loading figures
%% from other directories you can use the `import` package
%%   \usepackage{import}
%%
%% and then include the figures with
%%   \import{<path to file>}{<filename>.pgf}
%%
%% Matplotlib used the following preamble
%%   \def\mathdefault#1{#1}
%%   \everymath=\expandafter{\the\everymath\displaystyle}
%%   
%%   \ifdefined\pdftexversion\else  % non-pdftex case.
%%     \usepackage{fontspec}
%%     \setmainfont{DejaVuSans.ttf}[Path=\detokenize{/home/petr/Projects/PyRigi/.venv/lib/python3.12/site-packages/matplotlib/mpl-data/fonts/ttf/}]
%%     \setsansfont{DejaVuSans.ttf}[Path=\detokenize{/home/petr/Projects/PyRigi/.venv/lib/python3.12/site-packages/matplotlib/mpl-data/fonts/ttf/}]
%%     \setmonofont{DejaVuSansMono.ttf}[Path=\detokenize{/home/petr/Projects/PyRigi/.venv/lib/python3.12/site-packages/matplotlib/mpl-data/fonts/ttf/}]
%%   \fi
%%   \makeatletter\@ifpackageloaded{underscore}{}{\usepackage[strings]{underscore}}\makeatother
%%
\begingroup%
\makeatletter%
\begin{pgfpicture}%
\pgfpathrectangle{\pgfpointorigin}{\pgfqpoint{8.384376in}{2.25in}}%
\pgfusepath{use as bounding box, clip}%
\begin{pgfscope}%
\pgfsetbuttcap%
\pgfsetmiterjoin%
\definecolor{currentfill}{rgb}{1.000000,1.000000,1.000000}%
\pgfsetfillcolor{currentfill}%
\pgfsetlinewidth{0.000000pt}%
\definecolor{currentstroke}{rgb}{1.000000,1.000000,1.000000}%
\pgfsetstrokecolor{currentstroke}%
\pgfsetdash{}{0pt}%
\pgfpathmoveto{\pgfqpoint{0.000000in}{0.000000in}}%
\pgfpathlineto{\pgfqpoint{8.384376in}{0.000000in}}%
\pgfpathlineto{\pgfqpoint{8.384376in}{2.841849in}}%
\pgfpathlineto{\pgfqpoint{0.000000in}{2.841849in}}%
\pgfpathlineto{\pgfqpoint{0.000000in}{0.000000in}}%
\pgfpathclose%
\pgfusepath{fill}%
\end{pgfscope}%
\begin{pgfscope}%
\pgfsetbuttcap%
\pgfsetmiterjoin%
\definecolor{currentfill}{rgb}{1.000000,1.000000,1.000000}%
\pgfsetfillcolor{currentfill}%
\pgfsetlinewidth{0.000000pt}%
\definecolor{currentstroke}{rgb}{0.000000,0.000000,0.000000}%
\pgfsetstrokecolor{currentstroke}%
\pgfsetstrokeopacity{0.000000}%
\pgfsetdash{}{0pt}%
\pgfpathmoveto{\pgfqpoint{0.398419in}{0.521603in}}%
\pgfpathlineto{\pgfqpoint{5.487514in}{0.521603in}}%
\pgfpathlineto{\pgfqpoint{5.487514in}{2.206605in}}%
\pgfpathlineto{\pgfqpoint{0.398419in}{2.206605in}}%
\pgfpathlineto{\pgfqpoint{0.398419in}{0.521603in}}%
\pgfpathclose%
\pgfusepath{fill}%
\end{pgfscope}%
\begin{pgfscope}%
\pgfsetbuttcap%
\pgfsetroundjoin%
\definecolor{currentfill}{rgb}{0.000000,0.000000,0.000000}%
\pgfsetfillcolor{currentfill}%
\pgfsetlinewidth{0.803000pt}%
\definecolor{currentstroke}{rgb}{0.000000,0.000000,0.000000}%
\pgfsetstrokecolor{currentstroke}%
\pgfsetdash{}{0pt}%
\pgfsys@defobject{currentmarker}{\pgfqpoint{0.000000in}{-0.048611in}}{\pgfqpoint{0.000000in}{0.000000in}}{%
\pgfpathmoveto{\pgfqpoint{0.000000in}{0.000000in}}%
\pgfpathlineto{\pgfqpoint{0.000000in}{-0.048611in}}%
\pgfusepath{stroke,fill}%
}%
\begin{pgfscope}%
\pgfsys@transformshift{0.801091in}{0.521603in}%
\pgfsys@useobject{currentmarker}{}%
\end{pgfscope}%
\end{pgfscope}%
\begin{pgfscope}%
\definecolor{textcolor}{rgb}{0.000000,0.000000,0.000000}%
\pgfsetstrokecolor{textcolor}%
\pgfsetfillcolor{textcolor}%
\pgftext[x=0.801091in,y=0.424381in,,top]{\color{textcolor}{\rmfamily\fontsize{10.000000}{12.000000}\selectfont\catcode`\^=\active\def^{\ifmmode\sp\else\^{}\fi}\catcode`\%=\active\def%{\%}$\mathdefault{3}$}}%
\end{pgfscope}%
\begin{pgfscope}%
\pgfsetbuttcap%
\pgfsetroundjoin%
\definecolor{currentfill}{rgb}{0.000000,0.000000,0.000000}%
\pgfsetfillcolor{currentfill}%
\pgfsetlinewidth{0.803000pt}%
\definecolor{currentstroke}{rgb}{0.000000,0.000000,0.000000}%
\pgfsetstrokecolor{currentstroke}%
\pgfsetdash{}{0pt}%
\pgfsys@defobject{currentmarker}{\pgfqpoint{0.000000in}{-0.048611in}}{\pgfqpoint{0.000000in}{0.000000in}}{%
\pgfpathmoveto{\pgfqpoint{0.000000in}{0.000000in}}%
\pgfpathlineto{\pgfqpoint{0.000000in}{-0.048611in}}%
\pgfusepath{stroke,fill}%
}%
\begin{pgfscope}%
\pgfsys@transformshift{1.315141in}{0.521603in}%
\pgfsys@useobject{currentmarker}{}%
\end{pgfscope}%
\end{pgfscope}%
\begin{pgfscope}%
\definecolor{textcolor}{rgb}{0.000000,0.000000,0.000000}%
\pgfsetstrokecolor{textcolor}%
\pgfsetfillcolor{textcolor}%
\pgftext[x=1.315141in,y=0.424381in,,top]{\color{textcolor}{\rmfamily\fontsize{10.000000}{12.000000}\selectfont\catcode`\^=\active\def^{\ifmmode\sp\else\^{}\fi}\catcode`\%=\active\def%{\%}$\mathdefault{6}$}}%
\end{pgfscope}%
\begin{pgfscope}%
\pgfsetbuttcap%
\pgfsetroundjoin%
\definecolor{currentfill}{rgb}{0.000000,0.000000,0.000000}%
\pgfsetfillcolor{currentfill}%
\pgfsetlinewidth{0.803000pt}%
\definecolor{currentstroke}{rgb}{0.000000,0.000000,0.000000}%
\pgfsetstrokecolor{currentstroke}%
\pgfsetdash{}{0pt}%
\pgfsys@defobject{currentmarker}{\pgfqpoint{0.000000in}{-0.048611in}}{\pgfqpoint{0.000000in}{0.000000in}}{%
\pgfpathmoveto{\pgfqpoint{0.000000in}{0.000000in}}%
\pgfpathlineto{\pgfqpoint{0.000000in}{-0.048611in}}%
\pgfusepath{stroke,fill}%
}%
\begin{pgfscope}%
\pgfsys@transformshift{1.829191in}{0.521603in}%
\pgfsys@useobject{currentmarker}{}%
\end{pgfscope}%
\end{pgfscope}%
\begin{pgfscope}%
\definecolor{textcolor}{rgb}{0.000000,0.000000,0.000000}%
\pgfsetstrokecolor{textcolor}%
\pgfsetfillcolor{textcolor}%
\pgftext[x=1.829191in,y=0.424381in,,top]{\color{textcolor}{\rmfamily\fontsize{10.000000}{12.000000}\selectfont\catcode`\^=\active\def^{\ifmmode\sp\else\^{}\fi}\catcode`\%=\active\def%{\%}$\mathdefault{9}$}}%
\end{pgfscope}%
\begin{pgfscope}%
\pgfsetbuttcap%
\pgfsetroundjoin%
\definecolor{currentfill}{rgb}{0.000000,0.000000,0.000000}%
\pgfsetfillcolor{currentfill}%
\pgfsetlinewidth{0.803000pt}%
\definecolor{currentstroke}{rgb}{0.000000,0.000000,0.000000}%
\pgfsetstrokecolor{currentstroke}%
\pgfsetdash{}{0pt}%
\pgfsys@defobject{currentmarker}{\pgfqpoint{0.000000in}{-0.048611in}}{\pgfqpoint{0.000000in}{0.000000in}}{%
\pgfpathmoveto{\pgfqpoint{0.000000in}{0.000000in}}%
\pgfpathlineto{\pgfqpoint{0.000000in}{-0.048611in}}%
\pgfusepath{stroke,fill}%
}%
\begin{pgfscope}%
\pgfsys@transformshift{2.343241in}{0.521603in}%
\pgfsys@useobject{currentmarker}{}%
\end{pgfscope}%
\end{pgfscope}%
\begin{pgfscope}%
\definecolor{textcolor}{rgb}{0.000000,0.000000,0.000000}%
\pgfsetstrokecolor{textcolor}%
\pgfsetfillcolor{textcolor}%
\pgftext[x=2.343241in,y=0.424381in,,top]{\color{textcolor}{\rmfamily\fontsize{10.000000}{12.000000}\selectfont\catcode`\^=\active\def^{\ifmmode\sp\else\^{}\fi}\catcode`\%=\active\def%{\%}$\mathdefault{12}$}}%
\end{pgfscope}%
\begin{pgfscope}%
\pgfsetbuttcap%
\pgfsetroundjoin%
\definecolor{currentfill}{rgb}{0.000000,0.000000,0.000000}%
\pgfsetfillcolor{currentfill}%
\pgfsetlinewidth{0.803000pt}%
\definecolor{currentstroke}{rgb}{0.000000,0.000000,0.000000}%
\pgfsetstrokecolor{currentstroke}%
\pgfsetdash{}{0pt}%
\pgfsys@defobject{currentmarker}{\pgfqpoint{0.000000in}{-0.048611in}}{\pgfqpoint{0.000000in}{0.000000in}}{%
\pgfpathmoveto{\pgfqpoint{0.000000in}{0.000000in}}%
\pgfpathlineto{\pgfqpoint{0.000000in}{-0.048611in}}%
\pgfusepath{stroke,fill}%
}%
\begin{pgfscope}%
\pgfsys@transformshift{2.857291in}{0.521603in}%
\pgfsys@useobject{currentmarker}{}%
\end{pgfscope}%
\end{pgfscope}%
\begin{pgfscope}%
\definecolor{textcolor}{rgb}{0.000000,0.000000,0.000000}%
\pgfsetstrokecolor{textcolor}%
\pgfsetfillcolor{textcolor}%
\pgftext[x=2.857291in,y=0.424381in,,top]{\color{textcolor}{\rmfamily\fontsize{10.000000}{12.000000}\selectfont\catcode`\^=\active\def^{\ifmmode\sp\else\^{}\fi}\catcode`\%=\active\def%{\%}$\mathdefault{15}$}}%
\end{pgfscope}%
\begin{pgfscope}%
\pgfsetbuttcap%
\pgfsetroundjoin%
\definecolor{currentfill}{rgb}{0.000000,0.000000,0.000000}%
\pgfsetfillcolor{currentfill}%
\pgfsetlinewidth{0.803000pt}%
\definecolor{currentstroke}{rgb}{0.000000,0.000000,0.000000}%
\pgfsetstrokecolor{currentstroke}%
\pgfsetdash{}{0pt}%
\pgfsys@defobject{currentmarker}{\pgfqpoint{0.000000in}{-0.048611in}}{\pgfqpoint{0.000000in}{0.000000in}}{%
\pgfpathmoveto{\pgfqpoint{0.000000in}{0.000000in}}%
\pgfpathlineto{\pgfqpoint{0.000000in}{-0.048611in}}%
\pgfusepath{stroke,fill}%
}%
\begin{pgfscope}%
\pgfsys@transformshift{3.371341in}{0.521603in}%
\pgfsys@useobject{currentmarker}{}%
\end{pgfscope}%
\end{pgfscope}%
\begin{pgfscope}%
\definecolor{textcolor}{rgb}{0.000000,0.000000,0.000000}%
\pgfsetstrokecolor{textcolor}%
\pgfsetfillcolor{textcolor}%
\pgftext[x=3.371341in,y=0.424381in,,top]{\color{textcolor}{\rmfamily\fontsize{10.000000}{12.000000}\selectfont\catcode`\^=\active\def^{\ifmmode\sp\else\^{}\fi}\catcode`\%=\active\def%{\%}$\mathdefault{18}$}}%
\end{pgfscope}%
\begin{pgfscope}%
\pgfsetbuttcap%
\pgfsetroundjoin%
\definecolor{currentfill}{rgb}{0.000000,0.000000,0.000000}%
\pgfsetfillcolor{currentfill}%
\pgfsetlinewidth{0.803000pt}%
\definecolor{currentstroke}{rgb}{0.000000,0.000000,0.000000}%
\pgfsetstrokecolor{currentstroke}%
\pgfsetdash{}{0pt}%
\pgfsys@defobject{currentmarker}{\pgfqpoint{0.000000in}{-0.048611in}}{\pgfqpoint{0.000000in}{0.000000in}}{%
\pgfpathmoveto{\pgfqpoint{0.000000in}{0.000000in}}%
\pgfpathlineto{\pgfqpoint{0.000000in}{-0.048611in}}%
\pgfusepath{stroke,fill}%
}%
\begin{pgfscope}%
\pgfsys@transformshift{3.885391in}{0.521603in}%
\pgfsys@useobject{currentmarker}{}%
\end{pgfscope}%
\end{pgfscope}%
\begin{pgfscope}%
\definecolor{textcolor}{rgb}{0.000000,0.000000,0.000000}%
\pgfsetstrokecolor{textcolor}%
\pgfsetfillcolor{textcolor}%
\pgftext[x=3.885391in,y=0.424381in,,top]{\color{textcolor}{\rmfamily\fontsize{10.000000}{12.000000}\selectfont\catcode`\^=\active\def^{\ifmmode\sp\else\^{}\fi}\catcode`\%=\active\def%{\%}$\mathdefault{21}$}}%
\end{pgfscope}%
\begin{pgfscope}%
\pgfsetbuttcap%
\pgfsetroundjoin%
\definecolor{currentfill}{rgb}{0.000000,0.000000,0.000000}%
\pgfsetfillcolor{currentfill}%
\pgfsetlinewidth{0.803000pt}%
\definecolor{currentstroke}{rgb}{0.000000,0.000000,0.000000}%
\pgfsetstrokecolor{currentstroke}%
\pgfsetdash{}{0pt}%
\pgfsys@defobject{currentmarker}{\pgfqpoint{0.000000in}{-0.048611in}}{\pgfqpoint{0.000000in}{0.000000in}}{%
\pgfpathmoveto{\pgfqpoint{0.000000in}{0.000000in}}%
\pgfpathlineto{\pgfqpoint{0.000000in}{-0.048611in}}%
\pgfusepath{stroke,fill}%
}%
\begin{pgfscope}%
\pgfsys@transformshift{4.399441in}{0.521603in}%
\pgfsys@useobject{currentmarker}{}%
\end{pgfscope}%
\end{pgfscope}%
\begin{pgfscope}%
\definecolor{textcolor}{rgb}{0.000000,0.000000,0.000000}%
\pgfsetstrokecolor{textcolor}%
\pgfsetfillcolor{textcolor}%
\pgftext[x=4.399441in,y=0.424381in,,top]{\color{textcolor}{\rmfamily\fontsize{10.000000}{12.000000}\selectfont\catcode`\^=\active\def^{\ifmmode\sp\else\^{}\fi}\catcode`\%=\active\def%{\%}$\mathdefault{24}$}}%
\end{pgfscope}%
\begin{pgfscope}%
\pgfsetbuttcap%
\pgfsetroundjoin%
\definecolor{currentfill}{rgb}{0.000000,0.000000,0.000000}%
\pgfsetfillcolor{currentfill}%
\pgfsetlinewidth{0.803000pt}%
\definecolor{currentstroke}{rgb}{0.000000,0.000000,0.000000}%
\pgfsetstrokecolor{currentstroke}%
\pgfsetdash{}{0pt}%
\pgfsys@defobject{currentmarker}{\pgfqpoint{0.000000in}{-0.048611in}}{\pgfqpoint{0.000000in}{0.000000in}}{%
\pgfpathmoveto{\pgfqpoint{0.000000in}{0.000000in}}%
\pgfpathlineto{\pgfqpoint{0.000000in}{-0.048611in}}%
\pgfusepath{stroke,fill}%
}%
\begin{pgfscope}%
\pgfsys@transformshift{4.913491in}{0.521603in}%
\pgfsys@useobject{currentmarker}{}%
\end{pgfscope}%
\end{pgfscope}%
\begin{pgfscope}%
\definecolor{textcolor}{rgb}{0.000000,0.000000,0.000000}%
\pgfsetstrokecolor{textcolor}%
\pgfsetfillcolor{textcolor}%
\pgftext[x=4.913491in,y=0.424381in,,top]{\color{textcolor}{\rmfamily\fontsize{10.000000}{12.000000}\selectfont\catcode`\^=\active\def^{\ifmmode\sp\else\^{}\fi}\catcode`\%=\active\def%{\%}$\mathdefault{27}$}}%
\end{pgfscope}%
\begin{pgfscope}%
\pgfsetbuttcap%
\pgfsetroundjoin%
\definecolor{currentfill}{rgb}{0.000000,0.000000,0.000000}%
\pgfsetfillcolor{currentfill}%
\pgfsetlinewidth{0.803000pt}%
\definecolor{currentstroke}{rgb}{0.000000,0.000000,0.000000}%
\pgfsetstrokecolor{currentstroke}%
\pgfsetdash{}{0pt}%
\pgfsys@defobject{currentmarker}{\pgfqpoint{0.000000in}{-0.048611in}}{\pgfqpoint{0.000000in}{0.000000in}}{%
\pgfpathmoveto{\pgfqpoint{0.000000in}{0.000000in}}%
\pgfpathlineto{\pgfqpoint{0.000000in}{-0.048611in}}%
\pgfusepath{stroke,fill}%
}%
\begin{pgfscope}%
\pgfsys@transformshift{5.427541in}{0.521603in}%
\pgfsys@useobject{currentmarker}{}%
\end{pgfscope}%
\end{pgfscope}%
\begin{pgfscope}%
\definecolor{textcolor}{rgb}{0.000000,0.000000,0.000000}%
\pgfsetstrokecolor{textcolor}%
\pgfsetfillcolor{textcolor}%
\pgftext[x=5.427541in,y=0.424381in,,top]{\color{textcolor}{\rmfamily\fontsize{10.000000}{12.000000}\selectfont\catcode`\^=\active\def^{\ifmmode\sp\else\^{}\fi}\catcode`\%=\active\def%{\%}$\mathdefault{30}$}}%
\end{pgfscope}%
\begin{pgfscope}%
\definecolor{textcolor}{rgb}{0.000000,0.000000,0.000000}%
\pgfsetstrokecolor{textcolor}%
\pgfsetfillcolor{textcolor}%
\pgftext[x=2.942966in,y=0.234413in,,top]{\color{textcolor}{\rmfamily\fontsize{10.000000}{12.000000}\selectfont\catcode`\^=\active\def^{\ifmmode\sp\else\^{}\fi}\catcode`\%=\active\def%{\%}Monochromatic classes}}%
\end{pgfscope}%
\begin{pgfscope}%
\pgfsetbuttcap%
\pgfsetroundjoin%
\definecolor{currentfill}{rgb}{0.000000,0.000000,0.000000}%
\pgfsetfillcolor{currentfill}%
\pgfsetlinewidth{0.803000pt}%
\definecolor{currentstroke}{rgb}{0.000000,0.000000,0.000000}%
\pgfsetstrokecolor{currentstroke}%
\pgfsetdash{}{0pt}%
\pgfsys@defobject{currentmarker}{\pgfqpoint{-0.048611in}{0.000000in}}{\pgfqpoint{-0.000000in}{0.000000in}}{%
\pgfpathmoveto{\pgfqpoint{-0.000000in}{0.000000in}}%
\pgfpathlineto{\pgfqpoint{-0.048611in}{0.000000in}}%
\pgfusepath{stroke,fill}%
}%
\begin{pgfscope}%
\pgfsys@transformshift{0.398419in}{0.592804in}%
\pgfsys@useobject{currentmarker}{}%
\end{pgfscope}%
\end{pgfscope}%
\begin{pgfscope}%
\definecolor{textcolor}{rgb}{0.000000,0.000000,0.000000}%
\pgfsetstrokecolor{textcolor}%
\pgfsetfillcolor{textcolor}%
\pgftext[x=0.100000in, y=0.540043in, left, base]{\color{textcolor}{\rmfamily\fontsize{10.000000}{12.000000}\selectfont\catcode`\^=\active\def^{\ifmmode\sp\else\^{}\fi}\catcode`\%=\active\def%{\%}$\mathdefault{10^{0}}$}}%
\end{pgfscope}%
\begin{pgfscope}%
\pgfsetbuttcap%
\pgfsetroundjoin%
\definecolor{currentfill}{rgb}{0.000000,0.000000,0.000000}%
\pgfsetfillcolor{currentfill}%
\pgfsetlinewidth{0.803000pt}%
\definecolor{currentstroke}{rgb}{0.000000,0.000000,0.000000}%
\pgfsetstrokecolor{currentstroke}%
\pgfsetdash{}{0pt}%
\pgfsys@defobject{currentmarker}{\pgfqpoint{-0.048611in}{0.000000in}}{\pgfqpoint{-0.000000in}{0.000000in}}{%
\pgfpathmoveto{\pgfqpoint{-0.000000in}{0.000000in}}%
\pgfpathlineto{\pgfqpoint{-0.048611in}{0.000000in}}%
\pgfusepath{stroke,fill}%
}%
\begin{pgfscope}%
\pgfsys@transformshift{0.398419in}{0.996143in}%
\pgfsys@useobject{currentmarker}{}%
\end{pgfscope}%
\end{pgfscope}%
\begin{pgfscope}%
\definecolor{textcolor}{rgb}{0.000000,0.000000,0.000000}%
\pgfsetstrokecolor{textcolor}%
\pgfsetfillcolor{textcolor}%
\pgftext[x=0.100000in, y=0.943382in, left, base]{\color{textcolor}{\rmfamily\fontsize{10.000000}{12.000000}\selectfont\catcode`\^=\active\def^{\ifmmode\sp\else\^{}\fi}\catcode`\%=\active\def%{\%}$\mathdefault{10^{1}}$}}%
\end{pgfscope}%
\begin{pgfscope}%
\pgfsetbuttcap%
\pgfsetroundjoin%
\definecolor{currentfill}{rgb}{0.000000,0.000000,0.000000}%
\pgfsetfillcolor{currentfill}%
\pgfsetlinewidth{0.803000pt}%
\definecolor{currentstroke}{rgb}{0.000000,0.000000,0.000000}%
\pgfsetstrokecolor{currentstroke}%
\pgfsetdash{}{0pt}%
\pgfsys@defobject{currentmarker}{\pgfqpoint{-0.048611in}{0.000000in}}{\pgfqpoint{-0.000000in}{0.000000in}}{%
\pgfpathmoveto{\pgfqpoint{-0.000000in}{0.000000in}}%
\pgfpathlineto{\pgfqpoint{-0.048611in}{0.000000in}}%
\pgfusepath{stroke,fill}%
}%
\begin{pgfscope}%
\pgfsys@transformshift{0.398419in}{1.399482in}%
\pgfsys@useobject{currentmarker}{}%
\end{pgfscope}%
\end{pgfscope}%
\begin{pgfscope}%
\definecolor{textcolor}{rgb}{0.000000,0.000000,0.000000}%
\pgfsetstrokecolor{textcolor}%
\pgfsetfillcolor{textcolor}%
\pgftext[x=0.100000in, y=1.346721in, left, base]{\color{textcolor}{\rmfamily\fontsize{10.000000}{12.000000}\selectfont\catcode`\^=\active\def^{\ifmmode\sp\else\^{}\fi}\catcode`\%=\active\def%{\%}$\mathdefault{10^{2}}$}}%
\end{pgfscope}%
\begin{pgfscope}%
\pgfsetbuttcap%
\pgfsetroundjoin%
\definecolor{currentfill}{rgb}{0.000000,0.000000,0.000000}%
\pgfsetfillcolor{currentfill}%
\pgfsetlinewidth{0.803000pt}%
\definecolor{currentstroke}{rgb}{0.000000,0.000000,0.000000}%
\pgfsetstrokecolor{currentstroke}%
\pgfsetdash{}{0pt}%
\pgfsys@defobject{currentmarker}{\pgfqpoint{-0.048611in}{0.000000in}}{\pgfqpoint{-0.000000in}{0.000000in}}{%
\pgfpathmoveto{\pgfqpoint{-0.000000in}{0.000000in}}%
\pgfpathlineto{\pgfqpoint{-0.048611in}{0.000000in}}%
\pgfusepath{stroke,fill}%
}%
\begin{pgfscope}%
\pgfsys@transformshift{0.398419in}{1.802821in}%
\pgfsys@useobject{currentmarker}{}%
\end{pgfscope}%
\end{pgfscope}%
\begin{pgfscope}%
\definecolor{textcolor}{rgb}{0.000000,0.000000,0.000000}%
\pgfsetstrokecolor{textcolor}%
\pgfsetfillcolor{textcolor}%
\pgftext[x=0.100000in, y=1.750060in, left, base]{\color{textcolor}{\rmfamily\fontsize{10.000000}{12.000000}\selectfont\catcode`\^=\active\def^{\ifmmode\sp\else\^{}\fi}\catcode`\%=\active\def%{\%}$\mathdefault{10^{3}}$}}%
\end{pgfscope}%
\begin{pgfscope}%
\pgfsetbuttcap%
\pgfsetroundjoin%
\definecolor{currentfill}{rgb}{0.000000,0.000000,0.000000}%
\pgfsetfillcolor{currentfill}%
\pgfsetlinewidth{0.803000pt}%
\definecolor{currentstroke}{rgb}{0.000000,0.000000,0.000000}%
\pgfsetstrokecolor{currentstroke}%
\pgfsetdash{}{0pt}%
\pgfsys@defobject{currentmarker}{\pgfqpoint{-0.048611in}{0.000000in}}{\pgfqpoint{-0.000000in}{0.000000in}}{%
\pgfpathmoveto{\pgfqpoint{-0.000000in}{0.000000in}}%
\pgfpathlineto{\pgfqpoint{-0.048611in}{0.000000in}}%
\pgfusepath{stroke,fill}%
}%
\begin{pgfscope}%
\pgfsys@transformshift{0.398419in}{2.206160in}%
\pgfsys@useobject{currentmarker}{}%
\end{pgfscope}%
\end{pgfscope}%
\begin{pgfscope}%
\definecolor{textcolor}{rgb}{0.000000,0.000000,0.000000}%
\pgfsetstrokecolor{textcolor}%
\pgfsetfillcolor{textcolor}%
\pgftext[x=0.100000in, y=2.153398in, left, base]{\color{textcolor}{\rmfamily\fontsize{10.000000}{12.000000}\selectfont\catcode`\^=\active\def^{\ifmmode\sp\else\^{}\fi}\catcode`\%=\active\def%{\%}$\mathdefault{10^{4}}$}}%
\end{pgfscope}%
\begin{pgfscope}%
\pgfsetbuttcap%
\pgfsetroundjoin%
\definecolor{currentfill}{rgb}{0.000000,0.000000,0.000000}%
\pgfsetfillcolor{currentfill}%
\pgfsetlinewidth{0.602250pt}%
\definecolor{currentstroke}{rgb}{0.000000,0.000000,0.000000}%
\pgfsetstrokecolor{currentstroke}%
\pgfsetdash{}{0pt}%
\pgfsys@defobject{currentmarker}{\pgfqpoint{-0.027778in}{0.000000in}}{\pgfqpoint{-0.000000in}{0.000000in}}{%
\pgfpathmoveto{\pgfqpoint{-0.000000in}{0.000000in}}%
\pgfpathlineto{\pgfqpoint{-0.027778in}{0.000000in}}%
\pgfusepath{stroke,fill}%
}%
\begin{pgfscope}%
\pgfsys@transformshift{0.398419in}{0.530326in}%
\pgfsys@useobject{currentmarker}{}%
\end{pgfscope}%
\end{pgfscope}%
\begin{pgfscope}%
\pgfsetbuttcap%
\pgfsetroundjoin%
\definecolor{currentfill}{rgb}{0.000000,0.000000,0.000000}%
\pgfsetfillcolor{currentfill}%
\pgfsetlinewidth{0.602250pt}%
\definecolor{currentstroke}{rgb}{0.000000,0.000000,0.000000}%
\pgfsetstrokecolor{currentstroke}%
\pgfsetdash{}{0pt}%
\pgfsys@defobject{currentmarker}{\pgfqpoint{-0.027778in}{0.000000in}}{\pgfqpoint{-0.000000in}{0.000000in}}{%
\pgfpathmoveto{\pgfqpoint{-0.000000in}{0.000000in}}%
\pgfpathlineto{\pgfqpoint{-0.027778in}{0.000000in}}%
\pgfusepath{stroke,fill}%
}%
\begin{pgfscope}%
\pgfsys@transformshift{0.398419in}{0.553717in}%
\pgfsys@useobject{currentmarker}{}%
\end{pgfscope}%
\end{pgfscope}%
\begin{pgfscope}%
\pgfsetbuttcap%
\pgfsetroundjoin%
\definecolor{currentfill}{rgb}{0.000000,0.000000,0.000000}%
\pgfsetfillcolor{currentfill}%
\pgfsetlinewidth{0.602250pt}%
\definecolor{currentstroke}{rgb}{0.000000,0.000000,0.000000}%
\pgfsetstrokecolor{currentstroke}%
\pgfsetdash{}{0pt}%
\pgfsys@defobject{currentmarker}{\pgfqpoint{-0.027778in}{0.000000in}}{\pgfqpoint{-0.000000in}{0.000000in}}{%
\pgfpathmoveto{\pgfqpoint{-0.000000in}{0.000000in}}%
\pgfpathlineto{\pgfqpoint{-0.027778in}{0.000000in}}%
\pgfusepath{stroke,fill}%
}%
\begin{pgfscope}%
\pgfsys@transformshift{0.398419in}{0.574348in}%
\pgfsys@useobject{currentmarker}{}%
\end{pgfscope}%
\end{pgfscope}%
\begin{pgfscope}%
\pgfsetbuttcap%
\pgfsetroundjoin%
\definecolor{currentfill}{rgb}{0.000000,0.000000,0.000000}%
\pgfsetfillcolor{currentfill}%
\pgfsetlinewidth{0.602250pt}%
\definecolor{currentstroke}{rgb}{0.000000,0.000000,0.000000}%
\pgfsetstrokecolor{currentstroke}%
\pgfsetdash{}{0pt}%
\pgfsys@defobject{currentmarker}{\pgfqpoint{-0.027778in}{0.000000in}}{\pgfqpoint{-0.000000in}{0.000000in}}{%
\pgfpathmoveto{\pgfqpoint{-0.000000in}{0.000000in}}%
\pgfpathlineto{\pgfqpoint{-0.027778in}{0.000000in}}%
\pgfusepath{stroke,fill}%
}%
\begin{pgfscope}%
\pgfsys@transformshift{0.398419in}{0.714221in}%
\pgfsys@useobject{currentmarker}{}%
\end{pgfscope}%
\end{pgfscope}%
\begin{pgfscope}%
\pgfsetbuttcap%
\pgfsetroundjoin%
\definecolor{currentfill}{rgb}{0.000000,0.000000,0.000000}%
\pgfsetfillcolor{currentfill}%
\pgfsetlinewidth{0.602250pt}%
\definecolor{currentstroke}{rgb}{0.000000,0.000000,0.000000}%
\pgfsetstrokecolor{currentstroke}%
\pgfsetdash{}{0pt}%
\pgfsys@defobject{currentmarker}{\pgfqpoint{-0.027778in}{0.000000in}}{\pgfqpoint{-0.000000in}{0.000000in}}{%
\pgfpathmoveto{\pgfqpoint{-0.000000in}{0.000000in}}%
\pgfpathlineto{\pgfqpoint{-0.027778in}{0.000000in}}%
\pgfusepath{stroke,fill}%
}%
\begin{pgfscope}%
\pgfsys@transformshift{0.398419in}{0.785246in}%
\pgfsys@useobject{currentmarker}{}%
\end{pgfscope}%
\end{pgfscope}%
\begin{pgfscope}%
\pgfsetbuttcap%
\pgfsetroundjoin%
\definecolor{currentfill}{rgb}{0.000000,0.000000,0.000000}%
\pgfsetfillcolor{currentfill}%
\pgfsetlinewidth{0.602250pt}%
\definecolor{currentstroke}{rgb}{0.000000,0.000000,0.000000}%
\pgfsetstrokecolor{currentstroke}%
\pgfsetdash{}{0pt}%
\pgfsys@defobject{currentmarker}{\pgfqpoint{-0.027778in}{0.000000in}}{\pgfqpoint{-0.000000in}{0.000000in}}{%
\pgfpathmoveto{\pgfqpoint{-0.000000in}{0.000000in}}%
\pgfpathlineto{\pgfqpoint{-0.027778in}{0.000000in}}%
\pgfusepath{stroke,fill}%
}%
\begin{pgfscope}%
\pgfsys@transformshift{0.398419in}{0.835638in}%
\pgfsys@useobject{currentmarker}{}%
\end{pgfscope}%
\end{pgfscope}%
\begin{pgfscope}%
\pgfsetbuttcap%
\pgfsetroundjoin%
\definecolor{currentfill}{rgb}{0.000000,0.000000,0.000000}%
\pgfsetfillcolor{currentfill}%
\pgfsetlinewidth{0.602250pt}%
\definecolor{currentstroke}{rgb}{0.000000,0.000000,0.000000}%
\pgfsetstrokecolor{currentstroke}%
\pgfsetdash{}{0pt}%
\pgfsys@defobject{currentmarker}{\pgfqpoint{-0.027778in}{0.000000in}}{\pgfqpoint{-0.000000in}{0.000000in}}{%
\pgfpathmoveto{\pgfqpoint{-0.000000in}{0.000000in}}%
\pgfpathlineto{\pgfqpoint{-0.027778in}{0.000000in}}%
\pgfusepath{stroke,fill}%
}%
\begin{pgfscope}%
\pgfsys@transformshift{0.398419in}{0.874726in}%
\pgfsys@useobject{currentmarker}{}%
\end{pgfscope}%
\end{pgfscope}%
\begin{pgfscope}%
\pgfsetbuttcap%
\pgfsetroundjoin%
\definecolor{currentfill}{rgb}{0.000000,0.000000,0.000000}%
\pgfsetfillcolor{currentfill}%
\pgfsetlinewidth{0.602250pt}%
\definecolor{currentstroke}{rgb}{0.000000,0.000000,0.000000}%
\pgfsetstrokecolor{currentstroke}%
\pgfsetdash{}{0pt}%
\pgfsys@defobject{currentmarker}{\pgfqpoint{-0.027778in}{0.000000in}}{\pgfqpoint{-0.000000in}{0.000000in}}{%
\pgfpathmoveto{\pgfqpoint{-0.000000in}{0.000000in}}%
\pgfpathlineto{\pgfqpoint{-0.027778in}{0.000000in}}%
\pgfusepath{stroke,fill}%
}%
\begin{pgfscope}%
\pgfsys@transformshift{0.398419in}{0.906663in}%
\pgfsys@useobject{currentmarker}{}%
\end{pgfscope}%
\end{pgfscope}%
\begin{pgfscope}%
\pgfsetbuttcap%
\pgfsetroundjoin%
\definecolor{currentfill}{rgb}{0.000000,0.000000,0.000000}%
\pgfsetfillcolor{currentfill}%
\pgfsetlinewidth{0.602250pt}%
\definecolor{currentstroke}{rgb}{0.000000,0.000000,0.000000}%
\pgfsetstrokecolor{currentstroke}%
\pgfsetdash{}{0pt}%
\pgfsys@defobject{currentmarker}{\pgfqpoint{-0.027778in}{0.000000in}}{\pgfqpoint{-0.000000in}{0.000000in}}{%
\pgfpathmoveto{\pgfqpoint{-0.000000in}{0.000000in}}%
\pgfpathlineto{\pgfqpoint{-0.027778in}{0.000000in}}%
\pgfusepath{stroke,fill}%
}%
\begin{pgfscope}%
\pgfsys@transformshift{0.398419in}{0.933665in}%
\pgfsys@useobject{currentmarker}{}%
\end{pgfscope}%
\end{pgfscope}%
\begin{pgfscope}%
\pgfsetbuttcap%
\pgfsetroundjoin%
\definecolor{currentfill}{rgb}{0.000000,0.000000,0.000000}%
\pgfsetfillcolor{currentfill}%
\pgfsetlinewidth{0.602250pt}%
\definecolor{currentstroke}{rgb}{0.000000,0.000000,0.000000}%
\pgfsetstrokecolor{currentstroke}%
\pgfsetdash{}{0pt}%
\pgfsys@defobject{currentmarker}{\pgfqpoint{-0.027778in}{0.000000in}}{\pgfqpoint{-0.000000in}{0.000000in}}{%
\pgfpathmoveto{\pgfqpoint{-0.000000in}{0.000000in}}%
\pgfpathlineto{\pgfqpoint{-0.027778in}{0.000000in}}%
\pgfusepath{stroke,fill}%
}%
\begin{pgfscope}%
\pgfsys@transformshift{0.398419in}{0.957056in}%
\pgfsys@useobject{currentmarker}{}%
\end{pgfscope}%
\end{pgfscope}%
\begin{pgfscope}%
\pgfsetbuttcap%
\pgfsetroundjoin%
\definecolor{currentfill}{rgb}{0.000000,0.000000,0.000000}%
\pgfsetfillcolor{currentfill}%
\pgfsetlinewidth{0.602250pt}%
\definecolor{currentstroke}{rgb}{0.000000,0.000000,0.000000}%
\pgfsetstrokecolor{currentstroke}%
\pgfsetdash{}{0pt}%
\pgfsys@defobject{currentmarker}{\pgfqpoint{-0.027778in}{0.000000in}}{\pgfqpoint{-0.000000in}{0.000000in}}{%
\pgfpathmoveto{\pgfqpoint{-0.000000in}{0.000000in}}%
\pgfpathlineto{\pgfqpoint{-0.027778in}{0.000000in}}%
\pgfusepath{stroke,fill}%
}%
\begin{pgfscope}%
\pgfsys@transformshift{0.398419in}{0.977687in}%
\pgfsys@useobject{currentmarker}{}%
\end{pgfscope}%
\end{pgfscope}%
\begin{pgfscope}%
\pgfsetbuttcap%
\pgfsetroundjoin%
\definecolor{currentfill}{rgb}{0.000000,0.000000,0.000000}%
\pgfsetfillcolor{currentfill}%
\pgfsetlinewidth{0.602250pt}%
\definecolor{currentstroke}{rgb}{0.000000,0.000000,0.000000}%
\pgfsetstrokecolor{currentstroke}%
\pgfsetdash{}{0pt}%
\pgfsys@defobject{currentmarker}{\pgfqpoint{-0.027778in}{0.000000in}}{\pgfqpoint{-0.000000in}{0.000000in}}{%
\pgfpathmoveto{\pgfqpoint{-0.000000in}{0.000000in}}%
\pgfpathlineto{\pgfqpoint{-0.027778in}{0.000000in}}%
\pgfusepath{stroke,fill}%
}%
\begin{pgfscope}%
\pgfsys@transformshift{0.398419in}{1.117560in}%
\pgfsys@useobject{currentmarker}{}%
\end{pgfscope}%
\end{pgfscope}%
\begin{pgfscope}%
\pgfsetbuttcap%
\pgfsetroundjoin%
\definecolor{currentfill}{rgb}{0.000000,0.000000,0.000000}%
\pgfsetfillcolor{currentfill}%
\pgfsetlinewidth{0.602250pt}%
\definecolor{currentstroke}{rgb}{0.000000,0.000000,0.000000}%
\pgfsetstrokecolor{currentstroke}%
\pgfsetdash{}{0pt}%
\pgfsys@defobject{currentmarker}{\pgfqpoint{-0.027778in}{0.000000in}}{\pgfqpoint{-0.000000in}{0.000000in}}{%
\pgfpathmoveto{\pgfqpoint{-0.000000in}{0.000000in}}%
\pgfpathlineto{\pgfqpoint{-0.027778in}{0.000000in}}%
\pgfusepath{stroke,fill}%
}%
\begin{pgfscope}%
\pgfsys@transformshift{0.398419in}{1.188585in}%
\pgfsys@useobject{currentmarker}{}%
\end{pgfscope}%
\end{pgfscope}%
\begin{pgfscope}%
\pgfsetbuttcap%
\pgfsetroundjoin%
\definecolor{currentfill}{rgb}{0.000000,0.000000,0.000000}%
\pgfsetfillcolor{currentfill}%
\pgfsetlinewidth{0.602250pt}%
\definecolor{currentstroke}{rgb}{0.000000,0.000000,0.000000}%
\pgfsetstrokecolor{currentstroke}%
\pgfsetdash{}{0pt}%
\pgfsys@defobject{currentmarker}{\pgfqpoint{-0.027778in}{0.000000in}}{\pgfqpoint{-0.000000in}{0.000000in}}{%
\pgfpathmoveto{\pgfqpoint{-0.000000in}{0.000000in}}%
\pgfpathlineto{\pgfqpoint{-0.027778in}{0.000000in}}%
\pgfusepath{stroke,fill}%
}%
\begin{pgfscope}%
\pgfsys@transformshift{0.398419in}{1.238977in}%
\pgfsys@useobject{currentmarker}{}%
\end{pgfscope}%
\end{pgfscope}%
\begin{pgfscope}%
\pgfsetbuttcap%
\pgfsetroundjoin%
\definecolor{currentfill}{rgb}{0.000000,0.000000,0.000000}%
\pgfsetfillcolor{currentfill}%
\pgfsetlinewidth{0.602250pt}%
\definecolor{currentstroke}{rgb}{0.000000,0.000000,0.000000}%
\pgfsetstrokecolor{currentstroke}%
\pgfsetdash{}{0pt}%
\pgfsys@defobject{currentmarker}{\pgfqpoint{-0.027778in}{0.000000in}}{\pgfqpoint{-0.000000in}{0.000000in}}{%
\pgfpathmoveto{\pgfqpoint{-0.000000in}{0.000000in}}%
\pgfpathlineto{\pgfqpoint{-0.027778in}{0.000000in}}%
\pgfusepath{stroke,fill}%
}%
\begin{pgfscope}%
\pgfsys@transformshift{0.398419in}{1.278065in}%
\pgfsys@useobject{currentmarker}{}%
\end{pgfscope}%
\end{pgfscope}%
\begin{pgfscope}%
\pgfsetbuttcap%
\pgfsetroundjoin%
\definecolor{currentfill}{rgb}{0.000000,0.000000,0.000000}%
\pgfsetfillcolor{currentfill}%
\pgfsetlinewidth{0.602250pt}%
\definecolor{currentstroke}{rgb}{0.000000,0.000000,0.000000}%
\pgfsetstrokecolor{currentstroke}%
\pgfsetdash{}{0pt}%
\pgfsys@defobject{currentmarker}{\pgfqpoint{-0.027778in}{0.000000in}}{\pgfqpoint{-0.000000in}{0.000000in}}{%
\pgfpathmoveto{\pgfqpoint{-0.000000in}{0.000000in}}%
\pgfpathlineto{\pgfqpoint{-0.027778in}{0.000000in}}%
\pgfusepath{stroke,fill}%
}%
\begin{pgfscope}%
\pgfsys@transformshift{0.398419in}{1.310002in}%
\pgfsys@useobject{currentmarker}{}%
\end{pgfscope}%
\end{pgfscope}%
\begin{pgfscope}%
\pgfsetbuttcap%
\pgfsetroundjoin%
\definecolor{currentfill}{rgb}{0.000000,0.000000,0.000000}%
\pgfsetfillcolor{currentfill}%
\pgfsetlinewidth{0.602250pt}%
\definecolor{currentstroke}{rgb}{0.000000,0.000000,0.000000}%
\pgfsetstrokecolor{currentstroke}%
\pgfsetdash{}{0pt}%
\pgfsys@defobject{currentmarker}{\pgfqpoint{-0.027778in}{0.000000in}}{\pgfqpoint{-0.000000in}{0.000000in}}{%
\pgfpathmoveto{\pgfqpoint{-0.000000in}{0.000000in}}%
\pgfpathlineto{\pgfqpoint{-0.027778in}{0.000000in}}%
\pgfusepath{stroke,fill}%
}%
\begin{pgfscope}%
\pgfsys@transformshift{0.398419in}{1.337004in}%
\pgfsys@useobject{currentmarker}{}%
\end{pgfscope}%
\end{pgfscope}%
\begin{pgfscope}%
\pgfsetbuttcap%
\pgfsetroundjoin%
\definecolor{currentfill}{rgb}{0.000000,0.000000,0.000000}%
\pgfsetfillcolor{currentfill}%
\pgfsetlinewidth{0.602250pt}%
\definecolor{currentstroke}{rgb}{0.000000,0.000000,0.000000}%
\pgfsetstrokecolor{currentstroke}%
\pgfsetdash{}{0pt}%
\pgfsys@defobject{currentmarker}{\pgfqpoint{-0.027778in}{0.000000in}}{\pgfqpoint{-0.000000in}{0.000000in}}{%
\pgfpathmoveto{\pgfqpoint{-0.000000in}{0.000000in}}%
\pgfpathlineto{\pgfqpoint{-0.027778in}{0.000000in}}%
\pgfusepath{stroke,fill}%
}%
\begin{pgfscope}%
\pgfsys@transformshift{0.398419in}{1.360394in}%
\pgfsys@useobject{currentmarker}{}%
\end{pgfscope}%
\end{pgfscope}%
\begin{pgfscope}%
\pgfsetbuttcap%
\pgfsetroundjoin%
\definecolor{currentfill}{rgb}{0.000000,0.000000,0.000000}%
\pgfsetfillcolor{currentfill}%
\pgfsetlinewidth{0.602250pt}%
\definecolor{currentstroke}{rgb}{0.000000,0.000000,0.000000}%
\pgfsetstrokecolor{currentstroke}%
\pgfsetdash{}{0pt}%
\pgfsys@defobject{currentmarker}{\pgfqpoint{-0.027778in}{0.000000in}}{\pgfqpoint{-0.000000in}{0.000000in}}{%
\pgfpathmoveto{\pgfqpoint{-0.000000in}{0.000000in}}%
\pgfpathlineto{\pgfqpoint{-0.027778in}{0.000000in}}%
\pgfusepath{stroke,fill}%
}%
\begin{pgfscope}%
\pgfsys@transformshift{0.398419in}{1.381026in}%
\pgfsys@useobject{currentmarker}{}%
\end{pgfscope}%
\end{pgfscope}%
\begin{pgfscope}%
\pgfsetbuttcap%
\pgfsetroundjoin%
\definecolor{currentfill}{rgb}{0.000000,0.000000,0.000000}%
\pgfsetfillcolor{currentfill}%
\pgfsetlinewidth{0.602250pt}%
\definecolor{currentstroke}{rgb}{0.000000,0.000000,0.000000}%
\pgfsetstrokecolor{currentstroke}%
\pgfsetdash{}{0pt}%
\pgfsys@defobject{currentmarker}{\pgfqpoint{-0.027778in}{0.000000in}}{\pgfqpoint{-0.000000in}{0.000000in}}{%
\pgfpathmoveto{\pgfqpoint{-0.000000in}{0.000000in}}%
\pgfpathlineto{\pgfqpoint{-0.027778in}{0.000000in}}%
\pgfusepath{stroke,fill}%
}%
\begin{pgfscope}%
\pgfsys@transformshift{0.398419in}{1.520899in}%
\pgfsys@useobject{currentmarker}{}%
\end{pgfscope}%
\end{pgfscope}%
\begin{pgfscope}%
\pgfsetbuttcap%
\pgfsetroundjoin%
\definecolor{currentfill}{rgb}{0.000000,0.000000,0.000000}%
\pgfsetfillcolor{currentfill}%
\pgfsetlinewidth{0.602250pt}%
\definecolor{currentstroke}{rgb}{0.000000,0.000000,0.000000}%
\pgfsetstrokecolor{currentstroke}%
\pgfsetdash{}{0pt}%
\pgfsys@defobject{currentmarker}{\pgfqpoint{-0.027778in}{0.000000in}}{\pgfqpoint{-0.000000in}{0.000000in}}{%
\pgfpathmoveto{\pgfqpoint{-0.000000in}{0.000000in}}%
\pgfpathlineto{\pgfqpoint{-0.027778in}{0.000000in}}%
\pgfusepath{stroke,fill}%
}%
\begin{pgfscope}%
\pgfsys@transformshift{0.398419in}{1.591924in}%
\pgfsys@useobject{currentmarker}{}%
\end{pgfscope}%
\end{pgfscope}%
\begin{pgfscope}%
\pgfsetbuttcap%
\pgfsetroundjoin%
\definecolor{currentfill}{rgb}{0.000000,0.000000,0.000000}%
\pgfsetfillcolor{currentfill}%
\pgfsetlinewidth{0.602250pt}%
\definecolor{currentstroke}{rgb}{0.000000,0.000000,0.000000}%
\pgfsetstrokecolor{currentstroke}%
\pgfsetdash{}{0pt}%
\pgfsys@defobject{currentmarker}{\pgfqpoint{-0.027778in}{0.000000in}}{\pgfqpoint{-0.000000in}{0.000000in}}{%
\pgfpathmoveto{\pgfqpoint{-0.000000in}{0.000000in}}%
\pgfpathlineto{\pgfqpoint{-0.027778in}{0.000000in}}%
\pgfusepath{stroke,fill}%
}%
\begin{pgfscope}%
\pgfsys@transformshift{0.398419in}{1.642316in}%
\pgfsys@useobject{currentmarker}{}%
\end{pgfscope}%
\end{pgfscope}%
\begin{pgfscope}%
\pgfsetbuttcap%
\pgfsetroundjoin%
\definecolor{currentfill}{rgb}{0.000000,0.000000,0.000000}%
\pgfsetfillcolor{currentfill}%
\pgfsetlinewidth{0.602250pt}%
\definecolor{currentstroke}{rgb}{0.000000,0.000000,0.000000}%
\pgfsetstrokecolor{currentstroke}%
\pgfsetdash{}{0pt}%
\pgfsys@defobject{currentmarker}{\pgfqpoint{-0.027778in}{0.000000in}}{\pgfqpoint{-0.000000in}{0.000000in}}{%
\pgfpathmoveto{\pgfqpoint{-0.000000in}{0.000000in}}%
\pgfpathlineto{\pgfqpoint{-0.027778in}{0.000000in}}%
\pgfusepath{stroke,fill}%
}%
\begin{pgfscope}%
\pgfsys@transformshift{0.398419in}{1.681404in}%
\pgfsys@useobject{currentmarker}{}%
\end{pgfscope}%
\end{pgfscope}%
\begin{pgfscope}%
\pgfsetbuttcap%
\pgfsetroundjoin%
\definecolor{currentfill}{rgb}{0.000000,0.000000,0.000000}%
\pgfsetfillcolor{currentfill}%
\pgfsetlinewidth{0.602250pt}%
\definecolor{currentstroke}{rgb}{0.000000,0.000000,0.000000}%
\pgfsetstrokecolor{currentstroke}%
\pgfsetdash{}{0pt}%
\pgfsys@defobject{currentmarker}{\pgfqpoint{-0.027778in}{0.000000in}}{\pgfqpoint{-0.000000in}{0.000000in}}{%
\pgfpathmoveto{\pgfqpoint{-0.000000in}{0.000000in}}%
\pgfpathlineto{\pgfqpoint{-0.027778in}{0.000000in}}%
\pgfusepath{stroke,fill}%
}%
\begin{pgfscope}%
\pgfsys@transformshift{0.398419in}{1.713341in}%
\pgfsys@useobject{currentmarker}{}%
\end{pgfscope}%
\end{pgfscope}%
\begin{pgfscope}%
\pgfsetbuttcap%
\pgfsetroundjoin%
\definecolor{currentfill}{rgb}{0.000000,0.000000,0.000000}%
\pgfsetfillcolor{currentfill}%
\pgfsetlinewidth{0.602250pt}%
\definecolor{currentstroke}{rgb}{0.000000,0.000000,0.000000}%
\pgfsetstrokecolor{currentstroke}%
\pgfsetdash{}{0pt}%
\pgfsys@defobject{currentmarker}{\pgfqpoint{-0.027778in}{0.000000in}}{\pgfqpoint{-0.000000in}{0.000000in}}{%
\pgfpathmoveto{\pgfqpoint{-0.000000in}{0.000000in}}%
\pgfpathlineto{\pgfqpoint{-0.027778in}{0.000000in}}%
\pgfusepath{stroke,fill}%
}%
\begin{pgfscope}%
\pgfsys@transformshift{0.398419in}{1.740343in}%
\pgfsys@useobject{currentmarker}{}%
\end{pgfscope}%
\end{pgfscope}%
\begin{pgfscope}%
\pgfsetbuttcap%
\pgfsetroundjoin%
\definecolor{currentfill}{rgb}{0.000000,0.000000,0.000000}%
\pgfsetfillcolor{currentfill}%
\pgfsetlinewidth{0.602250pt}%
\definecolor{currentstroke}{rgb}{0.000000,0.000000,0.000000}%
\pgfsetstrokecolor{currentstroke}%
\pgfsetdash{}{0pt}%
\pgfsys@defobject{currentmarker}{\pgfqpoint{-0.027778in}{0.000000in}}{\pgfqpoint{-0.000000in}{0.000000in}}{%
\pgfpathmoveto{\pgfqpoint{-0.000000in}{0.000000in}}%
\pgfpathlineto{\pgfqpoint{-0.027778in}{0.000000in}}%
\pgfusepath{stroke,fill}%
}%
\begin{pgfscope}%
\pgfsys@transformshift{0.398419in}{1.763733in}%
\pgfsys@useobject{currentmarker}{}%
\end{pgfscope}%
\end{pgfscope}%
\begin{pgfscope}%
\pgfsetbuttcap%
\pgfsetroundjoin%
\definecolor{currentfill}{rgb}{0.000000,0.000000,0.000000}%
\pgfsetfillcolor{currentfill}%
\pgfsetlinewidth{0.602250pt}%
\definecolor{currentstroke}{rgb}{0.000000,0.000000,0.000000}%
\pgfsetstrokecolor{currentstroke}%
\pgfsetdash{}{0pt}%
\pgfsys@defobject{currentmarker}{\pgfqpoint{-0.027778in}{0.000000in}}{\pgfqpoint{-0.000000in}{0.000000in}}{%
\pgfpathmoveto{\pgfqpoint{-0.000000in}{0.000000in}}%
\pgfpathlineto{\pgfqpoint{-0.027778in}{0.000000in}}%
\pgfusepath{stroke,fill}%
}%
\begin{pgfscope}%
\pgfsys@transformshift{0.398419in}{1.784365in}%
\pgfsys@useobject{currentmarker}{}%
\end{pgfscope}%
\end{pgfscope}%
\begin{pgfscope}%
\pgfsetbuttcap%
\pgfsetroundjoin%
\definecolor{currentfill}{rgb}{0.000000,0.000000,0.000000}%
\pgfsetfillcolor{currentfill}%
\pgfsetlinewidth{0.602250pt}%
\definecolor{currentstroke}{rgb}{0.000000,0.000000,0.000000}%
\pgfsetstrokecolor{currentstroke}%
\pgfsetdash{}{0pt}%
\pgfsys@defobject{currentmarker}{\pgfqpoint{-0.027778in}{0.000000in}}{\pgfqpoint{-0.000000in}{0.000000in}}{%
\pgfpathmoveto{\pgfqpoint{-0.000000in}{0.000000in}}%
\pgfpathlineto{\pgfqpoint{-0.027778in}{0.000000in}}%
\pgfusepath{stroke,fill}%
}%
\begin{pgfscope}%
\pgfsys@transformshift{0.398419in}{1.924238in}%
\pgfsys@useobject{currentmarker}{}%
\end{pgfscope}%
\end{pgfscope}%
\begin{pgfscope}%
\pgfsetbuttcap%
\pgfsetroundjoin%
\definecolor{currentfill}{rgb}{0.000000,0.000000,0.000000}%
\pgfsetfillcolor{currentfill}%
\pgfsetlinewidth{0.602250pt}%
\definecolor{currentstroke}{rgb}{0.000000,0.000000,0.000000}%
\pgfsetstrokecolor{currentstroke}%
\pgfsetdash{}{0pt}%
\pgfsys@defobject{currentmarker}{\pgfqpoint{-0.027778in}{0.000000in}}{\pgfqpoint{-0.000000in}{0.000000in}}{%
\pgfpathmoveto{\pgfqpoint{-0.000000in}{0.000000in}}%
\pgfpathlineto{\pgfqpoint{-0.027778in}{0.000000in}}%
\pgfusepath{stroke,fill}%
}%
\begin{pgfscope}%
\pgfsys@transformshift{0.398419in}{1.995263in}%
\pgfsys@useobject{currentmarker}{}%
\end{pgfscope}%
\end{pgfscope}%
\begin{pgfscope}%
\pgfsetbuttcap%
\pgfsetroundjoin%
\definecolor{currentfill}{rgb}{0.000000,0.000000,0.000000}%
\pgfsetfillcolor{currentfill}%
\pgfsetlinewidth{0.602250pt}%
\definecolor{currentstroke}{rgb}{0.000000,0.000000,0.000000}%
\pgfsetstrokecolor{currentstroke}%
\pgfsetdash{}{0pt}%
\pgfsys@defobject{currentmarker}{\pgfqpoint{-0.027778in}{0.000000in}}{\pgfqpoint{-0.000000in}{0.000000in}}{%
\pgfpathmoveto{\pgfqpoint{-0.000000in}{0.000000in}}%
\pgfpathlineto{\pgfqpoint{-0.027778in}{0.000000in}}%
\pgfusepath{stroke,fill}%
}%
\begin{pgfscope}%
\pgfsys@transformshift{0.398419in}{2.045655in}%
\pgfsys@useobject{currentmarker}{}%
\end{pgfscope}%
\end{pgfscope}%
\begin{pgfscope}%
\pgfsetbuttcap%
\pgfsetroundjoin%
\definecolor{currentfill}{rgb}{0.000000,0.000000,0.000000}%
\pgfsetfillcolor{currentfill}%
\pgfsetlinewidth{0.602250pt}%
\definecolor{currentstroke}{rgb}{0.000000,0.000000,0.000000}%
\pgfsetstrokecolor{currentstroke}%
\pgfsetdash{}{0pt}%
\pgfsys@defobject{currentmarker}{\pgfqpoint{-0.027778in}{0.000000in}}{\pgfqpoint{-0.000000in}{0.000000in}}{%
\pgfpathmoveto{\pgfqpoint{-0.000000in}{0.000000in}}%
\pgfpathlineto{\pgfqpoint{-0.027778in}{0.000000in}}%
\pgfusepath{stroke,fill}%
}%
\begin{pgfscope}%
\pgfsys@transformshift{0.398419in}{2.084743in}%
\pgfsys@useobject{currentmarker}{}%
\end{pgfscope}%
\end{pgfscope}%
\begin{pgfscope}%
\pgfsetbuttcap%
\pgfsetroundjoin%
\definecolor{currentfill}{rgb}{0.000000,0.000000,0.000000}%
\pgfsetfillcolor{currentfill}%
\pgfsetlinewidth{0.602250pt}%
\definecolor{currentstroke}{rgb}{0.000000,0.000000,0.000000}%
\pgfsetstrokecolor{currentstroke}%
\pgfsetdash{}{0pt}%
\pgfsys@defobject{currentmarker}{\pgfqpoint{-0.027778in}{0.000000in}}{\pgfqpoint{-0.000000in}{0.000000in}}{%
\pgfpathmoveto{\pgfqpoint{-0.000000in}{0.000000in}}%
\pgfpathlineto{\pgfqpoint{-0.027778in}{0.000000in}}%
\pgfusepath{stroke,fill}%
}%
\begin{pgfscope}%
\pgfsys@transformshift{0.398419in}{2.116680in}%
\pgfsys@useobject{currentmarker}{}%
\end{pgfscope}%
\end{pgfscope}%
\begin{pgfscope}%
\pgfsetbuttcap%
\pgfsetroundjoin%
\definecolor{currentfill}{rgb}{0.000000,0.000000,0.000000}%
\pgfsetfillcolor{currentfill}%
\pgfsetlinewidth{0.602250pt}%
\definecolor{currentstroke}{rgb}{0.000000,0.000000,0.000000}%
\pgfsetstrokecolor{currentstroke}%
\pgfsetdash{}{0pt}%
\pgfsys@defobject{currentmarker}{\pgfqpoint{-0.027778in}{0.000000in}}{\pgfqpoint{-0.000000in}{0.000000in}}{%
\pgfpathmoveto{\pgfqpoint{-0.000000in}{0.000000in}}%
\pgfpathlineto{\pgfqpoint{-0.027778in}{0.000000in}}%
\pgfusepath{stroke,fill}%
}%
\begin{pgfscope}%
\pgfsys@transformshift{0.398419in}{2.143682in}%
\pgfsys@useobject{currentmarker}{}%
\end{pgfscope}%
\end{pgfscope}%
\begin{pgfscope}%
\pgfsetbuttcap%
\pgfsetroundjoin%
\definecolor{currentfill}{rgb}{0.000000,0.000000,0.000000}%
\pgfsetfillcolor{currentfill}%
\pgfsetlinewidth{0.602250pt}%
\definecolor{currentstroke}{rgb}{0.000000,0.000000,0.000000}%
\pgfsetstrokecolor{currentstroke}%
\pgfsetdash{}{0pt}%
\pgfsys@defobject{currentmarker}{\pgfqpoint{-0.027778in}{0.000000in}}{\pgfqpoint{-0.000000in}{0.000000in}}{%
\pgfpathmoveto{\pgfqpoint{-0.000000in}{0.000000in}}%
\pgfpathlineto{\pgfqpoint{-0.027778in}{0.000000in}}%
\pgfusepath{stroke,fill}%
}%
\begin{pgfscope}%
\pgfsys@transformshift{0.398419in}{2.167072in}%
\pgfsys@useobject{currentmarker}{}%
\end{pgfscope}%
\end{pgfscope}%
\begin{pgfscope}%
\pgfsetbuttcap%
\pgfsetroundjoin%
\definecolor{currentfill}{rgb}{0.000000,0.000000,0.000000}%
\pgfsetfillcolor{currentfill}%
\pgfsetlinewidth{0.602250pt}%
\definecolor{currentstroke}{rgb}{0.000000,0.000000,0.000000}%
\pgfsetstrokecolor{currentstroke}%
\pgfsetdash{}{0pt}%
\pgfsys@defobject{currentmarker}{\pgfqpoint{-0.027778in}{0.000000in}}{\pgfqpoint{-0.000000in}{0.000000in}}{%
\pgfpathmoveto{\pgfqpoint{-0.000000in}{0.000000in}}%
\pgfpathlineto{\pgfqpoint{-0.027778in}{0.000000in}}%
\pgfusepath{stroke,fill}%
}%
\begin{pgfscope}%
\pgfsys@transformshift{0.398419in}{2.187704in}%
\pgfsys@useobject{currentmarker}{}%
\end{pgfscope}%
\end{pgfscope}%
\begin{pgfscope}%
\pgfpathrectangle{\pgfqpoint{0.398419in}{0.521603in}}{\pgfqpoint{5.089095in}{1.685002in}}%
\pgfusepath{clip}%
\pgfsetrectcap%
\pgfsetroundjoin%
\pgfsetlinewidth{1.505625pt}%
\pgfsetstrokecolor{currentstroke1}%
\pgfsetdash{}{0pt}%
\pgfpathmoveto{\pgfqpoint{0.629741in}{0.611260in}}%
\pgfpathlineto{\pgfqpoint{0.801091in}{0.629176in}}%
\pgfpathlineto{\pgfqpoint{0.972441in}{0.688028in}}%
\pgfpathlineto{\pgfqpoint{1.143791in}{0.660728in}}%
\pgfpathlineto{\pgfqpoint{1.315141in}{0.641436in}}%
\pgfpathlineto{\pgfqpoint{1.486491in}{0.631892in}}%
\pgfpathlineto{\pgfqpoint{1.657841in}{0.598194in}}%
\pgfpathlineto{\pgfqpoint{1.829191in}{0.659755in}}%
\pgfpathlineto{\pgfqpoint{2.000541in}{0.689672in}}%
\pgfpathlineto{\pgfqpoint{2.171891in}{0.748062in}}%
\pgfpathlineto{\pgfqpoint{2.343241in}{0.829868in}}%
\pgfpathlineto{\pgfqpoint{2.514591in}{0.894330in}}%
\pgfpathlineto{\pgfqpoint{2.685941in}{1.011154in}}%
\pgfpathlineto{\pgfqpoint{2.857291in}{1.052380in}}%
\pgfpathlineto{\pgfqpoint{3.028641in}{1.186788in}}%
\pgfpathlineto{\pgfqpoint{3.199991in}{1.243241in}}%
\pgfpathlineto{\pgfqpoint{3.371341in}{1.358943in}}%
\pgfpathlineto{\pgfqpoint{3.542691in}{1.451268in}}%
\pgfpathlineto{\pgfqpoint{3.714041in}{1.578452in}}%
\pgfpathlineto{\pgfqpoint{3.885391in}{1.672298in}}%
\pgfpathlineto{\pgfqpoint{4.056741in}{1.799550in}}%
\pgfpathlineto{\pgfqpoint{4.228091in}{1.893170in}}%
\pgfpathlineto{\pgfqpoint{4.399441in}{2.027673in}}%
\pgfpathlineto{\pgfqpoint{4.570791in}{2.130014in}}%
\pgfusepath{stroke}%
\end{pgfscope}%
\begin{pgfscope}%
\pgfpathrectangle{\pgfqpoint{0.398419in}{0.521603in}}{\pgfqpoint{5.089095in}{1.685002in}}%
\pgfusepath{clip}%
\pgfsetrectcap%
\pgfsetroundjoin%
\pgfsetlinewidth{1.505625pt}%
\pgfsetstrokecolor{currentstroke2}%
\pgfsetdash{}{0pt}%
\pgfpathmoveto{\pgfqpoint{0.629741in}{0.611260in}}%
\pgfpathlineto{\pgfqpoint{0.801091in}{0.655282in}}%
\pgfpathlineto{\pgfqpoint{0.972441in}{0.701240in}}%
\pgfpathlineto{\pgfqpoint{1.143791in}{0.650121in}}%
\pgfpathlineto{\pgfqpoint{1.315141in}{0.636826in}}%
\pgfpathlineto{\pgfqpoint{1.486491in}{0.635048in}}%
\pgfpathlineto{\pgfqpoint{1.657841in}{0.610720in}}%
\pgfpathlineto{\pgfqpoint{1.829191in}{0.679389in}}%
\pgfpathlineto{\pgfqpoint{2.000541in}{0.728242in}}%
\pgfpathlineto{\pgfqpoint{2.171891in}{0.784345in}}%
\pgfpathlineto{\pgfqpoint{2.343241in}{0.899751in}}%
\pgfpathlineto{\pgfqpoint{2.514591in}{0.934944in}}%
\pgfpathlineto{\pgfqpoint{2.685941in}{1.027593in}}%
\pgfpathlineto{\pgfqpoint{2.857291in}{1.032620in}}%
\pgfpathlineto{\pgfqpoint{3.028641in}{1.159792in}}%
\pgfpathlineto{\pgfqpoint{3.199991in}{1.158836in}}%
\pgfpathlineto{\pgfqpoint{3.371341in}{1.256529in}}%
\pgfpathlineto{\pgfqpoint{3.542691in}{1.274271in}}%
\pgfpathlineto{\pgfqpoint{3.714041in}{1.401645in}}%
\pgfpathlineto{\pgfqpoint{3.885391in}{1.408607in}}%
\pgfpathlineto{\pgfqpoint{4.056741in}{1.542850in}}%
\pgfpathlineto{\pgfqpoint{4.228091in}{1.480854in}}%
\pgfpathlineto{\pgfqpoint{4.399441in}{1.676969in}}%
\pgfpathlineto{\pgfqpoint{4.570791in}{1.634511in}}%
\pgfpathlineto{\pgfqpoint{4.742141in}{1.731797in}}%
\pgfpathlineto{\pgfqpoint{4.913491in}{1.699052in}}%
\pgfpathlineto{\pgfqpoint{5.084841in}{1.983427in}}%
\pgfpathlineto{\pgfqpoint{5.256191in}{1.896326in}}%
\pgfusepath{stroke}%
\end{pgfscope}%
\begin{pgfscope}%
\pgfpathrectangle{\pgfqpoint{0.398419in}{0.521603in}}{\pgfqpoint{5.089095in}{1.685002in}}%
\pgfusepath{clip}%
\pgfsetrectcap%
\pgfsetroundjoin%
\pgfsetlinewidth{1.505625pt}%
\pgfsetstrokecolor{currentstroke3}%
\pgfsetdash{}{0pt}%
\pgfpathmoveto{\pgfqpoint{0.629741in}{0.611260in}}%
\pgfpathlineto{\pgfqpoint{0.801091in}{0.671977in}}%
\pgfpathlineto{\pgfqpoint{0.972441in}{0.714221in}}%
\pgfpathlineto{\pgfqpoint{1.143791in}{0.648587in}}%
\pgfpathlineto{\pgfqpoint{1.315141in}{0.636046in}}%
\pgfpathlineto{\pgfqpoint{1.486491in}{0.638148in}}%
\pgfpathlineto{\pgfqpoint{1.657841in}{0.610720in}}%
\pgfpathlineto{\pgfqpoint{1.829191in}{0.681132in}}%
\pgfpathlineto{\pgfqpoint{2.000541in}{0.727013in}}%
\pgfpathlineto{\pgfqpoint{2.171891in}{0.784345in}}%
\pgfpathlineto{\pgfqpoint{2.343241in}{0.899208in}}%
\pgfpathlineto{\pgfqpoint{2.514591in}{0.934625in}}%
\pgfpathlineto{\pgfqpoint{2.685941in}{1.028566in}}%
\pgfpathlineto{\pgfqpoint{2.857291in}{1.033203in}}%
\pgfpathlineto{\pgfqpoint{3.028641in}{1.160052in}}%
\pgfpathlineto{\pgfqpoint{3.199991in}{1.159625in}}%
\pgfpathlineto{\pgfqpoint{3.371341in}{1.248665in}}%
\pgfpathlineto{\pgfqpoint{3.542691in}{1.265775in}}%
\pgfpathlineto{\pgfqpoint{3.714041in}{1.406352in}}%
\pgfpathlineto{\pgfqpoint{3.885391in}{1.407286in}}%
\pgfpathlineto{\pgfqpoint{4.056741in}{1.533160in}}%
\pgfpathlineto{\pgfqpoint{4.228091in}{1.482091in}}%
\pgfpathlineto{\pgfqpoint{4.399441in}{1.646321in}}%
\pgfpathlineto{\pgfqpoint{4.570791in}{1.613212in}}%
\pgfpathlineto{\pgfqpoint{4.742141in}{1.707704in}}%
\pgfpathlineto{\pgfqpoint{4.913491in}{1.704100in}}%
\pgfpathlineto{\pgfqpoint{5.084841in}{1.951140in}}%
\pgfpathlineto{\pgfqpoint{5.256191in}{1.840586in}}%
\pgfusepath{stroke}%
\end{pgfscope}%
\begin{pgfscope}%
\pgfpathrectangle{\pgfqpoint{0.398419in}{0.521603in}}{\pgfqpoint{5.089095in}{1.685002in}}%
\pgfusepath{clip}%
\pgfsetrectcap%
\pgfsetroundjoin%
\pgfsetlinewidth{1.505625pt}%
\pgfsetstrokecolor{currentstroke4}%
\pgfsetdash{}{0pt}%
\pgfpathmoveto{\pgfqpoint{0.629741in}{0.611260in}}%
\pgfpathlineto{\pgfqpoint{0.801091in}{0.663829in}}%
\pgfpathlineto{\pgfqpoint{0.972441in}{0.701240in}}%
\pgfpathlineto{\pgfqpoint{1.143791in}{0.654148in}}%
\pgfpathlineto{\pgfqpoint{1.315141in}{0.636046in}}%
\pgfpathlineto{\pgfqpoint{1.486491in}{0.635048in}}%
\pgfpathlineto{\pgfqpoint{1.657841in}{0.610720in}}%
\pgfpathlineto{\pgfqpoint{1.829191in}{0.681132in}}%
\pgfpathlineto{\pgfqpoint{2.000541in}{0.728242in}}%
\pgfpathlineto{\pgfqpoint{2.171891in}{0.777908in}}%
\pgfpathlineto{\pgfqpoint{2.343241in}{0.900832in}}%
\pgfpathlineto{\pgfqpoint{2.514591in}{0.935897in}}%
\pgfpathlineto{\pgfqpoint{2.685941in}{1.026614in}}%
\pgfpathlineto{\pgfqpoint{2.857291in}{1.037614in}}%
\pgfpathlineto{\pgfqpoint{3.028641in}{1.159662in}}%
\pgfpathlineto{\pgfqpoint{3.199991in}{1.164382in}}%
\pgfpathlineto{\pgfqpoint{3.371341in}{1.278202in}}%
\pgfpathlineto{\pgfqpoint{3.542691in}{1.291347in}}%
\pgfpathlineto{\pgfqpoint{3.714041in}{1.437034in}}%
\pgfpathlineto{\pgfqpoint{3.885391in}{1.442363in}}%
\pgfpathlineto{\pgfqpoint{4.056741in}{1.565199in}}%
\pgfpathlineto{\pgfqpoint{4.228091in}{1.512228in}}%
\pgfpathlineto{\pgfqpoint{4.399441in}{1.702580in}}%
\pgfpathlineto{\pgfqpoint{4.570791in}{1.676753in}}%
\pgfpathlineto{\pgfqpoint{4.742141in}{1.815598in}}%
\pgfpathlineto{\pgfqpoint{4.913491in}{1.816356in}}%
\pgfpathlineto{\pgfqpoint{5.084841in}{1.976872in}}%
\pgfpathlineto{\pgfqpoint{5.256191in}{1.962074in}}%
\pgfusepath{stroke}%
\end{pgfscope}%
\begin{pgfscope}%
\pgfpathrectangle{\pgfqpoint{0.398419in}{0.521603in}}{\pgfqpoint{5.089095in}{1.685002in}}%
\pgfusepath{clip}%
\pgfsetrectcap%
\pgfsetroundjoin%
\pgfsetlinewidth{1.505625pt}%
\pgfsetstrokecolor{currentstroke5}%
\pgfsetdash{}{0pt}%
\pgfpathmoveto{\pgfqpoint{0.629741in}{0.611260in}}%
\pgfpathlineto{\pgfqpoint{0.801091in}{0.663829in}}%
\pgfpathlineto{\pgfqpoint{0.972441in}{0.709786in}}%
\pgfpathlineto{\pgfqpoint{1.143791in}{0.654148in}}%
\pgfpathlineto{\pgfqpoint{1.315141in}{0.633287in}}%
\pgfpathlineto{\pgfqpoint{1.486491in}{0.635048in}}%
\pgfpathlineto{\pgfqpoint{1.657841in}{0.610720in}}%
\pgfpathlineto{\pgfqpoint{1.829191in}{0.681132in}}%
\pgfpathlineto{\pgfqpoint{2.000541in}{0.728242in}}%
\pgfpathlineto{\pgfqpoint{2.171891in}{0.777908in}}%
\pgfpathlineto{\pgfqpoint{2.343241in}{0.900832in}}%
\pgfpathlineto{\pgfqpoint{2.514591in}{0.934625in}}%
\pgfpathlineto{\pgfqpoint{2.685941in}{1.025630in}}%
\pgfpathlineto{\pgfqpoint{2.857291in}{1.037614in}}%
\pgfpathlineto{\pgfqpoint{3.028641in}{1.160699in}}%
\pgfpathlineto{\pgfqpoint{3.199991in}{1.163903in}}%
\pgfpathlineto{\pgfqpoint{3.371341in}{1.258153in}}%
\pgfpathlineto{\pgfqpoint{3.542691in}{1.266769in}}%
\pgfpathlineto{\pgfqpoint{3.714041in}{1.402534in}}%
\pgfpathlineto{\pgfqpoint{3.885391in}{1.405391in}}%
\pgfpathlineto{\pgfqpoint{4.056741in}{1.532136in}}%
\pgfpathlineto{\pgfqpoint{4.228091in}{1.483088in}}%
\pgfpathlineto{\pgfqpoint{4.399441in}{1.637094in}}%
\pgfpathlineto{\pgfqpoint{4.570791in}{1.618893in}}%
\pgfpathlineto{\pgfqpoint{4.742141in}{1.710595in}}%
\pgfpathlineto{\pgfqpoint{4.913491in}{1.730257in}}%
\pgfpathlineto{\pgfqpoint{5.084841in}{1.993797in}}%
\pgfpathlineto{\pgfqpoint{5.256191in}{1.911545in}}%
\pgfusepath{stroke}%
\end{pgfscope}%
\begin{pgfscope}%
\pgfpathrectangle{\pgfqpoint{0.398419in}{0.521603in}}{\pgfqpoint{5.089095in}{1.685002in}}%
\pgfusepath{clip}%
\pgfsetrectcap%
\pgfsetroundjoin%
\pgfsetlinewidth{1.505625pt}%
\pgfsetstrokecolor{currentstroke6}%
\pgfsetdash{}{0pt}%
\pgfpathmoveto{\pgfqpoint{0.629741in}{0.611260in}}%
\pgfpathlineto{\pgfqpoint{0.801091in}{0.663829in}}%
\pgfpathlineto{\pgfqpoint{0.972441in}{0.695765in}}%
\pgfpathlineto{\pgfqpoint{1.143791in}{0.650121in}}%
\pgfpathlineto{\pgfqpoint{1.315141in}{0.630362in}}%
\pgfpathlineto{\pgfqpoint{1.486491in}{0.629430in}}%
\pgfpathlineto{\pgfqpoint{1.657841in}{0.605785in}}%
\pgfpathlineto{\pgfqpoint{1.829191in}{0.668550in}}%
\pgfpathlineto{\pgfqpoint{2.000541in}{0.712889in}}%
\pgfpathlineto{\pgfqpoint{2.171891in}{0.770249in}}%
\pgfpathlineto{\pgfqpoint{2.343241in}{0.913312in}}%
\pgfpathlineto{\pgfqpoint{2.514591in}{0.957616in}}%
\pgfpathlineto{\pgfqpoint{2.685941in}{1.055082in}}%
\pgfpathlineto{\pgfqpoint{2.857291in}{1.082035in}}%
\pgfpathlineto{\pgfqpoint{3.028641in}{1.200199in}}%
\pgfpathlineto{\pgfqpoint{3.199991in}{1.234927in}}%
\pgfpathlineto{\pgfqpoint{3.371341in}{1.332682in}}%
\pgfpathlineto{\pgfqpoint{3.542691in}{1.371686in}}%
\pgfpathlineto{\pgfqpoint{3.714041in}{1.506553in}}%
\pgfpathlineto{\pgfqpoint{3.885391in}{1.518659in}}%
\pgfpathlineto{\pgfqpoint{4.056741in}{1.672474in}}%
\pgfpathlineto{\pgfqpoint{4.228091in}{1.628972in}}%
\pgfpathlineto{\pgfqpoint{4.399441in}{1.751021in}}%
\pgfpathlineto{\pgfqpoint{4.570791in}{1.770119in}}%
\pgfpathlineto{\pgfqpoint{4.742141in}{1.840125in}}%
\pgfpathlineto{\pgfqpoint{4.913491in}{1.789291in}}%
\pgfpathlineto{\pgfqpoint{5.084841in}{1.966106in}}%
\pgfpathlineto{\pgfqpoint{5.256191in}{1.925981in}}%
\pgfusepath{stroke}%
\end{pgfscope}%
\begin{pgfscope}%
\pgfpathrectangle{\pgfqpoint{0.398419in}{0.521603in}}{\pgfqpoint{5.089095in}{1.685002in}}%
\pgfusepath{clip}%
\pgfsetrectcap%
\pgfsetroundjoin%
\pgfsetlinewidth{1.505625pt}%
\pgfsetstrokecolor{currentstroke7}%
\pgfsetdash{}{0pt}%
\pgfpathmoveto{\pgfqpoint{0.629741in}{0.611260in}}%
\pgfpathlineto{\pgfqpoint{0.801091in}{0.663829in}}%
\pgfpathlineto{\pgfqpoint{0.972441in}{0.695765in}}%
\pgfpathlineto{\pgfqpoint{1.143791in}{0.641778in}}%
\pgfpathlineto{\pgfqpoint{1.315141in}{0.639075in}}%
\pgfpathlineto{\pgfqpoint{1.486491in}{0.631035in}}%
\pgfpathlineto{\pgfqpoint{1.657841in}{0.605785in}}%
\pgfpathlineto{\pgfqpoint{1.829191in}{0.666677in}}%
\pgfpathlineto{\pgfqpoint{2.000541in}{0.715543in}}%
\pgfpathlineto{\pgfqpoint{2.171891in}{0.768281in}}%
\pgfpathlineto{\pgfqpoint{2.343241in}{0.914812in}}%
\pgfpathlineto{\pgfqpoint{2.514591in}{0.956493in}}%
\pgfpathlineto{\pgfqpoint{2.685941in}{1.055291in}}%
\pgfpathlineto{\pgfqpoint{2.857291in}{1.082182in}}%
\pgfpathlineto{\pgfqpoint{3.028641in}{1.200199in}}%
\pgfpathlineto{\pgfqpoint{3.199991in}{1.234157in}}%
\pgfpathlineto{\pgfqpoint{3.371341in}{1.301017in}}%
\pgfpathlineto{\pgfqpoint{3.542691in}{1.339891in}}%
\pgfpathlineto{\pgfqpoint{3.714041in}{1.481878in}}%
\pgfpathlineto{\pgfqpoint{3.885391in}{1.466752in}}%
\pgfpathlineto{\pgfqpoint{4.056741in}{1.584682in}}%
\pgfpathlineto{\pgfqpoint{4.228091in}{1.600032in}}%
\pgfpathlineto{\pgfqpoint{4.399441in}{1.681579in}}%
\pgfpathlineto{\pgfqpoint{4.570791in}{1.718235in}}%
\pgfpathlineto{\pgfqpoint{4.742141in}{1.868550in}}%
\pgfpathlineto{\pgfqpoint{4.913491in}{1.728526in}}%
\pgfpathlineto{\pgfqpoint{5.084841in}{1.860252in}}%
\pgfpathlineto{\pgfqpoint{5.256191in}{1.935450in}}%
\pgfusepath{stroke}%
\end{pgfscope}%
\begin{pgfscope}%
\pgfsetrectcap%
\pgfsetmiterjoin%
\pgfsetlinewidth{0.803000pt}%
\definecolor{currentstroke}{rgb}{0.000000,0.000000,0.000000}%
\pgfsetstrokecolor{currentstroke}%
\pgfsetdash{}{0pt}%
\pgfpathmoveto{\pgfqpoint{0.398419in}{0.521603in}}%
\pgfpathlineto{\pgfqpoint{0.398419in}{2.206605in}}%
\pgfusepath{stroke}%
\end{pgfscope}%
\begin{pgfscope}%
\pgfsetrectcap%
\pgfsetmiterjoin%
\pgfsetlinewidth{0.803000pt}%
\definecolor{currentstroke}{rgb}{0.000000,0.000000,0.000000}%
\pgfsetstrokecolor{currentstroke}%
\pgfsetdash{}{0pt}%
\pgfpathmoveto{\pgfqpoint{5.487514in}{0.521603in}}%
\pgfpathlineto{\pgfqpoint{5.487514in}{2.206605in}}%
\pgfusepath{stroke}%
\end{pgfscope}%
\begin{pgfscope}%
\pgfsetrectcap%
\pgfsetmiterjoin%
\pgfsetlinewidth{0.803000pt}%
\definecolor{currentstroke}{rgb}{0.000000,0.000000,0.000000}%
\pgfsetstrokecolor{currentstroke}%
\pgfsetdash{}{0pt}%
\pgfpathmoveto{\pgfqpoint{0.398419in}{0.521603in}}%
\pgfpathlineto{\pgfqpoint{5.487514in}{0.521603in}}%
\pgfusepath{stroke}%
\end{pgfscope}%
\begin{pgfscope}%
\pgfsetrectcap%
\pgfsetmiterjoin%
\pgfsetlinewidth{0.803000pt}%
\definecolor{currentstroke}{rgb}{0.000000,0.000000,0.000000}%
\pgfsetstrokecolor{currentstroke}%
\pgfsetdash{}{0pt}%
\pgfpathmoveto{\pgfqpoint{0.398419in}{2.206605in}}%
\pgfpathlineto{\pgfqpoint{5.487514in}{2.206605in}}%
\pgfusepath{stroke}%
\end{pgfscope}%
\begin{pgfscope}%
\pgfsetbuttcap%
\pgfsetmiterjoin%
\definecolor{currentfill}{rgb}{1.000000,1.000000,1.000000}%
\pgfsetfillcolor{currentfill}%
\pgfsetfillopacity{0.800000}%
\pgfsetlinewidth{1.003750pt}%
\definecolor{currentstroke}{rgb}{0.800000,0.800000,0.800000}%
\pgfsetstrokecolor{currentstroke}%
\pgfsetstrokeopacity{0.800000}%
\pgfsetdash{}{0pt}%
\pgfpathmoveto{\pgfqpoint{5.575014in}{0.808389in}}%
\pgfpathlineto{\pgfqpoint{8.259376in}{0.808389in}}%
\pgfpathquadraticcurveto{\pgfqpoint{8.284376in}{0.808389in}}{\pgfqpoint{8.284376in}{0.833389in}}%
\pgfpathlineto{\pgfqpoint{8.284376in}{2.119105in}}%
\pgfpathquadraticcurveto{\pgfqpoint{8.284376in}{2.144105in}}{\pgfqpoint{8.259376in}{2.144105in}}%
\pgfpathlineto{\pgfqpoint{5.575014in}{2.144105in}}%
\pgfpathquadraticcurveto{\pgfqpoint{5.550014in}{2.144105in}}{\pgfqpoint{5.550014in}{2.119105in}}%
\pgfpathlineto{\pgfqpoint{5.550014in}{0.833389in}}%
\pgfpathquadraticcurveto{\pgfqpoint{5.550014in}{0.808389in}}{\pgfqpoint{5.575014in}{0.808389in}}%
\pgfpathlineto{\pgfqpoint{5.575014in}{0.808389in}}%
\pgfpathclose%
\pgfusepath{stroke,fill}%
\end{pgfscope}%
\begin{pgfscope}%
\pgfsetrectcap%
\pgfsetroundjoin%
\pgfsetlinewidth{1.505625pt}%
\pgfsetstrokecolor{currentstroke1}%
\pgfsetdash{}{0pt}%
\pgfpathmoveto{\pgfqpoint{5.600014in}{2.042884in}}%
\pgfpathlineto{\pgfqpoint{5.725014in}{2.042884in}}%
\pgfpathlineto{\pgfqpoint{5.850014in}{2.042884in}}%
\pgfusepath{stroke}%
\end{pgfscope}%
\begin{pgfscope}%
\definecolor{textcolor}{rgb}{0.000000,0.000000,0.000000}%
\pgfsetstrokecolor{textcolor}%
\pgfsetfillcolor{textcolor}%
\pgftext[x=5.950014in,y=1.999134in,left,base]{\color{textcolor}{\rmfamily\fontsize{9.000000}{10.800000}\selectfont\catcode`\^=\active\def^{\ifmmode\sp\else\^{}\fi}\catcode`\%=\active\def%{\%}\NaiveCycles{}}}%
\end{pgfscope}%
\begin{pgfscope}%
\pgfsetrectcap%
\pgfsetroundjoin%
\pgfsetlinewidth{1.505625pt}%
\pgfsetstrokecolor{currentstroke2}%
\pgfsetdash{}{0pt}%
\pgfpathmoveto{\pgfqpoint{5.600014in}{1.859413in}}%
\pgfpathlineto{\pgfqpoint{5.725014in}{1.859413in}}%
\pgfpathlineto{\pgfqpoint{5.850014in}{1.859413in}}%
\pgfusepath{stroke}%
\end{pgfscope}%
\begin{pgfscope}%
\definecolor{textcolor}{rgb}{0.000000,0.000000,0.000000}%
\pgfsetstrokecolor{textcolor}%
\pgfsetfillcolor{textcolor}%
\pgftext[x=5.950014in,y=1.815663in,left,base]{\color{textcolor}{\rmfamily\fontsize{9.000000}{10.800000}\selectfont\catcode`\^=\active\def^{\ifmmode\sp\else\^{}\fi}\catcode`\%=\active\def%{\%}\Neighbors{} \& \MergeLinear{}}}%
\end{pgfscope}%
\begin{pgfscope}%
\pgfsetrectcap%
\pgfsetroundjoin%
\pgfsetlinewidth{1.505625pt}%
\pgfsetstrokecolor{currentstroke3}%
\pgfsetdash{}{0pt}%
\pgfpathmoveto{\pgfqpoint{5.600014in}{1.675941in}}%
\pgfpathlineto{\pgfqpoint{5.725014in}{1.675941in}}%
\pgfpathlineto{\pgfqpoint{5.850014in}{1.675941in}}%
\pgfusepath{stroke}%
\end{pgfscope}%
\begin{pgfscope}%
\definecolor{textcolor}{rgb}{0.000000,0.000000,0.000000}%
\pgfsetstrokecolor{textcolor}%
\pgfsetfillcolor{textcolor}%
\pgftext[x=5.950014in,y=1.632191in,left,base]{\color{textcolor}{\rmfamily\fontsize{9.000000}{10.800000}\selectfont\catcode`\^=\active\def^{\ifmmode\sp\else\^{}\fi}\catcode`\%=\active\def%{\%}\Neighbors{} \& \SharedVertices{}}}%
\end{pgfscope}%
\begin{pgfscope}%
\pgfsetrectcap%
\pgfsetroundjoin%
\pgfsetlinewidth{1.505625pt}%
\pgfsetstrokecolor{currentstroke4}%
\pgfsetdash{}{0pt}%
\pgfpathmoveto{\pgfqpoint{5.600014in}{1.488991in}}%
\pgfpathlineto{\pgfqpoint{5.725014in}{1.488991in}}%
\pgfpathlineto{\pgfqpoint{5.850014in}{1.488991in}}%
\pgfusepath{stroke}%
\end{pgfscope}%
\begin{pgfscope}%
\definecolor{textcolor}{rgb}{0.000000,0.000000,0.000000}%
\pgfsetstrokecolor{textcolor}%
\pgfsetfillcolor{textcolor}%
\pgftext[x=5.950014in,y=1.445241in,left,base]{\color{textcolor}{\rmfamily\fontsize{9.000000}{10.800000}\selectfont\catcode`\^=\active\def^{\ifmmode\sp\else\^{}\fi}\catcode`\%=\active\def%{\%}\NeighborsDegree{} \& \MergeLinear{}}}%
\end{pgfscope}%
\begin{pgfscope}%
\pgfsetrectcap%
\pgfsetroundjoin%
\pgfsetlinewidth{1.505625pt}%
\pgfsetstrokecolor{currentstroke5}%
\pgfsetdash{}{0pt}%
\pgfpathmoveto{\pgfqpoint{5.600014in}{1.302040in}}%
\pgfpathlineto{\pgfqpoint{5.725014in}{1.302040in}}%
\pgfpathlineto{\pgfqpoint{5.850014in}{1.302040in}}%
\pgfusepath{stroke}%
\end{pgfscope}%
\begin{pgfscope}%
\definecolor{textcolor}{rgb}{0.000000,0.000000,0.000000}%
\pgfsetstrokecolor{textcolor}%
\pgfsetfillcolor{textcolor}%
\pgftext[x=5.950014in,y=1.258290in,left,base]{\color{textcolor}{\rmfamily\fontsize{9.000000}{10.800000}\selectfont\catcode`\^=\active\def^{\ifmmode\sp\else\^{}\fi}\catcode`\%=\active\def%{\%}\NeighborsDegree{} \& \SharedVertices{}}}%
\end{pgfscope}%
\begin{pgfscope}%
\pgfsetrectcap%
\pgfsetroundjoin%
\pgfsetlinewidth{1.505625pt}%
\pgfsetstrokecolor{currentstroke6}%
\pgfsetdash{}{0pt}%
\pgfpathmoveto{\pgfqpoint{5.600014in}{1.115090in}}%
\pgfpathlineto{\pgfqpoint{5.725014in}{1.115090in}}%
\pgfpathlineto{\pgfqpoint{5.850014in}{1.115090in}}%
\pgfusepath{stroke}%
\end{pgfscope}%
\begin{pgfscope}%
\definecolor{textcolor}{rgb}{0.000000,0.000000,0.000000}%
\pgfsetstrokecolor{textcolor}%
\pgfsetfillcolor{textcolor}%
\pgftext[x=5.950014in,y=1.071340in,left,base]{\color{textcolor}{\rmfamily\fontsize{9.000000}{10.800000}\selectfont\catcode`\^=\active\def^{\ifmmode\sp\else\^{}\fi}\catcode`\%=\active\def%{\%}\None{} \& \MergeLinear{}}}%
\end{pgfscope}%
\begin{pgfscope}%
\pgfsetrectcap%
\pgfsetroundjoin%
\pgfsetlinewidth{1.505625pt}%
\pgfsetstrokecolor{currentstroke7}%
\pgfsetdash{}{0pt}%
\pgfpathmoveto{\pgfqpoint{5.600014in}{0.931619in}}%
\pgfpathlineto{\pgfqpoint{5.725014in}{0.931619in}}%
\pgfpathlineto{\pgfqpoint{5.850014in}{0.931619in}}%
\pgfusepath{stroke}%
\end{pgfscope}%
\begin{pgfscope}%
\definecolor{textcolor}{rgb}{0.000000,0.000000,0.000000}%
\pgfsetstrokecolor{textcolor}%
\pgfsetfillcolor{textcolor}%
\pgftext[x=5.950014in,y=0.887869in,left,base]{\color{textcolor}{\rmfamily\fontsize{9.000000}{10.800000}\selectfont\catcode`\^=\active\def^{\ifmmode\sp\else\^{}\fi}\catcode`\%=\active\def%{\%}\None{} \& \SharedVertices{}}}%
\end{pgfscope}%
\end{pgfpicture}%
\makeatother%
\endgroup%
}
	\caption[Running time for minimally rigid graphs.]{
		Mean running time (ms) to find all NAC-colorings for minimally rigid graphs.}%
	\label{fig:graph_time_minimally_rigid}
\end{figure}

If we analyze the number of \IsNACColoring{} calls performed by \NaiveCycles{} and \Subgraphs{} algorithms
as shown in \Cref{fig:graph_count_minimally_rigid},
you can see that the number of \IsNACColoring{} calls is reduced already for graphs
with eleven vertices,
even though the \NaiveCycles{} algorithm is still faster for these graphs.

\begin{figure}[ht]
	\centering
	\scalebox{0.5}{%% Creator: Matplotlib, PGF backend
%%
%% To include the figure in your LaTeX document, write
%%   \input{<filename>.pgf}
%%
%% Make sure the required packages are loaded in your preamble
%%   \usepackage{pgf}
%%
%% Also ensure that all the required font packages are loaded; for instance,
%% the lmodern package is sometimes necessary when using math font.
%%   \usepackage{lmodern}
%%
%% Figures using additional raster images can only be included by \input if
%% they are in the same directory as the main LaTeX file. For loading figures
%% from other directories you can use the `import` package
%%   \usepackage{import}
%%
%% and then include the figures with
%%   \import{<path to file>}{<filename>.pgf}
%%
%% Matplotlib used the following preamble
%%   \def\mathdefault#1{#1}
%%   \everymath=\expandafter{\the\everymath\displaystyle}
%%   
%%   \ifdefined\pdftexversion\else  % non-pdftex case.
%%     \usepackage{fontspec}
%%     \setmainfont{DejaVuSans.ttf}[Path=\detokenize{/home/petr/Projects/PyRigi/.venv/lib/python3.12/site-packages/matplotlib/mpl-data/fonts/ttf/}]
%%     \setsansfont{DejaVuSans.ttf}[Path=\detokenize{/home/petr/Projects/PyRigi/.venv/lib/python3.12/site-packages/matplotlib/mpl-data/fonts/ttf/}]
%%     \setmonofont{DejaVuSansMono.ttf}[Path=\detokenize{/home/petr/Projects/PyRigi/.venv/lib/python3.12/site-packages/matplotlib/mpl-data/fonts/ttf/}]
%%   \fi
%%   \makeatletter\@ifpackageloaded{underscore}{}{\usepackage[strings]{underscore}}\makeatother
%%
\begingroup%
\makeatletter%
\begin{pgfpicture}%
\pgfpathrectangle{\pgfpointorigin}{\pgfqpoint{8.384376in}{2.25in}}%
\pgfusepath{use as bounding box, clip}%
\begin{pgfscope}%
\pgfsetbuttcap%
\pgfsetmiterjoin%
\definecolor{currentfill}{rgb}{1.000000,1.000000,1.000000}%
\pgfsetfillcolor{currentfill}%
\pgfsetlinewidth{0.000000pt}%
\definecolor{currentstroke}{rgb}{1.000000,1.000000,1.000000}%
\pgfsetstrokecolor{currentstroke}%
\pgfsetdash{}{0pt}%
\pgfpathmoveto{\pgfqpoint{0.000000in}{0.000000in}}%
\pgfpathlineto{\pgfqpoint{8.384376in}{0.000000in}}%
\pgfpathlineto{\pgfqpoint{8.384376in}{2.841849in}}%
\pgfpathlineto{\pgfqpoint{0.000000in}{2.841849in}}%
\pgfpathlineto{\pgfqpoint{0.000000in}{0.000000in}}%
\pgfpathclose%
\pgfusepath{fill}%
\end{pgfscope}%
\begin{pgfscope}%
\pgfsetbuttcap%
\pgfsetmiterjoin%
\definecolor{currentfill}{rgb}{1.000000,1.000000,1.000000}%
\pgfsetfillcolor{currentfill}%
\pgfsetlinewidth{0.000000pt}%
\definecolor{currentstroke}{rgb}{0.000000,0.000000,0.000000}%
\pgfsetstrokecolor{currentstroke}%
\pgfsetstrokeopacity{0.000000}%
\pgfsetdash{}{0pt}%
\pgfpathmoveto{\pgfqpoint{0.398419in}{0.521603in}}%
\pgfpathlineto{\pgfqpoint{5.487514in}{0.521603in}}%
\pgfpathlineto{\pgfqpoint{5.487514in}{2.206605in}}%
\pgfpathlineto{\pgfqpoint{0.398419in}{2.206605in}}%
\pgfpathlineto{\pgfqpoint{0.398419in}{0.521603in}}%
\pgfpathclose%
\pgfusepath{fill}%
\end{pgfscope}%
\begin{pgfscope}%
\pgfsetbuttcap%
\pgfsetroundjoin%
\definecolor{currentfill}{rgb}{0.000000,0.000000,0.000000}%
\pgfsetfillcolor{currentfill}%
\pgfsetlinewidth{0.803000pt}%
\definecolor{currentstroke}{rgb}{0.000000,0.000000,0.000000}%
\pgfsetstrokecolor{currentstroke}%
\pgfsetdash{}{0pt}%
\pgfsys@defobject{currentmarker}{\pgfqpoint{0.000000in}{-0.048611in}}{\pgfqpoint{0.000000in}{0.000000in}}{%
\pgfpathmoveto{\pgfqpoint{0.000000in}{0.000000in}}%
\pgfpathlineto{\pgfqpoint{0.000000in}{-0.048611in}}%
\pgfusepath{stroke,fill}%
}%
\begin{pgfscope}%
\pgfsys@transformshift{0.801091in}{0.521603in}%
\pgfsys@useobject{currentmarker}{}%
\end{pgfscope}%
\end{pgfscope}%
\begin{pgfscope}%
\definecolor{textcolor}{rgb}{0.000000,0.000000,0.000000}%
\pgfsetstrokecolor{textcolor}%
\pgfsetfillcolor{textcolor}%
\pgftext[x=0.801091in,y=0.424381in,,top]{\color{textcolor}{\rmfamily\fontsize{10.000000}{12.000000}\selectfont\catcode`\^=\active\def^{\ifmmode\sp\else\^{}\fi}\catcode`\%=\active\def%{\%}$\mathdefault{3}$}}%
\end{pgfscope}%
\begin{pgfscope}%
\pgfsetbuttcap%
\pgfsetroundjoin%
\definecolor{currentfill}{rgb}{0.000000,0.000000,0.000000}%
\pgfsetfillcolor{currentfill}%
\pgfsetlinewidth{0.803000pt}%
\definecolor{currentstroke}{rgb}{0.000000,0.000000,0.000000}%
\pgfsetstrokecolor{currentstroke}%
\pgfsetdash{}{0pt}%
\pgfsys@defobject{currentmarker}{\pgfqpoint{0.000000in}{-0.048611in}}{\pgfqpoint{0.000000in}{0.000000in}}{%
\pgfpathmoveto{\pgfqpoint{0.000000in}{0.000000in}}%
\pgfpathlineto{\pgfqpoint{0.000000in}{-0.048611in}}%
\pgfusepath{stroke,fill}%
}%
\begin{pgfscope}%
\pgfsys@transformshift{1.315141in}{0.521603in}%
\pgfsys@useobject{currentmarker}{}%
\end{pgfscope}%
\end{pgfscope}%
\begin{pgfscope}%
\definecolor{textcolor}{rgb}{0.000000,0.000000,0.000000}%
\pgfsetstrokecolor{textcolor}%
\pgfsetfillcolor{textcolor}%
\pgftext[x=1.315141in,y=0.424381in,,top]{\color{textcolor}{\rmfamily\fontsize{10.000000}{12.000000}\selectfont\catcode`\^=\active\def^{\ifmmode\sp\else\^{}\fi}\catcode`\%=\active\def%{\%}$\mathdefault{6}$}}%
\end{pgfscope}%
\begin{pgfscope}%
\pgfsetbuttcap%
\pgfsetroundjoin%
\definecolor{currentfill}{rgb}{0.000000,0.000000,0.000000}%
\pgfsetfillcolor{currentfill}%
\pgfsetlinewidth{0.803000pt}%
\definecolor{currentstroke}{rgb}{0.000000,0.000000,0.000000}%
\pgfsetstrokecolor{currentstroke}%
\pgfsetdash{}{0pt}%
\pgfsys@defobject{currentmarker}{\pgfqpoint{0.000000in}{-0.048611in}}{\pgfqpoint{0.000000in}{0.000000in}}{%
\pgfpathmoveto{\pgfqpoint{0.000000in}{0.000000in}}%
\pgfpathlineto{\pgfqpoint{0.000000in}{-0.048611in}}%
\pgfusepath{stroke,fill}%
}%
\begin{pgfscope}%
\pgfsys@transformshift{1.829191in}{0.521603in}%
\pgfsys@useobject{currentmarker}{}%
\end{pgfscope}%
\end{pgfscope}%
\begin{pgfscope}%
\definecolor{textcolor}{rgb}{0.000000,0.000000,0.000000}%
\pgfsetstrokecolor{textcolor}%
\pgfsetfillcolor{textcolor}%
\pgftext[x=1.829191in,y=0.424381in,,top]{\color{textcolor}{\rmfamily\fontsize{10.000000}{12.000000}\selectfont\catcode`\^=\active\def^{\ifmmode\sp\else\^{}\fi}\catcode`\%=\active\def%{\%}$\mathdefault{9}$}}%
\end{pgfscope}%
\begin{pgfscope}%
\pgfsetbuttcap%
\pgfsetroundjoin%
\definecolor{currentfill}{rgb}{0.000000,0.000000,0.000000}%
\pgfsetfillcolor{currentfill}%
\pgfsetlinewidth{0.803000pt}%
\definecolor{currentstroke}{rgb}{0.000000,0.000000,0.000000}%
\pgfsetstrokecolor{currentstroke}%
\pgfsetdash{}{0pt}%
\pgfsys@defobject{currentmarker}{\pgfqpoint{0.000000in}{-0.048611in}}{\pgfqpoint{0.000000in}{0.000000in}}{%
\pgfpathmoveto{\pgfqpoint{0.000000in}{0.000000in}}%
\pgfpathlineto{\pgfqpoint{0.000000in}{-0.048611in}}%
\pgfusepath{stroke,fill}%
}%
\begin{pgfscope}%
\pgfsys@transformshift{2.343241in}{0.521603in}%
\pgfsys@useobject{currentmarker}{}%
\end{pgfscope}%
\end{pgfscope}%
\begin{pgfscope}%
\definecolor{textcolor}{rgb}{0.000000,0.000000,0.000000}%
\pgfsetstrokecolor{textcolor}%
\pgfsetfillcolor{textcolor}%
\pgftext[x=2.343241in,y=0.424381in,,top]{\color{textcolor}{\rmfamily\fontsize{10.000000}{12.000000}\selectfont\catcode`\^=\active\def^{\ifmmode\sp\else\^{}\fi}\catcode`\%=\active\def%{\%}$\mathdefault{12}$}}%
\end{pgfscope}%
\begin{pgfscope}%
\pgfsetbuttcap%
\pgfsetroundjoin%
\definecolor{currentfill}{rgb}{0.000000,0.000000,0.000000}%
\pgfsetfillcolor{currentfill}%
\pgfsetlinewidth{0.803000pt}%
\definecolor{currentstroke}{rgb}{0.000000,0.000000,0.000000}%
\pgfsetstrokecolor{currentstroke}%
\pgfsetdash{}{0pt}%
\pgfsys@defobject{currentmarker}{\pgfqpoint{0.000000in}{-0.048611in}}{\pgfqpoint{0.000000in}{0.000000in}}{%
\pgfpathmoveto{\pgfqpoint{0.000000in}{0.000000in}}%
\pgfpathlineto{\pgfqpoint{0.000000in}{-0.048611in}}%
\pgfusepath{stroke,fill}%
}%
\begin{pgfscope}%
\pgfsys@transformshift{2.857291in}{0.521603in}%
\pgfsys@useobject{currentmarker}{}%
\end{pgfscope}%
\end{pgfscope}%
\begin{pgfscope}%
\definecolor{textcolor}{rgb}{0.000000,0.000000,0.000000}%
\pgfsetstrokecolor{textcolor}%
\pgfsetfillcolor{textcolor}%
\pgftext[x=2.857291in,y=0.424381in,,top]{\color{textcolor}{\rmfamily\fontsize{10.000000}{12.000000}\selectfont\catcode`\^=\active\def^{\ifmmode\sp\else\^{}\fi}\catcode`\%=\active\def%{\%}$\mathdefault{15}$}}%
\end{pgfscope}%
\begin{pgfscope}%
\pgfsetbuttcap%
\pgfsetroundjoin%
\definecolor{currentfill}{rgb}{0.000000,0.000000,0.000000}%
\pgfsetfillcolor{currentfill}%
\pgfsetlinewidth{0.803000pt}%
\definecolor{currentstroke}{rgb}{0.000000,0.000000,0.000000}%
\pgfsetstrokecolor{currentstroke}%
\pgfsetdash{}{0pt}%
\pgfsys@defobject{currentmarker}{\pgfqpoint{0.000000in}{-0.048611in}}{\pgfqpoint{0.000000in}{0.000000in}}{%
\pgfpathmoveto{\pgfqpoint{0.000000in}{0.000000in}}%
\pgfpathlineto{\pgfqpoint{0.000000in}{-0.048611in}}%
\pgfusepath{stroke,fill}%
}%
\begin{pgfscope}%
\pgfsys@transformshift{3.371341in}{0.521603in}%
\pgfsys@useobject{currentmarker}{}%
\end{pgfscope}%
\end{pgfscope}%
\begin{pgfscope}%
\definecolor{textcolor}{rgb}{0.000000,0.000000,0.000000}%
\pgfsetstrokecolor{textcolor}%
\pgfsetfillcolor{textcolor}%
\pgftext[x=3.371341in,y=0.424381in,,top]{\color{textcolor}{\rmfamily\fontsize{10.000000}{12.000000}\selectfont\catcode`\^=\active\def^{\ifmmode\sp\else\^{}\fi}\catcode`\%=\active\def%{\%}$\mathdefault{18}$}}%
\end{pgfscope}%
\begin{pgfscope}%
\pgfsetbuttcap%
\pgfsetroundjoin%
\definecolor{currentfill}{rgb}{0.000000,0.000000,0.000000}%
\pgfsetfillcolor{currentfill}%
\pgfsetlinewidth{0.803000pt}%
\definecolor{currentstroke}{rgb}{0.000000,0.000000,0.000000}%
\pgfsetstrokecolor{currentstroke}%
\pgfsetdash{}{0pt}%
\pgfsys@defobject{currentmarker}{\pgfqpoint{0.000000in}{-0.048611in}}{\pgfqpoint{0.000000in}{0.000000in}}{%
\pgfpathmoveto{\pgfqpoint{0.000000in}{0.000000in}}%
\pgfpathlineto{\pgfqpoint{0.000000in}{-0.048611in}}%
\pgfusepath{stroke,fill}%
}%
\begin{pgfscope}%
\pgfsys@transformshift{3.885391in}{0.521603in}%
\pgfsys@useobject{currentmarker}{}%
\end{pgfscope}%
\end{pgfscope}%
\begin{pgfscope}%
\definecolor{textcolor}{rgb}{0.000000,0.000000,0.000000}%
\pgfsetstrokecolor{textcolor}%
\pgfsetfillcolor{textcolor}%
\pgftext[x=3.885391in,y=0.424381in,,top]{\color{textcolor}{\rmfamily\fontsize{10.000000}{12.000000}\selectfont\catcode`\^=\active\def^{\ifmmode\sp\else\^{}\fi}\catcode`\%=\active\def%{\%}$\mathdefault{21}$}}%
\end{pgfscope}%
\begin{pgfscope}%
\pgfsetbuttcap%
\pgfsetroundjoin%
\definecolor{currentfill}{rgb}{0.000000,0.000000,0.000000}%
\pgfsetfillcolor{currentfill}%
\pgfsetlinewidth{0.803000pt}%
\definecolor{currentstroke}{rgb}{0.000000,0.000000,0.000000}%
\pgfsetstrokecolor{currentstroke}%
\pgfsetdash{}{0pt}%
\pgfsys@defobject{currentmarker}{\pgfqpoint{0.000000in}{-0.048611in}}{\pgfqpoint{0.000000in}{0.000000in}}{%
\pgfpathmoveto{\pgfqpoint{0.000000in}{0.000000in}}%
\pgfpathlineto{\pgfqpoint{0.000000in}{-0.048611in}}%
\pgfusepath{stroke,fill}%
}%
\begin{pgfscope}%
\pgfsys@transformshift{4.399441in}{0.521603in}%
\pgfsys@useobject{currentmarker}{}%
\end{pgfscope}%
\end{pgfscope}%
\begin{pgfscope}%
\definecolor{textcolor}{rgb}{0.000000,0.000000,0.000000}%
\pgfsetstrokecolor{textcolor}%
\pgfsetfillcolor{textcolor}%
\pgftext[x=4.399441in,y=0.424381in,,top]{\color{textcolor}{\rmfamily\fontsize{10.000000}{12.000000}\selectfont\catcode`\^=\active\def^{\ifmmode\sp\else\^{}\fi}\catcode`\%=\active\def%{\%}$\mathdefault{24}$}}%
\end{pgfscope}%
\begin{pgfscope}%
\pgfsetbuttcap%
\pgfsetroundjoin%
\definecolor{currentfill}{rgb}{0.000000,0.000000,0.000000}%
\pgfsetfillcolor{currentfill}%
\pgfsetlinewidth{0.803000pt}%
\definecolor{currentstroke}{rgb}{0.000000,0.000000,0.000000}%
\pgfsetstrokecolor{currentstroke}%
\pgfsetdash{}{0pt}%
\pgfsys@defobject{currentmarker}{\pgfqpoint{0.000000in}{-0.048611in}}{\pgfqpoint{0.000000in}{0.000000in}}{%
\pgfpathmoveto{\pgfqpoint{0.000000in}{0.000000in}}%
\pgfpathlineto{\pgfqpoint{0.000000in}{-0.048611in}}%
\pgfusepath{stroke,fill}%
}%
\begin{pgfscope}%
\pgfsys@transformshift{4.913491in}{0.521603in}%
\pgfsys@useobject{currentmarker}{}%
\end{pgfscope}%
\end{pgfscope}%
\begin{pgfscope}%
\definecolor{textcolor}{rgb}{0.000000,0.000000,0.000000}%
\pgfsetstrokecolor{textcolor}%
\pgfsetfillcolor{textcolor}%
\pgftext[x=4.913491in,y=0.424381in,,top]{\color{textcolor}{\rmfamily\fontsize{10.000000}{12.000000}\selectfont\catcode`\^=\active\def^{\ifmmode\sp\else\^{}\fi}\catcode`\%=\active\def%{\%}$\mathdefault{27}$}}%
\end{pgfscope}%
\begin{pgfscope}%
\pgfsetbuttcap%
\pgfsetroundjoin%
\definecolor{currentfill}{rgb}{0.000000,0.000000,0.000000}%
\pgfsetfillcolor{currentfill}%
\pgfsetlinewidth{0.803000pt}%
\definecolor{currentstroke}{rgb}{0.000000,0.000000,0.000000}%
\pgfsetstrokecolor{currentstroke}%
\pgfsetdash{}{0pt}%
\pgfsys@defobject{currentmarker}{\pgfqpoint{0.000000in}{-0.048611in}}{\pgfqpoint{0.000000in}{0.000000in}}{%
\pgfpathmoveto{\pgfqpoint{0.000000in}{0.000000in}}%
\pgfpathlineto{\pgfqpoint{0.000000in}{-0.048611in}}%
\pgfusepath{stroke,fill}%
}%
\begin{pgfscope}%
\pgfsys@transformshift{5.427541in}{0.521603in}%
\pgfsys@useobject{currentmarker}{}%
\end{pgfscope}%
\end{pgfscope}%
\begin{pgfscope}%
\definecolor{textcolor}{rgb}{0.000000,0.000000,0.000000}%
\pgfsetstrokecolor{textcolor}%
\pgfsetfillcolor{textcolor}%
\pgftext[x=5.427541in,y=0.424381in,,top]{\color{textcolor}{\rmfamily\fontsize{10.000000}{12.000000}\selectfont\catcode`\^=\active\def^{\ifmmode\sp\else\^{}\fi}\catcode`\%=\active\def%{\%}$\mathdefault{30}$}}%
\end{pgfscope}%
\begin{pgfscope}%
\definecolor{textcolor}{rgb}{0.000000,0.000000,0.000000}%
\pgfsetstrokecolor{textcolor}%
\pgfsetfillcolor{textcolor}%
\pgftext[x=2.942966in,y=0.234413in,,top]{\color{textcolor}{\rmfamily\fontsize{10.000000}{12.000000}\selectfont\catcode`\^=\active\def^{\ifmmode\sp\else\^{}\fi}\catcode`\%=\active\def%{\%}Monochromatic classes}}%
\end{pgfscope}%
\begin{pgfscope}%
\pgfsetbuttcap%
\pgfsetroundjoin%
\definecolor{currentfill}{rgb}{0.000000,0.000000,0.000000}%
\pgfsetfillcolor{currentfill}%
\pgfsetlinewidth{0.803000pt}%
\definecolor{currentstroke}{rgb}{0.000000,0.000000,0.000000}%
\pgfsetstrokecolor{currentstroke}%
\pgfsetdash{}{0pt}%
\pgfsys@defobject{currentmarker}{\pgfqpoint{-0.048611in}{0.000000in}}{\pgfqpoint{-0.000000in}{0.000000in}}{%
\pgfpathmoveto{\pgfqpoint{-0.000000in}{0.000000in}}%
\pgfpathlineto{\pgfqpoint{-0.048611in}{0.000000in}}%
\pgfusepath{stroke,fill}%
}%
\begin{pgfscope}%
\pgfsys@transformshift{0.398419in}{0.810219in}%
\pgfsys@useobject{currentmarker}{}%
\end{pgfscope}%
\end{pgfscope}%
\begin{pgfscope}%
\definecolor{textcolor}{rgb}{0.000000,0.000000,0.000000}%
\pgfsetstrokecolor{textcolor}%
\pgfsetfillcolor{textcolor}%
\pgftext[x=0.100000in, y=0.757458in, left, base]{\color{textcolor}{\rmfamily\fontsize{10.000000}{12.000000}\selectfont\catcode`\^=\active\def^{\ifmmode\sp\else\^{}\fi}\catcode`\%=\active\def%{\%}$\mathdefault{10^{1}}$}}%
\end{pgfscope}%
\begin{pgfscope}%
\pgfsetbuttcap%
\pgfsetroundjoin%
\definecolor{currentfill}{rgb}{0.000000,0.000000,0.000000}%
\pgfsetfillcolor{currentfill}%
\pgfsetlinewidth{0.803000pt}%
\definecolor{currentstroke}{rgb}{0.000000,0.000000,0.000000}%
\pgfsetstrokecolor{currentstroke}%
\pgfsetdash{}{0pt}%
\pgfsys@defobject{currentmarker}{\pgfqpoint{-0.048611in}{0.000000in}}{\pgfqpoint{-0.000000in}{0.000000in}}{%
\pgfpathmoveto{\pgfqpoint{-0.000000in}{0.000000in}}%
\pgfpathlineto{\pgfqpoint{-0.048611in}{0.000000in}}%
\pgfusepath{stroke,fill}%
}%
\begin{pgfscope}%
\pgfsys@transformshift{0.398419in}{1.234269in}%
\pgfsys@useobject{currentmarker}{}%
\end{pgfscope}%
\end{pgfscope}%
\begin{pgfscope}%
\definecolor{textcolor}{rgb}{0.000000,0.000000,0.000000}%
\pgfsetstrokecolor{textcolor}%
\pgfsetfillcolor{textcolor}%
\pgftext[x=0.100000in, y=1.181507in, left, base]{\color{textcolor}{\rmfamily\fontsize{10.000000}{12.000000}\selectfont\catcode`\^=\active\def^{\ifmmode\sp\else\^{}\fi}\catcode`\%=\active\def%{\%}$\mathdefault{10^{3}}$}}%
\end{pgfscope}%
\begin{pgfscope}%
\pgfsetbuttcap%
\pgfsetroundjoin%
\definecolor{currentfill}{rgb}{0.000000,0.000000,0.000000}%
\pgfsetfillcolor{currentfill}%
\pgfsetlinewidth{0.803000pt}%
\definecolor{currentstroke}{rgb}{0.000000,0.000000,0.000000}%
\pgfsetstrokecolor{currentstroke}%
\pgfsetdash{}{0pt}%
\pgfsys@defobject{currentmarker}{\pgfqpoint{-0.048611in}{0.000000in}}{\pgfqpoint{-0.000000in}{0.000000in}}{%
\pgfpathmoveto{\pgfqpoint{-0.000000in}{0.000000in}}%
\pgfpathlineto{\pgfqpoint{-0.048611in}{0.000000in}}%
\pgfusepath{stroke,fill}%
}%
\begin{pgfscope}%
\pgfsys@transformshift{0.398419in}{1.658318in}%
\pgfsys@useobject{currentmarker}{}%
\end{pgfscope}%
\end{pgfscope}%
\begin{pgfscope}%
\definecolor{textcolor}{rgb}{0.000000,0.000000,0.000000}%
\pgfsetstrokecolor{textcolor}%
\pgfsetfillcolor{textcolor}%
\pgftext[x=0.100000in, y=1.605557in, left, base]{\color{textcolor}{\rmfamily\fontsize{10.000000}{12.000000}\selectfont\catcode`\^=\active\def^{\ifmmode\sp\else\^{}\fi}\catcode`\%=\active\def%{\%}$\mathdefault{10^{5}}$}}%
\end{pgfscope}%
\begin{pgfscope}%
\pgfsetbuttcap%
\pgfsetroundjoin%
\definecolor{currentfill}{rgb}{0.000000,0.000000,0.000000}%
\pgfsetfillcolor{currentfill}%
\pgfsetlinewidth{0.803000pt}%
\definecolor{currentstroke}{rgb}{0.000000,0.000000,0.000000}%
\pgfsetstrokecolor{currentstroke}%
\pgfsetdash{}{0pt}%
\pgfsys@defobject{currentmarker}{\pgfqpoint{-0.048611in}{0.000000in}}{\pgfqpoint{-0.000000in}{0.000000in}}{%
\pgfpathmoveto{\pgfqpoint{-0.000000in}{0.000000in}}%
\pgfpathlineto{\pgfqpoint{-0.048611in}{0.000000in}}%
\pgfusepath{stroke,fill}%
}%
\begin{pgfscope}%
\pgfsys@transformshift{0.398419in}{2.082368in}%
\pgfsys@useobject{currentmarker}{}%
\end{pgfscope}%
\end{pgfscope}%
\begin{pgfscope}%
\definecolor{textcolor}{rgb}{0.000000,0.000000,0.000000}%
\pgfsetstrokecolor{textcolor}%
\pgfsetfillcolor{textcolor}%
\pgftext[x=0.100000in, y=2.029606in, left, base]{\color{textcolor}{\rmfamily\fontsize{10.000000}{12.000000}\selectfont\catcode`\^=\active\def^{\ifmmode\sp\else\^{}\fi}\catcode`\%=\active\def%{\%}$\mathdefault{10^{7}}$}}%
\end{pgfscope}%
\begin{pgfscope}%
\pgfpathrectangle{\pgfqpoint{0.398419in}{0.521603in}}{\pgfqpoint{5.089095in}{1.685002in}}%
\pgfusepath{clip}%
\pgfsetrectcap%
\pgfsetroundjoin%
\pgfsetlinewidth{1.505625pt}%
\pgfsetstrokecolor{currentstroke1}%
\pgfsetdash{}{0pt}%
\pgfpathmoveto{\pgfqpoint{0.629741in}{0.598194in}}%
\pgfpathlineto{\pgfqpoint{0.801091in}{0.699356in}}%
\pgfpathlineto{\pgfqpoint{0.972441in}{0.777376in}}%
\pgfpathlineto{\pgfqpoint{1.143791in}{0.847555in}}%
\pgfpathlineto{\pgfqpoint{1.315141in}{0.914400in}}%
\pgfpathlineto{\pgfqpoint{1.486491in}{0.979699in}}%
\pgfpathlineto{\pgfqpoint{1.657841in}{1.044253in}}%
\pgfpathlineto{\pgfqpoint{1.829191in}{1.108441in}}%
\pgfpathlineto{\pgfqpoint{2.000541in}{1.172447in}}%
\pgfpathlineto{\pgfqpoint{2.171891in}{1.236363in}}%
\pgfpathlineto{\pgfqpoint{2.343241in}{1.300233in}}%
\pgfpathlineto{\pgfqpoint{2.514591in}{1.364082in}}%
\pgfpathlineto{\pgfqpoint{2.685941in}{1.427919in}}%
\pgfpathlineto{\pgfqpoint{2.857291in}{1.491750in}}%
\pgfpathlineto{\pgfqpoint{3.028641in}{1.555579in}}%
\pgfpathlineto{\pgfqpoint{3.199991in}{1.619406in}}%
\pgfpathlineto{\pgfqpoint{3.371341in}{1.683233in}}%
\pgfpathlineto{\pgfqpoint{3.542691in}{1.747059in}}%
\pgfpathlineto{\pgfqpoint{3.714041in}{1.810885in}}%
\pgfpathlineto{\pgfqpoint{3.885391in}{1.874711in}}%
\pgfpathlineto{\pgfqpoint{4.056741in}{1.938537in}}%
\pgfpathlineto{\pgfqpoint{4.228091in}{2.002362in}}%
\pgfpathlineto{\pgfqpoint{4.399441in}{2.066188in}}%
\pgfpathlineto{\pgfqpoint{4.570791in}{2.130014in}}%
\pgfusepath{stroke}%
\end{pgfscope}%
\begin{pgfscope}%
\pgfpathrectangle{\pgfqpoint{0.398419in}{0.521603in}}{\pgfqpoint{5.089095in}{1.685002in}}%
\pgfusepath{clip}%
\pgfsetrectcap%
\pgfsetroundjoin%
\pgfsetlinewidth{1.505625pt}%
\pgfsetstrokecolor{currentstroke2}%
\pgfsetdash{}{0pt}%
\pgfpathmoveto{\pgfqpoint{0.629741in}{0.662020in}}%
\pgfpathlineto{\pgfqpoint{0.801091in}{0.725846in}}%
\pgfpathlineto{\pgfqpoint{0.972441in}{0.789672in}}%
\pgfpathlineto{\pgfqpoint{1.143791in}{0.853498in}}%
\pgfpathlineto{\pgfqpoint{1.315141in}{0.917323in}}%
\pgfpathlineto{\pgfqpoint{1.486491in}{0.981149in}}%
\pgfpathlineto{\pgfqpoint{1.657841in}{1.044975in}}%
\pgfpathlineto{\pgfqpoint{1.829191in}{1.108801in}}%
\pgfpathlineto{\pgfqpoint{2.000541in}{1.172627in}}%
\pgfpathlineto{\pgfqpoint{2.171891in}{1.236453in}}%
\pgfpathlineto{\pgfqpoint{2.343241in}{1.214311in}}%
\pgfpathlineto{\pgfqpoint{2.514591in}{1.255336in}}%
\pgfpathlineto{\pgfqpoint{2.685941in}{1.324054in}}%
\pgfpathlineto{\pgfqpoint{2.857291in}{1.338158in}}%
\pgfpathlineto{\pgfqpoint{3.028641in}{1.413081in}}%
\pgfpathlineto{\pgfqpoint{3.199991in}{1.418030in}}%
\pgfpathlineto{\pgfqpoint{3.371341in}{1.458568in}}%
\pgfpathlineto{\pgfqpoint{3.542691in}{1.476266in}}%
\pgfpathlineto{\pgfqpoint{3.714041in}{1.532335in}}%
\pgfpathlineto{\pgfqpoint{3.885391in}{1.588728in}}%
\pgfpathlineto{\pgfqpoint{4.056741in}{1.647274in}}%
\pgfpathlineto{\pgfqpoint{4.228091in}{1.631205in}}%
\pgfpathlineto{\pgfqpoint{4.399441in}{1.690928in}}%
\pgfpathlineto{\pgfqpoint{4.570791in}{1.688935in}}%
\pgfpathlineto{\pgfqpoint{4.742141in}{1.810227in}}%
\pgfpathlineto{\pgfqpoint{4.913491in}{1.689196in}}%
\pgfpathlineto{\pgfqpoint{5.084841in}{1.821089in}}%
\pgfpathlineto{\pgfqpoint{5.256191in}{1.860280in}}%
\pgfusepath{stroke}%
\end{pgfscope}%
\begin{pgfscope}%
\pgfpathrectangle{\pgfqpoint{0.398419in}{0.521603in}}{\pgfqpoint{5.089095in}{1.685002in}}%
\pgfusepath{clip}%
\pgfsetrectcap%
\pgfsetroundjoin%
\pgfsetlinewidth{1.505625pt}%
\pgfsetstrokecolor{currentstroke3}%
\pgfsetdash{}{0pt}%
\pgfpathmoveto{\pgfqpoint{0.629741in}{0.662020in}}%
\pgfpathlineto{\pgfqpoint{0.801091in}{0.725846in}}%
\pgfpathlineto{\pgfqpoint{0.972441in}{0.789672in}}%
\pgfpathlineto{\pgfqpoint{1.143791in}{0.853498in}}%
\pgfpathlineto{\pgfqpoint{1.315141in}{0.917323in}}%
\pgfpathlineto{\pgfqpoint{1.486491in}{0.981149in}}%
\pgfpathlineto{\pgfqpoint{1.657841in}{1.044975in}}%
\pgfpathlineto{\pgfqpoint{1.829191in}{1.108801in}}%
\pgfpathlineto{\pgfqpoint{2.000541in}{1.172627in}}%
\pgfpathlineto{\pgfqpoint{2.171891in}{1.236453in}}%
\pgfpathlineto{\pgfqpoint{2.343241in}{1.214311in}}%
\pgfpathlineto{\pgfqpoint{2.514591in}{1.255336in}}%
\pgfpathlineto{\pgfqpoint{2.685941in}{1.324054in}}%
\pgfpathlineto{\pgfqpoint{2.857291in}{1.338158in}}%
\pgfpathlineto{\pgfqpoint{3.028641in}{1.413081in}}%
\pgfpathlineto{\pgfqpoint{3.199991in}{1.418030in}}%
\pgfpathlineto{\pgfqpoint{3.371341in}{1.426980in}}%
\pgfpathlineto{\pgfqpoint{3.542691in}{1.450439in}}%
\pgfpathlineto{\pgfqpoint{3.714041in}{1.535975in}}%
\pgfpathlineto{\pgfqpoint{3.885391in}{1.568266in}}%
\pgfpathlineto{\pgfqpoint{4.056741in}{1.594881in}}%
\pgfpathlineto{\pgfqpoint{4.228091in}{1.618614in}}%
\pgfpathlineto{\pgfqpoint{4.399441in}{1.638249in}}%
\pgfpathlineto{\pgfqpoint{4.570791in}{1.656387in}}%
\pgfpathlineto{\pgfqpoint{4.742141in}{1.687709in}}%
\pgfpathlineto{\pgfqpoint{4.913491in}{1.742457in}}%
\pgfpathlineto{\pgfqpoint{5.084841in}{1.821089in}}%
\pgfpathlineto{\pgfqpoint{5.256191in}{1.815659in}}%
\pgfusepath{stroke}%
\end{pgfscope}%
\begin{pgfscope}%
\pgfpathrectangle{\pgfqpoint{0.398419in}{0.521603in}}{\pgfqpoint{5.089095in}{1.685002in}}%
\pgfusepath{clip}%
\pgfsetrectcap%
\pgfsetroundjoin%
\pgfsetlinewidth{1.505625pt}%
\pgfsetstrokecolor{currentstroke4}%
\pgfsetdash{}{0pt}%
\pgfpathmoveto{\pgfqpoint{0.629741in}{0.662020in}}%
\pgfpathlineto{\pgfqpoint{0.801091in}{0.725846in}}%
\pgfpathlineto{\pgfqpoint{0.972441in}{0.789672in}}%
\pgfpathlineto{\pgfqpoint{1.143791in}{0.853498in}}%
\pgfpathlineto{\pgfqpoint{1.315141in}{0.917323in}}%
\pgfpathlineto{\pgfqpoint{1.486491in}{0.981149in}}%
\pgfpathlineto{\pgfqpoint{1.657841in}{1.044975in}}%
\pgfpathlineto{\pgfqpoint{1.829191in}{1.108801in}}%
\pgfpathlineto{\pgfqpoint{2.000541in}{1.172627in}}%
\pgfpathlineto{\pgfqpoint{2.171891in}{1.236453in}}%
\pgfpathlineto{\pgfqpoint{2.343241in}{1.228795in}}%
\pgfpathlineto{\pgfqpoint{2.514591in}{1.260447in}}%
\pgfpathlineto{\pgfqpoint{2.685941in}{1.317037in}}%
\pgfpathlineto{\pgfqpoint{2.857291in}{1.338415in}}%
\pgfpathlineto{\pgfqpoint{3.028641in}{1.406723in}}%
\pgfpathlineto{\pgfqpoint{3.199991in}{1.431470in}}%
\pgfpathlineto{\pgfqpoint{3.371341in}{1.489489in}}%
\pgfpathlineto{\pgfqpoint{3.542691in}{1.499457in}}%
\pgfpathlineto{\pgfqpoint{3.714041in}{1.577720in}}%
\pgfpathlineto{\pgfqpoint{3.885391in}{1.616065in}}%
\pgfpathlineto{\pgfqpoint{4.056741in}{1.664809in}}%
\pgfpathlineto{\pgfqpoint{4.228091in}{1.661415in}}%
\pgfpathlineto{\pgfqpoint{4.399441in}{1.709295in}}%
\pgfpathlineto{\pgfqpoint{4.570791in}{1.738983in}}%
\pgfpathlineto{\pgfqpoint{4.742141in}{1.848061in}}%
\pgfpathlineto{\pgfqpoint{4.913491in}{1.761779in}}%
\pgfpathlineto{\pgfqpoint{5.084841in}{1.765754in}}%
\pgfpathlineto{\pgfqpoint{5.256191in}{1.914782in}}%
\pgfusepath{stroke}%
\end{pgfscope}%
\begin{pgfscope}%
\pgfpathrectangle{\pgfqpoint{0.398419in}{0.521603in}}{\pgfqpoint{5.089095in}{1.685002in}}%
\pgfusepath{clip}%
\pgfsetrectcap%
\pgfsetroundjoin%
\pgfsetlinewidth{1.505625pt}%
\pgfsetstrokecolor{currentstroke5}%
\pgfsetdash{}{0pt}%
\pgfpathmoveto{\pgfqpoint{0.629741in}{0.662020in}}%
\pgfpathlineto{\pgfqpoint{0.801091in}{0.725846in}}%
\pgfpathlineto{\pgfqpoint{0.972441in}{0.789672in}}%
\pgfpathlineto{\pgfqpoint{1.143791in}{0.853498in}}%
\pgfpathlineto{\pgfqpoint{1.315141in}{0.917323in}}%
\pgfpathlineto{\pgfqpoint{1.486491in}{0.981149in}}%
\pgfpathlineto{\pgfqpoint{1.657841in}{1.044975in}}%
\pgfpathlineto{\pgfqpoint{1.829191in}{1.108801in}}%
\pgfpathlineto{\pgfqpoint{2.000541in}{1.172627in}}%
\pgfpathlineto{\pgfqpoint{2.171891in}{1.236453in}}%
\pgfpathlineto{\pgfqpoint{2.343241in}{1.228795in}}%
\pgfpathlineto{\pgfqpoint{2.514591in}{1.260447in}}%
\pgfpathlineto{\pgfqpoint{2.685941in}{1.317037in}}%
\pgfpathlineto{\pgfqpoint{2.857291in}{1.338415in}}%
\pgfpathlineto{\pgfqpoint{3.028641in}{1.406723in}}%
\pgfpathlineto{\pgfqpoint{3.199991in}{1.431470in}}%
\pgfpathlineto{\pgfqpoint{3.371341in}{1.444200in}}%
\pgfpathlineto{\pgfqpoint{3.542691in}{1.446713in}}%
\pgfpathlineto{\pgfqpoint{3.714041in}{1.527104in}}%
\pgfpathlineto{\pgfqpoint{3.885391in}{1.556996in}}%
\pgfpathlineto{\pgfqpoint{4.056741in}{1.604342in}}%
\pgfpathlineto{\pgfqpoint{4.228091in}{1.620035in}}%
\pgfpathlineto{\pgfqpoint{4.399441in}{1.658524in}}%
\pgfpathlineto{\pgfqpoint{4.570791in}{1.645684in}}%
\pgfpathlineto{\pgfqpoint{4.742141in}{1.747170in}}%
\pgfpathlineto{\pgfqpoint{4.913491in}{1.750917in}}%
\pgfpathlineto{\pgfqpoint{5.084841in}{1.764868in}}%
\pgfpathlineto{\pgfqpoint{5.256191in}{1.876573in}}%
\pgfusepath{stroke}%
\end{pgfscope}%
\begin{pgfscope}%
\pgfpathrectangle{\pgfqpoint{0.398419in}{0.521603in}}{\pgfqpoint{5.089095in}{1.685002in}}%
\pgfusepath{clip}%
\pgfsetrectcap%
\pgfsetroundjoin%
\pgfsetlinewidth{1.505625pt}%
\pgfsetstrokecolor{currentstroke6}%
\pgfsetdash{}{0pt}%
\pgfpathmoveto{\pgfqpoint{0.629741in}{0.662020in}}%
\pgfpathlineto{\pgfqpoint{0.801091in}{0.725846in}}%
\pgfpathlineto{\pgfqpoint{0.972441in}{0.789672in}}%
\pgfpathlineto{\pgfqpoint{1.143791in}{0.853498in}}%
\pgfpathlineto{\pgfqpoint{1.315141in}{0.917323in}}%
\pgfpathlineto{\pgfqpoint{1.486491in}{0.981149in}}%
\pgfpathlineto{\pgfqpoint{1.657841in}{1.044975in}}%
\pgfpathlineto{\pgfqpoint{1.829191in}{1.108801in}}%
\pgfpathlineto{\pgfqpoint{2.000541in}{1.172627in}}%
\pgfpathlineto{\pgfqpoint{2.171891in}{1.236453in}}%
\pgfpathlineto{\pgfqpoint{2.343241in}{1.266539in}}%
\pgfpathlineto{\pgfqpoint{2.514591in}{1.310255in}}%
\pgfpathlineto{\pgfqpoint{2.685941in}{1.379686in}}%
\pgfpathlineto{\pgfqpoint{2.857291in}{1.420215in}}%
\pgfpathlineto{\pgfqpoint{3.028641in}{1.489238in}}%
\pgfpathlineto{\pgfqpoint{3.199991in}{1.533803in}}%
\pgfpathlineto{\pgfqpoint{3.371341in}{1.548510in}}%
\pgfpathlineto{\pgfqpoint{3.542691in}{1.577643in}}%
\pgfpathlineto{\pgfqpoint{3.714041in}{1.648168in}}%
\pgfpathlineto{\pgfqpoint{3.885391in}{1.682898in}}%
\pgfpathlineto{\pgfqpoint{4.056741in}{1.765142in}}%
\pgfpathlineto{\pgfqpoint{4.228091in}{1.756412in}}%
\pgfpathlineto{\pgfqpoint{4.399441in}{1.774679in}}%
\pgfpathlineto{\pgfqpoint{4.570791in}{1.799522in}}%
\pgfpathlineto{\pgfqpoint{4.742141in}{1.809884in}}%
\pgfpathlineto{\pgfqpoint{4.913491in}{1.812692in}}%
\pgfpathlineto{\pgfqpoint{5.084841in}{1.924542in}}%
\pgfpathlineto{\pgfqpoint{5.256191in}{1.912054in}}%
\pgfusepath{stroke}%
\end{pgfscope}%
\begin{pgfscope}%
\pgfpathrectangle{\pgfqpoint{0.398419in}{0.521603in}}{\pgfqpoint{5.089095in}{1.685002in}}%
\pgfusepath{clip}%
\pgfsetrectcap%
\pgfsetroundjoin%
\pgfsetlinewidth{1.505625pt}%
\pgfsetstrokecolor{currentstroke7}%
\pgfsetdash{}{0pt}%
\pgfpathmoveto{\pgfqpoint{0.629741in}{0.662020in}}%
\pgfpathlineto{\pgfqpoint{0.801091in}{0.725846in}}%
\pgfpathlineto{\pgfqpoint{0.972441in}{0.789672in}}%
\pgfpathlineto{\pgfqpoint{1.143791in}{0.853498in}}%
\pgfpathlineto{\pgfqpoint{1.315141in}{0.917323in}}%
\pgfpathlineto{\pgfqpoint{1.486491in}{0.981149in}}%
\pgfpathlineto{\pgfqpoint{1.657841in}{1.044975in}}%
\pgfpathlineto{\pgfqpoint{1.829191in}{1.108801in}}%
\pgfpathlineto{\pgfqpoint{2.000541in}{1.172627in}}%
\pgfpathlineto{\pgfqpoint{2.171891in}{1.236453in}}%
\pgfpathlineto{\pgfqpoint{2.343241in}{1.266539in}}%
\pgfpathlineto{\pgfqpoint{2.514591in}{1.310255in}}%
\pgfpathlineto{\pgfqpoint{2.685941in}{1.379686in}}%
\pgfpathlineto{\pgfqpoint{2.857291in}{1.420215in}}%
\pgfpathlineto{\pgfqpoint{3.028641in}{1.489238in}}%
\pgfpathlineto{\pgfqpoint{3.199991in}{1.533803in}}%
\pgfpathlineto{\pgfqpoint{3.371341in}{1.516723in}}%
\pgfpathlineto{\pgfqpoint{3.542691in}{1.545393in}}%
\pgfpathlineto{\pgfqpoint{3.714041in}{1.629261in}}%
\pgfpathlineto{\pgfqpoint{3.885391in}{1.635095in}}%
\pgfpathlineto{\pgfqpoint{4.056741in}{1.691653in}}%
\pgfpathlineto{\pgfqpoint{4.228091in}{1.736354in}}%
\pgfpathlineto{\pgfqpoint{4.399441in}{1.706867in}}%
\pgfpathlineto{\pgfqpoint{4.570791in}{1.765431in}}%
\pgfpathlineto{\pgfqpoint{4.742141in}{1.838298in}}%
\pgfpathlineto{\pgfqpoint{4.913491in}{1.757069in}}%
\pgfpathlineto{\pgfqpoint{5.084841in}{1.779406in}}%
\pgfpathlineto{\pgfqpoint{5.256191in}{1.901756in}}%
\pgfusepath{stroke}%
\end{pgfscope}%
\begin{pgfscope}%
\pgfsetrectcap%
\pgfsetmiterjoin%
\pgfsetlinewidth{0.803000pt}%
\definecolor{currentstroke}{rgb}{0.000000,0.000000,0.000000}%
\pgfsetstrokecolor{currentstroke}%
\pgfsetdash{}{0pt}%
\pgfpathmoveto{\pgfqpoint{0.398419in}{0.521603in}}%
\pgfpathlineto{\pgfqpoint{0.398419in}{2.206605in}}%
\pgfusepath{stroke}%
\end{pgfscope}%
\begin{pgfscope}%
\pgfsetrectcap%
\pgfsetmiterjoin%
\pgfsetlinewidth{0.803000pt}%
\definecolor{currentstroke}{rgb}{0.000000,0.000000,0.000000}%
\pgfsetstrokecolor{currentstroke}%
\pgfsetdash{}{0pt}%
\pgfpathmoveto{\pgfqpoint{5.487514in}{0.521603in}}%
\pgfpathlineto{\pgfqpoint{5.487514in}{2.206605in}}%
\pgfusepath{stroke}%
\end{pgfscope}%
\begin{pgfscope}%
\pgfsetrectcap%
\pgfsetmiterjoin%
\pgfsetlinewidth{0.803000pt}%
\definecolor{currentstroke}{rgb}{0.000000,0.000000,0.000000}%
\pgfsetstrokecolor{currentstroke}%
\pgfsetdash{}{0pt}%
\pgfpathmoveto{\pgfqpoint{0.398419in}{0.521603in}}%
\pgfpathlineto{\pgfqpoint{5.487514in}{0.521603in}}%
\pgfusepath{stroke}%
\end{pgfscope}%
\begin{pgfscope}%
\pgfsetrectcap%
\pgfsetmiterjoin%
\pgfsetlinewidth{0.803000pt}%
\definecolor{currentstroke}{rgb}{0.000000,0.000000,0.000000}%
\pgfsetstrokecolor{currentstroke}%
\pgfsetdash{}{0pt}%
\pgfpathmoveto{\pgfqpoint{0.398419in}{2.206605in}}%
\pgfpathlineto{\pgfqpoint{5.487514in}{2.206605in}}%
\pgfusepath{stroke}%
\end{pgfscope}%
\begin{pgfscope}%
\pgfsetbuttcap%
\pgfsetmiterjoin%
\definecolor{currentfill}{rgb}{1.000000,1.000000,1.000000}%
\pgfsetfillcolor{currentfill}%
\pgfsetfillopacity{0.800000}%
\pgfsetlinewidth{1.003750pt}%
\definecolor{currentstroke}{rgb}{0.800000,0.800000,0.800000}%
\pgfsetstrokecolor{currentstroke}%
\pgfsetstrokeopacity{0.800000}%
\pgfsetdash{}{0pt}%
\pgfpathmoveto{\pgfqpoint{5.575014in}{0.808389in}}%
\pgfpathlineto{\pgfqpoint{8.259376in}{0.808389in}}%
\pgfpathquadraticcurveto{\pgfqpoint{8.284376in}{0.808389in}}{\pgfqpoint{8.284376in}{0.833389in}}%
\pgfpathlineto{\pgfqpoint{8.284376in}{2.119105in}}%
\pgfpathquadraticcurveto{\pgfqpoint{8.284376in}{2.144105in}}{\pgfqpoint{8.259376in}{2.144105in}}%
\pgfpathlineto{\pgfqpoint{5.575014in}{2.144105in}}%
\pgfpathquadraticcurveto{\pgfqpoint{5.550014in}{2.144105in}}{\pgfqpoint{5.550014in}{2.119105in}}%
\pgfpathlineto{\pgfqpoint{5.550014in}{0.833389in}}%
\pgfpathquadraticcurveto{\pgfqpoint{5.550014in}{0.808389in}}{\pgfqpoint{5.575014in}{0.808389in}}%
\pgfpathlineto{\pgfqpoint{5.575014in}{0.808389in}}%
\pgfpathclose%
\pgfusepath{stroke,fill}%
\end{pgfscope}%
\begin{pgfscope}%
\pgfsetrectcap%
\pgfsetroundjoin%
\pgfsetlinewidth{1.505625pt}%
\pgfsetstrokecolor{currentstroke1}%
\pgfsetdash{}{0pt}%
\pgfpathmoveto{\pgfqpoint{5.600014in}{2.042884in}}%
\pgfpathlineto{\pgfqpoint{5.725014in}{2.042884in}}%
\pgfpathlineto{\pgfqpoint{5.850014in}{2.042884in}}%
\pgfusepath{stroke}%
\end{pgfscope}%
\begin{pgfscope}%
\definecolor{textcolor}{rgb}{0.000000,0.000000,0.000000}%
\pgfsetstrokecolor{textcolor}%
\pgfsetfillcolor{textcolor}%
\pgftext[x=5.950014in,y=1.999134in,left,base]{\color{textcolor}{\rmfamily\fontsize{9.000000}{10.800000}\selectfont\catcode`\^=\active\def^{\ifmmode\sp\else\^{}\fi}\catcode`\%=\active\def%{\%}\NaiveCycles{}}}%
\end{pgfscope}%
\begin{pgfscope}%
\pgfsetrectcap%
\pgfsetroundjoin%
\pgfsetlinewidth{1.505625pt}%
\pgfsetstrokecolor{currentstroke2}%
\pgfsetdash{}{0pt}%
\pgfpathmoveto{\pgfqpoint{5.600014in}{1.859413in}}%
\pgfpathlineto{\pgfqpoint{5.725014in}{1.859413in}}%
\pgfpathlineto{\pgfqpoint{5.850014in}{1.859413in}}%
\pgfusepath{stroke}%
\end{pgfscope}%
\begin{pgfscope}%
\definecolor{textcolor}{rgb}{0.000000,0.000000,0.000000}%
\pgfsetstrokecolor{textcolor}%
\pgfsetfillcolor{textcolor}%
\pgftext[x=5.950014in,y=1.815663in,left,base]{\color{textcolor}{\rmfamily\fontsize{9.000000}{10.800000}\selectfont\catcode`\^=\active\def^{\ifmmode\sp\else\^{}\fi}\catcode`\%=\active\def%{\%}\Neighbors{} \& \MergeLinear{}}}%
\end{pgfscope}%
\begin{pgfscope}%
\pgfsetrectcap%
\pgfsetroundjoin%
\pgfsetlinewidth{1.505625pt}%
\pgfsetstrokecolor{currentstroke3}%
\pgfsetdash{}{0pt}%
\pgfpathmoveto{\pgfqpoint{5.600014in}{1.675941in}}%
\pgfpathlineto{\pgfqpoint{5.725014in}{1.675941in}}%
\pgfpathlineto{\pgfqpoint{5.850014in}{1.675941in}}%
\pgfusepath{stroke}%
\end{pgfscope}%
\begin{pgfscope}%
\definecolor{textcolor}{rgb}{0.000000,0.000000,0.000000}%
\pgfsetstrokecolor{textcolor}%
\pgfsetfillcolor{textcolor}%
\pgftext[x=5.950014in,y=1.632191in,left,base]{\color{textcolor}{\rmfamily\fontsize{9.000000}{10.800000}\selectfont\catcode`\^=\active\def^{\ifmmode\sp\else\^{}\fi}\catcode`\%=\active\def%{\%}\Neighbors{} \& \SharedVertices{}}}%
\end{pgfscope}%
\begin{pgfscope}%
\pgfsetrectcap%
\pgfsetroundjoin%
\pgfsetlinewidth{1.505625pt}%
\pgfsetstrokecolor{currentstroke4}%
\pgfsetdash{}{0pt}%
\pgfpathmoveto{\pgfqpoint{5.600014in}{1.488991in}}%
\pgfpathlineto{\pgfqpoint{5.725014in}{1.488991in}}%
\pgfpathlineto{\pgfqpoint{5.850014in}{1.488991in}}%
\pgfusepath{stroke}%
\end{pgfscope}%
\begin{pgfscope}%
\definecolor{textcolor}{rgb}{0.000000,0.000000,0.000000}%
\pgfsetstrokecolor{textcolor}%
\pgfsetfillcolor{textcolor}%
\pgftext[x=5.950014in,y=1.445241in,left,base]{\color{textcolor}{\rmfamily\fontsize{9.000000}{10.800000}\selectfont\catcode`\^=\active\def^{\ifmmode\sp\else\^{}\fi}\catcode`\%=\active\def%{\%}\NeighborsDegree{} \& \MergeLinear{}}}%
\end{pgfscope}%
\begin{pgfscope}%
\pgfsetrectcap%
\pgfsetroundjoin%
\pgfsetlinewidth{1.505625pt}%
\pgfsetstrokecolor{currentstroke5}%
\pgfsetdash{}{0pt}%
\pgfpathmoveto{\pgfqpoint{5.600014in}{1.302040in}}%
\pgfpathlineto{\pgfqpoint{5.725014in}{1.302040in}}%
\pgfpathlineto{\pgfqpoint{5.850014in}{1.302040in}}%
\pgfusepath{stroke}%
\end{pgfscope}%
\begin{pgfscope}%
\definecolor{textcolor}{rgb}{0.000000,0.000000,0.000000}%
\pgfsetstrokecolor{textcolor}%
\pgfsetfillcolor{textcolor}%
\pgftext[x=5.950014in,y=1.258290in,left,base]{\color{textcolor}{\rmfamily\fontsize{9.000000}{10.800000}\selectfont\catcode`\^=\active\def^{\ifmmode\sp\else\^{}\fi}\catcode`\%=\active\def%{\%}\NeighborsDegree{} \& \SharedVertices{}}}%
\end{pgfscope}%
\begin{pgfscope}%
\pgfsetrectcap%
\pgfsetroundjoin%
\pgfsetlinewidth{1.505625pt}%
\pgfsetstrokecolor{currentstroke6}%
\pgfsetdash{}{0pt}%
\pgfpathmoveto{\pgfqpoint{5.600014in}{1.115090in}}%
\pgfpathlineto{\pgfqpoint{5.725014in}{1.115090in}}%
\pgfpathlineto{\pgfqpoint{5.850014in}{1.115090in}}%
\pgfusepath{stroke}%
\end{pgfscope}%
\begin{pgfscope}%
\definecolor{textcolor}{rgb}{0.000000,0.000000,0.000000}%
\pgfsetstrokecolor{textcolor}%
\pgfsetfillcolor{textcolor}%
\pgftext[x=5.950014in,y=1.071340in,left,base]{\color{textcolor}{\rmfamily\fontsize{9.000000}{10.800000}\selectfont\catcode`\^=\active\def^{\ifmmode\sp\else\^{}\fi}\catcode`\%=\active\def%{\%}\None{} \& \MergeLinear{}}}%
\end{pgfscope}%
\begin{pgfscope}%
\pgfsetrectcap%
\pgfsetroundjoin%
\pgfsetlinewidth{1.505625pt}%
\pgfsetstrokecolor{currentstroke7}%
\pgfsetdash{}{0pt}%
\pgfpathmoveto{\pgfqpoint{5.600014in}{0.931619in}}%
\pgfpathlineto{\pgfqpoint{5.725014in}{0.931619in}}%
\pgfpathlineto{\pgfqpoint{5.850014in}{0.931619in}}%
\pgfusepath{stroke}%
\end{pgfscope}%
\begin{pgfscope}%
\definecolor{textcolor}{rgb}{0.000000,0.000000,0.000000}%
\pgfsetstrokecolor{textcolor}%
\pgfsetfillcolor{textcolor}%
\pgftext[x=5.950014in,y=0.887869in,left,base]{\color{textcolor}{\rmfamily\fontsize{9.000000}{10.800000}\selectfont\catcode`\^=\active\def^{\ifmmode\sp\else\^{}\fi}\catcode`\%=\active\def%{\%}\None{} \& \SharedVertices{}}}%
\end{pgfscope}%
\end{pgfpicture}%
\makeatother%
\endgroup%
}
	\caption[Checks performed or minimally rigid graphs.]{
		The number of checks performed to find all NAC-colorings for minimally rigid graphs.}%
	\label{fig:graph_count_minimally_rigid}
\end{figure}

In \Cref{fig:graph_summary}
we show the relation between the number of \IsNACColoring{} checks that
would \Naive{} algorithm perform compared to our solution.
The values are similar for graphs with few monochromatic classes,
which explains why the \NaiveCycles{} algorithm outperformed
the \NeighborsDegree{}\&\MergeLinear{} algorithm in \Cref{tab:all_min_rigid}. This should improve quickly for larger graphs.
We can also see how the use of \CycleMask{} routine
reduces the number of more expensive \IsNACColoring{} calls,
since these are called only when the small cycles check \CycleMask{} passes
(\CycleMask{} is called every time).

\begin{figure}[ht]
	\centering
	\scalebox{0.5}{%% Creator: Matplotlib, PGF backend
%%
%% To include the figure in your LaTeX document, write
%%   \input{<filename>.pgf}
%%
%% Make sure the required packages are loaded in your preamble
%%   \usepackage{pgf}
%%
%% Also ensure that all the required font packages are loaded; for instance,
%% the lmodern package is sometimes necessary when using math font.
%%   \usepackage{lmodern}
%%
%% Figures using additional raster images can only be included by \input if
%% they are in the same directory as the main LaTeX file. For loading figures
%% from other directories you can use the `import` package
%%   \usepackage{import}
%%
%% and then include the figures with
%%   \import{<path to file>}{<filename>.pgf}
%%
%% Matplotlib used the following preamble
%%   \def\mathdefault#1{#1}
%%   \everymath=\expandafter{\the\everymath\displaystyle}
%%   
%%   \ifdefined\pdftexversion\else  % non-pdftex case.
%%     \usepackage{fontspec}
%%     \setmainfont{DejaVuSans.ttf}[Path=\detokenize{/home/petr/Projects/PyRigi/.venv/lib/python3.12/site-packages/matplotlib/mpl-data/fonts/ttf/}]
%%     \setsansfont{DejaVuSans.ttf}[Path=\detokenize{/home/petr/Projects/PyRigi/.venv/lib/python3.12/site-packages/matplotlib/mpl-data/fonts/ttf/}]
%%     \setmonofont{DejaVuSansMono.ttf}[Path=\detokenize{/home/petr/Projects/PyRigi/.venv/lib/python3.12/site-packages/matplotlib/mpl-data/fonts/ttf/}]
%%   \fi
%%   \makeatletter\@ifpackageloaded{underscore}{}{\usepackage[strings]{underscore}}\makeatother
%%
\begingroup%
\makeatletter%
\begin{pgfpicture}%
\pgfpathrectangle{\pgfpointorigin}{\pgfqpoint{9.5in}{2.25in}}%
\pgfusepath{use as bounding box, clip}%
\begin{pgfscope}%
\pgfsetbuttcap%
\pgfsetmiterjoin%
\definecolor{currentfill}{rgb}{1.000000,1.000000,1.000000}%
\pgfsetfillcolor{currentfill}%
\pgfsetlinewidth{0.000000pt}%
\definecolor{currentstroke}{rgb}{1.000000,1.000000,1.000000}%
\pgfsetstrokecolor{currentstroke}%
\pgfsetdash{}{0pt}%
\pgfpathmoveto{\pgfqpoint{0.000000in}{0.000000in}}%
\pgfpathlineto{\pgfqpoint{10.058032in}{0.000000in}}%
\pgfpathlineto{\pgfqpoint{10.058032in}{2.841849in}}%
\pgfpathlineto{\pgfqpoint{0.000000in}{2.841849in}}%
\pgfpathlineto{\pgfqpoint{0.000000in}{0.000000in}}%
\pgfpathclose%
\pgfusepath{fill}%
\end{pgfscope}%
\begin{pgfscope}%
\pgfsetbuttcap%
\pgfsetmiterjoin%
\definecolor{currentfill}{rgb}{1.000000,1.000000,1.000000}%
\pgfsetfillcolor{currentfill}%
\pgfsetlinewidth{0.000000pt}%
\definecolor{currentstroke}{rgb}{0.000000,0.000000,0.000000}%
\pgfsetstrokecolor{currentstroke}%
\pgfsetstrokeopacity{0.000000}%
\pgfsetdash{}{0pt}%
\pgfpathmoveto{\pgfqpoint{1.480578in}{0.521603in}}%
\pgfpathlineto{\pgfqpoint{6.672425in}{0.521603in}}%
\pgfpathlineto{\pgfqpoint{6.672425in}{2.206605in}}%
\pgfpathlineto{\pgfqpoint{1.480578in}{2.206605in}}%
\pgfpathlineto{\pgfqpoint{1.480578in}{0.521603in}}%
\pgfpathclose%
\pgfusepath{fill}%
\end{pgfscope}%
\begin{pgfscope}%
\pgfsetbuttcap%
\pgfsetroundjoin%
\definecolor{currentfill}{rgb}{0.000000,0.000000,0.000000}%
\pgfsetfillcolor{currentfill}%
\pgfsetlinewidth{0.803000pt}%
\definecolor{currentstroke}{rgb}{0.000000,0.000000,0.000000}%
\pgfsetstrokecolor{currentstroke}%
\pgfsetdash{}{0pt}%
\pgfsys@defobject{currentmarker}{\pgfqpoint{0.000000in}{-0.048611in}}{\pgfqpoint{0.000000in}{0.000000in}}{%
\pgfpathmoveto{\pgfqpoint{0.000000in}{0.000000in}}%
\pgfpathlineto{\pgfqpoint{0.000000in}{-0.048611in}}%
\pgfusepath{stroke,fill}%
}%
\begin{pgfscope}%
\pgfsys@transformshift{1.891381in}{0.521603in}%
\pgfsys@useobject{currentmarker}{}%
\end{pgfscope}%
\end{pgfscope}%
\begin{pgfscope}%
\definecolor{textcolor}{rgb}{0.000000,0.000000,0.000000}%
\pgfsetstrokecolor{textcolor}%
\pgfsetfillcolor{textcolor}%
\pgftext[x=1.891381in,y=0.424381in,,top]{\color{textcolor}{\rmfamily\fontsize{10.000000}{12.000000}\selectfont\catcode`\^=\active\def^{\ifmmode\sp\else\^{}\fi}\catcode`\%=\active\def%{\%}$\mathdefault{3}$}}%
\end{pgfscope}%
\begin{pgfscope}%
\pgfsetbuttcap%
\pgfsetroundjoin%
\definecolor{currentfill}{rgb}{0.000000,0.000000,0.000000}%
\pgfsetfillcolor{currentfill}%
\pgfsetlinewidth{0.803000pt}%
\definecolor{currentstroke}{rgb}{0.000000,0.000000,0.000000}%
\pgfsetstrokecolor{currentstroke}%
\pgfsetdash{}{0pt}%
\pgfsys@defobject{currentmarker}{\pgfqpoint{0.000000in}{-0.048611in}}{\pgfqpoint{0.000000in}{0.000000in}}{%
\pgfpathmoveto{\pgfqpoint{0.000000in}{0.000000in}}%
\pgfpathlineto{\pgfqpoint{0.000000in}{-0.048611in}}%
\pgfusepath{stroke,fill}%
}%
\begin{pgfscope}%
\pgfsys@transformshift{2.415810in}{0.521603in}%
\pgfsys@useobject{currentmarker}{}%
\end{pgfscope}%
\end{pgfscope}%
\begin{pgfscope}%
\definecolor{textcolor}{rgb}{0.000000,0.000000,0.000000}%
\pgfsetstrokecolor{textcolor}%
\pgfsetfillcolor{textcolor}%
\pgftext[x=2.415810in,y=0.424381in,,top]{\color{textcolor}{\rmfamily\fontsize{10.000000}{12.000000}\selectfont\catcode`\^=\active\def^{\ifmmode\sp\else\^{}\fi}\catcode`\%=\active\def%{\%}$\mathdefault{6}$}}%
\end{pgfscope}%
\begin{pgfscope}%
\pgfsetbuttcap%
\pgfsetroundjoin%
\definecolor{currentfill}{rgb}{0.000000,0.000000,0.000000}%
\pgfsetfillcolor{currentfill}%
\pgfsetlinewidth{0.803000pt}%
\definecolor{currentstroke}{rgb}{0.000000,0.000000,0.000000}%
\pgfsetstrokecolor{currentstroke}%
\pgfsetdash{}{0pt}%
\pgfsys@defobject{currentmarker}{\pgfqpoint{0.000000in}{-0.048611in}}{\pgfqpoint{0.000000in}{0.000000in}}{%
\pgfpathmoveto{\pgfqpoint{0.000000in}{0.000000in}}%
\pgfpathlineto{\pgfqpoint{0.000000in}{-0.048611in}}%
\pgfusepath{stroke,fill}%
}%
\begin{pgfscope}%
\pgfsys@transformshift{2.940239in}{0.521603in}%
\pgfsys@useobject{currentmarker}{}%
\end{pgfscope}%
\end{pgfscope}%
\begin{pgfscope}%
\definecolor{textcolor}{rgb}{0.000000,0.000000,0.000000}%
\pgfsetstrokecolor{textcolor}%
\pgfsetfillcolor{textcolor}%
\pgftext[x=2.940239in,y=0.424381in,,top]{\color{textcolor}{\rmfamily\fontsize{10.000000}{12.000000}\selectfont\catcode`\^=\active\def^{\ifmmode\sp\else\^{}\fi}\catcode`\%=\active\def%{\%}$\mathdefault{9}$}}%
\end{pgfscope}%
\begin{pgfscope}%
\pgfsetbuttcap%
\pgfsetroundjoin%
\definecolor{currentfill}{rgb}{0.000000,0.000000,0.000000}%
\pgfsetfillcolor{currentfill}%
\pgfsetlinewidth{0.803000pt}%
\definecolor{currentstroke}{rgb}{0.000000,0.000000,0.000000}%
\pgfsetstrokecolor{currentstroke}%
\pgfsetdash{}{0pt}%
\pgfsys@defobject{currentmarker}{\pgfqpoint{0.000000in}{-0.048611in}}{\pgfqpoint{0.000000in}{0.000000in}}{%
\pgfpathmoveto{\pgfqpoint{0.000000in}{0.000000in}}%
\pgfpathlineto{\pgfqpoint{0.000000in}{-0.048611in}}%
\pgfusepath{stroke,fill}%
}%
\begin{pgfscope}%
\pgfsys@transformshift{3.464668in}{0.521603in}%
\pgfsys@useobject{currentmarker}{}%
\end{pgfscope}%
\end{pgfscope}%
\begin{pgfscope}%
\definecolor{textcolor}{rgb}{0.000000,0.000000,0.000000}%
\pgfsetstrokecolor{textcolor}%
\pgfsetfillcolor{textcolor}%
\pgftext[x=3.464668in,y=0.424381in,,top]{\color{textcolor}{\rmfamily\fontsize{10.000000}{12.000000}\selectfont\catcode`\^=\active\def^{\ifmmode\sp\else\^{}\fi}\catcode`\%=\active\def%{\%}$\mathdefault{12}$}}%
\end{pgfscope}%
\begin{pgfscope}%
\pgfsetbuttcap%
\pgfsetroundjoin%
\definecolor{currentfill}{rgb}{0.000000,0.000000,0.000000}%
\pgfsetfillcolor{currentfill}%
\pgfsetlinewidth{0.803000pt}%
\definecolor{currentstroke}{rgb}{0.000000,0.000000,0.000000}%
\pgfsetstrokecolor{currentstroke}%
\pgfsetdash{}{0pt}%
\pgfsys@defobject{currentmarker}{\pgfqpoint{0.000000in}{-0.048611in}}{\pgfqpoint{0.000000in}{0.000000in}}{%
\pgfpathmoveto{\pgfqpoint{0.000000in}{0.000000in}}%
\pgfpathlineto{\pgfqpoint{0.000000in}{-0.048611in}}%
\pgfusepath{stroke,fill}%
}%
\begin{pgfscope}%
\pgfsys@transformshift{3.989097in}{0.521603in}%
\pgfsys@useobject{currentmarker}{}%
\end{pgfscope}%
\end{pgfscope}%
\begin{pgfscope}%
\definecolor{textcolor}{rgb}{0.000000,0.000000,0.000000}%
\pgfsetstrokecolor{textcolor}%
\pgfsetfillcolor{textcolor}%
\pgftext[x=3.989097in,y=0.424381in,,top]{\color{textcolor}{\rmfamily\fontsize{10.000000}{12.000000}\selectfont\catcode`\^=\active\def^{\ifmmode\sp\else\^{}\fi}\catcode`\%=\active\def%{\%}$\mathdefault{15}$}}%
\end{pgfscope}%
\begin{pgfscope}%
\pgfsetbuttcap%
\pgfsetroundjoin%
\definecolor{currentfill}{rgb}{0.000000,0.000000,0.000000}%
\pgfsetfillcolor{currentfill}%
\pgfsetlinewidth{0.803000pt}%
\definecolor{currentstroke}{rgb}{0.000000,0.000000,0.000000}%
\pgfsetstrokecolor{currentstroke}%
\pgfsetdash{}{0pt}%
\pgfsys@defobject{currentmarker}{\pgfqpoint{0.000000in}{-0.048611in}}{\pgfqpoint{0.000000in}{0.000000in}}{%
\pgfpathmoveto{\pgfqpoint{0.000000in}{0.000000in}}%
\pgfpathlineto{\pgfqpoint{0.000000in}{-0.048611in}}%
\pgfusepath{stroke,fill}%
}%
\begin{pgfscope}%
\pgfsys@transformshift{4.513526in}{0.521603in}%
\pgfsys@useobject{currentmarker}{}%
\end{pgfscope}%
\end{pgfscope}%
\begin{pgfscope}%
\definecolor{textcolor}{rgb}{0.000000,0.000000,0.000000}%
\pgfsetstrokecolor{textcolor}%
\pgfsetfillcolor{textcolor}%
\pgftext[x=4.513526in,y=0.424381in,,top]{\color{textcolor}{\rmfamily\fontsize{10.000000}{12.000000}\selectfont\catcode`\^=\active\def^{\ifmmode\sp\else\^{}\fi}\catcode`\%=\active\def%{\%}$\mathdefault{18}$}}%
\end{pgfscope}%
\begin{pgfscope}%
\pgfsetbuttcap%
\pgfsetroundjoin%
\definecolor{currentfill}{rgb}{0.000000,0.000000,0.000000}%
\pgfsetfillcolor{currentfill}%
\pgfsetlinewidth{0.803000pt}%
\definecolor{currentstroke}{rgb}{0.000000,0.000000,0.000000}%
\pgfsetstrokecolor{currentstroke}%
\pgfsetdash{}{0pt}%
\pgfsys@defobject{currentmarker}{\pgfqpoint{0.000000in}{-0.048611in}}{\pgfqpoint{0.000000in}{0.000000in}}{%
\pgfpathmoveto{\pgfqpoint{0.000000in}{0.000000in}}%
\pgfpathlineto{\pgfqpoint{0.000000in}{-0.048611in}}%
\pgfusepath{stroke,fill}%
}%
\begin{pgfscope}%
\pgfsys@transformshift{5.037955in}{0.521603in}%
\pgfsys@useobject{currentmarker}{}%
\end{pgfscope}%
\end{pgfscope}%
\begin{pgfscope}%
\definecolor{textcolor}{rgb}{0.000000,0.000000,0.000000}%
\pgfsetstrokecolor{textcolor}%
\pgfsetfillcolor{textcolor}%
\pgftext[x=5.037955in,y=0.424381in,,top]{\color{textcolor}{\rmfamily\fontsize{10.000000}{12.000000}\selectfont\catcode`\^=\active\def^{\ifmmode\sp\else\^{}\fi}\catcode`\%=\active\def%{\%}$\mathdefault{21}$}}%
\end{pgfscope}%
\begin{pgfscope}%
\pgfsetbuttcap%
\pgfsetroundjoin%
\definecolor{currentfill}{rgb}{0.000000,0.000000,0.000000}%
\pgfsetfillcolor{currentfill}%
\pgfsetlinewidth{0.803000pt}%
\definecolor{currentstroke}{rgb}{0.000000,0.000000,0.000000}%
\pgfsetstrokecolor{currentstroke}%
\pgfsetdash{}{0pt}%
\pgfsys@defobject{currentmarker}{\pgfqpoint{0.000000in}{-0.048611in}}{\pgfqpoint{0.000000in}{0.000000in}}{%
\pgfpathmoveto{\pgfqpoint{0.000000in}{0.000000in}}%
\pgfpathlineto{\pgfqpoint{0.000000in}{-0.048611in}}%
\pgfusepath{stroke,fill}%
}%
\begin{pgfscope}%
\pgfsys@transformshift{5.562384in}{0.521603in}%
\pgfsys@useobject{currentmarker}{}%
\end{pgfscope}%
\end{pgfscope}%
\begin{pgfscope}%
\definecolor{textcolor}{rgb}{0.000000,0.000000,0.000000}%
\pgfsetstrokecolor{textcolor}%
\pgfsetfillcolor{textcolor}%
\pgftext[x=5.562384in,y=0.424381in,,top]{\color{textcolor}{\rmfamily\fontsize{10.000000}{12.000000}\selectfont\catcode`\^=\active\def^{\ifmmode\sp\else\^{}\fi}\catcode`\%=\active\def%{\%}$\mathdefault{24}$}}%
\end{pgfscope}%
\begin{pgfscope}%
\pgfsetbuttcap%
\pgfsetroundjoin%
\definecolor{currentfill}{rgb}{0.000000,0.000000,0.000000}%
\pgfsetfillcolor{currentfill}%
\pgfsetlinewidth{0.803000pt}%
\definecolor{currentstroke}{rgb}{0.000000,0.000000,0.000000}%
\pgfsetstrokecolor{currentstroke}%
\pgfsetdash{}{0pt}%
\pgfsys@defobject{currentmarker}{\pgfqpoint{0.000000in}{-0.048611in}}{\pgfqpoint{0.000000in}{0.000000in}}{%
\pgfpathmoveto{\pgfqpoint{0.000000in}{0.000000in}}%
\pgfpathlineto{\pgfqpoint{0.000000in}{-0.048611in}}%
\pgfusepath{stroke,fill}%
}%
\begin{pgfscope}%
\pgfsys@transformshift{6.086813in}{0.521603in}%
\pgfsys@useobject{currentmarker}{}%
\end{pgfscope}%
\end{pgfscope}%
\begin{pgfscope}%
\definecolor{textcolor}{rgb}{0.000000,0.000000,0.000000}%
\pgfsetstrokecolor{textcolor}%
\pgfsetfillcolor{textcolor}%
\pgftext[x=6.086813in,y=0.424381in,,top]{\color{textcolor}{\rmfamily\fontsize{10.000000}{12.000000}\selectfont\catcode`\^=\active\def^{\ifmmode\sp\else\^{}\fi}\catcode`\%=\active\def%{\%}$\mathdefault{27}$}}%
\end{pgfscope}%
\begin{pgfscope}%
\pgfsetbuttcap%
\pgfsetroundjoin%
\definecolor{currentfill}{rgb}{0.000000,0.000000,0.000000}%
\pgfsetfillcolor{currentfill}%
\pgfsetlinewidth{0.803000pt}%
\definecolor{currentstroke}{rgb}{0.000000,0.000000,0.000000}%
\pgfsetstrokecolor{currentstroke}%
\pgfsetdash{}{0pt}%
\pgfsys@defobject{currentmarker}{\pgfqpoint{0.000000in}{-0.048611in}}{\pgfqpoint{0.000000in}{0.000000in}}{%
\pgfpathmoveto{\pgfqpoint{0.000000in}{0.000000in}}%
\pgfpathlineto{\pgfqpoint{0.000000in}{-0.048611in}}%
\pgfusepath{stroke,fill}%
}%
\begin{pgfscope}%
\pgfsys@transformshift{6.611242in}{0.521603in}%
\pgfsys@useobject{currentmarker}{}%
\end{pgfscope}%
\end{pgfscope}%
\begin{pgfscope}%
\definecolor{textcolor}{rgb}{0.000000,0.000000,0.000000}%
\pgfsetstrokecolor{textcolor}%
\pgfsetfillcolor{textcolor}%
\pgftext[x=6.611242in,y=0.424381in,,top]{\color{textcolor}{\rmfamily\fontsize{10.000000}{12.000000}\selectfont\catcode`\^=\active\def^{\ifmmode\sp\else\^{}\fi}\catcode`\%=\active\def%{\%}$\mathdefault{30}$}}%
\end{pgfscope}%
\begin{pgfscope}%
\definecolor{textcolor}{rgb}{0.000000,0.000000,0.000000}%
\pgfsetstrokecolor{textcolor}%
\pgfsetfillcolor{textcolor}%
\pgftext[x=4.076502in,y=0.234413in,,top]{\color{textcolor}{\rmfamily\fontsize{10.000000}{12.000000}\selectfont\catcode`\^=\active\def^{\ifmmode\sp\else\^{}\fi}\catcode`\%=\active\def%{\%}Monochromatic classes}}%
\end{pgfscope}%
\begin{pgfscope}%
\pgfsetbuttcap%
\pgfsetroundjoin%
\definecolor{currentfill}{rgb}{0.000000,0.000000,0.000000}%
\pgfsetfillcolor{currentfill}%
\pgfsetlinewidth{0.803000pt}%
\definecolor{currentstroke}{rgb}{0.000000,0.000000,0.000000}%
\pgfsetstrokecolor{currentstroke}%
\pgfsetdash{}{0pt}%
\pgfsys@defobject{currentmarker}{\pgfqpoint{-0.048611in}{0.000000in}}{\pgfqpoint{-0.000000in}{0.000000in}}{%
\pgfpathmoveto{\pgfqpoint{-0.000000in}{0.000000in}}%
\pgfpathlineto{\pgfqpoint{-0.048611in}{0.000000in}}%
\pgfusepath{stroke,fill}%
}%
\begin{pgfscope}%
\pgfsys@transformshift{1.480578in}{0.810693in}%
\pgfsys@useobject{currentmarker}{}%
\end{pgfscope}%
\end{pgfscope}%
\begin{pgfscope}%
\definecolor{textcolor}{rgb}{0.000000,0.000000,0.000000}%
\pgfsetstrokecolor{textcolor}%
\pgfsetfillcolor{textcolor}%
\pgftext[x=1.182159in, y=0.757932in, left, base]{\color{textcolor}{\rmfamily\fontsize{10.000000}{12.000000}\selectfont\catcode`\^=\active\def^{\ifmmode\sp\else\^{}\fi}\catcode`\%=\active\def%{\%}$\mathdefault{10^{2}}$}}%
\end{pgfscope}%
\begin{pgfscope}%
\pgfsetbuttcap%
\pgfsetroundjoin%
\definecolor{currentfill}{rgb}{0.000000,0.000000,0.000000}%
\pgfsetfillcolor{currentfill}%
\pgfsetlinewidth{0.803000pt}%
\definecolor{currentstroke}{rgb}{0.000000,0.000000,0.000000}%
\pgfsetstrokecolor{currentstroke}%
\pgfsetdash{}{0pt}%
\pgfsys@defobject{currentmarker}{\pgfqpoint{-0.048611in}{0.000000in}}{\pgfqpoint{-0.000000in}{0.000000in}}{%
\pgfpathmoveto{\pgfqpoint{-0.000000in}{0.000000in}}%
\pgfpathlineto{\pgfqpoint{-0.048611in}{0.000000in}}%
\pgfusepath{stroke,fill}%
}%
\begin{pgfscope}%
\pgfsys@transformshift{1.480578in}{1.185919in}%
\pgfsys@useobject{currentmarker}{}%
\end{pgfscope}%
\end{pgfscope}%
\begin{pgfscope}%
\definecolor{textcolor}{rgb}{0.000000,0.000000,0.000000}%
\pgfsetstrokecolor{textcolor}%
\pgfsetfillcolor{textcolor}%
\pgftext[x=1.182159in, y=1.133158in, left, base]{\color{textcolor}{\rmfamily\fontsize{10.000000}{12.000000}\selectfont\catcode`\^=\active\def^{\ifmmode\sp\else\^{}\fi}\catcode`\%=\active\def%{\%}$\mathdefault{10^{5}}$}}%
\end{pgfscope}%
\begin{pgfscope}%
\pgfsetbuttcap%
\pgfsetroundjoin%
\definecolor{currentfill}{rgb}{0.000000,0.000000,0.000000}%
\pgfsetfillcolor{currentfill}%
\pgfsetlinewidth{0.803000pt}%
\definecolor{currentstroke}{rgb}{0.000000,0.000000,0.000000}%
\pgfsetstrokecolor{currentstroke}%
\pgfsetdash{}{0pt}%
\pgfsys@defobject{currentmarker}{\pgfqpoint{-0.048611in}{0.000000in}}{\pgfqpoint{-0.000000in}{0.000000in}}{%
\pgfpathmoveto{\pgfqpoint{-0.000000in}{0.000000in}}%
\pgfpathlineto{\pgfqpoint{-0.048611in}{0.000000in}}%
\pgfusepath{stroke,fill}%
}%
\begin{pgfscope}%
\pgfsys@transformshift{1.480578in}{1.561145in}%
\pgfsys@useobject{currentmarker}{}%
\end{pgfscope}%
\end{pgfscope}%
\begin{pgfscope}%
\definecolor{textcolor}{rgb}{0.000000,0.000000,0.000000}%
\pgfsetstrokecolor{textcolor}%
\pgfsetfillcolor{textcolor}%
\pgftext[x=1.182159in, y=1.508383in, left, base]{\color{textcolor}{\rmfamily\fontsize{10.000000}{12.000000}\selectfont\catcode`\^=\active\def^{\ifmmode\sp\else\^{}\fi}\catcode`\%=\active\def%{\%}$\mathdefault{10^{8}}$}}%
\end{pgfscope}%
\begin{pgfscope}%
\pgfsetbuttcap%
\pgfsetroundjoin%
\definecolor{currentfill}{rgb}{0.000000,0.000000,0.000000}%
\pgfsetfillcolor{currentfill}%
\pgfsetlinewidth{0.803000pt}%
\definecolor{currentstroke}{rgb}{0.000000,0.000000,0.000000}%
\pgfsetstrokecolor{currentstroke}%
\pgfsetdash{}{0pt}%
\pgfsys@defobject{currentmarker}{\pgfqpoint{-0.048611in}{0.000000in}}{\pgfqpoint{-0.000000in}{0.000000in}}{%
\pgfpathmoveto{\pgfqpoint{-0.000000in}{0.000000in}}%
\pgfpathlineto{\pgfqpoint{-0.048611in}{0.000000in}}%
\pgfusepath{stroke,fill}%
}%
\begin{pgfscope}%
\pgfsys@transformshift{1.480578in}{1.936371in}%
\pgfsys@useobject{currentmarker}{}%
\end{pgfscope}%
\end{pgfscope}%
\begin{pgfscope}%
\definecolor{textcolor}{rgb}{0.000000,0.000000,0.000000}%
\pgfsetstrokecolor{textcolor}%
\pgfsetfillcolor{textcolor}%
\pgftext[x=1.126796in, y=1.883609in, left, base]{\color{textcolor}{\rmfamily\fontsize{10.000000}{12.000000}\selectfont\catcode`\^=\active\def^{\ifmmode\sp\else\^{}\fi}\catcode`\%=\active\def%{\%}$\mathdefault{10^{11}}$}}%
\end{pgfscope}%
\begin{pgfscope}%
\definecolor{textcolor}{rgb}{0.000000,0.000000,0.000000}%
\pgfsetstrokecolor{textcolor}%
\pgfsetfillcolor{textcolor}%
\pgftext[x=1.071241in,y=1.364104in,,bottom,rotate=90.000000]{\color{textcolor}{\rmfamily\fontsize{10.000000}{12.000000}\selectfont\catcode`\^=\active\def^{\ifmmode\sp\else\^{}\fi}\catcode`\%=\active\def%{\%}Checks [call]}}%
\end{pgfscope}%
\begin{pgfscope}%
\pgfpathrectangle{\pgfqpoint{1.480578in}{0.521603in}}{\pgfqpoint{5.191847in}{1.685002in}}%
\pgfusepath{clip}%
\pgfsetrectcap%
\pgfsetroundjoin%
\pgfsetlinewidth{1.505625pt}%
\pgfsetstrokecolor{currentstroke1}%
\pgfsetdash{}{0pt}%
\pgfpathmoveto{\pgfqpoint{1.716571in}{2.029537in}}%
\pgfpathlineto{\pgfqpoint{1.891381in}{1.938322in}}%
\pgfpathlineto{\pgfqpoint{2.066190in}{2.130014in}}%
\pgfpathlineto{\pgfqpoint{2.241000in}{1.846748in}}%
\pgfpathlineto{\pgfqpoint{2.415810in}{1.851206in}}%
\pgfpathlineto{\pgfqpoint{2.590619in}{1.765869in}}%
\pgfpathlineto{\pgfqpoint{2.765429in}{1.763754in}}%
\pgfpathlineto{\pgfqpoint{2.940239in}{1.780400in}}%
\pgfpathlineto{\pgfqpoint{3.115048in}{1.774751in}}%
\pgfpathlineto{\pgfqpoint{3.289858in}{1.677190in}}%
\pgfpathlineto{\pgfqpoint{3.464668in}{1.791708in}}%
\pgfpathlineto{\pgfqpoint{3.639477in}{1.711752in}}%
\pgfpathlineto{\pgfqpoint{3.814287in}{1.641991in}}%
\pgfpathlineto{\pgfqpoint{3.989097in}{1.661887in}}%
\pgfpathlineto{\pgfqpoint{4.163906in}{1.714265in}}%
\pgfpathlineto{\pgfqpoint{4.338716in}{1.699637in}}%
\pgfpathlineto{\pgfqpoint{4.513526in}{1.684472in}}%
\pgfpathlineto{\pgfqpoint{4.688335in}{1.682131in}}%
\pgfpathlineto{\pgfqpoint{4.863145in}{1.723805in}}%
\pgfpathlineto{\pgfqpoint{5.037955in}{1.705323in}}%
\pgfpathlineto{\pgfqpoint{5.212764in}{1.747652in}}%
\pgfpathlineto{\pgfqpoint{5.387574in}{1.711334in}}%
\pgfpathlineto{\pgfqpoint{5.562384in}{1.746975in}}%
\pgfpathlineto{\pgfqpoint{5.737193in}{1.722329in}}%
\pgfpathlineto{\pgfqpoint{5.912003in}{1.765388in}}%
\pgfpathlineto{\pgfqpoint{6.086813in}{1.720483in}}%
\pgfpathlineto{\pgfqpoint{6.261622in}{1.765388in}}%
\pgfpathlineto{\pgfqpoint{6.436432in}{1.745205in}}%
\pgfusepath{stroke}%
\end{pgfscope}%
\begin{pgfscope}%
\pgfpathrectangle{\pgfqpoint{1.480578in}{0.521603in}}{\pgfqpoint{5.191847in}{1.685002in}}%
\pgfusepath{clip}%
\pgfsetrectcap%
\pgfsetroundjoin%
\pgfsetlinewidth{1.505625pt}%
\pgfsetstrokecolor{currentstroke2}%
\pgfsetdash{}{0pt}%
\pgfpathmoveto{\pgfqpoint{1.716571in}{0.859214in}}%
\pgfpathlineto{\pgfqpoint{1.891381in}{0.992169in}}%
\pgfpathlineto{\pgfqpoint{2.066190in}{0.930472in}}%
\pgfpathlineto{\pgfqpoint{2.241000in}{0.956959in}}%
\pgfpathlineto{\pgfqpoint{2.415810in}{0.980984in}}%
\pgfpathlineto{\pgfqpoint{2.590619in}{0.971066in}}%
\pgfpathlineto{\pgfqpoint{2.765429in}{0.964785in}}%
\pgfpathlineto{\pgfqpoint{2.940239in}{1.164808in}}%
\pgfpathlineto{\pgfqpoint{3.115048in}{1.045350in}}%
\pgfpathlineto{\pgfqpoint{3.289858in}{1.030231in}}%
\pgfpathlineto{\pgfqpoint{3.464668in}{1.070740in}}%
\pgfpathlineto{\pgfqpoint{3.639477in}{1.087032in}}%
\pgfpathlineto{\pgfqpoint{3.814287in}{1.122649in}}%
\pgfpathlineto{\pgfqpoint{3.989097in}{1.157273in}}%
\pgfpathlineto{\pgfqpoint{4.163906in}{1.190974in}}%
\pgfpathlineto{\pgfqpoint{4.338716in}{1.276308in}}%
\pgfpathlineto{\pgfqpoint{4.513526in}{1.252789in}}%
\pgfpathlineto{\pgfqpoint{4.688335in}{1.264760in}}%
\pgfpathlineto{\pgfqpoint{4.863145in}{1.322538in}}%
\pgfpathlineto{\pgfqpoint{5.037955in}{1.338752in}}%
\pgfpathlineto{\pgfqpoint{5.212764in}{1.392167in}}%
\pgfpathlineto{\pgfqpoint{5.387574in}{1.395400in}}%
\pgfpathlineto{\pgfqpoint{5.562384in}{1.458453in}}%
\pgfpathlineto{\pgfqpoint{5.737193in}{1.464177in}}%
\pgfpathlineto{\pgfqpoint{5.912003in}{1.539479in}}%
\pgfpathlineto{\pgfqpoint{6.086813in}{1.539479in}}%
\pgfpathlineto{\pgfqpoint{6.261622in}{1.614782in}}%
\pgfpathlineto{\pgfqpoint{6.436432in}{1.614782in}}%
\pgfusepath{stroke}%
\end{pgfscope}%
\begin{pgfscope}%
\pgfpathrectangle{\pgfqpoint{1.480578in}{0.521603in}}{\pgfqpoint{5.191847in}{1.685002in}}%
\pgfusepath{clip}%
\pgfsetrectcap%
\pgfsetroundjoin%
\pgfsetlinewidth{1.505625pt}%
\pgfsetstrokecolor{currentstroke3}%
\pgfsetdash{}{0pt}%
\pgfpathmoveto{\pgfqpoint{1.716571in}{0.598194in}}%
\pgfpathlineto{\pgfqpoint{1.891381in}{0.635846in}}%
\pgfpathlineto{\pgfqpoint{2.066190in}{0.673497in}}%
\pgfpathlineto{\pgfqpoint{2.241000in}{0.711149in}}%
\pgfpathlineto{\pgfqpoint{2.415810in}{0.748800in}}%
\pgfpathlineto{\pgfqpoint{2.590619in}{0.786451in}}%
\pgfpathlineto{\pgfqpoint{2.765429in}{0.824103in}}%
\pgfpathlineto{\pgfqpoint{2.940239in}{0.861754in}}%
\pgfpathlineto{\pgfqpoint{3.115048in}{0.899406in}}%
\pgfpathlineto{\pgfqpoint{3.289858in}{0.937057in}}%
\pgfpathlineto{\pgfqpoint{3.464668in}{0.974708in}}%
\pgfpathlineto{\pgfqpoint{3.639477in}{1.012360in}}%
\pgfpathlineto{\pgfqpoint{3.814287in}{1.050011in}}%
\pgfpathlineto{\pgfqpoint{3.989097in}{1.087663in}}%
\pgfpathlineto{\pgfqpoint{4.163906in}{1.125314in}}%
\pgfpathlineto{\pgfqpoint{4.338716in}{1.162965in}}%
\pgfpathlineto{\pgfqpoint{4.513526in}{1.200617in}}%
\pgfpathlineto{\pgfqpoint{4.688335in}{1.238268in}}%
\pgfpathlineto{\pgfqpoint{4.863145in}{1.275920in}}%
\pgfpathlineto{\pgfqpoint{5.037955in}{1.313571in}}%
\pgfpathlineto{\pgfqpoint{5.212764in}{1.351222in}}%
\pgfpathlineto{\pgfqpoint{5.387574in}{1.388874in}}%
\pgfpathlineto{\pgfqpoint{5.562384in}{1.426525in}}%
\pgfpathlineto{\pgfqpoint{5.737193in}{1.464177in}}%
\pgfpathlineto{\pgfqpoint{5.912003in}{1.501828in}}%
\pgfpathlineto{\pgfqpoint{6.086813in}{1.539479in}}%
\pgfpathlineto{\pgfqpoint{6.261622in}{1.577131in}}%
\pgfpathlineto{\pgfqpoint{6.436432in}{1.614782in}}%
\pgfusepath{stroke}%
\end{pgfscope}%
\begin{pgfscope}%
\pgfpathrectangle{\pgfqpoint{1.480578in}{0.521603in}}{\pgfqpoint{5.191847in}{1.685002in}}%
\pgfusepath{clip}%
\pgfsetrectcap%
\pgfsetroundjoin%
\pgfsetlinewidth{1.505625pt}%
\pgfsetstrokecolor{currentstroke4}%
\pgfsetdash{}{0pt}%
\pgfpathmoveto{\pgfqpoint{1.716571in}{0.598194in}}%
\pgfpathlineto{\pgfqpoint{1.891381in}{0.631917in}}%
\pgfpathlineto{\pgfqpoint{2.066190in}{0.669336in}}%
\pgfpathlineto{\pgfqpoint{2.241000in}{0.704572in}}%
\pgfpathlineto{\pgfqpoint{2.415810in}{0.740593in}}%
\pgfpathlineto{\pgfqpoint{2.590619in}{0.776585in}}%
\pgfpathlineto{\pgfqpoint{2.765429in}{0.812932in}}%
\pgfpathlineto{\pgfqpoint{2.940239in}{0.848787in}}%
\pgfpathlineto{\pgfqpoint{3.115048in}{0.891268in}}%
\pgfpathlineto{\pgfqpoint{3.289858in}{0.926684in}}%
\pgfpathlineto{\pgfqpoint{3.464668in}{0.932090in}}%
\pgfpathlineto{\pgfqpoint{3.639477in}{0.955904in}}%
\pgfpathlineto{\pgfqpoint{3.814287in}{0.992001in}}%
\pgfpathlineto{\pgfqpoint{3.989097in}{1.016140in}}%
\pgfpathlineto{\pgfqpoint{4.163906in}{1.050228in}}%
\pgfpathlineto{\pgfqpoint{4.338716in}{1.074000in}}%
\pgfpathlineto{\pgfqpoint{4.513526in}{1.083525in}}%
\pgfpathlineto{\pgfqpoint{4.688335in}{1.099106in}}%
\pgfpathlineto{\pgfqpoint{4.863145in}{1.139852in}}%
\pgfpathlineto{\pgfqpoint{5.037955in}{1.160227in}}%
\pgfpathlineto{\pgfqpoint{5.212764in}{1.193691in}}%
\pgfpathlineto{\pgfqpoint{5.387574in}{1.203703in}}%
\pgfpathlineto{\pgfqpoint{5.562384in}{1.204093in}}%
\pgfpathlineto{\pgfqpoint{5.737193in}{1.229038in}}%
\pgfpathlineto{\pgfqpoint{5.912003in}{1.272068in}}%
\pgfpathlineto{\pgfqpoint{6.086813in}{1.245507in}}%
\pgfpathlineto{\pgfqpoint{6.261622in}{1.288761in}}%
\pgfpathlineto{\pgfqpoint{6.436432in}{1.319666in}}%
\pgfusepath{stroke}%
\end{pgfscope}%
\begin{pgfscope}%
\pgfpathrectangle{\pgfqpoint{1.480578in}{0.521603in}}{\pgfqpoint{5.191847in}{1.685002in}}%
\pgfusepath{clip}%
\pgfsetrectcap%
\pgfsetroundjoin%
\pgfsetlinewidth{1.505625pt}%
\pgfsetstrokecolor{currentstroke5}%
\pgfsetdash{}{0pt}%
\pgfpathmoveto{\pgfqpoint{1.716571in}{0.598194in}}%
\pgfpathlineto{\pgfqpoint{1.891381in}{0.626695in}}%
\pgfpathlineto{\pgfqpoint{2.066190in}{0.640661in}}%
\pgfpathlineto{\pgfqpoint{2.241000in}{0.659168in}}%
\pgfpathlineto{\pgfqpoint{2.415810in}{0.675488in}}%
\pgfpathlineto{\pgfqpoint{2.590619in}{0.694135in}}%
\pgfpathlineto{\pgfqpoint{2.765429in}{0.714167in}}%
\pgfpathlineto{\pgfqpoint{2.940239in}{0.735077in}}%
\pgfpathlineto{\pgfqpoint{3.115048in}{0.752655in}}%
\pgfpathlineto{\pgfqpoint{3.289858in}{0.770721in}}%
\pgfpathlineto{\pgfqpoint{3.464668in}{0.812847in}}%
\pgfpathlineto{\pgfqpoint{3.639477in}{0.831022in}}%
\pgfpathlineto{\pgfqpoint{3.814287in}{0.854404in}}%
\pgfpathlineto{\pgfqpoint{3.989097in}{0.870574in}}%
\pgfpathlineto{\pgfqpoint{4.163906in}{0.896795in}}%
\pgfpathlineto{\pgfqpoint{4.338716in}{0.901357in}}%
\pgfpathlineto{\pgfqpoint{4.513526in}{0.943205in}}%
\pgfpathlineto{\pgfqpoint{4.688335in}{0.941148in}}%
\pgfpathlineto{\pgfqpoint{4.863145in}{0.982782in}}%
\pgfpathlineto{\pgfqpoint{5.037955in}{0.980805in}}%
\pgfpathlineto{\pgfqpoint{5.212764in}{1.020864in}}%
\pgfpathlineto{\pgfqpoint{5.387574in}{1.010697in}}%
\pgfpathlineto{\pgfqpoint{5.562384in}{1.053068in}}%
\pgfpathlineto{\pgfqpoint{5.737193in}{1.060496in}}%
\pgfpathlineto{\pgfqpoint{5.912003in}{1.096386in}}%
\pgfpathlineto{\pgfqpoint{6.086813in}{1.072780in}}%
\pgfpathlineto{\pgfqpoint{6.261622in}{1.110858in}}%
\pgfpathlineto{\pgfqpoint{6.436432in}{1.123404in}}%
\pgfusepath{stroke}%
\end{pgfscope}%
\begin{pgfscope}%
\pgfsetrectcap%
\pgfsetmiterjoin%
\pgfsetlinewidth{0.803000pt}%
\definecolor{currentstroke}{rgb}{0.000000,0.000000,0.000000}%
\pgfsetstrokecolor{currentstroke}%
\pgfsetdash{}{0pt}%
\pgfpathmoveto{\pgfqpoint{1.480578in}{0.521603in}}%
\pgfpathlineto{\pgfqpoint{1.480578in}{2.206605in}}%
\pgfusepath{stroke}%
\end{pgfscope}%
\begin{pgfscope}%
\pgfsetrectcap%
\pgfsetmiterjoin%
\pgfsetlinewidth{0.803000pt}%
\definecolor{currentstroke}{rgb}{0.000000,0.000000,0.000000}%
\pgfsetstrokecolor{currentstroke}%
\pgfsetdash{}{0pt}%
\pgfpathmoveto{\pgfqpoint{6.672425in}{0.521603in}}%
\pgfpathlineto{\pgfqpoint{6.672425in}{2.206605in}}%
\pgfusepath{stroke}%
\end{pgfscope}%
\begin{pgfscope}%
\pgfsetrectcap%
\pgfsetmiterjoin%
\pgfsetlinewidth{0.803000pt}%
\definecolor{currentstroke}{rgb}{0.000000,0.000000,0.000000}%
\pgfsetstrokecolor{currentstroke}%
\pgfsetdash{}{0pt}%
\pgfpathmoveto{\pgfqpoint{1.480578in}{0.521603in}}%
\pgfpathlineto{\pgfqpoint{6.672425in}{0.521603in}}%
\pgfusepath{stroke}%
\end{pgfscope}%
\begin{pgfscope}%
\pgfsetrectcap%
\pgfsetmiterjoin%
\pgfsetlinewidth{0.803000pt}%
\definecolor{currentstroke}{rgb}{0.000000,0.000000,0.000000}%
\pgfsetstrokecolor{currentstroke}%
\pgfsetdash{}{0pt}%
\pgfpathmoveto{\pgfqpoint{1.480578in}{2.206605in}}%
\pgfpathlineto{\pgfqpoint{6.672425in}{2.206605in}}%
\pgfusepath{stroke}%
\end{pgfscope}%
\begin{pgfscope}%
\pgfsetbuttcap%
\pgfsetmiterjoin%
\definecolor{currentfill}{rgb}{1.000000,1.000000,1.000000}%
\pgfsetfillcolor{currentfill}%
\pgfsetfillopacity{0.800000}%
\pgfsetlinewidth{1.003750pt}%
\definecolor{currentstroke}{rgb}{0.800000,0.800000,0.800000}%
\pgfsetstrokecolor{currentstroke}%
\pgfsetstrokeopacity{0.800000}%
\pgfsetdash{}{0pt}%
\pgfpathmoveto{\pgfqpoint{6.759925in}{1.189247in}}%
\pgfpathlineto{\pgfqpoint{9.096204in}{1.189247in}}%
\pgfpathquadraticcurveto{\pgfqpoint{9.121204in}{1.189247in}}{\pgfqpoint{9.121204in}{1.214247in}}%
\pgfpathlineto{\pgfqpoint{9.121204in}{2.119105in}}%
\pgfpathquadraticcurveto{\pgfqpoint{9.121204in}{2.144105in}}{\pgfqpoint{9.096204in}{2.144105in}}%
\pgfpathlineto{\pgfqpoint{6.759925in}{2.144105in}}%
\pgfpathquadraticcurveto{\pgfqpoint{6.734925in}{2.144105in}}{\pgfqpoint{6.734925in}{2.119105in}}%
\pgfpathlineto{\pgfqpoint{6.734925in}{1.214247in}}%
\pgfpathquadraticcurveto{\pgfqpoint{6.734925in}{1.189247in}}{\pgfqpoint{6.759925in}{1.189247in}}%
\pgfpathlineto{\pgfqpoint{6.759925in}{1.189247in}}%
\pgfpathclose%
\pgfusepath{stroke,fill}%
\end{pgfscope}%
\begin{pgfscope}%
\pgfsetrectcap%
\pgfsetroundjoin%
\pgfsetlinewidth{1.505625pt}%
\pgfsetstrokecolor{currentstroke1}%
\pgfsetdash{}{0pt}%
\pgfpathmoveto{\pgfqpoint{6.784925in}{2.042884in}}%
\pgfpathlineto{\pgfqpoint{6.909925in}{2.042884in}}%
\pgfpathlineto{\pgfqpoint{7.034925in}{2.042884in}}%
\pgfusepath{stroke}%
\end{pgfscope}%
\begin{pgfscope}%
\definecolor{textcolor}{rgb}{0.000000,0.000000,0.000000}%
\pgfsetstrokecolor{textcolor}%
\pgfsetfillcolor{textcolor}%
\pgftext[x=7.134925in,y=1.999134in,left,base]{\color{textcolor}{\rmfamily\fontsize{9.000000}{10.800000}\selectfont\catcode`\^=\active\def^{\ifmmode\sp\else\^{}\fi}\catcode`\%=\active\def%{\%}Naive - Edges}}%
\end{pgfscope}%
\begin{pgfscope}%
\pgfsetrectcap%
\pgfsetroundjoin%
\pgfsetlinewidth{1.505625pt}%
\pgfsetstrokecolor{currentstroke2}%
\pgfsetdash{}{0pt}%
\pgfpathmoveto{\pgfqpoint{6.784925in}{1.859413in}}%
\pgfpathlineto{\pgfqpoint{6.909925in}{1.859413in}}%
\pgfpathlineto{\pgfqpoint{7.034925in}{1.859413in}}%
\pgfusepath{stroke}%
\end{pgfscope}%
\begin{pgfscope}%
\definecolor{textcolor}{rgb}{0.000000,0.000000,0.000000}%
\pgfsetstrokecolor{textcolor}%
\pgfsetfillcolor{textcolor}%
\pgftext[x=7.134925in,y=1.815663in,left,base]{\color{textcolor}{\rmfamily\fontsize{9.000000}{10.800000}\selectfont\catcode`\^=\active\def^{\ifmmode\sp\else\^{}\fi}\catcode`\%=\active\def%{\%}Naive - Triangle-components}}%
\end{pgfscope}%
\begin{pgfscope}%
\pgfsetrectcap%
\pgfsetroundjoin%
\pgfsetlinewidth{1.505625pt}%
\pgfsetstrokecolor{currentstroke3}%
\pgfsetdash{}{0pt}%
\pgfpathmoveto{\pgfqpoint{6.784925in}{1.675941in}}%
\pgfpathlineto{\pgfqpoint{6.909925in}{1.675941in}}%
\pgfpathlineto{\pgfqpoint{7.034925in}{1.675941in}}%
\pgfusepath{stroke}%
\end{pgfscope}%
\begin{pgfscope}%
\definecolor{textcolor}{rgb}{0.000000,0.000000,0.000000}%
\pgfsetstrokecolor{textcolor}%
\pgfsetfillcolor{textcolor}%
\pgftext[x=7.134925in,y=1.632191in,left,base]{\color{textcolor}{\rmfamily\fontsize{9.000000}{10.800000}\selectfont\catcode`\^=\active\def^{\ifmmode\sp\else\^{}\fi}\catcode`\%=\active\def%{\%}Naive - Monochromatic classes}}%
\end{pgfscope}%
\begin{pgfscope}%
\pgfsetrectcap%
\pgfsetroundjoin%
\pgfsetlinewidth{1.505625pt}%
\pgfsetstrokecolor{currentstroke4}%
\pgfsetdash{}{0pt}%
\pgfpathmoveto{\pgfqpoint{6.784925in}{1.492470in}}%
\pgfpathlineto{\pgfqpoint{6.909925in}{1.492470in}}%
\pgfpathlineto{\pgfqpoint{7.034925in}{1.492470in}}%
\pgfusepath{stroke}%
\end{pgfscope}%
\begin{pgfscope}%
\definecolor{textcolor}{rgb}{0.000000,0.000000,0.000000}%
\pgfsetstrokecolor{textcolor}%
\pgfsetfillcolor{textcolor}%
\pgftext[x=7.134925in,y=1.448720in,left,base]{\color{textcolor}{\rmfamily\fontsize{9.000000}{10.800000}\selectfont\catcode`\^=\active\def^{\ifmmode\sp\else\^{}\fi}\catcode`\%=\active\def%{\%}Subgraphs - \CycleMask{}}}%
\end{pgfscope}%
\begin{pgfscope}%
\pgfsetrectcap%
\pgfsetroundjoin%
\pgfsetlinewidth{1.505625pt}%
\pgfsetstrokecolor{currentstroke5}%
\pgfsetdash{}{0pt}%
\pgfpathmoveto{\pgfqpoint{6.784925in}{1.308998in}}%
\pgfpathlineto{\pgfqpoint{6.909925in}{1.308998in}}%
\pgfpathlineto{\pgfqpoint{7.034925in}{1.308998in}}%
\pgfusepath{stroke}%
\end{pgfscope}%
\begin{pgfscope}%
\definecolor{textcolor}{rgb}{0.000000,0.000000,0.000000}%
\pgfsetstrokecolor{textcolor}%
\pgfsetfillcolor{textcolor}%
\pgftext[x=7.134925in,y=1.265248in,left,base]{\color{textcolor}{\rmfamily\fontsize{9.000000}{10.800000}\selectfont\catcode`\^=\active\def^{\ifmmode\sp\else\^{}\fi}\catcode`\%=\active\def%{\%}Subgraphs - \IsNACColoring{}}}%
\end{pgfscope}%
\end{pgfpicture}%
\makeatother%
\endgroup%
}
	\caption[The number of \IsNACColoring{} calls.]{
		The number of \IsNACColoring{} calls with respect to the number of monochromatic classes
		over all graphs used for benchmarking.}%
	\label{fig:graph_summary}
\end{figure}






\subsection{Performance on graph classes}%

\todo[inline]{Consider Laman deg 3+}
\todo[inline]{Consider line graphs of 3 nor 4 cycles}

Each benchmark was run two or three times and the mean was taken.
The graphs are grouped either by the number of vertices
monochromatic classes or \trcon{} components, see respective \(x\)-axis.
Overall, over 430k configurations were run
on over 28k graphs from multiple graph classes.
First, we only show strategies that performed well,
we show the others later in \Cref{sec:failing_strategies}.

In the previous section we showed performance of the algorithm for listing
all NAC-colorings of minimally rigid graphs.
Now we focus of finding some NAC-coloring of a minimally rigid graph.
Minimally rigid graphs as mostly flexible graphs have also
large number of NAC-colorings, therefore it is simple for both \NaiveCycles{}
and \Subgraphs{} algorithms to find some NAC-coloring.
It can be seen from the graphs, that for larger graphs, the required runtime
does not grow significantly.
Note that minimally rigid graphs have no NAC-coloring if and only if they are formed from
a single monochromatic class, therefore such instances do not worsen runtime performance.
\NaiveCycles{} is faster as it has lower internal overhead.
The number of \IsNACColoring{} checks is also lower,
that is probably because \Subgraphs{} strategies do additional checks
while merging, which are not needed for \NaiveCycles{}.

\todo[inline]{Graphs both}


From~\cite{extremal_graphs} we obtained all graphs with up to 52 vertices
that have no three nor four cycles. This class of graphs is interesting for us
as \trcon{} components and monochromatic classes optimizations cannot be used.
We also see that these graphs have numerous NAC-colorings.
Therefore, \NaiveCycles{} are again faster for finding some NAC-coloring
for the similar reasons as for minimally rigid graphs.
For listing all NAC-colorings, \Subgraphs{} are once again significantly faster.
It can be seen that \Neighbors{} and \CyclesMatchChunks{} strategies are not faster than \None{},
but the number of \IsNACColoring{} check calls is reduced.
The difference grows for larger graphs, therefore it can be expected
that \Neighbors{} and \CyclesMatchChunks{} will eventually become faster
for larger graphs then \None{}.

\todo[inline]{Graphs both}


\todo[inline]{Cite formula used to generate globally rigid graphs}
We also randomly generated a dataset of globally rigid graphs
up to 57 vertices.
The idea of monochromatic classes is so effective
that even large graphs collapse into just few monochromatic classes.
Most of the graphs in this dataset have a NAC-coloring
and the graph that do not often have only a single monochromatic class.
The statements from other graph classes in this section hold
--- \NaiveCycles{} is faster unless we list all NAC-colorings.
It can be also seen that the number of checks performed
by \NaiveCycles{} is not consisted while it is for \Subgraphs{}.

\todo[inline]{Graphs both + all}

\begin{figure}[ht]
	\centering
	\scalebox{0.5}{%% Creator: Matplotlib, PGF backend
%%
%% To include the figure in your LaTeX document, write
%%   \input{<filename>.pgf}
%%
%% Make sure the required packages are loaded in your preamble
%%   \usepackage{pgf}
%%
%% Also ensure that all the required font packages are loaded; for instance,
%% the lmodern package is sometimes necessary when using math font.
%%   \usepackage{lmodern}
%%
%% Figures using additional raster images can only be included by \input if
%% they are in the same directory as the main LaTeX file. For loading figures
%% from other directories you can use the `import` package
%%   \usepackage{import}
%%
%% and then include the figures with
%%   \import{<path to file>}{<filename>.pgf}
%%
%% Matplotlib used the following preamble
%%   \def\mathdefault#1{#1}
%%   \everymath=\expandafter{\the\everymath\displaystyle}
%%   
%%   \ifdefined\pdftexversion\else  % non-pdftex case.
%%     \usepackage{fontspec}
%%     \setmainfont{DejaVuSans.ttf}[Path=\detokenize{/home/petr/Projects/PyRigi/.venv/lib/python3.12/site-packages/matplotlib/mpl-data/fonts/ttf/}]
%%     \setsansfont{DejaVuSans.ttf}[Path=\detokenize{/home/petr/Projects/PyRigi/.venv/lib/python3.12/site-packages/matplotlib/mpl-data/fonts/ttf/}]
%%     \setmonofont{DejaVuSansMono.ttf}[Path=\detokenize{/home/petr/Projects/PyRigi/.venv/lib/python3.12/site-packages/matplotlib/mpl-data/fonts/ttf/}]
%%   \fi
%%   \makeatletter\@ifpackageloaded{underscore}{}{\usepackage[strings]{underscore}}\makeatother
%%
\begingroup%
\makeatletter%
\begin{pgfpicture}%
\pgfpathrectangle{\pgfpointorigin}{\pgfqpoint{8.384376in}{2.25in}}%
\pgfusepath{use as bounding box, clip}%
\begin{pgfscope}%
\pgfsetbuttcap%
\pgfsetmiterjoin%
\definecolor{currentfill}{rgb}{1.000000,1.000000,1.000000}%
\pgfsetfillcolor{currentfill}%
\pgfsetlinewidth{0.000000pt}%
\definecolor{currentstroke}{rgb}{1.000000,1.000000,1.000000}%
\pgfsetstrokecolor{currentstroke}%
\pgfsetdash{}{0pt}%
\pgfpathmoveto{\pgfqpoint{0.000000in}{0.000000in}}%
\pgfpathlineto{\pgfqpoint{8.384376in}{0.000000in}}%
\pgfpathlineto{\pgfqpoint{8.384376in}{2.841849in}}%
\pgfpathlineto{\pgfqpoint{0.000000in}{2.841849in}}%
\pgfpathlineto{\pgfqpoint{0.000000in}{0.000000in}}%
\pgfpathclose%
\pgfusepath{fill}%
\end{pgfscope}%
\begin{pgfscope}%
\pgfsetbuttcap%
\pgfsetmiterjoin%
\definecolor{currentfill}{rgb}{1.000000,1.000000,1.000000}%
\pgfsetfillcolor{currentfill}%
\pgfsetlinewidth{0.000000pt}%
\definecolor{currentstroke}{rgb}{0.000000,0.000000,0.000000}%
\pgfsetstrokecolor{currentstroke}%
\pgfsetstrokeopacity{0.000000}%
\pgfsetdash{}{0pt}%
\pgfpathmoveto{\pgfqpoint{0.336112in}{0.521603in}}%
\pgfpathlineto{\pgfqpoint{5.487514in}{0.521603in}}%
\pgfpathlineto{\pgfqpoint{5.487514in}{2.206605in}}%
\pgfpathlineto{\pgfqpoint{0.336112in}{2.206605in}}%
\pgfpathlineto{\pgfqpoint{0.336112in}{0.521603in}}%
\pgfpathclose%
\pgfusepath{fill}%
\end{pgfscope}%
\begin{pgfscope}%
\pgfsetbuttcap%
\pgfsetroundjoin%
\definecolor{currentfill}{rgb}{0.000000,0.000000,0.000000}%
\pgfsetfillcolor{currentfill}%
\pgfsetlinewidth{0.803000pt}%
\definecolor{currentstroke}{rgb}{0.000000,0.000000,0.000000}%
\pgfsetstrokecolor{currentstroke}%
\pgfsetdash{}{0pt}%
\pgfsys@defobject{currentmarker}{\pgfqpoint{0.000000in}{-0.048611in}}{\pgfqpoint{0.000000in}{0.000000in}}{%
\pgfpathmoveto{\pgfqpoint{0.000000in}{0.000000in}}%
\pgfpathlineto{\pgfqpoint{0.000000in}{-0.048611in}}%
\pgfusepath{stroke,fill}%
}%
\begin{pgfscope}%
\pgfsys@transformshift{0.769547in}{0.521603in}%
\pgfsys@useobject{currentmarker}{}%
\end{pgfscope}%
\end{pgfscope}%
\begin{pgfscope}%
\definecolor{textcolor}{rgb}{0.000000,0.000000,0.000000}%
\pgfsetstrokecolor{textcolor}%
\pgfsetfillcolor{textcolor}%
\pgftext[x=0.769547in,y=0.424381in,,top]{\color{textcolor}{\rmfamily\fontsize{10.000000}{12.000000}\selectfont\catcode`\^=\active\def^{\ifmmode\sp\else\^{}\fi}\catcode`\%=\active\def%{\%}$\mathdefault{12}$}}%
\end{pgfscope}%
\begin{pgfscope}%
\pgfsetbuttcap%
\pgfsetroundjoin%
\definecolor{currentfill}{rgb}{0.000000,0.000000,0.000000}%
\pgfsetfillcolor{currentfill}%
\pgfsetlinewidth{0.803000pt}%
\definecolor{currentstroke}{rgb}{0.000000,0.000000,0.000000}%
\pgfsetstrokecolor{currentstroke}%
\pgfsetdash{}{0pt}%
\pgfsys@defobject{currentmarker}{\pgfqpoint{0.000000in}{-0.048611in}}{\pgfqpoint{0.000000in}{0.000000in}}{%
\pgfpathmoveto{\pgfqpoint{0.000000in}{0.000000in}}%
\pgfpathlineto{\pgfqpoint{0.000000in}{-0.048611in}}%
\pgfusepath{stroke,fill}%
}%
\begin{pgfscope}%
\pgfsys@transformshift{1.367388in}{0.521603in}%
\pgfsys@useobject{currentmarker}{}%
\end{pgfscope}%
\end{pgfscope}%
\begin{pgfscope}%
\definecolor{textcolor}{rgb}{0.000000,0.000000,0.000000}%
\pgfsetstrokecolor{textcolor}%
\pgfsetfillcolor{textcolor}%
\pgftext[x=1.367388in,y=0.424381in,,top]{\color{textcolor}{\rmfamily\fontsize{10.000000}{12.000000}\selectfont\catcode`\^=\active\def^{\ifmmode\sp\else\^{}\fi}\catcode`\%=\active\def%{\%}$\mathdefault{18}$}}%
\end{pgfscope}%
\begin{pgfscope}%
\pgfsetbuttcap%
\pgfsetroundjoin%
\definecolor{currentfill}{rgb}{0.000000,0.000000,0.000000}%
\pgfsetfillcolor{currentfill}%
\pgfsetlinewidth{0.803000pt}%
\definecolor{currentstroke}{rgb}{0.000000,0.000000,0.000000}%
\pgfsetstrokecolor{currentstroke}%
\pgfsetdash{}{0pt}%
\pgfsys@defobject{currentmarker}{\pgfqpoint{0.000000in}{-0.048611in}}{\pgfqpoint{0.000000in}{0.000000in}}{%
\pgfpathmoveto{\pgfqpoint{0.000000in}{0.000000in}}%
\pgfpathlineto{\pgfqpoint{0.000000in}{-0.048611in}}%
\pgfusepath{stroke,fill}%
}%
\begin{pgfscope}%
\pgfsys@transformshift{1.965230in}{0.521603in}%
\pgfsys@useobject{currentmarker}{}%
\end{pgfscope}%
\end{pgfscope}%
\begin{pgfscope}%
\definecolor{textcolor}{rgb}{0.000000,0.000000,0.000000}%
\pgfsetstrokecolor{textcolor}%
\pgfsetfillcolor{textcolor}%
\pgftext[x=1.965230in,y=0.424381in,,top]{\color{textcolor}{\rmfamily\fontsize{10.000000}{12.000000}\selectfont\catcode`\^=\active\def^{\ifmmode\sp\else\^{}\fi}\catcode`\%=\active\def%{\%}$\mathdefault{24}$}}%
\end{pgfscope}%
\begin{pgfscope}%
\pgfsetbuttcap%
\pgfsetroundjoin%
\definecolor{currentfill}{rgb}{0.000000,0.000000,0.000000}%
\pgfsetfillcolor{currentfill}%
\pgfsetlinewidth{0.803000pt}%
\definecolor{currentstroke}{rgb}{0.000000,0.000000,0.000000}%
\pgfsetstrokecolor{currentstroke}%
\pgfsetdash{}{0pt}%
\pgfsys@defobject{currentmarker}{\pgfqpoint{0.000000in}{-0.048611in}}{\pgfqpoint{0.000000in}{0.000000in}}{%
\pgfpathmoveto{\pgfqpoint{0.000000in}{0.000000in}}%
\pgfpathlineto{\pgfqpoint{0.000000in}{-0.048611in}}%
\pgfusepath{stroke,fill}%
}%
\begin{pgfscope}%
\pgfsys@transformshift{2.563072in}{0.521603in}%
\pgfsys@useobject{currentmarker}{}%
\end{pgfscope}%
\end{pgfscope}%
\begin{pgfscope}%
\definecolor{textcolor}{rgb}{0.000000,0.000000,0.000000}%
\pgfsetstrokecolor{textcolor}%
\pgfsetfillcolor{textcolor}%
\pgftext[x=2.563072in,y=0.424381in,,top]{\color{textcolor}{\rmfamily\fontsize{10.000000}{12.000000}\selectfont\catcode`\^=\active\def^{\ifmmode\sp\else\^{}\fi}\catcode`\%=\active\def%{\%}$\mathdefault{30}$}}%
\end{pgfscope}%
\begin{pgfscope}%
\pgfsetbuttcap%
\pgfsetroundjoin%
\definecolor{currentfill}{rgb}{0.000000,0.000000,0.000000}%
\pgfsetfillcolor{currentfill}%
\pgfsetlinewidth{0.803000pt}%
\definecolor{currentstroke}{rgb}{0.000000,0.000000,0.000000}%
\pgfsetstrokecolor{currentstroke}%
\pgfsetdash{}{0pt}%
\pgfsys@defobject{currentmarker}{\pgfqpoint{0.000000in}{-0.048611in}}{\pgfqpoint{0.000000in}{0.000000in}}{%
\pgfpathmoveto{\pgfqpoint{0.000000in}{0.000000in}}%
\pgfpathlineto{\pgfqpoint{0.000000in}{-0.048611in}}%
\pgfusepath{stroke,fill}%
}%
\begin{pgfscope}%
\pgfsys@transformshift{3.160913in}{0.521603in}%
\pgfsys@useobject{currentmarker}{}%
\end{pgfscope}%
\end{pgfscope}%
\begin{pgfscope}%
\definecolor{textcolor}{rgb}{0.000000,0.000000,0.000000}%
\pgfsetstrokecolor{textcolor}%
\pgfsetfillcolor{textcolor}%
\pgftext[x=3.160913in,y=0.424381in,,top]{\color{textcolor}{\rmfamily\fontsize{10.000000}{12.000000}\selectfont\catcode`\^=\active\def^{\ifmmode\sp\else\^{}\fi}\catcode`\%=\active\def%{\%}$\mathdefault{36}$}}%
\end{pgfscope}%
\begin{pgfscope}%
\pgfsetbuttcap%
\pgfsetroundjoin%
\definecolor{currentfill}{rgb}{0.000000,0.000000,0.000000}%
\pgfsetfillcolor{currentfill}%
\pgfsetlinewidth{0.803000pt}%
\definecolor{currentstroke}{rgb}{0.000000,0.000000,0.000000}%
\pgfsetstrokecolor{currentstroke}%
\pgfsetdash{}{0pt}%
\pgfsys@defobject{currentmarker}{\pgfqpoint{0.000000in}{-0.048611in}}{\pgfqpoint{0.000000in}{0.000000in}}{%
\pgfpathmoveto{\pgfqpoint{0.000000in}{0.000000in}}%
\pgfpathlineto{\pgfqpoint{0.000000in}{-0.048611in}}%
\pgfusepath{stroke,fill}%
}%
\begin{pgfscope}%
\pgfsys@transformshift{3.758755in}{0.521603in}%
\pgfsys@useobject{currentmarker}{}%
\end{pgfscope}%
\end{pgfscope}%
\begin{pgfscope}%
\definecolor{textcolor}{rgb}{0.000000,0.000000,0.000000}%
\pgfsetstrokecolor{textcolor}%
\pgfsetfillcolor{textcolor}%
\pgftext[x=3.758755in,y=0.424381in,,top]{\color{textcolor}{\rmfamily\fontsize{10.000000}{12.000000}\selectfont\catcode`\^=\active\def^{\ifmmode\sp\else\^{}\fi}\catcode`\%=\active\def%{\%}$\mathdefault{42}$}}%
\end{pgfscope}%
\begin{pgfscope}%
\pgfsetbuttcap%
\pgfsetroundjoin%
\definecolor{currentfill}{rgb}{0.000000,0.000000,0.000000}%
\pgfsetfillcolor{currentfill}%
\pgfsetlinewidth{0.803000pt}%
\definecolor{currentstroke}{rgb}{0.000000,0.000000,0.000000}%
\pgfsetstrokecolor{currentstroke}%
\pgfsetdash{}{0pt}%
\pgfsys@defobject{currentmarker}{\pgfqpoint{0.000000in}{-0.048611in}}{\pgfqpoint{0.000000in}{0.000000in}}{%
\pgfpathmoveto{\pgfqpoint{0.000000in}{0.000000in}}%
\pgfpathlineto{\pgfqpoint{0.000000in}{-0.048611in}}%
\pgfusepath{stroke,fill}%
}%
\begin{pgfscope}%
\pgfsys@transformshift{4.356597in}{0.521603in}%
\pgfsys@useobject{currentmarker}{}%
\end{pgfscope}%
\end{pgfscope}%
\begin{pgfscope}%
\definecolor{textcolor}{rgb}{0.000000,0.000000,0.000000}%
\pgfsetstrokecolor{textcolor}%
\pgfsetfillcolor{textcolor}%
\pgftext[x=4.356597in,y=0.424381in,,top]{\color{textcolor}{\rmfamily\fontsize{10.000000}{12.000000}\selectfont\catcode`\^=\active\def^{\ifmmode\sp\else\^{}\fi}\catcode`\%=\active\def%{\%}$\mathdefault{48}$}}%
\end{pgfscope}%
\begin{pgfscope}%
\pgfsetbuttcap%
\pgfsetroundjoin%
\definecolor{currentfill}{rgb}{0.000000,0.000000,0.000000}%
\pgfsetfillcolor{currentfill}%
\pgfsetlinewidth{0.803000pt}%
\definecolor{currentstroke}{rgb}{0.000000,0.000000,0.000000}%
\pgfsetstrokecolor{currentstroke}%
\pgfsetdash{}{0pt}%
\pgfsys@defobject{currentmarker}{\pgfqpoint{0.000000in}{-0.048611in}}{\pgfqpoint{0.000000in}{0.000000in}}{%
\pgfpathmoveto{\pgfqpoint{0.000000in}{0.000000in}}%
\pgfpathlineto{\pgfqpoint{0.000000in}{-0.048611in}}%
\pgfusepath{stroke,fill}%
}%
\begin{pgfscope}%
\pgfsys@transformshift{4.954438in}{0.521603in}%
\pgfsys@useobject{currentmarker}{}%
\end{pgfscope}%
\end{pgfscope}%
\begin{pgfscope}%
\definecolor{textcolor}{rgb}{0.000000,0.000000,0.000000}%
\pgfsetstrokecolor{textcolor}%
\pgfsetfillcolor{textcolor}%
\pgftext[x=4.954438in,y=0.424381in,,top]{\color{textcolor}{\rmfamily\fontsize{10.000000}{12.000000}\selectfont\catcode`\^=\active\def^{\ifmmode\sp\else\^{}\fi}\catcode`\%=\active\def%{\%}$\mathdefault{54}$}}%
\end{pgfscope}%
\begin{pgfscope}%
\definecolor{textcolor}{rgb}{0.000000,0.000000,0.000000}%
\pgfsetstrokecolor{textcolor}%
\pgfsetfillcolor{textcolor}%
\pgftext[x=2.911813in,y=0.234413in,,top]{\color{textcolor}{\rmfamily\fontsize{10.000000}{12.000000}\selectfont\catcode`\^=\active\def^{\ifmmode\sp\else\^{}\fi}\catcode`\%=\active\def%{\%}Vertices}}%
\end{pgfscope}%
\begin{pgfscope}%
\pgfsetbuttcap%
\pgfsetroundjoin%
\definecolor{currentfill}{rgb}{0.000000,0.000000,0.000000}%
\pgfsetfillcolor{currentfill}%
\pgfsetlinewidth{0.803000pt}%
\definecolor{currentstroke}{rgb}{0.000000,0.000000,0.000000}%
\pgfsetstrokecolor{currentstroke}%
\pgfsetdash{}{0pt}%
\pgfsys@defobject{currentmarker}{\pgfqpoint{-0.048611in}{0.000000in}}{\pgfqpoint{-0.000000in}{0.000000in}}{%
\pgfpathmoveto{\pgfqpoint{-0.000000in}{0.000000in}}%
\pgfpathlineto{\pgfqpoint{-0.048611in}{0.000000in}}%
\pgfusepath{stroke,fill}%
}%
\begin{pgfscope}%
\pgfsys@transformshift{0.336112in}{0.569211in}%
\pgfsys@useobject{currentmarker}{}%
\end{pgfscope}%
\end{pgfscope}%
\begin{pgfscope}%
\definecolor{textcolor}{rgb}{0.000000,0.000000,0.000000}%
\pgfsetstrokecolor{textcolor}%
\pgfsetfillcolor{textcolor}%
\pgftext[x=0.169445in, y=0.516449in, left, base]{\color{textcolor}{\rmfamily\fontsize{10.000000}{12.000000}\selectfont\catcode`\^=\active\def^{\ifmmode\sp\else\^{}\fi}\catcode`\%=\active\def%{\%}$\mathdefault{0}$}}%
\end{pgfscope}%
\begin{pgfscope}%
\pgfsetbuttcap%
\pgfsetroundjoin%
\definecolor{currentfill}{rgb}{0.000000,0.000000,0.000000}%
\pgfsetfillcolor{currentfill}%
\pgfsetlinewidth{0.803000pt}%
\definecolor{currentstroke}{rgb}{0.000000,0.000000,0.000000}%
\pgfsetstrokecolor{currentstroke}%
\pgfsetdash{}{0pt}%
\pgfsys@defobject{currentmarker}{\pgfqpoint{-0.048611in}{0.000000in}}{\pgfqpoint{-0.000000in}{0.000000in}}{%
\pgfpathmoveto{\pgfqpoint{-0.000000in}{0.000000in}}%
\pgfpathlineto{\pgfqpoint{-0.048611in}{0.000000in}}%
\pgfusepath{stroke,fill}%
}%
\begin{pgfscope}%
\pgfsys@transformshift{0.336112in}{0.898494in}%
\pgfsys@useobject{currentmarker}{}%
\end{pgfscope}%
\end{pgfscope}%
\begin{pgfscope}%
\definecolor{textcolor}{rgb}{0.000000,0.000000,0.000000}%
\pgfsetstrokecolor{textcolor}%
\pgfsetfillcolor{textcolor}%
\pgftext[x=0.100000in, y=0.845732in, left, base]{\color{textcolor}{\rmfamily\fontsize{10.000000}{12.000000}\selectfont\catcode`\^=\active\def^{\ifmmode\sp\else\^{}\fi}\catcode`\%=\active\def%{\%}$\mathdefault{20}$}}%
\end{pgfscope}%
\begin{pgfscope}%
\pgfsetbuttcap%
\pgfsetroundjoin%
\definecolor{currentfill}{rgb}{0.000000,0.000000,0.000000}%
\pgfsetfillcolor{currentfill}%
\pgfsetlinewidth{0.803000pt}%
\definecolor{currentstroke}{rgb}{0.000000,0.000000,0.000000}%
\pgfsetstrokecolor{currentstroke}%
\pgfsetdash{}{0pt}%
\pgfsys@defobject{currentmarker}{\pgfqpoint{-0.048611in}{0.000000in}}{\pgfqpoint{-0.000000in}{0.000000in}}{%
\pgfpathmoveto{\pgfqpoint{-0.000000in}{0.000000in}}%
\pgfpathlineto{\pgfqpoint{-0.048611in}{0.000000in}}%
\pgfusepath{stroke,fill}%
}%
\begin{pgfscope}%
\pgfsys@transformshift{0.336112in}{1.227777in}%
\pgfsys@useobject{currentmarker}{}%
\end{pgfscope}%
\end{pgfscope}%
\begin{pgfscope}%
\definecolor{textcolor}{rgb}{0.000000,0.000000,0.000000}%
\pgfsetstrokecolor{textcolor}%
\pgfsetfillcolor{textcolor}%
\pgftext[x=0.100000in, y=1.175016in, left, base]{\color{textcolor}{\rmfamily\fontsize{10.000000}{12.000000}\selectfont\catcode`\^=\active\def^{\ifmmode\sp\else\^{}\fi}\catcode`\%=\active\def%{\%}$\mathdefault{40}$}}%
\end{pgfscope}%
\begin{pgfscope}%
\pgfsetbuttcap%
\pgfsetroundjoin%
\definecolor{currentfill}{rgb}{0.000000,0.000000,0.000000}%
\pgfsetfillcolor{currentfill}%
\pgfsetlinewidth{0.803000pt}%
\definecolor{currentstroke}{rgb}{0.000000,0.000000,0.000000}%
\pgfsetstrokecolor{currentstroke}%
\pgfsetdash{}{0pt}%
\pgfsys@defobject{currentmarker}{\pgfqpoint{-0.048611in}{0.000000in}}{\pgfqpoint{-0.000000in}{0.000000in}}{%
\pgfpathmoveto{\pgfqpoint{-0.000000in}{0.000000in}}%
\pgfpathlineto{\pgfqpoint{-0.048611in}{0.000000in}}%
\pgfusepath{stroke,fill}%
}%
\begin{pgfscope}%
\pgfsys@transformshift{0.336112in}{1.557061in}%
\pgfsys@useobject{currentmarker}{}%
\end{pgfscope}%
\end{pgfscope}%
\begin{pgfscope}%
\definecolor{textcolor}{rgb}{0.000000,0.000000,0.000000}%
\pgfsetstrokecolor{textcolor}%
\pgfsetfillcolor{textcolor}%
\pgftext[x=0.100000in, y=1.504299in, left, base]{\color{textcolor}{\rmfamily\fontsize{10.000000}{12.000000}\selectfont\catcode`\^=\active\def^{\ifmmode\sp\else\^{}\fi}\catcode`\%=\active\def%{\%}$\mathdefault{60}$}}%
\end{pgfscope}%
\begin{pgfscope}%
\pgfsetbuttcap%
\pgfsetroundjoin%
\definecolor{currentfill}{rgb}{0.000000,0.000000,0.000000}%
\pgfsetfillcolor{currentfill}%
\pgfsetlinewidth{0.803000pt}%
\definecolor{currentstroke}{rgb}{0.000000,0.000000,0.000000}%
\pgfsetstrokecolor{currentstroke}%
\pgfsetdash{}{0pt}%
\pgfsys@defobject{currentmarker}{\pgfqpoint{-0.048611in}{0.000000in}}{\pgfqpoint{-0.000000in}{0.000000in}}{%
\pgfpathmoveto{\pgfqpoint{-0.000000in}{0.000000in}}%
\pgfpathlineto{\pgfqpoint{-0.048611in}{0.000000in}}%
\pgfusepath{stroke,fill}%
}%
\begin{pgfscope}%
\pgfsys@transformshift{0.336112in}{1.886344in}%
\pgfsys@useobject{currentmarker}{}%
\end{pgfscope}%
\end{pgfscope}%
\begin{pgfscope}%
\definecolor{textcolor}{rgb}{0.000000,0.000000,0.000000}%
\pgfsetstrokecolor{textcolor}%
\pgfsetfillcolor{textcolor}%
\pgftext[x=0.100000in, y=1.833583in, left, base]{\color{textcolor}{\rmfamily\fontsize{10.000000}{12.000000}\selectfont\catcode`\^=\active\def^{\ifmmode\sp\else\^{}\fi}\catcode`\%=\active\def%{\%}$\mathdefault{80}$}}%
\end{pgfscope}%
\begin{pgfscope}%
\pgfpathrectangle{\pgfqpoint{0.336112in}{0.521603in}}{\pgfqpoint{5.151402in}{1.685002in}}%
\pgfusepath{clip}%
\pgfsetrectcap%
\pgfsetroundjoin%
\pgfsetlinewidth{1.505625pt}%
\pgfsetstrokecolor{currentstroke1}%
\pgfsetdash{}{0pt}%
\pgfpathmoveto{\pgfqpoint{0.570266in}{0.598194in}}%
\pgfpathlineto{\pgfqpoint{0.669906in}{0.600104in}}%
\pgfpathlineto{\pgfqpoint{0.769547in}{0.598574in}}%
\pgfpathlineto{\pgfqpoint{0.869187in}{0.606255in}}%
\pgfpathlineto{\pgfqpoint{0.968827in}{0.613664in}}%
\pgfpathlineto{\pgfqpoint{1.068468in}{0.614158in}}%
\pgfpathlineto{\pgfqpoint{1.168108in}{0.624695in}}%
\pgfpathlineto{\pgfqpoint{1.267748in}{0.630128in}}%
\pgfpathlineto{\pgfqpoint{1.367388in}{0.630787in}}%
\pgfpathlineto{\pgfqpoint{1.467029in}{0.645769in}}%
\pgfpathlineto{\pgfqpoint{1.566669in}{0.647580in}}%
\pgfpathlineto{\pgfqpoint{1.666309in}{0.652849in}}%
\pgfpathlineto{\pgfqpoint{1.765949in}{0.755750in}}%
\pgfpathlineto{\pgfqpoint{1.865590in}{0.678368in}}%
\pgfpathlineto{\pgfqpoint{1.965230in}{0.696149in}}%
\pgfpathlineto{\pgfqpoint{2.064870in}{0.691869in}}%
\pgfpathlineto{\pgfqpoint{2.164511in}{0.696314in}}%
\pgfpathlineto{\pgfqpoint{2.264151in}{0.714754in}}%
\pgfpathlineto{\pgfqpoint{2.363791in}{0.720187in}}%
\pgfpathlineto{\pgfqpoint{2.463431in}{0.869224in}}%
\pgfpathlineto{\pgfqpoint{2.563072in}{0.753445in}}%
\pgfpathlineto{\pgfqpoint{2.662712in}{0.757725in}}%
\pgfpathlineto{\pgfqpoint{2.762352in}{0.754926in}}%
\pgfpathlineto{\pgfqpoint{2.861992in}{2.130014in}}%
\pgfpathlineto{\pgfqpoint{2.961633in}{0.771884in}}%
\pgfpathlineto{\pgfqpoint{3.061273in}{0.795922in}}%
\pgfpathlineto{\pgfqpoint{3.160913in}{0.801685in}}%
\pgfpathlineto{\pgfqpoint{3.260554in}{0.813868in}}%
\pgfpathlineto{\pgfqpoint{3.360194in}{0.839717in}}%
\pgfpathlineto{\pgfqpoint{3.459834in}{0.834300in}}%
\pgfpathlineto{\pgfqpoint{3.559474in}{0.857334in}}%
\pgfpathlineto{\pgfqpoint{3.659115in}{0.857663in}}%
\pgfpathlineto{\pgfqpoint{3.758755in}{0.887298in}}%
\pgfpathlineto{\pgfqpoint{3.858395in}{0.906397in}}%
\pgfpathlineto{\pgfqpoint{3.958035in}{0.894872in}}%
\pgfpathlineto{\pgfqpoint{4.057676in}{0.903269in}}%
\pgfpathlineto{\pgfqpoint{4.157316in}{0.904092in}}%
\pgfpathlineto{\pgfqpoint{4.256956in}{0.930270in}}%
\pgfpathlineto{\pgfqpoint{4.356597in}{0.966820in}}%
\pgfpathlineto{\pgfqpoint{4.456237in}{0.956942in}}%
\pgfpathlineto{\pgfqpoint{4.555877in}{0.978345in}}%
\pgfpathlineto{\pgfqpoint{4.655517in}{1.001724in}}%
\pgfpathlineto{\pgfqpoint{4.755158in}{1.038769in}}%
\pgfpathlineto{\pgfqpoint{4.854798in}{1.034982in}}%
\pgfpathlineto{\pgfqpoint{4.954438in}{1.034982in}}%
\pgfpathlineto{\pgfqpoint{5.054079in}{1.056385in}}%
\pgfpathlineto{\pgfqpoint{5.153719in}{1.107918in}}%
\pgfpathlineto{\pgfqpoint{5.253359in}{1.126591in}}%
\pgfusepath{stroke}%
\end{pgfscope}%
\begin{pgfscope}%
\pgfpathrectangle{\pgfqpoint{0.336112in}{0.521603in}}{\pgfqpoint{5.151402in}{1.685002in}}%
\pgfusepath{clip}%
\pgfsetrectcap%
\pgfsetroundjoin%
\pgfsetlinewidth{1.505625pt}%
\pgfsetstrokecolor{currentstroke2}%
\pgfsetdash{}{0pt}%
\pgfpathmoveto{\pgfqpoint{0.570266in}{0.601645in}}%
\pgfpathlineto{\pgfqpoint{0.669906in}{0.599312in}}%
\pgfpathlineto{\pgfqpoint{0.769547in}{0.603127in}}%
\pgfpathlineto{\pgfqpoint{0.869187in}{0.614322in}}%
\pgfpathlineto{\pgfqpoint{0.968827in}{0.617451in}}%
\pgfpathlineto{\pgfqpoint{1.068468in}{0.624530in}}%
\pgfpathlineto{\pgfqpoint{1.168108in}{0.634573in}}%
\pgfpathlineto{\pgfqpoint{1.267748in}{0.637537in}}%
\pgfpathlineto{\pgfqpoint{1.367388in}{0.647909in}}%
\pgfpathlineto{\pgfqpoint{1.467029in}{0.655812in}}%
\pgfpathlineto{\pgfqpoint{1.566669in}{0.657294in}}%
\pgfpathlineto{\pgfqpoint{1.666309in}{0.663056in}}%
\pgfpathlineto{\pgfqpoint{1.765949in}{0.682813in}}%
\pgfpathlineto{\pgfqpoint{1.865590in}{0.686765in}}%
\pgfpathlineto{\pgfqpoint{1.965230in}{0.692198in}}%
\pgfpathlineto{\pgfqpoint{2.064870in}{0.719528in}}%
\pgfpathlineto{\pgfqpoint{2.164511in}{0.715742in}}%
\pgfpathlineto{\pgfqpoint{2.264151in}{0.732206in}}%
\pgfpathlineto{\pgfqpoint{2.363791in}{0.753939in}}%
\pgfpathlineto{\pgfqpoint{2.463431in}{0.748176in}}%
\pgfpathlineto{\pgfqpoint{2.563072in}{0.762335in}}%
\pgfpathlineto{\pgfqpoint{2.662712in}{0.781104in}}%
\pgfpathlineto{\pgfqpoint{2.762352in}{0.777812in}}%
\pgfpathlineto{\pgfqpoint{2.861992in}{0.790160in}}%
\pgfpathlineto{\pgfqpoint{2.961633in}{0.786373in}}%
\pgfpathlineto{\pgfqpoint{3.061273in}{0.817820in}}%
\pgfpathlineto{\pgfqpoint{3.160913in}{0.840375in}}%
\pgfpathlineto{\pgfqpoint{3.260554in}{0.828851in}}%
\pgfpathlineto{\pgfqpoint{3.360194in}{0.880219in}}%
\pgfpathlineto{\pgfqpoint{3.459834in}{0.880054in}}%
\pgfpathlineto{\pgfqpoint{3.559474in}{0.899153in}}%
\pgfpathlineto{\pgfqpoint{3.659115in}{0.892567in}}%
\pgfpathlineto{\pgfqpoint{3.758755in}{0.914958in}}%
\pgfpathlineto{\pgfqpoint{3.858395in}{0.930434in}}%
\pgfpathlineto{\pgfqpoint{3.958035in}{0.919403in}}%
\pgfpathlineto{\pgfqpoint{4.057676in}{0.933892in}}%
\pgfpathlineto{\pgfqpoint{4.157316in}{0.940972in}}%
\pgfpathlineto{\pgfqpoint{4.256956in}{0.956119in}}%
\pgfpathlineto{\pgfqpoint{4.356597in}{0.990858in}}%
\pgfpathlineto{\pgfqpoint{4.456237in}{0.986577in}}%
\pgfpathlineto{\pgfqpoint{4.555877in}{1.016213in}}%
\pgfpathlineto{\pgfqpoint{4.655517in}{1.028890in}}%
\pgfpathlineto{\pgfqpoint{4.755158in}{1.082399in}}%
\pgfpathlineto{\pgfqpoint{4.854798in}{1.076636in}}%
\pgfpathlineto{\pgfqpoint{4.954438in}{1.064947in}}%
\pgfpathlineto{\pgfqpoint{5.054079in}{1.115821in}}%
\pgfpathlineto{\pgfqpoint{5.153719in}{1.138377in}}%
\pgfpathlineto{\pgfqpoint{5.253359in}{1.167409in}}%
\pgfusepath{stroke}%
\end{pgfscope}%
\begin{pgfscope}%
\pgfpathrectangle{\pgfqpoint{0.336112in}{0.521603in}}{\pgfqpoint{5.151402in}{1.685002in}}%
\pgfusepath{clip}%
\pgfsetrectcap%
\pgfsetroundjoin%
\pgfsetlinewidth{1.505625pt}%
\pgfsetstrokecolor{currentstroke3}%
\pgfsetdash{}{0pt}%
\pgfpathmoveto{\pgfqpoint{0.570266in}{0.601151in}}%
\pgfpathlineto{\pgfqpoint{0.669906in}{0.599312in}}%
\pgfpathlineto{\pgfqpoint{0.769547in}{0.602962in}}%
\pgfpathlineto{\pgfqpoint{0.869187in}{0.614158in}}%
\pgfpathlineto{\pgfqpoint{0.968827in}{0.616133in}}%
\pgfpathlineto{\pgfqpoint{1.068468in}{0.627329in}}%
\pgfpathlineto{\pgfqpoint{1.168108in}{0.648403in}}%
\pgfpathlineto{\pgfqpoint{1.267748in}{0.636055in}}%
\pgfpathlineto{\pgfqpoint{1.367388in}{0.651202in}}%
\pgfpathlineto{\pgfqpoint{1.467029in}{0.657458in}}%
\pgfpathlineto{\pgfqpoint{1.566669in}{0.659599in}}%
\pgfpathlineto{\pgfqpoint{1.666309in}{0.678533in}}%
\pgfpathlineto{\pgfqpoint{1.765949in}{0.690387in}}%
\pgfpathlineto{\pgfqpoint{1.865590in}{0.686765in}}%
\pgfpathlineto{\pgfqpoint{1.965230in}{0.699442in}}%
\pgfpathlineto{\pgfqpoint{2.064870in}{0.738627in}}%
\pgfpathlineto{\pgfqpoint{2.164511in}{0.732370in}}%
\pgfpathlineto{\pgfqpoint{2.264151in}{0.741920in}}%
\pgfpathlineto{\pgfqpoint{2.363791in}{0.740932in}}%
\pgfpathlineto{\pgfqpoint{2.463431in}{0.965339in}}%
\pgfpathlineto{\pgfqpoint{2.563072in}{0.776824in}}%
\pgfpathlineto{\pgfqpoint{2.662712in}{0.787032in}}%
\pgfpathlineto{\pgfqpoint{2.762352in}{1.003700in}}%
\pgfpathlineto{\pgfqpoint{2.861992in}{0.892238in}}%
\pgfpathlineto{\pgfqpoint{2.961633in}{0.786867in}}%
\pgfpathlineto{\pgfqpoint{3.061273in}{0.817490in}}%
\pgfpathlineto{\pgfqpoint{3.160913in}{0.818002in}}%
\pgfpathlineto{\pgfqpoint{3.260554in}{0.847455in}}%
\pgfpathlineto{\pgfqpoint{3.360194in}{0.915123in}}%
\pgfpathlineto{\pgfqpoint{3.459834in}{0.868694in}}%
\pgfpathlineto{\pgfqpoint{3.559474in}{1.187440in}}%
\pgfpathlineto{\pgfqpoint{3.659115in}{0.979662in}}%
\pgfpathlineto{\pgfqpoint{3.758755in}{0.918086in}}%
\pgfpathlineto{\pgfqpoint{3.858395in}{0.942947in}}%
\pgfpathlineto{\pgfqpoint{3.958035in}{0.920227in}}%
\pgfpathlineto{\pgfqpoint{4.057676in}{0.933398in}}%
\pgfpathlineto{\pgfqpoint{4.157316in}{0.939984in}}%
\pgfpathlineto{\pgfqpoint{4.256956in}{0.958094in}}%
\pgfpathlineto{\pgfqpoint{4.356597in}{1.004359in}}%
\pgfpathlineto{\pgfqpoint{4.456237in}{1.003206in}}%
\pgfpathlineto{\pgfqpoint{4.555877in}{1.015390in}}%
\pgfpathlineto{\pgfqpoint{4.655517in}{1.027408in}}%
\pgfpathlineto{\pgfqpoint{4.755158in}{1.079600in}}%
\pgfpathlineto{\pgfqpoint{4.854798in}{1.108577in}}%
\pgfpathlineto{\pgfqpoint{4.954438in}{1.047659in}}%
\pgfpathlineto{\pgfqpoint{5.054079in}{1.082728in}}%
\pgfpathlineto{\pgfqpoint{5.153719in}{1.126687in}}%
\pgfpathlineto{\pgfqpoint{5.253359in}{1.131050in}}%
\pgfusepath{stroke}%
\end{pgfscope}%
\begin{pgfscope}%
\pgfpathrectangle{\pgfqpoint{0.336112in}{0.521603in}}{\pgfqpoint{5.151402in}{1.685002in}}%
\pgfusepath{clip}%
\pgfsetrectcap%
\pgfsetroundjoin%
\pgfsetlinewidth{1.505625pt}%
\pgfsetstrokecolor{currentstroke4}%
\pgfsetdash{}{0pt}%
\pgfpathmoveto{\pgfqpoint{0.570266in}{0.602139in}}%
\pgfpathlineto{\pgfqpoint{0.669906in}{0.599312in}}%
\pgfpathlineto{\pgfqpoint{0.769547in}{0.602304in}}%
\pgfpathlineto{\pgfqpoint{0.869187in}{0.612676in}}%
\pgfpathlineto{\pgfqpoint{0.968827in}{0.615969in}}%
\pgfpathlineto{\pgfqpoint{1.068468in}{0.624036in}}%
\pgfpathlineto{\pgfqpoint{1.168108in}{0.633256in}}%
\pgfpathlineto{\pgfqpoint{1.267748in}{0.636384in}}%
\pgfpathlineto{\pgfqpoint{1.367388in}{0.644452in}}%
\pgfpathlineto{\pgfqpoint{1.467029in}{0.653672in}}%
\pgfpathlineto{\pgfqpoint{1.566669in}{0.656471in}}%
\pgfpathlineto{\pgfqpoint{1.666309in}{0.662068in}}%
\pgfpathlineto{\pgfqpoint{1.765949in}{0.680014in}}%
\pgfpathlineto{\pgfqpoint{1.865590in}{0.685777in}}%
\pgfpathlineto{\pgfqpoint{1.965230in}{0.692527in}}%
\pgfpathlineto{\pgfqpoint{2.064870in}{0.705534in}}%
\pgfpathlineto{\pgfqpoint{2.164511in}{0.715083in}}%
\pgfpathlineto{\pgfqpoint{2.264151in}{0.731383in}}%
\pgfpathlineto{\pgfqpoint{2.363791in}{0.738133in}}%
\pgfpathlineto{\pgfqpoint{2.463431in}{0.762994in}}%
\pgfpathlineto{\pgfqpoint{2.563072in}{0.754103in}}%
\pgfpathlineto{\pgfqpoint{2.662712in}{0.780775in}}%
\pgfpathlineto{\pgfqpoint{2.762352in}{0.776001in}}%
\pgfpathlineto{\pgfqpoint{2.861992in}{0.789172in}}%
\pgfpathlineto{\pgfqpoint{2.961633in}{0.787032in}}%
\pgfpathlineto{\pgfqpoint{3.061273in}{0.817326in}}%
\pgfpathlineto{\pgfqpoint{3.160913in}{0.825393in}}%
\pgfpathlineto{\pgfqpoint{3.260554in}{0.828357in}}%
\pgfpathlineto{\pgfqpoint{3.360194in}{0.878737in}}%
\pgfpathlineto{\pgfqpoint{3.459834in}{0.864742in}}%
\pgfpathlineto{\pgfqpoint{3.559474in}{0.897835in}}%
\pgfpathlineto{\pgfqpoint{3.659115in}{0.890427in}}%
\pgfpathlineto{\pgfqpoint{3.758755in}{0.914464in}}%
\pgfpathlineto{\pgfqpoint{3.858395in}{0.940972in}}%
\pgfpathlineto{\pgfqpoint{3.958035in}{0.935703in}}%
\pgfpathlineto{\pgfqpoint{4.057676in}{0.947228in}}%
\pgfpathlineto{\pgfqpoint{4.157316in}{0.940972in}}%
\pgfpathlineto{\pgfqpoint{4.256956in}{0.953484in}}%
\pgfpathlineto{\pgfqpoint{4.356597in}{1.005346in}}%
\pgfpathlineto{\pgfqpoint{4.456237in}{0.986083in}}%
\pgfpathlineto{\pgfqpoint{4.555877in}{1.061160in}}%
\pgfpathlineto{\pgfqpoint{4.655517in}{1.042720in}}%
\pgfpathlineto{\pgfqpoint{4.755158in}{1.063136in}}%
\pgfpathlineto{\pgfqpoint{4.854798in}{1.074167in}}%
\pgfpathlineto{\pgfqpoint{4.954438in}{1.048153in}}%
\pgfpathlineto{\pgfqpoint{5.054079in}{1.083057in}}%
\pgfpathlineto{\pgfqpoint{5.153719in}{1.125700in}}%
\pgfpathlineto{\pgfqpoint{5.253359in}{1.130707in}}%
\pgfusepath{stroke}%
\end{pgfscope}%
\begin{pgfscope}%
\pgfpathrectangle{\pgfqpoint{0.336112in}{0.521603in}}{\pgfqpoint{5.151402in}{1.685002in}}%
\pgfusepath{clip}%
\pgfsetrectcap%
\pgfsetroundjoin%
\pgfsetlinewidth{1.505625pt}%
\pgfsetstrokecolor{currentstroke5}%
\pgfsetdash{}{0pt}%
\pgfpathmoveto{\pgfqpoint{0.570266in}{0.600986in}}%
\pgfpathlineto{\pgfqpoint{0.669906in}{0.599175in}}%
\pgfpathlineto{\pgfqpoint{0.769547in}{0.601810in}}%
\pgfpathlineto{\pgfqpoint{0.869187in}{0.613993in}}%
\pgfpathlineto{\pgfqpoint{0.968827in}{0.616133in}}%
\pgfpathlineto{\pgfqpoint{1.068468in}{0.626835in}}%
\pgfpathlineto{\pgfqpoint{1.168108in}{0.635561in}}%
\pgfpathlineto{\pgfqpoint{1.267748in}{0.636549in}}%
\pgfpathlineto{\pgfqpoint{1.367388in}{0.647909in}}%
\pgfpathlineto{\pgfqpoint{1.467029in}{0.672441in}}%
\pgfpathlineto{\pgfqpoint{1.566669in}{0.659763in}}%
\pgfpathlineto{\pgfqpoint{1.666309in}{0.710802in}}%
\pgfpathlineto{\pgfqpoint{1.765949in}{0.712613in}}%
\pgfpathlineto{\pgfqpoint{1.865590in}{0.716565in}}%
\pgfpathlineto{\pgfqpoint{1.965230in}{0.708827in}}%
\pgfpathlineto{\pgfqpoint{2.064870in}{0.751798in}}%
\pgfpathlineto{\pgfqpoint{2.164511in}{0.736157in}}%
\pgfpathlineto{\pgfqpoint{2.264151in}{0.871163in}}%
\pgfpathlineto{\pgfqpoint{2.363791in}{0.737639in}}%
\pgfpathlineto{\pgfqpoint{2.463431in}{1.040909in}}%
\pgfpathlineto{\pgfqpoint{2.563072in}{0.768921in}}%
\pgfpathlineto{\pgfqpoint{2.662712in}{0.870176in}}%
\pgfpathlineto{\pgfqpoint{2.762352in}{0.953978in}}%
\pgfpathlineto{\pgfqpoint{2.861992in}{0.849760in}}%
\pgfpathlineto{\pgfqpoint{2.961633in}{0.928129in}}%
\pgfpathlineto{\pgfqpoint{3.061273in}{0.818313in}}%
\pgfpathlineto{\pgfqpoint{3.160913in}{1.291823in}}%
\pgfpathlineto{\pgfqpoint{3.260554in}{0.829509in}}%
\pgfpathlineto{\pgfqpoint{3.360194in}{1.006170in}}%
\pgfpathlineto{\pgfqpoint{3.459834in}{1.150067in}}%
\pgfpathlineto{\pgfqpoint{3.559474in}{0.923684in}}%
\pgfpathlineto{\pgfqpoint{3.659115in}{1.272231in}}%
\pgfpathlineto{\pgfqpoint{3.758755in}{0.914135in}}%
\pgfpathlineto{\pgfqpoint{3.858395in}{0.984437in}}%
\pgfpathlineto{\pgfqpoint{3.958035in}{0.980815in}}%
\pgfpathlineto{\pgfqpoint{4.057676in}{0.937349in}}%
\pgfpathlineto{\pgfqpoint{4.157316in}{0.970278in}}%
\pgfpathlineto{\pgfqpoint{4.256956in}{0.958094in}}%
\pgfpathlineto{\pgfqpoint{4.356597in}{1.033994in}}%
\pgfpathlineto{\pgfqpoint{4.456237in}{0.992834in}}%
\pgfpathlineto{\pgfqpoint{4.555877in}{1.020658in}}%
\pgfpathlineto{\pgfqpoint{4.655517in}{1.047989in}}%
\pgfpathlineto{\pgfqpoint{4.755158in}{1.065770in}}%
\pgfpathlineto{\pgfqpoint{4.854798in}{1.075319in}}%
\pgfpathlineto{\pgfqpoint{4.954438in}{1.048977in}}%
\pgfpathlineto{\pgfqpoint{5.054079in}{1.092277in}}%
\pgfpathlineto{\pgfqpoint{5.153719in}{1.129486in}}%
\pgfpathlineto{\pgfqpoint{5.253359in}{1.125905in}}%
\pgfusepath{stroke}%
\end{pgfscope}%
\begin{pgfscope}%
\pgfpathrectangle{\pgfqpoint{0.336112in}{0.521603in}}{\pgfqpoint{5.151402in}{1.685002in}}%
\pgfusepath{clip}%
\pgfsetrectcap%
\pgfsetroundjoin%
\pgfsetlinewidth{1.505625pt}%
\pgfsetstrokecolor{currentstroke6}%
\pgfsetdash{}{0pt}%
\pgfpathmoveto{\pgfqpoint{0.570266in}{0.600986in}}%
\pgfpathlineto{\pgfqpoint{0.669906in}{0.599423in}}%
\pgfpathlineto{\pgfqpoint{0.769547in}{0.601151in}}%
\pgfpathlineto{\pgfqpoint{0.869187in}{0.612017in}}%
\pgfpathlineto{\pgfqpoint{0.968827in}{0.615639in}}%
\pgfpathlineto{\pgfqpoint{1.068468in}{0.621731in}}%
\pgfpathlineto{\pgfqpoint{1.168108in}{0.635396in}}%
\pgfpathlineto{\pgfqpoint{1.267748in}{0.636220in}}%
\pgfpathlineto{\pgfqpoint{1.367388in}{0.646263in}}%
\pgfpathlineto{\pgfqpoint{1.467029in}{0.656965in}}%
\pgfpathlineto{\pgfqpoint{1.566669in}{0.655812in}}%
\pgfpathlineto{\pgfqpoint{1.666309in}{0.664538in}}%
\pgfpathlineto{\pgfqpoint{1.765949in}{0.684460in}}%
\pgfpathlineto{\pgfqpoint{1.865590in}{0.693350in}}%
\pgfpathlineto{\pgfqpoint{1.965230in}{0.697466in}}%
\pgfpathlineto{\pgfqpoint{2.064870in}{0.707839in}}%
\pgfpathlineto{\pgfqpoint{2.164511in}{0.729572in}}%
\pgfpathlineto{\pgfqpoint{2.264151in}{0.737968in}}%
\pgfpathlineto{\pgfqpoint{2.363791in}{0.746694in}}%
\pgfpathlineto{\pgfqpoint{2.463431in}{0.760853in}}%
\pgfpathlineto{\pgfqpoint{2.563072in}{0.762500in}}%
\pgfpathlineto{\pgfqpoint{2.662712in}{0.782915in}}%
\pgfpathlineto{\pgfqpoint{2.762352in}{0.784727in}}%
\pgfpathlineto{\pgfqpoint{2.861992in}{0.816173in}}%
\pgfpathlineto{\pgfqpoint{2.961633in}{0.801520in}}%
\pgfpathlineto{\pgfqpoint{3.061273in}{0.847290in}}%
\pgfpathlineto{\pgfqpoint{3.160913in}{0.839882in}}%
\pgfpathlineto{\pgfqpoint{3.260554in}{0.858651in}}%
\pgfpathlineto{\pgfqpoint{3.360194in}{0.905244in}}%
\pgfpathlineto{\pgfqpoint{3.459834in}{0.870999in}}%
\pgfpathlineto{\pgfqpoint{3.559474in}{0.931422in}}%
\pgfpathlineto{\pgfqpoint{3.659115in}{0.908702in}}%
\pgfpathlineto{\pgfqpoint{3.758755in}{0.937679in}}%
\pgfpathlineto{\pgfqpoint{3.858395in}{0.943441in}}%
\pgfpathlineto{\pgfqpoint{3.958035in}{0.932081in}}%
\pgfpathlineto{\pgfqpoint{4.057676in}{0.944429in}}%
\pgfpathlineto{\pgfqpoint{4.157316in}{0.955131in}}%
\pgfpathlineto{\pgfqpoint{4.256956in}{0.988718in}}%
\pgfpathlineto{\pgfqpoint{4.356597in}{0.994480in}}%
\pgfpathlineto{\pgfqpoint{4.456237in}{1.006993in}}%
\pgfpathlineto{\pgfqpoint{4.555877in}{1.028067in}}%
\pgfpathlineto{\pgfqpoint{4.655517in}{1.046672in}}%
\pgfpathlineto{\pgfqpoint{4.755158in}{1.081740in}}%
\pgfpathlineto{\pgfqpoint{4.854798in}{1.097710in}}%
\pgfpathlineto{\pgfqpoint{4.954438in}{1.062642in}}%
\pgfpathlineto{\pgfqpoint{5.054079in}{1.093265in}}%
\pgfpathlineto{\pgfqpoint{5.153719in}{1.155335in}}%
\pgfpathlineto{\pgfqpoint{5.253359in}{1.131393in}}%
\pgfusepath{stroke}%
\end{pgfscope}%
\begin{pgfscope}%
\pgfpathrectangle{\pgfqpoint{0.336112in}{0.521603in}}{\pgfqpoint{5.151402in}{1.685002in}}%
\pgfusepath{clip}%
\pgfsetrectcap%
\pgfsetroundjoin%
\pgfsetlinewidth{1.505625pt}%
\pgfsetstrokecolor{currentstroke7}%
\pgfsetdash{}{0pt}%
\pgfpathmoveto{\pgfqpoint{0.570266in}{0.601799in}}%
\pgfpathlineto{\pgfqpoint{0.669906in}{0.599566in}}%
\pgfpathlineto{\pgfqpoint{0.769547in}{0.602139in}}%
\pgfpathlineto{\pgfqpoint{0.869187in}{0.612017in}}%
\pgfpathlineto{\pgfqpoint{0.968827in}{0.615310in}}%
\pgfpathlineto{\pgfqpoint{1.068468in}{0.637372in}}%
\pgfpathlineto{\pgfqpoint{1.168108in}{0.637701in}}%
\pgfpathlineto{\pgfqpoint{1.267748in}{0.639513in}}%
\pgfpathlineto{\pgfqpoint{1.367388in}{0.660422in}}%
\pgfpathlineto{\pgfqpoint{1.467029in}{0.679356in}}%
\pgfpathlineto{\pgfqpoint{1.566669in}{0.708662in}}%
\pgfpathlineto{\pgfqpoint{1.666309in}{0.667831in}}%
\pgfpathlineto{\pgfqpoint{1.765949in}{0.707016in}}%
\pgfpathlineto{\pgfqpoint{1.865590in}{0.693350in}}%
\pgfpathlineto{\pgfqpoint{1.965230in}{0.694174in}}%
\pgfpathlineto{\pgfqpoint{2.064870in}{1.989904in}}%
\pgfpathlineto{\pgfqpoint{2.164511in}{0.730889in}}%
\pgfpathlineto{\pgfqpoint{2.264151in}{1.398346in}}%
\pgfpathlineto{\pgfqpoint{2.363791in}{0.746200in}}%
\pgfpathlineto{\pgfqpoint{2.463431in}{1.143975in}}%
\pgfpathlineto{\pgfqpoint{2.563072in}{0.762829in}}%
\pgfpathlineto{\pgfqpoint{2.662712in}{1.007158in}}%
\pgfpathlineto{\pgfqpoint{2.762352in}{2.027936in}}%
\pgfpathlineto{\pgfqpoint{2.861992in}{0.801355in}}%
\pgfpathlineto{\pgfqpoint{2.961633in}{0.798227in}}%
\pgfpathlineto{\pgfqpoint{3.061273in}{0.815679in}}%
\pgfpathlineto{\pgfqpoint{3.160913in}{0.824899in}}%
\pgfpathlineto{\pgfqpoint{3.260554in}{0.829344in}}%
\pgfpathlineto{\pgfqpoint{3.360194in}{0.882359in}}%
\pgfpathlineto{\pgfqpoint{3.459834in}{1.370357in}}%
\pgfpathlineto{\pgfqpoint{3.559474in}{0.936362in}}%
\pgfpathlineto{\pgfqpoint{3.659115in}{0.890262in}}%
\pgfpathlineto{\pgfqpoint{3.758755in}{0.924178in}}%
\pgfpathlineto{\pgfqpoint{3.858395in}{1.176738in}}%
\pgfpathlineto{\pgfqpoint{3.958035in}{0.917428in}}%
\pgfpathlineto{\pgfqpoint{4.057676in}{0.934057in}}%
\pgfpathlineto{\pgfqpoint{4.157316in}{0.938831in}}%
\pgfpathlineto{\pgfqpoint{4.256956in}{0.969619in}}%
\pgfpathlineto{\pgfqpoint{4.356597in}{0.988388in}}%
\pgfpathlineto{\pgfqpoint{4.456237in}{1.014566in}}%
\pgfpathlineto{\pgfqpoint{4.555877in}{1.014237in}}%
\pgfpathlineto{\pgfqpoint{4.655517in}{1.043049in}}%
\pgfpathlineto{\pgfqpoint{4.755158in}{1.062642in}}%
\pgfpathlineto{\pgfqpoint{4.854798in}{1.071862in}}%
\pgfpathlineto{\pgfqpoint{4.954438in}{1.063136in}}%
\pgfpathlineto{\pgfqpoint{5.054079in}{1.086515in}}%
\pgfpathlineto{\pgfqpoint{5.153719in}{1.123888in}}%
\pgfpathlineto{\pgfqpoint{5.253359in}{1.128649in}}%
\pgfusepath{stroke}%
\end{pgfscope}%
\begin{pgfscope}%
\pgfsetrectcap%
\pgfsetmiterjoin%
\pgfsetlinewidth{0.803000pt}%
\definecolor{currentstroke}{rgb}{0.000000,0.000000,0.000000}%
\pgfsetstrokecolor{currentstroke}%
\pgfsetdash{}{0pt}%
\pgfpathmoveto{\pgfqpoint{0.336112in}{0.521603in}}%
\pgfpathlineto{\pgfqpoint{0.336112in}{2.206605in}}%
\pgfusepath{stroke}%
\end{pgfscope}%
\begin{pgfscope}%
\pgfsetrectcap%
\pgfsetmiterjoin%
\pgfsetlinewidth{0.803000pt}%
\definecolor{currentstroke}{rgb}{0.000000,0.000000,0.000000}%
\pgfsetstrokecolor{currentstroke}%
\pgfsetdash{}{0pt}%
\pgfpathmoveto{\pgfqpoint{5.487514in}{0.521603in}}%
\pgfpathlineto{\pgfqpoint{5.487514in}{2.206605in}}%
\pgfusepath{stroke}%
\end{pgfscope}%
\begin{pgfscope}%
\pgfsetrectcap%
\pgfsetmiterjoin%
\pgfsetlinewidth{0.803000pt}%
\definecolor{currentstroke}{rgb}{0.000000,0.000000,0.000000}%
\pgfsetstrokecolor{currentstroke}%
\pgfsetdash{}{0pt}%
\pgfpathmoveto{\pgfqpoint{0.336112in}{0.521603in}}%
\pgfpathlineto{\pgfqpoint{5.487514in}{0.521603in}}%
\pgfusepath{stroke}%
\end{pgfscope}%
\begin{pgfscope}%
\pgfsetrectcap%
\pgfsetmiterjoin%
\pgfsetlinewidth{0.803000pt}%
\definecolor{currentstroke}{rgb}{0.000000,0.000000,0.000000}%
\pgfsetstrokecolor{currentstroke}%
\pgfsetdash{}{0pt}%
\pgfpathmoveto{\pgfqpoint{0.336112in}{2.206605in}}%
\pgfpathlineto{\pgfqpoint{5.487514in}{2.206605in}}%
\pgfusepath{stroke}%
\end{pgfscope}%
\begin{pgfscope}%
\pgfsetbuttcap%
\pgfsetmiterjoin%
\definecolor{currentfill}{rgb}{1.000000,1.000000,1.000000}%
\pgfsetfillcolor{currentfill}%
\pgfsetfillopacity{0.800000}%
\pgfsetlinewidth{1.003750pt}%
\definecolor{currentstroke}{rgb}{0.800000,0.800000,0.800000}%
\pgfsetstrokecolor{currentstroke}%
\pgfsetstrokeopacity{0.800000}%
\pgfsetdash{}{0pt}%
\pgfpathmoveto{\pgfqpoint{5.575014in}{0.808389in}}%
\pgfpathlineto{\pgfqpoint{8.259376in}{0.808389in}}%
\pgfpathquadraticcurveto{\pgfqpoint{8.284376in}{0.808389in}}{\pgfqpoint{8.284376in}{0.833389in}}%
\pgfpathlineto{\pgfqpoint{8.284376in}{2.119105in}}%
\pgfpathquadraticcurveto{\pgfqpoint{8.284376in}{2.144105in}}{\pgfqpoint{8.259376in}{2.144105in}}%
\pgfpathlineto{\pgfqpoint{5.575014in}{2.144105in}}%
\pgfpathquadraticcurveto{\pgfqpoint{5.550014in}{2.144105in}}{\pgfqpoint{5.550014in}{2.119105in}}%
\pgfpathlineto{\pgfqpoint{5.550014in}{0.833389in}}%
\pgfpathquadraticcurveto{\pgfqpoint{5.550014in}{0.808389in}}{\pgfqpoint{5.575014in}{0.808389in}}%
\pgfpathlineto{\pgfqpoint{5.575014in}{0.808389in}}%
\pgfpathclose%
\pgfusepath{stroke,fill}%
\end{pgfscope}%
\begin{pgfscope}%
\pgfsetrectcap%
\pgfsetroundjoin%
\pgfsetlinewidth{1.505625pt}%
\pgfsetstrokecolor{currentstroke1}%
\pgfsetdash{}{0pt}%
\pgfpathmoveto{\pgfqpoint{5.600014in}{2.042884in}}%
\pgfpathlineto{\pgfqpoint{5.725014in}{2.042884in}}%
\pgfpathlineto{\pgfqpoint{5.850014in}{2.042884in}}%
\pgfusepath{stroke}%
\end{pgfscope}%
\begin{pgfscope}%
\definecolor{textcolor}{rgb}{0.000000,0.000000,0.000000}%
\pgfsetstrokecolor{textcolor}%
\pgfsetfillcolor{textcolor}%
\pgftext[x=5.950014in,y=1.999134in,left,base]{\color{textcolor}{\rmfamily\fontsize{9.000000}{10.800000}\selectfont\catcode`\^=\active\def^{\ifmmode\sp\else\^{}\fi}\catcode`\%=\active\def%{\%}\NaiveCycles{}}}%
\end{pgfscope}%
\begin{pgfscope}%
\pgfsetrectcap%
\pgfsetroundjoin%
\pgfsetlinewidth{1.505625pt}%
\pgfsetstrokecolor{currentstroke2}%
\pgfsetdash{}{0pt}%
\pgfpathmoveto{\pgfqpoint{5.600014in}{1.859413in}}%
\pgfpathlineto{\pgfqpoint{5.725014in}{1.859413in}}%
\pgfpathlineto{\pgfqpoint{5.850014in}{1.859413in}}%
\pgfusepath{stroke}%
\end{pgfscope}%
\begin{pgfscope}%
\definecolor{textcolor}{rgb}{0.000000,0.000000,0.000000}%
\pgfsetstrokecolor{textcolor}%
\pgfsetfillcolor{textcolor}%
\pgftext[x=5.950014in,y=1.815663in,left,base]{\color{textcolor}{\rmfamily\fontsize{9.000000}{10.800000}\selectfont\catcode`\^=\active\def^{\ifmmode\sp\else\^{}\fi}\catcode`\%=\active\def%{\%}\Neighbors{} \& \MergeLinear{}}}%
\end{pgfscope}%
\begin{pgfscope}%
\pgfsetrectcap%
\pgfsetroundjoin%
\pgfsetlinewidth{1.505625pt}%
\pgfsetstrokecolor{currentstroke3}%
\pgfsetdash{}{0pt}%
\pgfpathmoveto{\pgfqpoint{5.600014in}{1.675941in}}%
\pgfpathlineto{\pgfqpoint{5.725014in}{1.675941in}}%
\pgfpathlineto{\pgfqpoint{5.850014in}{1.675941in}}%
\pgfusepath{stroke}%
\end{pgfscope}%
\begin{pgfscope}%
\definecolor{textcolor}{rgb}{0.000000,0.000000,0.000000}%
\pgfsetstrokecolor{textcolor}%
\pgfsetfillcolor{textcolor}%
\pgftext[x=5.950014in,y=1.632191in,left,base]{\color{textcolor}{\rmfamily\fontsize{9.000000}{10.800000}\selectfont\catcode`\^=\active\def^{\ifmmode\sp\else\^{}\fi}\catcode`\%=\active\def%{\%}\Neighbors{} \& \SharedVertices{}}}%
\end{pgfscope}%
\begin{pgfscope}%
\pgfsetrectcap%
\pgfsetroundjoin%
\pgfsetlinewidth{1.505625pt}%
\pgfsetstrokecolor{currentstroke4}%
\pgfsetdash{}{0pt}%
\pgfpathmoveto{\pgfqpoint{5.600014in}{1.488991in}}%
\pgfpathlineto{\pgfqpoint{5.725014in}{1.488991in}}%
\pgfpathlineto{\pgfqpoint{5.850014in}{1.488991in}}%
\pgfusepath{stroke}%
\end{pgfscope}%
\begin{pgfscope}%
\definecolor{textcolor}{rgb}{0.000000,0.000000,0.000000}%
\pgfsetstrokecolor{textcolor}%
\pgfsetfillcolor{textcolor}%
\pgftext[x=5.950014in,y=1.445241in,left,base]{\color{textcolor}{\rmfamily\fontsize{9.000000}{10.800000}\selectfont\catcode`\^=\active\def^{\ifmmode\sp\else\^{}\fi}\catcode`\%=\active\def%{\%}\NeighborsDegree{} \& \MergeLinear{}}}%
\end{pgfscope}%
\begin{pgfscope}%
\pgfsetrectcap%
\pgfsetroundjoin%
\pgfsetlinewidth{1.505625pt}%
\pgfsetstrokecolor{currentstroke5}%
\pgfsetdash{}{0pt}%
\pgfpathmoveto{\pgfqpoint{5.600014in}{1.302040in}}%
\pgfpathlineto{\pgfqpoint{5.725014in}{1.302040in}}%
\pgfpathlineto{\pgfqpoint{5.850014in}{1.302040in}}%
\pgfusepath{stroke}%
\end{pgfscope}%
\begin{pgfscope}%
\definecolor{textcolor}{rgb}{0.000000,0.000000,0.000000}%
\pgfsetstrokecolor{textcolor}%
\pgfsetfillcolor{textcolor}%
\pgftext[x=5.950014in,y=1.258290in,left,base]{\color{textcolor}{\rmfamily\fontsize{9.000000}{10.800000}\selectfont\catcode`\^=\active\def^{\ifmmode\sp\else\^{}\fi}\catcode`\%=\active\def%{\%}\NeighborsDegree{} \& \SharedVertices{}}}%
\end{pgfscope}%
\begin{pgfscope}%
\pgfsetrectcap%
\pgfsetroundjoin%
\pgfsetlinewidth{1.505625pt}%
\pgfsetstrokecolor{currentstroke6}%
\pgfsetdash{}{0pt}%
\pgfpathmoveto{\pgfqpoint{5.600014in}{1.115090in}}%
\pgfpathlineto{\pgfqpoint{5.725014in}{1.115090in}}%
\pgfpathlineto{\pgfqpoint{5.850014in}{1.115090in}}%
\pgfusepath{stroke}%
\end{pgfscope}%
\begin{pgfscope}%
\definecolor{textcolor}{rgb}{0.000000,0.000000,0.000000}%
\pgfsetstrokecolor{textcolor}%
\pgfsetfillcolor{textcolor}%
\pgftext[x=5.950014in,y=1.071340in,left,base]{\color{textcolor}{\rmfamily\fontsize{9.000000}{10.800000}\selectfont\catcode`\^=\active\def^{\ifmmode\sp\else\^{}\fi}\catcode`\%=\active\def%{\%}\None{} \& \MergeLinear{}}}%
\end{pgfscope}%
\begin{pgfscope}%
\pgfsetrectcap%
\pgfsetroundjoin%
\pgfsetlinewidth{1.505625pt}%
\pgfsetstrokecolor{currentstroke7}%
\pgfsetdash{}{0pt}%
\pgfpathmoveto{\pgfqpoint{5.600014in}{0.931619in}}%
\pgfpathlineto{\pgfqpoint{5.725014in}{0.931619in}}%
\pgfpathlineto{\pgfqpoint{5.850014in}{0.931619in}}%
\pgfusepath{stroke}%
\end{pgfscope}%
\begin{pgfscope}%
\definecolor{textcolor}{rgb}{0.000000,0.000000,0.000000}%
\pgfsetstrokecolor{textcolor}%
\pgfsetfillcolor{textcolor}%
\pgftext[x=5.950014in,y=0.887869in,left,base]{\color{textcolor}{\rmfamily\fontsize{9.000000}{10.800000}\selectfont\catcode`\^=\active\def^{\ifmmode\sp\else\^{}\fi}\catcode`\%=\active\def%{\%}\None{} \& \SharedVertices{}}}%
\end{pgfscope}%
\end{pgfpicture}%
\makeatother%
\endgroup%
}
	\caption[Running time for globally rigid graphs.]{
		Mean running time (ms) to find a NAC-coloring for globally rigid graphs.}%
	\label{fig:graph_time_globally_rigid}
\end{figure}

To summarize this section, we mostly tested graphs having many NAC-colorings
or trivially having none.
Overall, if only a single NAC-coloring is requested
for these graph classes, you can notice that the complexity
is not growing fast neither for the \NaiveCycles{} nor for \Subgraphs{}.
We tested only graphs with up to one hundred vertices
as it is computationally hard to find larger graphs in these classes
and to run proper benchmarks for them.
For such graphs a NAC-coloring can be found in hundreds of milliseconds.
From graphs, we can see that both algorithms should scale well for larger graphs.
For finding a single NAC-coloring, \Subgraphs{} are outperformed by \NaiveCycles{},
for listing all NAC-colorings, the \NaiveCycles{} algorithm
is outperformed quickly even for small graphs.

\todo[inline]{Integrate heuristics evaluation}
It can be also seen that \MergeLinear\ is the most reliable one
while the \SharedVertices\ sometimes performs slightly better,
namely on globally rigid graphs.
%
Splitting strategies \None{}, \CycleMask{}, \Neighbors{} and \NeighborsDegree{}
performs similarly.

\subsection{Performance on graphs with no NAC-colorings}

%% No NAC-coloring
% Naive failes quickly
% Algo complexity grows exponentially
% only 29 graphs with more monochromatic classes found out of 10000
% Graphs that should have just few NAC-colorings \cite{}
% Small initial subgraphs are benefitial for smaller graphs, increases over time
% Runtime of all the strategies is similar
% no. of checks differs a lot
In the previous section, the \Subgraphs{} algorithm performed worse considering runtime
than doing no subgraphs splitting heuristics,
but better considering the number of checks called.
As explained, it is caused by additional overhead and problem simplicity.

For many NP-complete problems, interesting instances are usually
those where there are only few or no solutions.
In this section we focus on graphs with no NAC-colorings.
We searched for random graphs where \( |E| \ge 2|V(G)| - 2 \) that have
multiple monochromatic classes, but no NAC-coloring.
As this search was slow and unsuccessful, we searched only for
graphs with more than \( 2\sqrt(|V(G)|) \) \trcon{} components.
This once again shows how effective monochromatic classes are
in comparison with \trcon{} components.
We generated ten thousand of such graphs from 40 to 140 vertices in size.
Less than 30 of them had more than one monochromatic components.
The following benchmarks are run with monochromatic classes disabled.

For these graphs, \NaiveCycles{} algorithm needs to traverse all \( 2^{t-1} \)
where \( t \) is the number of \trcon{} components. It can be clearly seen that
this is not suitable for graphs as large as we use in this benchmark.
It can be also seen that \SharedVertices{} is faster than \MergeLinear{},
for runtime and the number of checks performed.
It can be also seen that \NeighborsDegree{} strategy is
faster than the other strategies, and it holds for both merging strategies.
Also notice, that in contrast with the previous section,
runtime grows strictly exponentially. This was not the case for search for a single NAC-coloring.

\begin{figure}[ht]
	\centering
	\scalebox{0.5}{%% Creator: Matplotlib, PGF backend
%%
%% To include the figure in your LaTeX document, write
%%   \input{<filename>.pgf}
%%
%% Make sure the required packages are loaded in your preamble
%%   \usepackage{pgf}
%%
%% Also ensure that all the required font packages are loaded; for instance,
%% the lmodern package is sometimes necessary when using math font.
%%   \usepackage{lmodern}
%%
%% Figures using additional raster images can only be included by \input if
%% they are in the same directory as the main LaTeX file. For loading figures
%% from other directories you can use the `import` package
%%   \usepackage{import}
%%
%% and then include the figures with
%%   \import{<path to file>}{<filename>.pgf}
%%
%% Matplotlib used the following preamble
%%   \def\mathdefault#1{#1}
%%   \everymath=\expandafter{\the\everymath\displaystyle}
%%   
%%   \ifdefined\pdftexversion\else  % non-pdftex case.
%%     \usepackage{fontspec}
%%     \setmainfont{DejaVuSans.ttf}[Path=\detokenize{/home/petr/Projects/PyRigi/.venv/lib/python3.12/site-packages/matplotlib/mpl-data/fonts/ttf/}]
%%     \setsansfont{DejaVuSans.ttf}[Path=\detokenize{/home/petr/Projects/PyRigi/.venv/lib/python3.12/site-packages/matplotlib/mpl-data/fonts/ttf/}]
%%     \setmonofont{DejaVuSansMono.ttf}[Path=\detokenize{/home/petr/Projects/PyRigi/.venv/lib/python3.12/site-packages/matplotlib/mpl-data/fonts/ttf/}]
%%   \fi
%%   \makeatletter\@ifpackageloaded{underscore}{}{\usepackage[strings]{underscore}}\makeatother
%%
\begingroup%
\makeatletter%
\begin{pgfpicture}%
\pgfpathrectangle{\pgfpointorigin}{\pgfqpoint{8.719409in}{2.25in}}%
\pgfusepath{use as bounding box, clip}%
\begin{pgfscope}%
\pgfsetbuttcap%
\pgfsetmiterjoin%
\definecolor{currentfill}{rgb}{1.000000,1.000000,1.000000}%
\pgfsetfillcolor{currentfill}%
\pgfsetlinewidth{0.000000pt}%
\definecolor{currentstroke}{rgb}{1.000000,1.000000,1.000000}%
\pgfsetstrokecolor{currentstroke}%
\pgfsetdash{}{0pt}%
\pgfpathmoveto{\pgfqpoint{0.000000in}{0.000000in}}%
\pgfpathlineto{\pgfqpoint{8.719409in}{0.000000in}}%
\pgfpathlineto{\pgfqpoint{8.719409in}{2.841849in}}%
\pgfpathlineto{\pgfqpoint{0.000000in}{2.841849in}}%
\pgfpathlineto{\pgfqpoint{0.000000in}{0.000000in}}%
\pgfpathclose%
\pgfusepath{fill}%
\end{pgfscope}%
\begin{pgfscope}%
\pgfsetbuttcap%
\pgfsetmiterjoin%
\definecolor{currentfill}{rgb}{1.000000,1.000000,1.000000}%
\pgfsetfillcolor{currentfill}%
\pgfsetlinewidth{0.000000pt}%
\definecolor{currentstroke}{rgb}{0.000000,0.000000,0.000000}%
\pgfsetstrokecolor{currentstroke}%
\pgfsetstrokeopacity{0.000000}%
\pgfsetdash{}{0pt}%
\pgfpathmoveto{\pgfqpoint{0.565935in}{0.521603in}}%
\pgfpathlineto{\pgfqpoint{5.655030in}{0.521603in}}%
\pgfpathlineto{\pgfqpoint{5.655030in}{2.206605in}}%
\pgfpathlineto{\pgfqpoint{0.565935in}{2.206605in}}%
\pgfpathlineto{\pgfqpoint{0.565935in}{0.521603in}}%
\pgfpathclose%
\pgfusepath{fill}%
\end{pgfscope}%
\begin{pgfscope}%
\pgfsetbuttcap%
\pgfsetroundjoin%
\definecolor{currentfill}{rgb}{0.000000,0.000000,0.000000}%
\pgfsetfillcolor{currentfill}%
\pgfsetlinewidth{0.803000pt}%
\definecolor{currentstroke}{rgb}{0.000000,0.000000,0.000000}%
\pgfsetstrokecolor{currentstroke}%
\pgfsetdash{}{0pt}%
\pgfsys@defobject{currentmarker}{\pgfqpoint{0.000000in}{-0.048611in}}{\pgfqpoint{0.000000in}{0.000000in}}{%
\pgfpathmoveto{\pgfqpoint{0.000000in}{0.000000in}}%
\pgfpathlineto{\pgfqpoint{0.000000in}{-0.048611in}}%
\pgfusepath{stroke,fill}%
}%
\begin{pgfscope}%
\pgfsys@transformshift{1.153138in}{0.521603in}%
\pgfsys@useobject{currentmarker}{}%
\end{pgfscope}%
\end{pgfscope}%
\begin{pgfscope}%
\definecolor{textcolor}{rgb}{0.000000,0.000000,0.000000}%
\pgfsetstrokecolor{textcolor}%
\pgfsetfillcolor{textcolor}%
\pgftext[x=1.153138in,y=0.424381in,,top]{\color{textcolor}{\rmfamily\fontsize{10.000000}{12.000000}\selectfont\catcode`\^=\active\def^{\ifmmode\sp\else\^{}\fi}\catcode`\%=\active\def%{\%}$\mathdefault{20}$}}%
\end{pgfscope}%
\begin{pgfscope}%
\pgfsetbuttcap%
\pgfsetroundjoin%
\definecolor{currentfill}{rgb}{0.000000,0.000000,0.000000}%
\pgfsetfillcolor{currentfill}%
\pgfsetlinewidth{0.803000pt}%
\definecolor{currentstroke}{rgb}{0.000000,0.000000,0.000000}%
\pgfsetstrokecolor{currentstroke}%
\pgfsetdash{}{0pt}%
\pgfsys@defobject{currentmarker}{\pgfqpoint{0.000000in}{-0.048611in}}{\pgfqpoint{0.000000in}{0.000000in}}{%
\pgfpathmoveto{\pgfqpoint{0.000000in}{0.000000in}}%
\pgfpathlineto{\pgfqpoint{0.000000in}{-0.048611in}}%
\pgfusepath{stroke,fill}%
}%
\begin{pgfscope}%
\pgfsys@transformshift{1.746273in}{0.521603in}%
\pgfsys@useobject{currentmarker}{}%
\end{pgfscope}%
\end{pgfscope}%
\begin{pgfscope}%
\definecolor{textcolor}{rgb}{0.000000,0.000000,0.000000}%
\pgfsetstrokecolor{textcolor}%
\pgfsetfillcolor{textcolor}%
\pgftext[x=1.746273in,y=0.424381in,,top]{\color{textcolor}{\rmfamily\fontsize{10.000000}{12.000000}\selectfont\catcode`\^=\active\def^{\ifmmode\sp\else\^{}\fi}\catcode`\%=\active\def%{\%}$\mathdefault{40}$}}%
\end{pgfscope}%
\begin{pgfscope}%
\pgfsetbuttcap%
\pgfsetroundjoin%
\definecolor{currentfill}{rgb}{0.000000,0.000000,0.000000}%
\pgfsetfillcolor{currentfill}%
\pgfsetlinewidth{0.803000pt}%
\definecolor{currentstroke}{rgb}{0.000000,0.000000,0.000000}%
\pgfsetstrokecolor{currentstroke}%
\pgfsetdash{}{0pt}%
\pgfsys@defobject{currentmarker}{\pgfqpoint{0.000000in}{-0.048611in}}{\pgfqpoint{0.000000in}{0.000000in}}{%
\pgfpathmoveto{\pgfqpoint{0.000000in}{0.000000in}}%
\pgfpathlineto{\pgfqpoint{0.000000in}{-0.048611in}}%
\pgfusepath{stroke,fill}%
}%
\begin{pgfscope}%
\pgfsys@transformshift{2.339408in}{0.521603in}%
\pgfsys@useobject{currentmarker}{}%
\end{pgfscope}%
\end{pgfscope}%
\begin{pgfscope}%
\definecolor{textcolor}{rgb}{0.000000,0.000000,0.000000}%
\pgfsetstrokecolor{textcolor}%
\pgfsetfillcolor{textcolor}%
\pgftext[x=2.339408in,y=0.424381in,,top]{\color{textcolor}{\rmfamily\fontsize{10.000000}{12.000000}\selectfont\catcode`\^=\active\def^{\ifmmode\sp\else\^{}\fi}\catcode`\%=\active\def%{\%}$\mathdefault{60}$}}%
\end{pgfscope}%
\begin{pgfscope}%
\pgfsetbuttcap%
\pgfsetroundjoin%
\definecolor{currentfill}{rgb}{0.000000,0.000000,0.000000}%
\pgfsetfillcolor{currentfill}%
\pgfsetlinewidth{0.803000pt}%
\definecolor{currentstroke}{rgb}{0.000000,0.000000,0.000000}%
\pgfsetstrokecolor{currentstroke}%
\pgfsetdash{}{0pt}%
\pgfsys@defobject{currentmarker}{\pgfqpoint{0.000000in}{-0.048611in}}{\pgfqpoint{0.000000in}{0.000000in}}{%
\pgfpathmoveto{\pgfqpoint{0.000000in}{0.000000in}}%
\pgfpathlineto{\pgfqpoint{0.000000in}{-0.048611in}}%
\pgfusepath{stroke,fill}%
}%
\begin{pgfscope}%
\pgfsys@transformshift{2.932542in}{0.521603in}%
\pgfsys@useobject{currentmarker}{}%
\end{pgfscope}%
\end{pgfscope}%
\begin{pgfscope}%
\definecolor{textcolor}{rgb}{0.000000,0.000000,0.000000}%
\pgfsetstrokecolor{textcolor}%
\pgfsetfillcolor{textcolor}%
\pgftext[x=2.932542in,y=0.424381in,,top]{\color{textcolor}{\rmfamily\fontsize{10.000000}{12.000000}\selectfont\catcode`\^=\active\def^{\ifmmode\sp\else\^{}\fi}\catcode`\%=\active\def%{\%}$\mathdefault{80}$}}%
\end{pgfscope}%
\begin{pgfscope}%
\pgfsetbuttcap%
\pgfsetroundjoin%
\definecolor{currentfill}{rgb}{0.000000,0.000000,0.000000}%
\pgfsetfillcolor{currentfill}%
\pgfsetlinewidth{0.803000pt}%
\definecolor{currentstroke}{rgb}{0.000000,0.000000,0.000000}%
\pgfsetstrokecolor{currentstroke}%
\pgfsetdash{}{0pt}%
\pgfsys@defobject{currentmarker}{\pgfqpoint{0.000000in}{-0.048611in}}{\pgfqpoint{0.000000in}{0.000000in}}{%
\pgfpathmoveto{\pgfqpoint{0.000000in}{0.000000in}}%
\pgfpathlineto{\pgfqpoint{0.000000in}{-0.048611in}}%
\pgfusepath{stroke,fill}%
}%
\begin{pgfscope}%
\pgfsys@transformshift{3.525677in}{0.521603in}%
\pgfsys@useobject{currentmarker}{}%
\end{pgfscope}%
\end{pgfscope}%
\begin{pgfscope}%
\definecolor{textcolor}{rgb}{0.000000,0.000000,0.000000}%
\pgfsetstrokecolor{textcolor}%
\pgfsetfillcolor{textcolor}%
\pgftext[x=3.525677in,y=0.424381in,,top]{\color{textcolor}{\rmfamily\fontsize{10.000000}{12.000000}\selectfont\catcode`\^=\active\def^{\ifmmode\sp\else\^{}\fi}\catcode`\%=\active\def%{\%}$\mathdefault{100}$}}%
\end{pgfscope}%
\begin{pgfscope}%
\pgfsetbuttcap%
\pgfsetroundjoin%
\definecolor{currentfill}{rgb}{0.000000,0.000000,0.000000}%
\pgfsetfillcolor{currentfill}%
\pgfsetlinewidth{0.803000pt}%
\definecolor{currentstroke}{rgb}{0.000000,0.000000,0.000000}%
\pgfsetstrokecolor{currentstroke}%
\pgfsetdash{}{0pt}%
\pgfsys@defobject{currentmarker}{\pgfqpoint{0.000000in}{-0.048611in}}{\pgfqpoint{0.000000in}{0.000000in}}{%
\pgfpathmoveto{\pgfqpoint{0.000000in}{0.000000in}}%
\pgfpathlineto{\pgfqpoint{0.000000in}{-0.048611in}}%
\pgfusepath{stroke,fill}%
}%
\begin{pgfscope}%
\pgfsys@transformshift{4.118811in}{0.521603in}%
\pgfsys@useobject{currentmarker}{}%
\end{pgfscope}%
\end{pgfscope}%
\begin{pgfscope}%
\definecolor{textcolor}{rgb}{0.000000,0.000000,0.000000}%
\pgfsetstrokecolor{textcolor}%
\pgfsetfillcolor{textcolor}%
\pgftext[x=4.118811in,y=0.424381in,,top]{\color{textcolor}{\rmfamily\fontsize{10.000000}{12.000000}\selectfont\catcode`\^=\active\def^{\ifmmode\sp\else\^{}\fi}\catcode`\%=\active\def%{\%}$\mathdefault{120}$}}%
\end{pgfscope}%
\begin{pgfscope}%
\pgfsetbuttcap%
\pgfsetroundjoin%
\definecolor{currentfill}{rgb}{0.000000,0.000000,0.000000}%
\pgfsetfillcolor{currentfill}%
\pgfsetlinewidth{0.803000pt}%
\definecolor{currentstroke}{rgb}{0.000000,0.000000,0.000000}%
\pgfsetstrokecolor{currentstroke}%
\pgfsetdash{}{0pt}%
\pgfsys@defobject{currentmarker}{\pgfqpoint{0.000000in}{-0.048611in}}{\pgfqpoint{0.000000in}{0.000000in}}{%
\pgfpathmoveto{\pgfqpoint{0.000000in}{0.000000in}}%
\pgfpathlineto{\pgfqpoint{0.000000in}{-0.048611in}}%
\pgfusepath{stroke,fill}%
}%
\begin{pgfscope}%
\pgfsys@transformshift{4.711946in}{0.521603in}%
\pgfsys@useobject{currentmarker}{}%
\end{pgfscope}%
\end{pgfscope}%
\begin{pgfscope}%
\definecolor{textcolor}{rgb}{0.000000,0.000000,0.000000}%
\pgfsetstrokecolor{textcolor}%
\pgfsetfillcolor{textcolor}%
\pgftext[x=4.711946in,y=0.424381in,,top]{\color{textcolor}{\rmfamily\fontsize{10.000000}{12.000000}\selectfont\catcode`\^=\active\def^{\ifmmode\sp\else\^{}\fi}\catcode`\%=\active\def%{\%}$\mathdefault{140}$}}%
\end{pgfscope}%
\begin{pgfscope}%
\pgfsetbuttcap%
\pgfsetroundjoin%
\definecolor{currentfill}{rgb}{0.000000,0.000000,0.000000}%
\pgfsetfillcolor{currentfill}%
\pgfsetlinewidth{0.803000pt}%
\definecolor{currentstroke}{rgb}{0.000000,0.000000,0.000000}%
\pgfsetstrokecolor{currentstroke}%
\pgfsetdash{}{0pt}%
\pgfsys@defobject{currentmarker}{\pgfqpoint{0.000000in}{-0.048611in}}{\pgfqpoint{0.000000in}{0.000000in}}{%
\pgfpathmoveto{\pgfqpoint{0.000000in}{0.000000in}}%
\pgfpathlineto{\pgfqpoint{0.000000in}{-0.048611in}}%
\pgfusepath{stroke,fill}%
}%
\begin{pgfscope}%
\pgfsys@transformshift{5.305081in}{0.521603in}%
\pgfsys@useobject{currentmarker}{}%
\end{pgfscope}%
\end{pgfscope}%
\begin{pgfscope}%
\definecolor{textcolor}{rgb}{0.000000,0.000000,0.000000}%
\pgfsetstrokecolor{textcolor}%
\pgfsetfillcolor{textcolor}%
\pgftext[x=5.305081in,y=0.424381in,,top]{\color{textcolor}{\rmfamily\fontsize{10.000000}{12.000000}\selectfont\catcode`\^=\active\def^{\ifmmode\sp\else\^{}\fi}\catcode`\%=\active\def%{\%}$\mathdefault{160}$}}%
\end{pgfscope}%
\begin{pgfscope}%
\definecolor{textcolor}{rgb}{0.000000,0.000000,0.000000}%
\pgfsetstrokecolor{textcolor}%
\pgfsetfillcolor{textcolor}%
\pgftext[x=3.110483in,y=0.234413in,,top]{\color{textcolor}{\rmfamily\fontsize{10.000000}{12.000000}\selectfont\catcode`\^=\active\def^{\ifmmode\sp\else\^{}\fi}\catcode`\%=\active\def%{\%}Triangle components}}%
\end{pgfscope}%
\begin{pgfscope}%
\pgfsetbuttcap%
\pgfsetroundjoin%
\definecolor{currentfill}{rgb}{0.000000,0.000000,0.000000}%
\pgfsetfillcolor{currentfill}%
\pgfsetlinewidth{0.803000pt}%
\definecolor{currentstroke}{rgb}{0.000000,0.000000,0.000000}%
\pgfsetstrokecolor{currentstroke}%
\pgfsetdash{}{0pt}%
\pgfsys@defobject{currentmarker}{\pgfqpoint{-0.048611in}{0.000000in}}{\pgfqpoint{-0.000000in}{0.000000in}}{%
\pgfpathmoveto{\pgfqpoint{-0.000000in}{0.000000in}}%
\pgfpathlineto{\pgfqpoint{-0.048611in}{0.000000in}}%
\pgfusepath{stroke,fill}%
}%
\begin{pgfscope}%
\pgfsys@transformshift{0.565935in}{0.546297in}%
\pgfsys@useobject{currentmarker}{}%
\end{pgfscope}%
\end{pgfscope}%
\begin{pgfscope}%
\definecolor{textcolor}{rgb}{0.000000,0.000000,0.000000}%
\pgfsetstrokecolor{textcolor}%
\pgfsetfillcolor{textcolor}%
\pgftext[x=0.267516in, y=0.493535in, left, base]{\color{textcolor}{\rmfamily\fontsize{10.000000}{12.000000}\selectfont\catcode`\^=\active\def^{\ifmmode\sp\else\^{}\fi}\catcode`\%=\active\def%{\%}$\mathdefault{10^{1}}$}}%
\end{pgfscope}%
\begin{pgfscope}%
\pgfsetbuttcap%
\pgfsetroundjoin%
\definecolor{currentfill}{rgb}{0.000000,0.000000,0.000000}%
\pgfsetfillcolor{currentfill}%
\pgfsetlinewidth{0.803000pt}%
\definecolor{currentstroke}{rgb}{0.000000,0.000000,0.000000}%
\pgfsetstrokecolor{currentstroke}%
\pgfsetdash{}{0pt}%
\pgfsys@defobject{currentmarker}{\pgfqpoint{-0.048611in}{0.000000in}}{\pgfqpoint{-0.000000in}{0.000000in}}{%
\pgfpathmoveto{\pgfqpoint{-0.000000in}{0.000000in}}%
\pgfpathlineto{\pgfqpoint{-0.048611in}{0.000000in}}%
\pgfusepath{stroke,fill}%
}%
\begin{pgfscope}%
\pgfsys@transformshift{0.565935in}{1.001765in}%
\pgfsys@useobject{currentmarker}{}%
\end{pgfscope}%
\end{pgfscope}%
\begin{pgfscope}%
\definecolor{textcolor}{rgb}{0.000000,0.000000,0.000000}%
\pgfsetstrokecolor{textcolor}%
\pgfsetfillcolor{textcolor}%
\pgftext[x=0.267516in, y=0.949003in, left, base]{\color{textcolor}{\rmfamily\fontsize{10.000000}{12.000000}\selectfont\catcode`\^=\active\def^{\ifmmode\sp\else\^{}\fi}\catcode`\%=\active\def%{\%}$\mathdefault{10^{2}}$}}%
\end{pgfscope}%
\begin{pgfscope}%
\pgfsetbuttcap%
\pgfsetroundjoin%
\definecolor{currentfill}{rgb}{0.000000,0.000000,0.000000}%
\pgfsetfillcolor{currentfill}%
\pgfsetlinewidth{0.803000pt}%
\definecolor{currentstroke}{rgb}{0.000000,0.000000,0.000000}%
\pgfsetstrokecolor{currentstroke}%
\pgfsetdash{}{0pt}%
\pgfsys@defobject{currentmarker}{\pgfqpoint{-0.048611in}{0.000000in}}{\pgfqpoint{-0.000000in}{0.000000in}}{%
\pgfpathmoveto{\pgfqpoint{-0.000000in}{0.000000in}}%
\pgfpathlineto{\pgfqpoint{-0.048611in}{0.000000in}}%
\pgfusepath{stroke,fill}%
}%
\begin{pgfscope}%
\pgfsys@transformshift{0.565935in}{1.457233in}%
\pgfsys@useobject{currentmarker}{}%
\end{pgfscope}%
\end{pgfscope}%
\begin{pgfscope}%
\definecolor{textcolor}{rgb}{0.000000,0.000000,0.000000}%
\pgfsetstrokecolor{textcolor}%
\pgfsetfillcolor{textcolor}%
\pgftext[x=0.267516in, y=1.404471in, left, base]{\color{textcolor}{\rmfamily\fontsize{10.000000}{12.000000}\selectfont\catcode`\^=\active\def^{\ifmmode\sp\else\^{}\fi}\catcode`\%=\active\def%{\%}$\mathdefault{10^{3}}$}}%
\end{pgfscope}%
\begin{pgfscope}%
\pgfsetbuttcap%
\pgfsetroundjoin%
\definecolor{currentfill}{rgb}{0.000000,0.000000,0.000000}%
\pgfsetfillcolor{currentfill}%
\pgfsetlinewidth{0.803000pt}%
\definecolor{currentstroke}{rgb}{0.000000,0.000000,0.000000}%
\pgfsetstrokecolor{currentstroke}%
\pgfsetdash{}{0pt}%
\pgfsys@defobject{currentmarker}{\pgfqpoint{-0.048611in}{0.000000in}}{\pgfqpoint{-0.000000in}{0.000000in}}{%
\pgfpathmoveto{\pgfqpoint{-0.000000in}{0.000000in}}%
\pgfpathlineto{\pgfqpoint{-0.048611in}{0.000000in}}%
\pgfusepath{stroke,fill}%
}%
\begin{pgfscope}%
\pgfsys@transformshift{0.565935in}{1.912701in}%
\pgfsys@useobject{currentmarker}{}%
\end{pgfscope}%
\end{pgfscope}%
\begin{pgfscope}%
\definecolor{textcolor}{rgb}{0.000000,0.000000,0.000000}%
\pgfsetstrokecolor{textcolor}%
\pgfsetfillcolor{textcolor}%
\pgftext[x=0.267516in, y=1.859939in, left, base]{\color{textcolor}{\rmfamily\fontsize{10.000000}{12.000000}\selectfont\catcode`\^=\active\def^{\ifmmode\sp\else\^{}\fi}\catcode`\%=\active\def%{\%}$\mathdefault{10^{4}}$}}%
\end{pgfscope}%
\begin{pgfscope}%
\pgfsetbuttcap%
\pgfsetroundjoin%
\definecolor{currentfill}{rgb}{0.000000,0.000000,0.000000}%
\pgfsetfillcolor{currentfill}%
\pgfsetlinewidth{0.602250pt}%
\definecolor{currentstroke}{rgb}{0.000000,0.000000,0.000000}%
\pgfsetstrokecolor{currentstroke}%
\pgfsetdash{}{0pt}%
\pgfsys@defobject{currentmarker}{\pgfqpoint{-0.027778in}{0.000000in}}{\pgfqpoint{-0.000000in}{0.000000in}}{%
\pgfpathmoveto{\pgfqpoint{-0.000000in}{0.000000in}}%
\pgfpathlineto{\pgfqpoint{-0.027778in}{0.000000in}}%
\pgfusepath{stroke,fill}%
}%
\begin{pgfscope}%
\pgfsys@transformshift{0.565935in}{0.525456in}%
\pgfsys@useobject{currentmarker}{}%
\end{pgfscope}%
\end{pgfscope}%
\begin{pgfscope}%
\pgfsetbuttcap%
\pgfsetroundjoin%
\definecolor{currentfill}{rgb}{0.000000,0.000000,0.000000}%
\pgfsetfillcolor{currentfill}%
\pgfsetlinewidth{0.602250pt}%
\definecolor{currentstroke}{rgb}{0.000000,0.000000,0.000000}%
\pgfsetstrokecolor{currentstroke}%
\pgfsetdash{}{0pt}%
\pgfsys@defobject{currentmarker}{\pgfqpoint{-0.027778in}{0.000000in}}{\pgfqpoint{-0.000000in}{0.000000in}}{%
\pgfpathmoveto{\pgfqpoint{-0.000000in}{0.000000in}}%
\pgfpathlineto{\pgfqpoint{-0.027778in}{0.000000in}}%
\pgfusepath{stroke,fill}%
}%
\begin{pgfscope}%
\pgfsys@transformshift{0.565935in}{0.683406in}%
\pgfsys@useobject{currentmarker}{}%
\end{pgfscope}%
\end{pgfscope}%
\begin{pgfscope}%
\pgfsetbuttcap%
\pgfsetroundjoin%
\definecolor{currentfill}{rgb}{0.000000,0.000000,0.000000}%
\pgfsetfillcolor{currentfill}%
\pgfsetlinewidth{0.602250pt}%
\definecolor{currentstroke}{rgb}{0.000000,0.000000,0.000000}%
\pgfsetstrokecolor{currentstroke}%
\pgfsetdash{}{0pt}%
\pgfsys@defobject{currentmarker}{\pgfqpoint{-0.027778in}{0.000000in}}{\pgfqpoint{-0.000000in}{0.000000in}}{%
\pgfpathmoveto{\pgfqpoint{-0.000000in}{0.000000in}}%
\pgfpathlineto{\pgfqpoint{-0.027778in}{0.000000in}}%
\pgfusepath{stroke,fill}%
}%
\begin{pgfscope}%
\pgfsys@transformshift{0.565935in}{0.763610in}%
\pgfsys@useobject{currentmarker}{}%
\end{pgfscope}%
\end{pgfscope}%
\begin{pgfscope}%
\pgfsetbuttcap%
\pgfsetroundjoin%
\definecolor{currentfill}{rgb}{0.000000,0.000000,0.000000}%
\pgfsetfillcolor{currentfill}%
\pgfsetlinewidth{0.602250pt}%
\definecolor{currentstroke}{rgb}{0.000000,0.000000,0.000000}%
\pgfsetstrokecolor{currentstroke}%
\pgfsetdash{}{0pt}%
\pgfsys@defobject{currentmarker}{\pgfqpoint{-0.027778in}{0.000000in}}{\pgfqpoint{-0.000000in}{0.000000in}}{%
\pgfpathmoveto{\pgfqpoint{-0.000000in}{0.000000in}}%
\pgfpathlineto{\pgfqpoint{-0.027778in}{0.000000in}}%
\pgfusepath{stroke,fill}%
}%
\begin{pgfscope}%
\pgfsys@transformshift{0.565935in}{0.820516in}%
\pgfsys@useobject{currentmarker}{}%
\end{pgfscope}%
\end{pgfscope}%
\begin{pgfscope}%
\pgfsetbuttcap%
\pgfsetroundjoin%
\definecolor{currentfill}{rgb}{0.000000,0.000000,0.000000}%
\pgfsetfillcolor{currentfill}%
\pgfsetlinewidth{0.602250pt}%
\definecolor{currentstroke}{rgb}{0.000000,0.000000,0.000000}%
\pgfsetstrokecolor{currentstroke}%
\pgfsetdash{}{0pt}%
\pgfsys@defobject{currentmarker}{\pgfqpoint{-0.027778in}{0.000000in}}{\pgfqpoint{-0.000000in}{0.000000in}}{%
\pgfpathmoveto{\pgfqpoint{-0.000000in}{0.000000in}}%
\pgfpathlineto{\pgfqpoint{-0.027778in}{0.000000in}}%
\pgfusepath{stroke,fill}%
}%
\begin{pgfscope}%
\pgfsys@transformshift{0.565935in}{0.864655in}%
\pgfsys@useobject{currentmarker}{}%
\end{pgfscope}%
\end{pgfscope}%
\begin{pgfscope}%
\pgfsetbuttcap%
\pgfsetroundjoin%
\definecolor{currentfill}{rgb}{0.000000,0.000000,0.000000}%
\pgfsetfillcolor{currentfill}%
\pgfsetlinewidth{0.602250pt}%
\definecolor{currentstroke}{rgb}{0.000000,0.000000,0.000000}%
\pgfsetstrokecolor{currentstroke}%
\pgfsetdash{}{0pt}%
\pgfsys@defobject{currentmarker}{\pgfqpoint{-0.027778in}{0.000000in}}{\pgfqpoint{-0.000000in}{0.000000in}}{%
\pgfpathmoveto{\pgfqpoint{-0.000000in}{0.000000in}}%
\pgfpathlineto{\pgfqpoint{-0.027778in}{0.000000in}}%
\pgfusepath{stroke,fill}%
}%
\begin{pgfscope}%
\pgfsys@transformshift{0.565935in}{0.900720in}%
\pgfsys@useobject{currentmarker}{}%
\end{pgfscope}%
\end{pgfscope}%
\begin{pgfscope}%
\pgfsetbuttcap%
\pgfsetroundjoin%
\definecolor{currentfill}{rgb}{0.000000,0.000000,0.000000}%
\pgfsetfillcolor{currentfill}%
\pgfsetlinewidth{0.602250pt}%
\definecolor{currentstroke}{rgb}{0.000000,0.000000,0.000000}%
\pgfsetstrokecolor{currentstroke}%
\pgfsetdash{}{0pt}%
\pgfsys@defobject{currentmarker}{\pgfqpoint{-0.027778in}{0.000000in}}{\pgfqpoint{-0.000000in}{0.000000in}}{%
\pgfpathmoveto{\pgfqpoint{-0.000000in}{0.000000in}}%
\pgfpathlineto{\pgfqpoint{-0.027778in}{0.000000in}}%
\pgfusepath{stroke,fill}%
}%
\begin{pgfscope}%
\pgfsys@transformshift{0.565935in}{0.931212in}%
\pgfsys@useobject{currentmarker}{}%
\end{pgfscope}%
\end{pgfscope}%
\begin{pgfscope}%
\pgfsetbuttcap%
\pgfsetroundjoin%
\definecolor{currentfill}{rgb}{0.000000,0.000000,0.000000}%
\pgfsetfillcolor{currentfill}%
\pgfsetlinewidth{0.602250pt}%
\definecolor{currentstroke}{rgb}{0.000000,0.000000,0.000000}%
\pgfsetstrokecolor{currentstroke}%
\pgfsetdash{}{0pt}%
\pgfsys@defobject{currentmarker}{\pgfqpoint{-0.027778in}{0.000000in}}{\pgfqpoint{-0.000000in}{0.000000in}}{%
\pgfpathmoveto{\pgfqpoint{-0.000000in}{0.000000in}}%
\pgfpathlineto{\pgfqpoint{-0.027778in}{0.000000in}}%
\pgfusepath{stroke,fill}%
}%
\begin{pgfscope}%
\pgfsys@transformshift{0.565935in}{0.957625in}%
\pgfsys@useobject{currentmarker}{}%
\end{pgfscope}%
\end{pgfscope}%
\begin{pgfscope}%
\pgfsetbuttcap%
\pgfsetroundjoin%
\definecolor{currentfill}{rgb}{0.000000,0.000000,0.000000}%
\pgfsetfillcolor{currentfill}%
\pgfsetlinewidth{0.602250pt}%
\definecolor{currentstroke}{rgb}{0.000000,0.000000,0.000000}%
\pgfsetstrokecolor{currentstroke}%
\pgfsetdash{}{0pt}%
\pgfsys@defobject{currentmarker}{\pgfqpoint{-0.027778in}{0.000000in}}{\pgfqpoint{-0.000000in}{0.000000in}}{%
\pgfpathmoveto{\pgfqpoint{-0.000000in}{0.000000in}}%
\pgfpathlineto{\pgfqpoint{-0.027778in}{0.000000in}}%
\pgfusepath{stroke,fill}%
}%
\begin{pgfscope}%
\pgfsys@transformshift{0.565935in}{0.980924in}%
\pgfsys@useobject{currentmarker}{}%
\end{pgfscope}%
\end{pgfscope}%
\begin{pgfscope}%
\pgfsetbuttcap%
\pgfsetroundjoin%
\definecolor{currentfill}{rgb}{0.000000,0.000000,0.000000}%
\pgfsetfillcolor{currentfill}%
\pgfsetlinewidth{0.602250pt}%
\definecolor{currentstroke}{rgb}{0.000000,0.000000,0.000000}%
\pgfsetstrokecolor{currentstroke}%
\pgfsetdash{}{0pt}%
\pgfsys@defobject{currentmarker}{\pgfqpoint{-0.027778in}{0.000000in}}{\pgfqpoint{-0.000000in}{0.000000in}}{%
\pgfpathmoveto{\pgfqpoint{-0.000000in}{0.000000in}}%
\pgfpathlineto{\pgfqpoint{-0.027778in}{0.000000in}}%
\pgfusepath{stroke,fill}%
}%
\begin{pgfscope}%
\pgfsys@transformshift{0.565935in}{1.138874in}%
\pgfsys@useobject{currentmarker}{}%
\end{pgfscope}%
\end{pgfscope}%
\begin{pgfscope}%
\pgfsetbuttcap%
\pgfsetroundjoin%
\definecolor{currentfill}{rgb}{0.000000,0.000000,0.000000}%
\pgfsetfillcolor{currentfill}%
\pgfsetlinewidth{0.602250pt}%
\definecolor{currentstroke}{rgb}{0.000000,0.000000,0.000000}%
\pgfsetstrokecolor{currentstroke}%
\pgfsetdash{}{0pt}%
\pgfsys@defobject{currentmarker}{\pgfqpoint{-0.027778in}{0.000000in}}{\pgfqpoint{-0.000000in}{0.000000in}}{%
\pgfpathmoveto{\pgfqpoint{-0.000000in}{0.000000in}}%
\pgfpathlineto{\pgfqpoint{-0.027778in}{0.000000in}}%
\pgfusepath{stroke,fill}%
}%
\begin{pgfscope}%
\pgfsys@transformshift{0.565935in}{1.219078in}%
\pgfsys@useobject{currentmarker}{}%
\end{pgfscope}%
\end{pgfscope}%
\begin{pgfscope}%
\pgfsetbuttcap%
\pgfsetroundjoin%
\definecolor{currentfill}{rgb}{0.000000,0.000000,0.000000}%
\pgfsetfillcolor{currentfill}%
\pgfsetlinewidth{0.602250pt}%
\definecolor{currentstroke}{rgb}{0.000000,0.000000,0.000000}%
\pgfsetstrokecolor{currentstroke}%
\pgfsetdash{}{0pt}%
\pgfsys@defobject{currentmarker}{\pgfqpoint{-0.027778in}{0.000000in}}{\pgfqpoint{-0.000000in}{0.000000in}}{%
\pgfpathmoveto{\pgfqpoint{-0.000000in}{0.000000in}}%
\pgfpathlineto{\pgfqpoint{-0.027778in}{0.000000in}}%
\pgfusepath{stroke,fill}%
}%
\begin{pgfscope}%
\pgfsys@transformshift{0.565935in}{1.275984in}%
\pgfsys@useobject{currentmarker}{}%
\end{pgfscope}%
\end{pgfscope}%
\begin{pgfscope}%
\pgfsetbuttcap%
\pgfsetroundjoin%
\definecolor{currentfill}{rgb}{0.000000,0.000000,0.000000}%
\pgfsetfillcolor{currentfill}%
\pgfsetlinewidth{0.602250pt}%
\definecolor{currentstroke}{rgb}{0.000000,0.000000,0.000000}%
\pgfsetstrokecolor{currentstroke}%
\pgfsetdash{}{0pt}%
\pgfsys@defobject{currentmarker}{\pgfqpoint{-0.027778in}{0.000000in}}{\pgfqpoint{-0.000000in}{0.000000in}}{%
\pgfpathmoveto{\pgfqpoint{-0.000000in}{0.000000in}}%
\pgfpathlineto{\pgfqpoint{-0.027778in}{0.000000in}}%
\pgfusepath{stroke,fill}%
}%
\begin{pgfscope}%
\pgfsys@transformshift{0.565935in}{1.320123in}%
\pgfsys@useobject{currentmarker}{}%
\end{pgfscope}%
\end{pgfscope}%
\begin{pgfscope}%
\pgfsetbuttcap%
\pgfsetroundjoin%
\definecolor{currentfill}{rgb}{0.000000,0.000000,0.000000}%
\pgfsetfillcolor{currentfill}%
\pgfsetlinewidth{0.602250pt}%
\definecolor{currentstroke}{rgb}{0.000000,0.000000,0.000000}%
\pgfsetstrokecolor{currentstroke}%
\pgfsetdash{}{0pt}%
\pgfsys@defobject{currentmarker}{\pgfqpoint{-0.027778in}{0.000000in}}{\pgfqpoint{-0.000000in}{0.000000in}}{%
\pgfpathmoveto{\pgfqpoint{-0.000000in}{0.000000in}}%
\pgfpathlineto{\pgfqpoint{-0.027778in}{0.000000in}}%
\pgfusepath{stroke,fill}%
}%
\begin{pgfscope}%
\pgfsys@transformshift{0.565935in}{1.356188in}%
\pgfsys@useobject{currentmarker}{}%
\end{pgfscope}%
\end{pgfscope}%
\begin{pgfscope}%
\pgfsetbuttcap%
\pgfsetroundjoin%
\definecolor{currentfill}{rgb}{0.000000,0.000000,0.000000}%
\pgfsetfillcolor{currentfill}%
\pgfsetlinewidth{0.602250pt}%
\definecolor{currentstroke}{rgb}{0.000000,0.000000,0.000000}%
\pgfsetstrokecolor{currentstroke}%
\pgfsetdash{}{0pt}%
\pgfsys@defobject{currentmarker}{\pgfqpoint{-0.027778in}{0.000000in}}{\pgfqpoint{-0.000000in}{0.000000in}}{%
\pgfpathmoveto{\pgfqpoint{-0.000000in}{0.000000in}}%
\pgfpathlineto{\pgfqpoint{-0.027778in}{0.000000in}}%
\pgfusepath{stroke,fill}%
}%
\begin{pgfscope}%
\pgfsys@transformshift{0.565935in}{1.386680in}%
\pgfsys@useobject{currentmarker}{}%
\end{pgfscope}%
\end{pgfscope}%
\begin{pgfscope}%
\pgfsetbuttcap%
\pgfsetroundjoin%
\definecolor{currentfill}{rgb}{0.000000,0.000000,0.000000}%
\pgfsetfillcolor{currentfill}%
\pgfsetlinewidth{0.602250pt}%
\definecolor{currentstroke}{rgb}{0.000000,0.000000,0.000000}%
\pgfsetstrokecolor{currentstroke}%
\pgfsetdash{}{0pt}%
\pgfsys@defobject{currentmarker}{\pgfqpoint{-0.027778in}{0.000000in}}{\pgfqpoint{-0.000000in}{0.000000in}}{%
\pgfpathmoveto{\pgfqpoint{-0.000000in}{0.000000in}}%
\pgfpathlineto{\pgfqpoint{-0.027778in}{0.000000in}}%
\pgfusepath{stroke,fill}%
}%
\begin{pgfscope}%
\pgfsys@transformshift{0.565935in}{1.413093in}%
\pgfsys@useobject{currentmarker}{}%
\end{pgfscope}%
\end{pgfscope}%
\begin{pgfscope}%
\pgfsetbuttcap%
\pgfsetroundjoin%
\definecolor{currentfill}{rgb}{0.000000,0.000000,0.000000}%
\pgfsetfillcolor{currentfill}%
\pgfsetlinewidth{0.602250pt}%
\definecolor{currentstroke}{rgb}{0.000000,0.000000,0.000000}%
\pgfsetstrokecolor{currentstroke}%
\pgfsetdash{}{0pt}%
\pgfsys@defobject{currentmarker}{\pgfqpoint{-0.027778in}{0.000000in}}{\pgfqpoint{-0.000000in}{0.000000in}}{%
\pgfpathmoveto{\pgfqpoint{-0.000000in}{0.000000in}}%
\pgfpathlineto{\pgfqpoint{-0.027778in}{0.000000in}}%
\pgfusepath{stroke,fill}%
}%
\begin{pgfscope}%
\pgfsys@transformshift{0.565935in}{1.436392in}%
\pgfsys@useobject{currentmarker}{}%
\end{pgfscope}%
\end{pgfscope}%
\begin{pgfscope}%
\pgfsetbuttcap%
\pgfsetroundjoin%
\definecolor{currentfill}{rgb}{0.000000,0.000000,0.000000}%
\pgfsetfillcolor{currentfill}%
\pgfsetlinewidth{0.602250pt}%
\definecolor{currentstroke}{rgb}{0.000000,0.000000,0.000000}%
\pgfsetstrokecolor{currentstroke}%
\pgfsetdash{}{0pt}%
\pgfsys@defobject{currentmarker}{\pgfqpoint{-0.027778in}{0.000000in}}{\pgfqpoint{-0.000000in}{0.000000in}}{%
\pgfpathmoveto{\pgfqpoint{-0.000000in}{0.000000in}}%
\pgfpathlineto{\pgfqpoint{-0.027778in}{0.000000in}}%
\pgfusepath{stroke,fill}%
}%
\begin{pgfscope}%
\pgfsys@transformshift{0.565935in}{1.594342in}%
\pgfsys@useobject{currentmarker}{}%
\end{pgfscope}%
\end{pgfscope}%
\begin{pgfscope}%
\pgfsetbuttcap%
\pgfsetroundjoin%
\definecolor{currentfill}{rgb}{0.000000,0.000000,0.000000}%
\pgfsetfillcolor{currentfill}%
\pgfsetlinewidth{0.602250pt}%
\definecolor{currentstroke}{rgb}{0.000000,0.000000,0.000000}%
\pgfsetstrokecolor{currentstroke}%
\pgfsetdash{}{0pt}%
\pgfsys@defobject{currentmarker}{\pgfqpoint{-0.027778in}{0.000000in}}{\pgfqpoint{-0.000000in}{0.000000in}}{%
\pgfpathmoveto{\pgfqpoint{-0.000000in}{0.000000in}}%
\pgfpathlineto{\pgfqpoint{-0.027778in}{0.000000in}}%
\pgfusepath{stroke,fill}%
}%
\begin{pgfscope}%
\pgfsys@transformshift{0.565935in}{1.674546in}%
\pgfsys@useobject{currentmarker}{}%
\end{pgfscope}%
\end{pgfscope}%
\begin{pgfscope}%
\pgfsetbuttcap%
\pgfsetroundjoin%
\definecolor{currentfill}{rgb}{0.000000,0.000000,0.000000}%
\pgfsetfillcolor{currentfill}%
\pgfsetlinewidth{0.602250pt}%
\definecolor{currentstroke}{rgb}{0.000000,0.000000,0.000000}%
\pgfsetstrokecolor{currentstroke}%
\pgfsetdash{}{0pt}%
\pgfsys@defobject{currentmarker}{\pgfqpoint{-0.027778in}{0.000000in}}{\pgfqpoint{-0.000000in}{0.000000in}}{%
\pgfpathmoveto{\pgfqpoint{-0.000000in}{0.000000in}}%
\pgfpathlineto{\pgfqpoint{-0.027778in}{0.000000in}}%
\pgfusepath{stroke,fill}%
}%
\begin{pgfscope}%
\pgfsys@transformshift{0.565935in}{1.731452in}%
\pgfsys@useobject{currentmarker}{}%
\end{pgfscope}%
\end{pgfscope}%
\begin{pgfscope}%
\pgfsetbuttcap%
\pgfsetroundjoin%
\definecolor{currentfill}{rgb}{0.000000,0.000000,0.000000}%
\pgfsetfillcolor{currentfill}%
\pgfsetlinewidth{0.602250pt}%
\definecolor{currentstroke}{rgb}{0.000000,0.000000,0.000000}%
\pgfsetstrokecolor{currentstroke}%
\pgfsetdash{}{0pt}%
\pgfsys@defobject{currentmarker}{\pgfqpoint{-0.027778in}{0.000000in}}{\pgfqpoint{-0.000000in}{0.000000in}}{%
\pgfpathmoveto{\pgfqpoint{-0.000000in}{0.000000in}}%
\pgfpathlineto{\pgfqpoint{-0.027778in}{0.000000in}}%
\pgfusepath{stroke,fill}%
}%
\begin{pgfscope}%
\pgfsys@transformshift{0.565935in}{1.775591in}%
\pgfsys@useobject{currentmarker}{}%
\end{pgfscope}%
\end{pgfscope}%
\begin{pgfscope}%
\pgfsetbuttcap%
\pgfsetroundjoin%
\definecolor{currentfill}{rgb}{0.000000,0.000000,0.000000}%
\pgfsetfillcolor{currentfill}%
\pgfsetlinewidth{0.602250pt}%
\definecolor{currentstroke}{rgb}{0.000000,0.000000,0.000000}%
\pgfsetstrokecolor{currentstroke}%
\pgfsetdash{}{0pt}%
\pgfsys@defobject{currentmarker}{\pgfqpoint{-0.027778in}{0.000000in}}{\pgfqpoint{-0.000000in}{0.000000in}}{%
\pgfpathmoveto{\pgfqpoint{-0.000000in}{0.000000in}}%
\pgfpathlineto{\pgfqpoint{-0.027778in}{0.000000in}}%
\pgfusepath{stroke,fill}%
}%
\begin{pgfscope}%
\pgfsys@transformshift{0.565935in}{1.811656in}%
\pgfsys@useobject{currentmarker}{}%
\end{pgfscope}%
\end{pgfscope}%
\begin{pgfscope}%
\pgfsetbuttcap%
\pgfsetroundjoin%
\definecolor{currentfill}{rgb}{0.000000,0.000000,0.000000}%
\pgfsetfillcolor{currentfill}%
\pgfsetlinewidth{0.602250pt}%
\definecolor{currentstroke}{rgb}{0.000000,0.000000,0.000000}%
\pgfsetstrokecolor{currentstroke}%
\pgfsetdash{}{0pt}%
\pgfsys@defobject{currentmarker}{\pgfqpoint{-0.027778in}{0.000000in}}{\pgfqpoint{-0.000000in}{0.000000in}}{%
\pgfpathmoveto{\pgfqpoint{-0.000000in}{0.000000in}}%
\pgfpathlineto{\pgfqpoint{-0.027778in}{0.000000in}}%
\pgfusepath{stroke,fill}%
}%
\begin{pgfscope}%
\pgfsys@transformshift{0.565935in}{1.842148in}%
\pgfsys@useobject{currentmarker}{}%
\end{pgfscope}%
\end{pgfscope}%
\begin{pgfscope}%
\pgfsetbuttcap%
\pgfsetroundjoin%
\definecolor{currentfill}{rgb}{0.000000,0.000000,0.000000}%
\pgfsetfillcolor{currentfill}%
\pgfsetlinewidth{0.602250pt}%
\definecolor{currentstroke}{rgb}{0.000000,0.000000,0.000000}%
\pgfsetstrokecolor{currentstroke}%
\pgfsetdash{}{0pt}%
\pgfsys@defobject{currentmarker}{\pgfqpoint{-0.027778in}{0.000000in}}{\pgfqpoint{-0.000000in}{0.000000in}}{%
\pgfpathmoveto{\pgfqpoint{-0.000000in}{0.000000in}}%
\pgfpathlineto{\pgfqpoint{-0.027778in}{0.000000in}}%
\pgfusepath{stroke,fill}%
}%
\begin{pgfscope}%
\pgfsys@transformshift{0.565935in}{1.868561in}%
\pgfsys@useobject{currentmarker}{}%
\end{pgfscope}%
\end{pgfscope}%
\begin{pgfscope}%
\pgfsetbuttcap%
\pgfsetroundjoin%
\definecolor{currentfill}{rgb}{0.000000,0.000000,0.000000}%
\pgfsetfillcolor{currentfill}%
\pgfsetlinewidth{0.602250pt}%
\definecolor{currentstroke}{rgb}{0.000000,0.000000,0.000000}%
\pgfsetstrokecolor{currentstroke}%
\pgfsetdash{}{0pt}%
\pgfsys@defobject{currentmarker}{\pgfqpoint{-0.027778in}{0.000000in}}{\pgfqpoint{-0.000000in}{0.000000in}}{%
\pgfpathmoveto{\pgfqpoint{-0.000000in}{0.000000in}}%
\pgfpathlineto{\pgfqpoint{-0.027778in}{0.000000in}}%
\pgfusepath{stroke,fill}%
}%
\begin{pgfscope}%
\pgfsys@transformshift{0.565935in}{1.891860in}%
\pgfsys@useobject{currentmarker}{}%
\end{pgfscope}%
\end{pgfscope}%
\begin{pgfscope}%
\pgfsetbuttcap%
\pgfsetroundjoin%
\definecolor{currentfill}{rgb}{0.000000,0.000000,0.000000}%
\pgfsetfillcolor{currentfill}%
\pgfsetlinewidth{0.602250pt}%
\definecolor{currentstroke}{rgb}{0.000000,0.000000,0.000000}%
\pgfsetstrokecolor{currentstroke}%
\pgfsetdash{}{0pt}%
\pgfsys@defobject{currentmarker}{\pgfqpoint{-0.027778in}{0.000000in}}{\pgfqpoint{-0.000000in}{0.000000in}}{%
\pgfpathmoveto{\pgfqpoint{-0.000000in}{0.000000in}}%
\pgfpathlineto{\pgfqpoint{-0.027778in}{0.000000in}}%
\pgfusepath{stroke,fill}%
}%
\begin{pgfscope}%
\pgfsys@transformshift{0.565935in}{2.049810in}%
\pgfsys@useobject{currentmarker}{}%
\end{pgfscope}%
\end{pgfscope}%
\begin{pgfscope}%
\pgfsetbuttcap%
\pgfsetroundjoin%
\definecolor{currentfill}{rgb}{0.000000,0.000000,0.000000}%
\pgfsetfillcolor{currentfill}%
\pgfsetlinewidth{0.602250pt}%
\definecolor{currentstroke}{rgb}{0.000000,0.000000,0.000000}%
\pgfsetstrokecolor{currentstroke}%
\pgfsetdash{}{0pt}%
\pgfsys@defobject{currentmarker}{\pgfqpoint{-0.027778in}{0.000000in}}{\pgfqpoint{-0.000000in}{0.000000in}}{%
\pgfpathmoveto{\pgfqpoint{-0.000000in}{0.000000in}}%
\pgfpathlineto{\pgfqpoint{-0.027778in}{0.000000in}}%
\pgfusepath{stroke,fill}%
}%
\begin{pgfscope}%
\pgfsys@transformshift{0.565935in}{2.130014in}%
\pgfsys@useobject{currentmarker}{}%
\end{pgfscope}%
\end{pgfscope}%
\begin{pgfscope}%
\pgfsetbuttcap%
\pgfsetroundjoin%
\definecolor{currentfill}{rgb}{0.000000,0.000000,0.000000}%
\pgfsetfillcolor{currentfill}%
\pgfsetlinewidth{0.602250pt}%
\definecolor{currentstroke}{rgb}{0.000000,0.000000,0.000000}%
\pgfsetstrokecolor{currentstroke}%
\pgfsetdash{}{0pt}%
\pgfsys@defobject{currentmarker}{\pgfqpoint{-0.027778in}{0.000000in}}{\pgfqpoint{-0.000000in}{0.000000in}}{%
\pgfpathmoveto{\pgfqpoint{-0.000000in}{0.000000in}}%
\pgfpathlineto{\pgfqpoint{-0.027778in}{0.000000in}}%
\pgfusepath{stroke,fill}%
}%
\begin{pgfscope}%
\pgfsys@transformshift{0.565935in}{2.186920in}%
\pgfsys@useobject{currentmarker}{}%
\end{pgfscope}%
\end{pgfscope}%
\begin{pgfscope}%
\pgfpathrectangle{\pgfqpoint{0.565935in}{0.521603in}}{\pgfqpoint{5.089095in}{1.685002in}}%
\pgfusepath{clip}%
\pgfsetrectcap%
\pgfsetroundjoin%
\pgfsetlinewidth{1.505625pt}%
\pgfsetstrokecolor{currentstroke1}%
\pgfsetdash{}{0pt}%
\pgfpathmoveto{\pgfqpoint{0.797258in}{0.727546in}}%
\pgfpathlineto{\pgfqpoint{0.945541in}{0.910371in}}%
\pgfpathlineto{\pgfqpoint{1.004855in}{0.962510in}}%
\pgfpathlineto{\pgfqpoint{1.034512in}{1.396734in}}%
\pgfpathlineto{\pgfqpoint{1.123482in}{1.286340in}}%
\pgfpathlineto{\pgfqpoint{1.153138in}{1.265229in}}%
\pgfpathlineto{\pgfqpoint{1.242109in}{1.595624in}}%
\pgfpathlineto{\pgfqpoint{1.271765in}{1.733738in}}%
\pgfpathlineto{\pgfqpoint{1.360736in}{2.127101in}}%
\pgfpathlineto{\pgfqpoint{1.390392in}{2.130014in}}%
\pgfpathlineto{\pgfqpoint{1.420049in}{2.130014in}}%
\pgfpathlineto{\pgfqpoint{1.449706in}{2.130014in}}%
\pgfpathlineto{\pgfqpoint{1.479363in}{2.130014in}}%
\pgfpathlineto{\pgfqpoint{1.509019in}{2.130014in}}%
\pgfpathlineto{\pgfqpoint{1.568333in}{2.130014in}}%
\pgfpathlineto{\pgfqpoint{1.597989in}{2.130014in}}%
\pgfpathlineto{\pgfqpoint{1.627646in}{2.130014in}}%
\pgfpathlineto{\pgfqpoint{1.686960in}{2.130014in}}%
\pgfpathlineto{\pgfqpoint{1.805587in}{2.130014in}}%
\pgfpathlineto{\pgfqpoint{1.835243in}{2.130014in}}%
\pgfpathlineto{\pgfqpoint{1.864900in}{2.130014in}}%
\pgfpathlineto{\pgfqpoint{1.894557in}{2.130014in}}%
\pgfpathlineto{\pgfqpoint{1.924213in}{2.130014in}}%
\pgfpathlineto{\pgfqpoint{1.953870in}{2.130014in}}%
\pgfpathlineto{\pgfqpoint{2.013184in}{2.130014in}}%
\pgfpathlineto{\pgfqpoint{2.042840in}{1.992977in}}%
\pgfpathlineto{\pgfqpoint{2.072497in}{0.598194in}}%
\pgfpathlineto{\pgfqpoint{2.102154in}{2.130014in}}%
\pgfpathlineto{\pgfqpoint{2.131811in}{2.130014in}}%
\pgfpathlineto{\pgfqpoint{2.191124in}{2.049856in}}%
\pgfpathlineto{\pgfqpoint{2.220781in}{2.049863in}}%
\pgfpathlineto{\pgfqpoint{2.250437in}{0.723549in}}%
\pgfpathlineto{\pgfqpoint{2.280094in}{2.130014in}}%
\pgfpathlineto{\pgfqpoint{2.339408in}{0.598194in}}%
\pgfpathlineto{\pgfqpoint{2.369064in}{0.756904in}}%
\pgfpathlineto{\pgfqpoint{2.398721in}{2.130014in}}%
\pgfpathlineto{\pgfqpoint{2.517348in}{2.130014in}}%
\pgfpathlineto{\pgfqpoint{2.547005in}{1.993155in}}%
\pgfpathlineto{\pgfqpoint{2.635975in}{2.130014in}}%
\pgfpathlineto{\pgfqpoint{2.724945in}{1.913070in}}%
\pgfpathlineto{\pgfqpoint{2.754602in}{2.130014in}}%
\pgfpathlineto{\pgfqpoint{2.784259in}{2.130014in}}%
\pgfpathlineto{\pgfqpoint{2.813915in}{2.130014in}}%
\pgfpathlineto{\pgfqpoint{2.873229in}{2.130014in}}%
\pgfpathlineto{\pgfqpoint{2.932542in}{2.130014in}}%
\pgfpathlineto{\pgfqpoint{2.991856in}{2.130014in}}%
\pgfpathlineto{\pgfqpoint{3.021512in}{0.794102in}}%
\pgfpathlineto{\pgfqpoint{3.080826in}{2.130014in}}%
\pgfpathlineto{\pgfqpoint{3.347736in}{2.130014in}}%
\pgfpathlineto{\pgfqpoint{3.377393in}{2.130014in}}%
\pgfpathlineto{\pgfqpoint{3.703617in}{2.130014in}}%
\pgfpathlineto{\pgfqpoint{3.762931in}{2.130014in}}%
\pgfpathlineto{\pgfqpoint{4.385722in}{2.130014in}}%
\pgfpathlineto{\pgfqpoint{4.415379in}{2.130014in}}%
\pgfpathlineto{\pgfqpoint{4.445035in}{2.130014in}}%
\pgfpathlineto{\pgfqpoint{5.038170in}{2.130014in}}%
\pgfpathlineto{\pgfqpoint{5.394051in}{2.130014in}}%
\pgfpathlineto{\pgfqpoint{5.423708in}{2.130014in}}%
\pgfusepath{stroke}%
\end{pgfscope}%
\begin{pgfscope}%
\pgfpathrectangle{\pgfqpoint{0.565935in}{0.521603in}}{\pgfqpoint{5.089095in}{1.685002in}}%
\pgfusepath{clip}%
\pgfsetrectcap%
\pgfsetroundjoin%
\pgfsetlinewidth{1.505625pt}%
\pgfsetstrokecolor{currentstroke2}%
\pgfsetdash{}{0pt}%
\pgfpathmoveto{\pgfqpoint{0.797258in}{0.711052in}}%
\pgfpathlineto{\pgfqpoint{0.945541in}{0.673260in}}%
\pgfpathlineto{\pgfqpoint{1.004855in}{0.735304in}}%
\pgfpathlineto{\pgfqpoint{1.034512in}{0.791256in}}%
\pgfpathlineto{\pgfqpoint{1.123482in}{0.727546in}}%
\pgfpathlineto{\pgfqpoint{1.153138in}{0.746399in}}%
\pgfpathlineto{\pgfqpoint{1.242109in}{0.791256in}}%
\pgfpathlineto{\pgfqpoint{1.271765in}{0.749963in}}%
\pgfpathlineto{\pgfqpoint{1.360736in}{0.846000in}}%
\pgfpathlineto{\pgfqpoint{1.390392in}{0.856580in}}%
\pgfpathlineto{\pgfqpoint{1.420049in}{0.863332in}}%
\pgfpathlineto{\pgfqpoint{1.449706in}{0.820516in}}%
\pgfpathlineto{\pgfqpoint{1.479363in}{0.868572in}}%
\pgfpathlineto{\pgfqpoint{1.509019in}{0.900720in}}%
\pgfpathlineto{\pgfqpoint{1.568333in}{1.317309in}}%
\pgfpathlineto{\pgfqpoint{1.597989in}{0.944859in}}%
\pgfpathlineto{\pgfqpoint{1.627646in}{0.910371in}}%
\pgfpathlineto{\pgfqpoint{1.686960in}{0.931212in}}%
\pgfpathlineto{\pgfqpoint{1.805587in}{0.960083in}}%
\pgfpathlineto{\pgfqpoint{1.835243in}{1.009523in}}%
\pgfpathlineto{\pgfqpoint{1.864900in}{1.015148in}}%
\pgfpathlineto{\pgfqpoint{1.894557in}{1.031974in}}%
\pgfpathlineto{\pgfqpoint{1.924213in}{1.015148in}}%
\pgfpathlineto{\pgfqpoint{1.953870in}{1.063313in}}%
\pgfpathlineto{\pgfqpoint{2.013184in}{1.034505in}}%
\pgfpathlineto{\pgfqpoint{2.042840in}{0.944859in}}%
\pgfpathlineto{\pgfqpoint{2.072497in}{0.864655in}}%
\pgfpathlineto{\pgfqpoint{2.102154in}{1.155011in}}%
\pgfpathlineto{\pgfqpoint{2.131811in}{1.007612in}}%
\pgfpathlineto{\pgfqpoint{2.191124in}{1.101619in}}%
\pgfpathlineto{\pgfqpoint{2.220781in}{1.069261in}}%
\pgfpathlineto{\pgfqpoint{2.250437in}{0.915025in}}%
\pgfpathlineto{\pgfqpoint{2.280094in}{1.142791in}}%
\pgfpathlineto{\pgfqpoint{2.339408in}{0.876181in}}%
\pgfpathlineto{\pgfqpoint{2.369064in}{0.950065in}}%
\pgfpathlineto{\pgfqpoint{2.398721in}{1.147581in}}%
\pgfpathlineto{\pgfqpoint{2.517348in}{1.167378in}}%
\pgfpathlineto{\pgfqpoint{2.547005in}{1.155011in}}%
\pgfpathlineto{\pgfqpoint{2.635975in}{1.276478in}}%
\pgfpathlineto{\pgfqpoint{2.724945in}{1.163632in}}%
\pgfpathlineto{\pgfqpoint{2.754602in}{1.259490in}}%
\pgfpathlineto{\pgfqpoint{2.784259in}{1.059657in}}%
\pgfpathlineto{\pgfqpoint{2.813915in}{1.234911in}}%
\pgfpathlineto{\pgfqpoint{2.873229in}{1.243254in}}%
\pgfpathlineto{\pgfqpoint{2.932542in}{1.242083in}}%
\pgfpathlineto{\pgfqpoint{2.991856in}{1.305342in}}%
\pgfpathlineto{\pgfqpoint{3.021512in}{1.071127in}}%
\pgfpathlineto{\pgfqpoint{3.080826in}{1.326737in}}%
\pgfpathlineto{\pgfqpoint{3.347736in}{1.318135in}}%
\pgfpathlineto{\pgfqpoint{3.377393in}{1.358482in}}%
\pgfpathlineto{\pgfqpoint{3.703617in}{1.492802in}}%
\pgfpathlineto{\pgfqpoint{3.762931in}{1.405533in}}%
\pgfpathlineto{\pgfqpoint{4.385722in}{1.547213in}}%
\pgfpathlineto{\pgfqpoint{4.415379in}{1.483324in}}%
\pgfpathlineto{\pgfqpoint{4.445035in}{1.499304in}}%
\pgfpathlineto{\pgfqpoint{5.038170in}{1.549708in}}%
\pgfpathlineto{\pgfqpoint{5.394051in}{1.655165in}}%
\pgfpathlineto{\pgfqpoint{5.423708in}{1.686755in}}%
\pgfusepath{stroke}%
\end{pgfscope}%
\begin{pgfscope}%
\pgfpathrectangle{\pgfqpoint{0.565935in}{0.521603in}}{\pgfqpoint{5.089095in}{1.685002in}}%
\pgfusepath{clip}%
\pgfsetrectcap%
\pgfsetroundjoin%
\pgfsetlinewidth{1.505625pt}%
\pgfsetstrokecolor{currentstroke3}%
\pgfsetdash{}{0pt}%
\pgfpathmoveto{\pgfqpoint{0.797258in}{0.711052in}}%
\pgfpathlineto{\pgfqpoint{0.945541in}{0.673260in}}%
\pgfpathlineto{\pgfqpoint{1.004855in}{0.727546in}}%
\pgfpathlineto{\pgfqpoint{1.034512in}{0.791256in}}%
\pgfpathlineto{\pgfqpoint{1.123482in}{0.731463in}}%
\pgfpathlineto{\pgfqpoint{1.153138in}{0.740312in}}%
\pgfpathlineto{\pgfqpoint{1.242109in}{0.791256in}}%
\pgfpathlineto{\pgfqpoint{1.271765in}{0.756904in}}%
\pgfpathlineto{\pgfqpoint{1.360736in}{0.846000in}}%
\pgfpathlineto{\pgfqpoint{1.390392in}{0.863664in}}%
\pgfpathlineto{\pgfqpoint{1.420049in}{0.865970in}}%
\pgfpathlineto{\pgfqpoint{1.449706in}{0.825400in}}%
\pgfpathlineto{\pgfqpoint{1.479363in}{0.867275in}}%
\pgfpathlineto{\pgfqpoint{1.509019in}{0.913486in}}%
\pgfpathlineto{\pgfqpoint{1.568333in}{1.282190in}}%
\pgfpathlineto{\pgfqpoint{1.597989in}{0.945517in}}%
\pgfpathlineto{\pgfqpoint{1.627646in}{0.913486in}}%
\pgfpathlineto{\pgfqpoint{1.686960in}{0.942204in}}%
\pgfpathlineto{\pgfqpoint{1.805587in}{0.980924in}}%
\pgfpathlineto{\pgfqpoint{1.835243in}{1.006649in}}%
\pgfpathlineto{\pgfqpoint{1.864900in}{1.003733in}}%
\pgfpathlineto{\pgfqpoint{1.894557in}{1.092247in}}%
\pgfpathlineto{\pgfqpoint{1.924213in}{0.995740in}}%
\pgfpathlineto{\pgfqpoint{1.953870in}{1.053662in}}%
\pgfpathlineto{\pgfqpoint{2.013184in}{1.050595in}}%
\pgfpathlineto{\pgfqpoint{2.042840in}{0.962510in}}%
\pgfpathlineto{\pgfqpoint{2.072497in}{0.852416in}}%
\pgfpathlineto{\pgfqpoint{2.102154in}{1.152258in}}%
\pgfpathlineto{\pgfqpoint{2.131811in}{1.047480in}}%
\pgfpathlineto{\pgfqpoint{2.191124in}{1.106727in}}%
\pgfpathlineto{\pgfqpoint{2.220781in}{1.110187in}}%
\pgfpathlineto{\pgfqpoint{2.250437in}{0.990575in}}%
\pgfpathlineto{\pgfqpoint{2.280094in}{1.138544in}}%
\pgfpathlineto{\pgfqpoint{2.339408in}{1.220065in}}%
\pgfpathlineto{\pgfqpoint{2.369064in}{1.098409in}}%
\pgfpathlineto{\pgfqpoint{2.398721in}{1.148525in}}%
\pgfpathlineto{\pgfqpoint{2.517348in}{1.163923in}}%
\pgfpathlineto{\pgfqpoint{2.547005in}{1.159517in}}%
\pgfpathlineto{\pgfqpoint{2.635975in}{1.266876in}}%
\pgfpathlineto{\pgfqpoint{2.724945in}{1.225777in}}%
\pgfpathlineto{\pgfqpoint{2.754602in}{1.252376in}}%
\pgfpathlineto{\pgfqpoint{2.784259in}{1.092247in}}%
\pgfpathlineto{\pgfqpoint{2.813915in}{1.227469in}}%
\pgfpathlineto{\pgfqpoint{2.873229in}{1.236125in}}%
\pgfpathlineto{\pgfqpoint{2.932542in}{1.236729in}}%
\pgfpathlineto{\pgfqpoint{2.991856in}{1.297072in}}%
\pgfpathlineto{\pgfqpoint{3.021512in}{1.280868in}}%
\pgfpathlineto{\pgfqpoint{3.080826in}{1.329774in}}%
\pgfpathlineto{\pgfqpoint{3.347736in}{1.341833in}}%
\pgfpathlineto{\pgfqpoint{3.377393in}{1.331276in}}%
\pgfpathlineto{\pgfqpoint{3.703617in}{1.489469in}}%
\pgfpathlineto{\pgfqpoint{3.762931in}{1.390597in}}%
\pgfpathlineto{\pgfqpoint{4.385722in}{1.522656in}}%
\pgfpathlineto{\pgfqpoint{4.415379in}{1.466507in}}%
\pgfpathlineto{\pgfqpoint{4.445035in}{1.492637in}}%
\pgfpathlineto{\pgfqpoint{5.038170in}{1.541224in}}%
\pgfpathlineto{\pgfqpoint{5.394051in}{1.660899in}}%
\pgfpathlineto{\pgfqpoint{5.423708in}{1.696374in}}%
\pgfusepath{stroke}%
\end{pgfscope}%
\begin{pgfscope}%
\pgfpathrectangle{\pgfqpoint{0.565935in}{0.521603in}}{\pgfqpoint{5.089095in}{1.685002in}}%
\pgfusepath{clip}%
\pgfsetrectcap%
\pgfsetroundjoin%
\pgfsetlinewidth{1.505625pt}%
\pgfsetstrokecolor{currentstroke4}%
\pgfsetdash{}{0pt}%
\pgfpathmoveto{\pgfqpoint{0.797258in}{0.711052in}}%
\pgfpathlineto{\pgfqpoint{0.945541in}{0.673260in}}%
\pgfpathlineto{\pgfqpoint{1.004855in}{0.719471in}}%
\pgfpathlineto{\pgfqpoint{1.034512in}{0.788368in}}%
\pgfpathlineto{\pgfqpoint{1.123482in}{0.735304in}}%
\pgfpathlineto{\pgfqpoint{1.153138in}{0.739072in}}%
\pgfpathlineto{\pgfqpoint{1.242109in}{0.779443in}}%
\pgfpathlineto{\pgfqpoint{1.271765in}{0.746399in}}%
\pgfpathlineto{\pgfqpoint{1.360736in}{0.850300in}}%
\pgfpathlineto{\pgfqpoint{1.390392in}{0.859647in}}%
\pgfpathlineto{\pgfqpoint{1.420049in}{0.869861in}}%
\pgfpathlineto{\pgfqpoint{1.449706in}{0.822973in}}%
\pgfpathlineto{\pgfqpoint{1.479363in}{0.873677in}}%
\pgfpathlineto{\pgfqpoint{1.509019in}{0.900720in}}%
\pgfpathlineto{\pgfqpoint{1.568333in}{1.391994in}}%
\pgfpathlineto{\pgfqpoint{1.597989in}{0.948776in}}%
\pgfpathlineto{\pgfqpoint{1.627646in}{0.907206in}}%
\pgfpathlineto{\pgfqpoint{1.686960in}{0.925478in}}%
\pgfpathlineto{\pgfqpoint{1.805587in}{0.974218in}}%
\pgfpathlineto{\pgfqpoint{1.835243in}{1.000773in}}%
\pgfpathlineto{\pgfqpoint{1.864900in}{1.013291in}}%
\pgfpathlineto{\pgfqpoint{1.894557in}{1.039471in}}%
\pgfpathlineto{\pgfqpoint{1.924213in}{1.007612in}}%
\pgfpathlineto{\pgfqpoint{1.953870in}{1.057430in}}%
\pgfpathlineto{\pgfqpoint{2.013184in}{1.025940in}}%
\pgfpathlineto{\pgfqpoint{2.042840in}{0.934018in}}%
\pgfpathlineto{\pgfqpoint{2.072497in}{0.852416in}}%
\pgfpathlineto{\pgfqpoint{2.102154in}{1.155011in}}%
\pgfpathlineto{\pgfqpoint{2.131811in}{1.002751in}}%
\pgfpathlineto{\pgfqpoint{2.191124in}{1.096376in}}%
\pgfpathlineto{\pgfqpoint{2.220781in}{1.077073in}}%
\pgfpathlineto{\pgfqpoint{2.250437in}{0.908795in}}%
\pgfpathlineto{\pgfqpoint{2.280094in}{1.147896in}}%
\pgfpathlineto{\pgfqpoint{2.339408in}{0.856580in}}%
\pgfpathlineto{\pgfqpoint{2.369064in}{0.955137in}}%
\pgfpathlineto{\pgfqpoint{2.398721in}{1.143759in}}%
\pgfpathlineto{\pgfqpoint{2.517348in}{1.159962in}}%
\pgfpathlineto{\pgfqpoint{2.547005in}{1.149933in}}%
\pgfpathlineto{\pgfqpoint{2.635975in}{1.255143in}}%
\pgfpathlineto{\pgfqpoint{2.724945in}{1.155921in}}%
\pgfpathlineto{\pgfqpoint{2.754602in}{1.250697in}}%
\pgfpathlineto{\pgfqpoint{2.784259in}{1.049044in}}%
\pgfpathlineto{\pgfqpoint{2.813915in}{1.221046in}}%
\pgfpathlineto{\pgfqpoint{2.873229in}{1.244417in}}%
\pgfpathlineto{\pgfqpoint{2.932542in}{1.244997in}}%
\pgfpathlineto{\pgfqpoint{2.991856in}{1.292576in}}%
\pgfpathlineto{\pgfqpoint{3.021512in}{1.056682in}}%
\pgfpathlineto{\pgfqpoint{3.080826in}{1.334613in}}%
\pgfpathlineto{\pgfqpoint{3.347736in}{1.316127in}}%
\pgfpathlineto{\pgfqpoint{3.377393in}{1.338256in}}%
\pgfpathlineto{\pgfqpoint{3.703617in}{1.464229in}}%
\pgfpathlineto{\pgfqpoint{3.762931in}{1.396600in}}%
\pgfpathlineto{\pgfqpoint{4.385722in}{1.534112in}}%
\pgfpathlineto{\pgfqpoint{4.415379in}{1.476804in}}%
\pgfpathlineto{\pgfqpoint{4.445035in}{1.508826in}}%
\pgfpathlineto{\pgfqpoint{5.038170in}{1.556290in}}%
\pgfpathlineto{\pgfqpoint{5.394051in}{1.659765in}}%
\pgfpathlineto{\pgfqpoint{5.423708in}{1.696137in}}%
\pgfusepath{stroke}%
\end{pgfscope}%
\begin{pgfscope}%
\pgfpathrectangle{\pgfqpoint{0.565935in}{0.521603in}}{\pgfqpoint{5.089095in}{1.685002in}}%
\pgfusepath{clip}%
\pgfsetrectcap%
\pgfsetroundjoin%
\pgfsetlinewidth{1.505625pt}%
\pgfsetstrokecolor{currentstroke5}%
\pgfsetdash{}{0pt}%
\pgfpathmoveto{\pgfqpoint{0.797258in}{0.711052in}}%
\pgfpathlineto{\pgfqpoint{0.945541in}{0.667985in}}%
\pgfpathlineto{\pgfqpoint{1.004855in}{0.719471in}}%
\pgfpathlineto{\pgfqpoint{1.034512in}{0.785438in}}%
\pgfpathlineto{\pgfqpoint{1.123482in}{0.735304in}}%
\pgfpathlineto{\pgfqpoint{1.153138in}{0.740312in}}%
\pgfpathlineto{\pgfqpoint{1.242109in}{0.776376in}}%
\pgfpathlineto{\pgfqpoint{1.271765in}{0.753464in}}%
\pgfpathlineto{\pgfqpoint{1.360736in}{0.850300in}}%
\pgfpathlineto{\pgfqpoint{1.390392in}{0.857608in}}%
\pgfpathlineto{\pgfqpoint{1.420049in}{0.864655in}}%
\pgfpathlineto{\pgfqpoint{1.449706in}{0.830167in}}%
\pgfpathlineto{\pgfqpoint{1.479363in}{0.873677in}}%
\pgfpathlineto{\pgfqpoint{1.509019in}{0.910371in}}%
\pgfpathlineto{\pgfqpoint{1.568333in}{1.278777in}}%
\pgfpathlineto{\pgfqpoint{1.597989in}{0.948129in}}%
\pgfpathlineto{\pgfqpoint{1.627646in}{0.913486in}}%
\pgfpathlineto{\pgfqpoint{1.686960in}{0.942204in}}%
\pgfpathlineto{\pgfqpoint{1.805587in}{0.971931in}}%
\pgfpathlineto{\pgfqpoint{1.835243in}{0.999277in}}%
\pgfpathlineto{\pgfqpoint{1.864900in}{1.015148in}}%
\pgfpathlineto{\pgfqpoint{1.894557in}{1.037003in}}%
\pgfpathlineto{\pgfqpoint{1.924213in}{1.003733in}}%
\pgfpathlineto{\pgfqpoint{1.953870in}{1.058917in}}%
\pgfpathlineto{\pgfqpoint{2.013184in}{1.039471in}}%
\pgfpathlineto{\pgfqpoint{2.042840in}{1.250697in}}%
\pgfpathlineto{\pgfqpoint{2.072497in}{1.005682in}}%
\pgfpathlineto{\pgfqpoint{2.102154in}{1.152258in}}%
\pgfpathlineto{\pgfqpoint{2.131811in}{1.015148in}}%
\pgfpathlineto{\pgfqpoint{2.191124in}{1.133866in}}%
\pgfpathlineto{\pgfqpoint{2.220781in}{1.069261in}}%
\pgfpathlineto{\pgfqpoint{2.250437in}{0.989525in}}%
\pgfpathlineto{\pgfqpoint{2.280094in}{1.144401in}}%
\pgfpathlineto{\pgfqpoint{2.339408in}{1.143276in}}%
\pgfpathlineto{\pgfqpoint{2.369064in}{1.293483in}}%
\pgfpathlineto{\pgfqpoint{2.398721in}{1.148525in}}%
\pgfpathlineto{\pgfqpoint{2.517348in}{1.157277in}}%
\pgfpathlineto{\pgfqpoint{2.547005in}{1.239126in}}%
\pgfpathlineto{\pgfqpoint{2.635975in}{1.295734in}}%
\pgfpathlineto{\pgfqpoint{2.724945in}{1.420455in}}%
\pgfpathlineto{\pgfqpoint{2.754602in}{1.257871in}}%
\pgfpathlineto{\pgfqpoint{2.784259in}{1.147581in}}%
\pgfpathlineto{\pgfqpoint{2.813915in}{1.221698in}}%
\pgfpathlineto{\pgfqpoint{2.873229in}{1.235519in}}%
\pgfpathlineto{\pgfqpoint{2.932542in}{1.239126in}}%
\pgfpathlineto{\pgfqpoint{2.991856in}{1.290749in}}%
\pgfpathlineto{\pgfqpoint{3.021512in}{1.097192in}}%
\pgfpathlineto{\pgfqpoint{3.080826in}{1.327881in}}%
\pgfpathlineto{\pgfqpoint{3.347736in}{1.324814in}}%
\pgfpathlineto{\pgfqpoint{3.377393in}{1.328641in}}%
\pgfpathlineto{\pgfqpoint{3.703617in}{1.475906in}}%
\pgfpathlineto{\pgfqpoint{3.762931in}{1.390597in}}%
\pgfpathlineto{\pgfqpoint{4.385722in}{1.521086in}}%
\pgfpathlineto{\pgfqpoint{4.415379in}{1.478054in}}%
\pgfpathlineto{\pgfqpoint{4.445035in}{1.500738in}}%
\pgfpathlineto{\pgfqpoint{5.038170in}{1.557127in}}%
\pgfpathlineto{\pgfqpoint{5.394051in}{1.661956in}}%
\pgfpathlineto{\pgfqpoint{5.423708in}{1.682368in}}%
\pgfusepath{stroke}%
\end{pgfscope}%
\begin{pgfscope}%
\pgfpathrectangle{\pgfqpoint{0.565935in}{0.521603in}}{\pgfqpoint{5.089095in}{1.685002in}}%
\pgfusepath{clip}%
\pgfsetrectcap%
\pgfsetroundjoin%
\pgfsetlinewidth{1.505625pt}%
\pgfsetstrokecolor{currentstroke6}%
\pgfsetdash{}{0pt}%
\pgfpathmoveto{\pgfqpoint{0.797258in}{0.711052in}}%
\pgfpathlineto{\pgfqpoint{0.945541in}{0.673260in}}%
\pgfpathlineto{\pgfqpoint{1.004855in}{0.711052in}}%
\pgfpathlineto{\pgfqpoint{1.034512in}{0.794102in}}%
\pgfpathlineto{\pgfqpoint{1.123482in}{0.723549in}}%
\pgfpathlineto{\pgfqpoint{1.153138in}{0.751137in}}%
\pgfpathlineto{\pgfqpoint{1.242109in}{0.799675in}}%
\pgfpathlineto{\pgfqpoint{1.271765in}{0.756904in}}%
\pgfpathlineto{\pgfqpoint{1.360736in}{0.850300in}}%
\pgfpathlineto{\pgfqpoint{1.390392in}{0.871460in}}%
\pgfpathlineto{\pgfqpoint{1.420049in}{0.874933in}}%
\pgfpathlineto{\pgfqpoint{1.449706in}{0.827798in}}%
\pgfpathlineto{\pgfqpoint{1.479363in}{0.890574in}}%
\pgfpathlineto{\pgfqpoint{1.509019in}{0.919573in}}%
\pgfpathlineto{\pgfqpoint{1.568333in}{1.560266in}}%
\pgfpathlineto{\pgfqpoint{1.597989in}{0.958858in}}%
\pgfpathlineto{\pgfqpoint{1.627646in}{0.919573in}}%
\pgfpathlineto{\pgfqpoint{1.686960in}{0.907206in}}%
\pgfpathlineto{\pgfqpoint{1.805587in}{0.960083in}}%
\pgfpathlineto{\pgfqpoint{1.835243in}{1.024182in}}%
\pgfpathlineto{\pgfqpoint{1.864900in}{0.995740in}}%
\pgfpathlineto{\pgfqpoint{1.894557in}{1.038652in}}%
\pgfpathlineto{\pgfqpoint{1.924213in}{1.016988in}}%
\pgfpathlineto{\pgfqpoint{1.953870in}{1.058175in}}%
\pgfpathlineto{\pgfqpoint{2.013184in}{1.053662in}}%
\pgfpathlineto{\pgfqpoint{2.042840in}{0.942204in}}%
\pgfpathlineto{\pgfqpoint{2.072497in}{0.872413in}}%
\pgfpathlineto{\pgfqpoint{2.102154in}{1.166950in}}%
\pgfpathlineto{\pgfqpoint{2.131811in}{0.997768in}}%
\pgfpathlineto{\pgfqpoint{2.191124in}{1.089304in}}%
\pgfpathlineto{\pgfqpoint{2.220781in}{1.073894in}}%
\pgfpathlineto{\pgfqpoint{2.250437in}{0.910371in}}%
\pgfpathlineto{\pgfqpoint{2.280094in}{1.156826in}}%
\pgfpathlineto{\pgfqpoint{2.339408in}{0.888831in}}%
\pgfpathlineto{\pgfqpoint{2.369064in}{0.960083in}}%
\pgfpathlineto{\pgfqpoint{2.398721in}{1.139861in}}%
\pgfpathlineto{\pgfqpoint{2.517348in}{1.185764in}}%
\pgfpathlineto{\pgfqpoint{2.547005in}{1.166520in}}%
\pgfpathlineto{\pgfqpoint{2.635975in}{1.261096in}}%
\pgfpathlineto{\pgfqpoint{2.724945in}{1.183803in}}%
\pgfpathlineto{\pgfqpoint{2.754602in}{1.310393in}}%
\pgfpathlineto{\pgfqpoint{2.784259in}{1.055178in}}%
\pgfpathlineto{\pgfqpoint{2.813915in}{1.236729in}}%
\pgfpathlineto{\pgfqpoint{2.873229in}{1.256239in}}%
\pgfpathlineto{\pgfqpoint{2.932542in}{1.217090in}}%
\pgfpathlineto{\pgfqpoint{2.991856in}{1.294837in}}%
\pgfpathlineto{\pgfqpoint{3.021512in}{1.064037in}}%
\pgfpathlineto{\pgfqpoint{3.080826in}{1.372325in}}%
\pgfpathlineto{\pgfqpoint{3.347736in}{1.314912in}}%
\pgfpathlineto{\pgfqpoint{3.377393in}{1.339336in}}%
\pgfpathlineto{\pgfqpoint{3.703617in}{1.476804in}}%
\pgfpathlineto{\pgfqpoint{3.762931in}{1.376534in}}%
\pgfpathlineto{\pgfqpoint{4.385722in}{1.523507in}}%
\pgfpathlineto{\pgfqpoint{4.415379in}{1.451004in}}%
\pgfpathlineto{\pgfqpoint{4.445035in}{1.487442in}}%
\pgfpathlineto{\pgfqpoint{5.038170in}{1.557008in}}%
\pgfpathlineto{\pgfqpoint{5.394051in}{1.681860in}}%
\pgfpathlineto{\pgfqpoint{5.423708in}{1.692498in}}%
\pgfusepath{stroke}%
\end{pgfscope}%
\begin{pgfscope}%
\pgfpathrectangle{\pgfqpoint{0.565935in}{0.521603in}}{\pgfqpoint{5.089095in}{1.685002in}}%
\pgfusepath{clip}%
\pgfsetrectcap%
\pgfsetroundjoin%
\pgfsetlinewidth{1.505625pt}%
\pgfsetstrokecolor{currentstroke7}%
\pgfsetdash{}{0pt}%
\pgfpathmoveto{\pgfqpoint{0.797258in}{0.711052in}}%
\pgfpathlineto{\pgfqpoint{0.945541in}{0.662565in}}%
\pgfpathlineto{\pgfqpoint{1.004855in}{0.702259in}}%
\pgfpathlineto{\pgfqpoint{1.034512in}{0.794102in}}%
\pgfpathlineto{\pgfqpoint{1.123482in}{0.723549in}}%
\pgfpathlineto{\pgfqpoint{1.153138in}{0.742769in}}%
\pgfpathlineto{\pgfqpoint{1.242109in}{0.796908in}}%
\pgfpathlineto{\pgfqpoint{1.271765in}{0.746399in}}%
\pgfpathlineto{\pgfqpoint{1.360736in}{0.846000in}}%
\pgfpathlineto{\pgfqpoint{1.390392in}{0.860659in}}%
\pgfpathlineto{\pgfqpoint{1.420049in}{0.873677in}}%
\pgfpathlineto{\pgfqpoint{1.449706in}{0.827798in}}%
\pgfpathlineto{\pgfqpoint{1.479363in}{0.867275in}}%
\pgfpathlineto{\pgfqpoint{1.509019in}{0.910371in}}%
\pgfpathlineto{\pgfqpoint{1.568333in}{1.433750in}}%
\pgfpathlineto{\pgfqpoint{1.597989in}{0.949422in}}%
\pgfpathlineto{\pgfqpoint{1.627646in}{0.916553in}}%
\pgfpathlineto{\pgfqpoint{1.686960in}{0.910371in}}%
\pgfpathlineto{\pgfqpoint{1.805587in}{0.939513in}}%
\pgfpathlineto{\pgfqpoint{1.835243in}{1.007612in}}%
\pgfpathlineto{\pgfqpoint{1.864900in}{0.974218in}}%
\pgfpathlineto{\pgfqpoint{1.894557in}{1.030269in}}%
\pgfpathlineto{\pgfqpoint{1.924213in}{0.993690in}}%
\pgfpathlineto{\pgfqpoint{1.953870in}{1.055932in}}%
\pgfpathlineto{\pgfqpoint{2.013184in}{1.036174in}}%
\pgfpathlineto{\pgfqpoint{2.042840in}{0.939513in}}%
\pgfpathlineto{\pgfqpoint{2.072497in}{0.864655in}}%
\pgfpathlineto{\pgfqpoint{2.102154in}{1.181823in}}%
\pgfpathlineto{\pgfqpoint{2.131811in}{0.984193in}}%
\pgfpathlineto{\pgfqpoint{2.191124in}{1.080203in}}%
\pgfpathlineto{\pgfqpoint{2.220781in}{1.070196in}}%
\pgfpathlineto{\pgfqpoint{2.250437in}{1.384834in}}%
\pgfpathlineto{\pgfqpoint{2.280094in}{1.166807in}}%
\pgfpathlineto{\pgfqpoint{2.339408in}{1.311636in}}%
\pgfpathlineto{\pgfqpoint{2.369064in}{0.957625in}}%
\pgfpathlineto{\pgfqpoint{2.398721in}{1.138874in}}%
\pgfpathlineto{\pgfqpoint{2.517348in}{1.162612in}}%
\pgfpathlineto{\pgfqpoint{2.547005in}{1.148996in}}%
\pgfpathlineto{\pgfqpoint{2.635975in}{1.241495in}}%
\pgfpathlineto{\pgfqpoint{2.724945in}{1.426245in}}%
\pgfpathlineto{\pgfqpoint{2.754602in}{1.287043in}}%
\pgfpathlineto{\pgfqpoint{2.784259in}{1.045904in}}%
\pgfpathlineto{\pgfqpoint{2.813915in}{1.229981in}}%
\pgfpathlineto{\pgfqpoint{2.873229in}{1.248437in}}%
\pgfpathlineto{\pgfqpoint{2.932542in}{1.212372in}}%
\pgfpathlineto{\pgfqpoint{2.991856in}{1.283742in}}%
\pgfpathlineto{\pgfqpoint{3.021512in}{2.130014in}}%
\pgfpathlineto{\pgfqpoint{3.080826in}{1.333507in}}%
\pgfpathlineto{\pgfqpoint{3.347736in}{1.294837in}}%
\pgfpathlineto{\pgfqpoint{3.377393in}{1.320123in}}%
\pgfpathlineto{\pgfqpoint{3.703617in}{1.469690in}}%
\pgfpathlineto{\pgfqpoint{3.762931in}{1.387244in}}%
\pgfpathlineto{\pgfqpoint{4.385722in}{1.510948in}}%
\pgfpathlineto{\pgfqpoint{4.415379in}{1.456042in}}%
\pgfpathlineto{\pgfqpoint{4.445035in}{1.477876in}}%
\pgfpathlineto{\pgfqpoint{5.038170in}{1.535449in}}%
\pgfpathlineto{\pgfqpoint{5.394051in}{1.657191in}}%
\pgfpathlineto{\pgfqpoint{5.423708in}{1.695605in}}%
\pgfusepath{stroke}%
\end{pgfscope}%
\begin{pgfscope}%
\pgfsetrectcap%
\pgfsetmiterjoin%
\pgfsetlinewidth{0.803000pt}%
\definecolor{currentstroke}{rgb}{0.000000,0.000000,0.000000}%
\pgfsetstrokecolor{currentstroke}%
\pgfsetdash{}{0pt}%
\pgfpathmoveto{\pgfqpoint{0.565935in}{0.521603in}}%
\pgfpathlineto{\pgfqpoint{0.565935in}{2.206605in}}%
\pgfusepath{stroke}%
\end{pgfscope}%
\begin{pgfscope}%
\pgfsetrectcap%
\pgfsetmiterjoin%
\pgfsetlinewidth{0.803000pt}%
\definecolor{currentstroke}{rgb}{0.000000,0.000000,0.000000}%
\pgfsetstrokecolor{currentstroke}%
\pgfsetdash{}{0pt}%
\pgfpathmoveto{\pgfqpoint{5.655030in}{0.521603in}}%
\pgfpathlineto{\pgfqpoint{5.655030in}{2.206605in}}%
\pgfusepath{stroke}%
\end{pgfscope}%
\begin{pgfscope}%
\pgfsetrectcap%
\pgfsetmiterjoin%
\pgfsetlinewidth{0.803000pt}%
\definecolor{currentstroke}{rgb}{0.000000,0.000000,0.000000}%
\pgfsetstrokecolor{currentstroke}%
\pgfsetdash{}{0pt}%
\pgfpathmoveto{\pgfqpoint{0.565935in}{0.521603in}}%
\pgfpathlineto{\pgfqpoint{5.655030in}{0.521603in}}%
\pgfusepath{stroke}%
\end{pgfscope}%
\begin{pgfscope}%
\pgfsetrectcap%
\pgfsetmiterjoin%
\pgfsetlinewidth{0.803000pt}%
\definecolor{currentstroke}{rgb}{0.000000,0.000000,0.000000}%
\pgfsetstrokecolor{currentstroke}%
\pgfsetdash{}{0pt}%
\pgfpathmoveto{\pgfqpoint{0.565935in}{2.206605in}}%
\pgfpathlineto{\pgfqpoint{5.655030in}{2.206605in}}%
\pgfusepath{stroke}%
\end{pgfscope}%
\begin{pgfscope}%
\pgfsetbuttcap%
\pgfsetmiterjoin%
\definecolor{currentfill}{rgb}{1.000000,1.000000,1.000000}%
\pgfsetfillcolor{currentfill}%
\pgfsetfillopacity{0.800000}%
\pgfsetlinewidth{1.003750pt}%
\definecolor{currentstroke}{rgb}{0.800000,0.800000,0.800000}%
\pgfsetstrokecolor{currentstroke}%
\pgfsetstrokeopacity{0.800000}%
\pgfsetdash{}{0pt}%
\pgfpathmoveto{\pgfqpoint{5.742530in}{0.808389in}}%
\pgfpathlineto{\pgfqpoint{8.426893in}{0.808389in}}%
\pgfpathquadraticcurveto{\pgfqpoint{8.451893in}{0.808389in}}{\pgfqpoint{8.451893in}{0.833389in}}%
\pgfpathlineto{\pgfqpoint{8.451893in}{2.119105in}}%
\pgfpathquadraticcurveto{\pgfqpoint{8.451893in}{2.144105in}}{\pgfqpoint{8.426893in}{2.144105in}}%
\pgfpathlineto{\pgfqpoint{5.742530in}{2.144105in}}%
\pgfpathquadraticcurveto{\pgfqpoint{5.717530in}{2.144105in}}{\pgfqpoint{5.717530in}{2.119105in}}%
\pgfpathlineto{\pgfqpoint{5.717530in}{0.833389in}}%
\pgfpathquadraticcurveto{\pgfqpoint{5.717530in}{0.808389in}}{\pgfqpoint{5.742530in}{0.808389in}}%
\pgfpathlineto{\pgfqpoint{5.742530in}{0.808389in}}%
\pgfpathclose%
\pgfusepath{stroke,fill}%
\end{pgfscope}%
\begin{pgfscope}%
\pgfsetrectcap%
\pgfsetroundjoin%
\pgfsetlinewidth{1.505625pt}%
\pgfsetstrokecolor{currentstroke1}%
\pgfsetdash{}{0pt}%
\pgfpathmoveto{\pgfqpoint{5.767530in}{2.042884in}}%
\pgfpathlineto{\pgfqpoint{5.892530in}{2.042884in}}%
\pgfpathlineto{\pgfqpoint{6.017530in}{2.042884in}}%
\pgfusepath{stroke}%
\end{pgfscope}%
\begin{pgfscope}%
\definecolor{textcolor}{rgb}{0.000000,0.000000,0.000000}%
\pgfsetstrokecolor{textcolor}%
\pgfsetfillcolor{textcolor}%
\pgftext[x=6.117530in,y=1.999134in,left,base]{\color{textcolor}{\rmfamily\fontsize{9.000000}{10.800000}\selectfont\catcode`\^=\active\def^{\ifmmode\sp\else\^{}\fi}\catcode`\%=\active\def%{\%}\NaiveCycles{}}}%
\end{pgfscope}%
\begin{pgfscope}%
\pgfsetrectcap%
\pgfsetroundjoin%
\pgfsetlinewidth{1.505625pt}%
\pgfsetstrokecolor{currentstroke2}%
\pgfsetdash{}{0pt}%
\pgfpathmoveto{\pgfqpoint{5.767530in}{1.859413in}}%
\pgfpathlineto{\pgfqpoint{5.892530in}{1.859413in}}%
\pgfpathlineto{\pgfqpoint{6.017530in}{1.859413in}}%
\pgfusepath{stroke}%
\end{pgfscope}%
\begin{pgfscope}%
\definecolor{textcolor}{rgb}{0.000000,0.000000,0.000000}%
\pgfsetstrokecolor{textcolor}%
\pgfsetfillcolor{textcolor}%
\pgftext[x=6.117530in,y=1.815663in,left,base]{\color{textcolor}{\rmfamily\fontsize{9.000000}{10.800000}\selectfont\catcode`\^=\active\def^{\ifmmode\sp\else\^{}\fi}\catcode`\%=\active\def%{\%}\Neighbors{} \& \MergeLinear{}}}%
\end{pgfscope}%
\begin{pgfscope}%
\pgfsetrectcap%
\pgfsetroundjoin%
\pgfsetlinewidth{1.505625pt}%
\pgfsetstrokecolor{currentstroke3}%
\pgfsetdash{}{0pt}%
\pgfpathmoveto{\pgfqpoint{5.767530in}{1.675941in}}%
\pgfpathlineto{\pgfqpoint{5.892530in}{1.675941in}}%
\pgfpathlineto{\pgfqpoint{6.017530in}{1.675941in}}%
\pgfusepath{stroke}%
\end{pgfscope}%
\begin{pgfscope}%
\definecolor{textcolor}{rgb}{0.000000,0.000000,0.000000}%
\pgfsetstrokecolor{textcolor}%
\pgfsetfillcolor{textcolor}%
\pgftext[x=6.117530in,y=1.632191in,left,base]{\color{textcolor}{\rmfamily\fontsize{9.000000}{10.800000}\selectfont\catcode`\^=\active\def^{\ifmmode\sp\else\^{}\fi}\catcode`\%=\active\def%{\%}\Neighbors{} \& \SharedVertices{}}}%
\end{pgfscope}%
\begin{pgfscope}%
\pgfsetrectcap%
\pgfsetroundjoin%
\pgfsetlinewidth{1.505625pt}%
\pgfsetstrokecolor{currentstroke4}%
\pgfsetdash{}{0pt}%
\pgfpathmoveto{\pgfqpoint{5.767530in}{1.488991in}}%
\pgfpathlineto{\pgfqpoint{5.892530in}{1.488991in}}%
\pgfpathlineto{\pgfqpoint{6.017530in}{1.488991in}}%
\pgfusepath{stroke}%
\end{pgfscope}%
\begin{pgfscope}%
\definecolor{textcolor}{rgb}{0.000000,0.000000,0.000000}%
\pgfsetstrokecolor{textcolor}%
\pgfsetfillcolor{textcolor}%
\pgftext[x=6.117530in,y=1.445241in,left,base]{\color{textcolor}{\rmfamily\fontsize{9.000000}{10.800000}\selectfont\catcode`\^=\active\def^{\ifmmode\sp\else\^{}\fi}\catcode`\%=\active\def%{\%}\NeighborsDegree{} \& \MergeLinear{}}}%
\end{pgfscope}%
\begin{pgfscope}%
\pgfsetrectcap%
\pgfsetroundjoin%
\pgfsetlinewidth{1.505625pt}%
\pgfsetstrokecolor{currentstroke5}%
\pgfsetdash{}{0pt}%
\pgfpathmoveto{\pgfqpoint{5.767530in}{1.302040in}}%
\pgfpathlineto{\pgfqpoint{5.892530in}{1.302040in}}%
\pgfpathlineto{\pgfqpoint{6.017530in}{1.302040in}}%
\pgfusepath{stroke}%
\end{pgfscope}%
\begin{pgfscope}%
\definecolor{textcolor}{rgb}{0.000000,0.000000,0.000000}%
\pgfsetstrokecolor{textcolor}%
\pgfsetfillcolor{textcolor}%
\pgftext[x=6.117530in,y=1.258290in,left,base]{\color{textcolor}{\rmfamily\fontsize{9.000000}{10.800000}\selectfont\catcode`\^=\active\def^{\ifmmode\sp\else\^{}\fi}\catcode`\%=\active\def%{\%}\NeighborsDegree{} \& \SharedVertices{}}}%
\end{pgfscope}%
\begin{pgfscope}%
\pgfsetrectcap%
\pgfsetroundjoin%
\pgfsetlinewidth{1.505625pt}%
\pgfsetstrokecolor{currentstroke6}%
\pgfsetdash{}{0pt}%
\pgfpathmoveto{\pgfqpoint{5.767530in}{1.115090in}}%
\pgfpathlineto{\pgfqpoint{5.892530in}{1.115090in}}%
\pgfpathlineto{\pgfqpoint{6.017530in}{1.115090in}}%
\pgfusepath{stroke}%
\end{pgfscope}%
\begin{pgfscope}%
\definecolor{textcolor}{rgb}{0.000000,0.000000,0.000000}%
\pgfsetstrokecolor{textcolor}%
\pgfsetfillcolor{textcolor}%
\pgftext[x=6.117530in,y=1.071340in,left,base]{\color{textcolor}{\rmfamily\fontsize{9.000000}{10.800000}\selectfont\catcode`\^=\active\def^{\ifmmode\sp\else\^{}\fi}\catcode`\%=\active\def%{\%}\None{} \& \MergeLinear{}}}%
\end{pgfscope}%
\begin{pgfscope}%
\pgfsetrectcap%
\pgfsetroundjoin%
\pgfsetlinewidth{1.505625pt}%
\pgfsetstrokecolor{currentstroke7}%
\pgfsetdash{}{0pt}%
\pgfpathmoveto{\pgfqpoint{5.767530in}{0.931619in}}%
\pgfpathlineto{\pgfqpoint{5.892530in}{0.931619in}}%
\pgfpathlineto{\pgfqpoint{6.017530in}{0.931619in}}%
\pgfusepath{stroke}%
\end{pgfscope}%
\begin{pgfscope}%
\definecolor{textcolor}{rgb}{0.000000,0.000000,0.000000}%
\pgfsetstrokecolor{textcolor}%
\pgfsetfillcolor{textcolor}%
\pgftext[x=6.117530in,y=0.887869in,left,base]{\color{textcolor}{\rmfamily\fontsize{9.000000}{10.800000}\selectfont\catcode`\^=\active\def^{\ifmmode\sp\else\^{}\fi}\catcode`\%=\active\def%{\%}\None{} \& \SharedVertices{}}}%
\end{pgfscope}%
\end{pgfpicture}%
\makeatother%
\endgroup%
}
	\caption[Running time on graphs without NAC-colorings.]{
		Mean running time (ms) on graphs without NAC-colorings.}%
	\label{fig:graph_time_no_nac_coloring}
\end{figure}


\subsubsection{Failing strategies}%
\label{sec:failing_strategies}

In this section we show performance of other strategies described in \Cref{chapter:alg}.
We do not show these strategies in previous graphs as they would influence
the scale and would make graphs and legends unreadable.

First smart split described in \Cref{sec:smart_split}
did not improve the runtime.
We expected minor performance hit for smaller graphs because heuristic is run
multiple times, but gains for larger graphs where subgraphs merging order
should join subgraphs near to each other together. This is not the case.

% Smart split
% \KernighanLin{}
% \Cuts{}

% \Log{}
% \MinMax{}
% \PromissingCycles{}

\todo[inline]{Run smart split with log}

\subsubsection{Final comparison}

Based on our benchmarks presented in the previous sections,
we choose algorithms that should be preserved and merged into PyRigi.
For graph classes with a lot of NAC-colorings,
\NaiveCycles{} is usually the best choice
when we search for a single NAC-coloring.
The user of the library has to pay attention while using this strategy
as if there is no NAC-coloring and the graph does not trivially collapse
into a few monochromatic classes, the runtime will be huge.
For these cases and cases where we search for all NAC-colorings,
we preserve the \NeighborsDegree{} strategy and both the merging strategies.
\MergeLinear{} performs better in general case and
\SharedVertices{} performed better for graphs with no NAC-colorings.

