\chapter{Implementation \& Benchmarks}%
\label{chapter:benchmarks}

\begin{chapterabstract}

	In this chapter, we first describe the structure of the project
	and discuss some design choices.
	After that, we evaluate performance of the algorithms
	proposed in \Cref{chapter:alg}.
	First, we compare the high-level approaches with previous approaches and among each other,
	and then we compare heuristics with each other
	for different  graph classes.
	We show reduction both in runtime and in the number
	of \IsNACColoring{} checks performed.
	Lastly, we evaluate which strategies should be preserved.

\end{chapterabstract}

\section{Implementation}

\todo[inline]{Popsat základní API knihovny}

In this section we first describe the structure of the project containing
the code of the algorithm.
Next, we mention libraries, relation to PyRigi and
some worth mentioning implementation details.
The code is available as an attachment of this thesis,
nevertheless the most recent version can be found on GitHub~\cite{my_code}.

The code is written in Python~\cite{python}, minimal supported version is Python 3.12.
To set up the project, create a virtual environment and install packages
from \texttt{requirements.txt}. On NixOS, \texttt{shell.nix} can be used.
See \texttt{README.md} for additional instructions.
We go through the main folders and files of the project,
to see the code structure in~\Cref{chapter:attachments}.

In \texttt{graphs\_store} we store datasets used for benchmarking.
Graphs are either obtained from~\cite{extremal_graphs},
generated using Nauty~\cite{nauty} with a plugin~\cite{nauty_plugin}
or generated using NetworkX~\cite{networkx} and checks from PyRigi~\cite{pyrigi}.
Graph are mostly stored in Graph6 format~\cite{graph6}.
Code for reading graphs from the store can be found in \texttt{benchmarks/dataset.py},
code for generating some graph classes can be found in  \texttt{benchmarks/generators.py}.
In the \texttt{benchmark/precomputed} directory, there are all result of the benchmarks that
we use for algorithms evaluation.
Individual runs are stored in a compressed CSV file.
%
The base directory contains tooling for running, visualizing and exporting benchmarks.
File \texttt{NAC\_playground.ipynb} presents a simple case
to show how the algorithm's API can be used.
File \texttt{NAC\_presentation.ipynb} shows how the benchmarks can be run and analyzed.

The code of the algorithm described in~\Cref{sec:stable_cuts_implementation}
with additional helper functions is implemented in directory \texttt{stablecut}.
Note that some changes were done when the code was merged into PyRigi.

The code of the algorithms described in~\Cref{chapter:alg}
is stored in directory \texttt{nac}.
%
Directory \texttt{nac/util} stores helper functions and classes
like an implementation of the \textsc{UnionFind} data structure.
%
File \texttt{check.py} implements \IsNACColoring{} check.
%
File \texttt{monochromatic\_classes.py} is used to find \trcon{} components
and monochromatic classes in a graph. With this, we can compare performance
between using monochromatic classes, \trcon{} components or just edges.
%
File \texttt{cycle\_detection.py} holds algorithms for finding cycles
used by \Cref{sec:small_cycles}
and some heuristics.
%
In \Cref{sec:polynomial_optimizations}
we presented checks that can in polynomial time
sometimes find a NAC-coloring or determine that there is none.
These checks are implemented in \texttt{existence.py} and
used mostly from \texttt{single.py} that is the entry-point
for a single NAC-coloring search.
%
General NAC-coloring searching is implemented in \texttt{search.py}
along with parameter parsing, graph vertices normalization and
optimizations like search for articulation vertices.
After that, the correct algorithm from \Naive{}, \NaiveCycles{} or \Subgraphs{}
is chosen and called.
%
These algorithms are implemented in \texttt{algorithms.py} alongside many helper functions.
Heuristics for \Subgraphs{} algorithm are stored in \texttt{strategies.py}.
%
Tests of both stable cuts and NAC-coloring parts are stored in directory \texttt{test}.

Common function parameters are:
\texttt{graph} repressing the subgraph where NAC-colorings should be found.
%
\texttt{comp\_graph} is a graph where vertices are some integer IDs of monochromatic classes
and edges exists if the classes are neighboring,
see \Cref{observ:monochromatic_classes_graph}.
%
Monochromatic class IDs also serve as indices into \texttt{component\_to\_edges}
that maps an ID of a monochromatic class to its edges.
%
NAC-colorings are represented as bit-masks where bit's offset correspond to a class ID\@.

As \IsNACColoring{} is a core component of all our algorithms,
we tried to optimized it well.
%
In the base implementation of \IsNACColoring{},
subgraphs from \( \red \) and \( \blue \) edges are created.
To create such subgraphs in code, edges can be added to an empty graph
using NetworkX's function \textsc{add\_edges\_from}.
%
This is rather slow as creating new vertices in the empty graph causes noticeable overhead.
Therefore, we create a graph with no edges and the same vertices as the original graph,
cache it and reuse it for the checks.
Every time only edges are added, the check is run, and the edges are cleared.
By doing this, the performance of \IsNACColoring{} is increased by roughly 40\%.
%
Another way how the performance could be increased is by reserving space in lists
when the final size is known.
To our knowledge, this is impossible in Python.

The code uses \textsc{Graph} class and related algorithms from NetworkX~\cite{networkx}
as the base of many operations. We use some utility functions from PyRigi~\cite{pyrigi}
related to (global) rigidity tests and rigidity components search.
Other than that, the code is not dependent on PyRigi.
%
Pytest~\cite{pytest} is used for testing and
Matplotlib~\cite{matplotlib} for visualizations.


\section{Benchmarks}

In this section we first set meaningful targets for our benchmarks,
then we compare the performance of our algorithms with the previous implementation
and show running time and internal search optimizations for various graph classes.

The main question regarding NAC-coloring search is whether a graph has a NAC-coloring.
We usually ask the algorithm to not only answer yes, but to also provide a certificate.
%
For flexible graphs, it is often algorithmically quite simple to find a NAC-coloring,
so this question is more interesting for rigid graphs.
%
For flexible graphs, it is more interesting to ask for the number of NAC-colorings
of a graph.
Note, that for larger flexible graphs with around thirty vertices
the number of NAC-colorings is huge as it often grows exponentially.
This slows our algorithm down as just materializing and iterating exponential
number of NAC-colorings takes exponential time.
%
For such cases, the FPT algorithm described in \Cref{chapter:fpt}
could be a better fit as it does not materialize any NAC-colorings.

The graphs the algorithm works with have integer vertices (we relabel the otherwise).
We noticed that for synthetically generated graphs,
the algorithm performs slightly better compared to
the same algorithm run on the same graph with vertices randomly relabeled.
To counteract this, we tried to relabel the graphs using BFS,
but we reached no performance gains compared to the random relabeling.

Visualizations in the following sections were created by grouping data per dataset,
graph size and some specified attribute --- usually the strategies used.
%
We show mean of the running time in milliseconds or
the number of \IsNACColoring{} calls.
%
Graphs with median and third quartile can be seen in \texttt{NAC\_presentation.ipynb}.


\subsection{Improvement over previous solutions}

The benchmarks comparing our algorithm with the previous implementations
were run on Linux on a laptop with Intel i7 of the 11th generation
with CPython 3.12~\cite{cpython} and SageMath 10.4~\cite{sagemath}.
The remaining benchmarks were run a laptop with Intel i5 of the 6th generation
using CPython 3.12.
On more modern hardware, the running times could be significantly shorter.

\Cref{tab:all_min_rigid}
shows the time required for finding all the NAC-colorings
of all minimally rigid graphs with given vertex count.
%
They are generated using Nauty~\cite{nauty}
with a corresponding plugin~\cite{nauty_plugin}.
%
We show results of the implementation
in \flexrilog{}~\cite{flexrilog} using \trcon{} components
run in SageMath~\cite{sagemath}
and compare them to our implementation of the same \Naive{} algorithm
using $\triangle$-connected components
and monochromatic classes as described in \Cref{sec:NACvalid}.
Next column shows \NaiveCycles{} from \Cref{sec:small_cycles}
using monochromatic classes.
The last column is for the \NeighborsDegree{} (each initial subgraph has $k=4$ monochromatic classes)
with \MergeLinear{} merging strategy.
%
In every case, our algorithms are significantly faster than implementation in \flexrilog{}~\cite{flexrilog}.
Notice also huge advantage gained by using monochromatic classes instead of \trcon{} components.
%
\begin{table}[ht]
	\caption[Running times on graphs]{
		The time (in seconds) needed to find all NAC-colorings for all graphs with a given size. Run by us.
		\textsc{FRLG} stands for \flexrilog{}, \textsc{ND} for \NeighborsDegree{}.}%
	\label{tab:all_min_rigid}
	\vspace{0.3cm}
	\centering
	\begin{tabular}{ccccccc}
		\hline
		\,$|V(G)|$\, & \,\#graphs\, & \,FRLG\, & \,$\triangle$-comps.\, & \,monochr.\, & \,cycles\, & \,\textsc{ND}\, \\
		\hline
		% 5        & 3           & 0.007 s      & 0.002 s            & 0.001 s       & 0.001 s & 0.002 s          \\
		% 6        & 13          & 0.063 s      & 0.030 s            & 0.010 s       & 0.005 s & 0.007 s          \\
		% 7        & 70          & 0.57 s       & 0.052 s            & 0.047 s       & 0.029 s & 0.041 s          \\
		8            & 608          & 14       & 1.09                   & 0.97         & 0.36       & 0.49            \\
		9            & 7\,222       & 509      & 34                     & 29           & 5.8        & 8.6             \\
		10           & 110\,132     & 27k      & 1\,725                 & 1\,446       & 151        & 213             \\
		11           & 2\,039\,273  & -        & -                      & -            & 5\,440     & 6\,650          \\
		\hline
	\end{tabular}
\end{table}

\Cref{fig:graph_time_minimally_rigid}
shows timings to compute all NAC-colorings of minimally rigid graphs
depending on the strategy used.
We did not list all NAC-coloring for all minimally rigid graphs with more than twelve vertices
as there is too many such graphs (around 44 millions for twelve vertices).
%
We rather randomly generated dataset of minimally rigid graphs
using NetworkX~\cite{networkx} and PyRigi~\cite{pyrigi}.
%
It can be seen that for graphs up to around fourteen vertices the \NaiveCycles{} algorithm
is still faster than \Subgraphs{}.
For graphs with more than eighteen vertices,
the growing advantage of \Subgraphs{} is already clear.
\Subgraphs{} can list all NAC-colorings on 30 monochromatic class graph in a few seconds.
This corresponds to \( 2^{29} \) checks done by \NaiveCycles{}.

\begin{figure}[ht]
	\centering
	\scalebox{\BenchFigureScale}{%% Creator: Matplotlib, PGF backend
%%
%% To include the figure in your LaTeX document, write
%%   \input{<filename>.pgf}
%%
%% Make sure the required packages are loaded in your preamble
%%   \usepackage{pgf}
%%
%% Also ensure that all the required font packages are loaded; for instance,
%% the lmodern package is sometimes necessary when using math font.
%%   \usepackage{lmodern}
%%
%% Figures using additional raster images can only be included by \input if
%% they are in the same directory as the main LaTeX file. For loading figures
%% from other directories you can use the `import` package
%%   \usepackage{import}
%%
%% and then include the figures with
%%   \import{<path to file>}{<filename>.pgf}
%%
%% Matplotlib used the following preamble
%%   \def\mathdefault#1{#1}
%%   \everymath=\expandafter{\the\everymath\displaystyle}
%%   \IfFileExists{scrextend.sty}{
%%     \usepackage[fontsize=10.000000pt]{scrextend}
%%   }{
%%     \renewcommand{\normalsize}{\fontsize{10.000000}{12.000000}\selectfont}
%%     \normalsize
%%   }
%%   
%%   \ifdefined\pdftexversion\else  % non-pdftex case.
%%     \usepackage{fontspec}
%%     \setmainfont{DejaVuSans.ttf}[Path=\detokenize{/home/petr/Projects/PyRigi/.venv/lib/python3.12/site-packages/matplotlib/mpl-data/fonts/ttf/}]
%%     \setsansfont{DejaVuSans.ttf}[Path=\detokenize{/home/petr/Projects/PyRigi/.venv/lib/python3.12/site-packages/matplotlib/mpl-data/fonts/ttf/}]
%%     \setmonofont{DejaVuSansMono.ttf}[Path=\detokenize{/home/petr/Projects/PyRigi/.venv/lib/python3.12/site-packages/matplotlib/mpl-data/fonts/ttf/}]
%%   \fi
%%   \makeatletter\@ifpackageloaded{under\Score{}}{}{\usepackage[strings]{under\Score{}}}\makeatother
%%
\begingroup%
\makeatletter%
\begin{pgfpicture}%
\pgfpathrectangle{\pgfpointorigin}{\pgfqpoint{8.384376in}{2.841849in}}%
\pgfusepath{use as bounding box, clip}%
\begin{pgfscope}%
\pgfsetbuttcap%
\pgfsetmiterjoin%
\definecolor{currentfill}{rgb}{1.000000,1.000000,1.000000}%
\pgfsetfillcolor{currentfill}%
\pgfsetlinewidth{0.000000pt}%
\definecolor{currentstroke}{rgb}{1.000000,1.000000,1.000000}%
\pgfsetstrokecolor{currentstroke}%
\pgfsetdash{}{0pt}%
\pgfpathmoveto{\pgfqpoint{0.000000in}{0.000000in}}%
\pgfpathlineto{\pgfqpoint{8.384376in}{0.000000in}}%
\pgfpathlineto{\pgfqpoint{8.384376in}{2.841849in}}%
\pgfpathlineto{\pgfqpoint{0.000000in}{2.841849in}}%
\pgfpathlineto{\pgfqpoint{0.000000in}{0.000000in}}%
\pgfpathclose%
\pgfusepath{fill}%
\end{pgfscope}%
\begin{pgfscope}%
\pgfsetbuttcap%
\pgfsetmiterjoin%
\definecolor{currentfill}{rgb}{1.000000,1.000000,1.000000}%
\pgfsetfillcolor{currentfill}%
\pgfsetlinewidth{0.000000pt}%
\definecolor{currentstroke}{rgb}{0.000000,0.000000,0.000000}%
\pgfsetstrokecolor{currentstroke}%
\pgfsetstrokeopacity{0.000000}%
\pgfsetdash{}{0pt}%
\pgfpathmoveto{\pgfqpoint{0.588387in}{0.521603in}}%
\pgfpathlineto{\pgfqpoint{4.248423in}{0.521603in}}%
\pgfpathlineto{\pgfqpoint{4.248423in}{2.741849in}}%
\pgfpathlineto{\pgfqpoint{0.588387in}{2.741849in}}%
\pgfpathlineto{\pgfqpoint{0.588387in}{0.521603in}}%
\pgfpathclose%
\pgfusepath{fill}%
\end{pgfscope}%
\begin{pgfscope}%
\pgfsetbuttcap%
\pgfsetroundjoin%
\definecolor{currentfill}{rgb}{0.000000,0.000000,0.000000}%
\pgfsetfillcolor{currentfill}%
\pgfsetlinewidth{0.803000pt}%
\definecolor{currentstroke}{rgb}{0.000000,0.000000,0.000000}%
\pgfsetstrokecolor{currentstroke}%
\pgfsetdash{}{0pt}%
\pgfsys@defobject{currentmarker}{\pgfqpoint{0.000000in}{-0.048611in}}{\pgfqpoint{0.000000in}{0.000000in}}{%
\pgfpathmoveto{\pgfqpoint{0.000000in}{0.000000in}}%
\pgfpathlineto{\pgfqpoint{0.000000in}{-0.048611in}}%
\pgfusepath{stroke,fill}%
}%
\begin{pgfscope}%
\pgfsys@transformshift{0.969417in}{0.521603in}%
\pgfsys@useobject{currentmarker}{}%
\end{pgfscope}%
\end{pgfscope}%
\begin{pgfscope}%
\definecolor{textcolor}{rgb}{0.000000,0.000000,0.000000}%
\pgfsetstrokecolor{textcolor}%
\pgfsetfillcolor{textcolor}%
\pgftext[x=0.969417in,y=0.424381in,,top]{\color{textcolor}{\rmfamily\fontsize{10.000000}{12.000000}\selectfont\catcode`\^=\active\def^{\ifmmode\sp\else\^{}\fi}\catcode`\%=\active\def%{\%}$\mathdefault{4}$}}%
\end{pgfscope}%
\begin{pgfscope}%
\pgfsetbuttcap%
\pgfsetroundjoin%
\definecolor{currentfill}{rgb}{0.000000,0.000000,0.000000}%
\pgfsetfillcolor{currentfill}%
\pgfsetlinewidth{0.803000pt}%
\definecolor{currentstroke}{rgb}{0.000000,0.000000,0.000000}%
\pgfsetstrokecolor{currentstroke}%
\pgfsetdash{}{0pt}%
\pgfsys@defobject{currentmarker}{\pgfqpoint{0.000000in}{-0.048611in}}{\pgfqpoint{0.000000in}{0.000000in}}{%
\pgfpathmoveto{\pgfqpoint{0.000000in}{0.000000in}}%
\pgfpathlineto{\pgfqpoint{0.000000in}{-0.048611in}}%
\pgfusepath{stroke,fill}%
}%
\begin{pgfscope}%
\pgfsys@transformshift{1.398747in}{0.521603in}%
\pgfsys@useobject{currentmarker}{}%
\end{pgfscope}%
\end{pgfscope}%
\begin{pgfscope}%
\definecolor{textcolor}{rgb}{0.000000,0.000000,0.000000}%
\pgfsetstrokecolor{textcolor}%
\pgfsetfillcolor{textcolor}%
\pgftext[x=1.398747in,y=0.424381in,,top]{\color{textcolor}{\rmfamily\fontsize{10.000000}{12.000000}\selectfont\catcode`\^=\active\def^{\ifmmode\sp\else\^{}\fi}\catcode`\%=\active\def%{\%}$\mathdefault{8}$}}%
\end{pgfscope}%
\begin{pgfscope}%
\pgfsetbuttcap%
\pgfsetroundjoin%
\definecolor{currentfill}{rgb}{0.000000,0.000000,0.000000}%
\pgfsetfillcolor{currentfill}%
\pgfsetlinewidth{0.803000pt}%
\definecolor{currentstroke}{rgb}{0.000000,0.000000,0.000000}%
\pgfsetstrokecolor{currentstroke}%
\pgfsetdash{}{0pt}%
\pgfsys@defobject{currentmarker}{\pgfqpoint{0.000000in}{-0.048611in}}{\pgfqpoint{0.000000in}{0.000000in}}{%
\pgfpathmoveto{\pgfqpoint{0.000000in}{0.000000in}}%
\pgfpathlineto{\pgfqpoint{0.000000in}{-0.048611in}}%
\pgfusepath{stroke,fill}%
}%
\begin{pgfscope}%
\pgfsys@transformshift{1.828077in}{0.521603in}%
\pgfsys@useobject{currentmarker}{}%
\end{pgfscope}%
\end{pgfscope}%
\begin{pgfscope}%
\definecolor{textcolor}{rgb}{0.000000,0.000000,0.000000}%
\pgfsetstrokecolor{textcolor}%
\pgfsetfillcolor{textcolor}%
\pgftext[x=1.828077in,y=0.424381in,,top]{\color{textcolor}{\rmfamily\fontsize{10.000000}{12.000000}\selectfont\catcode`\^=\active\def^{\ifmmode\sp\else\^{}\fi}\catcode`\%=\active\def%{\%}$\mathdefault{12}$}}%
\end{pgfscope}%
\begin{pgfscope}%
\pgfsetbuttcap%
\pgfsetroundjoin%
\definecolor{currentfill}{rgb}{0.000000,0.000000,0.000000}%
\pgfsetfillcolor{currentfill}%
\pgfsetlinewidth{0.803000pt}%
\definecolor{currentstroke}{rgb}{0.000000,0.000000,0.000000}%
\pgfsetstrokecolor{currentstroke}%
\pgfsetdash{}{0pt}%
\pgfsys@defobject{currentmarker}{\pgfqpoint{0.000000in}{-0.048611in}}{\pgfqpoint{0.000000in}{0.000000in}}{%
\pgfpathmoveto{\pgfqpoint{0.000000in}{0.000000in}}%
\pgfpathlineto{\pgfqpoint{0.000000in}{-0.048611in}}%
\pgfusepath{stroke,fill}%
}%
\begin{pgfscope}%
\pgfsys@transformshift{2.257406in}{0.521603in}%
\pgfsys@useobject{currentmarker}{}%
\end{pgfscope}%
\end{pgfscope}%
\begin{pgfscope}%
\definecolor{textcolor}{rgb}{0.000000,0.000000,0.000000}%
\pgfsetstrokecolor{textcolor}%
\pgfsetfillcolor{textcolor}%
\pgftext[x=2.257406in,y=0.424381in,,top]{\color{textcolor}{\rmfamily\fontsize{10.000000}{12.000000}\selectfont\catcode`\^=\active\def^{\ifmmode\sp\else\^{}\fi}\catcode`\%=\active\def%{\%}$\mathdefault{16}$}}%
\end{pgfscope}%
\begin{pgfscope}%
\pgfsetbuttcap%
\pgfsetroundjoin%
\definecolor{currentfill}{rgb}{0.000000,0.000000,0.000000}%
\pgfsetfillcolor{currentfill}%
\pgfsetlinewidth{0.803000pt}%
\definecolor{currentstroke}{rgb}{0.000000,0.000000,0.000000}%
\pgfsetstrokecolor{currentstroke}%
\pgfsetdash{}{0pt}%
\pgfsys@defobject{currentmarker}{\pgfqpoint{0.000000in}{-0.048611in}}{\pgfqpoint{0.000000in}{0.000000in}}{%
\pgfpathmoveto{\pgfqpoint{0.000000in}{0.000000in}}%
\pgfpathlineto{\pgfqpoint{0.000000in}{-0.048611in}}%
\pgfusepath{stroke,fill}%
}%
\begin{pgfscope}%
\pgfsys@transformshift{2.686736in}{0.521603in}%
\pgfsys@useobject{currentmarker}{}%
\end{pgfscope}%
\end{pgfscope}%
\begin{pgfscope}%
\definecolor{textcolor}{rgb}{0.000000,0.000000,0.000000}%
\pgfsetstrokecolor{textcolor}%
\pgfsetfillcolor{textcolor}%
\pgftext[x=2.686736in,y=0.424381in,,top]{\color{textcolor}{\rmfamily\fontsize{10.000000}{12.000000}\selectfont\catcode`\^=\active\def^{\ifmmode\sp\else\^{}\fi}\catcode`\%=\active\def%{\%}$\mathdefault{20}$}}%
\end{pgfscope}%
\begin{pgfscope}%
\pgfsetbuttcap%
\pgfsetroundjoin%
\definecolor{currentfill}{rgb}{0.000000,0.000000,0.000000}%
\pgfsetfillcolor{currentfill}%
\pgfsetlinewidth{0.803000pt}%
\definecolor{currentstroke}{rgb}{0.000000,0.000000,0.000000}%
\pgfsetstrokecolor{currentstroke}%
\pgfsetdash{}{0pt}%
\pgfsys@defobject{currentmarker}{\pgfqpoint{0.000000in}{-0.048611in}}{\pgfqpoint{0.000000in}{0.000000in}}{%
\pgfpathmoveto{\pgfqpoint{0.000000in}{0.000000in}}%
\pgfpathlineto{\pgfqpoint{0.000000in}{-0.048611in}}%
\pgfusepath{stroke,fill}%
}%
\begin{pgfscope}%
\pgfsys@transformshift{3.116066in}{0.521603in}%
\pgfsys@useobject{currentmarker}{}%
\end{pgfscope}%
\end{pgfscope}%
\begin{pgfscope}%
\definecolor{textcolor}{rgb}{0.000000,0.000000,0.000000}%
\pgfsetstrokecolor{textcolor}%
\pgfsetfillcolor{textcolor}%
\pgftext[x=3.116066in,y=0.424381in,,top]{\color{textcolor}{\rmfamily\fontsize{10.000000}{12.000000}\selectfont\catcode`\^=\active\def^{\ifmmode\sp\else\^{}\fi}\catcode`\%=\active\def%{\%}$\mathdefault{24}$}}%
\end{pgfscope}%
\begin{pgfscope}%
\pgfsetbuttcap%
\pgfsetroundjoin%
\definecolor{currentfill}{rgb}{0.000000,0.000000,0.000000}%
\pgfsetfillcolor{currentfill}%
\pgfsetlinewidth{0.803000pt}%
\definecolor{currentstroke}{rgb}{0.000000,0.000000,0.000000}%
\pgfsetstrokecolor{currentstroke}%
\pgfsetdash{}{0pt}%
\pgfsys@defobject{currentmarker}{\pgfqpoint{0.000000in}{-0.048611in}}{\pgfqpoint{0.000000in}{0.000000in}}{%
\pgfpathmoveto{\pgfqpoint{0.000000in}{0.000000in}}%
\pgfpathlineto{\pgfqpoint{0.000000in}{-0.048611in}}%
\pgfusepath{stroke,fill}%
}%
\begin{pgfscope}%
\pgfsys@transformshift{3.545395in}{0.521603in}%
\pgfsys@useobject{currentmarker}{}%
\end{pgfscope}%
\end{pgfscope}%
\begin{pgfscope}%
\definecolor{textcolor}{rgb}{0.000000,0.000000,0.000000}%
\pgfsetstrokecolor{textcolor}%
\pgfsetfillcolor{textcolor}%
\pgftext[x=3.545395in,y=0.424381in,,top]{\color{textcolor}{\rmfamily\fontsize{10.000000}{12.000000}\selectfont\catcode`\^=\active\def^{\ifmmode\sp\else\^{}\fi}\catcode`\%=\active\def%{\%}$\mathdefault{28}$}}%
\end{pgfscope}%
\begin{pgfscope}%
\pgfsetbuttcap%
\pgfsetroundjoin%
\definecolor{currentfill}{rgb}{0.000000,0.000000,0.000000}%
\pgfsetfillcolor{currentfill}%
\pgfsetlinewidth{0.803000pt}%
\definecolor{currentstroke}{rgb}{0.000000,0.000000,0.000000}%
\pgfsetstrokecolor{currentstroke}%
\pgfsetdash{}{0pt}%
\pgfsys@defobject{currentmarker}{\pgfqpoint{0.000000in}{-0.048611in}}{\pgfqpoint{0.000000in}{0.000000in}}{%
\pgfpathmoveto{\pgfqpoint{0.000000in}{0.000000in}}%
\pgfpathlineto{\pgfqpoint{0.000000in}{-0.048611in}}%
\pgfusepath{stroke,fill}%
}%
\begin{pgfscope}%
\pgfsys@transformshift{3.974725in}{0.521603in}%
\pgfsys@useobject{currentmarker}{}%
\end{pgfscope}%
\end{pgfscope}%
\begin{pgfscope}%
\definecolor{textcolor}{rgb}{0.000000,0.000000,0.000000}%
\pgfsetstrokecolor{textcolor}%
\pgfsetfillcolor{textcolor}%
\pgftext[x=3.974725in,y=0.424381in,,top]{\color{textcolor}{\rmfamily\fontsize{10.000000}{12.000000}\selectfont\catcode`\^=\active\def^{\ifmmode\sp\else\^{}\fi}\catcode`\%=\active\def%{\%}$\mathdefault{32}$}}%
\end{pgfscope}%
\begin{pgfscope}%
\definecolor{textcolor}{rgb}{0.000000,0.000000,0.000000}%
\pgfsetstrokecolor{textcolor}%
\pgfsetfillcolor{textcolor}%
\pgftext[x=2.418405in,y=0.234413in,,top]{\color{textcolor}{\rmfamily\fontsize{10.000000}{12.000000}\selectfont\catcode`\^=\active\def^{\ifmmode\sp\else\^{}\fi}\catcode`\%=\active\def%{\%}Monochromatic classes}}%
\end{pgfscope}%
\begin{pgfscope}%
\pgfsetbuttcap%
\pgfsetroundjoin%
\definecolor{currentfill}{rgb}{0.000000,0.000000,0.000000}%
\pgfsetfillcolor{currentfill}%
\pgfsetlinewidth{0.803000pt}%
\definecolor{currentstroke}{rgb}{0.000000,0.000000,0.000000}%
\pgfsetstrokecolor{currentstroke}%
\pgfsetdash{}{0pt}%
\pgfsys@defobject{currentmarker}{\pgfqpoint{-0.048611in}{0.000000in}}{\pgfqpoint{-0.000000in}{0.000000in}}{%
\pgfpathmoveto{\pgfqpoint{-0.000000in}{0.000000in}}%
\pgfpathlineto{\pgfqpoint{-0.048611in}{0.000000in}}%
\pgfusepath{stroke,fill}%
}%
\begin{pgfscope}%
\pgfsys@transformshift{0.588387in}{0.930505in}%
\pgfsys@useobject{currentmarker}{}%
\end{pgfscope}%
\end{pgfscope}%
\begin{pgfscope}%
\definecolor{textcolor}{rgb}{0.000000,0.000000,0.000000}%
\pgfsetstrokecolor{textcolor}%
\pgfsetfillcolor{textcolor}%
\pgftext[x=0.289968in, y=0.877743in, left, base]{\color{textcolor}{\rmfamily\fontsize{10.000000}{12.000000}\selectfont\catcode`\^=\active\def^{\ifmmode\sp\else\^{}\fi}\catcode`\%=\active\def%{\%}$\mathdefault{10^{1}}$}}%
\end{pgfscope}%
\begin{pgfscope}%
\pgfsetbuttcap%
\pgfsetroundjoin%
\definecolor{currentfill}{rgb}{0.000000,0.000000,0.000000}%
\pgfsetfillcolor{currentfill}%
\pgfsetlinewidth{0.803000pt}%
\definecolor{currentstroke}{rgb}{0.000000,0.000000,0.000000}%
\pgfsetstrokecolor{currentstroke}%
\pgfsetdash{}{0pt}%
\pgfsys@defobject{currentmarker}{\pgfqpoint{-0.048611in}{0.000000in}}{\pgfqpoint{-0.000000in}{0.000000in}}{%
\pgfpathmoveto{\pgfqpoint{-0.000000in}{0.000000in}}%
\pgfpathlineto{\pgfqpoint{-0.048611in}{0.000000in}}%
\pgfusepath{stroke,fill}%
}%
\begin{pgfscope}%
\pgfsys@transformshift{0.588387in}{1.552136in}%
\pgfsys@useobject{currentmarker}{}%
\end{pgfscope}%
\end{pgfscope}%
\begin{pgfscope}%
\definecolor{textcolor}{rgb}{0.000000,0.000000,0.000000}%
\pgfsetstrokecolor{textcolor}%
\pgfsetfillcolor{textcolor}%
\pgftext[x=0.289968in, y=1.499374in, left, base]{\color{textcolor}{\rmfamily\fontsize{10.000000}{12.000000}\selectfont\catcode`\^=\active\def^{\ifmmode\sp\else\^{}\fi}\catcode`\%=\active\def%{\%}$\mathdefault{10^{2}}$}}%
\end{pgfscope}%
\begin{pgfscope}%
\pgfsetbuttcap%
\pgfsetroundjoin%
\definecolor{currentfill}{rgb}{0.000000,0.000000,0.000000}%
\pgfsetfillcolor{currentfill}%
\pgfsetlinewidth{0.803000pt}%
\definecolor{currentstroke}{rgb}{0.000000,0.000000,0.000000}%
\pgfsetstrokecolor{currentstroke}%
\pgfsetdash{}{0pt}%
\pgfsys@defobject{currentmarker}{\pgfqpoint{-0.048611in}{0.000000in}}{\pgfqpoint{-0.000000in}{0.000000in}}{%
\pgfpathmoveto{\pgfqpoint{-0.000000in}{0.000000in}}%
\pgfpathlineto{\pgfqpoint{-0.048611in}{0.000000in}}%
\pgfusepath{stroke,fill}%
}%
\begin{pgfscope}%
\pgfsys@transformshift{0.588387in}{2.173767in}%
\pgfsys@useobject{currentmarker}{}%
\end{pgfscope}%
\end{pgfscope}%
\begin{pgfscope}%
\definecolor{textcolor}{rgb}{0.000000,0.000000,0.000000}%
\pgfsetstrokecolor{textcolor}%
\pgfsetfillcolor{textcolor}%
\pgftext[x=0.289968in, y=2.121005in, left, base]{\color{textcolor}{\rmfamily\fontsize{10.000000}{12.000000}\selectfont\catcode`\^=\active\def^{\ifmmode\sp\else\^{}\fi}\catcode`\%=\active\def%{\%}$\mathdefault{10^{3}}$}}%
\end{pgfscope}%
\begin{pgfscope}%
\pgfsetbuttcap%
\pgfsetroundjoin%
\definecolor{currentfill}{rgb}{0.000000,0.000000,0.000000}%
\pgfsetfillcolor{currentfill}%
\pgfsetlinewidth{0.602250pt}%
\definecolor{currentstroke}{rgb}{0.000000,0.000000,0.000000}%
\pgfsetstrokecolor{currentstroke}%
\pgfsetdash{}{0pt}%
\pgfsys@defobject{currentmarker}{\pgfqpoint{-0.027778in}{0.000000in}}{\pgfqpoint{-0.000000in}{0.000000in}}{%
\pgfpathmoveto{\pgfqpoint{-0.000000in}{0.000000in}}%
\pgfpathlineto{\pgfqpoint{-0.027778in}{0.000000in}}%
\pgfusepath{stroke,fill}%
}%
\begin{pgfscope}%
\pgfsys@transformshift{0.588387in}{0.605467in}%
\pgfsys@useobject{currentmarker}{}%
\end{pgfscope}%
\end{pgfscope}%
\begin{pgfscope}%
\pgfsetbuttcap%
\pgfsetroundjoin%
\definecolor{currentfill}{rgb}{0.000000,0.000000,0.000000}%
\pgfsetfillcolor{currentfill}%
\pgfsetlinewidth{0.602250pt}%
\definecolor{currentstroke}{rgb}{0.000000,0.000000,0.000000}%
\pgfsetstrokecolor{currentstroke}%
\pgfsetdash{}{0pt}%
\pgfsys@defobject{currentmarker}{\pgfqpoint{-0.027778in}{0.000000in}}{\pgfqpoint{-0.000000in}{0.000000in}}{%
\pgfpathmoveto{\pgfqpoint{-0.000000in}{0.000000in}}%
\pgfpathlineto{\pgfqpoint{-0.027778in}{0.000000in}}%
\pgfusepath{stroke,fill}%
}%
\begin{pgfscope}%
\pgfsys@transformshift{0.588387in}{0.683133in}%
\pgfsys@useobject{currentmarker}{}%
\end{pgfscope}%
\end{pgfscope}%
\begin{pgfscope}%
\pgfsetbuttcap%
\pgfsetroundjoin%
\definecolor{currentfill}{rgb}{0.000000,0.000000,0.000000}%
\pgfsetfillcolor{currentfill}%
\pgfsetlinewidth{0.602250pt}%
\definecolor{currentstroke}{rgb}{0.000000,0.000000,0.000000}%
\pgfsetstrokecolor{currentstroke}%
\pgfsetdash{}{0pt}%
\pgfsys@defobject{currentmarker}{\pgfqpoint{-0.027778in}{0.000000in}}{\pgfqpoint{-0.000000in}{0.000000in}}{%
\pgfpathmoveto{\pgfqpoint{-0.000000in}{0.000000in}}%
\pgfpathlineto{\pgfqpoint{-0.027778in}{0.000000in}}%
\pgfusepath{stroke,fill}%
}%
\begin{pgfscope}%
\pgfsys@transformshift{0.588387in}{0.743375in}%
\pgfsys@useobject{currentmarker}{}%
\end{pgfscope}%
\end{pgfscope}%
\begin{pgfscope}%
\pgfsetbuttcap%
\pgfsetroundjoin%
\definecolor{currentfill}{rgb}{0.000000,0.000000,0.000000}%
\pgfsetfillcolor{currentfill}%
\pgfsetlinewidth{0.602250pt}%
\definecolor{currentstroke}{rgb}{0.000000,0.000000,0.000000}%
\pgfsetstrokecolor{currentstroke}%
\pgfsetdash{}{0pt}%
\pgfsys@defobject{currentmarker}{\pgfqpoint{-0.027778in}{0.000000in}}{\pgfqpoint{-0.000000in}{0.000000in}}{%
\pgfpathmoveto{\pgfqpoint{-0.000000in}{0.000000in}}%
\pgfpathlineto{\pgfqpoint{-0.027778in}{0.000000in}}%
\pgfusepath{stroke,fill}%
}%
\begin{pgfscope}%
\pgfsys@transformshift{0.588387in}{0.792597in}%
\pgfsys@useobject{currentmarker}{}%
\end{pgfscope}%
\end{pgfscope}%
\begin{pgfscope}%
\pgfsetbuttcap%
\pgfsetroundjoin%
\definecolor{currentfill}{rgb}{0.000000,0.000000,0.000000}%
\pgfsetfillcolor{currentfill}%
\pgfsetlinewidth{0.602250pt}%
\definecolor{currentstroke}{rgb}{0.000000,0.000000,0.000000}%
\pgfsetstrokecolor{currentstroke}%
\pgfsetdash{}{0pt}%
\pgfsys@defobject{currentmarker}{\pgfqpoint{-0.027778in}{0.000000in}}{\pgfqpoint{-0.000000in}{0.000000in}}{%
\pgfpathmoveto{\pgfqpoint{-0.000000in}{0.000000in}}%
\pgfpathlineto{\pgfqpoint{-0.027778in}{0.000000in}}%
\pgfusepath{stroke,fill}%
}%
\begin{pgfscope}%
\pgfsys@transformshift{0.588387in}{0.834213in}%
\pgfsys@useobject{currentmarker}{}%
\end{pgfscope}%
\end{pgfscope}%
\begin{pgfscope}%
\pgfsetbuttcap%
\pgfsetroundjoin%
\definecolor{currentfill}{rgb}{0.000000,0.000000,0.000000}%
\pgfsetfillcolor{currentfill}%
\pgfsetlinewidth{0.602250pt}%
\definecolor{currentstroke}{rgb}{0.000000,0.000000,0.000000}%
\pgfsetstrokecolor{currentstroke}%
\pgfsetdash{}{0pt}%
\pgfsys@defobject{currentmarker}{\pgfqpoint{-0.027778in}{0.000000in}}{\pgfqpoint{-0.000000in}{0.000000in}}{%
\pgfpathmoveto{\pgfqpoint{-0.000000in}{0.000000in}}%
\pgfpathlineto{\pgfqpoint{-0.027778in}{0.000000in}}%
\pgfusepath{stroke,fill}%
}%
\begin{pgfscope}%
\pgfsys@transformshift{0.588387in}{0.870263in}%
\pgfsys@useobject{currentmarker}{}%
\end{pgfscope}%
\end{pgfscope}%
\begin{pgfscope}%
\pgfsetbuttcap%
\pgfsetroundjoin%
\definecolor{currentfill}{rgb}{0.000000,0.000000,0.000000}%
\pgfsetfillcolor{currentfill}%
\pgfsetlinewidth{0.602250pt}%
\definecolor{currentstroke}{rgb}{0.000000,0.000000,0.000000}%
\pgfsetstrokecolor{currentstroke}%
\pgfsetdash{}{0pt}%
\pgfsys@defobject{currentmarker}{\pgfqpoint{-0.027778in}{0.000000in}}{\pgfqpoint{-0.000000in}{0.000000in}}{%
\pgfpathmoveto{\pgfqpoint{-0.000000in}{0.000000in}}%
\pgfpathlineto{\pgfqpoint{-0.027778in}{0.000000in}}%
\pgfusepath{stroke,fill}%
}%
\begin{pgfscope}%
\pgfsys@transformshift{0.588387in}{0.902061in}%
\pgfsys@useobject{currentmarker}{}%
\end{pgfscope}%
\end{pgfscope}%
\begin{pgfscope}%
\pgfsetbuttcap%
\pgfsetroundjoin%
\definecolor{currentfill}{rgb}{0.000000,0.000000,0.000000}%
\pgfsetfillcolor{currentfill}%
\pgfsetlinewidth{0.602250pt}%
\definecolor{currentstroke}{rgb}{0.000000,0.000000,0.000000}%
\pgfsetstrokecolor{currentstroke}%
\pgfsetdash{}{0pt}%
\pgfsys@defobject{currentmarker}{\pgfqpoint{-0.027778in}{0.000000in}}{\pgfqpoint{-0.000000in}{0.000000in}}{%
\pgfpathmoveto{\pgfqpoint{-0.000000in}{0.000000in}}%
\pgfpathlineto{\pgfqpoint{-0.027778in}{0.000000in}}%
\pgfusepath{stroke,fill}%
}%
\begin{pgfscope}%
\pgfsys@transformshift{0.588387in}{1.117634in}%
\pgfsys@useobject{currentmarker}{}%
\end{pgfscope}%
\end{pgfscope}%
\begin{pgfscope}%
\pgfsetbuttcap%
\pgfsetroundjoin%
\definecolor{currentfill}{rgb}{0.000000,0.000000,0.000000}%
\pgfsetfillcolor{currentfill}%
\pgfsetlinewidth{0.602250pt}%
\definecolor{currentstroke}{rgb}{0.000000,0.000000,0.000000}%
\pgfsetstrokecolor{currentstroke}%
\pgfsetdash{}{0pt}%
\pgfsys@defobject{currentmarker}{\pgfqpoint{-0.027778in}{0.000000in}}{\pgfqpoint{-0.000000in}{0.000000in}}{%
\pgfpathmoveto{\pgfqpoint{-0.000000in}{0.000000in}}%
\pgfpathlineto{\pgfqpoint{-0.027778in}{0.000000in}}%
\pgfusepath{stroke,fill}%
}%
\begin{pgfscope}%
\pgfsys@transformshift{0.588387in}{1.227098in}%
\pgfsys@useobject{currentmarker}{}%
\end{pgfscope}%
\end{pgfscope}%
\begin{pgfscope}%
\pgfsetbuttcap%
\pgfsetroundjoin%
\definecolor{currentfill}{rgb}{0.000000,0.000000,0.000000}%
\pgfsetfillcolor{currentfill}%
\pgfsetlinewidth{0.602250pt}%
\definecolor{currentstroke}{rgb}{0.000000,0.000000,0.000000}%
\pgfsetstrokecolor{currentstroke}%
\pgfsetdash{}{0pt}%
\pgfsys@defobject{currentmarker}{\pgfqpoint{-0.027778in}{0.000000in}}{\pgfqpoint{-0.000000in}{0.000000in}}{%
\pgfpathmoveto{\pgfqpoint{-0.000000in}{0.000000in}}%
\pgfpathlineto{\pgfqpoint{-0.027778in}{0.000000in}}%
\pgfusepath{stroke,fill}%
}%
\begin{pgfscope}%
\pgfsys@transformshift{0.588387in}{1.304764in}%
\pgfsys@useobject{currentmarker}{}%
\end{pgfscope}%
\end{pgfscope}%
\begin{pgfscope}%
\pgfsetbuttcap%
\pgfsetroundjoin%
\definecolor{currentfill}{rgb}{0.000000,0.000000,0.000000}%
\pgfsetfillcolor{currentfill}%
\pgfsetlinewidth{0.602250pt}%
\definecolor{currentstroke}{rgb}{0.000000,0.000000,0.000000}%
\pgfsetstrokecolor{currentstroke}%
\pgfsetdash{}{0pt}%
\pgfsys@defobject{currentmarker}{\pgfqpoint{-0.027778in}{0.000000in}}{\pgfqpoint{-0.000000in}{0.000000in}}{%
\pgfpathmoveto{\pgfqpoint{-0.000000in}{0.000000in}}%
\pgfpathlineto{\pgfqpoint{-0.027778in}{0.000000in}}%
\pgfusepath{stroke,fill}%
}%
\begin{pgfscope}%
\pgfsys@transformshift{0.588387in}{1.365006in}%
\pgfsys@useobject{currentmarker}{}%
\end{pgfscope}%
\end{pgfscope}%
\begin{pgfscope}%
\pgfsetbuttcap%
\pgfsetroundjoin%
\definecolor{currentfill}{rgb}{0.000000,0.000000,0.000000}%
\pgfsetfillcolor{currentfill}%
\pgfsetlinewidth{0.602250pt}%
\definecolor{currentstroke}{rgb}{0.000000,0.000000,0.000000}%
\pgfsetstrokecolor{currentstroke}%
\pgfsetdash{}{0pt}%
\pgfsys@defobject{currentmarker}{\pgfqpoint{-0.027778in}{0.000000in}}{\pgfqpoint{-0.000000in}{0.000000in}}{%
\pgfpathmoveto{\pgfqpoint{-0.000000in}{0.000000in}}%
\pgfpathlineto{\pgfqpoint{-0.027778in}{0.000000in}}%
\pgfusepath{stroke,fill}%
}%
\begin{pgfscope}%
\pgfsys@transformshift{0.588387in}{1.414228in}%
\pgfsys@useobject{currentmarker}{}%
\end{pgfscope}%
\end{pgfscope}%
\begin{pgfscope}%
\pgfsetbuttcap%
\pgfsetroundjoin%
\definecolor{currentfill}{rgb}{0.000000,0.000000,0.000000}%
\pgfsetfillcolor{currentfill}%
\pgfsetlinewidth{0.602250pt}%
\definecolor{currentstroke}{rgb}{0.000000,0.000000,0.000000}%
\pgfsetstrokecolor{currentstroke}%
\pgfsetdash{}{0pt}%
\pgfsys@defobject{currentmarker}{\pgfqpoint{-0.027778in}{0.000000in}}{\pgfqpoint{-0.000000in}{0.000000in}}{%
\pgfpathmoveto{\pgfqpoint{-0.000000in}{0.000000in}}%
\pgfpathlineto{\pgfqpoint{-0.027778in}{0.000000in}}%
\pgfusepath{stroke,fill}%
}%
\begin{pgfscope}%
\pgfsys@transformshift{0.588387in}{1.455844in}%
\pgfsys@useobject{currentmarker}{}%
\end{pgfscope}%
\end{pgfscope}%
\begin{pgfscope}%
\pgfsetbuttcap%
\pgfsetroundjoin%
\definecolor{currentfill}{rgb}{0.000000,0.000000,0.000000}%
\pgfsetfillcolor{currentfill}%
\pgfsetlinewidth{0.602250pt}%
\definecolor{currentstroke}{rgb}{0.000000,0.000000,0.000000}%
\pgfsetstrokecolor{currentstroke}%
\pgfsetdash{}{0pt}%
\pgfsys@defobject{currentmarker}{\pgfqpoint{-0.027778in}{0.000000in}}{\pgfqpoint{-0.000000in}{0.000000in}}{%
\pgfpathmoveto{\pgfqpoint{-0.000000in}{0.000000in}}%
\pgfpathlineto{\pgfqpoint{-0.027778in}{0.000000in}}%
\pgfusepath{stroke,fill}%
}%
\begin{pgfscope}%
\pgfsys@transformshift{0.588387in}{1.491894in}%
\pgfsys@useobject{currentmarker}{}%
\end{pgfscope}%
\end{pgfscope}%
\begin{pgfscope}%
\pgfsetbuttcap%
\pgfsetroundjoin%
\definecolor{currentfill}{rgb}{0.000000,0.000000,0.000000}%
\pgfsetfillcolor{currentfill}%
\pgfsetlinewidth{0.602250pt}%
\definecolor{currentstroke}{rgb}{0.000000,0.000000,0.000000}%
\pgfsetstrokecolor{currentstroke}%
\pgfsetdash{}{0pt}%
\pgfsys@defobject{currentmarker}{\pgfqpoint{-0.027778in}{0.000000in}}{\pgfqpoint{-0.000000in}{0.000000in}}{%
\pgfpathmoveto{\pgfqpoint{-0.000000in}{0.000000in}}%
\pgfpathlineto{\pgfqpoint{-0.027778in}{0.000000in}}%
\pgfusepath{stroke,fill}%
}%
\begin{pgfscope}%
\pgfsys@transformshift{0.588387in}{1.523692in}%
\pgfsys@useobject{currentmarker}{}%
\end{pgfscope}%
\end{pgfscope}%
\begin{pgfscope}%
\pgfsetbuttcap%
\pgfsetroundjoin%
\definecolor{currentfill}{rgb}{0.000000,0.000000,0.000000}%
\pgfsetfillcolor{currentfill}%
\pgfsetlinewidth{0.602250pt}%
\definecolor{currentstroke}{rgb}{0.000000,0.000000,0.000000}%
\pgfsetstrokecolor{currentstroke}%
\pgfsetdash{}{0pt}%
\pgfsys@defobject{currentmarker}{\pgfqpoint{-0.027778in}{0.000000in}}{\pgfqpoint{-0.000000in}{0.000000in}}{%
\pgfpathmoveto{\pgfqpoint{-0.000000in}{0.000000in}}%
\pgfpathlineto{\pgfqpoint{-0.027778in}{0.000000in}}%
\pgfusepath{stroke,fill}%
}%
\begin{pgfscope}%
\pgfsys@transformshift{0.588387in}{1.739265in}%
\pgfsys@useobject{currentmarker}{}%
\end{pgfscope}%
\end{pgfscope}%
\begin{pgfscope}%
\pgfsetbuttcap%
\pgfsetroundjoin%
\definecolor{currentfill}{rgb}{0.000000,0.000000,0.000000}%
\pgfsetfillcolor{currentfill}%
\pgfsetlinewidth{0.602250pt}%
\definecolor{currentstroke}{rgb}{0.000000,0.000000,0.000000}%
\pgfsetstrokecolor{currentstroke}%
\pgfsetdash{}{0pt}%
\pgfsys@defobject{currentmarker}{\pgfqpoint{-0.027778in}{0.000000in}}{\pgfqpoint{-0.000000in}{0.000000in}}{%
\pgfpathmoveto{\pgfqpoint{-0.000000in}{0.000000in}}%
\pgfpathlineto{\pgfqpoint{-0.027778in}{0.000000in}}%
\pgfusepath{stroke,fill}%
}%
\begin{pgfscope}%
\pgfsys@transformshift{0.588387in}{1.848729in}%
\pgfsys@useobject{currentmarker}{}%
\end{pgfscope}%
\end{pgfscope}%
\begin{pgfscope}%
\pgfsetbuttcap%
\pgfsetroundjoin%
\definecolor{currentfill}{rgb}{0.000000,0.000000,0.000000}%
\pgfsetfillcolor{currentfill}%
\pgfsetlinewidth{0.602250pt}%
\definecolor{currentstroke}{rgb}{0.000000,0.000000,0.000000}%
\pgfsetstrokecolor{currentstroke}%
\pgfsetdash{}{0pt}%
\pgfsys@defobject{currentmarker}{\pgfqpoint{-0.027778in}{0.000000in}}{\pgfqpoint{-0.000000in}{0.000000in}}{%
\pgfpathmoveto{\pgfqpoint{-0.000000in}{0.000000in}}%
\pgfpathlineto{\pgfqpoint{-0.027778in}{0.000000in}}%
\pgfusepath{stroke,fill}%
}%
\begin{pgfscope}%
\pgfsys@transformshift{0.588387in}{1.926395in}%
\pgfsys@useobject{currentmarker}{}%
\end{pgfscope}%
\end{pgfscope}%
\begin{pgfscope}%
\pgfsetbuttcap%
\pgfsetroundjoin%
\definecolor{currentfill}{rgb}{0.000000,0.000000,0.000000}%
\pgfsetfillcolor{currentfill}%
\pgfsetlinewidth{0.602250pt}%
\definecolor{currentstroke}{rgb}{0.000000,0.000000,0.000000}%
\pgfsetstrokecolor{currentstroke}%
\pgfsetdash{}{0pt}%
\pgfsys@defobject{currentmarker}{\pgfqpoint{-0.027778in}{0.000000in}}{\pgfqpoint{-0.000000in}{0.000000in}}{%
\pgfpathmoveto{\pgfqpoint{-0.000000in}{0.000000in}}%
\pgfpathlineto{\pgfqpoint{-0.027778in}{0.000000in}}%
\pgfusepath{stroke,fill}%
}%
\begin{pgfscope}%
\pgfsys@transformshift{0.588387in}{1.986637in}%
\pgfsys@useobject{currentmarker}{}%
\end{pgfscope}%
\end{pgfscope}%
\begin{pgfscope}%
\pgfsetbuttcap%
\pgfsetroundjoin%
\definecolor{currentfill}{rgb}{0.000000,0.000000,0.000000}%
\pgfsetfillcolor{currentfill}%
\pgfsetlinewidth{0.602250pt}%
\definecolor{currentstroke}{rgb}{0.000000,0.000000,0.000000}%
\pgfsetstrokecolor{currentstroke}%
\pgfsetdash{}{0pt}%
\pgfsys@defobject{currentmarker}{\pgfqpoint{-0.027778in}{0.000000in}}{\pgfqpoint{-0.000000in}{0.000000in}}{%
\pgfpathmoveto{\pgfqpoint{-0.000000in}{0.000000in}}%
\pgfpathlineto{\pgfqpoint{-0.027778in}{0.000000in}}%
\pgfusepath{stroke,fill}%
}%
\begin{pgfscope}%
\pgfsys@transformshift{0.588387in}{2.035859in}%
\pgfsys@useobject{currentmarker}{}%
\end{pgfscope}%
\end{pgfscope}%
\begin{pgfscope}%
\pgfsetbuttcap%
\pgfsetroundjoin%
\definecolor{currentfill}{rgb}{0.000000,0.000000,0.000000}%
\pgfsetfillcolor{currentfill}%
\pgfsetlinewidth{0.602250pt}%
\definecolor{currentstroke}{rgb}{0.000000,0.000000,0.000000}%
\pgfsetstrokecolor{currentstroke}%
\pgfsetdash{}{0pt}%
\pgfsys@defobject{currentmarker}{\pgfqpoint{-0.027778in}{0.000000in}}{\pgfqpoint{-0.000000in}{0.000000in}}{%
\pgfpathmoveto{\pgfqpoint{-0.000000in}{0.000000in}}%
\pgfpathlineto{\pgfqpoint{-0.027778in}{0.000000in}}%
\pgfusepath{stroke,fill}%
}%
\begin{pgfscope}%
\pgfsys@transformshift{0.588387in}{2.077475in}%
\pgfsys@useobject{currentmarker}{}%
\end{pgfscope}%
\end{pgfscope}%
\begin{pgfscope}%
\pgfsetbuttcap%
\pgfsetroundjoin%
\definecolor{currentfill}{rgb}{0.000000,0.000000,0.000000}%
\pgfsetfillcolor{currentfill}%
\pgfsetlinewidth{0.602250pt}%
\definecolor{currentstroke}{rgb}{0.000000,0.000000,0.000000}%
\pgfsetstrokecolor{currentstroke}%
\pgfsetdash{}{0pt}%
\pgfsys@defobject{currentmarker}{\pgfqpoint{-0.027778in}{0.000000in}}{\pgfqpoint{-0.000000in}{0.000000in}}{%
\pgfpathmoveto{\pgfqpoint{-0.000000in}{0.000000in}}%
\pgfpathlineto{\pgfqpoint{-0.027778in}{0.000000in}}%
\pgfusepath{stroke,fill}%
}%
\begin{pgfscope}%
\pgfsys@transformshift{0.588387in}{2.113525in}%
\pgfsys@useobject{currentmarker}{}%
\end{pgfscope}%
\end{pgfscope}%
\begin{pgfscope}%
\pgfsetbuttcap%
\pgfsetroundjoin%
\definecolor{currentfill}{rgb}{0.000000,0.000000,0.000000}%
\pgfsetfillcolor{currentfill}%
\pgfsetlinewidth{0.602250pt}%
\definecolor{currentstroke}{rgb}{0.000000,0.000000,0.000000}%
\pgfsetstrokecolor{currentstroke}%
\pgfsetdash{}{0pt}%
\pgfsys@defobject{currentmarker}{\pgfqpoint{-0.027778in}{0.000000in}}{\pgfqpoint{-0.000000in}{0.000000in}}{%
\pgfpathmoveto{\pgfqpoint{-0.000000in}{0.000000in}}%
\pgfpathlineto{\pgfqpoint{-0.027778in}{0.000000in}}%
\pgfusepath{stroke,fill}%
}%
\begin{pgfscope}%
\pgfsys@transformshift{0.588387in}{2.145323in}%
\pgfsys@useobject{currentmarker}{}%
\end{pgfscope}%
\end{pgfscope}%
\begin{pgfscope}%
\pgfsetbuttcap%
\pgfsetroundjoin%
\definecolor{currentfill}{rgb}{0.000000,0.000000,0.000000}%
\pgfsetfillcolor{currentfill}%
\pgfsetlinewidth{0.602250pt}%
\definecolor{currentstroke}{rgb}{0.000000,0.000000,0.000000}%
\pgfsetstrokecolor{currentstroke}%
\pgfsetdash{}{0pt}%
\pgfsys@defobject{currentmarker}{\pgfqpoint{-0.027778in}{0.000000in}}{\pgfqpoint{-0.000000in}{0.000000in}}{%
\pgfpathmoveto{\pgfqpoint{-0.000000in}{0.000000in}}%
\pgfpathlineto{\pgfqpoint{-0.027778in}{0.000000in}}%
\pgfusepath{stroke,fill}%
}%
\begin{pgfscope}%
\pgfsys@transformshift{0.588387in}{2.360897in}%
\pgfsys@useobject{currentmarker}{}%
\end{pgfscope}%
\end{pgfscope}%
\begin{pgfscope}%
\pgfsetbuttcap%
\pgfsetroundjoin%
\definecolor{currentfill}{rgb}{0.000000,0.000000,0.000000}%
\pgfsetfillcolor{currentfill}%
\pgfsetlinewidth{0.602250pt}%
\definecolor{currentstroke}{rgb}{0.000000,0.000000,0.000000}%
\pgfsetstrokecolor{currentstroke}%
\pgfsetdash{}{0pt}%
\pgfsys@defobject{currentmarker}{\pgfqpoint{-0.027778in}{0.000000in}}{\pgfqpoint{-0.000000in}{0.000000in}}{%
\pgfpathmoveto{\pgfqpoint{-0.000000in}{0.000000in}}%
\pgfpathlineto{\pgfqpoint{-0.027778in}{0.000000in}}%
\pgfusepath{stroke,fill}%
}%
\begin{pgfscope}%
\pgfsys@transformshift{0.588387in}{2.470360in}%
\pgfsys@useobject{currentmarker}{}%
\end{pgfscope}%
\end{pgfscope}%
\begin{pgfscope}%
\pgfsetbuttcap%
\pgfsetroundjoin%
\definecolor{currentfill}{rgb}{0.000000,0.000000,0.000000}%
\pgfsetfillcolor{currentfill}%
\pgfsetlinewidth{0.602250pt}%
\definecolor{currentstroke}{rgb}{0.000000,0.000000,0.000000}%
\pgfsetstrokecolor{currentstroke}%
\pgfsetdash{}{0pt}%
\pgfsys@defobject{currentmarker}{\pgfqpoint{-0.027778in}{0.000000in}}{\pgfqpoint{-0.000000in}{0.000000in}}{%
\pgfpathmoveto{\pgfqpoint{-0.000000in}{0.000000in}}%
\pgfpathlineto{\pgfqpoint{-0.027778in}{0.000000in}}%
\pgfusepath{stroke,fill}%
}%
\begin{pgfscope}%
\pgfsys@transformshift{0.588387in}{2.548026in}%
\pgfsys@useobject{currentmarker}{}%
\end{pgfscope}%
\end{pgfscope}%
\begin{pgfscope}%
\pgfsetbuttcap%
\pgfsetroundjoin%
\definecolor{currentfill}{rgb}{0.000000,0.000000,0.000000}%
\pgfsetfillcolor{currentfill}%
\pgfsetlinewidth{0.602250pt}%
\definecolor{currentstroke}{rgb}{0.000000,0.000000,0.000000}%
\pgfsetstrokecolor{currentstroke}%
\pgfsetdash{}{0pt}%
\pgfsys@defobject{currentmarker}{\pgfqpoint{-0.027778in}{0.000000in}}{\pgfqpoint{-0.000000in}{0.000000in}}{%
\pgfpathmoveto{\pgfqpoint{-0.000000in}{0.000000in}}%
\pgfpathlineto{\pgfqpoint{-0.027778in}{0.000000in}}%
\pgfusepath{stroke,fill}%
}%
\begin{pgfscope}%
\pgfsys@transformshift{0.588387in}{2.608268in}%
\pgfsys@useobject{currentmarker}{}%
\end{pgfscope}%
\end{pgfscope}%
\begin{pgfscope}%
\pgfsetbuttcap%
\pgfsetroundjoin%
\definecolor{currentfill}{rgb}{0.000000,0.000000,0.000000}%
\pgfsetfillcolor{currentfill}%
\pgfsetlinewidth{0.602250pt}%
\definecolor{currentstroke}{rgb}{0.000000,0.000000,0.000000}%
\pgfsetstrokecolor{currentstroke}%
\pgfsetdash{}{0pt}%
\pgfsys@defobject{currentmarker}{\pgfqpoint{-0.027778in}{0.000000in}}{\pgfqpoint{-0.000000in}{0.000000in}}{%
\pgfpathmoveto{\pgfqpoint{-0.000000in}{0.000000in}}%
\pgfpathlineto{\pgfqpoint{-0.027778in}{0.000000in}}%
\pgfusepath{stroke,fill}%
}%
\begin{pgfscope}%
\pgfsys@transformshift{0.588387in}{2.657490in}%
\pgfsys@useobject{currentmarker}{}%
\end{pgfscope}%
\end{pgfscope}%
\begin{pgfscope}%
\pgfsetbuttcap%
\pgfsetroundjoin%
\definecolor{currentfill}{rgb}{0.000000,0.000000,0.000000}%
\pgfsetfillcolor{currentfill}%
\pgfsetlinewidth{0.602250pt}%
\definecolor{currentstroke}{rgb}{0.000000,0.000000,0.000000}%
\pgfsetstrokecolor{currentstroke}%
\pgfsetdash{}{0pt}%
\pgfsys@defobject{currentmarker}{\pgfqpoint{-0.027778in}{0.000000in}}{\pgfqpoint{-0.000000in}{0.000000in}}{%
\pgfpathmoveto{\pgfqpoint{-0.000000in}{0.000000in}}%
\pgfpathlineto{\pgfqpoint{-0.027778in}{0.000000in}}%
\pgfusepath{stroke,fill}%
}%
\begin{pgfscope}%
\pgfsys@transformshift{0.588387in}{2.699106in}%
\pgfsys@useobject{currentmarker}{}%
\end{pgfscope}%
\end{pgfscope}%
\begin{pgfscope}%
\pgfsetbuttcap%
\pgfsetroundjoin%
\definecolor{currentfill}{rgb}{0.000000,0.000000,0.000000}%
\pgfsetfillcolor{currentfill}%
\pgfsetlinewidth{0.602250pt}%
\definecolor{currentstroke}{rgb}{0.000000,0.000000,0.000000}%
\pgfsetstrokecolor{currentstroke}%
\pgfsetdash{}{0pt}%
\pgfsys@defobject{currentmarker}{\pgfqpoint{-0.027778in}{0.000000in}}{\pgfqpoint{-0.000000in}{0.000000in}}{%
\pgfpathmoveto{\pgfqpoint{-0.000000in}{0.000000in}}%
\pgfpathlineto{\pgfqpoint{-0.027778in}{0.000000in}}%
\pgfusepath{stroke,fill}%
}%
\begin{pgfscope}%
\pgfsys@transformshift{0.588387in}{2.735156in}%
\pgfsys@useobject{currentmarker}{}%
\end{pgfscope}%
\end{pgfscope}%
\begin{pgfscope}%
\definecolor{textcolor}{rgb}{0.000000,0.000000,0.000000}%
\pgfsetstrokecolor{textcolor}%
\pgfsetfillcolor{textcolor}%
\pgftext[x=0.234413in,y=1.631726in,,bottom,rotate=90.000000]{\color{textcolor}{\rmfamily\fontsize{10.000000}{12.000000}\selectfont\catcode`\^=\active\def^{\ifmmode\sp\else\^{}\fi}\catcode`\%=\active\def%{\%}Time [ms]}}%
\end{pgfscope}%
\begin{pgfscope}%
\pgfpathrectangle{\pgfqpoint{0.588387in}{0.521603in}}{\pgfqpoint{3.660036in}{2.220246in}}%
\pgfusepath{clip}%
\pgfsetrectcap%
\pgfsetroundjoin%
\pgfsetlinewidth{1.505625pt}%
\pgfsetstrokecolor{currentstroke1}%
\pgfsetdash{}{0pt}%
\pgfpathmoveto{\pgfqpoint{0.754752in}{0.697730in}}%
\pgfpathlineto{\pgfqpoint{0.862085in}{0.726228in}}%
\pgfpathlineto{\pgfqpoint{0.969417in}{0.738183in}}%
\pgfpathlineto{\pgfqpoint{1.076750in}{0.691192in}}%
\pgfpathlineto{\pgfqpoint{1.184082in}{0.622524in}}%
\pgfpathlineto{\pgfqpoint{1.291415in}{0.654331in}}%
\pgfpathlineto{\pgfqpoint{1.398747in}{0.782727in}}%
\pgfpathlineto{\pgfqpoint{1.506079in}{0.837356in}}%
\pgfpathlineto{\pgfqpoint{1.613412in}{0.913800in}}%
\pgfpathlineto{\pgfqpoint{1.720744in}{0.960230in}}%
\pgfpathlineto{\pgfqpoint{1.828077in}{1.111567in}}%
\pgfpathlineto{\pgfqpoint{1.935409in}{1.173002in}}%
\pgfpathlineto{\pgfqpoint{2.042742in}{1.278637in}}%
\pgfpathlineto{\pgfqpoint{2.150074in}{1.319716in}}%
\pgfpathlineto{\pgfqpoint{2.257406in}{1.543761in}}%
\pgfpathlineto{\pgfqpoint{2.364739in}{1.559478in}}%
\pgfpathlineto{\pgfqpoint{2.472071in}{1.696622in}}%
\pgfpathlineto{\pgfqpoint{2.579404in}{1.700807in}}%
\pgfpathlineto{\pgfqpoint{2.686736in}{1.923995in}}%
\pgfpathlineto{\pgfqpoint{2.794068in}{1.940559in}}%
\pgfpathlineto{\pgfqpoint{2.901401in}{2.109458in}}%
\pgfpathlineto{\pgfqpoint{3.008733in}{2.055558in}}%
\pgfpathlineto{\pgfqpoint{3.116066in}{2.303429in}}%
\pgfpathlineto{\pgfqpoint{3.223398in}{2.300273in}}%
\pgfpathlineto{\pgfqpoint{3.330731in}{2.482645in}}%
\pgfpathlineto{\pgfqpoint{3.438063in}{2.452371in}}%
\pgfpathlineto{\pgfqpoint{3.652728in}{2.506747in}}%
\pgfpathlineto{\pgfqpoint{3.867393in}{2.529521in}}%
\pgfusepath{stroke}%
\end{pgfscope}%
\begin{pgfscope}%
\pgfpathrectangle{\pgfqpoint{0.588387in}{0.521603in}}{\pgfqpoint{3.660036in}{2.220246in}}%
\pgfusepath{clip}%
\pgfsetrectcap%
\pgfsetroundjoin%
\pgfsetlinewidth{1.505625pt}%
\pgfsetstrokecolor{currentstroke2}%
\pgfsetdash{}{0pt}%
\pgfpathmoveto{\pgfqpoint{0.754752in}{0.697730in}}%
\pgfpathlineto{\pgfqpoint{0.862085in}{0.726228in}}%
\pgfpathlineto{\pgfqpoint{0.969417in}{0.721980in}}%
\pgfpathlineto{\pgfqpoint{1.076750in}{0.670574in}}%
\pgfpathlineto{\pgfqpoint{1.184082in}{0.629772in}}%
\pgfpathlineto{\pgfqpoint{1.291415in}{0.656113in}}%
\pgfpathlineto{\pgfqpoint{1.398747in}{0.787707in}}%
\pgfpathlineto{\pgfqpoint{1.506079in}{0.841695in}}%
\pgfpathlineto{\pgfqpoint{1.613412in}{0.910242in}}%
\pgfpathlineto{\pgfqpoint{1.720744in}{0.960230in}}%
\pgfpathlineto{\pgfqpoint{1.828077in}{1.117634in}}%
\pgfpathlineto{\pgfqpoint{1.935409in}{1.172423in}}%
\pgfpathlineto{\pgfqpoint{2.042742in}{1.271887in}}%
\pgfpathlineto{\pgfqpoint{2.150074in}{1.328528in}}%
\pgfpathlineto{\pgfqpoint{2.257406in}{1.547482in}}%
\pgfpathlineto{\pgfqpoint{2.364739in}{1.541229in}}%
\pgfpathlineto{\pgfqpoint{2.472071in}{1.667240in}}%
\pgfpathlineto{\pgfqpoint{2.579404in}{1.686129in}}%
\pgfpathlineto{\pgfqpoint{2.686736in}{1.913142in}}%
\pgfpathlineto{\pgfqpoint{2.794068in}{1.924575in}}%
\pgfpathlineto{\pgfqpoint{2.901401in}{2.088733in}}%
\pgfpathlineto{\pgfqpoint{3.008733in}{2.080314in}}%
\pgfpathlineto{\pgfqpoint{3.116066in}{2.285529in}}%
\pgfpathlineto{\pgfqpoint{3.223398in}{2.270920in}}%
\pgfpathlineto{\pgfqpoint{3.330731in}{2.425412in}}%
\pgfpathlineto{\pgfqpoint{3.438063in}{2.489646in}}%
\pgfpathlineto{\pgfqpoint{3.545395in}{2.579269in}}%
\pgfpathlineto{\pgfqpoint{3.652728in}{2.566955in}}%
\pgfpathlineto{\pgfqpoint{3.867393in}{2.595649in}}%
\pgfusepath{stroke}%
\end{pgfscope}%
\begin{pgfscope}%
\pgfpathrectangle{\pgfqpoint{0.588387in}{0.521603in}}{\pgfqpoint{3.660036in}{2.220246in}}%
\pgfusepath{clip}%
\pgfsetrectcap%
\pgfsetroundjoin%
\pgfsetlinewidth{1.505625pt}%
\pgfsetstrokecolor{currentstroke3}%
\pgfsetdash{}{0pt}%
\pgfpathmoveto{\pgfqpoint{0.754752in}{0.697730in}}%
\pgfpathlineto{\pgfqpoint{0.862085in}{0.721766in}}%
\pgfpathlineto{\pgfqpoint{0.969417in}{0.737631in}}%
\pgfpathlineto{\pgfqpoint{1.076750in}{0.683133in}}%
\pgfpathlineto{\pgfqpoint{1.184082in}{0.699500in}}%
\pgfpathlineto{\pgfqpoint{1.291415in}{0.647083in}}%
\pgfpathlineto{\pgfqpoint{1.398747in}{0.634451in}}%
\pgfpathlineto{\pgfqpoint{1.506079in}{0.721862in}}%
\pgfpathlineto{\pgfqpoint{1.613412in}{0.761835in}}%
\pgfpathlineto{\pgfqpoint{1.720744in}{0.827815in}}%
\pgfpathlineto{\pgfqpoint{1.828077in}{0.950726in}}%
\pgfpathlineto{\pgfqpoint{1.935409in}{1.037351in}}%
\pgfpathlineto{\pgfqpoint{2.042742in}{1.204115in}}%
\pgfpathlineto{\pgfqpoint{2.150074in}{1.284324in}}%
\pgfpathlineto{\pgfqpoint{2.257406in}{1.506697in}}%
\pgfpathlineto{\pgfqpoint{2.364739in}{1.578728in}}%
\pgfpathlineto{\pgfqpoint{2.472071in}{1.749397in}}%
\pgfpathlineto{\pgfqpoint{2.579404in}{1.871733in}}%
\pgfpathlineto{\pgfqpoint{2.686736in}{2.075802in}}%
\pgfpathlineto{\pgfqpoint{2.794068in}{2.216919in}}%
\pgfpathlineto{\pgfqpoint{2.901401in}{2.408930in}}%
\pgfpathlineto{\pgfqpoint{3.008733in}{2.556885in}}%
\pgfusepath{stroke}%
\end{pgfscope}%
\begin{pgfscope}%
\pgfpathrectangle{\pgfqpoint{0.588387in}{0.521603in}}{\pgfqpoint{3.660036in}{2.220246in}}%
\pgfusepath{clip}%
\pgfsetrectcap%
\pgfsetroundjoin%
\pgfsetlinewidth{1.505625pt}%
\pgfsetstrokecolor{currentstroke4}%
\pgfsetdash{}{0pt}%
\pgfpathmoveto{\pgfqpoint{0.754752in}{0.711577in}}%
\pgfpathlineto{\pgfqpoint{0.862085in}{0.726228in}}%
\pgfpathlineto{\pgfqpoint{0.969417in}{0.727490in}}%
\pgfpathlineto{\pgfqpoint{1.076750in}{0.652537in}}%
\pgfpathlineto{\pgfqpoint{1.184082in}{0.629772in}}%
\pgfpathlineto{\pgfqpoint{1.291415in}{0.654331in}}%
\pgfpathlineto{\pgfqpoint{1.398747in}{0.778384in}}%
\pgfpathlineto{\pgfqpoint{1.506079in}{0.833580in}}%
\pgfpathlineto{\pgfqpoint{1.613412in}{0.906636in}}%
\pgfpathlineto{\pgfqpoint{1.720744in}{0.946820in}}%
\pgfpathlineto{\pgfqpoint{1.828077in}{1.072580in}}%
\pgfpathlineto{\pgfqpoint{1.935409in}{1.146576in}}%
\pgfpathlineto{\pgfqpoint{2.042742in}{1.258551in}}%
\pgfpathlineto{\pgfqpoint{2.150074in}{1.296386in}}%
\pgfpathlineto{\pgfqpoint{2.257406in}{1.504773in}}%
\pgfpathlineto{\pgfqpoint{2.364739in}{1.493077in}}%
\pgfpathlineto{\pgfqpoint{2.472071in}{1.619444in}}%
\pgfpathlineto{\pgfqpoint{2.579404in}{1.626297in}}%
\pgfpathlineto{\pgfqpoint{2.686736in}{1.882176in}}%
\pgfpathlineto{\pgfqpoint{2.794068in}{1.881497in}}%
\pgfpathlineto{\pgfqpoint{2.901401in}{2.032918in}}%
\pgfpathlineto{\pgfqpoint{3.008733in}{1.996980in}}%
\pgfpathlineto{\pgfqpoint{3.116066in}{2.237907in}}%
\pgfpathlineto{\pgfqpoint{3.223398in}{2.218446in}}%
\pgfpathlineto{\pgfqpoint{3.330731in}{2.316130in}}%
\pgfpathlineto{\pgfqpoint{3.438063in}{2.423119in}}%
\pgfpathlineto{\pgfqpoint{3.545395in}{2.376288in}}%
\pgfpathlineto{\pgfqpoint{3.652728in}{2.496663in}}%
\pgfpathlineto{\pgfqpoint{3.867393in}{2.620255in}}%
\pgfusepath{stroke}%
\end{pgfscope}%
\begin{pgfscope}%
\pgfpathrectangle{\pgfqpoint{0.588387in}{0.521603in}}{\pgfqpoint{3.660036in}{2.220246in}}%
\pgfusepath{clip}%
\pgfsetrectcap%
\pgfsetroundjoin%
\pgfsetlinewidth{1.505625pt}%
\pgfsetstrokecolor{currentstroke5}%
\pgfsetdash{}{0pt}%
\pgfpathmoveto{\pgfqpoint{0.754752in}{0.704742in}}%
\pgfpathlineto{\pgfqpoint{0.862085in}{0.730618in}}%
\pgfpathlineto{\pgfqpoint{0.969417in}{0.727490in}}%
\pgfpathlineto{\pgfqpoint{1.076750in}{0.658195in}}%
\pgfpathlineto{\pgfqpoint{1.184082in}{0.633911in}}%
\pgfpathlineto{\pgfqpoint{1.291415in}{0.659642in}}%
\pgfpathlineto{\pgfqpoint{1.398747in}{0.779840in}}%
\pgfpathlineto{\pgfqpoint{1.506079in}{0.836103in}}%
\pgfpathlineto{\pgfqpoint{1.613412in}{0.905727in}}%
\pgfpathlineto{\pgfqpoint{1.720744in}{0.945985in}}%
\pgfpathlineto{\pgfqpoint{1.828077in}{1.090851in}}%
\pgfpathlineto{\pgfqpoint{1.935409in}{1.145136in}}%
\pgfpathlineto{\pgfqpoint{2.042742in}{1.257444in}}%
\pgfpathlineto{\pgfqpoint{2.150074in}{1.292098in}}%
\pgfpathlineto{\pgfqpoint{2.257406in}{1.493301in}}%
\pgfpathlineto{\pgfqpoint{2.364739in}{1.477854in}}%
\pgfpathlineto{\pgfqpoint{2.472071in}{1.615005in}}%
\pgfpathlineto{\pgfqpoint{2.579404in}{1.605381in}}%
\pgfpathlineto{\pgfqpoint{2.686736in}{1.810797in}}%
\pgfpathlineto{\pgfqpoint{2.794068in}{1.838377in}}%
\pgfpathlineto{\pgfqpoint{2.901401in}{1.993688in}}%
\pgfpathlineto{\pgfqpoint{3.008733in}{1.943627in}}%
\pgfpathlineto{\pgfqpoint{3.116066in}{2.210627in}}%
\pgfpathlineto{\pgfqpoint{3.223398in}{2.129025in}}%
\pgfpathlineto{\pgfqpoint{3.330731in}{2.252578in}}%
\pgfpathlineto{\pgfqpoint{3.438063in}{2.326610in}}%
\pgfpathlineto{\pgfqpoint{3.545395in}{2.412804in}}%
\pgfpathlineto{\pgfqpoint{3.652728in}{2.395527in}}%
\pgfpathlineto{\pgfqpoint{3.867393in}{2.629312in}}%
\pgfusepath{stroke}%
\end{pgfscope}%
\begin{pgfscope}%
\pgfpathrectangle{\pgfqpoint{0.588387in}{0.521603in}}{\pgfqpoint{3.660036in}{2.220246in}}%
\pgfusepath{clip}%
\pgfsetrectcap%
\pgfsetroundjoin%
\pgfsetlinewidth{1.505625pt}%
\pgfsetstrokecolor{currentstroke6}%
\pgfsetdash{}{0pt}%
\pgfpathmoveto{\pgfqpoint{0.754752in}{0.718244in}}%
\pgfpathlineto{\pgfqpoint{0.862085in}{0.734938in}}%
\pgfpathlineto{\pgfqpoint{0.969417in}{0.730203in}}%
\pgfpathlineto{\pgfqpoint{1.076750in}{0.656823in}}%
\pgfpathlineto{\pgfqpoint{1.184082in}{0.636830in}}%
\pgfpathlineto{\pgfqpoint{1.291415in}{0.652537in}}%
\pgfpathlineto{\pgfqpoint{1.398747in}{0.776921in}}%
\pgfpathlineto{\pgfqpoint{1.506079in}{0.839844in}}%
\pgfpathlineto{\pgfqpoint{1.613412in}{0.907542in}}%
\pgfpathlineto{\pgfqpoint{1.720744in}{0.951773in}}%
\pgfpathlineto{\pgfqpoint{1.828077in}{1.084991in}}%
\pgfpathlineto{\pgfqpoint{1.935409in}{1.156906in}}%
\pgfpathlineto{\pgfqpoint{2.042742in}{1.276578in}}%
\pgfpathlineto{\pgfqpoint{2.150074in}{1.306725in}}%
\pgfpathlineto{\pgfqpoint{2.257406in}{1.512500in}}%
\pgfpathlineto{\pgfqpoint{2.364739in}{1.521494in}}%
\pgfpathlineto{\pgfqpoint{2.472071in}{1.655873in}}%
\pgfpathlineto{\pgfqpoint{2.579404in}{1.667295in}}%
\pgfpathlineto{\pgfqpoint{2.686736in}{1.931795in}}%
\pgfpathlineto{\pgfqpoint{2.794068in}{1.923991in}}%
\pgfpathlineto{\pgfqpoint{2.901401in}{2.081056in}}%
\pgfpathlineto{\pgfqpoint{3.008733in}{2.056121in}}%
\pgfpathlineto{\pgfqpoint{3.116066in}{2.268970in}}%
\pgfpathlineto{\pgfqpoint{3.223398in}{2.265738in}}%
\pgfpathlineto{\pgfqpoint{3.330731in}{2.451705in}}%
\pgfpathlineto{\pgfqpoint{3.438063in}{2.424055in}}%
\pgfpathlineto{\pgfqpoint{3.545395in}{2.535431in}}%
\pgfpathlineto{\pgfqpoint{3.652728in}{2.518040in}}%
\pgfpathlineto{\pgfqpoint{3.867393in}{2.402243in}}%
\pgfusepath{stroke}%
\end{pgfscope}%
\begin{pgfscope}%
\pgfpathrectangle{\pgfqpoint{0.588387in}{0.521603in}}{\pgfqpoint{3.660036in}{2.220246in}}%
\pgfusepath{clip}%
\pgfsetrectcap%
\pgfsetroundjoin%
\pgfsetlinewidth{1.505625pt}%
\pgfsetstrokecolor{currentstroke7}%
\pgfsetdash{}{0pt}%
\pgfpathmoveto{\pgfqpoint{0.754752in}{0.704742in}}%
\pgfpathlineto{\pgfqpoint{0.862085in}{0.734938in}}%
\pgfpathlineto{\pgfqpoint{0.969417in}{0.737921in}}%
\pgfpathlineto{\pgfqpoint{1.076750in}{0.662514in}}%
\pgfpathlineto{\pgfqpoint{1.184082in}{0.631554in}}%
\pgfpathlineto{\pgfqpoint{1.291415in}{0.650732in}}%
\pgfpathlineto{\pgfqpoint{1.398747in}{0.776187in}}%
\pgfpathlineto{\pgfqpoint{1.506079in}{0.838603in}}%
\pgfpathlineto{\pgfqpoint{1.613412in}{0.902522in}}%
\pgfpathlineto{\pgfqpoint{1.720744in}{0.953811in}}%
\pgfpathlineto{\pgfqpoint{1.828077in}{1.091403in}}%
\pgfpathlineto{\pgfqpoint{1.935409in}{1.149906in}}%
\pgfpathlineto{\pgfqpoint{2.042742in}{1.261436in}}%
\pgfpathlineto{\pgfqpoint{2.150074in}{1.301178in}}%
\pgfpathlineto{\pgfqpoint{2.257406in}{1.494762in}}%
\pgfpathlineto{\pgfqpoint{2.364739in}{1.498649in}}%
\pgfpathlineto{\pgfqpoint{2.472071in}{1.624883in}}%
\pgfpathlineto{\pgfqpoint{2.579404in}{1.623439in}}%
\pgfpathlineto{\pgfqpoint{2.686736in}{1.818359in}}%
\pgfpathlineto{\pgfqpoint{2.794068in}{1.832422in}}%
\pgfpathlineto{\pgfqpoint{2.901401in}{2.037532in}}%
\pgfpathlineto{\pgfqpoint{3.008733in}{1.990884in}}%
\pgfpathlineto{\pgfqpoint{3.116066in}{2.226695in}}%
\pgfpathlineto{\pgfqpoint{3.223398in}{2.169249in}}%
\pgfpathlineto{\pgfqpoint{3.330731in}{2.330314in}}%
\pgfpathlineto{\pgfqpoint{3.438063in}{2.414594in}}%
\pgfpathlineto{\pgfqpoint{3.545395in}{2.422646in}}%
\pgfpathlineto{\pgfqpoint{3.652728in}{2.459847in}}%
\pgfpathlineto{\pgfqpoint{3.867393in}{2.453321in}}%
\pgfpathlineto{\pgfqpoint{4.082057in}{2.640929in}}%
\pgfusepath{stroke}%
\end{pgfscope}%
\begin{pgfscope}%
\pgfpathrectangle{\pgfqpoint{0.588387in}{0.521603in}}{\pgfqpoint{3.660036in}{2.220246in}}%
\pgfusepath{clip}%
\pgfsetrectcap%
\pgfsetroundjoin%
\pgfsetlinewidth{1.505625pt}%
\definecolor{currentstroke}{rgb}{0.498039,0.498039,0.498039}%
\pgfsetstrokecolor{currentstroke}%
\pgfsetdash{}{0pt}%
\pgfpathmoveto{\pgfqpoint{0.754752in}{0.704742in}}%
\pgfpathlineto{\pgfqpoint{0.862085in}{0.739190in}}%
\pgfpathlineto{\pgfqpoint{0.969417in}{0.751355in}}%
\pgfpathlineto{\pgfqpoint{1.076750in}{0.685170in}}%
\pgfpathlineto{\pgfqpoint{1.184082in}{0.642993in}}%
\pgfpathlineto{\pgfqpoint{1.291415in}{0.652807in}}%
\pgfpathlineto{\pgfqpoint{1.398747in}{0.784873in}}%
\pgfpathlineto{\pgfqpoint{1.506079in}{0.850766in}}%
\pgfpathlineto{\pgfqpoint{1.613412in}{0.914683in}}%
\pgfpathlineto{\pgfqpoint{1.720744in}{0.966499in}}%
\pgfpathlineto{\pgfqpoint{1.828077in}{1.128175in}}%
\pgfpathlineto{\pgfqpoint{1.935409in}{1.194143in}}%
\pgfpathlineto{\pgfqpoint{2.042742in}{1.322660in}}%
\pgfpathlineto{\pgfqpoint{2.150074in}{1.353907in}}%
\pgfpathlineto{\pgfqpoint{2.257406in}{1.578001in}}%
\pgfpathlineto{\pgfqpoint{2.364739in}{1.614442in}}%
\pgfpathlineto{\pgfqpoint{2.472071in}{1.756361in}}%
\pgfpathlineto{\pgfqpoint{2.579404in}{1.791764in}}%
\pgfpathlineto{\pgfqpoint{2.686736in}{1.992769in}}%
\pgfpathlineto{\pgfqpoint{2.794068in}{2.016147in}}%
\pgfpathlineto{\pgfqpoint{2.901401in}{2.198116in}}%
\pgfpathlineto{\pgfqpoint{3.008733in}{2.200838in}}%
\pgfpathlineto{\pgfqpoint{3.116066in}{2.452618in}}%
\pgfpathlineto{\pgfqpoint{3.223398in}{2.400988in}}%
\pgfpathlineto{\pgfqpoint{3.330731in}{2.493855in}}%
\pgfpathlineto{\pgfqpoint{3.438063in}{2.442625in}}%
\pgfpathlineto{\pgfqpoint{3.652728in}{2.486318in}}%
\pgfusepath{stroke}%
\end{pgfscope}%
\begin{pgfscope}%
\pgfpathrectangle{\pgfqpoint{0.588387in}{0.521603in}}{\pgfqpoint{3.660036in}{2.220246in}}%
\pgfusepath{clip}%
\pgfsetrectcap%
\pgfsetroundjoin%
\pgfsetlinewidth{1.505625pt}%
\definecolor{currentstroke}{rgb}{0.737255,0.741176,0.133333}%
\pgfsetstrokecolor{currentstroke}%
\pgfsetdash{}{0pt}%
\pgfpathmoveto{\pgfqpoint{0.754752in}{0.704742in}}%
\pgfpathlineto{\pgfqpoint{0.862085in}{0.734938in}}%
\pgfpathlineto{\pgfqpoint{0.969417in}{0.751769in}}%
\pgfpathlineto{\pgfqpoint{1.076750in}{0.669285in}}%
\pgfpathlineto{\pgfqpoint{1.184082in}{0.640417in}}%
\pgfpathlineto{\pgfqpoint{1.291415in}{0.670405in}}%
\pgfpathlineto{\pgfqpoint{1.398747in}{0.776187in}}%
\pgfpathlineto{\pgfqpoint{1.506079in}{0.844144in}}%
\pgfpathlineto{\pgfqpoint{1.613412in}{0.909345in}}%
\pgfpathlineto{\pgfqpoint{1.720744in}{0.963774in}}%
\pgfpathlineto{\pgfqpoint{1.828077in}{1.101673in}}%
\pgfpathlineto{\pgfqpoint{1.935409in}{1.169800in}}%
\pgfpathlineto{\pgfqpoint{2.042742in}{1.287589in}}%
\pgfpathlineto{\pgfqpoint{2.150074in}{1.336978in}}%
\pgfpathlineto{\pgfqpoint{2.257406in}{1.535484in}}%
\pgfpathlineto{\pgfqpoint{2.364739in}{1.567040in}}%
\pgfpathlineto{\pgfqpoint{2.472071in}{1.708142in}}%
\pgfpathlineto{\pgfqpoint{2.579404in}{1.734465in}}%
\pgfpathlineto{\pgfqpoint{2.686736in}{1.916247in}}%
\pgfpathlineto{\pgfqpoint{2.794068in}{1.922386in}}%
\pgfpathlineto{\pgfqpoint{2.901401in}{2.122184in}}%
\pgfpathlineto{\pgfqpoint{3.008733in}{2.121362in}}%
\pgfpathlineto{\pgfqpoint{3.116066in}{2.329061in}}%
\pgfpathlineto{\pgfqpoint{3.223398in}{2.323320in}}%
\pgfpathlineto{\pgfqpoint{3.330731in}{2.479865in}}%
\pgfpathlineto{\pgfqpoint{3.438063in}{2.427825in}}%
\pgfpathlineto{\pgfqpoint{3.545395in}{2.548532in}}%
\pgfpathlineto{\pgfqpoint{3.652728in}{2.527838in}}%
\pgfusepath{stroke}%
\end{pgfscope}%
\begin{pgfscope}%
\pgfsetrectcap%
\pgfsetmiterjoin%
\pgfsetlinewidth{0.803000pt}%
\definecolor{currentstroke}{rgb}{0.000000,0.000000,0.000000}%
\pgfsetstrokecolor{currentstroke}%
\pgfsetdash{}{0pt}%
\pgfpathmoveto{\pgfqpoint{0.588387in}{0.521603in}}%
\pgfpathlineto{\pgfqpoint{0.588387in}{2.741849in}}%
\pgfusepath{stroke}%
\end{pgfscope}%
\begin{pgfscope}%
\pgfsetrectcap%
\pgfsetmiterjoin%
\pgfsetlinewidth{0.803000pt}%
\definecolor{currentstroke}{rgb}{0.000000,0.000000,0.000000}%
\pgfsetstrokecolor{currentstroke}%
\pgfsetdash{}{0pt}%
\pgfpathmoveto{\pgfqpoint{4.248423in}{0.521603in}}%
\pgfpathlineto{\pgfqpoint{4.248423in}{2.741849in}}%
\pgfusepath{stroke}%
\end{pgfscope}%
\begin{pgfscope}%
\pgfsetrectcap%
\pgfsetmiterjoin%
\pgfsetlinewidth{0.803000pt}%
\definecolor{currentstroke}{rgb}{0.000000,0.000000,0.000000}%
\pgfsetstrokecolor{currentstroke}%
\pgfsetdash{}{0pt}%
\pgfpathmoveto{\pgfqpoint{0.588387in}{0.521603in}}%
\pgfpathlineto{\pgfqpoint{4.248423in}{0.521603in}}%
\pgfusepath{stroke}%
\end{pgfscope}%
\begin{pgfscope}%
\pgfsetrectcap%
\pgfsetmiterjoin%
\pgfsetlinewidth{0.803000pt}%
\definecolor{currentstroke}{rgb}{0.000000,0.000000,0.000000}%
\pgfsetstrokecolor{currentstroke}%
\pgfsetdash{}{0pt}%
\pgfpathmoveto{\pgfqpoint{0.588387in}{2.741849in}}%
\pgfpathlineto{\pgfqpoint{4.248423in}{2.741849in}}%
\pgfusepath{stroke}%
\end{pgfscope}%
\begin{pgfscope}%
\pgfsetbuttcap%
\pgfsetmiterjoin%
\definecolor{currentfill}{rgb}{1.000000,1.000000,1.000000}%
\pgfsetfillcolor{currentfill}%
\pgfsetfillopacity{0.800000}%
\pgfsetlinewidth{1.003750pt}%
\definecolor{currentstroke}{rgb}{0.800000,0.800000,0.800000}%
\pgfsetstrokecolor{currentstroke}%
\pgfsetstrokeopacity{0.800000}%
\pgfsetdash{}{0pt}%
\pgfpathmoveto{\pgfqpoint{4.365089in}{0.379025in}}%
\pgfpathlineto{\pgfqpoint{8.251043in}{0.379025in}}%
\pgfpathquadraticcurveto{\pgfqpoint{8.284376in}{0.379025in}}{\pgfqpoint{8.284376in}{0.412359in}}%
\pgfpathlineto{\pgfqpoint{8.284376in}{2.625183in}}%
\pgfpathquadraticcurveto{\pgfqpoint{8.284376in}{2.658516in}}{\pgfqpoint{8.251043in}{2.658516in}}%
\pgfpathlineto{\pgfqpoint{4.365089in}{2.658516in}}%
\pgfpathquadraticcurveto{\pgfqpoint{4.331756in}{2.658516in}}{\pgfqpoint{4.331756in}{2.625183in}}%
\pgfpathlineto{\pgfqpoint{4.331756in}{0.412359in}}%
\pgfpathquadraticcurveto{\pgfqpoint{4.331756in}{0.379025in}}{\pgfqpoint{4.365089in}{0.379025in}}%
\pgfpathlineto{\pgfqpoint{4.365089in}{0.379025in}}%
\pgfpathclose%
\pgfusepath{stroke,fill}%
\end{pgfscope}%
\begin{pgfscope}%
\pgfsetrectcap%
\pgfsetroundjoin%
\pgfsetlinewidth{1.505625pt}%
\pgfsetstrokecolor{currentstroke3}%
\pgfsetdash{}{0pt}%
\pgfpathmoveto{\pgfqpoint{4.398423in}{2.523555in}}%
\pgfpathlineto{\pgfqpoint{4.565089in}{2.523555in}}%
\pgfpathlineto{\pgfqpoint{4.731756in}{2.523555in}}%
\pgfusepath{stroke}%
\end{pgfscope}%
\begin{pgfscope}%
\definecolor{textcolor}{rgb}{0.000000,0.000000,0.000000}%
\pgfsetstrokecolor{textcolor}%
\pgfsetfillcolor{textcolor}%
\pgftext[x=4.865089in,y=2.465222in,left,base]{\color{textcolor}{\rmfamily\fontsize{12.000000}{14.400000}\selectfont\catcode`\^=\active\def^{\ifmmode\sp\else\^{}\fi}\catcode`\%=\active\def%{\%}\NaiveCycles{}}}%
\end{pgfscope}%
\begin{pgfscope}%
\pgfsetrectcap%
\pgfsetroundjoin%
\pgfsetlinewidth{1.505625pt}%
\pgfsetstrokecolor{currentstroke1}%
\pgfsetdash{}{0pt}%
\pgfpathmoveto{\pgfqpoint{4.398423in}{2.278926in}}%
\pgfpathlineto{\pgfqpoint{4.565089in}{2.278926in}}%
\pgfpathlineto{\pgfqpoint{4.731756in}{2.278926in}}%
\pgfusepath{stroke}%
\end{pgfscope}%
\begin{pgfscope}%
\definecolor{textcolor}{rgb}{0.000000,0.000000,0.000000}%
\pgfsetstrokecolor{textcolor}%
\pgfsetfillcolor{textcolor}%
\pgftext[x=4.865089in,y=2.220593in,left,base]{\color{textcolor}{\rmfamily\fontsize{12.000000}{14.400000}\selectfont\catcode`\^=\active\def^{\ifmmode\sp\else\^{}\fi}\catcode`\%=\active\def%{\%}\CyclesMatchChunks{} \& \MergeLinear{}}}%
\end{pgfscope}%
\begin{pgfscope}%
\pgfsetrectcap%
\pgfsetroundjoin%
\pgfsetlinewidth{1.505625pt}%
\pgfsetstrokecolor{currentstroke2}%
\pgfsetdash{}{0pt}%
\pgfpathmoveto{\pgfqpoint{4.398423in}{2.029659in}}%
\pgfpathlineto{\pgfqpoint{4.565089in}{2.029659in}}%
\pgfpathlineto{\pgfqpoint{4.731756in}{2.029659in}}%
\pgfusepath{stroke}%
\end{pgfscope}%
\begin{pgfscope}%
\definecolor{textcolor}{rgb}{0.000000,0.000000,0.000000}%
\pgfsetstrokecolor{textcolor}%
\pgfsetfillcolor{textcolor}%
\pgftext[x=4.865089in,y=1.971325in,left,base]{\color{textcolor}{\rmfamily\fontsize{12.000000}{14.400000}\selectfont\catcode`\^=\active\def^{\ifmmode\sp\else\^{}\fi}\catcode`\%=\active\def%{\%}\CyclesMatchChunks{} \& \SharedVertices{}}}%
\end{pgfscope}%
\begin{pgfscope}%
\pgfsetrectcap%
\pgfsetroundjoin%
\pgfsetlinewidth{1.505625pt}%
\pgfsetstrokecolor{currentstroke4}%
\pgfsetdash{}{0pt}%
\pgfpathmoveto{\pgfqpoint{4.398423in}{1.780391in}}%
\pgfpathlineto{\pgfqpoint{4.565089in}{1.780391in}}%
\pgfpathlineto{\pgfqpoint{4.731756in}{1.780391in}}%
\pgfusepath{stroke}%
\end{pgfscope}%
\begin{pgfscope}%
\definecolor{textcolor}{rgb}{0.000000,0.000000,0.000000}%
\pgfsetstrokecolor{textcolor}%
\pgfsetfillcolor{textcolor}%
\pgftext[x=4.865089in,y=1.722058in,left,base]{\color{textcolor}{\rmfamily\fontsize{12.000000}{14.400000}\selectfont\catcode`\^=\active\def^{\ifmmode\sp\else\^{}\fi}\catcode`\%=\active\def%{\%}\Neighbors{} \& \MergeLinear{}}}%
\end{pgfscope}%
\begin{pgfscope}%
\pgfsetrectcap%
\pgfsetroundjoin%
\pgfsetlinewidth{1.505625pt}%
\pgfsetstrokecolor{currentstroke5}%
\pgfsetdash{}{0pt}%
\pgfpathmoveto{\pgfqpoint{4.398423in}{1.535763in}}%
\pgfpathlineto{\pgfqpoint{4.565089in}{1.535763in}}%
\pgfpathlineto{\pgfqpoint{4.731756in}{1.535763in}}%
\pgfusepath{stroke}%
\end{pgfscope}%
\begin{pgfscope}%
\definecolor{textcolor}{rgb}{0.000000,0.000000,0.000000}%
\pgfsetstrokecolor{textcolor}%
\pgfsetfillcolor{textcolor}%
\pgftext[x=4.865089in,y=1.477429in,left,base]{\color{textcolor}{\rmfamily\fontsize{12.000000}{14.400000}\selectfont\catcode`\^=\active\def^{\ifmmode\sp\else\^{}\fi}\catcode`\%=\active\def%{\%}\Neighbors{} \& \SharedVertices{}}}%
\end{pgfscope}%
\begin{pgfscope}%
\pgfsetrectcap%
\pgfsetroundjoin%
\pgfsetlinewidth{1.505625pt}%
\pgfsetstrokecolor{currentstroke6}%
\pgfsetdash{}{0pt}%
\pgfpathmoveto{\pgfqpoint{4.398423in}{1.286495in}}%
\pgfpathlineto{\pgfqpoint{4.565089in}{1.286495in}}%
\pgfpathlineto{\pgfqpoint{4.731756in}{1.286495in}}%
\pgfusepath{stroke}%
\end{pgfscope}%
\begin{pgfscope}%
\definecolor{textcolor}{rgb}{0.000000,0.000000,0.000000}%
\pgfsetstrokecolor{textcolor}%
\pgfsetfillcolor{textcolor}%
\pgftext[x=4.865089in,y=1.228162in,left,base]{\color{textcolor}{\rmfamily\fontsize{12.000000}{14.400000}\selectfont\catcode`\^=\active\def^{\ifmmode\sp\else\^{}\fi}\catcode`\%=\active\def%{\%}\NeighborsDegree{} \& \MergeLinear{}}}%
\end{pgfscope}%
\begin{pgfscope}%
\pgfsetrectcap%
\pgfsetroundjoin%
\pgfsetlinewidth{1.505625pt}%
\pgfsetstrokecolor{currentstroke7}%
\pgfsetdash{}{0pt}%
\pgfpathmoveto{\pgfqpoint{4.398423in}{1.037228in}}%
\pgfpathlineto{\pgfqpoint{4.565089in}{1.037228in}}%
\pgfpathlineto{\pgfqpoint{4.731756in}{1.037228in}}%
\pgfusepath{stroke}%
\end{pgfscope}%
\begin{pgfscope}%
\definecolor{textcolor}{rgb}{0.000000,0.000000,0.000000}%
\pgfsetstrokecolor{textcolor}%
\pgfsetfillcolor{textcolor}%
\pgftext[x=4.865089in,y=0.978895in,left,base]{\color{textcolor}{\rmfamily\fontsize{12.000000}{14.400000}\selectfont\catcode`\^=\active\def^{\ifmmode\sp\else\^{}\fi}\catcode`\%=\active\def%{\%}\NeighborsDegree{} \& \SharedVertices{}}}%
\end{pgfscope}%
\begin{pgfscope}%
\pgfsetrectcap%
\pgfsetroundjoin%
\pgfsetlinewidth{1.505625pt}%
\definecolor{currentstroke}{rgb}{0.498039,0.498039,0.498039}%
\pgfsetstrokecolor{currentstroke}%
\pgfsetdash{}{0pt}%
\pgfpathmoveto{\pgfqpoint{4.398423in}{0.787961in}}%
\pgfpathlineto{\pgfqpoint{4.565089in}{0.787961in}}%
\pgfpathlineto{\pgfqpoint{4.731756in}{0.787961in}}%
\pgfusepath{stroke}%
\end{pgfscope}%
\begin{pgfscope}%
\definecolor{textcolor}{rgb}{0.000000,0.000000,0.000000}%
\pgfsetstrokecolor{textcolor}%
\pgfsetfillcolor{textcolor}%
\pgftext[x=4.865089in,y=0.729627in,left,base]{\color{textcolor}{\rmfamily\fontsize{12.000000}{14.400000}\selectfont\catcode`\^=\active\def^{\ifmmode\sp\else\^{}\fi}\catcode`\%=\active\def%{\%}\None{} \& \MergeLinear{}}}%
\end{pgfscope}%
\begin{pgfscope}%
\pgfsetrectcap%
\pgfsetroundjoin%
\pgfsetlinewidth{1.505625pt}%
\definecolor{currentstroke}{rgb}{0.737255,0.741176,0.133333}%
\pgfsetstrokecolor{currentstroke}%
\pgfsetdash{}{0pt}%
\pgfpathmoveto{\pgfqpoint{4.398423in}{0.543332in}}%
\pgfpathlineto{\pgfqpoint{4.565089in}{0.543332in}}%
\pgfpathlineto{\pgfqpoint{4.731756in}{0.543332in}}%
\pgfusepath{stroke}%
\end{pgfscope}%
\begin{pgfscope}%
\definecolor{textcolor}{rgb}{0.000000,0.000000,0.000000}%
\pgfsetstrokecolor{textcolor}%
\pgfsetfillcolor{textcolor}%
\pgftext[x=4.865089in,y=0.484999in,left,base]{\color{textcolor}{\rmfamily\fontsize{12.000000}{14.400000}\selectfont\catcode`\^=\active\def^{\ifmmode\sp\else\^{}\fi}\catcode`\%=\active\def%{\%}\None{} \& \SharedVertices{}}}%
\end{pgfscope}%
\end{pgfpicture}%
\makeatother%
\endgroup%
}
	\caption[Running time for minimally rigid graphs]{
		Mean running time to find all NAC-colorings for minimally rigid graphs.}%
	\label{fig:graph_time_minimally_rigid}
\end{figure}%

If we analyze the number of \IsNACColoring{} calls performed by \NaiveCycles{} and \Subgraphs{} algorithms
as shown in \Cref{fig:graph_count_minimally_rigid},
it can be seen that the number of \IsNACColoring{} calls is reduced already for graphs
with eleven vertices,
even though the \NaiveCycles{} algorithm is still faster for these graphs.
%
We think that the slowdown is caused by internal overhead
used for subgraphs splitting and merging.

\todo[inline]{Podívat se, co se kurňa děje - all vs some}
\begin{figure}[ht]
	\centering
	\scalebox{\BenchFigureScale}{%% Creator: Matplotlib, PGF backend
%%
%% To include the figure in your LaTeX document, write
%%   \input{<filename>.pgf}
%%
%% Make sure the required packages are loaded in your preamble
%%   \usepackage{pgf}
%%
%% Also ensure that all the required font packages are loaded; for instance,
%% the lmodern package is sometimes necessary when using math font.
%%   \usepackage{lmodern}
%%
%% Figures using additional raster images can only be included by \input if
%% they are in the same directory as the main LaTeX file. For loading figures
%% from other directories you can use the `import` package
%%   \usepackage{import}
%%
%% and then include the figures with
%%   \import{<path to file>}{<filename>.pgf}
%%
%% Matplotlib used the following preamble
%%   \def\mathdefault#1{#1}
%%   \everymath=\expandafter{\the\everymath\displaystyle}
%%   \IfFileExists{scrextend.sty}{
%%     \usepackage[fontsize=10.000000pt]{scrextend}
%%   }{
%%     \renewcommand{\normalsize}{\fontsize{10.000000}{12.000000}\selectfont}
%%     \normalsize
%%   }
%%   
%%   \ifdefined\pdftexversion\else  % non-pdftex case.
%%     \usepackage{fontspec}
%%     \setmainfont{DejaVuSans.ttf}[Path=\detokenize{/home/petr/Projects/PyRigi/.venv/lib/python3.12/site-packages/matplotlib/mpl-data/fonts/ttf/}]
%%     \setsansfont{DejaVuSans.ttf}[Path=\detokenize{/home/petr/Projects/PyRigi/.venv/lib/python3.12/site-packages/matplotlib/mpl-data/fonts/ttf/}]
%%     \setmonofont{DejaVuSansMono.ttf}[Path=\detokenize{/home/petr/Projects/PyRigi/.venv/lib/python3.12/site-packages/matplotlib/mpl-data/fonts/ttf/}]
%%   \fi
%%   \makeatletter\@ifpackageloaded{under\Score{}}{}{\usepackage[strings]{under\Score{}}}\makeatother
%%
\begingroup%
\makeatletter%
\begin{pgfpicture}%
\pgfpathrectangle{\pgfpointorigin}{\pgfqpoint{8.384376in}{2.841849in}}%
\pgfusepath{use as bounding box, clip}%
\begin{pgfscope}%
\pgfsetbuttcap%
\pgfsetmiterjoin%
\definecolor{currentfill}{rgb}{1.000000,1.000000,1.000000}%
\pgfsetfillcolor{currentfill}%
\pgfsetlinewidth{0.000000pt}%
\definecolor{currentstroke}{rgb}{1.000000,1.000000,1.000000}%
\pgfsetstrokecolor{currentstroke}%
\pgfsetdash{}{0pt}%
\pgfpathmoveto{\pgfqpoint{0.000000in}{0.000000in}}%
\pgfpathlineto{\pgfqpoint{8.384376in}{0.000000in}}%
\pgfpathlineto{\pgfqpoint{8.384376in}{2.841849in}}%
\pgfpathlineto{\pgfqpoint{0.000000in}{2.841849in}}%
\pgfpathlineto{\pgfqpoint{0.000000in}{0.000000in}}%
\pgfpathclose%
\pgfusepath{fill}%
\end{pgfscope}%
\begin{pgfscope}%
\pgfsetbuttcap%
\pgfsetmiterjoin%
\definecolor{currentfill}{rgb}{1.000000,1.000000,1.000000}%
\pgfsetfillcolor{currentfill}%
\pgfsetlinewidth{0.000000pt}%
\definecolor{currentstroke}{rgb}{0.000000,0.000000,0.000000}%
\pgfsetstrokecolor{currentstroke}%
\pgfsetstrokeopacity{0.000000}%
\pgfsetdash{}{0pt}%
\pgfpathmoveto{\pgfqpoint{0.588387in}{0.521603in}}%
\pgfpathlineto{\pgfqpoint{4.248423in}{0.521603in}}%
\pgfpathlineto{\pgfqpoint{4.248423in}{2.741849in}}%
\pgfpathlineto{\pgfqpoint{0.588387in}{2.741849in}}%
\pgfpathlineto{\pgfqpoint{0.588387in}{0.521603in}}%
\pgfpathclose%
\pgfusepath{fill}%
\end{pgfscope}%
\begin{pgfscope}%
\pgfsetbuttcap%
\pgfsetroundjoin%
\definecolor{currentfill}{rgb}{0.000000,0.000000,0.000000}%
\pgfsetfillcolor{currentfill}%
\pgfsetlinewidth{0.803000pt}%
\definecolor{currentstroke}{rgb}{0.000000,0.000000,0.000000}%
\pgfsetstrokecolor{currentstroke}%
\pgfsetdash{}{0pt}%
\pgfsys@defobject{currentmarker}{\pgfqpoint{0.000000in}{-0.048611in}}{\pgfqpoint{0.000000in}{0.000000in}}{%
\pgfpathmoveto{\pgfqpoint{0.000000in}{0.000000in}}%
\pgfpathlineto{\pgfqpoint{0.000000in}{-0.048611in}}%
\pgfusepath{stroke,fill}%
}%
\begin{pgfscope}%
\pgfsys@transformshift{0.969417in}{0.521603in}%
\pgfsys@useobject{currentmarker}{}%
\end{pgfscope}%
\end{pgfscope}%
\begin{pgfscope}%
\definecolor{textcolor}{rgb}{0.000000,0.000000,0.000000}%
\pgfsetstrokecolor{textcolor}%
\pgfsetfillcolor{textcolor}%
\pgftext[x=0.969417in,y=0.424381in,,top]{\color{textcolor}{\rmfamily\fontsize{10.000000}{12.000000}\selectfont\catcode`\^=\active\def^{\ifmmode\sp\else\^{}\fi}\catcode`\%=\active\def%{\%}$\mathdefault{4}$}}%
\end{pgfscope}%
\begin{pgfscope}%
\pgfsetbuttcap%
\pgfsetroundjoin%
\definecolor{currentfill}{rgb}{0.000000,0.000000,0.000000}%
\pgfsetfillcolor{currentfill}%
\pgfsetlinewidth{0.803000pt}%
\definecolor{currentstroke}{rgb}{0.000000,0.000000,0.000000}%
\pgfsetstrokecolor{currentstroke}%
\pgfsetdash{}{0pt}%
\pgfsys@defobject{currentmarker}{\pgfqpoint{0.000000in}{-0.048611in}}{\pgfqpoint{0.000000in}{0.000000in}}{%
\pgfpathmoveto{\pgfqpoint{0.000000in}{0.000000in}}%
\pgfpathlineto{\pgfqpoint{0.000000in}{-0.048611in}}%
\pgfusepath{stroke,fill}%
}%
\begin{pgfscope}%
\pgfsys@transformshift{1.398747in}{0.521603in}%
\pgfsys@useobject{currentmarker}{}%
\end{pgfscope}%
\end{pgfscope}%
\begin{pgfscope}%
\definecolor{textcolor}{rgb}{0.000000,0.000000,0.000000}%
\pgfsetstrokecolor{textcolor}%
\pgfsetfillcolor{textcolor}%
\pgftext[x=1.398747in,y=0.424381in,,top]{\color{textcolor}{\rmfamily\fontsize{10.000000}{12.000000}\selectfont\catcode`\^=\active\def^{\ifmmode\sp\else\^{}\fi}\catcode`\%=\active\def%{\%}$\mathdefault{8}$}}%
\end{pgfscope}%
\begin{pgfscope}%
\pgfsetbuttcap%
\pgfsetroundjoin%
\definecolor{currentfill}{rgb}{0.000000,0.000000,0.000000}%
\pgfsetfillcolor{currentfill}%
\pgfsetlinewidth{0.803000pt}%
\definecolor{currentstroke}{rgb}{0.000000,0.000000,0.000000}%
\pgfsetstrokecolor{currentstroke}%
\pgfsetdash{}{0pt}%
\pgfsys@defobject{currentmarker}{\pgfqpoint{0.000000in}{-0.048611in}}{\pgfqpoint{0.000000in}{0.000000in}}{%
\pgfpathmoveto{\pgfqpoint{0.000000in}{0.000000in}}%
\pgfpathlineto{\pgfqpoint{0.000000in}{-0.048611in}}%
\pgfusepath{stroke,fill}%
}%
\begin{pgfscope}%
\pgfsys@transformshift{1.828077in}{0.521603in}%
\pgfsys@useobject{currentmarker}{}%
\end{pgfscope}%
\end{pgfscope}%
\begin{pgfscope}%
\definecolor{textcolor}{rgb}{0.000000,0.000000,0.000000}%
\pgfsetstrokecolor{textcolor}%
\pgfsetfillcolor{textcolor}%
\pgftext[x=1.828077in,y=0.424381in,,top]{\color{textcolor}{\rmfamily\fontsize{10.000000}{12.000000}\selectfont\catcode`\^=\active\def^{\ifmmode\sp\else\^{}\fi}\catcode`\%=\active\def%{\%}$\mathdefault{12}$}}%
\end{pgfscope}%
\begin{pgfscope}%
\pgfsetbuttcap%
\pgfsetroundjoin%
\definecolor{currentfill}{rgb}{0.000000,0.000000,0.000000}%
\pgfsetfillcolor{currentfill}%
\pgfsetlinewidth{0.803000pt}%
\definecolor{currentstroke}{rgb}{0.000000,0.000000,0.000000}%
\pgfsetstrokecolor{currentstroke}%
\pgfsetdash{}{0pt}%
\pgfsys@defobject{currentmarker}{\pgfqpoint{0.000000in}{-0.048611in}}{\pgfqpoint{0.000000in}{0.000000in}}{%
\pgfpathmoveto{\pgfqpoint{0.000000in}{0.000000in}}%
\pgfpathlineto{\pgfqpoint{0.000000in}{-0.048611in}}%
\pgfusepath{stroke,fill}%
}%
\begin{pgfscope}%
\pgfsys@transformshift{2.257406in}{0.521603in}%
\pgfsys@useobject{currentmarker}{}%
\end{pgfscope}%
\end{pgfscope}%
\begin{pgfscope}%
\definecolor{textcolor}{rgb}{0.000000,0.000000,0.000000}%
\pgfsetstrokecolor{textcolor}%
\pgfsetfillcolor{textcolor}%
\pgftext[x=2.257406in,y=0.424381in,,top]{\color{textcolor}{\rmfamily\fontsize{10.000000}{12.000000}\selectfont\catcode`\^=\active\def^{\ifmmode\sp\else\^{}\fi}\catcode`\%=\active\def%{\%}$\mathdefault{16}$}}%
\end{pgfscope}%
\begin{pgfscope}%
\pgfsetbuttcap%
\pgfsetroundjoin%
\definecolor{currentfill}{rgb}{0.000000,0.000000,0.000000}%
\pgfsetfillcolor{currentfill}%
\pgfsetlinewidth{0.803000pt}%
\definecolor{currentstroke}{rgb}{0.000000,0.000000,0.000000}%
\pgfsetstrokecolor{currentstroke}%
\pgfsetdash{}{0pt}%
\pgfsys@defobject{currentmarker}{\pgfqpoint{0.000000in}{-0.048611in}}{\pgfqpoint{0.000000in}{0.000000in}}{%
\pgfpathmoveto{\pgfqpoint{0.000000in}{0.000000in}}%
\pgfpathlineto{\pgfqpoint{0.000000in}{-0.048611in}}%
\pgfusepath{stroke,fill}%
}%
\begin{pgfscope}%
\pgfsys@transformshift{2.686736in}{0.521603in}%
\pgfsys@useobject{currentmarker}{}%
\end{pgfscope}%
\end{pgfscope}%
\begin{pgfscope}%
\definecolor{textcolor}{rgb}{0.000000,0.000000,0.000000}%
\pgfsetstrokecolor{textcolor}%
\pgfsetfillcolor{textcolor}%
\pgftext[x=2.686736in,y=0.424381in,,top]{\color{textcolor}{\rmfamily\fontsize{10.000000}{12.000000}\selectfont\catcode`\^=\active\def^{\ifmmode\sp\else\^{}\fi}\catcode`\%=\active\def%{\%}$\mathdefault{20}$}}%
\end{pgfscope}%
\begin{pgfscope}%
\pgfsetbuttcap%
\pgfsetroundjoin%
\definecolor{currentfill}{rgb}{0.000000,0.000000,0.000000}%
\pgfsetfillcolor{currentfill}%
\pgfsetlinewidth{0.803000pt}%
\definecolor{currentstroke}{rgb}{0.000000,0.000000,0.000000}%
\pgfsetstrokecolor{currentstroke}%
\pgfsetdash{}{0pt}%
\pgfsys@defobject{currentmarker}{\pgfqpoint{0.000000in}{-0.048611in}}{\pgfqpoint{0.000000in}{0.000000in}}{%
\pgfpathmoveto{\pgfqpoint{0.000000in}{0.000000in}}%
\pgfpathlineto{\pgfqpoint{0.000000in}{-0.048611in}}%
\pgfusepath{stroke,fill}%
}%
\begin{pgfscope}%
\pgfsys@transformshift{3.116066in}{0.521603in}%
\pgfsys@useobject{currentmarker}{}%
\end{pgfscope}%
\end{pgfscope}%
\begin{pgfscope}%
\definecolor{textcolor}{rgb}{0.000000,0.000000,0.000000}%
\pgfsetstrokecolor{textcolor}%
\pgfsetfillcolor{textcolor}%
\pgftext[x=3.116066in,y=0.424381in,,top]{\color{textcolor}{\rmfamily\fontsize{10.000000}{12.000000}\selectfont\catcode`\^=\active\def^{\ifmmode\sp\else\^{}\fi}\catcode`\%=\active\def%{\%}$\mathdefault{24}$}}%
\end{pgfscope}%
\begin{pgfscope}%
\pgfsetbuttcap%
\pgfsetroundjoin%
\definecolor{currentfill}{rgb}{0.000000,0.000000,0.000000}%
\pgfsetfillcolor{currentfill}%
\pgfsetlinewidth{0.803000pt}%
\definecolor{currentstroke}{rgb}{0.000000,0.000000,0.000000}%
\pgfsetstrokecolor{currentstroke}%
\pgfsetdash{}{0pt}%
\pgfsys@defobject{currentmarker}{\pgfqpoint{0.000000in}{-0.048611in}}{\pgfqpoint{0.000000in}{0.000000in}}{%
\pgfpathmoveto{\pgfqpoint{0.000000in}{0.000000in}}%
\pgfpathlineto{\pgfqpoint{0.000000in}{-0.048611in}}%
\pgfusepath{stroke,fill}%
}%
\begin{pgfscope}%
\pgfsys@transformshift{3.545395in}{0.521603in}%
\pgfsys@useobject{currentmarker}{}%
\end{pgfscope}%
\end{pgfscope}%
\begin{pgfscope}%
\definecolor{textcolor}{rgb}{0.000000,0.000000,0.000000}%
\pgfsetstrokecolor{textcolor}%
\pgfsetfillcolor{textcolor}%
\pgftext[x=3.545395in,y=0.424381in,,top]{\color{textcolor}{\rmfamily\fontsize{10.000000}{12.000000}\selectfont\catcode`\^=\active\def^{\ifmmode\sp\else\^{}\fi}\catcode`\%=\active\def%{\%}$\mathdefault{28}$}}%
\end{pgfscope}%
\begin{pgfscope}%
\pgfsetbuttcap%
\pgfsetroundjoin%
\definecolor{currentfill}{rgb}{0.000000,0.000000,0.000000}%
\pgfsetfillcolor{currentfill}%
\pgfsetlinewidth{0.803000pt}%
\definecolor{currentstroke}{rgb}{0.000000,0.000000,0.000000}%
\pgfsetstrokecolor{currentstroke}%
\pgfsetdash{}{0pt}%
\pgfsys@defobject{currentmarker}{\pgfqpoint{0.000000in}{-0.048611in}}{\pgfqpoint{0.000000in}{0.000000in}}{%
\pgfpathmoveto{\pgfqpoint{0.000000in}{0.000000in}}%
\pgfpathlineto{\pgfqpoint{0.000000in}{-0.048611in}}%
\pgfusepath{stroke,fill}%
}%
\begin{pgfscope}%
\pgfsys@transformshift{3.974725in}{0.521603in}%
\pgfsys@useobject{currentmarker}{}%
\end{pgfscope}%
\end{pgfscope}%
\begin{pgfscope}%
\definecolor{textcolor}{rgb}{0.000000,0.000000,0.000000}%
\pgfsetstrokecolor{textcolor}%
\pgfsetfillcolor{textcolor}%
\pgftext[x=3.974725in,y=0.424381in,,top]{\color{textcolor}{\rmfamily\fontsize{10.000000}{12.000000}\selectfont\catcode`\^=\active\def^{\ifmmode\sp\else\^{}\fi}\catcode`\%=\active\def%{\%}$\mathdefault{32}$}}%
\end{pgfscope}%
\begin{pgfscope}%
\definecolor{textcolor}{rgb}{0.000000,0.000000,0.000000}%
\pgfsetstrokecolor{textcolor}%
\pgfsetfillcolor{textcolor}%
\pgftext[x=2.418405in,y=0.234413in,,top]{\color{textcolor}{\rmfamily\fontsize{10.000000}{12.000000}\selectfont\catcode`\^=\active\def^{\ifmmode\sp\else\^{}\fi}\catcode`\%=\active\def%{\%}Monochromatic classes}}%
\end{pgfscope}%
\begin{pgfscope}%
\pgfsetbuttcap%
\pgfsetroundjoin%
\definecolor{currentfill}{rgb}{0.000000,0.000000,0.000000}%
\pgfsetfillcolor{currentfill}%
\pgfsetlinewidth{0.803000pt}%
\definecolor{currentstroke}{rgb}{0.000000,0.000000,0.000000}%
\pgfsetstrokecolor{currentstroke}%
\pgfsetdash{}{0pt}%
\pgfsys@defobject{currentmarker}{\pgfqpoint{-0.048611in}{0.000000in}}{\pgfqpoint{-0.000000in}{0.000000in}}{%
\pgfpathmoveto{\pgfqpoint{-0.000000in}{0.000000in}}%
\pgfpathlineto{\pgfqpoint{-0.048611in}{0.000000in}}%
\pgfusepath{stroke,fill}%
}%
\begin{pgfscope}%
\pgfsys@transformshift{0.588387in}{0.622524in}%
\pgfsys@useobject{currentmarker}{}%
\end{pgfscope}%
\end{pgfscope}%
\begin{pgfscope}%
\definecolor{textcolor}{rgb}{0.000000,0.000000,0.000000}%
\pgfsetstrokecolor{textcolor}%
\pgfsetfillcolor{textcolor}%
\pgftext[x=0.289968in, y=0.569762in, left, base]{\color{textcolor}{\rmfamily\fontsize{10.000000}{12.000000}\selectfont\catcode`\^=\active\def^{\ifmmode\sp\else\^{}\fi}\catcode`\%=\active\def%{\%}$\mathdefault{10^{0}}$}}%
\end{pgfscope}%
\begin{pgfscope}%
\pgfsetbuttcap%
\pgfsetroundjoin%
\definecolor{currentfill}{rgb}{0.000000,0.000000,0.000000}%
\pgfsetfillcolor{currentfill}%
\pgfsetlinewidth{0.803000pt}%
\definecolor{currentstroke}{rgb}{0.000000,0.000000,0.000000}%
\pgfsetstrokecolor{currentstroke}%
\pgfsetdash{}{0pt}%
\pgfsys@defobject{currentmarker}{\pgfqpoint{-0.048611in}{0.000000in}}{\pgfqpoint{-0.000000in}{0.000000in}}{%
\pgfpathmoveto{\pgfqpoint{-0.000000in}{0.000000in}}%
\pgfpathlineto{\pgfqpoint{-0.048611in}{0.000000in}}%
\pgfusepath{stroke,fill}%
}%
\begin{pgfscope}%
\pgfsys@transformshift{0.588387in}{0.927296in}%
\pgfsys@useobject{currentmarker}{}%
\end{pgfscope}%
\end{pgfscope}%
\begin{pgfscope}%
\definecolor{textcolor}{rgb}{0.000000,0.000000,0.000000}%
\pgfsetstrokecolor{textcolor}%
\pgfsetfillcolor{textcolor}%
\pgftext[x=0.289968in, y=0.874535in, left, base]{\color{textcolor}{\rmfamily\fontsize{10.000000}{12.000000}\selectfont\catcode`\^=\active\def^{\ifmmode\sp\else\^{}\fi}\catcode`\%=\active\def%{\%}$\mathdefault{10^{1}}$}}%
\end{pgfscope}%
\begin{pgfscope}%
\pgfsetbuttcap%
\pgfsetroundjoin%
\definecolor{currentfill}{rgb}{0.000000,0.000000,0.000000}%
\pgfsetfillcolor{currentfill}%
\pgfsetlinewidth{0.803000pt}%
\definecolor{currentstroke}{rgb}{0.000000,0.000000,0.000000}%
\pgfsetstrokecolor{currentstroke}%
\pgfsetdash{}{0pt}%
\pgfsys@defobject{currentmarker}{\pgfqpoint{-0.048611in}{0.000000in}}{\pgfqpoint{-0.000000in}{0.000000in}}{%
\pgfpathmoveto{\pgfqpoint{-0.000000in}{0.000000in}}%
\pgfpathlineto{\pgfqpoint{-0.048611in}{0.000000in}}%
\pgfusepath{stroke,fill}%
}%
\begin{pgfscope}%
\pgfsys@transformshift{0.588387in}{1.232069in}%
\pgfsys@useobject{currentmarker}{}%
\end{pgfscope}%
\end{pgfscope}%
\begin{pgfscope}%
\definecolor{textcolor}{rgb}{0.000000,0.000000,0.000000}%
\pgfsetstrokecolor{textcolor}%
\pgfsetfillcolor{textcolor}%
\pgftext[x=0.289968in, y=1.179307in, left, base]{\color{textcolor}{\rmfamily\fontsize{10.000000}{12.000000}\selectfont\catcode`\^=\active\def^{\ifmmode\sp\else\^{}\fi}\catcode`\%=\active\def%{\%}$\mathdefault{10^{2}}$}}%
\end{pgfscope}%
\begin{pgfscope}%
\pgfsetbuttcap%
\pgfsetroundjoin%
\definecolor{currentfill}{rgb}{0.000000,0.000000,0.000000}%
\pgfsetfillcolor{currentfill}%
\pgfsetlinewidth{0.803000pt}%
\definecolor{currentstroke}{rgb}{0.000000,0.000000,0.000000}%
\pgfsetstrokecolor{currentstroke}%
\pgfsetdash{}{0pt}%
\pgfsys@defobject{currentmarker}{\pgfqpoint{-0.048611in}{0.000000in}}{\pgfqpoint{-0.000000in}{0.000000in}}{%
\pgfpathmoveto{\pgfqpoint{-0.000000in}{0.000000in}}%
\pgfpathlineto{\pgfqpoint{-0.048611in}{0.000000in}}%
\pgfusepath{stroke,fill}%
}%
\begin{pgfscope}%
\pgfsys@transformshift{0.588387in}{1.536841in}%
\pgfsys@useobject{currentmarker}{}%
\end{pgfscope}%
\end{pgfscope}%
\begin{pgfscope}%
\definecolor{textcolor}{rgb}{0.000000,0.000000,0.000000}%
\pgfsetstrokecolor{textcolor}%
\pgfsetfillcolor{textcolor}%
\pgftext[x=0.289968in, y=1.484080in, left, base]{\color{textcolor}{\rmfamily\fontsize{10.000000}{12.000000}\selectfont\catcode`\^=\active\def^{\ifmmode\sp\else\^{}\fi}\catcode`\%=\active\def%{\%}$\mathdefault{10^{3}}$}}%
\end{pgfscope}%
\begin{pgfscope}%
\pgfsetbuttcap%
\pgfsetroundjoin%
\definecolor{currentfill}{rgb}{0.000000,0.000000,0.000000}%
\pgfsetfillcolor{currentfill}%
\pgfsetlinewidth{0.803000pt}%
\definecolor{currentstroke}{rgb}{0.000000,0.000000,0.000000}%
\pgfsetstrokecolor{currentstroke}%
\pgfsetdash{}{0pt}%
\pgfsys@defobject{currentmarker}{\pgfqpoint{-0.048611in}{0.000000in}}{\pgfqpoint{-0.000000in}{0.000000in}}{%
\pgfpathmoveto{\pgfqpoint{-0.000000in}{0.000000in}}%
\pgfpathlineto{\pgfqpoint{-0.048611in}{0.000000in}}%
\pgfusepath{stroke,fill}%
}%
\begin{pgfscope}%
\pgfsys@transformshift{0.588387in}{1.841614in}%
\pgfsys@useobject{currentmarker}{}%
\end{pgfscope}%
\end{pgfscope}%
\begin{pgfscope}%
\definecolor{textcolor}{rgb}{0.000000,0.000000,0.000000}%
\pgfsetstrokecolor{textcolor}%
\pgfsetfillcolor{textcolor}%
\pgftext[x=0.289968in, y=1.788853in, left, base]{\color{textcolor}{\rmfamily\fontsize{10.000000}{12.000000}\selectfont\catcode`\^=\active\def^{\ifmmode\sp\else\^{}\fi}\catcode`\%=\active\def%{\%}$\mathdefault{10^{4}}$}}%
\end{pgfscope}%
\begin{pgfscope}%
\pgfsetbuttcap%
\pgfsetroundjoin%
\definecolor{currentfill}{rgb}{0.000000,0.000000,0.000000}%
\pgfsetfillcolor{currentfill}%
\pgfsetlinewidth{0.803000pt}%
\definecolor{currentstroke}{rgb}{0.000000,0.000000,0.000000}%
\pgfsetstrokecolor{currentstroke}%
\pgfsetdash{}{0pt}%
\pgfsys@defobject{currentmarker}{\pgfqpoint{-0.048611in}{0.000000in}}{\pgfqpoint{-0.000000in}{0.000000in}}{%
\pgfpathmoveto{\pgfqpoint{-0.000000in}{0.000000in}}%
\pgfpathlineto{\pgfqpoint{-0.048611in}{0.000000in}}%
\pgfusepath{stroke,fill}%
}%
\begin{pgfscope}%
\pgfsys@transformshift{0.588387in}{2.146387in}%
\pgfsys@useobject{currentmarker}{}%
\end{pgfscope}%
\end{pgfscope}%
\begin{pgfscope}%
\definecolor{textcolor}{rgb}{0.000000,0.000000,0.000000}%
\pgfsetstrokecolor{textcolor}%
\pgfsetfillcolor{textcolor}%
\pgftext[x=0.289968in, y=2.093625in, left, base]{\color{textcolor}{\rmfamily\fontsize{10.000000}{12.000000}\selectfont\catcode`\^=\active\def^{\ifmmode\sp\else\^{}\fi}\catcode`\%=\active\def%{\%}$\mathdefault{10^{5}}$}}%
\end{pgfscope}%
\begin{pgfscope}%
\pgfsetbuttcap%
\pgfsetroundjoin%
\definecolor{currentfill}{rgb}{0.000000,0.000000,0.000000}%
\pgfsetfillcolor{currentfill}%
\pgfsetlinewidth{0.803000pt}%
\definecolor{currentstroke}{rgb}{0.000000,0.000000,0.000000}%
\pgfsetstrokecolor{currentstroke}%
\pgfsetdash{}{0pt}%
\pgfsys@defobject{currentmarker}{\pgfqpoint{-0.048611in}{0.000000in}}{\pgfqpoint{-0.000000in}{0.000000in}}{%
\pgfpathmoveto{\pgfqpoint{-0.000000in}{0.000000in}}%
\pgfpathlineto{\pgfqpoint{-0.048611in}{0.000000in}}%
\pgfusepath{stroke,fill}%
}%
\begin{pgfscope}%
\pgfsys@transformshift{0.588387in}{2.451159in}%
\pgfsys@useobject{currentmarker}{}%
\end{pgfscope}%
\end{pgfscope}%
\begin{pgfscope}%
\definecolor{textcolor}{rgb}{0.000000,0.000000,0.000000}%
\pgfsetstrokecolor{textcolor}%
\pgfsetfillcolor{textcolor}%
\pgftext[x=0.289968in, y=2.398398in, left, base]{\color{textcolor}{\rmfamily\fontsize{10.000000}{12.000000}\selectfont\catcode`\^=\active\def^{\ifmmode\sp\else\^{}\fi}\catcode`\%=\active\def%{\%}$\mathdefault{10^{6}}$}}%
\end{pgfscope}%
\begin{pgfscope}%
\pgfsetbuttcap%
\pgfsetroundjoin%
\definecolor{currentfill}{rgb}{0.000000,0.000000,0.000000}%
\pgfsetfillcolor{currentfill}%
\pgfsetlinewidth{0.602250pt}%
\definecolor{currentstroke}{rgb}{0.000000,0.000000,0.000000}%
\pgfsetstrokecolor{currentstroke}%
\pgfsetdash{}{0pt}%
\pgfsys@defobject{currentmarker}{\pgfqpoint{-0.027778in}{0.000000in}}{\pgfqpoint{-0.000000in}{0.000000in}}{%
\pgfpathmoveto{\pgfqpoint{-0.000000in}{0.000000in}}%
\pgfpathlineto{\pgfqpoint{-0.027778in}{0.000000in}}%
\pgfusepath{stroke,fill}%
}%
\begin{pgfscope}%
\pgfsys@transformshift{0.588387in}{0.530778in}%
\pgfsys@useobject{currentmarker}{}%
\end{pgfscope}%
\end{pgfscope}%
\begin{pgfscope}%
\pgfsetbuttcap%
\pgfsetroundjoin%
\definecolor{currentfill}{rgb}{0.000000,0.000000,0.000000}%
\pgfsetfillcolor{currentfill}%
\pgfsetlinewidth{0.602250pt}%
\definecolor{currentstroke}{rgb}{0.000000,0.000000,0.000000}%
\pgfsetstrokecolor{currentstroke}%
\pgfsetdash{}{0pt}%
\pgfsys@defobject{currentmarker}{\pgfqpoint{-0.027778in}{0.000000in}}{\pgfqpoint{-0.000000in}{0.000000in}}{%
\pgfpathmoveto{\pgfqpoint{-0.000000in}{0.000000in}}%
\pgfpathlineto{\pgfqpoint{-0.027778in}{0.000000in}}%
\pgfusepath{stroke,fill}%
}%
\begin{pgfscope}%
\pgfsys@transformshift{0.588387in}{0.554910in}%
\pgfsys@useobject{currentmarker}{}%
\end{pgfscope}%
\end{pgfscope}%
\begin{pgfscope}%
\pgfsetbuttcap%
\pgfsetroundjoin%
\definecolor{currentfill}{rgb}{0.000000,0.000000,0.000000}%
\pgfsetfillcolor{currentfill}%
\pgfsetlinewidth{0.602250pt}%
\definecolor{currentstroke}{rgb}{0.000000,0.000000,0.000000}%
\pgfsetstrokecolor{currentstroke}%
\pgfsetdash{}{0pt}%
\pgfsys@defobject{currentmarker}{\pgfqpoint{-0.027778in}{0.000000in}}{\pgfqpoint{-0.000000in}{0.000000in}}{%
\pgfpathmoveto{\pgfqpoint{-0.000000in}{0.000000in}}%
\pgfpathlineto{\pgfqpoint{-0.027778in}{0.000000in}}%
\pgfusepath{stroke,fill}%
}%
\begin{pgfscope}%
\pgfsys@transformshift{0.588387in}{0.575314in}%
\pgfsys@useobject{currentmarker}{}%
\end{pgfscope}%
\end{pgfscope}%
\begin{pgfscope}%
\pgfsetbuttcap%
\pgfsetroundjoin%
\definecolor{currentfill}{rgb}{0.000000,0.000000,0.000000}%
\pgfsetfillcolor{currentfill}%
\pgfsetlinewidth{0.602250pt}%
\definecolor{currentstroke}{rgb}{0.000000,0.000000,0.000000}%
\pgfsetstrokecolor{currentstroke}%
\pgfsetdash{}{0pt}%
\pgfsys@defobject{currentmarker}{\pgfqpoint{-0.027778in}{0.000000in}}{\pgfqpoint{-0.000000in}{0.000000in}}{%
\pgfpathmoveto{\pgfqpoint{-0.000000in}{0.000000in}}%
\pgfpathlineto{\pgfqpoint{-0.027778in}{0.000000in}}%
\pgfusepath{stroke,fill}%
}%
\begin{pgfscope}%
\pgfsys@transformshift{0.588387in}{0.592988in}%
\pgfsys@useobject{currentmarker}{}%
\end{pgfscope}%
\end{pgfscope}%
\begin{pgfscope}%
\pgfsetbuttcap%
\pgfsetroundjoin%
\definecolor{currentfill}{rgb}{0.000000,0.000000,0.000000}%
\pgfsetfillcolor{currentfill}%
\pgfsetlinewidth{0.602250pt}%
\definecolor{currentstroke}{rgb}{0.000000,0.000000,0.000000}%
\pgfsetstrokecolor{currentstroke}%
\pgfsetdash{}{0pt}%
\pgfsys@defobject{currentmarker}{\pgfqpoint{-0.027778in}{0.000000in}}{\pgfqpoint{-0.000000in}{0.000000in}}{%
\pgfpathmoveto{\pgfqpoint{-0.000000in}{0.000000in}}%
\pgfpathlineto{\pgfqpoint{-0.027778in}{0.000000in}}%
\pgfusepath{stroke,fill}%
}%
\begin{pgfscope}%
\pgfsys@transformshift{0.588387in}{0.608578in}%
\pgfsys@useobject{currentmarker}{}%
\end{pgfscope}%
\end{pgfscope}%
\begin{pgfscope}%
\pgfsetbuttcap%
\pgfsetroundjoin%
\definecolor{currentfill}{rgb}{0.000000,0.000000,0.000000}%
\pgfsetfillcolor{currentfill}%
\pgfsetlinewidth{0.602250pt}%
\definecolor{currentstroke}{rgb}{0.000000,0.000000,0.000000}%
\pgfsetstrokecolor{currentstroke}%
\pgfsetdash{}{0pt}%
\pgfsys@defobject{currentmarker}{\pgfqpoint{-0.027778in}{0.000000in}}{\pgfqpoint{-0.000000in}{0.000000in}}{%
\pgfpathmoveto{\pgfqpoint{-0.000000in}{0.000000in}}%
\pgfpathlineto{\pgfqpoint{-0.027778in}{0.000000in}}%
\pgfusepath{stroke,fill}%
}%
\begin{pgfscope}%
\pgfsys@transformshift{0.588387in}{0.714269in}%
\pgfsys@useobject{currentmarker}{}%
\end{pgfscope}%
\end{pgfscope}%
\begin{pgfscope}%
\pgfsetbuttcap%
\pgfsetroundjoin%
\definecolor{currentfill}{rgb}{0.000000,0.000000,0.000000}%
\pgfsetfillcolor{currentfill}%
\pgfsetlinewidth{0.602250pt}%
\definecolor{currentstroke}{rgb}{0.000000,0.000000,0.000000}%
\pgfsetstrokecolor{currentstroke}%
\pgfsetdash{}{0pt}%
\pgfsys@defobject{currentmarker}{\pgfqpoint{-0.027778in}{0.000000in}}{\pgfqpoint{-0.000000in}{0.000000in}}{%
\pgfpathmoveto{\pgfqpoint{-0.000000in}{0.000000in}}%
\pgfpathlineto{\pgfqpoint{-0.027778in}{0.000000in}}%
\pgfusepath{stroke,fill}%
}%
\begin{pgfscope}%
\pgfsys@transformshift{0.588387in}{0.767937in}%
\pgfsys@useobject{currentmarker}{}%
\end{pgfscope}%
\end{pgfscope}%
\begin{pgfscope}%
\pgfsetbuttcap%
\pgfsetroundjoin%
\definecolor{currentfill}{rgb}{0.000000,0.000000,0.000000}%
\pgfsetfillcolor{currentfill}%
\pgfsetlinewidth{0.602250pt}%
\definecolor{currentstroke}{rgb}{0.000000,0.000000,0.000000}%
\pgfsetstrokecolor{currentstroke}%
\pgfsetdash{}{0pt}%
\pgfsys@defobject{currentmarker}{\pgfqpoint{-0.027778in}{0.000000in}}{\pgfqpoint{-0.000000in}{0.000000in}}{%
\pgfpathmoveto{\pgfqpoint{-0.000000in}{0.000000in}}%
\pgfpathlineto{\pgfqpoint{-0.027778in}{0.000000in}}%
\pgfusepath{stroke,fill}%
}%
\begin{pgfscope}%
\pgfsys@transformshift{0.588387in}{0.806015in}%
\pgfsys@useobject{currentmarker}{}%
\end{pgfscope}%
\end{pgfscope}%
\begin{pgfscope}%
\pgfsetbuttcap%
\pgfsetroundjoin%
\definecolor{currentfill}{rgb}{0.000000,0.000000,0.000000}%
\pgfsetfillcolor{currentfill}%
\pgfsetlinewidth{0.602250pt}%
\definecolor{currentstroke}{rgb}{0.000000,0.000000,0.000000}%
\pgfsetstrokecolor{currentstroke}%
\pgfsetdash{}{0pt}%
\pgfsys@defobject{currentmarker}{\pgfqpoint{-0.027778in}{0.000000in}}{\pgfqpoint{-0.000000in}{0.000000in}}{%
\pgfpathmoveto{\pgfqpoint{-0.000000in}{0.000000in}}%
\pgfpathlineto{\pgfqpoint{-0.027778in}{0.000000in}}%
\pgfusepath{stroke,fill}%
}%
\begin{pgfscope}%
\pgfsys@transformshift{0.588387in}{0.835551in}%
\pgfsys@useobject{currentmarker}{}%
\end{pgfscope}%
\end{pgfscope}%
\begin{pgfscope}%
\pgfsetbuttcap%
\pgfsetroundjoin%
\definecolor{currentfill}{rgb}{0.000000,0.000000,0.000000}%
\pgfsetfillcolor{currentfill}%
\pgfsetlinewidth{0.602250pt}%
\definecolor{currentstroke}{rgb}{0.000000,0.000000,0.000000}%
\pgfsetstrokecolor{currentstroke}%
\pgfsetdash{}{0pt}%
\pgfsys@defobject{currentmarker}{\pgfqpoint{-0.027778in}{0.000000in}}{\pgfqpoint{-0.000000in}{0.000000in}}{%
\pgfpathmoveto{\pgfqpoint{-0.000000in}{0.000000in}}%
\pgfpathlineto{\pgfqpoint{-0.027778in}{0.000000in}}%
\pgfusepath{stroke,fill}%
}%
\begin{pgfscope}%
\pgfsys@transformshift{0.588387in}{0.859683in}%
\pgfsys@useobject{currentmarker}{}%
\end{pgfscope}%
\end{pgfscope}%
\begin{pgfscope}%
\pgfsetbuttcap%
\pgfsetroundjoin%
\definecolor{currentfill}{rgb}{0.000000,0.000000,0.000000}%
\pgfsetfillcolor{currentfill}%
\pgfsetlinewidth{0.602250pt}%
\definecolor{currentstroke}{rgb}{0.000000,0.000000,0.000000}%
\pgfsetstrokecolor{currentstroke}%
\pgfsetdash{}{0pt}%
\pgfsys@defobject{currentmarker}{\pgfqpoint{-0.027778in}{0.000000in}}{\pgfqpoint{-0.000000in}{0.000000in}}{%
\pgfpathmoveto{\pgfqpoint{-0.000000in}{0.000000in}}%
\pgfpathlineto{\pgfqpoint{-0.027778in}{0.000000in}}%
\pgfusepath{stroke,fill}%
}%
\begin{pgfscope}%
\pgfsys@transformshift{0.588387in}{0.880086in}%
\pgfsys@useobject{currentmarker}{}%
\end{pgfscope}%
\end{pgfscope}%
\begin{pgfscope}%
\pgfsetbuttcap%
\pgfsetroundjoin%
\definecolor{currentfill}{rgb}{0.000000,0.000000,0.000000}%
\pgfsetfillcolor{currentfill}%
\pgfsetlinewidth{0.602250pt}%
\definecolor{currentstroke}{rgb}{0.000000,0.000000,0.000000}%
\pgfsetstrokecolor{currentstroke}%
\pgfsetdash{}{0pt}%
\pgfsys@defobject{currentmarker}{\pgfqpoint{-0.027778in}{0.000000in}}{\pgfqpoint{-0.000000in}{0.000000in}}{%
\pgfpathmoveto{\pgfqpoint{-0.000000in}{0.000000in}}%
\pgfpathlineto{\pgfqpoint{-0.027778in}{0.000000in}}%
\pgfusepath{stroke,fill}%
}%
\begin{pgfscope}%
\pgfsys@transformshift{0.588387in}{0.897761in}%
\pgfsys@useobject{currentmarker}{}%
\end{pgfscope}%
\end{pgfscope}%
\begin{pgfscope}%
\pgfsetbuttcap%
\pgfsetroundjoin%
\definecolor{currentfill}{rgb}{0.000000,0.000000,0.000000}%
\pgfsetfillcolor{currentfill}%
\pgfsetlinewidth{0.602250pt}%
\definecolor{currentstroke}{rgb}{0.000000,0.000000,0.000000}%
\pgfsetstrokecolor{currentstroke}%
\pgfsetdash{}{0pt}%
\pgfsys@defobject{currentmarker}{\pgfqpoint{-0.027778in}{0.000000in}}{\pgfqpoint{-0.000000in}{0.000000in}}{%
\pgfpathmoveto{\pgfqpoint{-0.000000in}{0.000000in}}%
\pgfpathlineto{\pgfqpoint{-0.027778in}{0.000000in}}%
\pgfusepath{stroke,fill}%
}%
\begin{pgfscope}%
\pgfsys@transformshift{0.588387in}{0.913351in}%
\pgfsys@useobject{currentmarker}{}%
\end{pgfscope}%
\end{pgfscope}%
\begin{pgfscope}%
\pgfsetbuttcap%
\pgfsetroundjoin%
\definecolor{currentfill}{rgb}{0.000000,0.000000,0.000000}%
\pgfsetfillcolor{currentfill}%
\pgfsetlinewidth{0.602250pt}%
\definecolor{currentstroke}{rgb}{0.000000,0.000000,0.000000}%
\pgfsetstrokecolor{currentstroke}%
\pgfsetdash{}{0pt}%
\pgfsys@defobject{currentmarker}{\pgfqpoint{-0.027778in}{0.000000in}}{\pgfqpoint{-0.000000in}{0.000000in}}{%
\pgfpathmoveto{\pgfqpoint{-0.000000in}{0.000000in}}%
\pgfpathlineto{\pgfqpoint{-0.027778in}{0.000000in}}%
\pgfusepath{stroke,fill}%
}%
\begin{pgfscope}%
\pgfsys@transformshift{0.588387in}{1.019042in}%
\pgfsys@useobject{currentmarker}{}%
\end{pgfscope}%
\end{pgfscope}%
\begin{pgfscope}%
\pgfsetbuttcap%
\pgfsetroundjoin%
\definecolor{currentfill}{rgb}{0.000000,0.000000,0.000000}%
\pgfsetfillcolor{currentfill}%
\pgfsetlinewidth{0.602250pt}%
\definecolor{currentstroke}{rgb}{0.000000,0.000000,0.000000}%
\pgfsetstrokecolor{currentstroke}%
\pgfsetdash{}{0pt}%
\pgfsys@defobject{currentmarker}{\pgfqpoint{-0.027778in}{0.000000in}}{\pgfqpoint{-0.000000in}{0.000000in}}{%
\pgfpathmoveto{\pgfqpoint{-0.000000in}{0.000000in}}%
\pgfpathlineto{\pgfqpoint{-0.027778in}{0.000000in}}%
\pgfusepath{stroke,fill}%
}%
\begin{pgfscope}%
\pgfsys@transformshift{0.588387in}{1.072710in}%
\pgfsys@useobject{currentmarker}{}%
\end{pgfscope}%
\end{pgfscope}%
\begin{pgfscope}%
\pgfsetbuttcap%
\pgfsetroundjoin%
\definecolor{currentfill}{rgb}{0.000000,0.000000,0.000000}%
\pgfsetfillcolor{currentfill}%
\pgfsetlinewidth{0.602250pt}%
\definecolor{currentstroke}{rgb}{0.000000,0.000000,0.000000}%
\pgfsetstrokecolor{currentstroke}%
\pgfsetdash{}{0pt}%
\pgfsys@defobject{currentmarker}{\pgfqpoint{-0.027778in}{0.000000in}}{\pgfqpoint{-0.000000in}{0.000000in}}{%
\pgfpathmoveto{\pgfqpoint{-0.000000in}{0.000000in}}%
\pgfpathlineto{\pgfqpoint{-0.027778in}{0.000000in}}%
\pgfusepath{stroke,fill}%
}%
\begin{pgfscope}%
\pgfsys@transformshift{0.588387in}{1.110788in}%
\pgfsys@useobject{currentmarker}{}%
\end{pgfscope}%
\end{pgfscope}%
\begin{pgfscope}%
\pgfsetbuttcap%
\pgfsetroundjoin%
\definecolor{currentfill}{rgb}{0.000000,0.000000,0.000000}%
\pgfsetfillcolor{currentfill}%
\pgfsetlinewidth{0.602250pt}%
\definecolor{currentstroke}{rgb}{0.000000,0.000000,0.000000}%
\pgfsetstrokecolor{currentstroke}%
\pgfsetdash{}{0pt}%
\pgfsys@defobject{currentmarker}{\pgfqpoint{-0.027778in}{0.000000in}}{\pgfqpoint{-0.000000in}{0.000000in}}{%
\pgfpathmoveto{\pgfqpoint{-0.000000in}{0.000000in}}%
\pgfpathlineto{\pgfqpoint{-0.027778in}{0.000000in}}%
\pgfusepath{stroke,fill}%
}%
\begin{pgfscope}%
\pgfsys@transformshift{0.588387in}{1.140323in}%
\pgfsys@useobject{currentmarker}{}%
\end{pgfscope}%
\end{pgfscope}%
\begin{pgfscope}%
\pgfsetbuttcap%
\pgfsetroundjoin%
\definecolor{currentfill}{rgb}{0.000000,0.000000,0.000000}%
\pgfsetfillcolor{currentfill}%
\pgfsetlinewidth{0.602250pt}%
\definecolor{currentstroke}{rgb}{0.000000,0.000000,0.000000}%
\pgfsetstrokecolor{currentstroke}%
\pgfsetdash{}{0pt}%
\pgfsys@defobject{currentmarker}{\pgfqpoint{-0.027778in}{0.000000in}}{\pgfqpoint{-0.000000in}{0.000000in}}{%
\pgfpathmoveto{\pgfqpoint{-0.000000in}{0.000000in}}%
\pgfpathlineto{\pgfqpoint{-0.027778in}{0.000000in}}%
\pgfusepath{stroke,fill}%
}%
\begin{pgfscope}%
\pgfsys@transformshift{0.588387in}{1.164455in}%
\pgfsys@useobject{currentmarker}{}%
\end{pgfscope}%
\end{pgfscope}%
\begin{pgfscope}%
\pgfsetbuttcap%
\pgfsetroundjoin%
\definecolor{currentfill}{rgb}{0.000000,0.000000,0.000000}%
\pgfsetfillcolor{currentfill}%
\pgfsetlinewidth{0.602250pt}%
\definecolor{currentstroke}{rgb}{0.000000,0.000000,0.000000}%
\pgfsetstrokecolor{currentstroke}%
\pgfsetdash{}{0pt}%
\pgfsys@defobject{currentmarker}{\pgfqpoint{-0.027778in}{0.000000in}}{\pgfqpoint{-0.000000in}{0.000000in}}{%
\pgfpathmoveto{\pgfqpoint{-0.000000in}{0.000000in}}%
\pgfpathlineto{\pgfqpoint{-0.027778in}{0.000000in}}%
\pgfusepath{stroke,fill}%
}%
\begin{pgfscope}%
\pgfsys@transformshift{0.588387in}{1.184859in}%
\pgfsys@useobject{currentmarker}{}%
\end{pgfscope}%
\end{pgfscope}%
\begin{pgfscope}%
\pgfsetbuttcap%
\pgfsetroundjoin%
\definecolor{currentfill}{rgb}{0.000000,0.000000,0.000000}%
\pgfsetfillcolor{currentfill}%
\pgfsetlinewidth{0.602250pt}%
\definecolor{currentstroke}{rgb}{0.000000,0.000000,0.000000}%
\pgfsetstrokecolor{currentstroke}%
\pgfsetdash{}{0pt}%
\pgfsys@defobject{currentmarker}{\pgfqpoint{-0.027778in}{0.000000in}}{\pgfqpoint{-0.000000in}{0.000000in}}{%
\pgfpathmoveto{\pgfqpoint{-0.000000in}{0.000000in}}%
\pgfpathlineto{\pgfqpoint{-0.027778in}{0.000000in}}%
\pgfusepath{stroke,fill}%
}%
\begin{pgfscope}%
\pgfsys@transformshift{0.588387in}{1.202533in}%
\pgfsys@useobject{currentmarker}{}%
\end{pgfscope}%
\end{pgfscope}%
\begin{pgfscope}%
\pgfsetbuttcap%
\pgfsetroundjoin%
\definecolor{currentfill}{rgb}{0.000000,0.000000,0.000000}%
\pgfsetfillcolor{currentfill}%
\pgfsetlinewidth{0.602250pt}%
\definecolor{currentstroke}{rgb}{0.000000,0.000000,0.000000}%
\pgfsetstrokecolor{currentstroke}%
\pgfsetdash{}{0pt}%
\pgfsys@defobject{currentmarker}{\pgfqpoint{-0.027778in}{0.000000in}}{\pgfqpoint{-0.000000in}{0.000000in}}{%
\pgfpathmoveto{\pgfqpoint{-0.000000in}{0.000000in}}%
\pgfpathlineto{\pgfqpoint{-0.027778in}{0.000000in}}%
\pgfusepath{stroke,fill}%
}%
\begin{pgfscope}%
\pgfsys@transformshift{0.588387in}{1.218123in}%
\pgfsys@useobject{currentmarker}{}%
\end{pgfscope}%
\end{pgfscope}%
\begin{pgfscope}%
\pgfsetbuttcap%
\pgfsetroundjoin%
\definecolor{currentfill}{rgb}{0.000000,0.000000,0.000000}%
\pgfsetfillcolor{currentfill}%
\pgfsetlinewidth{0.602250pt}%
\definecolor{currentstroke}{rgb}{0.000000,0.000000,0.000000}%
\pgfsetstrokecolor{currentstroke}%
\pgfsetdash{}{0pt}%
\pgfsys@defobject{currentmarker}{\pgfqpoint{-0.027778in}{0.000000in}}{\pgfqpoint{-0.000000in}{0.000000in}}{%
\pgfpathmoveto{\pgfqpoint{-0.000000in}{0.000000in}}%
\pgfpathlineto{\pgfqpoint{-0.027778in}{0.000000in}}%
\pgfusepath{stroke,fill}%
}%
\begin{pgfscope}%
\pgfsys@transformshift{0.588387in}{1.323815in}%
\pgfsys@useobject{currentmarker}{}%
\end{pgfscope}%
\end{pgfscope}%
\begin{pgfscope}%
\pgfsetbuttcap%
\pgfsetroundjoin%
\definecolor{currentfill}{rgb}{0.000000,0.000000,0.000000}%
\pgfsetfillcolor{currentfill}%
\pgfsetlinewidth{0.602250pt}%
\definecolor{currentstroke}{rgb}{0.000000,0.000000,0.000000}%
\pgfsetstrokecolor{currentstroke}%
\pgfsetdash{}{0pt}%
\pgfsys@defobject{currentmarker}{\pgfqpoint{-0.027778in}{0.000000in}}{\pgfqpoint{-0.000000in}{0.000000in}}{%
\pgfpathmoveto{\pgfqpoint{-0.000000in}{0.000000in}}%
\pgfpathlineto{\pgfqpoint{-0.027778in}{0.000000in}}%
\pgfusepath{stroke,fill}%
}%
\begin{pgfscope}%
\pgfsys@transformshift{0.588387in}{1.377482in}%
\pgfsys@useobject{currentmarker}{}%
\end{pgfscope}%
\end{pgfscope}%
\begin{pgfscope}%
\pgfsetbuttcap%
\pgfsetroundjoin%
\definecolor{currentfill}{rgb}{0.000000,0.000000,0.000000}%
\pgfsetfillcolor{currentfill}%
\pgfsetlinewidth{0.602250pt}%
\definecolor{currentstroke}{rgb}{0.000000,0.000000,0.000000}%
\pgfsetstrokecolor{currentstroke}%
\pgfsetdash{}{0pt}%
\pgfsys@defobject{currentmarker}{\pgfqpoint{-0.027778in}{0.000000in}}{\pgfqpoint{-0.000000in}{0.000000in}}{%
\pgfpathmoveto{\pgfqpoint{-0.000000in}{0.000000in}}%
\pgfpathlineto{\pgfqpoint{-0.027778in}{0.000000in}}%
\pgfusepath{stroke,fill}%
}%
\begin{pgfscope}%
\pgfsys@transformshift{0.588387in}{1.415560in}%
\pgfsys@useobject{currentmarker}{}%
\end{pgfscope}%
\end{pgfscope}%
\begin{pgfscope}%
\pgfsetbuttcap%
\pgfsetroundjoin%
\definecolor{currentfill}{rgb}{0.000000,0.000000,0.000000}%
\pgfsetfillcolor{currentfill}%
\pgfsetlinewidth{0.602250pt}%
\definecolor{currentstroke}{rgb}{0.000000,0.000000,0.000000}%
\pgfsetstrokecolor{currentstroke}%
\pgfsetdash{}{0pt}%
\pgfsys@defobject{currentmarker}{\pgfqpoint{-0.027778in}{0.000000in}}{\pgfqpoint{-0.000000in}{0.000000in}}{%
\pgfpathmoveto{\pgfqpoint{-0.000000in}{0.000000in}}%
\pgfpathlineto{\pgfqpoint{-0.027778in}{0.000000in}}%
\pgfusepath{stroke,fill}%
}%
\begin{pgfscope}%
\pgfsys@transformshift{0.588387in}{1.445096in}%
\pgfsys@useobject{currentmarker}{}%
\end{pgfscope}%
\end{pgfscope}%
\begin{pgfscope}%
\pgfsetbuttcap%
\pgfsetroundjoin%
\definecolor{currentfill}{rgb}{0.000000,0.000000,0.000000}%
\pgfsetfillcolor{currentfill}%
\pgfsetlinewidth{0.602250pt}%
\definecolor{currentstroke}{rgb}{0.000000,0.000000,0.000000}%
\pgfsetstrokecolor{currentstroke}%
\pgfsetdash{}{0pt}%
\pgfsys@defobject{currentmarker}{\pgfqpoint{-0.027778in}{0.000000in}}{\pgfqpoint{-0.000000in}{0.000000in}}{%
\pgfpathmoveto{\pgfqpoint{-0.000000in}{0.000000in}}%
\pgfpathlineto{\pgfqpoint{-0.027778in}{0.000000in}}%
\pgfusepath{stroke,fill}%
}%
\begin{pgfscope}%
\pgfsys@transformshift{0.588387in}{1.469228in}%
\pgfsys@useobject{currentmarker}{}%
\end{pgfscope}%
\end{pgfscope}%
\begin{pgfscope}%
\pgfsetbuttcap%
\pgfsetroundjoin%
\definecolor{currentfill}{rgb}{0.000000,0.000000,0.000000}%
\pgfsetfillcolor{currentfill}%
\pgfsetlinewidth{0.602250pt}%
\definecolor{currentstroke}{rgb}{0.000000,0.000000,0.000000}%
\pgfsetstrokecolor{currentstroke}%
\pgfsetdash{}{0pt}%
\pgfsys@defobject{currentmarker}{\pgfqpoint{-0.027778in}{0.000000in}}{\pgfqpoint{-0.000000in}{0.000000in}}{%
\pgfpathmoveto{\pgfqpoint{-0.000000in}{0.000000in}}%
\pgfpathlineto{\pgfqpoint{-0.027778in}{0.000000in}}%
\pgfusepath{stroke,fill}%
}%
\begin{pgfscope}%
\pgfsys@transformshift{0.588387in}{1.489632in}%
\pgfsys@useobject{currentmarker}{}%
\end{pgfscope}%
\end{pgfscope}%
\begin{pgfscope}%
\pgfsetbuttcap%
\pgfsetroundjoin%
\definecolor{currentfill}{rgb}{0.000000,0.000000,0.000000}%
\pgfsetfillcolor{currentfill}%
\pgfsetlinewidth{0.602250pt}%
\definecolor{currentstroke}{rgb}{0.000000,0.000000,0.000000}%
\pgfsetstrokecolor{currentstroke}%
\pgfsetdash{}{0pt}%
\pgfsys@defobject{currentmarker}{\pgfqpoint{-0.027778in}{0.000000in}}{\pgfqpoint{-0.000000in}{0.000000in}}{%
\pgfpathmoveto{\pgfqpoint{-0.000000in}{0.000000in}}%
\pgfpathlineto{\pgfqpoint{-0.027778in}{0.000000in}}%
\pgfusepath{stroke,fill}%
}%
\begin{pgfscope}%
\pgfsys@transformshift{0.588387in}{1.507306in}%
\pgfsys@useobject{currentmarker}{}%
\end{pgfscope}%
\end{pgfscope}%
\begin{pgfscope}%
\pgfsetbuttcap%
\pgfsetroundjoin%
\definecolor{currentfill}{rgb}{0.000000,0.000000,0.000000}%
\pgfsetfillcolor{currentfill}%
\pgfsetlinewidth{0.602250pt}%
\definecolor{currentstroke}{rgb}{0.000000,0.000000,0.000000}%
\pgfsetstrokecolor{currentstroke}%
\pgfsetdash{}{0pt}%
\pgfsys@defobject{currentmarker}{\pgfqpoint{-0.027778in}{0.000000in}}{\pgfqpoint{-0.000000in}{0.000000in}}{%
\pgfpathmoveto{\pgfqpoint{-0.000000in}{0.000000in}}%
\pgfpathlineto{\pgfqpoint{-0.027778in}{0.000000in}}%
\pgfusepath{stroke,fill}%
}%
\begin{pgfscope}%
\pgfsys@transformshift{0.588387in}{1.522896in}%
\pgfsys@useobject{currentmarker}{}%
\end{pgfscope}%
\end{pgfscope}%
\begin{pgfscope}%
\pgfsetbuttcap%
\pgfsetroundjoin%
\definecolor{currentfill}{rgb}{0.000000,0.000000,0.000000}%
\pgfsetfillcolor{currentfill}%
\pgfsetlinewidth{0.602250pt}%
\definecolor{currentstroke}{rgb}{0.000000,0.000000,0.000000}%
\pgfsetstrokecolor{currentstroke}%
\pgfsetdash{}{0pt}%
\pgfsys@defobject{currentmarker}{\pgfqpoint{-0.027778in}{0.000000in}}{\pgfqpoint{-0.000000in}{0.000000in}}{%
\pgfpathmoveto{\pgfqpoint{-0.000000in}{0.000000in}}%
\pgfpathlineto{\pgfqpoint{-0.027778in}{0.000000in}}%
\pgfusepath{stroke,fill}%
}%
\begin{pgfscope}%
\pgfsys@transformshift{0.588387in}{1.628587in}%
\pgfsys@useobject{currentmarker}{}%
\end{pgfscope}%
\end{pgfscope}%
\begin{pgfscope}%
\pgfsetbuttcap%
\pgfsetroundjoin%
\definecolor{currentfill}{rgb}{0.000000,0.000000,0.000000}%
\pgfsetfillcolor{currentfill}%
\pgfsetlinewidth{0.602250pt}%
\definecolor{currentstroke}{rgb}{0.000000,0.000000,0.000000}%
\pgfsetstrokecolor{currentstroke}%
\pgfsetdash{}{0pt}%
\pgfsys@defobject{currentmarker}{\pgfqpoint{-0.027778in}{0.000000in}}{\pgfqpoint{-0.000000in}{0.000000in}}{%
\pgfpathmoveto{\pgfqpoint{-0.000000in}{0.000000in}}%
\pgfpathlineto{\pgfqpoint{-0.027778in}{0.000000in}}%
\pgfusepath{stroke,fill}%
}%
\begin{pgfscope}%
\pgfsys@transformshift{0.588387in}{1.682255in}%
\pgfsys@useobject{currentmarker}{}%
\end{pgfscope}%
\end{pgfscope}%
\begin{pgfscope}%
\pgfsetbuttcap%
\pgfsetroundjoin%
\definecolor{currentfill}{rgb}{0.000000,0.000000,0.000000}%
\pgfsetfillcolor{currentfill}%
\pgfsetlinewidth{0.602250pt}%
\definecolor{currentstroke}{rgb}{0.000000,0.000000,0.000000}%
\pgfsetstrokecolor{currentstroke}%
\pgfsetdash{}{0pt}%
\pgfsys@defobject{currentmarker}{\pgfqpoint{-0.027778in}{0.000000in}}{\pgfqpoint{-0.000000in}{0.000000in}}{%
\pgfpathmoveto{\pgfqpoint{-0.000000in}{0.000000in}}%
\pgfpathlineto{\pgfqpoint{-0.027778in}{0.000000in}}%
\pgfusepath{stroke,fill}%
}%
\begin{pgfscope}%
\pgfsys@transformshift{0.588387in}{1.720333in}%
\pgfsys@useobject{currentmarker}{}%
\end{pgfscope}%
\end{pgfscope}%
\begin{pgfscope}%
\pgfsetbuttcap%
\pgfsetroundjoin%
\definecolor{currentfill}{rgb}{0.000000,0.000000,0.000000}%
\pgfsetfillcolor{currentfill}%
\pgfsetlinewidth{0.602250pt}%
\definecolor{currentstroke}{rgb}{0.000000,0.000000,0.000000}%
\pgfsetstrokecolor{currentstroke}%
\pgfsetdash{}{0pt}%
\pgfsys@defobject{currentmarker}{\pgfqpoint{-0.027778in}{0.000000in}}{\pgfqpoint{-0.000000in}{0.000000in}}{%
\pgfpathmoveto{\pgfqpoint{-0.000000in}{0.000000in}}%
\pgfpathlineto{\pgfqpoint{-0.027778in}{0.000000in}}%
\pgfusepath{stroke,fill}%
}%
\begin{pgfscope}%
\pgfsys@transformshift{0.588387in}{1.749868in}%
\pgfsys@useobject{currentmarker}{}%
\end{pgfscope}%
\end{pgfscope}%
\begin{pgfscope}%
\pgfsetbuttcap%
\pgfsetroundjoin%
\definecolor{currentfill}{rgb}{0.000000,0.000000,0.000000}%
\pgfsetfillcolor{currentfill}%
\pgfsetlinewidth{0.602250pt}%
\definecolor{currentstroke}{rgb}{0.000000,0.000000,0.000000}%
\pgfsetstrokecolor{currentstroke}%
\pgfsetdash{}{0pt}%
\pgfsys@defobject{currentmarker}{\pgfqpoint{-0.027778in}{0.000000in}}{\pgfqpoint{-0.000000in}{0.000000in}}{%
\pgfpathmoveto{\pgfqpoint{-0.000000in}{0.000000in}}%
\pgfpathlineto{\pgfqpoint{-0.027778in}{0.000000in}}%
\pgfusepath{stroke,fill}%
}%
\begin{pgfscope}%
\pgfsys@transformshift{0.588387in}{1.774001in}%
\pgfsys@useobject{currentmarker}{}%
\end{pgfscope}%
\end{pgfscope}%
\begin{pgfscope}%
\pgfsetbuttcap%
\pgfsetroundjoin%
\definecolor{currentfill}{rgb}{0.000000,0.000000,0.000000}%
\pgfsetfillcolor{currentfill}%
\pgfsetlinewidth{0.602250pt}%
\definecolor{currentstroke}{rgb}{0.000000,0.000000,0.000000}%
\pgfsetstrokecolor{currentstroke}%
\pgfsetdash{}{0pt}%
\pgfsys@defobject{currentmarker}{\pgfqpoint{-0.027778in}{0.000000in}}{\pgfqpoint{-0.000000in}{0.000000in}}{%
\pgfpathmoveto{\pgfqpoint{-0.000000in}{0.000000in}}%
\pgfpathlineto{\pgfqpoint{-0.027778in}{0.000000in}}%
\pgfusepath{stroke,fill}%
}%
\begin{pgfscope}%
\pgfsys@transformshift{0.588387in}{1.794404in}%
\pgfsys@useobject{currentmarker}{}%
\end{pgfscope}%
\end{pgfscope}%
\begin{pgfscope}%
\pgfsetbuttcap%
\pgfsetroundjoin%
\definecolor{currentfill}{rgb}{0.000000,0.000000,0.000000}%
\pgfsetfillcolor{currentfill}%
\pgfsetlinewidth{0.602250pt}%
\definecolor{currentstroke}{rgb}{0.000000,0.000000,0.000000}%
\pgfsetstrokecolor{currentstroke}%
\pgfsetdash{}{0pt}%
\pgfsys@defobject{currentmarker}{\pgfqpoint{-0.027778in}{0.000000in}}{\pgfqpoint{-0.000000in}{0.000000in}}{%
\pgfpathmoveto{\pgfqpoint{-0.000000in}{0.000000in}}%
\pgfpathlineto{\pgfqpoint{-0.027778in}{0.000000in}}%
\pgfusepath{stroke,fill}%
}%
\begin{pgfscope}%
\pgfsys@transformshift{0.588387in}{1.812079in}%
\pgfsys@useobject{currentmarker}{}%
\end{pgfscope}%
\end{pgfscope}%
\begin{pgfscope}%
\pgfsetbuttcap%
\pgfsetroundjoin%
\definecolor{currentfill}{rgb}{0.000000,0.000000,0.000000}%
\pgfsetfillcolor{currentfill}%
\pgfsetlinewidth{0.602250pt}%
\definecolor{currentstroke}{rgb}{0.000000,0.000000,0.000000}%
\pgfsetstrokecolor{currentstroke}%
\pgfsetdash{}{0pt}%
\pgfsys@defobject{currentmarker}{\pgfqpoint{-0.027778in}{0.000000in}}{\pgfqpoint{-0.000000in}{0.000000in}}{%
\pgfpathmoveto{\pgfqpoint{-0.000000in}{0.000000in}}%
\pgfpathlineto{\pgfqpoint{-0.027778in}{0.000000in}}%
\pgfusepath{stroke,fill}%
}%
\begin{pgfscope}%
\pgfsys@transformshift{0.588387in}{1.827668in}%
\pgfsys@useobject{currentmarker}{}%
\end{pgfscope}%
\end{pgfscope}%
\begin{pgfscope}%
\pgfsetbuttcap%
\pgfsetroundjoin%
\definecolor{currentfill}{rgb}{0.000000,0.000000,0.000000}%
\pgfsetfillcolor{currentfill}%
\pgfsetlinewidth{0.602250pt}%
\definecolor{currentstroke}{rgb}{0.000000,0.000000,0.000000}%
\pgfsetstrokecolor{currentstroke}%
\pgfsetdash{}{0pt}%
\pgfsys@defobject{currentmarker}{\pgfqpoint{-0.027778in}{0.000000in}}{\pgfqpoint{-0.000000in}{0.000000in}}{%
\pgfpathmoveto{\pgfqpoint{-0.000000in}{0.000000in}}%
\pgfpathlineto{\pgfqpoint{-0.027778in}{0.000000in}}%
\pgfusepath{stroke,fill}%
}%
\begin{pgfscope}%
\pgfsys@transformshift{0.588387in}{1.933360in}%
\pgfsys@useobject{currentmarker}{}%
\end{pgfscope}%
\end{pgfscope}%
\begin{pgfscope}%
\pgfsetbuttcap%
\pgfsetroundjoin%
\definecolor{currentfill}{rgb}{0.000000,0.000000,0.000000}%
\pgfsetfillcolor{currentfill}%
\pgfsetlinewidth{0.602250pt}%
\definecolor{currentstroke}{rgb}{0.000000,0.000000,0.000000}%
\pgfsetstrokecolor{currentstroke}%
\pgfsetdash{}{0pt}%
\pgfsys@defobject{currentmarker}{\pgfqpoint{-0.027778in}{0.000000in}}{\pgfqpoint{-0.000000in}{0.000000in}}{%
\pgfpathmoveto{\pgfqpoint{-0.000000in}{0.000000in}}%
\pgfpathlineto{\pgfqpoint{-0.027778in}{0.000000in}}%
\pgfusepath{stroke,fill}%
}%
\begin{pgfscope}%
\pgfsys@transformshift{0.588387in}{1.987028in}%
\pgfsys@useobject{currentmarker}{}%
\end{pgfscope}%
\end{pgfscope}%
\begin{pgfscope}%
\pgfsetbuttcap%
\pgfsetroundjoin%
\definecolor{currentfill}{rgb}{0.000000,0.000000,0.000000}%
\pgfsetfillcolor{currentfill}%
\pgfsetlinewidth{0.602250pt}%
\definecolor{currentstroke}{rgb}{0.000000,0.000000,0.000000}%
\pgfsetstrokecolor{currentstroke}%
\pgfsetdash{}{0pt}%
\pgfsys@defobject{currentmarker}{\pgfqpoint{-0.027778in}{0.000000in}}{\pgfqpoint{-0.000000in}{0.000000in}}{%
\pgfpathmoveto{\pgfqpoint{-0.000000in}{0.000000in}}%
\pgfpathlineto{\pgfqpoint{-0.027778in}{0.000000in}}%
\pgfusepath{stroke,fill}%
}%
\begin{pgfscope}%
\pgfsys@transformshift{0.588387in}{2.025105in}%
\pgfsys@useobject{currentmarker}{}%
\end{pgfscope}%
\end{pgfscope}%
\begin{pgfscope}%
\pgfsetbuttcap%
\pgfsetroundjoin%
\definecolor{currentfill}{rgb}{0.000000,0.000000,0.000000}%
\pgfsetfillcolor{currentfill}%
\pgfsetlinewidth{0.602250pt}%
\definecolor{currentstroke}{rgb}{0.000000,0.000000,0.000000}%
\pgfsetstrokecolor{currentstroke}%
\pgfsetdash{}{0pt}%
\pgfsys@defobject{currentmarker}{\pgfqpoint{-0.027778in}{0.000000in}}{\pgfqpoint{-0.000000in}{0.000000in}}{%
\pgfpathmoveto{\pgfqpoint{-0.000000in}{0.000000in}}%
\pgfpathlineto{\pgfqpoint{-0.027778in}{0.000000in}}%
\pgfusepath{stroke,fill}%
}%
\begin{pgfscope}%
\pgfsys@transformshift{0.588387in}{2.054641in}%
\pgfsys@useobject{currentmarker}{}%
\end{pgfscope}%
\end{pgfscope}%
\begin{pgfscope}%
\pgfsetbuttcap%
\pgfsetroundjoin%
\definecolor{currentfill}{rgb}{0.000000,0.000000,0.000000}%
\pgfsetfillcolor{currentfill}%
\pgfsetlinewidth{0.602250pt}%
\definecolor{currentstroke}{rgb}{0.000000,0.000000,0.000000}%
\pgfsetstrokecolor{currentstroke}%
\pgfsetdash{}{0pt}%
\pgfsys@defobject{currentmarker}{\pgfqpoint{-0.027778in}{0.000000in}}{\pgfqpoint{-0.000000in}{0.000000in}}{%
\pgfpathmoveto{\pgfqpoint{-0.000000in}{0.000000in}}%
\pgfpathlineto{\pgfqpoint{-0.027778in}{0.000000in}}%
\pgfusepath{stroke,fill}%
}%
\begin{pgfscope}%
\pgfsys@transformshift{0.588387in}{2.078773in}%
\pgfsys@useobject{currentmarker}{}%
\end{pgfscope}%
\end{pgfscope}%
\begin{pgfscope}%
\pgfsetbuttcap%
\pgfsetroundjoin%
\definecolor{currentfill}{rgb}{0.000000,0.000000,0.000000}%
\pgfsetfillcolor{currentfill}%
\pgfsetlinewidth{0.602250pt}%
\definecolor{currentstroke}{rgb}{0.000000,0.000000,0.000000}%
\pgfsetstrokecolor{currentstroke}%
\pgfsetdash{}{0pt}%
\pgfsys@defobject{currentmarker}{\pgfqpoint{-0.027778in}{0.000000in}}{\pgfqpoint{-0.000000in}{0.000000in}}{%
\pgfpathmoveto{\pgfqpoint{-0.000000in}{0.000000in}}%
\pgfpathlineto{\pgfqpoint{-0.027778in}{0.000000in}}%
\pgfusepath{stroke,fill}%
}%
\begin{pgfscope}%
\pgfsys@transformshift{0.588387in}{2.099177in}%
\pgfsys@useobject{currentmarker}{}%
\end{pgfscope}%
\end{pgfscope}%
\begin{pgfscope}%
\pgfsetbuttcap%
\pgfsetroundjoin%
\definecolor{currentfill}{rgb}{0.000000,0.000000,0.000000}%
\pgfsetfillcolor{currentfill}%
\pgfsetlinewidth{0.602250pt}%
\definecolor{currentstroke}{rgb}{0.000000,0.000000,0.000000}%
\pgfsetstrokecolor{currentstroke}%
\pgfsetdash{}{0pt}%
\pgfsys@defobject{currentmarker}{\pgfqpoint{-0.027778in}{0.000000in}}{\pgfqpoint{-0.000000in}{0.000000in}}{%
\pgfpathmoveto{\pgfqpoint{-0.000000in}{0.000000in}}%
\pgfpathlineto{\pgfqpoint{-0.027778in}{0.000000in}}%
\pgfusepath{stroke,fill}%
}%
\begin{pgfscope}%
\pgfsys@transformshift{0.588387in}{2.116851in}%
\pgfsys@useobject{currentmarker}{}%
\end{pgfscope}%
\end{pgfscope}%
\begin{pgfscope}%
\pgfsetbuttcap%
\pgfsetroundjoin%
\definecolor{currentfill}{rgb}{0.000000,0.000000,0.000000}%
\pgfsetfillcolor{currentfill}%
\pgfsetlinewidth{0.602250pt}%
\definecolor{currentstroke}{rgb}{0.000000,0.000000,0.000000}%
\pgfsetstrokecolor{currentstroke}%
\pgfsetdash{}{0pt}%
\pgfsys@defobject{currentmarker}{\pgfqpoint{-0.027778in}{0.000000in}}{\pgfqpoint{-0.000000in}{0.000000in}}{%
\pgfpathmoveto{\pgfqpoint{-0.000000in}{0.000000in}}%
\pgfpathlineto{\pgfqpoint{-0.027778in}{0.000000in}}%
\pgfusepath{stroke,fill}%
}%
\begin{pgfscope}%
\pgfsys@transformshift{0.588387in}{2.132441in}%
\pgfsys@useobject{currentmarker}{}%
\end{pgfscope}%
\end{pgfscope}%
\begin{pgfscope}%
\pgfsetbuttcap%
\pgfsetroundjoin%
\definecolor{currentfill}{rgb}{0.000000,0.000000,0.000000}%
\pgfsetfillcolor{currentfill}%
\pgfsetlinewidth{0.602250pt}%
\definecolor{currentstroke}{rgb}{0.000000,0.000000,0.000000}%
\pgfsetstrokecolor{currentstroke}%
\pgfsetdash{}{0pt}%
\pgfsys@defobject{currentmarker}{\pgfqpoint{-0.027778in}{0.000000in}}{\pgfqpoint{-0.000000in}{0.000000in}}{%
\pgfpathmoveto{\pgfqpoint{-0.000000in}{0.000000in}}%
\pgfpathlineto{\pgfqpoint{-0.027778in}{0.000000in}}%
\pgfusepath{stroke,fill}%
}%
\begin{pgfscope}%
\pgfsys@transformshift{0.588387in}{2.238132in}%
\pgfsys@useobject{currentmarker}{}%
\end{pgfscope}%
\end{pgfscope}%
\begin{pgfscope}%
\pgfsetbuttcap%
\pgfsetroundjoin%
\definecolor{currentfill}{rgb}{0.000000,0.000000,0.000000}%
\pgfsetfillcolor{currentfill}%
\pgfsetlinewidth{0.602250pt}%
\definecolor{currentstroke}{rgb}{0.000000,0.000000,0.000000}%
\pgfsetstrokecolor{currentstroke}%
\pgfsetdash{}{0pt}%
\pgfsys@defobject{currentmarker}{\pgfqpoint{-0.027778in}{0.000000in}}{\pgfqpoint{-0.000000in}{0.000000in}}{%
\pgfpathmoveto{\pgfqpoint{-0.000000in}{0.000000in}}%
\pgfpathlineto{\pgfqpoint{-0.027778in}{0.000000in}}%
\pgfusepath{stroke,fill}%
}%
\begin{pgfscope}%
\pgfsys@transformshift{0.588387in}{2.291800in}%
\pgfsys@useobject{currentmarker}{}%
\end{pgfscope}%
\end{pgfscope}%
\begin{pgfscope}%
\pgfsetbuttcap%
\pgfsetroundjoin%
\definecolor{currentfill}{rgb}{0.000000,0.000000,0.000000}%
\pgfsetfillcolor{currentfill}%
\pgfsetlinewidth{0.602250pt}%
\definecolor{currentstroke}{rgb}{0.000000,0.000000,0.000000}%
\pgfsetstrokecolor{currentstroke}%
\pgfsetdash{}{0pt}%
\pgfsys@defobject{currentmarker}{\pgfqpoint{-0.027778in}{0.000000in}}{\pgfqpoint{-0.000000in}{0.000000in}}{%
\pgfpathmoveto{\pgfqpoint{-0.000000in}{0.000000in}}%
\pgfpathlineto{\pgfqpoint{-0.027778in}{0.000000in}}%
\pgfusepath{stroke,fill}%
}%
\begin{pgfscope}%
\pgfsys@transformshift{0.588387in}{2.329878in}%
\pgfsys@useobject{currentmarker}{}%
\end{pgfscope}%
\end{pgfscope}%
\begin{pgfscope}%
\pgfsetbuttcap%
\pgfsetroundjoin%
\definecolor{currentfill}{rgb}{0.000000,0.000000,0.000000}%
\pgfsetfillcolor{currentfill}%
\pgfsetlinewidth{0.602250pt}%
\definecolor{currentstroke}{rgb}{0.000000,0.000000,0.000000}%
\pgfsetstrokecolor{currentstroke}%
\pgfsetdash{}{0pt}%
\pgfsys@defobject{currentmarker}{\pgfqpoint{-0.027778in}{0.000000in}}{\pgfqpoint{-0.000000in}{0.000000in}}{%
\pgfpathmoveto{\pgfqpoint{-0.000000in}{0.000000in}}%
\pgfpathlineto{\pgfqpoint{-0.027778in}{0.000000in}}%
\pgfusepath{stroke,fill}%
}%
\begin{pgfscope}%
\pgfsys@transformshift{0.588387in}{2.359414in}%
\pgfsys@useobject{currentmarker}{}%
\end{pgfscope}%
\end{pgfscope}%
\begin{pgfscope}%
\pgfsetbuttcap%
\pgfsetroundjoin%
\definecolor{currentfill}{rgb}{0.000000,0.000000,0.000000}%
\pgfsetfillcolor{currentfill}%
\pgfsetlinewidth{0.602250pt}%
\definecolor{currentstroke}{rgb}{0.000000,0.000000,0.000000}%
\pgfsetstrokecolor{currentstroke}%
\pgfsetdash{}{0pt}%
\pgfsys@defobject{currentmarker}{\pgfqpoint{-0.027778in}{0.000000in}}{\pgfqpoint{-0.000000in}{0.000000in}}{%
\pgfpathmoveto{\pgfqpoint{-0.000000in}{0.000000in}}%
\pgfpathlineto{\pgfqpoint{-0.027778in}{0.000000in}}%
\pgfusepath{stroke,fill}%
}%
\begin{pgfscope}%
\pgfsys@transformshift{0.588387in}{2.383546in}%
\pgfsys@useobject{currentmarker}{}%
\end{pgfscope}%
\end{pgfscope}%
\begin{pgfscope}%
\pgfsetbuttcap%
\pgfsetroundjoin%
\definecolor{currentfill}{rgb}{0.000000,0.000000,0.000000}%
\pgfsetfillcolor{currentfill}%
\pgfsetlinewidth{0.602250pt}%
\definecolor{currentstroke}{rgb}{0.000000,0.000000,0.000000}%
\pgfsetstrokecolor{currentstroke}%
\pgfsetdash{}{0pt}%
\pgfsys@defobject{currentmarker}{\pgfqpoint{-0.027778in}{0.000000in}}{\pgfqpoint{-0.000000in}{0.000000in}}{%
\pgfpathmoveto{\pgfqpoint{-0.000000in}{0.000000in}}%
\pgfpathlineto{\pgfqpoint{-0.027778in}{0.000000in}}%
\pgfusepath{stroke,fill}%
}%
\begin{pgfscope}%
\pgfsys@transformshift{0.588387in}{2.403949in}%
\pgfsys@useobject{currentmarker}{}%
\end{pgfscope}%
\end{pgfscope}%
\begin{pgfscope}%
\pgfsetbuttcap%
\pgfsetroundjoin%
\definecolor{currentfill}{rgb}{0.000000,0.000000,0.000000}%
\pgfsetfillcolor{currentfill}%
\pgfsetlinewidth{0.602250pt}%
\definecolor{currentstroke}{rgb}{0.000000,0.000000,0.000000}%
\pgfsetstrokecolor{currentstroke}%
\pgfsetdash{}{0pt}%
\pgfsys@defobject{currentmarker}{\pgfqpoint{-0.027778in}{0.000000in}}{\pgfqpoint{-0.000000in}{0.000000in}}{%
\pgfpathmoveto{\pgfqpoint{-0.000000in}{0.000000in}}%
\pgfpathlineto{\pgfqpoint{-0.027778in}{0.000000in}}%
\pgfusepath{stroke,fill}%
}%
\begin{pgfscope}%
\pgfsys@transformshift{0.588387in}{2.421624in}%
\pgfsys@useobject{currentmarker}{}%
\end{pgfscope}%
\end{pgfscope}%
\begin{pgfscope}%
\pgfsetbuttcap%
\pgfsetroundjoin%
\definecolor{currentfill}{rgb}{0.000000,0.000000,0.000000}%
\pgfsetfillcolor{currentfill}%
\pgfsetlinewidth{0.602250pt}%
\definecolor{currentstroke}{rgb}{0.000000,0.000000,0.000000}%
\pgfsetstrokecolor{currentstroke}%
\pgfsetdash{}{0pt}%
\pgfsys@defobject{currentmarker}{\pgfqpoint{-0.027778in}{0.000000in}}{\pgfqpoint{-0.000000in}{0.000000in}}{%
\pgfpathmoveto{\pgfqpoint{-0.000000in}{0.000000in}}%
\pgfpathlineto{\pgfqpoint{-0.027778in}{0.000000in}}%
\pgfusepath{stroke,fill}%
}%
\begin{pgfscope}%
\pgfsys@transformshift{0.588387in}{2.437214in}%
\pgfsys@useobject{currentmarker}{}%
\end{pgfscope}%
\end{pgfscope}%
\begin{pgfscope}%
\pgfsetbuttcap%
\pgfsetroundjoin%
\definecolor{currentfill}{rgb}{0.000000,0.000000,0.000000}%
\pgfsetfillcolor{currentfill}%
\pgfsetlinewidth{0.602250pt}%
\definecolor{currentstroke}{rgb}{0.000000,0.000000,0.000000}%
\pgfsetstrokecolor{currentstroke}%
\pgfsetdash{}{0pt}%
\pgfsys@defobject{currentmarker}{\pgfqpoint{-0.027778in}{0.000000in}}{\pgfqpoint{-0.000000in}{0.000000in}}{%
\pgfpathmoveto{\pgfqpoint{-0.000000in}{0.000000in}}%
\pgfpathlineto{\pgfqpoint{-0.027778in}{0.000000in}}%
\pgfusepath{stroke,fill}%
}%
\begin{pgfscope}%
\pgfsys@transformshift{0.588387in}{2.542905in}%
\pgfsys@useobject{currentmarker}{}%
\end{pgfscope}%
\end{pgfscope}%
\begin{pgfscope}%
\pgfsetbuttcap%
\pgfsetroundjoin%
\definecolor{currentfill}{rgb}{0.000000,0.000000,0.000000}%
\pgfsetfillcolor{currentfill}%
\pgfsetlinewidth{0.602250pt}%
\definecolor{currentstroke}{rgb}{0.000000,0.000000,0.000000}%
\pgfsetstrokecolor{currentstroke}%
\pgfsetdash{}{0pt}%
\pgfsys@defobject{currentmarker}{\pgfqpoint{-0.027778in}{0.000000in}}{\pgfqpoint{-0.000000in}{0.000000in}}{%
\pgfpathmoveto{\pgfqpoint{-0.000000in}{0.000000in}}%
\pgfpathlineto{\pgfqpoint{-0.027778in}{0.000000in}}%
\pgfusepath{stroke,fill}%
}%
\begin{pgfscope}%
\pgfsys@transformshift{0.588387in}{2.596573in}%
\pgfsys@useobject{currentmarker}{}%
\end{pgfscope}%
\end{pgfscope}%
\begin{pgfscope}%
\pgfsetbuttcap%
\pgfsetroundjoin%
\definecolor{currentfill}{rgb}{0.000000,0.000000,0.000000}%
\pgfsetfillcolor{currentfill}%
\pgfsetlinewidth{0.602250pt}%
\definecolor{currentstroke}{rgb}{0.000000,0.000000,0.000000}%
\pgfsetstrokecolor{currentstroke}%
\pgfsetdash{}{0pt}%
\pgfsys@defobject{currentmarker}{\pgfqpoint{-0.027778in}{0.000000in}}{\pgfqpoint{-0.000000in}{0.000000in}}{%
\pgfpathmoveto{\pgfqpoint{-0.000000in}{0.000000in}}%
\pgfpathlineto{\pgfqpoint{-0.027778in}{0.000000in}}%
\pgfusepath{stroke,fill}%
}%
\begin{pgfscope}%
\pgfsys@transformshift{0.588387in}{2.634651in}%
\pgfsys@useobject{currentmarker}{}%
\end{pgfscope}%
\end{pgfscope}%
\begin{pgfscope}%
\pgfsetbuttcap%
\pgfsetroundjoin%
\definecolor{currentfill}{rgb}{0.000000,0.000000,0.000000}%
\pgfsetfillcolor{currentfill}%
\pgfsetlinewidth{0.602250pt}%
\definecolor{currentstroke}{rgb}{0.000000,0.000000,0.000000}%
\pgfsetstrokecolor{currentstroke}%
\pgfsetdash{}{0pt}%
\pgfsys@defobject{currentmarker}{\pgfqpoint{-0.027778in}{0.000000in}}{\pgfqpoint{-0.000000in}{0.000000in}}{%
\pgfpathmoveto{\pgfqpoint{-0.000000in}{0.000000in}}%
\pgfpathlineto{\pgfqpoint{-0.027778in}{0.000000in}}%
\pgfusepath{stroke,fill}%
}%
\begin{pgfscope}%
\pgfsys@transformshift{0.588387in}{2.664186in}%
\pgfsys@useobject{currentmarker}{}%
\end{pgfscope}%
\end{pgfscope}%
\begin{pgfscope}%
\pgfsetbuttcap%
\pgfsetroundjoin%
\definecolor{currentfill}{rgb}{0.000000,0.000000,0.000000}%
\pgfsetfillcolor{currentfill}%
\pgfsetlinewidth{0.602250pt}%
\definecolor{currentstroke}{rgb}{0.000000,0.000000,0.000000}%
\pgfsetstrokecolor{currentstroke}%
\pgfsetdash{}{0pt}%
\pgfsys@defobject{currentmarker}{\pgfqpoint{-0.027778in}{0.000000in}}{\pgfqpoint{-0.000000in}{0.000000in}}{%
\pgfpathmoveto{\pgfqpoint{-0.000000in}{0.000000in}}%
\pgfpathlineto{\pgfqpoint{-0.027778in}{0.000000in}}%
\pgfusepath{stroke,fill}%
}%
\begin{pgfscope}%
\pgfsys@transformshift{0.588387in}{2.688318in}%
\pgfsys@useobject{currentmarker}{}%
\end{pgfscope}%
\end{pgfscope}%
\begin{pgfscope}%
\pgfsetbuttcap%
\pgfsetroundjoin%
\definecolor{currentfill}{rgb}{0.000000,0.000000,0.000000}%
\pgfsetfillcolor{currentfill}%
\pgfsetlinewidth{0.602250pt}%
\definecolor{currentstroke}{rgb}{0.000000,0.000000,0.000000}%
\pgfsetstrokecolor{currentstroke}%
\pgfsetdash{}{0pt}%
\pgfsys@defobject{currentmarker}{\pgfqpoint{-0.027778in}{0.000000in}}{\pgfqpoint{-0.000000in}{0.000000in}}{%
\pgfpathmoveto{\pgfqpoint{-0.000000in}{0.000000in}}%
\pgfpathlineto{\pgfqpoint{-0.027778in}{0.000000in}}%
\pgfusepath{stroke,fill}%
}%
\begin{pgfscope}%
\pgfsys@transformshift{0.588387in}{2.708722in}%
\pgfsys@useobject{currentmarker}{}%
\end{pgfscope}%
\end{pgfscope}%
\begin{pgfscope}%
\pgfsetbuttcap%
\pgfsetroundjoin%
\definecolor{currentfill}{rgb}{0.000000,0.000000,0.000000}%
\pgfsetfillcolor{currentfill}%
\pgfsetlinewidth{0.602250pt}%
\definecolor{currentstroke}{rgb}{0.000000,0.000000,0.000000}%
\pgfsetstrokecolor{currentstroke}%
\pgfsetdash{}{0pt}%
\pgfsys@defobject{currentmarker}{\pgfqpoint{-0.027778in}{0.000000in}}{\pgfqpoint{-0.000000in}{0.000000in}}{%
\pgfpathmoveto{\pgfqpoint{-0.000000in}{0.000000in}}%
\pgfpathlineto{\pgfqpoint{-0.027778in}{0.000000in}}%
\pgfusepath{stroke,fill}%
}%
\begin{pgfscope}%
\pgfsys@transformshift{0.588387in}{2.726396in}%
\pgfsys@useobject{currentmarker}{}%
\end{pgfscope}%
\end{pgfscope}%
\begin{pgfscope}%
\definecolor{textcolor}{rgb}{0.000000,0.000000,0.000000}%
\pgfsetstrokecolor{textcolor}%
\pgfsetfillcolor{textcolor}%
\pgftext[x=0.234413in,y=1.631726in,,bottom,rotate=90.000000]{\color{textcolor}{\rmfamily\fontsize{10.000000}{12.000000}\selectfont\catcode`\^=\active\def^{\ifmmode\sp\else\^{}\fi}\catcode`\%=\active\def%{\%}Checks [call]}}%
\end{pgfscope}%
\begin{pgfscope}%
\pgfpathrectangle{\pgfqpoint{0.588387in}{0.521603in}}{\pgfqpoint{3.660036in}{2.220246in}}%
\pgfusepath{clip}%
\pgfsetrectcap%
\pgfsetroundjoin%
\pgfsetlinewidth{1.505625pt}%
\pgfsetstrokecolor{currentstroke1}%
\pgfsetdash{}{0pt}%
\pgfpathmoveto{\pgfqpoint{0.754752in}{0.714269in}}%
\pgfpathlineto{\pgfqpoint{0.862085in}{0.806015in}}%
\pgfpathlineto{\pgfqpoint{0.969417in}{0.897761in}}%
\pgfpathlineto{\pgfqpoint{1.076750in}{0.989506in}}%
\pgfpathlineto{\pgfqpoint{1.184082in}{1.081252in}}%
\pgfpathlineto{\pgfqpoint{1.291415in}{1.172998in}}%
\pgfpathlineto{\pgfqpoint{1.398747in}{1.242632in}}%
\pgfpathlineto{\pgfqpoint{1.506079in}{1.284953in}}%
\pgfpathlineto{\pgfqpoint{1.613412in}{1.369768in}}%
\pgfpathlineto{\pgfqpoint{1.720744in}{1.434398in}}%
\pgfpathlineto{\pgfqpoint{1.828077in}{1.531928in}}%
\pgfpathlineto{\pgfqpoint{1.935409in}{1.571085in}}%
\pgfpathlineto{\pgfqpoint{2.042742in}{1.618955in}}%
\pgfpathlineto{\pgfqpoint{2.150074in}{1.700564in}}%
\pgfpathlineto{\pgfqpoint{2.257406in}{1.748720in}}%
\pgfpathlineto{\pgfqpoint{2.364739in}{1.776153in}}%
\pgfpathlineto{\pgfqpoint{2.472071in}{1.844116in}}%
\pgfpathlineto{\pgfqpoint{2.579404in}{1.865197in}}%
\pgfpathlineto{\pgfqpoint{2.686736in}{1.971424in}}%
\pgfpathlineto{\pgfqpoint{2.794068in}{1.986395in}}%
\pgfpathlineto{\pgfqpoint{2.901401in}{2.053367in}}%
\pgfpathlineto{\pgfqpoint{3.008733in}{2.037879in}}%
\pgfpathlineto{\pgfqpoint{3.116066in}{2.148797in}}%
\pgfpathlineto{\pgfqpoint{3.223398in}{2.159672in}}%
\pgfpathlineto{\pgfqpoint{3.330731in}{2.236159in}}%
\pgfpathlineto{\pgfqpoint{3.438063in}{2.221980in}}%
\pgfpathlineto{\pgfqpoint{3.652728in}{2.246990in}}%
\pgfpathlineto{\pgfqpoint{3.867393in}{2.246671in}}%
\pgfusepath{stroke}%
\end{pgfscope}%
\begin{pgfscope}%
\pgfpathrectangle{\pgfqpoint{0.588387in}{0.521603in}}{\pgfqpoint{3.660036in}{2.220246in}}%
\pgfusepath{clip}%
\pgfsetrectcap%
\pgfsetroundjoin%
\pgfsetlinewidth{1.505625pt}%
\pgfsetstrokecolor{currentstroke2}%
\pgfsetdash{}{0pt}%
\pgfpathmoveto{\pgfqpoint{0.754752in}{0.714269in}}%
\pgfpathlineto{\pgfqpoint{0.862085in}{0.806015in}}%
\pgfpathlineto{\pgfqpoint{0.969417in}{0.897761in}}%
\pgfpathlineto{\pgfqpoint{1.076750in}{0.989506in}}%
\pgfpathlineto{\pgfqpoint{1.184082in}{1.081252in}}%
\pgfpathlineto{\pgfqpoint{1.291415in}{1.172998in}}%
\pgfpathlineto{\pgfqpoint{1.398747in}{1.242632in}}%
\pgfpathlineto{\pgfqpoint{1.506079in}{1.284953in}}%
\pgfpathlineto{\pgfqpoint{1.613412in}{1.369768in}}%
\pgfpathlineto{\pgfqpoint{1.720744in}{1.434398in}}%
\pgfpathlineto{\pgfqpoint{1.828077in}{1.515250in}}%
\pgfpathlineto{\pgfqpoint{1.935409in}{1.556930in}}%
\pgfpathlineto{\pgfqpoint{2.042742in}{1.605659in}}%
\pgfpathlineto{\pgfqpoint{2.150074in}{1.675436in}}%
\pgfpathlineto{\pgfqpoint{2.257406in}{1.754488in}}%
\pgfpathlineto{\pgfqpoint{2.364739in}{1.777285in}}%
\pgfpathlineto{\pgfqpoint{2.472071in}{1.833320in}}%
\pgfpathlineto{\pgfqpoint{2.579404in}{1.870332in}}%
\pgfpathlineto{\pgfqpoint{2.686736in}{1.934211in}}%
\pgfpathlineto{\pgfqpoint{2.794068in}{1.967292in}}%
\pgfpathlineto{\pgfqpoint{2.901401in}{2.035772in}}%
\pgfpathlineto{\pgfqpoint{3.008733in}{2.067318in}}%
\pgfpathlineto{\pgfqpoint{3.116066in}{2.109640in}}%
\pgfpathlineto{\pgfqpoint{3.223398in}{2.136521in}}%
\pgfpathlineto{\pgfqpoint{3.330731in}{2.203448in}}%
\pgfpathlineto{\pgfqpoint{3.438063in}{2.230516in}}%
\pgfpathlineto{\pgfqpoint{3.545395in}{2.262375in}}%
\pgfpathlineto{\pgfqpoint{3.652728in}{2.243989in}}%
\pgfpathlineto{\pgfqpoint{3.867393in}{2.274635in}}%
\pgfusepath{stroke}%
\end{pgfscope}%
\begin{pgfscope}%
\pgfpathrectangle{\pgfqpoint{0.588387in}{0.521603in}}{\pgfqpoint{3.660036in}{2.220246in}}%
\pgfusepath{clip}%
\pgfsetrectcap%
\pgfsetroundjoin%
\pgfsetlinewidth{1.505625pt}%
\pgfsetstrokecolor{currentstroke3}%
\pgfsetdash{}{0pt}%
\pgfpathmoveto{\pgfqpoint{0.754752in}{0.622524in}}%
\pgfpathlineto{\pgfqpoint{0.862085in}{0.767937in}}%
\pgfpathlineto{\pgfqpoint{0.969417in}{0.880086in}}%
\pgfpathlineto{\pgfqpoint{1.076750in}{0.980964in}}%
\pgfpathlineto{\pgfqpoint{1.184082in}{1.077050in}}%
\pgfpathlineto{\pgfqpoint{1.291415in}{1.170913in}}%
\pgfpathlineto{\pgfqpoint{1.398747in}{1.263705in}}%
\pgfpathlineto{\pgfqpoint{1.506079in}{1.355971in}}%
\pgfpathlineto{\pgfqpoint{1.613412in}{1.447976in}}%
\pgfpathlineto{\pgfqpoint{1.720744in}{1.539851in}}%
\pgfpathlineto{\pgfqpoint{1.828077in}{1.631662in}}%
\pgfpathlineto{\pgfqpoint{1.935409in}{1.723440in}}%
\pgfpathlineto{\pgfqpoint{2.042742in}{1.815202in}}%
\pgfpathlineto{\pgfqpoint{2.150074in}{1.906955in}}%
\pgfpathlineto{\pgfqpoint{2.257406in}{1.998705in}}%
\pgfpathlineto{\pgfqpoint{2.364739in}{2.090453in}}%
\pgfpathlineto{\pgfqpoint{2.472071in}{2.182199in}}%
\pgfpathlineto{\pgfqpoint{2.579404in}{2.273946in}}%
\pgfpathlineto{\pgfqpoint{2.686736in}{2.365692in}}%
\pgfpathlineto{\pgfqpoint{2.794068in}{2.457437in}}%
\pgfpathlineto{\pgfqpoint{2.901401in}{2.549183in}}%
\pgfpathlineto{\pgfqpoint{3.008733in}{2.640929in}}%
\pgfusepath{stroke}%
\end{pgfscope}%
\begin{pgfscope}%
\pgfpathrectangle{\pgfqpoint{0.588387in}{0.521603in}}{\pgfqpoint{3.660036in}{2.220246in}}%
\pgfusepath{clip}%
\pgfsetrectcap%
\pgfsetroundjoin%
\pgfsetlinewidth{1.505625pt}%
\pgfsetstrokecolor{currentstroke4}%
\pgfsetdash{}{0pt}%
\pgfpathmoveto{\pgfqpoint{0.754752in}{0.714269in}}%
\pgfpathlineto{\pgfqpoint{0.862085in}{0.806015in}}%
\pgfpathlineto{\pgfqpoint{0.969417in}{0.897761in}}%
\pgfpathlineto{\pgfqpoint{1.076750in}{0.989506in}}%
\pgfpathlineto{\pgfqpoint{1.184082in}{1.081252in}}%
\pgfpathlineto{\pgfqpoint{1.291415in}{1.172998in}}%
\pgfpathlineto{\pgfqpoint{1.398747in}{1.235684in}}%
\pgfpathlineto{\pgfqpoint{1.506079in}{1.293517in}}%
\pgfpathlineto{\pgfqpoint{1.613412in}{1.379770in}}%
\pgfpathlineto{\pgfqpoint{1.720744in}{1.421334in}}%
\pgfpathlineto{\pgfqpoint{1.828077in}{1.485939in}}%
\pgfpathlineto{\pgfqpoint{1.935409in}{1.540957in}}%
\pgfpathlineto{\pgfqpoint{2.042742in}{1.588699in}}%
\pgfpathlineto{\pgfqpoint{2.150074in}{1.674853in}}%
\pgfpathlineto{\pgfqpoint{2.257406in}{1.715774in}}%
\pgfpathlineto{\pgfqpoint{2.364739in}{1.708422in}}%
\pgfpathlineto{\pgfqpoint{2.472071in}{1.778931in}}%
\pgfpathlineto{\pgfqpoint{2.579404in}{1.790648in}}%
\pgfpathlineto{\pgfqpoint{2.686736in}{1.938024in}}%
\pgfpathlineto{\pgfqpoint{2.794068in}{1.920925in}}%
\pgfpathlineto{\pgfqpoint{2.901401in}{2.000580in}}%
\pgfpathlineto{\pgfqpoint{3.008733in}{2.007648in}}%
\pgfpathlineto{\pgfqpoint{3.116066in}{2.094664in}}%
\pgfpathlineto{\pgfqpoint{3.223398in}{2.103440in}}%
\pgfpathlineto{\pgfqpoint{3.330731in}{2.133594in}}%
\pgfpathlineto{\pgfqpoint{3.438063in}{2.205353in}}%
\pgfpathlineto{\pgfqpoint{3.545395in}{2.140080in}}%
\pgfpathlineto{\pgfqpoint{3.652728in}{2.234055in}}%
\pgfpathlineto{\pgfqpoint{3.867393in}{2.304646in}}%
\pgfusepath{stroke}%
\end{pgfscope}%
\begin{pgfscope}%
\pgfpathrectangle{\pgfqpoint{0.588387in}{0.521603in}}{\pgfqpoint{3.660036in}{2.220246in}}%
\pgfusepath{clip}%
\pgfsetrectcap%
\pgfsetroundjoin%
\pgfsetlinewidth{1.505625pt}%
\pgfsetstrokecolor{currentstroke5}%
\pgfsetdash{}{0pt}%
\pgfpathmoveto{\pgfqpoint{0.754752in}{0.714269in}}%
\pgfpathlineto{\pgfqpoint{0.862085in}{0.806015in}}%
\pgfpathlineto{\pgfqpoint{0.969417in}{0.897761in}}%
\pgfpathlineto{\pgfqpoint{1.076750in}{0.989506in}}%
\pgfpathlineto{\pgfqpoint{1.184082in}{1.081252in}}%
\pgfpathlineto{\pgfqpoint{1.291415in}{1.172998in}}%
\pgfpathlineto{\pgfqpoint{1.398747in}{1.235684in}}%
\pgfpathlineto{\pgfqpoint{1.506079in}{1.293517in}}%
\pgfpathlineto{\pgfqpoint{1.613412in}{1.379770in}}%
\pgfpathlineto{\pgfqpoint{1.720744in}{1.421334in}}%
\pgfpathlineto{\pgfqpoint{1.828077in}{1.474859in}}%
\pgfpathlineto{\pgfqpoint{1.935409in}{1.511893in}}%
\pgfpathlineto{\pgfqpoint{2.042742in}{1.576086in}}%
\pgfpathlineto{\pgfqpoint{2.150074in}{1.631405in}}%
\pgfpathlineto{\pgfqpoint{2.257406in}{1.685316in}}%
\pgfpathlineto{\pgfqpoint{2.364739in}{1.690605in}}%
\pgfpathlineto{\pgfqpoint{2.472071in}{1.767153in}}%
\pgfpathlineto{\pgfqpoint{2.579404in}{1.780123in}}%
\pgfpathlineto{\pgfqpoint{2.686736in}{1.865090in}}%
\pgfpathlineto{\pgfqpoint{2.794068in}{1.891655in}}%
\pgfpathlineto{\pgfqpoint{2.901401in}{1.952611in}}%
\pgfpathlineto{\pgfqpoint{3.008733in}{1.971875in}}%
\pgfpathlineto{\pgfqpoint{3.116066in}{2.082076in}}%
\pgfpathlineto{\pgfqpoint{3.223398in}{2.022721in}}%
\pgfpathlineto{\pgfqpoint{3.330731in}{2.064199in}}%
\pgfpathlineto{\pgfqpoint{3.438063in}{2.123577in}}%
\pgfpathlineto{\pgfqpoint{3.545395in}{2.170227in}}%
\pgfpathlineto{\pgfqpoint{3.652728in}{2.142684in}}%
\pgfpathlineto{\pgfqpoint{3.867393in}{2.290090in}}%
\pgfusepath{stroke}%
\end{pgfscope}%
\begin{pgfscope}%
\pgfpathrectangle{\pgfqpoint{0.588387in}{0.521603in}}{\pgfqpoint{3.660036in}{2.220246in}}%
\pgfusepath{clip}%
\pgfsetrectcap%
\pgfsetroundjoin%
\pgfsetlinewidth{1.505625pt}%
\pgfsetstrokecolor{currentstroke6}%
\pgfsetdash{}{0pt}%
\pgfpathmoveto{\pgfqpoint{0.754752in}{0.714269in}}%
\pgfpathlineto{\pgfqpoint{0.862085in}{0.806015in}}%
\pgfpathlineto{\pgfqpoint{0.969417in}{0.897761in}}%
\pgfpathlineto{\pgfqpoint{1.076750in}{0.989506in}}%
\pgfpathlineto{\pgfqpoint{1.184082in}{1.081252in}}%
\pgfpathlineto{\pgfqpoint{1.291415in}{1.172998in}}%
\pgfpathlineto{\pgfqpoint{1.398747in}{1.240280in}}%
\pgfpathlineto{\pgfqpoint{1.506079in}{1.297812in}}%
\pgfpathlineto{\pgfqpoint{1.613412in}{1.382019in}}%
\pgfpathlineto{\pgfqpoint{1.720744in}{1.435765in}}%
\pgfpathlineto{\pgfqpoint{1.828077in}{1.509542in}}%
\pgfpathlineto{\pgfqpoint{1.935409in}{1.556138in}}%
\pgfpathlineto{\pgfqpoint{2.042742in}{1.628057in}}%
\pgfpathlineto{\pgfqpoint{2.150074in}{1.679085in}}%
\pgfpathlineto{\pgfqpoint{2.257406in}{1.731462in}}%
\pgfpathlineto{\pgfqpoint{2.364739in}{1.751998in}}%
\pgfpathlineto{\pgfqpoint{2.472071in}{1.831177in}}%
\pgfpathlineto{\pgfqpoint{2.579404in}{1.832720in}}%
\pgfpathlineto{\pgfqpoint{2.686736in}{1.966199in}}%
\pgfpathlineto{\pgfqpoint{2.794068in}{1.964782in}}%
\pgfpathlineto{\pgfqpoint{2.901401in}{2.031896in}}%
\pgfpathlineto{\pgfqpoint{3.008733in}{2.048123in}}%
\pgfpathlineto{\pgfqpoint{3.116066in}{2.134688in}}%
\pgfpathlineto{\pgfqpoint{3.223398in}{2.150671in}}%
\pgfpathlineto{\pgfqpoint{3.330731in}{2.224267in}}%
\pgfpathlineto{\pgfqpoint{3.438063in}{2.203115in}}%
\pgfpathlineto{\pgfqpoint{3.545395in}{2.274098in}}%
\pgfpathlineto{\pgfqpoint{3.652728in}{2.262308in}}%
\pgfpathlineto{\pgfqpoint{3.867393in}{2.148987in}}%
\pgfusepath{stroke}%
\end{pgfscope}%
\begin{pgfscope}%
\pgfpathrectangle{\pgfqpoint{0.588387in}{0.521603in}}{\pgfqpoint{3.660036in}{2.220246in}}%
\pgfusepath{clip}%
\pgfsetrectcap%
\pgfsetroundjoin%
\pgfsetlinewidth{1.505625pt}%
\pgfsetstrokecolor{currentstroke7}%
\pgfsetdash{}{0pt}%
\pgfpathmoveto{\pgfqpoint{0.754752in}{0.714269in}}%
\pgfpathlineto{\pgfqpoint{0.862085in}{0.806015in}}%
\pgfpathlineto{\pgfqpoint{0.969417in}{0.897761in}}%
\pgfpathlineto{\pgfqpoint{1.076750in}{0.989506in}}%
\pgfpathlineto{\pgfqpoint{1.184082in}{1.081252in}}%
\pgfpathlineto{\pgfqpoint{1.291415in}{1.172998in}}%
\pgfpathlineto{\pgfqpoint{1.398747in}{1.240280in}}%
\pgfpathlineto{\pgfqpoint{1.506079in}{1.297812in}}%
\pgfpathlineto{\pgfqpoint{1.613412in}{1.382019in}}%
\pgfpathlineto{\pgfqpoint{1.720744in}{1.435765in}}%
\pgfpathlineto{\pgfqpoint{1.828077in}{1.485448in}}%
\pgfpathlineto{\pgfqpoint{1.935409in}{1.522911in}}%
\pgfpathlineto{\pgfqpoint{2.042742in}{1.589090in}}%
\pgfpathlineto{\pgfqpoint{2.150074in}{1.629815in}}%
\pgfpathlineto{\pgfqpoint{2.257406in}{1.683152in}}%
\pgfpathlineto{\pgfqpoint{2.364739in}{1.729091in}}%
\pgfpathlineto{\pgfqpoint{2.472071in}{1.782473in}}%
\pgfpathlineto{\pgfqpoint{2.579404in}{1.799646in}}%
\pgfpathlineto{\pgfqpoint{2.686736in}{1.875716in}}%
\pgfpathlineto{\pgfqpoint{2.794068in}{1.882999in}}%
\pgfpathlineto{\pgfqpoint{2.901401in}{1.987682in}}%
\pgfpathlineto{\pgfqpoint{3.008733in}{1.999015in}}%
\pgfpathlineto{\pgfqpoint{3.116066in}{2.082665in}}%
\pgfpathlineto{\pgfqpoint{3.223398in}{2.058938in}}%
\pgfpathlineto{\pgfqpoint{3.330731in}{2.134929in}}%
\pgfpathlineto{\pgfqpoint{3.438063in}{2.155432in}}%
\pgfpathlineto{\pgfqpoint{3.545395in}{2.187865in}}%
\pgfpathlineto{\pgfqpoint{3.652728in}{2.221312in}}%
\pgfpathlineto{\pgfqpoint{3.867393in}{2.178132in}}%
\pgfpathlineto{\pgfqpoint{4.082057in}{2.210156in}}%
\pgfusepath{stroke}%
\end{pgfscope}%
\begin{pgfscope}%
\pgfpathrectangle{\pgfqpoint{0.588387in}{0.521603in}}{\pgfqpoint{3.660036in}{2.220246in}}%
\pgfusepath{clip}%
\pgfsetrectcap%
\pgfsetroundjoin%
\pgfsetlinewidth{1.505625pt}%
\definecolor{currentstroke}{rgb}{0.498039,0.498039,0.498039}%
\pgfsetstrokecolor{currentstroke}%
\pgfsetdash{}{0pt}%
\pgfpathmoveto{\pgfqpoint{0.754752in}{0.714269in}}%
\pgfpathlineto{\pgfqpoint{0.862085in}{0.806015in}}%
\pgfpathlineto{\pgfqpoint{0.969417in}{0.897761in}}%
\pgfpathlineto{\pgfqpoint{1.076750in}{0.989506in}}%
\pgfpathlineto{\pgfqpoint{1.184082in}{1.081252in}}%
\pgfpathlineto{\pgfqpoint{1.291415in}{1.172998in}}%
\pgfpathlineto{\pgfqpoint{1.398747in}{1.261523in}}%
\pgfpathlineto{\pgfqpoint{1.506079in}{1.350454in}}%
\pgfpathlineto{\pgfqpoint{1.613412in}{1.421513in}}%
\pgfpathlineto{\pgfqpoint{1.720744in}{1.482303in}}%
\pgfpathlineto{\pgfqpoint{1.828077in}{1.554682in}}%
\pgfpathlineto{\pgfqpoint{1.935409in}{1.602710in}}%
\pgfpathlineto{\pgfqpoint{2.042742in}{1.673377in}}%
\pgfpathlineto{\pgfqpoint{2.150074in}{1.739749in}}%
\pgfpathlineto{\pgfqpoint{2.257406in}{1.798122in}}%
\pgfpathlineto{\pgfqpoint{2.364739in}{1.833564in}}%
\pgfpathlineto{\pgfqpoint{2.472071in}{1.903431in}}%
\pgfpathlineto{\pgfqpoint{2.579404in}{1.950208in}}%
\pgfpathlineto{\pgfqpoint{2.686736in}{2.023355in}}%
\pgfpathlineto{\pgfqpoint{2.794068in}{2.035826in}}%
\pgfpathlineto{\pgfqpoint{2.901401in}{2.112486in}}%
\pgfpathlineto{\pgfqpoint{3.008733in}{2.138707in}}%
\pgfpathlineto{\pgfqpoint{3.116066in}{2.250798in}}%
\pgfpathlineto{\pgfqpoint{3.223398in}{2.230848in}}%
\pgfpathlineto{\pgfqpoint{3.330731in}{2.258756in}}%
\pgfpathlineto{\pgfqpoint{3.438063in}{2.237999in}}%
\pgfpathlineto{\pgfqpoint{3.652728in}{2.257960in}}%
\pgfusepath{stroke}%
\end{pgfscope}%
\begin{pgfscope}%
\pgfpathrectangle{\pgfqpoint{0.588387in}{0.521603in}}{\pgfqpoint{3.660036in}{2.220246in}}%
\pgfusepath{clip}%
\pgfsetrectcap%
\pgfsetroundjoin%
\pgfsetlinewidth{1.505625pt}%
\definecolor{currentstroke}{rgb}{0.737255,0.741176,0.133333}%
\pgfsetstrokecolor{currentstroke}%
\pgfsetdash{}{0pt}%
\pgfpathmoveto{\pgfqpoint{0.754752in}{0.714269in}}%
\pgfpathlineto{\pgfqpoint{0.862085in}{0.806015in}}%
\pgfpathlineto{\pgfqpoint{0.969417in}{0.897761in}}%
\pgfpathlineto{\pgfqpoint{1.076750in}{0.989506in}}%
\pgfpathlineto{\pgfqpoint{1.184082in}{1.081252in}}%
\pgfpathlineto{\pgfqpoint{1.291415in}{1.172998in}}%
\pgfpathlineto{\pgfqpoint{1.398747in}{1.261523in}}%
\pgfpathlineto{\pgfqpoint{1.506079in}{1.350454in}}%
\pgfpathlineto{\pgfqpoint{1.613412in}{1.421513in}}%
\pgfpathlineto{\pgfqpoint{1.720744in}{1.482303in}}%
\pgfpathlineto{\pgfqpoint{1.828077in}{1.524132in}}%
\pgfpathlineto{\pgfqpoint{1.935409in}{1.578866in}}%
\pgfpathlineto{\pgfqpoint{2.042742in}{1.656073in}}%
\pgfpathlineto{\pgfqpoint{2.150074in}{1.721234in}}%
\pgfpathlineto{\pgfqpoint{2.257406in}{1.754956in}}%
\pgfpathlineto{\pgfqpoint{2.364739in}{1.802361in}}%
\pgfpathlineto{\pgfqpoint{2.472071in}{1.874203in}}%
\pgfpathlineto{\pgfqpoint{2.579404in}{1.913225in}}%
\pgfpathlineto{\pgfqpoint{2.686736in}{1.964518in}}%
\pgfpathlineto{\pgfqpoint{2.794068in}{1.979917in}}%
\pgfpathlineto{\pgfqpoint{2.901401in}{2.071933in}}%
\pgfpathlineto{\pgfqpoint{3.008733in}{2.116274in}}%
\pgfpathlineto{\pgfqpoint{3.116066in}{2.170694in}}%
\pgfpathlineto{\pgfqpoint{3.223398in}{2.181030in}}%
\pgfpathlineto{\pgfqpoint{3.330731in}{2.306018in}}%
\pgfpathlineto{\pgfqpoint{3.438063in}{2.230342in}}%
\pgfpathlineto{\pgfqpoint{3.545395in}{2.303487in}}%
\pgfpathlineto{\pgfqpoint{3.652728in}{2.275557in}}%
\pgfusepath{stroke}%
\end{pgfscope}%
\begin{pgfscope}%
\pgfsetrectcap%
\pgfsetmiterjoin%
\pgfsetlinewidth{0.803000pt}%
\definecolor{currentstroke}{rgb}{0.000000,0.000000,0.000000}%
\pgfsetstrokecolor{currentstroke}%
\pgfsetdash{}{0pt}%
\pgfpathmoveto{\pgfqpoint{0.588387in}{0.521603in}}%
\pgfpathlineto{\pgfqpoint{0.588387in}{2.741849in}}%
\pgfusepath{stroke}%
\end{pgfscope}%
\begin{pgfscope}%
\pgfsetrectcap%
\pgfsetmiterjoin%
\pgfsetlinewidth{0.803000pt}%
\definecolor{currentstroke}{rgb}{0.000000,0.000000,0.000000}%
\pgfsetstrokecolor{currentstroke}%
\pgfsetdash{}{0pt}%
\pgfpathmoveto{\pgfqpoint{4.248423in}{0.521603in}}%
\pgfpathlineto{\pgfqpoint{4.248423in}{2.741849in}}%
\pgfusepath{stroke}%
\end{pgfscope}%
\begin{pgfscope}%
\pgfsetrectcap%
\pgfsetmiterjoin%
\pgfsetlinewidth{0.803000pt}%
\definecolor{currentstroke}{rgb}{0.000000,0.000000,0.000000}%
\pgfsetstrokecolor{currentstroke}%
\pgfsetdash{}{0pt}%
\pgfpathmoveto{\pgfqpoint{0.588387in}{0.521603in}}%
\pgfpathlineto{\pgfqpoint{4.248423in}{0.521603in}}%
\pgfusepath{stroke}%
\end{pgfscope}%
\begin{pgfscope}%
\pgfsetrectcap%
\pgfsetmiterjoin%
\pgfsetlinewidth{0.803000pt}%
\definecolor{currentstroke}{rgb}{0.000000,0.000000,0.000000}%
\pgfsetstrokecolor{currentstroke}%
\pgfsetdash{}{0pt}%
\pgfpathmoveto{\pgfqpoint{0.588387in}{2.741849in}}%
\pgfpathlineto{\pgfqpoint{4.248423in}{2.741849in}}%
\pgfusepath{stroke}%
\end{pgfscope}%
\begin{pgfscope}%
\pgfsetbuttcap%
\pgfsetmiterjoin%
\definecolor{currentfill}{rgb}{1.000000,1.000000,1.000000}%
\pgfsetfillcolor{currentfill}%
\pgfsetfillopacity{0.800000}%
\pgfsetlinewidth{1.003750pt}%
\definecolor{currentstroke}{rgb}{0.800000,0.800000,0.800000}%
\pgfsetstrokecolor{currentstroke}%
\pgfsetstrokeopacity{0.800000}%
\pgfsetdash{}{0pt}%
\pgfpathmoveto{\pgfqpoint{4.365089in}{0.379025in}}%
\pgfpathlineto{\pgfqpoint{8.251043in}{0.379025in}}%
\pgfpathquadraticcurveto{\pgfqpoint{8.284376in}{0.379025in}}{\pgfqpoint{8.284376in}{0.412359in}}%
\pgfpathlineto{\pgfqpoint{8.284376in}{2.625183in}}%
\pgfpathquadraticcurveto{\pgfqpoint{8.284376in}{2.658516in}}{\pgfqpoint{8.251043in}{2.658516in}}%
\pgfpathlineto{\pgfqpoint{4.365089in}{2.658516in}}%
\pgfpathquadraticcurveto{\pgfqpoint{4.331756in}{2.658516in}}{\pgfqpoint{4.331756in}{2.625183in}}%
\pgfpathlineto{\pgfqpoint{4.331756in}{0.412359in}}%
\pgfpathquadraticcurveto{\pgfqpoint{4.331756in}{0.379025in}}{\pgfqpoint{4.365089in}{0.379025in}}%
\pgfpathlineto{\pgfqpoint{4.365089in}{0.379025in}}%
\pgfpathclose%
\pgfusepath{stroke,fill}%
\end{pgfscope}%
\begin{pgfscope}%
\pgfsetrectcap%
\pgfsetroundjoin%
\pgfsetlinewidth{1.505625pt}%
\pgfsetstrokecolor{currentstroke3}%
\pgfsetdash{}{0pt}%
\pgfpathmoveto{\pgfqpoint{4.398423in}{2.523555in}}%
\pgfpathlineto{\pgfqpoint{4.565089in}{2.523555in}}%
\pgfpathlineto{\pgfqpoint{4.731756in}{2.523555in}}%
\pgfusepath{stroke}%
\end{pgfscope}%
\begin{pgfscope}%
\definecolor{textcolor}{rgb}{0.000000,0.000000,0.000000}%
\pgfsetstrokecolor{textcolor}%
\pgfsetfillcolor{textcolor}%
\pgftext[x=4.865089in,y=2.465222in,left,base]{\color{textcolor}{\rmfamily\fontsize{12.000000}{14.400000}\selectfont\catcode`\^=\active\def^{\ifmmode\sp\else\^{}\fi}\catcode`\%=\active\def%{\%}\NaiveCycles{}}}%
\end{pgfscope}%
\begin{pgfscope}%
\pgfsetrectcap%
\pgfsetroundjoin%
\pgfsetlinewidth{1.505625pt}%
\pgfsetstrokecolor{currentstroke1}%
\pgfsetdash{}{0pt}%
\pgfpathmoveto{\pgfqpoint{4.398423in}{2.278926in}}%
\pgfpathlineto{\pgfqpoint{4.565089in}{2.278926in}}%
\pgfpathlineto{\pgfqpoint{4.731756in}{2.278926in}}%
\pgfusepath{stroke}%
\end{pgfscope}%
\begin{pgfscope}%
\definecolor{textcolor}{rgb}{0.000000,0.000000,0.000000}%
\pgfsetstrokecolor{textcolor}%
\pgfsetfillcolor{textcolor}%
\pgftext[x=4.865089in,y=2.220593in,left,base]{\color{textcolor}{\rmfamily\fontsize{12.000000}{14.400000}\selectfont\catcode`\^=\active\def^{\ifmmode\sp\else\^{}\fi}\catcode`\%=\active\def%{\%}\CyclesMatchChunks{} \& \MergeLinear{}}}%
\end{pgfscope}%
\begin{pgfscope}%
\pgfsetrectcap%
\pgfsetroundjoin%
\pgfsetlinewidth{1.505625pt}%
\pgfsetstrokecolor{currentstroke2}%
\pgfsetdash{}{0pt}%
\pgfpathmoveto{\pgfqpoint{4.398423in}{2.029659in}}%
\pgfpathlineto{\pgfqpoint{4.565089in}{2.029659in}}%
\pgfpathlineto{\pgfqpoint{4.731756in}{2.029659in}}%
\pgfusepath{stroke}%
\end{pgfscope}%
\begin{pgfscope}%
\definecolor{textcolor}{rgb}{0.000000,0.000000,0.000000}%
\pgfsetstrokecolor{textcolor}%
\pgfsetfillcolor{textcolor}%
\pgftext[x=4.865089in,y=1.971325in,left,base]{\color{textcolor}{\rmfamily\fontsize{12.000000}{14.400000}\selectfont\catcode`\^=\active\def^{\ifmmode\sp\else\^{}\fi}\catcode`\%=\active\def%{\%}\CyclesMatchChunks{} \& \SharedVertices{}}}%
\end{pgfscope}%
\begin{pgfscope}%
\pgfsetrectcap%
\pgfsetroundjoin%
\pgfsetlinewidth{1.505625pt}%
\pgfsetstrokecolor{currentstroke4}%
\pgfsetdash{}{0pt}%
\pgfpathmoveto{\pgfqpoint{4.398423in}{1.780391in}}%
\pgfpathlineto{\pgfqpoint{4.565089in}{1.780391in}}%
\pgfpathlineto{\pgfqpoint{4.731756in}{1.780391in}}%
\pgfusepath{stroke}%
\end{pgfscope}%
\begin{pgfscope}%
\definecolor{textcolor}{rgb}{0.000000,0.000000,0.000000}%
\pgfsetstrokecolor{textcolor}%
\pgfsetfillcolor{textcolor}%
\pgftext[x=4.865089in,y=1.722058in,left,base]{\color{textcolor}{\rmfamily\fontsize{12.000000}{14.400000}\selectfont\catcode`\^=\active\def^{\ifmmode\sp\else\^{}\fi}\catcode`\%=\active\def%{\%}\Neighbors{} \& \MergeLinear{}}}%
\end{pgfscope}%
\begin{pgfscope}%
\pgfsetrectcap%
\pgfsetroundjoin%
\pgfsetlinewidth{1.505625pt}%
\pgfsetstrokecolor{currentstroke5}%
\pgfsetdash{}{0pt}%
\pgfpathmoveto{\pgfqpoint{4.398423in}{1.535763in}}%
\pgfpathlineto{\pgfqpoint{4.565089in}{1.535763in}}%
\pgfpathlineto{\pgfqpoint{4.731756in}{1.535763in}}%
\pgfusepath{stroke}%
\end{pgfscope}%
\begin{pgfscope}%
\definecolor{textcolor}{rgb}{0.000000,0.000000,0.000000}%
\pgfsetstrokecolor{textcolor}%
\pgfsetfillcolor{textcolor}%
\pgftext[x=4.865089in,y=1.477429in,left,base]{\color{textcolor}{\rmfamily\fontsize{12.000000}{14.400000}\selectfont\catcode`\^=\active\def^{\ifmmode\sp\else\^{}\fi}\catcode`\%=\active\def%{\%}\Neighbors{} \& \SharedVertices{}}}%
\end{pgfscope}%
\begin{pgfscope}%
\pgfsetrectcap%
\pgfsetroundjoin%
\pgfsetlinewidth{1.505625pt}%
\pgfsetstrokecolor{currentstroke6}%
\pgfsetdash{}{0pt}%
\pgfpathmoveto{\pgfqpoint{4.398423in}{1.286495in}}%
\pgfpathlineto{\pgfqpoint{4.565089in}{1.286495in}}%
\pgfpathlineto{\pgfqpoint{4.731756in}{1.286495in}}%
\pgfusepath{stroke}%
\end{pgfscope}%
\begin{pgfscope}%
\definecolor{textcolor}{rgb}{0.000000,0.000000,0.000000}%
\pgfsetstrokecolor{textcolor}%
\pgfsetfillcolor{textcolor}%
\pgftext[x=4.865089in,y=1.228162in,left,base]{\color{textcolor}{\rmfamily\fontsize{12.000000}{14.400000}\selectfont\catcode`\^=\active\def^{\ifmmode\sp\else\^{}\fi}\catcode`\%=\active\def%{\%}\NeighborsDegree{} \& \MergeLinear{}}}%
\end{pgfscope}%
\begin{pgfscope}%
\pgfsetrectcap%
\pgfsetroundjoin%
\pgfsetlinewidth{1.505625pt}%
\pgfsetstrokecolor{currentstroke7}%
\pgfsetdash{}{0pt}%
\pgfpathmoveto{\pgfqpoint{4.398423in}{1.037228in}}%
\pgfpathlineto{\pgfqpoint{4.565089in}{1.037228in}}%
\pgfpathlineto{\pgfqpoint{4.731756in}{1.037228in}}%
\pgfusepath{stroke}%
\end{pgfscope}%
\begin{pgfscope}%
\definecolor{textcolor}{rgb}{0.000000,0.000000,0.000000}%
\pgfsetstrokecolor{textcolor}%
\pgfsetfillcolor{textcolor}%
\pgftext[x=4.865089in,y=0.978895in,left,base]{\color{textcolor}{\rmfamily\fontsize{12.000000}{14.400000}\selectfont\catcode`\^=\active\def^{\ifmmode\sp\else\^{}\fi}\catcode`\%=\active\def%{\%}\NeighborsDegree{} \& \SharedVertices{}}}%
\end{pgfscope}%
\begin{pgfscope}%
\pgfsetrectcap%
\pgfsetroundjoin%
\pgfsetlinewidth{1.505625pt}%
\definecolor{currentstroke}{rgb}{0.498039,0.498039,0.498039}%
\pgfsetstrokecolor{currentstroke}%
\pgfsetdash{}{0pt}%
\pgfpathmoveto{\pgfqpoint{4.398423in}{0.787961in}}%
\pgfpathlineto{\pgfqpoint{4.565089in}{0.787961in}}%
\pgfpathlineto{\pgfqpoint{4.731756in}{0.787961in}}%
\pgfusepath{stroke}%
\end{pgfscope}%
\begin{pgfscope}%
\definecolor{textcolor}{rgb}{0.000000,0.000000,0.000000}%
\pgfsetstrokecolor{textcolor}%
\pgfsetfillcolor{textcolor}%
\pgftext[x=4.865089in,y=0.729627in,left,base]{\color{textcolor}{\rmfamily\fontsize{12.000000}{14.400000}\selectfont\catcode`\^=\active\def^{\ifmmode\sp\else\^{}\fi}\catcode`\%=\active\def%{\%}\None{} \& \MergeLinear{}}}%
\end{pgfscope}%
\begin{pgfscope}%
\pgfsetrectcap%
\pgfsetroundjoin%
\pgfsetlinewidth{1.505625pt}%
\definecolor{currentstroke}{rgb}{0.737255,0.741176,0.133333}%
\pgfsetstrokecolor{currentstroke}%
\pgfsetdash{}{0pt}%
\pgfpathmoveto{\pgfqpoint{4.398423in}{0.543332in}}%
\pgfpathlineto{\pgfqpoint{4.565089in}{0.543332in}}%
\pgfpathlineto{\pgfqpoint{4.731756in}{0.543332in}}%
\pgfusepath{stroke}%
\end{pgfscope}%
\begin{pgfscope}%
\definecolor{textcolor}{rgb}{0.000000,0.000000,0.000000}%
\pgfsetstrokecolor{textcolor}%
\pgfsetfillcolor{textcolor}%
\pgftext[x=4.865089in,y=0.484999in,left,base]{\color{textcolor}{\rmfamily\fontsize{12.000000}{14.400000}\selectfont\catcode`\^=\active\def^{\ifmmode\sp\else\^{}\fi}\catcode`\%=\active\def%{\%}\None{} \& \SharedVertices{}}}%
\end{pgfscope}%
\end{pgfpicture}%
\makeatother%
\endgroup%
}
	\caption[Checks performed for minimally rigid graphs]{
		The number of checks performed to find all NAC-colorings for minimally rigid graphs.}%
	\label{fig:graph_count_minimally_rigid}
\end{figure}%

In \Cref{fig:graph_summary}
we show the relation between the number of \IsNACColoring{} checks that
would \Naive{} algorithm perform compared to our solution.
%
The values are similar for graphs with few monochromatic classes,
which explains why the \NaiveCycles{} algorithm outperformed
the \NeighborsDegree{}\&\MergeLinear{} algorithm in \Cref{tab:all_min_rigid}. This should improve quickly for larger graphs.
We can also see how the use of \CycleMask{} routine
reduces the number of more expensive \IsNACColoring{} calls,
since these are called only when the small cycles check \CycleMask{} fails to decide
(\CycleMask{} is called every time).

\begin{figure}[ht]
	\centering
	\scalebox{\BenchFigureScale}{%% Creator: Matplotlib, PGF backend
%%
%% To include the figure in your LaTeX document, write
%%   \input{<filename>.pgf}
%%
%% Make sure the required packages are loaded in your preamble
%%   \usepackage{pgf}
%%
%% Also ensure that all the required font packages are loaded; for instance,
%% the lmodern package is sometimes necessary when using math font.
%%   \usepackage{lmodern}
%%
%% Figures using additional raster images can only be included by \input if
%% they are in the same directory as the main LaTeX file. For loading figures
%% from other directories you can use the `import` package
%%   \usepackage{import}
%%
%% and then include the figures with
%%   \import{<path to file>}{<filename>.pgf}
%%
%% Matplotlib used the following preamble
%%   \def\mathdefault#1{#1}
%%   \everymath=\expandafter{\the\everymath\displaystyle}
%%   \IfFileExists{scrextend.sty}{
%%     \usepackage[fontsize=10.000000pt]{scrextend}
%%   }{
%%     \renewcommand{\normalsize}{\fontsize{10.000000}{12.000000}\selectfont}
%%     \normalsize
%%   }
%%   
%%   \ifdefined\pdftexversion\else  % non-pdftex case.
%%     \usepackage{fontspec}
%%     \setmainfont{DejaVuSans.ttf}[Path=\detokenize{/home/petr/Projects/PyRigi/.venv/lib/python3.12/site-packages/matplotlib/mpl-data/fonts/ttf/}]
%%     \setsansfont{DejaVuSans.ttf}[Path=\detokenize{/home/petr/Projects/PyRigi/.venv/lib/python3.12/site-packages/matplotlib/mpl-data/fonts/ttf/}]
%%     \setmonofont{DejaVuSansMono.ttf}[Path=\detokenize{/home/petr/Projects/PyRigi/.venv/lib/python3.12/site-packages/matplotlib/mpl-data/fonts/ttf/}]
%%   \fi
%%   \makeatletter\@ifpackageloaded{under\Score{}}{}{\usepackage[strings]{under\Score{}}}\makeatother
%%
\begingroup%
\makeatletter%
\begin{pgfpicture}%
\pgfpathrectangle{\pgfpointorigin}{\pgfqpoint{8.384376in}{2.841849in}}%
\pgfusepath{use as bounding box, clip}%
\begin{pgfscope}%
\pgfsetbuttcap%
\pgfsetmiterjoin%
\definecolor{currentfill}{rgb}{1.000000,1.000000,1.000000}%
\pgfsetfillcolor{currentfill}%
\pgfsetlinewidth{0.000000pt}%
\definecolor{currentstroke}{rgb}{1.000000,1.000000,1.000000}%
\pgfsetstrokecolor{currentstroke}%
\pgfsetdash{}{0pt}%
\pgfpathmoveto{\pgfqpoint{0.000000in}{0.000000in}}%
\pgfpathlineto{\pgfqpoint{8.384376in}{0.000000in}}%
\pgfpathlineto{\pgfqpoint{8.384376in}{2.841849in}}%
\pgfpathlineto{\pgfqpoint{0.000000in}{2.841849in}}%
\pgfpathlineto{\pgfqpoint{0.000000in}{0.000000in}}%
\pgfpathclose%
\pgfusepath{fill}%
\end{pgfscope}%
\begin{pgfscope}%
\pgfsetbuttcap%
\pgfsetmiterjoin%
\definecolor{currentfill}{rgb}{1.000000,1.000000,1.000000}%
\pgfsetfillcolor{currentfill}%
\pgfsetlinewidth{0.000000pt}%
\definecolor{currentstroke}{rgb}{0.000000,0.000000,0.000000}%
\pgfsetstrokecolor{currentstroke}%
\pgfsetstrokeopacity{0.000000}%
\pgfsetdash{}{0pt}%
\pgfpathmoveto{\pgfqpoint{0.643750in}{0.521603in}}%
\pgfpathlineto{\pgfqpoint{8.284376in}{0.521603in}}%
\pgfpathlineto{\pgfqpoint{8.284376in}{2.741849in}}%
\pgfpathlineto{\pgfqpoint{0.643750in}{2.741849in}}%
\pgfpathlineto{\pgfqpoint{0.643750in}{0.521603in}}%
\pgfpathclose%
\pgfusepath{fill}%
\end{pgfscope}%
\begin{pgfscope}%
\pgfsetbuttcap%
\pgfsetroundjoin%
\definecolor{currentfill}{rgb}{0.000000,0.000000,0.000000}%
\pgfsetfillcolor{currentfill}%
\pgfsetlinewidth{0.803000pt}%
\definecolor{currentstroke}{rgb}{0.000000,0.000000,0.000000}%
\pgfsetstrokecolor{currentstroke}%
\pgfsetdash{}{0pt}%
\pgfsys@defobject{currentmarker}{\pgfqpoint{0.000000in}{-0.048611in}}{\pgfqpoint{0.000000in}{0.000000in}}{%
\pgfpathmoveto{\pgfqpoint{0.000000in}{0.000000in}}%
\pgfpathlineto{\pgfqpoint{0.000000in}{-0.048611in}}%
\pgfusepath{stroke,fill}%
}%
\begin{pgfscope}%
\pgfsys@transformshift{1.425178in}{0.521603in}%
\pgfsys@useobject{currentmarker}{}%
\end{pgfscope}%
\end{pgfscope}%
\begin{pgfscope}%
\definecolor{textcolor}{rgb}{0.000000,0.000000,0.000000}%
\pgfsetstrokecolor{textcolor}%
\pgfsetfillcolor{textcolor}%
\pgftext[x=1.425178in,y=0.424381in,,top]{\color{textcolor}{\rmfamily\fontsize{10.000000}{12.000000}\selectfont\catcode`\^=\active\def^{\ifmmode\sp\else\^{}\fi}\catcode`\%=\active\def%{\%}$\mathdefault{4}$}}%
\end{pgfscope}%
\begin{pgfscope}%
\pgfsetbuttcap%
\pgfsetroundjoin%
\definecolor{currentfill}{rgb}{0.000000,0.000000,0.000000}%
\pgfsetfillcolor{currentfill}%
\pgfsetlinewidth{0.803000pt}%
\definecolor{currentstroke}{rgb}{0.000000,0.000000,0.000000}%
\pgfsetstrokecolor{currentstroke}%
\pgfsetdash{}{0pt}%
\pgfsys@defobject{currentmarker}{\pgfqpoint{0.000000in}{-0.048611in}}{\pgfqpoint{0.000000in}{0.000000in}}{%
\pgfpathmoveto{\pgfqpoint{0.000000in}{0.000000in}}%
\pgfpathlineto{\pgfqpoint{0.000000in}{-0.048611in}}%
\pgfusepath{stroke,fill}%
}%
\begin{pgfscope}%
\pgfsys@transformshift{2.293431in}{0.521603in}%
\pgfsys@useobject{currentmarker}{}%
\end{pgfscope}%
\end{pgfscope}%
\begin{pgfscope}%
\definecolor{textcolor}{rgb}{0.000000,0.000000,0.000000}%
\pgfsetstrokecolor{textcolor}%
\pgfsetfillcolor{textcolor}%
\pgftext[x=2.293431in,y=0.424381in,,top]{\color{textcolor}{\rmfamily\fontsize{10.000000}{12.000000}\selectfont\catcode`\^=\active\def^{\ifmmode\sp\else\^{}\fi}\catcode`\%=\active\def%{\%}$\mathdefault{8}$}}%
\end{pgfscope}%
\begin{pgfscope}%
\pgfsetbuttcap%
\pgfsetroundjoin%
\definecolor{currentfill}{rgb}{0.000000,0.000000,0.000000}%
\pgfsetfillcolor{currentfill}%
\pgfsetlinewidth{0.803000pt}%
\definecolor{currentstroke}{rgb}{0.000000,0.000000,0.000000}%
\pgfsetstrokecolor{currentstroke}%
\pgfsetdash{}{0pt}%
\pgfsys@defobject{currentmarker}{\pgfqpoint{0.000000in}{-0.048611in}}{\pgfqpoint{0.000000in}{0.000000in}}{%
\pgfpathmoveto{\pgfqpoint{0.000000in}{0.000000in}}%
\pgfpathlineto{\pgfqpoint{0.000000in}{-0.048611in}}%
\pgfusepath{stroke,fill}%
}%
\begin{pgfscope}%
\pgfsys@transformshift{3.161684in}{0.521603in}%
\pgfsys@useobject{currentmarker}{}%
\end{pgfscope}%
\end{pgfscope}%
\begin{pgfscope}%
\definecolor{textcolor}{rgb}{0.000000,0.000000,0.000000}%
\pgfsetstrokecolor{textcolor}%
\pgfsetfillcolor{textcolor}%
\pgftext[x=3.161684in,y=0.424381in,,top]{\color{textcolor}{\rmfamily\fontsize{10.000000}{12.000000}\selectfont\catcode`\^=\active\def^{\ifmmode\sp\else\^{}\fi}\catcode`\%=\active\def%{\%}$\mathdefault{12}$}}%
\end{pgfscope}%
\begin{pgfscope}%
\pgfsetbuttcap%
\pgfsetroundjoin%
\definecolor{currentfill}{rgb}{0.000000,0.000000,0.000000}%
\pgfsetfillcolor{currentfill}%
\pgfsetlinewidth{0.803000pt}%
\definecolor{currentstroke}{rgb}{0.000000,0.000000,0.000000}%
\pgfsetstrokecolor{currentstroke}%
\pgfsetdash{}{0pt}%
\pgfsys@defobject{currentmarker}{\pgfqpoint{0.000000in}{-0.048611in}}{\pgfqpoint{0.000000in}{0.000000in}}{%
\pgfpathmoveto{\pgfqpoint{0.000000in}{0.000000in}}%
\pgfpathlineto{\pgfqpoint{0.000000in}{-0.048611in}}%
\pgfusepath{stroke,fill}%
}%
\begin{pgfscope}%
\pgfsys@transformshift{4.029937in}{0.521603in}%
\pgfsys@useobject{currentmarker}{}%
\end{pgfscope}%
\end{pgfscope}%
\begin{pgfscope}%
\definecolor{textcolor}{rgb}{0.000000,0.000000,0.000000}%
\pgfsetstrokecolor{textcolor}%
\pgfsetfillcolor{textcolor}%
\pgftext[x=4.029937in,y=0.424381in,,top]{\color{textcolor}{\rmfamily\fontsize{10.000000}{12.000000}\selectfont\catcode`\^=\active\def^{\ifmmode\sp\else\^{}\fi}\catcode`\%=\active\def%{\%}$\mathdefault{16}$}}%
\end{pgfscope}%
\begin{pgfscope}%
\pgfsetbuttcap%
\pgfsetroundjoin%
\definecolor{currentfill}{rgb}{0.000000,0.000000,0.000000}%
\pgfsetfillcolor{currentfill}%
\pgfsetlinewidth{0.803000pt}%
\definecolor{currentstroke}{rgb}{0.000000,0.000000,0.000000}%
\pgfsetstrokecolor{currentstroke}%
\pgfsetdash{}{0pt}%
\pgfsys@defobject{currentmarker}{\pgfqpoint{0.000000in}{-0.048611in}}{\pgfqpoint{0.000000in}{0.000000in}}{%
\pgfpathmoveto{\pgfqpoint{0.000000in}{0.000000in}}%
\pgfpathlineto{\pgfqpoint{0.000000in}{-0.048611in}}%
\pgfusepath{stroke,fill}%
}%
\begin{pgfscope}%
\pgfsys@transformshift{4.898190in}{0.521603in}%
\pgfsys@useobject{currentmarker}{}%
\end{pgfscope}%
\end{pgfscope}%
\begin{pgfscope}%
\definecolor{textcolor}{rgb}{0.000000,0.000000,0.000000}%
\pgfsetstrokecolor{textcolor}%
\pgfsetfillcolor{textcolor}%
\pgftext[x=4.898190in,y=0.424381in,,top]{\color{textcolor}{\rmfamily\fontsize{10.000000}{12.000000}\selectfont\catcode`\^=\active\def^{\ifmmode\sp\else\^{}\fi}\catcode`\%=\active\def%{\%}$\mathdefault{20}$}}%
\end{pgfscope}%
\begin{pgfscope}%
\pgfsetbuttcap%
\pgfsetroundjoin%
\definecolor{currentfill}{rgb}{0.000000,0.000000,0.000000}%
\pgfsetfillcolor{currentfill}%
\pgfsetlinewidth{0.803000pt}%
\definecolor{currentstroke}{rgb}{0.000000,0.000000,0.000000}%
\pgfsetstrokecolor{currentstroke}%
\pgfsetdash{}{0pt}%
\pgfsys@defobject{currentmarker}{\pgfqpoint{0.000000in}{-0.048611in}}{\pgfqpoint{0.000000in}{0.000000in}}{%
\pgfpathmoveto{\pgfqpoint{0.000000in}{0.000000in}}%
\pgfpathlineto{\pgfqpoint{0.000000in}{-0.048611in}}%
\pgfusepath{stroke,fill}%
}%
\begin{pgfscope}%
\pgfsys@transformshift{5.766443in}{0.521603in}%
\pgfsys@useobject{currentmarker}{}%
\end{pgfscope}%
\end{pgfscope}%
\begin{pgfscope}%
\definecolor{textcolor}{rgb}{0.000000,0.000000,0.000000}%
\pgfsetstrokecolor{textcolor}%
\pgfsetfillcolor{textcolor}%
\pgftext[x=5.766443in,y=0.424381in,,top]{\color{textcolor}{\rmfamily\fontsize{10.000000}{12.000000}\selectfont\catcode`\^=\active\def^{\ifmmode\sp\else\^{}\fi}\catcode`\%=\active\def%{\%}$\mathdefault{24}$}}%
\end{pgfscope}%
\begin{pgfscope}%
\pgfsetbuttcap%
\pgfsetroundjoin%
\definecolor{currentfill}{rgb}{0.000000,0.000000,0.000000}%
\pgfsetfillcolor{currentfill}%
\pgfsetlinewidth{0.803000pt}%
\definecolor{currentstroke}{rgb}{0.000000,0.000000,0.000000}%
\pgfsetstrokecolor{currentstroke}%
\pgfsetdash{}{0pt}%
\pgfsys@defobject{currentmarker}{\pgfqpoint{0.000000in}{-0.048611in}}{\pgfqpoint{0.000000in}{0.000000in}}{%
\pgfpathmoveto{\pgfqpoint{0.000000in}{0.000000in}}%
\pgfpathlineto{\pgfqpoint{0.000000in}{-0.048611in}}%
\pgfusepath{stroke,fill}%
}%
\begin{pgfscope}%
\pgfsys@transformshift{6.634696in}{0.521603in}%
\pgfsys@useobject{currentmarker}{}%
\end{pgfscope}%
\end{pgfscope}%
\begin{pgfscope}%
\definecolor{textcolor}{rgb}{0.000000,0.000000,0.000000}%
\pgfsetstrokecolor{textcolor}%
\pgfsetfillcolor{textcolor}%
\pgftext[x=6.634696in,y=0.424381in,,top]{\color{textcolor}{\rmfamily\fontsize{10.000000}{12.000000}\selectfont\catcode`\^=\active\def^{\ifmmode\sp\else\^{}\fi}\catcode`\%=\active\def%{\%}$\mathdefault{28}$}}%
\end{pgfscope}%
\begin{pgfscope}%
\pgfsetbuttcap%
\pgfsetroundjoin%
\definecolor{currentfill}{rgb}{0.000000,0.000000,0.000000}%
\pgfsetfillcolor{currentfill}%
\pgfsetlinewidth{0.803000pt}%
\definecolor{currentstroke}{rgb}{0.000000,0.000000,0.000000}%
\pgfsetstrokecolor{currentstroke}%
\pgfsetdash{}{0pt}%
\pgfsys@defobject{currentmarker}{\pgfqpoint{0.000000in}{-0.048611in}}{\pgfqpoint{0.000000in}{0.000000in}}{%
\pgfpathmoveto{\pgfqpoint{0.000000in}{0.000000in}}%
\pgfpathlineto{\pgfqpoint{0.000000in}{-0.048611in}}%
\pgfusepath{stroke,fill}%
}%
\begin{pgfscope}%
\pgfsys@transformshift{7.502949in}{0.521603in}%
\pgfsys@useobject{currentmarker}{}%
\end{pgfscope}%
\end{pgfscope}%
\begin{pgfscope}%
\definecolor{textcolor}{rgb}{0.000000,0.000000,0.000000}%
\pgfsetstrokecolor{textcolor}%
\pgfsetfillcolor{textcolor}%
\pgftext[x=7.502949in,y=0.424381in,,top]{\color{textcolor}{\rmfamily\fontsize{10.000000}{12.000000}\selectfont\catcode`\^=\active\def^{\ifmmode\sp\else\^{}\fi}\catcode`\%=\active\def%{\%}$\mathdefault{32}$}}%
\end{pgfscope}%
\begin{pgfscope}%
\definecolor{textcolor}{rgb}{0.000000,0.000000,0.000000}%
\pgfsetstrokecolor{textcolor}%
\pgfsetfillcolor{textcolor}%
\pgftext[x=4.464063in,y=0.234413in,,top]{\color{textcolor}{\rmfamily\fontsize{10.000000}{12.000000}\selectfont\catcode`\^=\active\def^{\ifmmode\sp\else\^{}\fi}\catcode`\%=\active\def%{\%}Monochromatic classes}}%
\end{pgfscope}%
\begin{pgfscope}%
\pgfsetbuttcap%
\pgfsetroundjoin%
\definecolor{currentfill}{rgb}{0.000000,0.000000,0.000000}%
\pgfsetfillcolor{currentfill}%
\pgfsetlinewidth{0.803000pt}%
\definecolor{currentstroke}{rgb}{0.000000,0.000000,0.000000}%
\pgfsetstrokecolor{currentstroke}%
\pgfsetdash{}{0pt}%
\pgfsys@defobject{currentmarker}{\pgfqpoint{-0.048611in}{0.000000in}}{\pgfqpoint{-0.000000in}{0.000000in}}{%
\pgfpathmoveto{\pgfqpoint{-0.000000in}{0.000000in}}%
\pgfpathlineto{\pgfqpoint{-0.048611in}{0.000000in}}%
\pgfusepath{stroke,fill}%
}%
\begin{pgfscope}%
\pgfsys@transformshift{0.643750in}{0.838318in}%
\pgfsys@useobject{currentmarker}{}%
\end{pgfscope}%
\end{pgfscope}%
\begin{pgfscope}%
\definecolor{textcolor}{rgb}{0.000000,0.000000,0.000000}%
\pgfsetstrokecolor{textcolor}%
\pgfsetfillcolor{textcolor}%
\pgftext[x=0.345331in, y=0.785556in, left, base]{\color{textcolor}{\rmfamily\fontsize{10.000000}{12.000000}\selectfont\catcode`\^=\active\def^{\ifmmode\sp\else\^{}\fi}\catcode`\%=\active\def%{\%}$\mathdefault{10^{2}}$}}%
\end{pgfscope}%
\begin{pgfscope}%
\pgfsetbuttcap%
\pgfsetroundjoin%
\definecolor{currentfill}{rgb}{0.000000,0.000000,0.000000}%
\pgfsetfillcolor{currentfill}%
\pgfsetlinewidth{0.803000pt}%
\definecolor{currentstroke}{rgb}{0.000000,0.000000,0.000000}%
\pgfsetstrokecolor{currentstroke}%
\pgfsetdash{}{0pt}%
\pgfsys@defobject{currentmarker}{\pgfqpoint{-0.048611in}{0.000000in}}{\pgfqpoint{-0.000000in}{0.000000in}}{%
\pgfpathmoveto{\pgfqpoint{-0.000000in}{0.000000in}}%
\pgfpathlineto{\pgfqpoint{-0.048611in}{0.000000in}}%
\pgfusepath{stroke,fill}%
}%
\begin{pgfscope}%
\pgfsys@transformshift{0.643750in}{1.219362in}%
\pgfsys@useobject{currentmarker}{}%
\end{pgfscope}%
\end{pgfscope}%
\begin{pgfscope}%
\definecolor{textcolor}{rgb}{0.000000,0.000000,0.000000}%
\pgfsetstrokecolor{textcolor}%
\pgfsetfillcolor{textcolor}%
\pgftext[x=0.345331in, y=1.166601in, left, base]{\color{textcolor}{\rmfamily\fontsize{10.000000}{12.000000}\selectfont\catcode`\^=\active\def^{\ifmmode\sp\else\^{}\fi}\catcode`\%=\active\def%{\%}$\mathdefault{10^{5}}$}}%
\end{pgfscope}%
\begin{pgfscope}%
\pgfsetbuttcap%
\pgfsetroundjoin%
\definecolor{currentfill}{rgb}{0.000000,0.000000,0.000000}%
\pgfsetfillcolor{currentfill}%
\pgfsetlinewidth{0.803000pt}%
\definecolor{currentstroke}{rgb}{0.000000,0.000000,0.000000}%
\pgfsetstrokecolor{currentstroke}%
\pgfsetdash{}{0pt}%
\pgfsys@defobject{currentmarker}{\pgfqpoint{-0.048611in}{0.000000in}}{\pgfqpoint{-0.000000in}{0.000000in}}{%
\pgfpathmoveto{\pgfqpoint{-0.000000in}{0.000000in}}%
\pgfpathlineto{\pgfqpoint{-0.048611in}{0.000000in}}%
\pgfusepath{stroke,fill}%
}%
\begin{pgfscope}%
\pgfsys@transformshift{0.643750in}{1.600407in}%
\pgfsys@useobject{currentmarker}{}%
\end{pgfscope}%
\end{pgfscope}%
\begin{pgfscope}%
\definecolor{textcolor}{rgb}{0.000000,0.000000,0.000000}%
\pgfsetstrokecolor{textcolor}%
\pgfsetfillcolor{textcolor}%
\pgftext[x=0.345331in, y=1.547645in, left, base]{\color{textcolor}{\rmfamily\fontsize{10.000000}{12.000000}\selectfont\catcode`\^=\active\def^{\ifmmode\sp\else\^{}\fi}\catcode`\%=\active\def%{\%}$\mathdefault{10^{8}}$}}%
\end{pgfscope}%
\begin{pgfscope}%
\pgfsetbuttcap%
\pgfsetroundjoin%
\definecolor{currentfill}{rgb}{0.000000,0.000000,0.000000}%
\pgfsetfillcolor{currentfill}%
\pgfsetlinewidth{0.803000pt}%
\definecolor{currentstroke}{rgb}{0.000000,0.000000,0.000000}%
\pgfsetstrokecolor{currentstroke}%
\pgfsetdash{}{0pt}%
\pgfsys@defobject{currentmarker}{\pgfqpoint{-0.048611in}{0.000000in}}{\pgfqpoint{-0.000000in}{0.000000in}}{%
\pgfpathmoveto{\pgfqpoint{-0.000000in}{0.000000in}}%
\pgfpathlineto{\pgfqpoint{-0.048611in}{0.000000in}}%
\pgfusepath{stroke,fill}%
}%
\begin{pgfscope}%
\pgfsys@transformshift{0.643750in}{1.981451in}%
\pgfsys@useobject{currentmarker}{}%
\end{pgfscope}%
\end{pgfscope}%
\begin{pgfscope}%
\definecolor{textcolor}{rgb}{0.000000,0.000000,0.000000}%
\pgfsetstrokecolor{textcolor}%
\pgfsetfillcolor{textcolor}%
\pgftext[x=0.289968in, y=1.928689in, left, base]{\color{textcolor}{\rmfamily\fontsize{10.000000}{12.000000}\selectfont\catcode`\^=\active\def^{\ifmmode\sp\else\^{}\fi}\catcode`\%=\active\def%{\%}$\mathdefault{10^{11}}$}}%
\end{pgfscope}%
\begin{pgfscope}%
\pgfsetbuttcap%
\pgfsetroundjoin%
\definecolor{currentfill}{rgb}{0.000000,0.000000,0.000000}%
\pgfsetfillcolor{currentfill}%
\pgfsetlinewidth{0.803000pt}%
\definecolor{currentstroke}{rgb}{0.000000,0.000000,0.000000}%
\pgfsetstrokecolor{currentstroke}%
\pgfsetdash{}{0pt}%
\pgfsys@defobject{currentmarker}{\pgfqpoint{-0.048611in}{0.000000in}}{\pgfqpoint{-0.000000in}{0.000000in}}{%
\pgfpathmoveto{\pgfqpoint{-0.000000in}{0.000000in}}%
\pgfpathlineto{\pgfqpoint{-0.048611in}{0.000000in}}%
\pgfusepath{stroke,fill}%
}%
\begin{pgfscope}%
\pgfsys@transformshift{0.643750in}{2.362495in}%
\pgfsys@useobject{currentmarker}{}%
\end{pgfscope}%
\end{pgfscope}%
\begin{pgfscope}%
\definecolor{textcolor}{rgb}{0.000000,0.000000,0.000000}%
\pgfsetstrokecolor{textcolor}%
\pgfsetfillcolor{textcolor}%
\pgftext[x=0.289968in, y=2.309734in, left, base]{\color{textcolor}{\rmfamily\fontsize{10.000000}{12.000000}\selectfont\catcode`\^=\active\def^{\ifmmode\sp\else\^{}\fi}\catcode`\%=\active\def%{\%}$\mathdefault{10^{14}}$}}%
\end{pgfscope}%
\begin{pgfscope}%
\definecolor{textcolor}{rgb}{0.000000,0.000000,0.000000}%
\pgfsetstrokecolor{textcolor}%
\pgfsetfillcolor{textcolor}%
\pgftext[x=0.234413in,y=1.631726in,,bottom,rotate=90.000000]{\color{textcolor}{\rmfamily\fontsize{10.000000}{12.000000}\selectfont\catcode`\^=\active\def^{\ifmmode\sp\else\^{}\fi}\catcode`\%=\active\def%{\%}Checks [call]}}%
\end{pgfscope}%
\begin{pgfscope}%
\pgfpathrectangle{\pgfqpoint{0.643750in}{0.521603in}}{\pgfqpoint{7.640626in}{2.220246in}}%
\pgfusepath{clip}%
\pgfsetrectcap%
\pgfsetroundjoin%
\pgfsetlinewidth{1.505625pt}%
\pgfsetstrokecolor{currentstroke1}%
\pgfsetdash{}{0pt}%
\pgfpathmoveto{\pgfqpoint{0.991051in}{2.618289in}}%
\pgfpathlineto{\pgfqpoint{1.208115in}{2.582972in}}%
\pgfpathlineto{\pgfqpoint{1.425178in}{2.640929in}}%
\pgfpathlineto{\pgfqpoint{1.642241in}{2.450208in}}%
\pgfpathlineto{\pgfqpoint{1.859304in}{2.556654in}}%
\pgfpathlineto{\pgfqpoint{2.076368in}{2.569786in}}%
\pgfpathlineto{\pgfqpoint{2.293431in}{2.405350in}}%
\pgfpathlineto{\pgfqpoint{2.510494in}{2.463250in}}%
\pgfpathlineto{\pgfqpoint{2.727557in}{2.353674in}}%
\pgfpathlineto{\pgfqpoint{2.944621in}{2.103172in}}%
\pgfpathlineto{\pgfqpoint{3.161684in}{2.428318in}}%
\pgfpathlineto{\pgfqpoint{3.378747in}{1.966533in}}%
\pgfpathlineto{\pgfqpoint{3.595810in}{2.221514in}}%
\pgfpathlineto{\pgfqpoint{3.812874in}{2.144665in}}%
\pgfpathlineto{\pgfqpoint{4.029937in}{2.262275in}}%
\pgfpathlineto{\pgfqpoint{4.247000in}{2.137084in}}%
\pgfpathlineto{\pgfqpoint{4.464063in}{2.114293in}}%
\pgfpathlineto{\pgfqpoint{4.681127in}{1.815053in}}%
\pgfpathlineto{\pgfqpoint{4.898190in}{1.956060in}}%
\pgfpathlineto{\pgfqpoint{5.115253in}{2.179567in}}%
\pgfpathlineto{\pgfqpoint{5.332316in}{1.986927in}}%
\pgfpathlineto{\pgfqpoint{5.549380in}{2.189365in}}%
\pgfpathlineto{\pgfqpoint{5.766443in}{2.061992in}}%
\pgfpathlineto{\pgfqpoint{5.983506in}{1.890173in}}%
\pgfpathlineto{\pgfqpoint{6.200569in}{2.160996in}}%
\pgfpathlineto{\pgfqpoint{6.417633in}{2.073458in}}%
\pgfpathlineto{\pgfqpoint{6.634696in}{1.724291in}}%
\pgfpathlineto{\pgfqpoint{6.851759in}{1.857906in}}%
\pgfpathlineto{\pgfqpoint{7.068822in}{2.304875in}}%
\pgfpathlineto{\pgfqpoint{7.285886in}{2.203264in}}%
\pgfpathlineto{\pgfqpoint{7.720012in}{1.884287in}}%
\pgfpathlineto{\pgfqpoint{7.937075in}{1.846052in}}%
\pgfusepath{stroke}%
\end{pgfscope}%
\begin{pgfscope}%
\pgfpathrectangle{\pgfqpoint{0.643750in}{0.521603in}}{\pgfqpoint{7.640626in}{2.220246in}}%
\pgfusepath{clip}%
\pgfsetrectcap%
\pgfsetroundjoin%
\pgfsetlinewidth{1.505625pt}%
\pgfsetstrokecolor{currentstroke2}%
\pgfsetdash{}{0pt}%
\pgfpathmoveto{\pgfqpoint{0.991051in}{1.151352in}}%
\pgfpathlineto{\pgfqpoint{1.208115in}{1.225999in}}%
\pgfpathlineto{\pgfqpoint{1.425178in}{1.201836in}}%
\pgfpathlineto{\pgfqpoint{1.642241in}{1.250919in}}%
\pgfpathlineto{\pgfqpoint{1.859304in}{1.228600in}}%
\pgfpathlineto{\pgfqpoint{2.076368in}{1.279886in}}%
\pgfpathlineto{\pgfqpoint{2.293431in}{1.214487in}}%
\pgfpathlineto{\pgfqpoint{2.510494in}{1.287904in}}%
\pgfpathlineto{\pgfqpoint{2.727557in}{1.318506in}}%
\pgfpathlineto{\pgfqpoint{2.944621in}{1.245160in}}%
\pgfpathlineto{\pgfqpoint{3.161684in}{1.331327in}}%
\pgfpathlineto{\pgfqpoint{3.378747in}{1.197408in}}%
\pgfpathlineto{\pgfqpoint{3.595810in}{1.316669in}}%
\pgfpathlineto{\pgfqpoint{3.812874in}{1.440288in}}%
\pgfpathlineto{\pgfqpoint{4.029937in}{1.450348in}}%
\pgfpathlineto{\pgfqpoint{4.247000in}{1.347073in}}%
\pgfpathlineto{\pgfqpoint{4.464063in}{1.357126in}}%
\pgfpathlineto{\pgfqpoint{4.681127in}{1.350706in}}%
\pgfpathlineto{\pgfqpoint{4.898190in}{1.377639in}}%
\pgfpathlineto{\pgfqpoint{5.115253in}{1.431670in}}%
\pgfpathlineto{\pgfqpoint{5.332316in}{1.443325in}}%
\pgfpathlineto{\pgfqpoint{5.549380in}{1.444025in}}%
\pgfpathlineto{\pgfqpoint{5.766443in}{1.519739in}}%
\pgfpathlineto{\pgfqpoint{5.983506in}{1.513442in}}%
\pgfpathlineto{\pgfqpoint{6.200569in}{1.641319in}}%
\pgfpathlineto{\pgfqpoint{6.417633in}{1.602465in}}%
\pgfpathlineto{\pgfqpoint{6.634696in}{1.619914in}}%
\pgfpathlineto{\pgfqpoint{6.851759in}{1.654876in}}%
\pgfpathlineto{\pgfqpoint{7.068822in}{1.846052in}}%
\pgfpathlineto{\pgfqpoint{7.285886in}{1.739458in}}%
\pgfpathlineto{\pgfqpoint{7.720012in}{1.807817in}}%
\pgfpathlineto{\pgfqpoint{7.937075in}{1.846052in}}%
\pgfusepath{stroke}%
\end{pgfscope}%
\begin{pgfscope}%
\pgfpathrectangle{\pgfqpoint{0.643750in}{0.521603in}}{\pgfqpoint{7.640626in}{2.220246in}}%
\pgfusepath{clip}%
\pgfsetrectcap%
\pgfsetroundjoin%
\pgfsetlinewidth{1.505625pt}%
\pgfsetstrokecolor{currentstroke3}%
\pgfsetdash{}{0pt}%
\pgfpathmoveto{\pgfqpoint{0.991051in}{0.622524in}}%
\pgfpathlineto{\pgfqpoint{1.208115in}{0.660759in}}%
\pgfpathlineto{\pgfqpoint{1.425178in}{0.698994in}}%
\pgfpathlineto{\pgfqpoint{1.642241in}{0.737229in}}%
\pgfpathlineto{\pgfqpoint{1.859304in}{0.775465in}}%
\pgfpathlineto{\pgfqpoint{2.076368in}{0.813700in}}%
\pgfpathlineto{\pgfqpoint{2.293431in}{0.851935in}}%
\pgfpathlineto{\pgfqpoint{2.510494in}{0.890170in}}%
\pgfpathlineto{\pgfqpoint{2.727557in}{0.928406in}}%
\pgfpathlineto{\pgfqpoint{2.944621in}{0.966641in}}%
\pgfpathlineto{\pgfqpoint{3.161684in}{1.004876in}}%
\pgfpathlineto{\pgfqpoint{3.378747in}{1.043111in}}%
\pgfpathlineto{\pgfqpoint{3.595810in}{1.081347in}}%
\pgfpathlineto{\pgfqpoint{3.812874in}{1.119582in}}%
\pgfpathlineto{\pgfqpoint{4.029937in}{1.157817in}}%
\pgfpathlineto{\pgfqpoint{4.247000in}{1.196053in}}%
\pgfpathlineto{\pgfqpoint{4.464063in}{1.234288in}}%
\pgfpathlineto{\pgfqpoint{4.681127in}{1.272523in}}%
\pgfpathlineto{\pgfqpoint{4.898190in}{1.310758in}}%
\pgfpathlineto{\pgfqpoint{5.115253in}{1.348994in}}%
\pgfpathlineto{\pgfqpoint{5.332316in}{1.387229in}}%
\pgfpathlineto{\pgfqpoint{5.549380in}{1.425464in}}%
\pgfpathlineto{\pgfqpoint{5.766443in}{1.463699in}}%
\pgfpathlineto{\pgfqpoint{5.983506in}{1.501935in}}%
\pgfpathlineto{\pgfqpoint{6.200569in}{1.540170in}}%
\pgfpathlineto{\pgfqpoint{6.417633in}{1.578405in}}%
\pgfpathlineto{\pgfqpoint{6.634696in}{1.616640in}}%
\pgfpathlineto{\pgfqpoint{6.851759in}{1.654876in}}%
\pgfpathlineto{\pgfqpoint{7.068822in}{1.693111in}}%
\pgfpathlineto{\pgfqpoint{7.285886in}{1.731346in}}%
\pgfpathlineto{\pgfqpoint{7.720012in}{1.807817in}}%
\pgfpathlineto{\pgfqpoint{7.937075in}{1.846052in}}%
\pgfusepath{stroke}%
\end{pgfscope}%
\begin{pgfscope}%
\pgfpathrectangle{\pgfqpoint{0.643750in}{0.521603in}}{\pgfqpoint{7.640626in}{2.220246in}}%
\pgfusepath{clip}%
\pgfsetrectcap%
\pgfsetroundjoin%
\pgfsetlinewidth{1.505625pt}%
\pgfsetstrokecolor{currentstroke4}%
\pgfsetdash{}{0pt}%
\pgfpathmoveto{\pgfqpoint{0.991051in}{0.622524in}}%
\pgfpathlineto{\pgfqpoint{1.208115in}{0.660759in}}%
\pgfpathlineto{\pgfqpoint{1.425178in}{0.698994in}}%
\pgfpathlineto{\pgfqpoint{1.642241in}{0.737229in}}%
\pgfpathlineto{\pgfqpoint{1.859304in}{0.775239in}}%
\pgfpathlineto{\pgfqpoint{2.076368in}{0.813700in}}%
\pgfpathlineto{\pgfqpoint{2.293431in}{0.845197in}}%
\pgfpathlineto{\pgfqpoint{2.510494in}{0.875033in}}%
\pgfpathlineto{\pgfqpoint{2.727557in}{0.910532in}}%
\pgfpathlineto{\pgfqpoint{2.944621in}{0.935624in}}%
\pgfpathlineto{\pgfqpoint{3.161684in}{0.950745in}}%
\pgfpathlineto{\pgfqpoint{3.378747in}{0.972291in}}%
\pgfpathlineto{\pgfqpoint{3.595810in}{0.998560in}}%
\pgfpathlineto{\pgfqpoint{3.812874in}{1.029885in}}%
\pgfpathlineto{\pgfqpoint{4.029937in}{1.052894in}}%
\pgfpathlineto{\pgfqpoint{4.247000in}{1.062798in}}%
\pgfpathlineto{\pgfqpoint{4.464063in}{1.112928in}}%
\pgfpathlineto{\pgfqpoint{4.681127in}{1.099172in}}%
\pgfpathlineto{\pgfqpoint{4.898190in}{1.132661in}}%
\pgfpathlineto{\pgfqpoint{5.115253in}{1.146570in}}%
\pgfpathlineto{\pgfqpoint{5.332316in}{1.166922in}}%
\pgfpathlineto{\pgfqpoint{5.549380in}{1.206202in}}%
\pgfpathlineto{\pgfqpoint{5.766443in}{1.209798in}}%
\pgfpathlineto{\pgfqpoint{5.983506in}{1.216637in}}%
\pgfpathlineto{\pgfqpoint{6.200569in}{1.257050in}}%
\pgfpathlineto{\pgfqpoint{6.417633in}{1.232813in}}%
\pgfpathlineto{\pgfqpoint{6.634696in}{1.328293in}}%
\pgfpathlineto{\pgfqpoint{6.851759in}{1.256856in}}%
\pgfpathlineto{\pgfqpoint{7.068822in}{1.197566in}}%
\pgfpathlineto{\pgfqpoint{7.285886in}{1.315866in}}%
\pgfpathlineto{\pgfqpoint{7.720012in}{1.245938in}}%
\pgfpathlineto{\pgfqpoint{7.937075in}{1.335354in}}%
\pgfusepath{stroke}%
\end{pgfscope}%
\begin{pgfscope}%
\pgfpathrectangle{\pgfqpoint{0.643750in}{0.521603in}}{\pgfqpoint{7.640626in}{2.220246in}}%
\pgfusepath{clip}%
\pgfsetrectcap%
\pgfsetroundjoin%
\pgfsetlinewidth{1.505625pt}%
\pgfsetstrokecolor{currentstroke5}%
\pgfsetdash{}{0pt}%
\pgfpathmoveto{\pgfqpoint{0.991051in}{0.622524in}}%
\pgfpathlineto{\pgfqpoint{1.208115in}{0.653568in}}%
\pgfpathlineto{\pgfqpoint{1.425178in}{0.664465in}}%
\pgfpathlineto{\pgfqpoint{1.642241in}{0.681651in}}%
\pgfpathlineto{\pgfqpoint{1.859304in}{0.700034in}}%
\pgfpathlineto{\pgfqpoint{2.076368in}{0.712586in}}%
\pgfpathlineto{\pgfqpoint{2.293431in}{0.765953in}}%
\pgfpathlineto{\pgfqpoint{2.510494in}{0.779809in}}%
\pgfpathlineto{\pgfqpoint{2.727557in}{0.798189in}}%
\pgfpathlineto{\pgfqpoint{2.944621in}{0.818015in}}%
\pgfpathlineto{\pgfqpoint{3.161684in}{0.846318in}}%
\pgfpathlineto{\pgfqpoint{3.378747in}{0.864005in}}%
\pgfpathlineto{\pgfqpoint{3.595810in}{0.883651in}}%
\pgfpathlineto{\pgfqpoint{3.812874in}{0.897401in}}%
\pgfpathlineto{\pgfqpoint{4.029937in}{0.945573in}}%
\pgfpathlineto{\pgfqpoint{4.247000in}{0.948404in}}%
\pgfpathlineto{\pgfqpoint{4.464063in}{1.003491in}}%
\pgfpathlineto{\pgfqpoint{4.681127in}{0.979329in}}%
\pgfpathlineto{\pgfqpoint{4.898190in}{1.017992in}}%
\pgfpathlineto{\pgfqpoint{5.115253in}{1.028499in}}%
\pgfpathlineto{\pgfqpoint{5.332316in}{1.053422in}}%
\pgfpathlineto{\pgfqpoint{5.549380in}{1.074575in}}%
\pgfpathlineto{\pgfqpoint{5.766443in}{1.098611in}}%
\pgfpathlineto{\pgfqpoint{5.983506in}{1.102044in}}%
\pgfpathlineto{\pgfqpoint{6.200569in}{1.128364in}}%
\pgfpathlineto{\pgfqpoint{6.417633in}{1.120351in}}%
\pgfpathlineto{\pgfqpoint{6.634696in}{1.178232in}}%
\pgfpathlineto{\pgfqpoint{6.851759in}{1.147348in}}%
\pgfpathlineto{\pgfqpoint{7.068822in}{1.023951in}}%
\pgfpathlineto{\pgfqpoint{7.285886in}{1.174764in}}%
\pgfpathlineto{\pgfqpoint{7.720012in}{1.177433in}}%
\pgfpathlineto{\pgfqpoint{7.937075in}{1.189897in}}%
\pgfusepath{stroke}%
\end{pgfscope}%
\begin{pgfscope}%
\pgfsetrectcap%
\pgfsetmiterjoin%
\pgfsetlinewidth{0.803000pt}%
\definecolor{currentstroke}{rgb}{0.000000,0.000000,0.000000}%
\pgfsetstrokecolor{currentstroke}%
\pgfsetdash{}{0pt}%
\pgfpathmoveto{\pgfqpoint{0.643750in}{0.521603in}}%
\pgfpathlineto{\pgfqpoint{0.643750in}{2.741849in}}%
\pgfusepath{stroke}%
\end{pgfscope}%
\begin{pgfscope}%
\pgfsetrectcap%
\pgfsetmiterjoin%
\pgfsetlinewidth{0.803000pt}%
\definecolor{currentstroke}{rgb}{0.000000,0.000000,0.000000}%
\pgfsetstrokecolor{currentstroke}%
\pgfsetdash{}{0pt}%
\pgfpathmoveto{\pgfqpoint{8.284376in}{0.521603in}}%
\pgfpathlineto{\pgfqpoint{8.284376in}{2.741849in}}%
\pgfusepath{stroke}%
\end{pgfscope}%
\begin{pgfscope}%
\pgfsetrectcap%
\pgfsetmiterjoin%
\pgfsetlinewidth{0.803000pt}%
\definecolor{currentstroke}{rgb}{0.000000,0.000000,0.000000}%
\pgfsetstrokecolor{currentstroke}%
\pgfsetdash{}{0pt}%
\pgfpathmoveto{\pgfqpoint{0.643750in}{0.521603in}}%
\pgfpathlineto{\pgfqpoint{8.284376in}{0.521603in}}%
\pgfusepath{stroke}%
\end{pgfscope}%
\begin{pgfscope}%
\pgfsetrectcap%
\pgfsetmiterjoin%
\pgfsetlinewidth{0.803000pt}%
\definecolor{currentstroke}{rgb}{0.000000,0.000000,0.000000}%
\pgfsetstrokecolor{currentstroke}%
\pgfsetdash{}{0pt}%
\pgfpathmoveto{\pgfqpoint{0.643750in}{2.741849in}}%
\pgfpathlineto{\pgfqpoint{8.284376in}{2.741849in}}%
\pgfusepath{stroke}%
\end{pgfscope}%
\begin{pgfscope}%
\pgfsetbuttcap%
\pgfsetmiterjoin%
\definecolor{currentfill}{rgb}{1.000000,1.000000,1.000000}%
\pgfsetfillcolor{currentfill}%
\pgfsetfillopacity{0.800000}%
\pgfsetlinewidth{1.003750pt}%
\definecolor{currentstroke}{rgb}{0.800000,0.800000,0.800000}%
\pgfsetstrokecolor{currentstroke}%
\pgfsetstrokeopacity{0.800000}%
\pgfsetdash{}{0pt}%
\pgfpathmoveto{\pgfqpoint{0.760417in}{1.385372in}}%
\pgfpathlineto{\pgfqpoint{4.097823in}{1.385372in}}%
\pgfpathquadraticcurveto{\pgfqpoint{4.131156in}{1.385372in}}{\pgfqpoint{4.131156in}{1.418705in}}%
\pgfpathlineto{\pgfqpoint{4.131156in}{2.625183in}}%
\pgfpathquadraticcurveto{\pgfqpoint{4.131156in}{2.658516in}}{\pgfqpoint{4.097823in}{2.658516in}}%
\pgfpathlineto{\pgfqpoint{0.760417in}{2.658516in}}%
\pgfpathquadraticcurveto{\pgfqpoint{0.727083in}{2.658516in}}{\pgfqpoint{0.727083in}{2.625183in}}%
\pgfpathlineto{\pgfqpoint{0.727083in}{1.418705in}}%
\pgfpathquadraticcurveto{\pgfqpoint{0.727083in}{1.385372in}}{\pgfqpoint{0.760417in}{1.385372in}}%
\pgfpathlineto{\pgfqpoint{0.760417in}{1.385372in}}%
\pgfpathclose%
\pgfusepath{stroke,fill}%
\end{pgfscope}%
\begin{pgfscope}%
\pgfsetrectcap%
\pgfsetroundjoin%
\pgfsetlinewidth{1.505625pt}%
\pgfsetstrokecolor{currentstroke1}%
\pgfsetdash{}{0pt}%
\pgfpathmoveto{\pgfqpoint{0.793750in}{2.523555in}}%
\pgfpathlineto{\pgfqpoint{0.960417in}{2.523555in}}%
\pgfpathlineto{\pgfqpoint{1.127083in}{2.523555in}}%
\pgfusepath{stroke}%
\end{pgfscope}%
\begin{pgfscope}%
\definecolor{textcolor}{rgb}{0.000000,0.000000,0.000000}%
\pgfsetstrokecolor{textcolor}%
\pgfsetfillcolor{textcolor}%
\pgftext[x=1.260417in,y=2.465222in,left,base]{\color{textcolor}{\rmfamily\fontsize{12.000000}{14.400000}\selectfont\catcode`\^=\active\def^{\ifmmode\sp\else\^{}\fi}\catcode`\%=\active\def%{\%}Naive - Edges}}%
\end{pgfscope}%
\begin{pgfscope}%
\pgfsetrectcap%
\pgfsetroundjoin%
\pgfsetlinewidth{1.505625pt}%
\pgfsetstrokecolor{currentstroke2}%
\pgfsetdash{}{0pt}%
\pgfpathmoveto{\pgfqpoint{0.793750in}{2.278926in}}%
\pgfpathlineto{\pgfqpoint{0.960417in}{2.278926in}}%
\pgfpathlineto{\pgfqpoint{1.127083in}{2.278926in}}%
\pgfusepath{stroke}%
\end{pgfscope}%
\begin{pgfscope}%
\definecolor{textcolor}{rgb}{0.000000,0.000000,0.000000}%
\pgfsetstrokecolor{textcolor}%
\pgfsetfillcolor{textcolor}%
\pgftext[x=1.260417in,y=2.220593in,left,base]{\color{textcolor}{\rmfamily\fontsize{12.000000}{14.400000}\selectfont\catcode`\^=\active\def^{\ifmmode\sp\else\^{}\fi}\catcode`\%=\active\def%{\%}Naive - $\triangle$-connected components}}%
\end{pgfscope}%
\begin{pgfscope}%
\pgfsetrectcap%
\pgfsetroundjoin%
\pgfsetlinewidth{1.505625pt}%
\pgfsetstrokecolor{currentstroke3}%
\pgfsetdash{}{0pt}%
\pgfpathmoveto{\pgfqpoint{0.793750in}{2.034297in}}%
\pgfpathlineto{\pgfqpoint{0.960417in}{2.034297in}}%
\pgfpathlineto{\pgfqpoint{1.127083in}{2.034297in}}%
\pgfusepath{stroke}%
\end{pgfscope}%
\begin{pgfscope}%
\definecolor{textcolor}{rgb}{0.000000,0.000000,0.000000}%
\pgfsetstrokecolor{textcolor}%
\pgfsetfillcolor{textcolor}%
\pgftext[x=1.260417in,y=1.975964in,left,base]{\color{textcolor}{\rmfamily\fontsize{12.000000}{14.400000}\selectfont\catcode`\^=\active\def^{\ifmmode\sp\else\^{}\fi}\catcode`\%=\active\def%{\%}Naive - Monochromatic classes}}%
\end{pgfscope}%
\begin{pgfscope}%
\pgfsetrectcap%
\pgfsetroundjoin%
\pgfsetlinewidth{1.505625pt}%
\pgfsetstrokecolor{currentstroke4}%
\pgfsetdash{}{0pt}%
\pgfpathmoveto{\pgfqpoint{0.793750in}{1.789669in}}%
\pgfpathlineto{\pgfqpoint{0.960417in}{1.789669in}}%
\pgfpathlineto{\pgfqpoint{1.127083in}{1.789669in}}%
\pgfusepath{stroke}%
\end{pgfscope}%
\begin{pgfscope}%
\definecolor{textcolor}{rgb}{0.000000,0.000000,0.000000}%
\pgfsetstrokecolor{textcolor}%
\pgfsetfillcolor{textcolor}%
\pgftext[x=1.260417in,y=1.731335in,left,base]{\color{textcolor}{\rmfamily\fontsize{12.000000}{14.400000}\selectfont\catcode`\^=\active\def^{\ifmmode\sp\else\^{}\fi}\catcode`\%=\active\def%{\%}Subgraphs - \CycleMask{}}}%
\end{pgfscope}%
\begin{pgfscope}%
\pgfsetrectcap%
\pgfsetroundjoin%
\pgfsetlinewidth{1.505625pt}%
\pgfsetstrokecolor{currentstroke5}%
\pgfsetdash{}{0pt}%
\pgfpathmoveto{\pgfqpoint{0.793750in}{1.545040in}}%
\pgfpathlineto{\pgfqpoint{0.960417in}{1.545040in}}%
\pgfpathlineto{\pgfqpoint{1.127083in}{1.545040in}}%
\pgfusepath{stroke}%
\end{pgfscope}%
\begin{pgfscope}%
\definecolor{textcolor}{rgb}{0.000000,0.000000,0.000000}%
\pgfsetstrokecolor{textcolor}%
\pgfsetfillcolor{textcolor}%
\pgftext[x=1.260417in,y=1.486707in,left,base]{\color{textcolor}{\rmfamily\fontsize{12.000000}{14.400000}\selectfont\catcode`\^=\active\def^{\ifmmode\sp\else\^{}\fi}\catcode`\%=\active\def%{\%}Subgraphs - \IsNACColoring{}}}%
\end{pgfscope}%
\end{pgfpicture}%
\makeatother%
\endgroup%
}
	\caption[The number of \IsNACColoring{} calls]{
		The number of \IsNACColoring{} calls with respect to the number of monochromatic classes
		over all graphs used for benchmarking.}%
	\label{fig:graph_summary}
\end{figure}%



\subsection{Performance on specific graph classes}%
\label{sec:bench_graph_classes}

\todo[inline]{Consider Laman deg 3+}
\todo[inline]{Consider line graphs of 3 nor 4 cycles}

Each benchmark was run two or three times and the mean was taken.
The graphs are grouped either by the number of vertices
monochromatic classes or \trcon{} components, see respective \(x\)-axis.
Overall, over 350k configurations (strategies and size of subgraphs combinations)
are presented (and 200k more were run for strategies not worth mentioning)
on over 12.8k graphs from multiple graph classes.
%
We tested only graphs with up to one hundred vertices
as it is computationally hard to find larger graphs in the classes given
and to run enough benchmarks for them
\todo{Uncomment footnote}
% \footnote{
{
	If it takes 250 milliseconds to find a NAC-coloring of a graph,
	in benchmarks, the configuration is run twice over often tens of strategies.
	The time grows quickly over a second per graph.
	If hundreds of graphs are benchmarked,
	the total running time is large.
	See our electric bill.
}.
A NAC-coloring can be found for these graphs in hundreds of milliseconds or seconds at most.

First, we only show strategies that performed well generally,
we show the others later in \Cref{sec:failing_strategies}.
If a configuration did not finish in the given time limit,
we replace the runtime field with the limit of the benchmark, usually 5 seconds.
These runs are excluded from figures with the number of check calls.


\subsubsection*{Minimally rigid graphs}

In the previous section, we showed performance of the algorithm for listing
all NAC-colorings of minimally rigid graphs,
but did not compare the strategies among each other.
%
\Neighbors{} and \NeighborsDegree{} perform slightly better than \None{} and
\SharedVertices{} outperforms \MergeLinear{} slightly.
The same also holds for the number of \IsNACColoring{} checks.
Notice that the runtime grows exponentially, but with lower factor than \NaiveCycles{}.
The growth is expected as the number of NAC-coloring grows fast.

In \Cref{fig:graph_minimally_rigid_first_runtime,fig:graph_minimally_rigid_first_checks}
we focus of finding some NAC-coloring of minimally rigid graphs.
%
Minimally rigid graphs as mostly flexible graphs have
large number of NAC-colorings, therefore it is simple for both \NaiveCycles{}
and \Subgraphs{} algorithms to find some NAC-coloring.
%
It can be seen from the graphs, that for larger graphs, the required runtime
does not grow significantly.
%
Note that minimally rigid graphs have no NAC-coloring if and only if they are formed from
a single \trcon{} component (resp. they are a 2-tree, recall \Cref{lemma:stable_cut_or_2_tree}).
Therefore, such instances do not worsen runtime performance as they are resolved instantly.
%
\NaiveCycles{} is faster as it has lower internal overhead.
The number of \IsNACColoring{} checks is also lower,
that is probably because \Subgraphs{} strategies do additional checks
while merging, which are not needed for \NaiveCycles{}.
\SharedVertices{} behaves unpredictably while \MergeLinear{} is consistent
for both the runtime and the number of \IsNACColoring{} checks.

\begin{figure}[thbp]
	\centering
	\scalebox{\BenchFigureScale}{%% Creator: Matplotlib, PGF backend
%%
%% To include the figure in your LaTeX document, write
%%   \input{<filename>.pgf}
%%
%% Make sure the required packages are loaded in your preamble
%%   \usepackage{pgf}
%%
%% Also ensure that all the required font packages are loaded; for instance,
%% the lmodern package is sometimes necessary when using math font.
%%   \usepackage{lmodern}
%%
%% Figures using additional raster images can only be included by \input if
%% they are in the same directory as the main LaTeX file. For loading figures
%% from other directories you can use the `import` package
%%   \usepackage{import}
%%
%% and then include the figures with
%%   \import{<path to file>}{<filename>.pgf}
%%
%% Matplotlib used the following preamble
%%   \def\mathdefault#1{#1}
%%   \everymath=\expandafter{\the\everymath\displaystyle}
%%   \IfFileExists{scrextend.sty}{
%%     \usepackage[fontsize=10.000000pt]{scrextend}
%%   }{
%%     \renewcommand{\normalsize}{\fontsize{10.000000}{12.000000}\selectfont}
%%     \normalsize
%%   }
%%   
%%   \ifdefined\pdftexversion\else  % non-pdftex case.
%%     \usepackage{fontspec}
%%     \setmainfont{DejaVuSans.ttf}[Path=\detokenize{/home/petr/Projects/PyRigi/.venv/lib/python3.12/site-packages/matplotlib/mpl-data/fonts/ttf/}]
%%     \setsansfont{DejaVuSans.ttf}[Path=\detokenize{/home/petr/Projects/PyRigi/.venv/lib/python3.12/site-packages/matplotlib/mpl-data/fonts/ttf/}]
%%     \setmonofont{DejaVuSansMono.ttf}[Path=\detokenize{/home/petr/Projects/PyRigi/.venv/lib/python3.12/site-packages/matplotlib/mpl-data/fonts/ttf/}]
%%   \fi
%%   \makeatletter\@ifpackageloaded{under\Score{}}{}{\usepackage[strings]{under\Score{}}}\makeatother
%%
\begingroup%
\makeatletter%
\begin{pgfpicture}%
\pgfpathrectangle{\pgfpointorigin}{\pgfqpoint{8.384376in}{2.841849in}}%
\pgfusepath{use as bounding box, clip}%
\begin{pgfscope}%
\pgfsetbuttcap%
\pgfsetmiterjoin%
\definecolor{currentfill}{rgb}{1.000000,1.000000,1.000000}%
\pgfsetfillcolor{currentfill}%
\pgfsetlinewidth{0.000000pt}%
\definecolor{currentstroke}{rgb}{1.000000,1.000000,1.000000}%
\pgfsetstrokecolor{currentstroke}%
\pgfsetdash{}{0pt}%
\pgfpathmoveto{\pgfqpoint{0.000000in}{0.000000in}}%
\pgfpathlineto{\pgfqpoint{8.384376in}{0.000000in}}%
\pgfpathlineto{\pgfqpoint{8.384376in}{2.841849in}}%
\pgfpathlineto{\pgfqpoint{0.000000in}{2.841849in}}%
\pgfpathlineto{\pgfqpoint{0.000000in}{0.000000in}}%
\pgfpathclose%
\pgfusepath{fill}%
\end{pgfscope}%
\begin{pgfscope}%
\pgfsetbuttcap%
\pgfsetmiterjoin%
\definecolor{currentfill}{rgb}{1.000000,1.000000,1.000000}%
\pgfsetfillcolor{currentfill}%
\pgfsetlinewidth{0.000000pt}%
\definecolor{currentstroke}{rgb}{0.000000,0.000000,0.000000}%
\pgfsetstrokecolor{currentstroke}%
\pgfsetstrokeopacity{0.000000}%
\pgfsetdash{}{0pt}%
\pgfpathmoveto{\pgfqpoint{0.588387in}{0.521603in}}%
\pgfpathlineto{\pgfqpoint{4.248423in}{0.521603in}}%
\pgfpathlineto{\pgfqpoint{4.248423in}{2.741849in}}%
\pgfpathlineto{\pgfqpoint{0.588387in}{2.741849in}}%
\pgfpathlineto{\pgfqpoint{0.588387in}{0.521603in}}%
\pgfpathclose%
\pgfusepath{fill}%
\end{pgfscope}%
\begin{pgfscope}%
\pgfsetbuttcap%
\pgfsetroundjoin%
\definecolor{currentfill}{rgb}{0.000000,0.000000,0.000000}%
\pgfsetfillcolor{currentfill}%
\pgfsetlinewidth{0.803000pt}%
\definecolor{currentstroke}{rgb}{0.000000,0.000000,0.000000}%
\pgfsetstrokecolor{currentstroke}%
\pgfsetdash{}{0pt}%
\pgfsys@defobject{currentmarker}{\pgfqpoint{0.000000in}{-0.048611in}}{\pgfqpoint{0.000000in}{0.000000in}}{%
\pgfpathmoveto{\pgfqpoint{0.000000in}{0.000000in}}%
\pgfpathlineto{\pgfqpoint{0.000000in}{-0.048611in}}%
\pgfusepath{stroke,fill}%
}%
\begin{pgfscope}%
\pgfsys@transformshift{0.690145in}{0.521603in}%
\pgfsys@useobject{currentmarker}{}%
\end{pgfscope}%
\end{pgfscope}%
\begin{pgfscope}%
\definecolor{textcolor}{rgb}{0.000000,0.000000,0.000000}%
\pgfsetstrokecolor{textcolor}%
\pgfsetfillcolor{textcolor}%
\pgftext[x=0.690145in,y=0.424381in,,top]{\color{textcolor}{\rmfamily\fontsize{10.000000}{12.000000}\selectfont\catcode`\^=\active\def^{\ifmmode\sp\else\^{}\fi}\catcode`\%=\active\def%{\%}$\mathdefault{0}$}}%
\end{pgfscope}%
\begin{pgfscope}%
\pgfsetbuttcap%
\pgfsetroundjoin%
\definecolor{currentfill}{rgb}{0.000000,0.000000,0.000000}%
\pgfsetfillcolor{currentfill}%
\pgfsetlinewidth{0.803000pt}%
\definecolor{currentstroke}{rgb}{0.000000,0.000000,0.000000}%
\pgfsetstrokecolor{currentstroke}%
\pgfsetdash{}{0pt}%
\pgfsys@defobject{currentmarker}{\pgfqpoint{0.000000in}{-0.048611in}}{\pgfqpoint{0.000000in}{0.000000in}}{%
\pgfpathmoveto{\pgfqpoint{0.000000in}{0.000000in}}%
\pgfpathlineto{\pgfqpoint{0.000000in}{-0.048611in}}%
\pgfusepath{stroke,fill}%
}%
\begin{pgfscope}%
\pgfsys@transformshift{1.174704in}{0.521603in}%
\pgfsys@useobject{currentmarker}{}%
\end{pgfscope}%
\end{pgfscope}%
\begin{pgfscope}%
\definecolor{textcolor}{rgb}{0.000000,0.000000,0.000000}%
\pgfsetstrokecolor{textcolor}%
\pgfsetfillcolor{textcolor}%
\pgftext[x=1.174704in,y=0.424381in,,top]{\color{textcolor}{\rmfamily\fontsize{10.000000}{12.000000}\selectfont\catcode`\^=\active\def^{\ifmmode\sp\else\^{}\fi}\catcode`\%=\active\def%{\%}$\mathdefault{15}$}}%
\end{pgfscope}%
\begin{pgfscope}%
\pgfsetbuttcap%
\pgfsetroundjoin%
\definecolor{currentfill}{rgb}{0.000000,0.000000,0.000000}%
\pgfsetfillcolor{currentfill}%
\pgfsetlinewidth{0.803000pt}%
\definecolor{currentstroke}{rgb}{0.000000,0.000000,0.000000}%
\pgfsetstrokecolor{currentstroke}%
\pgfsetdash{}{0pt}%
\pgfsys@defobject{currentmarker}{\pgfqpoint{0.000000in}{-0.048611in}}{\pgfqpoint{0.000000in}{0.000000in}}{%
\pgfpathmoveto{\pgfqpoint{0.000000in}{0.000000in}}%
\pgfpathlineto{\pgfqpoint{0.000000in}{-0.048611in}}%
\pgfusepath{stroke,fill}%
}%
\begin{pgfscope}%
\pgfsys@transformshift{1.659263in}{0.521603in}%
\pgfsys@useobject{currentmarker}{}%
\end{pgfscope}%
\end{pgfscope}%
\begin{pgfscope}%
\definecolor{textcolor}{rgb}{0.000000,0.000000,0.000000}%
\pgfsetstrokecolor{textcolor}%
\pgfsetfillcolor{textcolor}%
\pgftext[x=1.659263in,y=0.424381in,,top]{\color{textcolor}{\rmfamily\fontsize{10.000000}{12.000000}\selectfont\catcode`\^=\active\def^{\ifmmode\sp\else\^{}\fi}\catcode`\%=\active\def%{\%}$\mathdefault{30}$}}%
\end{pgfscope}%
\begin{pgfscope}%
\pgfsetbuttcap%
\pgfsetroundjoin%
\definecolor{currentfill}{rgb}{0.000000,0.000000,0.000000}%
\pgfsetfillcolor{currentfill}%
\pgfsetlinewidth{0.803000pt}%
\definecolor{currentstroke}{rgb}{0.000000,0.000000,0.000000}%
\pgfsetstrokecolor{currentstroke}%
\pgfsetdash{}{0pt}%
\pgfsys@defobject{currentmarker}{\pgfqpoint{0.000000in}{-0.048611in}}{\pgfqpoint{0.000000in}{0.000000in}}{%
\pgfpathmoveto{\pgfqpoint{0.000000in}{0.000000in}}%
\pgfpathlineto{\pgfqpoint{0.000000in}{-0.048611in}}%
\pgfusepath{stroke,fill}%
}%
\begin{pgfscope}%
\pgfsys@transformshift{2.143822in}{0.521603in}%
\pgfsys@useobject{currentmarker}{}%
\end{pgfscope}%
\end{pgfscope}%
\begin{pgfscope}%
\definecolor{textcolor}{rgb}{0.000000,0.000000,0.000000}%
\pgfsetstrokecolor{textcolor}%
\pgfsetfillcolor{textcolor}%
\pgftext[x=2.143822in,y=0.424381in,,top]{\color{textcolor}{\rmfamily\fontsize{10.000000}{12.000000}\selectfont\catcode`\^=\active\def^{\ifmmode\sp\else\^{}\fi}\catcode`\%=\active\def%{\%}$\mathdefault{45}$}}%
\end{pgfscope}%
\begin{pgfscope}%
\pgfsetbuttcap%
\pgfsetroundjoin%
\definecolor{currentfill}{rgb}{0.000000,0.000000,0.000000}%
\pgfsetfillcolor{currentfill}%
\pgfsetlinewidth{0.803000pt}%
\definecolor{currentstroke}{rgb}{0.000000,0.000000,0.000000}%
\pgfsetstrokecolor{currentstroke}%
\pgfsetdash{}{0pt}%
\pgfsys@defobject{currentmarker}{\pgfqpoint{0.000000in}{-0.048611in}}{\pgfqpoint{0.000000in}{0.000000in}}{%
\pgfpathmoveto{\pgfqpoint{0.000000in}{0.000000in}}%
\pgfpathlineto{\pgfqpoint{0.000000in}{-0.048611in}}%
\pgfusepath{stroke,fill}%
}%
\begin{pgfscope}%
\pgfsys@transformshift{2.628381in}{0.521603in}%
\pgfsys@useobject{currentmarker}{}%
\end{pgfscope}%
\end{pgfscope}%
\begin{pgfscope}%
\definecolor{textcolor}{rgb}{0.000000,0.000000,0.000000}%
\pgfsetstrokecolor{textcolor}%
\pgfsetfillcolor{textcolor}%
\pgftext[x=2.628381in,y=0.424381in,,top]{\color{textcolor}{\rmfamily\fontsize{10.000000}{12.000000}\selectfont\catcode`\^=\active\def^{\ifmmode\sp\else\^{}\fi}\catcode`\%=\active\def%{\%}$\mathdefault{60}$}}%
\end{pgfscope}%
\begin{pgfscope}%
\pgfsetbuttcap%
\pgfsetroundjoin%
\definecolor{currentfill}{rgb}{0.000000,0.000000,0.000000}%
\pgfsetfillcolor{currentfill}%
\pgfsetlinewidth{0.803000pt}%
\definecolor{currentstroke}{rgb}{0.000000,0.000000,0.000000}%
\pgfsetstrokecolor{currentstroke}%
\pgfsetdash{}{0pt}%
\pgfsys@defobject{currentmarker}{\pgfqpoint{0.000000in}{-0.048611in}}{\pgfqpoint{0.000000in}{0.000000in}}{%
\pgfpathmoveto{\pgfqpoint{0.000000in}{0.000000in}}%
\pgfpathlineto{\pgfqpoint{0.000000in}{-0.048611in}}%
\pgfusepath{stroke,fill}%
}%
\begin{pgfscope}%
\pgfsys@transformshift{3.112940in}{0.521603in}%
\pgfsys@useobject{currentmarker}{}%
\end{pgfscope}%
\end{pgfscope}%
\begin{pgfscope}%
\definecolor{textcolor}{rgb}{0.000000,0.000000,0.000000}%
\pgfsetstrokecolor{textcolor}%
\pgfsetfillcolor{textcolor}%
\pgftext[x=3.112940in,y=0.424381in,,top]{\color{textcolor}{\rmfamily\fontsize{10.000000}{12.000000}\selectfont\catcode`\^=\active\def^{\ifmmode\sp\else\^{}\fi}\catcode`\%=\active\def%{\%}$\mathdefault{75}$}}%
\end{pgfscope}%
\begin{pgfscope}%
\pgfsetbuttcap%
\pgfsetroundjoin%
\definecolor{currentfill}{rgb}{0.000000,0.000000,0.000000}%
\pgfsetfillcolor{currentfill}%
\pgfsetlinewidth{0.803000pt}%
\definecolor{currentstroke}{rgb}{0.000000,0.000000,0.000000}%
\pgfsetstrokecolor{currentstroke}%
\pgfsetdash{}{0pt}%
\pgfsys@defobject{currentmarker}{\pgfqpoint{0.000000in}{-0.048611in}}{\pgfqpoint{0.000000in}{0.000000in}}{%
\pgfpathmoveto{\pgfqpoint{0.000000in}{0.000000in}}%
\pgfpathlineto{\pgfqpoint{0.000000in}{-0.048611in}}%
\pgfusepath{stroke,fill}%
}%
\begin{pgfscope}%
\pgfsys@transformshift{3.597498in}{0.521603in}%
\pgfsys@useobject{currentmarker}{}%
\end{pgfscope}%
\end{pgfscope}%
\begin{pgfscope}%
\definecolor{textcolor}{rgb}{0.000000,0.000000,0.000000}%
\pgfsetstrokecolor{textcolor}%
\pgfsetfillcolor{textcolor}%
\pgftext[x=3.597498in,y=0.424381in,,top]{\color{textcolor}{\rmfamily\fontsize{10.000000}{12.000000}\selectfont\catcode`\^=\active\def^{\ifmmode\sp\else\^{}\fi}\catcode`\%=\active\def%{\%}$\mathdefault{90}$}}%
\end{pgfscope}%
\begin{pgfscope}%
\pgfsetbuttcap%
\pgfsetroundjoin%
\definecolor{currentfill}{rgb}{0.000000,0.000000,0.000000}%
\pgfsetfillcolor{currentfill}%
\pgfsetlinewidth{0.803000pt}%
\definecolor{currentstroke}{rgb}{0.000000,0.000000,0.000000}%
\pgfsetstrokecolor{currentstroke}%
\pgfsetdash{}{0pt}%
\pgfsys@defobject{currentmarker}{\pgfqpoint{0.000000in}{-0.048611in}}{\pgfqpoint{0.000000in}{0.000000in}}{%
\pgfpathmoveto{\pgfqpoint{0.000000in}{0.000000in}}%
\pgfpathlineto{\pgfqpoint{0.000000in}{-0.048611in}}%
\pgfusepath{stroke,fill}%
}%
\begin{pgfscope}%
\pgfsys@transformshift{4.082057in}{0.521603in}%
\pgfsys@useobject{currentmarker}{}%
\end{pgfscope}%
\end{pgfscope}%
\begin{pgfscope}%
\definecolor{textcolor}{rgb}{0.000000,0.000000,0.000000}%
\pgfsetstrokecolor{textcolor}%
\pgfsetfillcolor{textcolor}%
\pgftext[x=4.082057in,y=0.424381in,,top]{\color{textcolor}{\rmfamily\fontsize{10.000000}{12.000000}\selectfont\catcode`\^=\active\def^{\ifmmode\sp\else\^{}\fi}\catcode`\%=\active\def%{\%}$\mathdefault{105}$}}%
\end{pgfscope}%
\begin{pgfscope}%
\definecolor{textcolor}{rgb}{0.000000,0.000000,0.000000}%
\pgfsetstrokecolor{textcolor}%
\pgfsetfillcolor{textcolor}%
\pgftext[x=2.418405in,y=0.234413in,,top]{\color{textcolor}{\rmfamily\fontsize{10.000000}{12.000000}\selectfont\catcode`\^=\active\def^{\ifmmode\sp\else\^{}\fi}\catcode`\%=\active\def%{\%}Monochromatic classes}}%
\end{pgfscope}%
\begin{pgfscope}%
\pgfsetbuttcap%
\pgfsetroundjoin%
\definecolor{currentfill}{rgb}{0.000000,0.000000,0.000000}%
\pgfsetfillcolor{currentfill}%
\pgfsetlinewidth{0.803000pt}%
\definecolor{currentstroke}{rgb}{0.000000,0.000000,0.000000}%
\pgfsetstrokecolor{currentstroke}%
\pgfsetdash{}{0pt}%
\pgfsys@defobject{currentmarker}{\pgfqpoint{-0.048611in}{0.000000in}}{\pgfqpoint{-0.000000in}{0.000000in}}{%
\pgfpathmoveto{\pgfqpoint{-0.000000in}{0.000000in}}%
\pgfpathlineto{\pgfqpoint{-0.048611in}{0.000000in}}%
\pgfusepath{stroke,fill}%
}%
\begin{pgfscope}%
\pgfsys@transformshift{0.588387in}{1.046565in}%
\pgfsys@useobject{currentmarker}{}%
\end{pgfscope}%
\end{pgfscope}%
\begin{pgfscope}%
\definecolor{textcolor}{rgb}{0.000000,0.000000,0.000000}%
\pgfsetstrokecolor{textcolor}%
\pgfsetfillcolor{textcolor}%
\pgftext[x=0.289968in, y=0.993803in, left, base]{\color{textcolor}{\rmfamily\fontsize{10.000000}{12.000000}\selectfont\catcode`\^=\active\def^{\ifmmode\sp\else\^{}\fi}\catcode`\%=\active\def%{\%}$\mathdefault{10^{1}}$}}%
\end{pgfscope}%
\begin{pgfscope}%
\pgfsetbuttcap%
\pgfsetroundjoin%
\definecolor{currentfill}{rgb}{0.000000,0.000000,0.000000}%
\pgfsetfillcolor{currentfill}%
\pgfsetlinewidth{0.803000pt}%
\definecolor{currentstroke}{rgb}{0.000000,0.000000,0.000000}%
\pgfsetstrokecolor{currentstroke}%
\pgfsetdash{}{0pt}%
\pgfsys@defobject{currentmarker}{\pgfqpoint{-0.048611in}{0.000000in}}{\pgfqpoint{-0.000000in}{0.000000in}}{%
\pgfpathmoveto{\pgfqpoint{-0.000000in}{0.000000in}}%
\pgfpathlineto{\pgfqpoint{-0.048611in}{0.000000in}}%
\pgfusepath{stroke,fill}%
}%
\begin{pgfscope}%
\pgfsys@transformshift{0.588387in}{1.657004in}%
\pgfsys@useobject{currentmarker}{}%
\end{pgfscope}%
\end{pgfscope}%
\begin{pgfscope}%
\definecolor{textcolor}{rgb}{0.000000,0.000000,0.000000}%
\pgfsetstrokecolor{textcolor}%
\pgfsetfillcolor{textcolor}%
\pgftext[x=0.289968in, y=1.604243in, left, base]{\color{textcolor}{\rmfamily\fontsize{10.000000}{12.000000}\selectfont\catcode`\^=\active\def^{\ifmmode\sp\else\^{}\fi}\catcode`\%=\active\def%{\%}$\mathdefault{10^{2}}$}}%
\end{pgfscope}%
\begin{pgfscope}%
\pgfsetbuttcap%
\pgfsetroundjoin%
\definecolor{currentfill}{rgb}{0.000000,0.000000,0.000000}%
\pgfsetfillcolor{currentfill}%
\pgfsetlinewidth{0.803000pt}%
\definecolor{currentstroke}{rgb}{0.000000,0.000000,0.000000}%
\pgfsetstrokecolor{currentstroke}%
\pgfsetdash{}{0pt}%
\pgfsys@defobject{currentmarker}{\pgfqpoint{-0.048611in}{0.000000in}}{\pgfqpoint{-0.000000in}{0.000000in}}{%
\pgfpathmoveto{\pgfqpoint{-0.000000in}{0.000000in}}%
\pgfpathlineto{\pgfqpoint{-0.048611in}{0.000000in}}%
\pgfusepath{stroke,fill}%
}%
\begin{pgfscope}%
\pgfsys@transformshift{0.588387in}{2.267444in}%
\pgfsys@useobject{currentmarker}{}%
\end{pgfscope}%
\end{pgfscope}%
\begin{pgfscope}%
\definecolor{textcolor}{rgb}{0.000000,0.000000,0.000000}%
\pgfsetstrokecolor{textcolor}%
\pgfsetfillcolor{textcolor}%
\pgftext[x=0.289968in, y=2.214682in, left, base]{\color{textcolor}{\rmfamily\fontsize{10.000000}{12.000000}\selectfont\catcode`\^=\active\def^{\ifmmode\sp\else\^{}\fi}\catcode`\%=\active\def%{\%}$\mathdefault{10^{3}}$}}%
\end{pgfscope}%
\begin{pgfscope}%
\pgfsetbuttcap%
\pgfsetroundjoin%
\definecolor{currentfill}{rgb}{0.000000,0.000000,0.000000}%
\pgfsetfillcolor{currentfill}%
\pgfsetlinewidth{0.602250pt}%
\definecolor{currentstroke}{rgb}{0.000000,0.000000,0.000000}%
\pgfsetstrokecolor{currentstroke}%
\pgfsetdash{}{0pt}%
\pgfsys@defobject{currentmarker}{\pgfqpoint{-0.027778in}{0.000000in}}{\pgfqpoint{-0.000000in}{0.000000in}}{%
\pgfpathmoveto{\pgfqpoint{-0.000000in}{0.000000in}}%
\pgfpathlineto{\pgfqpoint{-0.027778in}{0.000000in}}%
\pgfusepath{stroke,fill}%
}%
\begin{pgfscope}%
\pgfsys@transformshift{0.588387in}{0.619886in}%
\pgfsys@useobject{currentmarker}{}%
\end{pgfscope}%
\end{pgfscope}%
\begin{pgfscope}%
\pgfsetbuttcap%
\pgfsetroundjoin%
\definecolor{currentfill}{rgb}{0.000000,0.000000,0.000000}%
\pgfsetfillcolor{currentfill}%
\pgfsetlinewidth{0.602250pt}%
\definecolor{currentstroke}{rgb}{0.000000,0.000000,0.000000}%
\pgfsetstrokecolor{currentstroke}%
\pgfsetdash{}{0pt}%
\pgfsys@defobject{currentmarker}{\pgfqpoint{-0.027778in}{0.000000in}}{\pgfqpoint{-0.000000in}{0.000000in}}{%
\pgfpathmoveto{\pgfqpoint{-0.000000in}{0.000000in}}%
\pgfpathlineto{\pgfqpoint{-0.027778in}{0.000000in}}%
\pgfusepath{stroke,fill}%
}%
\begin{pgfscope}%
\pgfsys@transformshift{0.588387in}{0.727379in}%
\pgfsys@useobject{currentmarker}{}%
\end{pgfscope}%
\end{pgfscope}%
\begin{pgfscope}%
\pgfsetbuttcap%
\pgfsetroundjoin%
\definecolor{currentfill}{rgb}{0.000000,0.000000,0.000000}%
\pgfsetfillcolor{currentfill}%
\pgfsetlinewidth{0.602250pt}%
\definecolor{currentstroke}{rgb}{0.000000,0.000000,0.000000}%
\pgfsetstrokecolor{currentstroke}%
\pgfsetdash{}{0pt}%
\pgfsys@defobject{currentmarker}{\pgfqpoint{-0.027778in}{0.000000in}}{\pgfqpoint{-0.000000in}{0.000000in}}{%
\pgfpathmoveto{\pgfqpoint{-0.000000in}{0.000000in}}%
\pgfpathlineto{\pgfqpoint{-0.027778in}{0.000000in}}%
\pgfusepath{stroke,fill}%
}%
\begin{pgfscope}%
\pgfsys@transformshift{0.588387in}{0.803646in}%
\pgfsys@useobject{currentmarker}{}%
\end{pgfscope}%
\end{pgfscope}%
\begin{pgfscope}%
\pgfsetbuttcap%
\pgfsetroundjoin%
\definecolor{currentfill}{rgb}{0.000000,0.000000,0.000000}%
\pgfsetfillcolor{currentfill}%
\pgfsetlinewidth{0.602250pt}%
\definecolor{currentstroke}{rgb}{0.000000,0.000000,0.000000}%
\pgfsetstrokecolor{currentstroke}%
\pgfsetdash{}{0pt}%
\pgfsys@defobject{currentmarker}{\pgfqpoint{-0.027778in}{0.000000in}}{\pgfqpoint{-0.000000in}{0.000000in}}{%
\pgfpathmoveto{\pgfqpoint{-0.000000in}{0.000000in}}%
\pgfpathlineto{\pgfqpoint{-0.027778in}{0.000000in}}%
\pgfusepath{stroke,fill}%
}%
\begin{pgfscope}%
\pgfsys@transformshift{0.588387in}{0.862804in}%
\pgfsys@useobject{currentmarker}{}%
\end{pgfscope}%
\end{pgfscope}%
\begin{pgfscope}%
\pgfsetbuttcap%
\pgfsetroundjoin%
\definecolor{currentfill}{rgb}{0.000000,0.000000,0.000000}%
\pgfsetfillcolor{currentfill}%
\pgfsetlinewidth{0.602250pt}%
\definecolor{currentstroke}{rgb}{0.000000,0.000000,0.000000}%
\pgfsetstrokecolor{currentstroke}%
\pgfsetdash{}{0pt}%
\pgfsys@defobject{currentmarker}{\pgfqpoint{-0.027778in}{0.000000in}}{\pgfqpoint{-0.000000in}{0.000000in}}{%
\pgfpathmoveto{\pgfqpoint{-0.000000in}{0.000000in}}%
\pgfpathlineto{\pgfqpoint{-0.027778in}{0.000000in}}%
\pgfusepath{stroke,fill}%
}%
\begin{pgfscope}%
\pgfsys@transformshift{0.588387in}{0.911139in}%
\pgfsys@useobject{currentmarker}{}%
\end{pgfscope}%
\end{pgfscope}%
\begin{pgfscope}%
\pgfsetbuttcap%
\pgfsetroundjoin%
\definecolor{currentfill}{rgb}{0.000000,0.000000,0.000000}%
\pgfsetfillcolor{currentfill}%
\pgfsetlinewidth{0.602250pt}%
\definecolor{currentstroke}{rgb}{0.000000,0.000000,0.000000}%
\pgfsetstrokecolor{currentstroke}%
\pgfsetdash{}{0pt}%
\pgfsys@defobject{currentmarker}{\pgfqpoint{-0.027778in}{0.000000in}}{\pgfqpoint{-0.000000in}{0.000000in}}{%
\pgfpathmoveto{\pgfqpoint{-0.000000in}{0.000000in}}%
\pgfpathlineto{\pgfqpoint{-0.027778in}{0.000000in}}%
\pgfusepath{stroke,fill}%
}%
\begin{pgfscope}%
\pgfsys@transformshift{0.588387in}{0.952006in}%
\pgfsys@useobject{currentmarker}{}%
\end{pgfscope}%
\end{pgfscope}%
\begin{pgfscope}%
\pgfsetbuttcap%
\pgfsetroundjoin%
\definecolor{currentfill}{rgb}{0.000000,0.000000,0.000000}%
\pgfsetfillcolor{currentfill}%
\pgfsetlinewidth{0.602250pt}%
\definecolor{currentstroke}{rgb}{0.000000,0.000000,0.000000}%
\pgfsetstrokecolor{currentstroke}%
\pgfsetdash{}{0pt}%
\pgfsys@defobject{currentmarker}{\pgfqpoint{-0.027778in}{0.000000in}}{\pgfqpoint{-0.000000in}{0.000000in}}{%
\pgfpathmoveto{\pgfqpoint{-0.000000in}{0.000000in}}%
\pgfpathlineto{\pgfqpoint{-0.027778in}{0.000000in}}%
\pgfusepath{stroke,fill}%
}%
\begin{pgfscope}%
\pgfsys@transformshift{0.588387in}{0.987407in}%
\pgfsys@useobject{currentmarker}{}%
\end{pgfscope}%
\end{pgfscope}%
\begin{pgfscope}%
\pgfsetbuttcap%
\pgfsetroundjoin%
\definecolor{currentfill}{rgb}{0.000000,0.000000,0.000000}%
\pgfsetfillcolor{currentfill}%
\pgfsetlinewidth{0.602250pt}%
\definecolor{currentstroke}{rgb}{0.000000,0.000000,0.000000}%
\pgfsetstrokecolor{currentstroke}%
\pgfsetdash{}{0pt}%
\pgfsys@defobject{currentmarker}{\pgfqpoint{-0.027778in}{0.000000in}}{\pgfqpoint{-0.000000in}{0.000000in}}{%
\pgfpathmoveto{\pgfqpoint{-0.000000in}{0.000000in}}%
\pgfpathlineto{\pgfqpoint{-0.027778in}{0.000000in}}%
\pgfusepath{stroke,fill}%
}%
\begin{pgfscope}%
\pgfsys@transformshift{0.588387in}{1.018632in}%
\pgfsys@useobject{currentmarker}{}%
\end{pgfscope}%
\end{pgfscope}%
\begin{pgfscope}%
\pgfsetbuttcap%
\pgfsetroundjoin%
\definecolor{currentfill}{rgb}{0.000000,0.000000,0.000000}%
\pgfsetfillcolor{currentfill}%
\pgfsetlinewidth{0.602250pt}%
\definecolor{currentstroke}{rgb}{0.000000,0.000000,0.000000}%
\pgfsetstrokecolor{currentstroke}%
\pgfsetdash{}{0pt}%
\pgfsys@defobject{currentmarker}{\pgfqpoint{-0.027778in}{0.000000in}}{\pgfqpoint{-0.000000in}{0.000000in}}{%
\pgfpathmoveto{\pgfqpoint{-0.000000in}{0.000000in}}%
\pgfpathlineto{\pgfqpoint{-0.027778in}{0.000000in}}%
\pgfusepath{stroke,fill}%
}%
\begin{pgfscope}%
\pgfsys@transformshift{0.588387in}{1.230325in}%
\pgfsys@useobject{currentmarker}{}%
\end{pgfscope}%
\end{pgfscope}%
\begin{pgfscope}%
\pgfsetbuttcap%
\pgfsetroundjoin%
\definecolor{currentfill}{rgb}{0.000000,0.000000,0.000000}%
\pgfsetfillcolor{currentfill}%
\pgfsetlinewidth{0.602250pt}%
\definecolor{currentstroke}{rgb}{0.000000,0.000000,0.000000}%
\pgfsetstrokecolor{currentstroke}%
\pgfsetdash{}{0pt}%
\pgfsys@defobject{currentmarker}{\pgfqpoint{-0.027778in}{0.000000in}}{\pgfqpoint{-0.000000in}{0.000000in}}{%
\pgfpathmoveto{\pgfqpoint{-0.000000in}{0.000000in}}%
\pgfpathlineto{\pgfqpoint{-0.027778in}{0.000000in}}%
\pgfusepath{stroke,fill}%
}%
\begin{pgfscope}%
\pgfsys@transformshift{0.588387in}{1.337818in}%
\pgfsys@useobject{currentmarker}{}%
\end{pgfscope}%
\end{pgfscope}%
\begin{pgfscope}%
\pgfsetbuttcap%
\pgfsetroundjoin%
\definecolor{currentfill}{rgb}{0.000000,0.000000,0.000000}%
\pgfsetfillcolor{currentfill}%
\pgfsetlinewidth{0.602250pt}%
\definecolor{currentstroke}{rgb}{0.000000,0.000000,0.000000}%
\pgfsetstrokecolor{currentstroke}%
\pgfsetdash{}{0pt}%
\pgfsys@defobject{currentmarker}{\pgfqpoint{-0.027778in}{0.000000in}}{\pgfqpoint{-0.000000in}{0.000000in}}{%
\pgfpathmoveto{\pgfqpoint{-0.000000in}{0.000000in}}%
\pgfpathlineto{\pgfqpoint{-0.027778in}{0.000000in}}%
\pgfusepath{stroke,fill}%
}%
\begin{pgfscope}%
\pgfsys@transformshift{0.588387in}{1.414086in}%
\pgfsys@useobject{currentmarker}{}%
\end{pgfscope}%
\end{pgfscope}%
\begin{pgfscope}%
\pgfsetbuttcap%
\pgfsetroundjoin%
\definecolor{currentfill}{rgb}{0.000000,0.000000,0.000000}%
\pgfsetfillcolor{currentfill}%
\pgfsetlinewidth{0.602250pt}%
\definecolor{currentstroke}{rgb}{0.000000,0.000000,0.000000}%
\pgfsetstrokecolor{currentstroke}%
\pgfsetdash{}{0pt}%
\pgfsys@defobject{currentmarker}{\pgfqpoint{-0.027778in}{0.000000in}}{\pgfqpoint{-0.000000in}{0.000000in}}{%
\pgfpathmoveto{\pgfqpoint{-0.000000in}{0.000000in}}%
\pgfpathlineto{\pgfqpoint{-0.027778in}{0.000000in}}%
\pgfusepath{stroke,fill}%
}%
\begin{pgfscope}%
\pgfsys@transformshift{0.588387in}{1.473244in}%
\pgfsys@useobject{currentmarker}{}%
\end{pgfscope}%
\end{pgfscope}%
\begin{pgfscope}%
\pgfsetbuttcap%
\pgfsetroundjoin%
\definecolor{currentfill}{rgb}{0.000000,0.000000,0.000000}%
\pgfsetfillcolor{currentfill}%
\pgfsetlinewidth{0.602250pt}%
\definecolor{currentstroke}{rgb}{0.000000,0.000000,0.000000}%
\pgfsetstrokecolor{currentstroke}%
\pgfsetdash{}{0pt}%
\pgfsys@defobject{currentmarker}{\pgfqpoint{-0.027778in}{0.000000in}}{\pgfqpoint{-0.000000in}{0.000000in}}{%
\pgfpathmoveto{\pgfqpoint{-0.000000in}{0.000000in}}%
\pgfpathlineto{\pgfqpoint{-0.027778in}{0.000000in}}%
\pgfusepath{stroke,fill}%
}%
\begin{pgfscope}%
\pgfsys@transformshift{0.588387in}{1.521579in}%
\pgfsys@useobject{currentmarker}{}%
\end{pgfscope}%
\end{pgfscope}%
\begin{pgfscope}%
\pgfsetbuttcap%
\pgfsetroundjoin%
\definecolor{currentfill}{rgb}{0.000000,0.000000,0.000000}%
\pgfsetfillcolor{currentfill}%
\pgfsetlinewidth{0.602250pt}%
\definecolor{currentstroke}{rgb}{0.000000,0.000000,0.000000}%
\pgfsetstrokecolor{currentstroke}%
\pgfsetdash{}{0pt}%
\pgfsys@defobject{currentmarker}{\pgfqpoint{-0.027778in}{0.000000in}}{\pgfqpoint{-0.000000in}{0.000000in}}{%
\pgfpathmoveto{\pgfqpoint{-0.000000in}{0.000000in}}%
\pgfpathlineto{\pgfqpoint{-0.027778in}{0.000000in}}%
\pgfusepath{stroke,fill}%
}%
\begin{pgfscope}%
\pgfsys@transformshift{0.588387in}{1.562446in}%
\pgfsys@useobject{currentmarker}{}%
\end{pgfscope}%
\end{pgfscope}%
\begin{pgfscope}%
\pgfsetbuttcap%
\pgfsetroundjoin%
\definecolor{currentfill}{rgb}{0.000000,0.000000,0.000000}%
\pgfsetfillcolor{currentfill}%
\pgfsetlinewidth{0.602250pt}%
\definecolor{currentstroke}{rgb}{0.000000,0.000000,0.000000}%
\pgfsetstrokecolor{currentstroke}%
\pgfsetdash{}{0pt}%
\pgfsys@defobject{currentmarker}{\pgfqpoint{-0.027778in}{0.000000in}}{\pgfqpoint{-0.000000in}{0.000000in}}{%
\pgfpathmoveto{\pgfqpoint{-0.000000in}{0.000000in}}%
\pgfpathlineto{\pgfqpoint{-0.027778in}{0.000000in}}%
\pgfusepath{stroke,fill}%
}%
\begin{pgfscope}%
\pgfsys@transformshift{0.588387in}{1.597847in}%
\pgfsys@useobject{currentmarker}{}%
\end{pgfscope}%
\end{pgfscope}%
\begin{pgfscope}%
\pgfsetbuttcap%
\pgfsetroundjoin%
\definecolor{currentfill}{rgb}{0.000000,0.000000,0.000000}%
\pgfsetfillcolor{currentfill}%
\pgfsetlinewidth{0.602250pt}%
\definecolor{currentstroke}{rgb}{0.000000,0.000000,0.000000}%
\pgfsetstrokecolor{currentstroke}%
\pgfsetdash{}{0pt}%
\pgfsys@defobject{currentmarker}{\pgfqpoint{-0.027778in}{0.000000in}}{\pgfqpoint{-0.000000in}{0.000000in}}{%
\pgfpathmoveto{\pgfqpoint{-0.000000in}{0.000000in}}%
\pgfpathlineto{\pgfqpoint{-0.027778in}{0.000000in}}%
\pgfusepath{stroke,fill}%
}%
\begin{pgfscope}%
\pgfsys@transformshift{0.588387in}{1.629072in}%
\pgfsys@useobject{currentmarker}{}%
\end{pgfscope}%
\end{pgfscope}%
\begin{pgfscope}%
\pgfsetbuttcap%
\pgfsetroundjoin%
\definecolor{currentfill}{rgb}{0.000000,0.000000,0.000000}%
\pgfsetfillcolor{currentfill}%
\pgfsetlinewidth{0.602250pt}%
\definecolor{currentstroke}{rgb}{0.000000,0.000000,0.000000}%
\pgfsetstrokecolor{currentstroke}%
\pgfsetdash{}{0pt}%
\pgfsys@defobject{currentmarker}{\pgfqpoint{-0.027778in}{0.000000in}}{\pgfqpoint{-0.000000in}{0.000000in}}{%
\pgfpathmoveto{\pgfqpoint{-0.000000in}{0.000000in}}%
\pgfpathlineto{\pgfqpoint{-0.027778in}{0.000000in}}%
\pgfusepath{stroke,fill}%
}%
\begin{pgfscope}%
\pgfsys@transformshift{0.588387in}{1.840765in}%
\pgfsys@useobject{currentmarker}{}%
\end{pgfscope}%
\end{pgfscope}%
\begin{pgfscope}%
\pgfsetbuttcap%
\pgfsetroundjoin%
\definecolor{currentfill}{rgb}{0.000000,0.000000,0.000000}%
\pgfsetfillcolor{currentfill}%
\pgfsetlinewidth{0.602250pt}%
\definecolor{currentstroke}{rgb}{0.000000,0.000000,0.000000}%
\pgfsetstrokecolor{currentstroke}%
\pgfsetdash{}{0pt}%
\pgfsys@defobject{currentmarker}{\pgfqpoint{-0.027778in}{0.000000in}}{\pgfqpoint{-0.000000in}{0.000000in}}{%
\pgfpathmoveto{\pgfqpoint{-0.000000in}{0.000000in}}%
\pgfpathlineto{\pgfqpoint{-0.027778in}{0.000000in}}%
\pgfusepath{stroke,fill}%
}%
\begin{pgfscope}%
\pgfsys@transformshift{0.588387in}{1.948258in}%
\pgfsys@useobject{currentmarker}{}%
\end{pgfscope}%
\end{pgfscope}%
\begin{pgfscope}%
\pgfsetbuttcap%
\pgfsetroundjoin%
\definecolor{currentfill}{rgb}{0.000000,0.000000,0.000000}%
\pgfsetfillcolor{currentfill}%
\pgfsetlinewidth{0.602250pt}%
\definecolor{currentstroke}{rgb}{0.000000,0.000000,0.000000}%
\pgfsetstrokecolor{currentstroke}%
\pgfsetdash{}{0pt}%
\pgfsys@defobject{currentmarker}{\pgfqpoint{-0.027778in}{0.000000in}}{\pgfqpoint{-0.000000in}{0.000000in}}{%
\pgfpathmoveto{\pgfqpoint{-0.000000in}{0.000000in}}%
\pgfpathlineto{\pgfqpoint{-0.027778in}{0.000000in}}%
\pgfusepath{stroke,fill}%
}%
\begin{pgfscope}%
\pgfsys@transformshift{0.588387in}{2.024526in}%
\pgfsys@useobject{currentmarker}{}%
\end{pgfscope}%
\end{pgfscope}%
\begin{pgfscope}%
\pgfsetbuttcap%
\pgfsetroundjoin%
\definecolor{currentfill}{rgb}{0.000000,0.000000,0.000000}%
\pgfsetfillcolor{currentfill}%
\pgfsetlinewidth{0.602250pt}%
\definecolor{currentstroke}{rgb}{0.000000,0.000000,0.000000}%
\pgfsetstrokecolor{currentstroke}%
\pgfsetdash{}{0pt}%
\pgfsys@defobject{currentmarker}{\pgfqpoint{-0.027778in}{0.000000in}}{\pgfqpoint{-0.000000in}{0.000000in}}{%
\pgfpathmoveto{\pgfqpoint{-0.000000in}{0.000000in}}%
\pgfpathlineto{\pgfqpoint{-0.027778in}{0.000000in}}%
\pgfusepath{stroke,fill}%
}%
\begin{pgfscope}%
\pgfsys@transformshift{0.588387in}{2.083683in}%
\pgfsys@useobject{currentmarker}{}%
\end{pgfscope}%
\end{pgfscope}%
\begin{pgfscope}%
\pgfsetbuttcap%
\pgfsetroundjoin%
\definecolor{currentfill}{rgb}{0.000000,0.000000,0.000000}%
\pgfsetfillcolor{currentfill}%
\pgfsetlinewidth{0.602250pt}%
\definecolor{currentstroke}{rgb}{0.000000,0.000000,0.000000}%
\pgfsetstrokecolor{currentstroke}%
\pgfsetdash{}{0pt}%
\pgfsys@defobject{currentmarker}{\pgfqpoint{-0.027778in}{0.000000in}}{\pgfqpoint{-0.000000in}{0.000000in}}{%
\pgfpathmoveto{\pgfqpoint{-0.000000in}{0.000000in}}%
\pgfpathlineto{\pgfqpoint{-0.027778in}{0.000000in}}%
\pgfusepath{stroke,fill}%
}%
\begin{pgfscope}%
\pgfsys@transformshift{0.588387in}{2.132019in}%
\pgfsys@useobject{currentmarker}{}%
\end{pgfscope}%
\end{pgfscope}%
\begin{pgfscope}%
\pgfsetbuttcap%
\pgfsetroundjoin%
\definecolor{currentfill}{rgb}{0.000000,0.000000,0.000000}%
\pgfsetfillcolor{currentfill}%
\pgfsetlinewidth{0.602250pt}%
\definecolor{currentstroke}{rgb}{0.000000,0.000000,0.000000}%
\pgfsetstrokecolor{currentstroke}%
\pgfsetdash{}{0pt}%
\pgfsys@defobject{currentmarker}{\pgfqpoint{-0.027778in}{0.000000in}}{\pgfqpoint{-0.000000in}{0.000000in}}{%
\pgfpathmoveto{\pgfqpoint{-0.000000in}{0.000000in}}%
\pgfpathlineto{\pgfqpoint{-0.027778in}{0.000000in}}%
\pgfusepath{stroke,fill}%
}%
\begin{pgfscope}%
\pgfsys@transformshift{0.588387in}{2.172886in}%
\pgfsys@useobject{currentmarker}{}%
\end{pgfscope}%
\end{pgfscope}%
\begin{pgfscope}%
\pgfsetbuttcap%
\pgfsetroundjoin%
\definecolor{currentfill}{rgb}{0.000000,0.000000,0.000000}%
\pgfsetfillcolor{currentfill}%
\pgfsetlinewidth{0.602250pt}%
\definecolor{currentstroke}{rgb}{0.000000,0.000000,0.000000}%
\pgfsetstrokecolor{currentstroke}%
\pgfsetdash{}{0pt}%
\pgfsys@defobject{currentmarker}{\pgfqpoint{-0.027778in}{0.000000in}}{\pgfqpoint{-0.000000in}{0.000000in}}{%
\pgfpathmoveto{\pgfqpoint{-0.000000in}{0.000000in}}%
\pgfpathlineto{\pgfqpoint{-0.027778in}{0.000000in}}%
\pgfusepath{stroke,fill}%
}%
\begin{pgfscope}%
\pgfsys@transformshift{0.588387in}{2.208286in}%
\pgfsys@useobject{currentmarker}{}%
\end{pgfscope}%
\end{pgfscope}%
\begin{pgfscope}%
\pgfsetbuttcap%
\pgfsetroundjoin%
\definecolor{currentfill}{rgb}{0.000000,0.000000,0.000000}%
\pgfsetfillcolor{currentfill}%
\pgfsetlinewidth{0.602250pt}%
\definecolor{currentstroke}{rgb}{0.000000,0.000000,0.000000}%
\pgfsetstrokecolor{currentstroke}%
\pgfsetdash{}{0pt}%
\pgfsys@defobject{currentmarker}{\pgfqpoint{-0.027778in}{0.000000in}}{\pgfqpoint{-0.000000in}{0.000000in}}{%
\pgfpathmoveto{\pgfqpoint{-0.000000in}{0.000000in}}%
\pgfpathlineto{\pgfqpoint{-0.027778in}{0.000000in}}%
\pgfusepath{stroke,fill}%
}%
\begin{pgfscope}%
\pgfsys@transformshift{0.588387in}{2.239512in}%
\pgfsys@useobject{currentmarker}{}%
\end{pgfscope}%
\end{pgfscope}%
\begin{pgfscope}%
\pgfsetbuttcap%
\pgfsetroundjoin%
\definecolor{currentfill}{rgb}{0.000000,0.000000,0.000000}%
\pgfsetfillcolor{currentfill}%
\pgfsetlinewidth{0.602250pt}%
\definecolor{currentstroke}{rgb}{0.000000,0.000000,0.000000}%
\pgfsetstrokecolor{currentstroke}%
\pgfsetdash{}{0pt}%
\pgfsys@defobject{currentmarker}{\pgfqpoint{-0.027778in}{0.000000in}}{\pgfqpoint{-0.000000in}{0.000000in}}{%
\pgfpathmoveto{\pgfqpoint{-0.000000in}{0.000000in}}%
\pgfpathlineto{\pgfqpoint{-0.027778in}{0.000000in}}%
\pgfusepath{stroke,fill}%
}%
\begin{pgfscope}%
\pgfsys@transformshift{0.588387in}{2.451205in}%
\pgfsys@useobject{currentmarker}{}%
\end{pgfscope}%
\end{pgfscope}%
\begin{pgfscope}%
\pgfsetbuttcap%
\pgfsetroundjoin%
\definecolor{currentfill}{rgb}{0.000000,0.000000,0.000000}%
\pgfsetfillcolor{currentfill}%
\pgfsetlinewidth{0.602250pt}%
\definecolor{currentstroke}{rgb}{0.000000,0.000000,0.000000}%
\pgfsetstrokecolor{currentstroke}%
\pgfsetdash{}{0pt}%
\pgfsys@defobject{currentmarker}{\pgfqpoint{-0.027778in}{0.000000in}}{\pgfqpoint{-0.000000in}{0.000000in}}{%
\pgfpathmoveto{\pgfqpoint{-0.000000in}{0.000000in}}%
\pgfpathlineto{\pgfqpoint{-0.027778in}{0.000000in}}%
\pgfusepath{stroke,fill}%
}%
\begin{pgfscope}%
\pgfsys@transformshift{0.588387in}{2.558698in}%
\pgfsys@useobject{currentmarker}{}%
\end{pgfscope}%
\end{pgfscope}%
\begin{pgfscope}%
\pgfsetbuttcap%
\pgfsetroundjoin%
\definecolor{currentfill}{rgb}{0.000000,0.000000,0.000000}%
\pgfsetfillcolor{currentfill}%
\pgfsetlinewidth{0.602250pt}%
\definecolor{currentstroke}{rgb}{0.000000,0.000000,0.000000}%
\pgfsetstrokecolor{currentstroke}%
\pgfsetdash{}{0pt}%
\pgfsys@defobject{currentmarker}{\pgfqpoint{-0.027778in}{0.000000in}}{\pgfqpoint{-0.000000in}{0.000000in}}{%
\pgfpathmoveto{\pgfqpoint{-0.000000in}{0.000000in}}%
\pgfpathlineto{\pgfqpoint{-0.027778in}{0.000000in}}%
\pgfusepath{stroke,fill}%
}%
\begin{pgfscope}%
\pgfsys@transformshift{0.588387in}{2.634965in}%
\pgfsys@useobject{currentmarker}{}%
\end{pgfscope}%
\end{pgfscope}%
\begin{pgfscope}%
\pgfsetbuttcap%
\pgfsetroundjoin%
\definecolor{currentfill}{rgb}{0.000000,0.000000,0.000000}%
\pgfsetfillcolor{currentfill}%
\pgfsetlinewidth{0.602250pt}%
\definecolor{currentstroke}{rgb}{0.000000,0.000000,0.000000}%
\pgfsetstrokecolor{currentstroke}%
\pgfsetdash{}{0pt}%
\pgfsys@defobject{currentmarker}{\pgfqpoint{-0.027778in}{0.000000in}}{\pgfqpoint{-0.000000in}{0.000000in}}{%
\pgfpathmoveto{\pgfqpoint{-0.000000in}{0.000000in}}%
\pgfpathlineto{\pgfqpoint{-0.027778in}{0.000000in}}%
\pgfusepath{stroke,fill}%
}%
\begin{pgfscope}%
\pgfsys@transformshift{0.588387in}{2.694123in}%
\pgfsys@useobject{currentmarker}{}%
\end{pgfscope}%
\end{pgfscope}%
\begin{pgfscope}%
\definecolor{textcolor}{rgb}{0.000000,0.000000,0.000000}%
\pgfsetstrokecolor{textcolor}%
\pgfsetfillcolor{textcolor}%
\pgftext[x=0.234413in,y=1.631726in,,bottom,rotate=90.000000]{\color{textcolor}{\rmfamily\fontsize{10.000000}{12.000000}\selectfont\catcode`\^=\active\def^{\ifmmode\sp\else\^{}\fi}\catcode`\%=\active\def%{\%}Time [ms]}}%
\end{pgfscope}%
\begin{pgfscope}%
\pgfpathrectangle{\pgfqpoint{0.588387in}{0.521603in}}{\pgfqpoint{3.660036in}{2.220246in}}%
\pgfusepath{clip}%
\pgfsetrectcap%
\pgfsetroundjoin%
\pgfsetlinewidth{1.505625pt}%
\pgfsetstrokecolor{currentstroke1}%
\pgfsetdash{}{0pt}%
\pgfpathmoveto{\pgfqpoint{0.754752in}{0.817980in}}%
\pgfpathlineto{\pgfqpoint{0.787056in}{0.841584in}}%
\pgfpathlineto{\pgfqpoint{0.819360in}{0.844513in}}%
\pgfpathlineto{\pgfqpoint{0.851664in}{0.780579in}}%
\pgfpathlineto{\pgfqpoint{0.883968in}{0.705357in}}%
\pgfpathlineto{\pgfqpoint{0.916272in}{0.698510in}}%
\pgfpathlineto{\pgfqpoint{0.948576in}{0.753264in}}%
\pgfpathlineto{\pgfqpoint{0.980880in}{0.779787in}}%
\pgfpathlineto{\pgfqpoint{1.013184in}{0.850277in}}%
\pgfpathlineto{\pgfqpoint{1.045488in}{0.845752in}}%
\pgfpathlineto{\pgfqpoint{1.077792in}{0.914392in}}%
\pgfpathlineto{\pgfqpoint{1.110096in}{0.931284in}}%
\pgfpathlineto{\pgfqpoint{1.142400in}{0.975929in}}%
\pgfpathlineto{\pgfqpoint{1.174704in}{0.993774in}}%
\pgfpathlineto{\pgfqpoint{1.207008in}{1.027066in}}%
\pgfpathlineto{\pgfqpoint{1.239311in}{1.045857in}}%
\pgfpathlineto{\pgfqpoint{1.271615in}{1.077790in}}%
\pgfpathlineto{\pgfqpoint{1.303919in}{1.097857in}}%
\pgfpathlineto{\pgfqpoint{1.336223in}{1.119419in}}%
\pgfpathlineto{\pgfqpoint{1.368527in}{1.134885in}}%
\pgfpathlineto{\pgfqpoint{1.400831in}{1.170061in}}%
\pgfpathlineto{\pgfqpoint{1.433135in}{1.180603in}}%
\pgfpathlineto{\pgfqpoint{1.465439in}{1.199617in}}%
\pgfpathlineto{\pgfqpoint{1.497743in}{1.220505in}}%
\pgfpathlineto{\pgfqpoint{1.530047in}{1.244441in}}%
\pgfpathlineto{\pgfqpoint{1.562351in}{1.253393in}}%
\pgfpathlineto{\pgfqpoint{1.594655in}{1.280163in}}%
\pgfpathlineto{\pgfqpoint{1.626959in}{1.283477in}}%
\pgfpathlineto{\pgfqpoint{1.659263in}{1.295166in}}%
\pgfpathlineto{\pgfqpoint{1.691567in}{1.324311in}}%
\pgfpathlineto{\pgfqpoint{1.723870in}{1.326535in}}%
\pgfpathlineto{\pgfqpoint{1.756174in}{1.344502in}}%
\pgfpathlineto{\pgfqpoint{1.788478in}{1.359039in}}%
\pgfpathlineto{\pgfqpoint{1.820782in}{1.368288in}}%
\pgfpathlineto{\pgfqpoint{1.853086in}{1.373954in}}%
\pgfpathlineto{\pgfqpoint{1.885390in}{1.404422in}}%
\pgfpathlineto{\pgfqpoint{1.917694in}{1.448823in}}%
\pgfpathlineto{\pgfqpoint{1.949998in}{1.418002in}}%
\pgfpathlineto{\pgfqpoint{1.982302in}{1.434138in}}%
\pgfpathlineto{\pgfqpoint{2.014606in}{1.450857in}}%
\pgfpathlineto{\pgfqpoint{2.046910in}{1.454263in}}%
\pgfpathlineto{\pgfqpoint{2.079214in}{1.471636in}}%
\pgfpathlineto{\pgfqpoint{2.111518in}{1.494260in}}%
\pgfpathlineto{\pgfqpoint{2.143822in}{1.489291in}}%
\pgfpathlineto{\pgfqpoint{2.176125in}{1.492132in}}%
\pgfpathlineto{\pgfqpoint{2.208429in}{1.517123in}}%
\pgfpathlineto{\pgfqpoint{2.240733in}{1.522021in}}%
\pgfpathlineto{\pgfqpoint{2.273037in}{1.535098in}}%
\pgfpathlineto{\pgfqpoint{2.305341in}{1.546272in}}%
\pgfpathlineto{\pgfqpoint{2.337645in}{1.559641in}}%
\pgfpathlineto{\pgfqpoint{2.369949in}{1.561090in}}%
\pgfpathlineto{\pgfqpoint{2.402253in}{1.568210in}}%
\pgfpathlineto{\pgfqpoint{2.434557in}{1.633164in}}%
\pgfpathlineto{\pgfqpoint{2.466861in}{1.588909in}}%
\pgfpathlineto{\pgfqpoint{2.499165in}{1.585987in}}%
\pgfpathlineto{\pgfqpoint{2.531469in}{1.624278in}}%
\pgfpathlineto{\pgfqpoint{2.563773in}{1.610781in}}%
\pgfpathlineto{\pgfqpoint{2.596077in}{1.648334in}}%
\pgfpathlineto{\pgfqpoint{2.628381in}{1.648587in}}%
\pgfpathlineto{\pgfqpoint{2.660684in}{1.631541in}}%
\pgfpathlineto{\pgfqpoint{2.692988in}{1.656474in}}%
\pgfpathlineto{\pgfqpoint{2.725292in}{1.642007in}}%
\pgfpathlineto{\pgfqpoint{2.757596in}{1.689406in}}%
\pgfpathlineto{\pgfqpoint{2.789900in}{1.674114in}}%
\pgfpathlineto{\pgfqpoint{2.822204in}{1.681064in}}%
\pgfpathlineto{\pgfqpoint{2.854508in}{1.689991in}}%
\pgfpathlineto{\pgfqpoint{2.886812in}{1.694057in}}%
\pgfpathlineto{\pgfqpoint{2.919116in}{1.700680in}}%
\pgfpathlineto{\pgfqpoint{2.951420in}{1.701632in}}%
\pgfpathlineto{\pgfqpoint{2.983724in}{1.719534in}}%
\pgfpathlineto{\pgfqpoint{3.016028in}{1.714033in}}%
\pgfpathlineto{\pgfqpoint{3.048332in}{1.735647in}}%
\pgfpathlineto{\pgfqpoint{3.080636in}{1.741430in}}%
\pgfpathlineto{\pgfqpoint{3.112940in}{1.748563in}}%
\pgfpathlineto{\pgfqpoint{3.177547in}{1.768931in}}%
\pgfpathlineto{\pgfqpoint{3.209851in}{1.756240in}}%
\pgfpathlineto{\pgfqpoint{3.242155in}{1.795330in}}%
\pgfpathlineto{\pgfqpoint{3.306763in}{1.772256in}}%
\pgfpathlineto{\pgfqpoint{3.339067in}{1.838101in}}%
\pgfpathlineto{\pgfqpoint{3.371371in}{1.804150in}}%
\pgfpathlineto{\pgfqpoint{3.403675in}{1.787344in}}%
\pgfpathlineto{\pgfqpoint{3.435979in}{1.805830in}}%
\pgfpathlineto{\pgfqpoint{3.468283in}{1.858702in}}%
\pgfpathlineto{\pgfqpoint{3.500587in}{1.824361in}}%
\pgfpathlineto{\pgfqpoint{3.565195in}{1.842464in}}%
\pgfpathlineto{\pgfqpoint{3.597498in}{1.863307in}}%
\pgfpathlineto{\pgfqpoint{3.629802in}{1.843028in}}%
\pgfpathlineto{\pgfqpoint{3.662106in}{1.864219in}}%
\pgfpathlineto{\pgfqpoint{3.694410in}{1.847957in}}%
\pgfpathlineto{\pgfqpoint{3.726714in}{1.845364in}}%
\pgfpathlineto{\pgfqpoint{3.759018in}{1.880493in}}%
\pgfpathlineto{\pgfqpoint{3.823626in}{1.880113in}}%
\pgfpathlineto{\pgfqpoint{3.855930in}{1.949580in}}%
\pgfpathlineto{\pgfqpoint{3.888234in}{1.909810in}}%
\pgfpathlineto{\pgfqpoint{3.952842in}{1.916369in}}%
\pgfpathlineto{\pgfqpoint{4.017450in}{1.930912in}}%
\pgfpathlineto{\pgfqpoint{4.049754in}{1.893482in}}%
\pgfpathlineto{\pgfqpoint{4.082057in}{1.912352in}}%
\pgfusepath{stroke}%
\end{pgfscope}%
\begin{pgfscope}%
\pgfpathrectangle{\pgfqpoint{0.588387in}{0.521603in}}{\pgfqpoint{3.660036in}{2.220246in}}%
\pgfusepath{clip}%
\pgfsetrectcap%
\pgfsetroundjoin%
\pgfsetlinewidth{1.505625pt}%
\pgfsetstrokecolor{currentstroke2}%
\pgfsetdash{}{0pt}%
\pgfpathmoveto{\pgfqpoint{0.754752in}{0.817980in}}%
\pgfpathlineto{\pgfqpoint{0.787056in}{0.832596in}}%
\pgfpathlineto{\pgfqpoint{0.819360in}{0.833465in}}%
\pgfpathlineto{\pgfqpoint{0.851664in}{0.764780in}}%
\pgfpathlineto{\pgfqpoint{0.883968in}{0.709499in}}%
\pgfpathlineto{\pgfqpoint{0.916272in}{0.696153in}}%
\pgfpathlineto{\pgfqpoint{0.948576in}{0.756937in}}%
\pgfpathlineto{\pgfqpoint{0.980880in}{0.783330in}}%
\pgfpathlineto{\pgfqpoint{1.013184in}{0.838025in}}%
\pgfpathlineto{\pgfqpoint{1.045488in}{0.847600in}}%
\pgfpathlineto{\pgfqpoint{1.077792in}{0.990458in}}%
\pgfpathlineto{\pgfqpoint{1.110096in}{0.988278in}}%
\pgfpathlineto{\pgfqpoint{1.142400in}{1.033675in}}%
\pgfpathlineto{\pgfqpoint{1.174704in}{1.083617in}}%
\pgfpathlineto{\pgfqpoint{1.207008in}{1.193959in}}%
\pgfpathlineto{\pgfqpoint{1.239311in}{1.172929in}}%
\pgfpathlineto{\pgfqpoint{1.271615in}{1.280038in}}%
\pgfpathlineto{\pgfqpoint{1.303919in}{1.384734in}}%
\pgfpathlineto{\pgfqpoint{1.336223in}{1.559256in}}%
\pgfpathlineto{\pgfqpoint{1.368527in}{1.454865in}}%
\pgfpathlineto{\pgfqpoint{1.400831in}{1.525671in}}%
\pgfpathlineto{\pgfqpoint{1.433135in}{1.604955in}}%
\pgfpathlineto{\pgfqpoint{1.465439in}{1.412370in}}%
\pgfpathlineto{\pgfqpoint{1.497743in}{1.684046in}}%
\pgfpathlineto{\pgfqpoint{1.530047in}{1.760042in}}%
\pgfpathlineto{\pgfqpoint{1.562351in}{1.977512in}}%
\pgfpathlineto{\pgfqpoint{1.594655in}{2.211966in}}%
\pgfpathlineto{\pgfqpoint{1.626959in}{2.262261in}}%
\pgfpathlineto{\pgfqpoint{1.659263in}{2.188021in}}%
\pgfpathlineto{\pgfqpoint{1.691567in}{2.081864in}}%
\pgfpathlineto{\pgfqpoint{1.723870in}{2.250369in}}%
\pgfpathlineto{\pgfqpoint{1.756174in}{2.111147in}}%
\pgfpathlineto{\pgfqpoint{1.788478in}{2.261671in}}%
\pgfpathlineto{\pgfqpoint{1.820782in}{2.175812in}}%
\pgfpathlineto{\pgfqpoint{1.853086in}{2.329405in}}%
\pgfpathlineto{\pgfqpoint{1.885390in}{2.309862in}}%
\pgfpathlineto{\pgfqpoint{1.917694in}{2.275212in}}%
\pgfpathlineto{\pgfqpoint{1.949998in}{2.255752in}}%
\pgfpathlineto{\pgfqpoint{1.982302in}{2.058431in}}%
\pgfpathlineto{\pgfqpoint{2.014606in}{2.350516in}}%
\pgfpathlineto{\pgfqpoint{2.046910in}{1.508403in}}%
\pgfpathlineto{\pgfqpoint{2.079214in}{2.415728in}}%
\pgfpathlineto{\pgfqpoint{2.111518in}{2.487203in}}%
\pgfpathlineto{\pgfqpoint{2.143822in}{2.465524in}}%
\pgfpathlineto{\pgfqpoint{2.176125in}{2.286313in}}%
\pgfpathlineto{\pgfqpoint{2.208429in}{2.281864in}}%
\pgfpathlineto{\pgfqpoint{2.240733in}{2.383955in}}%
\pgfpathlineto{\pgfqpoint{2.273037in}{2.416724in}}%
\pgfpathlineto{\pgfqpoint{2.305341in}{1.820710in}}%
\pgfpathlineto{\pgfqpoint{2.337645in}{2.257724in}}%
\pgfpathlineto{\pgfqpoint{2.369949in}{2.052576in}}%
\pgfpathlineto{\pgfqpoint{2.402253in}{2.275404in}}%
\pgfpathlineto{\pgfqpoint{2.434557in}{1.868193in}}%
\pgfpathlineto{\pgfqpoint{2.466861in}{2.365233in}}%
\pgfpathlineto{\pgfqpoint{2.499165in}{1.794147in}}%
\pgfpathlineto{\pgfqpoint{2.531469in}{2.461183in}}%
\pgfpathlineto{\pgfqpoint{2.563773in}{2.415417in}}%
\pgfpathlineto{\pgfqpoint{2.596077in}{2.393677in}}%
\pgfpathlineto{\pgfqpoint{2.628381in}{2.194382in}}%
\pgfpathlineto{\pgfqpoint{2.660684in}{2.383468in}}%
\pgfpathlineto{\pgfqpoint{2.692988in}{2.235714in}}%
\pgfpathlineto{\pgfqpoint{2.725292in}{2.336639in}}%
\pgfpathlineto{\pgfqpoint{2.757596in}{1.728591in}}%
\pgfpathlineto{\pgfqpoint{2.789900in}{2.399121in}}%
\pgfpathlineto{\pgfqpoint{2.822204in}{1.748094in}}%
\pgfpathlineto{\pgfqpoint{2.854508in}{2.259076in}}%
\pgfpathlineto{\pgfqpoint{2.886812in}{2.516753in}}%
\pgfpathlineto{\pgfqpoint{2.919116in}{2.468375in}}%
\pgfpathlineto{\pgfqpoint{2.951420in}{2.591412in}}%
\pgfpathlineto{\pgfqpoint{2.983724in}{2.311099in}}%
\pgfpathlineto{\pgfqpoint{3.016028in}{2.590342in}}%
\pgfpathlineto{\pgfqpoint{3.048332in}{2.466452in}}%
\pgfpathlineto{\pgfqpoint{3.080636in}{2.010927in}}%
\pgfpathlineto{\pgfqpoint{3.112940in}{1.859218in}}%
\pgfpathlineto{\pgfqpoint{3.177547in}{2.358931in}}%
\pgfpathlineto{\pgfqpoint{3.209851in}{2.326771in}}%
\pgfpathlineto{\pgfqpoint{3.242155in}{2.483850in}}%
\pgfpathlineto{\pgfqpoint{3.306763in}{2.475055in}}%
\pgfpathlineto{\pgfqpoint{3.339067in}{1.867235in}}%
\pgfpathlineto{\pgfqpoint{3.371371in}{2.482666in}}%
\pgfpathlineto{\pgfqpoint{3.403675in}{2.538507in}}%
\pgfpathlineto{\pgfqpoint{3.435979in}{2.530909in}}%
\pgfpathlineto{\pgfqpoint{3.500587in}{2.395994in}}%
\pgfpathlineto{\pgfqpoint{3.565195in}{2.532641in}}%
\pgfpathlineto{\pgfqpoint{3.597498in}{2.621419in}}%
\pgfpathlineto{\pgfqpoint{3.629802in}{2.554870in}}%
\pgfpathlineto{\pgfqpoint{3.662106in}{2.521473in}}%
\pgfpathlineto{\pgfqpoint{3.694410in}{2.382215in}}%
\pgfpathlineto{\pgfqpoint{3.726714in}{2.522235in}}%
\pgfpathlineto{\pgfqpoint{3.759018in}{2.373263in}}%
\pgfpathlineto{\pgfqpoint{3.823626in}{2.530945in}}%
\pgfpathlineto{\pgfqpoint{3.855930in}{2.524858in}}%
\pgfpathlineto{\pgfqpoint{3.888234in}{1.953074in}}%
\pgfpathlineto{\pgfqpoint{3.952842in}{2.076881in}}%
\pgfpathlineto{\pgfqpoint{4.017450in}{1.959504in}}%
\pgfusepath{stroke}%
\end{pgfscope}%
\begin{pgfscope}%
\pgfpathrectangle{\pgfqpoint{0.588387in}{0.521603in}}{\pgfqpoint{3.660036in}{2.220246in}}%
\pgfusepath{clip}%
\pgfsetrectcap%
\pgfsetroundjoin%
\pgfsetlinewidth{1.505625pt}%
\pgfsetstrokecolor{currentstroke3}%
\pgfsetdash{}{0pt}%
\pgfpathmoveto{\pgfqpoint{0.754752in}{0.817980in}}%
\pgfpathlineto{\pgfqpoint{0.787056in}{0.837128in}}%
\pgfpathlineto{\pgfqpoint{0.819360in}{0.823791in}}%
\pgfpathlineto{\pgfqpoint{0.851664in}{0.773140in}}%
\pgfpathlineto{\pgfqpoint{0.883968in}{0.782978in}}%
\pgfpathlineto{\pgfqpoint{0.916272in}{0.709815in}}%
\pgfpathlineto{\pgfqpoint{0.948576in}{0.622524in}}%
\pgfpathlineto{\pgfqpoint{0.980880in}{0.651737in}}%
\pgfpathlineto{\pgfqpoint{1.013184in}{0.645153in}}%
\pgfpathlineto{\pgfqpoint{1.045488in}{0.644758in}}%
\pgfpathlineto{\pgfqpoint{1.077792in}{0.666991in}}%
\pgfpathlineto{\pgfqpoint{1.110096in}{0.669092in}}%
\pgfpathlineto{\pgfqpoint{1.142400in}{0.736217in}}%
\pgfpathlineto{\pgfqpoint{1.174704in}{0.706976in}}%
\pgfpathlineto{\pgfqpoint{1.207008in}{0.736848in}}%
\pgfpathlineto{\pgfqpoint{1.239311in}{0.745591in}}%
\pgfpathlineto{\pgfqpoint{1.271615in}{0.774924in}}%
\pgfpathlineto{\pgfqpoint{1.303919in}{0.778251in}}%
\pgfpathlineto{\pgfqpoint{1.336223in}{0.812737in}}%
\pgfpathlineto{\pgfqpoint{1.368527in}{0.799765in}}%
\pgfpathlineto{\pgfqpoint{1.400831in}{0.834872in}}%
\pgfpathlineto{\pgfqpoint{1.433135in}{0.818586in}}%
\pgfpathlineto{\pgfqpoint{1.465439in}{0.852671in}}%
\pgfpathlineto{\pgfqpoint{1.497743in}{0.848782in}}%
\pgfpathlineto{\pgfqpoint{1.530047in}{0.866097in}}%
\pgfpathlineto{\pgfqpoint{1.562351in}{0.865203in}}%
\pgfpathlineto{\pgfqpoint{1.594655in}{0.883653in}}%
\pgfpathlineto{\pgfqpoint{1.626959in}{0.935603in}}%
\pgfpathlineto{\pgfqpoint{1.659263in}{0.905558in}}%
\pgfpathlineto{\pgfqpoint{1.691567in}{0.900534in}}%
\pgfpathlineto{\pgfqpoint{1.723870in}{0.961309in}}%
\pgfpathlineto{\pgfqpoint{1.756174in}{0.927543in}}%
\pgfpathlineto{\pgfqpoint{1.788478in}{0.955144in}}%
\pgfpathlineto{\pgfqpoint{1.820782in}{0.953459in}}%
\pgfpathlineto{\pgfqpoint{1.853086in}{0.992099in}}%
\pgfpathlineto{\pgfqpoint{1.885390in}{0.952866in}}%
\pgfpathlineto{\pgfqpoint{1.917694in}{0.957009in}}%
\pgfpathlineto{\pgfqpoint{1.949998in}{0.966909in}}%
\pgfpathlineto{\pgfqpoint{1.982302in}{0.996710in}}%
\pgfpathlineto{\pgfqpoint{2.014606in}{0.976686in}}%
\pgfpathlineto{\pgfqpoint{2.046910in}{0.984377in}}%
\pgfpathlineto{\pgfqpoint{2.079214in}{0.983525in}}%
\pgfpathlineto{\pgfqpoint{2.111518in}{1.018632in}}%
\pgfpathlineto{\pgfqpoint{2.143822in}{0.993953in}}%
\pgfpathlineto{\pgfqpoint{2.176125in}{1.016357in}}%
\pgfpathlineto{\pgfqpoint{2.208429in}{1.017681in}}%
\pgfpathlineto{\pgfqpoint{2.240733in}{1.259184in}}%
\pgfpathlineto{\pgfqpoint{2.273037in}{1.016166in}}%
\pgfpathlineto{\pgfqpoint{2.305341in}{1.034952in}}%
\pgfpathlineto{\pgfqpoint{2.337645in}{1.027775in}}%
\pgfpathlineto{\pgfqpoint{2.369949in}{1.055868in}}%
\pgfpathlineto{\pgfqpoint{2.402253in}{1.031891in}}%
\pgfpathlineto{\pgfqpoint{2.434557in}{1.066968in}}%
\pgfpathlineto{\pgfqpoint{2.466861in}{1.050000in}}%
\pgfpathlineto{\pgfqpoint{2.499165in}{1.071832in}}%
\pgfpathlineto{\pgfqpoint{2.531469in}{1.056633in}}%
\pgfpathlineto{\pgfqpoint{2.563773in}{1.084895in}}%
\pgfpathlineto{\pgfqpoint{2.596077in}{1.075991in}}%
\pgfpathlineto{\pgfqpoint{2.628381in}{1.162751in}}%
\pgfpathlineto{\pgfqpoint{2.660684in}{1.089022in}}%
\pgfpathlineto{\pgfqpoint{2.692988in}{1.103593in}}%
\pgfpathlineto{\pgfqpoint{2.725292in}{1.120614in}}%
\pgfpathlineto{\pgfqpoint{2.757596in}{1.145070in}}%
\pgfpathlineto{\pgfqpoint{2.789900in}{1.107835in}}%
\pgfpathlineto{\pgfqpoint{2.822204in}{1.116120in}}%
\pgfpathlineto{\pgfqpoint{2.854508in}{1.948258in}}%
\pgfpathlineto{\pgfqpoint{2.886812in}{1.126126in}}%
\pgfpathlineto{\pgfqpoint{2.919116in}{1.104757in}}%
\pgfpathlineto{\pgfqpoint{2.951420in}{1.129379in}}%
\pgfpathlineto{\pgfqpoint{2.983724in}{1.220422in}}%
\pgfpathlineto{\pgfqpoint{3.016028in}{1.234707in}}%
\pgfpathlineto{\pgfqpoint{3.048332in}{1.146373in}}%
\pgfpathlineto{\pgfqpoint{3.080636in}{1.158440in}}%
\pgfpathlineto{\pgfqpoint{3.112940in}{1.116120in}}%
\pgfpathlineto{\pgfqpoint{3.177547in}{1.159206in}}%
\pgfpathlineto{\pgfqpoint{3.209851in}{1.157569in}}%
\pgfpathlineto{\pgfqpoint{3.242155in}{1.171168in}}%
\pgfpathlineto{\pgfqpoint{3.306763in}{1.185818in}}%
\pgfpathlineto{\pgfqpoint{3.339067in}{1.187240in}}%
\pgfpathlineto{\pgfqpoint{3.371371in}{1.173368in}}%
\pgfpathlineto{\pgfqpoint{3.403675in}{1.154058in}}%
\pgfpathlineto{\pgfqpoint{3.435979in}{1.164717in}}%
\pgfpathlineto{\pgfqpoint{3.468283in}{1.202393in}}%
\pgfpathlineto{\pgfqpoint{3.500587in}{1.183312in}}%
\pgfpathlineto{\pgfqpoint{3.565195in}{1.220684in}}%
\pgfpathlineto{\pgfqpoint{3.597498in}{1.202393in}}%
\pgfpathlineto{\pgfqpoint{3.629802in}{1.224583in}}%
\pgfpathlineto{\pgfqpoint{3.662106in}{1.243260in}}%
\pgfpathlineto{\pgfqpoint{3.694410in}{1.230325in}}%
\pgfpathlineto{\pgfqpoint{3.726714in}{1.230325in}}%
\pgfpathlineto{\pgfqpoint{3.759018in}{1.357676in}}%
\pgfpathlineto{\pgfqpoint{3.823626in}{1.194925in}}%
\pgfpathlineto{\pgfqpoint{3.855930in}{1.261551in}}%
\pgfpathlineto{\pgfqpoint{3.888234in}{1.230325in}}%
\pgfpathlineto{\pgfqpoint{3.952842in}{1.216727in}}%
\pgfpathlineto{\pgfqpoint{4.017450in}{1.216727in}}%
\pgfpathlineto{\pgfqpoint{4.049754in}{1.216727in}}%
\pgfpathlineto{\pgfqpoint{4.082057in}{1.255593in}}%
\pgfusepath{stroke}%
\end{pgfscope}%
\begin{pgfscope}%
\pgfpathrectangle{\pgfqpoint{0.588387in}{0.521603in}}{\pgfqpoint{3.660036in}{2.220246in}}%
\pgfusepath{clip}%
\pgfsetrectcap%
\pgfsetroundjoin%
\pgfsetlinewidth{1.505625pt}%
\pgfsetstrokecolor{currentstroke4}%
\pgfsetdash{}{0pt}%
\pgfpathmoveto{\pgfqpoint{0.754752in}{0.824866in}}%
\pgfpathlineto{\pgfqpoint{0.787056in}{0.837128in}}%
\pgfpathlineto{\pgfqpoint{0.819360in}{0.833465in}}%
\pgfpathlineto{\pgfqpoint{0.851664in}{0.749181in}}%
\pgfpathlineto{\pgfqpoint{0.883968in}{0.719581in}}%
\pgfpathlineto{\pgfqpoint{0.916272in}{0.696153in}}%
\pgfpathlineto{\pgfqpoint{0.948576in}{0.745760in}}%
\pgfpathlineto{\pgfqpoint{0.980880in}{0.767624in}}%
\pgfpathlineto{\pgfqpoint{1.013184in}{0.830762in}}%
\pgfpathlineto{\pgfqpoint{1.045488in}{0.832446in}}%
\pgfpathlineto{\pgfqpoint{1.077792in}{0.899429in}}%
\pgfpathlineto{\pgfqpoint{1.110096in}{0.914605in}}%
\pgfpathlineto{\pgfqpoint{1.142400in}{0.962700in}}%
\pgfpathlineto{\pgfqpoint{1.174704in}{0.977561in}}%
\pgfpathlineto{\pgfqpoint{1.207008in}{1.015399in}}%
\pgfpathlineto{\pgfqpoint{1.239311in}{1.034635in}}%
\pgfpathlineto{\pgfqpoint{1.271615in}{1.059499in}}%
\pgfpathlineto{\pgfqpoint{1.303919in}{1.081167in}}%
\pgfpathlineto{\pgfqpoint{1.336223in}{1.116594in}}%
\pgfpathlineto{\pgfqpoint{1.368527in}{1.124752in}}%
\pgfpathlineto{\pgfqpoint{1.400831in}{1.152877in}}%
\pgfpathlineto{\pgfqpoint{1.433135in}{1.165831in}}%
\pgfpathlineto{\pgfqpoint{1.465439in}{1.193976in}}%
\pgfpathlineto{\pgfqpoint{1.497743in}{1.204615in}}%
\pgfpathlineto{\pgfqpoint{1.530047in}{1.227410in}}%
\pgfpathlineto{\pgfqpoint{1.562351in}{1.273335in}}%
\pgfpathlineto{\pgfqpoint{1.594655in}{1.281656in}}%
\pgfpathlineto{\pgfqpoint{1.626959in}{1.283260in}}%
\pgfpathlineto{\pgfqpoint{1.659263in}{1.290805in}}%
\pgfpathlineto{\pgfqpoint{1.691567in}{1.304833in}}%
\pgfpathlineto{\pgfqpoint{1.723870in}{1.324994in}}%
\pgfpathlineto{\pgfqpoint{1.756174in}{1.362915in}}%
\pgfpathlineto{\pgfqpoint{1.788478in}{1.352152in}}%
\pgfpathlineto{\pgfqpoint{1.820782in}{1.355247in}}%
\pgfpathlineto{\pgfqpoint{1.853086in}{1.373587in}}%
\pgfpathlineto{\pgfqpoint{1.885390in}{1.390140in}}%
\pgfpathlineto{\pgfqpoint{1.917694in}{1.397212in}}%
\pgfpathlineto{\pgfqpoint{1.949998in}{1.411230in}}%
\pgfpathlineto{\pgfqpoint{1.982302in}{1.425665in}}%
\pgfpathlineto{\pgfqpoint{2.014606in}{1.436992in}}%
\pgfpathlineto{\pgfqpoint{2.046910in}{1.449034in}}%
\pgfpathlineto{\pgfqpoint{2.079214in}{1.459840in}}%
\pgfpathlineto{\pgfqpoint{2.111518in}{1.479143in}}%
\pgfpathlineto{\pgfqpoint{2.143822in}{1.485065in}}%
\pgfpathlineto{\pgfqpoint{2.176125in}{1.490990in}}%
\pgfpathlineto{\pgfqpoint{2.208429in}{1.515232in}}%
\pgfpathlineto{\pgfqpoint{2.240733in}{1.518020in}}%
\pgfpathlineto{\pgfqpoint{2.273037in}{1.535447in}}%
\pgfpathlineto{\pgfqpoint{2.305341in}{1.582247in}}%
\pgfpathlineto{\pgfqpoint{2.337645in}{1.549790in}}%
\pgfpathlineto{\pgfqpoint{2.369949in}{1.555039in}}%
\pgfpathlineto{\pgfqpoint{2.402253in}{1.559222in}}%
\pgfpathlineto{\pgfqpoint{2.434557in}{1.591813in}}%
\pgfpathlineto{\pgfqpoint{2.466861in}{1.581810in}}%
\pgfpathlineto{\pgfqpoint{2.499165in}{1.580737in}}%
\pgfpathlineto{\pgfqpoint{2.531469in}{1.594944in}}%
\pgfpathlineto{\pgfqpoint{2.563773in}{1.598582in}}%
\pgfpathlineto{\pgfqpoint{2.596077in}{1.617555in}}%
\pgfpathlineto{\pgfqpoint{2.628381in}{1.623866in}}%
\pgfpathlineto{\pgfqpoint{2.660684in}{1.640897in}}%
\pgfpathlineto{\pgfqpoint{2.692988in}{1.634899in}}%
\pgfpathlineto{\pgfqpoint{2.725292in}{1.636495in}}%
\pgfpathlineto{\pgfqpoint{2.757596in}{1.671199in}}%
\pgfpathlineto{\pgfqpoint{2.789900in}{1.656474in}}%
\pgfpathlineto{\pgfqpoint{2.822204in}{1.657004in}}%
\pgfpathlineto{\pgfqpoint{2.854508in}{1.678326in}}%
\pgfpathlineto{\pgfqpoint{2.886812in}{1.688230in}}%
\pgfpathlineto{\pgfqpoint{2.919116in}{1.701088in}}%
\pgfpathlineto{\pgfqpoint{2.951420in}{1.696352in}}%
\pgfpathlineto{\pgfqpoint{2.983724in}{1.696961in}}%
\pgfpathlineto{\pgfqpoint{3.016028in}{1.693287in}}%
\pgfpathlineto{\pgfqpoint{3.048332in}{1.720787in}}%
\pgfpathlineto{\pgfqpoint{3.080636in}{1.722967in}}%
\pgfpathlineto{\pgfqpoint{3.112940in}{1.730272in}}%
\pgfpathlineto{\pgfqpoint{3.177547in}{1.753350in}}%
\pgfpathlineto{\pgfqpoint{3.209851in}{1.748094in}}%
\pgfpathlineto{\pgfqpoint{3.242155in}{1.763020in}}%
\pgfpathlineto{\pgfqpoint{3.306763in}{1.763693in}}%
\pgfpathlineto{\pgfqpoint{3.339067in}{1.809871in}}%
\pgfpathlineto{\pgfqpoint{3.371371in}{1.788369in}}%
\pgfpathlineto{\pgfqpoint{3.403675in}{1.778272in}}%
\pgfpathlineto{\pgfqpoint{3.435979in}{1.797559in}}%
\pgfpathlineto{\pgfqpoint{3.468283in}{1.832690in}}%
\pgfpathlineto{\pgfqpoint{3.500587in}{1.815398in}}%
\pgfpathlineto{\pgfqpoint{3.565195in}{1.832885in}}%
\pgfpathlineto{\pgfqpoint{3.597498in}{1.842087in}}%
\pgfpathlineto{\pgfqpoint{3.629802in}{1.835988in}}%
\pgfpathlineto{\pgfqpoint{3.662106in}{1.850525in}}%
\pgfpathlineto{\pgfqpoint{3.694410in}{1.851799in}}%
\pgfpathlineto{\pgfqpoint{3.726714in}{1.848601in}}%
\pgfpathlineto{\pgfqpoint{3.759018in}{1.861032in}}%
\pgfpathlineto{\pgfqpoint{3.823626in}{1.859321in}}%
\pgfpathlineto{\pgfqpoint{3.855930in}{1.925190in}}%
\pgfpathlineto{\pgfqpoint{3.888234in}{1.886045in}}%
\pgfpathlineto{\pgfqpoint{3.952842in}{1.891665in}}%
\pgfpathlineto{\pgfqpoint{4.017450in}{1.907244in}}%
\pgfpathlineto{\pgfqpoint{4.049754in}{1.875502in}}%
\pgfpathlineto{\pgfqpoint{4.082057in}{1.914870in}}%
\pgfusepath{stroke}%
\end{pgfscope}%
\begin{pgfscope}%
\pgfpathrectangle{\pgfqpoint{0.588387in}{0.521603in}}{\pgfqpoint{3.660036in}{2.220246in}}%
\pgfusepath{clip}%
\pgfsetrectcap%
\pgfsetroundjoin%
\pgfsetlinewidth{1.505625pt}%
\pgfsetstrokecolor{currentstroke5}%
\pgfsetdash{}{0pt}%
\pgfpathmoveto{\pgfqpoint{0.754752in}{0.824866in}}%
\pgfpathlineto{\pgfqpoint{0.787056in}{0.841584in}}%
\pgfpathlineto{\pgfqpoint{0.819360in}{0.836271in}}%
\pgfpathlineto{\pgfqpoint{0.851664in}{0.759181in}}%
\pgfpathlineto{\pgfqpoint{0.883968in}{0.713263in}}%
\pgfpathlineto{\pgfqpoint{0.916272in}{0.696153in}}%
\pgfpathlineto{\pgfqpoint{0.948576in}{0.740637in}}%
\pgfpathlineto{\pgfqpoint{0.980880in}{0.774989in}}%
\pgfpathlineto{\pgfqpoint{1.013184in}{0.831680in}}%
\pgfpathlineto{\pgfqpoint{1.045488in}{0.831470in}}%
\pgfpathlineto{\pgfqpoint{1.077792in}{0.944177in}}%
\pgfpathlineto{\pgfqpoint{1.110096in}{0.941848in}}%
\pgfpathlineto{\pgfqpoint{1.142400in}{0.996245in}}%
\pgfpathlineto{\pgfqpoint{1.174704in}{1.036814in}}%
\pgfpathlineto{\pgfqpoint{1.207008in}{1.094900in}}%
\pgfpathlineto{\pgfqpoint{1.239311in}{1.057302in}}%
\pgfpathlineto{\pgfqpoint{1.271615in}{1.090525in}}%
\pgfpathlineto{\pgfqpoint{1.303919in}{1.110772in}}%
\pgfpathlineto{\pgfqpoint{1.336223in}{1.238521in}}%
\pgfpathlineto{\pgfqpoint{1.368527in}{1.158945in}}%
\pgfpathlineto{\pgfqpoint{1.400831in}{1.190854in}}%
\pgfpathlineto{\pgfqpoint{1.433135in}{1.335632in}}%
\pgfpathlineto{\pgfqpoint{1.465439in}{1.471748in}}%
\pgfpathlineto{\pgfqpoint{1.497743in}{1.372297in}}%
\pgfpathlineto{\pgfqpoint{1.530047in}{1.274486in}}%
\pgfpathlineto{\pgfqpoint{1.562351in}{1.393255in}}%
\pgfpathlineto{\pgfqpoint{1.594655in}{1.956679in}}%
\pgfpathlineto{\pgfqpoint{1.626959in}{1.565234in}}%
\pgfpathlineto{\pgfqpoint{1.659263in}{1.533987in}}%
\pgfpathlineto{\pgfqpoint{1.691567in}{1.560733in}}%
\pgfpathlineto{\pgfqpoint{1.723870in}{2.092953in}}%
\pgfpathlineto{\pgfqpoint{1.756174in}{1.745278in}}%
\pgfpathlineto{\pgfqpoint{1.788478in}{1.910533in}}%
\pgfpathlineto{\pgfqpoint{1.820782in}{1.596184in}}%
\pgfpathlineto{\pgfqpoint{1.853086in}{2.093887in}}%
\pgfpathlineto{\pgfqpoint{1.885390in}{1.798670in}}%
\pgfpathlineto{\pgfqpoint{1.917694in}{1.705634in}}%
\pgfpathlineto{\pgfqpoint{1.949998in}{1.815901in}}%
\pgfpathlineto{\pgfqpoint{1.982302in}{1.580989in}}%
\pgfpathlineto{\pgfqpoint{2.014606in}{1.583054in}}%
\pgfpathlineto{\pgfqpoint{2.046910in}{1.538689in}}%
\pgfpathlineto{\pgfqpoint{2.079214in}{1.814370in}}%
\pgfpathlineto{\pgfqpoint{2.111518in}{1.547849in}}%
\pgfpathlineto{\pgfqpoint{2.143822in}{1.763151in}}%
\pgfpathlineto{\pgfqpoint{2.176125in}{1.536127in}}%
\pgfpathlineto{\pgfqpoint{2.208429in}{2.038233in}}%
\pgfpathlineto{\pgfqpoint{2.240733in}{2.206791in}}%
\pgfpathlineto{\pgfqpoint{2.273037in}{2.004811in}}%
\pgfpathlineto{\pgfqpoint{2.305341in}{1.595946in}}%
\pgfpathlineto{\pgfqpoint{2.337645in}{2.137952in}}%
\pgfpathlineto{\pgfqpoint{2.369949in}{2.089897in}}%
\pgfpathlineto{\pgfqpoint{2.402253in}{1.902601in}}%
\pgfpathlineto{\pgfqpoint{2.434557in}{1.676424in}}%
\pgfpathlineto{\pgfqpoint{2.466861in}{1.706490in}}%
\pgfpathlineto{\pgfqpoint{2.499165in}{2.514805in}}%
\pgfpathlineto{\pgfqpoint{2.531469in}{2.011377in}}%
\pgfpathlineto{\pgfqpoint{2.563773in}{2.166799in}}%
\pgfpathlineto{\pgfqpoint{2.596077in}{2.030058in}}%
\pgfpathlineto{\pgfqpoint{2.628381in}{1.987511in}}%
\pgfpathlineto{\pgfqpoint{2.660684in}{1.830451in}}%
\pgfpathlineto{\pgfqpoint{2.692988in}{2.295015in}}%
\pgfpathlineto{\pgfqpoint{2.725292in}{2.111986in}}%
\pgfpathlineto{\pgfqpoint{2.757596in}{1.777432in}}%
\pgfpathlineto{\pgfqpoint{2.789900in}{2.148049in}}%
\pgfpathlineto{\pgfqpoint{2.822204in}{1.720370in}}%
\pgfpathlineto{\pgfqpoint{2.854508in}{1.745139in}}%
\pgfpathlineto{\pgfqpoint{2.886812in}{1.711886in}}%
\pgfpathlineto{\pgfqpoint{2.919116in}{2.214215in}}%
\pgfpathlineto{\pgfqpoint{2.951420in}{1.826701in}}%
\pgfpathlineto{\pgfqpoint{2.983724in}{2.148935in}}%
\pgfpathlineto{\pgfqpoint{3.016028in}{2.230423in}}%
\pgfpathlineto{\pgfqpoint{3.048332in}{1.994177in}}%
\pgfpathlineto{\pgfqpoint{3.080636in}{1.829251in}}%
\pgfpathlineto{\pgfqpoint{3.112940in}{2.130208in}}%
\pgfpathlineto{\pgfqpoint{3.177547in}{2.106930in}}%
\pgfpathlineto{\pgfqpoint{3.209851in}{1.776589in}}%
\pgfpathlineto{\pgfqpoint{3.242155in}{2.289702in}}%
\pgfpathlineto{\pgfqpoint{3.306763in}{1.975273in}}%
\pgfpathlineto{\pgfqpoint{3.339067in}{1.822947in}}%
\pgfpathlineto{\pgfqpoint{3.371371in}{2.157153in}}%
\pgfpathlineto{\pgfqpoint{3.403675in}{1.979484in}}%
\pgfpathlineto{\pgfqpoint{3.435979in}{1.894899in}}%
\pgfpathlineto{\pgfqpoint{3.468283in}{1.880113in}}%
\pgfpathlineto{\pgfqpoint{3.500587in}{1.842087in}}%
\pgfpathlineto{\pgfqpoint{3.565195in}{1.985530in}}%
\pgfpathlineto{\pgfqpoint{3.597498in}{1.925431in}}%
\pgfpathlineto{\pgfqpoint{3.629802in}{2.244927in}}%
\pgfpathlineto{\pgfqpoint{3.662106in}{2.523802in}}%
\pgfpathlineto{\pgfqpoint{3.694410in}{2.360453in}}%
\pgfpathlineto{\pgfqpoint{3.726714in}{2.086059in}}%
\pgfpathlineto{\pgfqpoint{3.759018in}{2.393479in}}%
\pgfpathlineto{\pgfqpoint{3.823626in}{2.134875in}}%
\pgfpathlineto{\pgfqpoint{3.855930in}{2.538650in}}%
\pgfpathlineto{\pgfqpoint{3.888234in}{2.369536in}}%
\pgfpathlineto{\pgfqpoint{3.952842in}{2.292390in}}%
\pgfpathlineto{\pgfqpoint{4.017450in}{2.256898in}}%
\pgfpathlineto{\pgfqpoint{4.049754in}{2.025848in}}%
\pgfpathlineto{\pgfqpoint{4.082057in}{2.525860in}}%
\pgfusepath{stroke}%
\end{pgfscope}%
\begin{pgfscope}%
\pgfpathrectangle{\pgfqpoint{0.588387in}{0.521603in}}{\pgfqpoint{3.660036in}{2.220246in}}%
\pgfusepath{clip}%
\pgfsetrectcap%
\pgfsetroundjoin%
\pgfsetlinewidth{1.505625pt}%
\pgfsetstrokecolor{currentstroke6}%
\pgfsetdash{}{0pt}%
\pgfpathmoveto{\pgfqpoint{0.754752in}{0.838125in}}%
\pgfpathlineto{\pgfqpoint{0.787056in}{0.850277in}}%
\pgfpathlineto{\pgfqpoint{0.819360in}{0.836271in}}%
\pgfpathlineto{\pgfqpoint{0.851664in}{0.753792in}}%
\pgfpathlineto{\pgfqpoint{0.883968in}{0.719581in}}%
\pgfpathlineto{\pgfqpoint{0.916272in}{0.696153in}}%
\pgfpathlineto{\pgfqpoint{0.948576in}{0.738038in}}%
\pgfpathlineto{\pgfqpoint{0.980880in}{0.774989in}}%
\pgfpathlineto{\pgfqpoint{1.013184in}{0.832596in}}%
\pgfpathlineto{\pgfqpoint{1.045488in}{0.833419in}}%
\pgfpathlineto{\pgfqpoint{1.077792in}{0.899429in}}%
\pgfpathlineto{\pgfqpoint{1.110096in}{0.918592in}}%
\pgfpathlineto{\pgfqpoint{1.142400in}{0.972980in}}%
\pgfpathlineto{\pgfqpoint{1.174704in}{0.984165in}}%
\pgfpathlineto{\pgfqpoint{1.207008in}{1.012675in}}%
\pgfpathlineto{\pgfqpoint{1.239311in}{1.031287in}}%
\pgfpathlineto{\pgfqpoint{1.271615in}{1.062191in}}%
\pgfpathlineto{\pgfqpoint{1.303919in}{1.082061in}}%
\pgfpathlineto{\pgfqpoint{1.336223in}{1.111335in}}%
\pgfpathlineto{\pgfqpoint{1.368527in}{1.122447in}}%
\pgfpathlineto{\pgfqpoint{1.400831in}{1.154646in}}%
\pgfpathlineto{\pgfqpoint{1.433135in}{1.165831in}}%
\pgfpathlineto{\pgfqpoint{1.465439in}{1.192069in}}%
\pgfpathlineto{\pgfqpoint{1.497743in}{1.204394in}}%
\pgfpathlineto{\pgfqpoint{1.530047in}{1.226571in}}%
\pgfpathlineto{\pgfqpoint{1.562351in}{1.237752in}}%
\pgfpathlineto{\pgfqpoint{1.594655in}{1.259400in}}%
\pgfpathlineto{\pgfqpoint{1.626959in}{1.277776in}}%
\pgfpathlineto{\pgfqpoint{1.659263in}{1.285924in}}%
\pgfpathlineto{\pgfqpoint{1.691567in}{1.314656in}}%
\pgfpathlineto{\pgfqpoint{1.723870in}{1.318341in}}%
\pgfpathlineto{\pgfqpoint{1.756174in}{1.342108in}}%
\pgfpathlineto{\pgfqpoint{1.788478in}{1.348287in}}%
\pgfpathlineto{\pgfqpoint{1.820782in}{1.358725in}}%
\pgfpathlineto{\pgfqpoint{1.853086in}{1.369137in}}%
\pgfpathlineto{\pgfqpoint{1.885390in}{1.383462in}}%
\pgfpathlineto{\pgfqpoint{1.917694in}{1.397212in}}%
\pgfpathlineto{\pgfqpoint{1.949998in}{1.407859in}}%
\pgfpathlineto{\pgfqpoint{1.982302in}{1.443199in}}%
\pgfpathlineto{\pgfqpoint{2.014606in}{1.431297in}}%
\pgfpathlineto{\pgfqpoint{2.046910in}{1.444238in}}%
\pgfpathlineto{\pgfqpoint{2.079214in}{1.469393in}}%
\pgfpathlineto{\pgfqpoint{2.111518in}{1.482364in}}%
\pgfpathlineto{\pgfqpoint{2.143822in}{1.486279in}}%
\pgfpathlineto{\pgfqpoint{2.176125in}{1.489460in}}%
\pgfpathlineto{\pgfqpoint{2.208429in}{1.500371in}}%
\pgfpathlineto{\pgfqpoint{2.240733in}{1.515772in}}%
\pgfpathlineto{\pgfqpoint{2.273037in}{1.524266in}}%
\pgfpathlineto{\pgfqpoint{2.305341in}{1.563526in}}%
\pgfpathlineto{\pgfqpoint{2.337645in}{1.547902in}}%
\pgfpathlineto{\pgfqpoint{2.369949in}{1.553645in}}%
\pgfpathlineto{\pgfqpoint{2.402253in}{1.549613in}}%
\pgfpathlineto{\pgfqpoint{2.434557in}{1.576461in}}%
\pgfpathlineto{\pgfqpoint{2.466861in}{1.590987in}}%
\pgfpathlineto{\pgfqpoint{2.499165in}{1.575381in}}%
\pgfpathlineto{\pgfqpoint{2.531469in}{1.588711in}}%
\pgfpathlineto{\pgfqpoint{2.563773in}{1.604393in}}%
\pgfpathlineto{\pgfqpoint{2.596077in}{1.615136in}}%
\pgfpathlineto{\pgfqpoint{2.628381in}{1.615861in}}%
\pgfpathlineto{\pgfqpoint{2.660684in}{1.612437in}}%
\pgfpathlineto{\pgfqpoint{2.692988in}{1.628482in}}%
\pgfpathlineto{\pgfqpoint{2.725292in}{1.629726in}}%
\pgfpathlineto{\pgfqpoint{2.757596in}{1.655675in}}%
\pgfpathlineto{\pgfqpoint{2.789900in}{1.650383in}}%
\pgfpathlineto{\pgfqpoint{2.822204in}{1.678632in}}%
\pgfpathlineto{\pgfqpoint{2.854508in}{1.678020in}}%
\pgfpathlineto{\pgfqpoint{2.886812in}{1.672452in}}%
\pgfpathlineto{\pgfqpoint{2.919116in}{1.697596in}}%
\pgfpathlineto{\pgfqpoint{2.951420in}{1.680660in}}%
\pgfpathlineto{\pgfqpoint{2.983724in}{1.694824in}}%
\pgfpathlineto{\pgfqpoint{3.016028in}{1.690186in}}%
\pgfpathlineto{\pgfqpoint{3.048332in}{1.723001in}}%
\pgfpathlineto{\pgfqpoint{3.080636in}{1.713498in}}%
\pgfpathlineto{\pgfqpoint{3.112940in}{1.724855in}}%
\pgfpathlineto{\pgfqpoint{3.177547in}{1.746763in}}%
\pgfpathlineto{\pgfqpoint{3.209851in}{1.741237in}}%
\pgfpathlineto{\pgfqpoint{3.242155in}{1.752599in}}%
\pgfpathlineto{\pgfqpoint{3.306763in}{1.764658in}}%
\pgfpathlineto{\pgfqpoint{3.339067in}{1.789765in}}%
\pgfpathlineto{\pgfqpoint{3.371371in}{1.779166in}}%
\pgfpathlineto{\pgfqpoint{3.403675in}{1.766259in}}%
\pgfpathlineto{\pgfqpoint{3.435979in}{1.789518in}}%
\pgfpathlineto{\pgfqpoint{3.468283in}{1.821526in}}%
\pgfpathlineto{\pgfqpoint{3.500587in}{1.808751in}}%
\pgfpathlineto{\pgfqpoint{3.565195in}{1.823554in}}%
\pgfpathlineto{\pgfqpoint{3.597498in}{1.826818in}}%
\pgfpathlineto{\pgfqpoint{3.629802in}{1.833664in}}%
\pgfpathlineto{\pgfqpoint{3.662106in}{1.878393in}}%
\pgfpathlineto{\pgfqpoint{3.694410in}{1.839103in}}%
\pgfpathlineto{\pgfqpoint{3.726714in}{1.836758in}}%
\pgfpathlineto{\pgfqpoint{3.759018in}{1.854120in}}%
\pgfpathlineto{\pgfqpoint{3.823626in}{1.845581in}}%
\pgfpathlineto{\pgfqpoint{3.855930in}{1.890752in}}%
\pgfpathlineto{\pgfqpoint{3.888234in}{1.869920in}}%
\pgfpathlineto{\pgfqpoint{3.952842in}{1.898505in}}%
\pgfpathlineto{\pgfqpoint{4.017450in}{1.904131in}}%
\pgfpathlineto{\pgfqpoint{4.049754in}{1.887993in}}%
\pgfpathlineto{\pgfqpoint{4.082057in}{2.026177in}}%
\pgfusepath{stroke}%
\end{pgfscope}%
\begin{pgfscope}%
\pgfpathrectangle{\pgfqpoint{0.588387in}{0.521603in}}{\pgfqpoint{3.660036in}{2.220246in}}%
\pgfusepath{clip}%
\pgfsetrectcap%
\pgfsetroundjoin%
\pgfsetlinewidth{1.505625pt}%
\pgfsetstrokecolor{currentstroke7}%
\pgfsetdash{}{0pt}%
\pgfpathmoveto{\pgfqpoint{0.754752in}{0.824866in}}%
\pgfpathlineto{\pgfqpoint{0.787056in}{0.837128in}}%
\pgfpathlineto{\pgfqpoint{0.819360in}{0.843565in}}%
\pgfpathlineto{\pgfqpoint{0.851664in}{0.761477in}}%
\pgfpathlineto{\pgfqpoint{0.883968in}{0.711547in}}%
\pgfpathlineto{\pgfqpoint{0.916272in}{0.696153in}}%
\pgfpathlineto{\pgfqpoint{0.948576in}{0.740637in}}%
\pgfpathlineto{\pgfqpoint{0.980880in}{0.773775in}}%
\pgfpathlineto{\pgfqpoint{1.013184in}{0.831680in}}%
\pgfpathlineto{\pgfqpoint{1.045488in}{0.834389in}}%
\pgfpathlineto{\pgfqpoint{1.077792in}{0.949186in}}%
\pgfpathlineto{\pgfqpoint{1.110096in}{0.950507in}}%
\pgfpathlineto{\pgfqpoint{1.142400in}{1.021611in}}%
\pgfpathlineto{\pgfqpoint{1.174704in}{1.051670in}}%
\pgfpathlineto{\pgfqpoint{1.207008in}{1.143570in}}%
\pgfpathlineto{\pgfqpoint{1.239311in}{1.083156in}}%
\pgfpathlineto{\pgfqpoint{1.271615in}{1.116120in}}%
\pgfpathlineto{\pgfqpoint{1.303919in}{1.138844in}}%
\pgfpathlineto{\pgfqpoint{1.336223in}{1.308053in}}%
\pgfpathlineto{\pgfqpoint{1.368527in}{1.209990in}}%
\pgfpathlineto{\pgfqpoint{1.400831in}{1.167833in}}%
\pgfpathlineto{\pgfqpoint{1.433135in}{1.293903in}}%
\pgfpathlineto{\pgfqpoint{1.465439in}{1.537001in}}%
\pgfpathlineto{\pgfqpoint{1.497743in}{1.365027in}}%
\pgfpathlineto{\pgfqpoint{1.530047in}{1.454894in}}%
\pgfpathlineto{\pgfqpoint{1.562351in}{1.601586in}}%
\pgfpathlineto{\pgfqpoint{1.594655in}{1.560546in}}%
\pgfpathlineto{\pgfqpoint{1.626959in}{1.464073in}}%
\pgfpathlineto{\pgfqpoint{1.659263in}{2.109553in}}%
\pgfpathlineto{\pgfqpoint{1.691567in}{1.816105in}}%
\pgfpathlineto{\pgfqpoint{1.723870in}{2.078192in}}%
\pgfpathlineto{\pgfqpoint{1.756174in}{1.790175in}}%
\pgfpathlineto{\pgfqpoint{1.788478in}{1.721931in}}%
\pgfpathlineto{\pgfqpoint{1.820782in}{1.476386in}}%
\pgfpathlineto{\pgfqpoint{1.853086in}{2.144913in}}%
\pgfpathlineto{\pgfqpoint{1.885390in}{1.837003in}}%
\pgfpathlineto{\pgfqpoint{1.917694in}{1.929145in}}%
\pgfpathlineto{\pgfqpoint{1.949998in}{1.993401in}}%
\pgfpathlineto{\pgfqpoint{1.982302in}{2.199086in}}%
\pgfpathlineto{\pgfqpoint{2.014606in}{1.885711in}}%
\pgfpathlineto{\pgfqpoint{2.046910in}{1.877817in}}%
\pgfpathlineto{\pgfqpoint{2.079214in}{1.954293in}}%
\pgfpathlineto{\pgfqpoint{2.111518in}{2.224163in}}%
\pgfpathlineto{\pgfqpoint{2.143822in}{1.932605in}}%
\pgfpathlineto{\pgfqpoint{2.176125in}{1.561278in}}%
\pgfpathlineto{\pgfqpoint{2.208429in}{1.949793in}}%
\pgfpathlineto{\pgfqpoint{2.240733in}{2.234606in}}%
\pgfpathlineto{\pgfqpoint{2.273037in}{2.033895in}}%
\pgfpathlineto{\pgfqpoint{2.305341in}{1.621820in}}%
\pgfpathlineto{\pgfqpoint{2.337645in}{2.137223in}}%
\pgfpathlineto{\pgfqpoint{2.369949in}{2.115313in}}%
\pgfpathlineto{\pgfqpoint{2.402253in}{2.249430in}}%
\pgfpathlineto{\pgfqpoint{2.434557in}{2.331935in}}%
\pgfpathlineto{\pgfqpoint{2.466861in}{2.012228in}}%
\pgfpathlineto{\pgfqpoint{2.499165in}{1.742872in}}%
\pgfpathlineto{\pgfqpoint{2.531469in}{2.200850in}}%
\pgfpathlineto{\pgfqpoint{2.563773in}{1.998631in}}%
\pgfpathlineto{\pgfqpoint{2.596077in}{2.147648in}}%
\pgfpathlineto{\pgfqpoint{2.628381in}{2.201659in}}%
\pgfpathlineto{\pgfqpoint{2.660684in}{2.156863in}}%
\pgfpathlineto{\pgfqpoint{2.692988in}{2.326814in}}%
\pgfpathlineto{\pgfqpoint{2.725292in}{2.041704in}}%
\pgfpathlineto{\pgfqpoint{2.757596in}{2.516132in}}%
\pgfpathlineto{\pgfqpoint{2.789900in}{2.308190in}}%
\pgfpathlineto{\pgfqpoint{2.822204in}{1.847957in}}%
\pgfpathlineto{\pgfqpoint{2.854508in}{2.197744in}}%
\pgfpathlineto{\pgfqpoint{2.886812in}{1.792959in}}%
\pgfpathlineto{\pgfqpoint{2.919116in}{2.330050in}}%
\pgfpathlineto{\pgfqpoint{2.951420in}{1.923902in}}%
\pgfpathlineto{\pgfqpoint{2.983724in}{2.111318in}}%
\pgfpathlineto{\pgfqpoint{3.016028in}{1.931227in}}%
\pgfpathlineto{\pgfqpoint{3.048332in}{2.306517in}}%
\pgfpathlineto{\pgfqpoint{3.080636in}{2.060712in}}%
\pgfpathlineto{\pgfqpoint{3.112940in}{2.268216in}}%
\pgfpathlineto{\pgfqpoint{3.177547in}{2.131186in}}%
\pgfpathlineto{\pgfqpoint{3.209851in}{2.308023in}}%
\pgfpathlineto{\pgfqpoint{3.242155in}{1.810490in}}%
\pgfpathlineto{\pgfqpoint{3.306763in}{2.373649in}}%
\pgfpathlineto{\pgfqpoint{3.339067in}{1.887993in}}%
\pgfpathlineto{\pgfqpoint{3.371371in}{2.241039in}}%
\pgfpathlineto{\pgfqpoint{3.403675in}{2.362823in}}%
\pgfpathlineto{\pgfqpoint{3.435979in}{2.177134in}}%
\pgfpathlineto{\pgfqpoint{3.468283in}{1.867235in}}%
\pgfpathlineto{\pgfqpoint{3.500587in}{1.855900in}}%
\pgfpathlineto{\pgfqpoint{3.565195in}{1.891300in}}%
\pgfpathlineto{\pgfqpoint{3.597498in}{2.111947in}}%
\pgfpathlineto{\pgfqpoint{3.629802in}{2.070642in}}%
\pgfpathlineto{\pgfqpoint{3.662106in}{1.988746in}}%
\pgfpathlineto{\pgfqpoint{3.694410in}{2.115380in}}%
\pgfpathlineto{\pgfqpoint{3.726714in}{2.521676in}}%
\pgfpathlineto{\pgfqpoint{3.759018in}{2.272260in}}%
\pgfpathlineto{\pgfqpoint{3.823626in}{2.099381in}}%
\pgfpathlineto{\pgfqpoint{3.855930in}{1.982219in}}%
\pgfpathlineto{\pgfqpoint{3.888234in}{1.979091in}}%
\pgfpathlineto{\pgfqpoint{3.952842in}{2.456779in}}%
\pgfpathlineto{\pgfqpoint{4.017450in}{2.007416in}}%
\pgfpathlineto{\pgfqpoint{4.049754in}{1.995116in}}%
\pgfpathlineto{\pgfqpoint{4.082057in}{1.946929in}}%
\pgfusepath{stroke}%
\end{pgfscope}%
\begin{pgfscope}%
\pgfpathrectangle{\pgfqpoint{0.588387in}{0.521603in}}{\pgfqpoint{3.660036in}{2.220246in}}%
\pgfusepath{clip}%
\pgfsetrectcap%
\pgfsetroundjoin%
\pgfsetlinewidth{1.505625pt}%
\definecolor{currentstroke}{rgb}{0.498039,0.498039,0.498039}%
\pgfsetstrokecolor{currentstroke}%
\pgfsetdash{}{0pt}%
\pgfpathmoveto{\pgfqpoint{0.754752in}{0.824866in}}%
\pgfpathlineto{\pgfqpoint{0.787056in}{0.850277in}}%
\pgfpathlineto{\pgfqpoint{0.819360in}{0.851982in}}%
\pgfpathlineto{\pgfqpoint{0.851664in}{0.778379in}}%
\pgfpathlineto{\pgfqpoint{0.883968in}{0.720309in}}%
\pgfpathlineto{\pgfqpoint{0.916272in}{0.699447in}}%
\pgfpathlineto{\pgfqpoint{0.948576in}{0.745760in}}%
\pgfpathlineto{\pgfqpoint{0.980880in}{0.765123in}}%
\pgfpathlineto{\pgfqpoint{1.013184in}{0.820437in}}%
\pgfpathlineto{\pgfqpoint{1.045488in}{0.826533in}}%
\pgfpathlineto{\pgfqpoint{1.077792in}{0.876206in}}%
\pgfpathlineto{\pgfqpoint{1.110096in}{0.893098in}}%
\pgfpathlineto{\pgfqpoint{1.142400in}{0.944850in}}%
\pgfpathlineto{\pgfqpoint{1.174704in}{0.953056in}}%
\pgfpathlineto{\pgfqpoint{1.207008in}{0.980076in}}%
\pgfpathlineto{\pgfqpoint{1.239311in}{0.993953in}}%
\pgfpathlineto{\pgfqpoint{1.271615in}{1.031968in}}%
\pgfpathlineto{\pgfqpoint{1.303919in}{1.051114in}}%
\pgfpathlineto{\pgfqpoint{1.336223in}{1.072392in}}%
\pgfpathlineto{\pgfqpoint{1.368527in}{1.078474in}}%
\pgfpathlineto{\pgfqpoint{1.400831in}{1.111319in}}%
\pgfpathlineto{\pgfqpoint{1.433135in}{1.133285in}}%
\pgfpathlineto{\pgfqpoint{1.465439in}{1.143925in}}%
\pgfpathlineto{\pgfqpoint{1.497743in}{1.154860in}}%
\pgfpathlineto{\pgfqpoint{1.530047in}{1.188698in}}%
\pgfpathlineto{\pgfqpoint{1.562351in}{1.205720in}}%
\pgfpathlineto{\pgfqpoint{1.594655in}{1.219252in}}%
\pgfpathlineto{\pgfqpoint{1.626959in}{1.223341in}}%
\pgfpathlineto{\pgfqpoint{1.659263in}{1.242206in}}%
\pgfpathlineto{\pgfqpoint{1.691567in}{1.263966in}}%
\pgfpathlineto{\pgfqpoint{1.723870in}{1.273549in}}%
\pgfpathlineto{\pgfqpoint{1.756174in}{1.281466in}}%
\pgfpathlineto{\pgfqpoint{1.788478in}{1.301152in}}%
\pgfpathlineto{\pgfqpoint{1.820782in}{1.312517in}}%
\pgfpathlineto{\pgfqpoint{1.853086in}{1.313601in}}%
\pgfpathlineto{\pgfqpoint{1.885390in}{1.327163in}}%
\pgfpathlineto{\pgfqpoint{1.917694in}{1.343068in}}%
\pgfpathlineto{\pgfqpoint{1.949998in}{1.360008in}}%
\pgfpathlineto{\pgfqpoint{1.982302in}{1.372666in}}%
\pgfpathlineto{\pgfqpoint{2.014606in}{1.375339in}}%
\pgfpathlineto{\pgfqpoint{2.046910in}{1.387489in}}%
\pgfpathlineto{\pgfqpoint{2.079214in}{1.398206in}}%
\pgfpathlineto{\pgfqpoint{2.111518in}{1.411589in}}%
\pgfpathlineto{\pgfqpoint{2.143822in}{1.412757in}}%
\pgfpathlineto{\pgfqpoint{2.176125in}{1.466218in}}%
\pgfpathlineto{\pgfqpoint{2.208429in}{1.447771in}}%
\pgfpathlineto{\pgfqpoint{2.240733in}{1.458527in}}%
\pgfpathlineto{\pgfqpoint{2.273037in}{1.474566in}}%
\pgfpathlineto{\pgfqpoint{2.305341in}{1.469429in}}%
\pgfpathlineto{\pgfqpoint{2.337645in}{1.482566in}}%
\pgfpathlineto{\pgfqpoint{2.369949in}{1.491181in}}%
\pgfpathlineto{\pgfqpoint{2.402253in}{1.489460in}}%
\pgfpathlineto{\pgfqpoint{2.434557in}{1.509835in}}%
\pgfpathlineto{\pgfqpoint{2.466861in}{1.510196in}}%
\pgfpathlineto{\pgfqpoint{2.499165in}{1.510296in}}%
\pgfpathlineto{\pgfqpoint{2.531469in}{1.530823in}}%
\pgfpathlineto{\pgfqpoint{2.563773in}{1.542799in}}%
\pgfpathlineto{\pgfqpoint{2.596077in}{1.555607in}}%
\pgfpathlineto{\pgfqpoint{2.628381in}{1.551328in}}%
\pgfpathlineto{\pgfqpoint{2.660684in}{1.551250in}}%
\pgfpathlineto{\pgfqpoint{2.692988in}{1.566207in}}%
\pgfpathlineto{\pgfqpoint{2.725292in}{1.591512in}}%
\pgfpathlineto{\pgfqpoint{2.757596in}{1.627595in}}%
\pgfpathlineto{\pgfqpoint{2.789900in}{1.604608in}}%
\pgfpathlineto{\pgfqpoint{2.822204in}{1.599498in}}%
\pgfpathlineto{\pgfqpoint{2.854508in}{1.612942in}}%
\pgfpathlineto{\pgfqpoint{2.886812in}{1.604393in}}%
\pgfpathlineto{\pgfqpoint{2.919116in}{1.617020in}}%
\pgfpathlineto{\pgfqpoint{2.951420in}{1.623114in}}%
\pgfpathlineto{\pgfqpoint{2.983724in}{1.628679in}}%
\pgfpathlineto{\pgfqpoint{3.016028in}{1.614956in}}%
\pgfpathlineto{\pgfqpoint{3.048332in}{1.653087in}}%
\pgfpathlineto{\pgfqpoint{3.080636in}{1.642707in}}%
\pgfpathlineto{\pgfqpoint{3.112940in}{1.657225in}}%
\pgfpathlineto{\pgfqpoint{3.177547in}{1.686631in}}%
\pgfpathlineto{\pgfqpoint{3.209851in}{1.678388in}}%
\pgfpathlineto{\pgfqpoint{3.242155in}{1.686061in}}%
\pgfpathlineto{\pgfqpoint{3.306763in}{1.690470in}}%
\pgfpathlineto{\pgfqpoint{3.339067in}{1.724513in}}%
\pgfpathlineto{\pgfqpoint{3.371371in}{1.731676in}}%
\pgfpathlineto{\pgfqpoint{3.403675in}{1.703121in}}%
\pgfpathlineto{\pgfqpoint{3.435979in}{1.744746in}}%
\pgfpathlineto{\pgfqpoint{3.468283in}{1.736565in}}%
\pgfpathlineto{\pgfqpoint{3.500587in}{1.787749in}}%
\pgfpathlineto{\pgfqpoint{3.565195in}{1.780896in}}%
\pgfpathlineto{\pgfqpoint{3.597498in}{1.767572in}}%
\pgfpathlineto{\pgfqpoint{3.629802in}{1.792959in}}%
\pgfpathlineto{\pgfqpoint{3.662106in}{1.774895in}}%
\pgfpathlineto{\pgfqpoint{3.694410in}{1.750434in}}%
\pgfpathlineto{\pgfqpoint{3.726714in}{1.789765in}}%
\pgfpathlineto{\pgfqpoint{3.759018in}{1.783259in}}%
\pgfpathlineto{\pgfqpoint{3.823626in}{1.780777in}}%
\pgfpathlineto{\pgfqpoint{3.855930in}{1.820097in}}%
\pgfpathlineto{\pgfqpoint{3.888234in}{1.804985in}}%
\pgfpathlineto{\pgfqpoint{3.952842in}{1.801806in}}%
\pgfpathlineto{\pgfqpoint{4.017450in}{1.835409in}}%
\pgfpathlineto{\pgfqpoint{4.049754in}{1.809871in}}%
\pgfpathlineto{\pgfqpoint{4.082057in}{1.866033in}}%
\pgfusepath{stroke}%
\end{pgfscope}%
\begin{pgfscope}%
\pgfpathrectangle{\pgfqpoint{0.588387in}{0.521603in}}{\pgfqpoint{3.660036in}{2.220246in}}%
\pgfusepath{clip}%
\pgfsetrectcap%
\pgfsetroundjoin%
\pgfsetlinewidth{1.505625pt}%
\definecolor{currentstroke}{rgb}{0.737255,0.741176,0.133333}%
\pgfsetstrokecolor{currentstroke}%
\pgfsetdash{}{0pt}%
\pgfpathmoveto{\pgfqpoint{0.754752in}{0.824866in}}%
\pgfpathlineto{\pgfqpoint{0.787056in}{0.850277in}}%
\pgfpathlineto{\pgfqpoint{0.819360in}{0.854297in}}%
\pgfpathlineto{\pgfqpoint{0.851664in}{0.775714in}}%
\pgfpathlineto{\pgfqpoint{0.883968in}{0.723033in}}%
\pgfpathlineto{\pgfqpoint{0.916272in}{0.703040in}}%
\pgfpathlineto{\pgfqpoint{0.948576in}{0.731426in}}%
\pgfpathlineto{\pgfqpoint{0.980880in}{0.762598in}}%
\pgfpathlineto{\pgfqpoint{1.013184in}{0.814632in}}%
\pgfpathlineto{\pgfqpoint{1.045488in}{0.818437in}}%
\pgfpathlineto{\pgfqpoint{1.077792in}{0.907016in}}%
\pgfpathlineto{\pgfqpoint{1.110096in}{0.892475in}}%
\pgfpathlineto{\pgfqpoint{1.142400in}{1.003743in}}%
\pgfpathlineto{\pgfqpoint{1.174704in}{1.036814in}}%
\pgfpathlineto{\pgfqpoint{1.207008in}{1.188089in}}%
\pgfpathlineto{\pgfqpoint{1.239311in}{1.097829in}}%
\pgfpathlineto{\pgfqpoint{1.271615in}{1.150563in}}%
\pgfpathlineto{\pgfqpoint{1.303919in}{1.263695in}}%
\pgfpathlineto{\pgfqpoint{1.336223in}{1.493533in}}%
\pgfpathlineto{\pgfqpoint{1.368527in}{1.247916in}}%
\pgfpathlineto{\pgfqpoint{1.400831in}{1.580501in}}%
\pgfpathlineto{\pgfqpoint{1.433135in}{1.568470in}}%
\pgfpathlineto{\pgfqpoint{1.465439in}{1.694417in}}%
\pgfpathlineto{\pgfqpoint{1.497743in}{1.744046in}}%
\pgfpathlineto{\pgfqpoint{1.530047in}{1.758464in}}%
\pgfpathlineto{\pgfqpoint{1.562351in}{1.813301in}}%
\pgfpathlineto{\pgfqpoint{1.594655in}{2.240194in}}%
\pgfpathlineto{\pgfqpoint{1.626959in}{2.026375in}}%
\pgfpathlineto{\pgfqpoint{1.659263in}{2.065700in}}%
\pgfpathlineto{\pgfqpoint{1.691567in}{2.089900in}}%
\pgfpathlineto{\pgfqpoint{1.723870in}{2.251979in}}%
\pgfpathlineto{\pgfqpoint{1.756174in}{2.143899in}}%
\pgfpathlineto{\pgfqpoint{1.788478in}{2.368937in}}%
\pgfpathlineto{\pgfqpoint{1.820782in}{2.240175in}}%
\pgfpathlineto{\pgfqpoint{1.853086in}{2.433407in}}%
\pgfpathlineto{\pgfqpoint{1.885390in}{2.189916in}}%
\pgfpathlineto{\pgfqpoint{1.917694in}{2.369437in}}%
\pgfpathlineto{\pgfqpoint{1.949998in}{2.321730in}}%
\pgfpathlineto{\pgfqpoint{1.982302in}{2.325006in}}%
\pgfpathlineto{\pgfqpoint{2.014606in}{2.387531in}}%
\pgfpathlineto{\pgfqpoint{2.046910in}{2.487543in}}%
\pgfpathlineto{\pgfqpoint{2.079214in}{2.321869in}}%
\pgfpathlineto{\pgfqpoint{2.111518in}{2.432890in}}%
\pgfpathlineto{\pgfqpoint{2.143822in}{2.486193in}}%
\pgfpathlineto{\pgfqpoint{2.176125in}{2.414449in}}%
\pgfpathlineto{\pgfqpoint{2.208429in}{2.450961in}}%
\pgfpathlineto{\pgfqpoint{2.240733in}{2.475652in}}%
\pgfpathlineto{\pgfqpoint{2.273037in}{2.412906in}}%
\pgfpathlineto{\pgfqpoint{2.305341in}{2.455487in}}%
\pgfpathlineto{\pgfqpoint{2.337645in}{2.435830in}}%
\pgfpathlineto{\pgfqpoint{2.369949in}{2.592830in}}%
\pgfpathlineto{\pgfqpoint{2.402253in}{2.524962in}}%
\pgfpathlineto{\pgfqpoint{2.434557in}{2.560898in}}%
\pgfpathlineto{\pgfqpoint{2.466861in}{2.392472in}}%
\pgfpathlineto{\pgfqpoint{2.499165in}{2.516650in}}%
\pgfpathlineto{\pgfqpoint{2.531469in}{2.400795in}}%
\pgfpathlineto{\pgfqpoint{2.563773in}{2.212741in}}%
\pgfpathlineto{\pgfqpoint{2.596077in}{2.465453in}}%
\pgfpathlineto{\pgfqpoint{2.628381in}{2.573926in}}%
\pgfpathlineto{\pgfqpoint{2.660684in}{2.446226in}}%
\pgfpathlineto{\pgfqpoint{2.692988in}{2.565657in}}%
\pgfpathlineto{\pgfqpoint{2.725292in}{2.514907in}}%
\pgfpathlineto{\pgfqpoint{2.757596in}{2.640929in}}%
\pgfpathlineto{\pgfqpoint{2.789900in}{2.601472in}}%
\pgfpathlineto{\pgfqpoint{2.822204in}{2.515248in}}%
\pgfpathlineto{\pgfqpoint{2.854508in}{2.547484in}}%
\pgfpathlineto{\pgfqpoint{2.919116in}{2.581176in}}%
\pgfpathlineto{\pgfqpoint{2.983724in}{2.566820in}}%
\pgfpathlineto{\pgfqpoint{3.016028in}{2.589189in}}%
\pgfpathlineto{\pgfqpoint{3.048332in}{2.592290in}}%
\pgfpathlineto{\pgfqpoint{3.080636in}{2.435962in}}%
\pgfpathlineto{\pgfqpoint{3.112940in}{2.474487in}}%
\pgfpathlineto{\pgfqpoint{3.177547in}{2.583185in}}%
\pgfpathlineto{\pgfqpoint{3.209851in}{2.471166in}}%
\pgfpathlineto{\pgfqpoint{3.242155in}{2.542764in}}%
\pgfpathlineto{\pgfqpoint{3.306763in}{2.498019in}}%
\pgfpathlineto{\pgfqpoint{3.339067in}{2.611777in}}%
\pgfpathlineto{\pgfqpoint{3.371371in}{2.572663in}}%
\pgfpathlineto{\pgfqpoint{3.403675in}{2.517425in}}%
\pgfpathlineto{\pgfqpoint{3.435979in}{2.540858in}}%
\pgfpathlineto{\pgfqpoint{3.500587in}{2.621018in}}%
\pgfpathlineto{\pgfqpoint{3.565195in}{2.616438in}}%
\pgfpathlineto{\pgfqpoint{3.597498in}{2.591412in}}%
\pgfpathlineto{\pgfqpoint{3.629802in}{2.555906in}}%
\pgfpathlineto{\pgfqpoint{3.694410in}{2.371736in}}%
\pgfpathlineto{\pgfqpoint{3.726714in}{2.520403in}}%
\pgfpathlineto{\pgfqpoint{3.759018in}{2.434503in}}%
\pgfpathlineto{\pgfqpoint{3.823626in}{2.591477in}}%
\pgfpathlineto{\pgfqpoint{3.888234in}{2.362267in}}%
\pgfpathlineto{\pgfqpoint{3.952842in}{2.430082in}}%
\pgfpathlineto{\pgfqpoint{4.017450in}{1.868432in}}%
\pgfpathlineto{\pgfqpoint{4.049754in}{1.894567in}}%
\pgfusepath{stroke}%
\end{pgfscope}%
\begin{pgfscope}%
\pgfsetrectcap%
\pgfsetmiterjoin%
\pgfsetlinewidth{0.803000pt}%
\definecolor{currentstroke}{rgb}{0.000000,0.000000,0.000000}%
\pgfsetstrokecolor{currentstroke}%
\pgfsetdash{}{0pt}%
\pgfpathmoveto{\pgfqpoint{0.588387in}{0.521603in}}%
\pgfpathlineto{\pgfqpoint{0.588387in}{2.741849in}}%
\pgfusepath{stroke}%
\end{pgfscope}%
\begin{pgfscope}%
\pgfsetrectcap%
\pgfsetmiterjoin%
\pgfsetlinewidth{0.803000pt}%
\definecolor{currentstroke}{rgb}{0.000000,0.000000,0.000000}%
\pgfsetstrokecolor{currentstroke}%
\pgfsetdash{}{0pt}%
\pgfpathmoveto{\pgfqpoint{4.248423in}{0.521603in}}%
\pgfpathlineto{\pgfqpoint{4.248423in}{2.741849in}}%
\pgfusepath{stroke}%
\end{pgfscope}%
\begin{pgfscope}%
\pgfsetrectcap%
\pgfsetmiterjoin%
\pgfsetlinewidth{0.803000pt}%
\definecolor{currentstroke}{rgb}{0.000000,0.000000,0.000000}%
\pgfsetstrokecolor{currentstroke}%
\pgfsetdash{}{0pt}%
\pgfpathmoveto{\pgfqpoint{0.588387in}{0.521603in}}%
\pgfpathlineto{\pgfqpoint{4.248423in}{0.521603in}}%
\pgfusepath{stroke}%
\end{pgfscope}%
\begin{pgfscope}%
\pgfsetrectcap%
\pgfsetmiterjoin%
\pgfsetlinewidth{0.803000pt}%
\definecolor{currentstroke}{rgb}{0.000000,0.000000,0.000000}%
\pgfsetstrokecolor{currentstroke}%
\pgfsetdash{}{0pt}%
\pgfpathmoveto{\pgfqpoint{0.588387in}{2.741849in}}%
\pgfpathlineto{\pgfqpoint{4.248423in}{2.741849in}}%
\pgfusepath{stroke}%
\end{pgfscope}%
\begin{pgfscope}%
\pgfsetbuttcap%
\pgfsetmiterjoin%
\definecolor{currentfill}{rgb}{1.000000,1.000000,1.000000}%
\pgfsetfillcolor{currentfill}%
\pgfsetfillopacity{0.800000}%
\pgfsetlinewidth{1.003750pt}%
\definecolor{currentstroke}{rgb}{0.800000,0.800000,0.800000}%
\pgfsetstrokecolor{currentstroke}%
\pgfsetstrokeopacity{0.800000}%
\pgfsetdash{}{0pt}%
\pgfpathmoveto{\pgfqpoint{4.365089in}{0.379025in}}%
\pgfpathlineto{\pgfqpoint{8.251043in}{0.379025in}}%
\pgfpathquadraticcurveto{\pgfqpoint{8.284376in}{0.379025in}}{\pgfqpoint{8.284376in}{0.412359in}}%
\pgfpathlineto{\pgfqpoint{8.284376in}{2.625183in}}%
\pgfpathquadraticcurveto{\pgfqpoint{8.284376in}{2.658516in}}{\pgfqpoint{8.251043in}{2.658516in}}%
\pgfpathlineto{\pgfqpoint{4.365089in}{2.658516in}}%
\pgfpathquadraticcurveto{\pgfqpoint{4.331756in}{2.658516in}}{\pgfqpoint{4.331756in}{2.625183in}}%
\pgfpathlineto{\pgfqpoint{4.331756in}{0.412359in}}%
\pgfpathquadraticcurveto{\pgfqpoint{4.331756in}{0.379025in}}{\pgfqpoint{4.365089in}{0.379025in}}%
\pgfpathlineto{\pgfqpoint{4.365089in}{0.379025in}}%
\pgfpathclose%
\pgfusepath{stroke,fill}%
\end{pgfscope}%
\begin{pgfscope}%
\pgfsetrectcap%
\pgfsetroundjoin%
\pgfsetlinewidth{1.505625pt}%
\pgfsetstrokecolor{currentstroke3}%
\pgfsetdash{}{0pt}%
\pgfpathmoveto{\pgfqpoint{4.398423in}{2.523555in}}%
\pgfpathlineto{\pgfqpoint{4.565089in}{2.523555in}}%
\pgfpathlineto{\pgfqpoint{4.731756in}{2.523555in}}%
\pgfusepath{stroke}%
\end{pgfscope}%
\begin{pgfscope}%
\definecolor{textcolor}{rgb}{0.000000,0.000000,0.000000}%
\pgfsetstrokecolor{textcolor}%
\pgfsetfillcolor{textcolor}%
\pgftext[x=4.865089in,y=2.465222in,left,base]{\color{textcolor}{\rmfamily\fontsize{12.000000}{14.400000}\selectfont\catcode`\^=\active\def^{\ifmmode\sp\else\^{}\fi}\catcode`\%=\active\def%{\%}\NaiveCycles{}}}%
\end{pgfscope}%
\begin{pgfscope}%
\pgfsetrectcap%
\pgfsetroundjoin%
\pgfsetlinewidth{1.505625pt}%
\pgfsetstrokecolor{currentstroke1}%
\pgfsetdash{}{0pt}%
\pgfpathmoveto{\pgfqpoint{4.398423in}{2.278926in}}%
\pgfpathlineto{\pgfqpoint{4.565089in}{2.278926in}}%
\pgfpathlineto{\pgfqpoint{4.731756in}{2.278926in}}%
\pgfusepath{stroke}%
\end{pgfscope}%
\begin{pgfscope}%
\definecolor{textcolor}{rgb}{0.000000,0.000000,0.000000}%
\pgfsetstrokecolor{textcolor}%
\pgfsetfillcolor{textcolor}%
\pgftext[x=4.865089in,y=2.220593in,left,base]{\color{textcolor}{\rmfamily\fontsize{12.000000}{14.400000}\selectfont\catcode`\^=\active\def^{\ifmmode\sp\else\^{}\fi}\catcode`\%=\active\def%{\%}\CyclesMatchChunks{} \& \MergeLinear{}}}%
\end{pgfscope}%
\begin{pgfscope}%
\pgfsetrectcap%
\pgfsetroundjoin%
\pgfsetlinewidth{1.505625pt}%
\pgfsetstrokecolor{currentstroke2}%
\pgfsetdash{}{0pt}%
\pgfpathmoveto{\pgfqpoint{4.398423in}{2.029659in}}%
\pgfpathlineto{\pgfqpoint{4.565089in}{2.029659in}}%
\pgfpathlineto{\pgfqpoint{4.731756in}{2.029659in}}%
\pgfusepath{stroke}%
\end{pgfscope}%
\begin{pgfscope}%
\definecolor{textcolor}{rgb}{0.000000,0.000000,0.000000}%
\pgfsetstrokecolor{textcolor}%
\pgfsetfillcolor{textcolor}%
\pgftext[x=4.865089in,y=1.971325in,left,base]{\color{textcolor}{\rmfamily\fontsize{12.000000}{14.400000}\selectfont\catcode`\^=\active\def^{\ifmmode\sp\else\^{}\fi}\catcode`\%=\active\def%{\%}\CyclesMatchChunks{} \& \SharedVertices{}}}%
\end{pgfscope}%
\begin{pgfscope}%
\pgfsetrectcap%
\pgfsetroundjoin%
\pgfsetlinewidth{1.505625pt}%
\pgfsetstrokecolor{currentstroke4}%
\pgfsetdash{}{0pt}%
\pgfpathmoveto{\pgfqpoint{4.398423in}{1.780391in}}%
\pgfpathlineto{\pgfqpoint{4.565089in}{1.780391in}}%
\pgfpathlineto{\pgfqpoint{4.731756in}{1.780391in}}%
\pgfusepath{stroke}%
\end{pgfscope}%
\begin{pgfscope}%
\definecolor{textcolor}{rgb}{0.000000,0.000000,0.000000}%
\pgfsetstrokecolor{textcolor}%
\pgfsetfillcolor{textcolor}%
\pgftext[x=4.865089in,y=1.722058in,left,base]{\color{textcolor}{\rmfamily\fontsize{12.000000}{14.400000}\selectfont\catcode`\^=\active\def^{\ifmmode\sp\else\^{}\fi}\catcode`\%=\active\def%{\%}\Neighbors{} \& \MergeLinear{}}}%
\end{pgfscope}%
\begin{pgfscope}%
\pgfsetrectcap%
\pgfsetroundjoin%
\pgfsetlinewidth{1.505625pt}%
\pgfsetstrokecolor{currentstroke5}%
\pgfsetdash{}{0pt}%
\pgfpathmoveto{\pgfqpoint{4.398423in}{1.535763in}}%
\pgfpathlineto{\pgfqpoint{4.565089in}{1.535763in}}%
\pgfpathlineto{\pgfqpoint{4.731756in}{1.535763in}}%
\pgfusepath{stroke}%
\end{pgfscope}%
\begin{pgfscope}%
\definecolor{textcolor}{rgb}{0.000000,0.000000,0.000000}%
\pgfsetstrokecolor{textcolor}%
\pgfsetfillcolor{textcolor}%
\pgftext[x=4.865089in,y=1.477429in,left,base]{\color{textcolor}{\rmfamily\fontsize{12.000000}{14.400000}\selectfont\catcode`\^=\active\def^{\ifmmode\sp\else\^{}\fi}\catcode`\%=\active\def%{\%}\Neighbors{} \& \SharedVertices{}}}%
\end{pgfscope}%
\begin{pgfscope}%
\pgfsetrectcap%
\pgfsetroundjoin%
\pgfsetlinewidth{1.505625pt}%
\pgfsetstrokecolor{currentstroke6}%
\pgfsetdash{}{0pt}%
\pgfpathmoveto{\pgfqpoint{4.398423in}{1.286495in}}%
\pgfpathlineto{\pgfqpoint{4.565089in}{1.286495in}}%
\pgfpathlineto{\pgfqpoint{4.731756in}{1.286495in}}%
\pgfusepath{stroke}%
\end{pgfscope}%
\begin{pgfscope}%
\definecolor{textcolor}{rgb}{0.000000,0.000000,0.000000}%
\pgfsetstrokecolor{textcolor}%
\pgfsetfillcolor{textcolor}%
\pgftext[x=4.865089in,y=1.228162in,left,base]{\color{textcolor}{\rmfamily\fontsize{12.000000}{14.400000}\selectfont\catcode`\^=\active\def^{\ifmmode\sp\else\^{}\fi}\catcode`\%=\active\def%{\%}\NeighborsDegree{} \& \MergeLinear{}}}%
\end{pgfscope}%
\begin{pgfscope}%
\pgfsetrectcap%
\pgfsetroundjoin%
\pgfsetlinewidth{1.505625pt}%
\pgfsetstrokecolor{currentstroke7}%
\pgfsetdash{}{0pt}%
\pgfpathmoveto{\pgfqpoint{4.398423in}{1.037228in}}%
\pgfpathlineto{\pgfqpoint{4.565089in}{1.037228in}}%
\pgfpathlineto{\pgfqpoint{4.731756in}{1.037228in}}%
\pgfusepath{stroke}%
\end{pgfscope}%
\begin{pgfscope}%
\definecolor{textcolor}{rgb}{0.000000,0.000000,0.000000}%
\pgfsetstrokecolor{textcolor}%
\pgfsetfillcolor{textcolor}%
\pgftext[x=4.865089in,y=0.978895in,left,base]{\color{textcolor}{\rmfamily\fontsize{12.000000}{14.400000}\selectfont\catcode`\^=\active\def^{\ifmmode\sp\else\^{}\fi}\catcode`\%=\active\def%{\%}\NeighborsDegree{} \& \SharedVertices{}}}%
\end{pgfscope}%
\begin{pgfscope}%
\pgfsetrectcap%
\pgfsetroundjoin%
\pgfsetlinewidth{1.505625pt}%
\definecolor{currentstroke}{rgb}{0.498039,0.498039,0.498039}%
\pgfsetstrokecolor{currentstroke}%
\pgfsetdash{}{0pt}%
\pgfpathmoveto{\pgfqpoint{4.398423in}{0.787961in}}%
\pgfpathlineto{\pgfqpoint{4.565089in}{0.787961in}}%
\pgfpathlineto{\pgfqpoint{4.731756in}{0.787961in}}%
\pgfusepath{stroke}%
\end{pgfscope}%
\begin{pgfscope}%
\definecolor{textcolor}{rgb}{0.000000,0.000000,0.000000}%
\pgfsetstrokecolor{textcolor}%
\pgfsetfillcolor{textcolor}%
\pgftext[x=4.865089in,y=0.729627in,left,base]{\color{textcolor}{\rmfamily\fontsize{12.000000}{14.400000}\selectfont\catcode`\^=\active\def^{\ifmmode\sp\else\^{}\fi}\catcode`\%=\active\def%{\%}\None{} \& \MergeLinear{}}}%
\end{pgfscope}%
\begin{pgfscope}%
\pgfsetrectcap%
\pgfsetroundjoin%
\pgfsetlinewidth{1.505625pt}%
\definecolor{currentstroke}{rgb}{0.737255,0.741176,0.133333}%
\pgfsetstrokecolor{currentstroke}%
\pgfsetdash{}{0pt}%
\pgfpathmoveto{\pgfqpoint{4.398423in}{0.543332in}}%
\pgfpathlineto{\pgfqpoint{4.565089in}{0.543332in}}%
\pgfpathlineto{\pgfqpoint{4.731756in}{0.543332in}}%
\pgfusepath{stroke}%
\end{pgfscope}%
\begin{pgfscope}%
\definecolor{textcolor}{rgb}{0.000000,0.000000,0.000000}%
\pgfsetstrokecolor{textcolor}%
\pgfsetfillcolor{textcolor}%
\pgftext[x=4.865089in,y=0.484999in,left,base]{\color{textcolor}{\rmfamily\fontsize{12.000000}{14.400000}\selectfont\catcode`\^=\active\def^{\ifmmode\sp\else\^{}\fi}\catcode`\%=\active\def%{\%}\None{} \& \SharedVertices{}}}%
\end{pgfscope}%
\end{pgfpicture}%
\makeatother%
\endgroup%
}
	\caption[Mean runtime for minimally rigid graphs (some)]{
		Mean running time to find all NAC-colorings for minimally rigid graphs.}%
	\label{fig:graph_minimally_rigid_first_runtime}
\end{figure}%
\begin{figure}[thbp]
	\centering
	\scalebox{\BenchFigureScale}{%% Creator: Matplotlib, PGF backend
%%
%% To include the figure in your LaTeX document, write
%%   \input{<filename>.pgf}
%%
%% Make sure the required packages are loaded in your preamble
%%   \usepackage{pgf}
%%
%% Also ensure that all the required font packages are loaded; for instance,
%% the lmodern package is sometimes necessary when using math font.
%%   \usepackage{lmodern}
%%
%% Figures using additional raster images can only be included by \input if
%% they are in the same directory as the main LaTeX file. For loading figures
%% from other directories you can use the `import` package
%%   \usepackage{import}
%%
%% and then include the figures with
%%   \import{<path to file>}{<filename>.pgf}
%%
%% Matplotlib used the following preamble
%%   \def\mathdefault#1{#1}
%%   \everymath=\expandafter{\the\everymath\displaystyle}
%%   \IfFileExists{scrextend.sty}{
%%     \usepackage[fontsize=10.000000pt]{scrextend}
%%   }{
%%     \renewcommand{\normalsize}{\fontsize{10.000000}{12.000000}\selectfont}
%%     \normalsize
%%   }
%%   
%%   \ifdefined\pdftexversion\else  % non-pdftex case.
%%     \usepackage{fontspec}
%%     \setmainfont{DejaVuSans.ttf}[Path=\detokenize{/home/petr/Projects/PyRigi/.venv/lib/python3.12/site-packages/matplotlib/mpl-data/fonts/ttf/}]
%%     \setsansfont{DejaVuSans.ttf}[Path=\detokenize{/home/petr/Projects/PyRigi/.venv/lib/python3.12/site-packages/matplotlib/mpl-data/fonts/ttf/}]
%%     \setmonofont{DejaVuSansMono.ttf}[Path=\detokenize{/home/petr/Projects/PyRigi/.venv/lib/python3.12/site-packages/matplotlib/mpl-data/fonts/ttf/}]
%%   \fi
%%   \makeatletter\@ifpackageloaded{under\Score{}}{}{\usepackage[strings]{under\Score{}}}\makeatother
%%
\begingroup%
\makeatletter%
\begin{pgfpicture}%
\pgfpathrectangle{\pgfpointorigin}{\pgfqpoint{8.384376in}{2.841849in}}%
\pgfusepath{use as bounding box, clip}%
\begin{pgfscope}%
\pgfsetbuttcap%
\pgfsetmiterjoin%
\definecolor{currentfill}{rgb}{1.000000,1.000000,1.000000}%
\pgfsetfillcolor{currentfill}%
\pgfsetlinewidth{0.000000pt}%
\definecolor{currentstroke}{rgb}{1.000000,1.000000,1.000000}%
\pgfsetstrokecolor{currentstroke}%
\pgfsetdash{}{0pt}%
\pgfpathmoveto{\pgfqpoint{0.000000in}{0.000000in}}%
\pgfpathlineto{\pgfqpoint{8.384376in}{0.000000in}}%
\pgfpathlineto{\pgfqpoint{8.384376in}{2.841849in}}%
\pgfpathlineto{\pgfqpoint{0.000000in}{2.841849in}}%
\pgfpathlineto{\pgfqpoint{0.000000in}{0.000000in}}%
\pgfpathclose%
\pgfusepath{fill}%
\end{pgfscope}%
\begin{pgfscope}%
\pgfsetbuttcap%
\pgfsetmiterjoin%
\definecolor{currentfill}{rgb}{1.000000,1.000000,1.000000}%
\pgfsetfillcolor{currentfill}%
\pgfsetlinewidth{0.000000pt}%
\definecolor{currentstroke}{rgb}{0.000000,0.000000,0.000000}%
\pgfsetstrokecolor{currentstroke}%
\pgfsetstrokeopacity{0.000000}%
\pgfsetdash{}{0pt}%
\pgfpathmoveto{\pgfqpoint{0.588387in}{0.521603in}}%
\pgfpathlineto{\pgfqpoint{5.257411in}{0.521603in}}%
\pgfpathlineto{\pgfqpoint{5.257411in}{2.741849in}}%
\pgfpathlineto{\pgfqpoint{0.588387in}{2.741849in}}%
\pgfpathlineto{\pgfqpoint{0.588387in}{0.521603in}}%
\pgfpathclose%
\pgfusepath{fill}%
\end{pgfscope}%
\begin{pgfscope}%
\pgfsetbuttcap%
\pgfsetroundjoin%
\definecolor{currentfill}{rgb}{0.000000,0.000000,0.000000}%
\pgfsetfillcolor{currentfill}%
\pgfsetlinewidth{0.803000pt}%
\definecolor{currentstroke}{rgb}{0.000000,0.000000,0.000000}%
\pgfsetstrokecolor{currentstroke}%
\pgfsetdash{}{0pt}%
\pgfsys@defobject{currentmarker}{\pgfqpoint{0.000000in}{-0.048611in}}{\pgfqpoint{0.000000in}{0.000000in}}{%
\pgfpathmoveto{\pgfqpoint{0.000000in}{0.000000in}}%
\pgfpathlineto{\pgfqpoint{0.000000in}{-0.048611in}}%
\pgfusepath{stroke,fill}%
}%
\begin{pgfscope}%
\pgfsys@transformshift{0.718197in}{0.521603in}%
\pgfsys@useobject{currentmarker}{}%
\end{pgfscope}%
\end{pgfscope}%
\begin{pgfscope}%
\definecolor{textcolor}{rgb}{0.000000,0.000000,0.000000}%
\pgfsetstrokecolor{textcolor}%
\pgfsetfillcolor{textcolor}%
\pgftext[x=0.718197in,y=0.424381in,,top]{\color{textcolor}{\rmfamily\fontsize{10.000000}{12.000000}\selectfont\catcode`\^=\active\def^{\ifmmode\sp\else\^{}\fi}\catcode`\%=\active\def%{\%}$\mathdefault{0}$}}%
\end{pgfscope}%
\begin{pgfscope}%
\pgfsetbuttcap%
\pgfsetroundjoin%
\definecolor{currentfill}{rgb}{0.000000,0.000000,0.000000}%
\pgfsetfillcolor{currentfill}%
\pgfsetlinewidth{0.803000pt}%
\definecolor{currentstroke}{rgb}{0.000000,0.000000,0.000000}%
\pgfsetstrokecolor{currentstroke}%
\pgfsetdash{}{0pt}%
\pgfsys@defobject{currentmarker}{\pgfqpoint{0.000000in}{-0.048611in}}{\pgfqpoint{0.000000in}{0.000000in}}{%
\pgfpathmoveto{\pgfqpoint{0.000000in}{0.000000in}}%
\pgfpathlineto{\pgfqpoint{0.000000in}{-0.048611in}}%
\pgfusepath{stroke,fill}%
}%
\begin{pgfscope}%
\pgfsys@transformshift{1.336338in}{0.521603in}%
\pgfsys@useobject{currentmarker}{}%
\end{pgfscope}%
\end{pgfscope}%
\begin{pgfscope}%
\definecolor{textcolor}{rgb}{0.000000,0.000000,0.000000}%
\pgfsetstrokecolor{textcolor}%
\pgfsetfillcolor{textcolor}%
\pgftext[x=1.336338in,y=0.424381in,,top]{\color{textcolor}{\rmfamily\fontsize{10.000000}{12.000000}\selectfont\catcode`\^=\active\def^{\ifmmode\sp\else\^{}\fi}\catcode`\%=\active\def%{\%}$\mathdefault{15}$}}%
\end{pgfscope}%
\begin{pgfscope}%
\pgfsetbuttcap%
\pgfsetroundjoin%
\definecolor{currentfill}{rgb}{0.000000,0.000000,0.000000}%
\pgfsetfillcolor{currentfill}%
\pgfsetlinewidth{0.803000pt}%
\definecolor{currentstroke}{rgb}{0.000000,0.000000,0.000000}%
\pgfsetstrokecolor{currentstroke}%
\pgfsetdash{}{0pt}%
\pgfsys@defobject{currentmarker}{\pgfqpoint{0.000000in}{-0.048611in}}{\pgfqpoint{0.000000in}{0.000000in}}{%
\pgfpathmoveto{\pgfqpoint{0.000000in}{0.000000in}}%
\pgfpathlineto{\pgfqpoint{0.000000in}{-0.048611in}}%
\pgfusepath{stroke,fill}%
}%
\begin{pgfscope}%
\pgfsys@transformshift{1.954479in}{0.521603in}%
\pgfsys@useobject{currentmarker}{}%
\end{pgfscope}%
\end{pgfscope}%
\begin{pgfscope}%
\definecolor{textcolor}{rgb}{0.000000,0.000000,0.000000}%
\pgfsetstrokecolor{textcolor}%
\pgfsetfillcolor{textcolor}%
\pgftext[x=1.954479in,y=0.424381in,,top]{\color{textcolor}{\rmfamily\fontsize{10.000000}{12.000000}\selectfont\catcode`\^=\active\def^{\ifmmode\sp\else\^{}\fi}\catcode`\%=\active\def%{\%}$\mathdefault{30}$}}%
\end{pgfscope}%
\begin{pgfscope}%
\pgfsetbuttcap%
\pgfsetroundjoin%
\definecolor{currentfill}{rgb}{0.000000,0.000000,0.000000}%
\pgfsetfillcolor{currentfill}%
\pgfsetlinewidth{0.803000pt}%
\definecolor{currentstroke}{rgb}{0.000000,0.000000,0.000000}%
\pgfsetstrokecolor{currentstroke}%
\pgfsetdash{}{0pt}%
\pgfsys@defobject{currentmarker}{\pgfqpoint{0.000000in}{-0.048611in}}{\pgfqpoint{0.000000in}{0.000000in}}{%
\pgfpathmoveto{\pgfqpoint{0.000000in}{0.000000in}}%
\pgfpathlineto{\pgfqpoint{0.000000in}{-0.048611in}}%
\pgfusepath{stroke,fill}%
}%
\begin{pgfscope}%
\pgfsys@transformshift{2.572619in}{0.521603in}%
\pgfsys@useobject{currentmarker}{}%
\end{pgfscope}%
\end{pgfscope}%
\begin{pgfscope}%
\definecolor{textcolor}{rgb}{0.000000,0.000000,0.000000}%
\pgfsetstrokecolor{textcolor}%
\pgfsetfillcolor{textcolor}%
\pgftext[x=2.572619in,y=0.424381in,,top]{\color{textcolor}{\rmfamily\fontsize{10.000000}{12.000000}\selectfont\catcode`\^=\active\def^{\ifmmode\sp\else\^{}\fi}\catcode`\%=\active\def%{\%}$\mathdefault{45}$}}%
\end{pgfscope}%
\begin{pgfscope}%
\pgfsetbuttcap%
\pgfsetroundjoin%
\definecolor{currentfill}{rgb}{0.000000,0.000000,0.000000}%
\pgfsetfillcolor{currentfill}%
\pgfsetlinewidth{0.803000pt}%
\definecolor{currentstroke}{rgb}{0.000000,0.000000,0.000000}%
\pgfsetstrokecolor{currentstroke}%
\pgfsetdash{}{0pt}%
\pgfsys@defobject{currentmarker}{\pgfqpoint{0.000000in}{-0.048611in}}{\pgfqpoint{0.000000in}{0.000000in}}{%
\pgfpathmoveto{\pgfqpoint{0.000000in}{0.000000in}}%
\pgfpathlineto{\pgfqpoint{0.000000in}{-0.048611in}}%
\pgfusepath{stroke,fill}%
}%
\begin{pgfscope}%
\pgfsys@transformshift{3.190760in}{0.521603in}%
\pgfsys@useobject{currentmarker}{}%
\end{pgfscope}%
\end{pgfscope}%
\begin{pgfscope}%
\definecolor{textcolor}{rgb}{0.000000,0.000000,0.000000}%
\pgfsetstrokecolor{textcolor}%
\pgfsetfillcolor{textcolor}%
\pgftext[x=3.190760in,y=0.424381in,,top]{\color{textcolor}{\rmfamily\fontsize{10.000000}{12.000000}\selectfont\catcode`\^=\active\def^{\ifmmode\sp\else\^{}\fi}\catcode`\%=\active\def%{\%}$\mathdefault{60}$}}%
\end{pgfscope}%
\begin{pgfscope}%
\pgfsetbuttcap%
\pgfsetroundjoin%
\definecolor{currentfill}{rgb}{0.000000,0.000000,0.000000}%
\pgfsetfillcolor{currentfill}%
\pgfsetlinewidth{0.803000pt}%
\definecolor{currentstroke}{rgb}{0.000000,0.000000,0.000000}%
\pgfsetstrokecolor{currentstroke}%
\pgfsetdash{}{0pt}%
\pgfsys@defobject{currentmarker}{\pgfqpoint{0.000000in}{-0.048611in}}{\pgfqpoint{0.000000in}{0.000000in}}{%
\pgfpathmoveto{\pgfqpoint{0.000000in}{0.000000in}}%
\pgfpathlineto{\pgfqpoint{0.000000in}{-0.048611in}}%
\pgfusepath{stroke,fill}%
}%
\begin{pgfscope}%
\pgfsys@transformshift{3.808901in}{0.521603in}%
\pgfsys@useobject{currentmarker}{}%
\end{pgfscope}%
\end{pgfscope}%
\begin{pgfscope}%
\definecolor{textcolor}{rgb}{0.000000,0.000000,0.000000}%
\pgfsetstrokecolor{textcolor}%
\pgfsetfillcolor{textcolor}%
\pgftext[x=3.808901in,y=0.424381in,,top]{\color{textcolor}{\rmfamily\fontsize{10.000000}{12.000000}\selectfont\catcode`\^=\active\def^{\ifmmode\sp\else\^{}\fi}\catcode`\%=\active\def%{\%}$\mathdefault{75}$}}%
\end{pgfscope}%
\begin{pgfscope}%
\pgfsetbuttcap%
\pgfsetroundjoin%
\definecolor{currentfill}{rgb}{0.000000,0.000000,0.000000}%
\pgfsetfillcolor{currentfill}%
\pgfsetlinewidth{0.803000pt}%
\definecolor{currentstroke}{rgb}{0.000000,0.000000,0.000000}%
\pgfsetstrokecolor{currentstroke}%
\pgfsetdash{}{0pt}%
\pgfsys@defobject{currentmarker}{\pgfqpoint{0.000000in}{-0.048611in}}{\pgfqpoint{0.000000in}{0.000000in}}{%
\pgfpathmoveto{\pgfqpoint{0.000000in}{0.000000in}}%
\pgfpathlineto{\pgfqpoint{0.000000in}{-0.048611in}}%
\pgfusepath{stroke,fill}%
}%
\begin{pgfscope}%
\pgfsys@transformshift{4.427042in}{0.521603in}%
\pgfsys@useobject{currentmarker}{}%
\end{pgfscope}%
\end{pgfscope}%
\begin{pgfscope}%
\definecolor{textcolor}{rgb}{0.000000,0.000000,0.000000}%
\pgfsetstrokecolor{textcolor}%
\pgfsetfillcolor{textcolor}%
\pgftext[x=4.427042in,y=0.424381in,,top]{\color{textcolor}{\rmfamily\fontsize{10.000000}{12.000000}\selectfont\catcode`\^=\active\def^{\ifmmode\sp\else\^{}\fi}\catcode`\%=\active\def%{\%}$\mathdefault{90}$}}%
\end{pgfscope}%
\begin{pgfscope}%
\pgfsetbuttcap%
\pgfsetroundjoin%
\definecolor{currentfill}{rgb}{0.000000,0.000000,0.000000}%
\pgfsetfillcolor{currentfill}%
\pgfsetlinewidth{0.803000pt}%
\definecolor{currentstroke}{rgb}{0.000000,0.000000,0.000000}%
\pgfsetstrokecolor{currentstroke}%
\pgfsetdash{}{0pt}%
\pgfsys@defobject{currentmarker}{\pgfqpoint{0.000000in}{-0.048611in}}{\pgfqpoint{0.000000in}{0.000000in}}{%
\pgfpathmoveto{\pgfqpoint{0.000000in}{0.000000in}}%
\pgfpathlineto{\pgfqpoint{0.000000in}{-0.048611in}}%
\pgfusepath{stroke,fill}%
}%
\begin{pgfscope}%
\pgfsys@transformshift{5.045183in}{0.521603in}%
\pgfsys@useobject{currentmarker}{}%
\end{pgfscope}%
\end{pgfscope}%
\begin{pgfscope}%
\definecolor{textcolor}{rgb}{0.000000,0.000000,0.000000}%
\pgfsetstrokecolor{textcolor}%
\pgfsetfillcolor{textcolor}%
\pgftext[x=5.045183in,y=0.424381in,,top]{\color{textcolor}{\rmfamily\fontsize{10.000000}{12.000000}\selectfont\catcode`\^=\active\def^{\ifmmode\sp\else\^{}\fi}\catcode`\%=\active\def%{\%}$\mathdefault{105}$}}%
\end{pgfscope}%
\begin{pgfscope}%
\definecolor{textcolor}{rgb}{0.000000,0.000000,0.000000}%
\pgfsetstrokecolor{textcolor}%
\pgfsetfillcolor{textcolor}%
\pgftext[x=2.922899in,y=0.234413in,,top]{\color{textcolor}{\rmfamily\fontsize{10.000000}{12.000000}\selectfont\catcode`\^=\active\def^{\ifmmode\sp\else\^{}\fi}\catcode`\%=\active\def%{\%}Monochromatic classes}}%
\end{pgfscope}%
\begin{pgfscope}%
\pgfsetbuttcap%
\pgfsetroundjoin%
\definecolor{currentfill}{rgb}{0.000000,0.000000,0.000000}%
\pgfsetfillcolor{currentfill}%
\pgfsetlinewidth{0.803000pt}%
\definecolor{currentstroke}{rgb}{0.000000,0.000000,0.000000}%
\pgfsetstrokecolor{currentstroke}%
\pgfsetdash{}{0pt}%
\pgfsys@defobject{currentmarker}{\pgfqpoint{-0.048611in}{0.000000in}}{\pgfqpoint{-0.000000in}{0.000000in}}{%
\pgfpathmoveto{\pgfqpoint{-0.000000in}{0.000000in}}%
\pgfpathlineto{\pgfqpoint{-0.048611in}{0.000000in}}%
\pgfusepath{stroke,fill}%
}%
\begin{pgfscope}%
\pgfsys@transformshift{0.588387in}{0.622524in}%
\pgfsys@useobject{currentmarker}{}%
\end{pgfscope}%
\end{pgfscope}%
\begin{pgfscope}%
\definecolor{textcolor}{rgb}{0.000000,0.000000,0.000000}%
\pgfsetstrokecolor{textcolor}%
\pgfsetfillcolor{textcolor}%
\pgftext[x=0.289968in, y=0.569762in, left, base]{\color{textcolor}{\rmfamily\fontsize{10.000000}{12.000000}\selectfont\catcode`\^=\active\def^{\ifmmode\sp\else\^{}\fi}\catcode`\%=\active\def%{\%}$\mathdefault{10^{0}}$}}%
\end{pgfscope}%
\begin{pgfscope}%
\pgfsetbuttcap%
\pgfsetroundjoin%
\definecolor{currentfill}{rgb}{0.000000,0.000000,0.000000}%
\pgfsetfillcolor{currentfill}%
\pgfsetlinewidth{0.803000pt}%
\definecolor{currentstroke}{rgb}{0.000000,0.000000,0.000000}%
\pgfsetstrokecolor{currentstroke}%
\pgfsetdash{}{0pt}%
\pgfsys@defobject{currentmarker}{\pgfqpoint{-0.048611in}{0.000000in}}{\pgfqpoint{-0.000000in}{0.000000in}}{%
\pgfpathmoveto{\pgfqpoint{-0.000000in}{0.000000in}}%
\pgfpathlineto{\pgfqpoint{-0.048611in}{0.000000in}}%
\pgfusepath{stroke,fill}%
}%
\begin{pgfscope}%
\pgfsys@transformshift{0.588387in}{0.995960in}%
\pgfsys@useobject{currentmarker}{}%
\end{pgfscope}%
\end{pgfscope}%
\begin{pgfscope}%
\definecolor{textcolor}{rgb}{0.000000,0.000000,0.000000}%
\pgfsetstrokecolor{textcolor}%
\pgfsetfillcolor{textcolor}%
\pgftext[x=0.289968in, y=0.943198in, left, base]{\color{textcolor}{\rmfamily\fontsize{10.000000}{12.000000}\selectfont\catcode`\^=\active\def^{\ifmmode\sp\else\^{}\fi}\catcode`\%=\active\def%{\%}$\mathdefault{10^{1}}$}}%
\end{pgfscope}%
\begin{pgfscope}%
\pgfsetbuttcap%
\pgfsetroundjoin%
\definecolor{currentfill}{rgb}{0.000000,0.000000,0.000000}%
\pgfsetfillcolor{currentfill}%
\pgfsetlinewidth{0.803000pt}%
\definecolor{currentstroke}{rgb}{0.000000,0.000000,0.000000}%
\pgfsetstrokecolor{currentstroke}%
\pgfsetdash{}{0pt}%
\pgfsys@defobject{currentmarker}{\pgfqpoint{-0.048611in}{0.000000in}}{\pgfqpoint{-0.000000in}{0.000000in}}{%
\pgfpathmoveto{\pgfqpoint{-0.000000in}{0.000000in}}%
\pgfpathlineto{\pgfqpoint{-0.048611in}{0.000000in}}%
\pgfusepath{stroke,fill}%
}%
\begin{pgfscope}%
\pgfsys@transformshift{0.588387in}{1.369396in}%
\pgfsys@useobject{currentmarker}{}%
\end{pgfscope}%
\end{pgfscope}%
\begin{pgfscope}%
\definecolor{textcolor}{rgb}{0.000000,0.000000,0.000000}%
\pgfsetstrokecolor{textcolor}%
\pgfsetfillcolor{textcolor}%
\pgftext[x=0.289968in, y=1.316634in, left, base]{\color{textcolor}{\rmfamily\fontsize{10.000000}{12.000000}\selectfont\catcode`\^=\active\def^{\ifmmode\sp\else\^{}\fi}\catcode`\%=\active\def%{\%}$\mathdefault{10^{2}}$}}%
\end{pgfscope}%
\begin{pgfscope}%
\pgfsetbuttcap%
\pgfsetroundjoin%
\definecolor{currentfill}{rgb}{0.000000,0.000000,0.000000}%
\pgfsetfillcolor{currentfill}%
\pgfsetlinewidth{0.803000pt}%
\definecolor{currentstroke}{rgb}{0.000000,0.000000,0.000000}%
\pgfsetstrokecolor{currentstroke}%
\pgfsetdash{}{0pt}%
\pgfsys@defobject{currentmarker}{\pgfqpoint{-0.048611in}{0.000000in}}{\pgfqpoint{-0.000000in}{0.000000in}}{%
\pgfpathmoveto{\pgfqpoint{-0.000000in}{0.000000in}}%
\pgfpathlineto{\pgfqpoint{-0.048611in}{0.000000in}}%
\pgfusepath{stroke,fill}%
}%
\begin{pgfscope}%
\pgfsys@transformshift{0.588387in}{1.742832in}%
\pgfsys@useobject{currentmarker}{}%
\end{pgfscope}%
\end{pgfscope}%
\begin{pgfscope}%
\definecolor{textcolor}{rgb}{0.000000,0.000000,0.000000}%
\pgfsetstrokecolor{textcolor}%
\pgfsetfillcolor{textcolor}%
\pgftext[x=0.289968in, y=1.690071in, left, base]{\color{textcolor}{\rmfamily\fontsize{10.000000}{12.000000}\selectfont\catcode`\^=\active\def^{\ifmmode\sp\else\^{}\fi}\catcode`\%=\active\def%{\%}$\mathdefault{10^{3}}$}}%
\end{pgfscope}%
\begin{pgfscope}%
\pgfsetbuttcap%
\pgfsetroundjoin%
\definecolor{currentfill}{rgb}{0.000000,0.000000,0.000000}%
\pgfsetfillcolor{currentfill}%
\pgfsetlinewidth{0.803000pt}%
\definecolor{currentstroke}{rgb}{0.000000,0.000000,0.000000}%
\pgfsetstrokecolor{currentstroke}%
\pgfsetdash{}{0pt}%
\pgfsys@defobject{currentmarker}{\pgfqpoint{-0.048611in}{0.000000in}}{\pgfqpoint{-0.000000in}{0.000000in}}{%
\pgfpathmoveto{\pgfqpoint{-0.000000in}{0.000000in}}%
\pgfpathlineto{\pgfqpoint{-0.048611in}{0.000000in}}%
\pgfusepath{stroke,fill}%
}%
\begin{pgfscope}%
\pgfsys@transformshift{0.588387in}{2.116268in}%
\pgfsys@useobject{currentmarker}{}%
\end{pgfscope}%
\end{pgfscope}%
\begin{pgfscope}%
\definecolor{textcolor}{rgb}{0.000000,0.000000,0.000000}%
\pgfsetstrokecolor{textcolor}%
\pgfsetfillcolor{textcolor}%
\pgftext[x=0.289968in, y=2.063507in, left, base]{\color{textcolor}{\rmfamily\fontsize{10.000000}{12.000000}\selectfont\catcode`\^=\active\def^{\ifmmode\sp\else\^{}\fi}\catcode`\%=\active\def%{\%}$\mathdefault{10^{4}}$}}%
\end{pgfscope}%
\begin{pgfscope}%
\pgfsetbuttcap%
\pgfsetroundjoin%
\definecolor{currentfill}{rgb}{0.000000,0.000000,0.000000}%
\pgfsetfillcolor{currentfill}%
\pgfsetlinewidth{0.803000pt}%
\definecolor{currentstroke}{rgb}{0.000000,0.000000,0.000000}%
\pgfsetstrokecolor{currentstroke}%
\pgfsetdash{}{0pt}%
\pgfsys@defobject{currentmarker}{\pgfqpoint{-0.048611in}{0.000000in}}{\pgfqpoint{-0.000000in}{0.000000in}}{%
\pgfpathmoveto{\pgfqpoint{-0.000000in}{0.000000in}}%
\pgfpathlineto{\pgfqpoint{-0.048611in}{0.000000in}}%
\pgfusepath{stroke,fill}%
}%
\begin{pgfscope}%
\pgfsys@transformshift{0.588387in}{2.489705in}%
\pgfsys@useobject{currentmarker}{}%
\end{pgfscope}%
\end{pgfscope}%
\begin{pgfscope}%
\definecolor{textcolor}{rgb}{0.000000,0.000000,0.000000}%
\pgfsetstrokecolor{textcolor}%
\pgfsetfillcolor{textcolor}%
\pgftext[x=0.289968in, y=2.436943in, left, base]{\color{textcolor}{\rmfamily\fontsize{10.000000}{12.000000}\selectfont\catcode`\^=\active\def^{\ifmmode\sp\else\^{}\fi}\catcode`\%=\active\def%{\%}$\mathdefault{10^{5}}$}}%
\end{pgfscope}%
\begin{pgfscope}%
\pgfsetbuttcap%
\pgfsetroundjoin%
\definecolor{currentfill}{rgb}{0.000000,0.000000,0.000000}%
\pgfsetfillcolor{currentfill}%
\pgfsetlinewidth{0.602250pt}%
\definecolor{currentstroke}{rgb}{0.000000,0.000000,0.000000}%
\pgfsetstrokecolor{currentstroke}%
\pgfsetdash{}{0pt}%
\pgfsys@defobject{currentmarker}{\pgfqpoint{-0.027778in}{0.000000in}}{\pgfqpoint{-0.000000in}{0.000000in}}{%
\pgfpathmoveto{\pgfqpoint{-0.000000in}{0.000000in}}%
\pgfpathlineto{\pgfqpoint{-0.027778in}{0.000000in}}%
\pgfusepath{stroke,fill}%
}%
\begin{pgfscope}%
\pgfsys@transformshift{0.588387in}{0.539677in}%
\pgfsys@useobject{currentmarker}{}%
\end{pgfscope}%
\end{pgfscope}%
\begin{pgfscope}%
\pgfsetbuttcap%
\pgfsetroundjoin%
\definecolor{currentfill}{rgb}{0.000000,0.000000,0.000000}%
\pgfsetfillcolor{currentfill}%
\pgfsetlinewidth{0.602250pt}%
\definecolor{currentstroke}{rgb}{0.000000,0.000000,0.000000}%
\pgfsetstrokecolor{currentstroke}%
\pgfsetdash{}{0pt}%
\pgfsys@defobject{currentmarker}{\pgfqpoint{-0.027778in}{0.000000in}}{\pgfqpoint{-0.000000in}{0.000000in}}{%
\pgfpathmoveto{\pgfqpoint{-0.000000in}{0.000000in}}%
\pgfpathlineto{\pgfqpoint{-0.027778in}{0.000000in}}%
\pgfusepath{stroke,fill}%
}%
\begin{pgfscope}%
\pgfsys@transformshift{0.588387in}{0.564678in}%
\pgfsys@useobject{currentmarker}{}%
\end{pgfscope}%
\end{pgfscope}%
\begin{pgfscope}%
\pgfsetbuttcap%
\pgfsetroundjoin%
\definecolor{currentfill}{rgb}{0.000000,0.000000,0.000000}%
\pgfsetfillcolor{currentfill}%
\pgfsetlinewidth{0.602250pt}%
\definecolor{currentstroke}{rgb}{0.000000,0.000000,0.000000}%
\pgfsetstrokecolor{currentstroke}%
\pgfsetdash{}{0pt}%
\pgfsys@defobject{currentmarker}{\pgfqpoint{-0.027778in}{0.000000in}}{\pgfqpoint{-0.000000in}{0.000000in}}{%
\pgfpathmoveto{\pgfqpoint{-0.000000in}{0.000000in}}%
\pgfpathlineto{\pgfqpoint{-0.027778in}{0.000000in}}%
\pgfusepath{stroke,fill}%
}%
\begin{pgfscope}%
\pgfsys@transformshift{0.588387in}{0.586334in}%
\pgfsys@useobject{currentmarker}{}%
\end{pgfscope}%
\end{pgfscope}%
\begin{pgfscope}%
\pgfsetbuttcap%
\pgfsetroundjoin%
\definecolor{currentfill}{rgb}{0.000000,0.000000,0.000000}%
\pgfsetfillcolor{currentfill}%
\pgfsetlinewidth{0.602250pt}%
\definecolor{currentstroke}{rgb}{0.000000,0.000000,0.000000}%
\pgfsetstrokecolor{currentstroke}%
\pgfsetdash{}{0pt}%
\pgfsys@defobject{currentmarker}{\pgfqpoint{-0.027778in}{0.000000in}}{\pgfqpoint{-0.000000in}{0.000000in}}{%
\pgfpathmoveto{\pgfqpoint{-0.000000in}{0.000000in}}%
\pgfpathlineto{\pgfqpoint{-0.027778in}{0.000000in}}%
\pgfusepath{stroke,fill}%
}%
\begin{pgfscope}%
\pgfsys@transformshift{0.588387in}{0.605436in}%
\pgfsys@useobject{currentmarker}{}%
\end{pgfscope}%
\end{pgfscope}%
\begin{pgfscope}%
\pgfsetbuttcap%
\pgfsetroundjoin%
\definecolor{currentfill}{rgb}{0.000000,0.000000,0.000000}%
\pgfsetfillcolor{currentfill}%
\pgfsetlinewidth{0.602250pt}%
\definecolor{currentstroke}{rgb}{0.000000,0.000000,0.000000}%
\pgfsetstrokecolor{currentstroke}%
\pgfsetdash{}{0pt}%
\pgfsys@defobject{currentmarker}{\pgfqpoint{-0.027778in}{0.000000in}}{\pgfqpoint{-0.000000in}{0.000000in}}{%
\pgfpathmoveto{\pgfqpoint{-0.000000in}{0.000000in}}%
\pgfpathlineto{\pgfqpoint{-0.027778in}{0.000000in}}%
\pgfusepath{stroke,fill}%
}%
\begin{pgfscope}%
\pgfsys@transformshift{0.588387in}{0.734939in}%
\pgfsys@useobject{currentmarker}{}%
\end{pgfscope}%
\end{pgfscope}%
\begin{pgfscope}%
\pgfsetbuttcap%
\pgfsetroundjoin%
\definecolor{currentfill}{rgb}{0.000000,0.000000,0.000000}%
\pgfsetfillcolor{currentfill}%
\pgfsetlinewidth{0.602250pt}%
\definecolor{currentstroke}{rgb}{0.000000,0.000000,0.000000}%
\pgfsetstrokecolor{currentstroke}%
\pgfsetdash{}{0pt}%
\pgfsys@defobject{currentmarker}{\pgfqpoint{-0.027778in}{0.000000in}}{\pgfqpoint{-0.000000in}{0.000000in}}{%
\pgfpathmoveto{\pgfqpoint{-0.000000in}{0.000000in}}%
\pgfpathlineto{\pgfqpoint{-0.027778in}{0.000000in}}%
\pgfusepath{stroke,fill}%
}%
\begin{pgfscope}%
\pgfsys@transformshift{0.588387in}{0.800698in}%
\pgfsys@useobject{currentmarker}{}%
\end{pgfscope}%
\end{pgfscope}%
\begin{pgfscope}%
\pgfsetbuttcap%
\pgfsetroundjoin%
\definecolor{currentfill}{rgb}{0.000000,0.000000,0.000000}%
\pgfsetfillcolor{currentfill}%
\pgfsetlinewidth{0.602250pt}%
\definecolor{currentstroke}{rgb}{0.000000,0.000000,0.000000}%
\pgfsetstrokecolor{currentstroke}%
\pgfsetdash{}{0pt}%
\pgfsys@defobject{currentmarker}{\pgfqpoint{-0.027778in}{0.000000in}}{\pgfqpoint{-0.000000in}{0.000000in}}{%
\pgfpathmoveto{\pgfqpoint{-0.000000in}{0.000000in}}%
\pgfpathlineto{\pgfqpoint{-0.027778in}{0.000000in}}%
\pgfusepath{stroke,fill}%
}%
\begin{pgfscope}%
\pgfsys@transformshift{0.588387in}{0.847355in}%
\pgfsys@useobject{currentmarker}{}%
\end{pgfscope}%
\end{pgfscope}%
\begin{pgfscope}%
\pgfsetbuttcap%
\pgfsetroundjoin%
\definecolor{currentfill}{rgb}{0.000000,0.000000,0.000000}%
\pgfsetfillcolor{currentfill}%
\pgfsetlinewidth{0.602250pt}%
\definecolor{currentstroke}{rgb}{0.000000,0.000000,0.000000}%
\pgfsetstrokecolor{currentstroke}%
\pgfsetdash{}{0pt}%
\pgfsys@defobject{currentmarker}{\pgfqpoint{-0.027778in}{0.000000in}}{\pgfqpoint{-0.000000in}{0.000000in}}{%
\pgfpathmoveto{\pgfqpoint{-0.000000in}{0.000000in}}%
\pgfpathlineto{\pgfqpoint{-0.027778in}{0.000000in}}%
\pgfusepath{stroke,fill}%
}%
\begin{pgfscope}%
\pgfsys@transformshift{0.588387in}{0.883544in}%
\pgfsys@useobject{currentmarker}{}%
\end{pgfscope}%
\end{pgfscope}%
\begin{pgfscope}%
\pgfsetbuttcap%
\pgfsetroundjoin%
\definecolor{currentfill}{rgb}{0.000000,0.000000,0.000000}%
\pgfsetfillcolor{currentfill}%
\pgfsetlinewidth{0.602250pt}%
\definecolor{currentstroke}{rgb}{0.000000,0.000000,0.000000}%
\pgfsetstrokecolor{currentstroke}%
\pgfsetdash{}{0pt}%
\pgfsys@defobject{currentmarker}{\pgfqpoint{-0.027778in}{0.000000in}}{\pgfqpoint{-0.000000in}{0.000000in}}{%
\pgfpathmoveto{\pgfqpoint{-0.000000in}{0.000000in}}%
\pgfpathlineto{\pgfqpoint{-0.027778in}{0.000000in}}%
\pgfusepath{stroke,fill}%
}%
\begin{pgfscope}%
\pgfsys@transformshift{0.588387in}{0.913113in}%
\pgfsys@useobject{currentmarker}{}%
\end{pgfscope}%
\end{pgfscope}%
\begin{pgfscope}%
\pgfsetbuttcap%
\pgfsetroundjoin%
\definecolor{currentfill}{rgb}{0.000000,0.000000,0.000000}%
\pgfsetfillcolor{currentfill}%
\pgfsetlinewidth{0.602250pt}%
\definecolor{currentstroke}{rgb}{0.000000,0.000000,0.000000}%
\pgfsetstrokecolor{currentstroke}%
\pgfsetdash{}{0pt}%
\pgfsys@defobject{currentmarker}{\pgfqpoint{-0.027778in}{0.000000in}}{\pgfqpoint{-0.000000in}{0.000000in}}{%
\pgfpathmoveto{\pgfqpoint{-0.000000in}{0.000000in}}%
\pgfpathlineto{\pgfqpoint{-0.027778in}{0.000000in}}%
\pgfusepath{stroke,fill}%
}%
\begin{pgfscope}%
\pgfsys@transformshift{0.588387in}{0.938114in}%
\pgfsys@useobject{currentmarker}{}%
\end{pgfscope}%
\end{pgfscope}%
\begin{pgfscope}%
\pgfsetbuttcap%
\pgfsetroundjoin%
\definecolor{currentfill}{rgb}{0.000000,0.000000,0.000000}%
\pgfsetfillcolor{currentfill}%
\pgfsetlinewidth{0.602250pt}%
\definecolor{currentstroke}{rgb}{0.000000,0.000000,0.000000}%
\pgfsetstrokecolor{currentstroke}%
\pgfsetdash{}{0pt}%
\pgfsys@defobject{currentmarker}{\pgfqpoint{-0.027778in}{0.000000in}}{\pgfqpoint{-0.000000in}{0.000000in}}{%
\pgfpathmoveto{\pgfqpoint{-0.000000in}{0.000000in}}%
\pgfpathlineto{\pgfqpoint{-0.027778in}{0.000000in}}%
\pgfusepath{stroke,fill}%
}%
\begin{pgfscope}%
\pgfsys@transformshift{0.588387in}{0.959770in}%
\pgfsys@useobject{currentmarker}{}%
\end{pgfscope}%
\end{pgfscope}%
\begin{pgfscope}%
\pgfsetbuttcap%
\pgfsetroundjoin%
\definecolor{currentfill}{rgb}{0.000000,0.000000,0.000000}%
\pgfsetfillcolor{currentfill}%
\pgfsetlinewidth{0.602250pt}%
\definecolor{currentstroke}{rgb}{0.000000,0.000000,0.000000}%
\pgfsetstrokecolor{currentstroke}%
\pgfsetdash{}{0pt}%
\pgfsys@defobject{currentmarker}{\pgfqpoint{-0.027778in}{0.000000in}}{\pgfqpoint{-0.000000in}{0.000000in}}{%
\pgfpathmoveto{\pgfqpoint{-0.000000in}{0.000000in}}%
\pgfpathlineto{\pgfqpoint{-0.027778in}{0.000000in}}%
\pgfusepath{stroke,fill}%
}%
\begin{pgfscope}%
\pgfsys@transformshift{0.588387in}{0.978872in}%
\pgfsys@useobject{currentmarker}{}%
\end{pgfscope}%
\end{pgfscope}%
\begin{pgfscope}%
\pgfsetbuttcap%
\pgfsetroundjoin%
\definecolor{currentfill}{rgb}{0.000000,0.000000,0.000000}%
\pgfsetfillcolor{currentfill}%
\pgfsetlinewidth{0.602250pt}%
\definecolor{currentstroke}{rgb}{0.000000,0.000000,0.000000}%
\pgfsetstrokecolor{currentstroke}%
\pgfsetdash{}{0pt}%
\pgfsys@defobject{currentmarker}{\pgfqpoint{-0.027778in}{0.000000in}}{\pgfqpoint{-0.000000in}{0.000000in}}{%
\pgfpathmoveto{\pgfqpoint{-0.000000in}{0.000000in}}%
\pgfpathlineto{\pgfqpoint{-0.027778in}{0.000000in}}%
\pgfusepath{stroke,fill}%
}%
\begin{pgfscope}%
\pgfsys@transformshift{0.588387in}{1.108375in}%
\pgfsys@useobject{currentmarker}{}%
\end{pgfscope}%
\end{pgfscope}%
\begin{pgfscope}%
\pgfsetbuttcap%
\pgfsetroundjoin%
\definecolor{currentfill}{rgb}{0.000000,0.000000,0.000000}%
\pgfsetfillcolor{currentfill}%
\pgfsetlinewidth{0.602250pt}%
\definecolor{currentstroke}{rgb}{0.000000,0.000000,0.000000}%
\pgfsetstrokecolor{currentstroke}%
\pgfsetdash{}{0pt}%
\pgfsys@defobject{currentmarker}{\pgfqpoint{-0.027778in}{0.000000in}}{\pgfqpoint{-0.000000in}{0.000000in}}{%
\pgfpathmoveto{\pgfqpoint{-0.000000in}{0.000000in}}%
\pgfpathlineto{\pgfqpoint{-0.027778in}{0.000000in}}%
\pgfusepath{stroke,fill}%
}%
\begin{pgfscope}%
\pgfsys@transformshift{0.588387in}{1.174134in}%
\pgfsys@useobject{currentmarker}{}%
\end{pgfscope}%
\end{pgfscope}%
\begin{pgfscope}%
\pgfsetbuttcap%
\pgfsetroundjoin%
\definecolor{currentfill}{rgb}{0.000000,0.000000,0.000000}%
\pgfsetfillcolor{currentfill}%
\pgfsetlinewidth{0.602250pt}%
\definecolor{currentstroke}{rgb}{0.000000,0.000000,0.000000}%
\pgfsetstrokecolor{currentstroke}%
\pgfsetdash{}{0pt}%
\pgfsys@defobject{currentmarker}{\pgfqpoint{-0.027778in}{0.000000in}}{\pgfqpoint{-0.000000in}{0.000000in}}{%
\pgfpathmoveto{\pgfqpoint{-0.000000in}{0.000000in}}%
\pgfpathlineto{\pgfqpoint{-0.027778in}{0.000000in}}%
\pgfusepath{stroke,fill}%
}%
\begin{pgfscope}%
\pgfsys@transformshift{0.588387in}{1.220791in}%
\pgfsys@useobject{currentmarker}{}%
\end{pgfscope}%
\end{pgfscope}%
\begin{pgfscope}%
\pgfsetbuttcap%
\pgfsetroundjoin%
\definecolor{currentfill}{rgb}{0.000000,0.000000,0.000000}%
\pgfsetfillcolor{currentfill}%
\pgfsetlinewidth{0.602250pt}%
\definecolor{currentstroke}{rgb}{0.000000,0.000000,0.000000}%
\pgfsetstrokecolor{currentstroke}%
\pgfsetdash{}{0pt}%
\pgfsys@defobject{currentmarker}{\pgfqpoint{-0.027778in}{0.000000in}}{\pgfqpoint{-0.000000in}{0.000000in}}{%
\pgfpathmoveto{\pgfqpoint{-0.000000in}{0.000000in}}%
\pgfpathlineto{\pgfqpoint{-0.027778in}{0.000000in}}%
\pgfusepath{stroke,fill}%
}%
\begin{pgfscope}%
\pgfsys@transformshift{0.588387in}{1.256980in}%
\pgfsys@useobject{currentmarker}{}%
\end{pgfscope}%
\end{pgfscope}%
\begin{pgfscope}%
\pgfsetbuttcap%
\pgfsetroundjoin%
\definecolor{currentfill}{rgb}{0.000000,0.000000,0.000000}%
\pgfsetfillcolor{currentfill}%
\pgfsetlinewidth{0.602250pt}%
\definecolor{currentstroke}{rgb}{0.000000,0.000000,0.000000}%
\pgfsetstrokecolor{currentstroke}%
\pgfsetdash{}{0pt}%
\pgfsys@defobject{currentmarker}{\pgfqpoint{-0.027778in}{0.000000in}}{\pgfqpoint{-0.000000in}{0.000000in}}{%
\pgfpathmoveto{\pgfqpoint{-0.000000in}{0.000000in}}%
\pgfpathlineto{\pgfqpoint{-0.027778in}{0.000000in}}%
\pgfusepath{stroke,fill}%
}%
\begin{pgfscope}%
\pgfsys@transformshift{0.588387in}{1.286550in}%
\pgfsys@useobject{currentmarker}{}%
\end{pgfscope}%
\end{pgfscope}%
\begin{pgfscope}%
\pgfsetbuttcap%
\pgfsetroundjoin%
\definecolor{currentfill}{rgb}{0.000000,0.000000,0.000000}%
\pgfsetfillcolor{currentfill}%
\pgfsetlinewidth{0.602250pt}%
\definecolor{currentstroke}{rgb}{0.000000,0.000000,0.000000}%
\pgfsetstrokecolor{currentstroke}%
\pgfsetdash{}{0pt}%
\pgfsys@defobject{currentmarker}{\pgfqpoint{-0.027778in}{0.000000in}}{\pgfqpoint{-0.000000in}{0.000000in}}{%
\pgfpathmoveto{\pgfqpoint{-0.000000in}{0.000000in}}%
\pgfpathlineto{\pgfqpoint{-0.027778in}{0.000000in}}%
\pgfusepath{stroke,fill}%
}%
\begin{pgfscope}%
\pgfsys@transformshift{0.588387in}{1.311550in}%
\pgfsys@useobject{currentmarker}{}%
\end{pgfscope}%
\end{pgfscope}%
\begin{pgfscope}%
\pgfsetbuttcap%
\pgfsetroundjoin%
\definecolor{currentfill}{rgb}{0.000000,0.000000,0.000000}%
\pgfsetfillcolor{currentfill}%
\pgfsetlinewidth{0.602250pt}%
\definecolor{currentstroke}{rgb}{0.000000,0.000000,0.000000}%
\pgfsetstrokecolor{currentstroke}%
\pgfsetdash{}{0pt}%
\pgfsys@defobject{currentmarker}{\pgfqpoint{-0.027778in}{0.000000in}}{\pgfqpoint{-0.000000in}{0.000000in}}{%
\pgfpathmoveto{\pgfqpoint{-0.000000in}{0.000000in}}%
\pgfpathlineto{\pgfqpoint{-0.027778in}{0.000000in}}%
\pgfusepath{stroke,fill}%
}%
\begin{pgfscope}%
\pgfsys@transformshift{0.588387in}{1.333206in}%
\pgfsys@useobject{currentmarker}{}%
\end{pgfscope}%
\end{pgfscope}%
\begin{pgfscope}%
\pgfsetbuttcap%
\pgfsetroundjoin%
\definecolor{currentfill}{rgb}{0.000000,0.000000,0.000000}%
\pgfsetfillcolor{currentfill}%
\pgfsetlinewidth{0.602250pt}%
\definecolor{currentstroke}{rgb}{0.000000,0.000000,0.000000}%
\pgfsetstrokecolor{currentstroke}%
\pgfsetdash{}{0pt}%
\pgfsys@defobject{currentmarker}{\pgfqpoint{-0.027778in}{0.000000in}}{\pgfqpoint{-0.000000in}{0.000000in}}{%
\pgfpathmoveto{\pgfqpoint{-0.000000in}{0.000000in}}%
\pgfpathlineto{\pgfqpoint{-0.027778in}{0.000000in}}%
\pgfusepath{stroke,fill}%
}%
\begin{pgfscope}%
\pgfsys@transformshift{0.588387in}{1.352308in}%
\pgfsys@useobject{currentmarker}{}%
\end{pgfscope}%
\end{pgfscope}%
\begin{pgfscope}%
\pgfsetbuttcap%
\pgfsetroundjoin%
\definecolor{currentfill}{rgb}{0.000000,0.000000,0.000000}%
\pgfsetfillcolor{currentfill}%
\pgfsetlinewidth{0.602250pt}%
\definecolor{currentstroke}{rgb}{0.000000,0.000000,0.000000}%
\pgfsetstrokecolor{currentstroke}%
\pgfsetdash{}{0pt}%
\pgfsys@defobject{currentmarker}{\pgfqpoint{-0.027778in}{0.000000in}}{\pgfqpoint{-0.000000in}{0.000000in}}{%
\pgfpathmoveto{\pgfqpoint{-0.000000in}{0.000000in}}%
\pgfpathlineto{\pgfqpoint{-0.027778in}{0.000000in}}%
\pgfusepath{stroke,fill}%
}%
\begin{pgfscope}%
\pgfsys@transformshift{0.588387in}{1.481811in}%
\pgfsys@useobject{currentmarker}{}%
\end{pgfscope}%
\end{pgfscope}%
\begin{pgfscope}%
\pgfsetbuttcap%
\pgfsetroundjoin%
\definecolor{currentfill}{rgb}{0.000000,0.000000,0.000000}%
\pgfsetfillcolor{currentfill}%
\pgfsetlinewidth{0.602250pt}%
\definecolor{currentstroke}{rgb}{0.000000,0.000000,0.000000}%
\pgfsetstrokecolor{currentstroke}%
\pgfsetdash{}{0pt}%
\pgfsys@defobject{currentmarker}{\pgfqpoint{-0.027778in}{0.000000in}}{\pgfqpoint{-0.000000in}{0.000000in}}{%
\pgfpathmoveto{\pgfqpoint{-0.000000in}{0.000000in}}%
\pgfpathlineto{\pgfqpoint{-0.027778in}{0.000000in}}%
\pgfusepath{stroke,fill}%
}%
\begin{pgfscope}%
\pgfsys@transformshift{0.588387in}{1.547570in}%
\pgfsys@useobject{currentmarker}{}%
\end{pgfscope}%
\end{pgfscope}%
\begin{pgfscope}%
\pgfsetbuttcap%
\pgfsetroundjoin%
\definecolor{currentfill}{rgb}{0.000000,0.000000,0.000000}%
\pgfsetfillcolor{currentfill}%
\pgfsetlinewidth{0.602250pt}%
\definecolor{currentstroke}{rgb}{0.000000,0.000000,0.000000}%
\pgfsetstrokecolor{currentstroke}%
\pgfsetdash{}{0pt}%
\pgfsys@defobject{currentmarker}{\pgfqpoint{-0.027778in}{0.000000in}}{\pgfqpoint{-0.000000in}{0.000000in}}{%
\pgfpathmoveto{\pgfqpoint{-0.000000in}{0.000000in}}%
\pgfpathlineto{\pgfqpoint{-0.027778in}{0.000000in}}%
\pgfusepath{stroke,fill}%
}%
\begin{pgfscope}%
\pgfsys@transformshift{0.588387in}{1.594227in}%
\pgfsys@useobject{currentmarker}{}%
\end{pgfscope}%
\end{pgfscope}%
\begin{pgfscope}%
\pgfsetbuttcap%
\pgfsetroundjoin%
\definecolor{currentfill}{rgb}{0.000000,0.000000,0.000000}%
\pgfsetfillcolor{currentfill}%
\pgfsetlinewidth{0.602250pt}%
\definecolor{currentstroke}{rgb}{0.000000,0.000000,0.000000}%
\pgfsetstrokecolor{currentstroke}%
\pgfsetdash{}{0pt}%
\pgfsys@defobject{currentmarker}{\pgfqpoint{-0.027778in}{0.000000in}}{\pgfqpoint{-0.000000in}{0.000000in}}{%
\pgfpathmoveto{\pgfqpoint{-0.000000in}{0.000000in}}%
\pgfpathlineto{\pgfqpoint{-0.027778in}{0.000000in}}%
\pgfusepath{stroke,fill}%
}%
\begin{pgfscope}%
\pgfsys@transformshift{0.588387in}{1.630417in}%
\pgfsys@useobject{currentmarker}{}%
\end{pgfscope}%
\end{pgfscope}%
\begin{pgfscope}%
\pgfsetbuttcap%
\pgfsetroundjoin%
\definecolor{currentfill}{rgb}{0.000000,0.000000,0.000000}%
\pgfsetfillcolor{currentfill}%
\pgfsetlinewidth{0.602250pt}%
\definecolor{currentstroke}{rgb}{0.000000,0.000000,0.000000}%
\pgfsetstrokecolor{currentstroke}%
\pgfsetdash{}{0pt}%
\pgfsys@defobject{currentmarker}{\pgfqpoint{-0.027778in}{0.000000in}}{\pgfqpoint{-0.000000in}{0.000000in}}{%
\pgfpathmoveto{\pgfqpoint{-0.000000in}{0.000000in}}%
\pgfpathlineto{\pgfqpoint{-0.027778in}{0.000000in}}%
\pgfusepath{stroke,fill}%
}%
\begin{pgfscope}%
\pgfsys@transformshift{0.588387in}{1.659986in}%
\pgfsys@useobject{currentmarker}{}%
\end{pgfscope}%
\end{pgfscope}%
\begin{pgfscope}%
\pgfsetbuttcap%
\pgfsetroundjoin%
\definecolor{currentfill}{rgb}{0.000000,0.000000,0.000000}%
\pgfsetfillcolor{currentfill}%
\pgfsetlinewidth{0.602250pt}%
\definecolor{currentstroke}{rgb}{0.000000,0.000000,0.000000}%
\pgfsetstrokecolor{currentstroke}%
\pgfsetdash{}{0pt}%
\pgfsys@defobject{currentmarker}{\pgfqpoint{-0.027778in}{0.000000in}}{\pgfqpoint{-0.000000in}{0.000000in}}{%
\pgfpathmoveto{\pgfqpoint{-0.000000in}{0.000000in}}%
\pgfpathlineto{\pgfqpoint{-0.027778in}{0.000000in}}%
\pgfusepath{stroke,fill}%
}%
\begin{pgfscope}%
\pgfsys@transformshift{0.588387in}{1.684986in}%
\pgfsys@useobject{currentmarker}{}%
\end{pgfscope}%
\end{pgfscope}%
\begin{pgfscope}%
\pgfsetbuttcap%
\pgfsetroundjoin%
\definecolor{currentfill}{rgb}{0.000000,0.000000,0.000000}%
\pgfsetfillcolor{currentfill}%
\pgfsetlinewidth{0.602250pt}%
\definecolor{currentstroke}{rgb}{0.000000,0.000000,0.000000}%
\pgfsetstrokecolor{currentstroke}%
\pgfsetdash{}{0pt}%
\pgfsys@defobject{currentmarker}{\pgfqpoint{-0.027778in}{0.000000in}}{\pgfqpoint{-0.000000in}{0.000000in}}{%
\pgfpathmoveto{\pgfqpoint{-0.000000in}{0.000000in}}%
\pgfpathlineto{\pgfqpoint{-0.027778in}{0.000000in}}%
\pgfusepath{stroke,fill}%
}%
\begin{pgfscope}%
\pgfsys@transformshift{0.588387in}{1.706642in}%
\pgfsys@useobject{currentmarker}{}%
\end{pgfscope}%
\end{pgfscope}%
\begin{pgfscope}%
\pgfsetbuttcap%
\pgfsetroundjoin%
\definecolor{currentfill}{rgb}{0.000000,0.000000,0.000000}%
\pgfsetfillcolor{currentfill}%
\pgfsetlinewidth{0.602250pt}%
\definecolor{currentstroke}{rgb}{0.000000,0.000000,0.000000}%
\pgfsetstrokecolor{currentstroke}%
\pgfsetdash{}{0pt}%
\pgfsys@defobject{currentmarker}{\pgfqpoint{-0.027778in}{0.000000in}}{\pgfqpoint{-0.000000in}{0.000000in}}{%
\pgfpathmoveto{\pgfqpoint{-0.000000in}{0.000000in}}%
\pgfpathlineto{\pgfqpoint{-0.027778in}{0.000000in}}%
\pgfusepath{stroke,fill}%
}%
\begin{pgfscope}%
\pgfsys@transformshift{0.588387in}{1.725745in}%
\pgfsys@useobject{currentmarker}{}%
\end{pgfscope}%
\end{pgfscope}%
\begin{pgfscope}%
\pgfsetbuttcap%
\pgfsetroundjoin%
\definecolor{currentfill}{rgb}{0.000000,0.000000,0.000000}%
\pgfsetfillcolor{currentfill}%
\pgfsetlinewidth{0.602250pt}%
\definecolor{currentstroke}{rgb}{0.000000,0.000000,0.000000}%
\pgfsetstrokecolor{currentstroke}%
\pgfsetdash{}{0pt}%
\pgfsys@defobject{currentmarker}{\pgfqpoint{-0.027778in}{0.000000in}}{\pgfqpoint{-0.000000in}{0.000000in}}{%
\pgfpathmoveto{\pgfqpoint{-0.000000in}{0.000000in}}%
\pgfpathlineto{\pgfqpoint{-0.027778in}{0.000000in}}%
\pgfusepath{stroke,fill}%
}%
\begin{pgfscope}%
\pgfsys@transformshift{0.588387in}{1.855248in}%
\pgfsys@useobject{currentmarker}{}%
\end{pgfscope}%
\end{pgfscope}%
\begin{pgfscope}%
\pgfsetbuttcap%
\pgfsetroundjoin%
\definecolor{currentfill}{rgb}{0.000000,0.000000,0.000000}%
\pgfsetfillcolor{currentfill}%
\pgfsetlinewidth{0.602250pt}%
\definecolor{currentstroke}{rgb}{0.000000,0.000000,0.000000}%
\pgfsetstrokecolor{currentstroke}%
\pgfsetdash{}{0pt}%
\pgfsys@defobject{currentmarker}{\pgfqpoint{-0.027778in}{0.000000in}}{\pgfqpoint{-0.000000in}{0.000000in}}{%
\pgfpathmoveto{\pgfqpoint{-0.000000in}{0.000000in}}%
\pgfpathlineto{\pgfqpoint{-0.027778in}{0.000000in}}%
\pgfusepath{stroke,fill}%
}%
\begin{pgfscope}%
\pgfsys@transformshift{0.588387in}{1.921007in}%
\pgfsys@useobject{currentmarker}{}%
\end{pgfscope}%
\end{pgfscope}%
\begin{pgfscope}%
\pgfsetbuttcap%
\pgfsetroundjoin%
\definecolor{currentfill}{rgb}{0.000000,0.000000,0.000000}%
\pgfsetfillcolor{currentfill}%
\pgfsetlinewidth{0.602250pt}%
\definecolor{currentstroke}{rgb}{0.000000,0.000000,0.000000}%
\pgfsetstrokecolor{currentstroke}%
\pgfsetdash{}{0pt}%
\pgfsys@defobject{currentmarker}{\pgfqpoint{-0.027778in}{0.000000in}}{\pgfqpoint{-0.000000in}{0.000000in}}{%
\pgfpathmoveto{\pgfqpoint{-0.000000in}{0.000000in}}%
\pgfpathlineto{\pgfqpoint{-0.027778in}{0.000000in}}%
\pgfusepath{stroke,fill}%
}%
\begin{pgfscope}%
\pgfsys@transformshift{0.588387in}{1.967663in}%
\pgfsys@useobject{currentmarker}{}%
\end{pgfscope}%
\end{pgfscope}%
\begin{pgfscope}%
\pgfsetbuttcap%
\pgfsetroundjoin%
\definecolor{currentfill}{rgb}{0.000000,0.000000,0.000000}%
\pgfsetfillcolor{currentfill}%
\pgfsetlinewidth{0.602250pt}%
\definecolor{currentstroke}{rgb}{0.000000,0.000000,0.000000}%
\pgfsetstrokecolor{currentstroke}%
\pgfsetdash{}{0pt}%
\pgfsys@defobject{currentmarker}{\pgfqpoint{-0.027778in}{0.000000in}}{\pgfqpoint{-0.000000in}{0.000000in}}{%
\pgfpathmoveto{\pgfqpoint{-0.000000in}{0.000000in}}%
\pgfpathlineto{\pgfqpoint{-0.027778in}{0.000000in}}%
\pgfusepath{stroke,fill}%
}%
\begin{pgfscope}%
\pgfsys@transformshift{0.588387in}{2.003853in}%
\pgfsys@useobject{currentmarker}{}%
\end{pgfscope}%
\end{pgfscope}%
\begin{pgfscope}%
\pgfsetbuttcap%
\pgfsetroundjoin%
\definecolor{currentfill}{rgb}{0.000000,0.000000,0.000000}%
\pgfsetfillcolor{currentfill}%
\pgfsetlinewidth{0.602250pt}%
\definecolor{currentstroke}{rgb}{0.000000,0.000000,0.000000}%
\pgfsetstrokecolor{currentstroke}%
\pgfsetdash{}{0pt}%
\pgfsys@defobject{currentmarker}{\pgfqpoint{-0.027778in}{0.000000in}}{\pgfqpoint{-0.000000in}{0.000000in}}{%
\pgfpathmoveto{\pgfqpoint{-0.000000in}{0.000000in}}%
\pgfpathlineto{\pgfqpoint{-0.027778in}{0.000000in}}%
\pgfusepath{stroke,fill}%
}%
\begin{pgfscope}%
\pgfsys@transformshift{0.588387in}{2.033422in}%
\pgfsys@useobject{currentmarker}{}%
\end{pgfscope}%
\end{pgfscope}%
\begin{pgfscope}%
\pgfsetbuttcap%
\pgfsetroundjoin%
\definecolor{currentfill}{rgb}{0.000000,0.000000,0.000000}%
\pgfsetfillcolor{currentfill}%
\pgfsetlinewidth{0.602250pt}%
\definecolor{currentstroke}{rgb}{0.000000,0.000000,0.000000}%
\pgfsetstrokecolor{currentstroke}%
\pgfsetdash{}{0pt}%
\pgfsys@defobject{currentmarker}{\pgfqpoint{-0.027778in}{0.000000in}}{\pgfqpoint{-0.000000in}{0.000000in}}{%
\pgfpathmoveto{\pgfqpoint{-0.000000in}{0.000000in}}%
\pgfpathlineto{\pgfqpoint{-0.027778in}{0.000000in}}%
\pgfusepath{stroke,fill}%
}%
\begin{pgfscope}%
\pgfsys@transformshift{0.588387in}{2.058422in}%
\pgfsys@useobject{currentmarker}{}%
\end{pgfscope}%
\end{pgfscope}%
\begin{pgfscope}%
\pgfsetbuttcap%
\pgfsetroundjoin%
\definecolor{currentfill}{rgb}{0.000000,0.000000,0.000000}%
\pgfsetfillcolor{currentfill}%
\pgfsetlinewidth{0.602250pt}%
\definecolor{currentstroke}{rgb}{0.000000,0.000000,0.000000}%
\pgfsetstrokecolor{currentstroke}%
\pgfsetdash{}{0pt}%
\pgfsys@defobject{currentmarker}{\pgfqpoint{-0.027778in}{0.000000in}}{\pgfqpoint{-0.000000in}{0.000000in}}{%
\pgfpathmoveto{\pgfqpoint{-0.000000in}{0.000000in}}%
\pgfpathlineto{\pgfqpoint{-0.027778in}{0.000000in}}%
\pgfusepath{stroke,fill}%
}%
\begin{pgfscope}%
\pgfsys@transformshift{0.588387in}{2.080079in}%
\pgfsys@useobject{currentmarker}{}%
\end{pgfscope}%
\end{pgfscope}%
\begin{pgfscope}%
\pgfsetbuttcap%
\pgfsetroundjoin%
\definecolor{currentfill}{rgb}{0.000000,0.000000,0.000000}%
\pgfsetfillcolor{currentfill}%
\pgfsetlinewidth{0.602250pt}%
\definecolor{currentstroke}{rgb}{0.000000,0.000000,0.000000}%
\pgfsetstrokecolor{currentstroke}%
\pgfsetdash{}{0pt}%
\pgfsys@defobject{currentmarker}{\pgfqpoint{-0.027778in}{0.000000in}}{\pgfqpoint{-0.000000in}{0.000000in}}{%
\pgfpathmoveto{\pgfqpoint{-0.000000in}{0.000000in}}%
\pgfpathlineto{\pgfqpoint{-0.027778in}{0.000000in}}%
\pgfusepath{stroke,fill}%
}%
\begin{pgfscope}%
\pgfsys@transformshift{0.588387in}{2.099181in}%
\pgfsys@useobject{currentmarker}{}%
\end{pgfscope}%
\end{pgfscope}%
\begin{pgfscope}%
\pgfsetbuttcap%
\pgfsetroundjoin%
\definecolor{currentfill}{rgb}{0.000000,0.000000,0.000000}%
\pgfsetfillcolor{currentfill}%
\pgfsetlinewidth{0.602250pt}%
\definecolor{currentstroke}{rgb}{0.000000,0.000000,0.000000}%
\pgfsetstrokecolor{currentstroke}%
\pgfsetdash{}{0pt}%
\pgfsys@defobject{currentmarker}{\pgfqpoint{-0.027778in}{0.000000in}}{\pgfqpoint{-0.000000in}{0.000000in}}{%
\pgfpathmoveto{\pgfqpoint{-0.000000in}{0.000000in}}%
\pgfpathlineto{\pgfqpoint{-0.027778in}{0.000000in}}%
\pgfusepath{stroke,fill}%
}%
\begin{pgfscope}%
\pgfsys@transformshift{0.588387in}{2.228684in}%
\pgfsys@useobject{currentmarker}{}%
\end{pgfscope}%
\end{pgfscope}%
\begin{pgfscope}%
\pgfsetbuttcap%
\pgfsetroundjoin%
\definecolor{currentfill}{rgb}{0.000000,0.000000,0.000000}%
\pgfsetfillcolor{currentfill}%
\pgfsetlinewidth{0.602250pt}%
\definecolor{currentstroke}{rgb}{0.000000,0.000000,0.000000}%
\pgfsetstrokecolor{currentstroke}%
\pgfsetdash{}{0pt}%
\pgfsys@defobject{currentmarker}{\pgfqpoint{-0.027778in}{0.000000in}}{\pgfqpoint{-0.000000in}{0.000000in}}{%
\pgfpathmoveto{\pgfqpoint{-0.000000in}{0.000000in}}%
\pgfpathlineto{\pgfqpoint{-0.027778in}{0.000000in}}%
\pgfusepath{stroke,fill}%
}%
\begin{pgfscope}%
\pgfsys@transformshift{0.588387in}{2.294443in}%
\pgfsys@useobject{currentmarker}{}%
\end{pgfscope}%
\end{pgfscope}%
\begin{pgfscope}%
\pgfsetbuttcap%
\pgfsetroundjoin%
\definecolor{currentfill}{rgb}{0.000000,0.000000,0.000000}%
\pgfsetfillcolor{currentfill}%
\pgfsetlinewidth{0.602250pt}%
\definecolor{currentstroke}{rgb}{0.000000,0.000000,0.000000}%
\pgfsetstrokecolor{currentstroke}%
\pgfsetdash{}{0pt}%
\pgfsys@defobject{currentmarker}{\pgfqpoint{-0.027778in}{0.000000in}}{\pgfqpoint{-0.000000in}{0.000000in}}{%
\pgfpathmoveto{\pgfqpoint{-0.000000in}{0.000000in}}%
\pgfpathlineto{\pgfqpoint{-0.027778in}{0.000000in}}%
\pgfusepath{stroke,fill}%
}%
\begin{pgfscope}%
\pgfsys@transformshift{0.588387in}{2.341099in}%
\pgfsys@useobject{currentmarker}{}%
\end{pgfscope}%
\end{pgfscope}%
\begin{pgfscope}%
\pgfsetbuttcap%
\pgfsetroundjoin%
\definecolor{currentfill}{rgb}{0.000000,0.000000,0.000000}%
\pgfsetfillcolor{currentfill}%
\pgfsetlinewidth{0.602250pt}%
\definecolor{currentstroke}{rgb}{0.000000,0.000000,0.000000}%
\pgfsetstrokecolor{currentstroke}%
\pgfsetdash{}{0pt}%
\pgfsys@defobject{currentmarker}{\pgfqpoint{-0.027778in}{0.000000in}}{\pgfqpoint{-0.000000in}{0.000000in}}{%
\pgfpathmoveto{\pgfqpoint{-0.000000in}{0.000000in}}%
\pgfpathlineto{\pgfqpoint{-0.027778in}{0.000000in}}%
\pgfusepath{stroke,fill}%
}%
\begin{pgfscope}%
\pgfsys@transformshift{0.588387in}{2.377289in}%
\pgfsys@useobject{currentmarker}{}%
\end{pgfscope}%
\end{pgfscope}%
\begin{pgfscope}%
\pgfsetbuttcap%
\pgfsetroundjoin%
\definecolor{currentfill}{rgb}{0.000000,0.000000,0.000000}%
\pgfsetfillcolor{currentfill}%
\pgfsetlinewidth{0.602250pt}%
\definecolor{currentstroke}{rgb}{0.000000,0.000000,0.000000}%
\pgfsetstrokecolor{currentstroke}%
\pgfsetdash{}{0pt}%
\pgfsys@defobject{currentmarker}{\pgfqpoint{-0.027778in}{0.000000in}}{\pgfqpoint{-0.000000in}{0.000000in}}{%
\pgfpathmoveto{\pgfqpoint{-0.000000in}{0.000000in}}%
\pgfpathlineto{\pgfqpoint{-0.027778in}{0.000000in}}%
\pgfusepath{stroke,fill}%
}%
\begin{pgfscope}%
\pgfsys@transformshift{0.588387in}{2.406858in}%
\pgfsys@useobject{currentmarker}{}%
\end{pgfscope}%
\end{pgfscope}%
\begin{pgfscope}%
\pgfsetbuttcap%
\pgfsetroundjoin%
\definecolor{currentfill}{rgb}{0.000000,0.000000,0.000000}%
\pgfsetfillcolor{currentfill}%
\pgfsetlinewidth{0.602250pt}%
\definecolor{currentstroke}{rgb}{0.000000,0.000000,0.000000}%
\pgfsetstrokecolor{currentstroke}%
\pgfsetdash{}{0pt}%
\pgfsys@defobject{currentmarker}{\pgfqpoint{-0.027778in}{0.000000in}}{\pgfqpoint{-0.000000in}{0.000000in}}{%
\pgfpathmoveto{\pgfqpoint{-0.000000in}{0.000000in}}%
\pgfpathlineto{\pgfqpoint{-0.027778in}{0.000000in}}%
\pgfusepath{stroke,fill}%
}%
\begin{pgfscope}%
\pgfsys@transformshift{0.588387in}{2.431859in}%
\pgfsys@useobject{currentmarker}{}%
\end{pgfscope}%
\end{pgfscope}%
\begin{pgfscope}%
\pgfsetbuttcap%
\pgfsetroundjoin%
\definecolor{currentfill}{rgb}{0.000000,0.000000,0.000000}%
\pgfsetfillcolor{currentfill}%
\pgfsetlinewidth{0.602250pt}%
\definecolor{currentstroke}{rgb}{0.000000,0.000000,0.000000}%
\pgfsetstrokecolor{currentstroke}%
\pgfsetdash{}{0pt}%
\pgfsys@defobject{currentmarker}{\pgfqpoint{-0.027778in}{0.000000in}}{\pgfqpoint{-0.000000in}{0.000000in}}{%
\pgfpathmoveto{\pgfqpoint{-0.000000in}{0.000000in}}%
\pgfpathlineto{\pgfqpoint{-0.027778in}{0.000000in}}%
\pgfusepath{stroke,fill}%
}%
\begin{pgfscope}%
\pgfsys@transformshift{0.588387in}{2.453515in}%
\pgfsys@useobject{currentmarker}{}%
\end{pgfscope}%
\end{pgfscope}%
\begin{pgfscope}%
\pgfsetbuttcap%
\pgfsetroundjoin%
\definecolor{currentfill}{rgb}{0.000000,0.000000,0.000000}%
\pgfsetfillcolor{currentfill}%
\pgfsetlinewidth{0.602250pt}%
\definecolor{currentstroke}{rgb}{0.000000,0.000000,0.000000}%
\pgfsetstrokecolor{currentstroke}%
\pgfsetdash{}{0pt}%
\pgfsys@defobject{currentmarker}{\pgfqpoint{-0.027778in}{0.000000in}}{\pgfqpoint{-0.000000in}{0.000000in}}{%
\pgfpathmoveto{\pgfqpoint{-0.000000in}{0.000000in}}%
\pgfpathlineto{\pgfqpoint{-0.027778in}{0.000000in}}%
\pgfusepath{stroke,fill}%
}%
\begin{pgfscope}%
\pgfsys@transformshift{0.588387in}{2.472617in}%
\pgfsys@useobject{currentmarker}{}%
\end{pgfscope}%
\end{pgfscope}%
\begin{pgfscope}%
\pgfsetbuttcap%
\pgfsetroundjoin%
\definecolor{currentfill}{rgb}{0.000000,0.000000,0.000000}%
\pgfsetfillcolor{currentfill}%
\pgfsetlinewidth{0.602250pt}%
\definecolor{currentstroke}{rgb}{0.000000,0.000000,0.000000}%
\pgfsetstrokecolor{currentstroke}%
\pgfsetdash{}{0pt}%
\pgfsys@defobject{currentmarker}{\pgfqpoint{-0.027778in}{0.000000in}}{\pgfqpoint{-0.000000in}{0.000000in}}{%
\pgfpathmoveto{\pgfqpoint{-0.000000in}{0.000000in}}%
\pgfpathlineto{\pgfqpoint{-0.027778in}{0.000000in}}%
\pgfusepath{stroke,fill}%
}%
\begin{pgfscope}%
\pgfsys@transformshift{0.588387in}{2.602120in}%
\pgfsys@useobject{currentmarker}{}%
\end{pgfscope}%
\end{pgfscope}%
\begin{pgfscope}%
\pgfsetbuttcap%
\pgfsetroundjoin%
\definecolor{currentfill}{rgb}{0.000000,0.000000,0.000000}%
\pgfsetfillcolor{currentfill}%
\pgfsetlinewidth{0.602250pt}%
\definecolor{currentstroke}{rgb}{0.000000,0.000000,0.000000}%
\pgfsetstrokecolor{currentstroke}%
\pgfsetdash{}{0pt}%
\pgfsys@defobject{currentmarker}{\pgfqpoint{-0.027778in}{0.000000in}}{\pgfqpoint{-0.000000in}{0.000000in}}{%
\pgfpathmoveto{\pgfqpoint{-0.000000in}{0.000000in}}%
\pgfpathlineto{\pgfqpoint{-0.027778in}{0.000000in}}%
\pgfusepath{stroke,fill}%
}%
\begin{pgfscope}%
\pgfsys@transformshift{0.588387in}{2.667879in}%
\pgfsys@useobject{currentmarker}{}%
\end{pgfscope}%
\end{pgfscope}%
\begin{pgfscope}%
\pgfsetbuttcap%
\pgfsetroundjoin%
\definecolor{currentfill}{rgb}{0.000000,0.000000,0.000000}%
\pgfsetfillcolor{currentfill}%
\pgfsetlinewidth{0.602250pt}%
\definecolor{currentstroke}{rgb}{0.000000,0.000000,0.000000}%
\pgfsetstrokecolor{currentstroke}%
\pgfsetdash{}{0pt}%
\pgfsys@defobject{currentmarker}{\pgfqpoint{-0.027778in}{0.000000in}}{\pgfqpoint{-0.000000in}{0.000000in}}{%
\pgfpathmoveto{\pgfqpoint{-0.000000in}{0.000000in}}%
\pgfpathlineto{\pgfqpoint{-0.027778in}{0.000000in}}%
\pgfusepath{stroke,fill}%
}%
\begin{pgfscope}%
\pgfsys@transformshift{0.588387in}{2.714536in}%
\pgfsys@useobject{currentmarker}{}%
\end{pgfscope}%
\end{pgfscope}%
\begin{pgfscope}%
\definecolor{textcolor}{rgb}{0.000000,0.000000,0.000000}%
\pgfsetstrokecolor{textcolor}%
\pgfsetfillcolor{textcolor}%
\pgftext[x=0.234413in,y=1.631726in,,bottom,rotate=90.000000]{\color{textcolor}{\rmfamily\fontsize{10.000000}{12.000000}\selectfont\catcode`\^=\active\def^{\ifmmode\sp\else\^{}\fi}\catcode`\%=\active\def%{\%}Checks [call]}}%
\end{pgfscope}%
\begin{pgfscope}%
\pgfpathrectangle{\pgfqpoint{0.588387in}{0.521603in}}{\pgfqpoint{4.669024in}{2.220246in}}%
\pgfusepath{clip}%
\pgfsetrectcap%
\pgfsetroundjoin%
\pgfsetlinewidth{1.505625pt}%
\pgfsetstrokecolor{currentstroke1}%
\pgfsetdash{}{0pt}%
\pgfpathmoveto{\pgfqpoint{0.800616in}{0.734939in}}%
\pgfpathlineto{\pgfqpoint{0.841825in}{0.741060in}}%
\pgfpathlineto{\pgfqpoint{0.883034in}{0.813084in}}%
\pgfpathlineto{\pgfqpoint{0.924244in}{0.860336in}}%
\pgfpathlineto{\pgfqpoint{0.965453in}{0.897067in}}%
\pgfpathlineto{\pgfqpoint{1.006663in}{0.970959in}}%
\pgfpathlineto{\pgfqpoint{1.047872in}{1.066956in}}%
\pgfpathlineto{\pgfqpoint{1.089081in}{1.086795in}}%
\pgfpathlineto{\pgfqpoint{1.130291in}{1.147923in}}%
\pgfpathlineto{\pgfqpoint{1.171500in}{1.179195in}}%
\pgfpathlineto{\pgfqpoint{1.212709in}{1.190409in}}%
\pgfpathlineto{\pgfqpoint{1.253919in}{1.203466in}}%
\pgfpathlineto{\pgfqpoint{1.295128in}{1.211456in}}%
\pgfpathlineto{\pgfqpoint{1.336338in}{1.259971in}}%
\pgfpathlineto{\pgfqpoint{1.377547in}{1.226156in}}%
\pgfpathlineto{\pgfqpoint{1.418756in}{1.248639in}}%
\pgfpathlineto{\pgfqpoint{1.459966in}{1.271487in}}%
\pgfpathlineto{\pgfqpoint{1.501175in}{1.301771in}}%
\pgfpathlineto{\pgfqpoint{1.542385in}{1.262396in}}%
\pgfpathlineto{\pgfqpoint{1.583594in}{1.281051in}}%
\pgfpathlineto{\pgfqpoint{1.624803in}{1.297775in}}%
\pgfpathlineto{\pgfqpoint{1.666013in}{1.319671in}}%
\pgfpathlineto{\pgfqpoint{1.707222in}{1.301854in}}%
\pgfpathlineto{\pgfqpoint{1.748432in}{1.319897in}}%
\pgfpathlineto{\pgfqpoint{1.789641in}{1.324754in}}%
\pgfpathlineto{\pgfqpoint{1.830850in}{1.335357in}}%
\pgfpathlineto{\pgfqpoint{1.872060in}{1.324207in}}%
\pgfpathlineto{\pgfqpoint{1.913269in}{1.317688in}}%
\pgfpathlineto{\pgfqpoint{1.954479in}{1.338606in}}%
\pgfpathlineto{\pgfqpoint{1.995688in}{1.359192in}}%
\pgfpathlineto{\pgfqpoint{2.036897in}{1.334803in}}%
\pgfpathlineto{\pgfqpoint{2.078107in}{1.357701in}}%
\pgfpathlineto{\pgfqpoint{2.119316in}{1.360864in}}%
\pgfpathlineto{\pgfqpoint{2.160525in}{1.371349in}}%
\pgfpathlineto{\pgfqpoint{2.201735in}{1.355453in}}%
\pgfpathlineto{\pgfqpoint{2.242944in}{1.360104in}}%
\pgfpathlineto{\pgfqpoint{2.284154in}{1.371331in}}%
\pgfpathlineto{\pgfqpoint{2.325363in}{1.392130in}}%
\pgfpathlineto{\pgfqpoint{2.366572in}{1.363618in}}%
\pgfpathlineto{\pgfqpoint{2.407782in}{1.397175in}}%
\pgfpathlineto{\pgfqpoint{2.448991in}{1.382830in}}%
\pgfpathlineto{\pgfqpoint{2.490201in}{1.397671in}}%
\pgfpathlineto{\pgfqpoint{2.531410in}{1.395205in}}%
\pgfpathlineto{\pgfqpoint{2.572619in}{1.396404in}}%
\pgfpathlineto{\pgfqpoint{2.613829in}{1.399795in}}%
\pgfpathlineto{\pgfqpoint{2.655038in}{1.421708in}}%
\pgfpathlineto{\pgfqpoint{2.696248in}{1.400445in}}%
\pgfpathlineto{\pgfqpoint{2.737457in}{1.397911in}}%
\pgfpathlineto{\pgfqpoint{2.778666in}{1.410514in}}%
\pgfpathlineto{\pgfqpoint{2.819876in}{1.433726in}}%
\pgfpathlineto{\pgfqpoint{2.861085in}{1.411857in}}%
\pgfpathlineto{\pgfqpoint{2.902295in}{1.421767in}}%
\pgfpathlineto{\pgfqpoint{2.943504in}{1.442138in}}%
\pgfpathlineto{\pgfqpoint{2.984713in}{1.447854in}}%
\pgfpathlineto{\pgfqpoint{3.025923in}{1.429936in}}%
\pgfpathlineto{\pgfqpoint{3.067132in}{1.439185in}}%
\pgfpathlineto{\pgfqpoint{3.108341in}{1.435994in}}%
\pgfpathlineto{\pgfqpoint{3.149551in}{1.446457in}}%
\pgfpathlineto{\pgfqpoint{3.190760in}{1.456168in}}%
\pgfpathlineto{\pgfqpoint{3.231970in}{1.434699in}}%
\pgfpathlineto{\pgfqpoint{3.273179in}{1.456784in}}%
\pgfpathlineto{\pgfqpoint{3.314388in}{1.459691in}}%
\pgfpathlineto{\pgfqpoint{3.355598in}{1.446128in}}%
\pgfpathlineto{\pgfqpoint{3.396807in}{1.461631in}}%
\pgfpathlineto{\pgfqpoint{3.438017in}{1.452566in}}%
\pgfpathlineto{\pgfqpoint{3.479226in}{1.462341in}}%
\pgfpathlineto{\pgfqpoint{3.520435in}{1.483023in}}%
\pgfpathlineto{\pgfqpoint{3.561645in}{1.469963in}}%
\pgfpathlineto{\pgfqpoint{3.602854in}{1.455136in}}%
\pgfpathlineto{\pgfqpoint{3.644064in}{1.507505in}}%
\pgfpathlineto{\pgfqpoint{3.685273in}{1.464122in}}%
\pgfpathlineto{\pgfqpoint{3.726482in}{1.470854in}}%
\pgfpathlineto{\pgfqpoint{3.767692in}{1.466293in}}%
\pgfpathlineto{\pgfqpoint{3.808901in}{1.493603in}}%
\pgfpathlineto{\pgfqpoint{3.891320in}{1.482050in}}%
\pgfpathlineto{\pgfqpoint{3.932529in}{1.480345in}}%
\pgfpathlineto{\pgfqpoint{3.973739in}{1.499768in}}%
\pgfpathlineto{\pgfqpoint{4.056157in}{1.492647in}}%
\pgfpathlineto{\pgfqpoint{4.097367in}{1.493541in}}%
\pgfpathlineto{\pgfqpoint{4.138576in}{1.505369in}}%
\pgfpathlineto{\pgfqpoint{4.179786in}{1.511042in}}%
\pgfpathlineto{\pgfqpoint{4.220995in}{1.500414in}}%
\pgfpathlineto{\pgfqpoint{4.262204in}{1.533459in}}%
\pgfpathlineto{\pgfqpoint{4.303414in}{1.507966in}}%
\pgfpathlineto{\pgfqpoint{4.385833in}{1.514061in}}%
\pgfpathlineto{\pgfqpoint{4.427042in}{1.503949in}}%
\pgfpathlineto{\pgfqpoint{4.468251in}{1.519936in}}%
\pgfpathlineto{\pgfqpoint{4.509461in}{1.530483in}}%
\pgfpathlineto{\pgfqpoint{4.550670in}{1.499466in}}%
\pgfpathlineto{\pgfqpoint{4.591880in}{1.537535in}}%
\pgfpathlineto{\pgfqpoint{4.633089in}{1.523110in}}%
\pgfpathlineto{\pgfqpoint{4.715508in}{1.525088in}}%
\pgfpathlineto{\pgfqpoint{4.756717in}{1.535218in}}%
\pgfpathlineto{\pgfqpoint{4.797926in}{1.558796in}}%
\pgfpathlineto{\pgfqpoint{4.880345in}{1.530483in}}%
\pgfpathlineto{\pgfqpoint{4.962764in}{1.544845in}}%
\pgfpathlineto{\pgfqpoint{5.003973in}{1.510703in}}%
\pgfpathlineto{\pgfqpoint{5.045183in}{1.548916in}}%
\pgfusepath{stroke}%
\end{pgfscope}%
\begin{pgfscope}%
\pgfpathrectangle{\pgfqpoint{0.588387in}{0.521603in}}{\pgfqpoint{4.669024in}{2.220246in}}%
\pgfusepath{clip}%
\pgfsetrectcap%
\pgfsetroundjoin%
\pgfsetlinewidth{1.505625pt}%
\pgfsetstrokecolor{currentstroke2}%
\pgfsetdash{}{0pt}%
\pgfpathmoveto{\pgfqpoint{0.800616in}{0.734939in}}%
\pgfpathlineto{\pgfqpoint{0.841825in}{0.741060in}}%
\pgfpathlineto{\pgfqpoint{0.883034in}{0.808243in}}%
\pgfpathlineto{\pgfqpoint{0.924244in}{0.855735in}}%
\pgfpathlineto{\pgfqpoint{0.965453in}{0.897067in}}%
\pgfpathlineto{\pgfqpoint{1.006663in}{0.970959in}}%
\pgfpathlineto{\pgfqpoint{1.047872in}{1.066956in}}%
\pgfpathlineto{\pgfqpoint{1.089081in}{1.086795in}}%
\pgfpathlineto{\pgfqpoint{1.130291in}{1.147923in}}%
\pgfpathlineto{\pgfqpoint{1.171500in}{1.179195in}}%
\pgfpathlineto{\pgfqpoint{1.212709in}{1.317489in}}%
\pgfpathlineto{\pgfqpoint{1.253919in}{1.299925in}}%
\pgfpathlineto{\pgfqpoint{1.295128in}{1.313074in}}%
\pgfpathlineto{\pgfqpoint{1.336338in}{1.398269in}}%
\pgfpathlineto{\pgfqpoint{1.377547in}{1.523953in}}%
\pgfpathlineto{\pgfqpoint{1.418756in}{1.500394in}}%
\pgfpathlineto{\pgfqpoint{1.459966in}{1.619696in}}%
\pgfpathlineto{\pgfqpoint{1.501175in}{1.771848in}}%
\pgfpathlineto{\pgfqpoint{1.542385in}{1.854985in}}%
\pgfpathlineto{\pgfqpoint{1.583594in}{1.763753in}}%
\pgfpathlineto{\pgfqpoint{1.624803in}{1.766076in}}%
\pgfpathlineto{\pgfqpoint{1.666013in}{1.905646in}}%
\pgfpathlineto{\pgfqpoint{1.707222in}{1.748012in}}%
\pgfpathlineto{\pgfqpoint{1.748432in}{1.929434in}}%
\pgfpathlineto{\pgfqpoint{1.789641in}{1.913596in}}%
\pgfpathlineto{\pgfqpoint{1.830850in}{1.808943in}}%
\pgfpathlineto{\pgfqpoint{1.872060in}{2.087815in}}%
\pgfpathlineto{\pgfqpoint{1.913269in}{2.042391in}}%
\pgfpathlineto{\pgfqpoint{1.954479in}{1.912020in}}%
\pgfpathlineto{\pgfqpoint{1.995688in}{2.031089in}}%
\pgfpathlineto{\pgfqpoint{2.036897in}{1.853634in}}%
\pgfpathlineto{\pgfqpoint{2.078107in}{1.943016in}}%
\pgfpathlineto{\pgfqpoint{2.119316in}{1.993436in}}%
\pgfpathlineto{\pgfqpoint{2.160525in}{1.970594in}}%
\pgfpathlineto{\pgfqpoint{2.201735in}{2.080293in}}%
\pgfpathlineto{\pgfqpoint{2.242944in}{1.844661in}}%
\pgfpathlineto{\pgfqpoint{2.284154in}{2.047640in}}%
\pgfpathlineto{\pgfqpoint{2.325363in}{2.042989in}}%
\pgfpathlineto{\pgfqpoint{2.366572in}{1.703430in}}%
\pgfpathlineto{\pgfqpoint{2.407782in}{2.008057in}}%
\pgfpathlineto{\pgfqpoint{2.448991in}{1.488597in}}%
\pgfpathlineto{\pgfqpoint{2.490201in}{2.089075in}}%
\pgfpathlineto{\pgfqpoint{2.531410in}{2.001389in}}%
\pgfpathlineto{\pgfqpoint{2.572619in}{1.951099in}}%
\pgfpathlineto{\pgfqpoint{2.613829in}{1.957938in}}%
\pgfpathlineto{\pgfqpoint{2.655038in}{2.127265in}}%
\pgfpathlineto{\pgfqpoint{2.696248in}{1.490605in}}%
\pgfpathlineto{\pgfqpoint{2.737457in}{1.719413in}}%
\pgfpathlineto{\pgfqpoint{2.778666in}{1.745816in}}%
\pgfpathlineto{\pgfqpoint{2.819876in}{1.846100in}}%
\pgfpathlineto{\pgfqpoint{2.861085in}{1.558427in}}%
\pgfpathlineto{\pgfqpoint{2.902295in}{1.943259in}}%
\pgfpathlineto{\pgfqpoint{2.943504in}{1.786154in}}%
\pgfpathlineto{\pgfqpoint{2.984713in}{2.017991in}}%
\pgfpathlineto{\pgfqpoint{3.025923in}{1.712271in}}%
\pgfpathlineto{\pgfqpoint{3.067132in}{1.939267in}}%
\pgfpathlineto{\pgfqpoint{3.108341in}{1.664692in}}%
\pgfpathlineto{\pgfqpoint{3.149551in}{1.905610in}}%
\pgfpathlineto{\pgfqpoint{3.190760in}{1.652963in}}%
\pgfpathlineto{\pgfqpoint{3.231970in}{1.999616in}}%
\pgfpathlineto{\pgfqpoint{3.273179in}{2.177011in}}%
\pgfpathlineto{\pgfqpoint{3.314388in}{1.663625in}}%
\pgfpathlineto{\pgfqpoint{3.355598in}{1.541231in}}%
\pgfpathlineto{\pgfqpoint{3.396807in}{2.006511in}}%
\pgfpathlineto{\pgfqpoint{3.438017in}{1.718936in}}%
\pgfpathlineto{\pgfqpoint{3.479226in}{1.856117in}}%
\pgfpathlineto{\pgfqpoint{3.520435in}{1.468289in}}%
\pgfpathlineto{\pgfqpoint{3.561645in}{1.985888in}}%
\pgfpathlineto{\pgfqpoint{3.602854in}{1.648216in}}%
\pgfpathlineto{\pgfqpoint{3.644064in}{1.946712in}}%
\pgfpathlineto{\pgfqpoint{3.685273in}{1.550251in}}%
\pgfpathlineto{\pgfqpoint{3.726482in}{1.707720in}}%
\pgfpathlineto{\pgfqpoint{3.767692in}{1.853125in}}%
\pgfpathlineto{\pgfqpoint{3.808901in}{1.699811in}}%
\pgfpathlineto{\pgfqpoint{3.891320in}{1.781832in}}%
\pgfpathlineto{\pgfqpoint{3.932529in}{1.742710in}}%
\pgfpathlineto{\pgfqpoint{3.973739in}{1.824889in}}%
\pgfpathlineto{\pgfqpoint{4.056157in}{2.080289in}}%
\pgfpathlineto{\pgfqpoint{4.097367in}{1.568821in}}%
\pgfpathlineto{\pgfqpoint{4.138576in}{2.044766in}}%
\pgfpathlineto{\pgfqpoint{4.179786in}{1.884884in}}%
\pgfpathlineto{\pgfqpoint{4.220995in}{2.199078in}}%
\pgfpathlineto{\pgfqpoint{4.303414in}{1.847383in}}%
\pgfpathlineto{\pgfqpoint{4.385833in}{1.845586in}}%
\pgfpathlineto{\pgfqpoint{4.427042in}{1.503772in}}%
\pgfpathlineto{\pgfqpoint{4.468251in}{1.589426in}}%
\pgfpathlineto{\pgfqpoint{4.509461in}{1.571175in}}%
\pgfpathlineto{\pgfqpoint{4.550670in}{1.765264in}}%
\pgfpathlineto{\pgfqpoint{4.591880in}{1.583327in}}%
\pgfpathlineto{\pgfqpoint{4.633089in}{1.923444in}}%
\pgfpathlineto{\pgfqpoint{4.715508in}{1.775082in}}%
\pgfpathlineto{\pgfqpoint{4.756717in}{1.514725in}}%
\pgfpathlineto{\pgfqpoint{4.797926in}{1.646536in}}%
\pgfpathlineto{\pgfqpoint{4.880345in}{1.864953in}}%
\pgfpathlineto{\pgfqpoint{4.962764in}{1.644690in}}%
\pgfusepath{stroke}%
\end{pgfscope}%
\begin{pgfscope}%
\pgfpathrectangle{\pgfqpoint{0.588387in}{0.521603in}}{\pgfqpoint{4.669024in}{2.220246in}}%
\pgfusepath{clip}%
\pgfsetrectcap%
\pgfsetroundjoin%
\pgfsetlinewidth{1.505625pt}%
\pgfsetstrokecolor{currentstroke3}%
\pgfsetdash{}{0pt}%
\pgfpathmoveto{\pgfqpoint{0.800616in}{0.622524in}}%
\pgfpathlineto{\pgfqpoint{0.841825in}{0.634543in}}%
\pgfpathlineto{\pgfqpoint{0.883034in}{0.739152in}}%
\pgfpathlineto{\pgfqpoint{0.924244in}{0.798330in}}%
\pgfpathlineto{\pgfqpoint{0.965453in}{0.836888in}}%
\pgfpathlineto{\pgfqpoint{1.006663in}{0.934510in}}%
\pgfpathlineto{\pgfqpoint{1.047872in}{0.941779in}}%
\pgfpathlineto{\pgfqpoint{1.089081in}{0.988547in}}%
\pgfpathlineto{\pgfqpoint{1.130291in}{1.063044in}}%
\pgfpathlineto{\pgfqpoint{1.171500in}{1.149900in}}%
\pgfpathlineto{\pgfqpoint{1.212709in}{1.127210in}}%
\pgfpathlineto{\pgfqpoint{1.253919in}{1.195995in}}%
\pgfpathlineto{\pgfqpoint{1.295128in}{1.119528in}}%
\pgfpathlineto{\pgfqpoint{1.336338in}{1.217549in}}%
\pgfpathlineto{\pgfqpoint{1.377547in}{1.225226in}}%
\pgfpathlineto{\pgfqpoint{1.418756in}{1.337711in}}%
\pgfpathlineto{\pgfqpoint{1.459966in}{1.335471in}}%
\pgfpathlineto{\pgfqpoint{1.501175in}{1.440150in}}%
\pgfpathlineto{\pgfqpoint{1.542385in}{1.378169in}}%
\pgfpathlineto{\pgfqpoint{1.583594in}{1.433207in}}%
\pgfpathlineto{\pgfqpoint{1.624803in}{1.385246in}}%
\pgfpathlineto{\pgfqpoint{1.666013in}{1.444782in}}%
\pgfpathlineto{\pgfqpoint{1.707222in}{1.375659in}}%
\pgfpathlineto{\pgfqpoint{1.748432in}{1.566854in}}%
\pgfpathlineto{\pgfqpoint{1.789641in}{1.354101in}}%
\pgfpathlineto{\pgfqpoint{1.830850in}{1.450969in}}%
\pgfpathlineto{\pgfqpoint{1.872060in}{1.315709in}}%
\pgfpathlineto{\pgfqpoint{1.913269in}{1.792070in}}%
\pgfpathlineto{\pgfqpoint{1.954479in}{1.307444in}}%
\pgfpathlineto{\pgfqpoint{1.995688in}{1.449529in}}%
\pgfpathlineto{\pgfqpoint{2.036897in}{1.752588in}}%
\pgfpathlineto{\pgfqpoint{2.078107in}{1.579801in}}%
\pgfpathlineto{\pgfqpoint{2.119316in}{1.373467in}}%
\pgfpathlineto{\pgfqpoint{2.160525in}{1.637943in}}%
\pgfpathlineto{\pgfqpoint{2.201735in}{1.783801in}}%
\pgfpathlineto{\pgfqpoint{2.242944in}{1.390193in}}%
\pgfpathlineto{\pgfqpoint{2.284154in}{1.245791in}}%
\pgfpathlineto{\pgfqpoint{2.325363in}{1.569440in}}%
\pgfpathlineto{\pgfqpoint{2.366572in}{1.569564in}}%
\pgfpathlineto{\pgfqpoint{2.407782in}{1.521708in}}%
\pgfpathlineto{\pgfqpoint{2.448991in}{1.119761in}}%
\pgfpathlineto{\pgfqpoint{2.490201in}{1.262104in}}%
\pgfpathlineto{\pgfqpoint{2.531410in}{1.268710in}}%
\pgfpathlineto{\pgfqpoint{2.572619in}{1.433721in}}%
\pgfpathlineto{\pgfqpoint{2.613829in}{1.324559in}}%
\pgfpathlineto{\pgfqpoint{2.655038in}{1.436232in}}%
\pgfpathlineto{\pgfqpoint{2.696248in}{2.115618in}}%
\pgfpathlineto{\pgfqpoint{2.737457in}{1.101942in}}%
\pgfpathlineto{\pgfqpoint{2.778666in}{1.321499in}}%
\pgfpathlineto{\pgfqpoint{2.819876in}{1.238368in}}%
\pgfpathlineto{\pgfqpoint{2.861085in}{1.172582in}}%
\pgfpathlineto{\pgfqpoint{2.902295in}{1.291565in}}%
\pgfpathlineto{\pgfqpoint{2.943504in}{1.336418in}}%
\pgfpathlineto{\pgfqpoint{2.984713in}{1.460758in}}%
\pgfpathlineto{\pgfqpoint{3.025923in}{1.418367in}}%
\pgfpathlineto{\pgfqpoint{3.067132in}{1.256666in}}%
\pgfpathlineto{\pgfqpoint{3.108341in}{0.988587in}}%
\pgfpathlineto{\pgfqpoint{3.149551in}{1.399493in}}%
\pgfpathlineto{\pgfqpoint{3.190760in}{1.868544in}}%
\pgfpathlineto{\pgfqpoint{3.231970in}{1.412340in}}%
\pgfpathlineto{\pgfqpoint{3.273179in}{0.975228in}}%
\pgfpathlineto{\pgfqpoint{3.314388in}{1.806499in}}%
\pgfpathlineto{\pgfqpoint{3.355598in}{1.837935in}}%
\pgfpathlineto{\pgfqpoint{3.396807in}{1.138843in}}%
\pgfpathlineto{\pgfqpoint{3.438017in}{1.131042in}}%
\pgfpathlineto{\pgfqpoint{3.479226in}{2.640929in}}%
\pgfpathlineto{\pgfqpoint{3.520435in}{1.011417in}}%
\pgfpathlineto{\pgfqpoint{3.561645in}{1.054688in}}%
\pgfpathlineto{\pgfqpoint{3.602854in}{1.396697in}}%
\pgfpathlineto{\pgfqpoint{3.644064in}{1.994392in}}%
\pgfpathlineto{\pgfqpoint{3.685273in}{1.996623in}}%
\pgfpathlineto{\pgfqpoint{3.726482in}{1.449202in}}%
\pgfpathlineto{\pgfqpoint{3.767692in}{1.480386in}}%
\pgfpathlineto{\pgfqpoint{3.808901in}{1.123217in}}%
\pgfpathlineto{\pgfqpoint{3.891320in}{1.440288in}}%
\pgfpathlineto{\pgfqpoint{3.932529in}{1.460525in}}%
\pgfpathlineto{\pgfqpoint{3.973739in}{1.574375in}}%
\pgfpathlineto{\pgfqpoint{4.056157in}{1.078511in}}%
\pgfpathlineto{\pgfqpoint{4.097367in}{1.137944in}}%
\pgfpathlineto{\pgfqpoint{4.138576in}{1.546013in}}%
\pgfpathlineto{\pgfqpoint{4.179786in}{1.201435in}}%
\pgfpathlineto{\pgfqpoint{4.220995in}{1.200200in}}%
\pgfpathlineto{\pgfqpoint{4.262204in}{1.179452in}}%
\pgfpathlineto{\pgfqpoint{4.303414in}{1.305050in}}%
\pgfpathlineto{\pgfqpoint{4.385833in}{1.002766in}}%
\pgfpathlineto{\pgfqpoint{4.427042in}{1.309803in}}%
\pgfpathlineto{\pgfqpoint{4.468251in}{1.769341in}}%
\pgfpathlineto{\pgfqpoint{4.509461in}{1.179452in}}%
\pgfpathlineto{\pgfqpoint{4.550670in}{1.838363in}}%
\pgfpathlineto{\pgfqpoint{4.591880in}{1.331166in}}%
\pgfpathlineto{\pgfqpoint{4.633089in}{2.062693in}}%
\pgfpathlineto{\pgfqpoint{4.715508in}{1.236248in}}%
\pgfpathlineto{\pgfqpoint{4.756717in}{1.061719in}}%
\pgfpathlineto{\pgfqpoint{4.797926in}{1.514227in}}%
\pgfpathlineto{\pgfqpoint{4.880345in}{1.211043in}}%
\pgfpathlineto{\pgfqpoint{4.962764in}{1.061719in}}%
\pgfpathlineto{\pgfqpoint{5.003973in}{1.050529in}}%
\pgfpathlineto{\pgfqpoint{5.045183in}{1.273906in}}%
\pgfusepath{stroke}%
\end{pgfscope}%
\begin{pgfscope}%
\pgfpathrectangle{\pgfqpoint{0.588387in}{0.521603in}}{\pgfqpoint{4.669024in}{2.220246in}}%
\pgfusepath{clip}%
\pgfsetrectcap%
\pgfsetroundjoin%
\pgfsetlinewidth{1.505625pt}%
\pgfsetstrokecolor{currentstroke4}%
\pgfsetdash{}{0pt}%
\pgfpathmoveto{\pgfqpoint{0.800616in}{0.734939in}}%
\pgfpathlineto{\pgfqpoint{0.841825in}{0.734939in}}%
\pgfpathlineto{\pgfqpoint{0.883034in}{0.800698in}}%
\pgfpathlineto{\pgfqpoint{0.924244in}{0.855267in}}%
\pgfpathlineto{\pgfqpoint{0.965453in}{0.916601in}}%
\pgfpathlineto{\pgfqpoint{1.006663in}{0.953872in}}%
\pgfpathlineto{\pgfqpoint{1.047872in}{1.065500in}}%
\pgfpathlineto{\pgfqpoint{1.089081in}{1.075638in}}%
\pgfpathlineto{\pgfqpoint{1.130291in}{1.149867in}}%
\pgfpathlineto{\pgfqpoint{1.171500in}{1.150003in}}%
\pgfpathlineto{\pgfqpoint{1.212709in}{1.160132in}}%
\pgfpathlineto{\pgfqpoint{1.253919in}{1.163553in}}%
\pgfpathlineto{\pgfqpoint{1.295128in}{1.194271in}}%
\pgfpathlineto{\pgfqpoint{1.336338in}{1.249560in}}%
\pgfpathlineto{\pgfqpoint{1.377547in}{1.203949in}}%
\pgfpathlineto{\pgfqpoint{1.418756in}{1.204721in}}%
\pgfpathlineto{\pgfqpoint{1.459966in}{1.235192in}}%
\pgfpathlineto{\pgfqpoint{1.501175in}{1.269173in}}%
\pgfpathlineto{\pgfqpoint{1.542385in}{1.248065in}}%
\pgfpathlineto{\pgfqpoint{1.583594in}{1.250085in}}%
\pgfpathlineto{\pgfqpoint{1.624803in}{1.270957in}}%
\pgfpathlineto{\pgfqpoint{1.666013in}{1.296870in}}%
\pgfpathlineto{\pgfqpoint{1.707222in}{1.274726in}}%
\pgfpathlineto{\pgfqpoint{1.748432in}{1.284364in}}%
\pgfpathlineto{\pgfqpoint{1.789641in}{1.296541in}}%
\pgfpathlineto{\pgfqpoint{1.830850in}{1.323817in}}%
\pgfpathlineto{\pgfqpoint{1.872060in}{1.298735in}}%
\pgfpathlineto{\pgfqpoint{1.913269in}{1.299631in}}%
\pgfpathlineto{\pgfqpoint{1.954479in}{1.316868in}}%
\pgfpathlineto{\pgfqpoint{1.995688in}{1.335064in}}%
\pgfpathlineto{\pgfqpoint{2.036897in}{1.316774in}}%
\pgfpathlineto{\pgfqpoint{2.078107in}{1.320049in}}%
\pgfpathlineto{\pgfqpoint{2.119316in}{1.337458in}}%
\pgfpathlineto{\pgfqpoint{2.160525in}{1.350496in}}%
\pgfpathlineto{\pgfqpoint{2.201735in}{1.337305in}}%
\pgfpathlineto{\pgfqpoint{2.242944in}{1.343558in}}%
\pgfpathlineto{\pgfqpoint{2.284154in}{1.352068in}}%
\pgfpathlineto{\pgfqpoint{2.325363in}{1.368816in}}%
\pgfpathlineto{\pgfqpoint{2.366572in}{1.355621in}}%
\pgfpathlineto{\pgfqpoint{2.407782in}{1.358348in}}%
\pgfpathlineto{\pgfqpoint{2.448991in}{1.369691in}}%
\pgfpathlineto{\pgfqpoint{2.490201in}{1.382887in}}%
\pgfpathlineto{\pgfqpoint{2.531410in}{1.390824in}}%
\pgfpathlineto{\pgfqpoint{2.572619in}{1.365722in}}%
\pgfpathlineto{\pgfqpoint{2.613829in}{1.381993in}}%
\pgfpathlineto{\pgfqpoint{2.655038in}{1.395973in}}%
\pgfpathlineto{\pgfqpoint{2.696248in}{1.385148in}}%
\pgfpathlineto{\pgfqpoint{2.737457in}{1.397001in}}%
\pgfpathlineto{\pgfqpoint{2.778666in}{1.398191in}}%
\pgfpathlineto{\pgfqpoint{2.819876in}{1.405381in}}%
\pgfpathlineto{\pgfqpoint{2.861085in}{1.398191in}}%
\pgfpathlineto{\pgfqpoint{2.902295in}{1.398497in}}%
\pgfpathlineto{\pgfqpoint{2.943504in}{1.404544in}}%
\pgfpathlineto{\pgfqpoint{2.984713in}{1.425420in}}%
\pgfpathlineto{\pgfqpoint{3.025923in}{1.412880in}}%
\pgfpathlineto{\pgfqpoint{3.067132in}{1.411826in}}%
\pgfpathlineto{\pgfqpoint{3.108341in}{1.423708in}}%
\pgfpathlineto{\pgfqpoint{3.149551in}{1.428289in}}%
\pgfpathlineto{\pgfqpoint{3.190760in}{1.425837in}}%
\pgfpathlineto{\pgfqpoint{3.231970in}{1.434299in}}%
\pgfpathlineto{\pgfqpoint{3.273179in}{1.423269in}}%
\pgfpathlineto{\pgfqpoint{3.314388in}{1.437421in}}%
\pgfpathlineto{\pgfqpoint{3.355598in}{1.430215in}}%
\pgfpathlineto{\pgfqpoint{3.396807in}{1.427706in}}%
\pgfpathlineto{\pgfqpoint{3.438017in}{1.437836in}}%
\pgfpathlineto{\pgfqpoint{3.479226in}{1.443132in}}%
\pgfpathlineto{\pgfqpoint{3.520435in}{1.461998in}}%
\pgfpathlineto{\pgfqpoint{3.561645in}{1.440092in}}%
\pgfpathlineto{\pgfqpoint{3.602854in}{1.443582in}}%
\pgfpathlineto{\pgfqpoint{3.644064in}{1.451527in}}%
\pgfpathlineto{\pgfqpoint{3.685273in}{1.451592in}}%
\pgfpathlineto{\pgfqpoint{3.726482in}{1.450416in}}%
\pgfpathlineto{\pgfqpoint{3.767692in}{1.462227in}}%
\pgfpathlineto{\pgfqpoint{3.808901in}{1.466219in}}%
\pgfpathlineto{\pgfqpoint{3.891320in}{1.457240in}}%
\pgfpathlineto{\pgfqpoint{3.932529in}{1.467405in}}%
\pgfpathlineto{\pgfqpoint{3.973739in}{1.475261in}}%
\pgfpathlineto{\pgfqpoint{4.056157in}{1.463573in}}%
\pgfpathlineto{\pgfqpoint{4.097367in}{1.475191in}}%
\pgfpathlineto{\pgfqpoint{4.138576in}{1.480400in}}%
\pgfpathlineto{\pgfqpoint{4.179786in}{1.472637in}}%
\pgfpathlineto{\pgfqpoint{4.220995in}{1.475645in}}%
\pgfpathlineto{\pgfqpoint{4.262204in}{1.477705in}}%
\pgfpathlineto{\pgfqpoint{4.303414in}{1.485816in}}%
\pgfpathlineto{\pgfqpoint{4.385833in}{1.485363in}}%
\pgfpathlineto{\pgfqpoint{4.427042in}{1.485222in}}%
\pgfpathlineto{\pgfqpoint{4.468251in}{1.490933in}}%
\pgfpathlineto{\pgfqpoint{4.509461in}{1.499829in}}%
\pgfpathlineto{\pgfqpoint{4.550670in}{1.487977in}}%
\pgfpathlineto{\pgfqpoint{4.591880in}{1.488562in}}%
\pgfpathlineto{\pgfqpoint{4.633089in}{1.500272in}}%
\pgfpathlineto{\pgfqpoint{4.715508in}{1.493289in}}%
\pgfpathlineto{\pgfqpoint{4.756717in}{1.502349in}}%
\pgfpathlineto{\pgfqpoint{4.797926in}{1.503417in}}%
\pgfpathlineto{\pgfqpoint{4.880345in}{1.504125in}}%
\pgfpathlineto{\pgfqpoint{4.962764in}{1.521213in}}%
\pgfpathlineto{\pgfqpoint{5.003973in}{1.501633in}}%
\pgfpathlineto{\pgfqpoint{5.045183in}{1.512727in}}%
\pgfusepath{stroke}%
\end{pgfscope}%
\begin{pgfscope}%
\pgfpathrectangle{\pgfqpoint{0.588387in}{0.521603in}}{\pgfqpoint{4.669024in}{2.220246in}}%
\pgfusepath{clip}%
\pgfsetrectcap%
\pgfsetroundjoin%
\pgfsetlinewidth{1.505625pt}%
\pgfsetstrokecolor{currentstroke5}%
\pgfsetdash{}{0pt}%
\pgfpathmoveto{\pgfqpoint{0.800616in}{0.734939in}}%
\pgfpathlineto{\pgfqpoint{0.841825in}{0.734939in}}%
\pgfpathlineto{\pgfqpoint{0.883034in}{0.800698in}}%
\pgfpathlineto{\pgfqpoint{0.924244in}{0.850893in}}%
\pgfpathlineto{\pgfqpoint{0.965453in}{0.917849in}}%
\pgfpathlineto{\pgfqpoint{1.006663in}{0.953872in}}%
\pgfpathlineto{\pgfqpoint{1.047872in}{1.065500in}}%
\pgfpathlineto{\pgfqpoint{1.089081in}{1.075638in}}%
\pgfpathlineto{\pgfqpoint{1.130291in}{1.149867in}}%
\pgfpathlineto{\pgfqpoint{1.171500in}{1.150003in}}%
\pgfpathlineto{\pgfqpoint{1.212709in}{1.260251in}}%
\pgfpathlineto{\pgfqpoint{1.253919in}{1.213144in}}%
\pgfpathlineto{\pgfqpoint{1.295128in}{1.249153in}}%
\pgfpathlineto{\pgfqpoint{1.336338in}{1.390923in}}%
\pgfpathlineto{\pgfqpoint{1.377547in}{1.362067in}}%
\pgfpathlineto{\pgfqpoint{1.418756in}{1.321843in}}%
\pgfpathlineto{\pgfqpoint{1.459966in}{1.398021in}}%
\pgfpathlineto{\pgfqpoint{1.501175in}{1.366723in}}%
\pgfpathlineto{\pgfqpoint{1.542385in}{1.482056in}}%
\pgfpathlineto{\pgfqpoint{1.583594in}{1.343968in}}%
\pgfpathlineto{\pgfqpoint{1.624803in}{1.325668in}}%
\pgfpathlineto{\pgfqpoint{1.666013in}{1.649692in}}%
\pgfpathlineto{\pgfqpoint{1.707222in}{1.866701in}}%
\pgfpathlineto{\pgfqpoint{1.748432in}{1.639783in}}%
\pgfpathlineto{\pgfqpoint{1.789641in}{1.392107in}}%
\pgfpathlineto{\pgfqpoint{1.830850in}{1.738310in}}%
\pgfpathlineto{\pgfqpoint{1.872060in}{2.045317in}}%
\pgfpathlineto{\pgfqpoint{1.913269in}{1.625129in}}%
\pgfpathlineto{\pgfqpoint{1.954479in}{1.402640in}}%
\pgfpathlineto{\pgfqpoint{1.995688in}{1.664862in}}%
\pgfpathlineto{\pgfqpoint{2.036897in}{1.916262in}}%
\pgfpathlineto{\pgfqpoint{2.078107in}{2.027954in}}%
\pgfpathlineto{\pgfqpoint{2.119316in}{1.687891in}}%
\pgfpathlineto{\pgfqpoint{2.160525in}{1.872496in}}%
\pgfpathlineto{\pgfqpoint{2.201735in}{1.623814in}}%
\pgfpathlineto{\pgfqpoint{2.242944in}{1.656028in}}%
\pgfpathlineto{\pgfqpoint{2.284154in}{1.572045in}}%
\pgfpathlineto{\pgfqpoint{2.325363in}{1.673787in}}%
\pgfpathlineto{\pgfqpoint{2.366572in}{1.502655in}}%
\pgfpathlineto{\pgfqpoint{2.407782in}{1.595870in}}%
\pgfpathlineto{\pgfqpoint{2.448991in}{1.561862in}}%
\pgfpathlineto{\pgfqpoint{2.490201in}{1.740114in}}%
\pgfpathlineto{\pgfqpoint{2.531410in}{1.512223in}}%
\pgfpathlineto{\pgfqpoint{2.572619in}{1.929479in}}%
\pgfpathlineto{\pgfqpoint{2.613829in}{1.452192in}}%
\pgfpathlineto{\pgfqpoint{2.655038in}{1.679071in}}%
\pgfpathlineto{\pgfqpoint{2.696248in}{1.721895in}}%
\pgfpathlineto{\pgfqpoint{2.737457in}{2.015478in}}%
\pgfpathlineto{\pgfqpoint{2.778666in}{1.603185in}}%
\pgfpathlineto{\pgfqpoint{2.819876in}{1.499893in}}%
\pgfpathlineto{\pgfqpoint{2.861085in}{1.852993in}}%
\pgfpathlineto{\pgfqpoint{2.902295in}{1.871931in}}%
\pgfpathlineto{\pgfqpoint{2.943504in}{1.640111in}}%
\pgfpathlineto{\pgfqpoint{2.984713in}{1.591112in}}%
\pgfpathlineto{\pgfqpoint{3.025923in}{1.404936in}}%
\pgfpathlineto{\pgfqpoint{3.067132in}{1.864588in}}%
\pgfpathlineto{\pgfqpoint{3.108341in}{1.734236in}}%
\pgfpathlineto{\pgfqpoint{3.149551in}{1.579215in}}%
\pgfpathlineto{\pgfqpoint{3.190760in}{1.788120in}}%
\pgfpathlineto{\pgfqpoint{3.231970in}{1.980882in}}%
\pgfpathlineto{\pgfqpoint{3.273179in}{1.703469in}}%
\pgfpathlineto{\pgfqpoint{3.314388in}{1.925936in}}%
\pgfpathlineto{\pgfqpoint{3.355598in}{1.426266in}}%
\pgfpathlineto{\pgfqpoint{3.396807in}{1.771394in}}%
\pgfpathlineto{\pgfqpoint{3.438017in}{1.668925in}}%
\pgfpathlineto{\pgfqpoint{3.479226in}{1.568583in}}%
\pgfpathlineto{\pgfqpoint{3.520435in}{1.502706in}}%
\pgfpathlineto{\pgfqpoint{3.561645in}{1.941682in}}%
\pgfpathlineto{\pgfqpoint{3.602854in}{1.601236in}}%
\pgfpathlineto{\pgfqpoint{3.644064in}{1.687499in}}%
\pgfpathlineto{\pgfqpoint{3.685273in}{1.562864in}}%
\pgfpathlineto{\pgfqpoint{3.726482in}{1.922073in}}%
\pgfpathlineto{\pgfqpoint{3.767692in}{1.516699in}}%
\pgfpathlineto{\pgfqpoint{3.808901in}{1.782349in}}%
\pgfpathlineto{\pgfqpoint{3.891320in}{1.809474in}}%
\pgfpathlineto{\pgfqpoint{3.932529in}{1.522480in}}%
\pgfpathlineto{\pgfqpoint{3.973739in}{1.958600in}}%
\pgfpathlineto{\pgfqpoint{4.056157in}{1.838447in}}%
\pgfpathlineto{\pgfqpoint{4.097367in}{1.470911in}}%
\pgfpathlineto{\pgfqpoint{4.138576in}{1.639210in}}%
\pgfpathlineto{\pgfqpoint{4.179786in}{2.098621in}}%
\pgfpathlineto{\pgfqpoint{4.220995in}{1.718343in}}%
\pgfpathlineto{\pgfqpoint{4.262204in}{1.559048in}}%
\pgfpathlineto{\pgfqpoint{4.303414in}{1.530483in}}%
\pgfpathlineto{\pgfqpoint{4.385833in}{1.714003in}}%
\pgfpathlineto{\pgfqpoint{4.427042in}{1.709705in}}%
\pgfpathlineto{\pgfqpoint{4.468251in}{1.533753in}}%
\pgfpathlineto{\pgfqpoint{4.509461in}{1.483425in}}%
\pgfpathlineto{\pgfqpoint{4.550670in}{1.540762in}}%
\pgfpathlineto{\pgfqpoint{4.591880in}{2.088656in}}%
\pgfpathlineto{\pgfqpoint{4.633089in}{2.021893in}}%
\pgfpathlineto{\pgfqpoint{4.715508in}{2.025449in}}%
\pgfpathlineto{\pgfqpoint{4.756717in}{1.622098in}}%
\pgfpathlineto{\pgfqpoint{4.797926in}{1.624695in}}%
\pgfpathlineto{\pgfqpoint{4.880345in}{1.677688in}}%
\pgfpathlineto{\pgfqpoint{4.962764in}{2.043508in}}%
\pgfpathlineto{\pgfqpoint{5.003973in}{1.662667in}}%
\pgfpathlineto{\pgfqpoint{5.045183in}{1.577139in}}%
\pgfusepath{stroke}%
\end{pgfscope}%
\begin{pgfscope}%
\pgfpathrectangle{\pgfqpoint{0.588387in}{0.521603in}}{\pgfqpoint{4.669024in}{2.220246in}}%
\pgfusepath{clip}%
\pgfsetrectcap%
\pgfsetroundjoin%
\pgfsetlinewidth{1.505625pt}%
\pgfsetstrokecolor{currentstroke6}%
\pgfsetdash{}{0pt}%
\pgfpathmoveto{\pgfqpoint{0.800616in}{0.734939in}}%
\pgfpathlineto{\pgfqpoint{0.841825in}{0.734939in}}%
\pgfpathlineto{\pgfqpoint{0.883034in}{0.800698in}}%
\pgfpathlineto{\pgfqpoint{0.924244in}{0.856367in}}%
\pgfpathlineto{\pgfqpoint{0.965453in}{0.918886in}}%
\pgfpathlineto{\pgfqpoint{1.006663in}{0.938114in}}%
\pgfpathlineto{\pgfqpoint{1.047872in}{1.064522in}}%
\pgfpathlineto{\pgfqpoint{1.089081in}{1.088457in}}%
\pgfpathlineto{\pgfqpoint{1.130291in}{1.154344in}}%
\pgfpathlineto{\pgfqpoint{1.171500in}{1.148971in}}%
\pgfpathlineto{\pgfqpoint{1.212709in}{1.176322in}}%
\pgfpathlineto{\pgfqpoint{1.253919in}{1.193930in}}%
\pgfpathlineto{\pgfqpoint{1.295128in}{1.217177in}}%
\pgfpathlineto{\pgfqpoint{1.336338in}{1.271946in}}%
\pgfpathlineto{\pgfqpoint{1.377547in}{1.211694in}}%
\pgfpathlineto{\pgfqpoint{1.418756in}{1.230598in}}%
\pgfpathlineto{\pgfqpoint{1.459966in}{1.258135in}}%
\pgfpathlineto{\pgfqpoint{1.501175in}{1.280432in}}%
\pgfpathlineto{\pgfqpoint{1.542385in}{1.256073in}}%
\pgfpathlineto{\pgfqpoint{1.583594in}{1.279647in}}%
\pgfpathlineto{\pgfqpoint{1.624803in}{1.280304in}}%
\pgfpathlineto{\pgfqpoint{1.666013in}{1.311416in}}%
\pgfpathlineto{\pgfqpoint{1.707222in}{1.284595in}}%
\pgfpathlineto{\pgfqpoint{1.748432in}{1.305615in}}%
\pgfpathlineto{\pgfqpoint{1.789641in}{1.315272in}}%
\pgfpathlineto{\pgfqpoint{1.830850in}{1.325468in}}%
\pgfpathlineto{\pgfqpoint{1.872060in}{1.316221in}}%
\pgfpathlineto{\pgfqpoint{1.913269in}{1.312290in}}%
\pgfpathlineto{\pgfqpoint{1.954479in}{1.338279in}}%
\pgfpathlineto{\pgfqpoint{1.995688in}{1.364292in}}%
\pgfpathlineto{\pgfqpoint{2.036897in}{1.328144in}}%
\pgfpathlineto{\pgfqpoint{2.078107in}{1.337757in}}%
\pgfpathlineto{\pgfqpoint{2.119316in}{1.350572in}}%
\pgfpathlineto{\pgfqpoint{2.160525in}{1.358129in}}%
\pgfpathlineto{\pgfqpoint{2.201735in}{1.363338in}}%
\pgfpathlineto{\pgfqpoint{2.242944in}{1.378428in}}%
\pgfpathlineto{\pgfqpoint{2.284154in}{1.359706in}}%
\pgfpathlineto{\pgfqpoint{2.325363in}{1.398611in}}%
\pgfpathlineto{\pgfqpoint{2.366572in}{1.370665in}}%
\pgfpathlineto{\pgfqpoint{2.407782in}{1.369000in}}%
\pgfpathlineto{\pgfqpoint{2.448991in}{1.376182in}}%
\pgfpathlineto{\pgfqpoint{2.490201in}{1.401150in}}%
\pgfpathlineto{\pgfqpoint{2.531410in}{1.377309in}}%
\pgfpathlineto{\pgfqpoint{2.572619in}{1.385677in}}%
\pgfpathlineto{\pgfqpoint{2.613829in}{1.392496in}}%
\pgfpathlineto{\pgfqpoint{2.655038in}{1.407458in}}%
\pgfpathlineto{\pgfqpoint{2.696248in}{1.386175in}}%
\pgfpathlineto{\pgfqpoint{2.737457in}{1.399976in}}%
\pgfpathlineto{\pgfqpoint{2.778666in}{1.399736in}}%
\pgfpathlineto{\pgfqpoint{2.819876in}{1.427285in}}%
\pgfpathlineto{\pgfqpoint{2.861085in}{1.399351in}}%
\pgfpathlineto{\pgfqpoint{2.902295in}{1.407570in}}%
\pgfpathlineto{\pgfqpoint{2.943504in}{1.421397in}}%
\pgfpathlineto{\pgfqpoint{2.984713in}{1.450184in}}%
\pgfpathlineto{\pgfqpoint{3.025923in}{1.420452in}}%
\pgfpathlineto{\pgfqpoint{3.067132in}{1.417251in}}%
\pgfpathlineto{\pgfqpoint{3.108341in}{1.426393in}}%
\pgfpathlineto{\pgfqpoint{3.149551in}{1.432787in}}%
\pgfpathlineto{\pgfqpoint{3.190760in}{1.420896in}}%
\pgfpathlineto{\pgfqpoint{3.231970in}{1.429067in}}%
\pgfpathlineto{\pgfqpoint{3.273179in}{1.428984in}}%
\pgfpathlineto{\pgfqpoint{3.314388in}{1.445959in}}%
\pgfpathlineto{\pgfqpoint{3.355598in}{1.427970in}}%
\pgfpathlineto{\pgfqpoint{3.396807in}{1.433852in}}%
\pgfpathlineto{\pgfqpoint{3.438017in}{1.432429in}}%
\pgfpathlineto{\pgfqpoint{3.479226in}{1.464499in}}%
\pgfpathlineto{\pgfqpoint{3.520435in}{1.469168in}}%
\pgfpathlineto{\pgfqpoint{3.561645in}{1.446449in}}%
\pgfpathlineto{\pgfqpoint{3.602854in}{1.445284in}}%
\pgfpathlineto{\pgfqpoint{3.644064in}{1.464303in}}%
\pgfpathlineto{\pgfqpoint{3.685273in}{1.453856in}}%
\pgfpathlineto{\pgfqpoint{3.726482in}{1.454433in}}%
\pgfpathlineto{\pgfqpoint{3.767692in}{1.464499in}}%
\pgfpathlineto{\pgfqpoint{3.808901in}{1.472064in}}%
\pgfpathlineto{\pgfqpoint{3.891320in}{1.465781in}}%
\pgfpathlineto{\pgfqpoint{3.932529in}{1.470738in}}%
\pgfpathlineto{\pgfqpoint{3.973739in}{1.476662in}}%
\pgfpathlineto{\pgfqpoint{4.056157in}{1.471462in}}%
\pgfpathlineto{\pgfqpoint{4.097367in}{1.541512in}}%
\pgfpathlineto{\pgfqpoint{4.138576in}{1.488380in}}%
\pgfpathlineto{\pgfqpoint{4.179786in}{1.499102in}}%
\pgfpathlineto{\pgfqpoint{4.220995in}{1.482061in}}%
\pgfpathlineto{\pgfqpoint{4.262204in}{1.480181in}}%
\pgfpathlineto{\pgfqpoint{4.303414in}{1.493352in}}%
\pgfpathlineto{\pgfqpoint{4.385833in}{1.482851in}}%
\pgfpathlineto{\pgfqpoint{4.427042in}{1.489144in}}%
\pgfpathlineto{\pgfqpoint{4.468251in}{1.503772in}}%
\pgfpathlineto{\pgfqpoint{4.509461in}{1.486605in}}%
\pgfpathlineto{\pgfqpoint{4.550670in}{1.506406in}}%
\pgfpathlineto{\pgfqpoint{4.591880in}{1.492784in}}%
\pgfpathlineto{\pgfqpoint{4.633089in}{1.500914in}}%
\pgfpathlineto{\pgfqpoint{4.715508in}{1.510703in}}%
\pgfpathlineto{\pgfqpoint{4.756717in}{1.498005in}}%
\pgfpathlineto{\pgfqpoint{4.797926in}{1.516862in}}%
\pgfpathlineto{\pgfqpoint{4.880345in}{1.509341in}}%
\pgfpathlineto{\pgfqpoint{4.962764in}{1.532868in}}%
\pgfpathlineto{\pgfqpoint{5.003973in}{1.518001in}}%
\pgfpathlineto{\pgfqpoint{5.045183in}{1.508655in}}%
\pgfusepath{stroke}%
\end{pgfscope}%
\begin{pgfscope}%
\pgfpathrectangle{\pgfqpoint{0.588387in}{0.521603in}}{\pgfqpoint{4.669024in}{2.220246in}}%
\pgfusepath{clip}%
\pgfsetrectcap%
\pgfsetroundjoin%
\pgfsetlinewidth{1.505625pt}%
\pgfsetstrokecolor{currentstroke7}%
\pgfsetdash{}{0pt}%
\pgfpathmoveto{\pgfqpoint{0.800616in}{0.734939in}}%
\pgfpathlineto{\pgfqpoint{0.841825in}{0.734939in}}%
\pgfpathlineto{\pgfqpoint{0.883034in}{0.797972in}}%
\pgfpathlineto{\pgfqpoint{0.924244in}{0.855494in}}%
\pgfpathlineto{\pgfqpoint{0.965453in}{0.918886in}}%
\pgfpathlineto{\pgfqpoint{1.006663in}{0.938114in}}%
\pgfpathlineto{\pgfqpoint{1.047872in}{1.064522in}}%
\pgfpathlineto{\pgfqpoint{1.089081in}{1.088457in}}%
\pgfpathlineto{\pgfqpoint{1.130291in}{1.154344in}}%
\pgfpathlineto{\pgfqpoint{1.171500in}{1.148971in}}%
\pgfpathlineto{\pgfqpoint{1.212709in}{1.274014in}}%
\pgfpathlineto{\pgfqpoint{1.253919in}{1.272088in}}%
\pgfpathlineto{\pgfqpoint{1.295128in}{1.340233in}}%
\pgfpathlineto{\pgfqpoint{1.336338in}{1.402383in}}%
\pgfpathlineto{\pgfqpoint{1.377547in}{1.504080in}}%
\pgfpathlineto{\pgfqpoint{1.418756in}{1.349204in}}%
\pgfpathlineto{\pgfqpoint{1.459966in}{1.474314in}}%
\pgfpathlineto{\pgfqpoint{1.501175in}{1.400337in}}%
\pgfpathlineto{\pgfqpoint{1.542385in}{1.605701in}}%
\pgfpathlineto{\pgfqpoint{1.583594in}{1.378311in}}%
\pgfpathlineto{\pgfqpoint{1.624803in}{1.299365in}}%
\pgfpathlineto{\pgfqpoint{1.666013in}{1.637477in}}%
\pgfpathlineto{\pgfqpoint{1.707222in}{1.821441in}}%
\pgfpathlineto{\pgfqpoint{1.748432in}{1.632918in}}%
\pgfpathlineto{\pgfqpoint{1.789641in}{1.817466in}}%
\pgfpathlineto{\pgfqpoint{1.830850in}{1.814673in}}%
\pgfpathlineto{\pgfqpoint{1.872060in}{1.898525in}}%
\pgfpathlineto{\pgfqpoint{1.913269in}{1.711716in}}%
\pgfpathlineto{\pgfqpoint{1.954479in}{1.758623in}}%
\pgfpathlineto{\pgfqpoint{1.995688in}{2.049367in}}%
\pgfpathlineto{\pgfqpoint{2.036897in}{1.789780in}}%
\pgfpathlineto{\pgfqpoint{2.078107in}{1.970361in}}%
\pgfpathlineto{\pgfqpoint{2.119316in}{2.022192in}}%
\pgfpathlineto{\pgfqpoint{2.160525in}{1.716713in}}%
\pgfpathlineto{\pgfqpoint{2.201735in}{2.116866in}}%
\pgfpathlineto{\pgfqpoint{2.242944in}{1.817645in}}%
\pgfpathlineto{\pgfqpoint{2.284154in}{1.905632in}}%
\pgfpathlineto{\pgfqpoint{2.325363in}{1.751439in}}%
\pgfpathlineto{\pgfqpoint{2.366572in}{2.097272in}}%
\pgfpathlineto{\pgfqpoint{2.407782in}{1.808172in}}%
\pgfpathlineto{\pgfqpoint{2.448991in}{1.897531in}}%
\pgfpathlineto{\pgfqpoint{2.490201in}{1.632461in}}%
\pgfpathlineto{\pgfqpoint{2.531410in}{1.795068in}}%
\pgfpathlineto{\pgfqpoint{2.572619in}{1.700926in}}%
\pgfpathlineto{\pgfqpoint{2.613829in}{1.580534in}}%
\pgfpathlineto{\pgfqpoint{2.655038in}{1.803911in}}%
\pgfpathlineto{\pgfqpoint{2.696248in}{2.213446in}}%
\pgfpathlineto{\pgfqpoint{2.737457in}{1.843524in}}%
\pgfpathlineto{\pgfqpoint{2.778666in}{1.514156in}}%
\pgfpathlineto{\pgfqpoint{2.819876in}{1.541016in}}%
\pgfpathlineto{\pgfqpoint{2.861085in}{1.920331in}}%
\pgfpathlineto{\pgfqpoint{2.902295in}{1.951612in}}%
\pgfpathlineto{\pgfqpoint{2.943504in}{1.837121in}}%
\pgfpathlineto{\pgfqpoint{2.984713in}{1.684364in}}%
\pgfpathlineto{\pgfqpoint{3.025923in}{1.543603in}}%
\pgfpathlineto{\pgfqpoint{3.067132in}{1.788870in}}%
\pgfpathlineto{\pgfqpoint{3.108341in}{2.012505in}}%
\pgfpathlineto{\pgfqpoint{3.149551in}{1.936794in}}%
\pgfpathlineto{\pgfqpoint{3.190760in}{1.516137in}}%
\pgfpathlineto{\pgfqpoint{3.231970in}{1.754987in}}%
\pgfpathlineto{\pgfqpoint{3.273179in}{2.071877in}}%
\pgfpathlineto{\pgfqpoint{3.314388in}{1.661993in}}%
\pgfpathlineto{\pgfqpoint{3.355598in}{1.422803in}}%
\pgfpathlineto{\pgfqpoint{3.396807in}{1.928619in}}%
\pgfpathlineto{\pgfqpoint{3.438017in}{1.839103in}}%
\pgfpathlineto{\pgfqpoint{3.479226in}{1.737186in}}%
\pgfpathlineto{\pgfqpoint{3.520435in}{1.491262in}}%
\pgfpathlineto{\pgfqpoint{3.561645in}{1.992393in}}%
\pgfpathlineto{\pgfqpoint{3.602854in}{2.035410in}}%
\pgfpathlineto{\pgfqpoint{3.644064in}{1.883114in}}%
\pgfpathlineto{\pgfqpoint{3.685273in}{1.994174in}}%
\pgfpathlineto{\pgfqpoint{3.726482in}{2.003858in}}%
\pgfpathlineto{\pgfqpoint{3.767692in}{2.136205in}}%
\pgfpathlineto{\pgfqpoint{3.808901in}{1.716665in}}%
\pgfpathlineto{\pgfqpoint{3.891320in}{1.916911in}}%
\pgfpathlineto{\pgfqpoint{3.932529in}{1.564860in}}%
\pgfpathlineto{\pgfqpoint{3.973739in}{1.671736in}}%
\pgfpathlineto{\pgfqpoint{4.056157in}{1.519454in}}%
\pgfpathlineto{\pgfqpoint{4.097367in}{1.470042in}}%
\pgfpathlineto{\pgfqpoint{4.138576in}{1.803703in}}%
\pgfpathlineto{\pgfqpoint{4.179786in}{2.339817in}}%
\pgfpathlineto{\pgfqpoint{4.220995in}{1.649663in}}%
\pgfpathlineto{\pgfqpoint{4.262204in}{1.615121in}}%
\pgfpathlineto{\pgfqpoint{4.303414in}{1.695797in}}%
\pgfpathlineto{\pgfqpoint{4.385833in}{1.630833in}}%
\pgfpathlineto{\pgfqpoint{4.427042in}{1.933087in}}%
\pgfpathlineto{\pgfqpoint{4.468251in}{1.977239in}}%
\pgfpathlineto{\pgfqpoint{4.509461in}{2.044256in}}%
\pgfpathlineto{\pgfqpoint{4.550670in}{1.971183in}}%
\pgfpathlineto{\pgfqpoint{4.591880in}{1.621070in}}%
\pgfpathlineto{\pgfqpoint{4.633089in}{2.058796in}}%
\pgfpathlineto{\pgfqpoint{4.715508in}{2.044546in}}%
\pgfpathlineto{\pgfqpoint{4.756717in}{1.794361in}}%
\pgfpathlineto{\pgfqpoint{4.797926in}{1.858002in}}%
\pgfpathlineto{\pgfqpoint{4.880345in}{1.657054in}}%
\pgfpathlineto{\pgfqpoint{4.962764in}{1.630092in}}%
\pgfpathlineto{\pgfqpoint{5.003973in}{1.810526in}}%
\pgfpathlineto{\pgfqpoint{5.045183in}{1.577139in}}%
\pgfusepath{stroke}%
\end{pgfscope}%
\begin{pgfscope}%
\pgfpathrectangle{\pgfqpoint{0.588387in}{0.521603in}}{\pgfqpoint{4.669024in}{2.220246in}}%
\pgfusepath{clip}%
\pgfsetrectcap%
\pgfsetroundjoin%
\pgfsetlinewidth{1.505625pt}%
\definecolor{currentstroke}{rgb}{0.498039,0.498039,0.498039}%
\pgfsetstrokecolor{currentstroke}%
\pgfsetdash{}{0pt}%
\pgfpathmoveto{\pgfqpoint{0.800616in}{0.734939in}}%
\pgfpathlineto{\pgfqpoint{0.841825in}{0.741060in}}%
\pgfpathlineto{\pgfqpoint{0.883034in}{0.806016in}}%
\pgfpathlineto{\pgfqpoint{0.924244in}{0.838518in}}%
\pgfpathlineto{\pgfqpoint{0.965453in}{0.892672in}}%
\pgfpathlineto{\pgfqpoint{1.006663in}{0.948762in}}%
\pgfpathlineto{\pgfqpoint{1.047872in}{1.080098in}}%
\pgfpathlineto{\pgfqpoint{1.089081in}{1.144246in}}%
\pgfpathlineto{\pgfqpoint{1.130291in}{1.177183in}}%
\pgfpathlineto{\pgfqpoint{1.171500in}{1.214902in}}%
\pgfpathlineto{\pgfqpoint{1.212709in}{1.203202in}}%
\pgfpathlineto{\pgfqpoint{1.253919in}{1.262188in}}%
\pgfpathlineto{\pgfqpoint{1.295128in}{1.244530in}}%
\pgfpathlineto{\pgfqpoint{1.336338in}{1.300533in}}%
\pgfpathlineto{\pgfqpoint{1.377547in}{1.239762in}}%
\pgfpathlineto{\pgfqpoint{1.418756in}{1.271729in}}%
\pgfpathlineto{\pgfqpoint{1.459966in}{1.285241in}}%
\pgfpathlineto{\pgfqpoint{1.501175in}{1.327816in}}%
\pgfpathlineto{\pgfqpoint{1.542385in}{1.274076in}}%
\pgfpathlineto{\pgfqpoint{1.583594in}{1.301721in}}%
\pgfpathlineto{\pgfqpoint{1.624803in}{1.312628in}}%
\pgfpathlineto{\pgfqpoint{1.666013in}{1.366909in}}%
\pgfpathlineto{\pgfqpoint{1.707222in}{1.316961in}}%
\pgfpathlineto{\pgfqpoint{1.748432in}{1.334613in}}%
\pgfpathlineto{\pgfqpoint{1.789641in}{1.333459in}}%
\pgfpathlineto{\pgfqpoint{1.830850in}{1.365856in}}%
\pgfpathlineto{\pgfqpoint{1.872060in}{1.338732in}}%
\pgfpathlineto{\pgfqpoint{1.913269in}{1.341274in}}%
\pgfpathlineto{\pgfqpoint{1.954479in}{1.354471in}}%
\pgfpathlineto{\pgfqpoint{1.995688in}{1.377066in}}%
\pgfpathlineto{\pgfqpoint{2.036897in}{1.355579in}}%
\pgfpathlineto{\pgfqpoint{2.078107in}{1.377079in}}%
\pgfpathlineto{\pgfqpoint{2.119316in}{1.382689in}}%
\pgfpathlineto{\pgfqpoint{2.160525in}{1.392036in}}%
\pgfpathlineto{\pgfqpoint{2.201735in}{1.386112in}}%
\pgfpathlineto{\pgfqpoint{2.242944in}{1.376922in}}%
\pgfpathlineto{\pgfqpoint{2.284154in}{1.382078in}}%
\pgfpathlineto{\pgfqpoint{2.325363in}{1.412776in}}%
\pgfpathlineto{\pgfqpoint{2.366572in}{1.402970in}}%
\pgfpathlineto{\pgfqpoint{2.407782in}{1.403740in}}%
\pgfpathlineto{\pgfqpoint{2.448991in}{1.419156in}}%
\pgfpathlineto{\pgfqpoint{2.490201in}{1.435481in}}%
\pgfpathlineto{\pgfqpoint{2.531410in}{1.416407in}}%
\pgfpathlineto{\pgfqpoint{2.572619in}{1.415793in}}%
\pgfpathlineto{\pgfqpoint{2.613829in}{1.438448in}}%
\pgfpathlineto{\pgfqpoint{2.655038in}{1.452879in}}%
\pgfpathlineto{\pgfqpoint{2.696248in}{1.424544in}}%
\pgfpathlineto{\pgfqpoint{2.737457in}{1.429003in}}%
\pgfpathlineto{\pgfqpoint{2.778666in}{1.424791in}}%
\pgfpathlineto{\pgfqpoint{2.819876in}{1.447003in}}%
\pgfpathlineto{\pgfqpoint{2.861085in}{1.424791in}}%
\pgfpathlineto{\pgfqpoint{2.902295in}{1.423206in}}%
\pgfpathlineto{\pgfqpoint{2.943504in}{1.433634in}}%
\pgfpathlineto{\pgfqpoint{2.984713in}{1.442552in}}%
\pgfpathlineto{\pgfqpoint{3.025923in}{1.425980in}}%
\pgfpathlineto{\pgfqpoint{3.067132in}{1.446340in}}%
\pgfpathlineto{\pgfqpoint{3.108341in}{1.459846in}}%
\pgfpathlineto{\pgfqpoint{3.149551in}{1.477633in}}%
\pgfpathlineto{\pgfqpoint{3.190760in}{1.461194in}}%
\pgfpathlineto{\pgfqpoint{3.231970in}{1.462960in}}%
\pgfpathlineto{\pgfqpoint{3.273179in}{1.465981in}}%
\pgfpathlineto{\pgfqpoint{3.314388in}{1.475097in}}%
\pgfpathlineto{\pgfqpoint{3.355598in}{1.445622in}}%
\pgfpathlineto{\pgfqpoint{3.396807in}{1.474344in}}%
\pgfpathlineto{\pgfqpoint{3.438017in}{1.466961in}}%
\pgfpathlineto{\pgfqpoint{3.479226in}{1.465790in}}%
\pgfpathlineto{\pgfqpoint{3.520435in}{1.460618in}}%
\pgfpathlineto{\pgfqpoint{3.561645in}{1.469566in}}%
\pgfpathlineto{\pgfqpoint{3.602854in}{1.479360in}}%
\pgfpathlineto{\pgfqpoint{3.644064in}{1.491313in}}%
\pgfpathlineto{\pgfqpoint{3.685273in}{1.473208in}}%
\pgfpathlineto{\pgfqpoint{3.726482in}{1.481920in}}%
\pgfpathlineto{\pgfqpoint{3.767692in}{1.508998in}}%
\pgfpathlineto{\pgfqpoint{3.808901in}{1.482688in}}%
\pgfpathlineto{\pgfqpoint{3.891320in}{1.483236in}}%
\pgfpathlineto{\pgfqpoint{3.932529in}{1.476369in}}%
\pgfpathlineto{\pgfqpoint{3.973739in}{1.506232in}}%
\pgfpathlineto{\pgfqpoint{4.056157in}{1.496934in}}%
\pgfpathlineto{\pgfqpoint{4.097367in}{1.527451in}}%
\pgfpathlineto{\pgfqpoint{4.138576in}{1.514548in}}%
\pgfpathlineto{\pgfqpoint{4.179786in}{1.488172in}}%
\pgfpathlineto{\pgfqpoint{4.220995in}{1.504532in}}%
\pgfpathlineto{\pgfqpoint{4.262204in}{1.509341in}}%
\pgfpathlineto{\pgfqpoint{4.303414in}{1.503594in}}%
\pgfpathlineto{\pgfqpoint{4.385833in}{1.554894in}}%
\pgfpathlineto{\pgfqpoint{4.427042in}{1.503417in}}%
\pgfpathlineto{\pgfqpoint{4.468251in}{1.534047in}}%
\pgfpathlineto{\pgfqpoint{4.509461in}{1.489724in}}%
\pgfpathlineto{\pgfqpoint{4.550670in}{1.514559in}}%
\pgfpathlineto{\pgfqpoint{4.591880in}{1.544569in}}%
\pgfpathlineto{\pgfqpoint{4.633089in}{1.526565in}}%
\pgfpathlineto{\pgfqpoint{4.715508in}{1.530783in}}%
\pgfpathlineto{\pgfqpoint{4.756717in}{1.510364in}}%
\pgfpathlineto{\pgfqpoint{4.797926in}{1.542072in}}%
\pgfpathlineto{\pgfqpoint{4.880345in}{1.518864in}}%
\pgfpathlineto{\pgfqpoint{4.962764in}{1.560552in}}%
\pgfpathlineto{\pgfqpoint{5.003973in}{1.556510in}}%
\pgfpathlineto{\pgfqpoint{5.045183in}{1.558037in}}%
\pgfusepath{stroke}%
\end{pgfscope}%
\begin{pgfscope}%
\pgfpathrectangle{\pgfqpoint{0.588387in}{0.521603in}}{\pgfqpoint{4.669024in}{2.220246in}}%
\pgfusepath{clip}%
\pgfsetrectcap%
\pgfsetroundjoin%
\pgfsetlinewidth{1.505625pt}%
\definecolor{currentstroke}{rgb}{0.737255,0.741176,0.133333}%
\pgfsetstrokecolor{currentstroke}%
\pgfsetdash{}{0pt}%
\pgfpathmoveto{\pgfqpoint{0.800616in}{0.734939in}}%
\pgfpathlineto{\pgfqpoint{0.841825in}{0.741060in}}%
\pgfpathlineto{\pgfqpoint{0.883034in}{0.797827in}}%
\pgfpathlineto{\pgfqpoint{0.924244in}{0.841856in}}%
\pgfpathlineto{\pgfqpoint{0.965453in}{0.894258in}}%
\pgfpathlineto{\pgfqpoint{1.006663in}{0.949303in}}%
\pgfpathlineto{\pgfqpoint{1.047872in}{1.080098in}}%
\pgfpathlineto{\pgfqpoint{1.089081in}{1.144246in}}%
\pgfpathlineto{\pgfqpoint{1.130291in}{1.177183in}}%
\pgfpathlineto{\pgfqpoint{1.171500in}{1.214902in}}%
\pgfpathlineto{\pgfqpoint{1.212709in}{1.287249in}}%
\pgfpathlineto{\pgfqpoint{1.253919in}{1.276953in}}%
\pgfpathlineto{\pgfqpoint{1.295128in}{1.372230in}}%
\pgfpathlineto{\pgfqpoint{1.336338in}{1.441169in}}%
\pgfpathlineto{\pgfqpoint{1.377547in}{1.566275in}}%
\pgfpathlineto{\pgfqpoint{1.418756in}{1.462047in}}%
\pgfpathlineto{\pgfqpoint{1.459966in}{1.522671in}}%
\pgfpathlineto{\pgfqpoint{1.501175in}{1.707610in}}%
\pgfpathlineto{\pgfqpoint{1.542385in}{1.843612in}}%
\pgfpathlineto{\pgfqpoint{1.583594in}{1.617805in}}%
\pgfpathlineto{\pgfqpoint{1.624803in}{1.897000in}}%
\pgfpathlineto{\pgfqpoint{1.666013in}{1.909458in}}%
\pgfpathlineto{\pgfqpoint{1.707222in}{1.957997in}}%
\pgfpathlineto{\pgfqpoint{1.748432in}{2.012697in}}%
\pgfpathlineto{\pgfqpoint{1.789641in}{1.923858in}}%
\pgfpathlineto{\pgfqpoint{1.830850in}{2.066715in}}%
\pgfpathlineto{\pgfqpoint{1.872060in}{2.362602in}}%
\pgfpathlineto{\pgfqpoint{1.913269in}{2.198236in}}%
\pgfpathlineto{\pgfqpoint{1.954479in}{2.079235in}}%
\pgfpathlineto{\pgfqpoint{1.995688in}{2.184178in}}%
\pgfpathlineto{\pgfqpoint{2.036897in}{2.109633in}}%
\pgfpathlineto{\pgfqpoint{2.078107in}{2.189836in}}%
\pgfpathlineto{\pgfqpoint{2.119316in}{2.278080in}}%
\pgfpathlineto{\pgfqpoint{2.160525in}{2.062099in}}%
\pgfpathlineto{\pgfqpoint{2.201735in}{1.981788in}}%
\pgfpathlineto{\pgfqpoint{2.242944in}{1.906033in}}%
\pgfpathlineto{\pgfqpoint{2.284154in}{1.879204in}}%
\pgfpathlineto{\pgfqpoint{2.325363in}{2.095488in}}%
\pgfpathlineto{\pgfqpoint{2.366572in}{1.886977in}}%
\pgfpathlineto{\pgfqpoint{2.407782in}{1.879041in}}%
\pgfpathlineto{\pgfqpoint{2.448991in}{1.442552in}}%
\pgfpathlineto{\pgfqpoint{2.490201in}{1.886354in}}%
\pgfpathlineto{\pgfqpoint{2.531410in}{2.267826in}}%
\pgfpathlineto{\pgfqpoint{2.572619in}{2.089082in}}%
\pgfpathlineto{\pgfqpoint{2.613829in}{1.943134in}}%
\pgfpathlineto{\pgfqpoint{2.655038in}{1.929190in}}%
\pgfpathlineto{\pgfqpoint{2.696248in}{2.047531in}}%
\pgfpathlineto{\pgfqpoint{2.737457in}{2.057596in}}%
\pgfpathlineto{\pgfqpoint{2.778666in}{2.081849in}}%
\pgfpathlineto{\pgfqpoint{2.819876in}{1.590502in}}%
\pgfpathlineto{\pgfqpoint{2.861085in}{1.955466in}}%
\pgfpathlineto{\pgfqpoint{2.902295in}{1.847368in}}%
\pgfpathlineto{\pgfqpoint{2.943504in}{1.411322in}}%
\pgfpathlineto{\pgfqpoint{2.984713in}{1.654190in}}%
\pgfpathlineto{\pgfqpoint{3.025923in}{1.785445in}}%
\pgfpathlineto{\pgfqpoint{3.067132in}{1.958578in}}%
\pgfpathlineto{\pgfqpoint{3.108341in}{1.988793in}}%
\pgfpathlineto{\pgfqpoint{3.149551in}{2.113641in}}%
\pgfpathlineto{\pgfqpoint{3.190760in}{1.739335in}}%
\pgfpathlineto{\pgfqpoint{3.231970in}{1.734968in}}%
\pgfpathlineto{\pgfqpoint{3.273179in}{1.785133in}}%
\pgfpathlineto{\pgfqpoint{3.314388in}{2.072562in}}%
\pgfpathlineto{\pgfqpoint{3.355598in}{2.414232in}}%
\pgfpathlineto{\pgfqpoint{3.396807in}{1.959685in}}%
\pgfpathlineto{\pgfqpoint{3.438017in}{1.435155in}}%
\pgfpathlineto{\pgfqpoint{3.479226in}{1.637932in}}%
\pgfpathlineto{\pgfqpoint{3.561645in}{1.771112in}}%
\pgfpathlineto{\pgfqpoint{3.644064in}{1.781894in}}%
\pgfpathlineto{\pgfqpoint{3.685273in}{1.451592in}}%
\pgfpathlineto{\pgfqpoint{3.726482in}{1.826604in}}%
\pgfpathlineto{\pgfqpoint{3.767692in}{2.069943in}}%
\pgfpathlineto{\pgfqpoint{3.808901in}{1.721546in}}%
\pgfpathlineto{\pgfqpoint{3.891320in}{1.661287in}}%
\pgfpathlineto{\pgfqpoint{3.932529in}{1.748359in}}%
\pgfpathlineto{\pgfqpoint{3.973739in}{1.719387in}}%
\pgfpathlineto{\pgfqpoint{4.056157in}{1.707687in}}%
\pgfpathlineto{\pgfqpoint{4.097367in}{2.320067in}}%
\pgfpathlineto{\pgfqpoint{4.138576in}{1.836439in}}%
\pgfpathlineto{\pgfqpoint{4.179786in}{1.468289in}}%
\pgfpathlineto{\pgfqpoint{4.220995in}{1.718936in}}%
\pgfpathlineto{\pgfqpoint{4.303414in}{1.519293in}}%
\pgfpathlineto{\pgfqpoint{4.385833in}{1.848669in}}%
\pgfpathlineto{\pgfqpoint{4.427042in}{1.482620in}}%
\pgfpathlineto{\pgfqpoint{4.468251in}{1.761501in}}%
\pgfpathlineto{\pgfqpoint{4.550670in}{1.741148in}}%
\pgfpathlineto{\pgfqpoint{4.591880in}{1.566432in}}%
\pgfpathlineto{\pgfqpoint{4.633089in}{1.798249in}}%
\pgfpathlineto{\pgfqpoint{4.715508in}{1.537823in}}%
\pgfpathlineto{\pgfqpoint{4.797926in}{1.617363in}}%
\pgfpathlineto{\pgfqpoint{4.880345in}{1.695525in}}%
\pgfpathlineto{\pgfqpoint{4.962764in}{1.518001in}}%
\pgfpathlineto{\pgfqpoint{5.003973in}{1.630092in}}%
\pgfusepath{stroke}%
\end{pgfscope}%
\begin{pgfscope}%
\pgfsetrectcap%
\pgfsetmiterjoin%
\pgfsetlinewidth{0.803000pt}%
\definecolor{currentstroke}{rgb}{0.000000,0.000000,0.000000}%
\pgfsetstrokecolor{currentstroke}%
\pgfsetdash{}{0pt}%
\pgfpathmoveto{\pgfqpoint{0.588387in}{0.521603in}}%
\pgfpathlineto{\pgfqpoint{0.588387in}{2.741849in}}%
\pgfusepath{stroke}%
\end{pgfscope}%
\begin{pgfscope}%
\pgfsetrectcap%
\pgfsetmiterjoin%
\pgfsetlinewidth{0.803000pt}%
\definecolor{currentstroke}{rgb}{0.000000,0.000000,0.000000}%
\pgfsetstrokecolor{currentstroke}%
\pgfsetdash{}{0pt}%
\pgfpathmoveto{\pgfqpoint{5.257411in}{0.521603in}}%
\pgfpathlineto{\pgfqpoint{5.257411in}{2.741849in}}%
\pgfusepath{stroke}%
\end{pgfscope}%
\begin{pgfscope}%
\pgfsetrectcap%
\pgfsetmiterjoin%
\pgfsetlinewidth{0.803000pt}%
\definecolor{currentstroke}{rgb}{0.000000,0.000000,0.000000}%
\pgfsetstrokecolor{currentstroke}%
\pgfsetdash{}{0pt}%
\pgfpathmoveto{\pgfqpoint{0.588387in}{0.521603in}}%
\pgfpathlineto{\pgfqpoint{5.257411in}{0.521603in}}%
\pgfusepath{stroke}%
\end{pgfscope}%
\begin{pgfscope}%
\pgfsetrectcap%
\pgfsetmiterjoin%
\pgfsetlinewidth{0.803000pt}%
\definecolor{currentstroke}{rgb}{0.000000,0.000000,0.000000}%
\pgfsetstrokecolor{currentstroke}%
\pgfsetdash{}{0pt}%
\pgfpathmoveto{\pgfqpoint{0.588387in}{2.741849in}}%
\pgfpathlineto{\pgfqpoint{5.257411in}{2.741849in}}%
\pgfusepath{stroke}%
\end{pgfscope}%
\begin{pgfscope}%
\pgfsetbuttcap%
\pgfsetmiterjoin%
\definecolor{currentfill}{rgb}{1.000000,1.000000,1.000000}%
\pgfsetfillcolor{currentfill}%
\pgfsetfillopacity{0.800000}%
\pgfsetlinewidth{1.003750pt}%
\definecolor{currentstroke}{rgb}{0.800000,0.800000,0.800000}%
\pgfsetstrokecolor{currentstroke}%
\pgfsetstrokeopacity{0.800000}%
\pgfsetdash{}{0pt}%
\pgfpathmoveto{\pgfqpoint{5.344911in}{0.969732in}}%
\pgfpathlineto{\pgfqpoint{8.259376in}{0.969732in}}%
\pgfpathquadraticcurveto{\pgfqpoint{8.284376in}{0.969732in}}{\pgfqpoint{8.284376in}{0.994732in}}%
\pgfpathlineto{\pgfqpoint{8.284376in}{2.654349in}}%
\pgfpathquadraticcurveto{\pgfqpoint{8.284376in}{2.679349in}}{\pgfqpoint{8.259376in}{2.679349in}}%
\pgfpathlineto{\pgfqpoint{5.344911in}{2.679349in}}%
\pgfpathquadraticcurveto{\pgfqpoint{5.319911in}{2.679349in}}{\pgfqpoint{5.319911in}{2.654349in}}%
\pgfpathlineto{\pgfqpoint{5.319911in}{0.994732in}}%
\pgfpathquadraticcurveto{\pgfqpoint{5.319911in}{0.969732in}}{\pgfqpoint{5.344911in}{0.969732in}}%
\pgfpathlineto{\pgfqpoint{5.344911in}{0.969732in}}%
\pgfpathclose%
\pgfusepath{stroke,fill}%
\end{pgfscope}%
\begin{pgfscope}%
\pgfsetrectcap%
\pgfsetroundjoin%
\pgfsetlinewidth{1.505625pt}%
\pgfsetstrokecolor{currentstroke3}%
\pgfsetdash{}{0pt}%
\pgfpathmoveto{\pgfqpoint{5.369911in}{2.578129in}}%
\pgfpathlineto{\pgfqpoint{5.494911in}{2.578129in}}%
\pgfpathlineto{\pgfqpoint{5.619911in}{2.578129in}}%
\pgfusepath{stroke}%
\end{pgfscope}%
\begin{pgfscope}%
\definecolor{textcolor}{rgb}{0.000000,0.000000,0.000000}%
\pgfsetstrokecolor{textcolor}%
\pgfsetfillcolor{textcolor}%
\pgftext[x=5.719911in,y=2.534379in,left,base]{\color{textcolor}{\rmfamily\fontsize{9.000000}{10.800000}\selectfont\catcode`\^=\active\def^{\ifmmode\sp\else\^{}\fi}\catcode`\%=\active\def%{\%}\NaiveCycles{}}}%
\end{pgfscope}%
\begin{pgfscope}%
\pgfsetrectcap%
\pgfsetroundjoin%
\pgfsetlinewidth{1.505625pt}%
\pgfsetstrokecolor{currentstroke1}%
\pgfsetdash{}{0pt}%
\pgfpathmoveto{\pgfqpoint{5.369911in}{2.394657in}}%
\pgfpathlineto{\pgfqpoint{5.494911in}{2.394657in}}%
\pgfpathlineto{\pgfqpoint{5.619911in}{2.394657in}}%
\pgfusepath{stroke}%
\end{pgfscope}%
\begin{pgfscope}%
\definecolor{textcolor}{rgb}{0.000000,0.000000,0.000000}%
\pgfsetstrokecolor{textcolor}%
\pgfsetfillcolor{textcolor}%
\pgftext[x=5.719911in,y=2.350907in,left,base]{\color{textcolor}{\rmfamily\fontsize{9.000000}{10.800000}\selectfont\catcode`\^=\active\def^{\ifmmode\sp\else\^{}\fi}\catcode`\%=\active\def%{\%}\CyclesMatchChunks{} \& \MergeLinear{}}}%
\end{pgfscope}%
\begin{pgfscope}%
\pgfsetrectcap%
\pgfsetroundjoin%
\pgfsetlinewidth{1.505625pt}%
\pgfsetstrokecolor{currentstroke2}%
\pgfsetdash{}{0pt}%
\pgfpathmoveto{\pgfqpoint{5.369911in}{2.207707in}}%
\pgfpathlineto{\pgfqpoint{5.494911in}{2.207707in}}%
\pgfpathlineto{\pgfqpoint{5.619911in}{2.207707in}}%
\pgfusepath{stroke}%
\end{pgfscope}%
\begin{pgfscope}%
\definecolor{textcolor}{rgb}{0.000000,0.000000,0.000000}%
\pgfsetstrokecolor{textcolor}%
\pgfsetfillcolor{textcolor}%
\pgftext[x=5.719911in,y=2.163957in,left,base]{\color{textcolor}{\rmfamily\fontsize{9.000000}{10.800000}\selectfont\catcode`\^=\active\def^{\ifmmode\sp\else\^{}\fi}\catcode`\%=\active\def%{\%}\CyclesMatchChunks{} \& \SharedVertices{}}}%
\end{pgfscope}%
\begin{pgfscope}%
\pgfsetrectcap%
\pgfsetroundjoin%
\pgfsetlinewidth{1.505625pt}%
\pgfsetstrokecolor{currentstroke4}%
\pgfsetdash{}{0pt}%
\pgfpathmoveto{\pgfqpoint{5.369911in}{2.020756in}}%
\pgfpathlineto{\pgfqpoint{5.494911in}{2.020756in}}%
\pgfpathlineto{\pgfqpoint{5.619911in}{2.020756in}}%
\pgfusepath{stroke}%
\end{pgfscope}%
\begin{pgfscope}%
\definecolor{textcolor}{rgb}{0.000000,0.000000,0.000000}%
\pgfsetstrokecolor{textcolor}%
\pgfsetfillcolor{textcolor}%
\pgftext[x=5.719911in,y=1.977006in,left,base]{\color{textcolor}{\rmfamily\fontsize{9.000000}{10.800000}\selectfont\catcode`\^=\active\def^{\ifmmode\sp\else\^{}\fi}\catcode`\%=\active\def%{\%}\Neighbors{} \& \MergeLinear{}}}%
\end{pgfscope}%
\begin{pgfscope}%
\pgfsetrectcap%
\pgfsetroundjoin%
\pgfsetlinewidth{1.505625pt}%
\pgfsetstrokecolor{currentstroke5}%
\pgfsetdash{}{0pt}%
\pgfpathmoveto{\pgfqpoint{5.369911in}{1.837285in}}%
\pgfpathlineto{\pgfqpoint{5.494911in}{1.837285in}}%
\pgfpathlineto{\pgfqpoint{5.619911in}{1.837285in}}%
\pgfusepath{stroke}%
\end{pgfscope}%
\begin{pgfscope}%
\definecolor{textcolor}{rgb}{0.000000,0.000000,0.000000}%
\pgfsetstrokecolor{textcolor}%
\pgfsetfillcolor{textcolor}%
\pgftext[x=5.719911in,y=1.793535in,left,base]{\color{textcolor}{\rmfamily\fontsize{9.000000}{10.800000}\selectfont\catcode`\^=\active\def^{\ifmmode\sp\else\^{}\fi}\catcode`\%=\active\def%{\%}\Neighbors{} \& \SharedVertices{}}}%
\end{pgfscope}%
\begin{pgfscope}%
\pgfsetrectcap%
\pgfsetroundjoin%
\pgfsetlinewidth{1.505625pt}%
\pgfsetstrokecolor{currentstroke6}%
\pgfsetdash{}{0pt}%
\pgfpathmoveto{\pgfqpoint{5.369911in}{1.650334in}}%
\pgfpathlineto{\pgfqpoint{5.494911in}{1.650334in}}%
\pgfpathlineto{\pgfqpoint{5.619911in}{1.650334in}}%
\pgfusepath{stroke}%
\end{pgfscope}%
\begin{pgfscope}%
\definecolor{textcolor}{rgb}{0.000000,0.000000,0.000000}%
\pgfsetstrokecolor{textcolor}%
\pgfsetfillcolor{textcolor}%
\pgftext[x=5.719911in,y=1.606584in,left,base]{\color{textcolor}{\rmfamily\fontsize{9.000000}{10.800000}\selectfont\catcode`\^=\active\def^{\ifmmode\sp\else\^{}\fi}\catcode`\%=\active\def%{\%}\NeighborsDegree{} \& \MergeLinear{}}}%
\end{pgfscope}%
\begin{pgfscope}%
\pgfsetrectcap%
\pgfsetroundjoin%
\pgfsetlinewidth{1.505625pt}%
\pgfsetstrokecolor{currentstroke7}%
\pgfsetdash{}{0pt}%
\pgfpathmoveto{\pgfqpoint{5.369911in}{1.463384in}}%
\pgfpathlineto{\pgfqpoint{5.494911in}{1.463384in}}%
\pgfpathlineto{\pgfqpoint{5.619911in}{1.463384in}}%
\pgfusepath{stroke}%
\end{pgfscope}%
\begin{pgfscope}%
\definecolor{textcolor}{rgb}{0.000000,0.000000,0.000000}%
\pgfsetstrokecolor{textcolor}%
\pgfsetfillcolor{textcolor}%
\pgftext[x=5.719911in,y=1.419634in,left,base]{\color{textcolor}{\rmfamily\fontsize{9.000000}{10.800000}\selectfont\catcode`\^=\active\def^{\ifmmode\sp\else\^{}\fi}\catcode`\%=\active\def%{\%}\NeighborsDegree{} \& \SharedVertices{}}}%
\end{pgfscope}%
\begin{pgfscope}%
\pgfsetrectcap%
\pgfsetroundjoin%
\pgfsetlinewidth{1.505625pt}%
\definecolor{currentstroke}{rgb}{0.498039,0.498039,0.498039}%
\pgfsetstrokecolor{currentstroke}%
\pgfsetdash{}{0pt}%
\pgfpathmoveto{\pgfqpoint{5.369911in}{1.276433in}}%
\pgfpathlineto{\pgfqpoint{5.494911in}{1.276433in}}%
\pgfpathlineto{\pgfqpoint{5.619911in}{1.276433in}}%
\pgfusepath{stroke}%
\end{pgfscope}%
\begin{pgfscope}%
\definecolor{textcolor}{rgb}{0.000000,0.000000,0.000000}%
\pgfsetstrokecolor{textcolor}%
\pgfsetfillcolor{textcolor}%
\pgftext[x=5.719911in,y=1.232683in,left,base]{\color{textcolor}{\rmfamily\fontsize{9.000000}{10.800000}\selectfont\catcode`\^=\active\def^{\ifmmode\sp\else\^{}\fi}\catcode`\%=\active\def%{\%}\None{} \& \MergeLinear{}}}%
\end{pgfscope}%
\begin{pgfscope}%
\pgfsetrectcap%
\pgfsetroundjoin%
\pgfsetlinewidth{1.505625pt}%
\definecolor{currentstroke}{rgb}{0.737255,0.741176,0.133333}%
\pgfsetstrokecolor{currentstroke}%
\pgfsetdash{}{0pt}%
\pgfpathmoveto{\pgfqpoint{5.369911in}{1.092962in}}%
\pgfpathlineto{\pgfqpoint{5.494911in}{1.092962in}}%
\pgfpathlineto{\pgfqpoint{5.619911in}{1.092962in}}%
\pgfusepath{stroke}%
\end{pgfscope}%
\begin{pgfscope}%
\definecolor{textcolor}{rgb}{0.000000,0.000000,0.000000}%
\pgfsetstrokecolor{textcolor}%
\pgfsetfillcolor{textcolor}%
\pgftext[x=5.719911in,y=1.049212in,left,base]{\color{textcolor}{\rmfamily\fontsize{9.000000}{10.800000}\selectfont\catcode`\^=\active\def^{\ifmmode\sp\else\^{}\fi}\catcode`\%=\active\def%{\%}\None{} \& \SharedVertices{}}}%
\end{pgfscope}%
\end{pgfpicture}%
\makeatother%
\endgroup%
}
	\caption[Checks performed for minimally rigid graphs (some)]{
		The number of checks performed to find all NAC-colorings for minimally rigid graphs.}%
	\label{fig:graph_minimally_rigid_first_checks}
\end{figure}%



\subsubsection*{No three nor four cycle graphs}

From~\cite{extremal_graphs} we obtained all graphs with up to 52 vertices
that have no three nor four cycles. This class of graphs is interesting for us
as there cannot be any \trcon{} components.
These graphs have also many NAC-colorings.
Because of that, as seen in \Cref{fig:graph_count_no_3_nor_4_cycles_first_runtime},
\NaiveCycles{} is again faster for finding some NAC-coloring
for the similar reasons as for minimally rigid graphs.
%
Also notice that \SharedVertices{} performs worse and non-deterministically
for \CyclesMatchChunks{} and \None{}.
For \Neighbors{}, the performance is more stable, but still worse than \MergeLinear{}.

\begin{figure}[thbp]
	\centering
	\scalebox{\BenchFigureScale}{%% Creator: Matplotlib, PGF backend
%%
%% To include the figure in your LaTeX document, write
%%   \input{<filename>.pgf}
%%
%% Make sure the required packages are loaded in your preamble
%%   \usepackage{pgf}
%%
%% Also ensure that all the required font packages are loaded; for instance,
%% the lmodern package is sometimes necessary when using math font.
%%   \usepackage{lmodern}
%%
%% Figures using additional raster images can only be included by \input if
%% they are in the same directory as the main LaTeX file. For loading figures
%% from other directories you can use the `import` package
%%   \usepackage{import}
%%
%% and then include the figures with
%%   \import{<path to file>}{<filename>.pgf}
%%
%% Matplotlib used the following preamble
%%   \def\mathdefault#1{#1}
%%   \everymath=\expandafter{\the\everymath\displaystyle}
%%   \IfFileExists{scrextend.sty}{
%%     \usepackage[fontsize=10.000000pt]{scrextend}
%%   }{
%%     \renewcommand{\normalsize}{\fontsize{10.000000}{12.000000}\selectfont}
%%     \normalsize
%%   }
%%   
%%   \ifdefined\pdftexversion\else  % non-pdftex case.
%%     \usepackage{fontspec}
%%     \setmainfont{DejaVuSans.ttf}[Path=\detokenize{/home/petr/Projects/PyRigi/.venv/lib/python3.12/site-packages/matplotlib/mpl-data/fonts/ttf/}]
%%     \setsansfont{DejaVuSans.ttf}[Path=\detokenize{/home/petr/Projects/PyRigi/.venv/lib/python3.12/site-packages/matplotlib/mpl-data/fonts/ttf/}]
%%     \setmonofont{DejaVuSansMono.ttf}[Path=\detokenize{/home/petr/Projects/PyRigi/.venv/lib/python3.12/site-packages/matplotlib/mpl-data/fonts/ttf/}]
%%   \fi
%%   \makeatletter\@ifpackageloaded{underscore}{}{\usepackage[strings]{underscore}}\makeatother
%%
\begingroup%
\makeatletter%
\begin{pgfpicture}%
\pgfpathrectangle{\pgfpointorigin}{\pgfqpoint{8.384376in}{2.841849in}}%
\pgfusepath{use as bounding box, clip}%
\begin{pgfscope}%
\pgfsetbuttcap%
\pgfsetmiterjoin%
\definecolor{currentfill}{rgb}{1.000000,1.000000,1.000000}%
\pgfsetfillcolor{currentfill}%
\pgfsetlinewidth{0.000000pt}%
\definecolor{currentstroke}{rgb}{1.000000,1.000000,1.000000}%
\pgfsetstrokecolor{currentstroke}%
\pgfsetdash{}{0pt}%
\pgfpathmoveto{\pgfqpoint{0.000000in}{0.000000in}}%
\pgfpathlineto{\pgfqpoint{8.384376in}{0.000000in}}%
\pgfpathlineto{\pgfqpoint{8.384376in}{2.841849in}}%
\pgfpathlineto{\pgfqpoint{0.000000in}{2.841849in}}%
\pgfpathlineto{\pgfqpoint{0.000000in}{0.000000in}}%
\pgfpathclose%
\pgfusepath{fill}%
\end{pgfscope}%
\begin{pgfscope}%
\pgfsetbuttcap%
\pgfsetmiterjoin%
\definecolor{currentfill}{rgb}{1.000000,1.000000,1.000000}%
\pgfsetfillcolor{currentfill}%
\pgfsetlinewidth{0.000000pt}%
\definecolor{currentstroke}{rgb}{0.000000,0.000000,0.000000}%
\pgfsetstrokecolor{currentstroke}%
\pgfsetstrokeopacity{0.000000}%
\pgfsetdash{}{0pt}%
\pgfpathmoveto{\pgfqpoint{0.588387in}{0.521603in}}%
\pgfpathlineto{\pgfqpoint{5.257411in}{0.521603in}}%
\pgfpathlineto{\pgfqpoint{5.257411in}{2.531888in}}%
\pgfpathlineto{\pgfqpoint{0.588387in}{2.531888in}}%
\pgfpathlineto{\pgfqpoint{0.588387in}{0.521603in}}%
\pgfpathclose%
\pgfusepath{fill}%
\end{pgfscope}%
\begin{pgfscope}%
\pgfsetbuttcap%
\pgfsetroundjoin%
\definecolor{currentfill}{rgb}{0.000000,0.000000,0.000000}%
\pgfsetfillcolor{currentfill}%
\pgfsetlinewidth{0.803000pt}%
\definecolor{currentstroke}{rgb}{0.000000,0.000000,0.000000}%
\pgfsetstrokecolor{currentstroke}%
\pgfsetdash{}{0pt}%
\pgfsys@defobject{currentmarker}{\pgfqpoint{0.000000in}{-0.048611in}}{\pgfqpoint{0.000000in}{0.000000in}}{%
\pgfpathmoveto{\pgfqpoint{0.000000in}{0.000000in}}%
\pgfpathlineto{\pgfqpoint{0.000000in}{-0.048611in}}%
\pgfusepath{stroke,fill}%
}%
\begin{pgfscope}%
\pgfsys@transformshift{0.677940in}{0.521603in}%
\pgfsys@useobject{currentmarker}{}%
\end{pgfscope}%
\end{pgfscope}%
\begin{pgfscope}%
\definecolor{textcolor}{rgb}{0.000000,0.000000,0.000000}%
\pgfsetstrokecolor{textcolor}%
\pgfsetfillcolor{textcolor}%
\pgftext[x=0.677940in,y=0.424381in,,top]{\color{textcolor}{\rmfamily\fontsize{10.000000}{12.000000}\selectfont\catcode`\^=\active\def^{\ifmmode\sp\else\^{}\fi}\catcode`\%=\active\def%{\%}$\mathdefault{0}$}}%
\end{pgfscope}%
\begin{pgfscope}%
\pgfsetbuttcap%
\pgfsetroundjoin%
\definecolor{currentfill}{rgb}{0.000000,0.000000,0.000000}%
\pgfsetfillcolor{currentfill}%
\pgfsetlinewidth{0.803000pt}%
\definecolor{currentstroke}{rgb}{0.000000,0.000000,0.000000}%
\pgfsetstrokecolor{currentstroke}%
\pgfsetdash{}{0pt}%
\pgfsys@defobject{currentmarker}{\pgfqpoint{0.000000in}{-0.048611in}}{\pgfqpoint{0.000000in}{0.000000in}}{%
\pgfpathmoveto{\pgfqpoint{0.000000in}{0.000000in}}%
\pgfpathlineto{\pgfqpoint{0.000000in}{-0.048611in}}%
\pgfusepath{stroke,fill}%
}%
\begin{pgfscope}%
\pgfsys@transformshift{1.168642in}{0.521603in}%
\pgfsys@useobject{currentmarker}{}%
\end{pgfscope}%
\end{pgfscope}%
\begin{pgfscope}%
\definecolor{textcolor}{rgb}{0.000000,0.000000,0.000000}%
\pgfsetstrokecolor{textcolor}%
\pgfsetfillcolor{textcolor}%
\pgftext[x=1.168642in,y=0.424381in,,top]{\color{textcolor}{\rmfamily\fontsize{10.000000}{12.000000}\selectfont\catcode`\^=\active\def^{\ifmmode\sp\else\^{}\fi}\catcode`\%=\active\def%{\%}$\mathdefault{20}$}}%
\end{pgfscope}%
\begin{pgfscope}%
\pgfsetbuttcap%
\pgfsetroundjoin%
\definecolor{currentfill}{rgb}{0.000000,0.000000,0.000000}%
\pgfsetfillcolor{currentfill}%
\pgfsetlinewidth{0.803000pt}%
\definecolor{currentstroke}{rgb}{0.000000,0.000000,0.000000}%
\pgfsetstrokecolor{currentstroke}%
\pgfsetdash{}{0pt}%
\pgfsys@defobject{currentmarker}{\pgfqpoint{0.000000in}{-0.048611in}}{\pgfqpoint{0.000000in}{0.000000in}}{%
\pgfpathmoveto{\pgfqpoint{0.000000in}{0.000000in}}%
\pgfpathlineto{\pgfqpoint{0.000000in}{-0.048611in}}%
\pgfusepath{stroke,fill}%
}%
\begin{pgfscope}%
\pgfsys@transformshift{1.659343in}{0.521603in}%
\pgfsys@useobject{currentmarker}{}%
\end{pgfscope}%
\end{pgfscope}%
\begin{pgfscope}%
\definecolor{textcolor}{rgb}{0.000000,0.000000,0.000000}%
\pgfsetstrokecolor{textcolor}%
\pgfsetfillcolor{textcolor}%
\pgftext[x=1.659343in,y=0.424381in,,top]{\color{textcolor}{\rmfamily\fontsize{10.000000}{12.000000}\selectfont\catcode`\^=\active\def^{\ifmmode\sp\else\^{}\fi}\catcode`\%=\active\def%{\%}$\mathdefault{40}$}}%
\end{pgfscope}%
\begin{pgfscope}%
\pgfsetbuttcap%
\pgfsetroundjoin%
\definecolor{currentfill}{rgb}{0.000000,0.000000,0.000000}%
\pgfsetfillcolor{currentfill}%
\pgfsetlinewidth{0.803000pt}%
\definecolor{currentstroke}{rgb}{0.000000,0.000000,0.000000}%
\pgfsetstrokecolor{currentstroke}%
\pgfsetdash{}{0pt}%
\pgfsys@defobject{currentmarker}{\pgfqpoint{0.000000in}{-0.048611in}}{\pgfqpoint{0.000000in}{0.000000in}}{%
\pgfpathmoveto{\pgfqpoint{0.000000in}{0.000000in}}%
\pgfpathlineto{\pgfqpoint{0.000000in}{-0.048611in}}%
\pgfusepath{stroke,fill}%
}%
\begin{pgfscope}%
\pgfsys@transformshift{2.150044in}{0.521603in}%
\pgfsys@useobject{currentmarker}{}%
\end{pgfscope}%
\end{pgfscope}%
\begin{pgfscope}%
\definecolor{textcolor}{rgb}{0.000000,0.000000,0.000000}%
\pgfsetstrokecolor{textcolor}%
\pgfsetfillcolor{textcolor}%
\pgftext[x=2.150044in,y=0.424381in,,top]{\color{textcolor}{\rmfamily\fontsize{10.000000}{12.000000}\selectfont\catcode`\^=\active\def^{\ifmmode\sp\else\^{}\fi}\catcode`\%=\active\def%{\%}$\mathdefault{60}$}}%
\end{pgfscope}%
\begin{pgfscope}%
\pgfsetbuttcap%
\pgfsetroundjoin%
\definecolor{currentfill}{rgb}{0.000000,0.000000,0.000000}%
\pgfsetfillcolor{currentfill}%
\pgfsetlinewidth{0.803000pt}%
\definecolor{currentstroke}{rgb}{0.000000,0.000000,0.000000}%
\pgfsetstrokecolor{currentstroke}%
\pgfsetdash{}{0pt}%
\pgfsys@defobject{currentmarker}{\pgfqpoint{0.000000in}{-0.048611in}}{\pgfqpoint{0.000000in}{0.000000in}}{%
\pgfpathmoveto{\pgfqpoint{0.000000in}{0.000000in}}%
\pgfpathlineto{\pgfqpoint{0.000000in}{-0.048611in}}%
\pgfusepath{stroke,fill}%
}%
\begin{pgfscope}%
\pgfsys@transformshift{2.640746in}{0.521603in}%
\pgfsys@useobject{currentmarker}{}%
\end{pgfscope}%
\end{pgfscope}%
\begin{pgfscope}%
\definecolor{textcolor}{rgb}{0.000000,0.000000,0.000000}%
\pgfsetstrokecolor{textcolor}%
\pgfsetfillcolor{textcolor}%
\pgftext[x=2.640746in,y=0.424381in,,top]{\color{textcolor}{\rmfamily\fontsize{10.000000}{12.000000}\selectfont\catcode`\^=\active\def^{\ifmmode\sp\else\^{}\fi}\catcode`\%=\active\def%{\%}$\mathdefault{80}$}}%
\end{pgfscope}%
\begin{pgfscope}%
\pgfsetbuttcap%
\pgfsetroundjoin%
\definecolor{currentfill}{rgb}{0.000000,0.000000,0.000000}%
\pgfsetfillcolor{currentfill}%
\pgfsetlinewidth{0.803000pt}%
\definecolor{currentstroke}{rgb}{0.000000,0.000000,0.000000}%
\pgfsetstrokecolor{currentstroke}%
\pgfsetdash{}{0pt}%
\pgfsys@defobject{currentmarker}{\pgfqpoint{0.000000in}{-0.048611in}}{\pgfqpoint{0.000000in}{0.000000in}}{%
\pgfpathmoveto{\pgfqpoint{0.000000in}{0.000000in}}%
\pgfpathlineto{\pgfqpoint{0.000000in}{-0.048611in}}%
\pgfusepath{stroke,fill}%
}%
\begin{pgfscope}%
\pgfsys@transformshift{3.131447in}{0.521603in}%
\pgfsys@useobject{currentmarker}{}%
\end{pgfscope}%
\end{pgfscope}%
\begin{pgfscope}%
\definecolor{textcolor}{rgb}{0.000000,0.000000,0.000000}%
\pgfsetstrokecolor{textcolor}%
\pgfsetfillcolor{textcolor}%
\pgftext[x=3.131447in,y=0.424381in,,top]{\color{textcolor}{\rmfamily\fontsize{10.000000}{12.000000}\selectfont\catcode`\^=\active\def^{\ifmmode\sp\else\^{}\fi}\catcode`\%=\active\def%{\%}$\mathdefault{100}$}}%
\end{pgfscope}%
\begin{pgfscope}%
\pgfsetbuttcap%
\pgfsetroundjoin%
\definecolor{currentfill}{rgb}{0.000000,0.000000,0.000000}%
\pgfsetfillcolor{currentfill}%
\pgfsetlinewidth{0.803000pt}%
\definecolor{currentstroke}{rgb}{0.000000,0.000000,0.000000}%
\pgfsetstrokecolor{currentstroke}%
\pgfsetdash{}{0pt}%
\pgfsys@defobject{currentmarker}{\pgfqpoint{0.000000in}{-0.048611in}}{\pgfqpoint{0.000000in}{0.000000in}}{%
\pgfpathmoveto{\pgfqpoint{0.000000in}{0.000000in}}%
\pgfpathlineto{\pgfqpoint{0.000000in}{-0.048611in}}%
\pgfusepath{stroke,fill}%
}%
\begin{pgfscope}%
\pgfsys@transformshift{3.622149in}{0.521603in}%
\pgfsys@useobject{currentmarker}{}%
\end{pgfscope}%
\end{pgfscope}%
\begin{pgfscope}%
\definecolor{textcolor}{rgb}{0.000000,0.000000,0.000000}%
\pgfsetstrokecolor{textcolor}%
\pgfsetfillcolor{textcolor}%
\pgftext[x=3.622149in,y=0.424381in,,top]{\color{textcolor}{\rmfamily\fontsize{10.000000}{12.000000}\selectfont\catcode`\^=\active\def^{\ifmmode\sp\else\^{}\fi}\catcode`\%=\active\def%{\%}$\mathdefault{120}$}}%
\end{pgfscope}%
\begin{pgfscope}%
\pgfsetbuttcap%
\pgfsetroundjoin%
\definecolor{currentfill}{rgb}{0.000000,0.000000,0.000000}%
\pgfsetfillcolor{currentfill}%
\pgfsetlinewidth{0.803000pt}%
\definecolor{currentstroke}{rgb}{0.000000,0.000000,0.000000}%
\pgfsetstrokecolor{currentstroke}%
\pgfsetdash{}{0pt}%
\pgfsys@defobject{currentmarker}{\pgfqpoint{0.000000in}{-0.048611in}}{\pgfqpoint{0.000000in}{0.000000in}}{%
\pgfpathmoveto{\pgfqpoint{0.000000in}{0.000000in}}%
\pgfpathlineto{\pgfqpoint{0.000000in}{-0.048611in}}%
\pgfusepath{stroke,fill}%
}%
\begin{pgfscope}%
\pgfsys@transformshift{4.112850in}{0.521603in}%
\pgfsys@useobject{currentmarker}{}%
\end{pgfscope}%
\end{pgfscope}%
\begin{pgfscope}%
\definecolor{textcolor}{rgb}{0.000000,0.000000,0.000000}%
\pgfsetstrokecolor{textcolor}%
\pgfsetfillcolor{textcolor}%
\pgftext[x=4.112850in,y=0.424381in,,top]{\color{textcolor}{\rmfamily\fontsize{10.000000}{12.000000}\selectfont\catcode`\^=\active\def^{\ifmmode\sp\else\^{}\fi}\catcode`\%=\active\def%{\%}$\mathdefault{140}$}}%
\end{pgfscope}%
\begin{pgfscope}%
\pgfsetbuttcap%
\pgfsetroundjoin%
\definecolor{currentfill}{rgb}{0.000000,0.000000,0.000000}%
\pgfsetfillcolor{currentfill}%
\pgfsetlinewidth{0.803000pt}%
\definecolor{currentstroke}{rgb}{0.000000,0.000000,0.000000}%
\pgfsetstrokecolor{currentstroke}%
\pgfsetdash{}{0pt}%
\pgfsys@defobject{currentmarker}{\pgfqpoint{0.000000in}{-0.048611in}}{\pgfqpoint{0.000000in}{0.000000in}}{%
\pgfpathmoveto{\pgfqpoint{0.000000in}{0.000000in}}%
\pgfpathlineto{\pgfqpoint{0.000000in}{-0.048611in}}%
\pgfusepath{stroke,fill}%
}%
\begin{pgfscope}%
\pgfsys@transformshift{4.603552in}{0.521603in}%
\pgfsys@useobject{currentmarker}{}%
\end{pgfscope}%
\end{pgfscope}%
\begin{pgfscope}%
\definecolor{textcolor}{rgb}{0.000000,0.000000,0.000000}%
\pgfsetstrokecolor{textcolor}%
\pgfsetfillcolor{textcolor}%
\pgftext[x=4.603552in,y=0.424381in,,top]{\color{textcolor}{\rmfamily\fontsize{10.000000}{12.000000}\selectfont\catcode`\^=\active\def^{\ifmmode\sp\else\^{}\fi}\catcode`\%=\active\def%{\%}$\mathdefault{160}$}}%
\end{pgfscope}%
\begin{pgfscope}%
\pgfsetbuttcap%
\pgfsetroundjoin%
\definecolor{currentfill}{rgb}{0.000000,0.000000,0.000000}%
\pgfsetfillcolor{currentfill}%
\pgfsetlinewidth{0.803000pt}%
\definecolor{currentstroke}{rgb}{0.000000,0.000000,0.000000}%
\pgfsetstrokecolor{currentstroke}%
\pgfsetdash{}{0pt}%
\pgfsys@defobject{currentmarker}{\pgfqpoint{0.000000in}{-0.048611in}}{\pgfqpoint{0.000000in}{0.000000in}}{%
\pgfpathmoveto{\pgfqpoint{0.000000in}{0.000000in}}%
\pgfpathlineto{\pgfqpoint{0.000000in}{-0.048611in}}%
\pgfusepath{stroke,fill}%
}%
\begin{pgfscope}%
\pgfsys@transformshift{5.094253in}{0.521603in}%
\pgfsys@useobject{currentmarker}{}%
\end{pgfscope}%
\end{pgfscope}%
\begin{pgfscope}%
\definecolor{textcolor}{rgb}{0.000000,0.000000,0.000000}%
\pgfsetstrokecolor{textcolor}%
\pgfsetfillcolor{textcolor}%
\pgftext[x=5.094253in,y=0.424381in,,top]{\color{textcolor}{\rmfamily\fontsize{10.000000}{12.000000}\selectfont\catcode`\^=\active\def^{\ifmmode\sp\else\^{}\fi}\catcode`\%=\active\def%{\%}$\mathdefault{180}$}}%
\end{pgfscope}%
\begin{pgfscope}%
\definecolor{textcolor}{rgb}{0.000000,0.000000,0.000000}%
\pgfsetstrokecolor{textcolor}%
\pgfsetfillcolor{textcolor}%
\pgftext[x=2.922899in,y=0.234413in,,top]{\color{textcolor}{\rmfamily\fontsize{10.000000}{12.000000}\selectfont\catcode`\^=\active\def^{\ifmmode\sp\else\^{}\fi}\catcode`\%=\active\def%{\%}Monochromatic classes}}%
\end{pgfscope}%
\begin{pgfscope}%
\pgfsetbuttcap%
\pgfsetroundjoin%
\definecolor{currentfill}{rgb}{0.000000,0.000000,0.000000}%
\pgfsetfillcolor{currentfill}%
\pgfsetlinewidth{0.803000pt}%
\definecolor{currentstroke}{rgb}{0.000000,0.000000,0.000000}%
\pgfsetstrokecolor{currentstroke}%
\pgfsetdash{}{0pt}%
\pgfsys@defobject{currentmarker}{\pgfqpoint{-0.048611in}{0.000000in}}{\pgfqpoint{-0.000000in}{0.000000in}}{%
\pgfpathmoveto{\pgfqpoint{-0.000000in}{0.000000in}}%
\pgfpathlineto{\pgfqpoint{-0.048611in}{0.000000in}}%
\pgfusepath{stroke,fill}%
}%
\begin{pgfscope}%
\pgfsys@transformshift{0.588387in}{0.612980in}%
\pgfsys@useobject{currentmarker}{}%
\end{pgfscope}%
\end{pgfscope}%
\begin{pgfscope}%
\definecolor{textcolor}{rgb}{0.000000,0.000000,0.000000}%
\pgfsetstrokecolor{textcolor}%
\pgfsetfillcolor{textcolor}%
\pgftext[x=0.289968in, y=0.560218in, left, base]{\color{textcolor}{\rmfamily\fontsize{10.000000}{12.000000}\selectfont\catcode`\^=\active\def^{\ifmmode\sp\else\^{}\fi}\catcode`\%=\active\def%{\%}$\mathdefault{10^{0}}$}}%
\end{pgfscope}%
\begin{pgfscope}%
\pgfsetbuttcap%
\pgfsetroundjoin%
\definecolor{currentfill}{rgb}{0.000000,0.000000,0.000000}%
\pgfsetfillcolor{currentfill}%
\pgfsetlinewidth{0.803000pt}%
\definecolor{currentstroke}{rgb}{0.000000,0.000000,0.000000}%
\pgfsetstrokecolor{currentstroke}%
\pgfsetdash{}{0pt}%
\pgfsys@defobject{currentmarker}{\pgfqpoint{-0.048611in}{0.000000in}}{\pgfqpoint{-0.000000in}{0.000000in}}{%
\pgfpathmoveto{\pgfqpoint{-0.000000in}{0.000000in}}%
\pgfpathlineto{\pgfqpoint{-0.048611in}{0.000000in}}%
\pgfusepath{stroke,fill}%
}%
\begin{pgfscope}%
\pgfsys@transformshift{0.588387in}{1.107607in}%
\pgfsys@useobject{currentmarker}{}%
\end{pgfscope}%
\end{pgfscope}%
\begin{pgfscope}%
\definecolor{textcolor}{rgb}{0.000000,0.000000,0.000000}%
\pgfsetstrokecolor{textcolor}%
\pgfsetfillcolor{textcolor}%
\pgftext[x=0.289968in, y=1.054846in, left, base]{\color{textcolor}{\rmfamily\fontsize{10.000000}{12.000000}\selectfont\catcode`\^=\active\def^{\ifmmode\sp\else\^{}\fi}\catcode`\%=\active\def%{\%}$\mathdefault{10^{1}}$}}%
\end{pgfscope}%
\begin{pgfscope}%
\pgfsetbuttcap%
\pgfsetroundjoin%
\definecolor{currentfill}{rgb}{0.000000,0.000000,0.000000}%
\pgfsetfillcolor{currentfill}%
\pgfsetlinewidth{0.803000pt}%
\definecolor{currentstroke}{rgb}{0.000000,0.000000,0.000000}%
\pgfsetstrokecolor{currentstroke}%
\pgfsetdash{}{0pt}%
\pgfsys@defobject{currentmarker}{\pgfqpoint{-0.048611in}{0.000000in}}{\pgfqpoint{-0.000000in}{0.000000in}}{%
\pgfpathmoveto{\pgfqpoint{-0.000000in}{0.000000in}}%
\pgfpathlineto{\pgfqpoint{-0.048611in}{0.000000in}}%
\pgfusepath{stroke,fill}%
}%
\begin{pgfscope}%
\pgfsys@transformshift{0.588387in}{1.602235in}%
\pgfsys@useobject{currentmarker}{}%
\end{pgfscope}%
\end{pgfscope}%
\begin{pgfscope}%
\definecolor{textcolor}{rgb}{0.000000,0.000000,0.000000}%
\pgfsetstrokecolor{textcolor}%
\pgfsetfillcolor{textcolor}%
\pgftext[x=0.289968in, y=1.549473in, left, base]{\color{textcolor}{\rmfamily\fontsize{10.000000}{12.000000}\selectfont\catcode`\^=\active\def^{\ifmmode\sp\else\^{}\fi}\catcode`\%=\active\def%{\%}$\mathdefault{10^{2}}$}}%
\end{pgfscope}%
\begin{pgfscope}%
\pgfsetbuttcap%
\pgfsetroundjoin%
\definecolor{currentfill}{rgb}{0.000000,0.000000,0.000000}%
\pgfsetfillcolor{currentfill}%
\pgfsetlinewidth{0.803000pt}%
\definecolor{currentstroke}{rgb}{0.000000,0.000000,0.000000}%
\pgfsetstrokecolor{currentstroke}%
\pgfsetdash{}{0pt}%
\pgfsys@defobject{currentmarker}{\pgfqpoint{-0.048611in}{0.000000in}}{\pgfqpoint{-0.000000in}{0.000000in}}{%
\pgfpathmoveto{\pgfqpoint{-0.000000in}{0.000000in}}%
\pgfpathlineto{\pgfqpoint{-0.048611in}{0.000000in}}%
\pgfusepath{stroke,fill}%
}%
\begin{pgfscope}%
\pgfsys@transformshift{0.588387in}{2.096862in}%
\pgfsys@useobject{currentmarker}{}%
\end{pgfscope}%
\end{pgfscope}%
\begin{pgfscope}%
\definecolor{textcolor}{rgb}{0.000000,0.000000,0.000000}%
\pgfsetstrokecolor{textcolor}%
\pgfsetfillcolor{textcolor}%
\pgftext[x=0.289968in, y=2.044100in, left, base]{\color{textcolor}{\rmfamily\fontsize{10.000000}{12.000000}\selectfont\catcode`\^=\active\def^{\ifmmode\sp\else\^{}\fi}\catcode`\%=\active\def%{\%}$\mathdefault{10^{3}}$}}%
\end{pgfscope}%
\begin{pgfscope}%
\pgfsetbuttcap%
\pgfsetroundjoin%
\definecolor{currentfill}{rgb}{0.000000,0.000000,0.000000}%
\pgfsetfillcolor{currentfill}%
\pgfsetlinewidth{0.602250pt}%
\definecolor{currentstroke}{rgb}{0.000000,0.000000,0.000000}%
\pgfsetstrokecolor{currentstroke}%
\pgfsetdash{}{0pt}%
\pgfsys@defobject{currentmarker}{\pgfqpoint{-0.027778in}{0.000000in}}{\pgfqpoint{-0.000000in}{0.000000in}}{%
\pgfpathmoveto{\pgfqpoint{-0.000000in}{0.000000in}}%
\pgfpathlineto{\pgfqpoint{-0.027778in}{0.000000in}}%
\pgfusepath{stroke,fill}%
}%
\begin{pgfscope}%
\pgfsys@transformshift{0.588387in}{0.536361in}%
\pgfsys@useobject{currentmarker}{}%
\end{pgfscope}%
\end{pgfscope}%
\begin{pgfscope}%
\pgfsetbuttcap%
\pgfsetroundjoin%
\definecolor{currentfill}{rgb}{0.000000,0.000000,0.000000}%
\pgfsetfillcolor{currentfill}%
\pgfsetlinewidth{0.602250pt}%
\definecolor{currentstroke}{rgb}{0.000000,0.000000,0.000000}%
\pgfsetstrokecolor{currentstroke}%
\pgfsetdash{}{0pt}%
\pgfsys@defobject{currentmarker}{\pgfqpoint{-0.027778in}{0.000000in}}{\pgfqpoint{-0.000000in}{0.000000in}}{%
\pgfpathmoveto{\pgfqpoint{-0.000000in}{0.000000in}}%
\pgfpathlineto{\pgfqpoint{-0.027778in}{0.000000in}}%
\pgfusepath{stroke,fill}%
}%
\begin{pgfscope}%
\pgfsys@transformshift{0.588387in}{0.565046in}%
\pgfsys@useobject{currentmarker}{}%
\end{pgfscope}%
\end{pgfscope}%
\begin{pgfscope}%
\pgfsetbuttcap%
\pgfsetroundjoin%
\definecolor{currentfill}{rgb}{0.000000,0.000000,0.000000}%
\pgfsetfillcolor{currentfill}%
\pgfsetlinewidth{0.602250pt}%
\definecolor{currentstroke}{rgb}{0.000000,0.000000,0.000000}%
\pgfsetstrokecolor{currentstroke}%
\pgfsetdash{}{0pt}%
\pgfsys@defobject{currentmarker}{\pgfqpoint{-0.027778in}{0.000000in}}{\pgfqpoint{-0.000000in}{0.000000in}}{%
\pgfpathmoveto{\pgfqpoint{-0.000000in}{0.000000in}}%
\pgfpathlineto{\pgfqpoint{-0.027778in}{0.000000in}}%
\pgfusepath{stroke,fill}%
}%
\begin{pgfscope}%
\pgfsys@transformshift{0.588387in}{0.590347in}%
\pgfsys@useobject{currentmarker}{}%
\end{pgfscope}%
\end{pgfscope}%
\begin{pgfscope}%
\pgfsetbuttcap%
\pgfsetroundjoin%
\definecolor{currentfill}{rgb}{0.000000,0.000000,0.000000}%
\pgfsetfillcolor{currentfill}%
\pgfsetlinewidth{0.602250pt}%
\definecolor{currentstroke}{rgb}{0.000000,0.000000,0.000000}%
\pgfsetstrokecolor{currentstroke}%
\pgfsetdash{}{0pt}%
\pgfsys@defobject{currentmarker}{\pgfqpoint{-0.027778in}{0.000000in}}{\pgfqpoint{-0.000000in}{0.000000in}}{%
\pgfpathmoveto{\pgfqpoint{-0.000000in}{0.000000in}}%
\pgfpathlineto{\pgfqpoint{-0.027778in}{0.000000in}}%
\pgfusepath{stroke,fill}%
}%
\begin{pgfscope}%
\pgfsys@transformshift{0.588387in}{0.761878in}%
\pgfsys@useobject{currentmarker}{}%
\end{pgfscope}%
\end{pgfscope}%
\begin{pgfscope}%
\pgfsetbuttcap%
\pgfsetroundjoin%
\definecolor{currentfill}{rgb}{0.000000,0.000000,0.000000}%
\pgfsetfillcolor{currentfill}%
\pgfsetlinewidth{0.602250pt}%
\definecolor{currentstroke}{rgb}{0.000000,0.000000,0.000000}%
\pgfsetstrokecolor{currentstroke}%
\pgfsetdash{}{0pt}%
\pgfsys@defobject{currentmarker}{\pgfqpoint{-0.027778in}{0.000000in}}{\pgfqpoint{-0.000000in}{0.000000in}}{%
\pgfpathmoveto{\pgfqpoint{-0.000000in}{0.000000in}}%
\pgfpathlineto{\pgfqpoint{-0.027778in}{0.000000in}}%
\pgfusepath{stroke,fill}%
}%
\begin{pgfscope}%
\pgfsys@transformshift{0.588387in}{0.848977in}%
\pgfsys@useobject{currentmarker}{}%
\end{pgfscope}%
\end{pgfscope}%
\begin{pgfscope}%
\pgfsetbuttcap%
\pgfsetroundjoin%
\definecolor{currentfill}{rgb}{0.000000,0.000000,0.000000}%
\pgfsetfillcolor{currentfill}%
\pgfsetlinewidth{0.602250pt}%
\definecolor{currentstroke}{rgb}{0.000000,0.000000,0.000000}%
\pgfsetstrokecolor{currentstroke}%
\pgfsetdash{}{0pt}%
\pgfsys@defobject{currentmarker}{\pgfqpoint{-0.027778in}{0.000000in}}{\pgfqpoint{-0.000000in}{0.000000in}}{%
\pgfpathmoveto{\pgfqpoint{-0.000000in}{0.000000in}}%
\pgfpathlineto{\pgfqpoint{-0.027778in}{0.000000in}}%
\pgfusepath{stroke,fill}%
}%
\begin{pgfscope}%
\pgfsys@transformshift{0.588387in}{0.910775in}%
\pgfsys@useobject{currentmarker}{}%
\end{pgfscope}%
\end{pgfscope}%
\begin{pgfscope}%
\pgfsetbuttcap%
\pgfsetroundjoin%
\definecolor{currentfill}{rgb}{0.000000,0.000000,0.000000}%
\pgfsetfillcolor{currentfill}%
\pgfsetlinewidth{0.602250pt}%
\definecolor{currentstroke}{rgb}{0.000000,0.000000,0.000000}%
\pgfsetstrokecolor{currentstroke}%
\pgfsetdash{}{0pt}%
\pgfsys@defobject{currentmarker}{\pgfqpoint{-0.027778in}{0.000000in}}{\pgfqpoint{-0.000000in}{0.000000in}}{%
\pgfpathmoveto{\pgfqpoint{-0.000000in}{0.000000in}}%
\pgfpathlineto{\pgfqpoint{-0.027778in}{0.000000in}}%
\pgfusepath{stroke,fill}%
}%
\begin{pgfscope}%
\pgfsys@transformshift{0.588387in}{0.958710in}%
\pgfsys@useobject{currentmarker}{}%
\end{pgfscope}%
\end{pgfscope}%
\begin{pgfscope}%
\pgfsetbuttcap%
\pgfsetroundjoin%
\definecolor{currentfill}{rgb}{0.000000,0.000000,0.000000}%
\pgfsetfillcolor{currentfill}%
\pgfsetlinewidth{0.602250pt}%
\definecolor{currentstroke}{rgb}{0.000000,0.000000,0.000000}%
\pgfsetstrokecolor{currentstroke}%
\pgfsetdash{}{0pt}%
\pgfsys@defobject{currentmarker}{\pgfqpoint{-0.027778in}{0.000000in}}{\pgfqpoint{-0.000000in}{0.000000in}}{%
\pgfpathmoveto{\pgfqpoint{-0.000000in}{0.000000in}}%
\pgfpathlineto{\pgfqpoint{-0.027778in}{0.000000in}}%
\pgfusepath{stroke,fill}%
}%
\begin{pgfscope}%
\pgfsys@transformshift{0.588387in}{0.997875in}%
\pgfsys@useobject{currentmarker}{}%
\end{pgfscope}%
\end{pgfscope}%
\begin{pgfscope}%
\pgfsetbuttcap%
\pgfsetroundjoin%
\definecolor{currentfill}{rgb}{0.000000,0.000000,0.000000}%
\pgfsetfillcolor{currentfill}%
\pgfsetlinewidth{0.602250pt}%
\definecolor{currentstroke}{rgb}{0.000000,0.000000,0.000000}%
\pgfsetstrokecolor{currentstroke}%
\pgfsetdash{}{0pt}%
\pgfsys@defobject{currentmarker}{\pgfqpoint{-0.027778in}{0.000000in}}{\pgfqpoint{-0.000000in}{0.000000in}}{%
\pgfpathmoveto{\pgfqpoint{-0.000000in}{0.000000in}}%
\pgfpathlineto{\pgfqpoint{-0.027778in}{0.000000in}}%
\pgfusepath{stroke,fill}%
}%
\begin{pgfscope}%
\pgfsys@transformshift{0.588387in}{1.030989in}%
\pgfsys@useobject{currentmarker}{}%
\end{pgfscope}%
\end{pgfscope}%
\begin{pgfscope}%
\pgfsetbuttcap%
\pgfsetroundjoin%
\definecolor{currentfill}{rgb}{0.000000,0.000000,0.000000}%
\pgfsetfillcolor{currentfill}%
\pgfsetlinewidth{0.602250pt}%
\definecolor{currentstroke}{rgb}{0.000000,0.000000,0.000000}%
\pgfsetstrokecolor{currentstroke}%
\pgfsetdash{}{0pt}%
\pgfsys@defobject{currentmarker}{\pgfqpoint{-0.027778in}{0.000000in}}{\pgfqpoint{-0.000000in}{0.000000in}}{%
\pgfpathmoveto{\pgfqpoint{-0.000000in}{0.000000in}}%
\pgfpathlineto{\pgfqpoint{-0.027778in}{0.000000in}}%
\pgfusepath{stroke,fill}%
}%
\begin{pgfscope}%
\pgfsys@transformshift{0.588387in}{1.059673in}%
\pgfsys@useobject{currentmarker}{}%
\end{pgfscope}%
\end{pgfscope}%
\begin{pgfscope}%
\pgfsetbuttcap%
\pgfsetroundjoin%
\definecolor{currentfill}{rgb}{0.000000,0.000000,0.000000}%
\pgfsetfillcolor{currentfill}%
\pgfsetlinewidth{0.602250pt}%
\definecolor{currentstroke}{rgb}{0.000000,0.000000,0.000000}%
\pgfsetstrokecolor{currentstroke}%
\pgfsetdash{}{0pt}%
\pgfsys@defobject{currentmarker}{\pgfqpoint{-0.027778in}{0.000000in}}{\pgfqpoint{-0.000000in}{0.000000in}}{%
\pgfpathmoveto{\pgfqpoint{-0.000000in}{0.000000in}}%
\pgfpathlineto{\pgfqpoint{-0.027778in}{0.000000in}}%
\pgfusepath{stroke,fill}%
}%
\begin{pgfscope}%
\pgfsys@transformshift{0.588387in}{1.084974in}%
\pgfsys@useobject{currentmarker}{}%
\end{pgfscope}%
\end{pgfscope}%
\begin{pgfscope}%
\pgfsetbuttcap%
\pgfsetroundjoin%
\definecolor{currentfill}{rgb}{0.000000,0.000000,0.000000}%
\pgfsetfillcolor{currentfill}%
\pgfsetlinewidth{0.602250pt}%
\definecolor{currentstroke}{rgb}{0.000000,0.000000,0.000000}%
\pgfsetstrokecolor{currentstroke}%
\pgfsetdash{}{0pt}%
\pgfsys@defobject{currentmarker}{\pgfqpoint{-0.027778in}{0.000000in}}{\pgfqpoint{-0.000000in}{0.000000in}}{%
\pgfpathmoveto{\pgfqpoint{-0.000000in}{0.000000in}}%
\pgfpathlineto{\pgfqpoint{-0.027778in}{0.000000in}}%
\pgfusepath{stroke,fill}%
}%
\begin{pgfscope}%
\pgfsys@transformshift{0.588387in}{1.256505in}%
\pgfsys@useobject{currentmarker}{}%
\end{pgfscope}%
\end{pgfscope}%
\begin{pgfscope}%
\pgfsetbuttcap%
\pgfsetroundjoin%
\definecolor{currentfill}{rgb}{0.000000,0.000000,0.000000}%
\pgfsetfillcolor{currentfill}%
\pgfsetlinewidth{0.602250pt}%
\definecolor{currentstroke}{rgb}{0.000000,0.000000,0.000000}%
\pgfsetstrokecolor{currentstroke}%
\pgfsetdash{}{0pt}%
\pgfsys@defobject{currentmarker}{\pgfqpoint{-0.027778in}{0.000000in}}{\pgfqpoint{-0.000000in}{0.000000in}}{%
\pgfpathmoveto{\pgfqpoint{-0.000000in}{0.000000in}}%
\pgfpathlineto{\pgfqpoint{-0.027778in}{0.000000in}}%
\pgfusepath{stroke,fill}%
}%
\begin{pgfscope}%
\pgfsys@transformshift{0.588387in}{1.343605in}%
\pgfsys@useobject{currentmarker}{}%
\end{pgfscope}%
\end{pgfscope}%
\begin{pgfscope}%
\pgfsetbuttcap%
\pgfsetroundjoin%
\definecolor{currentfill}{rgb}{0.000000,0.000000,0.000000}%
\pgfsetfillcolor{currentfill}%
\pgfsetlinewidth{0.602250pt}%
\definecolor{currentstroke}{rgb}{0.000000,0.000000,0.000000}%
\pgfsetstrokecolor{currentstroke}%
\pgfsetdash{}{0pt}%
\pgfsys@defobject{currentmarker}{\pgfqpoint{-0.027778in}{0.000000in}}{\pgfqpoint{-0.000000in}{0.000000in}}{%
\pgfpathmoveto{\pgfqpoint{-0.000000in}{0.000000in}}%
\pgfpathlineto{\pgfqpoint{-0.027778in}{0.000000in}}%
\pgfusepath{stroke,fill}%
}%
\begin{pgfscope}%
\pgfsys@transformshift{0.588387in}{1.405403in}%
\pgfsys@useobject{currentmarker}{}%
\end{pgfscope}%
\end{pgfscope}%
\begin{pgfscope}%
\pgfsetbuttcap%
\pgfsetroundjoin%
\definecolor{currentfill}{rgb}{0.000000,0.000000,0.000000}%
\pgfsetfillcolor{currentfill}%
\pgfsetlinewidth{0.602250pt}%
\definecolor{currentstroke}{rgb}{0.000000,0.000000,0.000000}%
\pgfsetstrokecolor{currentstroke}%
\pgfsetdash{}{0pt}%
\pgfsys@defobject{currentmarker}{\pgfqpoint{-0.027778in}{0.000000in}}{\pgfqpoint{-0.000000in}{0.000000in}}{%
\pgfpathmoveto{\pgfqpoint{-0.000000in}{0.000000in}}%
\pgfpathlineto{\pgfqpoint{-0.027778in}{0.000000in}}%
\pgfusepath{stroke,fill}%
}%
\begin{pgfscope}%
\pgfsys@transformshift{0.588387in}{1.453337in}%
\pgfsys@useobject{currentmarker}{}%
\end{pgfscope}%
\end{pgfscope}%
\begin{pgfscope}%
\pgfsetbuttcap%
\pgfsetroundjoin%
\definecolor{currentfill}{rgb}{0.000000,0.000000,0.000000}%
\pgfsetfillcolor{currentfill}%
\pgfsetlinewidth{0.602250pt}%
\definecolor{currentstroke}{rgb}{0.000000,0.000000,0.000000}%
\pgfsetstrokecolor{currentstroke}%
\pgfsetdash{}{0pt}%
\pgfsys@defobject{currentmarker}{\pgfqpoint{-0.027778in}{0.000000in}}{\pgfqpoint{-0.000000in}{0.000000in}}{%
\pgfpathmoveto{\pgfqpoint{-0.000000in}{0.000000in}}%
\pgfpathlineto{\pgfqpoint{-0.027778in}{0.000000in}}%
\pgfusepath{stroke,fill}%
}%
\begin{pgfscope}%
\pgfsys@transformshift{0.588387in}{1.492502in}%
\pgfsys@useobject{currentmarker}{}%
\end{pgfscope}%
\end{pgfscope}%
\begin{pgfscope}%
\pgfsetbuttcap%
\pgfsetroundjoin%
\definecolor{currentfill}{rgb}{0.000000,0.000000,0.000000}%
\pgfsetfillcolor{currentfill}%
\pgfsetlinewidth{0.602250pt}%
\definecolor{currentstroke}{rgb}{0.000000,0.000000,0.000000}%
\pgfsetstrokecolor{currentstroke}%
\pgfsetdash{}{0pt}%
\pgfsys@defobject{currentmarker}{\pgfqpoint{-0.027778in}{0.000000in}}{\pgfqpoint{-0.000000in}{0.000000in}}{%
\pgfpathmoveto{\pgfqpoint{-0.000000in}{0.000000in}}%
\pgfpathlineto{\pgfqpoint{-0.027778in}{0.000000in}}%
\pgfusepath{stroke,fill}%
}%
\begin{pgfscope}%
\pgfsys@transformshift{0.588387in}{1.525616in}%
\pgfsys@useobject{currentmarker}{}%
\end{pgfscope}%
\end{pgfscope}%
\begin{pgfscope}%
\pgfsetbuttcap%
\pgfsetroundjoin%
\definecolor{currentfill}{rgb}{0.000000,0.000000,0.000000}%
\pgfsetfillcolor{currentfill}%
\pgfsetlinewidth{0.602250pt}%
\definecolor{currentstroke}{rgb}{0.000000,0.000000,0.000000}%
\pgfsetstrokecolor{currentstroke}%
\pgfsetdash{}{0pt}%
\pgfsys@defobject{currentmarker}{\pgfqpoint{-0.027778in}{0.000000in}}{\pgfqpoint{-0.000000in}{0.000000in}}{%
\pgfpathmoveto{\pgfqpoint{-0.000000in}{0.000000in}}%
\pgfpathlineto{\pgfqpoint{-0.027778in}{0.000000in}}%
\pgfusepath{stroke,fill}%
}%
\begin{pgfscope}%
\pgfsys@transformshift{0.588387in}{1.554300in}%
\pgfsys@useobject{currentmarker}{}%
\end{pgfscope}%
\end{pgfscope}%
\begin{pgfscope}%
\pgfsetbuttcap%
\pgfsetroundjoin%
\definecolor{currentfill}{rgb}{0.000000,0.000000,0.000000}%
\pgfsetfillcolor{currentfill}%
\pgfsetlinewidth{0.602250pt}%
\definecolor{currentstroke}{rgb}{0.000000,0.000000,0.000000}%
\pgfsetstrokecolor{currentstroke}%
\pgfsetdash{}{0pt}%
\pgfsys@defobject{currentmarker}{\pgfqpoint{-0.027778in}{0.000000in}}{\pgfqpoint{-0.000000in}{0.000000in}}{%
\pgfpathmoveto{\pgfqpoint{-0.000000in}{0.000000in}}%
\pgfpathlineto{\pgfqpoint{-0.027778in}{0.000000in}}%
\pgfusepath{stroke,fill}%
}%
\begin{pgfscope}%
\pgfsys@transformshift{0.588387in}{1.579602in}%
\pgfsys@useobject{currentmarker}{}%
\end{pgfscope}%
\end{pgfscope}%
\begin{pgfscope}%
\pgfsetbuttcap%
\pgfsetroundjoin%
\definecolor{currentfill}{rgb}{0.000000,0.000000,0.000000}%
\pgfsetfillcolor{currentfill}%
\pgfsetlinewidth{0.602250pt}%
\definecolor{currentstroke}{rgb}{0.000000,0.000000,0.000000}%
\pgfsetstrokecolor{currentstroke}%
\pgfsetdash{}{0pt}%
\pgfsys@defobject{currentmarker}{\pgfqpoint{-0.027778in}{0.000000in}}{\pgfqpoint{-0.000000in}{0.000000in}}{%
\pgfpathmoveto{\pgfqpoint{-0.000000in}{0.000000in}}%
\pgfpathlineto{\pgfqpoint{-0.027778in}{0.000000in}}%
\pgfusepath{stroke,fill}%
}%
\begin{pgfscope}%
\pgfsys@transformshift{0.588387in}{1.751132in}%
\pgfsys@useobject{currentmarker}{}%
\end{pgfscope}%
\end{pgfscope}%
\begin{pgfscope}%
\pgfsetbuttcap%
\pgfsetroundjoin%
\definecolor{currentfill}{rgb}{0.000000,0.000000,0.000000}%
\pgfsetfillcolor{currentfill}%
\pgfsetlinewidth{0.602250pt}%
\definecolor{currentstroke}{rgb}{0.000000,0.000000,0.000000}%
\pgfsetstrokecolor{currentstroke}%
\pgfsetdash{}{0pt}%
\pgfsys@defobject{currentmarker}{\pgfqpoint{-0.027778in}{0.000000in}}{\pgfqpoint{-0.000000in}{0.000000in}}{%
\pgfpathmoveto{\pgfqpoint{-0.000000in}{0.000000in}}%
\pgfpathlineto{\pgfqpoint{-0.027778in}{0.000000in}}%
\pgfusepath{stroke,fill}%
}%
\begin{pgfscope}%
\pgfsys@transformshift{0.588387in}{1.838232in}%
\pgfsys@useobject{currentmarker}{}%
\end{pgfscope}%
\end{pgfscope}%
\begin{pgfscope}%
\pgfsetbuttcap%
\pgfsetroundjoin%
\definecolor{currentfill}{rgb}{0.000000,0.000000,0.000000}%
\pgfsetfillcolor{currentfill}%
\pgfsetlinewidth{0.602250pt}%
\definecolor{currentstroke}{rgb}{0.000000,0.000000,0.000000}%
\pgfsetstrokecolor{currentstroke}%
\pgfsetdash{}{0pt}%
\pgfsys@defobject{currentmarker}{\pgfqpoint{-0.027778in}{0.000000in}}{\pgfqpoint{-0.000000in}{0.000000in}}{%
\pgfpathmoveto{\pgfqpoint{-0.000000in}{0.000000in}}%
\pgfpathlineto{\pgfqpoint{-0.027778in}{0.000000in}}%
\pgfusepath{stroke,fill}%
}%
\begin{pgfscope}%
\pgfsys@transformshift{0.588387in}{1.900030in}%
\pgfsys@useobject{currentmarker}{}%
\end{pgfscope}%
\end{pgfscope}%
\begin{pgfscope}%
\pgfsetbuttcap%
\pgfsetroundjoin%
\definecolor{currentfill}{rgb}{0.000000,0.000000,0.000000}%
\pgfsetfillcolor{currentfill}%
\pgfsetlinewidth{0.602250pt}%
\definecolor{currentstroke}{rgb}{0.000000,0.000000,0.000000}%
\pgfsetstrokecolor{currentstroke}%
\pgfsetdash{}{0pt}%
\pgfsys@defobject{currentmarker}{\pgfqpoint{-0.027778in}{0.000000in}}{\pgfqpoint{-0.000000in}{0.000000in}}{%
\pgfpathmoveto{\pgfqpoint{-0.000000in}{0.000000in}}%
\pgfpathlineto{\pgfqpoint{-0.027778in}{0.000000in}}%
\pgfusepath{stroke,fill}%
}%
\begin{pgfscope}%
\pgfsys@transformshift{0.588387in}{1.947964in}%
\pgfsys@useobject{currentmarker}{}%
\end{pgfscope}%
\end{pgfscope}%
\begin{pgfscope}%
\pgfsetbuttcap%
\pgfsetroundjoin%
\definecolor{currentfill}{rgb}{0.000000,0.000000,0.000000}%
\pgfsetfillcolor{currentfill}%
\pgfsetlinewidth{0.602250pt}%
\definecolor{currentstroke}{rgb}{0.000000,0.000000,0.000000}%
\pgfsetstrokecolor{currentstroke}%
\pgfsetdash{}{0pt}%
\pgfsys@defobject{currentmarker}{\pgfqpoint{-0.027778in}{0.000000in}}{\pgfqpoint{-0.000000in}{0.000000in}}{%
\pgfpathmoveto{\pgfqpoint{-0.000000in}{0.000000in}}%
\pgfpathlineto{\pgfqpoint{-0.027778in}{0.000000in}}%
\pgfusepath{stroke,fill}%
}%
\begin{pgfscope}%
\pgfsys@transformshift{0.588387in}{1.987130in}%
\pgfsys@useobject{currentmarker}{}%
\end{pgfscope}%
\end{pgfscope}%
\begin{pgfscope}%
\pgfsetbuttcap%
\pgfsetroundjoin%
\definecolor{currentfill}{rgb}{0.000000,0.000000,0.000000}%
\pgfsetfillcolor{currentfill}%
\pgfsetlinewidth{0.602250pt}%
\definecolor{currentstroke}{rgb}{0.000000,0.000000,0.000000}%
\pgfsetstrokecolor{currentstroke}%
\pgfsetdash{}{0pt}%
\pgfsys@defobject{currentmarker}{\pgfqpoint{-0.027778in}{0.000000in}}{\pgfqpoint{-0.000000in}{0.000000in}}{%
\pgfpathmoveto{\pgfqpoint{-0.000000in}{0.000000in}}%
\pgfpathlineto{\pgfqpoint{-0.027778in}{0.000000in}}%
\pgfusepath{stroke,fill}%
}%
\begin{pgfscope}%
\pgfsys@transformshift{0.588387in}{2.020243in}%
\pgfsys@useobject{currentmarker}{}%
\end{pgfscope}%
\end{pgfscope}%
\begin{pgfscope}%
\pgfsetbuttcap%
\pgfsetroundjoin%
\definecolor{currentfill}{rgb}{0.000000,0.000000,0.000000}%
\pgfsetfillcolor{currentfill}%
\pgfsetlinewidth{0.602250pt}%
\definecolor{currentstroke}{rgb}{0.000000,0.000000,0.000000}%
\pgfsetstrokecolor{currentstroke}%
\pgfsetdash{}{0pt}%
\pgfsys@defobject{currentmarker}{\pgfqpoint{-0.027778in}{0.000000in}}{\pgfqpoint{-0.000000in}{0.000000in}}{%
\pgfpathmoveto{\pgfqpoint{-0.000000in}{0.000000in}}%
\pgfpathlineto{\pgfqpoint{-0.027778in}{0.000000in}}%
\pgfusepath{stroke,fill}%
}%
\begin{pgfscope}%
\pgfsys@transformshift{0.588387in}{2.048928in}%
\pgfsys@useobject{currentmarker}{}%
\end{pgfscope}%
\end{pgfscope}%
\begin{pgfscope}%
\pgfsetbuttcap%
\pgfsetroundjoin%
\definecolor{currentfill}{rgb}{0.000000,0.000000,0.000000}%
\pgfsetfillcolor{currentfill}%
\pgfsetlinewidth{0.602250pt}%
\definecolor{currentstroke}{rgb}{0.000000,0.000000,0.000000}%
\pgfsetstrokecolor{currentstroke}%
\pgfsetdash{}{0pt}%
\pgfsys@defobject{currentmarker}{\pgfqpoint{-0.027778in}{0.000000in}}{\pgfqpoint{-0.000000in}{0.000000in}}{%
\pgfpathmoveto{\pgfqpoint{-0.000000in}{0.000000in}}%
\pgfpathlineto{\pgfqpoint{-0.027778in}{0.000000in}}%
\pgfusepath{stroke,fill}%
}%
\begin{pgfscope}%
\pgfsys@transformshift{0.588387in}{2.074229in}%
\pgfsys@useobject{currentmarker}{}%
\end{pgfscope}%
\end{pgfscope}%
\begin{pgfscope}%
\pgfsetbuttcap%
\pgfsetroundjoin%
\definecolor{currentfill}{rgb}{0.000000,0.000000,0.000000}%
\pgfsetfillcolor{currentfill}%
\pgfsetlinewidth{0.602250pt}%
\definecolor{currentstroke}{rgb}{0.000000,0.000000,0.000000}%
\pgfsetstrokecolor{currentstroke}%
\pgfsetdash{}{0pt}%
\pgfsys@defobject{currentmarker}{\pgfqpoint{-0.027778in}{0.000000in}}{\pgfqpoint{-0.000000in}{0.000000in}}{%
\pgfpathmoveto{\pgfqpoint{-0.000000in}{0.000000in}}%
\pgfpathlineto{\pgfqpoint{-0.027778in}{0.000000in}}%
\pgfusepath{stroke,fill}%
}%
\begin{pgfscope}%
\pgfsys@transformshift{0.588387in}{2.245760in}%
\pgfsys@useobject{currentmarker}{}%
\end{pgfscope}%
\end{pgfscope}%
\begin{pgfscope}%
\pgfsetbuttcap%
\pgfsetroundjoin%
\definecolor{currentfill}{rgb}{0.000000,0.000000,0.000000}%
\pgfsetfillcolor{currentfill}%
\pgfsetlinewidth{0.602250pt}%
\definecolor{currentstroke}{rgb}{0.000000,0.000000,0.000000}%
\pgfsetstrokecolor{currentstroke}%
\pgfsetdash{}{0pt}%
\pgfsys@defobject{currentmarker}{\pgfqpoint{-0.027778in}{0.000000in}}{\pgfqpoint{-0.000000in}{0.000000in}}{%
\pgfpathmoveto{\pgfqpoint{-0.000000in}{0.000000in}}%
\pgfpathlineto{\pgfqpoint{-0.027778in}{0.000000in}}%
\pgfusepath{stroke,fill}%
}%
\begin{pgfscope}%
\pgfsys@transformshift{0.588387in}{2.332859in}%
\pgfsys@useobject{currentmarker}{}%
\end{pgfscope}%
\end{pgfscope}%
\begin{pgfscope}%
\pgfsetbuttcap%
\pgfsetroundjoin%
\definecolor{currentfill}{rgb}{0.000000,0.000000,0.000000}%
\pgfsetfillcolor{currentfill}%
\pgfsetlinewidth{0.602250pt}%
\definecolor{currentstroke}{rgb}{0.000000,0.000000,0.000000}%
\pgfsetstrokecolor{currentstroke}%
\pgfsetdash{}{0pt}%
\pgfsys@defobject{currentmarker}{\pgfqpoint{-0.027778in}{0.000000in}}{\pgfqpoint{-0.000000in}{0.000000in}}{%
\pgfpathmoveto{\pgfqpoint{-0.000000in}{0.000000in}}%
\pgfpathlineto{\pgfqpoint{-0.027778in}{0.000000in}}%
\pgfusepath{stroke,fill}%
}%
\begin{pgfscope}%
\pgfsys@transformshift{0.588387in}{2.394657in}%
\pgfsys@useobject{currentmarker}{}%
\end{pgfscope}%
\end{pgfscope}%
\begin{pgfscope}%
\pgfsetbuttcap%
\pgfsetroundjoin%
\definecolor{currentfill}{rgb}{0.000000,0.000000,0.000000}%
\pgfsetfillcolor{currentfill}%
\pgfsetlinewidth{0.602250pt}%
\definecolor{currentstroke}{rgb}{0.000000,0.000000,0.000000}%
\pgfsetstrokecolor{currentstroke}%
\pgfsetdash{}{0pt}%
\pgfsys@defobject{currentmarker}{\pgfqpoint{-0.027778in}{0.000000in}}{\pgfqpoint{-0.000000in}{0.000000in}}{%
\pgfpathmoveto{\pgfqpoint{-0.000000in}{0.000000in}}%
\pgfpathlineto{\pgfqpoint{-0.027778in}{0.000000in}}%
\pgfusepath{stroke,fill}%
}%
\begin{pgfscope}%
\pgfsys@transformshift{0.588387in}{2.442592in}%
\pgfsys@useobject{currentmarker}{}%
\end{pgfscope}%
\end{pgfscope}%
\begin{pgfscope}%
\pgfsetbuttcap%
\pgfsetroundjoin%
\definecolor{currentfill}{rgb}{0.000000,0.000000,0.000000}%
\pgfsetfillcolor{currentfill}%
\pgfsetlinewidth{0.602250pt}%
\definecolor{currentstroke}{rgb}{0.000000,0.000000,0.000000}%
\pgfsetstrokecolor{currentstroke}%
\pgfsetdash{}{0pt}%
\pgfsys@defobject{currentmarker}{\pgfqpoint{-0.027778in}{0.000000in}}{\pgfqpoint{-0.000000in}{0.000000in}}{%
\pgfpathmoveto{\pgfqpoint{-0.000000in}{0.000000in}}%
\pgfpathlineto{\pgfqpoint{-0.027778in}{0.000000in}}%
\pgfusepath{stroke,fill}%
}%
\begin{pgfscope}%
\pgfsys@transformshift{0.588387in}{2.481757in}%
\pgfsys@useobject{currentmarker}{}%
\end{pgfscope}%
\end{pgfscope}%
\begin{pgfscope}%
\pgfsetbuttcap%
\pgfsetroundjoin%
\definecolor{currentfill}{rgb}{0.000000,0.000000,0.000000}%
\pgfsetfillcolor{currentfill}%
\pgfsetlinewidth{0.602250pt}%
\definecolor{currentstroke}{rgb}{0.000000,0.000000,0.000000}%
\pgfsetstrokecolor{currentstroke}%
\pgfsetdash{}{0pt}%
\pgfsys@defobject{currentmarker}{\pgfqpoint{-0.027778in}{0.000000in}}{\pgfqpoint{-0.000000in}{0.000000in}}{%
\pgfpathmoveto{\pgfqpoint{-0.000000in}{0.000000in}}%
\pgfpathlineto{\pgfqpoint{-0.027778in}{0.000000in}}%
\pgfusepath{stroke,fill}%
}%
\begin{pgfscope}%
\pgfsys@transformshift{0.588387in}{2.514871in}%
\pgfsys@useobject{currentmarker}{}%
\end{pgfscope}%
\end{pgfscope}%
\begin{pgfscope}%
\definecolor{textcolor}{rgb}{0.000000,0.000000,0.000000}%
\pgfsetstrokecolor{textcolor}%
\pgfsetfillcolor{textcolor}%
\pgftext[x=0.234413in,y=1.526746in,,bottom,rotate=90.000000]{\color{textcolor}{\rmfamily\fontsize{10.000000}{12.000000}\selectfont\catcode`\^=\active\def^{\ifmmode\sp\else\^{}\fi}\catcode`\%=\active\def%{\%}Time [ms]}}%
\end{pgfscope}%
\begin{pgfscope}%
\pgfpathrectangle{\pgfqpoint{0.588387in}{0.521603in}}{\pgfqpoint{4.669024in}{2.010285in}}%
\pgfusepath{clip}%
\pgfsetrectcap%
\pgfsetroundjoin%
\pgfsetlinewidth{1.505625pt}%
\pgfsetstrokecolor{currentstroke1}%
\pgfsetdash{}{0pt}%
\pgfpathmoveto{\pgfqpoint{0.800616in}{0.612980in}}%
\pgfpathlineto{\pgfqpoint{0.825151in}{0.612980in}}%
\pgfpathlineto{\pgfqpoint{0.874221in}{0.761878in}}%
\pgfpathlineto{\pgfqpoint{0.923291in}{0.848977in}}%
\pgfpathlineto{\pgfqpoint{0.972361in}{0.910775in}}%
\pgfpathlineto{\pgfqpoint{1.045966in}{0.997875in}}%
\pgfpathlineto{\pgfqpoint{1.070501in}{0.927970in}}%
\pgfpathlineto{\pgfqpoint{1.119572in}{1.055802in}}%
\pgfpathlineto{\pgfqpoint{1.193177in}{1.084974in}}%
\pgfpathlineto{\pgfqpoint{1.242247in}{1.146772in}}%
\pgfpathlineto{\pgfqpoint{1.315852in}{1.208571in}}%
\pgfpathlineto{\pgfqpoint{1.364922in}{1.229488in}}%
\pgfpathlineto{\pgfqpoint{1.438527in}{1.278369in}}%
\pgfpathlineto{\pgfqpoint{1.512133in}{1.319908in}}%
\pgfpathlineto{\pgfqpoint{1.610273in}{1.376718in}}%
\pgfpathlineto{\pgfqpoint{1.683878in}{1.420938in}}%
\pgfpathlineto{\pgfqpoint{1.757483in}{1.443071in}}%
\pgfpathlineto{\pgfqpoint{1.831089in}{1.492502in}}%
\pgfpathlineto{\pgfqpoint{1.904694in}{1.506845in}}%
\pgfpathlineto{\pgfqpoint{2.002834in}{1.548862in}}%
\pgfpathlineto{\pgfqpoint{2.076439in}{1.562208in}}%
\pgfpathlineto{\pgfqpoint{2.174580in}{1.596796in}}%
\pgfpathlineto{\pgfqpoint{2.272720in}{1.626579in}}%
\pgfpathlineto{\pgfqpoint{2.346325in}{1.652731in}}%
\pgfpathlineto{\pgfqpoint{2.444465in}{1.714954in}}%
\pgfpathlineto{\pgfqpoint{2.542606in}{1.712396in}}%
\pgfpathlineto{\pgfqpoint{2.640746in}{1.755912in}}%
\pgfpathlineto{\pgfqpoint{2.763421in}{1.772580in}}%
\pgfpathlineto{\pgfqpoint{2.812491in}{1.788048in}}%
\pgfpathlineto{\pgfqpoint{2.886097in}{1.804681in}}%
\pgfpathlineto{\pgfqpoint{3.008772in}{1.831541in}}%
\pgfpathlineto{\pgfqpoint{3.106912in}{1.857382in}}%
\pgfpathlineto{\pgfqpoint{3.229588in}{1.887309in}}%
\pgfpathlineto{\pgfqpoint{3.352263in}{1.904547in}}%
\pgfpathlineto{\pgfqpoint{3.474938in}{1.933297in}}%
\pgfpathlineto{\pgfqpoint{3.622149in}{1.963300in}}%
\pgfpathlineto{\pgfqpoint{3.720289in}{1.994173in}}%
\pgfpathlineto{\pgfqpoint{3.842964in}{2.021467in}}%
\pgfpathlineto{\pgfqpoint{3.965640in}{2.031308in}}%
\pgfpathlineto{\pgfqpoint{4.088315in}{2.064837in}}%
\pgfpathlineto{\pgfqpoint{4.235526in}{2.082884in}}%
\pgfpathlineto{\pgfqpoint{4.358201in}{2.110692in}}%
\pgfpathlineto{\pgfqpoint{4.505411in}{2.126885in}}%
\pgfpathlineto{\pgfqpoint{4.652622in}{2.157961in}}%
\pgfpathlineto{\pgfqpoint{4.799832in}{2.284118in}}%
\pgfpathlineto{\pgfqpoint{4.971578in}{2.204826in}}%
\pgfpathlineto{\pgfqpoint{4.996113in}{2.236607in}}%
\pgfpathlineto{\pgfqpoint{5.045183in}{2.222990in}}%
\pgfusepath{stroke}%
\end{pgfscope}%
\begin{pgfscope}%
\pgfpathrectangle{\pgfqpoint{0.588387in}{0.521603in}}{\pgfqpoint{4.669024in}{2.010285in}}%
\pgfusepath{clip}%
\pgfsetrectcap%
\pgfsetroundjoin%
\pgfsetlinewidth{1.505625pt}%
\pgfsetstrokecolor{currentstroke2}%
\pgfsetdash{}{0pt}%
\pgfpathmoveto{\pgfqpoint{0.800616in}{0.612980in}}%
\pgfpathlineto{\pgfqpoint{0.825151in}{0.612980in}}%
\pgfpathlineto{\pgfqpoint{0.874221in}{0.612980in}}%
\pgfpathlineto{\pgfqpoint{0.923291in}{0.848977in}}%
\pgfpathlineto{\pgfqpoint{0.972361in}{0.910775in}}%
\pgfpathlineto{\pgfqpoint{1.045966in}{0.997875in}}%
\pgfpathlineto{\pgfqpoint{1.070501in}{1.020508in}}%
\pgfpathlineto{\pgfqpoint{1.119572in}{1.538381in}}%
\pgfpathlineto{\pgfqpoint{1.193177in}{1.107607in}}%
\pgfpathlineto{\pgfqpoint{1.242247in}{2.057353in}}%
\pgfpathlineto{\pgfqpoint{1.315852in}{2.346185in}}%
\pgfpathlineto{\pgfqpoint{1.364922in}{1.891660in}}%
\pgfpathlineto{\pgfqpoint{1.438527in}{2.244860in}}%
\pgfpathlineto{\pgfqpoint{1.512133in}{2.231636in}}%
\pgfpathlineto{\pgfqpoint{1.831089in}{2.357310in}}%
\pgfpathlineto{\pgfqpoint{1.904694in}{2.404289in}}%
\pgfpathlineto{\pgfqpoint{2.076439in}{2.404164in}}%
\pgfpathlineto{\pgfqpoint{2.174580in}{2.404344in}}%
\pgfpathlineto{\pgfqpoint{2.346325in}{2.196883in}}%
\pgfpathlineto{\pgfqpoint{2.640746in}{2.315957in}}%
\pgfpathlineto{\pgfqpoint{2.812491in}{2.391492in}}%
\pgfpathlineto{\pgfqpoint{2.886097in}{2.416464in}}%
\pgfpathlineto{\pgfqpoint{3.008772in}{2.404072in}}%
\pgfpathlineto{\pgfqpoint{3.106912in}{2.420560in}}%
\pgfpathlineto{\pgfqpoint{3.229588in}{2.380089in}}%
\pgfpathlineto{\pgfqpoint{4.996113in}{2.366411in}}%
\pgfpathlineto{\pgfqpoint{5.045183in}{2.434149in}}%
\pgfusepath{stroke}%
\end{pgfscope}%
\begin{pgfscope}%
\pgfpathrectangle{\pgfqpoint{0.588387in}{0.521603in}}{\pgfqpoint{4.669024in}{2.010285in}}%
\pgfusepath{clip}%
\pgfsetrectcap%
\pgfsetroundjoin%
\pgfsetlinewidth{1.505625pt}%
\pgfsetstrokecolor{currentstroke3}%
\pgfsetdash{}{0pt}%
\pgfpathmoveto{\pgfqpoint{0.800616in}{0.722712in}}%
\pgfpathlineto{\pgfqpoint{0.825151in}{0.612980in}}%
\pgfpathlineto{\pgfqpoint{0.874221in}{0.674778in}}%
\pgfpathlineto{\pgfqpoint{0.923291in}{0.700079in}}%
\pgfpathlineto{\pgfqpoint{0.972361in}{0.910775in}}%
\pgfpathlineto{\pgfqpoint{1.045966in}{0.979184in}}%
\pgfpathlineto{\pgfqpoint{1.070501in}{0.991823in}}%
\pgfpathlineto{\pgfqpoint{1.119572in}{1.055802in}}%
\pgfpathlineto{\pgfqpoint{1.193177in}{1.128081in}}%
\pgfpathlineto{\pgfqpoint{1.242247in}{1.189879in}}%
\pgfpathlineto{\pgfqpoint{1.315852in}{1.208571in}}%
\pgfpathlineto{\pgfqpoint{1.364922in}{1.255363in}}%
\pgfpathlineto{\pgfqpoint{1.438527in}{1.282034in}}%
\pgfpathlineto{\pgfqpoint{1.512133in}{1.352641in}}%
\pgfpathlineto{\pgfqpoint{1.610273in}{1.376718in}}%
\pgfpathlineto{\pgfqpoint{1.683878in}{1.436976in}}%
\pgfpathlineto{\pgfqpoint{1.757483in}{1.448021in}}%
\pgfpathlineto{\pgfqpoint{1.831089in}{1.488081in}}%
\pgfpathlineto{\pgfqpoint{1.904694in}{1.503793in}}%
\pgfpathlineto{\pgfqpoint{2.002834in}{1.581975in}}%
\pgfpathlineto{\pgfqpoint{2.076439in}{1.563499in}}%
\pgfpathlineto{\pgfqpoint{2.174580in}{1.598260in}}%
\pgfpathlineto{\pgfqpoint{2.272720in}{1.631634in}}%
\pgfpathlineto{\pgfqpoint{2.346325in}{1.655124in}}%
\pgfpathlineto{\pgfqpoint{2.444465in}{1.685966in}}%
\pgfpathlineto{\pgfqpoint{2.542606in}{1.721217in}}%
\pgfpathlineto{\pgfqpoint{2.640746in}{1.742550in}}%
\pgfpathlineto{\pgfqpoint{2.763421in}{1.771281in}}%
\pgfpathlineto{\pgfqpoint{2.812491in}{1.781622in}}%
\pgfpathlineto{\pgfqpoint{2.886097in}{1.795438in}}%
\pgfpathlineto{\pgfqpoint{3.008772in}{1.820009in}}%
\pgfpathlineto{\pgfqpoint{3.106912in}{1.834311in}}%
\pgfpathlineto{\pgfqpoint{3.229588in}{1.857493in}}%
\pgfpathlineto{\pgfqpoint{3.352263in}{1.898683in}}%
\pgfpathlineto{\pgfqpoint{3.474938in}{1.904635in}}%
\pgfpathlineto{\pgfqpoint{3.622149in}{1.936644in}}%
\pgfpathlineto{\pgfqpoint{3.720289in}{1.961559in}}%
\pgfpathlineto{\pgfqpoint{3.842964in}{1.987010in}}%
\pgfpathlineto{\pgfqpoint{3.965640in}{2.044862in}}%
\pgfpathlineto{\pgfqpoint{4.088315in}{2.066576in}}%
\pgfpathlineto{\pgfqpoint{4.235526in}{2.092741in}}%
\pgfpathlineto{\pgfqpoint{4.358201in}{2.119183in}}%
\pgfpathlineto{\pgfqpoint{4.505411in}{2.146849in}}%
\pgfpathlineto{\pgfqpoint{4.652622in}{2.175192in}}%
\pgfpathlineto{\pgfqpoint{4.799832in}{2.199430in}}%
\pgfpathlineto{\pgfqpoint{4.971578in}{2.222529in}}%
\pgfpathlineto{\pgfqpoint{4.996113in}{2.193857in}}%
\pgfpathlineto{\pgfqpoint{5.045183in}{2.232669in}}%
\pgfusepath{stroke}%
\end{pgfscope}%
\begin{pgfscope}%
\pgfpathrectangle{\pgfqpoint{0.588387in}{0.521603in}}{\pgfqpoint{4.669024in}{2.010285in}}%
\pgfusepath{clip}%
\pgfsetrectcap%
\pgfsetroundjoin%
\pgfsetlinewidth{1.505625pt}%
\pgfsetstrokecolor{currentstroke4}%
\pgfsetdash{}{0pt}%
\pgfpathmoveto{\pgfqpoint{0.800616in}{0.612980in}}%
\pgfpathlineto{\pgfqpoint{0.825151in}{0.612980in}}%
\pgfpathlineto{\pgfqpoint{0.874221in}{0.700079in}}%
\pgfpathlineto{\pgfqpoint{0.923291in}{0.761878in}}%
\pgfpathlineto{\pgfqpoint{0.972361in}{0.958710in}}%
\pgfpathlineto{\pgfqpoint{1.045966in}{0.997875in}}%
\pgfpathlineto{\pgfqpoint{1.070501in}{1.009489in}}%
\pgfpathlineto{\pgfqpoint{1.119572in}{1.237255in}}%
\pgfpathlineto{\pgfqpoint{1.193177in}{1.442318in}}%
\pgfpathlineto{\pgfqpoint{1.242247in}{1.563499in}}%
\pgfpathlineto{\pgfqpoint{1.315852in}{1.266986in}}%
\pgfpathlineto{\pgfqpoint{1.364922in}{1.282023in}}%
\pgfpathlineto{\pgfqpoint{1.438527in}{1.351882in}}%
\pgfpathlineto{\pgfqpoint{1.512133in}{1.363427in}}%
\pgfpathlineto{\pgfqpoint{1.610273in}{2.009225in}}%
\pgfpathlineto{\pgfqpoint{1.683878in}{1.716221in}}%
\pgfpathlineto{\pgfqpoint{1.757483in}{1.507482in}}%
\pgfpathlineto{\pgfqpoint{1.831089in}{1.563927in}}%
\pgfpathlineto{\pgfqpoint{1.904694in}{1.560910in}}%
\pgfpathlineto{\pgfqpoint{2.002834in}{1.558291in}}%
\pgfpathlineto{\pgfqpoint{2.076439in}{1.583152in}}%
\pgfpathlineto{\pgfqpoint{2.174580in}{1.604903in}}%
\pgfpathlineto{\pgfqpoint{2.272720in}{1.648444in}}%
\pgfpathlineto{\pgfqpoint{2.346325in}{1.669272in}}%
\pgfpathlineto{\pgfqpoint{2.444465in}{1.734965in}}%
\pgfpathlineto{\pgfqpoint{2.542606in}{1.726702in}}%
\pgfpathlineto{\pgfqpoint{2.640746in}{1.910255in}}%
\pgfpathlineto{\pgfqpoint{2.763421in}{1.811989in}}%
\pgfpathlineto{\pgfqpoint{2.812491in}{1.962063in}}%
\pgfpathlineto{\pgfqpoint{2.886097in}{2.132048in}}%
\pgfpathlineto{\pgfqpoint{3.008772in}{1.914413in}}%
\pgfpathlineto{\pgfqpoint{3.106912in}{1.935717in}}%
\pgfpathlineto{\pgfqpoint{3.229588in}{1.963928in}}%
\pgfpathlineto{\pgfqpoint{3.352263in}{1.931913in}}%
\pgfpathlineto{\pgfqpoint{3.474938in}{1.937172in}}%
\pgfpathlineto{\pgfqpoint{3.622149in}{1.985151in}}%
\pgfpathlineto{\pgfqpoint{3.720289in}{1.992957in}}%
\pgfpathlineto{\pgfqpoint{3.842964in}{2.035779in}}%
\pgfpathlineto{\pgfqpoint{4.088315in}{2.102794in}}%
\pgfpathlineto{\pgfqpoint{4.358201in}{2.126136in}}%
\pgfpathlineto{\pgfqpoint{4.652622in}{2.178008in}}%
\pgfpathlineto{\pgfqpoint{4.971578in}{2.228896in}}%
\pgfpathlineto{\pgfqpoint{4.996113in}{2.242154in}}%
\pgfpathlineto{\pgfqpoint{5.045183in}{2.254068in}}%
\pgfusepath{stroke}%
\end{pgfscope}%
\begin{pgfscope}%
\pgfpathrectangle{\pgfqpoint{0.588387in}{0.521603in}}{\pgfqpoint{4.669024in}{2.010285in}}%
\pgfusepath{clip}%
\pgfsetrectcap%
\pgfsetroundjoin%
\pgfsetlinewidth{1.505625pt}%
\pgfsetstrokecolor{currentstroke5}%
\pgfsetdash{}{0pt}%
\pgfpathmoveto{\pgfqpoint{0.800616in}{0.612980in}}%
\pgfpathlineto{\pgfqpoint{0.825151in}{0.612980in}}%
\pgfpathlineto{\pgfqpoint{0.874221in}{0.612980in}}%
\pgfpathlineto{\pgfqpoint{0.923291in}{0.700079in}}%
\pgfpathlineto{\pgfqpoint{0.972361in}{0.910775in}}%
\pgfpathlineto{\pgfqpoint{1.045966in}{0.988732in}}%
\pgfpathlineto{\pgfqpoint{1.070501in}{0.969190in}}%
\pgfpathlineto{\pgfqpoint{1.119572in}{1.049862in}}%
\pgfpathlineto{\pgfqpoint{1.193177in}{1.132909in}}%
\pgfpathlineto{\pgfqpoint{1.242247in}{1.203476in}}%
\pgfpathlineto{\pgfqpoint{1.315852in}{1.201751in}}%
\pgfpathlineto{\pgfqpoint{1.364922in}{1.267218in}}%
\pgfpathlineto{\pgfqpoint{1.438527in}{1.281465in}}%
\pgfpathlineto{\pgfqpoint{1.512133in}{1.365592in}}%
\pgfpathlineto{\pgfqpoint{1.610273in}{1.391539in}}%
\pgfpathlineto{\pgfqpoint{1.683878in}{1.437748in}}%
\pgfpathlineto{\pgfqpoint{1.757483in}{1.443071in}}%
\pgfpathlineto{\pgfqpoint{1.831089in}{1.496639in}}%
\pgfpathlineto{\pgfqpoint{1.904694in}{1.501517in}}%
\pgfpathlineto{\pgfqpoint{2.002834in}{1.540437in}}%
\pgfpathlineto{\pgfqpoint{2.076439in}{1.565844in}}%
\pgfpathlineto{\pgfqpoint{2.174580in}{1.594024in}}%
\pgfpathlineto{\pgfqpoint{2.272720in}{1.631321in}}%
\pgfpathlineto{\pgfqpoint{2.346325in}{1.648444in}}%
\pgfpathlineto{\pgfqpoint{2.444465in}{1.684994in}}%
\pgfpathlineto{\pgfqpoint{2.542606in}{1.703198in}}%
\pgfpathlineto{\pgfqpoint{2.640746in}{1.718421in}}%
\pgfpathlineto{\pgfqpoint{2.763421in}{1.749515in}}%
\pgfpathlineto{\pgfqpoint{2.812491in}{1.761400in}}%
\pgfpathlineto{\pgfqpoint{2.886097in}{1.780153in}}%
\pgfpathlineto{\pgfqpoint{3.008772in}{1.811908in}}%
\pgfpathlineto{\pgfqpoint{3.106912in}{1.821468in}}%
\pgfpathlineto{\pgfqpoint{3.229588in}{1.838436in}}%
\pgfpathlineto{\pgfqpoint{3.352263in}{1.872875in}}%
\pgfpathlineto{\pgfqpoint{3.474938in}{1.887594in}}%
\pgfpathlineto{\pgfqpoint{3.622149in}{1.925093in}}%
\pgfpathlineto{\pgfqpoint{3.720289in}{1.944281in}}%
\pgfpathlineto{\pgfqpoint{3.842964in}{1.968243in}}%
\pgfusepath{stroke}%
\end{pgfscope}%
\begin{pgfscope}%
\pgfpathrectangle{\pgfqpoint{0.588387in}{0.521603in}}{\pgfqpoint{4.669024in}{2.010285in}}%
\pgfusepath{clip}%
\pgfsetrectcap%
\pgfsetroundjoin%
\pgfsetlinewidth{1.505625pt}%
\pgfsetstrokecolor{currentstroke6}%
\pgfsetdash{}{0pt}%
\pgfpathmoveto{\pgfqpoint{0.800616in}{0.700079in}}%
\pgfpathlineto{\pgfqpoint{0.825151in}{0.612980in}}%
\pgfpathlineto{\pgfqpoint{0.874221in}{0.674778in}}%
\pgfpathlineto{\pgfqpoint{0.923291in}{0.700079in}}%
\pgfpathlineto{\pgfqpoint{0.972361in}{0.882091in}}%
\pgfpathlineto{\pgfqpoint{1.045966in}{0.958710in}}%
\pgfpathlineto{\pgfqpoint{1.070501in}{0.958710in}}%
\pgfpathlineto{\pgfqpoint{1.119572in}{1.026559in}}%
\pgfpathlineto{\pgfqpoint{1.193177in}{1.102169in}}%
\pgfpathlineto{\pgfqpoint{1.242247in}{1.148259in}}%
\pgfpathlineto{\pgfqpoint{1.315852in}{1.184941in}}%
\pgfpathlineto{\pgfqpoint{1.364922in}{1.235493in}}%
\pgfpathlineto{\pgfqpoint{1.438527in}{1.254449in}}%
\pgfpathlineto{\pgfqpoint{1.512133in}{1.327072in}}%
\pgfpathlineto{\pgfqpoint{1.610273in}{1.350648in}}%
\pgfpathlineto{\pgfqpoint{1.683878in}{1.412446in}}%
\pgfpathlineto{\pgfqpoint{1.757483in}{1.422597in}}%
\pgfpathlineto{\pgfqpoint{1.831089in}{1.472068in}}%
\pgfpathlineto{\pgfqpoint{1.904694in}{1.479682in}}%
\pgfpathlineto{\pgfqpoint{2.002834in}{1.543282in}}%
\pgfpathlineto{\pgfqpoint{2.076439in}{1.544222in}}%
\pgfpathlineto{\pgfqpoint{2.174580in}{1.569419in}}%
\pgfpathlineto{\pgfqpoint{2.272720in}{1.621403in}}%
\pgfpathlineto{\pgfqpoint{2.346325in}{1.623359in}}%
\pgfpathlineto{\pgfqpoint{2.444465in}{1.650169in}}%
\pgfpathlineto{\pgfqpoint{2.542606in}{1.687416in}}%
\pgfpathlineto{\pgfqpoint{2.640746in}{1.716852in}}%
\pgfpathlineto{\pgfqpoint{2.763421in}{1.750774in}}%
\pgfpathlineto{\pgfqpoint{2.812491in}{1.773470in}}%
\pgfpathlineto{\pgfqpoint{2.886097in}{1.781349in}}%
\pgfpathlineto{\pgfqpoint{3.008772in}{1.801742in}}%
\pgfpathlineto{\pgfqpoint{3.106912in}{1.820356in}}%
\pgfpathlineto{\pgfqpoint{3.229588in}{1.833857in}}%
\pgfpathlineto{\pgfqpoint{3.352263in}{1.862789in}}%
\pgfpathlineto{\pgfqpoint{3.474938in}{1.904810in}}%
\pgfpathlineto{\pgfqpoint{3.622149in}{1.911022in}}%
\pgfpathlineto{\pgfqpoint{3.720289in}{1.940385in}}%
\pgfpathlineto{\pgfqpoint{3.842964in}{1.966608in}}%
\pgfpathlineto{\pgfqpoint{3.965640in}{2.027781in}}%
\pgfpathlineto{\pgfqpoint{4.088315in}{2.133053in}}%
\pgfpathlineto{\pgfqpoint{4.235526in}{2.084482in}}%
\pgfpathlineto{\pgfqpoint{4.358201in}{2.098999in}}%
\pgfpathlineto{\pgfqpoint{4.505411in}{2.168526in}}%
\pgfpathlineto{\pgfqpoint{4.652622in}{2.154375in}}%
\pgfpathlineto{\pgfqpoint{4.799832in}{2.181803in}}%
\pgfpathlineto{\pgfqpoint{4.971578in}{2.201948in}}%
\pgfpathlineto{\pgfqpoint{4.996113in}{2.181120in}}%
\pgfpathlineto{\pgfqpoint{5.045183in}{2.317788in}}%
\pgfusepath{stroke}%
\end{pgfscope}%
\begin{pgfscope}%
\pgfpathrectangle{\pgfqpoint{0.588387in}{0.521603in}}{\pgfqpoint{4.669024in}{2.010285in}}%
\pgfusepath{clip}%
\pgfsetrectcap%
\pgfsetroundjoin%
\pgfsetlinewidth{1.505625pt}%
\pgfsetstrokecolor{currentstroke7}%
\pgfsetdash{}{0pt}%
\pgfpathmoveto{\pgfqpoint{0.800616in}{0.612980in}}%
\pgfpathlineto{\pgfqpoint{0.825151in}{0.612980in}}%
\pgfpathlineto{\pgfqpoint{0.874221in}{0.612980in}}%
\pgfpathlineto{\pgfqpoint{0.923291in}{0.700079in}}%
\pgfpathlineto{\pgfqpoint{0.972361in}{0.958710in}}%
\pgfpathlineto{\pgfqpoint{1.045966in}{1.084974in}}%
\pgfpathlineto{\pgfqpoint{1.070501in}{1.223689in}}%
\pgfpathlineto{\pgfqpoint{1.119572in}{1.382770in}}%
\pgfpathlineto{\pgfqpoint{1.193177in}{1.580792in}}%
\pgfpathlineto{\pgfqpoint{1.242247in}{1.192909in}}%
\pgfpathlineto{\pgfqpoint{1.315852in}{1.905858in}}%
\pgfpathlineto{\pgfqpoint{1.364922in}{2.323640in}}%
\pgfpathlineto{\pgfqpoint{1.438527in}{2.229025in}}%
\pgfpathlineto{\pgfqpoint{1.512133in}{2.326903in}}%
\pgfpathlineto{\pgfqpoint{1.831089in}{2.408051in}}%
\pgfpathlineto{\pgfqpoint{2.812491in}{2.433936in}}%
\pgfpathlineto{\pgfqpoint{2.886097in}{2.439624in}}%
\pgfpathlineto{\pgfqpoint{3.106912in}{2.440512in}}%
\pgfpathlineto{\pgfqpoint{3.229588in}{2.427829in}}%
\pgfpathlineto{\pgfqpoint{4.996113in}{2.409557in}}%
\pgfusepath{stroke}%
\end{pgfscope}%
\begin{pgfscope}%
\pgfpathrectangle{\pgfqpoint{0.588387in}{0.521603in}}{\pgfqpoint{4.669024in}{2.010285in}}%
\pgfusepath{clip}%
\pgfsetrectcap%
\pgfsetroundjoin%
\pgfsetlinewidth{1.505625pt}%
\definecolor{currentstroke}{rgb}{0.498039,0.498039,0.498039}%
\pgfsetstrokecolor{currentstroke}%
\pgfsetdash{}{0pt}%
\pgfpathmoveto{\pgfqpoint{0.923291in}{0.612980in}}%
\pgfpathlineto{\pgfqpoint{0.972361in}{0.612980in}}%
\pgfpathlineto{\pgfqpoint{1.045966in}{0.761878in}}%
\pgfpathlineto{\pgfqpoint{1.070501in}{0.674778in}}%
\pgfpathlineto{\pgfqpoint{1.119572in}{0.761878in}}%
\pgfpathlineto{\pgfqpoint{1.193177in}{0.761878in}}%
\pgfpathlineto{\pgfqpoint{1.242247in}{0.848977in}}%
\pgfpathlineto{\pgfqpoint{1.315852in}{0.848977in}}%
\pgfpathlineto{\pgfqpoint{1.364922in}{0.864664in}}%
\pgfpathlineto{\pgfqpoint{1.438527in}{0.914577in}}%
\pgfpathlineto{\pgfqpoint{1.512133in}{0.949940in}}%
\pgfpathlineto{\pgfqpoint{1.610273in}{0.958710in}}%
\pgfpathlineto{\pgfqpoint{1.683878in}{1.030989in}}%
\pgfpathlineto{\pgfqpoint{1.757483in}{1.059673in}}%
\pgfpathlineto{\pgfqpoint{1.831089in}{1.059673in}}%
\pgfpathlineto{\pgfqpoint{1.904694in}{1.098198in}}%
\pgfpathlineto{\pgfqpoint{2.002834in}{1.128081in}}%
\pgfpathlineto{\pgfqpoint{2.076439in}{1.146772in}}%
\pgfpathlineto{\pgfqpoint{2.174580in}{1.179886in}}%
\pgfpathlineto{\pgfqpoint{2.272720in}{1.194707in}}%
\pgfpathlineto{\pgfqpoint{2.346325in}{1.215181in}}%
\pgfpathlineto{\pgfqpoint{2.444465in}{1.233872in}}%
\pgfpathlineto{\pgfqpoint{2.542606in}{1.266986in}}%
\pgfpathlineto{\pgfqpoint{2.640746in}{1.286528in}}%
\pgfpathlineto{\pgfqpoint{2.763421in}{1.312864in}}%
\pgfpathlineto{\pgfqpoint{2.812491in}{1.318973in}}%
\pgfpathlineto{\pgfqpoint{2.886097in}{1.328525in}}%
\pgfpathlineto{\pgfqpoint{3.008772in}{1.350648in}}%
\pgfpathlineto{\pgfqpoint{3.106912in}{1.369259in}}%
\pgfpathlineto{\pgfqpoint{3.229588in}{1.394384in}}%
\pgfpathlineto{\pgfqpoint{3.352263in}{1.415883in}}%
\pgfpathlineto{\pgfqpoint{3.474938in}{1.435425in}}%
\pgfpathlineto{\pgfqpoint{3.622149in}{1.457591in}}%
\pgfpathlineto{\pgfqpoint{3.720289in}{1.473811in}}%
\pgfpathlineto{\pgfqpoint{3.842964in}{1.492502in}}%
\pgfpathlineto{\pgfqpoint{3.965640in}{1.519389in}}%
\pgfpathlineto{\pgfqpoint{4.088315in}{1.533154in}}%
\pgfpathlineto{\pgfqpoint{4.235526in}{1.551598in}}%
\pgfpathlineto{\pgfqpoint{4.358201in}{1.571081in}}%
\pgfpathlineto{\pgfqpoint{4.505411in}{1.593465in}}%
\pgfpathlineto{\pgfqpoint{4.652622in}{1.610660in}}%
\pgfpathlineto{\pgfqpoint{4.799832in}{1.630381in}}%
\pgfpathlineto{\pgfqpoint{4.971578in}{1.650169in}}%
\pgfpathlineto{\pgfqpoint{4.996113in}{1.616482in}}%
\pgfpathlineto{\pgfqpoint{5.045183in}{1.653098in}}%
\pgfusepath{stroke}%
\end{pgfscope}%
\begin{pgfscope}%
\pgfsetrectcap%
\pgfsetmiterjoin%
\pgfsetlinewidth{0.803000pt}%
\definecolor{currentstroke}{rgb}{0.000000,0.000000,0.000000}%
\pgfsetstrokecolor{currentstroke}%
\pgfsetdash{}{0pt}%
\pgfpathmoveto{\pgfqpoint{0.588387in}{0.521603in}}%
\pgfpathlineto{\pgfqpoint{0.588387in}{2.531888in}}%
\pgfusepath{stroke}%
\end{pgfscope}%
\begin{pgfscope}%
\pgfsetrectcap%
\pgfsetmiterjoin%
\pgfsetlinewidth{0.803000pt}%
\definecolor{currentstroke}{rgb}{0.000000,0.000000,0.000000}%
\pgfsetstrokecolor{currentstroke}%
\pgfsetdash{}{0pt}%
\pgfpathmoveto{\pgfqpoint{5.257411in}{0.521603in}}%
\pgfpathlineto{\pgfqpoint{5.257411in}{2.531888in}}%
\pgfusepath{stroke}%
\end{pgfscope}%
\begin{pgfscope}%
\pgfsetrectcap%
\pgfsetmiterjoin%
\pgfsetlinewidth{0.803000pt}%
\definecolor{currentstroke}{rgb}{0.000000,0.000000,0.000000}%
\pgfsetstrokecolor{currentstroke}%
\pgfsetdash{}{0pt}%
\pgfpathmoveto{\pgfqpoint{0.588387in}{0.521603in}}%
\pgfpathlineto{\pgfqpoint{5.257411in}{0.521603in}}%
\pgfusepath{stroke}%
\end{pgfscope}%
\begin{pgfscope}%
\pgfsetrectcap%
\pgfsetmiterjoin%
\pgfsetlinewidth{0.803000pt}%
\definecolor{currentstroke}{rgb}{0.000000,0.000000,0.000000}%
\pgfsetstrokecolor{currentstroke}%
\pgfsetdash{}{0pt}%
\pgfpathmoveto{\pgfqpoint{0.588387in}{2.531888in}}%
\pgfpathlineto{\pgfqpoint{5.257411in}{2.531888in}}%
\pgfusepath{stroke}%
\end{pgfscope}%
\begin{pgfscope}%
\definecolor{textcolor}{rgb}{0.000000,0.000000,0.000000}%
\pgfsetstrokecolor{textcolor}%
\pgfsetfillcolor{textcolor}%
\pgftext[x=2.922899in,y=2.615222in,,base]{\color{textcolor}{\rmfamily\fontsize{12.000000}{14.400000}\selectfont\catcode`\^=\active\def^{\ifmmode\sp\else\^{}\fi}\catcode`\%=\active\def%{\%}Mean}}%
\end{pgfscope}%
\begin{pgfscope}%
\pgfsetbuttcap%
\pgfsetmiterjoin%
\definecolor{currentfill}{rgb}{1.000000,1.000000,1.000000}%
\pgfsetfillcolor{currentfill}%
\pgfsetfillopacity{0.800000}%
\pgfsetlinewidth{1.003750pt}%
\definecolor{currentstroke}{rgb}{0.800000,0.800000,0.800000}%
\pgfsetstrokecolor{currentstroke}%
\pgfsetstrokeopacity{0.800000}%
\pgfsetdash{}{0pt}%
\pgfpathmoveto{\pgfqpoint{5.344911in}{0.946722in}}%
\pgfpathlineto{\pgfqpoint{8.259376in}{0.946722in}}%
\pgfpathquadraticcurveto{\pgfqpoint{8.284376in}{0.946722in}}{\pgfqpoint{8.284376in}{0.971722in}}%
\pgfpathlineto{\pgfqpoint{8.284376in}{2.444388in}}%
\pgfpathquadraticcurveto{\pgfqpoint{8.284376in}{2.469388in}}{\pgfqpoint{8.259376in}{2.469388in}}%
\pgfpathlineto{\pgfqpoint{5.344911in}{2.469388in}}%
\pgfpathquadraticcurveto{\pgfqpoint{5.319911in}{2.469388in}}{\pgfqpoint{5.319911in}{2.444388in}}%
\pgfpathlineto{\pgfqpoint{5.319911in}{0.971722in}}%
\pgfpathquadraticcurveto{\pgfqpoint{5.319911in}{0.946722in}}{\pgfqpoint{5.344911in}{0.946722in}}%
\pgfpathlineto{\pgfqpoint{5.344911in}{0.946722in}}%
\pgfpathclose%
\pgfusepath{stroke,fill}%
\end{pgfscope}%
\begin{pgfscope}%
\pgfsetrectcap%
\pgfsetroundjoin%
\pgfsetlinewidth{1.505625pt}%
\definecolor{currentstroke}{rgb}{0.498039,0.498039,0.498039}%
\pgfsetstrokecolor{currentstroke}%
\pgfsetdash{}{0pt}%
\pgfpathmoveto{\pgfqpoint{5.369911in}{2.368168in}}%
\pgfpathlineto{\pgfqpoint{5.494911in}{2.368168in}}%
\pgfpathlineto{\pgfqpoint{5.619911in}{2.368168in}}%
\pgfusepath{stroke}%
\end{pgfscope}%
\begin{pgfscope}%
\definecolor{textcolor}{rgb}{0.000000,0.000000,0.000000}%
\pgfsetstrokecolor{textcolor}%
\pgfsetfillcolor{textcolor}%
\pgftext[x=5.719911in,y=2.324418in,left,base]{\color{textcolor}{\rmfamily\fontsize{9.000000}{10.800000}\selectfont\catcode`\^=\active\def^{\ifmmode\sp\else\^{}\fi}\catcode`\%=\active\def%{\%}\NaiveCycles{}}}%
\end{pgfscope}%
\begin{pgfscope}%
\pgfsetrectcap%
\pgfsetroundjoin%
\pgfsetlinewidth{1.505625pt}%
\pgfsetstrokecolor{currentstroke1}%
\pgfsetdash{}{0pt}%
\pgfpathmoveto{\pgfqpoint{5.369911in}{2.184696in}}%
\pgfpathlineto{\pgfqpoint{5.494911in}{2.184696in}}%
\pgfpathlineto{\pgfqpoint{5.619911in}{2.184696in}}%
\pgfusepath{stroke}%
\end{pgfscope}%
\begin{pgfscope}%
\definecolor{textcolor}{rgb}{0.000000,0.000000,0.000000}%
\pgfsetstrokecolor{textcolor}%
\pgfsetfillcolor{textcolor}%
\pgftext[x=5.719911in,y=2.140946in,left,base]{\color{textcolor}{\rmfamily\fontsize{9.000000}{10.800000}\selectfont\catcode`\^=\active\def^{\ifmmode\sp\else\^{}\fi}\catcode`\%=\active\def%{\%}\CyclesMatchChunks{} \& \MergeLinear{}}}%
\end{pgfscope}%
\begin{pgfscope}%
\pgfsetrectcap%
\pgfsetroundjoin%
\pgfsetlinewidth{1.505625pt}%
\pgfsetstrokecolor{currentstroke2}%
\pgfsetdash{}{0pt}%
\pgfpathmoveto{\pgfqpoint{5.369911in}{1.997746in}}%
\pgfpathlineto{\pgfqpoint{5.494911in}{1.997746in}}%
\pgfpathlineto{\pgfqpoint{5.619911in}{1.997746in}}%
\pgfusepath{stroke}%
\end{pgfscope}%
\begin{pgfscope}%
\definecolor{textcolor}{rgb}{0.000000,0.000000,0.000000}%
\pgfsetstrokecolor{textcolor}%
\pgfsetfillcolor{textcolor}%
\pgftext[x=5.719911in,y=1.953996in,left,base]{\color{textcolor}{\rmfamily\fontsize{9.000000}{10.800000}\selectfont\catcode`\^=\active\def^{\ifmmode\sp\else\^{}\fi}\catcode`\%=\active\def%{\%}\CyclesMatchChunks{} \& \SharedVertices{}}}%
\end{pgfscope}%
\begin{pgfscope}%
\pgfsetrectcap%
\pgfsetroundjoin%
\pgfsetlinewidth{1.505625pt}%
\pgfsetstrokecolor{currentstroke3}%
\pgfsetdash{}{0pt}%
\pgfpathmoveto{\pgfqpoint{5.369911in}{1.810795in}}%
\pgfpathlineto{\pgfqpoint{5.494911in}{1.810795in}}%
\pgfpathlineto{\pgfqpoint{5.619911in}{1.810795in}}%
\pgfusepath{stroke}%
\end{pgfscope}%
\begin{pgfscope}%
\definecolor{textcolor}{rgb}{0.000000,0.000000,0.000000}%
\pgfsetstrokecolor{textcolor}%
\pgfsetfillcolor{textcolor}%
\pgftext[x=5.719911in,y=1.767045in,left,base]{\color{textcolor}{\rmfamily\fontsize{9.000000}{10.800000}\selectfont\catcode`\^=\active\def^{\ifmmode\sp\else\^{}\fi}\catcode`\%=\active\def%{\%}\Neighbors{} \& \MergeLinear{}}}%
\end{pgfscope}%
\begin{pgfscope}%
\pgfsetrectcap%
\pgfsetroundjoin%
\pgfsetlinewidth{1.505625pt}%
\pgfsetstrokecolor{currentstroke4}%
\pgfsetdash{}{0pt}%
\pgfpathmoveto{\pgfqpoint{5.369911in}{1.627324in}}%
\pgfpathlineto{\pgfqpoint{5.494911in}{1.627324in}}%
\pgfpathlineto{\pgfqpoint{5.619911in}{1.627324in}}%
\pgfusepath{stroke}%
\end{pgfscope}%
\begin{pgfscope}%
\definecolor{textcolor}{rgb}{0.000000,0.000000,0.000000}%
\pgfsetstrokecolor{textcolor}%
\pgfsetfillcolor{textcolor}%
\pgftext[x=5.719911in,y=1.583574in,left,base]{\color{textcolor}{\rmfamily\fontsize{9.000000}{10.800000}\selectfont\catcode`\^=\active\def^{\ifmmode\sp\else\^{}\fi}\catcode`\%=\active\def%{\%}\Neighbors{} \& \SharedVertices{}}}%
\end{pgfscope}%
\begin{pgfscope}%
\pgfsetrectcap%
\pgfsetroundjoin%
\pgfsetlinewidth{1.505625pt}%
\pgfsetstrokecolor{currentstroke5}%
\pgfsetdash{}{0pt}%
\pgfpathmoveto{\pgfqpoint{5.369911in}{1.440373in}}%
\pgfpathlineto{\pgfqpoint{5.494911in}{1.440373in}}%
\pgfpathlineto{\pgfqpoint{5.619911in}{1.440373in}}%
\pgfusepath{stroke}%
\end{pgfscope}%
\begin{pgfscope}%
\definecolor{textcolor}{rgb}{0.000000,0.000000,0.000000}%
\pgfsetstrokecolor{textcolor}%
\pgfsetfillcolor{textcolor}%
\pgftext[x=5.719911in,y=1.396623in,left,base]{\color{textcolor}{\rmfamily\fontsize{9.000000}{10.800000}\selectfont\catcode`\^=\active\def^{\ifmmode\sp\else\^{}\fi}\catcode`\%=\active\def%{\%}\NeighborsDegree{} \& \MergeLinear{}}}%
\end{pgfscope}%
\begin{pgfscope}%
\pgfsetrectcap%
\pgfsetroundjoin%
\pgfsetlinewidth{1.505625pt}%
\pgfsetstrokecolor{currentstroke6}%
\pgfsetdash{}{0pt}%
\pgfpathmoveto{\pgfqpoint{5.369911in}{1.253423in}}%
\pgfpathlineto{\pgfqpoint{5.494911in}{1.253423in}}%
\pgfpathlineto{\pgfqpoint{5.619911in}{1.253423in}}%
\pgfusepath{stroke}%
\end{pgfscope}%
\begin{pgfscope}%
\definecolor{textcolor}{rgb}{0.000000,0.000000,0.000000}%
\pgfsetstrokecolor{textcolor}%
\pgfsetfillcolor{textcolor}%
\pgftext[x=5.719911in,y=1.209673in,left,base]{\color{textcolor}{\rmfamily\fontsize{9.000000}{10.800000}\selectfont\catcode`\^=\active\def^{\ifmmode\sp\else\^{}\fi}\catcode`\%=\active\def%{\%}\None{} \& \MergeLinear{}}}%
\end{pgfscope}%
\begin{pgfscope}%
\pgfsetrectcap%
\pgfsetroundjoin%
\pgfsetlinewidth{1.505625pt}%
\pgfsetstrokecolor{currentstroke7}%
\pgfsetdash{}{0pt}%
\pgfpathmoveto{\pgfqpoint{5.369911in}{1.069951in}}%
\pgfpathlineto{\pgfqpoint{5.494911in}{1.069951in}}%
\pgfpathlineto{\pgfqpoint{5.619911in}{1.069951in}}%
\pgfusepath{stroke}%
\end{pgfscope}%
\begin{pgfscope}%
\definecolor{textcolor}{rgb}{0.000000,0.000000,0.000000}%
\pgfsetstrokecolor{textcolor}%
\pgfsetfillcolor{textcolor}%
\pgftext[x=5.719911in,y=1.026201in,left,base]{\color{textcolor}{\rmfamily\fontsize{9.000000}{10.800000}\selectfont\catcode`\^=\active\def^{\ifmmode\sp\else\^{}\fi}\catcode`\%=\active\def%{\%}\None{} \& \SharedVertices{}}}%
\end{pgfscope}%
\end{pgfpicture}%
\makeatother%
\endgroup%
}
	\caption[Mean runtime for graphs with no 3 nor 4 cycles (some)]{
		Mean running time to find all NAC-colorings for graphs with no three nor four cycles.}%
	\label{fig:graph_count_no_3_nor_4_cycles_first_runtime}
\end{figure}%
% \begin{figure}[thbp]
% 	\centering
% 	\scalebox{\BenchFigureScale}{\input{./figures/graph_export_no_3_nor_4_cycles_first_monochromatic_checks_split_merging_mean.pgf}}
% 	\caption[Checks performed for graphs with no 3 nor 4 cycles (some)]{
% 		The number of checks performed to find all NAC-colorings for graphs with no three nor four cycles.}%
% 	\label{fig:graph_count_no_3_nor_4_cycles_first_checks}
% \end{figure}%

For listing all NAC-colorings shown
in \Cref{fig:graph_count_no_3_nor_4_cycles_all_runtime}
shows that \NaiveCycles{} is almost never faster.
We can see that \None{} and \CyclesMatchChunks{} is also slower than
\Neighbors{} and \NeighborsDegree{}.
At around twenty-eight vertices five second time limit is reached for all algorithms.
\MergeLinear{} and \SharedVertices{} have no significant influence.
%
\begin{figure}[thbp]
	\centering
	\scalebox{\BenchFigureScale}{%% Creator: Matplotlib, PGF backend
%%
%% To include the figure in your LaTeX document, write
%%   \input{<filename>.pgf}
%%
%% Make sure the required packages are loaded in your preamble
%%   \usepackage{pgf}
%%
%% Also ensure that all the required font packages are loaded; for instance,
%% the lmodern package is sometimes necessary when using math font.
%%   \usepackage{lmodern}
%%
%% Figures using additional raster images can only be included by \input if
%% they are in the same directory as the main LaTeX file. For loading figures
%% from other directories you can use the `import` package
%%   \usepackage{import}
%%
%% and then include the figures with
%%   \import{<path to file>}{<filename>.pgf}
%%
%% Matplotlib used the following preamble
%%   \def\mathdefault#1{#1}
%%   \everymath=\expandafter{\the\everymath\displaystyle}
%%   \IfFileExists{scrextend.sty}{
%%     \usepackage[fontsize=10.000000pt]{scrextend}
%%   }{
%%     \renewcommand{\normalsize}{\fontsize{10.000000}{12.000000}\selectfont}
%%     \normalsize
%%   }
%%   
%%   \ifdefined\pdftexversion\else  % non-pdftex case.
%%     \usepackage{fontspec}
%%     \setmainfont{DejaVuSans.ttf}[Path=\detokenize{/home/petr/Projects/PyRigi/.venv/lib/python3.12/site-packages/matplotlib/mpl-data/fonts/ttf/}]
%%     \setsansfont{DejaVuSans.ttf}[Path=\detokenize{/home/petr/Projects/PyRigi/.venv/lib/python3.12/site-packages/matplotlib/mpl-data/fonts/ttf/}]
%%     \setmonofont{DejaVuSansMono.ttf}[Path=\detokenize{/home/petr/Projects/PyRigi/.venv/lib/python3.12/site-packages/matplotlib/mpl-data/fonts/ttf/}]
%%   \fi
%%   \makeatletter\@ifpackageloaded{under\Score{}}{}{\usepackage[strings]{under\Score{}}}\makeatother
%%
\begingroup%
\makeatletter%
\begin{pgfpicture}%
\pgfpathrectangle{\pgfpointorigin}{\pgfqpoint{8.384376in}{2.841860in}}%
\pgfusepath{use as bounding box, clip}%
\begin{pgfscope}%
\pgfsetbuttcap%
\pgfsetmiterjoin%
\definecolor{currentfill}{rgb}{1.000000,1.000000,1.000000}%
\pgfsetfillcolor{currentfill}%
\pgfsetlinewidth{0.000000pt}%
\definecolor{currentstroke}{rgb}{1.000000,1.000000,1.000000}%
\pgfsetstrokecolor{currentstroke}%
\pgfsetdash{}{0pt}%
\pgfpathmoveto{\pgfqpoint{0.000000in}{0.000000in}}%
\pgfpathlineto{\pgfqpoint{8.384376in}{0.000000in}}%
\pgfpathlineto{\pgfqpoint{8.384376in}{2.841860in}}%
\pgfpathlineto{\pgfqpoint{0.000000in}{2.841860in}}%
\pgfpathlineto{\pgfqpoint{0.000000in}{0.000000in}}%
\pgfpathclose%
\pgfusepath{fill}%
\end{pgfscope}%
\begin{pgfscope}%
\pgfsetbuttcap%
\pgfsetmiterjoin%
\definecolor{currentfill}{rgb}{1.000000,1.000000,1.000000}%
\pgfsetfillcolor{currentfill}%
\pgfsetlinewidth{0.000000pt}%
\definecolor{currentstroke}{rgb}{0.000000,0.000000,0.000000}%
\pgfsetstrokecolor{currentstroke}%
\pgfsetstrokeopacity{0.000000}%
\pgfsetdash{}{0pt}%
\pgfpathmoveto{\pgfqpoint{0.588387in}{0.521603in}}%
\pgfpathlineto{\pgfqpoint{4.248423in}{0.521603in}}%
\pgfpathlineto{\pgfqpoint{4.248423in}{2.741376in}}%
\pgfpathlineto{\pgfqpoint{0.588387in}{2.741376in}}%
\pgfpathlineto{\pgfqpoint{0.588387in}{0.521603in}}%
\pgfpathclose%
\pgfusepath{fill}%
\end{pgfscope}%
\begin{pgfscope}%
\pgfsetbuttcap%
\pgfsetroundjoin%
\definecolor{currentfill}{rgb}{0.000000,0.000000,0.000000}%
\pgfsetfillcolor{currentfill}%
\pgfsetlinewidth{0.803000pt}%
\definecolor{currentstroke}{rgb}{0.000000,0.000000,0.000000}%
\pgfsetstrokecolor{currentstroke}%
\pgfsetdash{}{0pt}%
\pgfsys@defobject{currentmarker}{\pgfqpoint{0.000000in}{-0.048611in}}{\pgfqpoint{0.000000in}{0.000000in}}{%
\pgfpathmoveto{\pgfqpoint{0.000000in}{0.000000in}}%
\pgfpathlineto{\pgfqpoint{0.000000in}{-0.048611in}}%
\pgfusepath{stroke,fill}%
}%
\begin{pgfscope}%
\pgfsys@transformshift{0.882726in}{0.521603in}%
\pgfsys@useobject{currentmarker}{}%
\end{pgfscope}%
\end{pgfscope}%
\begin{pgfscope}%
\definecolor{textcolor}{rgb}{0.000000,0.000000,0.000000}%
\pgfsetstrokecolor{textcolor}%
\pgfsetfillcolor{textcolor}%
\pgftext[x=0.882726in,y=0.424381in,,top]{\color{textcolor}{\rmfamily\fontsize{10.000000}{12.000000}\selectfont\catcode`\^=\active\def^{\ifmmode\sp\else\^{}\fi}\catcode`\%=\active\def%{\%}$\mathdefault{9}$}}%
\end{pgfscope}%
\begin{pgfscope}%
\pgfsetbuttcap%
\pgfsetroundjoin%
\definecolor{currentfill}{rgb}{0.000000,0.000000,0.000000}%
\pgfsetfillcolor{currentfill}%
\pgfsetlinewidth{0.803000pt}%
\definecolor{currentstroke}{rgb}{0.000000,0.000000,0.000000}%
\pgfsetstrokecolor{currentstroke}%
\pgfsetdash{}{0pt}%
\pgfsys@defobject{currentmarker}{\pgfqpoint{0.000000in}{-0.048611in}}{\pgfqpoint{0.000000in}{0.000000in}}{%
\pgfpathmoveto{\pgfqpoint{0.000000in}{0.000000in}}%
\pgfpathlineto{\pgfqpoint{0.000000in}{-0.048611in}}%
\pgfusepath{stroke,fill}%
}%
\begin{pgfscope}%
\pgfsys@transformshift{1.266646in}{0.521603in}%
\pgfsys@useobject{currentmarker}{}%
\end{pgfscope}%
\end{pgfscope}%
\begin{pgfscope}%
\definecolor{textcolor}{rgb}{0.000000,0.000000,0.000000}%
\pgfsetstrokecolor{textcolor}%
\pgfsetfillcolor{textcolor}%
\pgftext[x=1.266646in,y=0.424381in,,top]{\color{textcolor}{\rmfamily\fontsize{10.000000}{12.000000}\selectfont\catcode`\^=\active\def^{\ifmmode\sp\else\^{}\fi}\catcode`\%=\active\def%{\%}$\mathdefault{12}$}}%
\end{pgfscope}%
\begin{pgfscope}%
\pgfsetbuttcap%
\pgfsetroundjoin%
\definecolor{currentfill}{rgb}{0.000000,0.000000,0.000000}%
\pgfsetfillcolor{currentfill}%
\pgfsetlinewidth{0.803000pt}%
\definecolor{currentstroke}{rgb}{0.000000,0.000000,0.000000}%
\pgfsetstrokecolor{currentstroke}%
\pgfsetdash{}{0pt}%
\pgfsys@defobject{currentmarker}{\pgfqpoint{0.000000in}{-0.048611in}}{\pgfqpoint{0.000000in}{0.000000in}}{%
\pgfpathmoveto{\pgfqpoint{0.000000in}{0.000000in}}%
\pgfpathlineto{\pgfqpoint{0.000000in}{-0.048611in}}%
\pgfusepath{stroke,fill}%
}%
\begin{pgfscope}%
\pgfsys@transformshift{1.650565in}{0.521603in}%
\pgfsys@useobject{currentmarker}{}%
\end{pgfscope}%
\end{pgfscope}%
\begin{pgfscope}%
\definecolor{textcolor}{rgb}{0.000000,0.000000,0.000000}%
\pgfsetstrokecolor{textcolor}%
\pgfsetfillcolor{textcolor}%
\pgftext[x=1.650565in,y=0.424381in,,top]{\color{textcolor}{\rmfamily\fontsize{10.000000}{12.000000}\selectfont\catcode`\^=\active\def^{\ifmmode\sp\else\^{}\fi}\catcode`\%=\active\def%{\%}$\mathdefault{15}$}}%
\end{pgfscope}%
\begin{pgfscope}%
\pgfsetbuttcap%
\pgfsetroundjoin%
\definecolor{currentfill}{rgb}{0.000000,0.000000,0.000000}%
\pgfsetfillcolor{currentfill}%
\pgfsetlinewidth{0.803000pt}%
\definecolor{currentstroke}{rgb}{0.000000,0.000000,0.000000}%
\pgfsetstrokecolor{currentstroke}%
\pgfsetdash{}{0pt}%
\pgfsys@defobject{currentmarker}{\pgfqpoint{0.000000in}{-0.048611in}}{\pgfqpoint{0.000000in}{0.000000in}}{%
\pgfpathmoveto{\pgfqpoint{0.000000in}{0.000000in}}%
\pgfpathlineto{\pgfqpoint{0.000000in}{-0.048611in}}%
\pgfusepath{stroke,fill}%
}%
\begin{pgfscope}%
\pgfsys@transformshift{2.034485in}{0.521603in}%
\pgfsys@useobject{currentmarker}{}%
\end{pgfscope}%
\end{pgfscope}%
\begin{pgfscope}%
\definecolor{textcolor}{rgb}{0.000000,0.000000,0.000000}%
\pgfsetstrokecolor{textcolor}%
\pgfsetfillcolor{textcolor}%
\pgftext[x=2.034485in,y=0.424381in,,top]{\color{textcolor}{\rmfamily\fontsize{10.000000}{12.000000}\selectfont\catcode`\^=\active\def^{\ifmmode\sp\else\^{}\fi}\catcode`\%=\active\def%{\%}$\mathdefault{18}$}}%
\end{pgfscope}%
\begin{pgfscope}%
\pgfsetbuttcap%
\pgfsetroundjoin%
\definecolor{currentfill}{rgb}{0.000000,0.000000,0.000000}%
\pgfsetfillcolor{currentfill}%
\pgfsetlinewidth{0.803000pt}%
\definecolor{currentstroke}{rgb}{0.000000,0.000000,0.000000}%
\pgfsetstrokecolor{currentstroke}%
\pgfsetdash{}{0pt}%
\pgfsys@defobject{currentmarker}{\pgfqpoint{0.000000in}{-0.048611in}}{\pgfqpoint{0.000000in}{0.000000in}}{%
\pgfpathmoveto{\pgfqpoint{0.000000in}{0.000000in}}%
\pgfpathlineto{\pgfqpoint{0.000000in}{-0.048611in}}%
\pgfusepath{stroke,fill}%
}%
\begin{pgfscope}%
\pgfsys@transformshift{2.418405in}{0.521603in}%
\pgfsys@useobject{currentmarker}{}%
\end{pgfscope}%
\end{pgfscope}%
\begin{pgfscope}%
\definecolor{textcolor}{rgb}{0.000000,0.000000,0.000000}%
\pgfsetstrokecolor{textcolor}%
\pgfsetfillcolor{textcolor}%
\pgftext[x=2.418405in,y=0.424381in,,top]{\color{textcolor}{\rmfamily\fontsize{10.000000}{12.000000}\selectfont\catcode`\^=\active\def^{\ifmmode\sp\else\^{}\fi}\catcode`\%=\active\def%{\%}$\mathdefault{21}$}}%
\end{pgfscope}%
\begin{pgfscope}%
\pgfsetbuttcap%
\pgfsetroundjoin%
\definecolor{currentfill}{rgb}{0.000000,0.000000,0.000000}%
\pgfsetfillcolor{currentfill}%
\pgfsetlinewidth{0.803000pt}%
\definecolor{currentstroke}{rgb}{0.000000,0.000000,0.000000}%
\pgfsetstrokecolor{currentstroke}%
\pgfsetdash{}{0pt}%
\pgfsys@defobject{currentmarker}{\pgfqpoint{0.000000in}{-0.048611in}}{\pgfqpoint{0.000000in}{0.000000in}}{%
\pgfpathmoveto{\pgfqpoint{0.000000in}{0.000000in}}%
\pgfpathlineto{\pgfqpoint{0.000000in}{-0.048611in}}%
\pgfusepath{stroke,fill}%
}%
\begin{pgfscope}%
\pgfsys@transformshift{2.802325in}{0.521603in}%
\pgfsys@useobject{currentmarker}{}%
\end{pgfscope}%
\end{pgfscope}%
\begin{pgfscope}%
\definecolor{textcolor}{rgb}{0.000000,0.000000,0.000000}%
\pgfsetstrokecolor{textcolor}%
\pgfsetfillcolor{textcolor}%
\pgftext[x=2.802325in,y=0.424381in,,top]{\color{textcolor}{\rmfamily\fontsize{10.000000}{12.000000}\selectfont\catcode`\^=\active\def^{\ifmmode\sp\else\^{}\fi}\catcode`\%=\active\def%{\%}$\mathdefault{24}$}}%
\end{pgfscope}%
\begin{pgfscope}%
\pgfsetbuttcap%
\pgfsetroundjoin%
\definecolor{currentfill}{rgb}{0.000000,0.000000,0.000000}%
\pgfsetfillcolor{currentfill}%
\pgfsetlinewidth{0.803000pt}%
\definecolor{currentstroke}{rgb}{0.000000,0.000000,0.000000}%
\pgfsetstrokecolor{currentstroke}%
\pgfsetdash{}{0pt}%
\pgfsys@defobject{currentmarker}{\pgfqpoint{0.000000in}{-0.048611in}}{\pgfqpoint{0.000000in}{0.000000in}}{%
\pgfpathmoveto{\pgfqpoint{0.000000in}{0.000000in}}%
\pgfpathlineto{\pgfqpoint{0.000000in}{-0.048611in}}%
\pgfusepath{stroke,fill}%
}%
\begin{pgfscope}%
\pgfsys@transformshift{3.186245in}{0.521603in}%
\pgfsys@useobject{currentmarker}{}%
\end{pgfscope}%
\end{pgfscope}%
\begin{pgfscope}%
\definecolor{textcolor}{rgb}{0.000000,0.000000,0.000000}%
\pgfsetstrokecolor{textcolor}%
\pgfsetfillcolor{textcolor}%
\pgftext[x=3.186245in,y=0.424381in,,top]{\color{textcolor}{\rmfamily\fontsize{10.000000}{12.000000}\selectfont\catcode`\^=\active\def^{\ifmmode\sp\else\^{}\fi}\catcode`\%=\active\def%{\%}$\mathdefault{27}$}}%
\end{pgfscope}%
\begin{pgfscope}%
\pgfsetbuttcap%
\pgfsetroundjoin%
\definecolor{currentfill}{rgb}{0.000000,0.000000,0.000000}%
\pgfsetfillcolor{currentfill}%
\pgfsetlinewidth{0.803000pt}%
\definecolor{currentstroke}{rgb}{0.000000,0.000000,0.000000}%
\pgfsetstrokecolor{currentstroke}%
\pgfsetdash{}{0pt}%
\pgfsys@defobject{currentmarker}{\pgfqpoint{0.000000in}{-0.048611in}}{\pgfqpoint{0.000000in}{0.000000in}}{%
\pgfpathmoveto{\pgfqpoint{0.000000in}{0.000000in}}%
\pgfpathlineto{\pgfqpoint{0.000000in}{-0.048611in}}%
\pgfusepath{stroke,fill}%
}%
\begin{pgfscope}%
\pgfsys@transformshift{3.570164in}{0.521603in}%
\pgfsys@useobject{currentmarker}{}%
\end{pgfscope}%
\end{pgfscope}%
\begin{pgfscope}%
\definecolor{textcolor}{rgb}{0.000000,0.000000,0.000000}%
\pgfsetstrokecolor{textcolor}%
\pgfsetfillcolor{textcolor}%
\pgftext[x=3.570164in,y=0.424381in,,top]{\color{textcolor}{\rmfamily\fontsize{10.000000}{12.000000}\selectfont\catcode`\^=\active\def^{\ifmmode\sp\else\^{}\fi}\catcode`\%=\active\def%{\%}$\mathdefault{30}$}}%
\end{pgfscope}%
\begin{pgfscope}%
\pgfsetbuttcap%
\pgfsetroundjoin%
\definecolor{currentfill}{rgb}{0.000000,0.000000,0.000000}%
\pgfsetfillcolor{currentfill}%
\pgfsetlinewidth{0.803000pt}%
\definecolor{currentstroke}{rgb}{0.000000,0.000000,0.000000}%
\pgfsetstrokecolor{currentstroke}%
\pgfsetdash{}{0pt}%
\pgfsys@defobject{currentmarker}{\pgfqpoint{0.000000in}{-0.048611in}}{\pgfqpoint{0.000000in}{0.000000in}}{%
\pgfpathmoveto{\pgfqpoint{0.000000in}{0.000000in}}%
\pgfpathlineto{\pgfqpoint{0.000000in}{-0.048611in}}%
\pgfusepath{stroke,fill}%
}%
\begin{pgfscope}%
\pgfsys@transformshift{3.954084in}{0.521603in}%
\pgfsys@useobject{currentmarker}{}%
\end{pgfscope}%
\end{pgfscope}%
\begin{pgfscope}%
\definecolor{textcolor}{rgb}{0.000000,0.000000,0.000000}%
\pgfsetstrokecolor{textcolor}%
\pgfsetfillcolor{textcolor}%
\pgftext[x=3.954084in,y=0.424381in,,top]{\color{textcolor}{\rmfamily\fontsize{10.000000}{12.000000}\selectfont\catcode`\^=\active\def^{\ifmmode\sp\else\^{}\fi}\catcode`\%=\active\def%{\%}$\mathdefault{33}$}}%
\end{pgfscope}%
\begin{pgfscope}%
\definecolor{textcolor}{rgb}{0.000000,0.000000,0.000000}%
\pgfsetstrokecolor{textcolor}%
\pgfsetfillcolor{textcolor}%
\pgftext[x=2.418405in,y=0.234413in,,top]{\color{textcolor}{\rmfamily\fontsize{10.000000}{12.000000}\selectfont\catcode`\^=\active\def^{\ifmmode\sp\else\^{}\fi}\catcode`\%=\active\def%{\%}Monochromatic classes}}%
\end{pgfscope}%
\begin{pgfscope}%
\pgfsetbuttcap%
\pgfsetroundjoin%
\definecolor{currentfill}{rgb}{0.000000,0.000000,0.000000}%
\pgfsetfillcolor{currentfill}%
\pgfsetlinewidth{0.803000pt}%
\definecolor{currentstroke}{rgb}{0.000000,0.000000,0.000000}%
\pgfsetstrokecolor{currentstroke}%
\pgfsetdash{}{0pt}%
\pgfsys@defobject{currentmarker}{\pgfqpoint{-0.048611in}{0.000000in}}{\pgfqpoint{-0.000000in}{0.000000in}}{%
\pgfpathmoveto{\pgfqpoint{-0.000000in}{0.000000in}}%
\pgfpathlineto{\pgfqpoint{-0.048611in}{0.000000in}}%
\pgfusepath{stroke,fill}%
}%
\begin{pgfscope}%
\pgfsys@transformshift{0.588387in}{0.810961in}%
\pgfsys@useobject{currentmarker}{}%
\end{pgfscope}%
\end{pgfscope}%
\begin{pgfscope}%
\definecolor{textcolor}{rgb}{0.000000,0.000000,0.000000}%
\pgfsetstrokecolor{textcolor}%
\pgfsetfillcolor{textcolor}%
\pgftext[x=0.289968in, y=0.758199in, left, base]{\color{textcolor}{\rmfamily\fontsize{10.000000}{12.000000}\selectfont\catcode`\^=\active\def^{\ifmmode\sp\else\^{}\fi}\catcode`\%=\active\def%{\%}$\mathdefault{10^{1}}$}}%
\end{pgfscope}%
\begin{pgfscope}%
\pgfsetbuttcap%
\pgfsetroundjoin%
\definecolor{currentfill}{rgb}{0.000000,0.000000,0.000000}%
\pgfsetfillcolor{currentfill}%
\pgfsetlinewidth{0.803000pt}%
\definecolor{currentstroke}{rgb}{0.000000,0.000000,0.000000}%
\pgfsetstrokecolor{currentstroke}%
\pgfsetdash{}{0pt}%
\pgfsys@defobject{currentmarker}{\pgfqpoint{-0.048611in}{0.000000in}}{\pgfqpoint{-0.000000in}{0.000000in}}{%
\pgfpathmoveto{\pgfqpoint{-0.000000in}{0.000000in}}%
\pgfpathlineto{\pgfqpoint{-0.048611in}{0.000000in}}%
\pgfusepath{stroke,fill}%
}%
\begin{pgfscope}%
\pgfsys@transformshift{0.588387in}{1.437007in}%
\pgfsys@useobject{currentmarker}{}%
\end{pgfscope}%
\end{pgfscope}%
\begin{pgfscope}%
\definecolor{textcolor}{rgb}{0.000000,0.000000,0.000000}%
\pgfsetstrokecolor{textcolor}%
\pgfsetfillcolor{textcolor}%
\pgftext[x=0.289968in, y=1.384245in, left, base]{\color{textcolor}{\rmfamily\fontsize{10.000000}{12.000000}\selectfont\catcode`\^=\active\def^{\ifmmode\sp\else\^{}\fi}\catcode`\%=\active\def%{\%}$\mathdefault{10^{2}}$}}%
\end{pgfscope}%
\begin{pgfscope}%
\pgfsetbuttcap%
\pgfsetroundjoin%
\definecolor{currentfill}{rgb}{0.000000,0.000000,0.000000}%
\pgfsetfillcolor{currentfill}%
\pgfsetlinewidth{0.803000pt}%
\definecolor{currentstroke}{rgb}{0.000000,0.000000,0.000000}%
\pgfsetstrokecolor{currentstroke}%
\pgfsetdash{}{0pt}%
\pgfsys@defobject{currentmarker}{\pgfqpoint{-0.048611in}{0.000000in}}{\pgfqpoint{-0.000000in}{0.000000in}}{%
\pgfpathmoveto{\pgfqpoint{-0.000000in}{0.000000in}}%
\pgfpathlineto{\pgfqpoint{-0.048611in}{0.000000in}}%
\pgfusepath{stroke,fill}%
}%
\begin{pgfscope}%
\pgfsys@transformshift{0.588387in}{2.063053in}%
\pgfsys@useobject{currentmarker}{}%
\end{pgfscope}%
\end{pgfscope}%
\begin{pgfscope}%
\definecolor{textcolor}{rgb}{0.000000,0.000000,0.000000}%
\pgfsetstrokecolor{textcolor}%
\pgfsetfillcolor{textcolor}%
\pgftext[x=0.289968in, y=2.010291in, left, base]{\color{textcolor}{\rmfamily\fontsize{10.000000}{12.000000}\selectfont\catcode`\^=\active\def^{\ifmmode\sp\else\^{}\fi}\catcode`\%=\active\def%{\%}$\mathdefault{10^{3}}$}}%
\end{pgfscope}%
\begin{pgfscope}%
\pgfsetbuttcap%
\pgfsetroundjoin%
\definecolor{currentfill}{rgb}{0.000000,0.000000,0.000000}%
\pgfsetfillcolor{currentfill}%
\pgfsetlinewidth{0.803000pt}%
\definecolor{currentstroke}{rgb}{0.000000,0.000000,0.000000}%
\pgfsetstrokecolor{currentstroke}%
\pgfsetdash{}{0pt}%
\pgfsys@defobject{currentmarker}{\pgfqpoint{-0.048611in}{0.000000in}}{\pgfqpoint{-0.000000in}{0.000000in}}{%
\pgfpathmoveto{\pgfqpoint{-0.000000in}{0.000000in}}%
\pgfpathlineto{\pgfqpoint{-0.048611in}{0.000000in}}%
\pgfusepath{stroke,fill}%
}%
\begin{pgfscope}%
\pgfsys@transformshift{0.588387in}{2.689099in}%
\pgfsys@useobject{currentmarker}{}%
\end{pgfscope}%
\end{pgfscope}%
\begin{pgfscope}%
\definecolor{textcolor}{rgb}{0.000000,0.000000,0.000000}%
\pgfsetstrokecolor{textcolor}%
\pgfsetfillcolor{textcolor}%
\pgftext[x=0.289968in, y=2.636337in, left, base]{\color{textcolor}{\rmfamily\fontsize{10.000000}{12.000000}\selectfont\catcode`\^=\active\def^{\ifmmode\sp\else\^{}\fi}\catcode`\%=\active\def%{\%}$\mathdefault{10^{4}}$}}%
\end{pgfscope}%
\begin{pgfscope}%
\pgfsetbuttcap%
\pgfsetroundjoin%
\definecolor{currentfill}{rgb}{0.000000,0.000000,0.000000}%
\pgfsetfillcolor{currentfill}%
\pgfsetlinewidth{0.602250pt}%
\definecolor{currentstroke}{rgb}{0.000000,0.000000,0.000000}%
\pgfsetstrokecolor{currentstroke}%
\pgfsetdash{}{0pt}%
\pgfsys@defobject{currentmarker}{\pgfqpoint{-0.027778in}{0.000000in}}{\pgfqpoint{-0.000000in}{0.000000in}}{%
\pgfpathmoveto{\pgfqpoint{-0.000000in}{0.000000in}}%
\pgfpathlineto{\pgfqpoint{-0.027778in}{0.000000in}}%
\pgfusepath{stroke,fill}%
}%
\begin{pgfscope}%
\pgfsys@transformshift{0.588387in}{0.561832in}%
\pgfsys@useobject{currentmarker}{}%
\end{pgfscope}%
\end{pgfscope}%
\begin{pgfscope}%
\pgfsetbuttcap%
\pgfsetroundjoin%
\definecolor{currentfill}{rgb}{0.000000,0.000000,0.000000}%
\pgfsetfillcolor{currentfill}%
\pgfsetlinewidth{0.602250pt}%
\definecolor{currentstroke}{rgb}{0.000000,0.000000,0.000000}%
\pgfsetstrokecolor{currentstroke}%
\pgfsetdash{}{0pt}%
\pgfsys@defobject{currentmarker}{\pgfqpoint{-0.027778in}{0.000000in}}{\pgfqpoint{-0.000000in}{0.000000in}}{%
\pgfpathmoveto{\pgfqpoint{-0.000000in}{0.000000in}}%
\pgfpathlineto{\pgfqpoint{-0.027778in}{0.000000in}}%
\pgfusepath{stroke,fill}%
}%
\begin{pgfscope}%
\pgfsys@transformshift{0.588387in}{0.622502in}%
\pgfsys@useobject{currentmarker}{}%
\end{pgfscope}%
\end{pgfscope}%
\begin{pgfscope}%
\pgfsetbuttcap%
\pgfsetroundjoin%
\definecolor{currentfill}{rgb}{0.000000,0.000000,0.000000}%
\pgfsetfillcolor{currentfill}%
\pgfsetlinewidth{0.602250pt}%
\definecolor{currentstroke}{rgb}{0.000000,0.000000,0.000000}%
\pgfsetstrokecolor{currentstroke}%
\pgfsetdash{}{0pt}%
\pgfsys@defobject{currentmarker}{\pgfqpoint{-0.027778in}{0.000000in}}{\pgfqpoint{-0.000000in}{0.000000in}}{%
\pgfpathmoveto{\pgfqpoint{-0.000000in}{0.000000in}}%
\pgfpathlineto{\pgfqpoint{-0.027778in}{0.000000in}}%
\pgfusepath{stroke,fill}%
}%
\begin{pgfscope}%
\pgfsys@transformshift{0.588387in}{0.672073in}%
\pgfsys@useobject{currentmarker}{}%
\end{pgfscope}%
\end{pgfscope}%
\begin{pgfscope}%
\pgfsetbuttcap%
\pgfsetroundjoin%
\definecolor{currentfill}{rgb}{0.000000,0.000000,0.000000}%
\pgfsetfillcolor{currentfill}%
\pgfsetlinewidth{0.602250pt}%
\definecolor{currentstroke}{rgb}{0.000000,0.000000,0.000000}%
\pgfsetstrokecolor{currentstroke}%
\pgfsetdash{}{0pt}%
\pgfsys@defobject{currentmarker}{\pgfqpoint{-0.027778in}{0.000000in}}{\pgfqpoint{-0.000000in}{0.000000in}}{%
\pgfpathmoveto{\pgfqpoint{-0.000000in}{0.000000in}}%
\pgfpathlineto{\pgfqpoint{-0.027778in}{0.000000in}}%
\pgfusepath{stroke,fill}%
}%
\begin{pgfscope}%
\pgfsys@transformshift{0.588387in}{0.713985in}%
\pgfsys@useobject{currentmarker}{}%
\end{pgfscope}%
\end{pgfscope}%
\begin{pgfscope}%
\pgfsetbuttcap%
\pgfsetroundjoin%
\definecolor{currentfill}{rgb}{0.000000,0.000000,0.000000}%
\pgfsetfillcolor{currentfill}%
\pgfsetlinewidth{0.602250pt}%
\definecolor{currentstroke}{rgb}{0.000000,0.000000,0.000000}%
\pgfsetstrokecolor{currentstroke}%
\pgfsetdash{}{0pt}%
\pgfsys@defobject{currentmarker}{\pgfqpoint{-0.027778in}{0.000000in}}{\pgfqpoint{-0.000000in}{0.000000in}}{%
\pgfpathmoveto{\pgfqpoint{-0.000000in}{0.000000in}}%
\pgfpathlineto{\pgfqpoint{-0.027778in}{0.000000in}}%
\pgfusepath{stroke,fill}%
}%
\begin{pgfscope}%
\pgfsys@transformshift{0.588387in}{0.750291in}%
\pgfsys@useobject{currentmarker}{}%
\end{pgfscope}%
\end{pgfscope}%
\begin{pgfscope}%
\pgfsetbuttcap%
\pgfsetroundjoin%
\definecolor{currentfill}{rgb}{0.000000,0.000000,0.000000}%
\pgfsetfillcolor{currentfill}%
\pgfsetlinewidth{0.602250pt}%
\definecolor{currentstroke}{rgb}{0.000000,0.000000,0.000000}%
\pgfsetstrokecolor{currentstroke}%
\pgfsetdash{}{0pt}%
\pgfsys@defobject{currentmarker}{\pgfqpoint{-0.027778in}{0.000000in}}{\pgfqpoint{-0.000000in}{0.000000in}}{%
\pgfpathmoveto{\pgfqpoint{-0.000000in}{0.000000in}}%
\pgfpathlineto{\pgfqpoint{-0.027778in}{0.000000in}}%
\pgfusepath{stroke,fill}%
}%
\begin{pgfscope}%
\pgfsys@transformshift{0.588387in}{0.782314in}%
\pgfsys@useobject{currentmarker}{}%
\end{pgfscope}%
\end{pgfscope}%
\begin{pgfscope}%
\pgfsetbuttcap%
\pgfsetroundjoin%
\definecolor{currentfill}{rgb}{0.000000,0.000000,0.000000}%
\pgfsetfillcolor{currentfill}%
\pgfsetlinewidth{0.602250pt}%
\definecolor{currentstroke}{rgb}{0.000000,0.000000,0.000000}%
\pgfsetstrokecolor{currentstroke}%
\pgfsetdash{}{0pt}%
\pgfsys@defobject{currentmarker}{\pgfqpoint{-0.027778in}{0.000000in}}{\pgfqpoint{-0.000000in}{0.000000in}}{%
\pgfpathmoveto{\pgfqpoint{-0.000000in}{0.000000in}}%
\pgfpathlineto{\pgfqpoint{-0.027778in}{0.000000in}}%
\pgfusepath{stroke,fill}%
}%
\begin{pgfscope}%
\pgfsys@transformshift{0.588387in}{0.999419in}%
\pgfsys@useobject{currentmarker}{}%
\end{pgfscope}%
\end{pgfscope}%
\begin{pgfscope}%
\pgfsetbuttcap%
\pgfsetroundjoin%
\definecolor{currentfill}{rgb}{0.000000,0.000000,0.000000}%
\pgfsetfillcolor{currentfill}%
\pgfsetlinewidth{0.602250pt}%
\definecolor{currentstroke}{rgb}{0.000000,0.000000,0.000000}%
\pgfsetstrokecolor{currentstroke}%
\pgfsetdash{}{0pt}%
\pgfsys@defobject{currentmarker}{\pgfqpoint{-0.027778in}{0.000000in}}{\pgfqpoint{-0.000000in}{0.000000in}}{%
\pgfpathmoveto{\pgfqpoint{-0.000000in}{0.000000in}}%
\pgfpathlineto{\pgfqpoint{-0.027778in}{0.000000in}}%
\pgfusepath{stroke,fill}%
}%
\begin{pgfscope}%
\pgfsys@transformshift{0.588387in}{1.109661in}%
\pgfsys@useobject{currentmarker}{}%
\end{pgfscope}%
\end{pgfscope}%
\begin{pgfscope}%
\pgfsetbuttcap%
\pgfsetroundjoin%
\definecolor{currentfill}{rgb}{0.000000,0.000000,0.000000}%
\pgfsetfillcolor{currentfill}%
\pgfsetlinewidth{0.602250pt}%
\definecolor{currentstroke}{rgb}{0.000000,0.000000,0.000000}%
\pgfsetstrokecolor{currentstroke}%
\pgfsetdash{}{0pt}%
\pgfsys@defobject{currentmarker}{\pgfqpoint{-0.027778in}{0.000000in}}{\pgfqpoint{-0.000000in}{0.000000in}}{%
\pgfpathmoveto{\pgfqpoint{-0.000000in}{0.000000in}}%
\pgfpathlineto{\pgfqpoint{-0.027778in}{0.000000in}}%
\pgfusepath{stroke,fill}%
}%
\begin{pgfscope}%
\pgfsys@transformshift{0.588387in}{1.187878in}%
\pgfsys@useobject{currentmarker}{}%
\end{pgfscope}%
\end{pgfscope}%
\begin{pgfscope}%
\pgfsetbuttcap%
\pgfsetroundjoin%
\definecolor{currentfill}{rgb}{0.000000,0.000000,0.000000}%
\pgfsetfillcolor{currentfill}%
\pgfsetlinewidth{0.602250pt}%
\definecolor{currentstroke}{rgb}{0.000000,0.000000,0.000000}%
\pgfsetstrokecolor{currentstroke}%
\pgfsetdash{}{0pt}%
\pgfsys@defobject{currentmarker}{\pgfqpoint{-0.027778in}{0.000000in}}{\pgfqpoint{-0.000000in}{0.000000in}}{%
\pgfpathmoveto{\pgfqpoint{-0.000000in}{0.000000in}}%
\pgfpathlineto{\pgfqpoint{-0.027778in}{0.000000in}}%
\pgfusepath{stroke,fill}%
}%
\begin{pgfscope}%
\pgfsys@transformshift{0.588387in}{1.248548in}%
\pgfsys@useobject{currentmarker}{}%
\end{pgfscope}%
\end{pgfscope}%
\begin{pgfscope}%
\pgfsetbuttcap%
\pgfsetroundjoin%
\definecolor{currentfill}{rgb}{0.000000,0.000000,0.000000}%
\pgfsetfillcolor{currentfill}%
\pgfsetlinewidth{0.602250pt}%
\definecolor{currentstroke}{rgb}{0.000000,0.000000,0.000000}%
\pgfsetstrokecolor{currentstroke}%
\pgfsetdash{}{0pt}%
\pgfsys@defobject{currentmarker}{\pgfqpoint{-0.027778in}{0.000000in}}{\pgfqpoint{-0.000000in}{0.000000in}}{%
\pgfpathmoveto{\pgfqpoint{-0.000000in}{0.000000in}}%
\pgfpathlineto{\pgfqpoint{-0.027778in}{0.000000in}}%
\pgfusepath{stroke,fill}%
}%
\begin{pgfscope}%
\pgfsys@transformshift{0.588387in}{1.298119in}%
\pgfsys@useobject{currentmarker}{}%
\end{pgfscope}%
\end{pgfscope}%
\begin{pgfscope}%
\pgfsetbuttcap%
\pgfsetroundjoin%
\definecolor{currentfill}{rgb}{0.000000,0.000000,0.000000}%
\pgfsetfillcolor{currentfill}%
\pgfsetlinewidth{0.602250pt}%
\definecolor{currentstroke}{rgb}{0.000000,0.000000,0.000000}%
\pgfsetstrokecolor{currentstroke}%
\pgfsetdash{}{0pt}%
\pgfsys@defobject{currentmarker}{\pgfqpoint{-0.027778in}{0.000000in}}{\pgfqpoint{-0.000000in}{0.000000in}}{%
\pgfpathmoveto{\pgfqpoint{-0.000000in}{0.000000in}}%
\pgfpathlineto{\pgfqpoint{-0.027778in}{0.000000in}}%
\pgfusepath{stroke,fill}%
}%
\begin{pgfscope}%
\pgfsys@transformshift{0.588387in}{1.340031in}%
\pgfsys@useobject{currentmarker}{}%
\end{pgfscope}%
\end{pgfscope}%
\begin{pgfscope}%
\pgfsetbuttcap%
\pgfsetroundjoin%
\definecolor{currentfill}{rgb}{0.000000,0.000000,0.000000}%
\pgfsetfillcolor{currentfill}%
\pgfsetlinewidth{0.602250pt}%
\definecolor{currentstroke}{rgb}{0.000000,0.000000,0.000000}%
\pgfsetstrokecolor{currentstroke}%
\pgfsetdash{}{0pt}%
\pgfsys@defobject{currentmarker}{\pgfqpoint{-0.027778in}{0.000000in}}{\pgfqpoint{-0.000000in}{0.000000in}}{%
\pgfpathmoveto{\pgfqpoint{-0.000000in}{0.000000in}}%
\pgfpathlineto{\pgfqpoint{-0.027778in}{0.000000in}}%
\pgfusepath{stroke,fill}%
}%
\begin{pgfscope}%
\pgfsys@transformshift{0.588387in}{1.376337in}%
\pgfsys@useobject{currentmarker}{}%
\end{pgfscope}%
\end{pgfscope}%
\begin{pgfscope}%
\pgfsetbuttcap%
\pgfsetroundjoin%
\definecolor{currentfill}{rgb}{0.000000,0.000000,0.000000}%
\pgfsetfillcolor{currentfill}%
\pgfsetlinewidth{0.602250pt}%
\definecolor{currentstroke}{rgb}{0.000000,0.000000,0.000000}%
\pgfsetstrokecolor{currentstroke}%
\pgfsetdash{}{0pt}%
\pgfsys@defobject{currentmarker}{\pgfqpoint{-0.027778in}{0.000000in}}{\pgfqpoint{-0.000000in}{0.000000in}}{%
\pgfpathmoveto{\pgfqpoint{-0.000000in}{0.000000in}}%
\pgfpathlineto{\pgfqpoint{-0.027778in}{0.000000in}}%
\pgfusepath{stroke,fill}%
}%
\begin{pgfscope}%
\pgfsys@transformshift{0.588387in}{1.408360in}%
\pgfsys@useobject{currentmarker}{}%
\end{pgfscope}%
\end{pgfscope}%
\begin{pgfscope}%
\pgfsetbuttcap%
\pgfsetroundjoin%
\definecolor{currentfill}{rgb}{0.000000,0.000000,0.000000}%
\pgfsetfillcolor{currentfill}%
\pgfsetlinewidth{0.602250pt}%
\definecolor{currentstroke}{rgb}{0.000000,0.000000,0.000000}%
\pgfsetstrokecolor{currentstroke}%
\pgfsetdash{}{0pt}%
\pgfsys@defobject{currentmarker}{\pgfqpoint{-0.027778in}{0.000000in}}{\pgfqpoint{-0.000000in}{0.000000in}}{%
\pgfpathmoveto{\pgfqpoint{-0.000000in}{0.000000in}}%
\pgfpathlineto{\pgfqpoint{-0.027778in}{0.000000in}}%
\pgfusepath{stroke,fill}%
}%
\begin{pgfscope}%
\pgfsys@transformshift{0.588387in}{1.625465in}%
\pgfsys@useobject{currentmarker}{}%
\end{pgfscope}%
\end{pgfscope}%
\begin{pgfscope}%
\pgfsetbuttcap%
\pgfsetroundjoin%
\definecolor{currentfill}{rgb}{0.000000,0.000000,0.000000}%
\pgfsetfillcolor{currentfill}%
\pgfsetlinewidth{0.602250pt}%
\definecolor{currentstroke}{rgb}{0.000000,0.000000,0.000000}%
\pgfsetstrokecolor{currentstroke}%
\pgfsetdash{}{0pt}%
\pgfsys@defobject{currentmarker}{\pgfqpoint{-0.027778in}{0.000000in}}{\pgfqpoint{-0.000000in}{0.000000in}}{%
\pgfpathmoveto{\pgfqpoint{-0.000000in}{0.000000in}}%
\pgfpathlineto{\pgfqpoint{-0.027778in}{0.000000in}}%
\pgfusepath{stroke,fill}%
}%
\begin{pgfscope}%
\pgfsys@transformshift{0.588387in}{1.735707in}%
\pgfsys@useobject{currentmarker}{}%
\end{pgfscope}%
\end{pgfscope}%
\begin{pgfscope}%
\pgfsetbuttcap%
\pgfsetroundjoin%
\definecolor{currentfill}{rgb}{0.000000,0.000000,0.000000}%
\pgfsetfillcolor{currentfill}%
\pgfsetlinewidth{0.602250pt}%
\definecolor{currentstroke}{rgb}{0.000000,0.000000,0.000000}%
\pgfsetstrokecolor{currentstroke}%
\pgfsetdash{}{0pt}%
\pgfsys@defobject{currentmarker}{\pgfqpoint{-0.027778in}{0.000000in}}{\pgfqpoint{-0.000000in}{0.000000in}}{%
\pgfpathmoveto{\pgfqpoint{-0.000000in}{0.000000in}}%
\pgfpathlineto{\pgfqpoint{-0.027778in}{0.000000in}}%
\pgfusepath{stroke,fill}%
}%
\begin{pgfscope}%
\pgfsys@transformshift{0.588387in}{1.813924in}%
\pgfsys@useobject{currentmarker}{}%
\end{pgfscope}%
\end{pgfscope}%
\begin{pgfscope}%
\pgfsetbuttcap%
\pgfsetroundjoin%
\definecolor{currentfill}{rgb}{0.000000,0.000000,0.000000}%
\pgfsetfillcolor{currentfill}%
\pgfsetlinewidth{0.602250pt}%
\definecolor{currentstroke}{rgb}{0.000000,0.000000,0.000000}%
\pgfsetstrokecolor{currentstroke}%
\pgfsetdash{}{0pt}%
\pgfsys@defobject{currentmarker}{\pgfqpoint{-0.027778in}{0.000000in}}{\pgfqpoint{-0.000000in}{0.000000in}}{%
\pgfpathmoveto{\pgfqpoint{-0.000000in}{0.000000in}}%
\pgfpathlineto{\pgfqpoint{-0.027778in}{0.000000in}}%
\pgfusepath{stroke,fill}%
}%
\begin{pgfscope}%
\pgfsys@transformshift{0.588387in}{1.874594in}%
\pgfsys@useobject{currentmarker}{}%
\end{pgfscope}%
\end{pgfscope}%
\begin{pgfscope}%
\pgfsetbuttcap%
\pgfsetroundjoin%
\definecolor{currentfill}{rgb}{0.000000,0.000000,0.000000}%
\pgfsetfillcolor{currentfill}%
\pgfsetlinewidth{0.602250pt}%
\definecolor{currentstroke}{rgb}{0.000000,0.000000,0.000000}%
\pgfsetstrokecolor{currentstroke}%
\pgfsetdash{}{0pt}%
\pgfsys@defobject{currentmarker}{\pgfqpoint{-0.027778in}{0.000000in}}{\pgfqpoint{-0.000000in}{0.000000in}}{%
\pgfpathmoveto{\pgfqpoint{-0.000000in}{0.000000in}}%
\pgfpathlineto{\pgfqpoint{-0.027778in}{0.000000in}}%
\pgfusepath{stroke,fill}%
}%
\begin{pgfscope}%
\pgfsys@transformshift{0.588387in}{1.924165in}%
\pgfsys@useobject{currentmarker}{}%
\end{pgfscope}%
\end{pgfscope}%
\begin{pgfscope}%
\pgfsetbuttcap%
\pgfsetroundjoin%
\definecolor{currentfill}{rgb}{0.000000,0.000000,0.000000}%
\pgfsetfillcolor{currentfill}%
\pgfsetlinewidth{0.602250pt}%
\definecolor{currentstroke}{rgb}{0.000000,0.000000,0.000000}%
\pgfsetstrokecolor{currentstroke}%
\pgfsetdash{}{0pt}%
\pgfsys@defobject{currentmarker}{\pgfqpoint{-0.027778in}{0.000000in}}{\pgfqpoint{-0.000000in}{0.000000in}}{%
\pgfpathmoveto{\pgfqpoint{-0.000000in}{0.000000in}}%
\pgfpathlineto{\pgfqpoint{-0.027778in}{0.000000in}}%
\pgfusepath{stroke,fill}%
}%
\begin{pgfscope}%
\pgfsys@transformshift{0.588387in}{1.966077in}%
\pgfsys@useobject{currentmarker}{}%
\end{pgfscope}%
\end{pgfscope}%
\begin{pgfscope}%
\pgfsetbuttcap%
\pgfsetroundjoin%
\definecolor{currentfill}{rgb}{0.000000,0.000000,0.000000}%
\pgfsetfillcolor{currentfill}%
\pgfsetlinewidth{0.602250pt}%
\definecolor{currentstroke}{rgb}{0.000000,0.000000,0.000000}%
\pgfsetstrokecolor{currentstroke}%
\pgfsetdash{}{0pt}%
\pgfsys@defobject{currentmarker}{\pgfqpoint{-0.027778in}{0.000000in}}{\pgfqpoint{-0.000000in}{0.000000in}}{%
\pgfpathmoveto{\pgfqpoint{-0.000000in}{0.000000in}}%
\pgfpathlineto{\pgfqpoint{-0.027778in}{0.000000in}}%
\pgfusepath{stroke,fill}%
}%
\begin{pgfscope}%
\pgfsys@transformshift{0.588387in}{2.002383in}%
\pgfsys@useobject{currentmarker}{}%
\end{pgfscope}%
\end{pgfscope}%
\begin{pgfscope}%
\pgfsetbuttcap%
\pgfsetroundjoin%
\definecolor{currentfill}{rgb}{0.000000,0.000000,0.000000}%
\pgfsetfillcolor{currentfill}%
\pgfsetlinewidth{0.602250pt}%
\definecolor{currentstroke}{rgb}{0.000000,0.000000,0.000000}%
\pgfsetstrokecolor{currentstroke}%
\pgfsetdash{}{0pt}%
\pgfsys@defobject{currentmarker}{\pgfqpoint{-0.027778in}{0.000000in}}{\pgfqpoint{-0.000000in}{0.000000in}}{%
\pgfpathmoveto{\pgfqpoint{-0.000000in}{0.000000in}}%
\pgfpathlineto{\pgfqpoint{-0.027778in}{0.000000in}}%
\pgfusepath{stroke,fill}%
}%
\begin{pgfscope}%
\pgfsys@transformshift{0.588387in}{2.034407in}%
\pgfsys@useobject{currentmarker}{}%
\end{pgfscope}%
\end{pgfscope}%
\begin{pgfscope}%
\pgfsetbuttcap%
\pgfsetroundjoin%
\definecolor{currentfill}{rgb}{0.000000,0.000000,0.000000}%
\pgfsetfillcolor{currentfill}%
\pgfsetlinewidth{0.602250pt}%
\definecolor{currentstroke}{rgb}{0.000000,0.000000,0.000000}%
\pgfsetstrokecolor{currentstroke}%
\pgfsetdash{}{0pt}%
\pgfsys@defobject{currentmarker}{\pgfqpoint{-0.027778in}{0.000000in}}{\pgfqpoint{-0.000000in}{0.000000in}}{%
\pgfpathmoveto{\pgfqpoint{-0.000000in}{0.000000in}}%
\pgfpathlineto{\pgfqpoint{-0.027778in}{0.000000in}}%
\pgfusepath{stroke,fill}%
}%
\begin{pgfscope}%
\pgfsys@transformshift{0.588387in}{2.251511in}%
\pgfsys@useobject{currentmarker}{}%
\end{pgfscope}%
\end{pgfscope}%
\begin{pgfscope}%
\pgfsetbuttcap%
\pgfsetroundjoin%
\definecolor{currentfill}{rgb}{0.000000,0.000000,0.000000}%
\pgfsetfillcolor{currentfill}%
\pgfsetlinewidth{0.602250pt}%
\definecolor{currentstroke}{rgb}{0.000000,0.000000,0.000000}%
\pgfsetstrokecolor{currentstroke}%
\pgfsetdash{}{0pt}%
\pgfsys@defobject{currentmarker}{\pgfqpoint{-0.027778in}{0.000000in}}{\pgfqpoint{-0.000000in}{0.000000in}}{%
\pgfpathmoveto{\pgfqpoint{-0.000000in}{0.000000in}}%
\pgfpathlineto{\pgfqpoint{-0.027778in}{0.000000in}}%
\pgfusepath{stroke,fill}%
}%
\begin{pgfscope}%
\pgfsys@transformshift{0.588387in}{2.361753in}%
\pgfsys@useobject{currentmarker}{}%
\end{pgfscope}%
\end{pgfscope}%
\begin{pgfscope}%
\pgfsetbuttcap%
\pgfsetroundjoin%
\definecolor{currentfill}{rgb}{0.000000,0.000000,0.000000}%
\pgfsetfillcolor{currentfill}%
\pgfsetlinewidth{0.602250pt}%
\definecolor{currentstroke}{rgb}{0.000000,0.000000,0.000000}%
\pgfsetstrokecolor{currentstroke}%
\pgfsetdash{}{0pt}%
\pgfsys@defobject{currentmarker}{\pgfqpoint{-0.027778in}{0.000000in}}{\pgfqpoint{-0.000000in}{0.000000in}}{%
\pgfpathmoveto{\pgfqpoint{-0.000000in}{0.000000in}}%
\pgfpathlineto{\pgfqpoint{-0.027778in}{0.000000in}}%
\pgfusepath{stroke,fill}%
}%
\begin{pgfscope}%
\pgfsys@transformshift{0.588387in}{2.439970in}%
\pgfsys@useobject{currentmarker}{}%
\end{pgfscope}%
\end{pgfscope}%
\begin{pgfscope}%
\pgfsetbuttcap%
\pgfsetroundjoin%
\definecolor{currentfill}{rgb}{0.000000,0.000000,0.000000}%
\pgfsetfillcolor{currentfill}%
\pgfsetlinewidth{0.602250pt}%
\definecolor{currentstroke}{rgb}{0.000000,0.000000,0.000000}%
\pgfsetstrokecolor{currentstroke}%
\pgfsetdash{}{0pt}%
\pgfsys@defobject{currentmarker}{\pgfqpoint{-0.027778in}{0.000000in}}{\pgfqpoint{-0.000000in}{0.000000in}}{%
\pgfpathmoveto{\pgfqpoint{-0.000000in}{0.000000in}}%
\pgfpathlineto{\pgfqpoint{-0.027778in}{0.000000in}}%
\pgfusepath{stroke,fill}%
}%
\begin{pgfscope}%
\pgfsys@transformshift{0.588387in}{2.500640in}%
\pgfsys@useobject{currentmarker}{}%
\end{pgfscope}%
\end{pgfscope}%
\begin{pgfscope}%
\pgfsetbuttcap%
\pgfsetroundjoin%
\definecolor{currentfill}{rgb}{0.000000,0.000000,0.000000}%
\pgfsetfillcolor{currentfill}%
\pgfsetlinewidth{0.602250pt}%
\definecolor{currentstroke}{rgb}{0.000000,0.000000,0.000000}%
\pgfsetstrokecolor{currentstroke}%
\pgfsetdash{}{0pt}%
\pgfsys@defobject{currentmarker}{\pgfqpoint{-0.027778in}{0.000000in}}{\pgfqpoint{-0.000000in}{0.000000in}}{%
\pgfpathmoveto{\pgfqpoint{-0.000000in}{0.000000in}}%
\pgfpathlineto{\pgfqpoint{-0.027778in}{0.000000in}}%
\pgfusepath{stroke,fill}%
}%
\begin{pgfscope}%
\pgfsys@transformshift{0.588387in}{2.550211in}%
\pgfsys@useobject{currentmarker}{}%
\end{pgfscope}%
\end{pgfscope}%
\begin{pgfscope}%
\pgfsetbuttcap%
\pgfsetroundjoin%
\definecolor{currentfill}{rgb}{0.000000,0.000000,0.000000}%
\pgfsetfillcolor{currentfill}%
\pgfsetlinewidth{0.602250pt}%
\definecolor{currentstroke}{rgb}{0.000000,0.000000,0.000000}%
\pgfsetstrokecolor{currentstroke}%
\pgfsetdash{}{0pt}%
\pgfsys@defobject{currentmarker}{\pgfqpoint{-0.027778in}{0.000000in}}{\pgfqpoint{-0.000000in}{0.000000in}}{%
\pgfpathmoveto{\pgfqpoint{-0.000000in}{0.000000in}}%
\pgfpathlineto{\pgfqpoint{-0.027778in}{0.000000in}}%
\pgfusepath{stroke,fill}%
}%
\begin{pgfscope}%
\pgfsys@transformshift{0.588387in}{2.592123in}%
\pgfsys@useobject{currentmarker}{}%
\end{pgfscope}%
\end{pgfscope}%
\begin{pgfscope}%
\pgfsetbuttcap%
\pgfsetroundjoin%
\definecolor{currentfill}{rgb}{0.000000,0.000000,0.000000}%
\pgfsetfillcolor{currentfill}%
\pgfsetlinewidth{0.602250pt}%
\definecolor{currentstroke}{rgb}{0.000000,0.000000,0.000000}%
\pgfsetstrokecolor{currentstroke}%
\pgfsetdash{}{0pt}%
\pgfsys@defobject{currentmarker}{\pgfqpoint{-0.027778in}{0.000000in}}{\pgfqpoint{-0.000000in}{0.000000in}}{%
\pgfpathmoveto{\pgfqpoint{-0.000000in}{0.000000in}}%
\pgfpathlineto{\pgfqpoint{-0.027778in}{0.000000in}}%
\pgfusepath{stroke,fill}%
}%
\begin{pgfscope}%
\pgfsys@transformshift{0.588387in}{2.628429in}%
\pgfsys@useobject{currentmarker}{}%
\end{pgfscope}%
\end{pgfscope}%
\begin{pgfscope}%
\pgfsetbuttcap%
\pgfsetroundjoin%
\definecolor{currentfill}{rgb}{0.000000,0.000000,0.000000}%
\pgfsetfillcolor{currentfill}%
\pgfsetlinewidth{0.602250pt}%
\definecolor{currentstroke}{rgb}{0.000000,0.000000,0.000000}%
\pgfsetstrokecolor{currentstroke}%
\pgfsetdash{}{0pt}%
\pgfsys@defobject{currentmarker}{\pgfqpoint{-0.027778in}{0.000000in}}{\pgfqpoint{-0.000000in}{0.000000in}}{%
\pgfpathmoveto{\pgfqpoint{-0.000000in}{0.000000in}}%
\pgfpathlineto{\pgfqpoint{-0.027778in}{0.000000in}}%
\pgfusepath{stroke,fill}%
}%
\begin{pgfscope}%
\pgfsys@transformshift{0.588387in}{2.660453in}%
\pgfsys@useobject{currentmarker}{}%
\end{pgfscope}%
\end{pgfscope}%
\begin{pgfscope}%
\definecolor{textcolor}{rgb}{0.000000,0.000000,0.000000}%
\pgfsetstrokecolor{textcolor}%
\pgfsetfillcolor{textcolor}%
\pgftext[x=0.234413in,y=1.631490in,,bottom,rotate=90.000000]{\color{textcolor}{\rmfamily\fontsize{10.000000}{12.000000}\selectfont\catcode`\^=\active\def^{\ifmmode\sp\else\^{}\fi}\catcode`\%=\active\def%{\%}Time [ms]}}%
\end{pgfscope}%
\begin{pgfscope}%
\pgfpathrectangle{\pgfqpoint{0.588387in}{0.521603in}}{\pgfqpoint{3.660036in}{2.219773in}}%
\pgfusepath{clip}%
\pgfsetrectcap%
\pgfsetroundjoin%
\pgfsetlinewidth{1.505625pt}%
\pgfsetstrokecolor{currentstroke1}%
\pgfsetdash{}{0pt}%
\pgfpathmoveto{\pgfqpoint{0.754752in}{0.622502in}}%
\pgfpathlineto{\pgfqpoint{1.010699in}{0.836874in}}%
\pgfpathlineto{\pgfqpoint{1.266646in}{1.048990in}}%
\pgfpathlineto{\pgfqpoint{1.650565in}{1.259212in}}%
\pgfpathlineto{\pgfqpoint{1.778539in}{1.577656in}}%
\pgfpathlineto{\pgfqpoint{2.034485in}{1.813000in}}%
\pgfpathlineto{\pgfqpoint{2.418405in}{2.040235in}}%
\pgfpathlineto{\pgfqpoint{2.674352in}{2.329559in}}%
\pgfpathlineto{\pgfqpoint{3.058271in}{2.618319in}}%
\pgfpathlineto{\pgfqpoint{3.314218in}{2.599732in}}%
\pgfusepath{stroke}%
\end{pgfscope}%
\begin{pgfscope}%
\pgfpathrectangle{\pgfqpoint{0.588387in}{0.521603in}}{\pgfqpoint{3.660036in}{2.219773in}}%
\pgfusepath{clip}%
\pgfsetrectcap%
\pgfsetroundjoin%
\pgfsetlinewidth{1.505625pt}%
\pgfsetstrokecolor{currentstroke2}%
\pgfsetdash{}{0pt}%
\pgfpathmoveto{\pgfqpoint{0.754752in}{0.622502in}}%
\pgfpathlineto{\pgfqpoint{1.010699in}{0.810961in}}%
\pgfpathlineto{\pgfqpoint{1.266646in}{1.054597in}}%
\pgfpathlineto{\pgfqpoint{1.650565in}{1.307034in}}%
\pgfpathlineto{\pgfqpoint{1.778539in}{1.549953in}}%
\pgfpathlineto{\pgfqpoint{2.034485in}{1.816052in}}%
\pgfpathlineto{\pgfqpoint{2.418405in}{2.035311in}}%
\pgfpathlineto{\pgfqpoint{2.674352in}{2.253189in}}%
\pgfpathlineto{\pgfqpoint{3.058271in}{2.611858in}}%
\pgfpathlineto{\pgfqpoint{3.314218in}{2.620278in}}%
\pgfpathlineto{\pgfqpoint{3.698138in}{2.526646in}}%
\pgfusepath{stroke}%
\end{pgfscope}%
\begin{pgfscope}%
\pgfpathrectangle{\pgfqpoint{0.588387in}{0.521603in}}{\pgfqpoint{3.660036in}{2.219773in}}%
\pgfusepath{clip}%
\pgfsetrectcap%
\pgfsetroundjoin%
\pgfsetlinewidth{1.505625pt}%
\pgfsetstrokecolor{currentstroke3}%
\pgfsetdash{}{0pt}%
\pgfpathmoveto{\pgfqpoint{0.754752in}{0.672073in}}%
\pgfpathlineto{\pgfqpoint{1.010699in}{0.810961in}}%
\pgfpathlineto{\pgfqpoint{1.266646in}{1.031443in}}%
\pgfpathlineto{\pgfqpoint{1.650565in}{1.259212in}}%
\pgfpathlineto{\pgfqpoint{1.778539in}{1.517593in}}%
\pgfpathlineto{\pgfqpoint{2.034485in}{1.780708in}}%
\pgfpathlineto{\pgfqpoint{2.418405in}{1.892990in}}%
\pgfpathlineto{\pgfqpoint{2.674352in}{2.211668in}}%
\pgfpathlineto{\pgfqpoint{3.058271in}{2.119116in}}%
\pgfpathlineto{\pgfqpoint{3.314218in}{2.454711in}}%
\pgfpathlineto{\pgfqpoint{3.698138in}{2.579553in}}%
\pgfpathlineto{\pgfqpoint{4.082057in}{2.605148in}}%
\pgfusepath{stroke}%
\end{pgfscope}%
\begin{pgfscope}%
\pgfpathrectangle{\pgfqpoint{0.588387in}{0.521603in}}{\pgfqpoint{3.660036in}{2.219773in}}%
\pgfusepath{clip}%
\pgfsetrectcap%
\pgfsetroundjoin%
\pgfsetlinewidth{1.505625pt}%
\pgfsetstrokecolor{currentstroke4}%
\pgfsetdash{}{0pt}%
\pgfpathmoveto{\pgfqpoint{0.754752in}{0.622502in}}%
\pgfpathlineto{\pgfqpoint{1.010699in}{0.810961in}}%
\pgfpathlineto{\pgfqpoint{1.266646in}{1.054597in}}%
\pgfpathlineto{\pgfqpoint{1.650565in}{1.251254in}}%
\pgfpathlineto{\pgfqpoint{1.778539in}{1.524085in}}%
\pgfpathlineto{\pgfqpoint{2.034485in}{1.773425in}}%
\pgfpathlineto{\pgfqpoint{2.418405in}{1.883158in}}%
\pgfpathlineto{\pgfqpoint{2.674352in}{2.147378in}}%
\pgfpathlineto{\pgfqpoint{3.058271in}{2.102585in}}%
\pgfpathlineto{\pgfqpoint{3.314218in}{2.444862in}}%
\pgfpathlineto{\pgfqpoint{3.698138in}{2.561125in}}%
\pgfpathlineto{\pgfqpoint{4.082057in}{2.613114in}}%
\pgfusepath{stroke}%
\end{pgfscope}%
\begin{pgfscope}%
\pgfpathrectangle{\pgfqpoint{0.588387in}{0.521603in}}{\pgfqpoint{3.660036in}{2.219773in}}%
\pgfusepath{clip}%
\pgfsetrectcap%
\pgfsetroundjoin%
\pgfsetlinewidth{1.505625pt}%
\pgfsetstrokecolor{currentstroke5}%
\pgfsetdash{}{0pt}%
\pgfpathmoveto{\pgfqpoint{0.754752in}{0.622502in}}%
\pgfpathlineto{\pgfqpoint{1.010699in}{0.810961in}}%
\pgfpathlineto{\pgfqpoint{1.266646in}{1.043266in}}%
\pgfpathlineto{\pgfqpoint{1.650565in}{1.253932in}}%
\pgfpathlineto{\pgfqpoint{1.778539in}{1.531387in}}%
\pgfpathlineto{\pgfqpoint{2.034485in}{1.775278in}}%
\pgfpathlineto{\pgfqpoint{2.418405in}{2.002553in}}%
\pgfpathlineto{\pgfqpoint{2.674352in}{2.200502in}}%
\pgfpathlineto{\pgfqpoint{3.058271in}{2.187765in}}%
\pgfpathlineto{\pgfqpoint{3.314218in}{2.516349in}}%
\pgfpathlineto{\pgfqpoint{3.698138in}{2.588396in}}%
\pgfpathlineto{\pgfqpoint{4.082057in}{2.595573in}}%
\pgfusepath{stroke}%
\end{pgfscope}%
\begin{pgfscope}%
\pgfpathrectangle{\pgfqpoint{0.588387in}{0.521603in}}{\pgfqpoint{3.660036in}{2.219773in}}%
\pgfusepath{clip}%
\pgfsetrectcap%
\pgfsetroundjoin%
\pgfsetlinewidth{1.505625pt}%
\pgfsetstrokecolor{currentstroke6}%
\pgfsetdash{}{0pt}%
\pgfpathmoveto{\pgfqpoint{0.754752in}{0.622502in}}%
\pgfpathlineto{\pgfqpoint{1.010699in}{0.810961in}}%
\pgfpathlineto{\pgfqpoint{1.266646in}{1.048990in}}%
\pgfpathlineto{\pgfqpoint{1.650565in}{1.245816in}}%
\pgfpathlineto{\pgfqpoint{1.778539in}{1.529459in}}%
\pgfpathlineto{\pgfqpoint{2.034485in}{1.765825in}}%
\pgfpathlineto{\pgfqpoint{2.418405in}{1.896775in}}%
\pgfpathlineto{\pgfqpoint{2.674352in}{2.183478in}}%
\pgfpathlineto{\pgfqpoint{3.058271in}{2.105624in}}%
\pgfpathlineto{\pgfqpoint{3.314218in}{2.479310in}}%
\pgfpathlineto{\pgfqpoint{3.698138in}{2.573163in}}%
\pgfpathlineto{\pgfqpoint{4.082057in}{2.633702in}}%
\pgfusepath{stroke}%
\end{pgfscope}%
\begin{pgfscope}%
\pgfpathrectangle{\pgfqpoint{0.588387in}{0.521603in}}{\pgfqpoint{3.660036in}{2.219773in}}%
\pgfusepath{clip}%
\pgfsetrectcap%
\pgfsetroundjoin%
\pgfsetlinewidth{1.505625pt}%
\pgfsetstrokecolor{currentstroke7}%
\pgfsetdash{}{0pt}%
\pgfpathmoveto{\pgfqpoint{0.754752in}{0.672073in}}%
\pgfpathlineto{\pgfqpoint{1.010699in}{0.810961in}}%
\pgfpathlineto{\pgfqpoint{1.266646in}{1.048990in}}%
\pgfpathlineto{\pgfqpoint{1.650565in}{1.274462in}}%
\pgfpathlineto{\pgfqpoint{1.778539in}{1.548153in}}%
\pgfpathlineto{\pgfqpoint{2.034485in}{1.837666in}}%
\pgfpathlineto{\pgfqpoint{2.418405in}{2.007933in}}%
\pgfpathlineto{\pgfqpoint{2.674352in}{2.183565in}}%
\pgfpathlineto{\pgfqpoint{3.058271in}{2.425702in}}%
\pgfpathlineto{\pgfqpoint{3.314218in}{2.614459in}}%
\pgfpathlineto{\pgfqpoint{3.698138in}{2.640478in}}%
\pgfusepath{stroke}%
\end{pgfscope}%
\begin{pgfscope}%
\pgfpathrectangle{\pgfqpoint{0.588387in}{0.521603in}}{\pgfqpoint{3.660036in}{2.219773in}}%
\pgfusepath{clip}%
\pgfsetrectcap%
\pgfsetroundjoin%
\pgfsetlinewidth{1.505625pt}%
\definecolor{currentstroke}{rgb}{0.498039,0.498039,0.498039}%
\pgfsetstrokecolor{currentstroke}%
\pgfsetdash{}{0pt}%
\pgfpathmoveto{\pgfqpoint{1.010699in}{0.810961in}}%
\pgfpathlineto{\pgfqpoint{1.266646in}{1.031443in}}%
\pgfpathlineto{\pgfqpoint{1.650565in}{1.243055in}}%
\pgfpathlineto{\pgfqpoint{1.778539in}{1.540829in}}%
\pgfpathlineto{\pgfqpoint{2.034485in}{1.784088in}}%
\pgfpathlineto{\pgfqpoint{2.418405in}{1.964714in}}%
\pgfpathlineto{\pgfqpoint{2.674352in}{2.188837in}}%
\pgfpathlineto{\pgfqpoint{3.058271in}{2.415483in}}%
\pgfpathlineto{\pgfqpoint{3.314218in}{2.537460in}}%
\pgfpathlineto{\pgfqpoint{3.698138in}{2.637042in}}%
\pgfpathlineto{\pgfqpoint{4.082057in}{2.547890in}}%
\pgfusepath{stroke}%
\end{pgfscope}%
\begin{pgfscope}%
\pgfpathrectangle{\pgfqpoint{0.588387in}{0.521603in}}{\pgfqpoint{3.660036in}{2.219773in}}%
\pgfusepath{clip}%
\pgfsetrectcap%
\pgfsetroundjoin%
\pgfsetlinewidth{1.505625pt}%
\definecolor{currentstroke}{rgb}{0.737255,0.741176,0.133333}%
\pgfsetstrokecolor{currentstroke}%
\pgfsetdash{}{0pt}%
\pgfpathmoveto{\pgfqpoint{2.034485in}{1.825613in}}%
\pgfpathlineto{\pgfqpoint{2.418405in}{2.194722in}}%
\pgfpathlineto{\pgfqpoint{2.674352in}{2.565251in}}%
\pgfusepath{stroke}%
\end{pgfscope}%
\begin{pgfscope}%
\pgfsetrectcap%
\pgfsetmiterjoin%
\pgfsetlinewidth{0.803000pt}%
\definecolor{currentstroke}{rgb}{0.000000,0.000000,0.000000}%
\pgfsetstrokecolor{currentstroke}%
\pgfsetdash{}{0pt}%
\pgfpathmoveto{\pgfqpoint{0.588387in}{0.521603in}}%
\pgfpathlineto{\pgfqpoint{0.588387in}{2.741376in}}%
\pgfusepath{stroke}%
\end{pgfscope}%
\begin{pgfscope}%
\pgfsetrectcap%
\pgfsetmiterjoin%
\pgfsetlinewidth{0.803000pt}%
\definecolor{currentstroke}{rgb}{0.000000,0.000000,0.000000}%
\pgfsetstrokecolor{currentstroke}%
\pgfsetdash{}{0pt}%
\pgfpathmoveto{\pgfqpoint{4.248423in}{0.521603in}}%
\pgfpathlineto{\pgfqpoint{4.248423in}{2.741376in}}%
\pgfusepath{stroke}%
\end{pgfscope}%
\begin{pgfscope}%
\pgfsetrectcap%
\pgfsetmiterjoin%
\pgfsetlinewidth{0.803000pt}%
\definecolor{currentstroke}{rgb}{0.000000,0.000000,0.000000}%
\pgfsetstrokecolor{currentstroke}%
\pgfsetdash{}{0pt}%
\pgfpathmoveto{\pgfqpoint{0.588387in}{0.521603in}}%
\pgfpathlineto{\pgfqpoint{4.248423in}{0.521603in}}%
\pgfusepath{stroke}%
\end{pgfscope}%
\begin{pgfscope}%
\pgfsetrectcap%
\pgfsetmiterjoin%
\pgfsetlinewidth{0.803000pt}%
\definecolor{currentstroke}{rgb}{0.000000,0.000000,0.000000}%
\pgfsetstrokecolor{currentstroke}%
\pgfsetdash{}{0pt}%
\pgfpathmoveto{\pgfqpoint{0.588387in}{2.741376in}}%
\pgfpathlineto{\pgfqpoint{4.248423in}{2.741376in}}%
\pgfusepath{stroke}%
\end{pgfscope}%
\begin{pgfscope}%
\pgfsetbuttcap%
\pgfsetmiterjoin%
\definecolor{currentfill}{rgb}{1.000000,1.000000,1.000000}%
\pgfsetfillcolor{currentfill}%
\pgfsetfillopacity{0.800000}%
\pgfsetlinewidth{1.003750pt}%
\definecolor{currentstroke}{rgb}{0.800000,0.800000,0.800000}%
\pgfsetstrokecolor{currentstroke}%
\pgfsetstrokeopacity{0.800000}%
\pgfsetdash{}{0pt}%
\pgfpathmoveto{\pgfqpoint{4.365089in}{0.378553in}}%
\pgfpathlineto{\pgfqpoint{8.251043in}{0.378553in}}%
\pgfpathquadraticcurveto{\pgfqpoint{8.284376in}{0.378553in}}{\pgfqpoint{8.284376in}{0.411886in}}%
\pgfpathlineto{\pgfqpoint{8.284376in}{2.624710in}}%
\pgfpathquadraticcurveto{\pgfqpoint{8.284376in}{2.658043in}}{\pgfqpoint{8.251043in}{2.658043in}}%
\pgfpathlineto{\pgfqpoint{4.365089in}{2.658043in}}%
\pgfpathquadraticcurveto{\pgfqpoint{4.331756in}{2.658043in}}{\pgfqpoint{4.331756in}{2.624710in}}%
\pgfpathlineto{\pgfqpoint{4.331756in}{0.411886in}}%
\pgfpathquadraticcurveto{\pgfqpoint{4.331756in}{0.378553in}}{\pgfqpoint{4.365089in}{0.378553in}}%
\pgfpathlineto{\pgfqpoint{4.365089in}{0.378553in}}%
\pgfpathclose%
\pgfusepath{stroke,fill}%
\end{pgfscope}%
\begin{pgfscope}%
\pgfsetrectcap%
\pgfsetroundjoin%
\pgfsetlinewidth{1.505625pt}%
\definecolor{currentstroke}{rgb}{0.737255,0.741176,0.133333}%
\pgfsetstrokecolor{currentstroke}%
\pgfsetdash{}{0pt}%
\pgfpathmoveto{\pgfqpoint{4.398423in}{2.523082in}}%
\pgfpathlineto{\pgfqpoint{4.565089in}{2.523082in}}%
\pgfpathlineto{\pgfqpoint{4.731756in}{2.523082in}}%
\pgfusepath{stroke}%
\end{pgfscope}%
\begin{pgfscope}%
\definecolor{textcolor}{rgb}{0.000000,0.000000,0.000000}%
\pgfsetstrokecolor{textcolor}%
\pgfsetfillcolor{textcolor}%
\pgftext[x=4.865089in,y=2.464749in,left,base]{\color{textcolor}{\rmfamily\fontsize{12.000000}{14.400000}\selectfont\catcode`\^=\active\def^{\ifmmode\sp\else\^{}\fi}\catcode`\%=\active\def%{\%}\NaiveCycles{}}}%
\end{pgfscope}%
\begin{pgfscope}%
\pgfsetrectcap%
\pgfsetroundjoin%
\pgfsetlinewidth{1.505625pt}%
\pgfsetstrokecolor{currentstroke1}%
\pgfsetdash{}{0pt}%
\pgfpathmoveto{\pgfqpoint{4.398423in}{2.278453in}}%
\pgfpathlineto{\pgfqpoint{4.565089in}{2.278453in}}%
\pgfpathlineto{\pgfqpoint{4.731756in}{2.278453in}}%
\pgfusepath{stroke}%
\end{pgfscope}%
\begin{pgfscope}%
\definecolor{textcolor}{rgb}{0.000000,0.000000,0.000000}%
\pgfsetstrokecolor{textcolor}%
\pgfsetfillcolor{textcolor}%
\pgftext[x=4.865089in,y=2.220120in,left,base]{\color{textcolor}{\rmfamily\fontsize{12.000000}{14.400000}\selectfont\catcode`\^=\active\def^{\ifmmode\sp\else\^{}\fi}\catcode`\%=\active\def%{\%}\CyclesMatchChunks{} \& \MergeLinear{}}}%
\end{pgfscope}%
\begin{pgfscope}%
\pgfsetrectcap%
\pgfsetroundjoin%
\pgfsetlinewidth{1.505625pt}%
\pgfsetstrokecolor{currentstroke2}%
\pgfsetdash{}{0pt}%
\pgfpathmoveto{\pgfqpoint{4.398423in}{2.029186in}}%
\pgfpathlineto{\pgfqpoint{4.565089in}{2.029186in}}%
\pgfpathlineto{\pgfqpoint{4.731756in}{2.029186in}}%
\pgfusepath{stroke}%
\end{pgfscope}%
\begin{pgfscope}%
\definecolor{textcolor}{rgb}{0.000000,0.000000,0.000000}%
\pgfsetstrokecolor{textcolor}%
\pgfsetfillcolor{textcolor}%
\pgftext[x=4.865089in,y=1.970853in,left,base]{\color{textcolor}{\rmfamily\fontsize{12.000000}{14.400000}\selectfont\catcode`\^=\active\def^{\ifmmode\sp\else\^{}\fi}\catcode`\%=\active\def%{\%}\CyclesMatchChunks{} \& \SharedVertices{}}}%
\end{pgfscope}%
\begin{pgfscope}%
\pgfsetrectcap%
\pgfsetroundjoin%
\pgfsetlinewidth{1.505625pt}%
\pgfsetstrokecolor{currentstroke3}%
\pgfsetdash{}{0pt}%
\pgfpathmoveto{\pgfqpoint{4.398423in}{1.779919in}}%
\pgfpathlineto{\pgfqpoint{4.565089in}{1.779919in}}%
\pgfpathlineto{\pgfqpoint{4.731756in}{1.779919in}}%
\pgfusepath{stroke}%
\end{pgfscope}%
\begin{pgfscope}%
\definecolor{textcolor}{rgb}{0.000000,0.000000,0.000000}%
\pgfsetstrokecolor{textcolor}%
\pgfsetfillcolor{textcolor}%
\pgftext[x=4.865089in,y=1.721585in,left,base]{\color{textcolor}{\rmfamily\fontsize{12.000000}{14.400000}\selectfont\catcode`\^=\active\def^{\ifmmode\sp\else\^{}\fi}\catcode`\%=\active\def%{\%}\Neighbors{} \& \MergeLinear{}}}%
\end{pgfscope}%
\begin{pgfscope}%
\pgfsetrectcap%
\pgfsetroundjoin%
\pgfsetlinewidth{1.505625pt}%
\pgfsetstrokecolor{currentstroke4}%
\pgfsetdash{}{0pt}%
\pgfpathmoveto{\pgfqpoint{4.398423in}{1.535290in}}%
\pgfpathlineto{\pgfqpoint{4.565089in}{1.535290in}}%
\pgfpathlineto{\pgfqpoint{4.731756in}{1.535290in}}%
\pgfusepath{stroke}%
\end{pgfscope}%
\begin{pgfscope}%
\definecolor{textcolor}{rgb}{0.000000,0.000000,0.000000}%
\pgfsetstrokecolor{textcolor}%
\pgfsetfillcolor{textcolor}%
\pgftext[x=4.865089in,y=1.476957in,left,base]{\color{textcolor}{\rmfamily\fontsize{12.000000}{14.400000}\selectfont\catcode`\^=\active\def^{\ifmmode\sp\else\^{}\fi}\catcode`\%=\active\def%{\%}\Neighbors{} \& \SharedVertices{}}}%
\end{pgfscope}%
\begin{pgfscope}%
\pgfsetrectcap%
\pgfsetroundjoin%
\pgfsetlinewidth{1.505625pt}%
\pgfsetstrokecolor{currentstroke5}%
\pgfsetdash{}{0pt}%
\pgfpathmoveto{\pgfqpoint{4.398423in}{1.286023in}}%
\pgfpathlineto{\pgfqpoint{4.565089in}{1.286023in}}%
\pgfpathlineto{\pgfqpoint{4.731756in}{1.286023in}}%
\pgfusepath{stroke}%
\end{pgfscope}%
\begin{pgfscope}%
\definecolor{textcolor}{rgb}{0.000000,0.000000,0.000000}%
\pgfsetstrokecolor{textcolor}%
\pgfsetfillcolor{textcolor}%
\pgftext[x=4.865089in,y=1.227689in,left,base]{\color{textcolor}{\rmfamily\fontsize{12.000000}{14.400000}\selectfont\catcode`\^=\active\def^{\ifmmode\sp\else\^{}\fi}\catcode`\%=\active\def%{\%}\NeighborsDegree{} \& \MergeLinear{}}}%
\end{pgfscope}%
\begin{pgfscope}%
\pgfsetrectcap%
\pgfsetroundjoin%
\pgfsetlinewidth{1.505625pt}%
\pgfsetstrokecolor{currentstroke6}%
\pgfsetdash{}{0pt}%
\pgfpathmoveto{\pgfqpoint{4.398423in}{1.036755in}}%
\pgfpathlineto{\pgfqpoint{4.565089in}{1.036755in}}%
\pgfpathlineto{\pgfqpoint{4.731756in}{1.036755in}}%
\pgfusepath{stroke}%
\end{pgfscope}%
\begin{pgfscope}%
\definecolor{textcolor}{rgb}{0.000000,0.000000,0.000000}%
\pgfsetstrokecolor{textcolor}%
\pgfsetfillcolor{textcolor}%
\pgftext[x=4.865089in,y=0.978422in,left,base]{\color{textcolor}{\rmfamily\fontsize{12.000000}{14.400000}\selectfont\catcode`\^=\active\def^{\ifmmode\sp\else\^{}\fi}\catcode`\%=\active\def%{\%}\NeighborsDegree{} \& \SharedVertices{}}}%
\end{pgfscope}%
\begin{pgfscope}%
\pgfsetrectcap%
\pgfsetroundjoin%
\pgfsetlinewidth{1.505625pt}%
\pgfsetstrokecolor{currentstroke7}%
\pgfsetdash{}{0pt}%
\pgfpathmoveto{\pgfqpoint{4.398423in}{0.787488in}}%
\pgfpathlineto{\pgfqpoint{4.565089in}{0.787488in}}%
\pgfpathlineto{\pgfqpoint{4.731756in}{0.787488in}}%
\pgfusepath{stroke}%
\end{pgfscope}%
\begin{pgfscope}%
\definecolor{textcolor}{rgb}{0.000000,0.000000,0.000000}%
\pgfsetstrokecolor{textcolor}%
\pgfsetfillcolor{textcolor}%
\pgftext[x=4.865089in,y=0.729155in,left,base]{\color{textcolor}{\rmfamily\fontsize{12.000000}{14.400000}\selectfont\catcode`\^=\active\def^{\ifmmode\sp\else\^{}\fi}\catcode`\%=\active\def%{\%}\None{} \& \MergeLinear{}}}%
\end{pgfscope}%
\begin{pgfscope}%
\pgfsetrectcap%
\pgfsetroundjoin%
\pgfsetlinewidth{1.505625pt}%
\definecolor{currentstroke}{rgb}{0.498039,0.498039,0.498039}%
\pgfsetstrokecolor{currentstroke}%
\pgfsetdash{}{0pt}%
\pgfpathmoveto{\pgfqpoint{4.398423in}{0.542859in}}%
\pgfpathlineto{\pgfqpoint{4.565089in}{0.542859in}}%
\pgfpathlineto{\pgfqpoint{4.731756in}{0.542859in}}%
\pgfusepath{stroke}%
\end{pgfscope}%
\begin{pgfscope}%
\definecolor{textcolor}{rgb}{0.000000,0.000000,0.000000}%
\pgfsetstrokecolor{textcolor}%
\pgfsetfillcolor{textcolor}%
\pgftext[x=4.865089in,y=0.484526in,left,base]{\color{textcolor}{\rmfamily\fontsize{12.000000}{14.400000}\selectfont\catcode`\^=\active\def^{\ifmmode\sp\else\^{}\fi}\catcode`\%=\active\def%{\%}\None{} \& \SharedVertices{}}}%
\end{pgfscope}%
\end{pgfpicture}%
\makeatother%
\endgroup%
}
	\caption[Mean runtime for graphs with no 3 nor 4 cycles (all)]{
		Mean running time to find all NAC-colorings for graphs with no three nor four cycles.}%
	\label{fig:graph_count_no_3_nor_4_cycles_all_runtime}
\end{figure}%
% \begin{figure}[thbp]
% 	\centering
% 	\scalebox{\BenchFigureScale}{%% Creator: Matplotlib, PGF backend
%%
%% To include the figure in your LaTeX document, write
%%   \input{<filename>.pgf}
%%
%% Make sure the required packages are loaded in your preamble
%%   \usepackage{pgf}
%%
%% Also ensure that all the required font packages are loaded; for instance,
%% the lmodern package is sometimes necessary when using math font.
%%   \usepackage{lmodern}
%%
%% Figures using additional raster images can only be included by \input if
%% they are in the same directory as the main LaTeX file. For loading figures
%% from other directories you can use the `import` package
%%   \usepackage{import}
%%
%% and then include the figures with
%%   \import{<path to file>}{<filename>.pgf}
%%
%% Matplotlib used the following preamble
%%   \def\mathdefault#1{#1}
%%   \everymath=\expandafter{\the\everymath\displaystyle}
%%   \IfFileExists{scrextend.sty}{
%%     \usepackage[fontsize=10.000000pt]{scrextend}
%%   }{
%%     \renewcommand{\normalsize}{\fontsize{10.000000}{12.000000}\selectfont}
%%     \normalsize
%%   }
%%   
%%   \ifdefined\pdftexversion\else  % non-pdftex case.
%%     \usepackage{fontspec}
%%     \setmainfont{DejaVuSans.ttf}[Path=\detokenize{/home/petr/Projects/PyRigi/.venv/lib/python3.12/site-packages/matplotlib/mpl-data/fonts/ttf/}]
%%     \setsansfont{DejaVuSans.ttf}[Path=\detokenize{/home/petr/Projects/PyRigi/.venv/lib/python3.12/site-packages/matplotlib/mpl-data/fonts/ttf/}]
%%     \setmonofont{DejaVuSansMono.ttf}[Path=\detokenize{/home/petr/Projects/PyRigi/.venv/lib/python3.12/site-packages/matplotlib/mpl-data/fonts/ttf/}]
%%   \fi
%%   \makeatletter\@ifpackageloaded{under\Score{}}{}{\usepackage[strings]{under\Score{}}}\makeatother
%%
\begingroup%
\makeatletter%
\begin{pgfpicture}%
\pgfpathrectangle{\pgfpointorigin}{\pgfqpoint{8.384376in}{2.841849in}}%
\pgfusepath{use as bounding box, clip}%
\begin{pgfscope}%
\pgfsetbuttcap%
\pgfsetmiterjoin%
\definecolor{currentfill}{rgb}{1.000000,1.000000,1.000000}%
\pgfsetfillcolor{currentfill}%
\pgfsetlinewidth{0.000000pt}%
\definecolor{currentstroke}{rgb}{1.000000,1.000000,1.000000}%
\pgfsetstrokecolor{currentstroke}%
\pgfsetdash{}{0pt}%
\pgfpathmoveto{\pgfqpoint{0.000000in}{0.000000in}}%
\pgfpathlineto{\pgfqpoint{8.384376in}{0.000000in}}%
\pgfpathlineto{\pgfqpoint{8.384376in}{2.841849in}}%
\pgfpathlineto{\pgfqpoint{0.000000in}{2.841849in}}%
\pgfpathlineto{\pgfqpoint{0.000000in}{0.000000in}}%
\pgfpathclose%
\pgfusepath{fill}%
\end{pgfscope}%
\begin{pgfscope}%
\pgfsetbuttcap%
\pgfsetmiterjoin%
\definecolor{currentfill}{rgb}{1.000000,1.000000,1.000000}%
\pgfsetfillcolor{currentfill}%
\pgfsetlinewidth{0.000000pt}%
\definecolor{currentstroke}{rgb}{0.000000,0.000000,0.000000}%
\pgfsetstrokecolor{currentstroke}%
\pgfsetstrokeopacity{0.000000}%
\pgfsetdash{}{0pt}%
\pgfpathmoveto{\pgfqpoint{0.588387in}{0.521603in}}%
\pgfpathlineto{\pgfqpoint{5.257411in}{0.521603in}}%
\pgfpathlineto{\pgfqpoint{5.257411in}{2.741849in}}%
\pgfpathlineto{\pgfqpoint{0.588387in}{2.741849in}}%
\pgfpathlineto{\pgfqpoint{0.588387in}{0.521603in}}%
\pgfpathclose%
\pgfusepath{fill}%
\end{pgfscope}%
\begin{pgfscope}%
\pgfsetbuttcap%
\pgfsetroundjoin%
\definecolor{currentfill}{rgb}{0.000000,0.000000,0.000000}%
\pgfsetfillcolor{currentfill}%
\pgfsetlinewidth{0.803000pt}%
\definecolor{currentstroke}{rgb}{0.000000,0.000000,0.000000}%
\pgfsetstrokecolor{currentstroke}%
\pgfsetdash{}{0pt}%
\pgfsys@defobject{currentmarker}{\pgfqpoint{0.000000in}{-0.048611in}}{\pgfqpoint{0.000000in}{0.000000in}}{%
\pgfpathmoveto{\pgfqpoint{0.000000in}{0.000000in}}%
\pgfpathlineto{\pgfqpoint{0.000000in}{-0.048611in}}%
\pgfusepath{stroke,fill}%
}%
\begin{pgfscope}%
\pgfsys@transformshift{0.654251in}{0.521603in}%
\pgfsys@useobject{currentmarker}{}%
\end{pgfscope}%
\end{pgfscope}%
\begin{pgfscope}%
\definecolor{textcolor}{rgb}{0.000000,0.000000,0.000000}%
\pgfsetstrokecolor{textcolor}%
\pgfsetfillcolor{textcolor}%
\pgftext[x=0.654251in,y=0.424381in,,top]{\color{textcolor}{\rmfamily\fontsize{10.000000}{12.000000}\selectfont\catcode`\^=\active\def^{\ifmmode\sp\else\^{}\fi}\catcode`\%=\active\def%{\%}$\mathdefault{4}$}}%
\end{pgfscope}%
\begin{pgfscope}%
\pgfsetbuttcap%
\pgfsetroundjoin%
\definecolor{currentfill}{rgb}{0.000000,0.000000,0.000000}%
\pgfsetfillcolor{currentfill}%
\pgfsetlinewidth{0.803000pt}%
\definecolor{currentstroke}{rgb}{0.000000,0.000000,0.000000}%
\pgfsetstrokecolor{currentstroke}%
\pgfsetdash{}{0pt}%
\pgfsys@defobject{currentmarker}{\pgfqpoint{0.000000in}{-0.048611in}}{\pgfqpoint{0.000000in}{0.000000in}}{%
\pgfpathmoveto{\pgfqpoint{0.000000in}{0.000000in}}%
\pgfpathlineto{\pgfqpoint{0.000000in}{-0.048611in}}%
\pgfusepath{stroke,fill}%
}%
\begin{pgfscope}%
\pgfsys@transformshift{1.239709in}{0.521603in}%
\pgfsys@useobject{currentmarker}{}%
\end{pgfscope}%
\end{pgfscope}%
\begin{pgfscope}%
\definecolor{textcolor}{rgb}{0.000000,0.000000,0.000000}%
\pgfsetstrokecolor{textcolor}%
\pgfsetfillcolor{textcolor}%
\pgftext[x=1.239709in,y=0.424381in,,top]{\color{textcolor}{\rmfamily\fontsize{10.000000}{12.000000}\selectfont\catcode`\^=\active\def^{\ifmmode\sp\else\^{}\fi}\catcode`\%=\active\def%{\%}$\mathdefault{8}$}}%
\end{pgfscope}%
\begin{pgfscope}%
\pgfsetbuttcap%
\pgfsetroundjoin%
\definecolor{currentfill}{rgb}{0.000000,0.000000,0.000000}%
\pgfsetfillcolor{currentfill}%
\pgfsetlinewidth{0.803000pt}%
\definecolor{currentstroke}{rgb}{0.000000,0.000000,0.000000}%
\pgfsetstrokecolor{currentstroke}%
\pgfsetdash{}{0pt}%
\pgfsys@defobject{currentmarker}{\pgfqpoint{0.000000in}{-0.048611in}}{\pgfqpoint{0.000000in}{0.000000in}}{%
\pgfpathmoveto{\pgfqpoint{0.000000in}{0.000000in}}%
\pgfpathlineto{\pgfqpoint{0.000000in}{-0.048611in}}%
\pgfusepath{stroke,fill}%
}%
\begin{pgfscope}%
\pgfsys@transformshift{1.825166in}{0.521603in}%
\pgfsys@useobject{currentmarker}{}%
\end{pgfscope}%
\end{pgfscope}%
\begin{pgfscope}%
\definecolor{textcolor}{rgb}{0.000000,0.000000,0.000000}%
\pgfsetstrokecolor{textcolor}%
\pgfsetfillcolor{textcolor}%
\pgftext[x=1.825166in,y=0.424381in,,top]{\color{textcolor}{\rmfamily\fontsize{10.000000}{12.000000}\selectfont\catcode`\^=\active\def^{\ifmmode\sp\else\^{}\fi}\catcode`\%=\active\def%{\%}$\mathdefault{12}$}}%
\end{pgfscope}%
\begin{pgfscope}%
\pgfsetbuttcap%
\pgfsetroundjoin%
\definecolor{currentfill}{rgb}{0.000000,0.000000,0.000000}%
\pgfsetfillcolor{currentfill}%
\pgfsetlinewidth{0.803000pt}%
\definecolor{currentstroke}{rgb}{0.000000,0.000000,0.000000}%
\pgfsetstrokecolor{currentstroke}%
\pgfsetdash{}{0pt}%
\pgfsys@defobject{currentmarker}{\pgfqpoint{0.000000in}{-0.048611in}}{\pgfqpoint{0.000000in}{0.000000in}}{%
\pgfpathmoveto{\pgfqpoint{0.000000in}{0.000000in}}%
\pgfpathlineto{\pgfqpoint{0.000000in}{-0.048611in}}%
\pgfusepath{stroke,fill}%
}%
\begin{pgfscope}%
\pgfsys@transformshift{2.410624in}{0.521603in}%
\pgfsys@useobject{currentmarker}{}%
\end{pgfscope}%
\end{pgfscope}%
\begin{pgfscope}%
\definecolor{textcolor}{rgb}{0.000000,0.000000,0.000000}%
\pgfsetstrokecolor{textcolor}%
\pgfsetfillcolor{textcolor}%
\pgftext[x=2.410624in,y=0.424381in,,top]{\color{textcolor}{\rmfamily\fontsize{10.000000}{12.000000}\selectfont\catcode`\^=\active\def^{\ifmmode\sp\else\^{}\fi}\catcode`\%=\active\def%{\%}$\mathdefault{16}$}}%
\end{pgfscope}%
\begin{pgfscope}%
\pgfsetbuttcap%
\pgfsetroundjoin%
\definecolor{currentfill}{rgb}{0.000000,0.000000,0.000000}%
\pgfsetfillcolor{currentfill}%
\pgfsetlinewidth{0.803000pt}%
\definecolor{currentstroke}{rgb}{0.000000,0.000000,0.000000}%
\pgfsetstrokecolor{currentstroke}%
\pgfsetdash{}{0pt}%
\pgfsys@defobject{currentmarker}{\pgfqpoint{0.000000in}{-0.048611in}}{\pgfqpoint{0.000000in}{0.000000in}}{%
\pgfpathmoveto{\pgfqpoint{0.000000in}{0.000000in}}%
\pgfpathlineto{\pgfqpoint{0.000000in}{-0.048611in}}%
\pgfusepath{stroke,fill}%
}%
\begin{pgfscope}%
\pgfsys@transformshift{2.996081in}{0.521603in}%
\pgfsys@useobject{currentmarker}{}%
\end{pgfscope}%
\end{pgfscope}%
\begin{pgfscope}%
\definecolor{textcolor}{rgb}{0.000000,0.000000,0.000000}%
\pgfsetstrokecolor{textcolor}%
\pgfsetfillcolor{textcolor}%
\pgftext[x=2.996081in,y=0.424381in,,top]{\color{textcolor}{\rmfamily\fontsize{10.000000}{12.000000}\selectfont\catcode`\^=\active\def^{\ifmmode\sp\else\^{}\fi}\catcode`\%=\active\def%{\%}$\mathdefault{20}$}}%
\end{pgfscope}%
\begin{pgfscope}%
\pgfsetbuttcap%
\pgfsetroundjoin%
\definecolor{currentfill}{rgb}{0.000000,0.000000,0.000000}%
\pgfsetfillcolor{currentfill}%
\pgfsetlinewidth{0.803000pt}%
\definecolor{currentstroke}{rgb}{0.000000,0.000000,0.000000}%
\pgfsetstrokecolor{currentstroke}%
\pgfsetdash{}{0pt}%
\pgfsys@defobject{currentmarker}{\pgfqpoint{0.000000in}{-0.048611in}}{\pgfqpoint{0.000000in}{0.000000in}}{%
\pgfpathmoveto{\pgfqpoint{0.000000in}{0.000000in}}%
\pgfpathlineto{\pgfqpoint{0.000000in}{-0.048611in}}%
\pgfusepath{stroke,fill}%
}%
\begin{pgfscope}%
\pgfsys@transformshift{3.581539in}{0.521603in}%
\pgfsys@useobject{currentmarker}{}%
\end{pgfscope}%
\end{pgfscope}%
\begin{pgfscope}%
\definecolor{textcolor}{rgb}{0.000000,0.000000,0.000000}%
\pgfsetstrokecolor{textcolor}%
\pgfsetfillcolor{textcolor}%
\pgftext[x=3.581539in,y=0.424381in,,top]{\color{textcolor}{\rmfamily\fontsize{10.000000}{12.000000}\selectfont\catcode`\^=\active\def^{\ifmmode\sp\else\^{}\fi}\catcode`\%=\active\def%{\%}$\mathdefault{24}$}}%
\end{pgfscope}%
\begin{pgfscope}%
\pgfsetbuttcap%
\pgfsetroundjoin%
\definecolor{currentfill}{rgb}{0.000000,0.000000,0.000000}%
\pgfsetfillcolor{currentfill}%
\pgfsetlinewidth{0.803000pt}%
\definecolor{currentstroke}{rgb}{0.000000,0.000000,0.000000}%
\pgfsetstrokecolor{currentstroke}%
\pgfsetdash{}{0pt}%
\pgfsys@defobject{currentmarker}{\pgfqpoint{0.000000in}{-0.048611in}}{\pgfqpoint{0.000000in}{0.000000in}}{%
\pgfpathmoveto{\pgfqpoint{0.000000in}{0.000000in}}%
\pgfpathlineto{\pgfqpoint{0.000000in}{-0.048611in}}%
\pgfusepath{stroke,fill}%
}%
\begin{pgfscope}%
\pgfsys@transformshift{4.166997in}{0.521603in}%
\pgfsys@useobject{currentmarker}{}%
\end{pgfscope}%
\end{pgfscope}%
\begin{pgfscope}%
\definecolor{textcolor}{rgb}{0.000000,0.000000,0.000000}%
\pgfsetstrokecolor{textcolor}%
\pgfsetfillcolor{textcolor}%
\pgftext[x=4.166997in,y=0.424381in,,top]{\color{textcolor}{\rmfamily\fontsize{10.000000}{12.000000}\selectfont\catcode`\^=\active\def^{\ifmmode\sp\else\^{}\fi}\catcode`\%=\active\def%{\%}$\mathdefault{28}$}}%
\end{pgfscope}%
\begin{pgfscope}%
\pgfsetbuttcap%
\pgfsetroundjoin%
\definecolor{currentfill}{rgb}{0.000000,0.000000,0.000000}%
\pgfsetfillcolor{currentfill}%
\pgfsetlinewidth{0.803000pt}%
\definecolor{currentstroke}{rgb}{0.000000,0.000000,0.000000}%
\pgfsetstrokecolor{currentstroke}%
\pgfsetdash{}{0pt}%
\pgfsys@defobject{currentmarker}{\pgfqpoint{0.000000in}{-0.048611in}}{\pgfqpoint{0.000000in}{0.000000in}}{%
\pgfpathmoveto{\pgfqpoint{0.000000in}{0.000000in}}%
\pgfpathlineto{\pgfqpoint{0.000000in}{-0.048611in}}%
\pgfusepath{stroke,fill}%
}%
\begin{pgfscope}%
\pgfsys@transformshift{4.752454in}{0.521603in}%
\pgfsys@useobject{currentmarker}{}%
\end{pgfscope}%
\end{pgfscope}%
\begin{pgfscope}%
\definecolor{textcolor}{rgb}{0.000000,0.000000,0.000000}%
\pgfsetstrokecolor{textcolor}%
\pgfsetfillcolor{textcolor}%
\pgftext[x=4.752454in,y=0.424381in,,top]{\color{textcolor}{\rmfamily\fontsize{10.000000}{12.000000}\selectfont\catcode`\^=\active\def^{\ifmmode\sp\else\^{}\fi}\catcode`\%=\active\def%{\%}$\mathdefault{32}$}}%
\end{pgfscope}%
\begin{pgfscope}%
\definecolor{textcolor}{rgb}{0.000000,0.000000,0.000000}%
\pgfsetstrokecolor{textcolor}%
\pgfsetfillcolor{textcolor}%
\pgftext[x=2.922899in,y=0.234413in,,top]{\color{textcolor}{\rmfamily\fontsize{10.000000}{12.000000}\selectfont\catcode`\^=\active\def^{\ifmmode\sp\else\^{}\fi}\catcode`\%=\active\def%{\%}Monochromatic classes}}%
\end{pgfscope}%
\begin{pgfscope}%
\pgfsetbuttcap%
\pgfsetroundjoin%
\definecolor{currentfill}{rgb}{0.000000,0.000000,0.000000}%
\pgfsetfillcolor{currentfill}%
\pgfsetlinewidth{0.803000pt}%
\definecolor{currentstroke}{rgb}{0.000000,0.000000,0.000000}%
\pgfsetstrokecolor{currentstroke}%
\pgfsetdash{}{0pt}%
\pgfsys@defobject{currentmarker}{\pgfqpoint{-0.048611in}{0.000000in}}{\pgfqpoint{-0.000000in}{0.000000in}}{%
\pgfpathmoveto{\pgfqpoint{-0.000000in}{0.000000in}}%
\pgfpathlineto{\pgfqpoint{-0.048611in}{0.000000in}}%
\pgfusepath{stroke,fill}%
}%
\begin{pgfscope}%
\pgfsys@transformshift{0.588387in}{0.546489in}%
\pgfsys@useobject{currentmarker}{}%
\end{pgfscope}%
\end{pgfscope}%
\begin{pgfscope}%
\definecolor{textcolor}{rgb}{0.000000,0.000000,0.000000}%
\pgfsetstrokecolor{textcolor}%
\pgfsetfillcolor{textcolor}%
\pgftext[x=0.289968in, y=0.493727in, left, base]{\color{textcolor}{\rmfamily\fontsize{10.000000}{12.000000}\selectfont\catcode`\^=\active\def^{\ifmmode\sp\else\^{}\fi}\catcode`\%=\active\def%{\%}$\mathdefault{10^{1}}$}}%
\end{pgfscope}%
\begin{pgfscope}%
\pgfsetbuttcap%
\pgfsetroundjoin%
\definecolor{currentfill}{rgb}{0.000000,0.000000,0.000000}%
\pgfsetfillcolor{currentfill}%
\pgfsetlinewidth{0.803000pt}%
\definecolor{currentstroke}{rgb}{0.000000,0.000000,0.000000}%
\pgfsetstrokecolor{currentstroke}%
\pgfsetdash{}{0pt}%
\pgfsys@defobject{currentmarker}{\pgfqpoint{-0.048611in}{0.000000in}}{\pgfqpoint{-0.000000in}{0.000000in}}{%
\pgfpathmoveto{\pgfqpoint{-0.000000in}{0.000000in}}%
\pgfpathlineto{\pgfqpoint{-0.048611in}{0.000000in}}%
\pgfusepath{stroke,fill}%
}%
\begin{pgfscope}%
\pgfsys@transformshift{0.588387in}{0.918989in}%
\pgfsys@useobject{currentmarker}{}%
\end{pgfscope}%
\end{pgfscope}%
\begin{pgfscope}%
\definecolor{textcolor}{rgb}{0.000000,0.000000,0.000000}%
\pgfsetstrokecolor{textcolor}%
\pgfsetfillcolor{textcolor}%
\pgftext[x=0.289968in, y=0.866227in, left, base]{\color{textcolor}{\rmfamily\fontsize{10.000000}{12.000000}\selectfont\catcode`\^=\active\def^{\ifmmode\sp\else\^{}\fi}\catcode`\%=\active\def%{\%}$\mathdefault{10^{2}}$}}%
\end{pgfscope}%
\begin{pgfscope}%
\pgfsetbuttcap%
\pgfsetroundjoin%
\definecolor{currentfill}{rgb}{0.000000,0.000000,0.000000}%
\pgfsetfillcolor{currentfill}%
\pgfsetlinewidth{0.803000pt}%
\definecolor{currentstroke}{rgb}{0.000000,0.000000,0.000000}%
\pgfsetstrokecolor{currentstroke}%
\pgfsetdash{}{0pt}%
\pgfsys@defobject{currentmarker}{\pgfqpoint{-0.048611in}{0.000000in}}{\pgfqpoint{-0.000000in}{0.000000in}}{%
\pgfpathmoveto{\pgfqpoint{-0.000000in}{0.000000in}}%
\pgfpathlineto{\pgfqpoint{-0.048611in}{0.000000in}}%
\pgfusepath{stroke,fill}%
}%
\begin{pgfscope}%
\pgfsys@transformshift{0.588387in}{1.291489in}%
\pgfsys@useobject{currentmarker}{}%
\end{pgfscope}%
\end{pgfscope}%
\begin{pgfscope}%
\definecolor{textcolor}{rgb}{0.000000,0.000000,0.000000}%
\pgfsetstrokecolor{textcolor}%
\pgfsetfillcolor{textcolor}%
\pgftext[x=0.289968in, y=1.238727in, left, base]{\color{textcolor}{\rmfamily\fontsize{10.000000}{12.000000}\selectfont\catcode`\^=\active\def^{\ifmmode\sp\else\^{}\fi}\catcode`\%=\active\def%{\%}$\mathdefault{10^{3}}$}}%
\end{pgfscope}%
\begin{pgfscope}%
\pgfsetbuttcap%
\pgfsetroundjoin%
\definecolor{currentfill}{rgb}{0.000000,0.000000,0.000000}%
\pgfsetfillcolor{currentfill}%
\pgfsetlinewidth{0.803000pt}%
\definecolor{currentstroke}{rgb}{0.000000,0.000000,0.000000}%
\pgfsetstrokecolor{currentstroke}%
\pgfsetdash{}{0pt}%
\pgfsys@defobject{currentmarker}{\pgfqpoint{-0.048611in}{0.000000in}}{\pgfqpoint{-0.000000in}{0.000000in}}{%
\pgfpathmoveto{\pgfqpoint{-0.000000in}{0.000000in}}%
\pgfpathlineto{\pgfqpoint{-0.048611in}{0.000000in}}%
\pgfusepath{stroke,fill}%
}%
\begin{pgfscope}%
\pgfsys@transformshift{0.588387in}{1.663989in}%
\pgfsys@useobject{currentmarker}{}%
\end{pgfscope}%
\end{pgfscope}%
\begin{pgfscope}%
\definecolor{textcolor}{rgb}{0.000000,0.000000,0.000000}%
\pgfsetstrokecolor{textcolor}%
\pgfsetfillcolor{textcolor}%
\pgftext[x=0.289968in, y=1.611227in, left, base]{\color{textcolor}{\rmfamily\fontsize{10.000000}{12.000000}\selectfont\catcode`\^=\active\def^{\ifmmode\sp\else\^{}\fi}\catcode`\%=\active\def%{\%}$\mathdefault{10^{4}}$}}%
\end{pgfscope}%
\begin{pgfscope}%
\pgfsetbuttcap%
\pgfsetroundjoin%
\definecolor{currentfill}{rgb}{0.000000,0.000000,0.000000}%
\pgfsetfillcolor{currentfill}%
\pgfsetlinewidth{0.803000pt}%
\definecolor{currentstroke}{rgb}{0.000000,0.000000,0.000000}%
\pgfsetstrokecolor{currentstroke}%
\pgfsetdash{}{0pt}%
\pgfsys@defobject{currentmarker}{\pgfqpoint{-0.048611in}{0.000000in}}{\pgfqpoint{-0.000000in}{0.000000in}}{%
\pgfpathmoveto{\pgfqpoint{-0.000000in}{0.000000in}}%
\pgfpathlineto{\pgfqpoint{-0.048611in}{0.000000in}}%
\pgfusepath{stroke,fill}%
}%
\begin{pgfscope}%
\pgfsys@transformshift{0.588387in}{2.036488in}%
\pgfsys@useobject{currentmarker}{}%
\end{pgfscope}%
\end{pgfscope}%
\begin{pgfscope}%
\definecolor{textcolor}{rgb}{0.000000,0.000000,0.000000}%
\pgfsetstrokecolor{textcolor}%
\pgfsetfillcolor{textcolor}%
\pgftext[x=0.289968in, y=1.983727in, left, base]{\color{textcolor}{\rmfamily\fontsize{10.000000}{12.000000}\selectfont\catcode`\^=\active\def^{\ifmmode\sp\else\^{}\fi}\catcode`\%=\active\def%{\%}$\mathdefault{10^{5}}$}}%
\end{pgfscope}%
\begin{pgfscope}%
\pgfsetbuttcap%
\pgfsetroundjoin%
\definecolor{currentfill}{rgb}{0.000000,0.000000,0.000000}%
\pgfsetfillcolor{currentfill}%
\pgfsetlinewidth{0.803000pt}%
\definecolor{currentstroke}{rgb}{0.000000,0.000000,0.000000}%
\pgfsetstrokecolor{currentstroke}%
\pgfsetdash{}{0pt}%
\pgfsys@defobject{currentmarker}{\pgfqpoint{-0.048611in}{0.000000in}}{\pgfqpoint{-0.000000in}{0.000000in}}{%
\pgfpathmoveto{\pgfqpoint{-0.000000in}{0.000000in}}%
\pgfpathlineto{\pgfqpoint{-0.048611in}{0.000000in}}%
\pgfusepath{stroke,fill}%
}%
\begin{pgfscope}%
\pgfsys@transformshift{0.588387in}{2.408988in}%
\pgfsys@useobject{currentmarker}{}%
\end{pgfscope}%
\end{pgfscope}%
\begin{pgfscope}%
\definecolor{textcolor}{rgb}{0.000000,0.000000,0.000000}%
\pgfsetstrokecolor{textcolor}%
\pgfsetfillcolor{textcolor}%
\pgftext[x=0.289968in, y=2.356227in, left, base]{\color{textcolor}{\rmfamily\fontsize{10.000000}{12.000000}\selectfont\catcode`\^=\active\def^{\ifmmode\sp\else\^{}\fi}\catcode`\%=\active\def%{\%}$\mathdefault{10^{6}}$}}%
\end{pgfscope}%
\begin{pgfscope}%
\pgfsetbuttcap%
\pgfsetroundjoin%
\definecolor{currentfill}{rgb}{0.000000,0.000000,0.000000}%
\pgfsetfillcolor{currentfill}%
\pgfsetlinewidth{0.602250pt}%
\definecolor{currentstroke}{rgb}{0.000000,0.000000,0.000000}%
\pgfsetstrokecolor{currentstroke}%
\pgfsetdash{}{0pt}%
\pgfsys@defobject{currentmarker}{\pgfqpoint{-0.027778in}{0.000000in}}{\pgfqpoint{-0.000000in}{0.000000in}}{%
\pgfpathmoveto{\pgfqpoint{-0.000000in}{0.000000in}}%
\pgfpathlineto{\pgfqpoint{-0.027778in}{0.000000in}}%
\pgfusepath{stroke,fill}%
}%
\begin{pgfscope}%
\pgfsys@transformshift{0.588387in}{0.529444in}%
\pgfsys@useobject{currentmarker}{}%
\end{pgfscope}%
\end{pgfscope}%
\begin{pgfscope}%
\pgfsetbuttcap%
\pgfsetroundjoin%
\definecolor{currentfill}{rgb}{0.000000,0.000000,0.000000}%
\pgfsetfillcolor{currentfill}%
\pgfsetlinewidth{0.602250pt}%
\definecolor{currentstroke}{rgb}{0.000000,0.000000,0.000000}%
\pgfsetstrokecolor{currentstroke}%
\pgfsetdash{}{0pt}%
\pgfsys@defobject{currentmarker}{\pgfqpoint{-0.027778in}{0.000000in}}{\pgfqpoint{-0.000000in}{0.000000in}}{%
\pgfpathmoveto{\pgfqpoint{-0.000000in}{0.000000in}}%
\pgfpathlineto{\pgfqpoint{-0.027778in}{0.000000in}}%
\pgfusepath{stroke,fill}%
}%
\begin{pgfscope}%
\pgfsys@transformshift{0.588387in}{0.658623in}%
\pgfsys@useobject{currentmarker}{}%
\end{pgfscope}%
\end{pgfscope}%
\begin{pgfscope}%
\pgfsetbuttcap%
\pgfsetroundjoin%
\definecolor{currentfill}{rgb}{0.000000,0.000000,0.000000}%
\pgfsetfillcolor{currentfill}%
\pgfsetlinewidth{0.602250pt}%
\definecolor{currentstroke}{rgb}{0.000000,0.000000,0.000000}%
\pgfsetstrokecolor{currentstroke}%
\pgfsetdash{}{0pt}%
\pgfsys@defobject{currentmarker}{\pgfqpoint{-0.027778in}{0.000000in}}{\pgfqpoint{-0.000000in}{0.000000in}}{%
\pgfpathmoveto{\pgfqpoint{-0.000000in}{0.000000in}}%
\pgfpathlineto{\pgfqpoint{-0.027778in}{0.000000in}}%
\pgfusepath{stroke,fill}%
}%
\begin{pgfscope}%
\pgfsys@transformshift{0.588387in}{0.724217in}%
\pgfsys@useobject{currentmarker}{}%
\end{pgfscope}%
\end{pgfscope}%
\begin{pgfscope}%
\pgfsetbuttcap%
\pgfsetroundjoin%
\definecolor{currentfill}{rgb}{0.000000,0.000000,0.000000}%
\pgfsetfillcolor{currentfill}%
\pgfsetlinewidth{0.602250pt}%
\definecolor{currentstroke}{rgb}{0.000000,0.000000,0.000000}%
\pgfsetstrokecolor{currentstroke}%
\pgfsetdash{}{0pt}%
\pgfsys@defobject{currentmarker}{\pgfqpoint{-0.027778in}{0.000000in}}{\pgfqpoint{-0.000000in}{0.000000in}}{%
\pgfpathmoveto{\pgfqpoint{-0.000000in}{0.000000in}}%
\pgfpathlineto{\pgfqpoint{-0.027778in}{0.000000in}}%
\pgfusepath{stroke,fill}%
}%
\begin{pgfscope}%
\pgfsys@transformshift{0.588387in}{0.770756in}%
\pgfsys@useobject{currentmarker}{}%
\end{pgfscope}%
\end{pgfscope}%
\begin{pgfscope}%
\pgfsetbuttcap%
\pgfsetroundjoin%
\definecolor{currentfill}{rgb}{0.000000,0.000000,0.000000}%
\pgfsetfillcolor{currentfill}%
\pgfsetlinewidth{0.602250pt}%
\definecolor{currentstroke}{rgb}{0.000000,0.000000,0.000000}%
\pgfsetstrokecolor{currentstroke}%
\pgfsetdash{}{0pt}%
\pgfsys@defobject{currentmarker}{\pgfqpoint{-0.027778in}{0.000000in}}{\pgfqpoint{-0.000000in}{0.000000in}}{%
\pgfpathmoveto{\pgfqpoint{-0.000000in}{0.000000in}}%
\pgfpathlineto{\pgfqpoint{-0.027778in}{0.000000in}}%
\pgfusepath{stroke,fill}%
}%
\begin{pgfscope}%
\pgfsys@transformshift{0.588387in}{0.806855in}%
\pgfsys@useobject{currentmarker}{}%
\end{pgfscope}%
\end{pgfscope}%
\begin{pgfscope}%
\pgfsetbuttcap%
\pgfsetroundjoin%
\definecolor{currentfill}{rgb}{0.000000,0.000000,0.000000}%
\pgfsetfillcolor{currentfill}%
\pgfsetlinewidth{0.602250pt}%
\definecolor{currentstroke}{rgb}{0.000000,0.000000,0.000000}%
\pgfsetstrokecolor{currentstroke}%
\pgfsetdash{}{0pt}%
\pgfsys@defobject{currentmarker}{\pgfqpoint{-0.027778in}{0.000000in}}{\pgfqpoint{-0.000000in}{0.000000in}}{%
\pgfpathmoveto{\pgfqpoint{-0.000000in}{0.000000in}}%
\pgfpathlineto{\pgfqpoint{-0.027778in}{0.000000in}}%
\pgfusepath{stroke,fill}%
}%
\begin{pgfscope}%
\pgfsys@transformshift{0.588387in}{0.836350in}%
\pgfsys@useobject{currentmarker}{}%
\end{pgfscope}%
\end{pgfscope}%
\begin{pgfscope}%
\pgfsetbuttcap%
\pgfsetroundjoin%
\definecolor{currentfill}{rgb}{0.000000,0.000000,0.000000}%
\pgfsetfillcolor{currentfill}%
\pgfsetlinewidth{0.602250pt}%
\definecolor{currentstroke}{rgb}{0.000000,0.000000,0.000000}%
\pgfsetstrokecolor{currentstroke}%
\pgfsetdash{}{0pt}%
\pgfsys@defobject{currentmarker}{\pgfqpoint{-0.027778in}{0.000000in}}{\pgfqpoint{-0.000000in}{0.000000in}}{%
\pgfpathmoveto{\pgfqpoint{-0.000000in}{0.000000in}}%
\pgfpathlineto{\pgfqpoint{-0.027778in}{0.000000in}}%
\pgfusepath{stroke,fill}%
}%
\begin{pgfscope}%
\pgfsys@transformshift{0.588387in}{0.861288in}%
\pgfsys@useobject{currentmarker}{}%
\end{pgfscope}%
\end{pgfscope}%
\begin{pgfscope}%
\pgfsetbuttcap%
\pgfsetroundjoin%
\definecolor{currentfill}{rgb}{0.000000,0.000000,0.000000}%
\pgfsetfillcolor{currentfill}%
\pgfsetlinewidth{0.602250pt}%
\definecolor{currentstroke}{rgb}{0.000000,0.000000,0.000000}%
\pgfsetstrokecolor{currentstroke}%
\pgfsetdash{}{0pt}%
\pgfsys@defobject{currentmarker}{\pgfqpoint{-0.027778in}{0.000000in}}{\pgfqpoint{-0.000000in}{0.000000in}}{%
\pgfpathmoveto{\pgfqpoint{-0.000000in}{0.000000in}}%
\pgfpathlineto{\pgfqpoint{-0.027778in}{0.000000in}}%
\pgfusepath{stroke,fill}%
}%
\begin{pgfscope}%
\pgfsys@transformshift{0.588387in}{0.882890in}%
\pgfsys@useobject{currentmarker}{}%
\end{pgfscope}%
\end{pgfscope}%
\begin{pgfscope}%
\pgfsetbuttcap%
\pgfsetroundjoin%
\definecolor{currentfill}{rgb}{0.000000,0.000000,0.000000}%
\pgfsetfillcolor{currentfill}%
\pgfsetlinewidth{0.602250pt}%
\definecolor{currentstroke}{rgb}{0.000000,0.000000,0.000000}%
\pgfsetstrokecolor{currentstroke}%
\pgfsetdash{}{0pt}%
\pgfsys@defobject{currentmarker}{\pgfqpoint{-0.027778in}{0.000000in}}{\pgfqpoint{-0.000000in}{0.000000in}}{%
\pgfpathmoveto{\pgfqpoint{-0.000000in}{0.000000in}}%
\pgfpathlineto{\pgfqpoint{-0.027778in}{0.000000in}}%
\pgfusepath{stroke,fill}%
}%
\begin{pgfscope}%
\pgfsys@transformshift{0.588387in}{0.901944in}%
\pgfsys@useobject{currentmarker}{}%
\end{pgfscope}%
\end{pgfscope}%
\begin{pgfscope}%
\pgfsetbuttcap%
\pgfsetroundjoin%
\definecolor{currentfill}{rgb}{0.000000,0.000000,0.000000}%
\pgfsetfillcolor{currentfill}%
\pgfsetlinewidth{0.602250pt}%
\definecolor{currentstroke}{rgb}{0.000000,0.000000,0.000000}%
\pgfsetstrokecolor{currentstroke}%
\pgfsetdash{}{0pt}%
\pgfsys@defobject{currentmarker}{\pgfqpoint{-0.027778in}{0.000000in}}{\pgfqpoint{-0.000000in}{0.000000in}}{%
\pgfpathmoveto{\pgfqpoint{-0.000000in}{0.000000in}}%
\pgfpathlineto{\pgfqpoint{-0.027778in}{0.000000in}}%
\pgfusepath{stroke,fill}%
}%
\begin{pgfscope}%
\pgfsys@transformshift{0.588387in}{1.031122in}%
\pgfsys@useobject{currentmarker}{}%
\end{pgfscope}%
\end{pgfscope}%
\begin{pgfscope}%
\pgfsetbuttcap%
\pgfsetroundjoin%
\definecolor{currentfill}{rgb}{0.000000,0.000000,0.000000}%
\pgfsetfillcolor{currentfill}%
\pgfsetlinewidth{0.602250pt}%
\definecolor{currentstroke}{rgb}{0.000000,0.000000,0.000000}%
\pgfsetstrokecolor{currentstroke}%
\pgfsetdash{}{0pt}%
\pgfsys@defobject{currentmarker}{\pgfqpoint{-0.027778in}{0.000000in}}{\pgfqpoint{-0.000000in}{0.000000in}}{%
\pgfpathmoveto{\pgfqpoint{-0.000000in}{0.000000in}}%
\pgfpathlineto{\pgfqpoint{-0.027778in}{0.000000in}}%
\pgfusepath{stroke,fill}%
}%
\begin{pgfscope}%
\pgfsys@transformshift{0.588387in}{1.096716in}%
\pgfsys@useobject{currentmarker}{}%
\end{pgfscope}%
\end{pgfscope}%
\begin{pgfscope}%
\pgfsetbuttcap%
\pgfsetroundjoin%
\definecolor{currentfill}{rgb}{0.000000,0.000000,0.000000}%
\pgfsetfillcolor{currentfill}%
\pgfsetlinewidth{0.602250pt}%
\definecolor{currentstroke}{rgb}{0.000000,0.000000,0.000000}%
\pgfsetstrokecolor{currentstroke}%
\pgfsetdash{}{0pt}%
\pgfsys@defobject{currentmarker}{\pgfqpoint{-0.027778in}{0.000000in}}{\pgfqpoint{-0.000000in}{0.000000in}}{%
\pgfpathmoveto{\pgfqpoint{-0.000000in}{0.000000in}}%
\pgfpathlineto{\pgfqpoint{-0.027778in}{0.000000in}}%
\pgfusepath{stroke,fill}%
}%
\begin{pgfscope}%
\pgfsys@transformshift{0.588387in}{1.143256in}%
\pgfsys@useobject{currentmarker}{}%
\end{pgfscope}%
\end{pgfscope}%
\begin{pgfscope}%
\pgfsetbuttcap%
\pgfsetroundjoin%
\definecolor{currentfill}{rgb}{0.000000,0.000000,0.000000}%
\pgfsetfillcolor{currentfill}%
\pgfsetlinewidth{0.602250pt}%
\definecolor{currentstroke}{rgb}{0.000000,0.000000,0.000000}%
\pgfsetstrokecolor{currentstroke}%
\pgfsetdash{}{0pt}%
\pgfsys@defobject{currentmarker}{\pgfqpoint{-0.027778in}{0.000000in}}{\pgfqpoint{-0.000000in}{0.000000in}}{%
\pgfpathmoveto{\pgfqpoint{-0.000000in}{0.000000in}}%
\pgfpathlineto{\pgfqpoint{-0.027778in}{0.000000in}}%
\pgfusepath{stroke,fill}%
}%
\begin{pgfscope}%
\pgfsys@transformshift{0.588387in}{1.179355in}%
\pgfsys@useobject{currentmarker}{}%
\end{pgfscope}%
\end{pgfscope}%
\begin{pgfscope}%
\pgfsetbuttcap%
\pgfsetroundjoin%
\definecolor{currentfill}{rgb}{0.000000,0.000000,0.000000}%
\pgfsetfillcolor{currentfill}%
\pgfsetlinewidth{0.602250pt}%
\definecolor{currentstroke}{rgb}{0.000000,0.000000,0.000000}%
\pgfsetstrokecolor{currentstroke}%
\pgfsetdash{}{0pt}%
\pgfsys@defobject{currentmarker}{\pgfqpoint{-0.027778in}{0.000000in}}{\pgfqpoint{-0.000000in}{0.000000in}}{%
\pgfpathmoveto{\pgfqpoint{-0.000000in}{0.000000in}}%
\pgfpathlineto{\pgfqpoint{-0.027778in}{0.000000in}}%
\pgfusepath{stroke,fill}%
}%
\begin{pgfscope}%
\pgfsys@transformshift{0.588387in}{1.208850in}%
\pgfsys@useobject{currentmarker}{}%
\end{pgfscope}%
\end{pgfscope}%
\begin{pgfscope}%
\pgfsetbuttcap%
\pgfsetroundjoin%
\definecolor{currentfill}{rgb}{0.000000,0.000000,0.000000}%
\pgfsetfillcolor{currentfill}%
\pgfsetlinewidth{0.602250pt}%
\definecolor{currentstroke}{rgb}{0.000000,0.000000,0.000000}%
\pgfsetstrokecolor{currentstroke}%
\pgfsetdash{}{0pt}%
\pgfsys@defobject{currentmarker}{\pgfqpoint{-0.027778in}{0.000000in}}{\pgfqpoint{-0.000000in}{0.000000in}}{%
\pgfpathmoveto{\pgfqpoint{-0.000000in}{0.000000in}}%
\pgfpathlineto{\pgfqpoint{-0.027778in}{0.000000in}}%
\pgfusepath{stroke,fill}%
}%
\begin{pgfscope}%
\pgfsys@transformshift{0.588387in}{1.233788in}%
\pgfsys@useobject{currentmarker}{}%
\end{pgfscope}%
\end{pgfscope}%
\begin{pgfscope}%
\pgfsetbuttcap%
\pgfsetroundjoin%
\definecolor{currentfill}{rgb}{0.000000,0.000000,0.000000}%
\pgfsetfillcolor{currentfill}%
\pgfsetlinewidth{0.602250pt}%
\definecolor{currentstroke}{rgb}{0.000000,0.000000,0.000000}%
\pgfsetstrokecolor{currentstroke}%
\pgfsetdash{}{0pt}%
\pgfsys@defobject{currentmarker}{\pgfqpoint{-0.027778in}{0.000000in}}{\pgfqpoint{-0.000000in}{0.000000in}}{%
\pgfpathmoveto{\pgfqpoint{-0.000000in}{0.000000in}}%
\pgfpathlineto{\pgfqpoint{-0.027778in}{0.000000in}}%
\pgfusepath{stroke,fill}%
}%
\begin{pgfscope}%
\pgfsys@transformshift{0.588387in}{1.255390in}%
\pgfsys@useobject{currentmarker}{}%
\end{pgfscope}%
\end{pgfscope}%
\begin{pgfscope}%
\pgfsetbuttcap%
\pgfsetroundjoin%
\definecolor{currentfill}{rgb}{0.000000,0.000000,0.000000}%
\pgfsetfillcolor{currentfill}%
\pgfsetlinewidth{0.602250pt}%
\definecolor{currentstroke}{rgb}{0.000000,0.000000,0.000000}%
\pgfsetstrokecolor{currentstroke}%
\pgfsetdash{}{0pt}%
\pgfsys@defobject{currentmarker}{\pgfqpoint{-0.027778in}{0.000000in}}{\pgfqpoint{-0.000000in}{0.000000in}}{%
\pgfpathmoveto{\pgfqpoint{-0.000000in}{0.000000in}}%
\pgfpathlineto{\pgfqpoint{-0.027778in}{0.000000in}}%
\pgfusepath{stroke,fill}%
}%
\begin{pgfscope}%
\pgfsys@transformshift{0.588387in}{1.274444in}%
\pgfsys@useobject{currentmarker}{}%
\end{pgfscope}%
\end{pgfscope}%
\begin{pgfscope}%
\pgfsetbuttcap%
\pgfsetroundjoin%
\definecolor{currentfill}{rgb}{0.000000,0.000000,0.000000}%
\pgfsetfillcolor{currentfill}%
\pgfsetlinewidth{0.602250pt}%
\definecolor{currentstroke}{rgb}{0.000000,0.000000,0.000000}%
\pgfsetstrokecolor{currentstroke}%
\pgfsetdash{}{0pt}%
\pgfsys@defobject{currentmarker}{\pgfqpoint{-0.027778in}{0.000000in}}{\pgfqpoint{-0.000000in}{0.000000in}}{%
\pgfpathmoveto{\pgfqpoint{-0.000000in}{0.000000in}}%
\pgfpathlineto{\pgfqpoint{-0.027778in}{0.000000in}}%
\pgfusepath{stroke,fill}%
}%
\begin{pgfscope}%
\pgfsys@transformshift{0.588387in}{1.403622in}%
\pgfsys@useobject{currentmarker}{}%
\end{pgfscope}%
\end{pgfscope}%
\begin{pgfscope}%
\pgfsetbuttcap%
\pgfsetroundjoin%
\definecolor{currentfill}{rgb}{0.000000,0.000000,0.000000}%
\pgfsetfillcolor{currentfill}%
\pgfsetlinewidth{0.602250pt}%
\definecolor{currentstroke}{rgb}{0.000000,0.000000,0.000000}%
\pgfsetstrokecolor{currentstroke}%
\pgfsetdash{}{0pt}%
\pgfsys@defobject{currentmarker}{\pgfqpoint{-0.027778in}{0.000000in}}{\pgfqpoint{-0.000000in}{0.000000in}}{%
\pgfpathmoveto{\pgfqpoint{-0.000000in}{0.000000in}}%
\pgfpathlineto{\pgfqpoint{-0.027778in}{0.000000in}}%
\pgfusepath{stroke,fill}%
}%
\begin{pgfscope}%
\pgfsys@transformshift{0.588387in}{1.469216in}%
\pgfsys@useobject{currentmarker}{}%
\end{pgfscope}%
\end{pgfscope}%
\begin{pgfscope}%
\pgfsetbuttcap%
\pgfsetroundjoin%
\definecolor{currentfill}{rgb}{0.000000,0.000000,0.000000}%
\pgfsetfillcolor{currentfill}%
\pgfsetlinewidth{0.602250pt}%
\definecolor{currentstroke}{rgb}{0.000000,0.000000,0.000000}%
\pgfsetstrokecolor{currentstroke}%
\pgfsetdash{}{0pt}%
\pgfsys@defobject{currentmarker}{\pgfqpoint{-0.027778in}{0.000000in}}{\pgfqpoint{-0.000000in}{0.000000in}}{%
\pgfpathmoveto{\pgfqpoint{-0.000000in}{0.000000in}}%
\pgfpathlineto{\pgfqpoint{-0.027778in}{0.000000in}}%
\pgfusepath{stroke,fill}%
}%
\begin{pgfscope}%
\pgfsys@transformshift{0.588387in}{1.515756in}%
\pgfsys@useobject{currentmarker}{}%
\end{pgfscope}%
\end{pgfscope}%
\begin{pgfscope}%
\pgfsetbuttcap%
\pgfsetroundjoin%
\definecolor{currentfill}{rgb}{0.000000,0.000000,0.000000}%
\pgfsetfillcolor{currentfill}%
\pgfsetlinewidth{0.602250pt}%
\definecolor{currentstroke}{rgb}{0.000000,0.000000,0.000000}%
\pgfsetstrokecolor{currentstroke}%
\pgfsetdash{}{0pt}%
\pgfsys@defobject{currentmarker}{\pgfqpoint{-0.027778in}{0.000000in}}{\pgfqpoint{-0.000000in}{0.000000in}}{%
\pgfpathmoveto{\pgfqpoint{-0.000000in}{0.000000in}}%
\pgfpathlineto{\pgfqpoint{-0.027778in}{0.000000in}}%
\pgfusepath{stroke,fill}%
}%
\begin{pgfscope}%
\pgfsys@transformshift{0.588387in}{1.551855in}%
\pgfsys@useobject{currentmarker}{}%
\end{pgfscope}%
\end{pgfscope}%
\begin{pgfscope}%
\pgfsetbuttcap%
\pgfsetroundjoin%
\definecolor{currentfill}{rgb}{0.000000,0.000000,0.000000}%
\pgfsetfillcolor{currentfill}%
\pgfsetlinewidth{0.602250pt}%
\definecolor{currentstroke}{rgb}{0.000000,0.000000,0.000000}%
\pgfsetstrokecolor{currentstroke}%
\pgfsetdash{}{0pt}%
\pgfsys@defobject{currentmarker}{\pgfqpoint{-0.027778in}{0.000000in}}{\pgfqpoint{-0.000000in}{0.000000in}}{%
\pgfpathmoveto{\pgfqpoint{-0.000000in}{0.000000in}}%
\pgfpathlineto{\pgfqpoint{-0.027778in}{0.000000in}}%
\pgfusepath{stroke,fill}%
}%
\begin{pgfscope}%
\pgfsys@transformshift{0.588387in}{1.581350in}%
\pgfsys@useobject{currentmarker}{}%
\end{pgfscope}%
\end{pgfscope}%
\begin{pgfscope}%
\pgfsetbuttcap%
\pgfsetroundjoin%
\definecolor{currentfill}{rgb}{0.000000,0.000000,0.000000}%
\pgfsetfillcolor{currentfill}%
\pgfsetlinewidth{0.602250pt}%
\definecolor{currentstroke}{rgb}{0.000000,0.000000,0.000000}%
\pgfsetstrokecolor{currentstroke}%
\pgfsetdash{}{0pt}%
\pgfsys@defobject{currentmarker}{\pgfqpoint{-0.027778in}{0.000000in}}{\pgfqpoint{-0.000000in}{0.000000in}}{%
\pgfpathmoveto{\pgfqpoint{-0.000000in}{0.000000in}}%
\pgfpathlineto{\pgfqpoint{-0.027778in}{0.000000in}}%
\pgfusepath{stroke,fill}%
}%
\begin{pgfscope}%
\pgfsys@transformshift{0.588387in}{1.606288in}%
\pgfsys@useobject{currentmarker}{}%
\end{pgfscope}%
\end{pgfscope}%
\begin{pgfscope}%
\pgfsetbuttcap%
\pgfsetroundjoin%
\definecolor{currentfill}{rgb}{0.000000,0.000000,0.000000}%
\pgfsetfillcolor{currentfill}%
\pgfsetlinewidth{0.602250pt}%
\definecolor{currentstroke}{rgb}{0.000000,0.000000,0.000000}%
\pgfsetstrokecolor{currentstroke}%
\pgfsetdash{}{0pt}%
\pgfsys@defobject{currentmarker}{\pgfqpoint{-0.027778in}{0.000000in}}{\pgfqpoint{-0.000000in}{0.000000in}}{%
\pgfpathmoveto{\pgfqpoint{-0.000000in}{0.000000in}}%
\pgfpathlineto{\pgfqpoint{-0.027778in}{0.000000in}}%
\pgfusepath{stroke,fill}%
}%
\begin{pgfscope}%
\pgfsys@transformshift{0.588387in}{1.627890in}%
\pgfsys@useobject{currentmarker}{}%
\end{pgfscope}%
\end{pgfscope}%
\begin{pgfscope}%
\pgfsetbuttcap%
\pgfsetroundjoin%
\definecolor{currentfill}{rgb}{0.000000,0.000000,0.000000}%
\pgfsetfillcolor{currentfill}%
\pgfsetlinewidth{0.602250pt}%
\definecolor{currentstroke}{rgb}{0.000000,0.000000,0.000000}%
\pgfsetstrokecolor{currentstroke}%
\pgfsetdash{}{0pt}%
\pgfsys@defobject{currentmarker}{\pgfqpoint{-0.027778in}{0.000000in}}{\pgfqpoint{-0.000000in}{0.000000in}}{%
\pgfpathmoveto{\pgfqpoint{-0.000000in}{0.000000in}}%
\pgfpathlineto{\pgfqpoint{-0.027778in}{0.000000in}}%
\pgfusepath{stroke,fill}%
}%
\begin{pgfscope}%
\pgfsys@transformshift{0.588387in}{1.646944in}%
\pgfsys@useobject{currentmarker}{}%
\end{pgfscope}%
\end{pgfscope}%
\begin{pgfscope}%
\pgfsetbuttcap%
\pgfsetroundjoin%
\definecolor{currentfill}{rgb}{0.000000,0.000000,0.000000}%
\pgfsetfillcolor{currentfill}%
\pgfsetlinewidth{0.602250pt}%
\definecolor{currentstroke}{rgb}{0.000000,0.000000,0.000000}%
\pgfsetstrokecolor{currentstroke}%
\pgfsetdash{}{0pt}%
\pgfsys@defobject{currentmarker}{\pgfqpoint{-0.027778in}{0.000000in}}{\pgfqpoint{-0.000000in}{0.000000in}}{%
\pgfpathmoveto{\pgfqpoint{-0.000000in}{0.000000in}}%
\pgfpathlineto{\pgfqpoint{-0.027778in}{0.000000in}}%
\pgfusepath{stroke,fill}%
}%
\begin{pgfscope}%
\pgfsys@transformshift{0.588387in}{1.776122in}%
\pgfsys@useobject{currentmarker}{}%
\end{pgfscope}%
\end{pgfscope}%
\begin{pgfscope}%
\pgfsetbuttcap%
\pgfsetroundjoin%
\definecolor{currentfill}{rgb}{0.000000,0.000000,0.000000}%
\pgfsetfillcolor{currentfill}%
\pgfsetlinewidth{0.602250pt}%
\definecolor{currentstroke}{rgb}{0.000000,0.000000,0.000000}%
\pgfsetstrokecolor{currentstroke}%
\pgfsetdash{}{0pt}%
\pgfsys@defobject{currentmarker}{\pgfqpoint{-0.027778in}{0.000000in}}{\pgfqpoint{-0.000000in}{0.000000in}}{%
\pgfpathmoveto{\pgfqpoint{-0.000000in}{0.000000in}}%
\pgfpathlineto{\pgfqpoint{-0.027778in}{0.000000in}}%
\pgfusepath{stroke,fill}%
}%
\begin{pgfscope}%
\pgfsys@transformshift{0.588387in}{1.841716in}%
\pgfsys@useobject{currentmarker}{}%
\end{pgfscope}%
\end{pgfscope}%
\begin{pgfscope}%
\pgfsetbuttcap%
\pgfsetroundjoin%
\definecolor{currentfill}{rgb}{0.000000,0.000000,0.000000}%
\pgfsetfillcolor{currentfill}%
\pgfsetlinewidth{0.602250pt}%
\definecolor{currentstroke}{rgb}{0.000000,0.000000,0.000000}%
\pgfsetstrokecolor{currentstroke}%
\pgfsetdash{}{0pt}%
\pgfsys@defobject{currentmarker}{\pgfqpoint{-0.027778in}{0.000000in}}{\pgfqpoint{-0.000000in}{0.000000in}}{%
\pgfpathmoveto{\pgfqpoint{-0.000000in}{0.000000in}}%
\pgfpathlineto{\pgfqpoint{-0.027778in}{0.000000in}}%
\pgfusepath{stroke,fill}%
}%
\begin{pgfscope}%
\pgfsys@transformshift{0.588387in}{1.888256in}%
\pgfsys@useobject{currentmarker}{}%
\end{pgfscope}%
\end{pgfscope}%
\begin{pgfscope}%
\pgfsetbuttcap%
\pgfsetroundjoin%
\definecolor{currentfill}{rgb}{0.000000,0.000000,0.000000}%
\pgfsetfillcolor{currentfill}%
\pgfsetlinewidth{0.602250pt}%
\definecolor{currentstroke}{rgb}{0.000000,0.000000,0.000000}%
\pgfsetstrokecolor{currentstroke}%
\pgfsetdash{}{0pt}%
\pgfsys@defobject{currentmarker}{\pgfqpoint{-0.027778in}{0.000000in}}{\pgfqpoint{-0.000000in}{0.000000in}}{%
\pgfpathmoveto{\pgfqpoint{-0.000000in}{0.000000in}}%
\pgfpathlineto{\pgfqpoint{-0.027778in}{0.000000in}}%
\pgfusepath{stroke,fill}%
}%
\begin{pgfscope}%
\pgfsys@transformshift{0.588387in}{1.924355in}%
\pgfsys@useobject{currentmarker}{}%
\end{pgfscope}%
\end{pgfscope}%
\begin{pgfscope}%
\pgfsetbuttcap%
\pgfsetroundjoin%
\definecolor{currentfill}{rgb}{0.000000,0.000000,0.000000}%
\pgfsetfillcolor{currentfill}%
\pgfsetlinewidth{0.602250pt}%
\definecolor{currentstroke}{rgb}{0.000000,0.000000,0.000000}%
\pgfsetstrokecolor{currentstroke}%
\pgfsetdash{}{0pt}%
\pgfsys@defobject{currentmarker}{\pgfqpoint{-0.027778in}{0.000000in}}{\pgfqpoint{-0.000000in}{0.000000in}}{%
\pgfpathmoveto{\pgfqpoint{-0.000000in}{0.000000in}}%
\pgfpathlineto{\pgfqpoint{-0.027778in}{0.000000in}}%
\pgfusepath{stroke,fill}%
}%
\begin{pgfscope}%
\pgfsys@transformshift{0.588387in}{1.953850in}%
\pgfsys@useobject{currentmarker}{}%
\end{pgfscope}%
\end{pgfscope}%
\begin{pgfscope}%
\pgfsetbuttcap%
\pgfsetroundjoin%
\definecolor{currentfill}{rgb}{0.000000,0.000000,0.000000}%
\pgfsetfillcolor{currentfill}%
\pgfsetlinewidth{0.602250pt}%
\definecolor{currentstroke}{rgb}{0.000000,0.000000,0.000000}%
\pgfsetstrokecolor{currentstroke}%
\pgfsetdash{}{0pt}%
\pgfsys@defobject{currentmarker}{\pgfqpoint{-0.027778in}{0.000000in}}{\pgfqpoint{-0.000000in}{0.000000in}}{%
\pgfpathmoveto{\pgfqpoint{-0.000000in}{0.000000in}}%
\pgfpathlineto{\pgfqpoint{-0.027778in}{0.000000in}}%
\pgfusepath{stroke,fill}%
}%
\begin{pgfscope}%
\pgfsys@transformshift{0.588387in}{1.978787in}%
\pgfsys@useobject{currentmarker}{}%
\end{pgfscope}%
\end{pgfscope}%
\begin{pgfscope}%
\pgfsetbuttcap%
\pgfsetroundjoin%
\definecolor{currentfill}{rgb}{0.000000,0.000000,0.000000}%
\pgfsetfillcolor{currentfill}%
\pgfsetlinewidth{0.602250pt}%
\definecolor{currentstroke}{rgb}{0.000000,0.000000,0.000000}%
\pgfsetstrokecolor{currentstroke}%
\pgfsetdash{}{0pt}%
\pgfsys@defobject{currentmarker}{\pgfqpoint{-0.027778in}{0.000000in}}{\pgfqpoint{-0.000000in}{0.000000in}}{%
\pgfpathmoveto{\pgfqpoint{-0.000000in}{0.000000in}}%
\pgfpathlineto{\pgfqpoint{-0.027778in}{0.000000in}}%
\pgfusepath{stroke,fill}%
}%
\begin{pgfscope}%
\pgfsys@transformshift{0.588387in}{2.000389in}%
\pgfsys@useobject{currentmarker}{}%
\end{pgfscope}%
\end{pgfscope}%
\begin{pgfscope}%
\pgfsetbuttcap%
\pgfsetroundjoin%
\definecolor{currentfill}{rgb}{0.000000,0.000000,0.000000}%
\pgfsetfillcolor{currentfill}%
\pgfsetlinewidth{0.602250pt}%
\definecolor{currentstroke}{rgb}{0.000000,0.000000,0.000000}%
\pgfsetstrokecolor{currentstroke}%
\pgfsetdash{}{0pt}%
\pgfsys@defobject{currentmarker}{\pgfqpoint{-0.027778in}{0.000000in}}{\pgfqpoint{-0.000000in}{0.000000in}}{%
\pgfpathmoveto{\pgfqpoint{-0.000000in}{0.000000in}}%
\pgfpathlineto{\pgfqpoint{-0.027778in}{0.000000in}}%
\pgfusepath{stroke,fill}%
}%
\begin{pgfscope}%
\pgfsys@transformshift{0.588387in}{2.019444in}%
\pgfsys@useobject{currentmarker}{}%
\end{pgfscope}%
\end{pgfscope}%
\begin{pgfscope}%
\pgfsetbuttcap%
\pgfsetroundjoin%
\definecolor{currentfill}{rgb}{0.000000,0.000000,0.000000}%
\pgfsetfillcolor{currentfill}%
\pgfsetlinewidth{0.602250pt}%
\definecolor{currentstroke}{rgb}{0.000000,0.000000,0.000000}%
\pgfsetstrokecolor{currentstroke}%
\pgfsetdash{}{0pt}%
\pgfsys@defobject{currentmarker}{\pgfqpoint{-0.027778in}{0.000000in}}{\pgfqpoint{-0.000000in}{0.000000in}}{%
\pgfpathmoveto{\pgfqpoint{-0.000000in}{0.000000in}}%
\pgfpathlineto{\pgfqpoint{-0.027778in}{0.000000in}}%
\pgfusepath{stroke,fill}%
}%
\begin{pgfscope}%
\pgfsys@transformshift{0.588387in}{2.148622in}%
\pgfsys@useobject{currentmarker}{}%
\end{pgfscope}%
\end{pgfscope}%
\begin{pgfscope}%
\pgfsetbuttcap%
\pgfsetroundjoin%
\definecolor{currentfill}{rgb}{0.000000,0.000000,0.000000}%
\pgfsetfillcolor{currentfill}%
\pgfsetlinewidth{0.602250pt}%
\definecolor{currentstroke}{rgb}{0.000000,0.000000,0.000000}%
\pgfsetstrokecolor{currentstroke}%
\pgfsetdash{}{0pt}%
\pgfsys@defobject{currentmarker}{\pgfqpoint{-0.027778in}{0.000000in}}{\pgfqpoint{-0.000000in}{0.000000in}}{%
\pgfpathmoveto{\pgfqpoint{-0.000000in}{0.000000in}}%
\pgfpathlineto{\pgfqpoint{-0.027778in}{0.000000in}}%
\pgfusepath{stroke,fill}%
}%
\begin{pgfscope}%
\pgfsys@transformshift{0.588387in}{2.214216in}%
\pgfsys@useobject{currentmarker}{}%
\end{pgfscope}%
\end{pgfscope}%
\begin{pgfscope}%
\pgfsetbuttcap%
\pgfsetroundjoin%
\definecolor{currentfill}{rgb}{0.000000,0.000000,0.000000}%
\pgfsetfillcolor{currentfill}%
\pgfsetlinewidth{0.602250pt}%
\definecolor{currentstroke}{rgb}{0.000000,0.000000,0.000000}%
\pgfsetstrokecolor{currentstroke}%
\pgfsetdash{}{0pt}%
\pgfsys@defobject{currentmarker}{\pgfqpoint{-0.027778in}{0.000000in}}{\pgfqpoint{-0.000000in}{0.000000in}}{%
\pgfpathmoveto{\pgfqpoint{-0.000000in}{0.000000in}}%
\pgfpathlineto{\pgfqpoint{-0.027778in}{0.000000in}}%
\pgfusepath{stroke,fill}%
}%
\begin{pgfscope}%
\pgfsys@transformshift{0.588387in}{2.260756in}%
\pgfsys@useobject{currentmarker}{}%
\end{pgfscope}%
\end{pgfscope}%
\begin{pgfscope}%
\pgfsetbuttcap%
\pgfsetroundjoin%
\definecolor{currentfill}{rgb}{0.000000,0.000000,0.000000}%
\pgfsetfillcolor{currentfill}%
\pgfsetlinewidth{0.602250pt}%
\definecolor{currentstroke}{rgb}{0.000000,0.000000,0.000000}%
\pgfsetstrokecolor{currentstroke}%
\pgfsetdash{}{0pt}%
\pgfsys@defobject{currentmarker}{\pgfqpoint{-0.027778in}{0.000000in}}{\pgfqpoint{-0.000000in}{0.000000in}}{%
\pgfpathmoveto{\pgfqpoint{-0.000000in}{0.000000in}}%
\pgfpathlineto{\pgfqpoint{-0.027778in}{0.000000in}}%
\pgfusepath{stroke,fill}%
}%
\begin{pgfscope}%
\pgfsys@transformshift{0.588387in}{2.296855in}%
\pgfsys@useobject{currentmarker}{}%
\end{pgfscope}%
\end{pgfscope}%
\begin{pgfscope}%
\pgfsetbuttcap%
\pgfsetroundjoin%
\definecolor{currentfill}{rgb}{0.000000,0.000000,0.000000}%
\pgfsetfillcolor{currentfill}%
\pgfsetlinewidth{0.602250pt}%
\definecolor{currentstroke}{rgb}{0.000000,0.000000,0.000000}%
\pgfsetstrokecolor{currentstroke}%
\pgfsetdash{}{0pt}%
\pgfsys@defobject{currentmarker}{\pgfqpoint{-0.027778in}{0.000000in}}{\pgfqpoint{-0.000000in}{0.000000in}}{%
\pgfpathmoveto{\pgfqpoint{-0.000000in}{0.000000in}}%
\pgfpathlineto{\pgfqpoint{-0.027778in}{0.000000in}}%
\pgfusepath{stroke,fill}%
}%
\begin{pgfscope}%
\pgfsys@transformshift{0.588387in}{2.326350in}%
\pgfsys@useobject{currentmarker}{}%
\end{pgfscope}%
\end{pgfscope}%
\begin{pgfscope}%
\pgfsetbuttcap%
\pgfsetroundjoin%
\definecolor{currentfill}{rgb}{0.000000,0.000000,0.000000}%
\pgfsetfillcolor{currentfill}%
\pgfsetlinewidth{0.602250pt}%
\definecolor{currentstroke}{rgb}{0.000000,0.000000,0.000000}%
\pgfsetstrokecolor{currentstroke}%
\pgfsetdash{}{0pt}%
\pgfsys@defobject{currentmarker}{\pgfqpoint{-0.027778in}{0.000000in}}{\pgfqpoint{-0.000000in}{0.000000in}}{%
\pgfpathmoveto{\pgfqpoint{-0.000000in}{0.000000in}}%
\pgfpathlineto{\pgfqpoint{-0.027778in}{0.000000in}}%
\pgfusepath{stroke,fill}%
}%
\begin{pgfscope}%
\pgfsys@transformshift{0.588387in}{2.351287in}%
\pgfsys@useobject{currentmarker}{}%
\end{pgfscope}%
\end{pgfscope}%
\begin{pgfscope}%
\pgfsetbuttcap%
\pgfsetroundjoin%
\definecolor{currentfill}{rgb}{0.000000,0.000000,0.000000}%
\pgfsetfillcolor{currentfill}%
\pgfsetlinewidth{0.602250pt}%
\definecolor{currentstroke}{rgb}{0.000000,0.000000,0.000000}%
\pgfsetstrokecolor{currentstroke}%
\pgfsetdash{}{0pt}%
\pgfsys@defobject{currentmarker}{\pgfqpoint{-0.027778in}{0.000000in}}{\pgfqpoint{-0.000000in}{0.000000in}}{%
\pgfpathmoveto{\pgfqpoint{-0.000000in}{0.000000in}}%
\pgfpathlineto{\pgfqpoint{-0.027778in}{0.000000in}}%
\pgfusepath{stroke,fill}%
}%
\begin{pgfscope}%
\pgfsys@transformshift{0.588387in}{2.372889in}%
\pgfsys@useobject{currentmarker}{}%
\end{pgfscope}%
\end{pgfscope}%
\begin{pgfscope}%
\pgfsetbuttcap%
\pgfsetroundjoin%
\definecolor{currentfill}{rgb}{0.000000,0.000000,0.000000}%
\pgfsetfillcolor{currentfill}%
\pgfsetlinewidth{0.602250pt}%
\definecolor{currentstroke}{rgb}{0.000000,0.000000,0.000000}%
\pgfsetstrokecolor{currentstroke}%
\pgfsetdash{}{0pt}%
\pgfsys@defobject{currentmarker}{\pgfqpoint{-0.027778in}{0.000000in}}{\pgfqpoint{-0.000000in}{0.000000in}}{%
\pgfpathmoveto{\pgfqpoint{-0.000000in}{0.000000in}}%
\pgfpathlineto{\pgfqpoint{-0.027778in}{0.000000in}}%
\pgfusepath{stroke,fill}%
}%
\begin{pgfscope}%
\pgfsys@transformshift{0.588387in}{2.391944in}%
\pgfsys@useobject{currentmarker}{}%
\end{pgfscope}%
\end{pgfscope}%
\begin{pgfscope}%
\pgfsetbuttcap%
\pgfsetroundjoin%
\definecolor{currentfill}{rgb}{0.000000,0.000000,0.000000}%
\pgfsetfillcolor{currentfill}%
\pgfsetlinewidth{0.602250pt}%
\definecolor{currentstroke}{rgb}{0.000000,0.000000,0.000000}%
\pgfsetstrokecolor{currentstroke}%
\pgfsetdash{}{0pt}%
\pgfsys@defobject{currentmarker}{\pgfqpoint{-0.027778in}{0.000000in}}{\pgfqpoint{-0.000000in}{0.000000in}}{%
\pgfpathmoveto{\pgfqpoint{-0.000000in}{0.000000in}}%
\pgfpathlineto{\pgfqpoint{-0.027778in}{0.000000in}}%
\pgfusepath{stroke,fill}%
}%
\begin{pgfscope}%
\pgfsys@transformshift{0.588387in}{2.521122in}%
\pgfsys@useobject{currentmarker}{}%
\end{pgfscope}%
\end{pgfscope}%
\begin{pgfscope}%
\pgfsetbuttcap%
\pgfsetroundjoin%
\definecolor{currentfill}{rgb}{0.000000,0.000000,0.000000}%
\pgfsetfillcolor{currentfill}%
\pgfsetlinewidth{0.602250pt}%
\definecolor{currentstroke}{rgb}{0.000000,0.000000,0.000000}%
\pgfsetstrokecolor{currentstroke}%
\pgfsetdash{}{0pt}%
\pgfsys@defobject{currentmarker}{\pgfqpoint{-0.027778in}{0.000000in}}{\pgfqpoint{-0.000000in}{0.000000in}}{%
\pgfpathmoveto{\pgfqpoint{-0.000000in}{0.000000in}}%
\pgfpathlineto{\pgfqpoint{-0.027778in}{0.000000in}}%
\pgfusepath{stroke,fill}%
}%
\begin{pgfscope}%
\pgfsys@transformshift{0.588387in}{2.586716in}%
\pgfsys@useobject{currentmarker}{}%
\end{pgfscope}%
\end{pgfscope}%
\begin{pgfscope}%
\pgfsetbuttcap%
\pgfsetroundjoin%
\definecolor{currentfill}{rgb}{0.000000,0.000000,0.000000}%
\pgfsetfillcolor{currentfill}%
\pgfsetlinewidth{0.602250pt}%
\definecolor{currentstroke}{rgb}{0.000000,0.000000,0.000000}%
\pgfsetstrokecolor{currentstroke}%
\pgfsetdash{}{0pt}%
\pgfsys@defobject{currentmarker}{\pgfqpoint{-0.027778in}{0.000000in}}{\pgfqpoint{-0.000000in}{0.000000in}}{%
\pgfpathmoveto{\pgfqpoint{-0.000000in}{0.000000in}}%
\pgfpathlineto{\pgfqpoint{-0.027778in}{0.000000in}}%
\pgfusepath{stroke,fill}%
}%
\begin{pgfscope}%
\pgfsys@transformshift{0.588387in}{2.633256in}%
\pgfsys@useobject{currentmarker}{}%
\end{pgfscope}%
\end{pgfscope}%
\begin{pgfscope}%
\pgfsetbuttcap%
\pgfsetroundjoin%
\definecolor{currentfill}{rgb}{0.000000,0.000000,0.000000}%
\pgfsetfillcolor{currentfill}%
\pgfsetlinewidth{0.602250pt}%
\definecolor{currentstroke}{rgb}{0.000000,0.000000,0.000000}%
\pgfsetstrokecolor{currentstroke}%
\pgfsetdash{}{0pt}%
\pgfsys@defobject{currentmarker}{\pgfqpoint{-0.027778in}{0.000000in}}{\pgfqpoint{-0.000000in}{0.000000in}}{%
\pgfpathmoveto{\pgfqpoint{-0.000000in}{0.000000in}}%
\pgfpathlineto{\pgfqpoint{-0.027778in}{0.000000in}}%
\pgfusepath{stroke,fill}%
}%
\begin{pgfscope}%
\pgfsys@transformshift{0.588387in}{2.669354in}%
\pgfsys@useobject{currentmarker}{}%
\end{pgfscope}%
\end{pgfscope}%
\begin{pgfscope}%
\pgfsetbuttcap%
\pgfsetroundjoin%
\definecolor{currentfill}{rgb}{0.000000,0.000000,0.000000}%
\pgfsetfillcolor{currentfill}%
\pgfsetlinewidth{0.602250pt}%
\definecolor{currentstroke}{rgb}{0.000000,0.000000,0.000000}%
\pgfsetstrokecolor{currentstroke}%
\pgfsetdash{}{0pt}%
\pgfsys@defobject{currentmarker}{\pgfqpoint{-0.027778in}{0.000000in}}{\pgfqpoint{-0.000000in}{0.000000in}}{%
\pgfpathmoveto{\pgfqpoint{-0.000000in}{0.000000in}}%
\pgfpathlineto{\pgfqpoint{-0.027778in}{0.000000in}}%
\pgfusepath{stroke,fill}%
}%
\begin{pgfscope}%
\pgfsys@transformshift{0.588387in}{2.698849in}%
\pgfsys@useobject{currentmarker}{}%
\end{pgfscope}%
\end{pgfscope}%
\begin{pgfscope}%
\pgfsetbuttcap%
\pgfsetroundjoin%
\definecolor{currentfill}{rgb}{0.000000,0.000000,0.000000}%
\pgfsetfillcolor{currentfill}%
\pgfsetlinewidth{0.602250pt}%
\definecolor{currentstroke}{rgb}{0.000000,0.000000,0.000000}%
\pgfsetstrokecolor{currentstroke}%
\pgfsetdash{}{0pt}%
\pgfsys@defobject{currentmarker}{\pgfqpoint{-0.027778in}{0.000000in}}{\pgfqpoint{-0.000000in}{0.000000in}}{%
\pgfpathmoveto{\pgfqpoint{-0.000000in}{0.000000in}}%
\pgfpathlineto{\pgfqpoint{-0.027778in}{0.000000in}}%
\pgfusepath{stroke,fill}%
}%
\begin{pgfscope}%
\pgfsys@transformshift{0.588387in}{2.723787in}%
\pgfsys@useobject{currentmarker}{}%
\end{pgfscope}%
\end{pgfscope}%
\begin{pgfscope}%
\definecolor{textcolor}{rgb}{0.000000,0.000000,0.000000}%
\pgfsetstrokecolor{textcolor}%
\pgfsetfillcolor{textcolor}%
\pgftext[x=0.234413in,y=1.631726in,,bottom,rotate=90.000000]{\color{textcolor}{\rmfamily\fontsize{10.000000}{12.000000}\selectfont\catcode`\^=\active\def^{\ifmmode\sp\else\^{}\fi}\catcode`\%=\active\def%{\%}Checks [call]}}%
\end{pgfscope}%
\begin{pgfscope}%
\pgfpathrectangle{\pgfqpoint{0.588387in}{0.521603in}}{\pgfqpoint{4.669024in}{2.220246in}}%
\pgfusepath{clip}%
\pgfsetrectcap%
\pgfsetroundjoin%
\pgfsetlinewidth{1.505625pt}%
\pgfsetstrokecolor{currentstroke1}%
\pgfsetdash{}{0pt}%
\pgfpathmoveto{\pgfqpoint{0.800616in}{0.622524in}}%
\pgfpathlineto{\pgfqpoint{0.946980in}{0.707172in}}%
\pgfpathlineto{\pgfqpoint{1.239709in}{0.968732in}}%
\pgfpathlineto{\pgfqpoint{1.532438in}{1.161590in}}%
\pgfpathlineto{\pgfqpoint{1.825166in}{1.358799in}}%
\pgfpathlineto{\pgfqpoint{2.264259in}{1.617708in}}%
\pgfpathlineto{\pgfqpoint{2.410624in}{1.702098in}}%
\pgfpathlineto{\pgfqpoint{2.703353in}{1.904705in}}%
\pgfpathlineto{\pgfqpoint{3.142446in}{2.153165in}}%
\pgfpathlineto{\pgfqpoint{3.435175in}{2.282590in}}%
\pgfpathlineto{\pgfqpoint{3.874268in}{2.445825in}}%
\pgfpathlineto{\pgfqpoint{4.166997in}{2.465487in}}%
\pgfusepath{stroke}%
\end{pgfscope}%
\begin{pgfscope}%
\pgfpathrectangle{\pgfqpoint{0.588387in}{0.521603in}}{\pgfqpoint{4.669024in}{2.220246in}}%
\pgfusepath{clip}%
\pgfsetrectcap%
\pgfsetroundjoin%
\pgfsetlinewidth{1.505625pt}%
\pgfsetstrokecolor{currentstroke2}%
\pgfsetdash{}{0pt}%
\pgfpathmoveto{\pgfqpoint{0.800616in}{0.622524in}}%
\pgfpathlineto{\pgfqpoint{0.946980in}{0.734657in}}%
\pgfpathlineto{\pgfqpoint{1.239709in}{0.968732in}}%
\pgfpathlineto{\pgfqpoint{1.532438in}{1.161590in}}%
\pgfpathlineto{\pgfqpoint{1.825166in}{1.371912in}}%
\pgfpathlineto{\pgfqpoint{2.264259in}{1.621454in}}%
\pgfpathlineto{\pgfqpoint{2.410624in}{1.701244in}}%
\pgfpathlineto{\pgfqpoint{2.703353in}{1.873410in}}%
\pgfpathlineto{\pgfqpoint{3.142446in}{2.069799in}}%
\pgfpathlineto{\pgfqpoint{3.435175in}{2.195290in}}%
\pgfpathlineto{\pgfqpoint{3.874268in}{2.519169in}}%
\pgfpathlineto{\pgfqpoint{4.166997in}{2.427642in}}%
\pgfpathlineto{\pgfqpoint{4.606090in}{2.408004in}}%
\pgfusepath{stroke}%
\end{pgfscope}%
\begin{pgfscope}%
\pgfpathrectangle{\pgfqpoint{0.588387in}{0.521603in}}{\pgfqpoint{4.669024in}{2.220246in}}%
\pgfusepath{clip}%
\pgfsetrectcap%
\pgfsetroundjoin%
\pgfsetlinewidth{1.505625pt}%
\pgfsetstrokecolor{currentstroke3}%
\pgfsetdash{}{0pt}%
\pgfpathmoveto{\pgfqpoint{0.800616in}{0.622524in}}%
\pgfpathlineto{\pgfqpoint{0.946980in}{0.734657in}}%
\pgfpathlineto{\pgfqpoint{1.239709in}{0.968732in}}%
\pgfpathlineto{\pgfqpoint{1.532438in}{1.161590in}}%
\pgfpathlineto{\pgfqpoint{1.825166in}{1.358799in}}%
\pgfpathlineto{\pgfqpoint{2.264259in}{1.621454in}}%
\pgfpathlineto{\pgfqpoint{2.410624in}{1.665454in}}%
\pgfpathlineto{\pgfqpoint{2.703353in}{1.829611in}}%
\pgfpathlineto{\pgfqpoint{3.142446in}{2.033655in}}%
\pgfpathlineto{\pgfqpoint{3.435175in}{2.178512in}}%
\pgfpathlineto{\pgfqpoint{3.874268in}{2.089417in}}%
\pgfpathlineto{\pgfqpoint{4.166997in}{2.320008in}}%
\pgfpathlineto{\pgfqpoint{4.606090in}{2.372161in}}%
\pgfpathlineto{\pgfqpoint{5.045183in}{2.393370in}}%
\pgfusepath{stroke}%
\end{pgfscope}%
\begin{pgfscope}%
\pgfpathrectangle{\pgfqpoint{0.588387in}{0.521603in}}{\pgfqpoint{4.669024in}{2.220246in}}%
\pgfusepath{clip}%
\pgfsetrectcap%
\pgfsetroundjoin%
\pgfsetlinewidth{1.505625pt}%
\pgfsetstrokecolor{currentstroke4}%
\pgfsetdash{}{0pt}%
\pgfpathmoveto{\pgfqpoint{0.800616in}{0.622524in}}%
\pgfpathlineto{\pgfqpoint{0.946980in}{0.734657in}}%
\pgfpathlineto{\pgfqpoint{1.239709in}{0.968732in}}%
\pgfpathlineto{\pgfqpoint{1.532438in}{1.161590in}}%
\pgfpathlineto{\pgfqpoint{1.825166in}{1.358799in}}%
\pgfpathlineto{\pgfqpoint{2.264259in}{1.613872in}}%
\pgfpathlineto{\pgfqpoint{2.410624in}{1.671331in}}%
\pgfpathlineto{\pgfqpoint{2.703353in}{1.817775in}}%
\pgfpathlineto{\pgfqpoint{3.142446in}{2.027111in}}%
\pgfpathlineto{\pgfqpoint{3.435175in}{2.166331in}}%
\pgfpathlineto{\pgfqpoint{3.874268in}{2.089361in}}%
\pgfpathlineto{\pgfqpoint{4.166997in}{2.325428in}}%
\pgfpathlineto{\pgfqpoint{4.606090in}{2.355525in}}%
\pgfpathlineto{\pgfqpoint{5.045183in}{2.370393in}}%
\pgfusepath{stroke}%
\end{pgfscope}%
\begin{pgfscope}%
\pgfpathrectangle{\pgfqpoint{0.588387in}{0.521603in}}{\pgfqpoint{4.669024in}{2.220246in}}%
\pgfusepath{clip}%
\pgfsetrectcap%
\pgfsetroundjoin%
\pgfsetlinewidth{1.505625pt}%
\pgfsetstrokecolor{currentstroke5}%
\pgfsetdash{}{0pt}%
\pgfpathmoveto{\pgfqpoint{0.800616in}{0.622524in}}%
\pgfpathlineto{\pgfqpoint{0.946980in}{0.734657in}}%
\pgfpathlineto{\pgfqpoint{1.239709in}{0.968732in}}%
\pgfpathlineto{\pgfqpoint{1.532438in}{1.161590in}}%
\pgfpathlineto{\pgfqpoint{1.825166in}{1.358799in}}%
\pgfpathlineto{\pgfqpoint{2.264259in}{1.621454in}}%
\pgfpathlineto{\pgfqpoint{2.410624in}{1.692948in}}%
\pgfpathlineto{\pgfqpoint{2.703353in}{1.885327in}}%
\pgfpathlineto{\pgfqpoint{3.142446in}{2.125519in}}%
\pgfpathlineto{\pgfqpoint{3.435175in}{2.224578in}}%
\pgfpathlineto{\pgfqpoint{3.874268in}{2.179042in}}%
\pgfpathlineto{\pgfqpoint{4.166997in}{2.361945in}}%
\pgfpathlineto{\pgfqpoint{4.606090in}{2.301688in}}%
\pgfpathlineto{\pgfqpoint{5.045183in}{2.374612in}}%
\pgfusepath{stroke}%
\end{pgfscope}%
\begin{pgfscope}%
\pgfpathrectangle{\pgfqpoint{0.588387in}{0.521603in}}{\pgfqpoint{4.669024in}{2.220246in}}%
\pgfusepath{clip}%
\pgfsetrectcap%
\pgfsetroundjoin%
\pgfsetlinewidth{1.505625pt}%
\pgfsetstrokecolor{currentstroke6}%
\pgfsetdash{}{0pt}%
\pgfpathmoveto{\pgfqpoint{0.800616in}{0.622524in}}%
\pgfpathlineto{\pgfqpoint{0.946980in}{0.734657in}}%
\pgfpathlineto{\pgfqpoint{1.239709in}{0.968732in}}%
\pgfpathlineto{\pgfqpoint{1.532438in}{1.161590in}}%
\pgfpathlineto{\pgfqpoint{1.825166in}{1.358799in}}%
\pgfpathlineto{\pgfqpoint{2.264259in}{1.613872in}}%
\pgfpathlineto{\pgfqpoint{2.410624in}{1.682673in}}%
\pgfpathlineto{\pgfqpoint{2.703353in}{1.834185in}}%
\pgfpathlineto{\pgfqpoint{3.142446in}{2.032439in}}%
\pgfpathlineto{\pgfqpoint{3.435175in}{2.186508in}}%
\pgfpathlineto{\pgfqpoint{3.874268in}{2.092320in}}%
\pgfpathlineto{\pgfqpoint{4.166997in}{2.327833in}}%
\pgfpathlineto{\pgfqpoint{4.606090in}{2.323026in}}%
\pgfpathlineto{\pgfqpoint{5.045183in}{2.344278in}}%
\pgfusepath{stroke}%
\end{pgfscope}%
\begin{pgfscope}%
\pgfpathrectangle{\pgfqpoint{0.588387in}{0.521603in}}{\pgfqpoint{4.669024in}{2.220246in}}%
\pgfusepath{clip}%
\pgfsetrectcap%
\pgfsetroundjoin%
\pgfsetlinewidth{1.505625pt}%
\pgfsetstrokecolor{currentstroke7}%
\pgfsetdash{}{0pt}%
\pgfpathmoveto{\pgfqpoint{0.800616in}{0.622524in}}%
\pgfpathlineto{\pgfqpoint{0.946980in}{0.707172in}}%
\pgfpathlineto{\pgfqpoint{1.239709in}{0.968732in}}%
\pgfpathlineto{\pgfqpoint{1.532438in}{1.161590in}}%
\pgfpathlineto{\pgfqpoint{1.825166in}{1.358799in}}%
\pgfpathlineto{\pgfqpoint{2.264259in}{1.630298in}}%
\pgfpathlineto{\pgfqpoint{2.410624in}{1.715596in}}%
\pgfpathlineto{\pgfqpoint{2.703353in}{1.907655in}}%
\pgfpathlineto{\pgfqpoint{3.142446in}{2.109114in}}%
\pgfpathlineto{\pgfqpoint{3.435175in}{2.165087in}}%
\pgfpathlineto{\pgfqpoint{3.874268in}{2.332761in}}%
\pgfpathlineto{\pgfqpoint{4.166997in}{2.417984in}}%
\pgfpathlineto{\pgfqpoint{4.606090in}{2.432487in}}%
\pgfusepath{stroke}%
\end{pgfscope}%
\begin{pgfscope}%
\pgfpathrectangle{\pgfqpoint{0.588387in}{0.521603in}}{\pgfqpoint{4.669024in}{2.220246in}}%
\pgfusepath{clip}%
\pgfsetrectcap%
\pgfsetroundjoin%
\pgfsetlinewidth{1.505625pt}%
\definecolor{currentstroke}{rgb}{0.498039,0.498039,0.498039}%
\pgfsetstrokecolor{currentstroke}%
\pgfsetdash{}{0pt}%
\pgfpathmoveto{\pgfqpoint{0.800616in}{0.622524in}}%
\pgfpathlineto{\pgfqpoint{0.946980in}{0.707172in}}%
\pgfpathlineto{\pgfqpoint{1.239709in}{0.968732in}}%
\pgfpathlineto{\pgfqpoint{1.532438in}{1.161590in}}%
\pgfpathlineto{\pgfqpoint{1.825166in}{1.327846in}}%
\pgfpathlineto{\pgfqpoint{2.264259in}{1.579833in}}%
\pgfpathlineto{\pgfqpoint{2.410624in}{1.701620in}}%
\pgfpathlineto{\pgfqpoint{2.703353in}{1.850303in}}%
\pgfpathlineto{\pgfqpoint{3.142446in}{2.082292in}}%
\pgfpathlineto{\pgfqpoint{3.435175in}{2.196495in}}%
\pgfpathlineto{\pgfqpoint{3.874268in}{2.306788in}}%
\pgfpathlineto{\pgfqpoint{4.166997in}{2.387798in}}%
\pgfpathlineto{\pgfqpoint{4.606090in}{2.454052in}}%
\pgfpathlineto{\pgfqpoint{5.045183in}{2.472561in}}%
\pgfusepath{stroke}%
\end{pgfscope}%
\begin{pgfscope}%
\pgfpathrectangle{\pgfqpoint{0.588387in}{0.521603in}}{\pgfqpoint{4.669024in}{2.220246in}}%
\pgfusepath{clip}%
\pgfsetrectcap%
\pgfsetroundjoin%
\pgfsetlinewidth{1.505625pt}%
\definecolor{currentstroke}{rgb}{0.737255,0.741176,0.133333}%
\pgfsetstrokecolor{currentstroke}%
\pgfsetdash{}{0pt}%
\pgfpathmoveto{\pgfqpoint{1.532438in}{1.182875in}}%
\pgfpathlineto{\pgfqpoint{1.825166in}{1.407380in}}%
\pgfpathlineto{\pgfqpoint{2.264259in}{1.743850in}}%
\pgfpathlineto{\pgfqpoint{2.410624in}{1.826493in}}%
\pgfpathlineto{\pgfqpoint{2.703353in}{2.080260in}}%
\pgfpathlineto{\pgfqpoint{3.142446in}{2.416662in}}%
\pgfpathlineto{\pgfqpoint{3.435175in}{2.640929in}}%
\pgfusepath{stroke}%
\end{pgfscope}%
\begin{pgfscope}%
\pgfsetrectcap%
\pgfsetmiterjoin%
\pgfsetlinewidth{0.803000pt}%
\definecolor{currentstroke}{rgb}{0.000000,0.000000,0.000000}%
\pgfsetstrokecolor{currentstroke}%
\pgfsetdash{}{0pt}%
\pgfpathmoveto{\pgfqpoint{0.588387in}{0.521603in}}%
\pgfpathlineto{\pgfqpoint{0.588387in}{2.741849in}}%
\pgfusepath{stroke}%
\end{pgfscope}%
\begin{pgfscope}%
\pgfsetrectcap%
\pgfsetmiterjoin%
\pgfsetlinewidth{0.803000pt}%
\definecolor{currentstroke}{rgb}{0.000000,0.000000,0.000000}%
\pgfsetstrokecolor{currentstroke}%
\pgfsetdash{}{0pt}%
\pgfpathmoveto{\pgfqpoint{5.257411in}{0.521603in}}%
\pgfpathlineto{\pgfqpoint{5.257411in}{2.741849in}}%
\pgfusepath{stroke}%
\end{pgfscope}%
\begin{pgfscope}%
\pgfsetrectcap%
\pgfsetmiterjoin%
\pgfsetlinewidth{0.803000pt}%
\definecolor{currentstroke}{rgb}{0.000000,0.000000,0.000000}%
\pgfsetstrokecolor{currentstroke}%
\pgfsetdash{}{0pt}%
\pgfpathmoveto{\pgfqpoint{0.588387in}{0.521603in}}%
\pgfpathlineto{\pgfqpoint{5.257411in}{0.521603in}}%
\pgfusepath{stroke}%
\end{pgfscope}%
\begin{pgfscope}%
\pgfsetrectcap%
\pgfsetmiterjoin%
\pgfsetlinewidth{0.803000pt}%
\definecolor{currentstroke}{rgb}{0.000000,0.000000,0.000000}%
\pgfsetstrokecolor{currentstroke}%
\pgfsetdash{}{0pt}%
\pgfpathmoveto{\pgfqpoint{0.588387in}{2.741849in}}%
\pgfpathlineto{\pgfqpoint{5.257411in}{2.741849in}}%
\pgfusepath{stroke}%
\end{pgfscope}%
\begin{pgfscope}%
\pgfsetbuttcap%
\pgfsetmiterjoin%
\definecolor{currentfill}{rgb}{1.000000,1.000000,1.000000}%
\pgfsetfillcolor{currentfill}%
\pgfsetfillopacity{0.800000}%
\pgfsetlinewidth{1.003750pt}%
\definecolor{currentstroke}{rgb}{0.800000,0.800000,0.800000}%
\pgfsetstrokecolor{currentstroke}%
\pgfsetstrokeopacity{0.800000}%
\pgfsetdash{}{0pt}%
\pgfpathmoveto{\pgfqpoint{5.344911in}{0.969732in}}%
\pgfpathlineto{\pgfqpoint{8.259376in}{0.969732in}}%
\pgfpathquadraticcurveto{\pgfqpoint{8.284376in}{0.969732in}}{\pgfqpoint{8.284376in}{0.994732in}}%
\pgfpathlineto{\pgfqpoint{8.284376in}{2.654349in}}%
\pgfpathquadraticcurveto{\pgfqpoint{8.284376in}{2.679349in}}{\pgfqpoint{8.259376in}{2.679349in}}%
\pgfpathlineto{\pgfqpoint{5.344911in}{2.679349in}}%
\pgfpathquadraticcurveto{\pgfqpoint{5.319911in}{2.679349in}}{\pgfqpoint{5.319911in}{2.654349in}}%
\pgfpathlineto{\pgfqpoint{5.319911in}{0.994732in}}%
\pgfpathquadraticcurveto{\pgfqpoint{5.319911in}{0.969732in}}{\pgfqpoint{5.344911in}{0.969732in}}%
\pgfpathlineto{\pgfqpoint{5.344911in}{0.969732in}}%
\pgfpathclose%
\pgfusepath{stroke,fill}%
\end{pgfscope}%
\begin{pgfscope}%
\pgfsetrectcap%
\pgfsetroundjoin%
\pgfsetlinewidth{1.505625pt}%
\definecolor{currentstroke}{rgb}{0.737255,0.741176,0.133333}%
\pgfsetstrokecolor{currentstroke}%
\pgfsetdash{}{0pt}%
\pgfpathmoveto{\pgfqpoint{5.369911in}{2.578129in}}%
\pgfpathlineto{\pgfqpoint{5.494911in}{2.578129in}}%
\pgfpathlineto{\pgfqpoint{5.619911in}{2.578129in}}%
\pgfusepath{stroke}%
\end{pgfscope}%
\begin{pgfscope}%
\definecolor{textcolor}{rgb}{0.000000,0.000000,0.000000}%
\pgfsetstrokecolor{textcolor}%
\pgfsetfillcolor{textcolor}%
\pgftext[x=5.719911in,y=2.534379in,left,base]{\color{textcolor}{\rmfamily\fontsize{9.000000}{10.800000}\selectfont\catcode`\^=\active\def^{\ifmmode\sp\else\^{}\fi}\catcode`\%=\active\def%{\%}\NaiveCycles{}}}%
\end{pgfscope}%
\begin{pgfscope}%
\pgfsetrectcap%
\pgfsetroundjoin%
\pgfsetlinewidth{1.505625pt}%
\pgfsetstrokecolor{currentstroke1}%
\pgfsetdash{}{0pt}%
\pgfpathmoveto{\pgfqpoint{5.369911in}{2.394657in}}%
\pgfpathlineto{\pgfqpoint{5.494911in}{2.394657in}}%
\pgfpathlineto{\pgfqpoint{5.619911in}{2.394657in}}%
\pgfusepath{stroke}%
\end{pgfscope}%
\begin{pgfscope}%
\definecolor{textcolor}{rgb}{0.000000,0.000000,0.000000}%
\pgfsetstrokecolor{textcolor}%
\pgfsetfillcolor{textcolor}%
\pgftext[x=5.719911in,y=2.350907in,left,base]{\color{textcolor}{\rmfamily\fontsize{9.000000}{10.800000}\selectfont\catcode`\^=\active\def^{\ifmmode\sp\else\^{}\fi}\catcode`\%=\active\def%{\%}\CyclesMatchChunks{} \& \MergeLinear{}}}%
\end{pgfscope}%
\begin{pgfscope}%
\pgfsetrectcap%
\pgfsetroundjoin%
\pgfsetlinewidth{1.505625pt}%
\pgfsetstrokecolor{currentstroke2}%
\pgfsetdash{}{0pt}%
\pgfpathmoveto{\pgfqpoint{5.369911in}{2.207707in}}%
\pgfpathlineto{\pgfqpoint{5.494911in}{2.207707in}}%
\pgfpathlineto{\pgfqpoint{5.619911in}{2.207707in}}%
\pgfusepath{stroke}%
\end{pgfscope}%
\begin{pgfscope}%
\definecolor{textcolor}{rgb}{0.000000,0.000000,0.000000}%
\pgfsetstrokecolor{textcolor}%
\pgfsetfillcolor{textcolor}%
\pgftext[x=5.719911in,y=2.163957in,left,base]{\color{textcolor}{\rmfamily\fontsize{9.000000}{10.800000}\selectfont\catcode`\^=\active\def^{\ifmmode\sp\else\^{}\fi}\catcode`\%=\active\def%{\%}\CyclesMatchChunks{} \& \SharedVertices{}}}%
\end{pgfscope}%
\begin{pgfscope}%
\pgfsetrectcap%
\pgfsetroundjoin%
\pgfsetlinewidth{1.505625pt}%
\pgfsetstrokecolor{currentstroke3}%
\pgfsetdash{}{0pt}%
\pgfpathmoveto{\pgfqpoint{5.369911in}{2.020756in}}%
\pgfpathlineto{\pgfqpoint{5.494911in}{2.020756in}}%
\pgfpathlineto{\pgfqpoint{5.619911in}{2.020756in}}%
\pgfusepath{stroke}%
\end{pgfscope}%
\begin{pgfscope}%
\definecolor{textcolor}{rgb}{0.000000,0.000000,0.000000}%
\pgfsetstrokecolor{textcolor}%
\pgfsetfillcolor{textcolor}%
\pgftext[x=5.719911in,y=1.977006in,left,base]{\color{textcolor}{\rmfamily\fontsize{9.000000}{10.800000}\selectfont\catcode`\^=\active\def^{\ifmmode\sp\else\^{}\fi}\catcode`\%=\active\def%{\%}\Neighbors{} \& \MergeLinear{}}}%
\end{pgfscope}%
\begin{pgfscope}%
\pgfsetrectcap%
\pgfsetroundjoin%
\pgfsetlinewidth{1.505625pt}%
\pgfsetstrokecolor{currentstroke4}%
\pgfsetdash{}{0pt}%
\pgfpathmoveto{\pgfqpoint{5.369911in}{1.837285in}}%
\pgfpathlineto{\pgfqpoint{5.494911in}{1.837285in}}%
\pgfpathlineto{\pgfqpoint{5.619911in}{1.837285in}}%
\pgfusepath{stroke}%
\end{pgfscope}%
\begin{pgfscope}%
\definecolor{textcolor}{rgb}{0.000000,0.000000,0.000000}%
\pgfsetstrokecolor{textcolor}%
\pgfsetfillcolor{textcolor}%
\pgftext[x=5.719911in,y=1.793535in,left,base]{\color{textcolor}{\rmfamily\fontsize{9.000000}{10.800000}\selectfont\catcode`\^=\active\def^{\ifmmode\sp\else\^{}\fi}\catcode`\%=\active\def%{\%}\Neighbors{} \& \SharedVertices{}}}%
\end{pgfscope}%
\begin{pgfscope}%
\pgfsetrectcap%
\pgfsetroundjoin%
\pgfsetlinewidth{1.505625pt}%
\pgfsetstrokecolor{currentstroke5}%
\pgfsetdash{}{0pt}%
\pgfpathmoveto{\pgfqpoint{5.369911in}{1.650334in}}%
\pgfpathlineto{\pgfqpoint{5.494911in}{1.650334in}}%
\pgfpathlineto{\pgfqpoint{5.619911in}{1.650334in}}%
\pgfusepath{stroke}%
\end{pgfscope}%
\begin{pgfscope}%
\definecolor{textcolor}{rgb}{0.000000,0.000000,0.000000}%
\pgfsetstrokecolor{textcolor}%
\pgfsetfillcolor{textcolor}%
\pgftext[x=5.719911in,y=1.606584in,left,base]{\color{textcolor}{\rmfamily\fontsize{9.000000}{10.800000}\selectfont\catcode`\^=\active\def^{\ifmmode\sp\else\^{}\fi}\catcode`\%=\active\def%{\%}\NeighborsDegree{} \& \MergeLinear{}}}%
\end{pgfscope}%
\begin{pgfscope}%
\pgfsetrectcap%
\pgfsetroundjoin%
\pgfsetlinewidth{1.505625pt}%
\pgfsetstrokecolor{currentstroke6}%
\pgfsetdash{}{0pt}%
\pgfpathmoveto{\pgfqpoint{5.369911in}{1.463384in}}%
\pgfpathlineto{\pgfqpoint{5.494911in}{1.463384in}}%
\pgfpathlineto{\pgfqpoint{5.619911in}{1.463384in}}%
\pgfusepath{stroke}%
\end{pgfscope}%
\begin{pgfscope}%
\definecolor{textcolor}{rgb}{0.000000,0.000000,0.000000}%
\pgfsetstrokecolor{textcolor}%
\pgfsetfillcolor{textcolor}%
\pgftext[x=5.719911in,y=1.419634in,left,base]{\color{textcolor}{\rmfamily\fontsize{9.000000}{10.800000}\selectfont\catcode`\^=\active\def^{\ifmmode\sp\else\^{}\fi}\catcode`\%=\active\def%{\%}\NeighborsDegree{} \& \SharedVertices{}}}%
\end{pgfscope}%
\begin{pgfscope}%
\pgfsetrectcap%
\pgfsetroundjoin%
\pgfsetlinewidth{1.505625pt}%
\pgfsetstrokecolor{currentstroke7}%
\pgfsetdash{}{0pt}%
\pgfpathmoveto{\pgfqpoint{5.369911in}{1.276433in}}%
\pgfpathlineto{\pgfqpoint{5.494911in}{1.276433in}}%
\pgfpathlineto{\pgfqpoint{5.619911in}{1.276433in}}%
\pgfusepath{stroke}%
\end{pgfscope}%
\begin{pgfscope}%
\definecolor{textcolor}{rgb}{0.000000,0.000000,0.000000}%
\pgfsetstrokecolor{textcolor}%
\pgfsetfillcolor{textcolor}%
\pgftext[x=5.719911in,y=1.232683in,left,base]{\color{textcolor}{\rmfamily\fontsize{9.000000}{10.800000}\selectfont\catcode`\^=\active\def^{\ifmmode\sp\else\^{}\fi}\catcode`\%=\active\def%{\%}\None{} \& \MergeLinear{}}}%
\end{pgfscope}%
\begin{pgfscope}%
\pgfsetrectcap%
\pgfsetroundjoin%
\pgfsetlinewidth{1.505625pt}%
\definecolor{currentstroke}{rgb}{0.498039,0.498039,0.498039}%
\pgfsetstrokecolor{currentstroke}%
\pgfsetdash{}{0pt}%
\pgfpathmoveto{\pgfqpoint{5.369911in}{1.092962in}}%
\pgfpathlineto{\pgfqpoint{5.494911in}{1.092962in}}%
\pgfpathlineto{\pgfqpoint{5.619911in}{1.092962in}}%
\pgfusepath{stroke}%
\end{pgfscope}%
\begin{pgfscope}%
\definecolor{textcolor}{rgb}{0.000000,0.000000,0.000000}%
\pgfsetstrokecolor{textcolor}%
\pgfsetfillcolor{textcolor}%
\pgftext[x=5.719911in,y=1.049212in,left,base]{\color{textcolor}{\rmfamily\fontsize{9.000000}{10.800000}\selectfont\catcode`\^=\active\def^{\ifmmode\sp\else\^{}\fi}\catcode`\%=\active\def%{\%}\None{} \& \SharedVertices{}}}%
\end{pgfscope}%
\end{pgfpicture}%
\makeatother%
\endgroup%
}
% 	\caption[Checks performed for graphs with no 3 nor 4 cycles (all)]{
% 		The number of checks performed to find all NAC-colorings for graphs with no three nor four cycles.}%
% 	\label{fig:graph_count_no_3_nor_4_cycles_all_checks}
% \end{figure}%


\subsubsection*{Globally rigid graphs}

We also randomly generated a dataset of globally rigid graphs
up to 57 vertices.
We used a formula from yet unpublished work of John Haslegrave
that for a number of vertices gives a number of edges,
such that graphs have no or just a few NAC-colorings.
For such random graphs, we checked if they are globally rigid using PyRigi~\cite{pyrigi}.
%
The idea of monochromatic classes is so effective
that even large graphs collapse into just a few monochromatic classes.
Majority of the graphs in this dataset either has a NAC-coloring,
or only a single monochromatic class and therefore no NAC-coloring.
In our dataset, 75 percent of graphs have less than ten	monochromatic classes,
but only 10 percent of graphs have less than ten \trcon{} components
as you can see in \Cref{fig:monochrom_vs_triangle_globally_rigid}.
For minimally rigid graphs, this is not the case as expected, see
\Cref{fig:monochrom_vs_triangle_minimally_rigid}.
%
\begin{figure}[h!]
	\centering
	\begin{subfigure}{0.48\textwidth}
		\centering
		\scalebox{0.6}{%% Creator: Matplotlib, PGF backend
%%
%% To include the figure in your LaTeX document, write
%%   \input{<filename>.pgf}
%%
%% Make sure the required packages are loaded in your preamble
%%   \usepackage{pgf}
%%
%% Also ensure that all the required font packages are loaded; for instance,
%% the lmodern package is sometimes necessary when using math font.
%%   \usepackage{lmodern}
%%
%% Figures using additional raster images can only be included by \input if
%% they are in the same directory as the main LaTeX file. For loading figures
%% from other directories you can use the `import` package
%%   \usepackage{import}
%%
%% and then include the figures with
%%   \import{<path to file>}{<filename>.pgf}
%%
%% Matplotlib used the following preamble
%%   \def\mathdefault#1{#1}
%%   \everymath=\expandafter{\the\everymath\displaystyle}
%%   \IfFileExists{scrextend.sty}{
%%     \usepackage[fontsize=10.000000pt]{scrextend}
%%   }{
%%     \renewcommand{\normalsize}{\fontsize{10.000000}{12.000000}\selectfont}
%%     \normalsize
%%   }
%%   
%%   \ifdefined\pdftexversion\else  % non-pdftex case.
%%     \usepackage{fontspec}
%%     \setmainfont{DejaVuSans.ttf}[Path=\detokenize{/home/petr/Projects/PyRigi/.venv/lib/python3.12/site-packages/matplotlib/mpl-data/fonts/ttf/}]
%%     \setsansfont{DejaVuSans.ttf}[Path=\detokenize{/home/petr/Projects/PyRigi/.venv/lib/python3.12/site-packages/matplotlib/mpl-data/fonts/ttf/}]
%%     \setmonofont{DejaVuSansMono.ttf}[Path=\detokenize{/home/petr/Projects/PyRigi/.venv/lib/python3.12/site-packages/matplotlib/mpl-data/fonts/ttf/}]
%%   \fi
%%   \makeatletter\@ifpackageloaded{under\Score{}}{}{\usepackage[strings]{under\Score{}}}\makeatother
%%
\begingroup%
\makeatletter%
\begin{pgfpicture}%
\pgfpathrectangle{\pgfpointorigin}{\pgfqpoint{3.965986in}{2.317798in}}%
\pgfusepath{use as bounding box, clip}%
\begin{pgfscope}%
\pgfsetbuttcap%
\pgfsetmiterjoin%
\definecolor{currentfill}{rgb}{1.000000,1.000000,1.000000}%
\pgfsetfillcolor{currentfill}%
\pgfsetlinewidth{0.000000pt}%
\definecolor{currentstroke}{rgb}{1.000000,1.000000,1.000000}%
\pgfsetstrokecolor{currentstroke}%
\pgfsetdash{}{0pt}%
\pgfpathmoveto{\pgfqpoint{0.000000in}{0.000000in}}%
\pgfpathlineto{\pgfqpoint{3.965986in}{0.000000in}}%
\pgfpathlineto{\pgfqpoint{3.965986in}{2.317798in}}%
\pgfpathlineto{\pgfqpoint{0.000000in}{2.317798in}}%
\pgfpathlineto{\pgfqpoint{0.000000in}{0.000000in}}%
\pgfpathclose%
\pgfusepath{fill}%
\end{pgfscope}%
\begin{pgfscope}%
\pgfsetbuttcap%
\pgfsetmiterjoin%
\definecolor{currentfill}{rgb}{1.000000,1.000000,1.000000}%
\pgfsetfillcolor{currentfill}%
\pgfsetlinewidth{0.000000pt}%
\definecolor{currentstroke}{rgb}{0.000000,0.000000,0.000000}%
\pgfsetstrokecolor{currentstroke}%
\pgfsetstrokeopacity{0.000000}%
\pgfsetdash{}{0pt}%
\pgfpathmoveto{\pgfqpoint{0.664969in}{0.521603in}}%
\pgfpathlineto{\pgfqpoint{3.865986in}{0.521603in}}%
\pgfpathlineto{\pgfqpoint{3.865986in}{2.217798in}}%
\pgfpathlineto{\pgfqpoint{0.664969in}{2.217798in}}%
\pgfpathlineto{\pgfqpoint{0.664969in}{0.521603in}}%
\pgfpathclose%
\pgfusepath{fill}%
\end{pgfscope}%
\begin{pgfscope}%
\pgfpathrectangle{\pgfqpoint{0.664969in}{0.521603in}}{\pgfqpoint{3.201017in}{1.696195in}}%
\pgfusepath{clip}%
\pgfsetbuttcap%
\pgfsetmiterjoin%
\definecolor{currentfill}{rgb}{0.121569,0.466667,0.705882}%
\pgfsetfillcolor{currentfill}%
\pgfsetfillopacity{0.700000}%
\pgfsetlinewidth{0.000000pt}%
\definecolor{currentstroke}{rgb}{0.000000,0.000000,0.000000}%
\pgfsetstrokecolor{currentstroke}%
\pgfsetstrokeopacity{0.700000}%
\pgfsetdash{}{0pt}%
\pgfpathmoveto{\pgfqpoint{0.810470in}{0.521603in}}%
\pgfpathlineto{\pgfqpoint{0.841011in}{0.521603in}}%
\pgfpathlineto{\pgfqpoint{0.841011in}{2.137027in}}%
\pgfpathlineto{\pgfqpoint{0.810470in}{2.137027in}}%
\pgfpathlineto{\pgfqpoint{0.810470in}{0.521603in}}%
\pgfpathclose%
\pgfusepath{fill}%
\end{pgfscope}%
\begin{pgfscope}%
\pgfpathrectangle{\pgfqpoint{0.664969in}{0.521603in}}{\pgfqpoint{3.201017in}{1.696195in}}%
\pgfusepath{clip}%
\pgfsetbuttcap%
\pgfsetmiterjoin%
\definecolor{currentfill}{rgb}{0.121569,0.466667,0.705882}%
\pgfsetfillcolor{currentfill}%
\pgfsetfillopacity{0.700000}%
\pgfsetlinewidth{0.000000pt}%
\definecolor{currentstroke}{rgb}{0.000000,0.000000,0.000000}%
\pgfsetstrokecolor{currentstroke}%
\pgfsetstrokeopacity{0.700000}%
\pgfsetdash{}{0pt}%
\pgfpathmoveto{\pgfqpoint{0.841011in}{0.521603in}}%
\pgfpathlineto{\pgfqpoint{0.871551in}{0.521603in}}%
\pgfpathlineto{\pgfqpoint{0.871551in}{1.243253in}}%
\pgfpathlineto{\pgfqpoint{0.841011in}{1.243253in}}%
\pgfpathlineto{\pgfqpoint{0.841011in}{0.521603in}}%
\pgfpathclose%
\pgfusepath{fill}%
\end{pgfscope}%
\begin{pgfscope}%
\pgfpathrectangle{\pgfqpoint{0.664969in}{0.521603in}}{\pgfqpoint{3.201017in}{1.696195in}}%
\pgfusepath{clip}%
\pgfsetbuttcap%
\pgfsetmiterjoin%
\definecolor{currentfill}{rgb}{0.121569,0.466667,0.705882}%
\pgfsetfillcolor{currentfill}%
\pgfsetfillopacity{0.700000}%
\pgfsetlinewidth{0.000000pt}%
\definecolor{currentstroke}{rgb}{0.000000,0.000000,0.000000}%
\pgfsetstrokecolor{currentstroke}%
\pgfsetstrokeopacity{0.700000}%
\pgfsetdash{}{0pt}%
\pgfpathmoveto{\pgfqpoint{0.871551in}{0.521603in}}%
\pgfpathlineto{\pgfqpoint{0.902092in}{0.521603in}}%
\pgfpathlineto{\pgfqpoint{0.902092in}{1.091528in}}%
\pgfpathlineto{\pgfqpoint{0.871551in}{1.091528in}}%
\pgfpathlineto{\pgfqpoint{0.871551in}{0.521603in}}%
\pgfpathclose%
\pgfusepath{fill}%
\end{pgfscope}%
\begin{pgfscope}%
\pgfpathrectangle{\pgfqpoint{0.664969in}{0.521603in}}{\pgfqpoint{3.201017in}{1.696195in}}%
\pgfusepath{clip}%
\pgfsetbuttcap%
\pgfsetmiterjoin%
\definecolor{currentfill}{rgb}{0.121569,0.466667,0.705882}%
\pgfsetfillcolor{currentfill}%
\pgfsetfillopacity{0.700000}%
\pgfsetlinewidth{0.000000pt}%
\definecolor{currentstroke}{rgb}{0.000000,0.000000,0.000000}%
\pgfsetstrokecolor{currentstroke}%
\pgfsetstrokeopacity{0.700000}%
\pgfsetdash{}{0pt}%
\pgfpathmoveto{\pgfqpoint{0.902092in}{0.521603in}}%
\pgfpathlineto{\pgfqpoint{0.932632in}{0.521603in}}%
\pgfpathlineto{\pgfqpoint{0.932632in}{1.057103in}}%
\pgfpathlineto{\pgfqpoint{0.902092in}{1.057103in}}%
\pgfpathlineto{\pgfqpoint{0.902092in}{0.521603in}}%
\pgfpathclose%
\pgfusepath{fill}%
\end{pgfscope}%
\begin{pgfscope}%
\pgfpathrectangle{\pgfqpoint{0.664969in}{0.521603in}}{\pgfqpoint{3.201017in}{1.696195in}}%
\pgfusepath{clip}%
\pgfsetbuttcap%
\pgfsetmiterjoin%
\definecolor{currentfill}{rgb}{0.121569,0.466667,0.705882}%
\pgfsetfillcolor{currentfill}%
\pgfsetfillopacity{0.700000}%
\pgfsetlinewidth{0.000000pt}%
\definecolor{currentstroke}{rgb}{0.000000,0.000000,0.000000}%
\pgfsetstrokecolor{currentstroke}%
\pgfsetstrokeopacity{0.700000}%
\pgfsetdash{}{0pt}%
\pgfpathmoveto{\pgfqpoint{0.932632in}{0.521603in}}%
\pgfpathlineto{\pgfqpoint{0.963173in}{0.521603in}}%
\pgfpathlineto{\pgfqpoint{0.963173in}{0.976778in}}%
\pgfpathlineto{\pgfqpoint{0.932632in}{0.976778in}}%
\pgfpathlineto{\pgfqpoint{0.932632in}{0.521603in}}%
\pgfpathclose%
\pgfusepath{fill}%
\end{pgfscope}%
\begin{pgfscope}%
\pgfpathrectangle{\pgfqpoint{0.664969in}{0.521603in}}{\pgfqpoint{3.201017in}{1.696195in}}%
\pgfusepath{clip}%
\pgfsetbuttcap%
\pgfsetmiterjoin%
\definecolor{currentfill}{rgb}{0.121569,0.466667,0.705882}%
\pgfsetfillcolor{currentfill}%
\pgfsetfillopacity{0.700000}%
\pgfsetlinewidth{0.000000pt}%
\definecolor{currentstroke}{rgb}{0.000000,0.000000,0.000000}%
\pgfsetstrokecolor{currentstroke}%
\pgfsetstrokeopacity{0.700000}%
\pgfsetdash{}{0pt}%
\pgfpathmoveto{\pgfqpoint{0.963173in}{0.521603in}}%
\pgfpathlineto{\pgfqpoint{0.993713in}{0.521603in}}%
\pgfpathlineto{\pgfqpoint{0.993713in}{1.127228in}}%
\pgfpathlineto{\pgfqpoint{0.963173in}{1.127228in}}%
\pgfpathlineto{\pgfqpoint{0.963173in}{0.521603in}}%
\pgfpathclose%
\pgfusepath{fill}%
\end{pgfscope}%
\begin{pgfscope}%
\pgfpathrectangle{\pgfqpoint{0.664969in}{0.521603in}}{\pgfqpoint{3.201017in}{1.696195in}}%
\pgfusepath{clip}%
\pgfsetbuttcap%
\pgfsetmiterjoin%
\definecolor{currentfill}{rgb}{0.121569,0.466667,0.705882}%
\pgfsetfillcolor{currentfill}%
\pgfsetfillopacity{0.700000}%
\pgfsetlinewidth{0.000000pt}%
\definecolor{currentstroke}{rgb}{0.000000,0.000000,0.000000}%
\pgfsetstrokecolor{currentstroke}%
\pgfsetstrokeopacity{0.700000}%
\pgfsetdash{}{0pt}%
\pgfpathmoveto{\pgfqpoint{0.993713in}{0.521603in}}%
\pgfpathlineto{\pgfqpoint{1.024254in}{0.521603in}}%
\pgfpathlineto{\pgfqpoint{1.024254in}{0.809753in}}%
\pgfpathlineto{\pgfqpoint{0.993713in}{0.809753in}}%
\pgfpathlineto{\pgfqpoint{0.993713in}{0.521603in}}%
\pgfpathclose%
\pgfusepath{fill}%
\end{pgfscope}%
\begin{pgfscope}%
\pgfpathrectangle{\pgfqpoint{0.664969in}{0.521603in}}{\pgfqpoint{3.201017in}{1.696195in}}%
\pgfusepath{clip}%
\pgfsetbuttcap%
\pgfsetmiterjoin%
\definecolor{currentfill}{rgb}{0.121569,0.466667,0.705882}%
\pgfsetfillcolor{currentfill}%
\pgfsetfillopacity{0.700000}%
\pgfsetlinewidth{0.000000pt}%
\definecolor{currentstroke}{rgb}{0.000000,0.000000,0.000000}%
\pgfsetstrokecolor{currentstroke}%
\pgfsetstrokeopacity{0.700000}%
\pgfsetdash{}{0pt}%
\pgfpathmoveto{\pgfqpoint{1.024254in}{0.521603in}}%
\pgfpathlineto{\pgfqpoint{1.054795in}{0.521603in}}%
\pgfpathlineto{\pgfqpoint{1.054795in}{0.688628in}}%
\pgfpathlineto{\pgfqpoint{1.024254in}{0.688628in}}%
\pgfpathlineto{\pgfqpoint{1.024254in}{0.521603in}}%
\pgfpathclose%
\pgfusepath{fill}%
\end{pgfscope}%
\begin{pgfscope}%
\pgfpathrectangle{\pgfqpoint{0.664969in}{0.521603in}}{\pgfqpoint{3.201017in}{1.696195in}}%
\pgfusepath{clip}%
\pgfsetbuttcap%
\pgfsetmiterjoin%
\definecolor{currentfill}{rgb}{0.121569,0.466667,0.705882}%
\pgfsetfillcolor{currentfill}%
\pgfsetfillopacity{0.700000}%
\pgfsetlinewidth{0.000000pt}%
\definecolor{currentstroke}{rgb}{0.000000,0.000000,0.000000}%
\pgfsetstrokecolor{currentstroke}%
\pgfsetstrokeopacity{0.700000}%
\pgfsetdash{}{0pt}%
\pgfpathmoveto{\pgfqpoint{1.054795in}{0.521603in}}%
\pgfpathlineto{\pgfqpoint{1.085335in}{0.521603in}}%
\pgfpathlineto{\pgfqpoint{1.085335in}{0.715403in}}%
\pgfpathlineto{\pgfqpoint{1.054795in}{0.715403in}}%
\pgfpathlineto{\pgfqpoint{1.054795in}{0.521603in}}%
\pgfpathclose%
\pgfusepath{fill}%
\end{pgfscope}%
\begin{pgfscope}%
\pgfpathrectangle{\pgfqpoint{0.664969in}{0.521603in}}{\pgfqpoint{3.201017in}{1.696195in}}%
\pgfusepath{clip}%
\pgfsetbuttcap%
\pgfsetmiterjoin%
\definecolor{currentfill}{rgb}{0.121569,0.466667,0.705882}%
\pgfsetfillcolor{currentfill}%
\pgfsetfillopacity{0.700000}%
\pgfsetlinewidth{0.000000pt}%
\definecolor{currentstroke}{rgb}{0.000000,0.000000,0.000000}%
\pgfsetstrokecolor{currentstroke}%
\pgfsetstrokeopacity{0.700000}%
\pgfsetdash{}{0pt}%
\pgfpathmoveto{\pgfqpoint{1.085335in}{0.521603in}}%
\pgfpathlineto{\pgfqpoint{1.115876in}{0.521603in}}%
\pgfpathlineto{\pgfqpoint{1.115876in}{0.659303in}}%
\pgfpathlineto{\pgfqpoint{1.085335in}{0.659303in}}%
\pgfpathlineto{\pgfqpoint{1.085335in}{0.521603in}}%
\pgfpathclose%
\pgfusepath{fill}%
\end{pgfscope}%
\begin{pgfscope}%
\pgfpathrectangle{\pgfqpoint{0.664969in}{0.521603in}}{\pgfqpoint{3.201017in}{1.696195in}}%
\pgfusepath{clip}%
\pgfsetbuttcap%
\pgfsetmiterjoin%
\definecolor{currentfill}{rgb}{0.121569,0.466667,0.705882}%
\pgfsetfillcolor{currentfill}%
\pgfsetfillopacity{0.700000}%
\pgfsetlinewidth{0.000000pt}%
\definecolor{currentstroke}{rgb}{0.000000,0.000000,0.000000}%
\pgfsetstrokecolor{currentstroke}%
\pgfsetstrokeopacity{0.700000}%
\pgfsetdash{}{0pt}%
\pgfpathmoveto{\pgfqpoint{1.115876in}{0.521603in}}%
\pgfpathlineto{\pgfqpoint{1.146416in}{0.521603in}}%
\pgfpathlineto{\pgfqpoint{1.146416in}{0.660578in}}%
\pgfpathlineto{\pgfqpoint{1.115876in}{0.660578in}}%
\pgfpathlineto{\pgfqpoint{1.115876in}{0.521603in}}%
\pgfpathclose%
\pgfusepath{fill}%
\end{pgfscope}%
\begin{pgfscope}%
\pgfpathrectangle{\pgfqpoint{0.664969in}{0.521603in}}{\pgfqpoint{3.201017in}{1.696195in}}%
\pgfusepath{clip}%
\pgfsetbuttcap%
\pgfsetmiterjoin%
\definecolor{currentfill}{rgb}{0.121569,0.466667,0.705882}%
\pgfsetfillcolor{currentfill}%
\pgfsetfillopacity{0.700000}%
\pgfsetlinewidth{0.000000pt}%
\definecolor{currentstroke}{rgb}{0.000000,0.000000,0.000000}%
\pgfsetstrokecolor{currentstroke}%
\pgfsetstrokeopacity{0.700000}%
\pgfsetdash{}{0pt}%
\pgfpathmoveto{\pgfqpoint{1.146416in}{0.521603in}}%
\pgfpathlineto{\pgfqpoint{1.176957in}{0.521603in}}%
\pgfpathlineto{\pgfqpoint{1.176957in}{0.710303in}}%
\pgfpathlineto{\pgfqpoint{1.146416in}{0.710303in}}%
\pgfpathlineto{\pgfqpoint{1.146416in}{0.521603in}}%
\pgfpathclose%
\pgfusepath{fill}%
\end{pgfscope}%
\begin{pgfscope}%
\pgfpathrectangle{\pgfqpoint{0.664969in}{0.521603in}}{\pgfqpoint{3.201017in}{1.696195in}}%
\pgfusepath{clip}%
\pgfsetbuttcap%
\pgfsetmiterjoin%
\definecolor{currentfill}{rgb}{0.121569,0.466667,0.705882}%
\pgfsetfillcolor{currentfill}%
\pgfsetfillopacity{0.700000}%
\pgfsetlinewidth{0.000000pt}%
\definecolor{currentstroke}{rgb}{0.000000,0.000000,0.000000}%
\pgfsetstrokecolor{currentstroke}%
\pgfsetstrokeopacity{0.700000}%
\pgfsetdash{}{0pt}%
\pgfpathmoveto{\pgfqpoint{1.176957in}{0.521603in}}%
\pgfpathlineto{\pgfqpoint{1.207497in}{0.521603in}}%
\pgfpathlineto{\pgfqpoint{1.207497in}{0.587903in}}%
\pgfpathlineto{\pgfqpoint{1.176957in}{0.587903in}}%
\pgfpathlineto{\pgfqpoint{1.176957in}{0.521603in}}%
\pgfpathclose%
\pgfusepath{fill}%
\end{pgfscope}%
\begin{pgfscope}%
\pgfpathrectangle{\pgfqpoint{0.664969in}{0.521603in}}{\pgfqpoint{3.201017in}{1.696195in}}%
\pgfusepath{clip}%
\pgfsetbuttcap%
\pgfsetmiterjoin%
\definecolor{currentfill}{rgb}{0.121569,0.466667,0.705882}%
\pgfsetfillcolor{currentfill}%
\pgfsetfillopacity{0.700000}%
\pgfsetlinewidth{0.000000pt}%
\definecolor{currentstroke}{rgb}{0.000000,0.000000,0.000000}%
\pgfsetstrokecolor{currentstroke}%
\pgfsetstrokeopacity{0.700000}%
\pgfsetdash{}{0pt}%
\pgfpathmoveto{\pgfqpoint{1.207497in}{0.521603in}}%
\pgfpathlineto{\pgfqpoint{1.238038in}{0.521603in}}%
\pgfpathlineto{\pgfqpoint{1.238038in}{0.577703in}}%
\pgfpathlineto{\pgfqpoint{1.207497in}{0.577703in}}%
\pgfpathlineto{\pgfqpoint{1.207497in}{0.521603in}}%
\pgfpathclose%
\pgfusepath{fill}%
\end{pgfscope}%
\begin{pgfscope}%
\pgfpathrectangle{\pgfqpoint{0.664969in}{0.521603in}}{\pgfqpoint{3.201017in}{1.696195in}}%
\pgfusepath{clip}%
\pgfsetbuttcap%
\pgfsetmiterjoin%
\definecolor{currentfill}{rgb}{0.121569,0.466667,0.705882}%
\pgfsetfillcolor{currentfill}%
\pgfsetfillopacity{0.700000}%
\pgfsetlinewidth{0.000000pt}%
\definecolor{currentstroke}{rgb}{0.000000,0.000000,0.000000}%
\pgfsetstrokecolor{currentstroke}%
\pgfsetstrokeopacity{0.700000}%
\pgfsetdash{}{0pt}%
\pgfpathmoveto{\pgfqpoint{1.238038in}{0.521603in}}%
\pgfpathlineto{\pgfqpoint{1.268579in}{0.521603in}}%
\pgfpathlineto{\pgfqpoint{1.268579in}{0.562403in}}%
\pgfpathlineto{\pgfqpoint{1.238038in}{0.562403in}}%
\pgfpathlineto{\pgfqpoint{1.238038in}{0.521603in}}%
\pgfpathclose%
\pgfusepath{fill}%
\end{pgfscope}%
\begin{pgfscope}%
\pgfpathrectangle{\pgfqpoint{0.664969in}{0.521603in}}{\pgfqpoint{3.201017in}{1.696195in}}%
\pgfusepath{clip}%
\pgfsetbuttcap%
\pgfsetmiterjoin%
\definecolor{currentfill}{rgb}{0.121569,0.466667,0.705882}%
\pgfsetfillcolor{currentfill}%
\pgfsetfillopacity{0.700000}%
\pgfsetlinewidth{0.000000pt}%
\definecolor{currentstroke}{rgb}{0.000000,0.000000,0.000000}%
\pgfsetstrokecolor{currentstroke}%
\pgfsetstrokeopacity{0.700000}%
\pgfsetdash{}{0pt}%
\pgfpathmoveto{\pgfqpoint{1.268579in}{0.521603in}}%
\pgfpathlineto{\pgfqpoint{1.299119in}{0.521603in}}%
\pgfpathlineto{\pgfqpoint{1.299119in}{0.563678in}}%
\pgfpathlineto{\pgfqpoint{1.268579in}{0.563678in}}%
\pgfpathlineto{\pgfqpoint{1.268579in}{0.521603in}}%
\pgfpathclose%
\pgfusepath{fill}%
\end{pgfscope}%
\begin{pgfscope}%
\pgfpathrectangle{\pgfqpoint{0.664969in}{0.521603in}}{\pgfqpoint{3.201017in}{1.696195in}}%
\pgfusepath{clip}%
\pgfsetbuttcap%
\pgfsetmiterjoin%
\definecolor{currentfill}{rgb}{0.121569,0.466667,0.705882}%
\pgfsetfillcolor{currentfill}%
\pgfsetfillopacity{0.700000}%
\pgfsetlinewidth{0.000000pt}%
\definecolor{currentstroke}{rgb}{0.000000,0.000000,0.000000}%
\pgfsetstrokecolor{currentstroke}%
\pgfsetstrokeopacity{0.700000}%
\pgfsetdash{}{0pt}%
\pgfpathmoveto{\pgfqpoint{1.299119in}{0.521603in}}%
\pgfpathlineto{\pgfqpoint{1.329660in}{0.521603in}}%
\pgfpathlineto{\pgfqpoint{1.329660in}{0.561128in}}%
\pgfpathlineto{\pgfqpoint{1.299119in}{0.561128in}}%
\pgfpathlineto{\pgfqpoint{1.299119in}{0.521603in}}%
\pgfpathclose%
\pgfusepath{fill}%
\end{pgfscope}%
\begin{pgfscope}%
\pgfpathrectangle{\pgfqpoint{0.664969in}{0.521603in}}{\pgfqpoint{3.201017in}{1.696195in}}%
\pgfusepath{clip}%
\pgfsetbuttcap%
\pgfsetmiterjoin%
\definecolor{currentfill}{rgb}{0.121569,0.466667,0.705882}%
\pgfsetfillcolor{currentfill}%
\pgfsetfillopacity{0.700000}%
\pgfsetlinewidth{0.000000pt}%
\definecolor{currentstroke}{rgb}{0.000000,0.000000,0.000000}%
\pgfsetstrokecolor{currentstroke}%
\pgfsetstrokeopacity{0.700000}%
\pgfsetdash{}{0pt}%
\pgfpathmoveto{\pgfqpoint{1.329660in}{0.521603in}}%
\pgfpathlineto{\pgfqpoint{1.360200in}{0.521603in}}%
\pgfpathlineto{\pgfqpoint{1.360200in}{0.567503in}}%
\pgfpathlineto{\pgfqpoint{1.329660in}{0.567503in}}%
\pgfpathlineto{\pgfqpoint{1.329660in}{0.521603in}}%
\pgfpathclose%
\pgfusepath{fill}%
\end{pgfscope}%
\begin{pgfscope}%
\pgfpathrectangle{\pgfqpoint{0.664969in}{0.521603in}}{\pgfqpoint{3.201017in}{1.696195in}}%
\pgfusepath{clip}%
\pgfsetbuttcap%
\pgfsetmiterjoin%
\definecolor{currentfill}{rgb}{0.121569,0.466667,0.705882}%
\pgfsetfillcolor{currentfill}%
\pgfsetfillopacity{0.700000}%
\pgfsetlinewidth{0.000000pt}%
\definecolor{currentstroke}{rgb}{0.000000,0.000000,0.000000}%
\pgfsetstrokecolor{currentstroke}%
\pgfsetstrokeopacity{0.700000}%
\pgfsetdash{}{0pt}%
\pgfpathmoveto{\pgfqpoint{1.360200in}{0.521603in}}%
\pgfpathlineto{\pgfqpoint{1.390741in}{0.521603in}}%
\pgfpathlineto{\pgfqpoint{1.390741in}{0.534353in}}%
\pgfpathlineto{\pgfqpoint{1.360200in}{0.534353in}}%
\pgfpathlineto{\pgfqpoint{1.360200in}{0.521603in}}%
\pgfpathclose%
\pgfusepath{fill}%
\end{pgfscope}%
\begin{pgfscope}%
\pgfpathrectangle{\pgfqpoint{0.664969in}{0.521603in}}{\pgfqpoint{3.201017in}{1.696195in}}%
\pgfusepath{clip}%
\pgfsetbuttcap%
\pgfsetmiterjoin%
\definecolor{currentfill}{rgb}{0.121569,0.466667,0.705882}%
\pgfsetfillcolor{currentfill}%
\pgfsetfillopacity{0.700000}%
\pgfsetlinewidth{0.000000pt}%
\definecolor{currentstroke}{rgb}{0.000000,0.000000,0.000000}%
\pgfsetstrokecolor{currentstroke}%
\pgfsetstrokeopacity{0.700000}%
\pgfsetdash{}{0pt}%
\pgfpathmoveto{\pgfqpoint{1.390741in}{0.521603in}}%
\pgfpathlineto{\pgfqpoint{1.421281in}{0.521603in}}%
\pgfpathlineto{\pgfqpoint{1.421281in}{0.529253in}}%
\pgfpathlineto{\pgfqpoint{1.390741in}{0.529253in}}%
\pgfpathlineto{\pgfqpoint{1.390741in}{0.521603in}}%
\pgfpathclose%
\pgfusepath{fill}%
\end{pgfscope}%
\begin{pgfscope}%
\pgfpathrectangle{\pgfqpoint{0.664969in}{0.521603in}}{\pgfqpoint{3.201017in}{1.696195in}}%
\pgfusepath{clip}%
\pgfsetbuttcap%
\pgfsetmiterjoin%
\definecolor{currentfill}{rgb}{0.121569,0.466667,0.705882}%
\pgfsetfillcolor{currentfill}%
\pgfsetfillopacity{0.700000}%
\pgfsetlinewidth{0.000000pt}%
\definecolor{currentstroke}{rgb}{0.000000,0.000000,0.000000}%
\pgfsetstrokecolor{currentstroke}%
\pgfsetstrokeopacity{0.700000}%
\pgfsetdash{}{0pt}%
\pgfpathmoveto{\pgfqpoint{1.421281in}{0.521603in}}%
\pgfpathlineto{\pgfqpoint{1.451822in}{0.521603in}}%
\pgfpathlineto{\pgfqpoint{1.451822in}{0.534353in}}%
\pgfpathlineto{\pgfqpoint{1.421281in}{0.534353in}}%
\pgfpathlineto{\pgfqpoint{1.421281in}{0.521603in}}%
\pgfpathclose%
\pgfusepath{fill}%
\end{pgfscope}%
\begin{pgfscope}%
\pgfpathrectangle{\pgfqpoint{0.664969in}{0.521603in}}{\pgfqpoint{3.201017in}{1.696195in}}%
\pgfusepath{clip}%
\pgfsetbuttcap%
\pgfsetmiterjoin%
\definecolor{currentfill}{rgb}{0.121569,0.466667,0.705882}%
\pgfsetfillcolor{currentfill}%
\pgfsetfillopacity{0.700000}%
\pgfsetlinewidth{0.000000pt}%
\definecolor{currentstroke}{rgb}{0.000000,0.000000,0.000000}%
\pgfsetstrokecolor{currentstroke}%
\pgfsetstrokeopacity{0.700000}%
\pgfsetdash{}{0pt}%
\pgfpathmoveto{\pgfqpoint{1.451822in}{0.521603in}}%
\pgfpathlineto{\pgfqpoint{1.482363in}{0.521603in}}%
\pgfpathlineto{\pgfqpoint{1.482363in}{0.529253in}}%
\pgfpathlineto{\pgfqpoint{1.451822in}{0.529253in}}%
\pgfpathlineto{\pgfqpoint{1.451822in}{0.521603in}}%
\pgfpathclose%
\pgfusepath{fill}%
\end{pgfscope}%
\begin{pgfscope}%
\pgfpathrectangle{\pgfqpoint{0.664969in}{0.521603in}}{\pgfqpoint{3.201017in}{1.696195in}}%
\pgfusepath{clip}%
\pgfsetbuttcap%
\pgfsetmiterjoin%
\definecolor{currentfill}{rgb}{0.121569,0.466667,0.705882}%
\pgfsetfillcolor{currentfill}%
\pgfsetfillopacity{0.700000}%
\pgfsetlinewidth{0.000000pt}%
\definecolor{currentstroke}{rgb}{0.000000,0.000000,0.000000}%
\pgfsetstrokecolor{currentstroke}%
\pgfsetstrokeopacity{0.700000}%
\pgfsetdash{}{0pt}%
\pgfpathmoveto{\pgfqpoint{1.482363in}{0.521603in}}%
\pgfpathlineto{\pgfqpoint{1.512903in}{0.521603in}}%
\pgfpathlineto{\pgfqpoint{1.512903in}{0.542003in}}%
\pgfpathlineto{\pgfqpoint{1.482363in}{0.542003in}}%
\pgfpathlineto{\pgfqpoint{1.482363in}{0.521603in}}%
\pgfpathclose%
\pgfusepath{fill}%
\end{pgfscope}%
\begin{pgfscope}%
\pgfpathrectangle{\pgfqpoint{0.664969in}{0.521603in}}{\pgfqpoint{3.201017in}{1.696195in}}%
\pgfusepath{clip}%
\pgfsetbuttcap%
\pgfsetmiterjoin%
\definecolor{currentfill}{rgb}{0.121569,0.466667,0.705882}%
\pgfsetfillcolor{currentfill}%
\pgfsetfillopacity{0.700000}%
\pgfsetlinewidth{0.000000pt}%
\definecolor{currentstroke}{rgb}{0.000000,0.000000,0.000000}%
\pgfsetstrokecolor{currentstroke}%
\pgfsetstrokeopacity{0.700000}%
\pgfsetdash{}{0pt}%
\pgfpathmoveto{\pgfqpoint{1.512903in}{0.521603in}}%
\pgfpathlineto{\pgfqpoint{1.543444in}{0.521603in}}%
\pgfpathlineto{\pgfqpoint{1.543444in}{0.534353in}}%
\pgfpathlineto{\pgfqpoint{1.512903in}{0.534353in}}%
\pgfpathlineto{\pgfqpoint{1.512903in}{0.521603in}}%
\pgfpathclose%
\pgfusepath{fill}%
\end{pgfscope}%
\begin{pgfscope}%
\pgfpathrectangle{\pgfqpoint{0.664969in}{0.521603in}}{\pgfqpoint{3.201017in}{1.696195in}}%
\pgfusepath{clip}%
\pgfsetbuttcap%
\pgfsetmiterjoin%
\definecolor{currentfill}{rgb}{0.121569,0.466667,0.705882}%
\pgfsetfillcolor{currentfill}%
\pgfsetfillopacity{0.700000}%
\pgfsetlinewidth{0.000000pt}%
\definecolor{currentstroke}{rgb}{0.000000,0.000000,0.000000}%
\pgfsetstrokecolor{currentstroke}%
\pgfsetstrokeopacity{0.700000}%
\pgfsetdash{}{0pt}%
\pgfpathmoveto{\pgfqpoint{1.543444in}{0.521603in}}%
\pgfpathlineto{\pgfqpoint{1.573984in}{0.521603in}}%
\pgfpathlineto{\pgfqpoint{1.573984in}{0.533078in}}%
\pgfpathlineto{\pgfqpoint{1.543444in}{0.533078in}}%
\pgfpathlineto{\pgfqpoint{1.543444in}{0.521603in}}%
\pgfpathclose%
\pgfusepath{fill}%
\end{pgfscope}%
\begin{pgfscope}%
\pgfpathrectangle{\pgfqpoint{0.664969in}{0.521603in}}{\pgfqpoint{3.201017in}{1.696195in}}%
\pgfusepath{clip}%
\pgfsetbuttcap%
\pgfsetmiterjoin%
\definecolor{currentfill}{rgb}{0.121569,0.466667,0.705882}%
\pgfsetfillcolor{currentfill}%
\pgfsetfillopacity{0.700000}%
\pgfsetlinewidth{0.000000pt}%
\definecolor{currentstroke}{rgb}{0.000000,0.000000,0.000000}%
\pgfsetstrokecolor{currentstroke}%
\pgfsetstrokeopacity{0.700000}%
\pgfsetdash{}{0pt}%
\pgfpathmoveto{\pgfqpoint{1.573984in}{0.521603in}}%
\pgfpathlineto{\pgfqpoint{1.604525in}{0.521603in}}%
\pgfpathlineto{\pgfqpoint{1.604525in}{0.526703in}}%
\pgfpathlineto{\pgfqpoint{1.573984in}{0.526703in}}%
\pgfpathlineto{\pgfqpoint{1.573984in}{0.521603in}}%
\pgfpathclose%
\pgfusepath{fill}%
\end{pgfscope}%
\begin{pgfscope}%
\pgfpathrectangle{\pgfqpoint{0.664969in}{0.521603in}}{\pgfqpoint{3.201017in}{1.696195in}}%
\pgfusepath{clip}%
\pgfsetbuttcap%
\pgfsetmiterjoin%
\definecolor{currentfill}{rgb}{0.121569,0.466667,0.705882}%
\pgfsetfillcolor{currentfill}%
\pgfsetfillopacity{0.700000}%
\pgfsetlinewidth{0.000000pt}%
\definecolor{currentstroke}{rgb}{0.000000,0.000000,0.000000}%
\pgfsetstrokecolor{currentstroke}%
\pgfsetstrokeopacity{0.700000}%
\pgfsetdash{}{0pt}%
\pgfpathmoveto{\pgfqpoint{1.604525in}{0.521603in}}%
\pgfpathlineto{\pgfqpoint{1.635065in}{0.521603in}}%
\pgfpathlineto{\pgfqpoint{1.635065in}{0.525428in}}%
\pgfpathlineto{\pgfqpoint{1.604525in}{0.525428in}}%
\pgfpathlineto{\pgfqpoint{1.604525in}{0.521603in}}%
\pgfpathclose%
\pgfusepath{fill}%
\end{pgfscope}%
\begin{pgfscope}%
\pgfpathrectangle{\pgfqpoint{0.664969in}{0.521603in}}{\pgfqpoint{3.201017in}{1.696195in}}%
\pgfusepath{clip}%
\pgfsetbuttcap%
\pgfsetmiterjoin%
\definecolor{currentfill}{rgb}{0.121569,0.466667,0.705882}%
\pgfsetfillcolor{currentfill}%
\pgfsetfillopacity{0.700000}%
\pgfsetlinewidth{0.000000pt}%
\definecolor{currentstroke}{rgb}{0.000000,0.000000,0.000000}%
\pgfsetstrokecolor{currentstroke}%
\pgfsetstrokeopacity{0.700000}%
\pgfsetdash{}{0pt}%
\pgfpathmoveto{\pgfqpoint{1.635065in}{0.521603in}}%
\pgfpathlineto{\pgfqpoint{1.665606in}{0.521603in}}%
\pgfpathlineto{\pgfqpoint{1.665606in}{0.522878in}}%
\pgfpathlineto{\pgfqpoint{1.635065in}{0.522878in}}%
\pgfpathlineto{\pgfqpoint{1.635065in}{0.521603in}}%
\pgfpathclose%
\pgfusepath{fill}%
\end{pgfscope}%
\begin{pgfscope}%
\pgfpathrectangle{\pgfqpoint{0.664969in}{0.521603in}}{\pgfqpoint{3.201017in}{1.696195in}}%
\pgfusepath{clip}%
\pgfsetbuttcap%
\pgfsetmiterjoin%
\definecolor{currentfill}{rgb}{0.121569,0.466667,0.705882}%
\pgfsetfillcolor{currentfill}%
\pgfsetfillopacity{0.700000}%
\pgfsetlinewidth{0.000000pt}%
\definecolor{currentstroke}{rgb}{0.000000,0.000000,0.000000}%
\pgfsetstrokecolor{currentstroke}%
\pgfsetstrokeopacity{0.700000}%
\pgfsetdash{}{0pt}%
\pgfpathmoveto{\pgfqpoint{1.665606in}{0.521603in}}%
\pgfpathlineto{\pgfqpoint{1.696147in}{0.521603in}}%
\pgfpathlineto{\pgfqpoint{1.696147in}{0.530528in}}%
\pgfpathlineto{\pgfqpoint{1.665606in}{0.530528in}}%
\pgfpathlineto{\pgfqpoint{1.665606in}{0.521603in}}%
\pgfpathclose%
\pgfusepath{fill}%
\end{pgfscope}%
\begin{pgfscope}%
\pgfpathrectangle{\pgfqpoint{0.664969in}{0.521603in}}{\pgfqpoint{3.201017in}{1.696195in}}%
\pgfusepath{clip}%
\pgfsetbuttcap%
\pgfsetmiterjoin%
\definecolor{currentfill}{rgb}{0.121569,0.466667,0.705882}%
\pgfsetfillcolor{currentfill}%
\pgfsetfillopacity{0.700000}%
\pgfsetlinewidth{0.000000pt}%
\definecolor{currentstroke}{rgb}{0.000000,0.000000,0.000000}%
\pgfsetstrokecolor{currentstroke}%
\pgfsetstrokeopacity{0.700000}%
\pgfsetdash{}{0pt}%
\pgfpathmoveto{\pgfqpoint{1.696147in}{0.521603in}}%
\pgfpathlineto{\pgfqpoint{1.726687in}{0.521603in}}%
\pgfpathlineto{\pgfqpoint{1.726687in}{0.524153in}}%
\pgfpathlineto{\pgfqpoint{1.696147in}{0.524153in}}%
\pgfpathlineto{\pgfqpoint{1.696147in}{0.521603in}}%
\pgfpathclose%
\pgfusepath{fill}%
\end{pgfscope}%
\begin{pgfscope}%
\pgfpathrectangle{\pgfqpoint{0.664969in}{0.521603in}}{\pgfqpoint{3.201017in}{1.696195in}}%
\pgfusepath{clip}%
\pgfsetbuttcap%
\pgfsetmiterjoin%
\definecolor{currentfill}{rgb}{0.121569,0.466667,0.705882}%
\pgfsetfillcolor{currentfill}%
\pgfsetfillopacity{0.700000}%
\pgfsetlinewidth{0.000000pt}%
\definecolor{currentstroke}{rgb}{0.000000,0.000000,0.000000}%
\pgfsetstrokecolor{currentstroke}%
\pgfsetstrokeopacity{0.700000}%
\pgfsetdash{}{0pt}%
\pgfpathmoveto{\pgfqpoint{1.726687in}{0.521603in}}%
\pgfpathlineto{\pgfqpoint{1.757228in}{0.521603in}}%
\pgfpathlineto{\pgfqpoint{1.757228in}{0.526703in}}%
\pgfpathlineto{\pgfqpoint{1.726687in}{0.526703in}}%
\pgfpathlineto{\pgfqpoint{1.726687in}{0.521603in}}%
\pgfpathclose%
\pgfusepath{fill}%
\end{pgfscope}%
\begin{pgfscope}%
\pgfpathrectangle{\pgfqpoint{0.664969in}{0.521603in}}{\pgfqpoint{3.201017in}{1.696195in}}%
\pgfusepath{clip}%
\pgfsetbuttcap%
\pgfsetmiterjoin%
\definecolor{currentfill}{rgb}{0.121569,0.466667,0.705882}%
\pgfsetfillcolor{currentfill}%
\pgfsetfillopacity{0.700000}%
\pgfsetlinewidth{0.000000pt}%
\definecolor{currentstroke}{rgb}{0.000000,0.000000,0.000000}%
\pgfsetstrokecolor{currentstroke}%
\pgfsetstrokeopacity{0.700000}%
\pgfsetdash{}{0pt}%
\pgfpathmoveto{\pgfqpoint{1.757228in}{0.521603in}}%
\pgfpathlineto{\pgfqpoint{1.787768in}{0.521603in}}%
\pgfpathlineto{\pgfqpoint{1.787768in}{0.522878in}}%
\pgfpathlineto{\pgfqpoint{1.757228in}{0.522878in}}%
\pgfpathlineto{\pgfqpoint{1.757228in}{0.521603in}}%
\pgfpathclose%
\pgfusepath{fill}%
\end{pgfscope}%
\begin{pgfscope}%
\pgfpathrectangle{\pgfqpoint{0.664969in}{0.521603in}}{\pgfqpoint{3.201017in}{1.696195in}}%
\pgfusepath{clip}%
\pgfsetbuttcap%
\pgfsetmiterjoin%
\definecolor{currentfill}{rgb}{0.121569,0.466667,0.705882}%
\pgfsetfillcolor{currentfill}%
\pgfsetfillopacity{0.700000}%
\pgfsetlinewidth{0.000000pt}%
\definecolor{currentstroke}{rgb}{0.000000,0.000000,0.000000}%
\pgfsetstrokecolor{currentstroke}%
\pgfsetstrokeopacity{0.700000}%
\pgfsetdash{}{0pt}%
\pgfpathmoveto{\pgfqpoint{1.787768in}{0.521603in}}%
\pgfpathlineto{\pgfqpoint{1.818309in}{0.521603in}}%
\pgfpathlineto{\pgfqpoint{1.818309in}{0.524153in}}%
\pgfpathlineto{\pgfqpoint{1.787768in}{0.524153in}}%
\pgfpathlineto{\pgfqpoint{1.787768in}{0.521603in}}%
\pgfpathclose%
\pgfusepath{fill}%
\end{pgfscope}%
\begin{pgfscope}%
\pgfpathrectangle{\pgfqpoint{0.664969in}{0.521603in}}{\pgfqpoint{3.201017in}{1.696195in}}%
\pgfusepath{clip}%
\pgfsetbuttcap%
\pgfsetmiterjoin%
\definecolor{currentfill}{rgb}{0.121569,0.466667,0.705882}%
\pgfsetfillcolor{currentfill}%
\pgfsetfillopacity{0.700000}%
\pgfsetlinewidth{0.000000pt}%
\definecolor{currentstroke}{rgb}{0.000000,0.000000,0.000000}%
\pgfsetstrokecolor{currentstroke}%
\pgfsetstrokeopacity{0.700000}%
\pgfsetdash{}{0pt}%
\pgfpathmoveto{\pgfqpoint{1.818309in}{0.521603in}}%
\pgfpathlineto{\pgfqpoint{1.848850in}{0.521603in}}%
\pgfpathlineto{\pgfqpoint{1.848850in}{0.524153in}}%
\pgfpathlineto{\pgfqpoint{1.818309in}{0.524153in}}%
\pgfpathlineto{\pgfqpoint{1.818309in}{0.521603in}}%
\pgfpathclose%
\pgfusepath{fill}%
\end{pgfscope}%
\begin{pgfscope}%
\pgfpathrectangle{\pgfqpoint{0.664969in}{0.521603in}}{\pgfqpoint{3.201017in}{1.696195in}}%
\pgfusepath{clip}%
\pgfsetbuttcap%
\pgfsetmiterjoin%
\definecolor{currentfill}{rgb}{0.121569,0.466667,0.705882}%
\pgfsetfillcolor{currentfill}%
\pgfsetfillopacity{0.700000}%
\pgfsetlinewidth{0.000000pt}%
\definecolor{currentstroke}{rgb}{0.000000,0.000000,0.000000}%
\pgfsetstrokecolor{currentstroke}%
\pgfsetstrokeopacity{0.700000}%
\pgfsetdash{}{0pt}%
\pgfpathmoveto{\pgfqpoint{1.848850in}{0.521603in}}%
\pgfpathlineto{\pgfqpoint{1.879390in}{0.521603in}}%
\pgfpathlineto{\pgfqpoint{1.879390in}{0.522878in}}%
\pgfpathlineto{\pgfqpoint{1.848850in}{0.522878in}}%
\pgfpathlineto{\pgfqpoint{1.848850in}{0.521603in}}%
\pgfpathclose%
\pgfusepath{fill}%
\end{pgfscope}%
\begin{pgfscope}%
\pgfpathrectangle{\pgfqpoint{0.664969in}{0.521603in}}{\pgfqpoint{3.201017in}{1.696195in}}%
\pgfusepath{clip}%
\pgfsetbuttcap%
\pgfsetmiterjoin%
\definecolor{currentfill}{rgb}{0.121569,0.466667,0.705882}%
\pgfsetfillcolor{currentfill}%
\pgfsetfillopacity{0.700000}%
\pgfsetlinewidth{0.000000pt}%
\definecolor{currentstroke}{rgb}{0.000000,0.000000,0.000000}%
\pgfsetstrokecolor{currentstroke}%
\pgfsetstrokeopacity{0.700000}%
\pgfsetdash{}{0pt}%
\pgfpathmoveto{\pgfqpoint{1.879390in}{0.521603in}}%
\pgfpathlineto{\pgfqpoint{1.909931in}{0.521603in}}%
\pgfpathlineto{\pgfqpoint{1.909931in}{0.521603in}}%
\pgfpathlineto{\pgfqpoint{1.879390in}{0.521603in}}%
\pgfpathlineto{\pgfqpoint{1.879390in}{0.521603in}}%
\pgfpathclose%
\pgfusepath{fill}%
\end{pgfscope}%
\begin{pgfscope}%
\pgfpathrectangle{\pgfqpoint{0.664969in}{0.521603in}}{\pgfqpoint{3.201017in}{1.696195in}}%
\pgfusepath{clip}%
\pgfsetbuttcap%
\pgfsetmiterjoin%
\definecolor{currentfill}{rgb}{0.121569,0.466667,0.705882}%
\pgfsetfillcolor{currentfill}%
\pgfsetfillopacity{0.700000}%
\pgfsetlinewidth{0.000000pt}%
\definecolor{currentstroke}{rgb}{0.000000,0.000000,0.000000}%
\pgfsetstrokecolor{currentstroke}%
\pgfsetstrokeopacity{0.700000}%
\pgfsetdash{}{0pt}%
\pgfpathmoveto{\pgfqpoint{1.909931in}{0.521603in}}%
\pgfpathlineto{\pgfqpoint{1.940471in}{0.521603in}}%
\pgfpathlineto{\pgfqpoint{1.940471in}{0.526703in}}%
\pgfpathlineto{\pgfqpoint{1.909931in}{0.526703in}}%
\pgfpathlineto{\pgfqpoint{1.909931in}{0.521603in}}%
\pgfpathclose%
\pgfusepath{fill}%
\end{pgfscope}%
\begin{pgfscope}%
\pgfpathrectangle{\pgfqpoint{0.664969in}{0.521603in}}{\pgfqpoint{3.201017in}{1.696195in}}%
\pgfusepath{clip}%
\pgfsetbuttcap%
\pgfsetmiterjoin%
\definecolor{currentfill}{rgb}{0.121569,0.466667,0.705882}%
\pgfsetfillcolor{currentfill}%
\pgfsetfillopacity{0.700000}%
\pgfsetlinewidth{0.000000pt}%
\definecolor{currentstroke}{rgb}{0.000000,0.000000,0.000000}%
\pgfsetstrokecolor{currentstroke}%
\pgfsetstrokeopacity{0.700000}%
\pgfsetdash{}{0pt}%
\pgfpathmoveto{\pgfqpoint{1.940471in}{0.521603in}}%
\pgfpathlineto{\pgfqpoint{1.971012in}{0.521603in}}%
\pgfpathlineto{\pgfqpoint{1.971012in}{0.521603in}}%
\pgfpathlineto{\pgfqpoint{1.940471in}{0.521603in}}%
\pgfpathlineto{\pgfqpoint{1.940471in}{0.521603in}}%
\pgfpathclose%
\pgfusepath{fill}%
\end{pgfscope}%
\begin{pgfscope}%
\pgfpathrectangle{\pgfqpoint{0.664969in}{0.521603in}}{\pgfqpoint{3.201017in}{1.696195in}}%
\pgfusepath{clip}%
\pgfsetbuttcap%
\pgfsetmiterjoin%
\definecolor{currentfill}{rgb}{0.121569,0.466667,0.705882}%
\pgfsetfillcolor{currentfill}%
\pgfsetfillopacity{0.700000}%
\pgfsetlinewidth{0.000000pt}%
\definecolor{currentstroke}{rgb}{0.000000,0.000000,0.000000}%
\pgfsetstrokecolor{currentstroke}%
\pgfsetstrokeopacity{0.700000}%
\pgfsetdash{}{0pt}%
\pgfpathmoveto{\pgfqpoint{1.971012in}{0.521603in}}%
\pgfpathlineto{\pgfqpoint{2.001552in}{0.521603in}}%
\pgfpathlineto{\pgfqpoint{2.001552in}{0.521603in}}%
\pgfpathlineto{\pgfqpoint{1.971012in}{0.521603in}}%
\pgfpathlineto{\pgfqpoint{1.971012in}{0.521603in}}%
\pgfpathclose%
\pgfusepath{fill}%
\end{pgfscope}%
\begin{pgfscope}%
\pgfpathrectangle{\pgfqpoint{0.664969in}{0.521603in}}{\pgfqpoint{3.201017in}{1.696195in}}%
\pgfusepath{clip}%
\pgfsetbuttcap%
\pgfsetmiterjoin%
\definecolor{currentfill}{rgb}{0.121569,0.466667,0.705882}%
\pgfsetfillcolor{currentfill}%
\pgfsetfillopacity{0.700000}%
\pgfsetlinewidth{0.000000pt}%
\definecolor{currentstroke}{rgb}{0.000000,0.000000,0.000000}%
\pgfsetstrokecolor{currentstroke}%
\pgfsetstrokeopacity{0.700000}%
\pgfsetdash{}{0pt}%
\pgfpathmoveto{\pgfqpoint{2.001552in}{0.521603in}}%
\pgfpathlineto{\pgfqpoint{2.032093in}{0.521603in}}%
\pgfpathlineto{\pgfqpoint{2.032093in}{0.525428in}}%
\pgfpathlineto{\pgfqpoint{2.001552in}{0.525428in}}%
\pgfpathlineto{\pgfqpoint{2.001552in}{0.521603in}}%
\pgfpathclose%
\pgfusepath{fill}%
\end{pgfscope}%
\begin{pgfscope}%
\pgfpathrectangle{\pgfqpoint{0.664969in}{0.521603in}}{\pgfqpoint{3.201017in}{1.696195in}}%
\pgfusepath{clip}%
\pgfsetbuttcap%
\pgfsetmiterjoin%
\definecolor{currentfill}{rgb}{0.121569,0.466667,0.705882}%
\pgfsetfillcolor{currentfill}%
\pgfsetfillopacity{0.700000}%
\pgfsetlinewidth{0.000000pt}%
\definecolor{currentstroke}{rgb}{0.000000,0.000000,0.000000}%
\pgfsetstrokecolor{currentstroke}%
\pgfsetstrokeopacity{0.700000}%
\pgfsetdash{}{0pt}%
\pgfpathmoveto{\pgfqpoint{2.032093in}{0.521603in}}%
\pgfpathlineto{\pgfqpoint{2.062634in}{0.521603in}}%
\pgfpathlineto{\pgfqpoint{2.062634in}{0.524153in}}%
\pgfpathlineto{\pgfqpoint{2.032093in}{0.524153in}}%
\pgfpathlineto{\pgfqpoint{2.032093in}{0.521603in}}%
\pgfpathclose%
\pgfusepath{fill}%
\end{pgfscope}%
\begin{pgfscope}%
\pgfpathrectangle{\pgfqpoint{0.664969in}{0.521603in}}{\pgfqpoint{3.201017in}{1.696195in}}%
\pgfusepath{clip}%
\pgfsetbuttcap%
\pgfsetmiterjoin%
\definecolor{currentfill}{rgb}{0.121569,0.466667,0.705882}%
\pgfsetfillcolor{currentfill}%
\pgfsetfillopacity{0.700000}%
\pgfsetlinewidth{0.000000pt}%
\definecolor{currentstroke}{rgb}{0.000000,0.000000,0.000000}%
\pgfsetstrokecolor{currentstroke}%
\pgfsetstrokeopacity{0.700000}%
\pgfsetdash{}{0pt}%
\pgfpathmoveto{\pgfqpoint{2.062634in}{0.521603in}}%
\pgfpathlineto{\pgfqpoint{2.093174in}{0.521603in}}%
\pgfpathlineto{\pgfqpoint{2.093174in}{0.524153in}}%
\pgfpathlineto{\pgfqpoint{2.062634in}{0.524153in}}%
\pgfpathlineto{\pgfqpoint{2.062634in}{0.521603in}}%
\pgfpathclose%
\pgfusepath{fill}%
\end{pgfscope}%
\begin{pgfscope}%
\pgfpathrectangle{\pgfqpoint{0.664969in}{0.521603in}}{\pgfqpoint{3.201017in}{1.696195in}}%
\pgfusepath{clip}%
\pgfsetbuttcap%
\pgfsetmiterjoin%
\definecolor{currentfill}{rgb}{0.121569,0.466667,0.705882}%
\pgfsetfillcolor{currentfill}%
\pgfsetfillopacity{0.700000}%
\pgfsetlinewidth{0.000000pt}%
\definecolor{currentstroke}{rgb}{0.000000,0.000000,0.000000}%
\pgfsetstrokecolor{currentstroke}%
\pgfsetstrokeopacity{0.700000}%
\pgfsetdash{}{0pt}%
\pgfpathmoveto{\pgfqpoint{2.093174in}{0.521603in}}%
\pgfpathlineto{\pgfqpoint{2.123715in}{0.521603in}}%
\pgfpathlineto{\pgfqpoint{2.123715in}{0.521603in}}%
\pgfpathlineto{\pgfqpoint{2.093174in}{0.521603in}}%
\pgfpathlineto{\pgfqpoint{2.093174in}{0.521603in}}%
\pgfpathclose%
\pgfusepath{fill}%
\end{pgfscope}%
\begin{pgfscope}%
\pgfpathrectangle{\pgfqpoint{0.664969in}{0.521603in}}{\pgfqpoint{3.201017in}{1.696195in}}%
\pgfusepath{clip}%
\pgfsetbuttcap%
\pgfsetmiterjoin%
\definecolor{currentfill}{rgb}{0.121569,0.466667,0.705882}%
\pgfsetfillcolor{currentfill}%
\pgfsetfillopacity{0.700000}%
\pgfsetlinewidth{0.000000pt}%
\definecolor{currentstroke}{rgb}{0.000000,0.000000,0.000000}%
\pgfsetstrokecolor{currentstroke}%
\pgfsetstrokeopacity{0.700000}%
\pgfsetdash{}{0pt}%
\pgfpathmoveto{\pgfqpoint{2.123715in}{0.521603in}}%
\pgfpathlineto{\pgfqpoint{2.154255in}{0.521603in}}%
\pgfpathlineto{\pgfqpoint{2.154255in}{0.522878in}}%
\pgfpathlineto{\pgfqpoint{2.123715in}{0.522878in}}%
\pgfpathlineto{\pgfqpoint{2.123715in}{0.521603in}}%
\pgfpathclose%
\pgfusepath{fill}%
\end{pgfscope}%
\begin{pgfscope}%
\pgfpathrectangle{\pgfqpoint{0.664969in}{0.521603in}}{\pgfqpoint{3.201017in}{1.696195in}}%
\pgfusepath{clip}%
\pgfsetbuttcap%
\pgfsetmiterjoin%
\definecolor{currentfill}{rgb}{0.121569,0.466667,0.705882}%
\pgfsetfillcolor{currentfill}%
\pgfsetfillopacity{0.700000}%
\pgfsetlinewidth{0.000000pt}%
\definecolor{currentstroke}{rgb}{0.000000,0.000000,0.000000}%
\pgfsetstrokecolor{currentstroke}%
\pgfsetstrokeopacity{0.700000}%
\pgfsetdash{}{0pt}%
\pgfpathmoveto{\pgfqpoint{2.154255in}{0.521603in}}%
\pgfpathlineto{\pgfqpoint{2.184796in}{0.521603in}}%
\pgfpathlineto{\pgfqpoint{2.184796in}{0.521603in}}%
\pgfpathlineto{\pgfqpoint{2.154255in}{0.521603in}}%
\pgfpathlineto{\pgfqpoint{2.154255in}{0.521603in}}%
\pgfpathclose%
\pgfusepath{fill}%
\end{pgfscope}%
\begin{pgfscope}%
\pgfpathrectangle{\pgfqpoint{0.664969in}{0.521603in}}{\pgfqpoint{3.201017in}{1.696195in}}%
\pgfusepath{clip}%
\pgfsetbuttcap%
\pgfsetmiterjoin%
\definecolor{currentfill}{rgb}{0.121569,0.466667,0.705882}%
\pgfsetfillcolor{currentfill}%
\pgfsetfillopacity{0.700000}%
\pgfsetlinewidth{0.000000pt}%
\definecolor{currentstroke}{rgb}{0.000000,0.000000,0.000000}%
\pgfsetstrokecolor{currentstroke}%
\pgfsetstrokeopacity{0.700000}%
\pgfsetdash{}{0pt}%
\pgfpathmoveto{\pgfqpoint{2.184796in}{0.521603in}}%
\pgfpathlineto{\pgfqpoint{2.215336in}{0.521603in}}%
\pgfpathlineto{\pgfqpoint{2.215336in}{0.524153in}}%
\pgfpathlineto{\pgfqpoint{2.184796in}{0.524153in}}%
\pgfpathlineto{\pgfqpoint{2.184796in}{0.521603in}}%
\pgfpathclose%
\pgfusepath{fill}%
\end{pgfscope}%
\begin{pgfscope}%
\pgfpathrectangle{\pgfqpoint{0.664969in}{0.521603in}}{\pgfqpoint{3.201017in}{1.696195in}}%
\pgfusepath{clip}%
\pgfsetbuttcap%
\pgfsetmiterjoin%
\definecolor{currentfill}{rgb}{0.121569,0.466667,0.705882}%
\pgfsetfillcolor{currentfill}%
\pgfsetfillopacity{0.700000}%
\pgfsetlinewidth{0.000000pt}%
\definecolor{currentstroke}{rgb}{0.000000,0.000000,0.000000}%
\pgfsetstrokecolor{currentstroke}%
\pgfsetstrokeopacity{0.700000}%
\pgfsetdash{}{0pt}%
\pgfpathmoveto{\pgfqpoint{2.215336in}{0.521603in}}%
\pgfpathlineto{\pgfqpoint{2.245877in}{0.521603in}}%
\pgfpathlineto{\pgfqpoint{2.245877in}{0.522878in}}%
\pgfpathlineto{\pgfqpoint{2.215336in}{0.522878in}}%
\pgfpathlineto{\pgfqpoint{2.215336in}{0.521603in}}%
\pgfpathclose%
\pgfusepath{fill}%
\end{pgfscope}%
\begin{pgfscope}%
\pgfpathrectangle{\pgfqpoint{0.664969in}{0.521603in}}{\pgfqpoint{3.201017in}{1.696195in}}%
\pgfusepath{clip}%
\pgfsetbuttcap%
\pgfsetmiterjoin%
\definecolor{currentfill}{rgb}{0.121569,0.466667,0.705882}%
\pgfsetfillcolor{currentfill}%
\pgfsetfillopacity{0.700000}%
\pgfsetlinewidth{0.000000pt}%
\definecolor{currentstroke}{rgb}{0.000000,0.000000,0.000000}%
\pgfsetstrokecolor{currentstroke}%
\pgfsetstrokeopacity{0.700000}%
\pgfsetdash{}{0pt}%
\pgfpathmoveto{\pgfqpoint{2.245877in}{0.521603in}}%
\pgfpathlineto{\pgfqpoint{2.276418in}{0.521603in}}%
\pgfpathlineto{\pgfqpoint{2.276418in}{0.521603in}}%
\pgfpathlineto{\pgfqpoint{2.245877in}{0.521603in}}%
\pgfpathlineto{\pgfqpoint{2.245877in}{0.521603in}}%
\pgfpathclose%
\pgfusepath{fill}%
\end{pgfscope}%
\begin{pgfscope}%
\pgfpathrectangle{\pgfqpoint{0.664969in}{0.521603in}}{\pgfqpoint{3.201017in}{1.696195in}}%
\pgfusepath{clip}%
\pgfsetbuttcap%
\pgfsetmiterjoin%
\definecolor{currentfill}{rgb}{0.121569,0.466667,0.705882}%
\pgfsetfillcolor{currentfill}%
\pgfsetfillopacity{0.700000}%
\pgfsetlinewidth{0.000000pt}%
\definecolor{currentstroke}{rgb}{0.000000,0.000000,0.000000}%
\pgfsetstrokecolor{currentstroke}%
\pgfsetstrokeopacity{0.700000}%
\pgfsetdash{}{0pt}%
\pgfpathmoveto{\pgfqpoint{2.276418in}{0.521603in}}%
\pgfpathlineto{\pgfqpoint{2.306958in}{0.521603in}}%
\pgfpathlineto{\pgfqpoint{2.306958in}{0.522878in}}%
\pgfpathlineto{\pgfqpoint{2.276418in}{0.522878in}}%
\pgfpathlineto{\pgfqpoint{2.276418in}{0.521603in}}%
\pgfpathclose%
\pgfusepath{fill}%
\end{pgfscope}%
\begin{pgfscope}%
\pgfpathrectangle{\pgfqpoint{0.664969in}{0.521603in}}{\pgfqpoint{3.201017in}{1.696195in}}%
\pgfusepath{clip}%
\pgfsetbuttcap%
\pgfsetmiterjoin%
\definecolor{currentfill}{rgb}{0.121569,0.466667,0.705882}%
\pgfsetfillcolor{currentfill}%
\pgfsetfillopacity{0.700000}%
\pgfsetlinewidth{0.000000pt}%
\definecolor{currentstroke}{rgb}{0.000000,0.000000,0.000000}%
\pgfsetstrokecolor{currentstroke}%
\pgfsetstrokeopacity{0.700000}%
\pgfsetdash{}{0pt}%
\pgfpathmoveto{\pgfqpoint{2.306958in}{0.521603in}}%
\pgfpathlineto{\pgfqpoint{2.337499in}{0.521603in}}%
\pgfpathlineto{\pgfqpoint{2.337499in}{0.521603in}}%
\pgfpathlineto{\pgfqpoint{2.306958in}{0.521603in}}%
\pgfpathlineto{\pgfqpoint{2.306958in}{0.521603in}}%
\pgfpathclose%
\pgfusepath{fill}%
\end{pgfscope}%
\begin{pgfscope}%
\pgfpathrectangle{\pgfqpoint{0.664969in}{0.521603in}}{\pgfqpoint{3.201017in}{1.696195in}}%
\pgfusepath{clip}%
\pgfsetbuttcap%
\pgfsetmiterjoin%
\definecolor{currentfill}{rgb}{0.121569,0.466667,0.705882}%
\pgfsetfillcolor{currentfill}%
\pgfsetfillopacity{0.700000}%
\pgfsetlinewidth{0.000000pt}%
\definecolor{currentstroke}{rgb}{0.000000,0.000000,0.000000}%
\pgfsetstrokecolor{currentstroke}%
\pgfsetstrokeopacity{0.700000}%
\pgfsetdash{}{0pt}%
\pgfpathmoveto{\pgfqpoint{2.337499in}{0.521603in}}%
\pgfpathlineto{\pgfqpoint{2.368039in}{0.521603in}}%
\pgfpathlineto{\pgfqpoint{2.368039in}{0.521603in}}%
\pgfpathlineto{\pgfqpoint{2.337499in}{0.521603in}}%
\pgfpathlineto{\pgfqpoint{2.337499in}{0.521603in}}%
\pgfpathclose%
\pgfusepath{fill}%
\end{pgfscope}%
\begin{pgfscope}%
\pgfpathrectangle{\pgfqpoint{0.664969in}{0.521603in}}{\pgfqpoint{3.201017in}{1.696195in}}%
\pgfusepath{clip}%
\pgfsetbuttcap%
\pgfsetmiterjoin%
\definecolor{currentfill}{rgb}{0.121569,0.466667,0.705882}%
\pgfsetfillcolor{currentfill}%
\pgfsetfillopacity{0.700000}%
\pgfsetlinewidth{0.000000pt}%
\definecolor{currentstroke}{rgb}{0.000000,0.000000,0.000000}%
\pgfsetstrokecolor{currentstroke}%
\pgfsetstrokeopacity{0.700000}%
\pgfsetdash{}{0pt}%
\pgfpathmoveto{\pgfqpoint{2.368039in}{0.521603in}}%
\pgfpathlineto{\pgfqpoint{2.398580in}{0.521603in}}%
\pgfpathlineto{\pgfqpoint{2.398580in}{0.524153in}}%
\pgfpathlineto{\pgfqpoint{2.368039in}{0.524153in}}%
\pgfpathlineto{\pgfqpoint{2.368039in}{0.521603in}}%
\pgfpathclose%
\pgfusepath{fill}%
\end{pgfscope}%
\begin{pgfscope}%
\pgfpathrectangle{\pgfqpoint{0.664969in}{0.521603in}}{\pgfqpoint{3.201017in}{1.696195in}}%
\pgfusepath{clip}%
\pgfsetbuttcap%
\pgfsetmiterjoin%
\definecolor{currentfill}{rgb}{0.121569,0.466667,0.705882}%
\pgfsetfillcolor{currentfill}%
\pgfsetfillopacity{0.700000}%
\pgfsetlinewidth{0.000000pt}%
\definecolor{currentstroke}{rgb}{0.000000,0.000000,0.000000}%
\pgfsetstrokecolor{currentstroke}%
\pgfsetstrokeopacity{0.700000}%
\pgfsetdash{}{0pt}%
\pgfpathmoveto{\pgfqpoint{2.398580in}{0.521603in}}%
\pgfpathlineto{\pgfqpoint{2.429120in}{0.521603in}}%
\pgfpathlineto{\pgfqpoint{2.429120in}{0.522878in}}%
\pgfpathlineto{\pgfqpoint{2.398580in}{0.522878in}}%
\pgfpathlineto{\pgfqpoint{2.398580in}{0.521603in}}%
\pgfpathclose%
\pgfusepath{fill}%
\end{pgfscope}%
\begin{pgfscope}%
\pgfpathrectangle{\pgfqpoint{0.664969in}{0.521603in}}{\pgfqpoint{3.201017in}{1.696195in}}%
\pgfusepath{clip}%
\pgfsetbuttcap%
\pgfsetmiterjoin%
\definecolor{currentfill}{rgb}{0.121569,0.466667,0.705882}%
\pgfsetfillcolor{currentfill}%
\pgfsetfillopacity{0.700000}%
\pgfsetlinewidth{0.000000pt}%
\definecolor{currentstroke}{rgb}{0.000000,0.000000,0.000000}%
\pgfsetstrokecolor{currentstroke}%
\pgfsetstrokeopacity{0.700000}%
\pgfsetdash{}{0pt}%
\pgfpathmoveto{\pgfqpoint{2.429120in}{0.521603in}}%
\pgfpathlineto{\pgfqpoint{2.459661in}{0.521603in}}%
\pgfpathlineto{\pgfqpoint{2.459661in}{0.521603in}}%
\pgfpathlineto{\pgfqpoint{2.429120in}{0.521603in}}%
\pgfpathlineto{\pgfqpoint{2.429120in}{0.521603in}}%
\pgfpathclose%
\pgfusepath{fill}%
\end{pgfscope}%
\begin{pgfscope}%
\pgfpathrectangle{\pgfqpoint{0.664969in}{0.521603in}}{\pgfqpoint{3.201017in}{1.696195in}}%
\pgfusepath{clip}%
\pgfsetbuttcap%
\pgfsetmiterjoin%
\definecolor{currentfill}{rgb}{0.121569,0.466667,0.705882}%
\pgfsetfillcolor{currentfill}%
\pgfsetfillopacity{0.700000}%
\pgfsetlinewidth{0.000000pt}%
\definecolor{currentstroke}{rgb}{0.000000,0.000000,0.000000}%
\pgfsetstrokecolor{currentstroke}%
\pgfsetstrokeopacity{0.700000}%
\pgfsetdash{}{0pt}%
\pgfpathmoveto{\pgfqpoint{2.459661in}{0.521603in}}%
\pgfpathlineto{\pgfqpoint{2.490202in}{0.521603in}}%
\pgfpathlineto{\pgfqpoint{2.490202in}{0.521603in}}%
\pgfpathlineto{\pgfqpoint{2.459661in}{0.521603in}}%
\pgfpathlineto{\pgfqpoint{2.459661in}{0.521603in}}%
\pgfpathclose%
\pgfusepath{fill}%
\end{pgfscope}%
\begin{pgfscope}%
\pgfpathrectangle{\pgfqpoint{0.664969in}{0.521603in}}{\pgfqpoint{3.201017in}{1.696195in}}%
\pgfusepath{clip}%
\pgfsetbuttcap%
\pgfsetmiterjoin%
\definecolor{currentfill}{rgb}{0.121569,0.466667,0.705882}%
\pgfsetfillcolor{currentfill}%
\pgfsetfillopacity{0.700000}%
\pgfsetlinewidth{0.000000pt}%
\definecolor{currentstroke}{rgb}{0.000000,0.000000,0.000000}%
\pgfsetstrokecolor{currentstroke}%
\pgfsetstrokeopacity{0.700000}%
\pgfsetdash{}{0pt}%
\pgfpathmoveto{\pgfqpoint{2.490202in}{0.521603in}}%
\pgfpathlineto{\pgfqpoint{2.520742in}{0.521603in}}%
\pgfpathlineto{\pgfqpoint{2.520742in}{0.521603in}}%
\pgfpathlineto{\pgfqpoint{2.490202in}{0.521603in}}%
\pgfpathlineto{\pgfqpoint{2.490202in}{0.521603in}}%
\pgfpathclose%
\pgfusepath{fill}%
\end{pgfscope}%
\begin{pgfscope}%
\pgfpathrectangle{\pgfqpoint{0.664969in}{0.521603in}}{\pgfqpoint{3.201017in}{1.696195in}}%
\pgfusepath{clip}%
\pgfsetbuttcap%
\pgfsetmiterjoin%
\definecolor{currentfill}{rgb}{0.121569,0.466667,0.705882}%
\pgfsetfillcolor{currentfill}%
\pgfsetfillopacity{0.700000}%
\pgfsetlinewidth{0.000000pt}%
\definecolor{currentstroke}{rgb}{0.000000,0.000000,0.000000}%
\pgfsetstrokecolor{currentstroke}%
\pgfsetstrokeopacity{0.700000}%
\pgfsetdash{}{0pt}%
\pgfpathmoveto{\pgfqpoint{2.520742in}{0.521603in}}%
\pgfpathlineto{\pgfqpoint{2.551283in}{0.521603in}}%
\pgfpathlineto{\pgfqpoint{2.551283in}{0.522878in}}%
\pgfpathlineto{\pgfqpoint{2.520742in}{0.522878in}}%
\pgfpathlineto{\pgfqpoint{2.520742in}{0.521603in}}%
\pgfpathclose%
\pgfusepath{fill}%
\end{pgfscope}%
\begin{pgfscope}%
\pgfpathrectangle{\pgfqpoint{0.664969in}{0.521603in}}{\pgfqpoint{3.201017in}{1.696195in}}%
\pgfusepath{clip}%
\pgfsetbuttcap%
\pgfsetmiterjoin%
\definecolor{currentfill}{rgb}{1.000000,0.498039,0.054902}%
\pgfsetfillcolor{currentfill}%
\pgfsetfillopacity{0.700000}%
\pgfsetlinewidth{0.000000pt}%
\definecolor{currentstroke}{rgb}{0.000000,0.000000,0.000000}%
\pgfsetstrokecolor{currentstroke}%
\pgfsetstrokeopacity{0.700000}%
\pgfsetdash{}{0pt}%
\pgfpathmoveto{\pgfqpoint{0.810470in}{0.521603in}}%
\pgfpathlineto{\pgfqpoint{0.861523in}{0.521603in}}%
\pgfpathlineto{\pgfqpoint{0.861523in}{0.525428in}}%
\pgfpathlineto{\pgfqpoint{0.810470in}{0.525428in}}%
\pgfpathlineto{\pgfqpoint{0.810470in}{0.521603in}}%
\pgfpathclose%
\pgfusepath{fill}%
\end{pgfscope}%
\begin{pgfscope}%
\pgfpathrectangle{\pgfqpoint{0.664969in}{0.521603in}}{\pgfqpoint{3.201017in}{1.696195in}}%
\pgfusepath{clip}%
\pgfsetbuttcap%
\pgfsetmiterjoin%
\definecolor{currentfill}{rgb}{1.000000,0.498039,0.054902}%
\pgfsetfillcolor{currentfill}%
\pgfsetfillopacity{0.700000}%
\pgfsetlinewidth{0.000000pt}%
\definecolor{currentstroke}{rgb}{0.000000,0.000000,0.000000}%
\pgfsetstrokecolor{currentstroke}%
\pgfsetstrokeopacity{0.700000}%
\pgfsetdash{}{0pt}%
\pgfpathmoveto{\pgfqpoint{0.861523in}{0.521603in}}%
\pgfpathlineto{\pgfqpoint{0.912576in}{0.521603in}}%
\pgfpathlineto{\pgfqpoint{0.912576in}{0.641453in}}%
\pgfpathlineto{\pgfqpoint{0.861523in}{0.641453in}}%
\pgfpathlineto{\pgfqpoint{0.861523in}{0.521603in}}%
\pgfpathclose%
\pgfusepath{fill}%
\end{pgfscope}%
\begin{pgfscope}%
\pgfpathrectangle{\pgfqpoint{0.664969in}{0.521603in}}{\pgfqpoint{3.201017in}{1.696195in}}%
\pgfusepath{clip}%
\pgfsetbuttcap%
\pgfsetmiterjoin%
\definecolor{currentfill}{rgb}{1.000000,0.498039,0.054902}%
\pgfsetfillcolor{currentfill}%
\pgfsetfillopacity{0.700000}%
\pgfsetlinewidth{0.000000pt}%
\definecolor{currentstroke}{rgb}{0.000000,0.000000,0.000000}%
\pgfsetstrokecolor{currentstroke}%
\pgfsetstrokeopacity{0.700000}%
\pgfsetdash{}{0pt}%
\pgfpathmoveto{\pgfqpoint{0.912576in}{0.521603in}}%
\pgfpathlineto{\pgfqpoint{0.963629in}{0.521603in}}%
\pgfpathlineto{\pgfqpoint{0.963629in}{0.743453in}}%
\pgfpathlineto{\pgfqpoint{0.912576in}{0.743453in}}%
\pgfpathlineto{\pgfqpoint{0.912576in}{0.521603in}}%
\pgfpathclose%
\pgfusepath{fill}%
\end{pgfscope}%
\begin{pgfscope}%
\pgfpathrectangle{\pgfqpoint{0.664969in}{0.521603in}}{\pgfqpoint{3.201017in}{1.696195in}}%
\pgfusepath{clip}%
\pgfsetbuttcap%
\pgfsetmiterjoin%
\definecolor{currentfill}{rgb}{1.000000,0.498039,0.054902}%
\pgfsetfillcolor{currentfill}%
\pgfsetfillopacity{0.700000}%
\pgfsetlinewidth{0.000000pt}%
\definecolor{currentstroke}{rgb}{0.000000,0.000000,0.000000}%
\pgfsetstrokecolor{currentstroke}%
\pgfsetstrokeopacity{0.700000}%
\pgfsetdash{}{0pt}%
\pgfpathmoveto{\pgfqpoint{0.963629in}{0.521603in}}%
\pgfpathlineto{\pgfqpoint{1.014682in}{0.521603in}}%
\pgfpathlineto{\pgfqpoint{1.014682in}{0.752378in}}%
\pgfpathlineto{\pgfqpoint{0.963629in}{0.752378in}}%
\pgfpathlineto{\pgfqpoint{0.963629in}{0.521603in}}%
\pgfpathclose%
\pgfusepath{fill}%
\end{pgfscope}%
\begin{pgfscope}%
\pgfpathrectangle{\pgfqpoint{0.664969in}{0.521603in}}{\pgfqpoint{3.201017in}{1.696195in}}%
\pgfusepath{clip}%
\pgfsetbuttcap%
\pgfsetmiterjoin%
\definecolor{currentfill}{rgb}{1.000000,0.498039,0.054902}%
\pgfsetfillcolor{currentfill}%
\pgfsetfillopacity{0.700000}%
\pgfsetlinewidth{0.000000pt}%
\definecolor{currentstroke}{rgb}{0.000000,0.000000,0.000000}%
\pgfsetstrokecolor{currentstroke}%
\pgfsetstrokeopacity{0.700000}%
\pgfsetdash{}{0pt}%
\pgfpathmoveto{\pgfqpoint{1.014682in}{0.521603in}}%
\pgfpathlineto{\pgfqpoint{1.065734in}{0.521603in}}%
\pgfpathlineto{\pgfqpoint{1.065734in}{0.731978in}}%
\pgfpathlineto{\pgfqpoint{1.014682in}{0.731978in}}%
\pgfpathlineto{\pgfqpoint{1.014682in}{0.521603in}}%
\pgfpathclose%
\pgfusepath{fill}%
\end{pgfscope}%
\begin{pgfscope}%
\pgfpathrectangle{\pgfqpoint{0.664969in}{0.521603in}}{\pgfqpoint{3.201017in}{1.696195in}}%
\pgfusepath{clip}%
\pgfsetbuttcap%
\pgfsetmiterjoin%
\definecolor{currentfill}{rgb}{1.000000,0.498039,0.054902}%
\pgfsetfillcolor{currentfill}%
\pgfsetfillopacity{0.700000}%
\pgfsetlinewidth{0.000000pt}%
\definecolor{currentstroke}{rgb}{0.000000,0.000000,0.000000}%
\pgfsetstrokecolor{currentstroke}%
\pgfsetstrokeopacity{0.700000}%
\pgfsetdash{}{0pt}%
\pgfpathmoveto{\pgfqpoint{1.065734in}{0.521603in}}%
\pgfpathlineto{\pgfqpoint{1.116787in}{0.521603in}}%
\pgfpathlineto{\pgfqpoint{1.116787in}{0.726878in}}%
\pgfpathlineto{\pgfqpoint{1.065734in}{0.726878in}}%
\pgfpathlineto{\pgfqpoint{1.065734in}{0.521603in}}%
\pgfpathclose%
\pgfusepath{fill}%
\end{pgfscope}%
\begin{pgfscope}%
\pgfpathrectangle{\pgfqpoint{0.664969in}{0.521603in}}{\pgfqpoint{3.201017in}{1.696195in}}%
\pgfusepath{clip}%
\pgfsetbuttcap%
\pgfsetmiterjoin%
\definecolor{currentfill}{rgb}{1.000000,0.498039,0.054902}%
\pgfsetfillcolor{currentfill}%
\pgfsetfillopacity{0.700000}%
\pgfsetlinewidth{0.000000pt}%
\definecolor{currentstroke}{rgb}{0.000000,0.000000,0.000000}%
\pgfsetstrokecolor{currentstroke}%
\pgfsetstrokeopacity{0.700000}%
\pgfsetdash{}{0pt}%
\pgfpathmoveto{\pgfqpoint{1.116787in}{0.521603in}}%
\pgfpathlineto{\pgfqpoint{1.167840in}{0.521603in}}%
\pgfpathlineto{\pgfqpoint{1.167840in}{0.726878in}}%
\pgfpathlineto{\pgfqpoint{1.116787in}{0.726878in}}%
\pgfpathlineto{\pgfqpoint{1.116787in}{0.521603in}}%
\pgfpathclose%
\pgfusepath{fill}%
\end{pgfscope}%
\begin{pgfscope}%
\pgfpathrectangle{\pgfqpoint{0.664969in}{0.521603in}}{\pgfqpoint{3.201017in}{1.696195in}}%
\pgfusepath{clip}%
\pgfsetbuttcap%
\pgfsetmiterjoin%
\definecolor{currentfill}{rgb}{1.000000,0.498039,0.054902}%
\pgfsetfillcolor{currentfill}%
\pgfsetfillopacity{0.700000}%
\pgfsetlinewidth{0.000000pt}%
\definecolor{currentstroke}{rgb}{0.000000,0.000000,0.000000}%
\pgfsetstrokecolor{currentstroke}%
\pgfsetstrokeopacity{0.700000}%
\pgfsetdash{}{0pt}%
\pgfpathmoveto{\pgfqpoint{1.167840in}{0.521603in}}%
\pgfpathlineto{\pgfqpoint{1.218893in}{0.521603in}}%
\pgfpathlineto{\pgfqpoint{1.218893in}{0.695003in}}%
\pgfpathlineto{\pgfqpoint{1.167840in}{0.695003in}}%
\pgfpathlineto{\pgfqpoint{1.167840in}{0.521603in}}%
\pgfpathclose%
\pgfusepath{fill}%
\end{pgfscope}%
\begin{pgfscope}%
\pgfpathrectangle{\pgfqpoint{0.664969in}{0.521603in}}{\pgfqpoint{3.201017in}{1.696195in}}%
\pgfusepath{clip}%
\pgfsetbuttcap%
\pgfsetmiterjoin%
\definecolor{currentfill}{rgb}{1.000000,0.498039,0.054902}%
\pgfsetfillcolor{currentfill}%
\pgfsetfillopacity{0.700000}%
\pgfsetlinewidth{0.000000pt}%
\definecolor{currentstroke}{rgb}{0.000000,0.000000,0.000000}%
\pgfsetstrokecolor{currentstroke}%
\pgfsetstrokeopacity{0.700000}%
\pgfsetdash{}{0pt}%
\pgfpathmoveto{\pgfqpoint{1.218893in}{0.521603in}}%
\pgfpathlineto{\pgfqpoint{1.269946in}{0.521603in}}%
\pgfpathlineto{\pgfqpoint{1.269946in}{0.744728in}}%
\pgfpathlineto{\pgfqpoint{1.218893in}{0.744728in}}%
\pgfpathlineto{\pgfqpoint{1.218893in}{0.521603in}}%
\pgfpathclose%
\pgfusepath{fill}%
\end{pgfscope}%
\begin{pgfscope}%
\pgfpathrectangle{\pgfqpoint{0.664969in}{0.521603in}}{\pgfqpoint{3.201017in}{1.696195in}}%
\pgfusepath{clip}%
\pgfsetbuttcap%
\pgfsetmiterjoin%
\definecolor{currentfill}{rgb}{1.000000,0.498039,0.054902}%
\pgfsetfillcolor{currentfill}%
\pgfsetfillopacity{0.700000}%
\pgfsetlinewidth{0.000000pt}%
\definecolor{currentstroke}{rgb}{0.000000,0.000000,0.000000}%
\pgfsetstrokecolor{currentstroke}%
\pgfsetstrokeopacity{0.700000}%
\pgfsetdash{}{0pt}%
\pgfpathmoveto{\pgfqpoint{1.269946in}{0.521603in}}%
\pgfpathlineto{\pgfqpoint{1.320999in}{0.521603in}}%
\pgfpathlineto{\pgfqpoint{1.320999in}{0.714128in}}%
\pgfpathlineto{\pgfqpoint{1.269946in}{0.714128in}}%
\pgfpathlineto{\pgfqpoint{1.269946in}{0.521603in}}%
\pgfpathclose%
\pgfusepath{fill}%
\end{pgfscope}%
\begin{pgfscope}%
\pgfpathrectangle{\pgfqpoint{0.664969in}{0.521603in}}{\pgfqpoint{3.201017in}{1.696195in}}%
\pgfusepath{clip}%
\pgfsetbuttcap%
\pgfsetmiterjoin%
\definecolor{currentfill}{rgb}{1.000000,0.498039,0.054902}%
\pgfsetfillcolor{currentfill}%
\pgfsetfillopacity{0.700000}%
\pgfsetlinewidth{0.000000pt}%
\definecolor{currentstroke}{rgb}{0.000000,0.000000,0.000000}%
\pgfsetstrokecolor{currentstroke}%
\pgfsetstrokeopacity{0.700000}%
\pgfsetdash{}{0pt}%
\pgfpathmoveto{\pgfqpoint{1.320999in}{0.521603in}}%
\pgfpathlineto{\pgfqpoint{1.372052in}{0.521603in}}%
\pgfpathlineto{\pgfqpoint{1.372052in}{0.700103in}}%
\pgfpathlineto{\pgfqpoint{1.320999in}{0.700103in}}%
\pgfpathlineto{\pgfqpoint{1.320999in}{0.521603in}}%
\pgfpathclose%
\pgfusepath{fill}%
\end{pgfscope}%
\begin{pgfscope}%
\pgfpathrectangle{\pgfqpoint{0.664969in}{0.521603in}}{\pgfqpoint{3.201017in}{1.696195in}}%
\pgfusepath{clip}%
\pgfsetbuttcap%
\pgfsetmiterjoin%
\definecolor{currentfill}{rgb}{1.000000,0.498039,0.054902}%
\pgfsetfillcolor{currentfill}%
\pgfsetfillopacity{0.700000}%
\pgfsetlinewidth{0.000000pt}%
\definecolor{currentstroke}{rgb}{0.000000,0.000000,0.000000}%
\pgfsetstrokecolor{currentstroke}%
\pgfsetstrokeopacity{0.700000}%
\pgfsetdash{}{0pt}%
\pgfpathmoveto{\pgfqpoint{1.372052in}{0.521603in}}%
\pgfpathlineto{\pgfqpoint{1.423105in}{0.521603in}}%
\pgfpathlineto{\pgfqpoint{1.423105in}{0.702653in}}%
\pgfpathlineto{\pgfqpoint{1.372052in}{0.702653in}}%
\pgfpathlineto{\pgfqpoint{1.372052in}{0.521603in}}%
\pgfpathclose%
\pgfusepath{fill}%
\end{pgfscope}%
\begin{pgfscope}%
\pgfpathrectangle{\pgfqpoint{0.664969in}{0.521603in}}{\pgfqpoint{3.201017in}{1.696195in}}%
\pgfusepath{clip}%
\pgfsetbuttcap%
\pgfsetmiterjoin%
\definecolor{currentfill}{rgb}{1.000000,0.498039,0.054902}%
\pgfsetfillcolor{currentfill}%
\pgfsetfillopacity{0.700000}%
\pgfsetlinewidth{0.000000pt}%
\definecolor{currentstroke}{rgb}{0.000000,0.000000,0.000000}%
\pgfsetstrokecolor{currentstroke}%
\pgfsetstrokeopacity{0.700000}%
\pgfsetdash{}{0pt}%
\pgfpathmoveto{\pgfqpoint{1.423105in}{0.521603in}}%
\pgfpathlineto{\pgfqpoint{1.474158in}{0.521603in}}%
\pgfpathlineto{\pgfqpoint{1.474158in}{0.686078in}}%
\pgfpathlineto{\pgfqpoint{1.423105in}{0.686078in}}%
\pgfpathlineto{\pgfqpoint{1.423105in}{0.521603in}}%
\pgfpathclose%
\pgfusepath{fill}%
\end{pgfscope}%
\begin{pgfscope}%
\pgfpathrectangle{\pgfqpoint{0.664969in}{0.521603in}}{\pgfqpoint{3.201017in}{1.696195in}}%
\pgfusepath{clip}%
\pgfsetbuttcap%
\pgfsetmiterjoin%
\definecolor{currentfill}{rgb}{1.000000,0.498039,0.054902}%
\pgfsetfillcolor{currentfill}%
\pgfsetfillopacity{0.700000}%
\pgfsetlinewidth{0.000000pt}%
\definecolor{currentstroke}{rgb}{0.000000,0.000000,0.000000}%
\pgfsetstrokecolor{currentstroke}%
\pgfsetstrokeopacity{0.700000}%
\pgfsetdash{}{0pt}%
\pgfpathmoveto{\pgfqpoint{1.474158in}{0.521603in}}%
\pgfpathlineto{\pgfqpoint{1.525211in}{0.521603in}}%
\pgfpathlineto{\pgfqpoint{1.525211in}{0.666953in}}%
\pgfpathlineto{\pgfqpoint{1.474158in}{0.666953in}}%
\pgfpathlineto{\pgfqpoint{1.474158in}{0.521603in}}%
\pgfpathclose%
\pgfusepath{fill}%
\end{pgfscope}%
\begin{pgfscope}%
\pgfpathrectangle{\pgfqpoint{0.664969in}{0.521603in}}{\pgfqpoint{3.201017in}{1.696195in}}%
\pgfusepath{clip}%
\pgfsetbuttcap%
\pgfsetmiterjoin%
\definecolor{currentfill}{rgb}{1.000000,0.498039,0.054902}%
\pgfsetfillcolor{currentfill}%
\pgfsetfillopacity{0.700000}%
\pgfsetlinewidth{0.000000pt}%
\definecolor{currentstroke}{rgb}{0.000000,0.000000,0.000000}%
\pgfsetstrokecolor{currentstroke}%
\pgfsetstrokeopacity{0.700000}%
\pgfsetdash{}{0pt}%
\pgfpathmoveto{\pgfqpoint{1.525211in}{0.521603in}}%
\pgfpathlineto{\pgfqpoint{1.576263in}{0.521603in}}%
\pgfpathlineto{\pgfqpoint{1.576263in}{0.695003in}}%
\pgfpathlineto{\pgfqpoint{1.525211in}{0.695003in}}%
\pgfpathlineto{\pgfqpoint{1.525211in}{0.521603in}}%
\pgfpathclose%
\pgfusepath{fill}%
\end{pgfscope}%
\begin{pgfscope}%
\pgfpathrectangle{\pgfqpoint{0.664969in}{0.521603in}}{\pgfqpoint{3.201017in}{1.696195in}}%
\pgfusepath{clip}%
\pgfsetbuttcap%
\pgfsetmiterjoin%
\definecolor{currentfill}{rgb}{1.000000,0.498039,0.054902}%
\pgfsetfillcolor{currentfill}%
\pgfsetfillopacity{0.700000}%
\pgfsetlinewidth{0.000000pt}%
\definecolor{currentstroke}{rgb}{0.000000,0.000000,0.000000}%
\pgfsetstrokecolor{currentstroke}%
\pgfsetstrokeopacity{0.700000}%
\pgfsetdash{}{0pt}%
\pgfpathmoveto{\pgfqpoint{1.576263in}{0.521603in}}%
\pgfpathlineto{\pgfqpoint{1.627316in}{0.521603in}}%
\pgfpathlineto{\pgfqpoint{1.627316in}{0.675878in}}%
\pgfpathlineto{\pgfqpoint{1.576263in}{0.675878in}}%
\pgfpathlineto{\pgfqpoint{1.576263in}{0.521603in}}%
\pgfpathclose%
\pgfusepath{fill}%
\end{pgfscope}%
\begin{pgfscope}%
\pgfpathrectangle{\pgfqpoint{0.664969in}{0.521603in}}{\pgfqpoint{3.201017in}{1.696195in}}%
\pgfusepath{clip}%
\pgfsetbuttcap%
\pgfsetmiterjoin%
\definecolor{currentfill}{rgb}{1.000000,0.498039,0.054902}%
\pgfsetfillcolor{currentfill}%
\pgfsetfillopacity{0.700000}%
\pgfsetlinewidth{0.000000pt}%
\definecolor{currentstroke}{rgb}{0.000000,0.000000,0.000000}%
\pgfsetstrokecolor{currentstroke}%
\pgfsetstrokeopacity{0.700000}%
\pgfsetdash{}{0pt}%
\pgfpathmoveto{\pgfqpoint{1.627316in}{0.521603in}}%
\pgfpathlineto{\pgfqpoint{1.678369in}{0.521603in}}%
\pgfpathlineto{\pgfqpoint{1.678369in}{0.670778in}}%
\pgfpathlineto{\pgfqpoint{1.627316in}{0.670778in}}%
\pgfpathlineto{\pgfqpoint{1.627316in}{0.521603in}}%
\pgfpathclose%
\pgfusepath{fill}%
\end{pgfscope}%
\begin{pgfscope}%
\pgfpathrectangle{\pgfqpoint{0.664969in}{0.521603in}}{\pgfqpoint{3.201017in}{1.696195in}}%
\pgfusepath{clip}%
\pgfsetbuttcap%
\pgfsetmiterjoin%
\definecolor{currentfill}{rgb}{1.000000,0.498039,0.054902}%
\pgfsetfillcolor{currentfill}%
\pgfsetfillopacity{0.700000}%
\pgfsetlinewidth{0.000000pt}%
\definecolor{currentstroke}{rgb}{0.000000,0.000000,0.000000}%
\pgfsetstrokecolor{currentstroke}%
\pgfsetstrokeopacity{0.700000}%
\pgfsetdash{}{0pt}%
\pgfpathmoveto{\pgfqpoint{1.678369in}{0.521603in}}%
\pgfpathlineto{\pgfqpoint{1.729422in}{0.521603in}}%
\pgfpathlineto{\pgfqpoint{1.729422in}{0.666953in}}%
\pgfpathlineto{\pgfqpoint{1.678369in}{0.666953in}}%
\pgfpathlineto{\pgfqpoint{1.678369in}{0.521603in}}%
\pgfpathclose%
\pgfusepath{fill}%
\end{pgfscope}%
\begin{pgfscope}%
\pgfpathrectangle{\pgfqpoint{0.664969in}{0.521603in}}{\pgfqpoint{3.201017in}{1.696195in}}%
\pgfusepath{clip}%
\pgfsetbuttcap%
\pgfsetmiterjoin%
\definecolor{currentfill}{rgb}{1.000000,0.498039,0.054902}%
\pgfsetfillcolor{currentfill}%
\pgfsetfillopacity{0.700000}%
\pgfsetlinewidth{0.000000pt}%
\definecolor{currentstroke}{rgb}{0.000000,0.000000,0.000000}%
\pgfsetstrokecolor{currentstroke}%
\pgfsetstrokeopacity{0.700000}%
\pgfsetdash{}{0pt}%
\pgfpathmoveto{\pgfqpoint{1.729422in}{0.521603in}}%
\pgfpathlineto{\pgfqpoint{1.780475in}{0.521603in}}%
\pgfpathlineto{\pgfqpoint{1.780475in}{0.656753in}}%
\pgfpathlineto{\pgfqpoint{1.729422in}{0.656753in}}%
\pgfpathlineto{\pgfqpoint{1.729422in}{0.521603in}}%
\pgfpathclose%
\pgfusepath{fill}%
\end{pgfscope}%
\begin{pgfscope}%
\pgfpathrectangle{\pgfqpoint{0.664969in}{0.521603in}}{\pgfqpoint{3.201017in}{1.696195in}}%
\pgfusepath{clip}%
\pgfsetbuttcap%
\pgfsetmiterjoin%
\definecolor{currentfill}{rgb}{1.000000,0.498039,0.054902}%
\pgfsetfillcolor{currentfill}%
\pgfsetfillopacity{0.700000}%
\pgfsetlinewidth{0.000000pt}%
\definecolor{currentstroke}{rgb}{0.000000,0.000000,0.000000}%
\pgfsetstrokecolor{currentstroke}%
\pgfsetstrokeopacity{0.700000}%
\pgfsetdash{}{0pt}%
\pgfpathmoveto{\pgfqpoint{1.780475in}{0.521603in}}%
\pgfpathlineto{\pgfqpoint{1.831528in}{0.521603in}}%
\pgfpathlineto{\pgfqpoint{1.831528in}{0.675878in}}%
\pgfpathlineto{\pgfqpoint{1.780475in}{0.675878in}}%
\pgfpathlineto{\pgfqpoint{1.780475in}{0.521603in}}%
\pgfpathclose%
\pgfusepath{fill}%
\end{pgfscope}%
\begin{pgfscope}%
\pgfpathrectangle{\pgfqpoint{0.664969in}{0.521603in}}{\pgfqpoint{3.201017in}{1.696195in}}%
\pgfusepath{clip}%
\pgfsetbuttcap%
\pgfsetmiterjoin%
\definecolor{currentfill}{rgb}{1.000000,0.498039,0.054902}%
\pgfsetfillcolor{currentfill}%
\pgfsetfillopacity{0.700000}%
\pgfsetlinewidth{0.000000pt}%
\definecolor{currentstroke}{rgb}{0.000000,0.000000,0.000000}%
\pgfsetstrokecolor{currentstroke}%
\pgfsetstrokeopacity{0.700000}%
\pgfsetdash{}{0pt}%
\pgfpathmoveto{\pgfqpoint{1.831528in}{0.521603in}}%
\pgfpathlineto{\pgfqpoint{1.882581in}{0.521603in}}%
\pgfpathlineto{\pgfqpoint{1.882581in}{0.664403in}}%
\pgfpathlineto{\pgfqpoint{1.831528in}{0.664403in}}%
\pgfpathlineto{\pgfqpoint{1.831528in}{0.521603in}}%
\pgfpathclose%
\pgfusepath{fill}%
\end{pgfscope}%
\begin{pgfscope}%
\pgfpathrectangle{\pgfqpoint{0.664969in}{0.521603in}}{\pgfqpoint{3.201017in}{1.696195in}}%
\pgfusepath{clip}%
\pgfsetbuttcap%
\pgfsetmiterjoin%
\definecolor{currentfill}{rgb}{1.000000,0.498039,0.054902}%
\pgfsetfillcolor{currentfill}%
\pgfsetfillopacity{0.700000}%
\pgfsetlinewidth{0.000000pt}%
\definecolor{currentstroke}{rgb}{0.000000,0.000000,0.000000}%
\pgfsetstrokecolor{currentstroke}%
\pgfsetstrokeopacity{0.700000}%
\pgfsetdash{}{0pt}%
\pgfpathmoveto{\pgfqpoint{1.882581in}{0.521603in}}%
\pgfpathlineto{\pgfqpoint{1.933634in}{0.521603in}}%
\pgfpathlineto{\pgfqpoint{1.933634in}{0.647828in}}%
\pgfpathlineto{\pgfqpoint{1.882581in}{0.647828in}}%
\pgfpathlineto{\pgfqpoint{1.882581in}{0.521603in}}%
\pgfpathclose%
\pgfusepath{fill}%
\end{pgfscope}%
\begin{pgfscope}%
\pgfpathrectangle{\pgfqpoint{0.664969in}{0.521603in}}{\pgfqpoint{3.201017in}{1.696195in}}%
\pgfusepath{clip}%
\pgfsetbuttcap%
\pgfsetmiterjoin%
\definecolor{currentfill}{rgb}{1.000000,0.498039,0.054902}%
\pgfsetfillcolor{currentfill}%
\pgfsetfillopacity{0.700000}%
\pgfsetlinewidth{0.000000pt}%
\definecolor{currentstroke}{rgb}{0.000000,0.000000,0.000000}%
\pgfsetstrokecolor{currentstroke}%
\pgfsetstrokeopacity{0.700000}%
\pgfsetdash{}{0pt}%
\pgfpathmoveto{\pgfqpoint{1.933634in}{0.521603in}}%
\pgfpathlineto{\pgfqpoint{1.984687in}{0.521603in}}%
\pgfpathlineto{\pgfqpoint{1.984687in}{0.633803in}}%
\pgfpathlineto{\pgfqpoint{1.933634in}{0.633803in}}%
\pgfpathlineto{\pgfqpoint{1.933634in}{0.521603in}}%
\pgfpathclose%
\pgfusepath{fill}%
\end{pgfscope}%
\begin{pgfscope}%
\pgfpathrectangle{\pgfqpoint{0.664969in}{0.521603in}}{\pgfqpoint{3.201017in}{1.696195in}}%
\pgfusepath{clip}%
\pgfsetbuttcap%
\pgfsetmiterjoin%
\definecolor{currentfill}{rgb}{1.000000,0.498039,0.054902}%
\pgfsetfillcolor{currentfill}%
\pgfsetfillopacity{0.700000}%
\pgfsetlinewidth{0.000000pt}%
\definecolor{currentstroke}{rgb}{0.000000,0.000000,0.000000}%
\pgfsetstrokecolor{currentstroke}%
\pgfsetstrokeopacity{0.700000}%
\pgfsetdash{}{0pt}%
\pgfpathmoveto{\pgfqpoint{1.984687in}{0.521603in}}%
\pgfpathlineto{\pgfqpoint{2.035740in}{0.521603in}}%
\pgfpathlineto{\pgfqpoint{2.035740in}{0.647828in}}%
\pgfpathlineto{\pgfqpoint{1.984687in}{0.647828in}}%
\pgfpathlineto{\pgfqpoint{1.984687in}{0.521603in}}%
\pgfpathclose%
\pgfusepath{fill}%
\end{pgfscope}%
\begin{pgfscope}%
\pgfpathrectangle{\pgfqpoint{0.664969in}{0.521603in}}{\pgfqpoint{3.201017in}{1.696195in}}%
\pgfusepath{clip}%
\pgfsetbuttcap%
\pgfsetmiterjoin%
\definecolor{currentfill}{rgb}{1.000000,0.498039,0.054902}%
\pgfsetfillcolor{currentfill}%
\pgfsetfillopacity{0.700000}%
\pgfsetlinewidth{0.000000pt}%
\definecolor{currentstroke}{rgb}{0.000000,0.000000,0.000000}%
\pgfsetstrokecolor{currentstroke}%
\pgfsetstrokeopacity{0.700000}%
\pgfsetdash{}{0pt}%
\pgfpathmoveto{\pgfqpoint{2.035740in}{0.521603in}}%
\pgfpathlineto{\pgfqpoint{2.086793in}{0.521603in}}%
\pgfpathlineto{\pgfqpoint{2.086793in}{0.623603in}}%
\pgfpathlineto{\pgfqpoint{2.035740in}{0.623603in}}%
\pgfpathlineto{\pgfqpoint{2.035740in}{0.521603in}}%
\pgfpathclose%
\pgfusepath{fill}%
\end{pgfscope}%
\begin{pgfscope}%
\pgfpathrectangle{\pgfqpoint{0.664969in}{0.521603in}}{\pgfqpoint{3.201017in}{1.696195in}}%
\pgfusepath{clip}%
\pgfsetbuttcap%
\pgfsetmiterjoin%
\definecolor{currentfill}{rgb}{1.000000,0.498039,0.054902}%
\pgfsetfillcolor{currentfill}%
\pgfsetfillopacity{0.700000}%
\pgfsetlinewidth{0.000000pt}%
\definecolor{currentstroke}{rgb}{0.000000,0.000000,0.000000}%
\pgfsetstrokecolor{currentstroke}%
\pgfsetstrokeopacity{0.700000}%
\pgfsetdash{}{0pt}%
\pgfpathmoveto{\pgfqpoint{2.086793in}{0.521603in}}%
\pgfpathlineto{\pgfqpoint{2.137845in}{0.521603in}}%
\pgfpathlineto{\pgfqpoint{2.137845in}{0.658028in}}%
\pgfpathlineto{\pgfqpoint{2.086793in}{0.658028in}}%
\pgfpathlineto{\pgfqpoint{2.086793in}{0.521603in}}%
\pgfpathclose%
\pgfusepath{fill}%
\end{pgfscope}%
\begin{pgfscope}%
\pgfpathrectangle{\pgfqpoint{0.664969in}{0.521603in}}{\pgfqpoint{3.201017in}{1.696195in}}%
\pgfusepath{clip}%
\pgfsetbuttcap%
\pgfsetmiterjoin%
\definecolor{currentfill}{rgb}{1.000000,0.498039,0.054902}%
\pgfsetfillcolor{currentfill}%
\pgfsetfillopacity{0.700000}%
\pgfsetlinewidth{0.000000pt}%
\definecolor{currentstroke}{rgb}{0.000000,0.000000,0.000000}%
\pgfsetstrokecolor{currentstroke}%
\pgfsetstrokeopacity{0.700000}%
\pgfsetdash{}{0pt}%
\pgfpathmoveto{\pgfqpoint{2.137845in}{0.521603in}}%
\pgfpathlineto{\pgfqpoint{2.188898in}{0.521603in}}%
\pgfpathlineto{\pgfqpoint{2.188898in}{0.632528in}}%
\pgfpathlineto{\pgfqpoint{2.137845in}{0.632528in}}%
\pgfpathlineto{\pgfqpoint{2.137845in}{0.521603in}}%
\pgfpathclose%
\pgfusepath{fill}%
\end{pgfscope}%
\begin{pgfscope}%
\pgfpathrectangle{\pgfqpoint{0.664969in}{0.521603in}}{\pgfqpoint{3.201017in}{1.696195in}}%
\pgfusepath{clip}%
\pgfsetbuttcap%
\pgfsetmiterjoin%
\definecolor{currentfill}{rgb}{1.000000,0.498039,0.054902}%
\pgfsetfillcolor{currentfill}%
\pgfsetfillopacity{0.700000}%
\pgfsetlinewidth{0.000000pt}%
\definecolor{currentstroke}{rgb}{0.000000,0.000000,0.000000}%
\pgfsetstrokecolor{currentstroke}%
\pgfsetstrokeopacity{0.700000}%
\pgfsetdash{}{0pt}%
\pgfpathmoveto{\pgfqpoint{2.188898in}{0.521603in}}%
\pgfpathlineto{\pgfqpoint{2.239951in}{0.521603in}}%
\pgfpathlineto{\pgfqpoint{2.239951in}{0.673328in}}%
\pgfpathlineto{\pgfqpoint{2.188898in}{0.673328in}}%
\pgfpathlineto{\pgfqpoint{2.188898in}{0.521603in}}%
\pgfpathclose%
\pgfusepath{fill}%
\end{pgfscope}%
\begin{pgfscope}%
\pgfpathrectangle{\pgfqpoint{0.664969in}{0.521603in}}{\pgfqpoint{3.201017in}{1.696195in}}%
\pgfusepath{clip}%
\pgfsetbuttcap%
\pgfsetmiterjoin%
\definecolor{currentfill}{rgb}{1.000000,0.498039,0.054902}%
\pgfsetfillcolor{currentfill}%
\pgfsetfillopacity{0.700000}%
\pgfsetlinewidth{0.000000pt}%
\definecolor{currentstroke}{rgb}{0.000000,0.000000,0.000000}%
\pgfsetstrokecolor{currentstroke}%
\pgfsetstrokeopacity{0.700000}%
\pgfsetdash{}{0pt}%
\pgfpathmoveto{\pgfqpoint{2.239951in}{0.521603in}}%
\pgfpathlineto{\pgfqpoint{2.291004in}{0.521603in}}%
\pgfpathlineto{\pgfqpoint{2.291004in}{0.598103in}}%
\pgfpathlineto{\pgfqpoint{2.239951in}{0.598103in}}%
\pgfpathlineto{\pgfqpoint{2.239951in}{0.521603in}}%
\pgfpathclose%
\pgfusepath{fill}%
\end{pgfscope}%
\begin{pgfscope}%
\pgfpathrectangle{\pgfqpoint{0.664969in}{0.521603in}}{\pgfqpoint{3.201017in}{1.696195in}}%
\pgfusepath{clip}%
\pgfsetbuttcap%
\pgfsetmiterjoin%
\definecolor{currentfill}{rgb}{1.000000,0.498039,0.054902}%
\pgfsetfillcolor{currentfill}%
\pgfsetfillopacity{0.700000}%
\pgfsetlinewidth{0.000000pt}%
\definecolor{currentstroke}{rgb}{0.000000,0.000000,0.000000}%
\pgfsetstrokecolor{currentstroke}%
\pgfsetstrokeopacity{0.700000}%
\pgfsetdash{}{0pt}%
\pgfpathmoveto{\pgfqpoint{2.291004in}{0.521603in}}%
\pgfpathlineto{\pgfqpoint{2.342057in}{0.521603in}}%
\pgfpathlineto{\pgfqpoint{2.342057in}{0.637628in}}%
\pgfpathlineto{\pgfqpoint{2.291004in}{0.637628in}}%
\pgfpathlineto{\pgfqpoint{2.291004in}{0.521603in}}%
\pgfpathclose%
\pgfusepath{fill}%
\end{pgfscope}%
\begin{pgfscope}%
\pgfpathrectangle{\pgfqpoint{0.664969in}{0.521603in}}{\pgfqpoint{3.201017in}{1.696195in}}%
\pgfusepath{clip}%
\pgfsetbuttcap%
\pgfsetmiterjoin%
\definecolor{currentfill}{rgb}{1.000000,0.498039,0.054902}%
\pgfsetfillcolor{currentfill}%
\pgfsetfillopacity{0.700000}%
\pgfsetlinewidth{0.000000pt}%
\definecolor{currentstroke}{rgb}{0.000000,0.000000,0.000000}%
\pgfsetstrokecolor{currentstroke}%
\pgfsetstrokeopacity{0.700000}%
\pgfsetdash{}{0pt}%
\pgfpathmoveto{\pgfqpoint{2.342057in}{0.521603in}}%
\pgfpathlineto{\pgfqpoint{2.393110in}{0.521603in}}%
\pgfpathlineto{\pgfqpoint{2.393110in}{0.650378in}}%
\pgfpathlineto{\pgfqpoint{2.342057in}{0.650378in}}%
\pgfpathlineto{\pgfqpoint{2.342057in}{0.521603in}}%
\pgfpathclose%
\pgfusepath{fill}%
\end{pgfscope}%
\begin{pgfscope}%
\pgfpathrectangle{\pgfqpoint{0.664969in}{0.521603in}}{\pgfqpoint{3.201017in}{1.696195in}}%
\pgfusepath{clip}%
\pgfsetbuttcap%
\pgfsetmiterjoin%
\definecolor{currentfill}{rgb}{1.000000,0.498039,0.054902}%
\pgfsetfillcolor{currentfill}%
\pgfsetfillopacity{0.700000}%
\pgfsetlinewidth{0.000000pt}%
\definecolor{currentstroke}{rgb}{0.000000,0.000000,0.000000}%
\pgfsetstrokecolor{currentstroke}%
\pgfsetstrokeopacity{0.700000}%
\pgfsetdash{}{0pt}%
\pgfpathmoveto{\pgfqpoint{2.393110in}{0.521603in}}%
\pgfpathlineto{\pgfqpoint{2.444163in}{0.521603in}}%
\pgfpathlineto{\pgfqpoint{2.444163in}{0.621053in}}%
\pgfpathlineto{\pgfqpoint{2.393110in}{0.621053in}}%
\pgfpathlineto{\pgfqpoint{2.393110in}{0.521603in}}%
\pgfpathclose%
\pgfusepath{fill}%
\end{pgfscope}%
\begin{pgfscope}%
\pgfpathrectangle{\pgfqpoint{0.664969in}{0.521603in}}{\pgfqpoint{3.201017in}{1.696195in}}%
\pgfusepath{clip}%
\pgfsetbuttcap%
\pgfsetmiterjoin%
\definecolor{currentfill}{rgb}{1.000000,0.498039,0.054902}%
\pgfsetfillcolor{currentfill}%
\pgfsetfillopacity{0.700000}%
\pgfsetlinewidth{0.000000pt}%
\definecolor{currentstroke}{rgb}{0.000000,0.000000,0.000000}%
\pgfsetstrokecolor{currentstroke}%
\pgfsetstrokeopacity{0.700000}%
\pgfsetdash{}{0pt}%
\pgfpathmoveto{\pgfqpoint{2.444163in}{0.521603in}}%
\pgfpathlineto{\pgfqpoint{2.495216in}{0.521603in}}%
\pgfpathlineto{\pgfqpoint{2.495216in}{0.632528in}}%
\pgfpathlineto{\pgfqpoint{2.444163in}{0.632528in}}%
\pgfpathlineto{\pgfqpoint{2.444163in}{0.521603in}}%
\pgfpathclose%
\pgfusepath{fill}%
\end{pgfscope}%
\begin{pgfscope}%
\pgfpathrectangle{\pgfqpoint{0.664969in}{0.521603in}}{\pgfqpoint{3.201017in}{1.696195in}}%
\pgfusepath{clip}%
\pgfsetbuttcap%
\pgfsetmiterjoin%
\definecolor{currentfill}{rgb}{1.000000,0.498039,0.054902}%
\pgfsetfillcolor{currentfill}%
\pgfsetfillopacity{0.700000}%
\pgfsetlinewidth{0.000000pt}%
\definecolor{currentstroke}{rgb}{0.000000,0.000000,0.000000}%
\pgfsetstrokecolor{currentstroke}%
\pgfsetstrokeopacity{0.700000}%
\pgfsetdash{}{0pt}%
\pgfpathmoveto{\pgfqpoint{2.495216in}{0.521603in}}%
\pgfpathlineto{\pgfqpoint{2.546269in}{0.521603in}}%
\pgfpathlineto{\pgfqpoint{2.546269in}{0.642728in}}%
\pgfpathlineto{\pgfqpoint{2.495216in}{0.642728in}}%
\pgfpathlineto{\pgfqpoint{2.495216in}{0.521603in}}%
\pgfpathclose%
\pgfusepath{fill}%
\end{pgfscope}%
\begin{pgfscope}%
\pgfpathrectangle{\pgfqpoint{0.664969in}{0.521603in}}{\pgfqpoint{3.201017in}{1.696195in}}%
\pgfusepath{clip}%
\pgfsetbuttcap%
\pgfsetmiterjoin%
\definecolor{currentfill}{rgb}{1.000000,0.498039,0.054902}%
\pgfsetfillcolor{currentfill}%
\pgfsetfillopacity{0.700000}%
\pgfsetlinewidth{0.000000pt}%
\definecolor{currentstroke}{rgb}{0.000000,0.000000,0.000000}%
\pgfsetstrokecolor{currentstroke}%
\pgfsetstrokeopacity{0.700000}%
\pgfsetdash{}{0pt}%
\pgfpathmoveto{\pgfqpoint{2.546269in}{0.521603in}}%
\pgfpathlineto{\pgfqpoint{2.597322in}{0.521603in}}%
\pgfpathlineto{\pgfqpoint{2.597322in}{0.632528in}}%
\pgfpathlineto{\pgfqpoint{2.546269in}{0.632528in}}%
\pgfpathlineto{\pgfqpoint{2.546269in}{0.521603in}}%
\pgfpathclose%
\pgfusepath{fill}%
\end{pgfscope}%
\begin{pgfscope}%
\pgfpathrectangle{\pgfqpoint{0.664969in}{0.521603in}}{\pgfqpoint{3.201017in}{1.696195in}}%
\pgfusepath{clip}%
\pgfsetbuttcap%
\pgfsetmiterjoin%
\definecolor{currentfill}{rgb}{1.000000,0.498039,0.054902}%
\pgfsetfillcolor{currentfill}%
\pgfsetfillopacity{0.700000}%
\pgfsetlinewidth{0.000000pt}%
\definecolor{currentstroke}{rgb}{0.000000,0.000000,0.000000}%
\pgfsetstrokecolor{currentstroke}%
\pgfsetstrokeopacity{0.700000}%
\pgfsetdash{}{0pt}%
\pgfpathmoveto{\pgfqpoint{2.597322in}{0.521603in}}%
\pgfpathlineto{\pgfqpoint{2.648374in}{0.521603in}}%
\pgfpathlineto{\pgfqpoint{2.648374in}{0.641453in}}%
\pgfpathlineto{\pgfqpoint{2.597322in}{0.641453in}}%
\pgfpathlineto{\pgfqpoint{2.597322in}{0.521603in}}%
\pgfpathclose%
\pgfusepath{fill}%
\end{pgfscope}%
\begin{pgfscope}%
\pgfpathrectangle{\pgfqpoint{0.664969in}{0.521603in}}{\pgfqpoint{3.201017in}{1.696195in}}%
\pgfusepath{clip}%
\pgfsetbuttcap%
\pgfsetmiterjoin%
\definecolor{currentfill}{rgb}{1.000000,0.498039,0.054902}%
\pgfsetfillcolor{currentfill}%
\pgfsetfillopacity{0.700000}%
\pgfsetlinewidth{0.000000pt}%
\definecolor{currentstroke}{rgb}{0.000000,0.000000,0.000000}%
\pgfsetstrokecolor{currentstroke}%
\pgfsetstrokeopacity{0.700000}%
\pgfsetdash{}{0pt}%
\pgfpathmoveto{\pgfqpoint{2.648374in}{0.521603in}}%
\pgfpathlineto{\pgfqpoint{2.699427in}{0.521603in}}%
\pgfpathlineto{\pgfqpoint{2.699427in}{0.610853in}}%
\pgfpathlineto{\pgfqpoint{2.648374in}{0.610853in}}%
\pgfpathlineto{\pgfqpoint{2.648374in}{0.521603in}}%
\pgfpathclose%
\pgfusepath{fill}%
\end{pgfscope}%
\begin{pgfscope}%
\pgfpathrectangle{\pgfqpoint{0.664969in}{0.521603in}}{\pgfqpoint{3.201017in}{1.696195in}}%
\pgfusepath{clip}%
\pgfsetbuttcap%
\pgfsetmiterjoin%
\definecolor{currentfill}{rgb}{1.000000,0.498039,0.054902}%
\pgfsetfillcolor{currentfill}%
\pgfsetfillopacity{0.700000}%
\pgfsetlinewidth{0.000000pt}%
\definecolor{currentstroke}{rgb}{0.000000,0.000000,0.000000}%
\pgfsetstrokecolor{currentstroke}%
\pgfsetstrokeopacity{0.700000}%
\pgfsetdash{}{0pt}%
\pgfpathmoveto{\pgfqpoint{2.699427in}{0.521603in}}%
\pgfpathlineto{\pgfqpoint{2.750480in}{0.521603in}}%
\pgfpathlineto{\pgfqpoint{2.750480in}{0.614678in}}%
\pgfpathlineto{\pgfqpoint{2.699427in}{0.614678in}}%
\pgfpathlineto{\pgfqpoint{2.699427in}{0.521603in}}%
\pgfpathclose%
\pgfusepath{fill}%
\end{pgfscope}%
\begin{pgfscope}%
\pgfpathrectangle{\pgfqpoint{0.664969in}{0.521603in}}{\pgfqpoint{3.201017in}{1.696195in}}%
\pgfusepath{clip}%
\pgfsetbuttcap%
\pgfsetmiterjoin%
\definecolor{currentfill}{rgb}{1.000000,0.498039,0.054902}%
\pgfsetfillcolor{currentfill}%
\pgfsetfillopacity{0.700000}%
\pgfsetlinewidth{0.000000pt}%
\definecolor{currentstroke}{rgb}{0.000000,0.000000,0.000000}%
\pgfsetstrokecolor{currentstroke}%
\pgfsetstrokeopacity{0.700000}%
\pgfsetdash{}{0pt}%
\pgfpathmoveto{\pgfqpoint{2.750480in}{0.521603in}}%
\pgfpathlineto{\pgfqpoint{2.801533in}{0.521603in}}%
\pgfpathlineto{\pgfqpoint{2.801533in}{0.604478in}}%
\pgfpathlineto{\pgfqpoint{2.750480in}{0.604478in}}%
\pgfpathlineto{\pgfqpoint{2.750480in}{0.521603in}}%
\pgfpathclose%
\pgfusepath{fill}%
\end{pgfscope}%
\begin{pgfscope}%
\pgfpathrectangle{\pgfqpoint{0.664969in}{0.521603in}}{\pgfqpoint{3.201017in}{1.696195in}}%
\pgfusepath{clip}%
\pgfsetbuttcap%
\pgfsetmiterjoin%
\definecolor{currentfill}{rgb}{1.000000,0.498039,0.054902}%
\pgfsetfillcolor{currentfill}%
\pgfsetfillopacity{0.700000}%
\pgfsetlinewidth{0.000000pt}%
\definecolor{currentstroke}{rgb}{0.000000,0.000000,0.000000}%
\pgfsetstrokecolor{currentstroke}%
\pgfsetstrokeopacity{0.700000}%
\pgfsetdash{}{0pt}%
\pgfpathmoveto{\pgfqpoint{2.801533in}{0.521603in}}%
\pgfpathlineto{\pgfqpoint{2.852586in}{0.521603in}}%
\pgfpathlineto{\pgfqpoint{2.852586in}{0.596828in}}%
\pgfpathlineto{\pgfqpoint{2.801533in}{0.596828in}}%
\pgfpathlineto{\pgfqpoint{2.801533in}{0.521603in}}%
\pgfpathclose%
\pgfusepath{fill}%
\end{pgfscope}%
\begin{pgfscope}%
\pgfpathrectangle{\pgfqpoint{0.664969in}{0.521603in}}{\pgfqpoint{3.201017in}{1.696195in}}%
\pgfusepath{clip}%
\pgfsetbuttcap%
\pgfsetmiterjoin%
\definecolor{currentfill}{rgb}{1.000000,0.498039,0.054902}%
\pgfsetfillcolor{currentfill}%
\pgfsetfillopacity{0.700000}%
\pgfsetlinewidth{0.000000pt}%
\definecolor{currentstroke}{rgb}{0.000000,0.000000,0.000000}%
\pgfsetstrokecolor{currentstroke}%
\pgfsetstrokeopacity{0.700000}%
\pgfsetdash{}{0pt}%
\pgfpathmoveto{\pgfqpoint{2.852586in}{0.521603in}}%
\pgfpathlineto{\pgfqpoint{2.903639in}{0.521603in}}%
\pgfpathlineto{\pgfqpoint{2.903639in}{0.594278in}}%
\pgfpathlineto{\pgfqpoint{2.852586in}{0.594278in}}%
\pgfpathlineto{\pgfqpoint{2.852586in}{0.521603in}}%
\pgfpathclose%
\pgfusepath{fill}%
\end{pgfscope}%
\begin{pgfscope}%
\pgfpathrectangle{\pgfqpoint{0.664969in}{0.521603in}}{\pgfqpoint{3.201017in}{1.696195in}}%
\pgfusepath{clip}%
\pgfsetbuttcap%
\pgfsetmiterjoin%
\definecolor{currentfill}{rgb}{1.000000,0.498039,0.054902}%
\pgfsetfillcolor{currentfill}%
\pgfsetfillopacity{0.700000}%
\pgfsetlinewidth{0.000000pt}%
\definecolor{currentstroke}{rgb}{0.000000,0.000000,0.000000}%
\pgfsetstrokecolor{currentstroke}%
\pgfsetstrokeopacity{0.700000}%
\pgfsetdash{}{0pt}%
\pgfpathmoveto{\pgfqpoint{2.903639in}{0.521603in}}%
\pgfpathlineto{\pgfqpoint{2.954692in}{0.521603in}}%
\pgfpathlineto{\pgfqpoint{2.954692in}{0.586628in}}%
\pgfpathlineto{\pgfqpoint{2.903639in}{0.586628in}}%
\pgfpathlineto{\pgfqpoint{2.903639in}{0.521603in}}%
\pgfpathclose%
\pgfusepath{fill}%
\end{pgfscope}%
\begin{pgfscope}%
\pgfpathrectangle{\pgfqpoint{0.664969in}{0.521603in}}{\pgfqpoint{3.201017in}{1.696195in}}%
\pgfusepath{clip}%
\pgfsetbuttcap%
\pgfsetmiterjoin%
\definecolor{currentfill}{rgb}{1.000000,0.498039,0.054902}%
\pgfsetfillcolor{currentfill}%
\pgfsetfillopacity{0.700000}%
\pgfsetlinewidth{0.000000pt}%
\definecolor{currentstroke}{rgb}{0.000000,0.000000,0.000000}%
\pgfsetstrokecolor{currentstroke}%
\pgfsetstrokeopacity{0.700000}%
\pgfsetdash{}{0pt}%
\pgfpathmoveto{\pgfqpoint{2.954692in}{0.521603in}}%
\pgfpathlineto{\pgfqpoint{3.005745in}{0.521603in}}%
\pgfpathlineto{\pgfqpoint{3.005745in}{0.587903in}}%
\pgfpathlineto{\pgfqpoint{2.954692in}{0.587903in}}%
\pgfpathlineto{\pgfqpoint{2.954692in}{0.521603in}}%
\pgfpathclose%
\pgfusepath{fill}%
\end{pgfscope}%
\begin{pgfscope}%
\pgfpathrectangle{\pgfqpoint{0.664969in}{0.521603in}}{\pgfqpoint{3.201017in}{1.696195in}}%
\pgfusepath{clip}%
\pgfsetbuttcap%
\pgfsetmiterjoin%
\definecolor{currentfill}{rgb}{1.000000,0.498039,0.054902}%
\pgfsetfillcolor{currentfill}%
\pgfsetfillopacity{0.700000}%
\pgfsetlinewidth{0.000000pt}%
\definecolor{currentstroke}{rgb}{0.000000,0.000000,0.000000}%
\pgfsetstrokecolor{currentstroke}%
\pgfsetstrokeopacity{0.700000}%
\pgfsetdash{}{0pt}%
\pgfpathmoveto{\pgfqpoint{3.005745in}{0.521603in}}%
\pgfpathlineto{\pgfqpoint{3.056798in}{0.521603in}}%
\pgfpathlineto{\pgfqpoint{3.056798in}{0.567503in}}%
\pgfpathlineto{\pgfqpoint{3.005745in}{0.567503in}}%
\pgfpathlineto{\pgfqpoint{3.005745in}{0.521603in}}%
\pgfpathclose%
\pgfusepath{fill}%
\end{pgfscope}%
\begin{pgfscope}%
\pgfpathrectangle{\pgfqpoint{0.664969in}{0.521603in}}{\pgfqpoint{3.201017in}{1.696195in}}%
\pgfusepath{clip}%
\pgfsetbuttcap%
\pgfsetmiterjoin%
\definecolor{currentfill}{rgb}{1.000000,0.498039,0.054902}%
\pgfsetfillcolor{currentfill}%
\pgfsetfillopacity{0.700000}%
\pgfsetlinewidth{0.000000pt}%
\definecolor{currentstroke}{rgb}{0.000000,0.000000,0.000000}%
\pgfsetstrokecolor{currentstroke}%
\pgfsetstrokeopacity{0.700000}%
\pgfsetdash{}{0pt}%
\pgfpathmoveto{\pgfqpoint{3.056798in}{0.521603in}}%
\pgfpathlineto{\pgfqpoint{3.107851in}{0.521603in}}%
\pgfpathlineto{\pgfqpoint{3.107851in}{0.566228in}}%
\pgfpathlineto{\pgfqpoint{3.056798in}{0.566228in}}%
\pgfpathlineto{\pgfqpoint{3.056798in}{0.521603in}}%
\pgfpathclose%
\pgfusepath{fill}%
\end{pgfscope}%
\begin{pgfscope}%
\pgfpathrectangle{\pgfqpoint{0.664969in}{0.521603in}}{\pgfqpoint{3.201017in}{1.696195in}}%
\pgfusepath{clip}%
\pgfsetbuttcap%
\pgfsetmiterjoin%
\definecolor{currentfill}{rgb}{1.000000,0.498039,0.054902}%
\pgfsetfillcolor{currentfill}%
\pgfsetfillopacity{0.700000}%
\pgfsetlinewidth{0.000000pt}%
\definecolor{currentstroke}{rgb}{0.000000,0.000000,0.000000}%
\pgfsetstrokecolor{currentstroke}%
\pgfsetstrokeopacity{0.700000}%
\pgfsetdash{}{0pt}%
\pgfpathmoveto{\pgfqpoint{3.107851in}{0.521603in}}%
\pgfpathlineto{\pgfqpoint{3.158903in}{0.521603in}}%
\pgfpathlineto{\pgfqpoint{3.158903in}{0.559853in}}%
\pgfpathlineto{\pgfqpoint{3.107851in}{0.559853in}}%
\pgfpathlineto{\pgfqpoint{3.107851in}{0.521603in}}%
\pgfpathclose%
\pgfusepath{fill}%
\end{pgfscope}%
\begin{pgfscope}%
\pgfpathrectangle{\pgfqpoint{0.664969in}{0.521603in}}{\pgfqpoint{3.201017in}{1.696195in}}%
\pgfusepath{clip}%
\pgfsetbuttcap%
\pgfsetmiterjoin%
\definecolor{currentfill}{rgb}{1.000000,0.498039,0.054902}%
\pgfsetfillcolor{currentfill}%
\pgfsetfillopacity{0.700000}%
\pgfsetlinewidth{0.000000pt}%
\definecolor{currentstroke}{rgb}{0.000000,0.000000,0.000000}%
\pgfsetstrokecolor{currentstroke}%
\pgfsetstrokeopacity{0.700000}%
\pgfsetdash{}{0pt}%
\pgfpathmoveto{\pgfqpoint{3.158903in}{0.521603in}}%
\pgfpathlineto{\pgfqpoint{3.209956in}{0.521603in}}%
\pgfpathlineto{\pgfqpoint{3.209956in}{0.542003in}}%
\pgfpathlineto{\pgfqpoint{3.158903in}{0.542003in}}%
\pgfpathlineto{\pgfqpoint{3.158903in}{0.521603in}}%
\pgfpathclose%
\pgfusepath{fill}%
\end{pgfscope}%
\begin{pgfscope}%
\pgfpathrectangle{\pgfqpoint{0.664969in}{0.521603in}}{\pgfqpoint{3.201017in}{1.696195in}}%
\pgfusepath{clip}%
\pgfsetbuttcap%
\pgfsetmiterjoin%
\definecolor{currentfill}{rgb}{1.000000,0.498039,0.054902}%
\pgfsetfillcolor{currentfill}%
\pgfsetfillopacity{0.700000}%
\pgfsetlinewidth{0.000000pt}%
\definecolor{currentstroke}{rgb}{0.000000,0.000000,0.000000}%
\pgfsetstrokecolor{currentstroke}%
\pgfsetstrokeopacity{0.700000}%
\pgfsetdash{}{0pt}%
\pgfpathmoveto{\pgfqpoint{3.209956in}{0.521603in}}%
\pgfpathlineto{\pgfqpoint{3.261009in}{0.521603in}}%
\pgfpathlineto{\pgfqpoint{3.261009in}{0.543278in}}%
\pgfpathlineto{\pgfqpoint{3.209956in}{0.543278in}}%
\pgfpathlineto{\pgfqpoint{3.209956in}{0.521603in}}%
\pgfpathclose%
\pgfusepath{fill}%
\end{pgfscope}%
\begin{pgfscope}%
\pgfpathrectangle{\pgfqpoint{0.664969in}{0.521603in}}{\pgfqpoint{3.201017in}{1.696195in}}%
\pgfusepath{clip}%
\pgfsetbuttcap%
\pgfsetmiterjoin%
\definecolor{currentfill}{rgb}{1.000000,0.498039,0.054902}%
\pgfsetfillcolor{currentfill}%
\pgfsetfillopacity{0.700000}%
\pgfsetlinewidth{0.000000pt}%
\definecolor{currentstroke}{rgb}{0.000000,0.000000,0.000000}%
\pgfsetstrokecolor{currentstroke}%
\pgfsetstrokeopacity{0.700000}%
\pgfsetdash{}{0pt}%
\pgfpathmoveto{\pgfqpoint{3.261009in}{0.521603in}}%
\pgfpathlineto{\pgfqpoint{3.312062in}{0.521603in}}%
\pgfpathlineto{\pgfqpoint{3.312062in}{0.544553in}}%
\pgfpathlineto{\pgfqpoint{3.261009in}{0.544553in}}%
\pgfpathlineto{\pgfqpoint{3.261009in}{0.521603in}}%
\pgfpathclose%
\pgfusepath{fill}%
\end{pgfscope}%
\begin{pgfscope}%
\pgfpathrectangle{\pgfqpoint{0.664969in}{0.521603in}}{\pgfqpoint{3.201017in}{1.696195in}}%
\pgfusepath{clip}%
\pgfsetbuttcap%
\pgfsetmiterjoin%
\definecolor{currentfill}{rgb}{1.000000,0.498039,0.054902}%
\pgfsetfillcolor{currentfill}%
\pgfsetfillopacity{0.700000}%
\pgfsetlinewidth{0.000000pt}%
\definecolor{currentstroke}{rgb}{0.000000,0.000000,0.000000}%
\pgfsetstrokecolor{currentstroke}%
\pgfsetstrokeopacity{0.700000}%
\pgfsetdash{}{0pt}%
\pgfpathmoveto{\pgfqpoint{3.312062in}{0.521603in}}%
\pgfpathlineto{\pgfqpoint{3.363115in}{0.521603in}}%
\pgfpathlineto{\pgfqpoint{3.363115in}{0.525428in}}%
\pgfpathlineto{\pgfqpoint{3.312062in}{0.525428in}}%
\pgfpathlineto{\pgfqpoint{3.312062in}{0.521603in}}%
\pgfpathclose%
\pgfusepath{fill}%
\end{pgfscope}%
\begin{pgfscope}%
\pgfpathrectangle{\pgfqpoint{0.664969in}{0.521603in}}{\pgfqpoint{3.201017in}{1.696195in}}%
\pgfusepath{clip}%
\pgfsetbuttcap%
\pgfsetmiterjoin%
\definecolor{currentfill}{rgb}{1.000000,0.498039,0.054902}%
\pgfsetfillcolor{currentfill}%
\pgfsetfillopacity{0.700000}%
\pgfsetlinewidth{0.000000pt}%
\definecolor{currentstroke}{rgb}{0.000000,0.000000,0.000000}%
\pgfsetstrokecolor{currentstroke}%
\pgfsetstrokeopacity{0.700000}%
\pgfsetdash{}{0pt}%
\pgfpathmoveto{\pgfqpoint{3.363115in}{0.521603in}}%
\pgfpathlineto{\pgfqpoint{3.414168in}{0.521603in}}%
\pgfpathlineto{\pgfqpoint{3.414168in}{0.540728in}}%
\pgfpathlineto{\pgfqpoint{3.363115in}{0.540728in}}%
\pgfpathlineto{\pgfqpoint{3.363115in}{0.521603in}}%
\pgfpathclose%
\pgfusepath{fill}%
\end{pgfscope}%
\begin{pgfscope}%
\pgfpathrectangle{\pgfqpoint{0.664969in}{0.521603in}}{\pgfqpoint{3.201017in}{1.696195in}}%
\pgfusepath{clip}%
\pgfsetbuttcap%
\pgfsetmiterjoin%
\definecolor{currentfill}{rgb}{1.000000,0.498039,0.054902}%
\pgfsetfillcolor{currentfill}%
\pgfsetfillopacity{0.700000}%
\pgfsetlinewidth{0.000000pt}%
\definecolor{currentstroke}{rgb}{0.000000,0.000000,0.000000}%
\pgfsetstrokecolor{currentstroke}%
\pgfsetstrokeopacity{0.700000}%
\pgfsetdash{}{0pt}%
\pgfpathmoveto{\pgfqpoint{3.414168in}{0.521603in}}%
\pgfpathlineto{\pgfqpoint{3.465221in}{0.521603in}}%
\pgfpathlineto{\pgfqpoint{3.465221in}{0.531803in}}%
\pgfpathlineto{\pgfqpoint{3.414168in}{0.531803in}}%
\pgfpathlineto{\pgfqpoint{3.414168in}{0.521603in}}%
\pgfpathclose%
\pgfusepath{fill}%
\end{pgfscope}%
\begin{pgfscope}%
\pgfpathrectangle{\pgfqpoint{0.664969in}{0.521603in}}{\pgfqpoint{3.201017in}{1.696195in}}%
\pgfusepath{clip}%
\pgfsetbuttcap%
\pgfsetmiterjoin%
\definecolor{currentfill}{rgb}{1.000000,0.498039,0.054902}%
\pgfsetfillcolor{currentfill}%
\pgfsetfillopacity{0.700000}%
\pgfsetlinewidth{0.000000pt}%
\definecolor{currentstroke}{rgb}{0.000000,0.000000,0.000000}%
\pgfsetstrokecolor{currentstroke}%
\pgfsetstrokeopacity{0.700000}%
\pgfsetdash{}{0pt}%
\pgfpathmoveto{\pgfqpoint{3.465221in}{0.521603in}}%
\pgfpathlineto{\pgfqpoint{3.516274in}{0.521603in}}%
\pgfpathlineto{\pgfqpoint{3.516274in}{0.526703in}}%
\pgfpathlineto{\pgfqpoint{3.465221in}{0.526703in}}%
\pgfpathlineto{\pgfqpoint{3.465221in}{0.521603in}}%
\pgfpathclose%
\pgfusepath{fill}%
\end{pgfscope}%
\begin{pgfscope}%
\pgfpathrectangle{\pgfqpoint{0.664969in}{0.521603in}}{\pgfqpoint{3.201017in}{1.696195in}}%
\pgfusepath{clip}%
\pgfsetbuttcap%
\pgfsetmiterjoin%
\definecolor{currentfill}{rgb}{1.000000,0.498039,0.054902}%
\pgfsetfillcolor{currentfill}%
\pgfsetfillopacity{0.700000}%
\pgfsetlinewidth{0.000000pt}%
\definecolor{currentstroke}{rgb}{0.000000,0.000000,0.000000}%
\pgfsetstrokecolor{currentstroke}%
\pgfsetstrokeopacity{0.700000}%
\pgfsetdash{}{0pt}%
\pgfpathmoveto{\pgfqpoint{3.516274in}{0.521603in}}%
\pgfpathlineto{\pgfqpoint{3.567327in}{0.521603in}}%
\pgfpathlineto{\pgfqpoint{3.567327in}{0.529253in}}%
\pgfpathlineto{\pgfqpoint{3.516274in}{0.529253in}}%
\pgfpathlineto{\pgfqpoint{3.516274in}{0.521603in}}%
\pgfpathclose%
\pgfusepath{fill}%
\end{pgfscope}%
\begin{pgfscope}%
\pgfpathrectangle{\pgfqpoint{0.664969in}{0.521603in}}{\pgfqpoint{3.201017in}{1.696195in}}%
\pgfusepath{clip}%
\pgfsetbuttcap%
\pgfsetmiterjoin%
\definecolor{currentfill}{rgb}{1.000000,0.498039,0.054902}%
\pgfsetfillcolor{currentfill}%
\pgfsetfillopacity{0.700000}%
\pgfsetlinewidth{0.000000pt}%
\definecolor{currentstroke}{rgb}{0.000000,0.000000,0.000000}%
\pgfsetstrokecolor{currentstroke}%
\pgfsetstrokeopacity{0.700000}%
\pgfsetdash{}{0pt}%
\pgfpathmoveto{\pgfqpoint{3.567327in}{0.521603in}}%
\pgfpathlineto{\pgfqpoint{3.618380in}{0.521603in}}%
\pgfpathlineto{\pgfqpoint{3.618380in}{0.526703in}}%
\pgfpathlineto{\pgfqpoint{3.567327in}{0.526703in}}%
\pgfpathlineto{\pgfqpoint{3.567327in}{0.521603in}}%
\pgfpathclose%
\pgfusepath{fill}%
\end{pgfscope}%
\begin{pgfscope}%
\pgfpathrectangle{\pgfqpoint{0.664969in}{0.521603in}}{\pgfqpoint{3.201017in}{1.696195in}}%
\pgfusepath{clip}%
\pgfsetbuttcap%
\pgfsetmiterjoin%
\definecolor{currentfill}{rgb}{1.000000,0.498039,0.054902}%
\pgfsetfillcolor{currentfill}%
\pgfsetfillopacity{0.700000}%
\pgfsetlinewidth{0.000000pt}%
\definecolor{currentstroke}{rgb}{0.000000,0.000000,0.000000}%
\pgfsetstrokecolor{currentstroke}%
\pgfsetstrokeopacity{0.700000}%
\pgfsetdash{}{0pt}%
\pgfpathmoveto{\pgfqpoint{3.618380in}{0.521603in}}%
\pgfpathlineto{\pgfqpoint{3.669433in}{0.521603in}}%
\pgfpathlineto{\pgfqpoint{3.669433in}{0.522878in}}%
\pgfpathlineto{\pgfqpoint{3.618380in}{0.522878in}}%
\pgfpathlineto{\pgfqpoint{3.618380in}{0.521603in}}%
\pgfpathclose%
\pgfusepath{fill}%
\end{pgfscope}%
\begin{pgfscope}%
\pgfpathrectangle{\pgfqpoint{0.664969in}{0.521603in}}{\pgfqpoint{3.201017in}{1.696195in}}%
\pgfusepath{clip}%
\pgfsetbuttcap%
\pgfsetmiterjoin%
\definecolor{currentfill}{rgb}{1.000000,0.498039,0.054902}%
\pgfsetfillcolor{currentfill}%
\pgfsetfillopacity{0.700000}%
\pgfsetlinewidth{0.000000pt}%
\definecolor{currentstroke}{rgb}{0.000000,0.000000,0.000000}%
\pgfsetstrokecolor{currentstroke}%
\pgfsetstrokeopacity{0.700000}%
\pgfsetdash{}{0pt}%
\pgfpathmoveto{\pgfqpoint{3.669433in}{0.521603in}}%
\pgfpathlineto{\pgfqpoint{3.720485in}{0.521603in}}%
\pgfpathlineto{\pgfqpoint{3.720485in}{0.525428in}}%
\pgfpathlineto{\pgfqpoint{3.669433in}{0.525428in}}%
\pgfpathlineto{\pgfqpoint{3.669433in}{0.521603in}}%
\pgfpathclose%
\pgfusepath{fill}%
\end{pgfscope}%
\begin{pgfscope}%
\pgfsetbuttcap%
\pgfsetroundjoin%
\definecolor{currentfill}{rgb}{0.000000,0.000000,0.000000}%
\pgfsetfillcolor{currentfill}%
\pgfsetlinewidth{0.803000pt}%
\definecolor{currentstroke}{rgb}{0.000000,0.000000,0.000000}%
\pgfsetstrokecolor{currentstroke}%
\pgfsetdash{}{0pt}%
\pgfsys@defobject{currentmarker}{\pgfqpoint{0.000000in}{-0.048611in}}{\pgfqpoint{0.000000in}{0.000000in}}{%
\pgfpathmoveto{\pgfqpoint{0.000000in}{0.000000in}}%
\pgfpathlineto{\pgfqpoint{0.000000in}{-0.048611in}}%
\pgfusepath{stroke,fill}%
}%
\begin{pgfscope}%
\pgfsys@transformshift{0.758505in}{0.521603in}%
\pgfsys@useobject{currentmarker}{}%
\end{pgfscope}%
\end{pgfscope}%
\begin{pgfscope}%
\definecolor{textcolor}{rgb}{0.000000,0.000000,0.000000}%
\pgfsetstrokecolor{textcolor}%
\pgfsetfillcolor{textcolor}%
\pgftext[x=0.758505in,y=0.424381in,,top]{\color{textcolor}{\rmfamily\fontsize{10.000000}{12.000000}\selectfont\catcode`\^=\active\def^{\ifmmode\sp\else\^{}\fi}\catcode`\%=\active\def%{\%}$\mathdefault{0}$}}%
\end{pgfscope}%
\begin{pgfscope}%
\pgfsetbuttcap%
\pgfsetroundjoin%
\definecolor{currentfill}{rgb}{0.000000,0.000000,0.000000}%
\pgfsetfillcolor{currentfill}%
\pgfsetlinewidth{0.803000pt}%
\definecolor{currentstroke}{rgb}{0.000000,0.000000,0.000000}%
\pgfsetstrokecolor{currentstroke}%
\pgfsetdash{}{0pt}%
\pgfsys@defobject{currentmarker}{\pgfqpoint{0.000000in}{-0.048611in}}{\pgfqpoint{0.000000in}{0.000000in}}{%
\pgfpathmoveto{\pgfqpoint{0.000000in}{0.000000in}}%
\pgfpathlineto{\pgfqpoint{0.000000in}{-0.048611in}}%
\pgfusepath{stroke,fill}%
}%
\begin{pgfscope}%
\pgfsys@transformshift{1.278151in}{0.521603in}%
\pgfsys@useobject{currentmarker}{}%
\end{pgfscope}%
\end{pgfscope}%
\begin{pgfscope}%
\definecolor{textcolor}{rgb}{0.000000,0.000000,0.000000}%
\pgfsetstrokecolor{textcolor}%
\pgfsetfillcolor{textcolor}%
\pgftext[x=1.278151in,y=0.424381in,,top]{\color{textcolor}{\rmfamily\fontsize{10.000000}{12.000000}\selectfont\catcode`\^=\active\def^{\ifmmode\sp\else\^{}\fi}\catcode`\%=\active\def%{\%}$\mathdefault{20}$}}%
\end{pgfscope}%
\begin{pgfscope}%
\pgfsetbuttcap%
\pgfsetroundjoin%
\definecolor{currentfill}{rgb}{0.000000,0.000000,0.000000}%
\pgfsetfillcolor{currentfill}%
\pgfsetlinewidth{0.803000pt}%
\definecolor{currentstroke}{rgb}{0.000000,0.000000,0.000000}%
\pgfsetstrokecolor{currentstroke}%
\pgfsetdash{}{0pt}%
\pgfsys@defobject{currentmarker}{\pgfqpoint{0.000000in}{-0.048611in}}{\pgfqpoint{0.000000in}{0.000000in}}{%
\pgfpathmoveto{\pgfqpoint{0.000000in}{0.000000in}}%
\pgfpathlineto{\pgfqpoint{0.000000in}{-0.048611in}}%
\pgfusepath{stroke,fill}%
}%
\begin{pgfscope}%
\pgfsys@transformshift{1.797797in}{0.521603in}%
\pgfsys@useobject{currentmarker}{}%
\end{pgfscope}%
\end{pgfscope}%
\begin{pgfscope}%
\definecolor{textcolor}{rgb}{0.000000,0.000000,0.000000}%
\pgfsetstrokecolor{textcolor}%
\pgfsetfillcolor{textcolor}%
\pgftext[x=1.797797in,y=0.424381in,,top]{\color{textcolor}{\rmfamily\fontsize{10.000000}{12.000000}\selectfont\catcode`\^=\active\def^{\ifmmode\sp\else\^{}\fi}\catcode`\%=\active\def%{\%}$\mathdefault{40}$}}%
\end{pgfscope}%
\begin{pgfscope}%
\pgfsetbuttcap%
\pgfsetroundjoin%
\definecolor{currentfill}{rgb}{0.000000,0.000000,0.000000}%
\pgfsetfillcolor{currentfill}%
\pgfsetlinewidth{0.803000pt}%
\definecolor{currentstroke}{rgb}{0.000000,0.000000,0.000000}%
\pgfsetstrokecolor{currentstroke}%
\pgfsetdash{}{0pt}%
\pgfsys@defobject{currentmarker}{\pgfqpoint{0.000000in}{-0.048611in}}{\pgfqpoint{0.000000in}{0.000000in}}{%
\pgfpathmoveto{\pgfqpoint{0.000000in}{0.000000in}}%
\pgfpathlineto{\pgfqpoint{0.000000in}{-0.048611in}}%
\pgfusepath{stroke,fill}%
}%
\begin{pgfscope}%
\pgfsys@transformshift{2.317442in}{0.521603in}%
\pgfsys@useobject{currentmarker}{}%
\end{pgfscope}%
\end{pgfscope}%
\begin{pgfscope}%
\definecolor{textcolor}{rgb}{0.000000,0.000000,0.000000}%
\pgfsetstrokecolor{textcolor}%
\pgfsetfillcolor{textcolor}%
\pgftext[x=2.317442in,y=0.424381in,,top]{\color{textcolor}{\rmfamily\fontsize{10.000000}{12.000000}\selectfont\catcode`\^=\active\def^{\ifmmode\sp\else\^{}\fi}\catcode`\%=\active\def%{\%}$\mathdefault{60}$}}%
\end{pgfscope}%
\begin{pgfscope}%
\pgfsetbuttcap%
\pgfsetroundjoin%
\definecolor{currentfill}{rgb}{0.000000,0.000000,0.000000}%
\pgfsetfillcolor{currentfill}%
\pgfsetlinewidth{0.803000pt}%
\definecolor{currentstroke}{rgb}{0.000000,0.000000,0.000000}%
\pgfsetstrokecolor{currentstroke}%
\pgfsetdash{}{0pt}%
\pgfsys@defobject{currentmarker}{\pgfqpoint{0.000000in}{-0.048611in}}{\pgfqpoint{0.000000in}{0.000000in}}{%
\pgfpathmoveto{\pgfqpoint{0.000000in}{0.000000in}}%
\pgfpathlineto{\pgfqpoint{0.000000in}{-0.048611in}}%
\pgfusepath{stroke,fill}%
}%
\begin{pgfscope}%
\pgfsys@transformshift{2.837088in}{0.521603in}%
\pgfsys@useobject{currentmarker}{}%
\end{pgfscope}%
\end{pgfscope}%
\begin{pgfscope}%
\definecolor{textcolor}{rgb}{0.000000,0.000000,0.000000}%
\pgfsetstrokecolor{textcolor}%
\pgfsetfillcolor{textcolor}%
\pgftext[x=2.837088in,y=0.424381in,,top]{\color{textcolor}{\rmfamily\fontsize{10.000000}{12.000000}\selectfont\catcode`\^=\active\def^{\ifmmode\sp\else\^{}\fi}\catcode`\%=\active\def%{\%}$\mathdefault{80}$}}%
\end{pgfscope}%
\begin{pgfscope}%
\pgfsetbuttcap%
\pgfsetroundjoin%
\definecolor{currentfill}{rgb}{0.000000,0.000000,0.000000}%
\pgfsetfillcolor{currentfill}%
\pgfsetlinewidth{0.803000pt}%
\definecolor{currentstroke}{rgb}{0.000000,0.000000,0.000000}%
\pgfsetstrokecolor{currentstroke}%
\pgfsetdash{}{0pt}%
\pgfsys@defobject{currentmarker}{\pgfqpoint{0.000000in}{-0.048611in}}{\pgfqpoint{0.000000in}{0.000000in}}{%
\pgfpathmoveto{\pgfqpoint{0.000000in}{0.000000in}}%
\pgfpathlineto{\pgfqpoint{0.000000in}{-0.048611in}}%
\pgfusepath{stroke,fill}%
}%
\begin{pgfscope}%
\pgfsys@transformshift{3.356733in}{0.521603in}%
\pgfsys@useobject{currentmarker}{}%
\end{pgfscope}%
\end{pgfscope}%
\begin{pgfscope}%
\definecolor{textcolor}{rgb}{0.000000,0.000000,0.000000}%
\pgfsetstrokecolor{textcolor}%
\pgfsetfillcolor{textcolor}%
\pgftext[x=3.356733in,y=0.424381in,,top]{\color{textcolor}{\rmfamily\fontsize{10.000000}{12.000000}\selectfont\catcode`\^=\active\def^{\ifmmode\sp\else\^{}\fi}\catcode`\%=\active\def%{\%}$\mathdefault{100}$}}%
\end{pgfscope}%
\begin{pgfscope}%
\definecolor{textcolor}{rgb}{0.000000,0.000000,0.000000}%
\pgfsetstrokecolor{textcolor}%
\pgfsetfillcolor{textcolor}%
\pgftext[x=2.265478in,y=0.234413in,,top]{\color{textcolor}{\rmfamily\fontsize{10.000000}{12.000000}\selectfont\catcode`\^=\active\def^{\ifmmode\sp\else\^{}\fi}\catcode`\%=\active\def%{\%}Number of m. classes or t. components}}%
\end{pgfscope}%
\begin{pgfscope}%
\pgfsetbuttcap%
\pgfsetroundjoin%
\definecolor{currentfill}{rgb}{0.000000,0.000000,0.000000}%
\pgfsetfillcolor{currentfill}%
\pgfsetlinewidth{0.803000pt}%
\definecolor{currentstroke}{rgb}{0.000000,0.000000,0.000000}%
\pgfsetstrokecolor{currentstroke}%
\pgfsetdash{}{0pt}%
\pgfsys@defobject{currentmarker}{\pgfqpoint{-0.048611in}{0.000000in}}{\pgfqpoint{-0.000000in}{0.000000in}}{%
\pgfpathmoveto{\pgfqpoint{-0.000000in}{0.000000in}}%
\pgfpathlineto{\pgfqpoint{-0.048611in}{0.000000in}}%
\pgfusepath{stroke,fill}%
}%
\begin{pgfscope}%
\pgfsys@transformshift{0.664969in}{0.521603in}%
\pgfsys@useobject{currentmarker}{}%
\end{pgfscope}%
\end{pgfscope}%
\begin{pgfscope}%
\definecolor{textcolor}{rgb}{0.000000,0.000000,0.000000}%
\pgfsetstrokecolor{textcolor}%
\pgfsetfillcolor{textcolor}%
\pgftext[x=0.498302in, y=0.468842in, left, base]{\color{textcolor}{\rmfamily\fontsize{10.000000}{12.000000}\selectfont\catcode`\^=\active\def^{\ifmmode\sp\else\^{}\fi}\catcode`\%=\active\def%{\%}$\mathdefault{0}$}}%
\end{pgfscope}%
\begin{pgfscope}%
\pgfsetbuttcap%
\pgfsetroundjoin%
\definecolor{currentfill}{rgb}{0.000000,0.000000,0.000000}%
\pgfsetfillcolor{currentfill}%
\pgfsetlinewidth{0.803000pt}%
\definecolor{currentstroke}{rgb}{0.000000,0.000000,0.000000}%
\pgfsetstrokecolor{currentstroke}%
\pgfsetdash{}{0pt}%
\pgfsys@defobject{currentmarker}{\pgfqpoint{-0.048611in}{0.000000in}}{\pgfqpoint{-0.000000in}{0.000000in}}{%
\pgfpathmoveto{\pgfqpoint{-0.000000in}{0.000000in}}%
\pgfpathlineto{\pgfqpoint{-0.048611in}{0.000000in}}%
\pgfusepath{stroke,fill}%
}%
\begin{pgfscope}%
\pgfsys@transformshift{0.664969in}{0.840353in}%
\pgfsys@useobject{currentmarker}{}%
\end{pgfscope}%
\end{pgfscope}%
\begin{pgfscope}%
\definecolor{textcolor}{rgb}{0.000000,0.000000,0.000000}%
\pgfsetstrokecolor{textcolor}%
\pgfsetfillcolor{textcolor}%
\pgftext[x=0.359413in, y=0.787592in, left, base]{\color{textcolor}{\rmfamily\fontsize{10.000000}{12.000000}\selectfont\catcode`\^=\active\def^{\ifmmode\sp\else\^{}\fi}\catcode`\%=\active\def%{\%}$\mathdefault{250}$}}%
\end{pgfscope}%
\begin{pgfscope}%
\pgfsetbuttcap%
\pgfsetroundjoin%
\definecolor{currentfill}{rgb}{0.000000,0.000000,0.000000}%
\pgfsetfillcolor{currentfill}%
\pgfsetlinewidth{0.803000pt}%
\definecolor{currentstroke}{rgb}{0.000000,0.000000,0.000000}%
\pgfsetstrokecolor{currentstroke}%
\pgfsetdash{}{0pt}%
\pgfsys@defobject{currentmarker}{\pgfqpoint{-0.048611in}{0.000000in}}{\pgfqpoint{-0.000000in}{0.000000in}}{%
\pgfpathmoveto{\pgfqpoint{-0.000000in}{0.000000in}}%
\pgfpathlineto{\pgfqpoint{-0.048611in}{0.000000in}}%
\pgfusepath{stroke,fill}%
}%
\begin{pgfscope}%
\pgfsys@transformshift{0.664969in}{1.159103in}%
\pgfsys@useobject{currentmarker}{}%
\end{pgfscope}%
\end{pgfscope}%
\begin{pgfscope}%
\definecolor{textcolor}{rgb}{0.000000,0.000000,0.000000}%
\pgfsetstrokecolor{textcolor}%
\pgfsetfillcolor{textcolor}%
\pgftext[x=0.359413in, y=1.106341in, left, base]{\color{textcolor}{\rmfamily\fontsize{10.000000}{12.000000}\selectfont\catcode`\^=\active\def^{\ifmmode\sp\else\^{}\fi}\catcode`\%=\active\def%{\%}$\mathdefault{500}$}}%
\end{pgfscope}%
\begin{pgfscope}%
\pgfsetbuttcap%
\pgfsetroundjoin%
\definecolor{currentfill}{rgb}{0.000000,0.000000,0.000000}%
\pgfsetfillcolor{currentfill}%
\pgfsetlinewidth{0.803000pt}%
\definecolor{currentstroke}{rgb}{0.000000,0.000000,0.000000}%
\pgfsetstrokecolor{currentstroke}%
\pgfsetdash{}{0pt}%
\pgfsys@defobject{currentmarker}{\pgfqpoint{-0.048611in}{0.000000in}}{\pgfqpoint{-0.000000in}{0.000000in}}{%
\pgfpathmoveto{\pgfqpoint{-0.000000in}{0.000000in}}%
\pgfpathlineto{\pgfqpoint{-0.048611in}{0.000000in}}%
\pgfusepath{stroke,fill}%
}%
\begin{pgfscope}%
\pgfsys@transformshift{0.664969in}{1.477853in}%
\pgfsys@useobject{currentmarker}{}%
\end{pgfscope}%
\end{pgfscope}%
\begin{pgfscope}%
\definecolor{textcolor}{rgb}{0.000000,0.000000,0.000000}%
\pgfsetstrokecolor{textcolor}%
\pgfsetfillcolor{textcolor}%
\pgftext[x=0.359413in, y=1.425091in, left, base]{\color{textcolor}{\rmfamily\fontsize{10.000000}{12.000000}\selectfont\catcode`\^=\active\def^{\ifmmode\sp\else\^{}\fi}\catcode`\%=\active\def%{\%}$\mathdefault{750}$}}%
\end{pgfscope}%
\begin{pgfscope}%
\pgfsetbuttcap%
\pgfsetroundjoin%
\definecolor{currentfill}{rgb}{0.000000,0.000000,0.000000}%
\pgfsetfillcolor{currentfill}%
\pgfsetlinewidth{0.803000pt}%
\definecolor{currentstroke}{rgb}{0.000000,0.000000,0.000000}%
\pgfsetstrokecolor{currentstroke}%
\pgfsetdash{}{0pt}%
\pgfsys@defobject{currentmarker}{\pgfqpoint{-0.048611in}{0.000000in}}{\pgfqpoint{-0.000000in}{0.000000in}}{%
\pgfpathmoveto{\pgfqpoint{-0.000000in}{0.000000in}}%
\pgfpathlineto{\pgfqpoint{-0.048611in}{0.000000in}}%
\pgfusepath{stroke,fill}%
}%
\begin{pgfscope}%
\pgfsys@transformshift{0.664969in}{1.796602in}%
\pgfsys@useobject{currentmarker}{}%
\end{pgfscope}%
\end{pgfscope}%
\begin{pgfscope}%
\definecolor{textcolor}{rgb}{0.000000,0.000000,0.000000}%
\pgfsetstrokecolor{textcolor}%
\pgfsetfillcolor{textcolor}%
\pgftext[x=0.289968in, y=1.743841in, left, base]{\color{textcolor}{\rmfamily\fontsize{10.000000}{12.000000}\selectfont\catcode`\^=\active\def^{\ifmmode\sp\else\^{}\fi}\catcode`\%=\active\def%{\%}$\mathdefault{1000}$}}%
\end{pgfscope}%
\begin{pgfscope}%
\pgfsetbuttcap%
\pgfsetroundjoin%
\definecolor{currentfill}{rgb}{0.000000,0.000000,0.000000}%
\pgfsetfillcolor{currentfill}%
\pgfsetlinewidth{0.803000pt}%
\definecolor{currentstroke}{rgb}{0.000000,0.000000,0.000000}%
\pgfsetstrokecolor{currentstroke}%
\pgfsetdash{}{0pt}%
\pgfsys@defobject{currentmarker}{\pgfqpoint{-0.048611in}{0.000000in}}{\pgfqpoint{-0.000000in}{0.000000in}}{%
\pgfpathmoveto{\pgfqpoint{-0.000000in}{0.000000in}}%
\pgfpathlineto{\pgfqpoint{-0.048611in}{0.000000in}}%
\pgfusepath{stroke,fill}%
}%
\begin{pgfscope}%
\pgfsys@transformshift{0.664969in}{2.115352in}%
\pgfsys@useobject{currentmarker}{}%
\end{pgfscope}%
\end{pgfscope}%
\begin{pgfscope}%
\definecolor{textcolor}{rgb}{0.000000,0.000000,0.000000}%
\pgfsetstrokecolor{textcolor}%
\pgfsetfillcolor{textcolor}%
\pgftext[x=0.289968in, y=2.062590in, left, base]{\color{textcolor}{\rmfamily\fontsize{10.000000}{12.000000}\selectfont\catcode`\^=\active\def^{\ifmmode\sp\else\^{}\fi}\catcode`\%=\active\def%{\%}$\mathdefault{1250}$}}%
\end{pgfscope}%
\begin{pgfscope}%
\definecolor{textcolor}{rgb}{0.000000,0.000000,0.000000}%
\pgfsetstrokecolor{textcolor}%
\pgfsetfillcolor{textcolor}%
\pgftext[x=0.234413in,y=1.369701in,,bottom,rotate=90.000000]{\color{textcolor}{\rmfamily\fontsize{10.000000}{12.000000}\selectfont\catcode`\^=\active\def^{\ifmmode\sp\else\^{}\fi}\catcode`\%=\active\def%{\%}Number of graphs}}%
\end{pgfscope}%
\begin{pgfscope}%
\pgfsetrectcap%
\pgfsetmiterjoin%
\pgfsetlinewidth{0.803000pt}%
\definecolor{currentstroke}{rgb}{0.000000,0.000000,0.000000}%
\pgfsetstrokecolor{currentstroke}%
\pgfsetdash{}{0pt}%
\pgfpathmoveto{\pgfqpoint{0.664969in}{0.521603in}}%
\pgfpathlineto{\pgfqpoint{0.664969in}{2.217798in}}%
\pgfusepath{stroke}%
\end{pgfscope}%
\begin{pgfscope}%
\pgfsetrectcap%
\pgfsetmiterjoin%
\pgfsetlinewidth{0.803000pt}%
\definecolor{currentstroke}{rgb}{0.000000,0.000000,0.000000}%
\pgfsetstrokecolor{currentstroke}%
\pgfsetdash{}{0pt}%
\pgfpathmoveto{\pgfqpoint{3.865986in}{0.521603in}}%
\pgfpathlineto{\pgfqpoint{3.865986in}{2.217798in}}%
\pgfusepath{stroke}%
\end{pgfscope}%
\begin{pgfscope}%
\pgfsetrectcap%
\pgfsetmiterjoin%
\pgfsetlinewidth{0.803000pt}%
\definecolor{currentstroke}{rgb}{0.000000,0.000000,0.000000}%
\pgfsetstrokecolor{currentstroke}%
\pgfsetdash{}{0pt}%
\pgfpathmoveto{\pgfqpoint{0.664969in}{0.521603in}}%
\pgfpathlineto{\pgfqpoint{3.865986in}{0.521603in}}%
\pgfusepath{stroke}%
\end{pgfscope}%
\begin{pgfscope}%
\pgfsetrectcap%
\pgfsetmiterjoin%
\pgfsetlinewidth{0.803000pt}%
\definecolor{currentstroke}{rgb}{0.000000,0.000000,0.000000}%
\pgfsetstrokecolor{currentstroke}%
\pgfsetdash{}{0pt}%
\pgfpathmoveto{\pgfqpoint{0.664969in}{2.217798in}}%
\pgfpathlineto{\pgfqpoint{3.865986in}{2.217798in}}%
\pgfusepath{stroke}%
\end{pgfscope}%
\end{pgfpicture}%
\makeatother%
\endgroup%
}
		\caption[Monoch. classes vs tr. con. components for globally rigid]{%
			\centering Globally rigid graphs}%
		\label{fig:monochrom_vs_triangle_globally_rigid}
	\end{subfigure}
	\hfill
	\begin{subfigure}{0.48\textwidth}
		\centering
		\scalebox{0.6}{%% Creator: Matplotlib, PGF backend
%%
%% To include the figure in your LaTeX document, write
%%   \input{<filename>.pgf}
%%
%% Make sure the required packages are loaded in your preamble
%%   \usepackage{pgf}
%%
%% Also ensure that all the required font packages are loaded; for instance,
%% the lmodern package is sometimes necessary when using math font.
%%   \usepackage{lmodern}
%%
%% Figures using additional raster images can only be included by \input if
%% they are in the same directory as the main LaTeX file. For loading figures
%% from other directories you can use the `import` package
%%   \usepackage{import}
%%
%% and then include the figures with
%%   \import{<path to file>}{<filename>.pgf}
%%
%% Matplotlib used the following preamble
%%   \def\mathdefault#1{#1}
%%   \everymath=\expandafter{\the\everymath\displaystyle}
%%   \IfFileExists{scrextend.sty}{
%%     \usepackage[fontsize=10.000000pt]{scrextend}
%%   }{
%%     \renewcommand{\normalsize}{\fontsize{10.000000}{12.000000}\selectfont}
%%     \normalsize
%%   }
%%   
%%   \ifdefined\pdftexversion\else  % non-pdftex case.
%%     \usepackage{fontspec}
%%     \setmainfont{DejaVuSans.ttf}[Path=\detokenize{/home/petr/Projects/PyRigi/.venv/lib/python3.12/site-packages/matplotlib/mpl-data/fonts/ttf/}]
%%     \setsansfont{DejaVuSans.ttf}[Path=\detokenize{/home/petr/Projects/PyRigi/.venv/lib/python3.12/site-packages/matplotlib/mpl-data/fonts/ttf/}]
%%     \setmonofont{DejaVuSansMono.ttf}[Path=\detokenize{/home/petr/Projects/PyRigi/.venv/lib/python3.12/site-packages/matplotlib/mpl-data/fonts/ttf/}]
%%   \fi
%%   \makeatletter\@ifpackageloaded{under\Score{}}{}{\usepackage[strings]{under\Score{}}}\makeatother
%%
\begingroup%
\makeatletter%
\begin{pgfpicture}%
\pgfpathrectangle{\pgfpointorigin}{\pgfqpoint{3.896542in}{2.317798in}}%
\pgfusepath{use as bounding box, clip}%
\begin{pgfscope}%
\pgfsetbuttcap%
\pgfsetmiterjoin%
\definecolor{currentfill}{rgb}{1.000000,1.000000,1.000000}%
\pgfsetfillcolor{currentfill}%
\pgfsetlinewidth{0.000000pt}%
\definecolor{currentstroke}{rgb}{1.000000,1.000000,1.000000}%
\pgfsetstrokecolor{currentstroke}%
\pgfsetdash{}{0pt}%
\pgfpathmoveto{\pgfqpoint{0.000000in}{0.000000in}}%
\pgfpathlineto{\pgfqpoint{3.896542in}{0.000000in}}%
\pgfpathlineto{\pgfqpoint{3.896542in}{2.317798in}}%
\pgfpathlineto{\pgfqpoint{0.000000in}{2.317798in}}%
\pgfpathlineto{\pgfqpoint{0.000000in}{0.000000in}}%
\pgfpathclose%
\pgfusepath{fill}%
\end{pgfscope}%
\begin{pgfscope}%
\pgfsetbuttcap%
\pgfsetmiterjoin%
\definecolor{currentfill}{rgb}{1.000000,1.000000,1.000000}%
\pgfsetfillcolor{currentfill}%
\pgfsetlinewidth{0.000000pt}%
\definecolor{currentstroke}{rgb}{0.000000,0.000000,0.000000}%
\pgfsetstrokecolor{currentstroke}%
\pgfsetstrokeopacity{0.000000}%
\pgfsetdash{}{0pt}%
\pgfpathmoveto{\pgfqpoint{0.595525in}{0.521603in}}%
\pgfpathlineto{\pgfqpoint{3.796542in}{0.521603in}}%
\pgfpathlineto{\pgfqpoint{3.796542in}{2.217798in}}%
\pgfpathlineto{\pgfqpoint{0.595525in}{2.217798in}}%
\pgfpathlineto{\pgfqpoint{0.595525in}{0.521603in}}%
\pgfpathclose%
\pgfusepath{fill}%
\end{pgfscope}%
\begin{pgfscope}%
\pgfpathrectangle{\pgfqpoint{0.595525in}{0.521603in}}{\pgfqpoint{3.201017in}{1.696195in}}%
\pgfusepath{clip}%
\pgfsetbuttcap%
\pgfsetmiterjoin%
\definecolor{currentfill}{rgb}{0.121569,0.466667,0.705882}%
\pgfsetfillcolor{currentfill}%
\pgfsetfillopacity{0.600000}%
\pgfsetlinewidth{0.000000pt}%
\definecolor{currentstroke}{rgb}{0.000000,0.000000,0.000000}%
\pgfsetstrokecolor{currentstroke}%
\pgfsetstrokeopacity{0.600000}%
\pgfsetdash{}{0pt}%
\pgfpathmoveto{\pgfqpoint{0.741025in}{0.521603in}}%
\pgfpathlineto{\pgfqpoint{0.852949in}{0.521603in}}%
\pgfpathlineto{\pgfqpoint{0.852949in}{0.973052in}}%
\pgfpathlineto{\pgfqpoint{0.741025in}{0.973052in}}%
\pgfpathlineto{\pgfqpoint{0.741025in}{0.521603in}}%
\pgfpathclose%
\pgfusepath{fill}%
\end{pgfscope}%
\begin{pgfscope}%
\pgfpathrectangle{\pgfqpoint{0.595525in}{0.521603in}}{\pgfqpoint{3.201017in}{1.696195in}}%
\pgfusepath{clip}%
\pgfsetbuttcap%
\pgfsetmiterjoin%
\definecolor{currentfill}{rgb}{0.121569,0.466667,0.705882}%
\pgfsetfillcolor{currentfill}%
\pgfsetfillopacity{0.600000}%
\pgfsetlinewidth{0.000000pt}%
\definecolor{currentstroke}{rgb}{0.000000,0.000000,0.000000}%
\pgfsetstrokecolor{currentstroke}%
\pgfsetstrokeopacity{0.600000}%
\pgfsetdash{}{0pt}%
\pgfpathmoveto{\pgfqpoint{0.852949in}{0.521603in}}%
\pgfpathlineto{\pgfqpoint{0.964873in}{0.521603in}}%
\pgfpathlineto{\pgfqpoint{0.964873in}{1.685579in}}%
\pgfpathlineto{\pgfqpoint{0.852949in}{1.685579in}}%
\pgfpathlineto{\pgfqpoint{0.852949in}{0.521603in}}%
\pgfpathclose%
\pgfusepath{fill}%
\end{pgfscope}%
\begin{pgfscope}%
\pgfpathrectangle{\pgfqpoint{0.595525in}{0.521603in}}{\pgfqpoint{3.201017in}{1.696195in}}%
\pgfusepath{clip}%
\pgfsetbuttcap%
\pgfsetmiterjoin%
\definecolor{currentfill}{rgb}{0.121569,0.466667,0.705882}%
\pgfsetfillcolor{currentfill}%
\pgfsetfillopacity{0.600000}%
\pgfsetlinewidth{0.000000pt}%
\definecolor{currentstroke}{rgb}{0.000000,0.000000,0.000000}%
\pgfsetstrokecolor{currentstroke}%
\pgfsetstrokeopacity{0.600000}%
\pgfsetdash{}{0pt}%
\pgfpathmoveto{\pgfqpoint{0.964873in}{0.521603in}}%
\pgfpathlineto{\pgfqpoint{1.076796in}{0.521603in}}%
\pgfpathlineto{\pgfqpoint{1.076796in}{1.914022in}}%
\pgfpathlineto{\pgfqpoint{0.964873in}{1.914022in}}%
\pgfpathlineto{\pgfqpoint{0.964873in}{0.521603in}}%
\pgfpathclose%
\pgfusepath{fill}%
\end{pgfscope}%
\begin{pgfscope}%
\pgfpathrectangle{\pgfqpoint{0.595525in}{0.521603in}}{\pgfqpoint{3.201017in}{1.696195in}}%
\pgfusepath{clip}%
\pgfsetbuttcap%
\pgfsetmiterjoin%
\definecolor{currentfill}{rgb}{0.121569,0.466667,0.705882}%
\pgfsetfillcolor{currentfill}%
\pgfsetfillopacity{0.600000}%
\pgfsetlinewidth{0.000000pt}%
\definecolor{currentstroke}{rgb}{0.000000,0.000000,0.000000}%
\pgfsetstrokecolor{currentstroke}%
\pgfsetstrokeopacity{0.600000}%
\pgfsetdash{}{0pt}%
\pgfpathmoveto{\pgfqpoint{1.076796in}{0.521603in}}%
\pgfpathlineto{\pgfqpoint{1.188720in}{0.521603in}}%
\pgfpathlineto{\pgfqpoint{1.188720in}{1.941218in}}%
\pgfpathlineto{\pgfqpoint{1.076796in}{1.941218in}}%
\pgfpathlineto{\pgfqpoint{1.076796in}{0.521603in}}%
\pgfpathclose%
\pgfusepath{fill}%
\end{pgfscope}%
\begin{pgfscope}%
\pgfpathrectangle{\pgfqpoint{0.595525in}{0.521603in}}{\pgfqpoint{3.201017in}{1.696195in}}%
\pgfusepath{clip}%
\pgfsetbuttcap%
\pgfsetmiterjoin%
\definecolor{currentfill}{rgb}{0.121569,0.466667,0.705882}%
\pgfsetfillcolor{currentfill}%
\pgfsetfillopacity{0.600000}%
\pgfsetlinewidth{0.000000pt}%
\definecolor{currentstroke}{rgb}{0.000000,0.000000,0.000000}%
\pgfsetstrokecolor{currentstroke}%
\pgfsetstrokeopacity{0.600000}%
\pgfsetdash{}{0pt}%
\pgfpathmoveto{\pgfqpoint{1.188720in}{0.521603in}}%
\pgfpathlineto{\pgfqpoint{1.300644in}{0.521603in}}%
\pgfpathlineto{\pgfqpoint{1.300644in}{2.093514in}}%
\pgfpathlineto{\pgfqpoint{1.188720in}{2.093514in}}%
\pgfpathlineto{\pgfqpoint{1.188720in}{0.521603in}}%
\pgfpathclose%
\pgfusepath{fill}%
\end{pgfscope}%
\begin{pgfscope}%
\pgfpathrectangle{\pgfqpoint{0.595525in}{0.521603in}}{\pgfqpoint{3.201017in}{1.696195in}}%
\pgfusepath{clip}%
\pgfsetbuttcap%
\pgfsetmiterjoin%
\definecolor{currentfill}{rgb}{0.121569,0.466667,0.705882}%
\pgfsetfillcolor{currentfill}%
\pgfsetfillopacity{0.600000}%
\pgfsetlinewidth{0.000000pt}%
\definecolor{currentstroke}{rgb}{0.000000,0.000000,0.000000}%
\pgfsetstrokecolor{currentstroke}%
\pgfsetstrokeopacity{0.600000}%
\pgfsetdash{}{0pt}%
\pgfpathmoveto{\pgfqpoint{1.300644in}{0.521603in}}%
\pgfpathlineto{\pgfqpoint{1.412567in}{0.521603in}}%
\pgfpathlineto{\pgfqpoint{1.412567in}{1.593113in}}%
\pgfpathlineto{\pgfqpoint{1.300644in}{1.593113in}}%
\pgfpathlineto{\pgfqpoint{1.300644in}{0.521603in}}%
\pgfpathclose%
\pgfusepath{fill}%
\end{pgfscope}%
\begin{pgfscope}%
\pgfpathrectangle{\pgfqpoint{0.595525in}{0.521603in}}{\pgfqpoint{3.201017in}{1.696195in}}%
\pgfusepath{clip}%
\pgfsetbuttcap%
\pgfsetmiterjoin%
\definecolor{currentfill}{rgb}{0.121569,0.466667,0.705882}%
\pgfsetfillcolor{currentfill}%
\pgfsetfillopacity{0.600000}%
\pgfsetlinewidth{0.000000pt}%
\definecolor{currentstroke}{rgb}{0.000000,0.000000,0.000000}%
\pgfsetstrokecolor{currentstroke}%
\pgfsetstrokeopacity{0.600000}%
\pgfsetdash{}{0pt}%
\pgfpathmoveto{\pgfqpoint{1.412567in}{0.521603in}}%
\pgfpathlineto{\pgfqpoint{1.524491in}{0.521603in}}%
\pgfpathlineto{\pgfqpoint{1.524491in}{1.326596in}}%
\pgfpathlineto{\pgfqpoint{1.412567in}{1.326596in}}%
\pgfpathlineto{\pgfqpoint{1.412567in}{0.521603in}}%
\pgfpathclose%
\pgfusepath{fill}%
\end{pgfscope}%
\begin{pgfscope}%
\pgfpathrectangle{\pgfqpoint{0.595525in}{0.521603in}}{\pgfqpoint{3.201017in}{1.696195in}}%
\pgfusepath{clip}%
\pgfsetbuttcap%
\pgfsetmiterjoin%
\definecolor{currentfill}{rgb}{0.121569,0.466667,0.705882}%
\pgfsetfillcolor{currentfill}%
\pgfsetfillopacity{0.600000}%
\pgfsetlinewidth{0.000000pt}%
\definecolor{currentstroke}{rgb}{0.000000,0.000000,0.000000}%
\pgfsetstrokecolor{currentstroke}%
\pgfsetstrokeopacity{0.600000}%
\pgfsetdash{}{0pt}%
\pgfpathmoveto{\pgfqpoint{1.524491in}{0.521603in}}%
\pgfpathlineto{\pgfqpoint{1.636415in}{0.521603in}}%
\pgfpathlineto{\pgfqpoint{1.636415in}{1.315717in}}%
\pgfpathlineto{\pgfqpoint{1.524491in}{1.315717in}}%
\pgfpathlineto{\pgfqpoint{1.524491in}{0.521603in}}%
\pgfpathclose%
\pgfusepath{fill}%
\end{pgfscope}%
\begin{pgfscope}%
\pgfpathrectangle{\pgfqpoint{0.595525in}{0.521603in}}{\pgfqpoint{3.201017in}{1.696195in}}%
\pgfusepath{clip}%
\pgfsetbuttcap%
\pgfsetmiterjoin%
\definecolor{currentfill}{rgb}{0.121569,0.466667,0.705882}%
\pgfsetfillcolor{currentfill}%
\pgfsetfillopacity{0.600000}%
\pgfsetlinewidth{0.000000pt}%
\definecolor{currentstroke}{rgb}{0.000000,0.000000,0.000000}%
\pgfsetstrokecolor{currentstroke}%
\pgfsetstrokeopacity{0.600000}%
\pgfsetdash{}{0pt}%
\pgfpathmoveto{\pgfqpoint{1.636415in}{0.521603in}}%
\pgfpathlineto{\pgfqpoint{1.748338in}{0.521603in}}%
\pgfpathlineto{\pgfqpoint{1.748338in}{1.288522in}}%
\pgfpathlineto{\pgfqpoint{1.636415in}{1.288522in}}%
\pgfpathlineto{\pgfqpoint{1.636415in}{0.521603in}}%
\pgfpathclose%
\pgfusepath{fill}%
\end{pgfscope}%
\begin{pgfscope}%
\pgfpathrectangle{\pgfqpoint{0.595525in}{0.521603in}}{\pgfqpoint{3.201017in}{1.696195in}}%
\pgfusepath{clip}%
\pgfsetbuttcap%
\pgfsetmiterjoin%
\definecolor{currentfill}{rgb}{0.121569,0.466667,0.705882}%
\pgfsetfillcolor{currentfill}%
\pgfsetfillopacity{0.600000}%
\pgfsetlinewidth{0.000000pt}%
\definecolor{currentstroke}{rgb}{0.000000,0.000000,0.000000}%
\pgfsetstrokecolor{currentstroke}%
\pgfsetstrokeopacity{0.600000}%
\pgfsetdash{}{0pt}%
\pgfpathmoveto{\pgfqpoint{1.748338in}{0.521603in}}%
\pgfpathlineto{\pgfqpoint{1.860262in}{0.521603in}}%
\pgfpathlineto{\pgfqpoint{1.860262in}{1.130787in}}%
\pgfpathlineto{\pgfqpoint{1.748338in}{1.130787in}}%
\pgfpathlineto{\pgfqpoint{1.748338in}{0.521603in}}%
\pgfpathclose%
\pgfusepath{fill}%
\end{pgfscope}%
\begin{pgfscope}%
\pgfpathrectangle{\pgfqpoint{0.595525in}{0.521603in}}{\pgfqpoint{3.201017in}{1.696195in}}%
\pgfusepath{clip}%
\pgfsetbuttcap%
\pgfsetmiterjoin%
\definecolor{currentfill}{rgb}{0.121569,0.466667,0.705882}%
\pgfsetfillcolor{currentfill}%
\pgfsetfillopacity{0.600000}%
\pgfsetlinewidth{0.000000pt}%
\definecolor{currentstroke}{rgb}{0.000000,0.000000,0.000000}%
\pgfsetstrokecolor{currentstroke}%
\pgfsetstrokeopacity{0.600000}%
\pgfsetdash{}{0pt}%
\pgfpathmoveto{\pgfqpoint{1.860262in}{0.521603in}}%
\pgfpathlineto{\pgfqpoint{1.972186in}{0.521603in}}%
\pgfpathlineto{\pgfqpoint{1.972186in}{0.994808in}}%
\pgfpathlineto{\pgfqpoint{1.860262in}{0.994808in}}%
\pgfpathlineto{\pgfqpoint{1.860262in}{0.521603in}}%
\pgfpathclose%
\pgfusepath{fill}%
\end{pgfscope}%
\begin{pgfscope}%
\pgfpathrectangle{\pgfqpoint{0.595525in}{0.521603in}}{\pgfqpoint{3.201017in}{1.696195in}}%
\pgfusepath{clip}%
\pgfsetbuttcap%
\pgfsetmiterjoin%
\definecolor{currentfill}{rgb}{0.121569,0.466667,0.705882}%
\pgfsetfillcolor{currentfill}%
\pgfsetfillopacity{0.600000}%
\pgfsetlinewidth{0.000000pt}%
\definecolor{currentstroke}{rgb}{0.000000,0.000000,0.000000}%
\pgfsetstrokecolor{currentstroke}%
\pgfsetstrokeopacity{0.600000}%
\pgfsetdash{}{0pt}%
\pgfpathmoveto{\pgfqpoint{1.972186in}{0.521603in}}%
\pgfpathlineto{\pgfqpoint{2.084109in}{0.521603in}}%
\pgfpathlineto{\pgfqpoint{2.084109in}{1.011126in}}%
\pgfpathlineto{\pgfqpoint{1.972186in}{1.011126in}}%
\pgfpathlineto{\pgfqpoint{1.972186in}{0.521603in}}%
\pgfpathclose%
\pgfusepath{fill}%
\end{pgfscope}%
\begin{pgfscope}%
\pgfpathrectangle{\pgfqpoint{0.595525in}{0.521603in}}{\pgfqpoint{3.201017in}{1.696195in}}%
\pgfusepath{clip}%
\pgfsetbuttcap%
\pgfsetmiterjoin%
\definecolor{currentfill}{rgb}{0.121569,0.466667,0.705882}%
\pgfsetfillcolor{currentfill}%
\pgfsetfillopacity{0.600000}%
\pgfsetlinewidth{0.000000pt}%
\definecolor{currentstroke}{rgb}{0.000000,0.000000,0.000000}%
\pgfsetstrokecolor{currentstroke}%
\pgfsetstrokeopacity{0.600000}%
\pgfsetdash{}{0pt}%
\pgfpathmoveto{\pgfqpoint{2.084109in}{0.521603in}}%
\pgfpathlineto{\pgfqpoint{2.196033in}{0.521603in}}%
\pgfpathlineto{\pgfqpoint{2.196033in}{0.880586in}}%
\pgfpathlineto{\pgfqpoint{2.084109in}{0.880586in}}%
\pgfpathlineto{\pgfqpoint{2.084109in}{0.521603in}}%
\pgfpathclose%
\pgfusepath{fill}%
\end{pgfscope}%
\begin{pgfscope}%
\pgfpathrectangle{\pgfqpoint{0.595525in}{0.521603in}}{\pgfqpoint{3.201017in}{1.696195in}}%
\pgfusepath{clip}%
\pgfsetbuttcap%
\pgfsetmiterjoin%
\definecolor{currentfill}{rgb}{0.121569,0.466667,0.705882}%
\pgfsetfillcolor{currentfill}%
\pgfsetfillopacity{0.600000}%
\pgfsetlinewidth{0.000000pt}%
\definecolor{currentstroke}{rgb}{0.000000,0.000000,0.000000}%
\pgfsetstrokecolor{currentstroke}%
\pgfsetstrokeopacity{0.600000}%
\pgfsetdash{}{0pt}%
\pgfpathmoveto{\pgfqpoint{2.196033in}{0.521603in}}%
\pgfpathlineto{\pgfqpoint{2.307957in}{0.521603in}}%
\pgfpathlineto{\pgfqpoint{2.307957in}{0.864269in}}%
\pgfpathlineto{\pgfqpoint{2.196033in}{0.864269in}}%
\pgfpathlineto{\pgfqpoint{2.196033in}{0.521603in}}%
\pgfpathclose%
\pgfusepath{fill}%
\end{pgfscope}%
\begin{pgfscope}%
\pgfpathrectangle{\pgfqpoint{0.595525in}{0.521603in}}{\pgfqpoint{3.201017in}{1.696195in}}%
\pgfusepath{clip}%
\pgfsetbuttcap%
\pgfsetmiterjoin%
\definecolor{currentfill}{rgb}{0.121569,0.466667,0.705882}%
\pgfsetfillcolor{currentfill}%
\pgfsetfillopacity{0.600000}%
\pgfsetlinewidth{0.000000pt}%
\definecolor{currentstroke}{rgb}{0.000000,0.000000,0.000000}%
\pgfsetstrokecolor{currentstroke}%
\pgfsetstrokeopacity{0.600000}%
\pgfsetdash{}{0pt}%
\pgfpathmoveto{\pgfqpoint{2.307957in}{0.521603in}}%
\pgfpathlineto{\pgfqpoint{2.419880in}{0.521603in}}%
\pgfpathlineto{\pgfqpoint{2.419880in}{0.842512in}}%
\pgfpathlineto{\pgfqpoint{2.307957in}{0.842512in}}%
\pgfpathlineto{\pgfqpoint{2.307957in}{0.521603in}}%
\pgfpathclose%
\pgfusepath{fill}%
\end{pgfscope}%
\begin{pgfscope}%
\pgfpathrectangle{\pgfqpoint{0.595525in}{0.521603in}}{\pgfqpoint{3.201017in}{1.696195in}}%
\pgfusepath{clip}%
\pgfsetbuttcap%
\pgfsetmiterjoin%
\definecolor{currentfill}{rgb}{0.121569,0.466667,0.705882}%
\pgfsetfillcolor{currentfill}%
\pgfsetfillopacity{0.600000}%
\pgfsetlinewidth{0.000000pt}%
\definecolor{currentstroke}{rgb}{0.000000,0.000000,0.000000}%
\pgfsetstrokecolor{currentstroke}%
\pgfsetstrokeopacity{0.600000}%
\pgfsetdash{}{0pt}%
\pgfpathmoveto{\pgfqpoint{2.419880in}{0.521603in}}%
\pgfpathlineto{\pgfqpoint{2.531804in}{0.521603in}}%
\pgfpathlineto{\pgfqpoint{2.531804in}{0.739169in}}%
\pgfpathlineto{\pgfqpoint{2.419880in}{0.739169in}}%
\pgfpathlineto{\pgfqpoint{2.419880in}{0.521603in}}%
\pgfpathclose%
\pgfusepath{fill}%
\end{pgfscope}%
\begin{pgfscope}%
\pgfpathrectangle{\pgfqpoint{0.595525in}{0.521603in}}{\pgfqpoint{3.201017in}{1.696195in}}%
\pgfusepath{clip}%
\pgfsetbuttcap%
\pgfsetmiterjoin%
\definecolor{currentfill}{rgb}{0.121569,0.466667,0.705882}%
\pgfsetfillcolor{currentfill}%
\pgfsetfillopacity{0.600000}%
\pgfsetlinewidth{0.000000pt}%
\definecolor{currentstroke}{rgb}{0.000000,0.000000,0.000000}%
\pgfsetstrokecolor{currentstroke}%
\pgfsetstrokeopacity{0.600000}%
\pgfsetdash{}{0pt}%
\pgfpathmoveto{\pgfqpoint{2.531804in}{0.521603in}}%
\pgfpathlineto{\pgfqpoint{2.643728in}{0.521603in}}%
\pgfpathlineto{\pgfqpoint{2.643728in}{0.690217in}}%
\pgfpathlineto{\pgfqpoint{2.531804in}{0.690217in}}%
\pgfpathlineto{\pgfqpoint{2.531804in}{0.521603in}}%
\pgfpathclose%
\pgfusepath{fill}%
\end{pgfscope}%
\begin{pgfscope}%
\pgfpathrectangle{\pgfqpoint{0.595525in}{0.521603in}}{\pgfqpoint{3.201017in}{1.696195in}}%
\pgfusepath{clip}%
\pgfsetbuttcap%
\pgfsetmiterjoin%
\definecolor{currentfill}{rgb}{0.121569,0.466667,0.705882}%
\pgfsetfillcolor{currentfill}%
\pgfsetfillopacity{0.600000}%
\pgfsetlinewidth{0.000000pt}%
\definecolor{currentstroke}{rgb}{0.000000,0.000000,0.000000}%
\pgfsetstrokecolor{currentstroke}%
\pgfsetstrokeopacity{0.600000}%
\pgfsetdash{}{0pt}%
\pgfpathmoveto{\pgfqpoint{2.643728in}{0.521603in}}%
\pgfpathlineto{\pgfqpoint{2.755651in}{0.521603in}}%
\pgfpathlineto{\pgfqpoint{2.755651in}{0.717412in}}%
\pgfpathlineto{\pgfqpoint{2.643728in}{0.717412in}}%
\pgfpathlineto{\pgfqpoint{2.643728in}{0.521603in}}%
\pgfpathclose%
\pgfusepath{fill}%
\end{pgfscope}%
\begin{pgfscope}%
\pgfpathrectangle{\pgfqpoint{0.595525in}{0.521603in}}{\pgfqpoint{3.201017in}{1.696195in}}%
\pgfusepath{clip}%
\pgfsetbuttcap%
\pgfsetmiterjoin%
\definecolor{currentfill}{rgb}{0.121569,0.466667,0.705882}%
\pgfsetfillcolor{currentfill}%
\pgfsetfillopacity{0.600000}%
\pgfsetlinewidth{0.000000pt}%
\definecolor{currentstroke}{rgb}{0.000000,0.000000,0.000000}%
\pgfsetstrokecolor{currentstroke}%
\pgfsetstrokeopacity{0.600000}%
\pgfsetdash{}{0pt}%
\pgfpathmoveto{\pgfqpoint{2.755651in}{0.521603in}}%
\pgfpathlineto{\pgfqpoint{2.867575in}{0.521603in}}%
\pgfpathlineto{\pgfqpoint{2.867575in}{0.701095in}}%
\pgfpathlineto{\pgfqpoint{2.755651in}{0.701095in}}%
\pgfpathlineto{\pgfqpoint{2.755651in}{0.521603in}}%
\pgfpathclose%
\pgfusepath{fill}%
\end{pgfscope}%
\begin{pgfscope}%
\pgfpathrectangle{\pgfqpoint{0.595525in}{0.521603in}}{\pgfqpoint{3.201017in}{1.696195in}}%
\pgfusepath{clip}%
\pgfsetbuttcap%
\pgfsetmiterjoin%
\definecolor{currentfill}{rgb}{0.121569,0.466667,0.705882}%
\pgfsetfillcolor{currentfill}%
\pgfsetfillopacity{0.600000}%
\pgfsetlinewidth{0.000000pt}%
\definecolor{currentstroke}{rgb}{0.000000,0.000000,0.000000}%
\pgfsetstrokecolor{currentstroke}%
\pgfsetstrokeopacity{0.600000}%
\pgfsetdash{}{0pt}%
\pgfpathmoveto{\pgfqpoint{2.867575in}{0.521603in}}%
\pgfpathlineto{\pgfqpoint{2.979499in}{0.521603in}}%
\pgfpathlineto{\pgfqpoint{2.979499in}{0.673899in}}%
\pgfpathlineto{\pgfqpoint{2.867575in}{0.673899in}}%
\pgfpathlineto{\pgfqpoint{2.867575in}{0.521603in}}%
\pgfpathclose%
\pgfusepath{fill}%
\end{pgfscope}%
\begin{pgfscope}%
\pgfpathrectangle{\pgfqpoint{0.595525in}{0.521603in}}{\pgfqpoint{3.201017in}{1.696195in}}%
\pgfusepath{clip}%
\pgfsetbuttcap%
\pgfsetmiterjoin%
\definecolor{currentfill}{rgb}{0.121569,0.466667,0.705882}%
\pgfsetfillcolor{currentfill}%
\pgfsetfillopacity{0.600000}%
\pgfsetlinewidth{0.000000pt}%
\definecolor{currentstroke}{rgb}{0.000000,0.000000,0.000000}%
\pgfsetstrokecolor{currentstroke}%
\pgfsetstrokeopacity{0.600000}%
\pgfsetdash{}{0pt}%
\pgfpathmoveto{\pgfqpoint{2.979499in}{0.521603in}}%
\pgfpathlineto{\pgfqpoint{3.091422in}{0.521603in}}%
\pgfpathlineto{\pgfqpoint{3.091422in}{0.690217in}}%
\pgfpathlineto{\pgfqpoint{2.979499in}{0.690217in}}%
\pgfpathlineto{\pgfqpoint{2.979499in}{0.521603in}}%
\pgfpathclose%
\pgfusepath{fill}%
\end{pgfscope}%
\begin{pgfscope}%
\pgfpathrectangle{\pgfqpoint{0.595525in}{0.521603in}}{\pgfqpoint{3.201017in}{1.696195in}}%
\pgfusepath{clip}%
\pgfsetbuttcap%
\pgfsetmiterjoin%
\definecolor{currentfill}{rgb}{0.121569,0.466667,0.705882}%
\pgfsetfillcolor{currentfill}%
\pgfsetfillopacity{0.600000}%
\pgfsetlinewidth{0.000000pt}%
\definecolor{currentstroke}{rgb}{0.000000,0.000000,0.000000}%
\pgfsetstrokecolor{currentstroke}%
\pgfsetstrokeopacity{0.600000}%
\pgfsetdash{}{0pt}%
\pgfpathmoveto{\pgfqpoint{3.091422in}{0.521603in}}%
\pgfpathlineto{\pgfqpoint{3.203346in}{0.521603in}}%
\pgfpathlineto{\pgfqpoint{3.203346in}{0.586873in}}%
\pgfpathlineto{\pgfqpoint{3.091422in}{0.586873in}}%
\pgfpathlineto{\pgfqpoint{3.091422in}{0.521603in}}%
\pgfpathclose%
\pgfusepath{fill}%
\end{pgfscope}%
\begin{pgfscope}%
\pgfpathrectangle{\pgfqpoint{0.595525in}{0.521603in}}{\pgfqpoint{3.201017in}{1.696195in}}%
\pgfusepath{clip}%
\pgfsetbuttcap%
\pgfsetmiterjoin%
\definecolor{currentfill}{rgb}{0.121569,0.466667,0.705882}%
\pgfsetfillcolor{currentfill}%
\pgfsetfillopacity{0.600000}%
\pgfsetlinewidth{0.000000pt}%
\definecolor{currentstroke}{rgb}{0.000000,0.000000,0.000000}%
\pgfsetstrokecolor{currentstroke}%
\pgfsetstrokeopacity{0.600000}%
\pgfsetdash{}{0pt}%
\pgfpathmoveto{\pgfqpoint{3.203346in}{0.521603in}}%
\pgfpathlineto{\pgfqpoint{3.315270in}{0.521603in}}%
\pgfpathlineto{\pgfqpoint{3.315270in}{0.614069in}}%
\pgfpathlineto{\pgfqpoint{3.203346in}{0.614069in}}%
\pgfpathlineto{\pgfqpoint{3.203346in}{0.521603in}}%
\pgfpathclose%
\pgfusepath{fill}%
\end{pgfscope}%
\begin{pgfscope}%
\pgfpathrectangle{\pgfqpoint{0.595525in}{0.521603in}}{\pgfqpoint{3.201017in}{1.696195in}}%
\pgfusepath{clip}%
\pgfsetbuttcap%
\pgfsetmiterjoin%
\definecolor{currentfill}{rgb}{0.121569,0.466667,0.705882}%
\pgfsetfillcolor{currentfill}%
\pgfsetfillopacity{0.600000}%
\pgfsetlinewidth{0.000000pt}%
\definecolor{currentstroke}{rgb}{0.000000,0.000000,0.000000}%
\pgfsetstrokecolor{currentstroke}%
\pgfsetstrokeopacity{0.600000}%
\pgfsetdash{}{0pt}%
\pgfpathmoveto{\pgfqpoint{3.315270in}{0.521603in}}%
\pgfpathlineto{\pgfqpoint{3.427193in}{0.521603in}}%
\pgfpathlineto{\pgfqpoint{3.427193in}{0.614069in}}%
\pgfpathlineto{\pgfqpoint{3.315270in}{0.614069in}}%
\pgfpathlineto{\pgfqpoint{3.315270in}{0.521603in}}%
\pgfpathclose%
\pgfusepath{fill}%
\end{pgfscope}%
\begin{pgfscope}%
\pgfpathrectangle{\pgfqpoint{0.595525in}{0.521603in}}{\pgfqpoint{3.201017in}{1.696195in}}%
\pgfusepath{clip}%
\pgfsetbuttcap%
\pgfsetmiterjoin%
\definecolor{currentfill}{rgb}{0.121569,0.466667,0.705882}%
\pgfsetfillcolor{currentfill}%
\pgfsetfillopacity{0.600000}%
\pgfsetlinewidth{0.000000pt}%
\definecolor{currentstroke}{rgb}{0.000000,0.000000,0.000000}%
\pgfsetstrokecolor{currentstroke}%
\pgfsetstrokeopacity{0.600000}%
\pgfsetdash{}{0pt}%
\pgfpathmoveto{\pgfqpoint{3.427193in}{0.521603in}}%
\pgfpathlineto{\pgfqpoint{3.539117in}{0.521603in}}%
\pgfpathlineto{\pgfqpoint{3.539117in}{0.586873in}}%
\pgfpathlineto{\pgfqpoint{3.427193in}{0.586873in}}%
\pgfpathlineto{\pgfqpoint{3.427193in}{0.521603in}}%
\pgfpathclose%
\pgfusepath{fill}%
\end{pgfscope}%
\begin{pgfscope}%
\pgfpathrectangle{\pgfqpoint{0.595525in}{0.521603in}}{\pgfqpoint{3.201017in}{1.696195in}}%
\pgfusepath{clip}%
\pgfsetbuttcap%
\pgfsetmiterjoin%
\definecolor{currentfill}{rgb}{0.121569,0.466667,0.705882}%
\pgfsetfillcolor{currentfill}%
\pgfsetfillopacity{0.600000}%
\pgfsetlinewidth{0.000000pt}%
\definecolor{currentstroke}{rgb}{0.000000,0.000000,0.000000}%
\pgfsetstrokecolor{currentstroke}%
\pgfsetstrokeopacity{0.600000}%
\pgfsetdash{}{0pt}%
\pgfpathmoveto{\pgfqpoint{3.539117in}{0.521603in}}%
\pgfpathlineto{\pgfqpoint{3.651041in}{0.521603in}}%
\pgfpathlineto{\pgfqpoint{3.651041in}{0.543360in}}%
\pgfpathlineto{\pgfqpoint{3.539117in}{0.543360in}}%
\pgfpathlineto{\pgfqpoint{3.539117in}{0.521603in}}%
\pgfpathclose%
\pgfusepath{fill}%
\end{pgfscope}%
\begin{pgfscope}%
\pgfpathrectangle{\pgfqpoint{0.595525in}{0.521603in}}{\pgfqpoint{3.201017in}{1.696195in}}%
\pgfusepath{clip}%
\pgfsetbuttcap%
\pgfsetmiterjoin%
\definecolor{currentfill}{rgb}{1.000000,0.498039,0.054902}%
\pgfsetfillcolor{currentfill}%
\pgfsetfillopacity{0.600000}%
\pgfsetlinewidth{0.000000pt}%
\definecolor{currentstroke}{rgb}{0.000000,0.000000,0.000000}%
\pgfsetstrokecolor{currentstroke}%
\pgfsetstrokeopacity{0.600000}%
\pgfsetdash{}{0pt}%
\pgfpathmoveto{\pgfqpoint{0.769278in}{0.521603in}}%
\pgfpathlineto{\pgfqpoint{0.880115in}{0.521603in}}%
\pgfpathlineto{\pgfqpoint{0.880115in}{0.755486in}}%
\pgfpathlineto{\pgfqpoint{0.769278in}{0.755486in}}%
\pgfpathlineto{\pgfqpoint{0.769278in}{0.521603in}}%
\pgfpathclose%
\pgfusepath{fill}%
\end{pgfscope}%
\begin{pgfscope}%
\pgfpathrectangle{\pgfqpoint{0.595525in}{0.521603in}}{\pgfqpoint{3.201017in}{1.696195in}}%
\pgfusepath{clip}%
\pgfsetbuttcap%
\pgfsetmiterjoin%
\definecolor{currentfill}{rgb}{1.000000,0.498039,0.054902}%
\pgfsetfillcolor{currentfill}%
\pgfsetfillopacity{0.600000}%
\pgfsetlinewidth{0.000000pt}%
\definecolor{currentstroke}{rgb}{0.000000,0.000000,0.000000}%
\pgfsetstrokecolor{currentstroke}%
\pgfsetstrokeopacity{0.600000}%
\pgfsetdash{}{0pt}%
\pgfpathmoveto{\pgfqpoint{0.880115in}{0.521603in}}%
\pgfpathlineto{\pgfqpoint{0.990952in}{0.521603in}}%
\pgfpathlineto{\pgfqpoint{0.990952in}{1.593113in}}%
\pgfpathlineto{\pgfqpoint{0.880115in}{1.593113in}}%
\pgfpathlineto{\pgfqpoint{0.880115in}{0.521603in}}%
\pgfpathclose%
\pgfusepath{fill}%
\end{pgfscope}%
\begin{pgfscope}%
\pgfpathrectangle{\pgfqpoint{0.595525in}{0.521603in}}{\pgfqpoint{3.201017in}{1.696195in}}%
\pgfusepath{clip}%
\pgfsetbuttcap%
\pgfsetmiterjoin%
\definecolor{currentfill}{rgb}{1.000000,0.498039,0.054902}%
\pgfsetfillcolor{currentfill}%
\pgfsetfillopacity{0.600000}%
\pgfsetlinewidth{0.000000pt}%
\definecolor{currentstroke}{rgb}{0.000000,0.000000,0.000000}%
\pgfsetstrokecolor{currentstroke}%
\pgfsetstrokeopacity{0.600000}%
\pgfsetdash{}{0pt}%
\pgfpathmoveto{\pgfqpoint{0.990952in}{0.521603in}}%
\pgfpathlineto{\pgfqpoint{1.101789in}{0.521603in}}%
\pgfpathlineto{\pgfqpoint{1.101789in}{2.006488in}}%
\pgfpathlineto{\pgfqpoint{0.990952in}{2.006488in}}%
\pgfpathlineto{\pgfqpoint{0.990952in}{0.521603in}}%
\pgfpathclose%
\pgfusepath{fill}%
\end{pgfscope}%
\begin{pgfscope}%
\pgfpathrectangle{\pgfqpoint{0.595525in}{0.521603in}}{\pgfqpoint{3.201017in}{1.696195in}}%
\pgfusepath{clip}%
\pgfsetbuttcap%
\pgfsetmiterjoin%
\definecolor{currentfill}{rgb}{1.000000,0.498039,0.054902}%
\pgfsetfillcolor{currentfill}%
\pgfsetfillopacity{0.600000}%
\pgfsetlinewidth{0.000000pt}%
\definecolor{currentstroke}{rgb}{0.000000,0.000000,0.000000}%
\pgfsetstrokecolor{currentstroke}%
\pgfsetstrokeopacity{0.600000}%
\pgfsetdash{}{0pt}%
\pgfpathmoveto{\pgfqpoint{1.101789in}{0.521603in}}%
\pgfpathlineto{\pgfqpoint{1.212626in}{0.521603in}}%
\pgfpathlineto{\pgfqpoint{1.212626in}{1.952096in}}%
\pgfpathlineto{\pgfqpoint{1.101789in}{1.952096in}}%
\pgfpathlineto{\pgfqpoint{1.101789in}{0.521603in}}%
\pgfpathclose%
\pgfusepath{fill}%
\end{pgfscope}%
\begin{pgfscope}%
\pgfpathrectangle{\pgfqpoint{0.595525in}{0.521603in}}{\pgfqpoint{3.201017in}{1.696195in}}%
\pgfusepath{clip}%
\pgfsetbuttcap%
\pgfsetmiterjoin%
\definecolor{currentfill}{rgb}{1.000000,0.498039,0.054902}%
\pgfsetfillcolor{currentfill}%
\pgfsetfillopacity{0.600000}%
\pgfsetlinewidth{0.000000pt}%
\definecolor{currentstroke}{rgb}{0.000000,0.000000,0.000000}%
\pgfsetstrokecolor{currentstroke}%
\pgfsetstrokeopacity{0.600000}%
\pgfsetdash{}{0pt}%
\pgfpathmoveto{\pgfqpoint{1.212626in}{0.521603in}}%
\pgfpathlineto{\pgfqpoint{1.323463in}{0.521603in}}%
\pgfpathlineto{\pgfqpoint{1.323463in}{2.137027in}}%
\pgfpathlineto{\pgfqpoint{1.212626in}{2.137027in}}%
\pgfpathlineto{\pgfqpoint{1.212626in}{0.521603in}}%
\pgfpathclose%
\pgfusepath{fill}%
\end{pgfscope}%
\begin{pgfscope}%
\pgfpathrectangle{\pgfqpoint{0.595525in}{0.521603in}}{\pgfqpoint{3.201017in}{1.696195in}}%
\pgfusepath{clip}%
\pgfsetbuttcap%
\pgfsetmiterjoin%
\definecolor{currentfill}{rgb}{1.000000,0.498039,0.054902}%
\pgfsetfillcolor{currentfill}%
\pgfsetfillopacity{0.600000}%
\pgfsetlinewidth{0.000000pt}%
\definecolor{currentstroke}{rgb}{0.000000,0.000000,0.000000}%
\pgfsetstrokecolor{currentstroke}%
\pgfsetstrokeopacity{0.600000}%
\pgfsetdash{}{0pt}%
\pgfpathmoveto{\pgfqpoint{1.323463in}{0.521603in}}%
\pgfpathlineto{\pgfqpoint{1.434300in}{0.521603in}}%
\pgfpathlineto{\pgfqpoint{1.434300in}{1.652944in}}%
\pgfpathlineto{\pgfqpoint{1.323463in}{1.652944in}}%
\pgfpathlineto{\pgfqpoint{1.323463in}{0.521603in}}%
\pgfpathclose%
\pgfusepath{fill}%
\end{pgfscope}%
\begin{pgfscope}%
\pgfpathrectangle{\pgfqpoint{0.595525in}{0.521603in}}{\pgfqpoint{3.201017in}{1.696195in}}%
\pgfusepath{clip}%
\pgfsetbuttcap%
\pgfsetmiterjoin%
\definecolor{currentfill}{rgb}{1.000000,0.498039,0.054902}%
\pgfsetfillcolor{currentfill}%
\pgfsetfillopacity{0.600000}%
\pgfsetlinewidth{0.000000pt}%
\definecolor{currentstroke}{rgb}{0.000000,0.000000,0.000000}%
\pgfsetstrokecolor{currentstroke}%
\pgfsetstrokeopacity{0.600000}%
\pgfsetdash{}{0pt}%
\pgfpathmoveto{\pgfqpoint{1.434300in}{0.521603in}}%
\pgfpathlineto{\pgfqpoint{1.545137in}{0.521603in}}%
\pgfpathlineto{\pgfqpoint{1.545137in}{1.391865in}}%
\pgfpathlineto{\pgfqpoint{1.434300in}{1.391865in}}%
\pgfpathlineto{\pgfqpoint{1.434300in}{0.521603in}}%
\pgfpathclose%
\pgfusepath{fill}%
\end{pgfscope}%
\begin{pgfscope}%
\pgfpathrectangle{\pgfqpoint{0.595525in}{0.521603in}}{\pgfqpoint{3.201017in}{1.696195in}}%
\pgfusepath{clip}%
\pgfsetbuttcap%
\pgfsetmiterjoin%
\definecolor{currentfill}{rgb}{1.000000,0.498039,0.054902}%
\pgfsetfillcolor{currentfill}%
\pgfsetfillopacity{0.600000}%
\pgfsetlinewidth{0.000000pt}%
\definecolor{currentstroke}{rgb}{0.000000,0.000000,0.000000}%
\pgfsetstrokecolor{currentstroke}%
\pgfsetstrokeopacity{0.600000}%
\pgfsetdash{}{0pt}%
\pgfpathmoveto{\pgfqpoint{1.545137in}{0.521603in}}%
\pgfpathlineto{\pgfqpoint{1.655974in}{0.521603in}}%
\pgfpathlineto{\pgfqpoint{1.655974in}{1.304839in}}%
\pgfpathlineto{\pgfqpoint{1.545137in}{1.304839in}}%
\pgfpathlineto{\pgfqpoint{1.545137in}{0.521603in}}%
\pgfpathclose%
\pgfusepath{fill}%
\end{pgfscope}%
\begin{pgfscope}%
\pgfpathrectangle{\pgfqpoint{0.595525in}{0.521603in}}{\pgfqpoint{3.201017in}{1.696195in}}%
\pgfusepath{clip}%
\pgfsetbuttcap%
\pgfsetmiterjoin%
\definecolor{currentfill}{rgb}{1.000000,0.498039,0.054902}%
\pgfsetfillcolor{currentfill}%
\pgfsetfillopacity{0.600000}%
\pgfsetlinewidth{0.000000pt}%
\definecolor{currentstroke}{rgb}{0.000000,0.000000,0.000000}%
\pgfsetstrokecolor{currentstroke}%
\pgfsetstrokeopacity{0.600000}%
\pgfsetdash{}{0pt}%
\pgfpathmoveto{\pgfqpoint{1.655974in}{0.521603in}}%
\pgfpathlineto{\pgfqpoint{1.766811in}{0.521603in}}%
\pgfpathlineto{\pgfqpoint{1.766811in}{1.299400in}}%
\pgfpathlineto{\pgfqpoint{1.655974in}{1.299400in}}%
\pgfpathlineto{\pgfqpoint{1.655974in}{0.521603in}}%
\pgfpathclose%
\pgfusepath{fill}%
\end{pgfscope}%
\begin{pgfscope}%
\pgfpathrectangle{\pgfqpoint{0.595525in}{0.521603in}}{\pgfqpoint{3.201017in}{1.696195in}}%
\pgfusepath{clip}%
\pgfsetbuttcap%
\pgfsetmiterjoin%
\definecolor{currentfill}{rgb}{1.000000,0.498039,0.054902}%
\pgfsetfillcolor{currentfill}%
\pgfsetfillopacity{0.600000}%
\pgfsetlinewidth{0.000000pt}%
\definecolor{currentstroke}{rgb}{0.000000,0.000000,0.000000}%
\pgfsetstrokecolor{currentstroke}%
\pgfsetstrokeopacity{0.600000}%
\pgfsetdash{}{0pt}%
\pgfpathmoveto{\pgfqpoint{1.766811in}{0.521603in}}%
\pgfpathlineto{\pgfqpoint{1.877648in}{0.521603in}}%
\pgfpathlineto{\pgfqpoint{1.877648in}{1.147104in}}%
\pgfpathlineto{\pgfqpoint{1.766811in}{1.147104in}}%
\pgfpathlineto{\pgfqpoint{1.766811in}{0.521603in}}%
\pgfpathclose%
\pgfusepath{fill}%
\end{pgfscope}%
\begin{pgfscope}%
\pgfpathrectangle{\pgfqpoint{0.595525in}{0.521603in}}{\pgfqpoint{3.201017in}{1.696195in}}%
\pgfusepath{clip}%
\pgfsetbuttcap%
\pgfsetmiterjoin%
\definecolor{currentfill}{rgb}{1.000000,0.498039,0.054902}%
\pgfsetfillcolor{currentfill}%
\pgfsetfillopacity{0.600000}%
\pgfsetlinewidth{0.000000pt}%
\definecolor{currentstroke}{rgb}{0.000000,0.000000,0.000000}%
\pgfsetstrokecolor{currentstroke}%
\pgfsetstrokeopacity{0.600000}%
\pgfsetdash{}{0pt}%
\pgfpathmoveto{\pgfqpoint{1.877648in}{0.521603in}}%
\pgfpathlineto{\pgfqpoint{1.988485in}{0.521603in}}%
\pgfpathlineto{\pgfqpoint{1.988485in}{1.016565in}}%
\pgfpathlineto{\pgfqpoint{1.877648in}{1.016565in}}%
\pgfpathlineto{\pgfqpoint{1.877648in}{0.521603in}}%
\pgfpathclose%
\pgfusepath{fill}%
\end{pgfscope}%
\begin{pgfscope}%
\pgfpathrectangle{\pgfqpoint{0.595525in}{0.521603in}}{\pgfqpoint{3.201017in}{1.696195in}}%
\pgfusepath{clip}%
\pgfsetbuttcap%
\pgfsetmiterjoin%
\definecolor{currentfill}{rgb}{1.000000,0.498039,0.054902}%
\pgfsetfillcolor{currentfill}%
\pgfsetfillopacity{0.600000}%
\pgfsetlinewidth{0.000000pt}%
\definecolor{currentstroke}{rgb}{0.000000,0.000000,0.000000}%
\pgfsetstrokecolor{currentstroke}%
\pgfsetstrokeopacity{0.600000}%
\pgfsetdash{}{0pt}%
\pgfpathmoveto{\pgfqpoint{1.988485in}{0.521603in}}%
\pgfpathlineto{\pgfqpoint{2.099322in}{0.521603in}}%
\pgfpathlineto{\pgfqpoint{2.099322in}{0.983930in}}%
\pgfpathlineto{\pgfqpoint{1.988485in}{0.983930in}}%
\pgfpathlineto{\pgfqpoint{1.988485in}{0.521603in}}%
\pgfpathclose%
\pgfusepath{fill}%
\end{pgfscope}%
\begin{pgfscope}%
\pgfpathrectangle{\pgfqpoint{0.595525in}{0.521603in}}{\pgfqpoint{3.201017in}{1.696195in}}%
\pgfusepath{clip}%
\pgfsetbuttcap%
\pgfsetmiterjoin%
\definecolor{currentfill}{rgb}{1.000000,0.498039,0.054902}%
\pgfsetfillcolor{currentfill}%
\pgfsetfillopacity{0.600000}%
\pgfsetlinewidth{0.000000pt}%
\definecolor{currentstroke}{rgb}{0.000000,0.000000,0.000000}%
\pgfsetstrokecolor{currentstroke}%
\pgfsetstrokeopacity{0.600000}%
\pgfsetdash{}{0pt}%
\pgfpathmoveto{\pgfqpoint{2.099322in}{0.521603in}}%
\pgfpathlineto{\pgfqpoint{2.210159in}{0.521603in}}%
\pgfpathlineto{\pgfqpoint{2.210159in}{0.891465in}}%
\pgfpathlineto{\pgfqpoint{2.099322in}{0.891465in}}%
\pgfpathlineto{\pgfqpoint{2.099322in}{0.521603in}}%
\pgfpathclose%
\pgfusepath{fill}%
\end{pgfscope}%
\begin{pgfscope}%
\pgfpathrectangle{\pgfqpoint{0.595525in}{0.521603in}}{\pgfqpoint{3.201017in}{1.696195in}}%
\pgfusepath{clip}%
\pgfsetbuttcap%
\pgfsetmiterjoin%
\definecolor{currentfill}{rgb}{1.000000,0.498039,0.054902}%
\pgfsetfillcolor{currentfill}%
\pgfsetfillopacity{0.600000}%
\pgfsetlinewidth{0.000000pt}%
\definecolor{currentstroke}{rgb}{0.000000,0.000000,0.000000}%
\pgfsetstrokecolor{currentstroke}%
\pgfsetstrokeopacity{0.600000}%
\pgfsetdash{}{0pt}%
\pgfpathmoveto{\pgfqpoint{2.210159in}{0.521603in}}%
\pgfpathlineto{\pgfqpoint{2.320996in}{0.521603in}}%
\pgfpathlineto{\pgfqpoint{2.320996in}{0.875147in}}%
\pgfpathlineto{\pgfqpoint{2.210159in}{0.875147in}}%
\pgfpathlineto{\pgfqpoint{2.210159in}{0.521603in}}%
\pgfpathclose%
\pgfusepath{fill}%
\end{pgfscope}%
\begin{pgfscope}%
\pgfpathrectangle{\pgfqpoint{0.595525in}{0.521603in}}{\pgfqpoint{3.201017in}{1.696195in}}%
\pgfusepath{clip}%
\pgfsetbuttcap%
\pgfsetmiterjoin%
\definecolor{currentfill}{rgb}{1.000000,0.498039,0.054902}%
\pgfsetfillcolor{currentfill}%
\pgfsetfillopacity{0.600000}%
\pgfsetlinewidth{0.000000pt}%
\definecolor{currentstroke}{rgb}{0.000000,0.000000,0.000000}%
\pgfsetstrokecolor{currentstroke}%
\pgfsetstrokeopacity{0.600000}%
\pgfsetdash{}{0pt}%
\pgfpathmoveto{\pgfqpoint{2.320996in}{0.521603in}}%
\pgfpathlineto{\pgfqpoint{2.431833in}{0.521603in}}%
\pgfpathlineto{\pgfqpoint{2.431833in}{0.847952in}}%
\pgfpathlineto{\pgfqpoint{2.320996in}{0.847952in}}%
\pgfpathlineto{\pgfqpoint{2.320996in}{0.521603in}}%
\pgfpathclose%
\pgfusepath{fill}%
\end{pgfscope}%
\begin{pgfscope}%
\pgfpathrectangle{\pgfqpoint{0.595525in}{0.521603in}}{\pgfqpoint{3.201017in}{1.696195in}}%
\pgfusepath{clip}%
\pgfsetbuttcap%
\pgfsetmiterjoin%
\definecolor{currentfill}{rgb}{1.000000,0.498039,0.054902}%
\pgfsetfillcolor{currentfill}%
\pgfsetfillopacity{0.600000}%
\pgfsetlinewidth{0.000000pt}%
\definecolor{currentstroke}{rgb}{0.000000,0.000000,0.000000}%
\pgfsetstrokecolor{currentstroke}%
\pgfsetstrokeopacity{0.600000}%
\pgfsetdash{}{0pt}%
\pgfpathmoveto{\pgfqpoint{2.431833in}{0.521603in}}%
\pgfpathlineto{\pgfqpoint{2.542670in}{0.521603in}}%
\pgfpathlineto{\pgfqpoint{2.542670in}{0.728291in}}%
\pgfpathlineto{\pgfqpoint{2.431833in}{0.728291in}}%
\pgfpathlineto{\pgfqpoint{2.431833in}{0.521603in}}%
\pgfpathclose%
\pgfusepath{fill}%
\end{pgfscope}%
\begin{pgfscope}%
\pgfpathrectangle{\pgfqpoint{0.595525in}{0.521603in}}{\pgfqpoint{3.201017in}{1.696195in}}%
\pgfusepath{clip}%
\pgfsetbuttcap%
\pgfsetmiterjoin%
\definecolor{currentfill}{rgb}{1.000000,0.498039,0.054902}%
\pgfsetfillcolor{currentfill}%
\pgfsetfillopacity{0.600000}%
\pgfsetlinewidth{0.000000pt}%
\definecolor{currentstroke}{rgb}{0.000000,0.000000,0.000000}%
\pgfsetstrokecolor{currentstroke}%
\pgfsetstrokeopacity{0.600000}%
\pgfsetdash{}{0pt}%
\pgfpathmoveto{\pgfqpoint{2.542670in}{0.521603in}}%
\pgfpathlineto{\pgfqpoint{2.653508in}{0.521603in}}%
\pgfpathlineto{\pgfqpoint{2.653508in}{0.701095in}}%
\pgfpathlineto{\pgfqpoint{2.542670in}{0.701095in}}%
\pgfpathlineto{\pgfqpoint{2.542670in}{0.521603in}}%
\pgfpathclose%
\pgfusepath{fill}%
\end{pgfscope}%
\begin{pgfscope}%
\pgfpathrectangle{\pgfqpoint{0.595525in}{0.521603in}}{\pgfqpoint{3.201017in}{1.696195in}}%
\pgfusepath{clip}%
\pgfsetbuttcap%
\pgfsetmiterjoin%
\definecolor{currentfill}{rgb}{1.000000,0.498039,0.054902}%
\pgfsetfillcolor{currentfill}%
\pgfsetfillopacity{0.600000}%
\pgfsetlinewidth{0.000000pt}%
\definecolor{currentstroke}{rgb}{0.000000,0.000000,0.000000}%
\pgfsetstrokecolor{currentstroke}%
\pgfsetstrokeopacity{0.600000}%
\pgfsetdash{}{0pt}%
\pgfpathmoveto{\pgfqpoint{2.653508in}{0.521603in}}%
\pgfpathlineto{\pgfqpoint{2.764345in}{0.521603in}}%
\pgfpathlineto{\pgfqpoint{2.764345in}{0.711973in}}%
\pgfpathlineto{\pgfqpoint{2.653508in}{0.711973in}}%
\pgfpathlineto{\pgfqpoint{2.653508in}{0.521603in}}%
\pgfpathclose%
\pgfusepath{fill}%
\end{pgfscope}%
\begin{pgfscope}%
\pgfpathrectangle{\pgfqpoint{0.595525in}{0.521603in}}{\pgfqpoint{3.201017in}{1.696195in}}%
\pgfusepath{clip}%
\pgfsetbuttcap%
\pgfsetmiterjoin%
\definecolor{currentfill}{rgb}{1.000000,0.498039,0.054902}%
\pgfsetfillcolor{currentfill}%
\pgfsetfillopacity{0.600000}%
\pgfsetlinewidth{0.000000pt}%
\definecolor{currentstroke}{rgb}{0.000000,0.000000,0.000000}%
\pgfsetstrokecolor{currentstroke}%
\pgfsetstrokeopacity{0.600000}%
\pgfsetdash{}{0pt}%
\pgfpathmoveto{\pgfqpoint{2.764345in}{0.521603in}}%
\pgfpathlineto{\pgfqpoint{2.875182in}{0.521603in}}%
\pgfpathlineto{\pgfqpoint{2.875182in}{0.695656in}}%
\pgfpathlineto{\pgfqpoint{2.764345in}{0.695656in}}%
\pgfpathlineto{\pgfqpoint{2.764345in}{0.521603in}}%
\pgfpathclose%
\pgfusepath{fill}%
\end{pgfscope}%
\begin{pgfscope}%
\pgfpathrectangle{\pgfqpoint{0.595525in}{0.521603in}}{\pgfqpoint{3.201017in}{1.696195in}}%
\pgfusepath{clip}%
\pgfsetbuttcap%
\pgfsetmiterjoin%
\definecolor{currentfill}{rgb}{1.000000,0.498039,0.054902}%
\pgfsetfillcolor{currentfill}%
\pgfsetfillopacity{0.600000}%
\pgfsetlinewidth{0.000000pt}%
\definecolor{currentstroke}{rgb}{0.000000,0.000000,0.000000}%
\pgfsetstrokecolor{currentstroke}%
\pgfsetstrokeopacity{0.600000}%
\pgfsetdash{}{0pt}%
\pgfpathmoveto{\pgfqpoint{2.875182in}{0.521603in}}%
\pgfpathlineto{\pgfqpoint{2.986019in}{0.521603in}}%
\pgfpathlineto{\pgfqpoint{2.986019in}{0.684777in}}%
\pgfpathlineto{\pgfqpoint{2.875182in}{0.684777in}}%
\pgfpathlineto{\pgfqpoint{2.875182in}{0.521603in}}%
\pgfpathclose%
\pgfusepath{fill}%
\end{pgfscope}%
\begin{pgfscope}%
\pgfpathrectangle{\pgfqpoint{0.595525in}{0.521603in}}{\pgfqpoint{3.201017in}{1.696195in}}%
\pgfusepath{clip}%
\pgfsetbuttcap%
\pgfsetmiterjoin%
\definecolor{currentfill}{rgb}{1.000000,0.498039,0.054902}%
\pgfsetfillcolor{currentfill}%
\pgfsetfillopacity{0.600000}%
\pgfsetlinewidth{0.000000pt}%
\definecolor{currentstroke}{rgb}{0.000000,0.000000,0.000000}%
\pgfsetstrokecolor{currentstroke}%
\pgfsetstrokeopacity{0.600000}%
\pgfsetdash{}{0pt}%
\pgfpathmoveto{\pgfqpoint{2.986019in}{0.521603in}}%
\pgfpathlineto{\pgfqpoint{3.096856in}{0.521603in}}%
\pgfpathlineto{\pgfqpoint{3.096856in}{0.684777in}}%
\pgfpathlineto{\pgfqpoint{2.986019in}{0.684777in}}%
\pgfpathlineto{\pgfqpoint{2.986019in}{0.521603in}}%
\pgfpathclose%
\pgfusepath{fill}%
\end{pgfscope}%
\begin{pgfscope}%
\pgfpathrectangle{\pgfqpoint{0.595525in}{0.521603in}}{\pgfqpoint{3.201017in}{1.696195in}}%
\pgfusepath{clip}%
\pgfsetbuttcap%
\pgfsetmiterjoin%
\definecolor{currentfill}{rgb}{1.000000,0.498039,0.054902}%
\pgfsetfillcolor{currentfill}%
\pgfsetfillopacity{0.600000}%
\pgfsetlinewidth{0.000000pt}%
\definecolor{currentstroke}{rgb}{0.000000,0.000000,0.000000}%
\pgfsetstrokecolor{currentstroke}%
\pgfsetstrokeopacity{0.600000}%
\pgfsetdash{}{0pt}%
\pgfpathmoveto{\pgfqpoint{3.096856in}{0.521603in}}%
\pgfpathlineto{\pgfqpoint{3.207693in}{0.521603in}}%
\pgfpathlineto{\pgfqpoint{3.207693in}{0.592312in}}%
\pgfpathlineto{\pgfqpoint{3.096856in}{0.592312in}}%
\pgfpathlineto{\pgfqpoint{3.096856in}{0.521603in}}%
\pgfpathclose%
\pgfusepath{fill}%
\end{pgfscope}%
\begin{pgfscope}%
\pgfpathrectangle{\pgfqpoint{0.595525in}{0.521603in}}{\pgfqpoint{3.201017in}{1.696195in}}%
\pgfusepath{clip}%
\pgfsetbuttcap%
\pgfsetmiterjoin%
\definecolor{currentfill}{rgb}{1.000000,0.498039,0.054902}%
\pgfsetfillcolor{currentfill}%
\pgfsetfillopacity{0.600000}%
\pgfsetlinewidth{0.000000pt}%
\definecolor{currentstroke}{rgb}{0.000000,0.000000,0.000000}%
\pgfsetstrokecolor{currentstroke}%
\pgfsetstrokeopacity{0.600000}%
\pgfsetdash{}{0pt}%
\pgfpathmoveto{\pgfqpoint{3.207693in}{0.521603in}}%
\pgfpathlineto{\pgfqpoint{3.318530in}{0.521603in}}%
\pgfpathlineto{\pgfqpoint{3.318530in}{0.614069in}}%
\pgfpathlineto{\pgfqpoint{3.207693in}{0.614069in}}%
\pgfpathlineto{\pgfqpoint{3.207693in}{0.521603in}}%
\pgfpathclose%
\pgfusepath{fill}%
\end{pgfscope}%
\begin{pgfscope}%
\pgfpathrectangle{\pgfqpoint{0.595525in}{0.521603in}}{\pgfqpoint{3.201017in}{1.696195in}}%
\pgfusepath{clip}%
\pgfsetbuttcap%
\pgfsetmiterjoin%
\definecolor{currentfill}{rgb}{1.000000,0.498039,0.054902}%
\pgfsetfillcolor{currentfill}%
\pgfsetfillopacity{0.600000}%
\pgfsetlinewidth{0.000000pt}%
\definecolor{currentstroke}{rgb}{0.000000,0.000000,0.000000}%
\pgfsetstrokecolor{currentstroke}%
\pgfsetstrokeopacity{0.600000}%
\pgfsetdash{}{0pt}%
\pgfpathmoveto{\pgfqpoint{3.318530in}{0.521603in}}%
\pgfpathlineto{\pgfqpoint{3.429367in}{0.521603in}}%
\pgfpathlineto{\pgfqpoint{3.429367in}{0.614069in}}%
\pgfpathlineto{\pgfqpoint{3.318530in}{0.614069in}}%
\pgfpathlineto{\pgfqpoint{3.318530in}{0.521603in}}%
\pgfpathclose%
\pgfusepath{fill}%
\end{pgfscope}%
\begin{pgfscope}%
\pgfpathrectangle{\pgfqpoint{0.595525in}{0.521603in}}{\pgfqpoint{3.201017in}{1.696195in}}%
\pgfusepath{clip}%
\pgfsetbuttcap%
\pgfsetmiterjoin%
\definecolor{currentfill}{rgb}{1.000000,0.498039,0.054902}%
\pgfsetfillcolor{currentfill}%
\pgfsetfillopacity{0.600000}%
\pgfsetlinewidth{0.000000pt}%
\definecolor{currentstroke}{rgb}{0.000000,0.000000,0.000000}%
\pgfsetstrokecolor{currentstroke}%
\pgfsetstrokeopacity{0.600000}%
\pgfsetdash{}{0pt}%
\pgfpathmoveto{\pgfqpoint{3.429367in}{0.521603in}}%
\pgfpathlineto{\pgfqpoint{3.540204in}{0.521603in}}%
\pgfpathlineto{\pgfqpoint{3.540204in}{0.586873in}}%
\pgfpathlineto{\pgfqpoint{3.429367in}{0.586873in}}%
\pgfpathlineto{\pgfqpoint{3.429367in}{0.521603in}}%
\pgfpathclose%
\pgfusepath{fill}%
\end{pgfscope}%
\begin{pgfscope}%
\pgfpathrectangle{\pgfqpoint{0.595525in}{0.521603in}}{\pgfqpoint{3.201017in}{1.696195in}}%
\pgfusepath{clip}%
\pgfsetbuttcap%
\pgfsetmiterjoin%
\definecolor{currentfill}{rgb}{1.000000,0.498039,0.054902}%
\pgfsetfillcolor{currentfill}%
\pgfsetfillopacity{0.600000}%
\pgfsetlinewidth{0.000000pt}%
\definecolor{currentstroke}{rgb}{0.000000,0.000000,0.000000}%
\pgfsetstrokecolor{currentstroke}%
\pgfsetstrokeopacity{0.600000}%
\pgfsetdash{}{0pt}%
\pgfpathmoveto{\pgfqpoint{3.540204in}{0.521603in}}%
\pgfpathlineto{\pgfqpoint{3.651041in}{0.521603in}}%
\pgfpathlineto{\pgfqpoint{3.651041in}{0.543360in}}%
\pgfpathlineto{\pgfqpoint{3.540204in}{0.543360in}}%
\pgfpathlineto{\pgfqpoint{3.540204in}{0.521603in}}%
\pgfpathclose%
\pgfusepath{fill}%
\end{pgfscope}%
\begin{pgfscope}%
\pgfsetbuttcap%
\pgfsetroundjoin%
\definecolor{currentfill}{rgb}{0.000000,0.000000,0.000000}%
\pgfsetfillcolor{currentfill}%
\pgfsetlinewidth{0.803000pt}%
\definecolor{currentstroke}{rgb}{0.000000,0.000000,0.000000}%
\pgfsetstrokecolor{currentstroke}%
\pgfsetdash{}{0pt}%
\pgfsys@defobject{currentmarker}{\pgfqpoint{0.000000in}{-0.048611in}}{\pgfqpoint{0.000000in}{0.000000in}}{%
\pgfpathmoveto{\pgfqpoint{0.000000in}{0.000000in}}%
\pgfpathlineto{\pgfqpoint{0.000000in}{-0.048611in}}%
\pgfusepath{stroke,fill}%
}%
\begin{pgfscope}%
\pgfsys@transformshift{0.684520in}{0.521603in}%
\pgfsys@useobject{currentmarker}{}%
\end{pgfscope}%
\end{pgfscope}%
\begin{pgfscope}%
\definecolor{textcolor}{rgb}{0.000000,0.000000,0.000000}%
\pgfsetstrokecolor{textcolor}%
\pgfsetfillcolor{textcolor}%
\pgftext[x=0.684520in,y=0.424381in,,top]{\color{textcolor}{\rmfamily\fontsize{10.000000}{12.000000}\selectfont\catcode`\^=\active\def^{\ifmmode\sp\else\^{}\fi}\catcode`\%=\active\def%{\%}$\mathdefault{0}$}}%
\end{pgfscope}%
\begin{pgfscope}%
\pgfsetbuttcap%
\pgfsetroundjoin%
\definecolor{currentfill}{rgb}{0.000000,0.000000,0.000000}%
\pgfsetfillcolor{currentfill}%
\pgfsetlinewidth{0.803000pt}%
\definecolor{currentstroke}{rgb}{0.000000,0.000000,0.000000}%
\pgfsetstrokecolor{currentstroke}%
\pgfsetdash{}{0pt}%
\pgfsys@defobject{currentmarker}{\pgfqpoint{0.000000in}{-0.048611in}}{\pgfqpoint{0.000000in}{0.000000in}}{%
\pgfpathmoveto{\pgfqpoint{0.000000in}{0.000000in}}%
\pgfpathlineto{\pgfqpoint{0.000000in}{-0.048611in}}%
\pgfusepath{stroke,fill}%
}%
\begin{pgfscope}%
\pgfsys@transformshift{1.249572in}{0.521603in}%
\pgfsys@useobject{currentmarker}{}%
\end{pgfscope}%
\end{pgfscope}%
\begin{pgfscope}%
\definecolor{textcolor}{rgb}{0.000000,0.000000,0.000000}%
\pgfsetstrokecolor{textcolor}%
\pgfsetfillcolor{textcolor}%
\pgftext[x=1.249572in,y=0.424381in,,top]{\color{textcolor}{\rmfamily\fontsize{10.000000}{12.000000}\selectfont\catcode`\^=\active\def^{\ifmmode\sp\else\^{}\fi}\catcode`\%=\active\def%{\%}$\mathdefault{20}$}}%
\end{pgfscope}%
\begin{pgfscope}%
\pgfsetbuttcap%
\pgfsetroundjoin%
\definecolor{currentfill}{rgb}{0.000000,0.000000,0.000000}%
\pgfsetfillcolor{currentfill}%
\pgfsetlinewidth{0.803000pt}%
\definecolor{currentstroke}{rgb}{0.000000,0.000000,0.000000}%
\pgfsetstrokecolor{currentstroke}%
\pgfsetdash{}{0pt}%
\pgfsys@defobject{currentmarker}{\pgfqpoint{0.000000in}{-0.048611in}}{\pgfqpoint{0.000000in}{0.000000in}}{%
\pgfpathmoveto{\pgfqpoint{0.000000in}{0.000000in}}%
\pgfpathlineto{\pgfqpoint{0.000000in}{-0.048611in}}%
\pgfusepath{stroke,fill}%
}%
\begin{pgfscope}%
\pgfsys@transformshift{1.814623in}{0.521603in}%
\pgfsys@useobject{currentmarker}{}%
\end{pgfscope}%
\end{pgfscope}%
\begin{pgfscope}%
\definecolor{textcolor}{rgb}{0.000000,0.000000,0.000000}%
\pgfsetstrokecolor{textcolor}%
\pgfsetfillcolor{textcolor}%
\pgftext[x=1.814623in,y=0.424381in,,top]{\color{textcolor}{\rmfamily\fontsize{10.000000}{12.000000}\selectfont\catcode`\^=\active\def^{\ifmmode\sp\else\^{}\fi}\catcode`\%=\active\def%{\%}$\mathdefault{40}$}}%
\end{pgfscope}%
\begin{pgfscope}%
\pgfsetbuttcap%
\pgfsetroundjoin%
\definecolor{currentfill}{rgb}{0.000000,0.000000,0.000000}%
\pgfsetfillcolor{currentfill}%
\pgfsetlinewidth{0.803000pt}%
\definecolor{currentstroke}{rgb}{0.000000,0.000000,0.000000}%
\pgfsetstrokecolor{currentstroke}%
\pgfsetdash{}{0pt}%
\pgfsys@defobject{currentmarker}{\pgfqpoint{0.000000in}{-0.048611in}}{\pgfqpoint{0.000000in}{0.000000in}}{%
\pgfpathmoveto{\pgfqpoint{0.000000in}{0.000000in}}%
\pgfpathlineto{\pgfqpoint{0.000000in}{-0.048611in}}%
\pgfusepath{stroke,fill}%
}%
\begin{pgfscope}%
\pgfsys@transformshift{2.379675in}{0.521603in}%
\pgfsys@useobject{currentmarker}{}%
\end{pgfscope}%
\end{pgfscope}%
\begin{pgfscope}%
\definecolor{textcolor}{rgb}{0.000000,0.000000,0.000000}%
\pgfsetstrokecolor{textcolor}%
\pgfsetfillcolor{textcolor}%
\pgftext[x=2.379675in,y=0.424381in,,top]{\color{textcolor}{\rmfamily\fontsize{10.000000}{12.000000}\selectfont\catcode`\^=\active\def^{\ifmmode\sp\else\^{}\fi}\catcode`\%=\active\def%{\%}$\mathdefault{60}$}}%
\end{pgfscope}%
\begin{pgfscope}%
\pgfsetbuttcap%
\pgfsetroundjoin%
\definecolor{currentfill}{rgb}{0.000000,0.000000,0.000000}%
\pgfsetfillcolor{currentfill}%
\pgfsetlinewidth{0.803000pt}%
\definecolor{currentstroke}{rgb}{0.000000,0.000000,0.000000}%
\pgfsetstrokecolor{currentstroke}%
\pgfsetdash{}{0pt}%
\pgfsys@defobject{currentmarker}{\pgfqpoint{0.000000in}{-0.048611in}}{\pgfqpoint{0.000000in}{0.000000in}}{%
\pgfpathmoveto{\pgfqpoint{0.000000in}{0.000000in}}%
\pgfpathlineto{\pgfqpoint{0.000000in}{-0.048611in}}%
\pgfusepath{stroke,fill}%
}%
\begin{pgfscope}%
\pgfsys@transformshift{2.944726in}{0.521603in}%
\pgfsys@useobject{currentmarker}{}%
\end{pgfscope}%
\end{pgfscope}%
\begin{pgfscope}%
\definecolor{textcolor}{rgb}{0.000000,0.000000,0.000000}%
\pgfsetstrokecolor{textcolor}%
\pgfsetfillcolor{textcolor}%
\pgftext[x=2.944726in,y=0.424381in,,top]{\color{textcolor}{\rmfamily\fontsize{10.000000}{12.000000}\selectfont\catcode`\^=\active\def^{\ifmmode\sp\else\^{}\fi}\catcode`\%=\active\def%{\%}$\mathdefault{80}$}}%
\end{pgfscope}%
\begin{pgfscope}%
\pgfsetbuttcap%
\pgfsetroundjoin%
\definecolor{currentfill}{rgb}{0.000000,0.000000,0.000000}%
\pgfsetfillcolor{currentfill}%
\pgfsetlinewidth{0.803000pt}%
\definecolor{currentstroke}{rgb}{0.000000,0.000000,0.000000}%
\pgfsetstrokecolor{currentstroke}%
\pgfsetdash{}{0pt}%
\pgfsys@defobject{currentmarker}{\pgfqpoint{0.000000in}{-0.048611in}}{\pgfqpoint{0.000000in}{0.000000in}}{%
\pgfpathmoveto{\pgfqpoint{0.000000in}{0.000000in}}%
\pgfpathlineto{\pgfqpoint{0.000000in}{-0.048611in}}%
\pgfusepath{stroke,fill}%
}%
\begin{pgfscope}%
\pgfsys@transformshift{3.509778in}{0.521603in}%
\pgfsys@useobject{currentmarker}{}%
\end{pgfscope}%
\end{pgfscope}%
\begin{pgfscope}%
\definecolor{textcolor}{rgb}{0.000000,0.000000,0.000000}%
\pgfsetstrokecolor{textcolor}%
\pgfsetfillcolor{textcolor}%
\pgftext[x=3.509778in,y=0.424381in,,top]{\color{textcolor}{\rmfamily\fontsize{10.000000}{12.000000}\selectfont\catcode`\^=\active\def^{\ifmmode\sp\else\^{}\fi}\catcode`\%=\active\def%{\%}$\mathdefault{100}$}}%
\end{pgfscope}%
\begin{pgfscope}%
\definecolor{textcolor}{rgb}{0.000000,0.000000,0.000000}%
\pgfsetstrokecolor{textcolor}%
\pgfsetfillcolor{textcolor}%
\pgftext[x=2.196033in,y=0.234413in,,top]{\color{textcolor}{\rmfamily\fontsize{10.000000}{12.000000}\selectfont\catcode`\^=\active\def^{\ifmmode\sp\else\^{}\fi}\catcode`\%=\active\def%{\%}Number of m. classes or $\triangle$-conn. components}}%
\end{pgfscope}%
\begin{pgfscope}%
\pgfsetbuttcap%
\pgfsetroundjoin%
\definecolor{currentfill}{rgb}{0.000000,0.000000,0.000000}%
\pgfsetfillcolor{currentfill}%
\pgfsetlinewidth{0.803000pt}%
\definecolor{currentstroke}{rgb}{0.000000,0.000000,0.000000}%
\pgfsetstrokecolor{currentstroke}%
\pgfsetdash{}{0pt}%
\pgfsys@defobject{currentmarker}{\pgfqpoint{-0.048611in}{0.000000in}}{\pgfqpoint{-0.000000in}{0.000000in}}{%
\pgfpathmoveto{\pgfqpoint{-0.000000in}{0.000000in}}%
\pgfpathlineto{\pgfqpoint{-0.048611in}{0.000000in}}%
\pgfusepath{stroke,fill}%
}%
\begin{pgfscope}%
\pgfsys@transformshift{0.595525in}{0.521603in}%
\pgfsys@useobject{currentmarker}{}%
\end{pgfscope}%
\end{pgfscope}%
\begin{pgfscope}%
\definecolor{textcolor}{rgb}{0.000000,0.000000,0.000000}%
\pgfsetstrokecolor{textcolor}%
\pgfsetfillcolor{textcolor}%
\pgftext[x=0.428858in, y=0.468842in, left, base]{\color{textcolor}{\rmfamily\fontsize{10.000000}{12.000000}\selectfont\catcode`\^=\active\def^{\ifmmode\sp\else\^{}\fi}\catcode`\%=\active\def%{\%}$\mathdefault{0}$}}%
\end{pgfscope}%
\begin{pgfscope}%
\pgfsetbuttcap%
\pgfsetroundjoin%
\definecolor{currentfill}{rgb}{0.000000,0.000000,0.000000}%
\pgfsetfillcolor{currentfill}%
\pgfsetlinewidth{0.803000pt}%
\definecolor{currentstroke}{rgb}{0.000000,0.000000,0.000000}%
\pgfsetstrokecolor{currentstroke}%
\pgfsetdash{}{0pt}%
\pgfsys@defobject{currentmarker}{\pgfqpoint{-0.048611in}{0.000000in}}{\pgfqpoint{-0.000000in}{0.000000in}}{%
\pgfpathmoveto{\pgfqpoint{-0.000000in}{0.000000in}}%
\pgfpathlineto{\pgfqpoint{-0.048611in}{0.000000in}}%
\pgfusepath{stroke,fill}%
}%
\begin{pgfscope}%
\pgfsys@transformshift{0.595525in}{1.065517in}%
\pgfsys@useobject{currentmarker}{}%
\end{pgfscope}%
\end{pgfscope}%
\begin{pgfscope}%
\definecolor{textcolor}{rgb}{0.000000,0.000000,0.000000}%
\pgfsetstrokecolor{textcolor}%
\pgfsetfillcolor{textcolor}%
\pgftext[x=0.289968in, y=1.012755in, left, base]{\color{textcolor}{\rmfamily\fontsize{10.000000}{12.000000}\selectfont\catcode`\^=\active\def^{\ifmmode\sp\else\^{}\fi}\catcode`\%=\active\def%{\%}$\mathdefault{100}$}}%
\end{pgfscope}%
\begin{pgfscope}%
\pgfsetbuttcap%
\pgfsetroundjoin%
\definecolor{currentfill}{rgb}{0.000000,0.000000,0.000000}%
\pgfsetfillcolor{currentfill}%
\pgfsetlinewidth{0.803000pt}%
\definecolor{currentstroke}{rgb}{0.000000,0.000000,0.000000}%
\pgfsetstrokecolor{currentstroke}%
\pgfsetdash{}{0pt}%
\pgfsys@defobject{currentmarker}{\pgfqpoint{-0.048611in}{0.000000in}}{\pgfqpoint{-0.000000in}{0.000000in}}{%
\pgfpathmoveto{\pgfqpoint{-0.000000in}{0.000000in}}%
\pgfpathlineto{\pgfqpoint{-0.048611in}{0.000000in}}%
\pgfusepath{stroke,fill}%
}%
\begin{pgfscope}%
\pgfsys@transformshift{0.595525in}{1.609431in}%
\pgfsys@useobject{currentmarker}{}%
\end{pgfscope}%
\end{pgfscope}%
\begin{pgfscope}%
\definecolor{textcolor}{rgb}{0.000000,0.000000,0.000000}%
\pgfsetstrokecolor{textcolor}%
\pgfsetfillcolor{textcolor}%
\pgftext[x=0.289968in, y=1.556669in, left, base]{\color{textcolor}{\rmfamily\fontsize{10.000000}{12.000000}\selectfont\catcode`\^=\active\def^{\ifmmode\sp\else\^{}\fi}\catcode`\%=\active\def%{\%}$\mathdefault{200}$}}%
\end{pgfscope}%
\begin{pgfscope}%
\pgfsetbuttcap%
\pgfsetroundjoin%
\definecolor{currentfill}{rgb}{0.000000,0.000000,0.000000}%
\pgfsetfillcolor{currentfill}%
\pgfsetlinewidth{0.803000pt}%
\definecolor{currentstroke}{rgb}{0.000000,0.000000,0.000000}%
\pgfsetstrokecolor{currentstroke}%
\pgfsetdash{}{0pt}%
\pgfsys@defobject{currentmarker}{\pgfqpoint{-0.048611in}{0.000000in}}{\pgfqpoint{-0.000000in}{0.000000in}}{%
\pgfpathmoveto{\pgfqpoint{-0.000000in}{0.000000in}}%
\pgfpathlineto{\pgfqpoint{-0.048611in}{0.000000in}}%
\pgfusepath{stroke,fill}%
}%
\begin{pgfscope}%
\pgfsys@transformshift{0.595525in}{2.153344in}%
\pgfsys@useobject{currentmarker}{}%
\end{pgfscope}%
\end{pgfscope}%
\begin{pgfscope}%
\definecolor{textcolor}{rgb}{0.000000,0.000000,0.000000}%
\pgfsetstrokecolor{textcolor}%
\pgfsetfillcolor{textcolor}%
\pgftext[x=0.289968in, y=2.100583in, left, base]{\color{textcolor}{\rmfamily\fontsize{10.000000}{12.000000}\selectfont\catcode`\^=\active\def^{\ifmmode\sp\else\^{}\fi}\catcode`\%=\active\def%{\%}$\mathdefault{300}$}}%
\end{pgfscope}%
\begin{pgfscope}%
\definecolor{textcolor}{rgb}{0.000000,0.000000,0.000000}%
\pgfsetstrokecolor{textcolor}%
\pgfsetfillcolor{textcolor}%
\pgftext[x=0.234413in,y=1.369701in,,bottom,rotate=90.000000]{\color{textcolor}{\rmfamily\fontsize{10.000000}{12.000000}\selectfont\catcode`\^=\active\def^{\ifmmode\sp\else\^{}\fi}\catcode`\%=\active\def%{\%}Number of graphs}}%
\end{pgfscope}%
\begin{pgfscope}%
\pgfsetrectcap%
\pgfsetmiterjoin%
\pgfsetlinewidth{0.803000pt}%
\definecolor{currentstroke}{rgb}{0.000000,0.000000,0.000000}%
\pgfsetstrokecolor{currentstroke}%
\pgfsetdash{}{0pt}%
\pgfpathmoveto{\pgfqpoint{0.595525in}{0.521603in}}%
\pgfpathlineto{\pgfqpoint{0.595525in}{2.217798in}}%
\pgfusepath{stroke}%
\end{pgfscope}%
\begin{pgfscope}%
\pgfsetrectcap%
\pgfsetmiterjoin%
\pgfsetlinewidth{0.803000pt}%
\definecolor{currentstroke}{rgb}{0.000000,0.000000,0.000000}%
\pgfsetstrokecolor{currentstroke}%
\pgfsetdash{}{0pt}%
\pgfpathmoveto{\pgfqpoint{3.796542in}{0.521603in}}%
\pgfpathlineto{\pgfqpoint{3.796542in}{2.217798in}}%
\pgfusepath{stroke}%
\end{pgfscope}%
\begin{pgfscope}%
\pgfsetrectcap%
\pgfsetmiterjoin%
\pgfsetlinewidth{0.803000pt}%
\definecolor{currentstroke}{rgb}{0.000000,0.000000,0.000000}%
\pgfsetstrokecolor{currentstroke}%
\pgfsetdash{}{0pt}%
\pgfpathmoveto{\pgfqpoint{0.595525in}{0.521603in}}%
\pgfpathlineto{\pgfqpoint{3.796542in}{0.521603in}}%
\pgfusepath{stroke}%
\end{pgfscope}%
\begin{pgfscope}%
\pgfsetrectcap%
\pgfsetmiterjoin%
\pgfsetlinewidth{0.803000pt}%
\definecolor{currentstroke}{rgb}{0.000000,0.000000,0.000000}%
\pgfsetstrokecolor{currentstroke}%
\pgfsetdash{}{0pt}%
\pgfpathmoveto{\pgfqpoint{0.595525in}{2.217798in}}%
\pgfpathlineto{\pgfqpoint{3.796542in}{2.217798in}}%
\pgfusepath{stroke}%
\end{pgfscope}%
\begin{pgfscope}%
\pgfsetbuttcap%
\pgfsetmiterjoin%
\definecolor{currentfill}{rgb}{1.000000,1.000000,1.000000}%
\pgfsetfillcolor{currentfill}%
\pgfsetfillopacity{0.800000}%
\pgfsetlinewidth{1.003750pt}%
\definecolor{currentstroke}{rgb}{0.800000,0.800000,0.800000}%
\pgfsetstrokecolor{currentstroke}%
\pgfsetstrokeopacity{0.800000}%
\pgfsetdash{}{0pt}%
\pgfpathmoveto{\pgfqpoint{0.982850in}{1.595207in}}%
\pgfpathlineto{\pgfqpoint{3.679875in}{1.595207in}}%
\pgfpathquadraticcurveto{\pgfqpoint{3.713208in}{1.595207in}}{\pgfqpoint{3.713208in}{1.628541in}}%
\pgfpathlineto{\pgfqpoint{3.713208in}{2.101131in}}%
\pgfpathquadraticcurveto{\pgfqpoint{3.713208in}{2.134465in}}{\pgfqpoint{3.679875in}{2.134465in}}%
\pgfpathlineto{\pgfqpoint{0.982850in}{2.134465in}}%
\pgfpathquadraticcurveto{\pgfqpoint{0.949516in}{2.134465in}}{\pgfqpoint{0.949516in}{2.101131in}}%
\pgfpathlineto{\pgfqpoint{0.949516in}{1.628541in}}%
\pgfpathquadraticcurveto{\pgfqpoint{0.949516in}{1.595207in}}{\pgfqpoint{0.982850in}{1.595207in}}%
\pgfpathlineto{\pgfqpoint{0.982850in}{1.595207in}}%
\pgfpathclose%
\pgfusepath{stroke,fill}%
\end{pgfscope}%
\begin{pgfscope}%
\pgfsetbuttcap%
\pgfsetmiterjoin%
\definecolor{currentfill}{rgb}{0.121569,0.466667,0.705882}%
\pgfsetfillcolor{currentfill}%
\pgfsetfillopacity{0.600000}%
\pgfsetlinewidth{0.000000pt}%
\definecolor{currentstroke}{rgb}{0.000000,0.000000,0.000000}%
\pgfsetstrokecolor{currentstroke}%
\pgfsetstrokeopacity{0.600000}%
\pgfsetdash{}{0pt}%
\pgfpathmoveto{\pgfqpoint{1.016183in}{1.941171in}}%
\pgfpathlineto{\pgfqpoint{1.349516in}{1.941171in}}%
\pgfpathlineto{\pgfqpoint{1.349516in}{2.057837in}}%
\pgfpathlineto{\pgfqpoint{1.016183in}{2.057837in}}%
\pgfpathlineto{\pgfqpoint{1.016183in}{1.941171in}}%
\pgfpathclose%
\pgfusepath{fill}%
\end{pgfscope}%
\begin{pgfscope}%
\definecolor{textcolor}{rgb}{0.000000,0.000000,0.000000}%
\pgfsetstrokecolor{textcolor}%
\pgfsetfillcolor{textcolor}%
\pgftext[x=1.482850in,y=1.941171in,left,base]{\color{textcolor}{\rmfamily\fontsize{12.000000}{14.400000}\selectfont\catcode`\^=\active\def^{\ifmmode\sp\else\^{}\fi}\catcode`\%=\active\def%{\%}Monochromatic classes}}%
\end{pgfscope}%
\begin{pgfscope}%
\pgfsetbuttcap%
\pgfsetmiterjoin%
\definecolor{currentfill}{rgb}{1.000000,0.498039,0.054902}%
\pgfsetfillcolor{currentfill}%
\pgfsetfillopacity{0.600000}%
\pgfsetlinewidth{0.000000pt}%
\definecolor{currentstroke}{rgb}{0.000000,0.000000,0.000000}%
\pgfsetstrokecolor{currentstroke}%
\pgfsetstrokeopacity{0.600000}%
\pgfsetdash{}{0pt}%
\pgfpathmoveto{\pgfqpoint{1.016183in}{1.696542in}}%
\pgfpathlineto{\pgfqpoint{1.349516in}{1.696542in}}%
\pgfpathlineto{\pgfqpoint{1.349516in}{1.813208in}}%
\pgfpathlineto{\pgfqpoint{1.016183in}{1.813208in}}%
\pgfpathlineto{\pgfqpoint{1.016183in}{1.696542in}}%
\pgfpathclose%
\pgfusepath{fill}%
\end{pgfscope}%
\begin{pgfscope}%
\definecolor{textcolor}{rgb}{0.000000,0.000000,0.000000}%
\pgfsetstrokecolor{textcolor}%
\pgfsetfillcolor{textcolor}%
\pgftext[x=1.482850in,y=1.696542in,left,base]{\color{textcolor}{\rmfamily\fontsize{12.000000}{14.400000}\selectfont\catcode`\^=\active\def^{\ifmmode\sp\else\^{}\fi}\catcode`\%=\active\def%{\%}$\triangle$-connected components}}%
\end{pgfscope}%
\end{pgfpicture}%
\makeatother%
\endgroup%
}
		\caption[Monoch. classes vs tr. con. components for minimally rigid]{%
			\centering Minimally rigid graphs}%
		\label{fig:monochrom_vs_triangle_minimally_rigid}
	\end{subfigure}
	\caption{Monochromatic classes vs \trcon{} components}%
	\label{fig:monochrom_vs_triangle}
\end{figure}

\NaiveCycles{} is no longer faster for finding a NAC-coloring
as shown in \Cref{fig:graph_globally_rigid_first_runtime}.
\None{} and \Neighbors{} strategies match the performance and
the other are not far behind.
%
\begin{figure}[thbp]
	\centering
	\scalebox{\BenchFigureScale}{%% Creator: Matplotlib, PGF backend
%%
%% To include the figure in your LaTeX document, write
%%   \input{<filename>.pgf}
%%
%% Make sure the required packages are loaded in your preamble
%%   \usepackage{pgf}
%%
%% Also ensure that all the required font packages are loaded; for instance,
%% the lmodern package is sometimes necessary when using math font.
%%   \usepackage{lmodern}
%%
%% Figures using additional raster images can only be included by \input if
%% they are in the same directory as the main LaTeX file. For loading figures
%% from other directories you can use the `import` package
%%   \usepackage{import}
%%
%% and then include the figures with
%%   \import{<path to file>}{<filename>.pgf}
%%
%% Matplotlib used the following preamble
%%   \def\mathdefault#1{#1}
%%   \everymath=\expandafter{\the\everymath\displaystyle}
%%   \IfFileExists{scrextend.sty}{
%%     \usepackage[fontsize=10.000000pt]{scrextend}
%%   }{
%%     \renewcommand{\normalsize}{\fontsize{10.000000}{12.000000}\selectfont}
%%     \normalsize
%%   }
%%   
%%   \ifdefined\pdftexversion\else  % non-pdftex case.
%%     \usepackage{fontspec}
%%     \setmainfont{DejaVuSans.ttf}[Path=\detokenize{/home/petr/Projects/PyRigi/.venv/lib/python3.12/site-packages/matplotlib/mpl-data/fonts/ttf/}]
%%     \setsansfont{DejaVuSans.ttf}[Path=\detokenize{/home/petr/Projects/PyRigi/.venv/lib/python3.12/site-packages/matplotlib/mpl-data/fonts/ttf/}]
%%     \setmonofont{DejaVuSansMono.ttf}[Path=\detokenize{/home/petr/Projects/PyRigi/.venv/lib/python3.12/site-packages/matplotlib/mpl-data/fonts/ttf/}]
%%   \fi
%%   \makeatletter\@ifpackageloaded{under\Score{}}{}{\usepackage[strings]{under\Score{}}}\makeatother
%%
\begingroup%
\makeatletter%
\begin{pgfpicture}%
\pgfpathrectangle{\pgfpointorigin}{\pgfqpoint{8.384376in}{2.841849in}}%
\pgfusepath{use as bounding box, clip}%
\begin{pgfscope}%
\pgfsetbuttcap%
\pgfsetmiterjoin%
\definecolor{currentfill}{rgb}{1.000000,1.000000,1.000000}%
\pgfsetfillcolor{currentfill}%
\pgfsetlinewidth{0.000000pt}%
\definecolor{currentstroke}{rgb}{1.000000,1.000000,1.000000}%
\pgfsetstrokecolor{currentstroke}%
\pgfsetdash{}{0pt}%
\pgfpathmoveto{\pgfqpoint{0.000000in}{0.000000in}}%
\pgfpathlineto{\pgfqpoint{8.384376in}{0.000000in}}%
\pgfpathlineto{\pgfqpoint{8.384376in}{2.841849in}}%
\pgfpathlineto{\pgfqpoint{0.000000in}{2.841849in}}%
\pgfpathlineto{\pgfqpoint{0.000000in}{0.000000in}}%
\pgfpathclose%
\pgfusepath{fill}%
\end{pgfscope}%
\begin{pgfscope}%
\pgfsetbuttcap%
\pgfsetmiterjoin%
\definecolor{currentfill}{rgb}{1.000000,1.000000,1.000000}%
\pgfsetfillcolor{currentfill}%
\pgfsetlinewidth{0.000000pt}%
\definecolor{currentstroke}{rgb}{0.000000,0.000000,0.000000}%
\pgfsetstrokecolor{currentstroke}%
\pgfsetstrokeopacity{0.000000}%
\pgfsetdash{}{0pt}%
\pgfpathmoveto{\pgfqpoint{0.588387in}{0.521603in}}%
\pgfpathlineto{\pgfqpoint{4.248423in}{0.521603in}}%
\pgfpathlineto{\pgfqpoint{4.248423in}{2.741849in}}%
\pgfpathlineto{\pgfqpoint{0.588387in}{2.741849in}}%
\pgfpathlineto{\pgfqpoint{0.588387in}{0.521603in}}%
\pgfpathclose%
\pgfusepath{fill}%
\end{pgfscope}%
\begin{pgfscope}%
\pgfsetbuttcap%
\pgfsetroundjoin%
\definecolor{currentfill}{rgb}{0.000000,0.000000,0.000000}%
\pgfsetfillcolor{currentfill}%
\pgfsetlinewidth{0.803000pt}%
\definecolor{currentstroke}{rgb}{0.000000,0.000000,0.000000}%
\pgfsetstrokecolor{currentstroke}%
\pgfsetdash{}{0pt}%
\pgfsys@defobject{currentmarker}{\pgfqpoint{0.000000in}{-0.048611in}}{\pgfqpoint{0.000000in}{0.000000in}}{%
\pgfpathmoveto{\pgfqpoint{0.000000in}{0.000000in}}%
\pgfpathlineto{\pgfqpoint{0.000000in}{-0.048611in}}%
\pgfusepath{stroke,fill}%
}%
\begin{pgfscope}%
\pgfsys@transformshift{0.896340in}{0.521603in}%
\pgfsys@useobject{currentmarker}{}%
\end{pgfscope}%
\end{pgfscope}%
\begin{pgfscope}%
\definecolor{textcolor}{rgb}{0.000000,0.000000,0.000000}%
\pgfsetstrokecolor{textcolor}%
\pgfsetfillcolor{textcolor}%
\pgftext[x=0.896340in,y=0.424381in,,top]{\color{textcolor}{\rmfamily\fontsize{10.000000}{12.000000}\selectfont\catcode`\^=\active\def^{\ifmmode\sp\else\^{}\fi}\catcode`\%=\active\def%{\%}$\mathdefault{12}$}}%
\end{pgfscope}%
\begin{pgfscope}%
\pgfsetbuttcap%
\pgfsetroundjoin%
\definecolor{currentfill}{rgb}{0.000000,0.000000,0.000000}%
\pgfsetfillcolor{currentfill}%
\pgfsetlinewidth{0.803000pt}%
\definecolor{currentstroke}{rgb}{0.000000,0.000000,0.000000}%
\pgfsetstrokecolor{currentstroke}%
\pgfsetdash{}{0pt}%
\pgfsys@defobject{currentmarker}{\pgfqpoint{0.000000in}{-0.048611in}}{\pgfqpoint{0.000000in}{0.000000in}}{%
\pgfpathmoveto{\pgfqpoint{0.000000in}{0.000000in}}%
\pgfpathlineto{\pgfqpoint{0.000000in}{-0.048611in}}%
\pgfusepath{stroke,fill}%
}%
\begin{pgfscope}%
\pgfsys@transformshift{1.321102in}{0.521603in}%
\pgfsys@useobject{currentmarker}{}%
\end{pgfscope}%
\end{pgfscope}%
\begin{pgfscope}%
\definecolor{textcolor}{rgb}{0.000000,0.000000,0.000000}%
\pgfsetstrokecolor{textcolor}%
\pgfsetfillcolor{textcolor}%
\pgftext[x=1.321102in,y=0.424381in,,top]{\color{textcolor}{\rmfamily\fontsize{10.000000}{12.000000}\selectfont\catcode`\^=\active\def^{\ifmmode\sp\else\^{}\fi}\catcode`\%=\active\def%{\%}$\mathdefault{18}$}}%
\end{pgfscope}%
\begin{pgfscope}%
\pgfsetbuttcap%
\pgfsetroundjoin%
\definecolor{currentfill}{rgb}{0.000000,0.000000,0.000000}%
\pgfsetfillcolor{currentfill}%
\pgfsetlinewidth{0.803000pt}%
\definecolor{currentstroke}{rgb}{0.000000,0.000000,0.000000}%
\pgfsetstrokecolor{currentstroke}%
\pgfsetdash{}{0pt}%
\pgfsys@defobject{currentmarker}{\pgfqpoint{0.000000in}{-0.048611in}}{\pgfqpoint{0.000000in}{0.000000in}}{%
\pgfpathmoveto{\pgfqpoint{0.000000in}{0.000000in}}%
\pgfpathlineto{\pgfqpoint{0.000000in}{-0.048611in}}%
\pgfusepath{stroke,fill}%
}%
\begin{pgfscope}%
\pgfsys@transformshift{1.745865in}{0.521603in}%
\pgfsys@useobject{currentmarker}{}%
\end{pgfscope}%
\end{pgfscope}%
\begin{pgfscope}%
\definecolor{textcolor}{rgb}{0.000000,0.000000,0.000000}%
\pgfsetstrokecolor{textcolor}%
\pgfsetfillcolor{textcolor}%
\pgftext[x=1.745865in,y=0.424381in,,top]{\color{textcolor}{\rmfamily\fontsize{10.000000}{12.000000}\selectfont\catcode`\^=\active\def^{\ifmmode\sp\else\^{}\fi}\catcode`\%=\active\def%{\%}$\mathdefault{24}$}}%
\end{pgfscope}%
\begin{pgfscope}%
\pgfsetbuttcap%
\pgfsetroundjoin%
\definecolor{currentfill}{rgb}{0.000000,0.000000,0.000000}%
\pgfsetfillcolor{currentfill}%
\pgfsetlinewidth{0.803000pt}%
\definecolor{currentstroke}{rgb}{0.000000,0.000000,0.000000}%
\pgfsetstrokecolor{currentstroke}%
\pgfsetdash{}{0pt}%
\pgfsys@defobject{currentmarker}{\pgfqpoint{0.000000in}{-0.048611in}}{\pgfqpoint{0.000000in}{0.000000in}}{%
\pgfpathmoveto{\pgfqpoint{0.000000in}{0.000000in}}%
\pgfpathlineto{\pgfqpoint{0.000000in}{-0.048611in}}%
\pgfusepath{stroke,fill}%
}%
\begin{pgfscope}%
\pgfsys@transformshift{2.170627in}{0.521603in}%
\pgfsys@useobject{currentmarker}{}%
\end{pgfscope}%
\end{pgfscope}%
\begin{pgfscope}%
\definecolor{textcolor}{rgb}{0.000000,0.000000,0.000000}%
\pgfsetstrokecolor{textcolor}%
\pgfsetfillcolor{textcolor}%
\pgftext[x=2.170627in,y=0.424381in,,top]{\color{textcolor}{\rmfamily\fontsize{10.000000}{12.000000}\selectfont\catcode`\^=\active\def^{\ifmmode\sp\else\^{}\fi}\catcode`\%=\active\def%{\%}$\mathdefault{30}$}}%
\end{pgfscope}%
\begin{pgfscope}%
\pgfsetbuttcap%
\pgfsetroundjoin%
\definecolor{currentfill}{rgb}{0.000000,0.000000,0.000000}%
\pgfsetfillcolor{currentfill}%
\pgfsetlinewidth{0.803000pt}%
\definecolor{currentstroke}{rgb}{0.000000,0.000000,0.000000}%
\pgfsetstrokecolor{currentstroke}%
\pgfsetdash{}{0pt}%
\pgfsys@defobject{currentmarker}{\pgfqpoint{0.000000in}{-0.048611in}}{\pgfqpoint{0.000000in}{0.000000in}}{%
\pgfpathmoveto{\pgfqpoint{0.000000in}{0.000000in}}%
\pgfpathlineto{\pgfqpoint{0.000000in}{-0.048611in}}%
\pgfusepath{stroke,fill}%
}%
\begin{pgfscope}%
\pgfsys@transformshift{2.595389in}{0.521603in}%
\pgfsys@useobject{currentmarker}{}%
\end{pgfscope}%
\end{pgfscope}%
\begin{pgfscope}%
\definecolor{textcolor}{rgb}{0.000000,0.000000,0.000000}%
\pgfsetstrokecolor{textcolor}%
\pgfsetfillcolor{textcolor}%
\pgftext[x=2.595389in,y=0.424381in,,top]{\color{textcolor}{\rmfamily\fontsize{10.000000}{12.000000}\selectfont\catcode`\^=\active\def^{\ifmmode\sp\else\^{}\fi}\catcode`\%=\active\def%{\%}$\mathdefault{36}$}}%
\end{pgfscope}%
\begin{pgfscope}%
\pgfsetbuttcap%
\pgfsetroundjoin%
\definecolor{currentfill}{rgb}{0.000000,0.000000,0.000000}%
\pgfsetfillcolor{currentfill}%
\pgfsetlinewidth{0.803000pt}%
\definecolor{currentstroke}{rgb}{0.000000,0.000000,0.000000}%
\pgfsetstrokecolor{currentstroke}%
\pgfsetdash{}{0pt}%
\pgfsys@defobject{currentmarker}{\pgfqpoint{0.000000in}{-0.048611in}}{\pgfqpoint{0.000000in}{0.000000in}}{%
\pgfpathmoveto{\pgfqpoint{0.000000in}{0.000000in}}%
\pgfpathlineto{\pgfqpoint{0.000000in}{-0.048611in}}%
\pgfusepath{stroke,fill}%
}%
\begin{pgfscope}%
\pgfsys@transformshift{3.020152in}{0.521603in}%
\pgfsys@useobject{currentmarker}{}%
\end{pgfscope}%
\end{pgfscope}%
\begin{pgfscope}%
\definecolor{textcolor}{rgb}{0.000000,0.000000,0.000000}%
\pgfsetstrokecolor{textcolor}%
\pgfsetfillcolor{textcolor}%
\pgftext[x=3.020152in,y=0.424381in,,top]{\color{textcolor}{\rmfamily\fontsize{10.000000}{12.000000}\selectfont\catcode`\^=\active\def^{\ifmmode\sp\else\^{}\fi}\catcode`\%=\active\def%{\%}$\mathdefault{42}$}}%
\end{pgfscope}%
\begin{pgfscope}%
\pgfsetbuttcap%
\pgfsetroundjoin%
\definecolor{currentfill}{rgb}{0.000000,0.000000,0.000000}%
\pgfsetfillcolor{currentfill}%
\pgfsetlinewidth{0.803000pt}%
\definecolor{currentstroke}{rgb}{0.000000,0.000000,0.000000}%
\pgfsetstrokecolor{currentstroke}%
\pgfsetdash{}{0pt}%
\pgfsys@defobject{currentmarker}{\pgfqpoint{0.000000in}{-0.048611in}}{\pgfqpoint{0.000000in}{0.000000in}}{%
\pgfpathmoveto{\pgfqpoint{0.000000in}{0.000000in}}%
\pgfpathlineto{\pgfqpoint{0.000000in}{-0.048611in}}%
\pgfusepath{stroke,fill}%
}%
\begin{pgfscope}%
\pgfsys@transformshift{3.444914in}{0.521603in}%
\pgfsys@useobject{currentmarker}{}%
\end{pgfscope}%
\end{pgfscope}%
\begin{pgfscope}%
\definecolor{textcolor}{rgb}{0.000000,0.000000,0.000000}%
\pgfsetstrokecolor{textcolor}%
\pgfsetfillcolor{textcolor}%
\pgftext[x=3.444914in,y=0.424381in,,top]{\color{textcolor}{\rmfamily\fontsize{10.000000}{12.000000}\selectfont\catcode`\^=\active\def^{\ifmmode\sp\else\^{}\fi}\catcode`\%=\active\def%{\%}$\mathdefault{48}$}}%
\end{pgfscope}%
\begin{pgfscope}%
\pgfsetbuttcap%
\pgfsetroundjoin%
\definecolor{currentfill}{rgb}{0.000000,0.000000,0.000000}%
\pgfsetfillcolor{currentfill}%
\pgfsetlinewidth{0.803000pt}%
\definecolor{currentstroke}{rgb}{0.000000,0.000000,0.000000}%
\pgfsetstrokecolor{currentstroke}%
\pgfsetdash{}{0pt}%
\pgfsys@defobject{currentmarker}{\pgfqpoint{0.000000in}{-0.048611in}}{\pgfqpoint{0.000000in}{0.000000in}}{%
\pgfpathmoveto{\pgfqpoint{0.000000in}{0.000000in}}%
\pgfpathlineto{\pgfqpoint{0.000000in}{-0.048611in}}%
\pgfusepath{stroke,fill}%
}%
\begin{pgfscope}%
\pgfsys@transformshift{3.869676in}{0.521603in}%
\pgfsys@useobject{currentmarker}{}%
\end{pgfscope}%
\end{pgfscope}%
\begin{pgfscope}%
\definecolor{textcolor}{rgb}{0.000000,0.000000,0.000000}%
\pgfsetstrokecolor{textcolor}%
\pgfsetfillcolor{textcolor}%
\pgftext[x=3.869676in,y=0.424381in,,top]{\color{textcolor}{\rmfamily\fontsize{10.000000}{12.000000}\selectfont\catcode`\^=\active\def^{\ifmmode\sp\else\^{}\fi}\catcode`\%=\active\def%{\%}$\mathdefault{54}$}}%
\end{pgfscope}%
\begin{pgfscope}%
\definecolor{textcolor}{rgb}{0.000000,0.000000,0.000000}%
\pgfsetstrokecolor{textcolor}%
\pgfsetfillcolor{textcolor}%
\pgftext[x=2.418405in,y=0.234413in,,top]{\color{textcolor}{\rmfamily\fontsize{10.000000}{12.000000}\selectfont\catcode`\^=\active\def^{\ifmmode\sp\else\^{}\fi}\catcode`\%=\active\def%{\%}Vertices}}%
\end{pgfscope}%
\begin{pgfscope}%
\pgfsetbuttcap%
\pgfsetroundjoin%
\definecolor{currentfill}{rgb}{0.000000,0.000000,0.000000}%
\pgfsetfillcolor{currentfill}%
\pgfsetlinewidth{0.803000pt}%
\definecolor{currentstroke}{rgb}{0.000000,0.000000,0.000000}%
\pgfsetstrokecolor{currentstroke}%
\pgfsetdash{}{0pt}%
\pgfsys@defobject{currentmarker}{\pgfqpoint{-0.048611in}{0.000000in}}{\pgfqpoint{-0.000000in}{0.000000in}}{%
\pgfpathmoveto{\pgfqpoint{-0.000000in}{0.000000in}}%
\pgfpathlineto{\pgfqpoint{-0.048611in}{0.000000in}}%
\pgfusepath{stroke,fill}%
}%
\begin{pgfscope}%
\pgfsys@transformshift{0.588387in}{1.110268in}%
\pgfsys@useobject{currentmarker}{}%
\end{pgfscope}%
\end{pgfscope}%
\begin{pgfscope}%
\definecolor{textcolor}{rgb}{0.000000,0.000000,0.000000}%
\pgfsetstrokecolor{textcolor}%
\pgfsetfillcolor{textcolor}%
\pgftext[x=0.289968in, y=1.057506in, left, base]{\color{textcolor}{\rmfamily\fontsize{10.000000}{12.000000}\selectfont\catcode`\^=\active\def^{\ifmmode\sp\else\^{}\fi}\catcode`\%=\active\def%{\%}$\mathdefault{10^{1}}$}}%
\end{pgfscope}%
\begin{pgfscope}%
\pgfsetbuttcap%
\pgfsetroundjoin%
\definecolor{currentfill}{rgb}{0.000000,0.000000,0.000000}%
\pgfsetfillcolor{currentfill}%
\pgfsetlinewidth{0.803000pt}%
\definecolor{currentstroke}{rgb}{0.000000,0.000000,0.000000}%
\pgfsetstrokecolor{currentstroke}%
\pgfsetdash{}{0pt}%
\pgfsys@defobject{currentmarker}{\pgfqpoint{-0.048611in}{0.000000in}}{\pgfqpoint{-0.000000in}{0.000000in}}{%
\pgfpathmoveto{\pgfqpoint{-0.000000in}{0.000000in}}%
\pgfpathlineto{\pgfqpoint{-0.048611in}{0.000000in}}%
\pgfusepath{stroke,fill}%
}%
\begin{pgfscope}%
\pgfsys@transformshift{0.588387in}{2.667503in}%
\pgfsys@useobject{currentmarker}{}%
\end{pgfscope}%
\end{pgfscope}%
\begin{pgfscope}%
\definecolor{textcolor}{rgb}{0.000000,0.000000,0.000000}%
\pgfsetstrokecolor{textcolor}%
\pgfsetfillcolor{textcolor}%
\pgftext[x=0.289968in, y=2.614741in, left, base]{\color{textcolor}{\rmfamily\fontsize{10.000000}{12.000000}\selectfont\catcode`\^=\active\def^{\ifmmode\sp\else\^{}\fi}\catcode`\%=\active\def%{\%}$\mathdefault{10^{2}}$}}%
\end{pgfscope}%
\begin{pgfscope}%
\pgfsetbuttcap%
\pgfsetroundjoin%
\definecolor{currentfill}{rgb}{0.000000,0.000000,0.000000}%
\pgfsetfillcolor{currentfill}%
\pgfsetlinewidth{0.602250pt}%
\definecolor{currentstroke}{rgb}{0.000000,0.000000,0.000000}%
\pgfsetstrokecolor{currentstroke}%
\pgfsetdash{}{0pt}%
\pgfsys@defobject{currentmarker}{\pgfqpoint{-0.027778in}{0.000000in}}{\pgfqpoint{-0.000000in}{0.000000in}}{%
\pgfpathmoveto{\pgfqpoint{-0.000000in}{0.000000in}}%
\pgfpathlineto{\pgfqpoint{-0.027778in}{0.000000in}}%
\pgfusepath{stroke,fill}%
}%
\begin{pgfscope}%
\pgfsys@transformshift{0.588387in}{0.641493in}%
\pgfsys@useobject{currentmarker}{}%
\end{pgfscope}%
\end{pgfscope}%
\begin{pgfscope}%
\pgfsetbuttcap%
\pgfsetroundjoin%
\definecolor{currentfill}{rgb}{0.000000,0.000000,0.000000}%
\pgfsetfillcolor{currentfill}%
\pgfsetlinewidth{0.602250pt}%
\definecolor{currentstroke}{rgb}{0.000000,0.000000,0.000000}%
\pgfsetstrokecolor{currentstroke}%
\pgfsetdash{}{0pt}%
\pgfsys@defobject{currentmarker}{\pgfqpoint{-0.027778in}{0.000000in}}{\pgfqpoint{-0.000000in}{0.000000in}}{%
\pgfpathmoveto{\pgfqpoint{-0.000000in}{0.000000in}}%
\pgfpathlineto{\pgfqpoint{-0.027778in}{0.000000in}}%
\pgfusepath{stroke,fill}%
}%
\begin{pgfscope}%
\pgfsys@transformshift{0.588387in}{0.764797in}%
\pgfsys@useobject{currentmarker}{}%
\end{pgfscope}%
\end{pgfscope}%
\begin{pgfscope}%
\pgfsetbuttcap%
\pgfsetroundjoin%
\definecolor{currentfill}{rgb}{0.000000,0.000000,0.000000}%
\pgfsetfillcolor{currentfill}%
\pgfsetlinewidth{0.602250pt}%
\definecolor{currentstroke}{rgb}{0.000000,0.000000,0.000000}%
\pgfsetstrokecolor{currentstroke}%
\pgfsetdash{}{0pt}%
\pgfsys@defobject{currentmarker}{\pgfqpoint{-0.027778in}{0.000000in}}{\pgfqpoint{-0.000000in}{0.000000in}}{%
\pgfpathmoveto{\pgfqpoint{-0.000000in}{0.000000in}}%
\pgfpathlineto{\pgfqpoint{-0.027778in}{0.000000in}}%
\pgfusepath{stroke,fill}%
}%
\begin{pgfscope}%
\pgfsys@transformshift{0.588387in}{0.869049in}%
\pgfsys@useobject{currentmarker}{}%
\end{pgfscope}%
\end{pgfscope}%
\begin{pgfscope}%
\pgfsetbuttcap%
\pgfsetroundjoin%
\definecolor{currentfill}{rgb}{0.000000,0.000000,0.000000}%
\pgfsetfillcolor{currentfill}%
\pgfsetlinewidth{0.602250pt}%
\definecolor{currentstroke}{rgb}{0.000000,0.000000,0.000000}%
\pgfsetstrokecolor{currentstroke}%
\pgfsetdash{}{0pt}%
\pgfsys@defobject{currentmarker}{\pgfqpoint{-0.027778in}{0.000000in}}{\pgfqpoint{-0.000000in}{0.000000in}}{%
\pgfpathmoveto{\pgfqpoint{-0.000000in}{0.000000in}}%
\pgfpathlineto{\pgfqpoint{-0.027778in}{0.000000in}}%
\pgfusepath{stroke,fill}%
}%
\begin{pgfscope}%
\pgfsys@transformshift{0.588387in}{0.959356in}%
\pgfsys@useobject{currentmarker}{}%
\end{pgfscope}%
\end{pgfscope}%
\begin{pgfscope}%
\pgfsetbuttcap%
\pgfsetroundjoin%
\definecolor{currentfill}{rgb}{0.000000,0.000000,0.000000}%
\pgfsetfillcolor{currentfill}%
\pgfsetlinewidth{0.602250pt}%
\definecolor{currentstroke}{rgb}{0.000000,0.000000,0.000000}%
\pgfsetstrokecolor{currentstroke}%
\pgfsetdash{}{0pt}%
\pgfsys@defobject{currentmarker}{\pgfqpoint{-0.027778in}{0.000000in}}{\pgfqpoint{-0.000000in}{0.000000in}}{%
\pgfpathmoveto{\pgfqpoint{-0.000000in}{0.000000in}}%
\pgfpathlineto{\pgfqpoint{-0.027778in}{0.000000in}}%
\pgfusepath{stroke,fill}%
}%
\begin{pgfscope}%
\pgfsys@transformshift{0.588387in}{1.039013in}%
\pgfsys@useobject{currentmarker}{}%
\end{pgfscope}%
\end{pgfscope}%
\begin{pgfscope}%
\pgfsetbuttcap%
\pgfsetroundjoin%
\definecolor{currentfill}{rgb}{0.000000,0.000000,0.000000}%
\pgfsetfillcolor{currentfill}%
\pgfsetlinewidth{0.602250pt}%
\definecolor{currentstroke}{rgb}{0.000000,0.000000,0.000000}%
\pgfsetstrokecolor{currentstroke}%
\pgfsetdash{}{0pt}%
\pgfsys@defobject{currentmarker}{\pgfqpoint{-0.027778in}{0.000000in}}{\pgfqpoint{-0.000000in}{0.000000in}}{%
\pgfpathmoveto{\pgfqpoint{-0.000000in}{0.000000in}}%
\pgfpathlineto{\pgfqpoint{-0.027778in}{0.000000in}}%
\pgfusepath{stroke,fill}%
}%
\begin{pgfscope}%
\pgfsys@transformshift{0.588387in}{1.579042in}%
\pgfsys@useobject{currentmarker}{}%
\end{pgfscope}%
\end{pgfscope}%
\begin{pgfscope}%
\pgfsetbuttcap%
\pgfsetroundjoin%
\definecolor{currentfill}{rgb}{0.000000,0.000000,0.000000}%
\pgfsetfillcolor{currentfill}%
\pgfsetlinewidth{0.602250pt}%
\definecolor{currentstroke}{rgb}{0.000000,0.000000,0.000000}%
\pgfsetstrokecolor{currentstroke}%
\pgfsetdash{}{0pt}%
\pgfsys@defobject{currentmarker}{\pgfqpoint{-0.027778in}{0.000000in}}{\pgfqpoint{-0.000000in}{0.000000in}}{%
\pgfpathmoveto{\pgfqpoint{-0.000000in}{0.000000in}}%
\pgfpathlineto{\pgfqpoint{-0.027778in}{0.000000in}}%
\pgfusepath{stroke,fill}%
}%
\begin{pgfscope}%
\pgfsys@transformshift{0.588387in}{1.853258in}%
\pgfsys@useobject{currentmarker}{}%
\end{pgfscope}%
\end{pgfscope}%
\begin{pgfscope}%
\pgfsetbuttcap%
\pgfsetroundjoin%
\definecolor{currentfill}{rgb}{0.000000,0.000000,0.000000}%
\pgfsetfillcolor{currentfill}%
\pgfsetlinewidth{0.602250pt}%
\definecolor{currentstroke}{rgb}{0.000000,0.000000,0.000000}%
\pgfsetstrokecolor{currentstroke}%
\pgfsetdash{}{0pt}%
\pgfsys@defobject{currentmarker}{\pgfqpoint{-0.027778in}{0.000000in}}{\pgfqpoint{-0.000000in}{0.000000in}}{%
\pgfpathmoveto{\pgfqpoint{-0.000000in}{0.000000in}}%
\pgfpathlineto{\pgfqpoint{-0.027778in}{0.000000in}}%
\pgfusepath{stroke,fill}%
}%
\begin{pgfscope}%
\pgfsys@transformshift{0.588387in}{2.047817in}%
\pgfsys@useobject{currentmarker}{}%
\end{pgfscope}%
\end{pgfscope}%
\begin{pgfscope}%
\pgfsetbuttcap%
\pgfsetroundjoin%
\definecolor{currentfill}{rgb}{0.000000,0.000000,0.000000}%
\pgfsetfillcolor{currentfill}%
\pgfsetlinewidth{0.602250pt}%
\definecolor{currentstroke}{rgb}{0.000000,0.000000,0.000000}%
\pgfsetstrokecolor{currentstroke}%
\pgfsetdash{}{0pt}%
\pgfsys@defobject{currentmarker}{\pgfqpoint{-0.027778in}{0.000000in}}{\pgfqpoint{-0.000000in}{0.000000in}}{%
\pgfpathmoveto{\pgfqpoint{-0.000000in}{0.000000in}}%
\pgfpathlineto{\pgfqpoint{-0.027778in}{0.000000in}}%
\pgfusepath{stroke,fill}%
}%
\begin{pgfscope}%
\pgfsys@transformshift{0.588387in}{2.198728in}%
\pgfsys@useobject{currentmarker}{}%
\end{pgfscope}%
\end{pgfscope}%
\begin{pgfscope}%
\pgfsetbuttcap%
\pgfsetroundjoin%
\definecolor{currentfill}{rgb}{0.000000,0.000000,0.000000}%
\pgfsetfillcolor{currentfill}%
\pgfsetlinewidth{0.602250pt}%
\definecolor{currentstroke}{rgb}{0.000000,0.000000,0.000000}%
\pgfsetstrokecolor{currentstroke}%
\pgfsetdash{}{0pt}%
\pgfsys@defobject{currentmarker}{\pgfqpoint{-0.027778in}{0.000000in}}{\pgfqpoint{-0.000000in}{0.000000in}}{%
\pgfpathmoveto{\pgfqpoint{-0.000000in}{0.000000in}}%
\pgfpathlineto{\pgfqpoint{-0.027778in}{0.000000in}}%
\pgfusepath{stroke,fill}%
}%
\begin{pgfscope}%
\pgfsys@transformshift{0.588387in}{2.322032in}%
\pgfsys@useobject{currentmarker}{}%
\end{pgfscope}%
\end{pgfscope}%
\begin{pgfscope}%
\pgfsetbuttcap%
\pgfsetroundjoin%
\definecolor{currentfill}{rgb}{0.000000,0.000000,0.000000}%
\pgfsetfillcolor{currentfill}%
\pgfsetlinewidth{0.602250pt}%
\definecolor{currentstroke}{rgb}{0.000000,0.000000,0.000000}%
\pgfsetstrokecolor{currentstroke}%
\pgfsetdash{}{0pt}%
\pgfsys@defobject{currentmarker}{\pgfqpoint{-0.027778in}{0.000000in}}{\pgfqpoint{-0.000000in}{0.000000in}}{%
\pgfpathmoveto{\pgfqpoint{-0.000000in}{0.000000in}}%
\pgfpathlineto{\pgfqpoint{-0.027778in}{0.000000in}}%
\pgfusepath{stroke,fill}%
}%
\begin{pgfscope}%
\pgfsys@transformshift{0.588387in}{2.426284in}%
\pgfsys@useobject{currentmarker}{}%
\end{pgfscope}%
\end{pgfscope}%
\begin{pgfscope}%
\pgfsetbuttcap%
\pgfsetroundjoin%
\definecolor{currentfill}{rgb}{0.000000,0.000000,0.000000}%
\pgfsetfillcolor{currentfill}%
\pgfsetlinewidth{0.602250pt}%
\definecolor{currentstroke}{rgb}{0.000000,0.000000,0.000000}%
\pgfsetstrokecolor{currentstroke}%
\pgfsetdash{}{0pt}%
\pgfsys@defobject{currentmarker}{\pgfqpoint{-0.027778in}{0.000000in}}{\pgfqpoint{-0.000000in}{0.000000in}}{%
\pgfpathmoveto{\pgfqpoint{-0.000000in}{0.000000in}}%
\pgfpathlineto{\pgfqpoint{-0.027778in}{0.000000in}}%
\pgfusepath{stroke,fill}%
}%
\begin{pgfscope}%
\pgfsys@transformshift{0.588387in}{2.516591in}%
\pgfsys@useobject{currentmarker}{}%
\end{pgfscope}%
\end{pgfscope}%
\begin{pgfscope}%
\pgfsetbuttcap%
\pgfsetroundjoin%
\definecolor{currentfill}{rgb}{0.000000,0.000000,0.000000}%
\pgfsetfillcolor{currentfill}%
\pgfsetlinewidth{0.602250pt}%
\definecolor{currentstroke}{rgb}{0.000000,0.000000,0.000000}%
\pgfsetstrokecolor{currentstroke}%
\pgfsetdash{}{0pt}%
\pgfsys@defobject{currentmarker}{\pgfqpoint{-0.027778in}{0.000000in}}{\pgfqpoint{-0.000000in}{0.000000in}}{%
\pgfpathmoveto{\pgfqpoint{-0.000000in}{0.000000in}}%
\pgfpathlineto{\pgfqpoint{-0.027778in}{0.000000in}}%
\pgfusepath{stroke,fill}%
}%
\begin{pgfscope}%
\pgfsys@transformshift{0.588387in}{2.596248in}%
\pgfsys@useobject{currentmarker}{}%
\end{pgfscope}%
\end{pgfscope}%
\begin{pgfscope}%
\definecolor{textcolor}{rgb}{0.000000,0.000000,0.000000}%
\pgfsetstrokecolor{textcolor}%
\pgfsetfillcolor{textcolor}%
\pgftext[x=0.234413in,y=1.631726in,,bottom,rotate=90.000000]{\color{textcolor}{\rmfamily\fontsize{10.000000}{12.000000}\selectfont\catcode`\^=\active\def^{\ifmmode\sp\else\^{}\fi}\catcode`\%=\active\def%{\%}Time [ms]}}%
\end{pgfscope}%
\begin{pgfscope}%
\pgfpathrectangle{\pgfqpoint{0.588387in}{0.521603in}}{\pgfqpoint{3.660036in}{2.220246in}}%
\pgfusepath{clip}%
\pgfsetrectcap%
\pgfsetroundjoin%
\pgfsetlinewidth{1.505625pt}%
\pgfsetstrokecolor{currentstroke1}%
\pgfsetdash{}{0pt}%
\pgfpathmoveto{\pgfqpoint{0.754752in}{0.703411in}}%
\pgfpathlineto{\pgfqpoint{0.825546in}{0.764207in}}%
\pgfpathlineto{\pgfqpoint{0.896340in}{0.854526in}}%
\pgfpathlineto{\pgfqpoint{0.967134in}{0.945554in}}%
\pgfpathlineto{\pgfqpoint{1.037927in}{1.019574in}}%
\pgfpathlineto{\pgfqpoint{1.108721in}{1.103129in}}%
\pgfpathlineto{\pgfqpoint{1.179515in}{1.194716in}}%
\pgfpathlineto{\pgfqpoint{1.250309in}{1.245581in}}%
\pgfpathlineto{\pgfqpoint{1.321102in}{1.309208in}}%
\pgfpathlineto{\pgfqpoint{1.391896in}{1.374262in}}%
\pgfpathlineto{\pgfqpoint{1.462690in}{1.392815in}}%
\pgfpathlineto{\pgfqpoint{1.533483in}{1.453639in}}%
\pgfpathlineto{\pgfqpoint{1.604277in}{1.567618in}}%
\pgfpathlineto{\pgfqpoint{1.675071in}{1.568134in}}%
\pgfpathlineto{\pgfqpoint{1.745865in}{1.614623in}}%
\pgfpathlineto{\pgfqpoint{1.816658in}{1.668546in}}%
\pgfpathlineto{\pgfqpoint{1.887452in}{1.702487in}}%
\pgfpathlineto{\pgfqpoint{1.958246in}{1.740289in}}%
\pgfpathlineto{\pgfqpoint{2.029039in}{1.804421in}}%
\pgfpathlineto{\pgfqpoint{2.099833in}{1.864104in}}%
\pgfpathlineto{\pgfqpoint{2.170627in}{1.885610in}}%
\pgfpathlineto{\pgfqpoint{2.241421in}{1.956833in}}%
\pgfpathlineto{\pgfqpoint{2.312214in}{1.969100in}}%
\pgfpathlineto{\pgfqpoint{2.383008in}{2.016161in}}%
\pgfpathlineto{\pgfqpoint{2.453802in}{2.000124in}}%
\pgfpathlineto{\pgfqpoint{2.524596in}{2.060380in}}%
\pgfpathlineto{\pgfqpoint{2.595389in}{2.099422in}}%
\pgfpathlineto{\pgfqpoint{2.666183in}{2.103803in}}%
\pgfpathlineto{\pgfqpoint{2.736977in}{2.180969in}}%
\pgfpathlineto{\pgfqpoint{2.807770in}{2.196724in}}%
\pgfpathlineto{\pgfqpoint{2.878564in}{2.201552in}}%
\pgfpathlineto{\pgfqpoint{2.949358in}{2.221480in}}%
\pgfpathlineto{\pgfqpoint{3.020152in}{2.314666in}}%
\pgfpathlineto{\pgfqpoint{3.090945in}{2.286503in}}%
\pgfpathlineto{\pgfqpoint{3.161739in}{2.299105in}}%
\pgfpathlineto{\pgfqpoint{3.232533in}{2.329244in}}%
\pgfpathlineto{\pgfqpoint{3.303327in}{2.337978in}}%
\pgfpathlineto{\pgfqpoint{3.374120in}{2.389176in}}%
\pgfpathlineto{\pgfqpoint{3.444914in}{2.449302in}}%
\pgfpathlineto{\pgfqpoint{3.515708in}{2.449776in}}%
\pgfpathlineto{\pgfqpoint{3.586501in}{2.470720in}}%
\pgfpathlineto{\pgfqpoint{3.657295in}{2.491459in}}%
\pgfpathlineto{\pgfqpoint{3.728089in}{2.550124in}}%
\pgfpathlineto{\pgfqpoint{3.798883in}{2.527628in}}%
\pgfpathlineto{\pgfqpoint{3.869676in}{2.498544in}}%
\pgfpathlineto{\pgfqpoint{3.940470in}{2.581480in}}%
\pgfpathlineto{\pgfqpoint{4.011264in}{2.629696in}}%
\pgfpathlineto{\pgfqpoint{4.082057in}{2.629244in}}%
\pgfusepath{stroke}%
\end{pgfscope}%
\begin{pgfscope}%
\pgfpathrectangle{\pgfqpoint{0.588387in}{0.521603in}}{\pgfqpoint{3.660036in}{2.220246in}}%
\pgfusepath{clip}%
\pgfsetrectcap%
\pgfsetroundjoin%
\pgfsetlinewidth{1.505625pt}%
\pgfsetstrokecolor{currentstroke2}%
\pgfsetdash{}{0pt}%
\pgfpathmoveto{\pgfqpoint{0.754752in}{0.691366in}}%
\pgfpathlineto{\pgfqpoint{0.825546in}{0.761228in}}%
\pgfpathlineto{\pgfqpoint{0.896340in}{0.864024in}}%
\pgfpathlineto{\pgfqpoint{0.967134in}{0.989419in}}%
\pgfpathlineto{\pgfqpoint{1.037927in}{1.036377in}}%
\pgfpathlineto{\pgfqpoint{1.108721in}{1.136142in}}%
\pgfpathlineto{\pgfqpoint{1.179515in}{1.227884in}}%
\pgfpathlineto{\pgfqpoint{1.250309in}{1.249721in}}%
\pgfpathlineto{\pgfqpoint{1.321102in}{1.313479in}}%
\pgfpathlineto{\pgfqpoint{1.391896in}{1.400523in}}%
\pgfpathlineto{\pgfqpoint{1.462690in}{1.397053in}}%
\pgfpathlineto{\pgfqpoint{1.533483in}{1.453231in}}%
\pgfpathlineto{\pgfqpoint{1.604277in}{1.561658in}}%
\pgfpathlineto{\pgfqpoint{1.675071in}{1.572554in}}%
\pgfpathlineto{\pgfqpoint{1.745865in}{1.614771in}}%
\pgfpathlineto{\pgfqpoint{1.816658in}{1.652337in}}%
\pgfpathlineto{\pgfqpoint{1.887452in}{1.697514in}}%
\pgfpathlineto{\pgfqpoint{1.958246in}{1.720599in}}%
\pgfpathlineto{\pgfqpoint{2.029039in}{1.807201in}}%
\pgfpathlineto{\pgfqpoint{2.099833in}{1.869737in}}%
\pgfpathlineto{\pgfqpoint{2.170627in}{1.886576in}}%
\pgfpathlineto{\pgfqpoint{2.241421in}{1.935994in}}%
\pgfpathlineto{\pgfqpoint{2.312214in}{1.958668in}}%
\pgfpathlineto{\pgfqpoint{2.383008in}{1.994999in}}%
\pgfpathlineto{\pgfqpoint{2.453802in}{1.997737in}}%
\pgfpathlineto{\pgfqpoint{2.524596in}{2.052136in}}%
\pgfpathlineto{\pgfqpoint{2.595389in}{2.105644in}}%
\pgfpathlineto{\pgfqpoint{2.666183in}{2.102131in}}%
\pgfpathlineto{\pgfqpoint{2.736977in}{2.179879in}}%
\pgfpathlineto{\pgfqpoint{2.807770in}{2.197201in}}%
\pgfpathlineto{\pgfqpoint{2.878564in}{2.203199in}}%
\pgfpathlineto{\pgfqpoint{2.949358in}{2.219779in}}%
\pgfpathlineto{\pgfqpoint{3.020152in}{2.309166in}}%
\pgfpathlineto{\pgfqpoint{3.090945in}{2.289676in}}%
\pgfpathlineto{\pgfqpoint{3.161739in}{2.296827in}}%
\pgfpathlineto{\pgfqpoint{3.232533in}{2.329988in}}%
\pgfpathlineto{\pgfqpoint{3.303327in}{2.332483in}}%
\pgfpathlineto{\pgfqpoint{3.374120in}{2.379310in}}%
\pgfpathlineto{\pgfqpoint{3.444914in}{2.432418in}}%
\pgfpathlineto{\pgfqpoint{3.515708in}{2.451697in}}%
\pgfpathlineto{\pgfqpoint{3.586501in}{2.462395in}}%
\pgfpathlineto{\pgfqpoint{3.657295in}{2.485434in}}%
\pgfpathlineto{\pgfqpoint{3.728089in}{2.546204in}}%
\pgfpathlineto{\pgfqpoint{3.798883in}{2.519027in}}%
\pgfpathlineto{\pgfqpoint{3.869676in}{2.497441in}}%
\pgfpathlineto{\pgfqpoint{3.940470in}{2.580693in}}%
\pgfpathlineto{\pgfqpoint{4.011264in}{2.625432in}}%
\pgfpathlineto{\pgfqpoint{4.082057in}{2.614109in}}%
\pgfusepath{stroke}%
\end{pgfscope}%
\begin{pgfscope}%
\pgfpathrectangle{\pgfqpoint{0.588387in}{0.521603in}}{\pgfqpoint{3.660036in}{2.220246in}}%
\pgfusepath{clip}%
\pgfsetrectcap%
\pgfsetroundjoin%
\pgfsetlinewidth{1.505625pt}%
\pgfsetstrokecolor{currentstroke3}%
\pgfsetdash{}{0pt}%
\pgfpathmoveto{\pgfqpoint{0.754752in}{0.622524in}}%
\pgfpathlineto{\pgfqpoint{0.825546in}{0.732009in}}%
\pgfpathlineto{\pgfqpoint{0.896340in}{0.772616in}}%
\pgfpathlineto{\pgfqpoint{0.967134in}{0.860502in}}%
\pgfpathlineto{\pgfqpoint{1.037927in}{0.962020in}}%
\pgfpathlineto{\pgfqpoint{1.108721in}{0.970367in}}%
\pgfpathlineto{\pgfqpoint{1.179515in}{1.073439in}}%
\pgfpathlineto{\pgfqpoint{1.250309in}{1.149037in}}%
\pgfpathlineto{\pgfqpoint{1.321102in}{1.157919in}}%
\pgfpathlineto{\pgfqpoint{1.391896in}{1.288224in}}%
\pgfpathlineto{\pgfqpoint{1.533483in}{1.345986in}}%
\pgfpathlineto{\pgfqpoint{1.604277in}{1.411981in}}%
\pgfpathlineto{\pgfqpoint{1.675071in}{1.500990in}}%
\pgfpathlineto{\pgfqpoint{1.745865in}{1.549318in}}%
\pgfpathlineto{\pgfqpoint{1.816658in}{1.576671in}}%
\pgfpathlineto{\pgfqpoint{1.887452in}{1.582079in}}%
\pgfpathlineto{\pgfqpoint{1.958246in}{1.641345in}}%
\pgfpathlineto{\pgfqpoint{2.029039in}{1.705158in}}%
\pgfpathlineto{\pgfqpoint{2.099833in}{1.744158in}}%
\pgfpathlineto{\pgfqpoint{2.170627in}{1.746687in}}%
\pgfpathlineto{\pgfqpoint{2.241421in}{1.849868in}}%
\pgfpathlineto{\pgfqpoint{2.312214in}{1.854159in}}%
\pgfpathlineto{\pgfqpoint{2.383008in}{1.896905in}}%
\pgfpathlineto{\pgfqpoint{2.453802in}{1.928491in}}%
\pgfpathlineto{\pgfqpoint{2.524596in}{1.970713in}}%
\pgfpathlineto{\pgfqpoint{2.595389in}{2.012593in}}%
\pgfpathlineto{\pgfqpoint{2.666183in}{2.057719in}}%
\pgfpathlineto{\pgfqpoint{2.736977in}{2.072548in}}%
\pgfpathlineto{\pgfqpoint{2.807770in}{2.099610in}}%
\pgfpathlineto{\pgfqpoint{2.878564in}{2.118090in}}%
\pgfpathlineto{\pgfqpoint{2.949358in}{2.137172in}}%
\pgfusepath{stroke}%
\end{pgfscope}%
\begin{pgfscope}%
\pgfpathrectangle{\pgfqpoint{0.588387in}{0.521603in}}{\pgfqpoint{3.660036in}{2.220246in}}%
\pgfusepath{clip}%
\pgfsetrectcap%
\pgfsetroundjoin%
\pgfsetlinewidth{1.505625pt}%
\pgfsetstrokecolor{currentstroke4}%
\pgfsetdash{}{0pt}%
\pgfpathmoveto{\pgfqpoint{0.754752in}{0.705005in}}%
\pgfpathlineto{\pgfqpoint{0.825546in}{0.761247in}}%
\pgfpathlineto{\pgfqpoint{0.896340in}{0.857375in}}%
\pgfpathlineto{\pgfqpoint{0.967134in}{0.942496in}}%
\pgfpathlineto{\pgfqpoint{1.037927in}{1.040908in}}%
\pgfpathlineto{\pgfqpoint{1.108721in}{1.065538in}}%
\pgfpathlineto{\pgfqpoint{1.179515in}{1.140036in}}%
\pgfpathlineto{\pgfqpoint{1.250309in}{1.190225in}}%
\pgfpathlineto{\pgfqpoint{1.321102in}{1.285881in}}%
\pgfpathlineto{\pgfqpoint{1.391896in}{1.354523in}}%
\pgfpathlineto{\pgfqpoint{1.462690in}{1.385834in}}%
\pgfpathlineto{\pgfqpoint{1.533483in}{1.406877in}}%
\pgfpathlineto{\pgfqpoint{1.604277in}{1.502317in}}%
\pgfpathlineto{\pgfqpoint{1.675071in}{1.521364in}}%
\pgfpathlineto{\pgfqpoint{1.745865in}{1.576162in}}%
\pgfpathlineto{\pgfqpoint{1.816658in}{1.620161in}}%
\pgfpathlineto{\pgfqpoint{1.887452in}{1.658248in}}%
\pgfpathlineto{\pgfqpoint{1.958246in}{1.702487in}}%
\pgfpathlineto{\pgfqpoint{2.029039in}{1.753611in}}%
\pgfpathlineto{\pgfqpoint{2.099833in}{1.808407in}}%
\pgfpathlineto{\pgfqpoint{2.170627in}{1.816548in}}%
\pgfpathlineto{\pgfqpoint{2.241421in}{1.886740in}}%
\pgfpathlineto{\pgfqpoint{2.312214in}{1.913605in}}%
\pgfpathlineto{\pgfqpoint{2.383008in}{1.958577in}}%
\pgfpathlineto{\pgfqpoint{2.453802in}{1.955671in}}%
\pgfpathlineto{\pgfqpoint{2.524596in}{2.005934in}}%
\pgfpathlineto{\pgfqpoint{2.595389in}{2.055667in}}%
\pgfpathlineto{\pgfqpoint{2.666183in}{2.056134in}}%
\pgfpathlineto{\pgfqpoint{2.736977in}{2.136083in}}%
\pgfpathlineto{\pgfqpoint{2.807770in}{2.173904in}}%
\pgfpathlineto{\pgfqpoint{2.878564in}{2.219723in}}%
\pgfpathlineto{\pgfqpoint{2.949358in}{2.188731in}}%
\pgfpathlineto{\pgfqpoint{3.020152in}{2.233309in}}%
\pgfpathlineto{\pgfqpoint{3.090945in}{2.236716in}}%
\pgfpathlineto{\pgfqpoint{3.161739in}{2.285442in}}%
\pgfpathlineto{\pgfqpoint{3.232533in}{2.288779in}}%
\pgfpathlineto{\pgfqpoint{3.303327in}{2.272622in}}%
\pgfpathlineto{\pgfqpoint{3.374120in}{2.316832in}}%
\pgfpathlineto{\pgfqpoint{3.444914in}{2.382276in}}%
\pgfpathlineto{\pgfqpoint{3.515708in}{2.388622in}}%
\pgfpathlineto{\pgfqpoint{3.586501in}{2.419412in}}%
\pgfpathlineto{\pgfqpoint{3.657295in}{2.513490in}}%
\pgfpathlineto{\pgfqpoint{3.728089in}{2.482127in}}%
\pgfpathlineto{\pgfqpoint{3.798883in}{2.524691in}}%
\pgfpathlineto{\pgfqpoint{3.869676in}{2.496602in}}%
\pgfpathlineto{\pgfqpoint{3.940470in}{2.581326in}}%
\pgfpathlineto{\pgfqpoint{4.011264in}{2.614389in}}%
\pgfpathlineto{\pgfqpoint{4.082057in}{2.615585in}}%
\pgfusepath{stroke}%
\end{pgfscope}%
\begin{pgfscope}%
\pgfpathrectangle{\pgfqpoint{0.588387in}{0.521603in}}{\pgfqpoint{3.660036in}{2.220246in}}%
\pgfusepath{clip}%
\pgfsetrectcap%
\pgfsetroundjoin%
\pgfsetlinewidth{1.505625pt}%
\pgfsetstrokecolor{currentstroke5}%
\pgfsetdash{}{0pt}%
\pgfpathmoveto{\pgfqpoint{0.754752in}{0.699298in}}%
\pgfpathlineto{\pgfqpoint{0.825546in}{0.766603in}}%
\pgfpathlineto{\pgfqpoint{0.896340in}{0.854811in}}%
\pgfpathlineto{\pgfqpoint{0.967134in}{0.969681in}}%
\pgfpathlineto{\pgfqpoint{1.037927in}{1.045776in}}%
\pgfpathlineto{\pgfqpoint{1.108721in}{1.094185in}}%
\pgfpathlineto{\pgfqpoint{1.179515in}{1.165440in}}%
\pgfpathlineto{\pgfqpoint{1.250309in}{1.219909in}}%
\pgfpathlineto{\pgfqpoint{1.321102in}{1.329028in}}%
\pgfpathlineto{\pgfqpoint{1.391896in}{1.391213in}}%
\pgfpathlineto{\pgfqpoint{1.462690in}{1.384934in}}%
\pgfpathlineto{\pgfqpoint{1.533483in}{1.413748in}}%
\pgfpathlineto{\pgfqpoint{1.604277in}{1.545246in}}%
\pgfpathlineto{\pgfqpoint{1.675071in}{1.513989in}}%
\pgfpathlineto{\pgfqpoint{1.745865in}{1.573269in}}%
\pgfpathlineto{\pgfqpoint{1.816658in}{1.626851in}}%
\pgfpathlineto{\pgfqpoint{1.887452in}{1.656591in}}%
\pgfpathlineto{\pgfqpoint{1.958246in}{1.699494in}}%
\pgfpathlineto{\pgfqpoint{2.029039in}{1.763208in}}%
\pgfpathlineto{\pgfqpoint{2.099833in}{1.811944in}}%
\pgfpathlineto{\pgfqpoint{2.170627in}{1.825036in}}%
\pgfpathlineto{\pgfqpoint{2.241421in}{1.899343in}}%
\pgfpathlineto{\pgfqpoint{2.312214in}{1.920987in}}%
\pgfpathlineto{\pgfqpoint{2.383008in}{1.948867in}}%
\pgfpathlineto{\pgfqpoint{2.453802in}{1.952370in}}%
\pgfpathlineto{\pgfqpoint{2.524596in}{2.008036in}}%
\pgfpathlineto{\pgfqpoint{2.595389in}{2.062430in}}%
\pgfpathlineto{\pgfqpoint{2.666183in}{2.055756in}}%
\pgfpathlineto{\pgfqpoint{2.736977in}{2.132930in}}%
\pgfpathlineto{\pgfqpoint{2.807770in}{2.168369in}}%
\pgfpathlineto{\pgfqpoint{2.878564in}{2.222727in}}%
\pgfpathlineto{\pgfqpoint{2.949358in}{2.180118in}}%
\pgfpathlineto{\pgfqpoint{3.020152in}{2.238391in}}%
\pgfpathlineto{\pgfqpoint{3.090945in}{2.237941in}}%
\pgfpathlineto{\pgfqpoint{3.161739in}{2.290080in}}%
\pgfpathlineto{\pgfqpoint{3.232533in}{2.285084in}}%
\pgfpathlineto{\pgfqpoint{3.303327in}{2.274145in}}%
\pgfpathlineto{\pgfqpoint{3.374120in}{2.326694in}}%
\pgfpathlineto{\pgfqpoint{3.444914in}{2.383871in}}%
\pgfpathlineto{\pgfqpoint{3.515708in}{2.389088in}}%
\pgfpathlineto{\pgfqpoint{3.586501in}{2.436226in}}%
\pgfpathlineto{\pgfqpoint{3.657295in}{2.523417in}}%
\pgfpathlineto{\pgfqpoint{3.728089in}{2.489655in}}%
\pgfpathlineto{\pgfqpoint{3.798883in}{2.535830in}}%
\pgfpathlineto{\pgfqpoint{3.869676in}{2.497963in}}%
\pgfpathlineto{\pgfqpoint{3.940470in}{2.594735in}}%
\pgfpathlineto{\pgfqpoint{4.011264in}{2.640929in}}%
\pgfpathlineto{\pgfqpoint{4.082057in}{2.623457in}}%
\pgfusepath{stroke}%
\end{pgfscope}%
\begin{pgfscope}%
\pgfpathrectangle{\pgfqpoint{0.588387in}{0.521603in}}{\pgfqpoint{3.660036in}{2.220246in}}%
\pgfusepath{clip}%
\pgfsetrectcap%
\pgfsetroundjoin%
\pgfsetlinewidth{1.505625pt}%
\pgfsetstrokecolor{currentstroke6}%
\pgfsetdash{}{0pt}%
\pgfpathmoveto{\pgfqpoint{0.754752in}{0.667069in}}%
\pgfpathlineto{\pgfqpoint{0.825546in}{0.746893in}}%
\pgfpathlineto{\pgfqpoint{0.896340in}{0.829682in}}%
\pgfpathlineto{\pgfqpoint{0.967134in}{0.919549in}}%
\pgfpathlineto{\pgfqpoint{1.037927in}{0.997566in}}%
\pgfpathlineto{\pgfqpoint{1.108721in}{1.067119in}}%
\pgfpathlineto{\pgfqpoint{1.179515in}{1.169170in}}%
\pgfpathlineto{\pgfqpoint{1.250309in}{1.226205in}}%
\pgfpathlineto{\pgfqpoint{1.321102in}{1.288484in}}%
\pgfpathlineto{\pgfqpoint{1.391896in}{1.352872in}}%
\pgfpathlineto{\pgfqpoint{1.462690in}{1.383355in}}%
\pgfpathlineto{\pgfqpoint{1.533483in}{1.430872in}}%
\pgfpathlineto{\pgfqpoint{1.604277in}{1.532503in}}%
\pgfpathlineto{\pgfqpoint{1.675071in}{1.557571in}}%
\pgfpathlineto{\pgfqpoint{1.745865in}{1.608163in}}%
\pgfpathlineto{\pgfqpoint{1.816658in}{1.657646in}}%
\pgfpathlineto{\pgfqpoint{1.887452in}{1.689545in}}%
\pgfpathlineto{\pgfqpoint{1.958246in}{1.724113in}}%
\pgfpathlineto{\pgfqpoint{2.029039in}{1.787643in}}%
\pgfpathlineto{\pgfqpoint{2.099833in}{1.845663in}}%
\pgfpathlineto{\pgfqpoint{2.170627in}{1.844865in}}%
\pgfpathlineto{\pgfqpoint{2.241421in}{1.939594in}}%
\pgfpathlineto{\pgfqpoint{2.312214in}{1.942267in}}%
\pgfpathlineto{\pgfqpoint{2.383008in}{1.976561in}}%
\pgfpathlineto{\pgfqpoint{2.453802in}{1.986631in}}%
\pgfpathlineto{\pgfqpoint{2.524596in}{2.036049in}}%
\pgfpathlineto{\pgfqpoint{2.595389in}{2.067923in}}%
\pgfpathlineto{\pgfqpoint{2.666183in}{2.099865in}}%
\pgfpathlineto{\pgfqpoint{2.736977in}{2.164402in}}%
\pgfpathlineto{\pgfqpoint{2.807770in}{2.170622in}}%
\pgfpathlineto{\pgfqpoint{2.878564in}{2.172492in}}%
\pgfpathlineto{\pgfqpoint{2.949358in}{2.195647in}}%
\pgfpathlineto{\pgfqpoint{3.020152in}{2.309977in}}%
\pgfpathlineto{\pgfqpoint{3.090945in}{2.288185in}}%
\pgfpathlineto{\pgfqpoint{3.161739in}{2.292955in}}%
\pgfpathlineto{\pgfqpoint{3.232533in}{2.321919in}}%
\pgfpathlineto{\pgfqpoint{3.303327in}{2.347615in}}%
\pgfpathlineto{\pgfqpoint{3.374120in}{2.392313in}}%
\pgfpathlineto{\pgfqpoint{3.444914in}{2.456034in}}%
\pgfpathlineto{\pgfqpoint{3.515708in}{2.460471in}}%
\pgfpathlineto{\pgfqpoint{3.586501in}{2.449946in}}%
\pgfpathlineto{\pgfqpoint{3.657295in}{2.460859in}}%
\pgfpathlineto{\pgfqpoint{3.728089in}{2.565662in}}%
\pgfpathlineto{\pgfqpoint{3.798883in}{2.516634in}}%
\pgfpathlineto{\pgfqpoint{3.869676in}{2.526678in}}%
\pgfpathlineto{\pgfqpoint{3.940470in}{2.576819in}}%
\pgfpathlineto{\pgfqpoint{4.011264in}{2.608008in}}%
\pgfpathlineto{\pgfqpoint{4.082057in}{2.626192in}}%
\pgfusepath{stroke}%
\end{pgfscope}%
\begin{pgfscope}%
\pgfpathrectangle{\pgfqpoint{0.588387in}{0.521603in}}{\pgfqpoint{3.660036in}{2.220246in}}%
\pgfusepath{clip}%
\pgfsetrectcap%
\pgfsetroundjoin%
\pgfsetlinewidth{1.505625pt}%
\pgfsetstrokecolor{currentstroke7}%
\pgfsetdash{}{0pt}%
\pgfpathmoveto{\pgfqpoint{0.754752in}{0.682494in}}%
\pgfpathlineto{\pgfqpoint{0.825546in}{0.749475in}}%
\pgfpathlineto{\pgfqpoint{0.896340in}{0.833258in}}%
\pgfpathlineto{\pgfqpoint{0.967134in}{0.949031in}}%
\pgfpathlineto{\pgfqpoint{1.037927in}{1.009445in}}%
\pgfpathlineto{\pgfqpoint{1.108721in}{1.091409in}}%
\pgfpathlineto{\pgfqpoint{1.179515in}{1.178099in}}%
\pgfpathlineto{\pgfqpoint{1.250309in}{1.238345in}}%
\pgfpathlineto{\pgfqpoint{1.321102in}{1.310216in}}%
\pgfpathlineto{\pgfqpoint{1.391896in}{1.371510in}}%
\pgfpathlineto{\pgfqpoint{1.462690in}{1.393886in}}%
\pgfpathlineto{\pgfqpoint{1.533483in}{1.443179in}}%
\pgfpathlineto{\pgfqpoint{1.604277in}{1.591668in}}%
\pgfpathlineto{\pgfqpoint{1.675071in}{1.567146in}}%
\pgfpathlineto{\pgfqpoint{1.745865in}{1.620679in}}%
\pgfpathlineto{\pgfqpoint{1.816658in}{1.651526in}}%
\pgfpathlineto{\pgfqpoint{1.887452in}{1.686070in}}%
\pgfpathlineto{\pgfqpoint{1.958246in}{1.710208in}}%
\pgfpathlineto{\pgfqpoint{2.029039in}{1.781749in}}%
\pgfpathlineto{\pgfqpoint{2.099833in}{1.877394in}}%
\pgfpathlineto{\pgfqpoint{2.170627in}{1.859976in}}%
\pgfpathlineto{\pgfqpoint{2.241421in}{1.941832in}}%
\pgfpathlineto{\pgfqpoint{2.312214in}{1.927834in}}%
\pgfpathlineto{\pgfqpoint{2.383008in}{1.968352in}}%
\pgfpathlineto{\pgfqpoint{2.453802in}{1.983511in}}%
\pgfpathlineto{\pgfqpoint{2.524596in}{2.025747in}}%
\pgfpathlineto{\pgfqpoint{2.595389in}{2.070983in}}%
\pgfpathlineto{\pgfqpoint{2.666183in}{2.091491in}}%
\pgfpathlineto{\pgfqpoint{2.736977in}{2.179248in}}%
\pgfpathlineto{\pgfqpoint{2.807770in}{2.165807in}}%
\pgfpathlineto{\pgfqpoint{2.878564in}{2.175991in}}%
\pgfpathlineto{\pgfqpoint{2.949358in}{2.202904in}}%
\pgfpathlineto{\pgfqpoint{3.020152in}{2.310549in}}%
\pgfpathlineto{\pgfqpoint{3.090945in}{2.289867in}}%
\pgfpathlineto{\pgfqpoint{3.161739in}{2.296710in}}%
\pgfpathlineto{\pgfqpoint{3.232533in}{2.322772in}}%
\pgfpathlineto{\pgfqpoint{3.303327in}{2.346665in}}%
\pgfpathlineto{\pgfqpoint{3.374120in}{2.392057in}}%
\pgfpathlineto{\pgfqpoint{3.444914in}{2.454808in}}%
\pgfpathlineto{\pgfqpoint{3.515708in}{2.453409in}}%
\pgfpathlineto{\pgfqpoint{3.586501in}{2.456767in}}%
\pgfpathlineto{\pgfqpoint{3.657295in}{2.464881in}}%
\pgfpathlineto{\pgfqpoint{3.728089in}{2.563543in}}%
\pgfpathlineto{\pgfqpoint{3.798883in}{2.517281in}}%
\pgfpathlineto{\pgfqpoint{3.869676in}{2.520761in}}%
\pgfpathlineto{\pgfqpoint{3.940470in}{2.584842in}}%
\pgfpathlineto{\pgfqpoint{4.011264in}{2.611298in}}%
\pgfpathlineto{\pgfqpoint{4.082057in}{2.626739in}}%
\pgfusepath{stroke}%
\end{pgfscope}%
\begin{pgfscope}%
\pgfpathrectangle{\pgfqpoint{0.588387in}{0.521603in}}{\pgfqpoint{3.660036in}{2.220246in}}%
\pgfusepath{clip}%
\pgfsetrectcap%
\pgfsetroundjoin%
\pgfsetlinewidth{1.505625pt}%
\definecolor{currentstroke}{rgb}{0.498039,0.498039,0.498039}%
\pgfsetstrokecolor{currentstroke}%
\pgfsetdash{}{0pt}%
\pgfpathmoveto{\pgfqpoint{0.754752in}{0.716122in}}%
\pgfpathlineto{\pgfqpoint{0.825546in}{0.761091in}}%
\pgfpathlineto{\pgfqpoint{0.896340in}{0.863998in}}%
\pgfpathlineto{\pgfqpoint{0.967134in}{0.942678in}}%
\pgfpathlineto{\pgfqpoint{1.037927in}{1.018881in}}%
\pgfpathlineto{\pgfqpoint{1.108721in}{1.058638in}}%
\pgfpathlineto{\pgfqpoint{1.179515in}{1.144551in}}%
\pgfpathlineto{\pgfqpoint{1.250309in}{1.203317in}}%
\pgfpathlineto{\pgfqpoint{1.321102in}{1.280907in}}%
\pgfpathlineto{\pgfqpoint{1.391896in}{1.361089in}}%
\pgfpathlineto{\pgfqpoint{1.462690in}{1.380640in}}%
\pgfpathlineto{\pgfqpoint{1.533483in}{1.406004in}}%
\pgfpathlineto{\pgfqpoint{1.604277in}{1.502317in}}%
\pgfpathlineto{\pgfqpoint{1.675071in}{1.527048in}}%
\pgfpathlineto{\pgfqpoint{1.745865in}{1.566758in}}%
\pgfpathlineto{\pgfqpoint{1.816658in}{1.645181in}}%
\pgfpathlineto{\pgfqpoint{1.887452in}{1.659900in}}%
\pgfpathlineto{\pgfqpoint{1.958246in}{1.700229in}}%
\pgfpathlineto{\pgfqpoint{2.029039in}{1.762048in}}%
\pgfpathlineto{\pgfqpoint{2.099833in}{1.796892in}}%
\pgfpathlineto{\pgfqpoint{2.170627in}{1.842807in}}%
\pgfpathlineto{\pgfqpoint{2.241421in}{1.902932in}}%
\pgfpathlineto{\pgfqpoint{2.312214in}{1.919864in}}%
\pgfpathlineto{\pgfqpoint{2.383008in}{1.943480in}}%
\pgfpathlineto{\pgfqpoint{2.453802in}{1.961171in}}%
\pgfpathlineto{\pgfqpoint{2.524596in}{2.010631in}}%
\pgfpathlineto{\pgfqpoint{2.595389in}{2.059274in}}%
\pgfpathlineto{\pgfqpoint{2.666183in}{2.054713in}}%
\pgfpathlineto{\pgfqpoint{2.736977in}{2.135436in}}%
\pgfpathlineto{\pgfqpoint{2.807770in}{2.168780in}}%
\pgfpathlineto{\pgfqpoint{2.878564in}{2.218820in}}%
\pgfpathlineto{\pgfqpoint{2.949358in}{2.185412in}}%
\pgfpathlineto{\pgfqpoint{3.020152in}{2.234922in}}%
\pgfpathlineto{\pgfqpoint{3.090945in}{2.238981in}}%
\pgfpathlineto{\pgfqpoint{3.161739in}{2.287698in}}%
\pgfpathlineto{\pgfqpoint{3.232533in}{2.286149in}}%
\pgfpathlineto{\pgfqpoint{3.303327in}{2.273968in}}%
\pgfpathlineto{\pgfqpoint{3.374120in}{2.330012in}}%
\pgfpathlineto{\pgfqpoint{3.444914in}{2.395994in}}%
\pgfpathlineto{\pgfqpoint{3.515708in}{2.392052in}}%
\pgfpathlineto{\pgfqpoint{3.586501in}{2.434457in}}%
\pgfpathlineto{\pgfqpoint{3.657295in}{2.534115in}}%
\pgfpathlineto{\pgfqpoint{3.728089in}{2.496765in}}%
\pgfpathlineto{\pgfqpoint{3.798883in}{2.532643in}}%
\pgfpathlineto{\pgfqpoint{3.869676in}{2.500927in}}%
\pgfpathlineto{\pgfqpoint{3.940470in}{2.585992in}}%
\pgfpathlineto{\pgfqpoint{4.011264in}{2.624146in}}%
\pgfpathlineto{\pgfqpoint{4.082057in}{2.605049in}}%
\pgfusepath{stroke}%
\end{pgfscope}%
\begin{pgfscope}%
\pgfpathrectangle{\pgfqpoint{0.588387in}{0.521603in}}{\pgfqpoint{3.660036in}{2.220246in}}%
\pgfusepath{clip}%
\pgfsetrectcap%
\pgfsetroundjoin%
\pgfsetlinewidth{1.505625pt}%
\definecolor{currentstroke}{rgb}{0.737255,0.741176,0.133333}%
\pgfsetstrokecolor{currentstroke}%
\pgfsetdash{}{0pt}%
\pgfpathmoveto{\pgfqpoint{0.754752in}{0.682494in}}%
\pgfpathlineto{\pgfqpoint{0.825546in}{0.755391in}}%
\pgfpathlineto{\pgfqpoint{0.896340in}{0.861337in}}%
\pgfpathlineto{\pgfqpoint{0.967134in}{0.938330in}}%
\pgfpathlineto{\pgfqpoint{1.037927in}{1.013620in}}%
\pgfpathlineto{\pgfqpoint{1.108721in}{1.049452in}}%
\pgfpathlineto{\pgfqpoint{1.179515in}{1.170718in}}%
\pgfpathlineto{\pgfqpoint{1.250309in}{1.199895in}}%
\pgfpathlineto{\pgfqpoint{1.321102in}{1.305506in}}%
\pgfpathlineto{\pgfqpoint{1.391896in}{1.345067in}}%
\pgfpathlineto{\pgfqpoint{1.462690in}{1.362732in}}%
\pgfpathlineto{\pgfqpoint{1.533483in}{1.400303in}}%
\pgfpathlineto{\pgfqpoint{1.604277in}{1.508166in}}%
\pgfpathlineto{\pgfqpoint{1.675071in}{1.515632in}}%
\pgfpathlineto{\pgfqpoint{1.745865in}{1.548276in}}%
\pgfpathlineto{\pgfqpoint{1.816658in}{1.592300in}}%
\pgfpathlineto{\pgfqpoint{1.887452in}{1.643655in}}%
\pgfpathlineto{\pgfqpoint{1.958246in}{1.699508in}}%
\pgfpathlineto{\pgfqpoint{2.029039in}{1.770891in}}%
\pgfpathlineto{\pgfqpoint{2.099833in}{1.778267in}}%
\pgfpathlineto{\pgfqpoint{2.170627in}{1.832319in}}%
\pgfpathlineto{\pgfqpoint{2.241421in}{1.863127in}}%
\pgfpathlineto{\pgfqpoint{2.312214in}{1.912339in}}%
\pgfpathlineto{\pgfqpoint{2.383008in}{1.946986in}}%
\pgfpathlineto{\pgfqpoint{2.453802in}{1.949834in}}%
\pgfpathlineto{\pgfqpoint{2.524596in}{2.012232in}}%
\pgfpathlineto{\pgfqpoint{2.595389in}{2.051207in}}%
\pgfpathlineto{\pgfqpoint{2.666183in}{2.044410in}}%
\pgfpathlineto{\pgfqpoint{2.736977in}{2.127781in}}%
\pgfpathlineto{\pgfqpoint{2.807770in}{2.152994in}}%
\pgfpathlineto{\pgfqpoint{2.878564in}{2.209999in}}%
\pgfpathlineto{\pgfqpoint{2.949358in}{2.184371in}}%
\pgfpathlineto{\pgfqpoint{3.020152in}{2.225649in}}%
\pgfpathlineto{\pgfqpoint{3.090945in}{2.238783in}}%
\pgfpathlineto{\pgfqpoint{3.161739in}{2.289120in}}%
\pgfpathlineto{\pgfqpoint{3.232533in}{2.286155in}}%
\pgfpathlineto{\pgfqpoint{3.303327in}{2.272731in}}%
\pgfpathlineto{\pgfqpoint{3.374120in}{2.322933in}}%
\pgfpathlineto{\pgfqpoint{3.444914in}{2.389743in}}%
\pgfpathlineto{\pgfqpoint{3.515708in}{2.388422in}}%
\pgfpathlineto{\pgfqpoint{3.586501in}{2.426040in}}%
\pgfpathlineto{\pgfqpoint{3.657295in}{2.514258in}}%
\pgfpathlineto{\pgfqpoint{3.728089in}{2.477233in}}%
\pgfpathlineto{\pgfqpoint{3.798883in}{2.528290in}}%
\pgfpathlineto{\pgfqpoint{3.869676in}{2.488253in}}%
\pgfpathlineto{\pgfqpoint{3.940470in}{2.568399in}}%
\pgfpathlineto{\pgfqpoint{4.011264in}{2.619309in}}%
\pgfpathlineto{\pgfqpoint{4.082057in}{2.594694in}}%
\pgfusepath{stroke}%
\end{pgfscope}%
\begin{pgfscope}%
\pgfsetrectcap%
\pgfsetmiterjoin%
\pgfsetlinewidth{0.803000pt}%
\definecolor{currentstroke}{rgb}{0.000000,0.000000,0.000000}%
\pgfsetstrokecolor{currentstroke}%
\pgfsetdash{}{0pt}%
\pgfpathmoveto{\pgfqpoint{0.588387in}{0.521603in}}%
\pgfpathlineto{\pgfqpoint{0.588387in}{2.741849in}}%
\pgfusepath{stroke}%
\end{pgfscope}%
\begin{pgfscope}%
\pgfsetrectcap%
\pgfsetmiterjoin%
\pgfsetlinewidth{0.803000pt}%
\definecolor{currentstroke}{rgb}{0.000000,0.000000,0.000000}%
\pgfsetstrokecolor{currentstroke}%
\pgfsetdash{}{0pt}%
\pgfpathmoveto{\pgfqpoint{4.248423in}{0.521603in}}%
\pgfpathlineto{\pgfqpoint{4.248423in}{2.741849in}}%
\pgfusepath{stroke}%
\end{pgfscope}%
\begin{pgfscope}%
\pgfsetrectcap%
\pgfsetmiterjoin%
\pgfsetlinewidth{0.803000pt}%
\definecolor{currentstroke}{rgb}{0.000000,0.000000,0.000000}%
\pgfsetstrokecolor{currentstroke}%
\pgfsetdash{}{0pt}%
\pgfpathmoveto{\pgfqpoint{0.588387in}{0.521603in}}%
\pgfpathlineto{\pgfqpoint{4.248423in}{0.521603in}}%
\pgfusepath{stroke}%
\end{pgfscope}%
\begin{pgfscope}%
\pgfsetrectcap%
\pgfsetmiterjoin%
\pgfsetlinewidth{0.803000pt}%
\definecolor{currentstroke}{rgb}{0.000000,0.000000,0.000000}%
\pgfsetstrokecolor{currentstroke}%
\pgfsetdash{}{0pt}%
\pgfpathmoveto{\pgfqpoint{0.588387in}{2.741849in}}%
\pgfpathlineto{\pgfqpoint{4.248423in}{2.741849in}}%
\pgfusepath{stroke}%
\end{pgfscope}%
\begin{pgfscope}%
\pgfsetbuttcap%
\pgfsetmiterjoin%
\definecolor{currentfill}{rgb}{1.000000,1.000000,1.000000}%
\pgfsetfillcolor{currentfill}%
\pgfsetfillopacity{0.800000}%
\pgfsetlinewidth{1.003750pt}%
\definecolor{currentstroke}{rgb}{0.800000,0.800000,0.800000}%
\pgfsetstrokecolor{currentstroke}%
\pgfsetstrokeopacity{0.800000}%
\pgfsetdash{}{0pt}%
\pgfpathmoveto{\pgfqpoint{4.365089in}{0.379025in}}%
\pgfpathlineto{\pgfqpoint{8.251043in}{0.379025in}}%
\pgfpathquadraticcurveto{\pgfqpoint{8.284376in}{0.379025in}}{\pgfqpoint{8.284376in}{0.412359in}}%
\pgfpathlineto{\pgfqpoint{8.284376in}{2.625183in}}%
\pgfpathquadraticcurveto{\pgfqpoint{8.284376in}{2.658516in}}{\pgfqpoint{8.251043in}{2.658516in}}%
\pgfpathlineto{\pgfqpoint{4.365089in}{2.658516in}}%
\pgfpathquadraticcurveto{\pgfqpoint{4.331756in}{2.658516in}}{\pgfqpoint{4.331756in}{2.625183in}}%
\pgfpathlineto{\pgfqpoint{4.331756in}{0.412359in}}%
\pgfpathquadraticcurveto{\pgfqpoint{4.331756in}{0.379025in}}{\pgfqpoint{4.365089in}{0.379025in}}%
\pgfpathlineto{\pgfqpoint{4.365089in}{0.379025in}}%
\pgfpathclose%
\pgfusepath{stroke,fill}%
\end{pgfscope}%
\begin{pgfscope}%
\pgfsetrectcap%
\pgfsetroundjoin%
\pgfsetlinewidth{1.505625pt}%
\pgfsetstrokecolor{currentstroke3}%
\pgfsetdash{}{0pt}%
\pgfpathmoveto{\pgfqpoint{4.398423in}{2.523555in}}%
\pgfpathlineto{\pgfqpoint{4.565089in}{2.523555in}}%
\pgfpathlineto{\pgfqpoint{4.731756in}{2.523555in}}%
\pgfusepath{stroke}%
\end{pgfscope}%
\begin{pgfscope}%
\definecolor{textcolor}{rgb}{0.000000,0.000000,0.000000}%
\pgfsetstrokecolor{textcolor}%
\pgfsetfillcolor{textcolor}%
\pgftext[x=4.865089in,y=2.465222in,left,base]{\color{textcolor}{\rmfamily\fontsize{12.000000}{14.400000}\selectfont\catcode`\^=\active\def^{\ifmmode\sp\else\^{}\fi}\catcode`\%=\active\def%{\%}\NaiveCycles{}}}%
\end{pgfscope}%
\begin{pgfscope}%
\pgfsetrectcap%
\pgfsetroundjoin%
\pgfsetlinewidth{1.505625pt}%
\pgfsetstrokecolor{currentstroke1}%
\pgfsetdash{}{0pt}%
\pgfpathmoveto{\pgfqpoint{4.398423in}{2.278926in}}%
\pgfpathlineto{\pgfqpoint{4.565089in}{2.278926in}}%
\pgfpathlineto{\pgfqpoint{4.731756in}{2.278926in}}%
\pgfusepath{stroke}%
\end{pgfscope}%
\begin{pgfscope}%
\definecolor{textcolor}{rgb}{0.000000,0.000000,0.000000}%
\pgfsetstrokecolor{textcolor}%
\pgfsetfillcolor{textcolor}%
\pgftext[x=4.865089in,y=2.220593in,left,base]{\color{textcolor}{\rmfamily\fontsize{12.000000}{14.400000}\selectfont\catcode`\^=\active\def^{\ifmmode\sp\else\^{}\fi}\catcode`\%=\active\def%{\%}\CyclesMatchChunks{} \& \MergeLinear{}}}%
\end{pgfscope}%
\begin{pgfscope}%
\pgfsetrectcap%
\pgfsetroundjoin%
\pgfsetlinewidth{1.505625pt}%
\pgfsetstrokecolor{currentstroke2}%
\pgfsetdash{}{0pt}%
\pgfpathmoveto{\pgfqpoint{4.398423in}{2.029659in}}%
\pgfpathlineto{\pgfqpoint{4.565089in}{2.029659in}}%
\pgfpathlineto{\pgfqpoint{4.731756in}{2.029659in}}%
\pgfusepath{stroke}%
\end{pgfscope}%
\begin{pgfscope}%
\definecolor{textcolor}{rgb}{0.000000,0.000000,0.000000}%
\pgfsetstrokecolor{textcolor}%
\pgfsetfillcolor{textcolor}%
\pgftext[x=4.865089in,y=1.971325in,left,base]{\color{textcolor}{\rmfamily\fontsize{12.000000}{14.400000}\selectfont\catcode`\^=\active\def^{\ifmmode\sp\else\^{}\fi}\catcode`\%=\active\def%{\%}\CyclesMatchChunks{} \& \SharedVertices{}}}%
\end{pgfscope}%
\begin{pgfscope}%
\pgfsetrectcap%
\pgfsetroundjoin%
\pgfsetlinewidth{1.505625pt}%
\pgfsetstrokecolor{currentstroke4}%
\pgfsetdash{}{0pt}%
\pgfpathmoveto{\pgfqpoint{4.398423in}{1.780391in}}%
\pgfpathlineto{\pgfqpoint{4.565089in}{1.780391in}}%
\pgfpathlineto{\pgfqpoint{4.731756in}{1.780391in}}%
\pgfusepath{stroke}%
\end{pgfscope}%
\begin{pgfscope}%
\definecolor{textcolor}{rgb}{0.000000,0.000000,0.000000}%
\pgfsetstrokecolor{textcolor}%
\pgfsetfillcolor{textcolor}%
\pgftext[x=4.865089in,y=1.722058in,left,base]{\color{textcolor}{\rmfamily\fontsize{12.000000}{14.400000}\selectfont\catcode`\^=\active\def^{\ifmmode\sp\else\^{}\fi}\catcode`\%=\active\def%{\%}\Neighbors{} \& \MergeLinear{}}}%
\end{pgfscope}%
\begin{pgfscope}%
\pgfsetrectcap%
\pgfsetroundjoin%
\pgfsetlinewidth{1.505625pt}%
\pgfsetstrokecolor{currentstroke5}%
\pgfsetdash{}{0pt}%
\pgfpathmoveto{\pgfqpoint{4.398423in}{1.535763in}}%
\pgfpathlineto{\pgfqpoint{4.565089in}{1.535763in}}%
\pgfpathlineto{\pgfqpoint{4.731756in}{1.535763in}}%
\pgfusepath{stroke}%
\end{pgfscope}%
\begin{pgfscope}%
\definecolor{textcolor}{rgb}{0.000000,0.000000,0.000000}%
\pgfsetstrokecolor{textcolor}%
\pgfsetfillcolor{textcolor}%
\pgftext[x=4.865089in,y=1.477429in,left,base]{\color{textcolor}{\rmfamily\fontsize{12.000000}{14.400000}\selectfont\catcode`\^=\active\def^{\ifmmode\sp\else\^{}\fi}\catcode`\%=\active\def%{\%}\Neighbors{} \& \SharedVertices{}}}%
\end{pgfscope}%
\begin{pgfscope}%
\pgfsetrectcap%
\pgfsetroundjoin%
\pgfsetlinewidth{1.505625pt}%
\pgfsetstrokecolor{currentstroke6}%
\pgfsetdash{}{0pt}%
\pgfpathmoveto{\pgfqpoint{4.398423in}{1.286495in}}%
\pgfpathlineto{\pgfqpoint{4.565089in}{1.286495in}}%
\pgfpathlineto{\pgfqpoint{4.731756in}{1.286495in}}%
\pgfusepath{stroke}%
\end{pgfscope}%
\begin{pgfscope}%
\definecolor{textcolor}{rgb}{0.000000,0.000000,0.000000}%
\pgfsetstrokecolor{textcolor}%
\pgfsetfillcolor{textcolor}%
\pgftext[x=4.865089in,y=1.228162in,left,base]{\color{textcolor}{\rmfamily\fontsize{12.000000}{14.400000}\selectfont\catcode`\^=\active\def^{\ifmmode\sp\else\^{}\fi}\catcode`\%=\active\def%{\%}\NeighborsDegree{} \& \MergeLinear{}}}%
\end{pgfscope}%
\begin{pgfscope}%
\pgfsetrectcap%
\pgfsetroundjoin%
\pgfsetlinewidth{1.505625pt}%
\pgfsetstrokecolor{currentstroke7}%
\pgfsetdash{}{0pt}%
\pgfpathmoveto{\pgfqpoint{4.398423in}{1.037228in}}%
\pgfpathlineto{\pgfqpoint{4.565089in}{1.037228in}}%
\pgfpathlineto{\pgfqpoint{4.731756in}{1.037228in}}%
\pgfusepath{stroke}%
\end{pgfscope}%
\begin{pgfscope}%
\definecolor{textcolor}{rgb}{0.000000,0.000000,0.000000}%
\pgfsetstrokecolor{textcolor}%
\pgfsetfillcolor{textcolor}%
\pgftext[x=4.865089in,y=0.978895in,left,base]{\color{textcolor}{\rmfamily\fontsize{12.000000}{14.400000}\selectfont\catcode`\^=\active\def^{\ifmmode\sp\else\^{}\fi}\catcode`\%=\active\def%{\%}\NeighborsDegree{} \& \SharedVertices{}}}%
\end{pgfscope}%
\begin{pgfscope}%
\pgfsetrectcap%
\pgfsetroundjoin%
\pgfsetlinewidth{1.505625pt}%
\definecolor{currentstroke}{rgb}{0.498039,0.498039,0.498039}%
\pgfsetstrokecolor{currentstroke}%
\pgfsetdash{}{0pt}%
\pgfpathmoveto{\pgfqpoint{4.398423in}{0.787961in}}%
\pgfpathlineto{\pgfqpoint{4.565089in}{0.787961in}}%
\pgfpathlineto{\pgfqpoint{4.731756in}{0.787961in}}%
\pgfusepath{stroke}%
\end{pgfscope}%
\begin{pgfscope}%
\definecolor{textcolor}{rgb}{0.000000,0.000000,0.000000}%
\pgfsetstrokecolor{textcolor}%
\pgfsetfillcolor{textcolor}%
\pgftext[x=4.865089in,y=0.729627in,left,base]{\color{textcolor}{\rmfamily\fontsize{12.000000}{14.400000}\selectfont\catcode`\^=\active\def^{\ifmmode\sp\else\^{}\fi}\catcode`\%=\active\def%{\%}\None{} \& \MergeLinear{}}}%
\end{pgfscope}%
\begin{pgfscope}%
\pgfsetrectcap%
\pgfsetroundjoin%
\pgfsetlinewidth{1.505625pt}%
\definecolor{currentstroke}{rgb}{0.737255,0.741176,0.133333}%
\pgfsetstrokecolor{currentstroke}%
\pgfsetdash{}{0pt}%
\pgfpathmoveto{\pgfqpoint{4.398423in}{0.543332in}}%
\pgfpathlineto{\pgfqpoint{4.565089in}{0.543332in}}%
\pgfpathlineto{\pgfqpoint{4.731756in}{0.543332in}}%
\pgfusepath{stroke}%
\end{pgfscope}%
\begin{pgfscope}%
\definecolor{textcolor}{rgb}{0.000000,0.000000,0.000000}%
\pgfsetstrokecolor{textcolor}%
\pgfsetfillcolor{textcolor}%
\pgftext[x=4.865089in,y=0.484999in,left,base]{\color{textcolor}{\rmfamily\fontsize{12.000000}{14.400000}\selectfont\catcode`\^=\active\def^{\ifmmode\sp\else\^{}\fi}\catcode`\%=\active\def%{\%}\None{} \& \SharedVertices{}}}%
\end{pgfscope}%
\end{pgfpicture}%
\makeatother%
\endgroup%
}
	\caption[Mean runtime for globally rigid graphs (some)]{
		Mean running time to find some NAC-colorings for globally rigid graphs.}%
	\label{fig:graph_globally_rigid_first_runtime}
\end{figure}%
% \begin{figure}[thbp]
% 	\centering
% 	\scalebox{\BenchFigureScale}{%% Creator: Matplotlib, PGF backend
%%
%% To include the figure in your LaTeX document, write
%%   \input{<filename>.pgf}
%%
%% Make sure the required packages are loaded in your preamble
%%   \usepackage{pgf}
%%
%% Also ensure that all the required font packages are loaded; for instance,
%% the lmodern package is sometimes necessary when using math font.
%%   \usepackage{lmodern}
%%
%% Figures using additional raster images can only be included by \input if
%% they are in the same directory as the main LaTeX file. For loading figures
%% from other directories you can use the `import` package
%%   \usepackage{import}
%%
%% and then include the figures with
%%   \import{<path to file>}{<filename>.pgf}
%%
%% Matplotlib used the following preamble
%%   \def\mathdefault#1{#1}
%%   \everymath=\expandafter{\the\everymath\displaystyle}
%%   \IfFileExists{scrextend.sty}{
%%     \usepackage[fontsize=10.000000pt]{scrextend}
%%   }{
%%     \renewcommand{\normalsize}{\fontsize{10.000000}{12.000000}\selectfont}
%%     \normalsize
%%   }
%%   
%%   \ifdefined\pdftexversion\else  % non-pdftex case.
%%     \usepackage{fontspec}
%%     \setmainfont{DejaVuSans.ttf}[Path=\detokenize{/home/petr/Projects/PyRigi/.venv/lib/python3.12/site-packages/matplotlib/mpl-data/fonts/ttf/}]
%%     \setsansfont{DejaVuSans.ttf}[Path=\detokenize{/home/petr/Projects/PyRigi/.venv/lib/python3.12/site-packages/matplotlib/mpl-data/fonts/ttf/}]
%%     \setmonofont{DejaVuSansMono.ttf}[Path=\detokenize{/home/petr/Projects/PyRigi/.venv/lib/python3.12/site-packages/matplotlib/mpl-data/fonts/ttf/}]
%%   \fi
%%   \makeatletter\@ifpackageloaded{under\Score{}}{}{\usepackage[strings]{under\Score{}}}\makeatother
%%
\begingroup%
\makeatletter%
\begin{pgfpicture}%
\pgfpathrectangle{\pgfpointorigin}{\pgfqpoint{8.384376in}{2.841849in}}%
\pgfusepath{use as bounding box, clip}%
\begin{pgfscope}%
\pgfsetbuttcap%
\pgfsetmiterjoin%
\definecolor{currentfill}{rgb}{1.000000,1.000000,1.000000}%
\pgfsetfillcolor{currentfill}%
\pgfsetlinewidth{0.000000pt}%
\definecolor{currentstroke}{rgb}{1.000000,1.000000,1.000000}%
\pgfsetstrokecolor{currentstroke}%
\pgfsetdash{}{0pt}%
\pgfpathmoveto{\pgfqpoint{0.000000in}{0.000000in}}%
\pgfpathlineto{\pgfqpoint{8.384376in}{0.000000in}}%
\pgfpathlineto{\pgfqpoint{8.384376in}{2.841849in}}%
\pgfpathlineto{\pgfqpoint{0.000000in}{2.841849in}}%
\pgfpathlineto{\pgfqpoint{0.000000in}{0.000000in}}%
\pgfpathclose%
\pgfusepath{fill}%
\end{pgfscope}%
\begin{pgfscope}%
\pgfsetbuttcap%
\pgfsetmiterjoin%
\definecolor{currentfill}{rgb}{1.000000,1.000000,1.000000}%
\pgfsetfillcolor{currentfill}%
\pgfsetlinewidth{0.000000pt}%
\definecolor{currentstroke}{rgb}{0.000000,0.000000,0.000000}%
\pgfsetstrokecolor{currentstroke}%
\pgfsetstrokeopacity{0.000000}%
\pgfsetdash{}{0pt}%
\pgfpathmoveto{\pgfqpoint{0.588387in}{0.521603in}}%
\pgfpathlineto{\pgfqpoint{4.248423in}{0.521603in}}%
\pgfpathlineto{\pgfqpoint{4.248423in}{2.741849in}}%
\pgfpathlineto{\pgfqpoint{0.588387in}{2.741849in}}%
\pgfpathlineto{\pgfqpoint{0.588387in}{0.521603in}}%
\pgfpathclose%
\pgfusepath{fill}%
\end{pgfscope}%
\begin{pgfscope}%
\pgfsetbuttcap%
\pgfsetroundjoin%
\definecolor{currentfill}{rgb}{0.000000,0.000000,0.000000}%
\pgfsetfillcolor{currentfill}%
\pgfsetlinewidth{0.803000pt}%
\definecolor{currentstroke}{rgb}{0.000000,0.000000,0.000000}%
\pgfsetstrokecolor{currentstroke}%
\pgfsetdash{}{0pt}%
\pgfsys@defobject{currentmarker}{\pgfqpoint{0.000000in}{-0.048611in}}{\pgfqpoint{0.000000in}{0.000000in}}{%
\pgfpathmoveto{\pgfqpoint{0.000000in}{0.000000in}}%
\pgfpathlineto{\pgfqpoint{0.000000in}{-0.048611in}}%
\pgfusepath{stroke,fill}%
}%
\begin{pgfscope}%
\pgfsys@transformshift{0.655430in}{0.521603in}%
\pgfsys@useobject{currentmarker}{}%
\end{pgfscope}%
\end{pgfscope}%
\begin{pgfscope}%
\definecolor{textcolor}{rgb}{0.000000,0.000000,0.000000}%
\pgfsetstrokecolor{textcolor}%
\pgfsetfillcolor{textcolor}%
\pgftext[x=0.655430in,y=0.424381in,,top]{\color{textcolor}{\rmfamily\fontsize{10.000000}{12.000000}\selectfont\catcode`\^=\active\def^{\ifmmode\sp\else\^{}\fi}\catcode`\%=\active\def%{\%}$\mathdefault{0}$}}%
\end{pgfscope}%
\begin{pgfscope}%
\pgfsetbuttcap%
\pgfsetroundjoin%
\definecolor{currentfill}{rgb}{0.000000,0.000000,0.000000}%
\pgfsetfillcolor{currentfill}%
\pgfsetlinewidth{0.803000pt}%
\definecolor{currentstroke}{rgb}{0.000000,0.000000,0.000000}%
\pgfsetstrokecolor{currentstroke}%
\pgfsetdash{}{0pt}%
\pgfsys@defobject{currentmarker}{\pgfqpoint{0.000000in}{-0.048611in}}{\pgfqpoint{0.000000in}{0.000000in}}{%
\pgfpathmoveto{\pgfqpoint{0.000000in}{0.000000in}}%
\pgfpathlineto{\pgfqpoint{0.000000in}{-0.048611in}}%
\pgfusepath{stroke,fill}%
}%
\begin{pgfscope}%
\pgfsys@transformshift{1.052720in}{0.521603in}%
\pgfsys@useobject{currentmarker}{}%
\end{pgfscope}%
\end{pgfscope}%
\begin{pgfscope}%
\definecolor{textcolor}{rgb}{0.000000,0.000000,0.000000}%
\pgfsetstrokecolor{textcolor}%
\pgfsetfillcolor{textcolor}%
\pgftext[x=1.052720in,y=0.424381in,,top]{\color{textcolor}{\rmfamily\fontsize{10.000000}{12.000000}\selectfont\catcode`\^=\active\def^{\ifmmode\sp\else\^{}\fi}\catcode`\%=\active\def%{\%}$\mathdefault{8}$}}%
\end{pgfscope}%
\begin{pgfscope}%
\pgfsetbuttcap%
\pgfsetroundjoin%
\definecolor{currentfill}{rgb}{0.000000,0.000000,0.000000}%
\pgfsetfillcolor{currentfill}%
\pgfsetlinewidth{0.803000pt}%
\definecolor{currentstroke}{rgb}{0.000000,0.000000,0.000000}%
\pgfsetstrokecolor{currentstroke}%
\pgfsetdash{}{0pt}%
\pgfsys@defobject{currentmarker}{\pgfqpoint{0.000000in}{-0.048611in}}{\pgfqpoint{0.000000in}{0.000000in}}{%
\pgfpathmoveto{\pgfqpoint{0.000000in}{0.000000in}}%
\pgfpathlineto{\pgfqpoint{0.000000in}{-0.048611in}}%
\pgfusepath{stroke,fill}%
}%
\begin{pgfscope}%
\pgfsys@transformshift{1.450010in}{0.521603in}%
\pgfsys@useobject{currentmarker}{}%
\end{pgfscope}%
\end{pgfscope}%
\begin{pgfscope}%
\definecolor{textcolor}{rgb}{0.000000,0.000000,0.000000}%
\pgfsetstrokecolor{textcolor}%
\pgfsetfillcolor{textcolor}%
\pgftext[x=1.450010in,y=0.424381in,,top]{\color{textcolor}{\rmfamily\fontsize{10.000000}{12.000000}\selectfont\catcode`\^=\active\def^{\ifmmode\sp\else\^{}\fi}\catcode`\%=\active\def%{\%}$\mathdefault{16}$}}%
\end{pgfscope}%
\begin{pgfscope}%
\pgfsetbuttcap%
\pgfsetroundjoin%
\definecolor{currentfill}{rgb}{0.000000,0.000000,0.000000}%
\pgfsetfillcolor{currentfill}%
\pgfsetlinewidth{0.803000pt}%
\definecolor{currentstroke}{rgb}{0.000000,0.000000,0.000000}%
\pgfsetstrokecolor{currentstroke}%
\pgfsetdash{}{0pt}%
\pgfsys@defobject{currentmarker}{\pgfqpoint{0.000000in}{-0.048611in}}{\pgfqpoint{0.000000in}{0.000000in}}{%
\pgfpathmoveto{\pgfqpoint{0.000000in}{0.000000in}}%
\pgfpathlineto{\pgfqpoint{0.000000in}{-0.048611in}}%
\pgfusepath{stroke,fill}%
}%
\begin{pgfscope}%
\pgfsys@transformshift{1.847300in}{0.521603in}%
\pgfsys@useobject{currentmarker}{}%
\end{pgfscope}%
\end{pgfscope}%
\begin{pgfscope}%
\definecolor{textcolor}{rgb}{0.000000,0.000000,0.000000}%
\pgfsetstrokecolor{textcolor}%
\pgfsetfillcolor{textcolor}%
\pgftext[x=1.847300in,y=0.424381in,,top]{\color{textcolor}{\rmfamily\fontsize{10.000000}{12.000000}\selectfont\catcode`\^=\active\def^{\ifmmode\sp\else\^{}\fi}\catcode`\%=\active\def%{\%}$\mathdefault{24}$}}%
\end{pgfscope}%
\begin{pgfscope}%
\pgfsetbuttcap%
\pgfsetroundjoin%
\definecolor{currentfill}{rgb}{0.000000,0.000000,0.000000}%
\pgfsetfillcolor{currentfill}%
\pgfsetlinewidth{0.803000pt}%
\definecolor{currentstroke}{rgb}{0.000000,0.000000,0.000000}%
\pgfsetstrokecolor{currentstroke}%
\pgfsetdash{}{0pt}%
\pgfsys@defobject{currentmarker}{\pgfqpoint{0.000000in}{-0.048611in}}{\pgfqpoint{0.000000in}{0.000000in}}{%
\pgfpathmoveto{\pgfqpoint{0.000000in}{0.000000in}}%
\pgfpathlineto{\pgfqpoint{0.000000in}{-0.048611in}}%
\pgfusepath{stroke,fill}%
}%
\begin{pgfscope}%
\pgfsys@transformshift{2.244591in}{0.521603in}%
\pgfsys@useobject{currentmarker}{}%
\end{pgfscope}%
\end{pgfscope}%
\begin{pgfscope}%
\definecolor{textcolor}{rgb}{0.000000,0.000000,0.000000}%
\pgfsetstrokecolor{textcolor}%
\pgfsetfillcolor{textcolor}%
\pgftext[x=2.244591in,y=0.424381in,,top]{\color{textcolor}{\rmfamily\fontsize{10.000000}{12.000000}\selectfont\catcode`\^=\active\def^{\ifmmode\sp\else\^{}\fi}\catcode`\%=\active\def%{\%}$\mathdefault{32}$}}%
\end{pgfscope}%
\begin{pgfscope}%
\pgfsetbuttcap%
\pgfsetroundjoin%
\definecolor{currentfill}{rgb}{0.000000,0.000000,0.000000}%
\pgfsetfillcolor{currentfill}%
\pgfsetlinewidth{0.803000pt}%
\definecolor{currentstroke}{rgb}{0.000000,0.000000,0.000000}%
\pgfsetstrokecolor{currentstroke}%
\pgfsetdash{}{0pt}%
\pgfsys@defobject{currentmarker}{\pgfqpoint{0.000000in}{-0.048611in}}{\pgfqpoint{0.000000in}{0.000000in}}{%
\pgfpathmoveto{\pgfqpoint{0.000000in}{0.000000in}}%
\pgfpathlineto{\pgfqpoint{0.000000in}{-0.048611in}}%
\pgfusepath{stroke,fill}%
}%
\begin{pgfscope}%
\pgfsys@transformshift{2.641881in}{0.521603in}%
\pgfsys@useobject{currentmarker}{}%
\end{pgfscope}%
\end{pgfscope}%
\begin{pgfscope}%
\definecolor{textcolor}{rgb}{0.000000,0.000000,0.000000}%
\pgfsetstrokecolor{textcolor}%
\pgfsetfillcolor{textcolor}%
\pgftext[x=2.641881in,y=0.424381in,,top]{\color{textcolor}{\rmfamily\fontsize{10.000000}{12.000000}\selectfont\catcode`\^=\active\def^{\ifmmode\sp\else\^{}\fi}\catcode`\%=\active\def%{\%}$\mathdefault{40}$}}%
\end{pgfscope}%
\begin{pgfscope}%
\pgfsetbuttcap%
\pgfsetroundjoin%
\definecolor{currentfill}{rgb}{0.000000,0.000000,0.000000}%
\pgfsetfillcolor{currentfill}%
\pgfsetlinewidth{0.803000pt}%
\definecolor{currentstroke}{rgb}{0.000000,0.000000,0.000000}%
\pgfsetstrokecolor{currentstroke}%
\pgfsetdash{}{0pt}%
\pgfsys@defobject{currentmarker}{\pgfqpoint{0.000000in}{-0.048611in}}{\pgfqpoint{0.000000in}{0.000000in}}{%
\pgfpathmoveto{\pgfqpoint{0.000000in}{0.000000in}}%
\pgfpathlineto{\pgfqpoint{0.000000in}{-0.048611in}}%
\pgfusepath{stroke,fill}%
}%
\begin{pgfscope}%
\pgfsys@transformshift{3.039171in}{0.521603in}%
\pgfsys@useobject{currentmarker}{}%
\end{pgfscope}%
\end{pgfscope}%
\begin{pgfscope}%
\definecolor{textcolor}{rgb}{0.000000,0.000000,0.000000}%
\pgfsetstrokecolor{textcolor}%
\pgfsetfillcolor{textcolor}%
\pgftext[x=3.039171in,y=0.424381in,,top]{\color{textcolor}{\rmfamily\fontsize{10.000000}{12.000000}\selectfont\catcode`\^=\active\def^{\ifmmode\sp\else\^{}\fi}\catcode`\%=\active\def%{\%}$\mathdefault{48}$}}%
\end{pgfscope}%
\begin{pgfscope}%
\pgfsetbuttcap%
\pgfsetroundjoin%
\definecolor{currentfill}{rgb}{0.000000,0.000000,0.000000}%
\pgfsetfillcolor{currentfill}%
\pgfsetlinewidth{0.803000pt}%
\definecolor{currentstroke}{rgb}{0.000000,0.000000,0.000000}%
\pgfsetstrokecolor{currentstroke}%
\pgfsetdash{}{0pt}%
\pgfsys@defobject{currentmarker}{\pgfqpoint{0.000000in}{-0.048611in}}{\pgfqpoint{0.000000in}{0.000000in}}{%
\pgfpathmoveto{\pgfqpoint{0.000000in}{0.000000in}}%
\pgfpathlineto{\pgfqpoint{0.000000in}{-0.048611in}}%
\pgfusepath{stroke,fill}%
}%
\begin{pgfscope}%
\pgfsys@transformshift{3.436461in}{0.521603in}%
\pgfsys@useobject{currentmarker}{}%
\end{pgfscope}%
\end{pgfscope}%
\begin{pgfscope}%
\definecolor{textcolor}{rgb}{0.000000,0.000000,0.000000}%
\pgfsetstrokecolor{textcolor}%
\pgfsetfillcolor{textcolor}%
\pgftext[x=3.436461in,y=0.424381in,,top]{\color{textcolor}{\rmfamily\fontsize{10.000000}{12.000000}\selectfont\catcode`\^=\active\def^{\ifmmode\sp\else\^{}\fi}\catcode`\%=\active\def%{\%}$\mathdefault{56}$}}%
\end{pgfscope}%
\begin{pgfscope}%
\pgfsetbuttcap%
\pgfsetroundjoin%
\definecolor{currentfill}{rgb}{0.000000,0.000000,0.000000}%
\pgfsetfillcolor{currentfill}%
\pgfsetlinewidth{0.803000pt}%
\definecolor{currentstroke}{rgb}{0.000000,0.000000,0.000000}%
\pgfsetstrokecolor{currentstroke}%
\pgfsetdash{}{0pt}%
\pgfsys@defobject{currentmarker}{\pgfqpoint{0.000000in}{-0.048611in}}{\pgfqpoint{0.000000in}{0.000000in}}{%
\pgfpathmoveto{\pgfqpoint{0.000000in}{0.000000in}}%
\pgfpathlineto{\pgfqpoint{0.000000in}{-0.048611in}}%
\pgfusepath{stroke,fill}%
}%
\begin{pgfscope}%
\pgfsys@transformshift{3.833751in}{0.521603in}%
\pgfsys@useobject{currentmarker}{}%
\end{pgfscope}%
\end{pgfscope}%
\begin{pgfscope}%
\definecolor{textcolor}{rgb}{0.000000,0.000000,0.000000}%
\pgfsetstrokecolor{textcolor}%
\pgfsetfillcolor{textcolor}%
\pgftext[x=3.833751in,y=0.424381in,,top]{\color{textcolor}{\rmfamily\fontsize{10.000000}{12.000000}\selectfont\catcode`\^=\active\def^{\ifmmode\sp\else\^{}\fi}\catcode`\%=\active\def%{\%}$\mathdefault{64}$}}%
\end{pgfscope}%
\begin{pgfscope}%
\pgfsetbuttcap%
\pgfsetroundjoin%
\definecolor{currentfill}{rgb}{0.000000,0.000000,0.000000}%
\pgfsetfillcolor{currentfill}%
\pgfsetlinewidth{0.803000pt}%
\definecolor{currentstroke}{rgb}{0.000000,0.000000,0.000000}%
\pgfsetstrokecolor{currentstroke}%
\pgfsetdash{}{0pt}%
\pgfsys@defobject{currentmarker}{\pgfqpoint{0.000000in}{-0.048611in}}{\pgfqpoint{0.000000in}{0.000000in}}{%
\pgfpathmoveto{\pgfqpoint{0.000000in}{0.000000in}}%
\pgfpathlineto{\pgfqpoint{0.000000in}{-0.048611in}}%
\pgfusepath{stroke,fill}%
}%
\begin{pgfscope}%
\pgfsys@transformshift{4.231041in}{0.521603in}%
\pgfsys@useobject{currentmarker}{}%
\end{pgfscope}%
\end{pgfscope}%
\begin{pgfscope}%
\definecolor{textcolor}{rgb}{0.000000,0.000000,0.000000}%
\pgfsetstrokecolor{textcolor}%
\pgfsetfillcolor{textcolor}%
\pgftext[x=4.231041in,y=0.424381in,,top]{\color{textcolor}{\rmfamily\fontsize{10.000000}{12.000000}\selectfont\catcode`\^=\active\def^{\ifmmode\sp\else\^{}\fi}\catcode`\%=\active\def%{\%}$\mathdefault{72}$}}%
\end{pgfscope}%
\begin{pgfscope}%
\definecolor{textcolor}{rgb}{0.000000,0.000000,0.000000}%
\pgfsetstrokecolor{textcolor}%
\pgfsetfillcolor{textcolor}%
\pgftext[x=2.418405in,y=0.234413in,,top]{\color{textcolor}{\rmfamily\fontsize{10.000000}{12.000000}\selectfont\catcode`\^=\active\def^{\ifmmode\sp\else\^{}\fi}\catcode`\%=\active\def%{\%}Monochromatic classes}}%
\end{pgfscope}%
\begin{pgfscope}%
\pgfsetbuttcap%
\pgfsetroundjoin%
\definecolor{currentfill}{rgb}{0.000000,0.000000,0.000000}%
\pgfsetfillcolor{currentfill}%
\pgfsetlinewidth{0.803000pt}%
\definecolor{currentstroke}{rgb}{0.000000,0.000000,0.000000}%
\pgfsetstrokecolor{currentstroke}%
\pgfsetdash{}{0pt}%
\pgfsys@defobject{currentmarker}{\pgfqpoint{-0.048611in}{0.000000in}}{\pgfqpoint{-0.000000in}{0.000000in}}{%
\pgfpathmoveto{\pgfqpoint{-0.000000in}{0.000000in}}%
\pgfpathlineto{\pgfqpoint{-0.048611in}{0.000000in}}%
\pgfusepath{stroke,fill}%
}%
\begin{pgfscope}%
\pgfsys@transformshift{0.588387in}{0.622524in}%
\pgfsys@useobject{currentmarker}{}%
\end{pgfscope}%
\end{pgfscope}%
\begin{pgfscope}%
\definecolor{textcolor}{rgb}{0.000000,0.000000,0.000000}%
\pgfsetstrokecolor{textcolor}%
\pgfsetfillcolor{textcolor}%
\pgftext[x=0.289968in, y=0.569762in, left, base]{\color{textcolor}{\rmfamily\fontsize{10.000000}{12.000000}\selectfont\catcode`\^=\active\def^{\ifmmode\sp\else\^{}\fi}\catcode`\%=\active\def%{\%}$\mathdefault{10^{0}}$}}%
\end{pgfscope}%
\begin{pgfscope}%
\pgfsetbuttcap%
\pgfsetroundjoin%
\definecolor{currentfill}{rgb}{0.000000,0.000000,0.000000}%
\pgfsetfillcolor{currentfill}%
\pgfsetlinewidth{0.803000pt}%
\definecolor{currentstroke}{rgb}{0.000000,0.000000,0.000000}%
\pgfsetstrokecolor{currentstroke}%
\pgfsetdash{}{0pt}%
\pgfsys@defobject{currentmarker}{\pgfqpoint{-0.048611in}{0.000000in}}{\pgfqpoint{-0.000000in}{0.000000in}}{%
\pgfpathmoveto{\pgfqpoint{-0.000000in}{0.000000in}}%
\pgfpathlineto{\pgfqpoint{-0.048611in}{0.000000in}}%
\pgfusepath{stroke,fill}%
}%
\begin{pgfscope}%
\pgfsys@transformshift{0.588387in}{1.072296in}%
\pgfsys@useobject{currentmarker}{}%
\end{pgfscope}%
\end{pgfscope}%
\begin{pgfscope}%
\definecolor{textcolor}{rgb}{0.000000,0.000000,0.000000}%
\pgfsetstrokecolor{textcolor}%
\pgfsetfillcolor{textcolor}%
\pgftext[x=0.289968in, y=1.019534in, left, base]{\color{textcolor}{\rmfamily\fontsize{10.000000}{12.000000}\selectfont\catcode`\^=\active\def^{\ifmmode\sp\else\^{}\fi}\catcode`\%=\active\def%{\%}$\mathdefault{10^{1}}$}}%
\end{pgfscope}%
\begin{pgfscope}%
\pgfsetbuttcap%
\pgfsetroundjoin%
\definecolor{currentfill}{rgb}{0.000000,0.000000,0.000000}%
\pgfsetfillcolor{currentfill}%
\pgfsetlinewidth{0.803000pt}%
\definecolor{currentstroke}{rgb}{0.000000,0.000000,0.000000}%
\pgfsetstrokecolor{currentstroke}%
\pgfsetdash{}{0pt}%
\pgfsys@defobject{currentmarker}{\pgfqpoint{-0.048611in}{0.000000in}}{\pgfqpoint{-0.000000in}{0.000000in}}{%
\pgfpathmoveto{\pgfqpoint{-0.000000in}{0.000000in}}%
\pgfpathlineto{\pgfqpoint{-0.048611in}{0.000000in}}%
\pgfusepath{stroke,fill}%
}%
\begin{pgfscope}%
\pgfsys@transformshift{0.588387in}{1.522068in}%
\pgfsys@useobject{currentmarker}{}%
\end{pgfscope}%
\end{pgfscope}%
\begin{pgfscope}%
\definecolor{textcolor}{rgb}{0.000000,0.000000,0.000000}%
\pgfsetstrokecolor{textcolor}%
\pgfsetfillcolor{textcolor}%
\pgftext[x=0.289968in, y=1.469306in, left, base]{\color{textcolor}{\rmfamily\fontsize{10.000000}{12.000000}\selectfont\catcode`\^=\active\def^{\ifmmode\sp\else\^{}\fi}\catcode`\%=\active\def%{\%}$\mathdefault{10^{2}}$}}%
\end{pgfscope}%
\begin{pgfscope}%
\pgfsetbuttcap%
\pgfsetroundjoin%
\definecolor{currentfill}{rgb}{0.000000,0.000000,0.000000}%
\pgfsetfillcolor{currentfill}%
\pgfsetlinewidth{0.803000pt}%
\definecolor{currentstroke}{rgb}{0.000000,0.000000,0.000000}%
\pgfsetstrokecolor{currentstroke}%
\pgfsetdash{}{0pt}%
\pgfsys@defobject{currentmarker}{\pgfqpoint{-0.048611in}{0.000000in}}{\pgfqpoint{-0.000000in}{0.000000in}}{%
\pgfpathmoveto{\pgfqpoint{-0.000000in}{0.000000in}}%
\pgfpathlineto{\pgfqpoint{-0.048611in}{0.000000in}}%
\pgfusepath{stroke,fill}%
}%
\begin{pgfscope}%
\pgfsys@transformshift{0.588387in}{1.971840in}%
\pgfsys@useobject{currentmarker}{}%
\end{pgfscope}%
\end{pgfscope}%
\begin{pgfscope}%
\definecolor{textcolor}{rgb}{0.000000,0.000000,0.000000}%
\pgfsetstrokecolor{textcolor}%
\pgfsetfillcolor{textcolor}%
\pgftext[x=0.289968in, y=1.919078in, left, base]{\color{textcolor}{\rmfamily\fontsize{10.000000}{12.000000}\selectfont\catcode`\^=\active\def^{\ifmmode\sp\else\^{}\fi}\catcode`\%=\active\def%{\%}$\mathdefault{10^{3}}$}}%
\end{pgfscope}%
\begin{pgfscope}%
\pgfsetbuttcap%
\pgfsetroundjoin%
\definecolor{currentfill}{rgb}{0.000000,0.000000,0.000000}%
\pgfsetfillcolor{currentfill}%
\pgfsetlinewidth{0.803000pt}%
\definecolor{currentstroke}{rgb}{0.000000,0.000000,0.000000}%
\pgfsetstrokecolor{currentstroke}%
\pgfsetdash{}{0pt}%
\pgfsys@defobject{currentmarker}{\pgfqpoint{-0.048611in}{0.000000in}}{\pgfqpoint{-0.000000in}{0.000000in}}{%
\pgfpathmoveto{\pgfqpoint{-0.000000in}{0.000000in}}%
\pgfpathlineto{\pgfqpoint{-0.048611in}{0.000000in}}%
\pgfusepath{stroke,fill}%
}%
\begin{pgfscope}%
\pgfsys@transformshift{0.588387in}{2.421612in}%
\pgfsys@useobject{currentmarker}{}%
\end{pgfscope}%
\end{pgfscope}%
\begin{pgfscope}%
\definecolor{textcolor}{rgb}{0.000000,0.000000,0.000000}%
\pgfsetstrokecolor{textcolor}%
\pgfsetfillcolor{textcolor}%
\pgftext[x=0.289968in, y=2.368850in, left, base]{\color{textcolor}{\rmfamily\fontsize{10.000000}{12.000000}\selectfont\catcode`\^=\active\def^{\ifmmode\sp\else\^{}\fi}\catcode`\%=\active\def%{\%}$\mathdefault{10^{4}}$}}%
\end{pgfscope}%
\begin{pgfscope}%
\pgfsetbuttcap%
\pgfsetroundjoin%
\definecolor{currentfill}{rgb}{0.000000,0.000000,0.000000}%
\pgfsetfillcolor{currentfill}%
\pgfsetlinewidth{0.602250pt}%
\definecolor{currentstroke}{rgb}{0.000000,0.000000,0.000000}%
\pgfsetstrokecolor{currentstroke}%
\pgfsetdash{}{0pt}%
\pgfsys@defobject{currentmarker}{\pgfqpoint{-0.027778in}{0.000000in}}{\pgfqpoint{-0.000000in}{0.000000in}}{%
\pgfpathmoveto{\pgfqpoint{-0.000000in}{0.000000in}}%
\pgfpathlineto{\pgfqpoint{-0.027778in}{0.000000in}}%
\pgfusepath{stroke,fill}%
}%
\begin{pgfscope}%
\pgfsys@transformshift{0.588387in}{0.522742in}%
\pgfsys@useobject{currentmarker}{}%
\end{pgfscope}%
\end{pgfscope}%
\begin{pgfscope}%
\pgfsetbuttcap%
\pgfsetroundjoin%
\definecolor{currentfill}{rgb}{0.000000,0.000000,0.000000}%
\pgfsetfillcolor{currentfill}%
\pgfsetlinewidth{0.602250pt}%
\definecolor{currentstroke}{rgb}{0.000000,0.000000,0.000000}%
\pgfsetstrokecolor{currentstroke}%
\pgfsetdash{}{0pt}%
\pgfsys@defobject{currentmarker}{\pgfqpoint{-0.027778in}{0.000000in}}{\pgfqpoint{-0.000000in}{0.000000in}}{%
\pgfpathmoveto{\pgfqpoint{-0.000000in}{0.000000in}}%
\pgfpathlineto{\pgfqpoint{-0.027778in}{0.000000in}}%
\pgfusepath{stroke,fill}%
}%
\begin{pgfscope}%
\pgfsys@transformshift{0.588387in}{0.552853in}%
\pgfsys@useobject{currentmarker}{}%
\end{pgfscope}%
\end{pgfscope}%
\begin{pgfscope}%
\pgfsetbuttcap%
\pgfsetroundjoin%
\definecolor{currentfill}{rgb}{0.000000,0.000000,0.000000}%
\pgfsetfillcolor{currentfill}%
\pgfsetlinewidth{0.602250pt}%
\definecolor{currentstroke}{rgb}{0.000000,0.000000,0.000000}%
\pgfsetstrokecolor{currentstroke}%
\pgfsetdash{}{0pt}%
\pgfsys@defobject{currentmarker}{\pgfqpoint{-0.027778in}{0.000000in}}{\pgfqpoint{-0.000000in}{0.000000in}}{%
\pgfpathmoveto{\pgfqpoint{-0.000000in}{0.000000in}}%
\pgfpathlineto{\pgfqpoint{-0.027778in}{0.000000in}}%
\pgfusepath{stroke,fill}%
}%
\begin{pgfscope}%
\pgfsys@transformshift{0.588387in}{0.578936in}%
\pgfsys@useobject{currentmarker}{}%
\end{pgfscope}%
\end{pgfscope}%
\begin{pgfscope}%
\pgfsetbuttcap%
\pgfsetroundjoin%
\definecolor{currentfill}{rgb}{0.000000,0.000000,0.000000}%
\pgfsetfillcolor{currentfill}%
\pgfsetlinewidth{0.602250pt}%
\definecolor{currentstroke}{rgb}{0.000000,0.000000,0.000000}%
\pgfsetstrokecolor{currentstroke}%
\pgfsetdash{}{0pt}%
\pgfsys@defobject{currentmarker}{\pgfqpoint{-0.027778in}{0.000000in}}{\pgfqpoint{-0.000000in}{0.000000in}}{%
\pgfpathmoveto{\pgfqpoint{-0.000000in}{0.000000in}}%
\pgfpathlineto{\pgfqpoint{-0.027778in}{0.000000in}}%
\pgfusepath{stroke,fill}%
}%
\begin{pgfscope}%
\pgfsys@transformshift{0.588387in}{0.601943in}%
\pgfsys@useobject{currentmarker}{}%
\end{pgfscope}%
\end{pgfscope}%
\begin{pgfscope}%
\pgfsetbuttcap%
\pgfsetroundjoin%
\definecolor{currentfill}{rgb}{0.000000,0.000000,0.000000}%
\pgfsetfillcolor{currentfill}%
\pgfsetlinewidth{0.602250pt}%
\definecolor{currentstroke}{rgb}{0.000000,0.000000,0.000000}%
\pgfsetstrokecolor{currentstroke}%
\pgfsetdash{}{0pt}%
\pgfsys@defobject{currentmarker}{\pgfqpoint{-0.027778in}{0.000000in}}{\pgfqpoint{-0.000000in}{0.000000in}}{%
\pgfpathmoveto{\pgfqpoint{-0.000000in}{0.000000in}}%
\pgfpathlineto{\pgfqpoint{-0.027778in}{0.000000in}}%
\pgfusepath{stroke,fill}%
}%
\begin{pgfscope}%
\pgfsys@transformshift{0.588387in}{0.757918in}%
\pgfsys@useobject{currentmarker}{}%
\end{pgfscope}%
\end{pgfscope}%
\begin{pgfscope}%
\pgfsetbuttcap%
\pgfsetroundjoin%
\definecolor{currentfill}{rgb}{0.000000,0.000000,0.000000}%
\pgfsetfillcolor{currentfill}%
\pgfsetlinewidth{0.602250pt}%
\definecolor{currentstroke}{rgb}{0.000000,0.000000,0.000000}%
\pgfsetstrokecolor{currentstroke}%
\pgfsetdash{}{0pt}%
\pgfsys@defobject{currentmarker}{\pgfqpoint{-0.027778in}{0.000000in}}{\pgfqpoint{-0.000000in}{0.000000in}}{%
\pgfpathmoveto{\pgfqpoint{-0.000000in}{0.000000in}}%
\pgfpathlineto{\pgfqpoint{-0.027778in}{0.000000in}}%
\pgfusepath{stroke,fill}%
}%
\begin{pgfscope}%
\pgfsys@transformshift{0.588387in}{0.837119in}%
\pgfsys@useobject{currentmarker}{}%
\end{pgfscope}%
\end{pgfscope}%
\begin{pgfscope}%
\pgfsetbuttcap%
\pgfsetroundjoin%
\definecolor{currentfill}{rgb}{0.000000,0.000000,0.000000}%
\pgfsetfillcolor{currentfill}%
\pgfsetlinewidth{0.602250pt}%
\definecolor{currentstroke}{rgb}{0.000000,0.000000,0.000000}%
\pgfsetstrokecolor{currentstroke}%
\pgfsetdash{}{0pt}%
\pgfsys@defobject{currentmarker}{\pgfqpoint{-0.027778in}{0.000000in}}{\pgfqpoint{-0.000000in}{0.000000in}}{%
\pgfpathmoveto{\pgfqpoint{-0.000000in}{0.000000in}}%
\pgfpathlineto{\pgfqpoint{-0.027778in}{0.000000in}}%
\pgfusepath{stroke,fill}%
}%
\begin{pgfscope}%
\pgfsys@transformshift{0.588387in}{0.893313in}%
\pgfsys@useobject{currentmarker}{}%
\end{pgfscope}%
\end{pgfscope}%
\begin{pgfscope}%
\pgfsetbuttcap%
\pgfsetroundjoin%
\definecolor{currentfill}{rgb}{0.000000,0.000000,0.000000}%
\pgfsetfillcolor{currentfill}%
\pgfsetlinewidth{0.602250pt}%
\definecolor{currentstroke}{rgb}{0.000000,0.000000,0.000000}%
\pgfsetstrokecolor{currentstroke}%
\pgfsetdash{}{0pt}%
\pgfsys@defobject{currentmarker}{\pgfqpoint{-0.027778in}{0.000000in}}{\pgfqpoint{-0.000000in}{0.000000in}}{%
\pgfpathmoveto{\pgfqpoint{-0.000000in}{0.000000in}}%
\pgfpathlineto{\pgfqpoint{-0.027778in}{0.000000in}}%
\pgfusepath{stroke,fill}%
}%
\begin{pgfscope}%
\pgfsys@transformshift{0.588387in}{0.936901in}%
\pgfsys@useobject{currentmarker}{}%
\end{pgfscope}%
\end{pgfscope}%
\begin{pgfscope}%
\pgfsetbuttcap%
\pgfsetroundjoin%
\definecolor{currentfill}{rgb}{0.000000,0.000000,0.000000}%
\pgfsetfillcolor{currentfill}%
\pgfsetlinewidth{0.602250pt}%
\definecolor{currentstroke}{rgb}{0.000000,0.000000,0.000000}%
\pgfsetstrokecolor{currentstroke}%
\pgfsetdash{}{0pt}%
\pgfsys@defobject{currentmarker}{\pgfqpoint{-0.027778in}{0.000000in}}{\pgfqpoint{-0.000000in}{0.000000in}}{%
\pgfpathmoveto{\pgfqpoint{-0.000000in}{0.000000in}}%
\pgfpathlineto{\pgfqpoint{-0.027778in}{0.000000in}}%
\pgfusepath{stroke,fill}%
}%
\begin{pgfscope}%
\pgfsys@transformshift{0.588387in}{0.972514in}%
\pgfsys@useobject{currentmarker}{}%
\end{pgfscope}%
\end{pgfscope}%
\begin{pgfscope}%
\pgfsetbuttcap%
\pgfsetroundjoin%
\definecolor{currentfill}{rgb}{0.000000,0.000000,0.000000}%
\pgfsetfillcolor{currentfill}%
\pgfsetlinewidth{0.602250pt}%
\definecolor{currentstroke}{rgb}{0.000000,0.000000,0.000000}%
\pgfsetstrokecolor{currentstroke}%
\pgfsetdash{}{0pt}%
\pgfsys@defobject{currentmarker}{\pgfqpoint{-0.027778in}{0.000000in}}{\pgfqpoint{-0.000000in}{0.000000in}}{%
\pgfpathmoveto{\pgfqpoint{-0.000000in}{0.000000in}}%
\pgfpathlineto{\pgfqpoint{-0.027778in}{0.000000in}}%
\pgfusepath{stroke,fill}%
}%
\begin{pgfscope}%
\pgfsys@transformshift{0.588387in}{1.002625in}%
\pgfsys@useobject{currentmarker}{}%
\end{pgfscope}%
\end{pgfscope}%
\begin{pgfscope}%
\pgfsetbuttcap%
\pgfsetroundjoin%
\definecolor{currentfill}{rgb}{0.000000,0.000000,0.000000}%
\pgfsetfillcolor{currentfill}%
\pgfsetlinewidth{0.602250pt}%
\definecolor{currentstroke}{rgb}{0.000000,0.000000,0.000000}%
\pgfsetstrokecolor{currentstroke}%
\pgfsetdash{}{0pt}%
\pgfsys@defobject{currentmarker}{\pgfqpoint{-0.027778in}{0.000000in}}{\pgfqpoint{-0.000000in}{0.000000in}}{%
\pgfpathmoveto{\pgfqpoint{-0.000000in}{0.000000in}}%
\pgfpathlineto{\pgfqpoint{-0.027778in}{0.000000in}}%
\pgfusepath{stroke,fill}%
}%
\begin{pgfscope}%
\pgfsys@transformshift{0.588387in}{1.028708in}%
\pgfsys@useobject{currentmarker}{}%
\end{pgfscope}%
\end{pgfscope}%
\begin{pgfscope}%
\pgfsetbuttcap%
\pgfsetroundjoin%
\definecolor{currentfill}{rgb}{0.000000,0.000000,0.000000}%
\pgfsetfillcolor{currentfill}%
\pgfsetlinewidth{0.602250pt}%
\definecolor{currentstroke}{rgb}{0.000000,0.000000,0.000000}%
\pgfsetstrokecolor{currentstroke}%
\pgfsetdash{}{0pt}%
\pgfsys@defobject{currentmarker}{\pgfqpoint{-0.027778in}{0.000000in}}{\pgfqpoint{-0.000000in}{0.000000in}}{%
\pgfpathmoveto{\pgfqpoint{-0.000000in}{0.000000in}}%
\pgfpathlineto{\pgfqpoint{-0.027778in}{0.000000in}}%
\pgfusepath{stroke,fill}%
}%
\begin{pgfscope}%
\pgfsys@transformshift{0.588387in}{1.051715in}%
\pgfsys@useobject{currentmarker}{}%
\end{pgfscope}%
\end{pgfscope}%
\begin{pgfscope}%
\pgfsetbuttcap%
\pgfsetroundjoin%
\definecolor{currentfill}{rgb}{0.000000,0.000000,0.000000}%
\pgfsetfillcolor{currentfill}%
\pgfsetlinewidth{0.602250pt}%
\definecolor{currentstroke}{rgb}{0.000000,0.000000,0.000000}%
\pgfsetstrokecolor{currentstroke}%
\pgfsetdash{}{0pt}%
\pgfsys@defobject{currentmarker}{\pgfqpoint{-0.027778in}{0.000000in}}{\pgfqpoint{-0.000000in}{0.000000in}}{%
\pgfpathmoveto{\pgfqpoint{-0.000000in}{0.000000in}}%
\pgfpathlineto{\pgfqpoint{-0.027778in}{0.000000in}}%
\pgfusepath{stroke,fill}%
}%
\begin{pgfscope}%
\pgfsys@transformshift{0.588387in}{1.207690in}%
\pgfsys@useobject{currentmarker}{}%
\end{pgfscope}%
\end{pgfscope}%
\begin{pgfscope}%
\pgfsetbuttcap%
\pgfsetroundjoin%
\definecolor{currentfill}{rgb}{0.000000,0.000000,0.000000}%
\pgfsetfillcolor{currentfill}%
\pgfsetlinewidth{0.602250pt}%
\definecolor{currentstroke}{rgb}{0.000000,0.000000,0.000000}%
\pgfsetstrokecolor{currentstroke}%
\pgfsetdash{}{0pt}%
\pgfsys@defobject{currentmarker}{\pgfqpoint{-0.027778in}{0.000000in}}{\pgfqpoint{-0.000000in}{0.000000in}}{%
\pgfpathmoveto{\pgfqpoint{-0.000000in}{0.000000in}}%
\pgfpathlineto{\pgfqpoint{-0.027778in}{0.000000in}}%
\pgfusepath{stroke,fill}%
}%
\begin{pgfscope}%
\pgfsys@transformshift{0.588387in}{1.286891in}%
\pgfsys@useobject{currentmarker}{}%
\end{pgfscope}%
\end{pgfscope}%
\begin{pgfscope}%
\pgfsetbuttcap%
\pgfsetroundjoin%
\definecolor{currentfill}{rgb}{0.000000,0.000000,0.000000}%
\pgfsetfillcolor{currentfill}%
\pgfsetlinewidth{0.602250pt}%
\definecolor{currentstroke}{rgb}{0.000000,0.000000,0.000000}%
\pgfsetstrokecolor{currentstroke}%
\pgfsetdash{}{0pt}%
\pgfsys@defobject{currentmarker}{\pgfqpoint{-0.027778in}{0.000000in}}{\pgfqpoint{-0.000000in}{0.000000in}}{%
\pgfpathmoveto{\pgfqpoint{-0.000000in}{0.000000in}}%
\pgfpathlineto{\pgfqpoint{-0.027778in}{0.000000in}}%
\pgfusepath{stroke,fill}%
}%
\begin{pgfscope}%
\pgfsys@transformshift{0.588387in}{1.343085in}%
\pgfsys@useobject{currentmarker}{}%
\end{pgfscope}%
\end{pgfscope}%
\begin{pgfscope}%
\pgfsetbuttcap%
\pgfsetroundjoin%
\definecolor{currentfill}{rgb}{0.000000,0.000000,0.000000}%
\pgfsetfillcolor{currentfill}%
\pgfsetlinewidth{0.602250pt}%
\definecolor{currentstroke}{rgb}{0.000000,0.000000,0.000000}%
\pgfsetstrokecolor{currentstroke}%
\pgfsetdash{}{0pt}%
\pgfsys@defobject{currentmarker}{\pgfqpoint{-0.027778in}{0.000000in}}{\pgfqpoint{-0.000000in}{0.000000in}}{%
\pgfpathmoveto{\pgfqpoint{-0.000000in}{0.000000in}}%
\pgfpathlineto{\pgfqpoint{-0.027778in}{0.000000in}}%
\pgfusepath{stroke,fill}%
}%
\begin{pgfscope}%
\pgfsys@transformshift{0.588387in}{1.386673in}%
\pgfsys@useobject{currentmarker}{}%
\end{pgfscope}%
\end{pgfscope}%
\begin{pgfscope}%
\pgfsetbuttcap%
\pgfsetroundjoin%
\definecolor{currentfill}{rgb}{0.000000,0.000000,0.000000}%
\pgfsetfillcolor{currentfill}%
\pgfsetlinewidth{0.602250pt}%
\definecolor{currentstroke}{rgb}{0.000000,0.000000,0.000000}%
\pgfsetstrokecolor{currentstroke}%
\pgfsetdash{}{0pt}%
\pgfsys@defobject{currentmarker}{\pgfqpoint{-0.027778in}{0.000000in}}{\pgfqpoint{-0.000000in}{0.000000in}}{%
\pgfpathmoveto{\pgfqpoint{-0.000000in}{0.000000in}}%
\pgfpathlineto{\pgfqpoint{-0.027778in}{0.000000in}}%
\pgfusepath{stroke,fill}%
}%
\begin{pgfscope}%
\pgfsys@transformshift{0.588387in}{1.422286in}%
\pgfsys@useobject{currentmarker}{}%
\end{pgfscope}%
\end{pgfscope}%
\begin{pgfscope}%
\pgfsetbuttcap%
\pgfsetroundjoin%
\definecolor{currentfill}{rgb}{0.000000,0.000000,0.000000}%
\pgfsetfillcolor{currentfill}%
\pgfsetlinewidth{0.602250pt}%
\definecolor{currentstroke}{rgb}{0.000000,0.000000,0.000000}%
\pgfsetstrokecolor{currentstroke}%
\pgfsetdash{}{0pt}%
\pgfsys@defobject{currentmarker}{\pgfqpoint{-0.027778in}{0.000000in}}{\pgfqpoint{-0.000000in}{0.000000in}}{%
\pgfpathmoveto{\pgfqpoint{-0.000000in}{0.000000in}}%
\pgfpathlineto{\pgfqpoint{-0.027778in}{0.000000in}}%
\pgfusepath{stroke,fill}%
}%
\begin{pgfscope}%
\pgfsys@transformshift{0.588387in}{1.452397in}%
\pgfsys@useobject{currentmarker}{}%
\end{pgfscope}%
\end{pgfscope}%
\begin{pgfscope}%
\pgfsetbuttcap%
\pgfsetroundjoin%
\definecolor{currentfill}{rgb}{0.000000,0.000000,0.000000}%
\pgfsetfillcolor{currentfill}%
\pgfsetlinewidth{0.602250pt}%
\definecolor{currentstroke}{rgb}{0.000000,0.000000,0.000000}%
\pgfsetstrokecolor{currentstroke}%
\pgfsetdash{}{0pt}%
\pgfsys@defobject{currentmarker}{\pgfqpoint{-0.027778in}{0.000000in}}{\pgfqpoint{-0.000000in}{0.000000in}}{%
\pgfpathmoveto{\pgfqpoint{-0.000000in}{0.000000in}}%
\pgfpathlineto{\pgfqpoint{-0.027778in}{0.000000in}}%
\pgfusepath{stroke,fill}%
}%
\begin{pgfscope}%
\pgfsys@transformshift{0.588387in}{1.478480in}%
\pgfsys@useobject{currentmarker}{}%
\end{pgfscope}%
\end{pgfscope}%
\begin{pgfscope}%
\pgfsetbuttcap%
\pgfsetroundjoin%
\definecolor{currentfill}{rgb}{0.000000,0.000000,0.000000}%
\pgfsetfillcolor{currentfill}%
\pgfsetlinewidth{0.602250pt}%
\definecolor{currentstroke}{rgb}{0.000000,0.000000,0.000000}%
\pgfsetstrokecolor{currentstroke}%
\pgfsetdash{}{0pt}%
\pgfsys@defobject{currentmarker}{\pgfqpoint{-0.027778in}{0.000000in}}{\pgfqpoint{-0.000000in}{0.000000in}}{%
\pgfpathmoveto{\pgfqpoint{-0.000000in}{0.000000in}}%
\pgfpathlineto{\pgfqpoint{-0.027778in}{0.000000in}}%
\pgfusepath{stroke,fill}%
}%
\begin{pgfscope}%
\pgfsys@transformshift{0.588387in}{1.501487in}%
\pgfsys@useobject{currentmarker}{}%
\end{pgfscope}%
\end{pgfscope}%
\begin{pgfscope}%
\pgfsetbuttcap%
\pgfsetroundjoin%
\definecolor{currentfill}{rgb}{0.000000,0.000000,0.000000}%
\pgfsetfillcolor{currentfill}%
\pgfsetlinewidth{0.602250pt}%
\definecolor{currentstroke}{rgb}{0.000000,0.000000,0.000000}%
\pgfsetstrokecolor{currentstroke}%
\pgfsetdash{}{0pt}%
\pgfsys@defobject{currentmarker}{\pgfqpoint{-0.027778in}{0.000000in}}{\pgfqpoint{-0.000000in}{0.000000in}}{%
\pgfpathmoveto{\pgfqpoint{-0.000000in}{0.000000in}}%
\pgfpathlineto{\pgfqpoint{-0.027778in}{0.000000in}}%
\pgfusepath{stroke,fill}%
}%
\begin{pgfscope}%
\pgfsys@transformshift{0.588387in}{1.657462in}%
\pgfsys@useobject{currentmarker}{}%
\end{pgfscope}%
\end{pgfscope}%
\begin{pgfscope}%
\pgfsetbuttcap%
\pgfsetroundjoin%
\definecolor{currentfill}{rgb}{0.000000,0.000000,0.000000}%
\pgfsetfillcolor{currentfill}%
\pgfsetlinewidth{0.602250pt}%
\definecolor{currentstroke}{rgb}{0.000000,0.000000,0.000000}%
\pgfsetstrokecolor{currentstroke}%
\pgfsetdash{}{0pt}%
\pgfsys@defobject{currentmarker}{\pgfqpoint{-0.027778in}{0.000000in}}{\pgfqpoint{-0.000000in}{0.000000in}}{%
\pgfpathmoveto{\pgfqpoint{-0.000000in}{0.000000in}}%
\pgfpathlineto{\pgfqpoint{-0.027778in}{0.000000in}}%
\pgfusepath{stroke,fill}%
}%
\begin{pgfscope}%
\pgfsys@transformshift{0.588387in}{1.736663in}%
\pgfsys@useobject{currentmarker}{}%
\end{pgfscope}%
\end{pgfscope}%
\begin{pgfscope}%
\pgfsetbuttcap%
\pgfsetroundjoin%
\definecolor{currentfill}{rgb}{0.000000,0.000000,0.000000}%
\pgfsetfillcolor{currentfill}%
\pgfsetlinewidth{0.602250pt}%
\definecolor{currentstroke}{rgb}{0.000000,0.000000,0.000000}%
\pgfsetstrokecolor{currentstroke}%
\pgfsetdash{}{0pt}%
\pgfsys@defobject{currentmarker}{\pgfqpoint{-0.027778in}{0.000000in}}{\pgfqpoint{-0.000000in}{0.000000in}}{%
\pgfpathmoveto{\pgfqpoint{-0.000000in}{0.000000in}}%
\pgfpathlineto{\pgfqpoint{-0.027778in}{0.000000in}}%
\pgfusepath{stroke,fill}%
}%
\begin{pgfscope}%
\pgfsys@transformshift{0.588387in}{1.792857in}%
\pgfsys@useobject{currentmarker}{}%
\end{pgfscope}%
\end{pgfscope}%
\begin{pgfscope}%
\pgfsetbuttcap%
\pgfsetroundjoin%
\definecolor{currentfill}{rgb}{0.000000,0.000000,0.000000}%
\pgfsetfillcolor{currentfill}%
\pgfsetlinewidth{0.602250pt}%
\definecolor{currentstroke}{rgb}{0.000000,0.000000,0.000000}%
\pgfsetstrokecolor{currentstroke}%
\pgfsetdash{}{0pt}%
\pgfsys@defobject{currentmarker}{\pgfqpoint{-0.027778in}{0.000000in}}{\pgfqpoint{-0.000000in}{0.000000in}}{%
\pgfpathmoveto{\pgfqpoint{-0.000000in}{0.000000in}}%
\pgfpathlineto{\pgfqpoint{-0.027778in}{0.000000in}}%
\pgfusepath{stroke,fill}%
}%
\begin{pgfscope}%
\pgfsys@transformshift{0.588387in}{1.836445in}%
\pgfsys@useobject{currentmarker}{}%
\end{pgfscope}%
\end{pgfscope}%
\begin{pgfscope}%
\pgfsetbuttcap%
\pgfsetroundjoin%
\definecolor{currentfill}{rgb}{0.000000,0.000000,0.000000}%
\pgfsetfillcolor{currentfill}%
\pgfsetlinewidth{0.602250pt}%
\definecolor{currentstroke}{rgb}{0.000000,0.000000,0.000000}%
\pgfsetstrokecolor{currentstroke}%
\pgfsetdash{}{0pt}%
\pgfsys@defobject{currentmarker}{\pgfqpoint{-0.027778in}{0.000000in}}{\pgfqpoint{-0.000000in}{0.000000in}}{%
\pgfpathmoveto{\pgfqpoint{-0.000000in}{0.000000in}}%
\pgfpathlineto{\pgfqpoint{-0.027778in}{0.000000in}}%
\pgfusepath{stroke,fill}%
}%
\begin{pgfscope}%
\pgfsys@transformshift{0.588387in}{1.872058in}%
\pgfsys@useobject{currentmarker}{}%
\end{pgfscope}%
\end{pgfscope}%
\begin{pgfscope}%
\pgfsetbuttcap%
\pgfsetroundjoin%
\definecolor{currentfill}{rgb}{0.000000,0.000000,0.000000}%
\pgfsetfillcolor{currentfill}%
\pgfsetlinewidth{0.602250pt}%
\definecolor{currentstroke}{rgb}{0.000000,0.000000,0.000000}%
\pgfsetstrokecolor{currentstroke}%
\pgfsetdash{}{0pt}%
\pgfsys@defobject{currentmarker}{\pgfqpoint{-0.027778in}{0.000000in}}{\pgfqpoint{-0.000000in}{0.000000in}}{%
\pgfpathmoveto{\pgfqpoint{-0.000000in}{0.000000in}}%
\pgfpathlineto{\pgfqpoint{-0.027778in}{0.000000in}}%
\pgfusepath{stroke,fill}%
}%
\begin{pgfscope}%
\pgfsys@transformshift{0.588387in}{1.902169in}%
\pgfsys@useobject{currentmarker}{}%
\end{pgfscope}%
\end{pgfscope}%
\begin{pgfscope}%
\pgfsetbuttcap%
\pgfsetroundjoin%
\definecolor{currentfill}{rgb}{0.000000,0.000000,0.000000}%
\pgfsetfillcolor{currentfill}%
\pgfsetlinewidth{0.602250pt}%
\definecolor{currentstroke}{rgb}{0.000000,0.000000,0.000000}%
\pgfsetstrokecolor{currentstroke}%
\pgfsetdash{}{0pt}%
\pgfsys@defobject{currentmarker}{\pgfqpoint{-0.027778in}{0.000000in}}{\pgfqpoint{-0.000000in}{0.000000in}}{%
\pgfpathmoveto{\pgfqpoint{-0.000000in}{0.000000in}}%
\pgfpathlineto{\pgfqpoint{-0.027778in}{0.000000in}}%
\pgfusepath{stroke,fill}%
}%
\begin{pgfscope}%
\pgfsys@transformshift{0.588387in}{1.928252in}%
\pgfsys@useobject{currentmarker}{}%
\end{pgfscope}%
\end{pgfscope}%
\begin{pgfscope}%
\pgfsetbuttcap%
\pgfsetroundjoin%
\definecolor{currentfill}{rgb}{0.000000,0.000000,0.000000}%
\pgfsetfillcolor{currentfill}%
\pgfsetlinewidth{0.602250pt}%
\definecolor{currentstroke}{rgb}{0.000000,0.000000,0.000000}%
\pgfsetstrokecolor{currentstroke}%
\pgfsetdash{}{0pt}%
\pgfsys@defobject{currentmarker}{\pgfqpoint{-0.027778in}{0.000000in}}{\pgfqpoint{-0.000000in}{0.000000in}}{%
\pgfpathmoveto{\pgfqpoint{-0.000000in}{0.000000in}}%
\pgfpathlineto{\pgfqpoint{-0.027778in}{0.000000in}}%
\pgfusepath{stroke,fill}%
}%
\begin{pgfscope}%
\pgfsys@transformshift{0.588387in}{1.951259in}%
\pgfsys@useobject{currentmarker}{}%
\end{pgfscope}%
\end{pgfscope}%
\begin{pgfscope}%
\pgfsetbuttcap%
\pgfsetroundjoin%
\definecolor{currentfill}{rgb}{0.000000,0.000000,0.000000}%
\pgfsetfillcolor{currentfill}%
\pgfsetlinewidth{0.602250pt}%
\definecolor{currentstroke}{rgb}{0.000000,0.000000,0.000000}%
\pgfsetstrokecolor{currentstroke}%
\pgfsetdash{}{0pt}%
\pgfsys@defobject{currentmarker}{\pgfqpoint{-0.027778in}{0.000000in}}{\pgfqpoint{-0.000000in}{0.000000in}}{%
\pgfpathmoveto{\pgfqpoint{-0.000000in}{0.000000in}}%
\pgfpathlineto{\pgfqpoint{-0.027778in}{0.000000in}}%
\pgfusepath{stroke,fill}%
}%
\begin{pgfscope}%
\pgfsys@transformshift{0.588387in}{2.107234in}%
\pgfsys@useobject{currentmarker}{}%
\end{pgfscope}%
\end{pgfscope}%
\begin{pgfscope}%
\pgfsetbuttcap%
\pgfsetroundjoin%
\definecolor{currentfill}{rgb}{0.000000,0.000000,0.000000}%
\pgfsetfillcolor{currentfill}%
\pgfsetlinewidth{0.602250pt}%
\definecolor{currentstroke}{rgb}{0.000000,0.000000,0.000000}%
\pgfsetstrokecolor{currentstroke}%
\pgfsetdash{}{0pt}%
\pgfsys@defobject{currentmarker}{\pgfqpoint{-0.027778in}{0.000000in}}{\pgfqpoint{-0.000000in}{0.000000in}}{%
\pgfpathmoveto{\pgfqpoint{-0.000000in}{0.000000in}}%
\pgfpathlineto{\pgfqpoint{-0.027778in}{0.000000in}}%
\pgfusepath{stroke,fill}%
}%
\begin{pgfscope}%
\pgfsys@transformshift{0.588387in}{2.186435in}%
\pgfsys@useobject{currentmarker}{}%
\end{pgfscope}%
\end{pgfscope}%
\begin{pgfscope}%
\pgfsetbuttcap%
\pgfsetroundjoin%
\definecolor{currentfill}{rgb}{0.000000,0.000000,0.000000}%
\pgfsetfillcolor{currentfill}%
\pgfsetlinewidth{0.602250pt}%
\definecolor{currentstroke}{rgb}{0.000000,0.000000,0.000000}%
\pgfsetstrokecolor{currentstroke}%
\pgfsetdash{}{0pt}%
\pgfsys@defobject{currentmarker}{\pgfqpoint{-0.027778in}{0.000000in}}{\pgfqpoint{-0.000000in}{0.000000in}}{%
\pgfpathmoveto{\pgfqpoint{-0.000000in}{0.000000in}}%
\pgfpathlineto{\pgfqpoint{-0.027778in}{0.000000in}}%
\pgfusepath{stroke,fill}%
}%
\begin{pgfscope}%
\pgfsys@transformshift{0.588387in}{2.242629in}%
\pgfsys@useobject{currentmarker}{}%
\end{pgfscope}%
\end{pgfscope}%
\begin{pgfscope}%
\pgfsetbuttcap%
\pgfsetroundjoin%
\definecolor{currentfill}{rgb}{0.000000,0.000000,0.000000}%
\pgfsetfillcolor{currentfill}%
\pgfsetlinewidth{0.602250pt}%
\definecolor{currentstroke}{rgb}{0.000000,0.000000,0.000000}%
\pgfsetstrokecolor{currentstroke}%
\pgfsetdash{}{0pt}%
\pgfsys@defobject{currentmarker}{\pgfqpoint{-0.027778in}{0.000000in}}{\pgfqpoint{-0.000000in}{0.000000in}}{%
\pgfpathmoveto{\pgfqpoint{-0.000000in}{0.000000in}}%
\pgfpathlineto{\pgfqpoint{-0.027778in}{0.000000in}}%
\pgfusepath{stroke,fill}%
}%
\begin{pgfscope}%
\pgfsys@transformshift{0.588387in}{2.286217in}%
\pgfsys@useobject{currentmarker}{}%
\end{pgfscope}%
\end{pgfscope}%
\begin{pgfscope}%
\pgfsetbuttcap%
\pgfsetroundjoin%
\definecolor{currentfill}{rgb}{0.000000,0.000000,0.000000}%
\pgfsetfillcolor{currentfill}%
\pgfsetlinewidth{0.602250pt}%
\definecolor{currentstroke}{rgb}{0.000000,0.000000,0.000000}%
\pgfsetstrokecolor{currentstroke}%
\pgfsetdash{}{0pt}%
\pgfsys@defobject{currentmarker}{\pgfqpoint{-0.027778in}{0.000000in}}{\pgfqpoint{-0.000000in}{0.000000in}}{%
\pgfpathmoveto{\pgfqpoint{-0.000000in}{0.000000in}}%
\pgfpathlineto{\pgfqpoint{-0.027778in}{0.000000in}}%
\pgfusepath{stroke,fill}%
}%
\begin{pgfscope}%
\pgfsys@transformshift{0.588387in}{2.321830in}%
\pgfsys@useobject{currentmarker}{}%
\end{pgfscope}%
\end{pgfscope}%
\begin{pgfscope}%
\pgfsetbuttcap%
\pgfsetroundjoin%
\definecolor{currentfill}{rgb}{0.000000,0.000000,0.000000}%
\pgfsetfillcolor{currentfill}%
\pgfsetlinewidth{0.602250pt}%
\definecolor{currentstroke}{rgb}{0.000000,0.000000,0.000000}%
\pgfsetstrokecolor{currentstroke}%
\pgfsetdash{}{0pt}%
\pgfsys@defobject{currentmarker}{\pgfqpoint{-0.027778in}{0.000000in}}{\pgfqpoint{-0.000000in}{0.000000in}}{%
\pgfpathmoveto{\pgfqpoint{-0.000000in}{0.000000in}}%
\pgfpathlineto{\pgfqpoint{-0.027778in}{0.000000in}}%
\pgfusepath{stroke,fill}%
}%
\begin{pgfscope}%
\pgfsys@transformshift{0.588387in}{2.351941in}%
\pgfsys@useobject{currentmarker}{}%
\end{pgfscope}%
\end{pgfscope}%
\begin{pgfscope}%
\pgfsetbuttcap%
\pgfsetroundjoin%
\definecolor{currentfill}{rgb}{0.000000,0.000000,0.000000}%
\pgfsetfillcolor{currentfill}%
\pgfsetlinewidth{0.602250pt}%
\definecolor{currentstroke}{rgb}{0.000000,0.000000,0.000000}%
\pgfsetstrokecolor{currentstroke}%
\pgfsetdash{}{0pt}%
\pgfsys@defobject{currentmarker}{\pgfqpoint{-0.027778in}{0.000000in}}{\pgfqpoint{-0.000000in}{0.000000in}}{%
\pgfpathmoveto{\pgfqpoint{-0.000000in}{0.000000in}}%
\pgfpathlineto{\pgfqpoint{-0.027778in}{0.000000in}}%
\pgfusepath{stroke,fill}%
}%
\begin{pgfscope}%
\pgfsys@transformshift{0.588387in}{2.378024in}%
\pgfsys@useobject{currentmarker}{}%
\end{pgfscope}%
\end{pgfscope}%
\begin{pgfscope}%
\pgfsetbuttcap%
\pgfsetroundjoin%
\definecolor{currentfill}{rgb}{0.000000,0.000000,0.000000}%
\pgfsetfillcolor{currentfill}%
\pgfsetlinewidth{0.602250pt}%
\definecolor{currentstroke}{rgb}{0.000000,0.000000,0.000000}%
\pgfsetstrokecolor{currentstroke}%
\pgfsetdash{}{0pt}%
\pgfsys@defobject{currentmarker}{\pgfqpoint{-0.027778in}{0.000000in}}{\pgfqpoint{-0.000000in}{0.000000in}}{%
\pgfpathmoveto{\pgfqpoint{-0.000000in}{0.000000in}}%
\pgfpathlineto{\pgfqpoint{-0.027778in}{0.000000in}}%
\pgfusepath{stroke,fill}%
}%
\begin{pgfscope}%
\pgfsys@transformshift{0.588387in}{2.401031in}%
\pgfsys@useobject{currentmarker}{}%
\end{pgfscope}%
\end{pgfscope}%
\begin{pgfscope}%
\pgfsetbuttcap%
\pgfsetroundjoin%
\definecolor{currentfill}{rgb}{0.000000,0.000000,0.000000}%
\pgfsetfillcolor{currentfill}%
\pgfsetlinewidth{0.602250pt}%
\definecolor{currentstroke}{rgb}{0.000000,0.000000,0.000000}%
\pgfsetstrokecolor{currentstroke}%
\pgfsetdash{}{0pt}%
\pgfsys@defobject{currentmarker}{\pgfqpoint{-0.027778in}{0.000000in}}{\pgfqpoint{-0.000000in}{0.000000in}}{%
\pgfpathmoveto{\pgfqpoint{-0.000000in}{0.000000in}}%
\pgfpathlineto{\pgfqpoint{-0.027778in}{0.000000in}}%
\pgfusepath{stroke,fill}%
}%
\begin{pgfscope}%
\pgfsys@transformshift{0.588387in}{2.557006in}%
\pgfsys@useobject{currentmarker}{}%
\end{pgfscope}%
\end{pgfscope}%
\begin{pgfscope}%
\pgfsetbuttcap%
\pgfsetroundjoin%
\definecolor{currentfill}{rgb}{0.000000,0.000000,0.000000}%
\pgfsetfillcolor{currentfill}%
\pgfsetlinewidth{0.602250pt}%
\definecolor{currentstroke}{rgb}{0.000000,0.000000,0.000000}%
\pgfsetstrokecolor{currentstroke}%
\pgfsetdash{}{0pt}%
\pgfsys@defobject{currentmarker}{\pgfqpoint{-0.027778in}{0.000000in}}{\pgfqpoint{-0.000000in}{0.000000in}}{%
\pgfpathmoveto{\pgfqpoint{-0.000000in}{0.000000in}}%
\pgfpathlineto{\pgfqpoint{-0.027778in}{0.000000in}}%
\pgfusepath{stroke,fill}%
}%
\begin{pgfscope}%
\pgfsys@transformshift{0.588387in}{2.636207in}%
\pgfsys@useobject{currentmarker}{}%
\end{pgfscope}%
\end{pgfscope}%
\begin{pgfscope}%
\pgfsetbuttcap%
\pgfsetroundjoin%
\definecolor{currentfill}{rgb}{0.000000,0.000000,0.000000}%
\pgfsetfillcolor{currentfill}%
\pgfsetlinewidth{0.602250pt}%
\definecolor{currentstroke}{rgb}{0.000000,0.000000,0.000000}%
\pgfsetstrokecolor{currentstroke}%
\pgfsetdash{}{0pt}%
\pgfsys@defobject{currentmarker}{\pgfqpoint{-0.027778in}{0.000000in}}{\pgfqpoint{-0.000000in}{0.000000in}}{%
\pgfpathmoveto{\pgfqpoint{-0.000000in}{0.000000in}}%
\pgfpathlineto{\pgfqpoint{-0.027778in}{0.000000in}}%
\pgfusepath{stroke,fill}%
}%
\begin{pgfscope}%
\pgfsys@transformshift{0.588387in}{2.692401in}%
\pgfsys@useobject{currentmarker}{}%
\end{pgfscope}%
\end{pgfscope}%
\begin{pgfscope}%
\pgfsetbuttcap%
\pgfsetroundjoin%
\definecolor{currentfill}{rgb}{0.000000,0.000000,0.000000}%
\pgfsetfillcolor{currentfill}%
\pgfsetlinewidth{0.602250pt}%
\definecolor{currentstroke}{rgb}{0.000000,0.000000,0.000000}%
\pgfsetstrokecolor{currentstroke}%
\pgfsetdash{}{0pt}%
\pgfsys@defobject{currentmarker}{\pgfqpoint{-0.027778in}{0.000000in}}{\pgfqpoint{-0.000000in}{0.000000in}}{%
\pgfpathmoveto{\pgfqpoint{-0.000000in}{0.000000in}}%
\pgfpathlineto{\pgfqpoint{-0.027778in}{0.000000in}}%
\pgfusepath{stroke,fill}%
}%
\begin{pgfscope}%
\pgfsys@transformshift{0.588387in}{2.735989in}%
\pgfsys@useobject{currentmarker}{}%
\end{pgfscope}%
\end{pgfscope}%
\begin{pgfscope}%
\definecolor{textcolor}{rgb}{0.000000,0.000000,0.000000}%
\pgfsetstrokecolor{textcolor}%
\pgfsetfillcolor{textcolor}%
\pgftext[x=0.234413in,y=1.631726in,,bottom,rotate=90.000000]{\color{textcolor}{\rmfamily\fontsize{10.000000}{12.000000}\selectfont\catcode`\^=\active\def^{\ifmmode\sp\else\^{}\fi}\catcode`\%=\active\def%{\%}Checks [call]}}%
\end{pgfscope}%
\begin{pgfscope}%
\pgfpathrectangle{\pgfqpoint{0.588387in}{0.521603in}}{\pgfqpoint{3.660036in}{2.220246in}}%
\pgfusepath{clip}%
\pgfsetrectcap%
\pgfsetroundjoin%
\pgfsetlinewidth{1.505625pt}%
\pgfsetstrokecolor{currentstroke1}%
\pgfsetdash{}{0pt}%
\pgfpathmoveto{\pgfqpoint{0.754752in}{0.757918in}}%
\pgfpathlineto{\pgfqpoint{0.804414in}{0.787671in}}%
\pgfpathlineto{\pgfqpoint{0.854075in}{0.849456in}}%
\pgfpathlineto{\pgfqpoint{0.903736in}{0.883651in}}%
\pgfpathlineto{\pgfqpoint{0.953398in}{0.943638in}}%
\pgfpathlineto{\pgfqpoint{1.003059in}{1.007369in}}%
\pgfpathlineto{\pgfqpoint{1.052720in}{1.100530in}}%
\pgfpathlineto{\pgfqpoint{1.102381in}{1.168276in}}%
\pgfpathlineto{\pgfqpoint{1.152043in}{1.206775in}}%
\pgfpathlineto{\pgfqpoint{1.201704in}{1.247378in}}%
\pgfpathlineto{\pgfqpoint{1.251365in}{1.333712in}}%
\pgfpathlineto{\pgfqpoint{1.301026in}{1.328792in}}%
\pgfpathlineto{\pgfqpoint{1.350688in}{1.430543in}}%
\pgfpathlineto{\pgfqpoint{1.400349in}{1.448124in}}%
\pgfpathlineto{\pgfqpoint{1.450010in}{1.513866in}}%
\pgfpathlineto{\pgfqpoint{1.499672in}{1.518695in}}%
\pgfpathlineto{\pgfqpoint{1.549333in}{1.439689in}}%
\pgfpathlineto{\pgfqpoint{1.598994in}{1.492390in}}%
\pgfpathlineto{\pgfqpoint{1.648655in}{1.511424in}}%
\pgfpathlineto{\pgfqpoint{1.698317in}{1.535952in}}%
\pgfpathlineto{\pgfqpoint{1.747978in}{1.563635in}}%
\pgfpathlineto{\pgfqpoint{1.797639in}{1.746442in}}%
\pgfpathlineto{\pgfqpoint{1.847300in}{1.533818in}}%
\pgfpathlineto{\pgfqpoint{1.896962in}{1.494661in}}%
\pgfpathlineto{\pgfqpoint{1.946623in}{1.626806in}}%
\pgfpathlineto{\pgfqpoint{1.996284in}{1.639933in}}%
\pgfpathlineto{\pgfqpoint{2.045945in}{1.631007in}}%
\pgfpathlineto{\pgfqpoint{2.095607in}{1.641529in}}%
\pgfpathlineto{\pgfqpoint{2.145268in}{1.640784in}}%
\pgfpathlineto{\pgfqpoint{2.194929in}{1.608538in}}%
\pgfpathlineto{\pgfqpoint{2.244591in}{1.637559in}}%
\pgfpathlineto{\pgfqpoint{2.294252in}{1.636701in}}%
\pgfpathlineto{\pgfqpoint{2.343913in}{1.638503in}}%
\pgfpathlineto{\pgfqpoint{2.393574in}{1.726986in}}%
\pgfpathlineto{\pgfqpoint{2.443236in}{1.691851in}}%
\pgfpathlineto{\pgfqpoint{2.492897in}{1.593295in}}%
\pgfpathlineto{\pgfqpoint{2.542558in}{1.641705in}}%
\pgfpathlineto{\pgfqpoint{2.592219in}{1.714631in}}%
\pgfpathlineto{\pgfqpoint{2.641881in}{1.692872in}}%
\pgfpathlineto{\pgfqpoint{2.691542in}{1.690825in}}%
\pgfpathlineto{\pgfqpoint{2.741203in}{1.653516in}}%
\pgfpathlineto{\pgfqpoint{2.890187in}{1.688652in}}%
\pgfpathlineto{\pgfqpoint{3.039171in}{1.681336in}}%
\pgfpathlineto{\pgfqpoint{3.138493in}{1.760537in}}%
\pgfpathlineto{\pgfqpoint{3.188155in}{1.718241in}}%
\pgfpathlineto{\pgfqpoint{3.436461in}{1.761686in}}%
\pgfusepath{stroke}%
\end{pgfscope}%
\begin{pgfscope}%
\pgfpathrectangle{\pgfqpoint{0.588387in}{0.521603in}}{\pgfqpoint{3.660036in}{2.220246in}}%
\pgfusepath{clip}%
\pgfsetrectcap%
\pgfsetroundjoin%
\pgfsetlinewidth{1.505625pt}%
\pgfsetstrokecolor{currentstroke2}%
\pgfsetdash{}{0pt}%
\pgfpathmoveto{\pgfqpoint{0.754752in}{0.757918in}}%
\pgfpathlineto{\pgfqpoint{0.804414in}{0.787582in}}%
\pgfpathlineto{\pgfqpoint{0.854075in}{0.849510in}}%
\pgfpathlineto{\pgfqpoint{0.903736in}{0.883777in}}%
\pgfpathlineto{\pgfqpoint{0.953398in}{0.944470in}}%
\pgfpathlineto{\pgfqpoint{1.003059in}{1.006984in}}%
\pgfpathlineto{\pgfqpoint{1.052720in}{1.100720in}}%
\pgfpathlineto{\pgfqpoint{1.102381in}{1.167494in}}%
\pgfpathlineto{\pgfqpoint{1.152043in}{1.206493in}}%
\pgfpathlineto{\pgfqpoint{1.201704in}{1.247424in}}%
\pgfpathlineto{\pgfqpoint{1.251365in}{1.342423in}}%
\pgfpathlineto{\pgfqpoint{1.301026in}{1.334859in}}%
\pgfpathlineto{\pgfqpoint{1.350688in}{1.431601in}}%
\pgfpathlineto{\pgfqpoint{1.400349in}{1.453375in}}%
\pgfpathlineto{\pgfqpoint{1.450010in}{1.539781in}}%
\pgfpathlineto{\pgfqpoint{1.499672in}{1.520975in}}%
\pgfpathlineto{\pgfqpoint{1.549333in}{1.440145in}}%
\pgfpathlineto{\pgfqpoint{1.598994in}{1.553354in}}%
\pgfpathlineto{\pgfqpoint{1.648655in}{1.687075in}}%
\pgfpathlineto{\pgfqpoint{1.698317in}{1.633985in}}%
\pgfpathlineto{\pgfqpoint{1.747978in}{1.562504in}}%
\pgfpathlineto{\pgfqpoint{1.797639in}{1.716396in}}%
\pgfpathlineto{\pgfqpoint{1.847300in}{1.874865in}}%
\pgfpathlineto{\pgfqpoint{1.896962in}{1.508844in}}%
\pgfpathlineto{\pgfqpoint{1.946623in}{1.656238in}}%
\pgfpathlineto{\pgfqpoint{1.996284in}{1.798726in}}%
\pgfpathlineto{\pgfqpoint{2.045945in}{1.610937in}}%
\pgfpathlineto{\pgfqpoint{2.095607in}{1.643112in}}%
\pgfpathlineto{\pgfqpoint{2.145268in}{1.719742in}}%
\pgfpathlineto{\pgfqpoint{2.194929in}{1.655499in}}%
\pgfpathlineto{\pgfqpoint{2.244591in}{1.630540in}}%
\pgfpathlineto{\pgfqpoint{2.294252in}{1.625334in}}%
\pgfpathlineto{\pgfqpoint{2.343913in}{1.654014in}}%
\pgfpathlineto{\pgfqpoint{2.393574in}{1.722628in}}%
\pgfpathlineto{\pgfqpoint{2.443236in}{1.739465in}}%
\pgfpathlineto{\pgfqpoint{2.492897in}{1.629416in}}%
\pgfpathlineto{\pgfqpoint{2.542558in}{1.655217in}}%
\pgfpathlineto{\pgfqpoint{2.641881in}{1.683698in}}%
\pgfpathlineto{\pgfqpoint{2.691542in}{1.692872in}}%
\pgfpathlineto{\pgfqpoint{2.741203in}{1.613875in}}%
\pgfpathlineto{\pgfqpoint{2.890187in}{1.734151in}}%
\pgfpathlineto{\pgfqpoint{3.039171in}{1.726644in}}%
\pgfpathlineto{\pgfqpoint{3.138493in}{1.756166in}}%
\pgfpathlineto{\pgfqpoint{3.436461in}{1.742437in}}%
\pgfusepath{stroke}%
\end{pgfscope}%
\begin{pgfscope}%
\pgfpathrectangle{\pgfqpoint{0.588387in}{0.521603in}}{\pgfqpoint{3.660036in}{2.220246in}}%
\pgfusepath{clip}%
\pgfsetrectcap%
\pgfsetroundjoin%
\pgfsetlinewidth{1.505625pt}%
\pgfsetstrokecolor{currentstroke3}%
\pgfsetdash{}{0pt}%
\pgfpathmoveto{\pgfqpoint{0.754752in}{0.622524in}}%
\pgfpathlineto{\pgfqpoint{0.804414in}{0.680439in}}%
\pgfpathlineto{\pgfqpoint{0.854075in}{0.785871in}}%
\pgfpathlineto{\pgfqpoint{0.903736in}{0.830424in}}%
\pgfpathlineto{\pgfqpoint{0.953398in}{0.928890in}}%
\pgfpathlineto{\pgfqpoint{1.003059in}{0.992758in}}%
\pgfpathlineto{\pgfqpoint{1.052720in}{1.066940in}}%
\pgfpathlineto{\pgfqpoint{1.102381in}{1.153574in}}%
\pgfpathlineto{\pgfqpoint{1.152043in}{1.172759in}}%
\pgfpathlineto{\pgfqpoint{1.201704in}{1.171294in}}%
\pgfpathlineto{\pgfqpoint{1.251365in}{1.361279in}}%
\pgfpathlineto{\pgfqpoint{1.301026in}{1.392025in}}%
\pgfpathlineto{\pgfqpoint{1.350688in}{1.614844in}}%
\pgfpathlineto{\pgfqpoint{1.400349in}{1.600633in}}%
\pgfpathlineto{\pgfqpoint{1.450010in}{1.870672in}}%
\pgfpathlineto{\pgfqpoint{1.499672in}{1.830217in}}%
\pgfpathlineto{\pgfqpoint{1.549333in}{1.297826in}}%
\pgfpathlineto{\pgfqpoint{1.598994in}{2.315293in}}%
\pgfpathlineto{\pgfqpoint{1.648655in}{1.869023in}}%
\pgfpathlineto{\pgfqpoint{1.698317in}{2.130848in}}%
\pgfpathlineto{\pgfqpoint{1.747978in}{1.430572in}}%
\pgfpathlineto{\pgfqpoint{1.797639in}{2.034551in}}%
\pgfpathlineto{\pgfqpoint{1.847300in}{2.252477in}}%
\pgfpathlineto{\pgfqpoint{1.896962in}{1.371796in}}%
\pgfpathlineto{\pgfqpoint{1.946623in}{2.069362in}}%
\pgfpathlineto{\pgfqpoint{1.996284in}{2.236720in}}%
\pgfpathlineto{\pgfqpoint{2.045945in}{2.583252in}}%
\pgfpathlineto{\pgfqpoint{2.095607in}{2.078755in}}%
\pgfpathlineto{\pgfqpoint{2.145268in}{2.156704in}}%
\pgfpathlineto{\pgfqpoint{2.194929in}{2.078001in}}%
\pgfpathlineto{\pgfqpoint{2.244591in}{1.956188in}}%
\pgfpathlineto{\pgfqpoint{2.294252in}{2.396649in}}%
\pgfpathlineto{\pgfqpoint{2.343913in}{2.370050in}}%
\pgfpathlineto{\pgfqpoint{2.393574in}{1.701310in}}%
\pgfpathlineto{\pgfqpoint{2.443236in}{1.343085in}}%
\pgfpathlineto{\pgfqpoint{2.492897in}{1.002625in}}%
\pgfpathlineto{\pgfqpoint{2.542558in}{1.424445in}}%
\pgfpathlineto{\pgfqpoint{2.592219in}{2.259463in}}%
\pgfpathlineto{\pgfqpoint{2.641881in}{1.456539in}}%
\pgfpathlineto{\pgfqpoint{2.691542in}{1.815864in}}%
\pgfpathlineto{\pgfqpoint{2.741203in}{1.151497in}}%
\pgfpathlineto{\pgfqpoint{2.890187in}{2.113356in}}%
\pgfpathlineto{\pgfqpoint{3.039171in}{2.640929in}}%
\pgfpathlineto{\pgfqpoint{3.138493in}{1.151497in}}%
\pgfpathlineto{\pgfqpoint{3.188155in}{1.319773in}}%
\pgfpathlineto{\pgfqpoint{3.287477in}{2.099261in}}%
\pgfpathlineto{\pgfqpoint{3.486122in}{2.105665in}}%
\pgfpathlineto{\pgfqpoint{3.784090in}{1.557681in}}%
\pgfpathlineto{\pgfqpoint{4.082057in}{1.293296in}}%
\pgfusepath{stroke}%
\end{pgfscope}%
\begin{pgfscope}%
\pgfpathrectangle{\pgfqpoint{0.588387in}{0.521603in}}{\pgfqpoint{3.660036in}{2.220246in}}%
\pgfusepath{clip}%
\pgfsetrectcap%
\pgfsetroundjoin%
\pgfsetlinewidth{1.505625pt}%
\pgfsetstrokecolor{currentstroke4}%
\pgfsetdash{}{0pt}%
\pgfpathmoveto{\pgfqpoint{0.754752in}{0.757918in}}%
\pgfpathlineto{\pgfqpoint{0.804414in}{0.759169in}}%
\pgfpathlineto{\pgfqpoint{0.854075in}{0.828949in}}%
\pgfpathlineto{\pgfqpoint{0.903736in}{0.855004in}}%
\pgfpathlineto{\pgfqpoint{0.953398in}{0.925385in}}%
\pgfpathlineto{\pgfqpoint{1.003059in}{0.968649in}}%
\pgfpathlineto{\pgfqpoint{1.052720in}{1.086439in}}%
\pgfpathlineto{\pgfqpoint{1.102381in}{1.180922in}}%
\pgfpathlineto{\pgfqpoint{1.152043in}{1.199748in}}%
\pgfpathlineto{\pgfqpoint{1.201704in}{1.243651in}}%
\pgfpathlineto{\pgfqpoint{1.251365in}{1.340199in}}%
\pgfpathlineto{\pgfqpoint{1.301026in}{1.316375in}}%
\pgfpathlineto{\pgfqpoint{1.350688in}{1.416361in}}%
\pgfpathlineto{\pgfqpoint{1.400349in}{1.436547in}}%
\pgfpathlineto{\pgfqpoint{1.450010in}{1.516198in}}%
\pgfpathlineto{\pgfqpoint{1.499672in}{1.505290in}}%
\pgfpathlineto{\pgfqpoint{1.549333in}{1.427331in}}%
\pgfpathlineto{\pgfqpoint{1.598994in}{1.448376in}}%
\pgfpathlineto{\pgfqpoint{1.648655in}{1.486170in}}%
\pgfpathlineto{\pgfqpoint{1.698317in}{1.535469in}}%
\pgfpathlineto{\pgfqpoint{1.747978in}{1.538408in}}%
\pgfpathlineto{\pgfqpoint{1.797639in}{1.781134in}}%
\pgfpathlineto{\pgfqpoint{1.847300in}{1.491239in}}%
\pgfpathlineto{\pgfqpoint{1.896962in}{1.496491in}}%
\pgfpathlineto{\pgfqpoint{1.946623in}{1.550526in}}%
\pgfpathlineto{\pgfqpoint{1.996284in}{1.584391in}}%
\pgfpathlineto{\pgfqpoint{2.045945in}{1.593521in}}%
\pgfpathlineto{\pgfqpoint{2.095607in}{1.597211in}}%
\pgfpathlineto{\pgfqpoint{2.145268in}{1.554315in}}%
\pgfpathlineto{\pgfqpoint{2.194929in}{1.579145in}}%
\pgfpathlineto{\pgfqpoint{2.244591in}{1.633461in}}%
\pgfpathlineto{\pgfqpoint{2.294252in}{1.649827in}}%
\pgfpathlineto{\pgfqpoint{2.343913in}{1.710208in}}%
\pgfpathlineto{\pgfqpoint{2.393574in}{1.659544in}}%
\pgfpathlineto{\pgfqpoint{2.443236in}{1.612650in}}%
\pgfpathlineto{\pgfqpoint{2.492897in}{1.602243in}}%
\pgfpathlineto{\pgfqpoint{2.542558in}{1.616603in}}%
\pgfpathlineto{\pgfqpoint{2.592219in}{1.686874in}}%
\pgfpathlineto{\pgfqpoint{2.641881in}{1.662286in}}%
\pgfpathlineto{\pgfqpoint{2.691542in}{1.791755in}}%
\pgfpathlineto{\pgfqpoint{2.741203in}{1.608303in}}%
\pgfpathlineto{\pgfqpoint{2.890187in}{1.650250in}}%
\pgfpathlineto{\pgfqpoint{3.039171in}{1.635521in}}%
\pgfpathlineto{\pgfqpoint{3.138493in}{1.733601in}}%
\pgfpathlineto{\pgfqpoint{3.188155in}{1.748964in}}%
\pgfpathlineto{\pgfqpoint{3.436461in}{1.744324in}}%
\pgfusepath{stroke}%
\end{pgfscope}%
\begin{pgfscope}%
\pgfpathrectangle{\pgfqpoint{0.588387in}{0.521603in}}{\pgfqpoint{3.660036in}{2.220246in}}%
\pgfusepath{clip}%
\pgfsetrectcap%
\pgfsetroundjoin%
\pgfsetlinewidth{1.505625pt}%
\pgfsetstrokecolor{currentstroke5}%
\pgfsetdash{}{0pt}%
\pgfpathmoveto{\pgfqpoint{0.754752in}{0.757918in}}%
\pgfpathlineto{\pgfqpoint{0.804414in}{0.759168in}}%
\pgfpathlineto{\pgfqpoint{0.854075in}{0.829206in}}%
\pgfpathlineto{\pgfqpoint{0.903736in}{0.855157in}}%
\pgfpathlineto{\pgfqpoint{0.953398in}{0.924649in}}%
\pgfpathlineto{\pgfqpoint{1.003059in}{0.969099in}}%
\pgfpathlineto{\pgfqpoint{1.052720in}{1.086816in}}%
\pgfpathlineto{\pgfqpoint{1.102381in}{1.195512in}}%
\pgfpathlineto{\pgfqpoint{1.152043in}{1.197624in}}%
\pgfpathlineto{\pgfqpoint{1.201704in}{1.243209in}}%
\pgfpathlineto{\pgfqpoint{1.251365in}{1.358972in}}%
\pgfpathlineto{\pgfqpoint{1.301026in}{1.333575in}}%
\pgfpathlineto{\pgfqpoint{1.350688in}{1.420522in}}%
\pgfpathlineto{\pgfqpoint{1.400349in}{1.471798in}}%
\pgfpathlineto{\pgfqpoint{1.450010in}{1.525162in}}%
\pgfpathlineto{\pgfqpoint{1.499672in}{1.515478in}}%
\pgfpathlineto{\pgfqpoint{1.549333in}{1.578862in}}%
\pgfpathlineto{\pgfqpoint{1.598994in}{1.510278in}}%
\pgfpathlineto{\pgfqpoint{1.648655in}{1.510296in}}%
\pgfpathlineto{\pgfqpoint{1.698317in}{1.607715in}}%
\pgfpathlineto{\pgfqpoint{1.747978in}{1.694736in}}%
\pgfpathlineto{\pgfqpoint{1.797639in}{1.799294in}}%
\pgfpathlineto{\pgfqpoint{1.847300in}{2.116476in}}%
\pgfpathlineto{\pgfqpoint{1.896962in}{1.547662in}}%
\pgfpathlineto{\pgfqpoint{1.946623in}{1.572524in}}%
\pgfpathlineto{\pgfqpoint{1.996284in}{1.593664in}}%
\pgfpathlineto{\pgfqpoint{2.045945in}{1.774399in}}%
\pgfpathlineto{\pgfqpoint{2.095607in}{1.593972in}}%
\pgfpathlineto{\pgfqpoint{2.145268in}{1.650503in}}%
\pgfpathlineto{\pgfqpoint{2.194929in}{1.730570in}}%
\pgfpathlineto{\pgfqpoint{2.244591in}{1.666644in}}%
\pgfpathlineto{\pgfqpoint{2.294252in}{1.665593in}}%
\pgfpathlineto{\pgfqpoint{2.343913in}{1.677847in}}%
\pgfpathlineto{\pgfqpoint{2.393574in}{1.696038in}}%
\pgfpathlineto{\pgfqpoint{2.443236in}{1.620870in}}%
\pgfpathlineto{\pgfqpoint{2.492897in}{1.603534in}}%
\pgfpathlineto{\pgfqpoint{2.542558in}{1.580977in}}%
\pgfpathlineto{\pgfqpoint{2.592219in}{1.685610in}}%
\pgfpathlineto{\pgfqpoint{2.641881in}{1.695703in}}%
\pgfpathlineto{\pgfqpoint{2.691542in}{1.741962in}}%
\pgfpathlineto{\pgfqpoint{2.741203in}{1.540685in}}%
\pgfpathlineto{\pgfqpoint{2.890187in}{1.617504in}}%
\pgfpathlineto{\pgfqpoint{3.039171in}{1.684763in}}%
\pgfpathlineto{\pgfqpoint{3.138493in}{1.724115in}}%
\pgfpathlineto{\pgfqpoint{3.188155in}{1.805616in}}%
\pgfpathlineto{\pgfqpoint{3.436461in}{1.744324in}}%
\pgfusepath{stroke}%
\end{pgfscope}%
\begin{pgfscope}%
\pgfpathrectangle{\pgfqpoint{0.588387in}{0.521603in}}{\pgfqpoint{3.660036in}{2.220246in}}%
\pgfusepath{clip}%
\pgfsetrectcap%
\pgfsetroundjoin%
\pgfsetlinewidth{1.505625pt}%
\pgfsetstrokecolor{currentstroke6}%
\pgfsetdash{}{0pt}%
\pgfpathmoveto{\pgfqpoint{0.754752in}{0.757918in}}%
\pgfpathlineto{\pgfqpoint{0.804414in}{0.759165in}}%
\pgfpathlineto{\pgfqpoint{0.854075in}{0.828159in}}%
\pgfpathlineto{\pgfqpoint{0.903736in}{0.852822in}}%
\pgfpathlineto{\pgfqpoint{0.953398in}{0.926983in}}%
\pgfpathlineto{\pgfqpoint{1.003059in}{0.964655in}}%
\pgfpathlineto{\pgfqpoint{1.052720in}{1.089243in}}%
\pgfpathlineto{\pgfqpoint{1.102381in}{1.132351in}}%
\pgfpathlineto{\pgfqpoint{1.152043in}{1.196753in}}%
\pgfpathlineto{\pgfqpoint{1.201704in}{1.229268in}}%
\pgfpathlineto{\pgfqpoint{1.251365in}{1.333270in}}%
\pgfpathlineto{\pgfqpoint{1.301026in}{1.322933in}}%
\pgfpathlineto{\pgfqpoint{1.350688in}{1.424049in}}%
\pgfpathlineto{\pgfqpoint{1.400349in}{1.448242in}}%
\pgfpathlineto{\pgfqpoint{1.450010in}{1.516472in}}%
\pgfpathlineto{\pgfqpoint{1.499672in}{1.516466in}}%
\pgfpathlineto{\pgfqpoint{1.549333in}{1.438163in}}%
\pgfpathlineto{\pgfqpoint{1.598994in}{1.490788in}}%
\pgfpathlineto{\pgfqpoint{1.648655in}{1.514217in}}%
\pgfpathlineto{\pgfqpoint{1.698317in}{1.547774in}}%
\pgfpathlineto{\pgfqpoint{1.747978in}{1.607058in}}%
\pgfpathlineto{\pgfqpoint{1.797639in}{1.704943in}}%
\pgfpathlineto{\pgfqpoint{1.847300in}{1.530010in}}%
\pgfpathlineto{\pgfqpoint{1.896962in}{1.497097in}}%
\pgfpathlineto{\pgfqpoint{1.946623in}{1.534552in}}%
\pgfpathlineto{\pgfqpoint{1.996284in}{1.621653in}}%
\pgfpathlineto{\pgfqpoint{2.045945in}{1.656919in}}%
\pgfpathlineto{\pgfqpoint{2.095607in}{1.603534in}}%
\pgfpathlineto{\pgfqpoint{2.145268in}{1.568294in}}%
\pgfpathlineto{\pgfqpoint{2.194929in}{1.687054in}}%
\pgfpathlineto{\pgfqpoint{2.244591in}{1.668036in}}%
\pgfpathlineto{\pgfqpoint{2.294252in}{1.611005in}}%
\pgfpathlineto{\pgfqpoint{2.343913in}{1.760537in}}%
\pgfpathlineto{\pgfqpoint{2.393574in}{1.838278in}}%
\pgfpathlineto{\pgfqpoint{2.443236in}{1.596212in}}%
\pgfpathlineto{\pgfqpoint{2.492897in}{1.578139in}}%
\pgfpathlineto{\pgfqpoint{2.542558in}{1.634425in}}%
\pgfpathlineto{\pgfqpoint{2.592219in}{1.629698in}}%
\pgfpathlineto{\pgfqpoint{2.641881in}{1.679381in}}%
\pgfpathlineto{\pgfqpoint{2.691542in}{1.673173in}}%
\pgfpathlineto{\pgfqpoint{2.741203in}{1.749270in}}%
\pgfpathlineto{\pgfqpoint{2.890187in}{1.671362in}}%
\pgfpathlineto{\pgfqpoint{3.039171in}{1.689379in}}%
\pgfpathlineto{\pgfqpoint{3.138493in}{1.690206in}}%
\pgfpathlineto{\pgfqpoint{3.188155in}{1.707958in}}%
\pgfpathlineto{\pgfqpoint{3.436461in}{1.779206in}}%
\pgfusepath{stroke}%
\end{pgfscope}%
\begin{pgfscope}%
\pgfpathrectangle{\pgfqpoint{0.588387in}{0.521603in}}{\pgfqpoint{3.660036in}{2.220246in}}%
\pgfusepath{clip}%
\pgfsetrectcap%
\pgfsetroundjoin%
\pgfsetlinewidth{1.505625pt}%
\pgfsetstrokecolor{currentstroke7}%
\pgfsetdash{}{0pt}%
\pgfpathmoveto{\pgfqpoint{0.754752in}{0.757918in}}%
\pgfpathlineto{\pgfqpoint{0.804414in}{0.759164in}}%
\pgfpathlineto{\pgfqpoint{0.854075in}{0.828421in}}%
\pgfpathlineto{\pgfqpoint{0.903736in}{0.852923in}}%
\pgfpathlineto{\pgfqpoint{0.953398in}{0.927837in}}%
\pgfpathlineto{\pgfqpoint{1.003059in}{0.963852in}}%
\pgfpathlineto{\pgfqpoint{1.052720in}{1.089609in}}%
\pgfpathlineto{\pgfqpoint{1.102381in}{1.132449in}}%
\pgfpathlineto{\pgfqpoint{1.152043in}{1.196551in}}%
\pgfpathlineto{\pgfqpoint{1.201704in}{1.229111in}}%
\pgfpathlineto{\pgfqpoint{1.251365in}{1.386711in}}%
\pgfpathlineto{\pgfqpoint{1.301026in}{1.364135in}}%
\pgfpathlineto{\pgfqpoint{1.350688in}{1.438643in}}%
\pgfpathlineto{\pgfqpoint{1.400349in}{1.511951in}}%
\pgfpathlineto{\pgfqpoint{1.450010in}{1.590047in}}%
\pgfpathlineto{\pgfqpoint{1.499672in}{1.590036in}}%
\pgfpathlineto{\pgfqpoint{1.549333in}{1.614649in}}%
\pgfpathlineto{\pgfqpoint{1.598994in}{1.653111in}}%
\pgfpathlineto{\pgfqpoint{1.648655in}{1.638768in}}%
\pgfpathlineto{\pgfqpoint{1.698317in}{1.778544in}}%
\pgfpathlineto{\pgfqpoint{1.747978in}{1.742729in}}%
\pgfpathlineto{\pgfqpoint{1.797639in}{1.954815in}}%
\pgfpathlineto{\pgfqpoint{1.847300in}{1.704966in}}%
\pgfpathlineto{\pgfqpoint{1.896962in}{1.535739in}}%
\pgfpathlineto{\pgfqpoint{1.946623in}{1.605437in}}%
\pgfpathlineto{\pgfqpoint{1.996284in}{1.731607in}}%
\pgfpathlineto{\pgfqpoint{2.045945in}{1.799323in}}%
\pgfpathlineto{\pgfqpoint{2.095607in}{1.580029in}}%
\pgfpathlineto{\pgfqpoint{2.145268in}{1.570211in}}%
\pgfpathlineto{\pgfqpoint{2.194929in}{1.731229in}}%
\pgfpathlineto{\pgfqpoint{2.244591in}{1.623405in}}%
\pgfpathlineto{\pgfqpoint{2.294252in}{2.124783in}}%
\pgfpathlineto{\pgfqpoint{2.343913in}{1.750487in}}%
\pgfpathlineto{\pgfqpoint{2.393574in}{1.807438in}}%
\pgfpathlineto{\pgfqpoint{2.443236in}{1.618301in}}%
\pgfpathlineto{\pgfqpoint{2.492897in}{1.653267in}}%
\pgfpathlineto{\pgfqpoint{2.542558in}{1.661011in}}%
\pgfpathlineto{\pgfqpoint{2.592219in}{1.529729in}}%
\pgfpathlineto{\pgfqpoint{2.641881in}{1.626863in}}%
\pgfpathlineto{\pgfqpoint{2.691542in}{1.593972in}}%
\pgfpathlineto{\pgfqpoint{2.741203in}{2.032618in}}%
\pgfpathlineto{\pgfqpoint{2.890187in}{1.825291in}}%
\pgfpathlineto{\pgfqpoint{3.138493in}{1.732052in}}%
\pgfpathlineto{\pgfqpoint{3.188155in}{1.786908in}}%
\pgfpathlineto{\pgfqpoint{3.436461in}{1.761686in}}%
\pgfusepath{stroke}%
\end{pgfscope}%
\begin{pgfscope}%
\pgfpathrectangle{\pgfqpoint{0.588387in}{0.521603in}}{\pgfqpoint{3.660036in}{2.220246in}}%
\pgfusepath{clip}%
\pgfsetrectcap%
\pgfsetroundjoin%
\pgfsetlinewidth{1.505625pt}%
\definecolor{currentstroke}{rgb}{0.498039,0.498039,0.498039}%
\pgfsetstrokecolor{currentstroke}%
\pgfsetdash{}{0pt}%
\pgfpathmoveto{\pgfqpoint{0.754752in}{0.757918in}}%
\pgfpathlineto{\pgfqpoint{0.804414in}{0.787760in}}%
\pgfpathlineto{\pgfqpoint{0.854075in}{0.847726in}}%
\pgfpathlineto{\pgfqpoint{0.903736in}{0.877055in}}%
\pgfpathlineto{\pgfqpoint{0.953398in}{0.941855in}}%
\pgfpathlineto{\pgfqpoint{1.003059in}{0.985600in}}%
\pgfpathlineto{\pgfqpoint{1.052720in}{1.115139in}}%
\pgfpathlineto{\pgfqpoint{1.102381in}{1.206716in}}%
\pgfpathlineto{\pgfqpoint{1.152043in}{1.215335in}}%
\pgfpathlineto{\pgfqpoint{1.201704in}{1.229538in}}%
\pgfpathlineto{\pgfqpoint{1.251365in}{1.351206in}}%
\pgfpathlineto{\pgfqpoint{1.301026in}{1.377617in}}%
\pgfpathlineto{\pgfqpoint{1.350688in}{1.468195in}}%
\pgfpathlineto{\pgfqpoint{1.400349in}{1.468277in}}%
\pgfpathlineto{\pgfqpoint{1.450010in}{1.540583in}}%
\pgfpathlineto{\pgfqpoint{1.499672in}{1.564872in}}%
\pgfpathlineto{\pgfqpoint{1.549333in}{1.494456in}}%
\pgfpathlineto{\pgfqpoint{1.598994in}{1.491037in}}%
\pgfpathlineto{\pgfqpoint{1.648655in}{1.556865in}}%
\pgfpathlineto{\pgfqpoint{1.698317in}{1.595125in}}%
\pgfpathlineto{\pgfqpoint{1.747978in}{1.592031in}}%
\pgfpathlineto{\pgfqpoint{1.797639in}{1.790614in}}%
\pgfpathlineto{\pgfqpoint{1.847300in}{1.768208in}}%
\pgfpathlineto{\pgfqpoint{1.896962in}{1.585805in}}%
\pgfpathlineto{\pgfqpoint{1.946623in}{1.564637in}}%
\pgfpathlineto{\pgfqpoint{1.996284in}{1.623017in}}%
\pgfpathlineto{\pgfqpoint{2.045945in}{1.680104in}}%
\pgfpathlineto{\pgfqpoint{2.095607in}{1.679381in}}%
\pgfpathlineto{\pgfqpoint{2.145268in}{1.639683in}}%
\pgfpathlineto{\pgfqpoint{2.194929in}{1.703225in}}%
\pgfpathlineto{\pgfqpoint{2.244591in}{1.676523in}}%
\pgfpathlineto{\pgfqpoint{2.294252in}{1.651848in}}%
\pgfpathlineto{\pgfqpoint{2.343913in}{1.751093in}}%
\pgfpathlineto{\pgfqpoint{2.393574in}{1.897985in}}%
\pgfpathlineto{\pgfqpoint{2.443236in}{1.648469in}}%
\pgfpathlineto{\pgfqpoint{2.492897in}{1.664182in}}%
\pgfpathlineto{\pgfqpoint{2.542558in}{1.693584in}}%
\pgfpathlineto{\pgfqpoint{2.592219in}{1.688547in}}%
\pgfpathlineto{\pgfqpoint{2.641881in}{1.750790in}}%
\pgfpathlineto{\pgfqpoint{2.691542in}{1.826220in}}%
\pgfpathlineto{\pgfqpoint{2.741203in}{1.631937in}}%
\pgfpathlineto{\pgfqpoint{2.890187in}{1.676586in}}%
\pgfpathlineto{\pgfqpoint{3.039171in}{1.804814in}}%
\pgfpathlineto{\pgfqpoint{3.138493in}{1.704535in}}%
\pgfpathlineto{\pgfqpoint{3.188155in}{1.867613in}}%
\pgfpathlineto{\pgfqpoint{3.436461in}{1.948417in}}%
\pgfpathlineto{\pgfqpoint{3.486122in}{1.827655in}}%
\pgfpathlineto{\pgfqpoint{3.585445in}{1.825601in}}%
\pgfpathlineto{\pgfqpoint{3.784090in}{1.937783in}}%
\pgfusepath{stroke}%
\end{pgfscope}%
\begin{pgfscope}%
\pgfpathrectangle{\pgfqpoint{0.588387in}{0.521603in}}{\pgfqpoint{3.660036in}{2.220246in}}%
\pgfusepath{clip}%
\pgfsetrectcap%
\pgfsetroundjoin%
\pgfsetlinewidth{1.505625pt}%
\definecolor{currentstroke}{rgb}{0.737255,0.741176,0.133333}%
\pgfsetstrokecolor{currentstroke}%
\pgfsetdash{}{0pt}%
\pgfpathmoveto{\pgfqpoint{0.754752in}{0.757918in}}%
\pgfpathlineto{\pgfqpoint{0.804414in}{0.787790in}}%
\pgfpathlineto{\pgfqpoint{0.854075in}{0.847562in}}%
\pgfpathlineto{\pgfqpoint{0.903736in}{0.877033in}}%
\pgfpathlineto{\pgfqpoint{0.953398in}{0.940831in}}%
\pgfpathlineto{\pgfqpoint{1.003059in}{0.984923in}}%
\pgfpathlineto{\pgfqpoint{1.052720in}{1.114915in}}%
\pgfpathlineto{\pgfqpoint{1.102381in}{1.205834in}}%
\pgfpathlineto{\pgfqpoint{1.152043in}{1.216368in}}%
\pgfpathlineto{\pgfqpoint{1.201704in}{1.230664in}}%
\pgfpathlineto{\pgfqpoint{1.251365in}{1.363206in}}%
\pgfpathlineto{\pgfqpoint{1.301026in}{1.380930in}}%
\pgfpathlineto{\pgfqpoint{1.350688in}{1.483587in}}%
\pgfpathlineto{\pgfqpoint{1.400349in}{1.500912in}}%
\pgfpathlineto{\pgfqpoint{1.450010in}{1.554608in}}%
\pgfpathlineto{\pgfqpoint{1.499672in}{1.587631in}}%
\pgfpathlineto{\pgfqpoint{1.549333in}{1.500887in}}%
\pgfpathlineto{\pgfqpoint{1.598994in}{1.518219in}}%
\pgfpathlineto{\pgfqpoint{1.648655in}{1.704337in}}%
\pgfpathlineto{\pgfqpoint{1.698317in}{1.619025in}}%
\pgfpathlineto{\pgfqpoint{1.747978in}{1.756102in}}%
\pgfpathlineto{\pgfqpoint{1.797639in}{1.819690in}}%
\pgfpathlineto{\pgfqpoint{1.847300in}{1.966805in}}%
\pgfpathlineto{\pgfqpoint{1.896962in}{1.737638in}}%
\pgfpathlineto{\pgfqpoint{1.946623in}{1.777490in}}%
\pgfpathlineto{\pgfqpoint{1.996284in}{1.954041in}}%
\pgfpathlineto{\pgfqpoint{2.045945in}{1.904080in}}%
\pgfpathlineto{\pgfqpoint{2.095607in}{1.662286in}}%
\pgfpathlineto{\pgfqpoint{2.145268in}{1.753165in}}%
\pgfpathlineto{\pgfqpoint{2.194929in}{1.850837in}}%
\pgfpathlineto{\pgfqpoint{2.244591in}{1.675190in}}%
\pgfpathlineto{\pgfqpoint{2.294252in}{1.858116in}}%
\pgfpathlineto{\pgfqpoint{2.343913in}{1.707958in}}%
\pgfpathlineto{\pgfqpoint{2.393574in}{1.659406in}}%
\pgfpathlineto{\pgfqpoint{2.443236in}{1.721927in}}%
\pgfpathlineto{\pgfqpoint{2.492897in}{1.636882in}}%
\pgfpathlineto{\pgfqpoint{2.542558in}{1.684763in}}%
\pgfpathlineto{\pgfqpoint{2.641881in}{1.707455in}}%
\pgfpathlineto{\pgfqpoint{2.691542in}{1.939115in}}%
\pgfpathlineto{\pgfqpoint{2.890187in}{1.812139in}}%
\pgfpathlineto{\pgfqpoint{3.039171in}{1.727328in}}%
\pgfpathlineto{\pgfqpoint{3.138493in}{1.809691in}}%
\pgfpathlineto{\pgfqpoint{3.188155in}{1.746194in}}%
\pgfpathlineto{\pgfqpoint{3.436461in}{1.778156in}}%
\pgfpathlineto{\pgfqpoint{3.486122in}{1.874002in}}%
\pgfpathlineto{\pgfqpoint{3.784090in}{1.892150in}}%
\pgfusepath{stroke}%
\end{pgfscope}%
\begin{pgfscope}%
\pgfsetrectcap%
\pgfsetmiterjoin%
\pgfsetlinewidth{0.803000pt}%
\definecolor{currentstroke}{rgb}{0.000000,0.000000,0.000000}%
\pgfsetstrokecolor{currentstroke}%
\pgfsetdash{}{0pt}%
\pgfpathmoveto{\pgfqpoint{0.588387in}{0.521603in}}%
\pgfpathlineto{\pgfqpoint{0.588387in}{2.741849in}}%
\pgfusepath{stroke}%
\end{pgfscope}%
\begin{pgfscope}%
\pgfsetrectcap%
\pgfsetmiterjoin%
\pgfsetlinewidth{0.803000pt}%
\definecolor{currentstroke}{rgb}{0.000000,0.000000,0.000000}%
\pgfsetstrokecolor{currentstroke}%
\pgfsetdash{}{0pt}%
\pgfpathmoveto{\pgfqpoint{4.248423in}{0.521603in}}%
\pgfpathlineto{\pgfqpoint{4.248423in}{2.741849in}}%
\pgfusepath{stroke}%
\end{pgfscope}%
\begin{pgfscope}%
\pgfsetrectcap%
\pgfsetmiterjoin%
\pgfsetlinewidth{0.803000pt}%
\definecolor{currentstroke}{rgb}{0.000000,0.000000,0.000000}%
\pgfsetstrokecolor{currentstroke}%
\pgfsetdash{}{0pt}%
\pgfpathmoveto{\pgfqpoint{0.588387in}{0.521603in}}%
\pgfpathlineto{\pgfqpoint{4.248423in}{0.521603in}}%
\pgfusepath{stroke}%
\end{pgfscope}%
\begin{pgfscope}%
\pgfsetrectcap%
\pgfsetmiterjoin%
\pgfsetlinewidth{0.803000pt}%
\definecolor{currentstroke}{rgb}{0.000000,0.000000,0.000000}%
\pgfsetstrokecolor{currentstroke}%
\pgfsetdash{}{0pt}%
\pgfpathmoveto{\pgfqpoint{0.588387in}{2.741849in}}%
\pgfpathlineto{\pgfqpoint{4.248423in}{2.741849in}}%
\pgfusepath{stroke}%
\end{pgfscope}%
\begin{pgfscope}%
\pgfsetbuttcap%
\pgfsetmiterjoin%
\definecolor{currentfill}{rgb}{1.000000,1.000000,1.000000}%
\pgfsetfillcolor{currentfill}%
\pgfsetfillopacity{0.800000}%
\pgfsetlinewidth{1.003750pt}%
\definecolor{currentstroke}{rgb}{0.800000,0.800000,0.800000}%
\pgfsetstrokecolor{currentstroke}%
\pgfsetstrokeopacity{0.800000}%
\pgfsetdash{}{0pt}%
\pgfpathmoveto{\pgfqpoint{4.365089in}{0.379025in}}%
\pgfpathlineto{\pgfqpoint{8.251043in}{0.379025in}}%
\pgfpathquadraticcurveto{\pgfqpoint{8.284376in}{0.379025in}}{\pgfqpoint{8.284376in}{0.412359in}}%
\pgfpathlineto{\pgfqpoint{8.284376in}{2.625183in}}%
\pgfpathquadraticcurveto{\pgfqpoint{8.284376in}{2.658516in}}{\pgfqpoint{8.251043in}{2.658516in}}%
\pgfpathlineto{\pgfqpoint{4.365089in}{2.658516in}}%
\pgfpathquadraticcurveto{\pgfqpoint{4.331756in}{2.658516in}}{\pgfqpoint{4.331756in}{2.625183in}}%
\pgfpathlineto{\pgfqpoint{4.331756in}{0.412359in}}%
\pgfpathquadraticcurveto{\pgfqpoint{4.331756in}{0.379025in}}{\pgfqpoint{4.365089in}{0.379025in}}%
\pgfpathlineto{\pgfqpoint{4.365089in}{0.379025in}}%
\pgfpathclose%
\pgfusepath{stroke,fill}%
\end{pgfscope}%
\begin{pgfscope}%
\pgfsetrectcap%
\pgfsetroundjoin%
\pgfsetlinewidth{1.505625pt}%
\pgfsetstrokecolor{currentstroke3}%
\pgfsetdash{}{0pt}%
\pgfpathmoveto{\pgfqpoint{4.398423in}{2.523555in}}%
\pgfpathlineto{\pgfqpoint{4.565089in}{2.523555in}}%
\pgfpathlineto{\pgfqpoint{4.731756in}{2.523555in}}%
\pgfusepath{stroke}%
\end{pgfscope}%
\begin{pgfscope}%
\definecolor{textcolor}{rgb}{0.000000,0.000000,0.000000}%
\pgfsetstrokecolor{textcolor}%
\pgfsetfillcolor{textcolor}%
\pgftext[x=4.865089in,y=2.465222in,left,base]{\color{textcolor}{\rmfamily\fontsize{12.000000}{14.400000}\selectfont\catcode`\^=\active\def^{\ifmmode\sp\else\^{}\fi}\catcode`\%=\active\def%{\%}\NaiveCycles{}}}%
\end{pgfscope}%
\begin{pgfscope}%
\pgfsetrectcap%
\pgfsetroundjoin%
\pgfsetlinewidth{1.505625pt}%
\pgfsetstrokecolor{currentstroke1}%
\pgfsetdash{}{0pt}%
\pgfpathmoveto{\pgfqpoint{4.398423in}{2.278926in}}%
\pgfpathlineto{\pgfqpoint{4.565089in}{2.278926in}}%
\pgfpathlineto{\pgfqpoint{4.731756in}{2.278926in}}%
\pgfusepath{stroke}%
\end{pgfscope}%
\begin{pgfscope}%
\definecolor{textcolor}{rgb}{0.000000,0.000000,0.000000}%
\pgfsetstrokecolor{textcolor}%
\pgfsetfillcolor{textcolor}%
\pgftext[x=4.865089in,y=2.220593in,left,base]{\color{textcolor}{\rmfamily\fontsize{12.000000}{14.400000}\selectfont\catcode`\^=\active\def^{\ifmmode\sp\else\^{}\fi}\catcode`\%=\active\def%{\%}\CyclesMatchChunks{} \& \MergeLinear{}}}%
\end{pgfscope}%
\begin{pgfscope}%
\pgfsetrectcap%
\pgfsetroundjoin%
\pgfsetlinewidth{1.505625pt}%
\pgfsetstrokecolor{currentstroke2}%
\pgfsetdash{}{0pt}%
\pgfpathmoveto{\pgfqpoint{4.398423in}{2.029659in}}%
\pgfpathlineto{\pgfqpoint{4.565089in}{2.029659in}}%
\pgfpathlineto{\pgfqpoint{4.731756in}{2.029659in}}%
\pgfusepath{stroke}%
\end{pgfscope}%
\begin{pgfscope}%
\definecolor{textcolor}{rgb}{0.000000,0.000000,0.000000}%
\pgfsetstrokecolor{textcolor}%
\pgfsetfillcolor{textcolor}%
\pgftext[x=4.865089in,y=1.971325in,left,base]{\color{textcolor}{\rmfamily\fontsize{12.000000}{14.400000}\selectfont\catcode`\^=\active\def^{\ifmmode\sp\else\^{}\fi}\catcode`\%=\active\def%{\%}\CyclesMatchChunks{} \& \SharedVertices{}}}%
\end{pgfscope}%
\begin{pgfscope}%
\pgfsetrectcap%
\pgfsetroundjoin%
\pgfsetlinewidth{1.505625pt}%
\pgfsetstrokecolor{currentstroke4}%
\pgfsetdash{}{0pt}%
\pgfpathmoveto{\pgfqpoint{4.398423in}{1.780391in}}%
\pgfpathlineto{\pgfqpoint{4.565089in}{1.780391in}}%
\pgfpathlineto{\pgfqpoint{4.731756in}{1.780391in}}%
\pgfusepath{stroke}%
\end{pgfscope}%
\begin{pgfscope}%
\definecolor{textcolor}{rgb}{0.000000,0.000000,0.000000}%
\pgfsetstrokecolor{textcolor}%
\pgfsetfillcolor{textcolor}%
\pgftext[x=4.865089in,y=1.722058in,left,base]{\color{textcolor}{\rmfamily\fontsize{12.000000}{14.400000}\selectfont\catcode`\^=\active\def^{\ifmmode\sp\else\^{}\fi}\catcode`\%=\active\def%{\%}\Neighbors{} \& \MergeLinear{}}}%
\end{pgfscope}%
\begin{pgfscope}%
\pgfsetrectcap%
\pgfsetroundjoin%
\pgfsetlinewidth{1.505625pt}%
\pgfsetstrokecolor{currentstroke5}%
\pgfsetdash{}{0pt}%
\pgfpathmoveto{\pgfqpoint{4.398423in}{1.535763in}}%
\pgfpathlineto{\pgfqpoint{4.565089in}{1.535763in}}%
\pgfpathlineto{\pgfqpoint{4.731756in}{1.535763in}}%
\pgfusepath{stroke}%
\end{pgfscope}%
\begin{pgfscope}%
\definecolor{textcolor}{rgb}{0.000000,0.000000,0.000000}%
\pgfsetstrokecolor{textcolor}%
\pgfsetfillcolor{textcolor}%
\pgftext[x=4.865089in,y=1.477429in,left,base]{\color{textcolor}{\rmfamily\fontsize{12.000000}{14.400000}\selectfont\catcode`\^=\active\def^{\ifmmode\sp\else\^{}\fi}\catcode`\%=\active\def%{\%}\Neighbors{} \& \SharedVertices{}}}%
\end{pgfscope}%
\begin{pgfscope}%
\pgfsetrectcap%
\pgfsetroundjoin%
\pgfsetlinewidth{1.505625pt}%
\pgfsetstrokecolor{currentstroke6}%
\pgfsetdash{}{0pt}%
\pgfpathmoveto{\pgfqpoint{4.398423in}{1.286495in}}%
\pgfpathlineto{\pgfqpoint{4.565089in}{1.286495in}}%
\pgfpathlineto{\pgfqpoint{4.731756in}{1.286495in}}%
\pgfusepath{stroke}%
\end{pgfscope}%
\begin{pgfscope}%
\definecolor{textcolor}{rgb}{0.000000,0.000000,0.000000}%
\pgfsetstrokecolor{textcolor}%
\pgfsetfillcolor{textcolor}%
\pgftext[x=4.865089in,y=1.228162in,left,base]{\color{textcolor}{\rmfamily\fontsize{12.000000}{14.400000}\selectfont\catcode`\^=\active\def^{\ifmmode\sp\else\^{}\fi}\catcode`\%=\active\def%{\%}\NeighborsDegree{} \& \MergeLinear{}}}%
\end{pgfscope}%
\begin{pgfscope}%
\pgfsetrectcap%
\pgfsetroundjoin%
\pgfsetlinewidth{1.505625pt}%
\pgfsetstrokecolor{currentstroke7}%
\pgfsetdash{}{0pt}%
\pgfpathmoveto{\pgfqpoint{4.398423in}{1.037228in}}%
\pgfpathlineto{\pgfqpoint{4.565089in}{1.037228in}}%
\pgfpathlineto{\pgfqpoint{4.731756in}{1.037228in}}%
\pgfusepath{stroke}%
\end{pgfscope}%
\begin{pgfscope}%
\definecolor{textcolor}{rgb}{0.000000,0.000000,0.000000}%
\pgfsetstrokecolor{textcolor}%
\pgfsetfillcolor{textcolor}%
\pgftext[x=4.865089in,y=0.978895in,left,base]{\color{textcolor}{\rmfamily\fontsize{12.000000}{14.400000}\selectfont\catcode`\^=\active\def^{\ifmmode\sp\else\^{}\fi}\catcode`\%=\active\def%{\%}\NeighborsDegree{} \& \SharedVertices{}}}%
\end{pgfscope}%
\begin{pgfscope}%
\pgfsetrectcap%
\pgfsetroundjoin%
\pgfsetlinewidth{1.505625pt}%
\definecolor{currentstroke}{rgb}{0.498039,0.498039,0.498039}%
\pgfsetstrokecolor{currentstroke}%
\pgfsetdash{}{0pt}%
\pgfpathmoveto{\pgfqpoint{4.398423in}{0.787961in}}%
\pgfpathlineto{\pgfqpoint{4.565089in}{0.787961in}}%
\pgfpathlineto{\pgfqpoint{4.731756in}{0.787961in}}%
\pgfusepath{stroke}%
\end{pgfscope}%
\begin{pgfscope}%
\definecolor{textcolor}{rgb}{0.000000,0.000000,0.000000}%
\pgfsetstrokecolor{textcolor}%
\pgfsetfillcolor{textcolor}%
\pgftext[x=4.865089in,y=0.729627in,left,base]{\color{textcolor}{\rmfamily\fontsize{12.000000}{14.400000}\selectfont\catcode`\^=\active\def^{\ifmmode\sp\else\^{}\fi}\catcode`\%=\active\def%{\%}\None{} \& \MergeLinear{}}}%
\end{pgfscope}%
\begin{pgfscope}%
\pgfsetrectcap%
\pgfsetroundjoin%
\pgfsetlinewidth{1.505625pt}%
\definecolor{currentstroke}{rgb}{0.737255,0.741176,0.133333}%
\pgfsetstrokecolor{currentstroke}%
\pgfsetdash{}{0pt}%
\pgfpathmoveto{\pgfqpoint{4.398423in}{0.543332in}}%
\pgfpathlineto{\pgfqpoint{4.565089in}{0.543332in}}%
\pgfpathlineto{\pgfqpoint{4.731756in}{0.543332in}}%
\pgfusepath{stroke}%
\end{pgfscope}%
\begin{pgfscope}%
\definecolor{textcolor}{rgb}{0.000000,0.000000,0.000000}%
\pgfsetstrokecolor{textcolor}%
\pgfsetfillcolor{textcolor}%
\pgftext[x=4.865089in,y=0.484999in,left,base]{\color{textcolor}{\rmfamily\fontsize{12.000000}{14.400000}\selectfont\catcode`\^=\active\def^{\ifmmode\sp\else\^{}\fi}\catcode`\%=\active\def%{\%}\None{} \& \SharedVertices{}}}%
\end{pgfscope}%
\end{pgfpicture}%
\makeatother%
\endgroup%
}
% 	\caption[Checks performed for globally rigid graphs (some)]{
% 		The number of checks performed to find some NAC-colorings for globally rigid graphs.}%
% 	\label{fig:graph_globally_rigid_first_checks}
% \end{figure}%

\NaiveCycles{} is significantly slower when we list all NAC-colorings
as expected, see \Cref{fig:graph_globally_rigid_all_runtime}.
\None{} and \CycleMask{} strategies also lack behind \Neighbors{} and \NeighborsDegree{}.
We do not see a advantage of \MergeLinear{} over \SharedVertices{} any more.
%
It can be also seen in \Cref{fig:graph_globally_rigid_all_checks}
that for the number of checks performed the same statements hold.
%
\begin{figure}[thbp]
	\centering
	\scalebox{\BenchFigureScale}{%% Creator: Matplotlib, PGF backend
%%
%% To include the figure in your LaTeX document, write
%%   \input{<filename>.pgf}
%%
%% Make sure the required packages are loaded in your preamble
%%   \usepackage{pgf}
%%
%% Also ensure that all the required font packages are loaded; for instance,
%% the lmodern package is sometimes necessary when using math font.
%%   \usepackage{lmodern}
%%
%% Figures using additional raster images can only be included by \input if
%% they are in the same directory as the main LaTeX file. For loading figures
%% from other directories you can use the `import` package
%%   \usepackage{import}
%%
%% and then include the figures with
%%   \import{<path to file>}{<filename>.pgf}
%%
%% Matplotlib used the following preamble
%%   \def\mathdefault#1{#1}
%%   \everymath=\expandafter{\the\everymath\displaystyle}
%%   \IfFileExists{scrextend.sty}{
%%     \usepackage[fontsize=10.000000pt]{scrextend}
%%   }{
%%     \renewcommand{\normalsize}{\fontsize{10.000000}{12.000000}\selectfont}
%%     \normalsize
%%   }
%%   
%%   \ifdefined\pdftexversion\else  % non-pdftex case.
%%     \usepackage{fontspec}
%%     \setmainfont{DejaVuSans.ttf}[Path=\detokenize{/home/petr/Projects/PyRigi/.venv/lib/python3.12/site-packages/matplotlib/mpl-data/fonts/ttf/}]
%%     \setsansfont{DejaVuSans.ttf}[Path=\detokenize{/home/petr/Projects/PyRigi/.venv/lib/python3.12/site-packages/matplotlib/mpl-data/fonts/ttf/}]
%%     \setmonofont{DejaVuSansMono.ttf}[Path=\detokenize{/home/petr/Projects/PyRigi/.venv/lib/python3.12/site-packages/matplotlib/mpl-data/fonts/ttf/}]
%%   \fi
%%   \makeatletter\@ifpackageloaded{under\Score{}}{}{\usepackage[strings]{under\Score{}}}\makeatother
%%
\begingroup%
\makeatletter%
\begin{pgfpicture}%
\pgfpathrectangle{\pgfpointorigin}{\pgfqpoint{8.384376in}{2.841853in}}%
\pgfusepath{use as bounding box, clip}%
\begin{pgfscope}%
\pgfsetbuttcap%
\pgfsetmiterjoin%
\definecolor{currentfill}{rgb}{1.000000,1.000000,1.000000}%
\pgfsetfillcolor{currentfill}%
\pgfsetlinewidth{0.000000pt}%
\definecolor{currentstroke}{rgb}{1.000000,1.000000,1.000000}%
\pgfsetstrokecolor{currentstroke}%
\pgfsetdash{}{0pt}%
\pgfpathmoveto{\pgfqpoint{0.000000in}{0.000000in}}%
\pgfpathlineto{\pgfqpoint{8.384376in}{0.000000in}}%
\pgfpathlineto{\pgfqpoint{8.384376in}{2.841853in}}%
\pgfpathlineto{\pgfqpoint{0.000000in}{2.841853in}}%
\pgfpathlineto{\pgfqpoint{0.000000in}{0.000000in}}%
\pgfpathclose%
\pgfusepath{fill}%
\end{pgfscope}%
\begin{pgfscope}%
\pgfsetbuttcap%
\pgfsetmiterjoin%
\definecolor{currentfill}{rgb}{1.000000,1.000000,1.000000}%
\pgfsetfillcolor{currentfill}%
\pgfsetlinewidth{0.000000pt}%
\definecolor{currentstroke}{rgb}{0.000000,0.000000,0.000000}%
\pgfsetstrokecolor{currentstroke}%
\pgfsetstrokeopacity{0.000000}%
\pgfsetdash{}{0pt}%
\pgfpathmoveto{\pgfqpoint{0.588387in}{0.521603in}}%
\pgfpathlineto{\pgfqpoint{5.257411in}{0.521603in}}%
\pgfpathlineto{\pgfqpoint{5.257411in}{2.713741in}}%
\pgfpathlineto{\pgfqpoint{0.588387in}{2.713741in}}%
\pgfpathlineto{\pgfqpoint{0.588387in}{0.521603in}}%
\pgfpathclose%
\pgfusepath{fill}%
\end{pgfscope}%
\begin{pgfscope}%
\pgfsetbuttcap%
\pgfsetroundjoin%
\definecolor{currentfill}{rgb}{0.000000,0.000000,0.000000}%
\pgfsetfillcolor{currentfill}%
\pgfsetlinewidth{0.803000pt}%
\definecolor{currentstroke}{rgb}{0.000000,0.000000,0.000000}%
\pgfsetstrokecolor{currentstroke}%
\pgfsetdash{}{0pt}%
\pgfsys@defobject{currentmarker}{\pgfqpoint{0.000000in}{-0.048611in}}{\pgfqpoint{0.000000in}{0.000000in}}{%
\pgfpathmoveto{\pgfqpoint{0.000000in}{0.000000in}}%
\pgfpathlineto{\pgfqpoint{0.000000in}{-0.048611in}}%
\pgfusepath{stroke,fill}%
}%
\begin{pgfscope}%
\pgfsys@transformshift{1.093344in}{0.521603in}%
\pgfsys@useobject{currentmarker}{}%
\end{pgfscope}%
\end{pgfscope}%
\begin{pgfscope}%
\definecolor{textcolor}{rgb}{0.000000,0.000000,0.000000}%
\pgfsetstrokecolor{textcolor}%
\pgfsetfillcolor{textcolor}%
\pgftext[x=1.093344in,y=0.424381in,,top]{\color{textcolor}{\rmfamily\fontsize{10.000000}{12.000000}\selectfont\catcode`\^=\active\def^{\ifmmode\sp\else\^{}\fi}\catcode`\%=\active\def%{\%}$\mathdefault{4}$}}%
\end{pgfscope}%
\begin{pgfscope}%
\pgfsetbuttcap%
\pgfsetroundjoin%
\definecolor{currentfill}{rgb}{0.000000,0.000000,0.000000}%
\pgfsetfillcolor{currentfill}%
\pgfsetlinewidth{0.803000pt}%
\definecolor{currentstroke}{rgb}{0.000000,0.000000,0.000000}%
\pgfsetstrokecolor{currentstroke}%
\pgfsetdash{}{0pt}%
\pgfsys@defobject{currentmarker}{\pgfqpoint{0.000000in}{-0.048611in}}{\pgfqpoint{0.000000in}{0.000000in}}{%
\pgfpathmoveto{\pgfqpoint{0.000000in}{0.000000in}}%
\pgfpathlineto{\pgfqpoint{0.000000in}{-0.048611in}}%
\pgfusepath{stroke,fill}%
}%
\begin{pgfscope}%
\pgfsys@transformshift{1.678802in}{0.521603in}%
\pgfsys@useobject{currentmarker}{}%
\end{pgfscope}%
\end{pgfscope}%
\begin{pgfscope}%
\definecolor{textcolor}{rgb}{0.000000,0.000000,0.000000}%
\pgfsetstrokecolor{textcolor}%
\pgfsetfillcolor{textcolor}%
\pgftext[x=1.678802in,y=0.424381in,,top]{\color{textcolor}{\rmfamily\fontsize{10.000000}{12.000000}\selectfont\catcode`\^=\active\def^{\ifmmode\sp\else\^{}\fi}\catcode`\%=\active\def%{\%}$\mathdefault{8}$}}%
\end{pgfscope}%
\begin{pgfscope}%
\pgfsetbuttcap%
\pgfsetroundjoin%
\definecolor{currentfill}{rgb}{0.000000,0.000000,0.000000}%
\pgfsetfillcolor{currentfill}%
\pgfsetlinewidth{0.803000pt}%
\definecolor{currentstroke}{rgb}{0.000000,0.000000,0.000000}%
\pgfsetstrokecolor{currentstroke}%
\pgfsetdash{}{0pt}%
\pgfsys@defobject{currentmarker}{\pgfqpoint{0.000000in}{-0.048611in}}{\pgfqpoint{0.000000in}{0.000000in}}{%
\pgfpathmoveto{\pgfqpoint{0.000000in}{0.000000in}}%
\pgfpathlineto{\pgfqpoint{0.000000in}{-0.048611in}}%
\pgfusepath{stroke,fill}%
}%
\begin{pgfscope}%
\pgfsys@transformshift{2.264259in}{0.521603in}%
\pgfsys@useobject{currentmarker}{}%
\end{pgfscope}%
\end{pgfscope}%
\begin{pgfscope}%
\definecolor{textcolor}{rgb}{0.000000,0.000000,0.000000}%
\pgfsetstrokecolor{textcolor}%
\pgfsetfillcolor{textcolor}%
\pgftext[x=2.264259in,y=0.424381in,,top]{\color{textcolor}{\rmfamily\fontsize{10.000000}{12.000000}\selectfont\catcode`\^=\active\def^{\ifmmode\sp\else\^{}\fi}\catcode`\%=\active\def%{\%}$\mathdefault{12}$}}%
\end{pgfscope}%
\begin{pgfscope}%
\pgfsetbuttcap%
\pgfsetroundjoin%
\definecolor{currentfill}{rgb}{0.000000,0.000000,0.000000}%
\pgfsetfillcolor{currentfill}%
\pgfsetlinewidth{0.803000pt}%
\definecolor{currentstroke}{rgb}{0.000000,0.000000,0.000000}%
\pgfsetstrokecolor{currentstroke}%
\pgfsetdash{}{0pt}%
\pgfsys@defobject{currentmarker}{\pgfqpoint{0.000000in}{-0.048611in}}{\pgfqpoint{0.000000in}{0.000000in}}{%
\pgfpathmoveto{\pgfqpoint{0.000000in}{0.000000in}}%
\pgfpathlineto{\pgfqpoint{0.000000in}{-0.048611in}}%
\pgfusepath{stroke,fill}%
}%
\begin{pgfscope}%
\pgfsys@transformshift{2.849717in}{0.521603in}%
\pgfsys@useobject{currentmarker}{}%
\end{pgfscope}%
\end{pgfscope}%
\begin{pgfscope}%
\definecolor{textcolor}{rgb}{0.000000,0.000000,0.000000}%
\pgfsetstrokecolor{textcolor}%
\pgfsetfillcolor{textcolor}%
\pgftext[x=2.849717in,y=0.424381in,,top]{\color{textcolor}{\rmfamily\fontsize{10.000000}{12.000000}\selectfont\catcode`\^=\active\def^{\ifmmode\sp\else\^{}\fi}\catcode`\%=\active\def%{\%}$\mathdefault{16}$}}%
\end{pgfscope}%
\begin{pgfscope}%
\pgfsetbuttcap%
\pgfsetroundjoin%
\definecolor{currentfill}{rgb}{0.000000,0.000000,0.000000}%
\pgfsetfillcolor{currentfill}%
\pgfsetlinewidth{0.803000pt}%
\definecolor{currentstroke}{rgb}{0.000000,0.000000,0.000000}%
\pgfsetstrokecolor{currentstroke}%
\pgfsetdash{}{0pt}%
\pgfsys@defobject{currentmarker}{\pgfqpoint{0.000000in}{-0.048611in}}{\pgfqpoint{0.000000in}{0.000000in}}{%
\pgfpathmoveto{\pgfqpoint{0.000000in}{0.000000in}}%
\pgfpathlineto{\pgfqpoint{0.000000in}{-0.048611in}}%
\pgfusepath{stroke,fill}%
}%
\begin{pgfscope}%
\pgfsys@transformshift{3.435175in}{0.521603in}%
\pgfsys@useobject{currentmarker}{}%
\end{pgfscope}%
\end{pgfscope}%
\begin{pgfscope}%
\definecolor{textcolor}{rgb}{0.000000,0.000000,0.000000}%
\pgfsetstrokecolor{textcolor}%
\pgfsetfillcolor{textcolor}%
\pgftext[x=3.435175in,y=0.424381in,,top]{\color{textcolor}{\rmfamily\fontsize{10.000000}{12.000000}\selectfont\catcode`\^=\active\def^{\ifmmode\sp\else\^{}\fi}\catcode`\%=\active\def%{\%}$\mathdefault{20}$}}%
\end{pgfscope}%
\begin{pgfscope}%
\pgfsetbuttcap%
\pgfsetroundjoin%
\definecolor{currentfill}{rgb}{0.000000,0.000000,0.000000}%
\pgfsetfillcolor{currentfill}%
\pgfsetlinewidth{0.803000pt}%
\definecolor{currentstroke}{rgb}{0.000000,0.000000,0.000000}%
\pgfsetstrokecolor{currentstroke}%
\pgfsetdash{}{0pt}%
\pgfsys@defobject{currentmarker}{\pgfqpoint{0.000000in}{-0.048611in}}{\pgfqpoint{0.000000in}{0.000000in}}{%
\pgfpathmoveto{\pgfqpoint{0.000000in}{0.000000in}}%
\pgfpathlineto{\pgfqpoint{0.000000in}{-0.048611in}}%
\pgfusepath{stroke,fill}%
}%
\begin{pgfscope}%
\pgfsys@transformshift{4.020632in}{0.521603in}%
\pgfsys@useobject{currentmarker}{}%
\end{pgfscope}%
\end{pgfscope}%
\begin{pgfscope}%
\definecolor{textcolor}{rgb}{0.000000,0.000000,0.000000}%
\pgfsetstrokecolor{textcolor}%
\pgfsetfillcolor{textcolor}%
\pgftext[x=4.020632in,y=0.424381in,,top]{\color{textcolor}{\rmfamily\fontsize{10.000000}{12.000000}\selectfont\catcode`\^=\active\def^{\ifmmode\sp\else\^{}\fi}\catcode`\%=\active\def%{\%}$\mathdefault{24}$}}%
\end{pgfscope}%
\begin{pgfscope}%
\pgfsetbuttcap%
\pgfsetroundjoin%
\definecolor{currentfill}{rgb}{0.000000,0.000000,0.000000}%
\pgfsetfillcolor{currentfill}%
\pgfsetlinewidth{0.803000pt}%
\definecolor{currentstroke}{rgb}{0.000000,0.000000,0.000000}%
\pgfsetstrokecolor{currentstroke}%
\pgfsetdash{}{0pt}%
\pgfsys@defobject{currentmarker}{\pgfqpoint{0.000000in}{-0.048611in}}{\pgfqpoint{0.000000in}{0.000000in}}{%
\pgfpathmoveto{\pgfqpoint{0.000000in}{0.000000in}}%
\pgfpathlineto{\pgfqpoint{0.000000in}{-0.048611in}}%
\pgfusepath{stroke,fill}%
}%
\begin{pgfscope}%
\pgfsys@transformshift{4.606090in}{0.521603in}%
\pgfsys@useobject{currentmarker}{}%
\end{pgfscope}%
\end{pgfscope}%
\begin{pgfscope}%
\definecolor{textcolor}{rgb}{0.000000,0.000000,0.000000}%
\pgfsetstrokecolor{textcolor}%
\pgfsetfillcolor{textcolor}%
\pgftext[x=4.606090in,y=0.424381in,,top]{\color{textcolor}{\rmfamily\fontsize{10.000000}{12.000000}\selectfont\catcode`\^=\active\def^{\ifmmode\sp\else\^{}\fi}\catcode`\%=\active\def%{\%}$\mathdefault{28}$}}%
\end{pgfscope}%
\begin{pgfscope}%
\pgfsetbuttcap%
\pgfsetroundjoin%
\definecolor{currentfill}{rgb}{0.000000,0.000000,0.000000}%
\pgfsetfillcolor{currentfill}%
\pgfsetlinewidth{0.803000pt}%
\definecolor{currentstroke}{rgb}{0.000000,0.000000,0.000000}%
\pgfsetstrokecolor{currentstroke}%
\pgfsetdash{}{0pt}%
\pgfsys@defobject{currentmarker}{\pgfqpoint{0.000000in}{-0.048611in}}{\pgfqpoint{0.000000in}{0.000000in}}{%
\pgfpathmoveto{\pgfqpoint{0.000000in}{0.000000in}}%
\pgfpathlineto{\pgfqpoint{0.000000in}{-0.048611in}}%
\pgfusepath{stroke,fill}%
}%
\begin{pgfscope}%
\pgfsys@transformshift{5.191547in}{0.521603in}%
\pgfsys@useobject{currentmarker}{}%
\end{pgfscope}%
\end{pgfscope}%
\begin{pgfscope}%
\definecolor{textcolor}{rgb}{0.000000,0.000000,0.000000}%
\pgfsetstrokecolor{textcolor}%
\pgfsetfillcolor{textcolor}%
\pgftext[x=5.191547in,y=0.424381in,,top]{\color{textcolor}{\rmfamily\fontsize{10.000000}{12.000000}\selectfont\catcode`\^=\active\def^{\ifmmode\sp\else\^{}\fi}\catcode`\%=\active\def%{\%}$\mathdefault{32}$}}%
\end{pgfscope}%
\begin{pgfscope}%
\definecolor{textcolor}{rgb}{0.000000,0.000000,0.000000}%
\pgfsetstrokecolor{textcolor}%
\pgfsetfillcolor{textcolor}%
\pgftext[x=2.922899in,y=0.234413in,,top]{\color{textcolor}{\rmfamily\fontsize{10.000000}{12.000000}\selectfont\catcode`\^=\active\def^{\ifmmode\sp\else\^{}\fi}\catcode`\%=\active\def%{\%}Monochromatic classes}}%
\end{pgfscope}%
\begin{pgfscope}%
\pgfsetbuttcap%
\pgfsetroundjoin%
\definecolor{currentfill}{rgb}{0.000000,0.000000,0.000000}%
\pgfsetfillcolor{currentfill}%
\pgfsetlinewidth{0.803000pt}%
\definecolor{currentstroke}{rgb}{0.000000,0.000000,0.000000}%
\pgfsetstrokecolor{currentstroke}%
\pgfsetdash{}{0pt}%
\pgfsys@defobject{currentmarker}{\pgfqpoint{-0.048611in}{0.000000in}}{\pgfqpoint{-0.000000in}{0.000000in}}{%
\pgfpathmoveto{\pgfqpoint{-0.000000in}{0.000000in}}%
\pgfpathlineto{\pgfqpoint{-0.048611in}{0.000000in}}%
\pgfusepath{stroke,fill}%
}%
\begin{pgfscope}%
\pgfsys@transformshift{0.588387in}{0.670551in}%
\pgfsys@useobject{currentmarker}{}%
\end{pgfscope}%
\end{pgfscope}%
\begin{pgfscope}%
\definecolor{textcolor}{rgb}{0.000000,0.000000,0.000000}%
\pgfsetstrokecolor{textcolor}%
\pgfsetfillcolor{textcolor}%
\pgftext[x=0.289968in, y=0.617790in, left, base]{\color{textcolor}{\rmfamily\fontsize{10.000000}{12.000000}\selectfont\catcode`\^=\active\def^{\ifmmode\sp\else\^{}\fi}\catcode`\%=\active\def%{\%}$\mathdefault{10^{1}}$}}%
\end{pgfscope}%
\begin{pgfscope}%
\pgfsetbuttcap%
\pgfsetroundjoin%
\definecolor{currentfill}{rgb}{0.000000,0.000000,0.000000}%
\pgfsetfillcolor{currentfill}%
\pgfsetlinewidth{0.803000pt}%
\definecolor{currentstroke}{rgb}{0.000000,0.000000,0.000000}%
\pgfsetstrokecolor{currentstroke}%
\pgfsetdash{}{0pt}%
\pgfsys@defobject{currentmarker}{\pgfqpoint{-0.048611in}{0.000000in}}{\pgfqpoint{-0.000000in}{0.000000in}}{%
\pgfpathmoveto{\pgfqpoint{-0.000000in}{0.000000in}}%
\pgfpathlineto{\pgfqpoint{-0.048611in}{0.000000in}}%
\pgfusepath{stroke,fill}%
}%
\begin{pgfscope}%
\pgfsys@transformshift{0.588387in}{1.343398in}%
\pgfsys@useobject{currentmarker}{}%
\end{pgfscope}%
\end{pgfscope}%
\begin{pgfscope}%
\definecolor{textcolor}{rgb}{0.000000,0.000000,0.000000}%
\pgfsetstrokecolor{textcolor}%
\pgfsetfillcolor{textcolor}%
\pgftext[x=0.289968in, y=1.290636in, left, base]{\color{textcolor}{\rmfamily\fontsize{10.000000}{12.000000}\selectfont\catcode`\^=\active\def^{\ifmmode\sp\else\^{}\fi}\catcode`\%=\active\def%{\%}$\mathdefault{10^{2}}$}}%
\end{pgfscope}%
\begin{pgfscope}%
\pgfsetbuttcap%
\pgfsetroundjoin%
\definecolor{currentfill}{rgb}{0.000000,0.000000,0.000000}%
\pgfsetfillcolor{currentfill}%
\pgfsetlinewidth{0.803000pt}%
\definecolor{currentstroke}{rgb}{0.000000,0.000000,0.000000}%
\pgfsetstrokecolor{currentstroke}%
\pgfsetdash{}{0pt}%
\pgfsys@defobject{currentmarker}{\pgfqpoint{-0.048611in}{0.000000in}}{\pgfqpoint{-0.000000in}{0.000000in}}{%
\pgfpathmoveto{\pgfqpoint{-0.000000in}{0.000000in}}%
\pgfpathlineto{\pgfqpoint{-0.048611in}{0.000000in}}%
\pgfusepath{stroke,fill}%
}%
\begin{pgfscope}%
\pgfsys@transformshift{0.588387in}{2.016245in}%
\pgfsys@useobject{currentmarker}{}%
\end{pgfscope}%
\end{pgfscope}%
\begin{pgfscope}%
\definecolor{textcolor}{rgb}{0.000000,0.000000,0.000000}%
\pgfsetstrokecolor{textcolor}%
\pgfsetfillcolor{textcolor}%
\pgftext[x=0.289968in, y=1.963483in, left, base]{\color{textcolor}{\rmfamily\fontsize{10.000000}{12.000000}\selectfont\catcode`\^=\active\def^{\ifmmode\sp\else\^{}\fi}\catcode`\%=\active\def%{\%}$\mathdefault{10^{3}}$}}%
\end{pgfscope}%
\begin{pgfscope}%
\pgfsetbuttcap%
\pgfsetroundjoin%
\definecolor{currentfill}{rgb}{0.000000,0.000000,0.000000}%
\pgfsetfillcolor{currentfill}%
\pgfsetlinewidth{0.803000pt}%
\definecolor{currentstroke}{rgb}{0.000000,0.000000,0.000000}%
\pgfsetstrokecolor{currentstroke}%
\pgfsetdash{}{0pt}%
\pgfsys@defobject{currentmarker}{\pgfqpoint{-0.048611in}{0.000000in}}{\pgfqpoint{-0.000000in}{0.000000in}}{%
\pgfpathmoveto{\pgfqpoint{-0.000000in}{0.000000in}}%
\pgfpathlineto{\pgfqpoint{-0.048611in}{0.000000in}}%
\pgfusepath{stroke,fill}%
}%
\begin{pgfscope}%
\pgfsys@transformshift{0.588387in}{2.689091in}%
\pgfsys@useobject{currentmarker}{}%
\end{pgfscope}%
\end{pgfscope}%
\begin{pgfscope}%
\definecolor{textcolor}{rgb}{0.000000,0.000000,0.000000}%
\pgfsetstrokecolor{textcolor}%
\pgfsetfillcolor{textcolor}%
\pgftext[x=0.289968in, y=2.636330in, left, base]{\color{textcolor}{\rmfamily\fontsize{10.000000}{12.000000}\selectfont\catcode`\^=\active\def^{\ifmmode\sp\else\^{}\fi}\catcode`\%=\active\def%{\%}$\mathdefault{10^{4}}$}}%
\end{pgfscope}%
\begin{pgfscope}%
\pgfsetbuttcap%
\pgfsetroundjoin%
\definecolor{currentfill}{rgb}{0.000000,0.000000,0.000000}%
\pgfsetfillcolor{currentfill}%
\pgfsetlinewidth{0.602250pt}%
\definecolor{currentstroke}{rgb}{0.000000,0.000000,0.000000}%
\pgfsetstrokecolor{currentstroke}%
\pgfsetdash{}{0pt}%
\pgfsys@defobject{currentmarker}{\pgfqpoint{-0.027778in}{0.000000in}}{\pgfqpoint{-0.000000in}{0.000000in}}{%
\pgfpathmoveto{\pgfqpoint{-0.000000in}{0.000000in}}%
\pgfpathlineto{\pgfqpoint{-0.027778in}{0.000000in}}%
\pgfusepath{stroke,fill}%
}%
\begin{pgfscope}%
\pgfsys@transformshift{0.588387in}{0.566326in}%
\pgfsys@useobject{currentmarker}{}%
\end{pgfscope}%
\end{pgfscope}%
\begin{pgfscope}%
\pgfsetbuttcap%
\pgfsetroundjoin%
\definecolor{currentfill}{rgb}{0.000000,0.000000,0.000000}%
\pgfsetfillcolor{currentfill}%
\pgfsetlinewidth{0.602250pt}%
\definecolor{currentstroke}{rgb}{0.000000,0.000000,0.000000}%
\pgfsetstrokecolor{currentstroke}%
\pgfsetdash{}{0pt}%
\pgfsys@defobject{currentmarker}{\pgfqpoint{-0.027778in}{0.000000in}}{\pgfqpoint{-0.000000in}{0.000000in}}{%
\pgfpathmoveto{\pgfqpoint{-0.000000in}{0.000000in}}%
\pgfpathlineto{\pgfqpoint{-0.027778in}{0.000000in}}%
\pgfusepath{stroke,fill}%
}%
\begin{pgfscope}%
\pgfsys@transformshift{0.588387in}{0.605346in}%
\pgfsys@useobject{currentmarker}{}%
\end{pgfscope}%
\end{pgfscope}%
\begin{pgfscope}%
\pgfsetbuttcap%
\pgfsetroundjoin%
\definecolor{currentfill}{rgb}{0.000000,0.000000,0.000000}%
\pgfsetfillcolor{currentfill}%
\pgfsetlinewidth{0.602250pt}%
\definecolor{currentstroke}{rgb}{0.000000,0.000000,0.000000}%
\pgfsetstrokecolor{currentstroke}%
\pgfsetdash{}{0pt}%
\pgfsys@defobject{currentmarker}{\pgfqpoint{-0.027778in}{0.000000in}}{\pgfqpoint{-0.000000in}{0.000000in}}{%
\pgfpathmoveto{\pgfqpoint{-0.000000in}{0.000000in}}%
\pgfpathlineto{\pgfqpoint{-0.027778in}{0.000000in}}%
\pgfusepath{stroke,fill}%
}%
\begin{pgfscope}%
\pgfsys@transformshift{0.588387in}{0.639763in}%
\pgfsys@useobject{currentmarker}{}%
\end{pgfscope}%
\end{pgfscope}%
\begin{pgfscope}%
\pgfsetbuttcap%
\pgfsetroundjoin%
\definecolor{currentfill}{rgb}{0.000000,0.000000,0.000000}%
\pgfsetfillcolor{currentfill}%
\pgfsetlinewidth{0.602250pt}%
\definecolor{currentstroke}{rgb}{0.000000,0.000000,0.000000}%
\pgfsetstrokecolor{currentstroke}%
\pgfsetdash{}{0pt}%
\pgfsys@defobject{currentmarker}{\pgfqpoint{-0.027778in}{0.000000in}}{\pgfqpoint{-0.000000in}{0.000000in}}{%
\pgfpathmoveto{\pgfqpoint{-0.000000in}{0.000000in}}%
\pgfpathlineto{\pgfqpoint{-0.027778in}{0.000000in}}%
\pgfusepath{stroke,fill}%
}%
\begin{pgfscope}%
\pgfsys@transformshift{0.588387in}{0.873098in}%
\pgfsys@useobject{currentmarker}{}%
\end{pgfscope}%
\end{pgfscope}%
\begin{pgfscope}%
\pgfsetbuttcap%
\pgfsetroundjoin%
\definecolor{currentfill}{rgb}{0.000000,0.000000,0.000000}%
\pgfsetfillcolor{currentfill}%
\pgfsetlinewidth{0.602250pt}%
\definecolor{currentstroke}{rgb}{0.000000,0.000000,0.000000}%
\pgfsetstrokecolor{currentstroke}%
\pgfsetdash{}{0pt}%
\pgfsys@defobject{currentmarker}{\pgfqpoint{-0.027778in}{0.000000in}}{\pgfqpoint{-0.000000in}{0.000000in}}{%
\pgfpathmoveto{\pgfqpoint{-0.000000in}{0.000000in}}%
\pgfpathlineto{\pgfqpoint{-0.027778in}{0.000000in}}%
\pgfusepath{stroke,fill}%
}%
\begin{pgfscope}%
\pgfsys@transformshift{0.588387in}{0.991581in}%
\pgfsys@useobject{currentmarker}{}%
\end{pgfscope}%
\end{pgfscope}%
\begin{pgfscope}%
\pgfsetbuttcap%
\pgfsetroundjoin%
\definecolor{currentfill}{rgb}{0.000000,0.000000,0.000000}%
\pgfsetfillcolor{currentfill}%
\pgfsetlinewidth{0.602250pt}%
\definecolor{currentstroke}{rgb}{0.000000,0.000000,0.000000}%
\pgfsetstrokecolor{currentstroke}%
\pgfsetdash{}{0pt}%
\pgfsys@defobject{currentmarker}{\pgfqpoint{-0.027778in}{0.000000in}}{\pgfqpoint{-0.000000in}{0.000000in}}{%
\pgfpathmoveto{\pgfqpoint{-0.000000in}{0.000000in}}%
\pgfpathlineto{\pgfqpoint{-0.027778in}{0.000000in}}%
\pgfusepath{stroke,fill}%
}%
\begin{pgfscope}%
\pgfsys@transformshift{0.588387in}{1.075645in}%
\pgfsys@useobject{currentmarker}{}%
\end{pgfscope}%
\end{pgfscope}%
\begin{pgfscope}%
\pgfsetbuttcap%
\pgfsetroundjoin%
\definecolor{currentfill}{rgb}{0.000000,0.000000,0.000000}%
\pgfsetfillcolor{currentfill}%
\pgfsetlinewidth{0.602250pt}%
\definecolor{currentstroke}{rgb}{0.000000,0.000000,0.000000}%
\pgfsetstrokecolor{currentstroke}%
\pgfsetdash{}{0pt}%
\pgfsys@defobject{currentmarker}{\pgfqpoint{-0.027778in}{0.000000in}}{\pgfqpoint{-0.000000in}{0.000000in}}{%
\pgfpathmoveto{\pgfqpoint{-0.000000in}{0.000000in}}%
\pgfpathlineto{\pgfqpoint{-0.027778in}{0.000000in}}%
\pgfusepath{stroke,fill}%
}%
\begin{pgfscope}%
\pgfsys@transformshift{0.588387in}{1.140851in}%
\pgfsys@useobject{currentmarker}{}%
\end{pgfscope}%
\end{pgfscope}%
\begin{pgfscope}%
\pgfsetbuttcap%
\pgfsetroundjoin%
\definecolor{currentfill}{rgb}{0.000000,0.000000,0.000000}%
\pgfsetfillcolor{currentfill}%
\pgfsetlinewidth{0.602250pt}%
\definecolor{currentstroke}{rgb}{0.000000,0.000000,0.000000}%
\pgfsetstrokecolor{currentstroke}%
\pgfsetdash{}{0pt}%
\pgfsys@defobject{currentmarker}{\pgfqpoint{-0.027778in}{0.000000in}}{\pgfqpoint{-0.000000in}{0.000000in}}{%
\pgfpathmoveto{\pgfqpoint{-0.000000in}{0.000000in}}%
\pgfpathlineto{\pgfqpoint{-0.027778in}{0.000000in}}%
\pgfusepath{stroke,fill}%
}%
\begin{pgfscope}%
\pgfsys@transformshift{0.588387in}{1.194128in}%
\pgfsys@useobject{currentmarker}{}%
\end{pgfscope}%
\end{pgfscope}%
\begin{pgfscope}%
\pgfsetbuttcap%
\pgfsetroundjoin%
\definecolor{currentfill}{rgb}{0.000000,0.000000,0.000000}%
\pgfsetfillcolor{currentfill}%
\pgfsetlinewidth{0.602250pt}%
\definecolor{currentstroke}{rgb}{0.000000,0.000000,0.000000}%
\pgfsetstrokecolor{currentstroke}%
\pgfsetdash{}{0pt}%
\pgfsys@defobject{currentmarker}{\pgfqpoint{-0.027778in}{0.000000in}}{\pgfqpoint{-0.000000in}{0.000000in}}{%
\pgfpathmoveto{\pgfqpoint{-0.000000in}{0.000000in}}%
\pgfpathlineto{\pgfqpoint{-0.027778in}{0.000000in}}%
\pgfusepath{stroke,fill}%
}%
\begin{pgfscope}%
\pgfsys@transformshift{0.588387in}{1.239173in}%
\pgfsys@useobject{currentmarker}{}%
\end{pgfscope}%
\end{pgfscope}%
\begin{pgfscope}%
\pgfsetbuttcap%
\pgfsetroundjoin%
\definecolor{currentfill}{rgb}{0.000000,0.000000,0.000000}%
\pgfsetfillcolor{currentfill}%
\pgfsetlinewidth{0.602250pt}%
\definecolor{currentstroke}{rgb}{0.000000,0.000000,0.000000}%
\pgfsetstrokecolor{currentstroke}%
\pgfsetdash{}{0pt}%
\pgfsys@defobject{currentmarker}{\pgfqpoint{-0.027778in}{0.000000in}}{\pgfqpoint{-0.000000in}{0.000000in}}{%
\pgfpathmoveto{\pgfqpoint{-0.000000in}{0.000000in}}%
\pgfpathlineto{\pgfqpoint{-0.027778in}{0.000000in}}%
\pgfusepath{stroke,fill}%
}%
\begin{pgfscope}%
\pgfsys@transformshift{0.588387in}{1.278192in}%
\pgfsys@useobject{currentmarker}{}%
\end{pgfscope}%
\end{pgfscope}%
\begin{pgfscope}%
\pgfsetbuttcap%
\pgfsetroundjoin%
\definecolor{currentfill}{rgb}{0.000000,0.000000,0.000000}%
\pgfsetfillcolor{currentfill}%
\pgfsetlinewidth{0.602250pt}%
\definecolor{currentstroke}{rgb}{0.000000,0.000000,0.000000}%
\pgfsetstrokecolor{currentstroke}%
\pgfsetdash{}{0pt}%
\pgfsys@defobject{currentmarker}{\pgfqpoint{-0.027778in}{0.000000in}}{\pgfqpoint{-0.000000in}{0.000000in}}{%
\pgfpathmoveto{\pgfqpoint{-0.000000in}{0.000000in}}%
\pgfpathlineto{\pgfqpoint{-0.027778in}{0.000000in}}%
\pgfusepath{stroke,fill}%
}%
\begin{pgfscope}%
\pgfsys@transformshift{0.588387in}{1.312610in}%
\pgfsys@useobject{currentmarker}{}%
\end{pgfscope}%
\end{pgfscope}%
\begin{pgfscope}%
\pgfsetbuttcap%
\pgfsetroundjoin%
\definecolor{currentfill}{rgb}{0.000000,0.000000,0.000000}%
\pgfsetfillcolor{currentfill}%
\pgfsetlinewidth{0.602250pt}%
\definecolor{currentstroke}{rgb}{0.000000,0.000000,0.000000}%
\pgfsetstrokecolor{currentstroke}%
\pgfsetdash{}{0pt}%
\pgfsys@defobject{currentmarker}{\pgfqpoint{-0.027778in}{0.000000in}}{\pgfqpoint{-0.000000in}{0.000000in}}{%
\pgfpathmoveto{\pgfqpoint{-0.000000in}{0.000000in}}%
\pgfpathlineto{\pgfqpoint{-0.027778in}{0.000000in}}%
\pgfusepath{stroke,fill}%
}%
\begin{pgfscope}%
\pgfsys@transformshift{0.588387in}{1.545945in}%
\pgfsys@useobject{currentmarker}{}%
\end{pgfscope}%
\end{pgfscope}%
\begin{pgfscope}%
\pgfsetbuttcap%
\pgfsetroundjoin%
\definecolor{currentfill}{rgb}{0.000000,0.000000,0.000000}%
\pgfsetfillcolor{currentfill}%
\pgfsetlinewidth{0.602250pt}%
\definecolor{currentstroke}{rgb}{0.000000,0.000000,0.000000}%
\pgfsetstrokecolor{currentstroke}%
\pgfsetdash{}{0pt}%
\pgfsys@defobject{currentmarker}{\pgfqpoint{-0.027778in}{0.000000in}}{\pgfqpoint{-0.000000in}{0.000000in}}{%
\pgfpathmoveto{\pgfqpoint{-0.000000in}{0.000000in}}%
\pgfpathlineto{\pgfqpoint{-0.027778in}{0.000000in}}%
\pgfusepath{stroke,fill}%
}%
\begin{pgfscope}%
\pgfsys@transformshift{0.588387in}{1.664427in}%
\pgfsys@useobject{currentmarker}{}%
\end{pgfscope}%
\end{pgfscope}%
\begin{pgfscope}%
\pgfsetbuttcap%
\pgfsetroundjoin%
\definecolor{currentfill}{rgb}{0.000000,0.000000,0.000000}%
\pgfsetfillcolor{currentfill}%
\pgfsetlinewidth{0.602250pt}%
\definecolor{currentstroke}{rgb}{0.000000,0.000000,0.000000}%
\pgfsetstrokecolor{currentstroke}%
\pgfsetdash{}{0pt}%
\pgfsys@defobject{currentmarker}{\pgfqpoint{-0.027778in}{0.000000in}}{\pgfqpoint{-0.000000in}{0.000000in}}{%
\pgfpathmoveto{\pgfqpoint{-0.000000in}{0.000000in}}%
\pgfpathlineto{\pgfqpoint{-0.027778in}{0.000000in}}%
\pgfusepath{stroke,fill}%
}%
\begin{pgfscope}%
\pgfsys@transformshift{0.588387in}{1.748492in}%
\pgfsys@useobject{currentmarker}{}%
\end{pgfscope}%
\end{pgfscope}%
\begin{pgfscope}%
\pgfsetbuttcap%
\pgfsetroundjoin%
\definecolor{currentfill}{rgb}{0.000000,0.000000,0.000000}%
\pgfsetfillcolor{currentfill}%
\pgfsetlinewidth{0.602250pt}%
\definecolor{currentstroke}{rgb}{0.000000,0.000000,0.000000}%
\pgfsetstrokecolor{currentstroke}%
\pgfsetdash{}{0pt}%
\pgfsys@defobject{currentmarker}{\pgfqpoint{-0.027778in}{0.000000in}}{\pgfqpoint{-0.000000in}{0.000000in}}{%
\pgfpathmoveto{\pgfqpoint{-0.000000in}{0.000000in}}%
\pgfpathlineto{\pgfqpoint{-0.027778in}{0.000000in}}%
\pgfusepath{stroke,fill}%
}%
\begin{pgfscope}%
\pgfsys@transformshift{0.588387in}{1.813697in}%
\pgfsys@useobject{currentmarker}{}%
\end{pgfscope}%
\end{pgfscope}%
\begin{pgfscope}%
\pgfsetbuttcap%
\pgfsetroundjoin%
\definecolor{currentfill}{rgb}{0.000000,0.000000,0.000000}%
\pgfsetfillcolor{currentfill}%
\pgfsetlinewidth{0.602250pt}%
\definecolor{currentstroke}{rgb}{0.000000,0.000000,0.000000}%
\pgfsetstrokecolor{currentstroke}%
\pgfsetdash{}{0pt}%
\pgfsys@defobject{currentmarker}{\pgfqpoint{-0.027778in}{0.000000in}}{\pgfqpoint{-0.000000in}{0.000000in}}{%
\pgfpathmoveto{\pgfqpoint{-0.000000in}{0.000000in}}%
\pgfpathlineto{\pgfqpoint{-0.027778in}{0.000000in}}%
\pgfusepath{stroke,fill}%
}%
\begin{pgfscope}%
\pgfsys@transformshift{0.588387in}{1.866974in}%
\pgfsys@useobject{currentmarker}{}%
\end{pgfscope}%
\end{pgfscope}%
\begin{pgfscope}%
\pgfsetbuttcap%
\pgfsetroundjoin%
\definecolor{currentfill}{rgb}{0.000000,0.000000,0.000000}%
\pgfsetfillcolor{currentfill}%
\pgfsetlinewidth{0.602250pt}%
\definecolor{currentstroke}{rgb}{0.000000,0.000000,0.000000}%
\pgfsetstrokecolor{currentstroke}%
\pgfsetdash{}{0pt}%
\pgfsys@defobject{currentmarker}{\pgfqpoint{-0.027778in}{0.000000in}}{\pgfqpoint{-0.000000in}{0.000000in}}{%
\pgfpathmoveto{\pgfqpoint{-0.000000in}{0.000000in}}%
\pgfpathlineto{\pgfqpoint{-0.027778in}{0.000000in}}%
\pgfusepath{stroke,fill}%
}%
\begin{pgfscope}%
\pgfsys@transformshift{0.588387in}{1.912019in}%
\pgfsys@useobject{currentmarker}{}%
\end{pgfscope}%
\end{pgfscope}%
\begin{pgfscope}%
\pgfsetbuttcap%
\pgfsetroundjoin%
\definecolor{currentfill}{rgb}{0.000000,0.000000,0.000000}%
\pgfsetfillcolor{currentfill}%
\pgfsetlinewidth{0.602250pt}%
\definecolor{currentstroke}{rgb}{0.000000,0.000000,0.000000}%
\pgfsetstrokecolor{currentstroke}%
\pgfsetdash{}{0pt}%
\pgfsys@defobject{currentmarker}{\pgfqpoint{-0.027778in}{0.000000in}}{\pgfqpoint{-0.000000in}{0.000000in}}{%
\pgfpathmoveto{\pgfqpoint{-0.000000in}{0.000000in}}%
\pgfpathlineto{\pgfqpoint{-0.027778in}{0.000000in}}%
\pgfusepath{stroke,fill}%
}%
\begin{pgfscope}%
\pgfsys@transformshift{0.588387in}{1.951039in}%
\pgfsys@useobject{currentmarker}{}%
\end{pgfscope}%
\end{pgfscope}%
\begin{pgfscope}%
\pgfsetbuttcap%
\pgfsetroundjoin%
\definecolor{currentfill}{rgb}{0.000000,0.000000,0.000000}%
\pgfsetfillcolor{currentfill}%
\pgfsetlinewidth{0.602250pt}%
\definecolor{currentstroke}{rgb}{0.000000,0.000000,0.000000}%
\pgfsetstrokecolor{currentstroke}%
\pgfsetdash{}{0pt}%
\pgfsys@defobject{currentmarker}{\pgfqpoint{-0.027778in}{0.000000in}}{\pgfqpoint{-0.000000in}{0.000000in}}{%
\pgfpathmoveto{\pgfqpoint{-0.000000in}{0.000000in}}%
\pgfpathlineto{\pgfqpoint{-0.027778in}{0.000000in}}%
\pgfusepath{stroke,fill}%
}%
\begin{pgfscope}%
\pgfsys@transformshift{0.588387in}{1.985457in}%
\pgfsys@useobject{currentmarker}{}%
\end{pgfscope}%
\end{pgfscope}%
\begin{pgfscope}%
\pgfsetbuttcap%
\pgfsetroundjoin%
\definecolor{currentfill}{rgb}{0.000000,0.000000,0.000000}%
\pgfsetfillcolor{currentfill}%
\pgfsetlinewidth{0.602250pt}%
\definecolor{currentstroke}{rgb}{0.000000,0.000000,0.000000}%
\pgfsetstrokecolor{currentstroke}%
\pgfsetdash{}{0pt}%
\pgfsys@defobject{currentmarker}{\pgfqpoint{-0.027778in}{0.000000in}}{\pgfqpoint{-0.000000in}{0.000000in}}{%
\pgfpathmoveto{\pgfqpoint{-0.000000in}{0.000000in}}%
\pgfpathlineto{\pgfqpoint{-0.027778in}{0.000000in}}%
\pgfusepath{stroke,fill}%
}%
\begin{pgfscope}%
\pgfsys@transformshift{0.588387in}{2.218792in}%
\pgfsys@useobject{currentmarker}{}%
\end{pgfscope}%
\end{pgfscope}%
\begin{pgfscope}%
\pgfsetbuttcap%
\pgfsetroundjoin%
\definecolor{currentfill}{rgb}{0.000000,0.000000,0.000000}%
\pgfsetfillcolor{currentfill}%
\pgfsetlinewidth{0.602250pt}%
\definecolor{currentstroke}{rgb}{0.000000,0.000000,0.000000}%
\pgfsetstrokecolor{currentstroke}%
\pgfsetdash{}{0pt}%
\pgfsys@defobject{currentmarker}{\pgfqpoint{-0.027778in}{0.000000in}}{\pgfqpoint{-0.000000in}{0.000000in}}{%
\pgfpathmoveto{\pgfqpoint{-0.000000in}{0.000000in}}%
\pgfpathlineto{\pgfqpoint{-0.027778in}{0.000000in}}%
\pgfusepath{stroke,fill}%
}%
\begin{pgfscope}%
\pgfsys@transformshift{0.588387in}{2.337274in}%
\pgfsys@useobject{currentmarker}{}%
\end{pgfscope}%
\end{pgfscope}%
\begin{pgfscope}%
\pgfsetbuttcap%
\pgfsetroundjoin%
\definecolor{currentfill}{rgb}{0.000000,0.000000,0.000000}%
\pgfsetfillcolor{currentfill}%
\pgfsetlinewidth{0.602250pt}%
\definecolor{currentstroke}{rgb}{0.000000,0.000000,0.000000}%
\pgfsetstrokecolor{currentstroke}%
\pgfsetdash{}{0pt}%
\pgfsys@defobject{currentmarker}{\pgfqpoint{-0.027778in}{0.000000in}}{\pgfqpoint{-0.000000in}{0.000000in}}{%
\pgfpathmoveto{\pgfqpoint{-0.000000in}{0.000000in}}%
\pgfpathlineto{\pgfqpoint{-0.027778in}{0.000000in}}%
\pgfusepath{stroke,fill}%
}%
\begin{pgfscope}%
\pgfsys@transformshift{0.588387in}{2.421339in}%
\pgfsys@useobject{currentmarker}{}%
\end{pgfscope}%
\end{pgfscope}%
\begin{pgfscope}%
\pgfsetbuttcap%
\pgfsetroundjoin%
\definecolor{currentfill}{rgb}{0.000000,0.000000,0.000000}%
\pgfsetfillcolor{currentfill}%
\pgfsetlinewidth{0.602250pt}%
\definecolor{currentstroke}{rgb}{0.000000,0.000000,0.000000}%
\pgfsetstrokecolor{currentstroke}%
\pgfsetdash{}{0pt}%
\pgfsys@defobject{currentmarker}{\pgfqpoint{-0.027778in}{0.000000in}}{\pgfqpoint{-0.000000in}{0.000000in}}{%
\pgfpathmoveto{\pgfqpoint{-0.000000in}{0.000000in}}%
\pgfpathlineto{\pgfqpoint{-0.027778in}{0.000000in}}%
\pgfusepath{stroke,fill}%
}%
\begin{pgfscope}%
\pgfsys@transformshift{0.588387in}{2.486544in}%
\pgfsys@useobject{currentmarker}{}%
\end{pgfscope}%
\end{pgfscope}%
\begin{pgfscope}%
\pgfsetbuttcap%
\pgfsetroundjoin%
\definecolor{currentfill}{rgb}{0.000000,0.000000,0.000000}%
\pgfsetfillcolor{currentfill}%
\pgfsetlinewidth{0.602250pt}%
\definecolor{currentstroke}{rgb}{0.000000,0.000000,0.000000}%
\pgfsetstrokecolor{currentstroke}%
\pgfsetdash{}{0pt}%
\pgfsys@defobject{currentmarker}{\pgfqpoint{-0.027778in}{0.000000in}}{\pgfqpoint{-0.000000in}{0.000000in}}{%
\pgfpathmoveto{\pgfqpoint{-0.000000in}{0.000000in}}%
\pgfpathlineto{\pgfqpoint{-0.027778in}{0.000000in}}%
\pgfusepath{stroke,fill}%
}%
\begin{pgfscope}%
\pgfsys@transformshift{0.588387in}{2.539821in}%
\pgfsys@useobject{currentmarker}{}%
\end{pgfscope}%
\end{pgfscope}%
\begin{pgfscope}%
\pgfsetbuttcap%
\pgfsetroundjoin%
\definecolor{currentfill}{rgb}{0.000000,0.000000,0.000000}%
\pgfsetfillcolor{currentfill}%
\pgfsetlinewidth{0.602250pt}%
\definecolor{currentstroke}{rgb}{0.000000,0.000000,0.000000}%
\pgfsetstrokecolor{currentstroke}%
\pgfsetdash{}{0pt}%
\pgfsys@defobject{currentmarker}{\pgfqpoint{-0.027778in}{0.000000in}}{\pgfqpoint{-0.000000in}{0.000000in}}{%
\pgfpathmoveto{\pgfqpoint{-0.000000in}{0.000000in}}%
\pgfpathlineto{\pgfqpoint{-0.027778in}{0.000000in}}%
\pgfusepath{stroke,fill}%
}%
\begin{pgfscope}%
\pgfsys@transformshift{0.588387in}{2.584866in}%
\pgfsys@useobject{currentmarker}{}%
\end{pgfscope}%
\end{pgfscope}%
\begin{pgfscope}%
\pgfsetbuttcap%
\pgfsetroundjoin%
\definecolor{currentfill}{rgb}{0.000000,0.000000,0.000000}%
\pgfsetfillcolor{currentfill}%
\pgfsetlinewidth{0.602250pt}%
\definecolor{currentstroke}{rgb}{0.000000,0.000000,0.000000}%
\pgfsetstrokecolor{currentstroke}%
\pgfsetdash{}{0pt}%
\pgfsys@defobject{currentmarker}{\pgfqpoint{-0.027778in}{0.000000in}}{\pgfqpoint{-0.000000in}{0.000000in}}{%
\pgfpathmoveto{\pgfqpoint{-0.000000in}{0.000000in}}%
\pgfpathlineto{\pgfqpoint{-0.027778in}{0.000000in}}%
\pgfusepath{stroke,fill}%
}%
\begin{pgfscope}%
\pgfsys@transformshift{0.588387in}{2.623886in}%
\pgfsys@useobject{currentmarker}{}%
\end{pgfscope}%
\end{pgfscope}%
\begin{pgfscope}%
\pgfsetbuttcap%
\pgfsetroundjoin%
\definecolor{currentfill}{rgb}{0.000000,0.000000,0.000000}%
\pgfsetfillcolor{currentfill}%
\pgfsetlinewidth{0.602250pt}%
\definecolor{currentstroke}{rgb}{0.000000,0.000000,0.000000}%
\pgfsetstrokecolor{currentstroke}%
\pgfsetdash{}{0pt}%
\pgfsys@defobject{currentmarker}{\pgfqpoint{-0.027778in}{0.000000in}}{\pgfqpoint{-0.000000in}{0.000000in}}{%
\pgfpathmoveto{\pgfqpoint{-0.000000in}{0.000000in}}%
\pgfpathlineto{\pgfqpoint{-0.027778in}{0.000000in}}%
\pgfusepath{stroke,fill}%
}%
\begin{pgfscope}%
\pgfsys@transformshift{0.588387in}{2.658303in}%
\pgfsys@useobject{currentmarker}{}%
\end{pgfscope}%
\end{pgfscope}%
\begin{pgfscope}%
\definecolor{textcolor}{rgb}{0.000000,0.000000,0.000000}%
\pgfsetstrokecolor{textcolor}%
\pgfsetfillcolor{textcolor}%
\pgftext[x=0.234413in,y=1.617672in,,bottom,rotate=90.000000]{\color{textcolor}{\rmfamily\fontsize{10.000000}{12.000000}\selectfont\catcode`\^=\active\def^{\ifmmode\sp\else\^{}\fi}\catcode`\%=\active\def%{\%}Time [ms]}}%
\end{pgfscope}%
\begin{pgfscope}%
\pgfpathrectangle{\pgfqpoint{0.588387in}{0.521603in}}{\pgfqpoint{4.669024in}{2.192138in}}%
\pgfusepath{clip}%
\pgfsetrectcap%
\pgfsetroundjoin%
\pgfsetlinewidth{1.505625pt}%
\pgfsetstrokecolor{currentstroke1}%
\pgfsetdash{}{0pt}%
\pgfpathmoveto{\pgfqpoint{0.800616in}{0.650495in}}%
\pgfpathlineto{\pgfqpoint{0.946980in}{0.708985in}}%
\pgfpathlineto{\pgfqpoint{1.093344in}{0.681895in}}%
\pgfpathlineto{\pgfqpoint{1.239709in}{0.712675in}}%
\pgfpathlineto{\pgfqpoint{1.386073in}{0.670369in}}%
\pgfpathlineto{\pgfqpoint{1.532438in}{0.669607in}}%
\pgfpathlineto{\pgfqpoint{1.678802in}{0.699589in}}%
\pgfpathlineto{\pgfqpoint{1.825166in}{0.690039in}}%
\pgfpathlineto{\pgfqpoint{1.971531in}{0.708009in}}%
\pgfpathlineto{\pgfqpoint{2.117895in}{0.752784in}}%
\pgfpathlineto{\pgfqpoint{2.264259in}{0.854173in}}%
\pgfpathlineto{\pgfqpoint{2.410624in}{0.868496in}}%
\pgfpathlineto{\pgfqpoint{2.556988in}{0.937271in}}%
\pgfpathlineto{\pgfqpoint{2.703353in}{0.995673in}}%
\pgfpathlineto{\pgfqpoint{2.849717in}{1.073200in}}%
\pgfpathlineto{\pgfqpoint{2.996081in}{1.206990in}}%
\pgfpathlineto{\pgfqpoint{3.142446in}{1.203120in}}%
\pgfpathlineto{\pgfqpoint{3.288810in}{1.221092in}}%
\pgfpathlineto{\pgfqpoint{3.435175in}{1.337196in}}%
\pgfpathlineto{\pgfqpoint{3.581539in}{1.379628in}}%
\pgfpathlineto{\pgfqpoint{3.727903in}{1.442241in}}%
\pgfpathlineto{\pgfqpoint{3.874268in}{1.372574in}}%
\pgfpathlineto{\pgfqpoint{4.020632in}{1.362485in}}%
\pgfpathlineto{\pgfqpoint{4.166997in}{1.427828in}}%
\pgfpathlineto{\pgfqpoint{4.313361in}{1.408603in}}%
\pgfpathlineto{\pgfqpoint{4.459725in}{1.570120in}}%
\pgfpathlineto{\pgfqpoint{4.898818in}{1.814573in}}%
\pgfpathlineto{\pgfqpoint{5.045183in}{2.117887in}}%
\pgfusepath{stroke}%
\end{pgfscope}%
\begin{pgfscope}%
\pgfpathrectangle{\pgfqpoint{0.588387in}{0.521603in}}{\pgfqpoint{4.669024in}{2.192138in}}%
\pgfusepath{clip}%
\pgfsetrectcap%
\pgfsetroundjoin%
\pgfsetlinewidth{1.505625pt}%
\pgfsetstrokecolor{currentstroke2}%
\pgfsetdash{}{0pt}%
\pgfpathmoveto{\pgfqpoint{0.800616in}{0.649984in}}%
\pgfpathlineto{\pgfqpoint{0.946980in}{0.706070in}}%
\pgfpathlineto{\pgfqpoint{1.093344in}{0.679740in}}%
\pgfpathlineto{\pgfqpoint{1.239709in}{0.713475in}}%
\pgfpathlineto{\pgfqpoint{1.386073in}{0.668892in}}%
\pgfpathlineto{\pgfqpoint{1.532438in}{0.671503in}}%
\pgfpathlineto{\pgfqpoint{1.678802in}{0.696049in}}%
\pgfpathlineto{\pgfqpoint{1.825166in}{0.689568in}}%
\pgfpathlineto{\pgfqpoint{1.971531in}{0.710268in}}%
\pgfpathlineto{\pgfqpoint{2.117895in}{0.748819in}}%
\pgfpathlineto{\pgfqpoint{2.264259in}{0.858878in}}%
\pgfpathlineto{\pgfqpoint{2.410624in}{0.862877in}}%
\pgfpathlineto{\pgfqpoint{2.556988in}{0.943753in}}%
\pgfpathlineto{\pgfqpoint{2.703353in}{0.994191in}}%
\pgfpathlineto{\pgfqpoint{2.849717in}{1.064160in}}%
\pgfpathlineto{\pgfqpoint{2.996081in}{1.181183in}}%
\pgfpathlineto{\pgfqpoint{3.142446in}{1.154411in}}%
\pgfpathlineto{\pgfqpoint{3.288810in}{1.230702in}}%
\pgfpathlineto{\pgfqpoint{3.435175in}{1.317121in}}%
\pgfpathlineto{\pgfqpoint{3.581539in}{1.367506in}}%
\pgfpathlineto{\pgfqpoint{3.727903in}{1.442241in}}%
\pgfpathlineto{\pgfqpoint{3.874268in}{1.304378in}}%
\pgfpathlineto{\pgfqpoint{4.020632in}{1.435390in}}%
\pgfpathlineto{\pgfqpoint{4.166997in}{1.409770in}}%
\pgfpathlineto{\pgfqpoint{4.313361in}{1.598612in}}%
\pgfpathlineto{\pgfqpoint{4.459725in}{1.598612in}}%
\pgfpathlineto{\pgfqpoint{4.898818in}{1.749222in}}%
\pgfpathlineto{\pgfqpoint{5.045183in}{1.965991in}}%
\pgfusepath{stroke}%
\end{pgfscope}%
\begin{pgfscope}%
\pgfpathrectangle{\pgfqpoint{0.588387in}{0.521603in}}{\pgfqpoint{4.669024in}{2.192138in}}%
\pgfusepath{clip}%
\pgfsetrectcap%
\pgfsetroundjoin%
\pgfsetlinewidth{1.505625pt}%
\pgfsetstrokecolor{currentstroke3}%
\pgfsetdash{}{0pt}%
\pgfpathmoveto{\pgfqpoint{0.800616in}{0.622189in}}%
\pgfpathlineto{\pgfqpoint{0.946980in}{0.680427in}}%
\pgfpathlineto{\pgfqpoint{1.093344in}{0.645249in}}%
\pgfpathlineto{\pgfqpoint{1.239709in}{0.676747in}}%
\pgfpathlineto{\pgfqpoint{1.386073in}{0.666666in}}%
\pgfpathlineto{\pgfqpoint{1.532438in}{0.621246in}}%
\pgfpathlineto{\pgfqpoint{1.678802in}{0.644913in}}%
\pgfpathlineto{\pgfqpoint{1.825166in}{0.646520in}}%
\pgfpathlineto{\pgfqpoint{1.971531in}{0.642073in}}%
\pgfpathlineto{\pgfqpoint{2.117895in}{0.711392in}}%
\pgfpathlineto{\pgfqpoint{2.264259in}{0.694916in}}%
\pgfpathlineto{\pgfqpoint{2.410624in}{0.737214in}}%
\pgfpathlineto{\pgfqpoint{2.556988in}{0.838540in}}%
\pgfpathlineto{\pgfqpoint{2.703353in}{0.947714in}}%
\pgfpathlineto{\pgfqpoint{2.849717in}{1.105482in}}%
\pgfpathlineto{\pgfqpoint{2.996081in}{1.295908in}}%
\pgfpathlineto{\pgfqpoint{3.142446in}{1.466710in}}%
\pgfpathlineto{\pgfqpoint{3.288810in}{1.643221in}}%
\pgfpathlineto{\pgfqpoint{3.435175in}{1.849411in}}%
\pgfpathlineto{\pgfqpoint{3.581539in}{2.014682in}}%
\pgfpathlineto{\pgfqpoint{3.727903in}{2.227500in}}%
\pgfpathlineto{\pgfqpoint{4.020632in}{2.614099in}}%
\pgfusepath{stroke}%
\end{pgfscope}%
\begin{pgfscope}%
\pgfpathrectangle{\pgfqpoint{0.588387in}{0.521603in}}{\pgfqpoint{4.669024in}{2.192138in}}%
\pgfusepath{clip}%
\pgfsetrectcap%
\pgfsetroundjoin%
\pgfsetlinewidth{1.505625pt}%
\pgfsetstrokecolor{currentstroke4}%
\pgfsetdash{}{0pt}%
\pgfpathmoveto{\pgfqpoint{0.800616in}{0.648839in}}%
\pgfpathlineto{\pgfqpoint{0.946980in}{0.703864in}}%
\pgfpathlineto{\pgfqpoint{1.093344in}{0.677371in}}%
\pgfpathlineto{\pgfqpoint{1.239709in}{0.708157in}}%
\pgfpathlineto{\pgfqpoint{1.386073in}{0.661085in}}%
\pgfpathlineto{\pgfqpoint{1.532438in}{0.668959in}}%
\pgfpathlineto{\pgfqpoint{1.678802in}{0.701206in}}%
\pgfpathlineto{\pgfqpoint{1.825166in}{0.682603in}}%
\pgfpathlineto{\pgfqpoint{1.971531in}{0.692393in}}%
\pgfpathlineto{\pgfqpoint{2.117895in}{0.735757in}}%
\pgfpathlineto{\pgfqpoint{2.264259in}{0.822440in}}%
\pgfpathlineto{\pgfqpoint{2.410624in}{0.842817in}}%
\pgfpathlineto{\pgfqpoint{2.556988in}{0.896223in}}%
\pgfpathlineto{\pgfqpoint{2.703353in}{0.944531in}}%
\pgfpathlineto{\pgfqpoint{2.849717in}{0.999587in}}%
\pgfpathlineto{\pgfqpoint{2.996081in}{1.074181in}}%
\pgfpathlineto{\pgfqpoint{3.142446in}{1.118069in}}%
\pgfpathlineto{\pgfqpoint{3.288810in}{1.153993in}}%
\pgfpathlineto{\pgfqpoint{3.435175in}{1.261275in}}%
\pgfpathlineto{\pgfqpoint{3.581539in}{1.290705in}}%
\pgfpathlineto{\pgfqpoint{3.727903in}{1.399401in}}%
\pgfpathlineto{\pgfqpoint{3.874268in}{1.208385in}}%
\pgfpathlineto{\pgfqpoint{4.020632in}{1.198958in}}%
\pgfpathlineto{\pgfqpoint{4.166997in}{1.546675in}}%
\pgfpathlineto{\pgfqpoint{4.313361in}{1.307699in}}%
\pgfpathlineto{\pgfqpoint{4.459725in}{1.625964in}}%
\pgfpathlineto{\pgfqpoint{4.898818in}{1.542269in}}%
\pgfpathlineto{\pgfqpoint{5.045183in}{1.483466in}}%
\pgfusepath{stroke}%
\end{pgfscope}%
\begin{pgfscope}%
\pgfpathrectangle{\pgfqpoint{0.588387in}{0.521603in}}{\pgfqpoint{4.669024in}{2.192138in}}%
\pgfusepath{clip}%
\pgfsetrectcap%
\pgfsetroundjoin%
\pgfsetlinewidth{1.505625pt}%
\pgfsetstrokecolor{currentstroke5}%
\pgfsetdash{}{0pt}%
\pgfpathmoveto{\pgfqpoint{0.800616in}{0.648732in}}%
\pgfpathlineto{\pgfqpoint{0.946980in}{0.702385in}}%
\pgfpathlineto{\pgfqpoint{1.093344in}{0.678261in}}%
\pgfpathlineto{\pgfqpoint{1.239709in}{0.706527in}}%
\pgfpathlineto{\pgfqpoint{1.386073in}{0.670364in}}%
\pgfpathlineto{\pgfqpoint{1.532438in}{0.694916in}}%
\pgfpathlineto{\pgfqpoint{1.678802in}{0.689547in}}%
\pgfpathlineto{\pgfqpoint{1.825166in}{0.685566in}}%
\pgfpathlineto{\pgfqpoint{1.971531in}{0.696788in}}%
\pgfpathlineto{\pgfqpoint{2.117895in}{0.736590in}}%
\pgfpathlineto{\pgfqpoint{2.264259in}{0.825608in}}%
\pgfpathlineto{\pgfqpoint{2.410624in}{0.840785in}}%
\pgfpathlineto{\pgfqpoint{2.556988in}{0.900558in}}%
\pgfpathlineto{\pgfqpoint{2.703353in}{0.948899in}}%
\pgfpathlineto{\pgfqpoint{2.849717in}{0.998796in}}%
\pgfpathlineto{\pgfqpoint{2.996081in}{1.072708in}}%
\pgfpathlineto{\pgfqpoint{3.142446in}{1.082861in}}%
\pgfpathlineto{\pgfqpoint{3.288810in}{1.111359in}}%
\pgfpathlineto{\pgfqpoint{3.435175in}{1.262819in}}%
\pgfpathlineto{\pgfqpoint{3.581539in}{1.314229in}}%
\pgfpathlineto{\pgfqpoint{3.727903in}{1.367237in}}%
\pgfpathlineto{\pgfqpoint{3.874268in}{1.191682in}}%
\pgfpathlineto{\pgfqpoint{4.020632in}{1.312610in}}%
\pgfpathlineto{\pgfqpoint{4.166997in}{1.473341in}}%
\pgfpathlineto{\pgfqpoint{4.313361in}{1.325317in}}%
\pgfpathlineto{\pgfqpoint{4.459725in}{1.547039in}}%
\pgfpathlineto{\pgfqpoint{4.898818in}{1.535534in}}%
\pgfpathlineto{\pgfqpoint{5.045183in}{1.527086in}}%
\pgfusepath{stroke}%
\end{pgfscope}%
\begin{pgfscope}%
\pgfpathrectangle{\pgfqpoint{0.588387in}{0.521603in}}{\pgfqpoint{4.669024in}{2.192138in}}%
\pgfusepath{clip}%
\pgfsetrectcap%
\pgfsetroundjoin%
\pgfsetlinewidth{1.505625pt}%
\pgfsetstrokecolor{currentstroke6}%
\pgfsetdash{}{0pt}%
\pgfpathmoveto{\pgfqpoint{0.800616in}{0.646306in}}%
\pgfpathlineto{\pgfqpoint{0.946980in}{0.706679in}}%
\pgfpathlineto{\pgfqpoint{1.093344in}{0.676179in}}%
\pgfpathlineto{\pgfqpoint{1.239709in}{0.708482in}}%
\pgfpathlineto{\pgfqpoint{1.386073in}{0.670174in}}%
\pgfpathlineto{\pgfqpoint{1.532438in}{0.660206in}}%
\pgfpathlineto{\pgfqpoint{1.678802in}{0.692798in}}%
\pgfpathlineto{\pgfqpoint{1.825166in}{0.684048in}}%
\pgfpathlineto{\pgfqpoint{1.971531in}{0.696812in}}%
\pgfpathlineto{\pgfqpoint{2.117895in}{0.739902in}}%
\pgfpathlineto{\pgfqpoint{2.264259in}{0.829026in}}%
\pgfpathlineto{\pgfqpoint{2.410624in}{0.848832in}}%
\pgfpathlineto{\pgfqpoint{2.556988in}{0.904444in}}%
\pgfpathlineto{\pgfqpoint{2.703353in}{0.941878in}}%
\pgfpathlineto{\pgfqpoint{2.849717in}{0.999192in}}%
\pgfpathlineto{\pgfqpoint{2.996081in}{1.080715in}}%
\pgfpathlineto{\pgfqpoint{3.142446in}{1.085521in}}%
\pgfpathlineto{\pgfqpoint{3.288810in}{1.163340in}}%
\pgfpathlineto{\pgfqpoint{3.435175in}{1.287184in}}%
\pgfpathlineto{\pgfqpoint{3.581539in}{1.341051in}}%
\pgfpathlineto{\pgfqpoint{3.727903in}{1.449698in}}%
\pgfpathlineto{\pgfqpoint{3.874268in}{1.276360in}}%
\pgfpathlineto{\pgfqpoint{4.020632in}{1.220870in}}%
\pgfpathlineto{\pgfqpoint{4.166997in}{1.674480in}}%
\pgfpathlineto{\pgfqpoint{4.313361in}{1.377816in}}%
\pgfpathlineto{\pgfqpoint{4.459725in}{1.632555in}}%
\pgfpathlineto{\pgfqpoint{4.898818in}{1.649439in}}%
\pgfpathlineto{\pgfqpoint{5.045183in}{1.593070in}}%
\pgfusepath{stroke}%
\end{pgfscope}%
\begin{pgfscope}%
\pgfpathrectangle{\pgfqpoint{0.588387in}{0.521603in}}{\pgfqpoint{4.669024in}{2.192138in}}%
\pgfusepath{clip}%
\pgfsetrectcap%
\pgfsetroundjoin%
\pgfsetlinewidth{1.505625pt}%
\pgfsetstrokecolor{currentstroke7}%
\pgfsetdash{}{0pt}%
\pgfpathmoveto{\pgfqpoint{0.800616in}{0.646776in}}%
\pgfpathlineto{\pgfqpoint{0.946980in}{0.706314in}}%
\pgfpathlineto{\pgfqpoint{1.093344in}{0.677569in}}%
\pgfpathlineto{\pgfqpoint{1.239709in}{0.707506in}}%
\pgfpathlineto{\pgfqpoint{1.386073in}{0.662298in}}%
\pgfpathlineto{\pgfqpoint{1.532438in}{0.663729in}}%
\pgfpathlineto{\pgfqpoint{1.678802in}{0.691890in}}%
\pgfpathlineto{\pgfqpoint{1.825166in}{0.680301in}}%
\pgfpathlineto{\pgfqpoint{1.971531in}{0.695213in}}%
\pgfpathlineto{\pgfqpoint{2.117895in}{0.737422in}}%
\pgfpathlineto{\pgfqpoint{2.264259in}{0.830720in}}%
\pgfpathlineto{\pgfqpoint{2.410624in}{0.841294in}}%
\pgfpathlineto{\pgfqpoint{2.556988in}{0.902896in}}%
\pgfpathlineto{\pgfqpoint{2.703353in}{0.946723in}}%
\pgfpathlineto{\pgfqpoint{2.849717in}{1.002338in}}%
\pgfpathlineto{\pgfqpoint{2.996081in}{1.082147in}}%
\pgfpathlineto{\pgfqpoint{3.142446in}{1.083750in}}%
\pgfpathlineto{\pgfqpoint{3.288810in}{1.167104in}}%
\pgfpathlineto{\pgfqpoint{3.435175in}{1.262819in}}%
\pgfpathlineto{\pgfqpoint{3.581539in}{1.320932in}}%
\pgfpathlineto{\pgfqpoint{3.727903in}{1.420626in}}%
\pgfpathlineto{\pgfqpoint{3.874268in}{1.226373in}}%
\pgfpathlineto{\pgfqpoint{4.020632in}{1.182959in}}%
\pgfpathlineto{\pgfqpoint{4.166997in}{1.555997in}}%
\pgfpathlineto{\pgfqpoint{4.313361in}{1.285408in}}%
\pgfpathlineto{\pgfqpoint{4.459725in}{1.581011in}}%
\pgfpathlineto{\pgfqpoint{4.898818in}{1.526306in}}%
\pgfpathlineto{\pgfqpoint{5.045183in}{1.562972in}}%
\pgfusepath{stroke}%
\end{pgfscope}%
\begin{pgfscope}%
\pgfpathrectangle{\pgfqpoint{0.588387in}{0.521603in}}{\pgfqpoint{4.669024in}{2.192138in}}%
\pgfusepath{clip}%
\pgfsetrectcap%
\pgfsetroundjoin%
\pgfsetlinewidth{1.505625pt}%
\definecolor{currentstroke}{rgb}{0.498039,0.498039,0.498039}%
\pgfsetstrokecolor{currentstroke}%
\pgfsetdash{}{0pt}%
\pgfpathmoveto{\pgfqpoint{0.800616in}{0.653001in}}%
\pgfpathlineto{\pgfqpoint{0.946980in}{0.707288in}}%
\pgfpathlineto{\pgfqpoint{1.093344in}{0.681504in}}%
\pgfpathlineto{\pgfqpoint{1.239709in}{0.710909in}}%
\pgfpathlineto{\pgfqpoint{1.386073in}{0.671131in}}%
\pgfpathlineto{\pgfqpoint{1.532438in}{0.668575in}}%
\pgfpathlineto{\pgfqpoint{1.678802in}{0.694349in}}%
\pgfpathlineto{\pgfqpoint{1.825166in}{0.687578in}}%
\pgfpathlineto{\pgfqpoint{1.971531in}{0.703122in}}%
\pgfpathlineto{\pgfqpoint{2.117895in}{0.754616in}}%
\pgfpathlineto{\pgfqpoint{2.264259in}{0.841498in}}%
\pgfpathlineto{\pgfqpoint{2.410624in}{0.864762in}}%
\pgfpathlineto{\pgfqpoint{2.556988in}{0.933101in}}%
\pgfpathlineto{\pgfqpoint{2.703353in}{1.006906in}}%
\pgfpathlineto{\pgfqpoint{2.849717in}{1.070424in}}%
\pgfpathlineto{\pgfqpoint{2.996081in}{1.204650in}}%
\pgfpathlineto{\pgfqpoint{3.142446in}{1.169365in}}%
\pgfpathlineto{\pgfqpoint{3.288810in}{1.206524in}}%
\pgfpathlineto{\pgfqpoint{3.435175in}{1.434749in}}%
\pgfpathlineto{\pgfqpoint{3.581539in}{1.373893in}}%
\pgfpathlineto{\pgfqpoint{3.727903in}{1.494125in}}%
\pgfpathlineto{\pgfqpoint{3.874268in}{1.186729in}}%
\pgfpathlineto{\pgfqpoint{4.020632in}{1.521580in}}%
\pgfpathlineto{\pgfqpoint{4.166997in}{1.651991in}}%
\pgfpathlineto{\pgfqpoint{4.313361in}{1.463822in}}%
\pgfpathlineto{\pgfqpoint{4.459725in}{1.867582in}}%
\pgfpathlineto{\pgfqpoint{4.898818in}{2.146188in}}%
\pgfpathlineto{\pgfqpoint{5.045183in}{1.946437in}}%
\pgfusepath{stroke}%
\end{pgfscope}%
\begin{pgfscope}%
\pgfpathrectangle{\pgfqpoint{0.588387in}{0.521603in}}{\pgfqpoint{4.669024in}{2.192138in}}%
\pgfusepath{clip}%
\pgfsetrectcap%
\pgfsetroundjoin%
\pgfsetlinewidth{1.505625pt}%
\definecolor{currentstroke}{rgb}{0.737255,0.741176,0.133333}%
\pgfsetstrokecolor{currentstroke}%
\pgfsetdash{}{0pt}%
\pgfpathmoveto{\pgfqpoint{0.800616in}{0.645339in}}%
\pgfpathlineto{\pgfqpoint{0.946980in}{0.701034in}}%
\pgfpathlineto{\pgfqpoint{1.093344in}{0.674081in}}%
\pgfpathlineto{\pgfqpoint{1.239709in}{0.703568in}}%
\pgfpathlineto{\pgfqpoint{1.386073in}{0.667811in}}%
\pgfpathlineto{\pgfqpoint{1.532438in}{0.662319in}}%
\pgfpathlineto{\pgfqpoint{1.678802in}{0.689410in}}%
\pgfpathlineto{\pgfqpoint{1.825166in}{0.689755in}}%
\pgfpathlineto{\pgfqpoint{1.971531in}{0.698402in}}%
\pgfpathlineto{\pgfqpoint{2.117895in}{0.748019in}}%
\pgfpathlineto{\pgfqpoint{2.264259in}{0.834357in}}%
\pgfpathlineto{\pgfqpoint{2.410624in}{0.856181in}}%
\pgfpathlineto{\pgfqpoint{2.556988in}{0.919857in}}%
\pgfpathlineto{\pgfqpoint{2.703353in}{0.990078in}}%
\pgfpathlineto{\pgfqpoint{2.849717in}{1.059050in}}%
\pgfpathlineto{\pgfqpoint{2.996081in}{1.192173in}}%
\pgfpathlineto{\pgfqpoint{3.142446in}{1.131196in}}%
\pgfpathlineto{\pgfqpoint{3.288810in}{1.207456in}}%
\pgfpathlineto{\pgfqpoint{3.435175in}{1.379370in}}%
\pgfpathlineto{\pgfqpoint{3.581539in}{1.343690in}}%
\pgfpathlineto{\pgfqpoint{3.727903in}{1.482560in}}%
\pgfpathlineto{\pgfqpoint{3.874268in}{1.184221in}}%
\pgfpathlineto{\pgfqpoint{4.020632in}{1.444317in}}%
\pgfpathlineto{\pgfqpoint{4.166997in}{1.781934in}}%
\pgfpathlineto{\pgfqpoint{4.313361in}{1.464788in}}%
\pgfpathlineto{\pgfqpoint{4.459725in}{1.799323in}}%
\pgfpathlineto{\pgfqpoint{4.898818in}{1.685107in}}%
\pgfusepath{stroke}%
\end{pgfscope}%
\begin{pgfscope}%
\pgfsetrectcap%
\pgfsetmiterjoin%
\pgfsetlinewidth{0.803000pt}%
\definecolor{currentstroke}{rgb}{0.000000,0.000000,0.000000}%
\pgfsetstrokecolor{currentstroke}%
\pgfsetdash{}{0pt}%
\pgfpathmoveto{\pgfqpoint{0.588387in}{0.521603in}}%
\pgfpathlineto{\pgfqpoint{0.588387in}{2.713741in}}%
\pgfusepath{stroke}%
\end{pgfscope}%
\begin{pgfscope}%
\pgfsetrectcap%
\pgfsetmiterjoin%
\pgfsetlinewidth{0.803000pt}%
\definecolor{currentstroke}{rgb}{0.000000,0.000000,0.000000}%
\pgfsetstrokecolor{currentstroke}%
\pgfsetdash{}{0pt}%
\pgfpathmoveto{\pgfqpoint{5.257411in}{0.521603in}}%
\pgfpathlineto{\pgfqpoint{5.257411in}{2.713741in}}%
\pgfusepath{stroke}%
\end{pgfscope}%
\begin{pgfscope}%
\pgfsetrectcap%
\pgfsetmiterjoin%
\pgfsetlinewidth{0.803000pt}%
\definecolor{currentstroke}{rgb}{0.000000,0.000000,0.000000}%
\pgfsetstrokecolor{currentstroke}%
\pgfsetdash{}{0pt}%
\pgfpathmoveto{\pgfqpoint{0.588387in}{0.521603in}}%
\pgfpathlineto{\pgfqpoint{5.257411in}{0.521603in}}%
\pgfusepath{stroke}%
\end{pgfscope}%
\begin{pgfscope}%
\pgfsetrectcap%
\pgfsetmiterjoin%
\pgfsetlinewidth{0.803000pt}%
\definecolor{currentstroke}{rgb}{0.000000,0.000000,0.000000}%
\pgfsetstrokecolor{currentstroke}%
\pgfsetdash{}{0pt}%
\pgfpathmoveto{\pgfqpoint{0.588387in}{2.713741in}}%
\pgfpathlineto{\pgfqpoint{5.257411in}{2.713741in}}%
\pgfusepath{stroke}%
\end{pgfscope}%
\begin{pgfscope}%
\pgfsetbuttcap%
\pgfsetmiterjoin%
\definecolor{currentfill}{rgb}{1.000000,1.000000,1.000000}%
\pgfsetfillcolor{currentfill}%
\pgfsetfillopacity{0.800000}%
\pgfsetlinewidth{1.003750pt}%
\definecolor{currentstroke}{rgb}{0.800000,0.800000,0.800000}%
\pgfsetstrokecolor{currentstroke}%
\pgfsetstrokeopacity{0.800000}%
\pgfsetdash{}{0pt}%
\pgfpathmoveto{\pgfqpoint{5.344911in}{0.941624in}}%
\pgfpathlineto{\pgfqpoint{8.259376in}{0.941624in}}%
\pgfpathquadraticcurveto{\pgfqpoint{8.284376in}{0.941624in}}{\pgfqpoint{8.284376in}{0.966624in}}%
\pgfpathlineto{\pgfqpoint{8.284376in}{2.626241in}}%
\pgfpathquadraticcurveto{\pgfqpoint{8.284376in}{2.651241in}}{\pgfqpoint{8.259376in}{2.651241in}}%
\pgfpathlineto{\pgfqpoint{5.344911in}{2.651241in}}%
\pgfpathquadraticcurveto{\pgfqpoint{5.319911in}{2.651241in}}{\pgfqpoint{5.319911in}{2.626241in}}%
\pgfpathlineto{\pgfqpoint{5.319911in}{0.966624in}}%
\pgfpathquadraticcurveto{\pgfqpoint{5.319911in}{0.941624in}}{\pgfqpoint{5.344911in}{0.941624in}}%
\pgfpathlineto{\pgfqpoint{5.344911in}{0.941624in}}%
\pgfpathclose%
\pgfusepath{stroke,fill}%
\end{pgfscope}%
\begin{pgfscope}%
\pgfsetrectcap%
\pgfsetroundjoin%
\pgfsetlinewidth{1.505625pt}%
\pgfsetstrokecolor{currentstroke3}%
\pgfsetdash{}{0pt}%
\pgfpathmoveto{\pgfqpoint{5.369911in}{2.550021in}}%
\pgfpathlineto{\pgfqpoint{5.494911in}{2.550021in}}%
\pgfpathlineto{\pgfqpoint{5.619911in}{2.550021in}}%
\pgfusepath{stroke}%
\end{pgfscope}%
\begin{pgfscope}%
\definecolor{textcolor}{rgb}{0.000000,0.000000,0.000000}%
\pgfsetstrokecolor{textcolor}%
\pgfsetfillcolor{textcolor}%
\pgftext[x=5.719911in,y=2.506271in,left,base]{\color{textcolor}{\rmfamily\fontsize{9.000000}{10.800000}\selectfont\catcode`\^=\active\def^{\ifmmode\sp\else\^{}\fi}\catcode`\%=\active\def%{\%}\NaiveCycles{}}}%
\end{pgfscope}%
\begin{pgfscope}%
\pgfsetrectcap%
\pgfsetroundjoin%
\pgfsetlinewidth{1.505625pt}%
\pgfsetstrokecolor{currentstroke1}%
\pgfsetdash{}{0pt}%
\pgfpathmoveto{\pgfqpoint{5.369911in}{2.366549in}}%
\pgfpathlineto{\pgfqpoint{5.494911in}{2.366549in}}%
\pgfpathlineto{\pgfqpoint{5.619911in}{2.366549in}}%
\pgfusepath{stroke}%
\end{pgfscope}%
\begin{pgfscope}%
\definecolor{textcolor}{rgb}{0.000000,0.000000,0.000000}%
\pgfsetstrokecolor{textcolor}%
\pgfsetfillcolor{textcolor}%
\pgftext[x=5.719911in,y=2.322799in,left,base]{\color{textcolor}{\rmfamily\fontsize{9.000000}{10.800000}\selectfont\catcode`\^=\active\def^{\ifmmode\sp\else\^{}\fi}\catcode`\%=\active\def%{\%}\CyclesMatchChunks{} \& \MergeLinear{}}}%
\end{pgfscope}%
\begin{pgfscope}%
\pgfsetrectcap%
\pgfsetroundjoin%
\pgfsetlinewidth{1.505625pt}%
\pgfsetstrokecolor{currentstroke2}%
\pgfsetdash{}{0pt}%
\pgfpathmoveto{\pgfqpoint{5.369911in}{2.179599in}}%
\pgfpathlineto{\pgfqpoint{5.494911in}{2.179599in}}%
\pgfpathlineto{\pgfqpoint{5.619911in}{2.179599in}}%
\pgfusepath{stroke}%
\end{pgfscope}%
\begin{pgfscope}%
\definecolor{textcolor}{rgb}{0.000000,0.000000,0.000000}%
\pgfsetstrokecolor{textcolor}%
\pgfsetfillcolor{textcolor}%
\pgftext[x=5.719911in,y=2.135849in,left,base]{\color{textcolor}{\rmfamily\fontsize{9.000000}{10.800000}\selectfont\catcode`\^=\active\def^{\ifmmode\sp\else\^{}\fi}\catcode`\%=\active\def%{\%}\CyclesMatchChunks{} \& \SharedVertices{}}}%
\end{pgfscope}%
\begin{pgfscope}%
\pgfsetrectcap%
\pgfsetroundjoin%
\pgfsetlinewidth{1.505625pt}%
\pgfsetstrokecolor{currentstroke4}%
\pgfsetdash{}{0pt}%
\pgfpathmoveto{\pgfqpoint{5.369911in}{1.992648in}}%
\pgfpathlineto{\pgfqpoint{5.494911in}{1.992648in}}%
\pgfpathlineto{\pgfqpoint{5.619911in}{1.992648in}}%
\pgfusepath{stroke}%
\end{pgfscope}%
\begin{pgfscope}%
\definecolor{textcolor}{rgb}{0.000000,0.000000,0.000000}%
\pgfsetstrokecolor{textcolor}%
\pgfsetfillcolor{textcolor}%
\pgftext[x=5.719911in,y=1.948898in,left,base]{\color{textcolor}{\rmfamily\fontsize{9.000000}{10.800000}\selectfont\catcode`\^=\active\def^{\ifmmode\sp\else\^{}\fi}\catcode`\%=\active\def%{\%}\Neighbors{} \& \MergeLinear{}}}%
\end{pgfscope}%
\begin{pgfscope}%
\pgfsetrectcap%
\pgfsetroundjoin%
\pgfsetlinewidth{1.505625pt}%
\pgfsetstrokecolor{currentstroke5}%
\pgfsetdash{}{0pt}%
\pgfpathmoveto{\pgfqpoint{5.369911in}{1.809177in}}%
\pgfpathlineto{\pgfqpoint{5.494911in}{1.809177in}}%
\pgfpathlineto{\pgfqpoint{5.619911in}{1.809177in}}%
\pgfusepath{stroke}%
\end{pgfscope}%
\begin{pgfscope}%
\definecolor{textcolor}{rgb}{0.000000,0.000000,0.000000}%
\pgfsetstrokecolor{textcolor}%
\pgfsetfillcolor{textcolor}%
\pgftext[x=5.719911in,y=1.765427in,left,base]{\color{textcolor}{\rmfamily\fontsize{9.000000}{10.800000}\selectfont\catcode`\^=\active\def^{\ifmmode\sp\else\^{}\fi}\catcode`\%=\active\def%{\%}\Neighbors{} \& \SharedVertices{}}}%
\end{pgfscope}%
\begin{pgfscope}%
\pgfsetrectcap%
\pgfsetroundjoin%
\pgfsetlinewidth{1.505625pt}%
\pgfsetstrokecolor{currentstroke6}%
\pgfsetdash{}{0pt}%
\pgfpathmoveto{\pgfqpoint{5.369911in}{1.622226in}}%
\pgfpathlineto{\pgfqpoint{5.494911in}{1.622226in}}%
\pgfpathlineto{\pgfqpoint{5.619911in}{1.622226in}}%
\pgfusepath{stroke}%
\end{pgfscope}%
\begin{pgfscope}%
\definecolor{textcolor}{rgb}{0.000000,0.000000,0.000000}%
\pgfsetstrokecolor{textcolor}%
\pgfsetfillcolor{textcolor}%
\pgftext[x=5.719911in,y=1.578476in,left,base]{\color{textcolor}{\rmfamily\fontsize{9.000000}{10.800000}\selectfont\catcode`\^=\active\def^{\ifmmode\sp\else\^{}\fi}\catcode`\%=\active\def%{\%}\NeighborsDegree{} \& \MergeLinear{}}}%
\end{pgfscope}%
\begin{pgfscope}%
\pgfsetrectcap%
\pgfsetroundjoin%
\pgfsetlinewidth{1.505625pt}%
\pgfsetstrokecolor{currentstroke7}%
\pgfsetdash{}{0pt}%
\pgfpathmoveto{\pgfqpoint{5.369911in}{1.435276in}}%
\pgfpathlineto{\pgfqpoint{5.494911in}{1.435276in}}%
\pgfpathlineto{\pgfqpoint{5.619911in}{1.435276in}}%
\pgfusepath{stroke}%
\end{pgfscope}%
\begin{pgfscope}%
\definecolor{textcolor}{rgb}{0.000000,0.000000,0.000000}%
\pgfsetstrokecolor{textcolor}%
\pgfsetfillcolor{textcolor}%
\pgftext[x=5.719911in,y=1.391526in,left,base]{\color{textcolor}{\rmfamily\fontsize{9.000000}{10.800000}\selectfont\catcode`\^=\active\def^{\ifmmode\sp\else\^{}\fi}\catcode`\%=\active\def%{\%}\NeighborsDegree{} \& \SharedVertices{}}}%
\end{pgfscope}%
\begin{pgfscope}%
\pgfsetrectcap%
\pgfsetroundjoin%
\pgfsetlinewidth{1.505625pt}%
\definecolor{currentstroke}{rgb}{0.498039,0.498039,0.498039}%
\pgfsetstrokecolor{currentstroke}%
\pgfsetdash{}{0pt}%
\pgfpathmoveto{\pgfqpoint{5.369911in}{1.248326in}}%
\pgfpathlineto{\pgfqpoint{5.494911in}{1.248326in}}%
\pgfpathlineto{\pgfqpoint{5.619911in}{1.248326in}}%
\pgfusepath{stroke}%
\end{pgfscope}%
\begin{pgfscope}%
\definecolor{textcolor}{rgb}{0.000000,0.000000,0.000000}%
\pgfsetstrokecolor{textcolor}%
\pgfsetfillcolor{textcolor}%
\pgftext[x=5.719911in,y=1.204576in,left,base]{\color{textcolor}{\rmfamily\fontsize{9.000000}{10.800000}\selectfont\catcode`\^=\active\def^{\ifmmode\sp\else\^{}\fi}\catcode`\%=\active\def%{\%}\None{} \& \MergeLinear{}}}%
\end{pgfscope}%
\begin{pgfscope}%
\pgfsetrectcap%
\pgfsetroundjoin%
\pgfsetlinewidth{1.505625pt}%
\definecolor{currentstroke}{rgb}{0.737255,0.741176,0.133333}%
\pgfsetstrokecolor{currentstroke}%
\pgfsetdash{}{0pt}%
\pgfpathmoveto{\pgfqpoint{5.369911in}{1.064854in}}%
\pgfpathlineto{\pgfqpoint{5.494911in}{1.064854in}}%
\pgfpathlineto{\pgfqpoint{5.619911in}{1.064854in}}%
\pgfusepath{stroke}%
\end{pgfscope}%
\begin{pgfscope}%
\definecolor{textcolor}{rgb}{0.000000,0.000000,0.000000}%
\pgfsetstrokecolor{textcolor}%
\pgfsetfillcolor{textcolor}%
\pgftext[x=5.719911in,y=1.021104in,left,base]{\color{textcolor}{\rmfamily\fontsize{9.000000}{10.800000}\selectfont\catcode`\^=\active\def^{\ifmmode\sp\else\^{}\fi}\catcode`\%=\active\def%{\%}\None{} \& \SharedVertices{}}}%
\end{pgfscope}%
\end{pgfpicture}%
\makeatother%
\endgroup%
}
	\caption[Mean runtime for globally rigid graphs (all)]{
		Mean running time to find all NAC-colorings for globally rigid graphs.}%
	\label{fig:graph_globally_rigid_all_runtime}
\end{figure}%
\begin{figure}[thbp]
	\centering
	\scalebox{\BenchFigureScale}{%% Creator: Matplotlib, PGF backend
%%
%% To include the figure in your LaTeX document, write
%%   \input{<filename>.pgf}
%%
%% Make sure the required packages are loaded in your preamble
%%   \usepackage{pgf}
%%
%% Also ensure that all the required font packages are loaded; for instance,
%% the lmodern package is sometimes necessary when using math font.
%%   \usepackage{lmodern}
%%
%% Figures using additional raster images can only be included by \input if
%% they are in the same directory as the main LaTeX file. For loading figures
%% from other directories you can use the `import` package
%%   \usepackage{import}
%%
%% and then include the figures with
%%   \import{<path to file>}{<filename>.pgf}
%%
%% Matplotlib used the following preamble
%%   \def\mathdefault#1{#1}
%%   \everymath=\expandafter{\the\everymath\displaystyle}
%%   \IfFileExists{scrextend.sty}{
%%     \usepackage[fontsize=10.000000pt]{scrextend}
%%   }{
%%     \renewcommand{\normalsize}{\fontsize{10.000000}{12.000000}\selectfont}
%%     \normalsize
%%   }
%%   
%%   \ifdefined\pdftexversion\else  % non-pdftex case.
%%     \usepackage{fontspec}
%%     \setmainfont{DejaVuSans.ttf}[Path=\detokenize{/home/petr/Projects/PyRigi/.venv/lib/python3.12/site-packages/matplotlib/mpl-data/fonts/ttf/}]
%%     \setsansfont{DejaVuSans.ttf}[Path=\detokenize{/home/petr/Projects/PyRigi/.venv/lib/python3.12/site-packages/matplotlib/mpl-data/fonts/ttf/}]
%%     \setmonofont{DejaVuSansMono.ttf}[Path=\detokenize{/home/petr/Projects/PyRigi/.venv/lib/python3.12/site-packages/matplotlib/mpl-data/fonts/ttf/}]
%%   \fi
%%   \makeatletter\@ifpackageloaded{under\Score{}}{}{\usepackage[strings]{under\Score{}}}\makeatother
%%
\begingroup%
\makeatletter%
\begin{pgfpicture}%
\pgfpathrectangle{\pgfpointorigin}{\pgfqpoint{8.384376in}{2.841849in}}%
\pgfusepath{use as bounding box, clip}%
\begin{pgfscope}%
\pgfsetbuttcap%
\pgfsetmiterjoin%
\definecolor{currentfill}{rgb}{1.000000,1.000000,1.000000}%
\pgfsetfillcolor{currentfill}%
\pgfsetlinewidth{0.000000pt}%
\definecolor{currentstroke}{rgb}{1.000000,1.000000,1.000000}%
\pgfsetstrokecolor{currentstroke}%
\pgfsetdash{}{0pt}%
\pgfpathmoveto{\pgfqpoint{0.000000in}{0.000000in}}%
\pgfpathlineto{\pgfqpoint{8.384376in}{0.000000in}}%
\pgfpathlineto{\pgfqpoint{8.384376in}{2.841849in}}%
\pgfpathlineto{\pgfqpoint{0.000000in}{2.841849in}}%
\pgfpathlineto{\pgfqpoint{0.000000in}{0.000000in}}%
\pgfpathclose%
\pgfusepath{fill}%
\end{pgfscope}%
\begin{pgfscope}%
\pgfsetbuttcap%
\pgfsetmiterjoin%
\definecolor{currentfill}{rgb}{1.000000,1.000000,1.000000}%
\pgfsetfillcolor{currentfill}%
\pgfsetlinewidth{0.000000pt}%
\definecolor{currentstroke}{rgb}{0.000000,0.000000,0.000000}%
\pgfsetstrokecolor{currentstroke}%
\pgfsetstrokeopacity{0.000000}%
\pgfsetdash{}{0pt}%
\pgfpathmoveto{\pgfqpoint{0.588387in}{0.521603in}}%
\pgfpathlineto{\pgfqpoint{5.257411in}{0.521603in}}%
\pgfpathlineto{\pgfqpoint{5.257411in}{2.531888in}}%
\pgfpathlineto{\pgfqpoint{0.588387in}{2.531888in}}%
\pgfpathlineto{\pgfqpoint{0.588387in}{0.521603in}}%
\pgfpathclose%
\pgfusepath{fill}%
\end{pgfscope}%
\begin{pgfscope}%
\pgfsetbuttcap%
\pgfsetroundjoin%
\definecolor{currentfill}{rgb}{0.000000,0.000000,0.000000}%
\pgfsetfillcolor{currentfill}%
\pgfsetlinewidth{0.803000pt}%
\definecolor{currentstroke}{rgb}{0.000000,0.000000,0.000000}%
\pgfsetstrokecolor{currentstroke}%
\pgfsetdash{}{0pt}%
\pgfsys@defobject{currentmarker}{\pgfqpoint{0.000000in}{-0.048611in}}{\pgfqpoint{0.000000in}{0.000000in}}{%
\pgfpathmoveto{\pgfqpoint{0.000000in}{0.000000in}}%
\pgfpathlineto{\pgfqpoint{0.000000in}{-0.048611in}}%
\pgfusepath{stroke,fill}%
}%
\begin{pgfscope}%
\pgfsys@transformshift{0.993550in}{0.521603in}%
\pgfsys@useobject{currentmarker}{}%
\end{pgfscope}%
\end{pgfscope}%
\begin{pgfscope}%
\definecolor{textcolor}{rgb}{0.000000,0.000000,0.000000}%
\pgfsetstrokecolor{textcolor}%
\pgfsetfillcolor{textcolor}%
\pgftext[x=0.993550in,y=0.424381in,,top]{\color{textcolor}{\rmfamily\fontsize{10.000000}{12.000000}\selectfont\catcode`\^=\active\def^{\ifmmode\sp\else\^{}\fi}\catcode`\%=\active\def%{\%}$\mathdefault{3}$}}%
\end{pgfscope}%
\begin{pgfscope}%
\pgfsetbuttcap%
\pgfsetroundjoin%
\definecolor{currentfill}{rgb}{0.000000,0.000000,0.000000}%
\pgfsetfillcolor{currentfill}%
\pgfsetlinewidth{0.803000pt}%
\definecolor{currentstroke}{rgb}{0.000000,0.000000,0.000000}%
\pgfsetstrokecolor{currentstroke}%
\pgfsetdash{}{0pt}%
\pgfsys@defobject{currentmarker}{\pgfqpoint{0.000000in}{-0.048611in}}{\pgfqpoint{0.000000in}{0.000000in}}{%
\pgfpathmoveto{\pgfqpoint{0.000000in}{0.000000in}}%
\pgfpathlineto{\pgfqpoint{0.000000in}{-0.048611in}}%
\pgfusepath{stroke,fill}%
}%
\begin{pgfscope}%
\pgfsys@transformshift{1.572355in}{0.521603in}%
\pgfsys@useobject{currentmarker}{}%
\end{pgfscope}%
\end{pgfscope}%
\begin{pgfscope}%
\definecolor{textcolor}{rgb}{0.000000,0.000000,0.000000}%
\pgfsetstrokecolor{textcolor}%
\pgfsetfillcolor{textcolor}%
\pgftext[x=1.572355in,y=0.424381in,,top]{\color{textcolor}{\rmfamily\fontsize{10.000000}{12.000000}\selectfont\catcode`\^=\active\def^{\ifmmode\sp\else\^{}\fi}\catcode`\%=\active\def%{\%}$\mathdefault{6}$}}%
\end{pgfscope}%
\begin{pgfscope}%
\pgfsetbuttcap%
\pgfsetroundjoin%
\definecolor{currentfill}{rgb}{0.000000,0.000000,0.000000}%
\pgfsetfillcolor{currentfill}%
\pgfsetlinewidth{0.803000pt}%
\definecolor{currentstroke}{rgb}{0.000000,0.000000,0.000000}%
\pgfsetstrokecolor{currentstroke}%
\pgfsetdash{}{0pt}%
\pgfsys@defobject{currentmarker}{\pgfqpoint{0.000000in}{-0.048611in}}{\pgfqpoint{0.000000in}{0.000000in}}{%
\pgfpathmoveto{\pgfqpoint{0.000000in}{0.000000in}}%
\pgfpathlineto{\pgfqpoint{0.000000in}{-0.048611in}}%
\pgfusepath{stroke,fill}%
}%
\begin{pgfscope}%
\pgfsys@transformshift{2.151160in}{0.521603in}%
\pgfsys@useobject{currentmarker}{}%
\end{pgfscope}%
\end{pgfscope}%
\begin{pgfscope}%
\definecolor{textcolor}{rgb}{0.000000,0.000000,0.000000}%
\pgfsetstrokecolor{textcolor}%
\pgfsetfillcolor{textcolor}%
\pgftext[x=2.151160in,y=0.424381in,,top]{\color{textcolor}{\rmfamily\fontsize{10.000000}{12.000000}\selectfont\catcode`\^=\active\def^{\ifmmode\sp\else\^{}\fi}\catcode`\%=\active\def%{\%}$\mathdefault{9}$}}%
\end{pgfscope}%
\begin{pgfscope}%
\pgfsetbuttcap%
\pgfsetroundjoin%
\definecolor{currentfill}{rgb}{0.000000,0.000000,0.000000}%
\pgfsetfillcolor{currentfill}%
\pgfsetlinewidth{0.803000pt}%
\definecolor{currentstroke}{rgb}{0.000000,0.000000,0.000000}%
\pgfsetstrokecolor{currentstroke}%
\pgfsetdash{}{0pt}%
\pgfsys@defobject{currentmarker}{\pgfqpoint{0.000000in}{-0.048611in}}{\pgfqpoint{0.000000in}{0.000000in}}{%
\pgfpathmoveto{\pgfqpoint{0.000000in}{0.000000in}}%
\pgfpathlineto{\pgfqpoint{0.000000in}{-0.048611in}}%
\pgfusepath{stroke,fill}%
}%
\begin{pgfscope}%
\pgfsys@transformshift{2.729964in}{0.521603in}%
\pgfsys@useobject{currentmarker}{}%
\end{pgfscope}%
\end{pgfscope}%
\begin{pgfscope}%
\definecolor{textcolor}{rgb}{0.000000,0.000000,0.000000}%
\pgfsetstrokecolor{textcolor}%
\pgfsetfillcolor{textcolor}%
\pgftext[x=2.729964in,y=0.424381in,,top]{\color{textcolor}{\rmfamily\fontsize{10.000000}{12.000000}\selectfont\catcode`\^=\active\def^{\ifmmode\sp\else\^{}\fi}\catcode`\%=\active\def%{\%}$\mathdefault{12}$}}%
\end{pgfscope}%
\begin{pgfscope}%
\pgfsetbuttcap%
\pgfsetroundjoin%
\definecolor{currentfill}{rgb}{0.000000,0.000000,0.000000}%
\pgfsetfillcolor{currentfill}%
\pgfsetlinewidth{0.803000pt}%
\definecolor{currentstroke}{rgb}{0.000000,0.000000,0.000000}%
\pgfsetstrokecolor{currentstroke}%
\pgfsetdash{}{0pt}%
\pgfsys@defobject{currentmarker}{\pgfqpoint{0.000000in}{-0.048611in}}{\pgfqpoint{0.000000in}{0.000000in}}{%
\pgfpathmoveto{\pgfqpoint{0.000000in}{0.000000in}}%
\pgfpathlineto{\pgfqpoint{0.000000in}{-0.048611in}}%
\pgfusepath{stroke,fill}%
}%
\begin{pgfscope}%
\pgfsys@transformshift{3.308769in}{0.521603in}%
\pgfsys@useobject{currentmarker}{}%
\end{pgfscope}%
\end{pgfscope}%
\begin{pgfscope}%
\definecolor{textcolor}{rgb}{0.000000,0.000000,0.000000}%
\pgfsetstrokecolor{textcolor}%
\pgfsetfillcolor{textcolor}%
\pgftext[x=3.308769in,y=0.424381in,,top]{\color{textcolor}{\rmfamily\fontsize{10.000000}{12.000000}\selectfont\catcode`\^=\active\def^{\ifmmode\sp\else\^{}\fi}\catcode`\%=\active\def%{\%}$\mathdefault{15}$}}%
\end{pgfscope}%
\begin{pgfscope}%
\pgfsetbuttcap%
\pgfsetroundjoin%
\definecolor{currentfill}{rgb}{0.000000,0.000000,0.000000}%
\pgfsetfillcolor{currentfill}%
\pgfsetlinewidth{0.803000pt}%
\definecolor{currentstroke}{rgb}{0.000000,0.000000,0.000000}%
\pgfsetstrokecolor{currentstroke}%
\pgfsetdash{}{0pt}%
\pgfsys@defobject{currentmarker}{\pgfqpoint{0.000000in}{-0.048611in}}{\pgfqpoint{0.000000in}{0.000000in}}{%
\pgfpathmoveto{\pgfqpoint{0.000000in}{0.000000in}}%
\pgfpathlineto{\pgfqpoint{0.000000in}{-0.048611in}}%
\pgfusepath{stroke,fill}%
}%
\begin{pgfscope}%
\pgfsys@transformshift{3.887574in}{0.521603in}%
\pgfsys@useobject{currentmarker}{}%
\end{pgfscope}%
\end{pgfscope}%
\begin{pgfscope}%
\definecolor{textcolor}{rgb}{0.000000,0.000000,0.000000}%
\pgfsetstrokecolor{textcolor}%
\pgfsetfillcolor{textcolor}%
\pgftext[x=3.887574in,y=0.424381in,,top]{\color{textcolor}{\rmfamily\fontsize{10.000000}{12.000000}\selectfont\catcode`\^=\active\def^{\ifmmode\sp\else\^{}\fi}\catcode`\%=\active\def%{\%}$\mathdefault{18}$}}%
\end{pgfscope}%
\begin{pgfscope}%
\pgfsetbuttcap%
\pgfsetroundjoin%
\definecolor{currentfill}{rgb}{0.000000,0.000000,0.000000}%
\pgfsetfillcolor{currentfill}%
\pgfsetlinewidth{0.803000pt}%
\definecolor{currentstroke}{rgb}{0.000000,0.000000,0.000000}%
\pgfsetstrokecolor{currentstroke}%
\pgfsetdash{}{0pt}%
\pgfsys@defobject{currentmarker}{\pgfqpoint{0.000000in}{-0.048611in}}{\pgfqpoint{0.000000in}{0.000000in}}{%
\pgfpathmoveto{\pgfqpoint{0.000000in}{0.000000in}}%
\pgfpathlineto{\pgfqpoint{0.000000in}{-0.048611in}}%
\pgfusepath{stroke,fill}%
}%
\begin{pgfscope}%
\pgfsys@transformshift{4.466378in}{0.521603in}%
\pgfsys@useobject{currentmarker}{}%
\end{pgfscope}%
\end{pgfscope}%
\begin{pgfscope}%
\definecolor{textcolor}{rgb}{0.000000,0.000000,0.000000}%
\pgfsetstrokecolor{textcolor}%
\pgfsetfillcolor{textcolor}%
\pgftext[x=4.466378in,y=0.424381in,,top]{\color{textcolor}{\rmfamily\fontsize{10.000000}{12.000000}\selectfont\catcode`\^=\active\def^{\ifmmode\sp\else\^{}\fi}\catcode`\%=\active\def%{\%}$\mathdefault{21}$}}%
\end{pgfscope}%
\begin{pgfscope}%
\pgfsetbuttcap%
\pgfsetroundjoin%
\definecolor{currentfill}{rgb}{0.000000,0.000000,0.000000}%
\pgfsetfillcolor{currentfill}%
\pgfsetlinewidth{0.803000pt}%
\definecolor{currentstroke}{rgb}{0.000000,0.000000,0.000000}%
\pgfsetstrokecolor{currentstroke}%
\pgfsetdash{}{0pt}%
\pgfsys@defobject{currentmarker}{\pgfqpoint{0.000000in}{-0.048611in}}{\pgfqpoint{0.000000in}{0.000000in}}{%
\pgfpathmoveto{\pgfqpoint{0.000000in}{0.000000in}}%
\pgfpathlineto{\pgfqpoint{0.000000in}{-0.048611in}}%
\pgfusepath{stroke,fill}%
}%
\begin{pgfscope}%
\pgfsys@transformshift{5.045183in}{0.521603in}%
\pgfsys@useobject{currentmarker}{}%
\end{pgfscope}%
\end{pgfscope}%
\begin{pgfscope}%
\definecolor{textcolor}{rgb}{0.000000,0.000000,0.000000}%
\pgfsetstrokecolor{textcolor}%
\pgfsetfillcolor{textcolor}%
\pgftext[x=5.045183in,y=0.424381in,,top]{\color{textcolor}{\rmfamily\fontsize{10.000000}{12.000000}\selectfont\catcode`\^=\active\def^{\ifmmode\sp\else\^{}\fi}\catcode`\%=\active\def%{\%}$\mathdefault{24}$}}%
\end{pgfscope}%
\begin{pgfscope}%
\definecolor{textcolor}{rgb}{0.000000,0.000000,0.000000}%
\pgfsetstrokecolor{textcolor}%
\pgfsetfillcolor{textcolor}%
\pgftext[x=2.922899in,y=0.234413in,,top]{\color{textcolor}{\rmfamily\fontsize{10.000000}{12.000000}\selectfont\catcode`\^=\active\def^{\ifmmode\sp\else\^{}\fi}\catcode`\%=\active\def%{\%}Monochromatic classes}}%
\end{pgfscope}%
\begin{pgfscope}%
\pgfsetbuttcap%
\pgfsetroundjoin%
\definecolor{currentfill}{rgb}{0.000000,0.000000,0.000000}%
\pgfsetfillcolor{currentfill}%
\pgfsetlinewidth{0.803000pt}%
\definecolor{currentstroke}{rgb}{0.000000,0.000000,0.000000}%
\pgfsetstrokecolor{currentstroke}%
\pgfsetdash{}{0pt}%
\pgfsys@defobject{currentmarker}{\pgfqpoint{-0.048611in}{0.000000in}}{\pgfqpoint{-0.000000in}{0.000000in}}{%
\pgfpathmoveto{\pgfqpoint{-0.000000in}{0.000000in}}%
\pgfpathlineto{\pgfqpoint{-0.048611in}{0.000000in}}%
\pgfusepath{stroke,fill}%
}%
\begin{pgfscope}%
\pgfsys@transformshift{0.588387in}{0.876933in}%
\pgfsys@useobject{currentmarker}{}%
\end{pgfscope}%
\end{pgfscope}%
\begin{pgfscope}%
\definecolor{textcolor}{rgb}{0.000000,0.000000,0.000000}%
\pgfsetstrokecolor{textcolor}%
\pgfsetfillcolor{textcolor}%
\pgftext[x=0.289968in, y=0.824172in, left, base]{\color{textcolor}{\rmfamily\fontsize{10.000000}{12.000000}\selectfont\catcode`\^=\active\def^{\ifmmode\sp\else\^{}\fi}\catcode`\%=\active\def%{\%}$\mathdefault{10^{1}}$}}%
\end{pgfscope}%
\begin{pgfscope}%
\pgfsetbuttcap%
\pgfsetroundjoin%
\definecolor{currentfill}{rgb}{0.000000,0.000000,0.000000}%
\pgfsetfillcolor{currentfill}%
\pgfsetlinewidth{0.803000pt}%
\definecolor{currentstroke}{rgb}{0.000000,0.000000,0.000000}%
\pgfsetstrokecolor{currentstroke}%
\pgfsetdash{}{0pt}%
\pgfsys@defobject{currentmarker}{\pgfqpoint{-0.048611in}{0.000000in}}{\pgfqpoint{-0.000000in}{0.000000in}}{%
\pgfpathmoveto{\pgfqpoint{-0.000000in}{0.000000in}}%
\pgfpathlineto{\pgfqpoint{-0.048611in}{0.000000in}}%
\pgfusepath{stroke,fill}%
}%
\begin{pgfscope}%
\pgfsys@transformshift{0.588387in}{1.404840in}%
\pgfsys@useobject{currentmarker}{}%
\end{pgfscope}%
\end{pgfscope}%
\begin{pgfscope}%
\definecolor{textcolor}{rgb}{0.000000,0.000000,0.000000}%
\pgfsetstrokecolor{textcolor}%
\pgfsetfillcolor{textcolor}%
\pgftext[x=0.289968in, y=1.352079in, left, base]{\color{textcolor}{\rmfamily\fontsize{10.000000}{12.000000}\selectfont\catcode`\^=\active\def^{\ifmmode\sp\else\^{}\fi}\catcode`\%=\active\def%{\%}$\mathdefault{10^{3}}$}}%
\end{pgfscope}%
\begin{pgfscope}%
\pgfsetbuttcap%
\pgfsetroundjoin%
\definecolor{currentfill}{rgb}{0.000000,0.000000,0.000000}%
\pgfsetfillcolor{currentfill}%
\pgfsetlinewidth{0.803000pt}%
\definecolor{currentstroke}{rgb}{0.000000,0.000000,0.000000}%
\pgfsetstrokecolor{currentstroke}%
\pgfsetdash{}{0pt}%
\pgfsys@defobject{currentmarker}{\pgfqpoint{-0.048611in}{0.000000in}}{\pgfqpoint{-0.000000in}{0.000000in}}{%
\pgfpathmoveto{\pgfqpoint{-0.000000in}{0.000000in}}%
\pgfpathlineto{\pgfqpoint{-0.048611in}{0.000000in}}%
\pgfusepath{stroke,fill}%
}%
\begin{pgfscope}%
\pgfsys@transformshift{0.588387in}{1.932747in}%
\pgfsys@useobject{currentmarker}{}%
\end{pgfscope}%
\end{pgfscope}%
\begin{pgfscope}%
\definecolor{textcolor}{rgb}{0.000000,0.000000,0.000000}%
\pgfsetstrokecolor{textcolor}%
\pgfsetfillcolor{textcolor}%
\pgftext[x=0.289968in, y=1.879986in, left, base]{\color{textcolor}{\rmfamily\fontsize{10.000000}{12.000000}\selectfont\catcode`\^=\active\def^{\ifmmode\sp\else\^{}\fi}\catcode`\%=\active\def%{\%}$\mathdefault{10^{5}}$}}%
\end{pgfscope}%
\begin{pgfscope}%
\pgfsetbuttcap%
\pgfsetroundjoin%
\definecolor{currentfill}{rgb}{0.000000,0.000000,0.000000}%
\pgfsetfillcolor{currentfill}%
\pgfsetlinewidth{0.803000pt}%
\definecolor{currentstroke}{rgb}{0.000000,0.000000,0.000000}%
\pgfsetstrokecolor{currentstroke}%
\pgfsetdash{}{0pt}%
\pgfsys@defobject{currentmarker}{\pgfqpoint{-0.048611in}{0.000000in}}{\pgfqpoint{-0.000000in}{0.000000in}}{%
\pgfpathmoveto{\pgfqpoint{-0.000000in}{0.000000in}}%
\pgfpathlineto{\pgfqpoint{-0.048611in}{0.000000in}}%
\pgfusepath{stroke,fill}%
}%
\begin{pgfscope}%
\pgfsys@transformshift{0.588387in}{2.460654in}%
\pgfsys@useobject{currentmarker}{}%
\end{pgfscope}%
\end{pgfscope}%
\begin{pgfscope}%
\definecolor{textcolor}{rgb}{0.000000,0.000000,0.000000}%
\pgfsetstrokecolor{textcolor}%
\pgfsetfillcolor{textcolor}%
\pgftext[x=0.289968in, y=2.407892in, left, base]{\color{textcolor}{\rmfamily\fontsize{10.000000}{12.000000}\selectfont\catcode`\^=\active\def^{\ifmmode\sp\else\^{}\fi}\catcode`\%=\active\def%{\%}$\mathdefault{10^{7}}$}}%
\end{pgfscope}%
\begin{pgfscope}%
\definecolor{textcolor}{rgb}{0.000000,0.000000,0.000000}%
\pgfsetstrokecolor{textcolor}%
\pgfsetfillcolor{textcolor}%
\pgftext[x=0.234413in,y=1.526746in,,bottom,rotate=90.000000]{\color{textcolor}{\rmfamily\fontsize{10.000000}{12.000000}\selectfont\catcode`\^=\active\def^{\ifmmode\sp\else\^{}\fi}\catcode`\%=\active\def%{\%}Checks [call]}}%
\end{pgfscope}%
\begin{pgfscope}%
\pgfpathrectangle{\pgfqpoint{0.588387in}{0.521603in}}{\pgfqpoint{4.669024in}{2.010285in}}%
\pgfusepath{clip}%
\pgfsetrectcap%
\pgfsetroundjoin%
\pgfsetlinewidth{1.505625pt}%
\pgfsetstrokecolor{currentstroke1}%
\pgfsetdash{}{0pt}%
\pgfpathmoveto{\pgfqpoint{0.800616in}{0.692438in}}%
\pgfpathlineto{\pgfqpoint{0.993550in}{0.771896in}}%
\pgfpathlineto{\pgfqpoint{1.186485in}{0.851354in}}%
\pgfpathlineto{\pgfqpoint{1.379420in}{0.930812in}}%
\pgfpathlineto{\pgfqpoint{1.572355in}{1.010269in}}%
\pgfpathlineto{\pgfqpoint{1.765290in}{1.089727in}}%
\pgfpathlineto{\pgfqpoint{1.958225in}{1.146115in}}%
\pgfpathlineto{\pgfqpoint{2.151160in}{1.191398in}}%
\pgfpathlineto{\pgfqpoint{2.344095in}{1.253427in}}%
\pgfpathlineto{\pgfqpoint{2.537029in}{1.298966in}}%
\pgfpathlineto{\pgfqpoint{2.729964in}{1.345900in}}%
\pgfpathlineto{\pgfqpoint{2.922899in}{1.396881in}}%
\pgfpathlineto{\pgfqpoint{3.115834in}{1.403146in}}%
\pgfpathlineto{\pgfqpoint{3.308769in}{1.514989in}}%
\pgfpathlineto{\pgfqpoint{3.501704in}{1.487782in}}%
\pgfpathlineto{\pgfqpoint{3.694639in}{1.574316in}}%
\pgfpathlineto{\pgfqpoint{3.887574in}{1.564214in}}%
\pgfpathlineto{\pgfqpoint{4.080508in}{1.592207in}}%
\pgfpathlineto{\pgfqpoint{4.273443in}{1.589336in}}%
\pgfpathlineto{\pgfqpoint{4.466378in}{1.627382in}}%
\pgfpathlineto{\pgfqpoint{4.659313in}{1.633780in}}%
\pgfpathlineto{\pgfqpoint{5.045183in}{1.669160in}}%
\pgfusepath{stroke}%
\end{pgfscope}%
\begin{pgfscope}%
\pgfpathrectangle{\pgfqpoint{0.588387in}{0.521603in}}{\pgfqpoint{4.669024in}{2.010285in}}%
\pgfusepath{clip}%
\pgfsetrectcap%
\pgfsetroundjoin%
\pgfsetlinewidth{1.505625pt}%
\pgfsetstrokecolor{currentstroke2}%
\pgfsetdash{}{0pt}%
\pgfpathmoveto{\pgfqpoint{0.800616in}{0.692438in}}%
\pgfpathlineto{\pgfqpoint{0.993550in}{0.771896in}}%
\pgfpathlineto{\pgfqpoint{1.186485in}{0.851354in}}%
\pgfpathlineto{\pgfqpoint{1.379420in}{0.930812in}}%
\pgfpathlineto{\pgfqpoint{1.572355in}{1.010269in}}%
\pgfpathlineto{\pgfqpoint{1.765290in}{1.089727in}}%
\pgfpathlineto{\pgfqpoint{1.958225in}{1.146115in}}%
\pgfpathlineto{\pgfqpoint{2.151160in}{1.191398in}}%
\pgfpathlineto{\pgfqpoint{2.344095in}{1.253427in}}%
\pgfpathlineto{\pgfqpoint{2.537029in}{1.298966in}}%
\pgfpathlineto{\pgfqpoint{2.729964in}{1.338701in}}%
\pgfpathlineto{\pgfqpoint{2.922899in}{1.355864in}}%
\pgfpathlineto{\pgfqpoint{3.115834in}{1.419323in}}%
\pgfpathlineto{\pgfqpoint{3.308769in}{1.454634in}}%
\pgfpathlineto{\pgfqpoint{3.501704in}{1.502715in}}%
\pgfpathlineto{\pgfqpoint{3.694639in}{1.556297in}}%
\pgfpathlineto{\pgfqpoint{3.887574in}{1.564214in}}%
\pgfpathlineto{\pgfqpoint{4.080508in}{1.621972in}}%
\pgfpathlineto{\pgfqpoint{4.273443in}{1.593655in}}%
\pgfpathlineto{\pgfqpoint{4.466378in}{1.628950in}}%
\pgfpathlineto{\pgfqpoint{4.659313in}{1.613995in}}%
\pgfusepath{stroke}%
\end{pgfscope}%
\begin{pgfscope}%
\pgfpathrectangle{\pgfqpoint{0.588387in}{0.521603in}}{\pgfqpoint{4.669024in}{2.010285in}}%
\pgfusepath{clip}%
\pgfsetrectcap%
\pgfsetroundjoin%
\pgfsetlinewidth{1.505625pt}%
\pgfsetstrokecolor{currentstroke3}%
\pgfsetdash{}{0pt}%
\pgfpathmoveto{\pgfqpoint{0.800616in}{0.612980in}}%
\pgfpathlineto{\pgfqpoint{0.993550in}{0.738918in}}%
\pgfpathlineto{\pgfqpoint{1.186485in}{0.836046in}}%
\pgfpathlineto{\pgfqpoint{1.379420in}{0.923413in}}%
\pgfpathlineto{\pgfqpoint{1.572355in}{1.006630in}}%
\pgfpathlineto{\pgfqpoint{1.765290in}{1.087922in}}%
\pgfpathlineto{\pgfqpoint{1.958225in}{1.168286in}}%
\pgfpathlineto{\pgfqpoint{2.151160in}{1.248194in}}%
\pgfpathlineto{\pgfqpoint{2.344095in}{1.327877in}}%
\pgfpathlineto{\pgfqpoint{2.537029in}{1.407447in}}%
\pgfpathlineto{\pgfqpoint{2.729964in}{1.486961in}}%
\pgfpathlineto{\pgfqpoint{2.922899in}{1.566447in}}%
\pgfpathlineto{\pgfqpoint{3.115834in}{1.645919in}}%
\pgfpathlineto{\pgfqpoint{3.308769in}{1.725384in}}%
\pgfpathlineto{\pgfqpoint{3.501704in}{1.804845in}}%
\pgfpathlineto{\pgfqpoint{3.694639in}{1.884305in}}%
\pgfpathlineto{\pgfqpoint{3.887574in}{1.963763in}}%
\pgfpathlineto{\pgfqpoint{4.080508in}{2.043222in}}%
\pgfpathlineto{\pgfqpoint{4.273443in}{2.122680in}}%
\pgfpathlineto{\pgfqpoint{4.466378in}{2.202138in}}%
\pgfpathlineto{\pgfqpoint{4.659313in}{2.281596in}}%
\pgfpathlineto{\pgfqpoint{5.045183in}{2.440512in}}%
\pgfusepath{stroke}%
\end{pgfscope}%
\begin{pgfscope}%
\pgfpathrectangle{\pgfqpoint{0.588387in}{0.521603in}}{\pgfqpoint{4.669024in}{2.010285in}}%
\pgfusepath{clip}%
\pgfsetrectcap%
\pgfsetroundjoin%
\pgfsetlinewidth{1.505625pt}%
\pgfsetstrokecolor{currentstroke4}%
\pgfsetdash{}{0pt}%
\pgfpathmoveto{\pgfqpoint{0.800616in}{0.692438in}}%
\pgfpathlineto{\pgfqpoint{0.993550in}{0.771896in}}%
\pgfpathlineto{\pgfqpoint{1.186485in}{0.851354in}}%
\pgfpathlineto{\pgfqpoint{1.379420in}{0.930812in}}%
\pgfpathlineto{\pgfqpoint{1.572355in}{1.010269in}}%
\pgfpathlineto{\pgfqpoint{1.765290in}{1.089727in}}%
\pgfpathlineto{\pgfqpoint{1.958225in}{1.148596in}}%
\pgfpathlineto{\pgfqpoint{2.151160in}{1.225672in}}%
\pgfpathlineto{\pgfqpoint{2.344095in}{1.296668in}}%
\pgfpathlineto{\pgfqpoint{2.537029in}{1.368449in}}%
\pgfpathlineto{\pgfqpoint{2.729964in}{1.343508in}}%
\pgfpathlineto{\pgfqpoint{2.922899in}{1.411153in}}%
\pgfpathlineto{\pgfqpoint{3.115834in}{1.453140in}}%
\pgfpathlineto{\pgfqpoint{3.308769in}{1.503213in}}%
\pgfpathlineto{\pgfqpoint{3.501704in}{1.508375in}}%
\pgfpathlineto{\pgfqpoint{3.694639in}{1.600387in}}%
\pgfpathlineto{\pgfqpoint{3.887574in}{1.629726in}}%
\pgfpathlineto{\pgfqpoint{4.080508in}{1.633394in}}%
\pgfpathlineto{\pgfqpoint{4.273443in}{1.681717in}}%
\pgfpathlineto{\pgfqpoint{4.466378in}{1.677764in}}%
\pgfpathlineto{\pgfqpoint{4.659313in}{1.694080in}}%
\pgfpathlineto{\pgfqpoint{5.045183in}{1.573002in}}%
\pgfusepath{stroke}%
\end{pgfscope}%
\begin{pgfscope}%
\pgfpathrectangle{\pgfqpoint{0.588387in}{0.521603in}}{\pgfqpoint{4.669024in}{2.010285in}}%
\pgfusepath{clip}%
\pgfsetrectcap%
\pgfsetroundjoin%
\pgfsetlinewidth{1.505625pt}%
\pgfsetstrokecolor{currentstroke5}%
\pgfsetdash{}{0pt}%
\pgfpathmoveto{\pgfqpoint{0.800616in}{0.692438in}}%
\pgfpathlineto{\pgfqpoint{0.993550in}{0.771896in}}%
\pgfpathlineto{\pgfqpoint{1.186485in}{0.851354in}}%
\pgfpathlineto{\pgfqpoint{1.379420in}{0.930812in}}%
\pgfpathlineto{\pgfqpoint{1.572355in}{1.010269in}}%
\pgfpathlineto{\pgfqpoint{1.765290in}{1.089727in}}%
\pgfpathlineto{\pgfqpoint{1.958225in}{1.149075in}}%
\pgfpathlineto{\pgfqpoint{2.151160in}{1.224209in}}%
\pgfpathlineto{\pgfqpoint{2.344095in}{1.295803in}}%
\pgfpathlineto{\pgfqpoint{2.537029in}{1.368449in}}%
\pgfpathlineto{\pgfqpoint{2.729964in}{1.330242in}}%
\pgfpathlineto{\pgfqpoint{2.922899in}{1.406242in}}%
\pgfpathlineto{\pgfqpoint{3.115834in}{1.447592in}}%
\pgfpathlineto{\pgfqpoint{3.308769in}{1.506528in}}%
\pgfpathlineto{\pgfqpoint{3.501704in}{1.502938in}}%
\pgfpathlineto{\pgfqpoint{3.694639in}{1.598497in}}%
\pgfpathlineto{\pgfqpoint{3.887574in}{1.564897in}}%
\pgfpathlineto{\pgfqpoint{4.080508in}{1.573235in}}%
\pgfpathlineto{\pgfqpoint{4.273443in}{1.663171in}}%
\pgfpathlineto{\pgfqpoint{4.466378in}{1.667503in}}%
\pgfpathlineto{\pgfqpoint{4.659313in}{1.633842in}}%
\pgfpathlineto{\pgfqpoint{5.045183in}{1.735002in}}%
\pgfusepath{stroke}%
\end{pgfscope}%
\begin{pgfscope}%
\pgfpathrectangle{\pgfqpoint{0.588387in}{0.521603in}}{\pgfqpoint{4.669024in}{2.010285in}}%
\pgfusepath{clip}%
\pgfsetrectcap%
\pgfsetroundjoin%
\pgfsetlinewidth{1.505625pt}%
\pgfsetstrokecolor{currentstroke6}%
\pgfsetdash{}{0pt}%
\pgfpathmoveto{\pgfqpoint{0.800616in}{0.692438in}}%
\pgfpathlineto{\pgfqpoint{0.993550in}{0.771896in}}%
\pgfpathlineto{\pgfqpoint{1.186485in}{0.851354in}}%
\pgfpathlineto{\pgfqpoint{1.379420in}{0.930812in}}%
\pgfpathlineto{\pgfqpoint{1.572355in}{1.010269in}}%
\pgfpathlineto{\pgfqpoint{1.765290in}{1.089727in}}%
\pgfpathlineto{\pgfqpoint{1.958225in}{1.166027in}}%
\pgfpathlineto{\pgfqpoint{2.151160in}{1.236224in}}%
\pgfpathlineto{\pgfqpoint{2.344095in}{1.311742in}}%
\pgfpathlineto{\pgfqpoint{2.537029in}{1.383861in}}%
\pgfpathlineto{\pgfqpoint{2.729964in}{1.385126in}}%
\pgfpathlineto{\pgfqpoint{2.922899in}{1.440140in}}%
\pgfpathlineto{\pgfqpoint{3.115834in}{1.500087in}}%
\pgfpathlineto{\pgfqpoint{3.308769in}{1.582096in}}%
\pgfpathlineto{\pgfqpoint{3.501704in}{1.564962in}}%
\pgfpathlineto{\pgfqpoint{3.694639in}{1.700810in}}%
\pgfpathlineto{\pgfqpoint{3.887574in}{1.623364in}}%
\pgfpathlineto{\pgfqpoint{4.080508in}{1.650788in}}%
\pgfpathlineto{\pgfqpoint{4.273443in}{1.761168in}}%
\pgfpathlineto{\pgfqpoint{4.466378in}{1.779206in}}%
\pgfpathlineto{\pgfqpoint{4.659313in}{1.646769in}}%
\pgfpathlineto{\pgfqpoint{5.045183in}{1.693969in}}%
\pgfusepath{stroke}%
\end{pgfscope}%
\begin{pgfscope}%
\pgfpathrectangle{\pgfqpoint{0.588387in}{0.521603in}}{\pgfqpoint{4.669024in}{2.010285in}}%
\pgfusepath{clip}%
\pgfsetrectcap%
\pgfsetroundjoin%
\pgfsetlinewidth{1.505625pt}%
\pgfsetstrokecolor{currentstroke7}%
\pgfsetdash{}{0pt}%
\pgfpathmoveto{\pgfqpoint{0.800616in}{0.692438in}}%
\pgfpathlineto{\pgfqpoint{0.993550in}{0.771896in}}%
\pgfpathlineto{\pgfqpoint{1.186485in}{0.851354in}}%
\pgfpathlineto{\pgfqpoint{1.379420in}{0.930812in}}%
\pgfpathlineto{\pgfqpoint{1.572355in}{1.010269in}}%
\pgfpathlineto{\pgfqpoint{1.765290in}{1.089727in}}%
\pgfpathlineto{\pgfqpoint{1.958225in}{1.166027in}}%
\pgfpathlineto{\pgfqpoint{2.151160in}{1.235426in}}%
\pgfpathlineto{\pgfqpoint{2.344095in}{1.311322in}}%
\pgfpathlineto{\pgfqpoint{2.537029in}{1.385052in}}%
\pgfpathlineto{\pgfqpoint{2.729964in}{1.378729in}}%
\pgfpathlineto{\pgfqpoint{2.922899in}{1.439046in}}%
\pgfpathlineto{\pgfqpoint{3.115834in}{1.504014in}}%
\pgfpathlineto{\pgfqpoint{3.308769in}{1.575512in}}%
\pgfpathlineto{\pgfqpoint{3.501704in}{1.558152in}}%
\pgfpathlineto{\pgfqpoint{3.694639in}{1.697074in}}%
\pgfpathlineto{\pgfqpoint{3.887574in}{1.597306in}}%
\pgfpathlineto{\pgfqpoint{4.080508in}{1.650228in}}%
\pgfpathlineto{\pgfqpoint{4.273443in}{1.753546in}}%
\pgfpathlineto{\pgfqpoint{4.466378in}{1.768038in}}%
\pgfpathlineto{\pgfqpoint{4.659313in}{1.679092in}}%
\pgfpathlineto{\pgfqpoint{5.045183in}{1.792914in}}%
\pgfusepath{stroke}%
\end{pgfscope}%
\begin{pgfscope}%
\pgfsetrectcap%
\pgfsetmiterjoin%
\pgfsetlinewidth{0.803000pt}%
\definecolor{currentstroke}{rgb}{0.000000,0.000000,0.000000}%
\pgfsetstrokecolor{currentstroke}%
\pgfsetdash{}{0pt}%
\pgfpathmoveto{\pgfqpoint{0.588387in}{0.521603in}}%
\pgfpathlineto{\pgfqpoint{0.588387in}{2.531888in}}%
\pgfusepath{stroke}%
\end{pgfscope}%
\begin{pgfscope}%
\pgfsetrectcap%
\pgfsetmiterjoin%
\pgfsetlinewidth{0.803000pt}%
\definecolor{currentstroke}{rgb}{0.000000,0.000000,0.000000}%
\pgfsetstrokecolor{currentstroke}%
\pgfsetdash{}{0pt}%
\pgfpathmoveto{\pgfqpoint{5.257411in}{0.521603in}}%
\pgfpathlineto{\pgfqpoint{5.257411in}{2.531888in}}%
\pgfusepath{stroke}%
\end{pgfscope}%
\begin{pgfscope}%
\pgfsetrectcap%
\pgfsetmiterjoin%
\pgfsetlinewidth{0.803000pt}%
\definecolor{currentstroke}{rgb}{0.000000,0.000000,0.000000}%
\pgfsetstrokecolor{currentstroke}%
\pgfsetdash{}{0pt}%
\pgfpathmoveto{\pgfqpoint{0.588387in}{0.521603in}}%
\pgfpathlineto{\pgfqpoint{5.257411in}{0.521603in}}%
\pgfusepath{stroke}%
\end{pgfscope}%
\begin{pgfscope}%
\pgfsetrectcap%
\pgfsetmiterjoin%
\pgfsetlinewidth{0.803000pt}%
\definecolor{currentstroke}{rgb}{0.000000,0.000000,0.000000}%
\pgfsetstrokecolor{currentstroke}%
\pgfsetdash{}{0pt}%
\pgfpathmoveto{\pgfqpoint{0.588387in}{2.531888in}}%
\pgfpathlineto{\pgfqpoint{5.257411in}{2.531888in}}%
\pgfusepath{stroke}%
\end{pgfscope}%
\begin{pgfscope}%
\definecolor{textcolor}{rgb}{0.000000,0.000000,0.000000}%
\pgfsetstrokecolor{textcolor}%
\pgfsetfillcolor{textcolor}%
\pgftext[x=2.922899in,y=2.615222in,,base]{\color{textcolor}{\rmfamily\fontsize{12.000000}{14.400000}\selectfont\catcode`\^=\active\def^{\ifmmode\sp\else\^{}\fi}\catcode`\%=\active\def%{\%}Mean}}%
\end{pgfscope}%
\begin{pgfscope}%
\pgfsetbuttcap%
\pgfsetmiterjoin%
\definecolor{currentfill}{rgb}{1.000000,1.000000,1.000000}%
\pgfsetfillcolor{currentfill}%
\pgfsetfillopacity{0.800000}%
\pgfsetlinewidth{1.003750pt}%
\definecolor{currentstroke}{rgb}{0.800000,0.800000,0.800000}%
\pgfsetstrokecolor{currentstroke}%
\pgfsetstrokeopacity{0.800000}%
\pgfsetdash{}{0pt}%
\pgfpathmoveto{\pgfqpoint{5.344911in}{1.133672in}}%
\pgfpathlineto{\pgfqpoint{8.259376in}{1.133672in}}%
\pgfpathquadraticcurveto{\pgfqpoint{8.284376in}{1.133672in}}{\pgfqpoint{8.284376in}{1.158672in}}%
\pgfpathlineto{\pgfqpoint{8.284376in}{2.444388in}}%
\pgfpathquadraticcurveto{\pgfqpoint{8.284376in}{2.469388in}}{\pgfqpoint{8.259376in}{2.469388in}}%
\pgfpathlineto{\pgfqpoint{5.344911in}{2.469388in}}%
\pgfpathquadraticcurveto{\pgfqpoint{5.319911in}{2.469388in}}{\pgfqpoint{5.319911in}{2.444388in}}%
\pgfpathlineto{\pgfqpoint{5.319911in}{1.158672in}}%
\pgfpathquadraticcurveto{\pgfqpoint{5.319911in}{1.133672in}}{\pgfqpoint{5.344911in}{1.133672in}}%
\pgfpathlineto{\pgfqpoint{5.344911in}{1.133672in}}%
\pgfpathclose%
\pgfusepath{stroke,fill}%
\end{pgfscope}%
\begin{pgfscope}%
\pgfsetrectcap%
\pgfsetroundjoin%
\pgfsetlinewidth{1.505625pt}%
\pgfsetstrokecolor{currentstroke3}%
\pgfsetdash{}{0pt}%
\pgfpathmoveto{\pgfqpoint{5.369911in}{2.368168in}}%
\pgfpathlineto{\pgfqpoint{5.494911in}{2.368168in}}%
\pgfpathlineto{\pgfqpoint{5.619911in}{2.368168in}}%
\pgfusepath{stroke}%
\end{pgfscope}%
\begin{pgfscope}%
\definecolor{textcolor}{rgb}{0.000000,0.000000,0.000000}%
\pgfsetstrokecolor{textcolor}%
\pgfsetfillcolor{textcolor}%
\pgftext[x=5.719911in,y=2.324418in,left,base]{\color{textcolor}{\rmfamily\fontsize{9.000000}{10.800000}\selectfont\catcode`\^=\active\def^{\ifmmode\sp\else\^{}\fi}\catcode`\%=\active\def%{\%}\NaiveCycles{}}}%
\end{pgfscope}%
\begin{pgfscope}%
\pgfsetrectcap%
\pgfsetroundjoin%
\pgfsetlinewidth{1.505625pt}%
\pgfsetstrokecolor{currentstroke1}%
\pgfsetdash{}{0pt}%
\pgfpathmoveto{\pgfqpoint{5.369911in}{2.184696in}}%
\pgfpathlineto{\pgfqpoint{5.494911in}{2.184696in}}%
\pgfpathlineto{\pgfqpoint{5.619911in}{2.184696in}}%
\pgfusepath{stroke}%
\end{pgfscope}%
\begin{pgfscope}%
\definecolor{textcolor}{rgb}{0.000000,0.000000,0.000000}%
\pgfsetstrokecolor{textcolor}%
\pgfsetfillcolor{textcolor}%
\pgftext[x=5.719911in,y=2.140946in,left,base]{\color{textcolor}{\rmfamily\fontsize{9.000000}{10.800000}\selectfont\catcode`\^=\active\def^{\ifmmode\sp\else\^{}\fi}\catcode`\%=\active\def%{\%}\CyclesMatchChunks{} \& \MergeLinear{}}}%
\end{pgfscope}%
\begin{pgfscope}%
\pgfsetrectcap%
\pgfsetroundjoin%
\pgfsetlinewidth{1.505625pt}%
\pgfsetstrokecolor{currentstroke2}%
\pgfsetdash{}{0pt}%
\pgfpathmoveto{\pgfqpoint{5.369911in}{1.997746in}}%
\pgfpathlineto{\pgfqpoint{5.494911in}{1.997746in}}%
\pgfpathlineto{\pgfqpoint{5.619911in}{1.997746in}}%
\pgfusepath{stroke}%
\end{pgfscope}%
\begin{pgfscope}%
\definecolor{textcolor}{rgb}{0.000000,0.000000,0.000000}%
\pgfsetstrokecolor{textcolor}%
\pgfsetfillcolor{textcolor}%
\pgftext[x=5.719911in,y=1.953996in,left,base]{\color{textcolor}{\rmfamily\fontsize{9.000000}{10.800000}\selectfont\catcode`\^=\active\def^{\ifmmode\sp\else\^{}\fi}\catcode`\%=\active\def%{\%}\CyclesMatchChunks{} \& \SharedVertices{}}}%
\end{pgfscope}%
\begin{pgfscope}%
\pgfsetrectcap%
\pgfsetroundjoin%
\pgfsetlinewidth{1.505625pt}%
\pgfsetstrokecolor{currentstroke4}%
\pgfsetdash{}{0pt}%
\pgfpathmoveto{\pgfqpoint{5.369911in}{1.810795in}}%
\pgfpathlineto{\pgfqpoint{5.494911in}{1.810795in}}%
\pgfpathlineto{\pgfqpoint{5.619911in}{1.810795in}}%
\pgfusepath{stroke}%
\end{pgfscope}%
\begin{pgfscope}%
\definecolor{textcolor}{rgb}{0.000000,0.000000,0.000000}%
\pgfsetstrokecolor{textcolor}%
\pgfsetfillcolor{textcolor}%
\pgftext[x=5.719911in,y=1.767045in,left,base]{\color{textcolor}{\rmfamily\fontsize{9.000000}{10.800000}\selectfont\catcode`\^=\active\def^{\ifmmode\sp\else\^{}\fi}\catcode`\%=\active\def%{\%}\Neighbors{} \& \MergeLinear{}}}%
\end{pgfscope}%
\begin{pgfscope}%
\pgfsetrectcap%
\pgfsetroundjoin%
\pgfsetlinewidth{1.505625pt}%
\pgfsetstrokecolor{currentstroke5}%
\pgfsetdash{}{0pt}%
\pgfpathmoveto{\pgfqpoint{5.369911in}{1.627324in}}%
\pgfpathlineto{\pgfqpoint{5.494911in}{1.627324in}}%
\pgfpathlineto{\pgfqpoint{5.619911in}{1.627324in}}%
\pgfusepath{stroke}%
\end{pgfscope}%
\begin{pgfscope}%
\definecolor{textcolor}{rgb}{0.000000,0.000000,0.000000}%
\pgfsetstrokecolor{textcolor}%
\pgfsetfillcolor{textcolor}%
\pgftext[x=5.719911in,y=1.583574in,left,base]{\color{textcolor}{\rmfamily\fontsize{9.000000}{10.800000}\selectfont\catcode`\^=\active\def^{\ifmmode\sp\else\^{}\fi}\catcode`\%=\active\def%{\%}\Neighbors{} \& \SharedVertices{}}}%
\end{pgfscope}%
\begin{pgfscope}%
\pgfsetrectcap%
\pgfsetroundjoin%
\pgfsetlinewidth{1.505625pt}%
\pgfsetstrokecolor{currentstroke6}%
\pgfsetdash{}{0pt}%
\pgfpathmoveto{\pgfqpoint{5.369911in}{1.440373in}}%
\pgfpathlineto{\pgfqpoint{5.494911in}{1.440373in}}%
\pgfpathlineto{\pgfqpoint{5.619911in}{1.440373in}}%
\pgfusepath{stroke}%
\end{pgfscope}%
\begin{pgfscope}%
\definecolor{textcolor}{rgb}{0.000000,0.000000,0.000000}%
\pgfsetstrokecolor{textcolor}%
\pgfsetfillcolor{textcolor}%
\pgftext[x=5.719911in,y=1.396623in,left,base]{\color{textcolor}{\rmfamily\fontsize{9.000000}{10.800000}\selectfont\catcode`\^=\active\def^{\ifmmode\sp\else\^{}\fi}\catcode`\%=\active\def%{\%}\None{} \& \MergeLinear{}}}%
\end{pgfscope}%
\begin{pgfscope}%
\pgfsetrectcap%
\pgfsetroundjoin%
\pgfsetlinewidth{1.505625pt}%
\pgfsetstrokecolor{currentstroke7}%
\pgfsetdash{}{0pt}%
\pgfpathmoveto{\pgfqpoint{5.369911in}{1.256902in}}%
\pgfpathlineto{\pgfqpoint{5.494911in}{1.256902in}}%
\pgfpathlineto{\pgfqpoint{5.619911in}{1.256902in}}%
\pgfusepath{stroke}%
\end{pgfscope}%
\begin{pgfscope}%
\definecolor{textcolor}{rgb}{0.000000,0.000000,0.000000}%
\pgfsetstrokecolor{textcolor}%
\pgfsetfillcolor{textcolor}%
\pgftext[x=5.719911in,y=1.213152in,left,base]{\color{textcolor}{\rmfamily\fontsize{9.000000}{10.800000}\selectfont\catcode`\^=\active\def^{\ifmmode\sp\else\^{}\fi}\catcode`\%=\active\def%{\%}\None{} \& \SharedVertices{}}}%
\end{pgfscope}%
\end{pgfpicture}%
\makeatother%
\endgroup%
}
	\caption[Checks performed for globally rigid graphs (all)]{
		The number of checks performed to find all NAC-colorings for globally rigid graphs.}%
	\label{fig:graph_globally_rigid_all_checks}
\end{figure}%



\subsubsection*{Summary}
\todo[inline]{Je tohle vůbec třeba? Plus je to takové divné, když se globally chovají jinak.}

To summarize this section, we mostly tested graphs having many NAC-colorings
or trivially having none.
The only exception were some globally rigid graphs.
%
Overall, if only a single NAC-coloring is requested
for these graph classes, you can notice that the complexity
is not growing fast neither for the \NaiveCycles{} nor for \Subgraphs{}.

From figures above, we can see that both algorithms should scale well for larger graphs.
For finding a single NAC-coloring, \Subgraphs{} are outperformed or matched by \NaiveCycles{}.
For listing all NAC-colorings, the \NaiveCycles{} algorithm
is not suitable even for small graphs.

It can be also seen that \MergeLinear{} is the most reliable one
while the \SharedVertices{} sometimes performs slightly better,
but sometimes behave non-deterministically, mostly for easy instances.
%
Splitting strategies \None{}, \Neighbors{} and \NeighborsDegree{}
performs similarly.


\subsection{Performance on graphs with no NAC-colorings}

In the previous section, the \Subgraphs{} algorithm
often performed worse considering runtime.
As explained, it is caused by the graphs being too simple
--- having plenty of NAC-colorings.
%
For many NP-complete problems, studied instances are often
those where there are only few or no solutions.
In this section, we focus on graphs with no NAC-colorings.

We searched for random graphs where \( |E| \ge 2|V(G)| - 2 \) that have
multiple monochromatic classes, but no NAC-coloring
\todo{Uncoment footnote}
% \footnote{
{We could have also used the same formula as for globally rigid graphs.}.
%
As this graph generation was slow and unsuccessful, we searched only for
graphs with more than \( 2\sqrt{|V(G)|} \) \trcon{} components.
%
This once again shows how effective monochromatic classes are
in comparison with \trcon{} components.
We generated and tested five thousand of such graphs from 40 to 130 vertices in size.
Just one of them had more than one monochromatic class
\todo{Footnote}
{We also found around fifteen globally rigid graphs with this property.}.
%
The following benchmarks are run with monochromatic classes disabled.

For these graphs, \NaiveCycles{} algorithm needs to traverse all \( 2^{t-1} \)
where \( t \) is the number of \trcon{} components. It can be clearly seen that
this is not suitable for graphs as large as we use in this benchmark,
therefore, they are not present as they did not finish in reasonable time.
It can be seen from \Cref{fig:graph_no_nac_coloring_first_runtime}
that \SharedVertices{} is faster than \MergeLinear{},
for runtime and also for the number of checks performed
as shown in \Cref{fig:graph_no_nac_coloring_first_checks}.
%
It can be also seen that \NeighborsDegree{} strategy is
faster than the other strategies, and it holds for both merging strategies.
Also notice that,
runtime grows strictly exponentially.
This is in contrast with the previous section
as these graphs are not simple anymore.

\begin{figure}[thbp]
	\centering
	\scalebox{\BenchFigureScale}{%% Creator: Matplotlib, PGF backend
%%
%% To include the figure in your LaTeX document, write
%%   \input{<filename>.pgf}
%%
%% Make sure the required packages are loaded in your preamble
%%   \usepackage{pgf}
%%
%% Also ensure that all the required font packages are loaded; for instance,
%% the lmodern package is sometimes necessary when using math font.
%%   \usepackage{lmodern}
%%
%% Figures using additional raster images can only be included by \input if
%% they are in the same directory as the main LaTeX file. For loading figures
%% from other directories you can use the `import` package
%%   \usepackage{import}
%%
%% and then include the figures with
%%   \import{<path to file>}{<filename>.pgf}
%%
%% Matplotlib used the following preamble
%%   \def\mathdefault#1{#1}
%%   \everymath=\expandafter{\the\everymath\displaystyle}
%%   \IfFileExists{scrextend.sty}{
%%     \usepackage[fontsize=10.000000pt]{scrextend}
%%   }{
%%     \renewcommand{\normalsize}{\fontsize{10.000000}{12.000000}\selectfont}
%%     \normalsize
%%   }
%%   
%%   \ifdefined\pdftexversion\else  % non-pdftex case.
%%     \usepackage{fontspec}
%%     \setmainfont{DejaVuSans.ttf}[Path=\detokenize{/home/petr/Projects/PyRigi/.venv/lib/python3.12/site-packages/matplotlib/mpl-data/fonts/ttf/}]
%%     \setsansfont{DejaVuSans.ttf}[Path=\detokenize{/home/petr/Projects/PyRigi/.venv/lib/python3.12/site-packages/matplotlib/mpl-data/fonts/ttf/}]
%%     \setmonofont{DejaVuSansMono.ttf}[Path=\detokenize{/home/petr/Projects/PyRigi/.venv/lib/python3.12/site-packages/matplotlib/mpl-data/fonts/ttf/}]
%%   \fi
%%   \makeatletter\@ifpackageloaded{underscore}{}{\usepackage[strings]{underscore}}\makeatother
%%
\begingroup%
\makeatletter%
\begin{pgfpicture}%
\pgfpathrectangle{\pgfpointorigin}{\pgfqpoint{8.384376in}{2.841849in}}%
\pgfusepath{use as bounding box, clip}%
\begin{pgfscope}%
\pgfsetbuttcap%
\pgfsetmiterjoin%
\definecolor{currentfill}{rgb}{1.000000,1.000000,1.000000}%
\pgfsetfillcolor{currentfill}%
\pgfsetlinewidth{0.000000pt}%
\definecolor{currentstroke}{rgb}{1.000000,1.000000,1.000000}%
\pgfsetstrokecolor{currentstroke}%
\pgfsetdash{}{0pt}%
\pgfpathmoveto{\pgfqpoint{0.000000in}{0.000000in}}%
\pgfpathlineto{\pgfqpoint{8.384376in}{0.000000in}}%
\pgfpathlineto{\pgfqpoint{8.384376in}{2.841849in}}%
\pgfpathlineto{\pgfqpoint{0.000000in}{2.841849in}}%
\pgfpathlineto{\pgfqpoint{0.000000in}{0.000000in}}%
\pgfpathclose%
\pgfusepath{fill}%
\end{pgfscope}%
\begin{pgfscope}%
\pgfsetbuttcap%
\pgfsetmiterjoin%
\definecolor{currentfill}{rgb}{1.000000,1.000000,1.000000}%
\pgfsetfillcolor{currentfill}%
\pgfsetlinewidth{0.000000pt}%
\definecolor{currentstroke}{rgb}{0.000000,0.000000,0.000000}%
\pgfsetstrokecolor{currentstroke}%
\pgfsetstrokeopacity{0.000000}%
\pgfsetdash{}{0pt}%
\pgfpathmoveto{\pgfqpoint{0.588387in}{0.521603in}}%
\pgfpathlineto{\pgfqpoint{5.257411in}{0.521603in}}%
\pgfpathlineto{\pgfqpoint{5.257411in}{2.531888in}}%
\pgfpathlineto{\pgfqpoint{0.588387in}{2.531888in}}%
\pgfpathlineto{\pgfqpoint{0.588387in}{0.521603in}}%
\pgfpathclose%
\pgfusepath{fill}%
\end{pgfscope}%
\begin{pgfscope}%
\pgfsetbuttcap%
\pgfsetroundjoin%
\definecolor{currentfill}{rgb}{0.000000,0.000000,0.000000}%
\pgfsetfillcolor{currentfill}%
\pgfsetlinewidth{0.803000pt}%
\definecolor{currentstroke}{rgb}{0.000000,0.000000,0.000000}%
\pgfsetstrokecolor{currentstroke}%
\pgfsetdash{}{0pt}%
\pgfsys@defobject{currentmarker}{\pgfqpoint{0.000000in}{-0.048611in}}{\pgfqpoint{0.000000in}{0.000000in}}{%
\pgfpathmoveto{\pgfqpoint{0.000000in}{0.000000in}}%
\pgfpathlineto{\pgfqpoint{0.000000in}{-0.048611in}}%
\pgfusepath{stroke,fill}%
}%
\begin{pgfscope}%
\pgfsys@transformshift{1.017491in}{0.521603in}%
\pgfsys@useobject{currentmarker}{}%
\end{pgfscope}%
\end{pgfscope}%
\begin{pgfscope}%
\definecolor{textcolor}{rgb}{0.000000,0.000000,0.000000}%
\pgfsetstrokecolor{textcolor}%
\pgfsetfillcolor{textcolor}%
\pgftext[x=1.017491in,y=0.424381in,,top]{\color{textcolor}{\rmfamily\fontsize{10.000000}{12.000000}\selectfont\catcode`\^=\active\def^{\ifmmode\sp\else\^{}\fi}\catcode`\%=\active\def%{\%}$\mathdefault{20}$}}%
\end{pgfscope}%
\begin{pgfscope}%
\pgfsetbuttcap%
\pgfsetroundjoin%
\definecolor{currentfill}{rgb}{0.000000,0.000000,0.000000}%
\pgfsetfillcolor{currentfill}%
\pgfsetlinewidth{0.803000pt}%
\definecolor{currentstroke}{rgb}{0.000000,0.000000,0.000000}%
\pgfsetstrokecolor{currentstroke}%
\pgfsetdash{}{0pt}%
\pgfsys@defobject{currentmarker}{\pgfqpoint{0.000000in}{-0.048611in}}{\pgfqpoint{0.000000in}{0.000000in}}{%
\pgfpathmoveto{\pgfqpoint{0.000000in}{0.000000in}}%
\pgfpathlineto{\pgfqpoint{0.000000in}{-0.048611in}}%
\pgfusepath{stroke,fill}%
}%
\begin{pgfscope}%
\pgfsys@transformshift{1.637136in}{0.521603in}%
\pgfsys@useobject{currentmarker}{}%
\end{pgfscope}%
\end{pgfscope}%
\begin{pgfscope}%
\definecolor{textcolor}{rgb}{0.000000,0.000000,0.000000}%
\pgfsetstrokecolor{textcolor}%
\pgfsetfillcolor{textcolor}%
\pgftext[x=1.637136in,y=0.424381in,,top]{\color{textcolor}{\rmfamily\fontsize{10.000000}{12.000000}\selectfont\catcode`\^=\active\def^{\ifmmode\sp\else\^{}\fi}\catcode`\%=\active\def%{\%}$\mathdefault{40}$}}%
\end{pgfscope}%
\begin{pgfscope}%
\pgfsetbuttcap%
\pgfsetroundjoin%
\definecolor{currentfill}{rgb}{0.000000,0.000000,0.000000}%
\pgfsetfillcolor{currentfill}%
\pgfsetlinewidth{0.803000pt}%
\definecolor{currentstroke}{rgb}{0.000000,0.000000,0.000000}%
\pgfsetstrokecolor{currentstroke}%
\pgfsetdash{}{0pt}%
\pgfsys@defobject{currentmarker}{\pgfqpoint{0.000000in}{-0.048611in}}{\pgfqpoint{0.000000in}{0.000000in}}{%
\pgfpathmoveto{\pgfqpoint{0.000000in}{0.000000in}}%
\pgfpathlineto{\pgfqpoint{0.000000in}{-0.048611in}}%
\pgfusepath{stroke,fill}%
}%
\begin{pgfscope}%
\pgfsys@transformshift{2.256781in}{0.521603in}%
\pgfsys@useobject{currentmarker}{}%
\end{pgfscope}%
\end{pgfscope}%
\begin{pgfscope}%
\definecolor{textcolor}{rgb}{0.000000,0.000000,0.000000}%
\pgfsetstrokecolor{textcolor}%
\pgfsetfillcolor{textcolor}%
\pgftext[x=2.256781in,y=0.424381in,,top]{\color{textcolor}{\rmfamily\fontsize{10.000000}{12.000000}\selectfont\catcode`\^=\active\def^{\ifmmode\sp\else\^{}\fi}\catcode`\%=\active\def%{\%}$\mathdefault{60}$}}%
\end{pgfscope}%
\begin{pgfscope}%
\pgfsetbuttcap%
\pgfsetroundjoin%
\definecolor{currentfill}{rgb}{0.000000,0.000000,0.000000}%
\pgfsetfillcolor{currentfill}%
\pgfsetlinewidth{0.803000pt}%
\definecolor{currentstroke}{rgb}{0.000000,0.000000,0.000000}%
\pgfsetstrokecolor{currentstroke}%
\pgfsetdash{}{0pt}%
\pgfsys@defobject{currentmarker}{\pgfqpoint{0.000000in}{-0.048611in}}{\pgfqpoint{0.000000in}{0.000000in}}{%
\pgfpathmoveto{\pgfqpoint{0.000000in}{0.000000in}}%
\pgfpathlineto{\pgfqpoint{0.000000in}{-0.048611in}}%
\pgfusepath{stroke,fill}%
}%
\begin{pgfscope}%
\pgfsys@transformshift{2.876426in}{0.521603in}%
\pgfsys@useobject{currentmarker}{}%
\end{pgfscope}%
\end{pgfscope}%
\begin{pgfscope}%
\definecolor{textcolor}{rgb}{0.000000,0.000000,0.000000}%
\pgfsetstrokecolor{textcolor}%
\pgfsetfillcolor{textcolor}%
\pgftext[x=2.876426in,y=0.424381in,,top]{\color{textcolor}{\rmfamily\fontsize{10.000000}{12.000000}\selectfont\catcode`\^=\active\def^{\ifmmode\sp\else\^{}\fi}\catcode`\%=\active\def%{\%}$\mathdefault{80}$}}%
\end{pgfscope}%
\begin{pgfscope}%
\pgfsetbuttcap%
\pgfsetroundjoin%
\definecolor{currentfill}{rgb}{0.000000,0.000000,0.000000}%
\pgfsetfillcolor{currentfill}%
\pgfsetlinewidth{0.803000pt}%
\definecolor{currentstroke}{rgb}{0.000000,0.000000,0.000000}%
\pgfsetstrokecolor{currentstroke}%
\pgfsetdash{}{0pt}%
\pgfsys@defobject{currentmarker}{\pgfqpoint{0.000000in}{-0.048611in}}{\pgfqpoint{0.000000in}{0.000000in}}{%
\pgfpathmoveto{\pgfqpoint{0.000000in}{0.000000in}}%
\pgfpathlineto{\pgfqpoint{0.000000in}{-0.048611in}}%
\pgfusepath{stroke,fill}%
}%
\begin{pgfscope}%
\pgfsys@transformshift{3.496071in}{0.521603in}%
\pgfsys@useobject{currentmarker}{}%
\end{pgfscope}%
\end{pgfscope}%
\begin{pgfscope}%
\definecolor{textcolor}{rgb}{0.000000,0.000000,0.000000}%
\pgfsetstrokecolor{textcolor}%
\pgfsetfillcolor{textcolor}%
\pgftext[x=3.496071in,y=0.424381in,,top]{\color{textcolor}{\rmfamily\fontsize{10.000000}{12.000000}\selectfont\catcode`\^=\active\def^{\ifmmode\sp\else\^{}\fi}\catcode`\%=\active\def%{\%}$\mathdefault{100}$}}%
\end{pgfscope}%
\begin{pgfscope}%
\pgfsetbuttcap%
\pgfsetroundjoin%
\definecolor{currentfill}{rgb}{0.000000,0.000000,0.000000}%
\pgfsetfillcolor{currentfill}%
\pgfsetlinewidth{0.803000pt}%
\definecolor{currentstroke}{rgb}{0.000000,0.000000,0.000000}%
\pgfsetstrokecolor{currentstroke}%
\pgfsetdash{}{0pt}%
\pgfsys@defobject{currentmarker}{\pgfqpoint{0.000000in}{-0.048611in}}{\pgfqpoint{0.000000in}{0.000000in}}{%
\pgfpathmoveto{\pgfqpoint{0.000000in}{0.000000in}}%
\pgfpathlineto{\pgfqpoint{0.000000in}{-0.048611in}}%
\pgfusepath{stroke,fill}%
}%
\begin{pgfscope}%
\pgfsys@transformshift{4.115716in}{0.521603in}%
\pgfsys@useobject{currentmarker}{}%
\end{pgfscope}%
\end{pgfscope}%
\begin{pgfscope}%
\definecolor{textcolor}{rgb}{0.000000,0.000000,0.000000}%
\pgfsetstrokecolor{textcolor}%
\pgfsetfillcolor{textcolor}%
\pgftext[x=4.115716in,y=0.424381in,,top]{\color{textcolor}{\rmfamily\fontsize{10.000000}{12.000000}\selectfont\catcode`\^=\active\def^{\ifmmode\sp\else\^{}\fi}\catcode`\%=\active\def%{\%}$\mathdefault{120}$}}%
\end{pgfscope}%
\begin{pgfscope}%
\pgfsetbuttcap%
\pgfsetroundjoin%
\definecolor{currentfill}{rgb}{0.000000,0.000000,0.000000}%
\pgfsetfillcolor{currentfill}%
\pgfsetlinewidth{0.803000pt}%
\definecolor{currentstroke}{rgb}{0.000000,0.000000,0.000000}%
\pgfsetstrokecolor{currentstroke}%
\pgfsetdash{}{0pt}%
\pgfsys@defobject{currentmarker}{\pgfqpoint{0.000000in}{-0.048611in}}{\pgfqpoint{0.000000in}{0.000000in}}{%
\pgfpathmoveto{\pgfqpoint{0.000000in}{0.000000in}}%
\pgfpathlineto{\pgfqpoint{0.000000in}{-0.048611in}}%
\pgfusepath{stroke,fill}%
}%
\begin{pgfscope}%
\pgfsys@transformshift{4.735360in}{0.521603in}%
\pgfsys@useobject{currentmarker}{}%
\end{pgfscope}%
\end{pgfscope}%
\begin{pgfscope}%
\definecolor{textcolor}{rgb}{0.000000,0.000000,0.000000}%
\pgfsetstrokecolor{textcolor}%
\pgfsetfillcolor{textcolor}%
\pgftext[x=4.735360in,y=0.424381in,,top]{\color{textcolor}{\rmfamily\fontsize{10.000000}{12.000000}\selectfont\catcode`\^=\active\def^{\ifmmode\sp\else\^{}\fi}\catcode`\%=\active\def%{\%}$\mathdefault{140}$}}%
\end{pgfscope}%
\begin{pgfscope}%
\definecolor{textcolor}{rgb}{0.000000,0.000000,0.000000}%
\pgfsetstrokecolor{textcolor}%
\pgfsetfillcolor{textcolor}%
\pgftext[x=2.922899in,y=0.234413in,,top]{\color{textcolor}{\rmfamily\fontsize{10.000000}{12.000000}\selectfont\catcode`\^=\active\def^{\ifmmode\sp\else\^{}\fi}\catcode`\%=\active\def%{\%}Triangle components}}%
\end{pgfscope}%
\begin{pgfscope}%
\pgfsetbuttcap%
\pgfsetroundjoin%
\definecolor{currentfill}{rgb}{0.000000,0.000000,0.000000}%
\pgfsetfillcolor{currentfill}%
\pgfsetlinewidth{0.803000pt}%
\definecolor{currentstroke}{rgb}{0.000000,0.000000,0.000000}%
\pgfsetstrokecolor{currentstroke}%
\pgfsetdash{}{0pt}%
\pgfsys@defobject{currentmarker}{\pgfqpoint{-0.048611in}{0.000000in}}{\pgfqpoint{-0.000000in}{0.000000in}}{%
\pgfpathmoveto{\pgfqpoint{-0.000000in}{0.000000in}}%
\pgfpathlineto{\pgfqpoint{-0.048611in}{0.000000in}}%
\pgfusepath{stroke,fill}%
}%
\begin{pgfscope}%
\pgfsys@transformshift{0.588387in}{1.696092in}%
\pgfsys@useobject{currentmarker}{}%
\end{pgfscope}%
\end{pgfscope}%
\begin{pgfscope}%
\definecolor{textcolor}{rgb}{0.000000,0.000000,0.000000}%
\pgfsetstrokecolor{textcolor}%
\pgfsetfillcolor{textcolor}%
\pgftext[x=0.289968in, y=1.643330in, left, base]{\color{textcolor}{\rmfamily\fontsize{10.000000}{12.000000}\selectfont\catcode`\^=\active\def^{\ifmmode\sp\else\^{}\fi}\catcode`\%=\active\def%{\%}$\mathdefault{10^{3}}$}}%
\end{pgfscope}%
\begin{pgfscope}%
\pgfsetbuttcap%
\pgfsetroundjoin%
\definecolor{currentfill}{rgb}{0.000000,0.000000,0.000000}%
\pgfsetfillcolor{currentfill}%
\pgfsetlinewidth{0.602250pt}%
\definecolor{currentstroke}{rgb}{0.000000,0.000000,0.000000}%
\pgfsetstrokecolor{currentstroke}%
\pgfsetdash{}{0pt}%
\pgfsys@defobject{currentmarker}{\pgfqpoint{-0.027778in}{0.000000in}}{\pgfqpoint{-0.000000in}{0.000000in}}{%
\pgfpathmoveto{\pgfqpoint{-0.000000in}{0.000000in}}%
\pgfpathlineto{\pgfqpoint{-0.027778in}{0.000000in}}%
\pgfusepath{stroke,fill}%
}%
\begin{pgfscope}%
\pgfsys@transformshift{0.588387in}{0.835560in}%
\pgfsys@useobject{currentmarker}{}%
\end{pgfscope}%
\end{pgfscope}%
\begin{pgfscope}%
\pgfsetbuttcap%
\pgfsetroundjoin%
\definecolor{currentfill}{rgb}{0.000000,0.000000,0.000000}%
\pgfsetfillcolor{currentfill}%
\pgfsetlinewidth{0.602250pt}%
\definecolor{currentstroke}{rgb}{0.000000,0.000000,0.000000}%
\pgfsetstrokecolor{currentstroke}%
\pgfsetdash{}{0pt}%
\pgfsys@defobject{currentmarker}{\pgfqpoint{-0.027778in}{0.000000in}}{\pgfqpoint{-0.000000in}{0.000000in}}{%
\pgfpathmoveto{\pgfqpoint{-0.000000in}{0.000000in}}%
\pgfpathlineto{\pgfqpoint{-0.027778in}{0.000000in}}%
\pgfusepath{stroke,fill}%
}%
\begin{pgfscope}%
\pgfsys@transformshift{0.588387in}{1.052354in}%
\pgfsys@useobject{currentmarker}{}%
\end{pgfscope}%
\end{pgfscope}%
\begin{pgfscope}%
\pgfsetbuttcap%
\pgfsetroundjoin%
\definecolor{currentfill}{rgb}{0.000000,0.000000,0.000000}%
\pgfsetfillcolor{currentfill}%
\pgfsetlinewidth{0.602250pt}%
\definecolor{currentstroke}{rgb}{0.000000,0.000000,0.000000}%
\pgfsetstrokecolor{currentstroke}%
\pgfsetdash{}{0pt}%
\pgfsys@defobject{currentmarker}{\pgfqpoint{-0.027778in}{0.000000in}}{\pgfqpoint{-0.000000in}{0.000000in}}{%
\pgfpathmoveto{\pgfqpoint{-0.000000in}{0.000000in}}%
\pgfpathlineto{\pgfqpoint{-0.027778in}{0.000000in}}%
\pgfusepath{stroke,fill}%
}%
\begin{pgfscope}%
\pgfsys@transformshift{0.588387in}{1.206171in}%
\pgfsys@useobject{currentmarker}{}%
\end{pgfscope}%
\end{pgfscope}%
\begin{pgfscope}%
\pgfsetbuttcap%
\pgfsetroundjoin%
\definecolor{currentfill}{rgb}{0.000000,0.000000,0.000000}%
\pgfsetfillcolor{currentfill}%
\pgfsetlinewidth{0.602250pt}%
\definecolor{currentstroke}{rgb}{0.000000,0.000000,0.000000}%
\pgfsetstrokecolor{currentstroke}%
\pgfsetdash{}{0pt}%
\pgfsys@defobject{currentmarker}{\pgfqpoint{-0.027778in}{0.000000in}}{\pgfqpoint{-0.000000in}{0.000000in}}{%
\pgfpathmoveto{\pgfqpoint{-0.000000in}{0.000000in}}%
\pgfpathlineto{\pgfqpoint{-0.027778in}{0.000000in}}%
\pgfusepath{stroke,fill}%
}%
\begin{pgfscope}%
\pgfsys@transformshift{0.588387in}{1.325481in}%
\pgfsys@useobject{currentmarker}{}%
\end{pgfscope}%
\end{pgfscope}%
\begin{pgfscope}%
\pgfsetbuttcap%
\pgfsetroundjoin%
\definecolor{currentfill}{rgb}{0.000000,0.000000,0.000000}%
\pgfsetfillcolor{currentfill}%
\pgfsetlinewidth{0.602250pt}%
\definecolor{currentstroke}{rgb}{0.000000,0.000000,0.000000}%
\pgfsetstrokecolor{currentstroke}%
\pgfsetdash{}{0pt}%
\pgfsys@defobject{currentmarker}{\pgfqpoint{-0.027778in}{0.000000in}}{\pgfqpoint{-0.000000in}{0.000000in}}{%
\pgfpathmoveto{\pgfqpoint{-0.000000in}{0.000000in}}%
\pgfpathlineto{\pgfqpoint{-0.027778in}{0.000000in}}%
\pgfusepath{stroke,fill}%
}%
\begin{pgfscope}%
\pgfsys@transformshift{0.588387in}{1.422964in}%
\pgfsys@useobject{currentmarker}{}%
\end{pgfscope}%
\end{pgfscope}%
\begin{pgfscope}%
\pgfsetbuttcap%
\pgfsetroundjoin%
\definecolor{currentfill}{rgb}{0.000000,0.000000,0.000000}%
\pgfsetfillcolor{currentfill}%
\pgfsetlinewidth{0.602250pt}%
\definecolor{currentstroke}{rgb}{0.000000,0.000000,0.000000}%
\pgfsetstrokecolor{currentstroke}%
\pgfsetdash{}{0pt}%
\pgfsys@defobject{currentmarker}{\pgfqpoint{-0.027778in}{0.000000in}}{\pgfqpoint{-0.000000in}{0.000000in}}{%
\pgfpathmoveto{\pgfqpoint{-0.000000in}{0.000000in}}%
\pgfpathlineto{\pgfqpoint{-0.027778in}{0.000000in}}%
\pgfusepath{stroke,fill}%
}%
\begin{pgfscope}%
\pgfsys@transformshift{0.588387in}{1.505385in}%
\pgfsys@useobject{currentmarker}{}%
\end{pgfscope}%
\end{pgfscope}%
\begin{pgfscope}%
\pgfsetbuttcap%
\pgfsetroundjoin%
\definecolor{currentfill}{rgb}{0.000000,0.000000,0.000000}%
\pgfsetfillcolor{currentfill}%
\pgfsetlinewidth{0.602250pt}%
\definecolor{currentstroke}{rgb}{0.000000,0.000000,0.000000}%
\pgfsetstrokecolor{currentstroke}%
\pgfsetdash{}{0pt}%
\pgfsys@defobject{currentmarker}{\pgfqpoint{-0.027778in}{0.000000in}}{\pgfqpoint{-0.000000in}{0.000000in}}{%
\pgfpathmoveto{\pgfqpoint{-0.000000in}{0.000000in}}%
\pgfpathlineto{\pgfqpoint{-0.027778in}{0.000000in}}%
\pgfusepath{stroke,fill}%
}%
\begin{pgfscope}%
\pgfsys@transformshift{0.588387in}{1.576782in}%
\pgfsys@useobject{currentmarker}{}%
\end{pgfscope}%
\end{pgfscope}%
\begin{pgfscope}%
\pgfsetbuttcap%
\pgfsetroundjoin%
\definecolor{currentfill}{rgb}{0.000000,0.000000,0.000000}%
\pgfsetfillcolor{currentfill}%
\pgfsetlinewidth{0.602250pt}%
\definecolor{currentstroke}{rgb}{0.000000,0.000000,0.000000}%
\pgfsetstrokecolor{currentstroke}%
\pgfsetdash{}{0pt}%
\pgfsys@defobject{currentmarker}{\pgfqpoint{-0.027778in}{0.000000in}}{\pgfqpoint{-0.000000in}{0.000000in}}{%
\pgfpathmoveto{\pgfqpoint{-0.000000in}{0.000000in}}%
\pgfpathlineto{\pgfqpoint{-0.027778in}{0.000000in}}%
\pgfusepath{stroke,fill}%
}%
\begin{pgfscope}%
\pgfsys@transformshift{0.588387in}{1.639758in}%
\pgfsys@useobject{currentmarker}{}%
\end{pgfscope}%
\end{pgfscope}%
\begin{pgfscope}%
\pgfsetbuttcap%
\pgfsetroundjoin%
\definecolor{currentfill}{rgb}{0.000000,0.000000,0.000000}%
\pgfsetfillcolor{currentfill}%
\pgfsetlinewidth{0.602250pt}%
\definecolor{currentstroke}{rgb}{0.000000,0.000000,0.000000}%
\pgfsetstrokecolor{currentstroke}%
\pgfsetdash{}{0pt}%
\pgfsys@defobject{currentmarker}{\pgfqpoint{-0.027778in}{0.000000in}}{\pgfqpoint{-0.000000in}{0.000000in}}{%
\pgfpathmoveto{\pgfqpoint{-0.000000in}{0.000000in}}%
\pgfpathlineto{\pgfqpoint{-0.027778in}{0.000000in}}%
\pgfusepath{stroke,fill}%
}%
\begin{pgfscope}%
\pgfsys@transformshift{0.588387in}{2.066702in}%
\pgfsys@useobject{currentmarker}{}%
\end{pgfscope}%
\end{pgfscope}%
\begin{pgfscope}%
\pgfsetbuttcap%
\pgfsetroundjoin%
\definecolor{currentfill}{rgb}{0.000000,0.000000,0.000000}%
\pgfsetfillcolor{currentfill}%
\pgfsetlinewidth{0.602250pt}%
\definecolor{currentstroke}{rgb}{0.000000,0.000000,0.000000}%
\pgfsetstrokecolor{currentstroke}%
\pgfsetdash{}{0pt}%
\pgfsys@defobject{currentmarker}{\pgfqpoint{-0.027778in}{0.000000in}}{\pgfqpoint{-0.000000in}{0.000000in}}{%
\pgfpathmoveto{\pgfqpoint{-0.000000in}{0.000000in}}%
\pgfpathlineto{\pgfqpoint{-0.027778in}{0.000000in}}%
\pgfusepath{stroke,fill}%
}%
\begin{pgfscope}%
\pgfsys@transformshift{0.588387in}{2.283496in}%
\pgfsys@useobject{currentmarker}{}%
\end{pgfscope}%
\end{pgfscope}%
\begin{pgfscope}%
\pgfsetbuttcap%
\pgfsetroundjoin%
\definecolor{currentfill}{rgb}{0.000000,0.000000,0.000000}%
\pgfsetfillcolor{currentfill}%
\pgfsetlinewidth{0.602250pt}%
\definecolor{currentstroke}{rgb}{0.000000,0.000000,0.000000}%
\pgfsetstrokecolor{currentstroke}%
\pgfsetdash{}{0pt}%
\pgfsys@defobject{currentmarker}{\pgfqpoint{-0.027778in}{0.000000in}}{\pgfqpoint{-0.000000in}{0.000000in}}{%
\pgfpathmoveto{\pgfqpoint{-0.000000in}{0.000000in}}%
\pgfpathlineto{\pgfqpoint{-0.027778in}{0.000000in}}%
\pgfusepath{stroke,fill}%
}%
\begin{pgfscope}%
\pgfsys@transformshift{0.588387in}{2.437313in}%
\pgfsys@useobject{currentmarker}{}%
\end{pgfscope}%
\end{pgfscope}%
\begin{pgfscope}%
\definecolor{textcolor}{rgb}{0.000000,0.000000,0.000000}%
\pgfsetstrokecolor{textcolor}%
\pgfsetfillcolor{textcolor}%
\pgftext[x=0.234413in,y=1.526746in,,bottom,rotate=90.000000]{\color{textcolor}{\rmfamily\fontsize{10.000000}{12.000000}\selectfont\catcode`\^=\active\def^{\ifmmode\sp\else\^{}\fi}\catcode`\%=\active\def%{\%}Time [ms]}}%
\end{pgfscope}%
\begin{pgfscope}%
\pgfpathrectangle{\pgfqpoint{0.588387in}{0.521603in}}{\pgfqpoint{4.669024in}{2.010285in}}%
\pgfusepath{clip}%
\pgfsetrectcap%
\pgfsetroundjoin%
\pgfsetlinewidth{1.505625pt}%
\pgfsetstrokecolor{currentstroke1}%
\pgfsetdash{}{0pt}%
\pgfpathmoveto{\pgfqpoint{0.800616in}{1.014144in}}%
\pgfpathlineto{\pgfqpoint{0.831598in}{0.632380in}}%
\pgfpathlineto{\pgfqpoint{0.862580in}{0.700493in}}%
\pgfpathlineto{\pgfqpoint{0.893562in}{0.844372in}}%
\pgfpathlineto{\pgfqpoint{0.924545in}{0.922774in}}%
\pgfpathlineto{\pgfqpoint{0.955527in}{0.964352in}}%
\pgfpathlineto{\pgfqpoint{0.986509in}{0.948393in}}%
\pgfpathlineto{\pgfqpoint{1.017491in}{1.044323in}}%
\pgfpathlineto{\pgfqpoint{1.048474in}{1.077766in}}%
\pgfpathlineto{\pgfqpoint{1.079456in}{0.990339in}}%
\pgfpathlineto{\pgfqpoint{1.110438in}{1.024191in}}%
\pgfpathlineto{\pgfqpoint{1.141420in}{0.957825in}}%
\pgfpathlineto{\pgfqpoint{1.172402in}{1.016072in}}%
\pgfpathlineto{\pgfqpoint{1.203385in}{0.978019in}}%
\pgfpathlineto{\pgfqpoint{1.234367in}{1.061563in}}%
\pgfpathlineto{\pgfqpoint{1.265349in}{1.072881in}}%
\pgfpathlineto{\pgfqpoint{1.296331in}{1.087597in}}%
\pgfpathlineto{\pgfqpoint{1.327314in}{1.058847in}}%
\pgfpathlineto{\pgfqpoint{1.358296in}{1.055025in}}%
\pgfpathlineto{\pgfqpoint{1.389278in}{1.055178in}}%
\pgfpathlineto{\pgfqpoint{1.420260in}{1.135704in}}%
\pgfpathlineto{\pgfqpoint{1.451243in}{1.159891in}}%
\pgfpathlineto{\pgfqpoint{1.482225in}{1.083863in}}%
\pgfpathlineto{\pgfqpoint{1.513207in}{1.146726in}}%
\pgfpathlineto{\pgfqpoint{1.544189in}{1.130002in}}%
\pgfpathlineto{\pgfqpoint{1.575172in}{1.202030in}}%
\pgfpathlineto{\pgfqpoint{1.606154in}{1.183083in}}%
\pgfpathlineto{\pgfqpoint{1.637136in}{1.207661in}}%
\pgfpathlineto{\pgfqpoint{1.668118in}{1.194878in}}%
\pgfpathlineto{\pgfqpoint{1.699101in}{1.292110in}}%
\pgfpathlineto{\pgfqpoint{1.730083in}{1.268411in}}%
\pgfpathlineto{\pgfqpoint{1.761065in}{1.289792in}}%
\pgfpathlineto{\pgfqpoint{1.792047in}{1.317988in}}%
\pgfpathlineto{\pgfqpoint{1.823030in}{1.278435in}}%
\pgfpathlineto{\pgfqpoint{1.854012in}{1.340789in}}%
\pgfpathlineto{\pgfqpoint{1.884994in}{1.323185in}}%
\pgfpathlineto{\pgfqpoint{1.915976in}{1.329756in}}%
\pgfpathlineto{\pgfqpoint{1.946959in}{1.329308in}}%
\pgfpathlineto{\pgfqpoint{1.977941in}{1.368281in}}%
\pgfpathlineto{\pgfqpoint{2.008923in}{1.387997in}}%
\pgfpathlineto{\pgfqpoint{2.039905in}{1.433115in}}%
\pgfpathlineto{\pgfqpoint{2.070888in}{1.392169in}}%
\pgfpathlineto{\pgfqpoint{2.101870in}{1.455025in}}%
\pgfpathlineto{\pgfqpoint{2.132852in}{1.421362in}}%
\pgfpathlineto{\pgfqpoint{2.163834in}{1.490891in}}%
\pgfpathlineto{\pgfqpoint{2.194817in}{1.488755in}}%
\pgfpathlineto{\pgfqpoint{2.225799in}{1.494857in}}%
\pgfpathlineto{\pgfqpoint{2.256781in}{1.530761in}}%
\pgfpathlineto{\pgfqpoint{2.287763in}{1.520190in}}%
\pgfpathlineto{\pgfqpoint{2.318745in}{1.595894in}}%
\pgfpathlineto{\pgfqpoint{2.349728in}{1.531836in}}%
\pgfpathlineto{\pgfqpoint{2.380710in}{1.624395in}}%
\pgfpathlineto{\pgfqpoint{2.411692in}{1.618629in}}%
\pgfpathlineto{\pgfqpoint{2.442674in}{1.592766in}}%
\pgfpathlineto{\pgfqpoint{2.473657in}{1.649919in}}%
\pgfpathlineto{\pgfqpoint{2.504639in}{1.652613in}}%
\pgfpathlineto{\pgfqpoint{2.535621in}{1.648585in}}%
\pgfpathlineto{\pgfqpoint{2.566603in}{1.645100in}}%
\pgfpathlineto{\pgfqpoint{2.597586in}{1.703540in}}%
\pgfpathlineto{\pgfqpoint{2.628568in}{1.719959in}}%
\pgfpathlineto{\pgfqpoint{2.659550in}{1.685012in}}%
\pgfpathlineto{\pgfqpoint{2.690532in}{1.713727in}}%
\pgfpathlineto{\pgfqpoint{2.721515in}{1.696009in}}%
\pgfpathlineto{\pgfqpoint{2.752497in}{1.762318in}}%
\pgfpathlineto{\pgfqpoint{2.783479in}{1.823581in}}%
\pgfpathlineto{\pgfqpoint{2.814461in}{1.810349in}}%
\pgfpathlineto{\pgfqpoint{2.845444in}{1.742087in}}%
\pgfpathlineto{\pgfqpoint{2.876426in}{1.805067in}}%
\pgfpathlineto{\pgfqpoint{2.907408in}{1.835267in}}%
\pgfpathlineto{\pgfqpoint{2.938390in}{1.845268in}}%
\pgfpathlineto{\pgfqpoint{2.969373in}{1.857752in}}%
\pgfpathlineto{\pgfqpoint{3.000355in}{1.865305in}}%
\pgfpathlineto{\pgfqpoint{3.031337in}{1.863926in}}%
\pgfpathlineto{\pgfqpoint{3.062319in}{1.872560in}}%
\pgfpathlineto{\pgfqpoint{3.093302in}{1.961099in}}%
\pgfpathlineto{\pgfqpoint{3.124284in}{1.957375in}}%
\pgfpathlineto{\pgfqpoint{3.155266in}{1.947671in}}%
\pgfpathlineto{\pgfqpoint{3.186248in}{1.962548in}}%
\pgfpathlineto{\pgfqpoint{3.217231in}{1.986825in}}%
\pgfpathlineto{\pgfqpoint{3.248213in}{2.070508in}}%
\pgfpathlineto{\pgfqpoint{3.279195in}{1.983292in}}%
\pgfpathlineto{\pgfqpoint{3.310177in}{2.031472in}}%
\pgfpathlineto{\pgfqpoint{3.341159in}{2.109397in}}%
\pgfpathlineto{\pgfqpoint{3.372142in}{2.036838in}}%
\pgfpathlineto{\pgfqpoint{3.403124in}{2.052212in}}%
\pgfpathlineto{\pgfqpoint{3.434106in}{2.082413in}}%
\pgfpathlineto{\pgfqpoint{3.465088in}{2.136925in}}%
\pgfpathlineto{\pgfqpoint{3.496071in}{2.162624in}}%
\pgfpathlineto{\pgfqpoint{3.527053in}{2.122487in}}%
\pgfpathlineto{\pgfqpoint{3.558035in}{2.191809in}}%
\pgfpathlineto{\pgfqpoint{3.589017in}{2.166720in}}%
\pgfpathlineto{\pgfqpoint{3.620000in}{2.164896in}}%
\pgfpathlineto{\pgfqpoint{3.650982in}{2.197497in}}%
\pgfpathlineto{\pgfqpoint{3.681964in}{2.158708in}}%
\pgfpathlineto{\pgfqpoint{3.712946in}{2.208855in}}%
\pgfpathlineto{\pgfqpoint{3.743929in}{2.257054in}}%
\pgfpathlineto{\pgfqpoint{3.774911in}{2.219801in}}%
\pgfpathlineto{\pgfqpoint{3.805893in}{2.248437in}}%
\pgfpathlineto{\pgfqpoint{3.836875in}{2.254358in}}%
\pgfpathlineto{\pgfqpoint{3.867858in}{2.258893in}}%
\pgfpathlineto{\pgfqpoint{3.898840in}{2.310465in}}%
\pgfpathlineto{\pgfqpoint{3.929822in}{2.266198in}}%
\pgfpathlineto{\pgfqpoint{3.960804in}{2.257756in}}%
\pgfpathlineto{\pgfqpoint{3.991787in}{2.266106in}}%
\pgfpathlineto{\pgfqpoint{4.022769in}{2.316609in}}%
\pgfpathlineto{\pgfqpoint{4.053751in}{2.309847in}}%
\pgfpathlineto{\pgfqpoint{4.084733in}{2.320004in}}%
\pgfpathlineto{\pgfqpoint{4.115716in}{2.313641in}}%
\pgfpathlineto{\pgfqpoint{4.146698in}{2.344408in}}%
\pgfpathlineto{\pgfqpoint{4.177680in}{2.376917in}}%
\pgfpathlineto{\pgfqpoint{4.208662in}{2.374254in}}%
\pgfpathlineto{\pgfqpoint{4.239645in}{2.396712in}}%
\pgfpathlineto{\pgfqpoint{4.270627in}{2.355115in}}%
\pgfpathlineto{\pgfqpoint{4.332591in}{2.388501in}}%
\pgfpathlineto{\pgfqpoint{4.363573in}{2.335266in}}%
\pgfpathlineto{\pgfqpoint{4.394556in}{2.427873in}}%
\pgfpathlineto{\pgfqpoint{4.425538in}{2.428689in}}%
\pgfpathlineto{\pgfqpoint{4.456520in}{2.387181in}}%
\pgfpathlineto{\pgfqpoint{4.549467in}{2.432614in}}%
\pgfpathlineto{\pgfqpoint{4.704378in}{2.416599in}}%
\pgfusepath{stroke}%
\end{pgfscope}%
\begin{pgfscope}%
\pgfpathrectangle{\pgfqpoint{0.588387in}{0.521603in}}{\pgfqpoint{4.669024in}{2.010285in}}%
\pgfusepath{clip}%
\pgfsetrectcap%
\pgfsetroundjoin%
\pgfsetlinewidth{1.505625pt}%
\pgfsetstrokecolor{currentstroke2}%
\pgfsetdash{}{0pt}%
\pgfpathmoveto{\pgfqpoint{0.800616in}{1.034808in}}%
\pgfpathlineto{\pgfqpoint{0.831598in}{0.631309in}}%
\pgfpathlineto{\pgfqpoint{0.862580in}{0.684641in}}%
\pgfpathlineto{\pgfqpoint{0.893562in}{0.806969in}}%
\pgfpathlineto{\pgfqpoint{0.924545in}{0.885302in}}%
\pgfpathlineto{\pgfqpoint{0.955527in}{0.957722in}}%
\pgfpathlineto{\pgfqpoint{0.986509in}{0.952576in}}%
\pgfpathlineto{\pgfqpoint{1.017491in}{1.042919in}}%
\pgfpathlineto{\pgfqpoint{1.048474in}{1.086571in}}%
\pgfpathlineto{\pgfqpoint{1.079456in}{0.974285in}}%
\pgfpathlineto{\pgfqpoint{1.110438in}{1.016670in}}%
\pgfpathlineto{\pgfqpoint{1.141420in}{0.969305in}}%
\pgfpathlineto{\pgfqpoint{1.172402in}{1.018907in}}%
\pgfpathlineto{\pgfqpoint{1.203385in}{0.968021in}}%
\pgfpathlineto{\pgfqpoint{1.234367in}{1.064339in}}%
\pgfpathlineto{\pgfqpoint{1.265349in}{1.075928in}}%
\pgfpathlineto{\pgfqpoint{1.296331in}{1.082347in}}%
\pgfpathlineto{\pgfqpoint{1.327314in}{1.054081in}}%
\pgfpathlineto{\pgfqpoint{1.358296in}{1.050704in}}%
\pgfpathlineto{\pgfqpoint{1.389278in}{1.050634in}}%
\pgfpathlineto{\pgfqpoint{1.420260in}{1.136811in}}%
\pgfpathlineto{\pgfqpoint{1.451243in}{1.158636in}}%
\pgfpathlineto{\pgfqpoint{1.482225in}{1.085746in}}%
\pgfpathlineto{\pgfqpoint{1.513207in}{1.150818in}}%
\pgfpathlineto{\pgfqpoint{1.544189in}{1.113517in}}%
\pgfpathlineto{\pgfqpoint{1.575172in}{1.202419in}}%
\pgfpathlineto{\pgfqpoint{1.637136in}{1.192155in}}%
\pgfpathlineto{\pgfqpoint{1.668118in}{1.193903in}}%
\pgfpathlineto{\pgfqpoint{1.699101in}{1.315390in}}%
\pgfpathlineto{\pgfqpoint{1.730083in}{1.262641in}}%
\pgfpathlineto{\pgfqpoint{1.761065in}{1.279848in}}%
\pgfpathlineto{\pgfqpoint{1.792047in}{1.314284in}}%
\pgfpathlineto{\pgfqpoint{1.823030in}{1.270299in}}%
\pgfpathlineto{\pgfqpoint{1.854012in}{1.334154in}}%
\pgfpathlineto{\pgfqpoint{1.884994in}{1.307467in}}%
\pgfpathlineto{\pgfqpoint{1.915976in}{1.312418in}}%
\pgfpathlineto{\pgfqpoint{1.946959in}{1.309846in}}%
\pgfpathlineto{\pgfqpoint{1.977941in}{1.360739in}}%
\pgfpathlineto{\pgfqpoint{2.008923in}{1.373063in}}%
\pgfpathlineto{\pgfqpoint{2.039905in}{1.420146in}}%
\pgfpathlineto{\pgfqpoint{2.070888in}{1.396541in}}%
\pgfpathlineto{\pgfqpoint{2.101870in}{1.424801in}}%
\pgfpathlineto{\pgfqpoint{2.132852in}{1.418199in}}%
\pgfpathlineto{\pgfqpoint{2.163834in}{1.474669in}}%
\pgfpathlineto{\pgfqpoint{2.194817in}{1.436359in}}%
\pgfpathlineto{\pgfqpoint{2.225799in}{1.481344in}}%
\pgfpathlineto{\pgfqpoint{2.256781in}{1.508885in}}%
\pgfpathlineto{\pgfqpoint{2.287763in}{1.485618in}}%
\pgfpathlineto{\pgfqpoint{2.318745in}{1.555825in}}%
\pgfpathlineto{\pgfqpoint{2.349728in}{1.499143in}}%
\pgfpathlineto{\pgfqpoint{2.380710in}{1.587571in}}%
\pgfpathlineto{\pgfqpoint{2.411692in}{1.582751in}}%
\pgfpathlineto{\pgfqpoint{2.442674in}{1.541676in}}%
\pgfpathlineto{\pgfqpoint{2.473657in}{1.616038in}}%
\pgfpathlineto{\pgfqpoint{2.504639in}{1.607470in}}%
\pgfpathlineto{\pgfqpoint{2.535621in}{1.639187in}}%
\pgfpathlineto{\pgfqpoint{2.566603in}{1.605899in}}%
\pgfpathlineto{\pgfqpoint{2.597586in}{1.668997in}}%
\pgfpathlineto{\pgfqpoint{2.628568in}{1.668119in}}%
\pgfpathlineto{\pgfqpoint{2.659550in}{1.648078in}}%
\pgfpathlineto{\pgfqpoint{2.690532in}{1.663509in}}%
\pgfpathlineto{\pgfqpoint{2.721515in}{1.645820in}}%
\pgfpathlineto{\pgfqpoint{2.752497in}{1.715714in}}%
\pgfpathlineto{\pgfqpoint{2.783479in}{1.753082in}}%
\pgfpathlineto{\pgfqpoint{2.814461in}{1.752565in}}%
\pgfpathlineto{\pgfqpoint{2.845444in}{1.694887in}}%
\pgfpathlineto{\pgfqpoint{2.876426in}{1.746023in}}%
\pgfpathlineto{\pgfqpoint{2.907408in}{1.799948in}}%
\pgfpathlineto{\pgfqpoint{2.938390in}{1.788754in}}%
\pgfpathlineto{\pgfqpoint{2.969373in}{1.791757in}}%
\pgfpathlineto{\pgfqpoint{3.000355in}{1.817962in}}%
\pgfpathlineto{\pgfqpoint{3.031337in}{1.804319in}}%
\pgfpathlineto{\pgfqpoint{3.062319in}{1.811666in}}%
\pgfpathlineto{\pgfqpoint{3.093302in}{1.915197in}}%
\pgfpathlineto{\pgfqpoint{3.124284in}{1.877643in}}%
\pgfpathlineto{\pgfqpoint{3.155266in}{1.893472in}}%
\pgfpathlineto{\pgfqpoint{3.186248in}{1.897835in}}%
\pgfpathlineto{\pgfqpoint{3.217231in}{1.927026in}}%
\pgfpathlineto{\pgfqpoint{3.248213in}{2.011270in}}%
\pgfpathlineto{\pgfqpoint{3.279195in}{1.933613in}}%
\pgfpathlineto{\pgfqpoint{3.310177in}{1.937638in}}%
\pgfpathlineto{\pgfqpoint{3.341159in}{1.995561in}}%
\pgfpathlineto{\pgfqpoint{3.372142in}{1.961464in}}%
\pgfpathlineto{\pgfqpoint{3.403124in}{1.990032in}}%
\pgfpathlineto{\pgfqpoint{3.434106in}{1.995120in}}%
\pgfpathlineto{\pgfqpoint{3.465088in}{2.063400in}}%
\pgfpathlineto{\pgfqpoint{3.496071in}{2.055733in}}%
\pgfpathlineto{\pgfqpoint{3.527053in}{2.025442in}}%
\pgfpathlineto{\pgfqpoint{3.558035in}{2.090365in}}%
\pgfpathlineto{\pgfqpoint{3.589017in}{2.092535in}}%
\pgfpathlineto{\pgfqpoint{3.620000in}{2.104506in}}%
\pgfpathlineto{\pgfqpoint{3.650982in}{2.129986in}}%
\pgfpathlineto{\pgfqpoint{3.681964in}{2.134107in}}%
\pgfpathlineto{\pgfqpoint{3.712946in}{2.127849in}}%
\pgfpathlineto{\pgfqpoint{3.743929in}{2.175865in}}%
\pgfpathlineto{\pgfqpoint{3.774911in}{2.162940in}}%
\pgfpathlineto{\pgfqpoint{3.805893in}{2.227756in}}%
\pgfpathlineto{\pgfqpoint{3.836875in}{2.188339in}}%
\pgfpathlineto{\pgfqpoint{3.867858in}{2.166066in}}%
\pgfpathlineto{\pgfqpoint{3.898840in}{2.246173in}}%
\pgfpathlineto{\pgfqpoint{3.929822in}{2.181583in}}%
\pgfpathlineto{\pgfqpoint{3.991787in}{2.217329in}}%
\pgfpathlineto{\pgfqpoint{4.022769in}{2.241330in}}%
\pgfpathlineto{\pgfqpoint{4.053751in}{2.243329in}}%
\pgfpathlineto{\pgfqpoint{4.084733in}{2.240075in}}%
\pgfpathlineto{\pgfqpoint{4.115716in}{2.218075in}}%
\pgfpathlineto{\pgfqpoint{4.146698in}{2.277762in}}%
\pgfpathlineto{\pgfqpoint{4.177680in}{2.320138in}}%
\pgfpathlineto{\pgfqpoint{4.208662in}{2.310113in}}%
\pgfpathlineto{\pgfqpoint{4.239645in}{2.315961in}}%
\pgfpathlineto{\pgfqpoint{4.270627in}{2.371426in}}%
\pgfpathlineto{\pgfqpoint{4.301609in}{2.299127in}}%
\pgfpathlineto{\pgfqpoint{4.332591in}{2.376542in}}%
\pgfpathlineto{\pgfqpoint{4.363573in}{2.250602in}}%
\pgfpathlineto{\pgfqpoint{4.394556in}{2.376388in}}%
\pgfpathlineto{\pgfqpoint{4.425538in}{2.376130in}}%
\pgfpathlineto{\pgfqpoint{4.456520in}{2.366426in}}%
\pgfpathlineto{\pgfqpoint{4.487502in}{2.371161in}}%
\pgfpathlineto{\pgfqpoint{4.518485in}{2.431838in}}%
\pgfpathlineto{\pgfqpoint{4.549467in}{2.371741in}}%
\pgfpathlineto{\pgfqpoint{4.580449in}{2.409325in}}%
\pgfpathlineto{\pgfqpoint{4.611431in}{2.381573in}}%
\pgfpathlineto{\pgfqpoint{4.642414in}{2.417016in}}%
\pgfpathlineto{\pgfqpoint{4.673396in}{2.400376in}}%
\pgfpathlineto{\pgfqpoint{4.704378in}{2.340257in}}%
\pgfpathlineto{\pgfqpoint{4.797325in}{2.422816in}}%
\pgfpathlineto{\pgfqpoint{4.797325in}{2.422816in}}%
\pgfusepath{stroke}%
\end{pgfscope}%
\begin{pgfscope}%
\pgfpathrectangle{\pgfqpoint{0.588387in}{0.521603in}}{\pgfqpoint{4.669024in}{2.010285in}}%
\pgfusepath{clip}%
\pgfsetrectcap%
\pgfsetroundjoin%
\pgfsetlinewidth{1.505625pt}%
\pgfsetstrokecolor{currentstroke3}%
\pgfsetdash{}{0pt}%
\pgfpathmoveto{\pgfqpoint{0.800616in}{0.956326in}}%
\pgfpathlineto{\pgfqpoint{0.831598in}{0.614711in}}%
\pgfpathlineto{\pgfqpoint{0.862580in}{0.708424in}}%
\pgfpathlineto{\pgfqpoint{0.893562in}{0.811535in}}%
\pgfpathlineto{\pgfqpoint{0.924545in}{0.922019in}}%
\pgfpathlineto{\pgfqpoint{0.955527in}{0.960988in}}%
\pgfpathlineto{\pgfqpoint{0.986509in}{0.971209in}}%
\pgfpathlineto{\pgfqpoint{1.017491in}{1.075728in}}%
\pgfpathlineto{\pgfqpoint{1.048474in}{1.139752in}}%
\pgfpathlineto{\pgfqpoint{1.079456in}{1.045616in}}%
\pgfpathlineto{\pgfqpoint{1.110438in}{1.087972in}}%
\pgfpathlineto{\pgfqpoint{1.141420in}{1.070775in}}%
\pgfpathlineto{\pgfqpoint{1.172402in}{1.093147in}}%
\pgfpathlineto{\pgfqpoint{1.203385in}{1.051807in}}%
\pgfpathlineto{\pgfqpoint{1.234367in}{1.133226in}}%
\pgfpathlineto{\pgfqpoint{1.265349in}{1.150074in}}%
\pgfpathlineto{\pgfqpoint{1.296331in}{1.157987in}}%
\pgfpathlineto{\pgfqpoint{1.327314in}{1.133766in}}%
\pgfpathlineto{\pgfqpoint{1.358296in}{1.105464in}}%
\pgfpathlineto{\pgfqpoint{1.389278in}{1.132596in}}%
\pgfpathlineto{\pgfqpoint{1.420260in}{1.282193in}}%
\pgfpathlineto{\pgfqpoint{1.451243in}{1.237701in}}%
\pgfpathlineto{\pgfqpoint{1.482225in}{1.150634in}}%
\pgfpathlineto{\pgfqpoint{1.513207in}{1.225774in}}%
\pgfpathlineto{\pgfqpoint{1.544189in}{1.202752in}}%
\pgfpathlineto{\pgfqpoint{1.575172in}{1.332878in}}%
\pgfpathlineto{\pgfqpoint{1.606154in}{1.284402in}}%
\pgfpathlineto{\pgfqpoint{1.637136in}{1.302891in}}%
\pgfpathlineto{\pgfqpoint{1.668118in}{1.265782in}}%
\pgfpathlineto{\pgfqpoint{1.699101in}{1.388853in}}%
\pgfpathlineto{\pgfqpoint{1.730083in}{1.348678in}}%
\pgfpathlineto{\pgfqpoint{1.792047in}{1.404158in}}%
\pgfpathlineto{\pgfqpoint{1.823030in}{1.378160in}}%
\pgfpathlineto{\pgfqpoint{1.854012in}{1.440838in}}%
\pgfpathlineto{\pgfqpoint{1.884994in}{1.374933in}}%
\pgfpathlineto{\pgfqpoint{1.915976in}{1.384245in}}%
\pgfpathlineto{\pgfqpoint{1.946959in}{1.410603in}}%
\pgfpathlineto{\pgfqpoint{1.977941in}{1.444081in}}%
\pgfpathlineto{\pgfqpoint{2.008923in}{1.479431in}}%
\pgfpathlineto{\pgfqpoint{2.039905in}{1.516979in}}%
\pgfpathlineto{\pgfqpoint{2.070888in}{1.477337in}}%
\pgfpathlineto{\pgfqpoint{2.101870in}{1.531538in}}%
\pgfpathlineto{\pgfqpoint{2.132852in}{1.514954in}}%
\pgfpathlineto{\pgfqpoint{2.163834in}{1.596954in}}%
\pgfpathlineto{\pgfqpoint{2.194817in}{1.517516in}}%
\pgfpathlineto{\pgfqpoint{2.225799in}{1.567792in}}%
\pgfpathlineto{\pgfqpoint{2.256781in}{1.573434in}}%
\pgfpathlineto{\pgfqpoint{2.287763in}{1.555059in}}%
\pgfpathlineto{\pgfqpoint{2.318745in}{1.672444in}}%
\pgfpathlineto{\pgfqpoint{2.349728in}{1.593445in}}%
\pgfpathlineto{\pgfqpoint{2.380710in}{1.646499in}}%
\pgfpathlineto{\pgfqpoint{2.411692in}{1.653124in}}%
\pgfpathlineto{\pgfqpoint{2.442674in}{1.621053in}}%
\pgfpathlineto{\pgfqpoint{2.473657in}{1.690109in}}%
\pgfpathlineto{\pgfqpoint{2.504639in}{1.676702in}}%
\pgfpathlineto{\pgfqpoint{2.535621in}{1.657129in}}%
\pgfpathlineto{\pgfqpoint{2.566603in}{1.680399in}}%
\pgfpathlineto{\pgfqpoint{2.597586in}{1.759926in}}%
\pgfpathlineto{\pgfqpoint{2.628568in}{1.729608in}}%
\pgfpathlineto{\pgfqpoint{2.659550in}{1.720094in}}%
\pgfpathlineto{\pgfqpoint{2.690532in}{1.765607in}}%
\pgfpathlineto{\pgfqpoint{2.721515in}{1.688406in}}%
\pgfpathlineto{\pgfqpoint{2.752497in}{1.758024in}}%
\pgfpathlineto{\pgfqpoint{2.783479in}{1.853765in}}%
\pgfpathlineto{\pgfqpoint{2.814461in}{1.811297in}}%
\pgfpathlineto{\pgfqpoint{2.845444in}{1.754563in}}%
\pgfpathlineto{\pgfqpoint{2.876426in}{1.801915in}}%
\pgfpathlineto{\pgfqpoint{2.907408in}{1.877918in}}%
\pgfpathlineto{\pgfqpoint{2.938390in}{1.820685in}}%
\pgfpathlineto{\pgfqpoint{2.969373in}{1.870225in}}%
\pgfpathlineto{\pgfqpoint{3.000355in}{1.876714in}}%
\pgfpathlineto{\pgfqpoint{3.031337in}{1.898651in}}%
\pgfpathlineto{\pgfqpoint{3.062319in}{1.901168in}}%
\pgfpathlineto{\pgfqpoint{3.093302in}{1.920729in}}%
\pgfpathlineto{\pgfqpoint{3.124284in}{1.928749in}}%
\pgfpathlineto{\pgfqpoint{3.155266in}{1.917962in}}%
\pgfpathlineto{\pgfqpoint{3.186248in}{1.932897in}}%
\pgfpathlineto{\pgfqpoint{3.217231in}{1.950365in}}%
\pgfpathlineto{\pgfqpoint{3.248213in}{2.027114in}}%
\pgfpathlineto{\pgfqpoint{3.279195in}{1.958314in}}%
\pgfpathlineto{\pgfqpoint{3.310177in}{2.004089in}}%
\pgfpathlineto{\pgfqpoint{3.341159in}{2.067937in}}%
\pgfpathlineto{\pgfqpoint{3.372142in}{1.989417in}}%
\pgfpathlineto{\pgfqpoint{3.403124in}{2.028429in}}%
\pgfpathlineto{\pgfqpoint{3.434106in}{2.031285in}}%
\pgfpathlineto{\pgfqpoint{3.465088in}{2.061408in}}%
\pgfpathlineto{\pgfqpoint{3.496071in}{2.073800in}}%
\pgfpathlineto{\pgfqpoint{3.527053in}{2.094742in}}%
\pgfpathlineto{\pgfqpoint{3.558035in}{2.105238in}}%
\pgfpathlineto{\pgfqpoint{3.589017in}{2.117203in}}%
\pgfpathlineto{\pgfqpoint{3.620000in}{2.124160in}}%
\pgfpathlineto{\pgfqpoint{3.650982in}{2.114577in}}%
\pgfpathlineto{\pgfqpoint{3.681964in}{2.150342in}}%
\pgfpathlineto{\pgfqpoint{3.712946in}{2.142213in}}%
\pgfpathlineto{\pgfqpoint{3.743929in}{2.156105in}}%
\pgfpathlineto{\pgfqpoint{3.774911in}{2.149741in}}%
\pgfpathlineto{\pgfqpoint{3.805893in}{2.196163in}}%
\pgfpathlineto{\pgfqpoint{3.836875in}{2.199808in}}%
\pgfpathlineto{\pgfqpoint{3.867858in}{2.199815in}}%
\pgfpathlineto{\pgfqpoint{3.898840in}{2.234411in}}%
\pgfpathlineto{\pgfqpoint{3.929822in}{2.198118in}}%
\pgfpathlineto{\pgfqpoint{3.960804in}{2.201183in}}%
\pgfpathlineto{\pgfqpoint{3.991787in}{2.227991in}}%
\pgfpathlineto{\pgfqpoint{4.022769in}{2.266886in}}%
\pgfpathlineto{\pgfqpoint{4.053751in}{2.256867in}}%
\pgfpathlineto{\pgfqpoint{4.084733in}{2.239540in}}%
\pgfpathlineto{\pgfqpoint{4.115716in}{2.275355in}}%
\pgfpathlineto{\pgfqpoint{4.146698in}{2.272008in}}%
\pgfpathlineto{\pgfqpoint{4.177680in}{2.307553in}}%
\pgfpathlineto{\pgfqpoint{4.208662in}{2.304112in}}%
\pgfpathlineto{\pgfqpoint{4.239645in}{2.327211in}}%
\pgfpathlineto{\pgfqpoint{4.270627in}{2.330477in}}%
\pgfpathlineto{\pgfqpoint{4.301609in}{2.296999in}}%
\pgfpathlineto{\pgfqpoint{4.332591in}{2.363548in}}%
\pgfpathlineto{\pgfqpoint{4.363573in}{2.340578in}}%
\pgfpathlineto{\pgfqpoint{4.394556in}{2.370427in}}%
\pgfpathlineto{\pgfqpoint{4.425538in}{2.348825in}}%
\pgfpathlineto{\pgfqpoint{4.456520in}{2.366022in}}%
\pgfpathlineto{\pgfqpoint{4.487502in}{2.358785in}}%
\pgfpathlineto{\pgfqpoint{4.518485in}{2.380775in}}%
\pgfpathlineto{\pgfqpoint{4.549467in}{2.390802in}}%
\pgfpathlineto{\pgfqpoint{4.580449in}{2.374015in}}%
\pgfpathlineto{\pgfqpoint{4.611431in}{2.386299in}}%
\pgfpathlineto{\pgfqpoint{4.642414in}{2.390182in}}%
\pgfpathlineto{\pgfqpoint{4.673396in}{2.409958in}}%
\pgfpathlineto{\pgfqpoint{4.704378in}{2.396663in}}%
\pgfpathlineto{\pgfqpoint{4.735360in}{2.394954in}}%
\pgfpathlineto{\pgfqpoint{4.766343in}{2.434095in}}%
\pgfpathlineto{\pgfqpoint{4.797325in}{2.408055in}}%
\pgfpathlineto{\pgfqpoint{4.828307in}{2.394472in}}%
\pgfpathlineto{\pgfqpoint{4.859289in}{2.407702in}}%
\pgfpathlineto{\pgfqpoint{4.921254in}{2.364693in}}%
\pgfpathlineto{\pgfqpoint{4.952236in}{2.420752in}}%
\pgfpathlineto{\pgfqpoint{4.952236in}{2.420752in}}%
\pgfusepath{stroke}%
\end{pgfscope}%
\begin{pgfscope}%
\pgfpathrectangle{\pgfqpoint{0.588387in}{0.521603in}}{\pgfqpoint{4.669024in}{2.010285in}}%
\pgfusepath{clip}%
\pgfsetrectcap%
\pgfsetroundjoin%
\pgfsetlinewidth{1.505625pt}%
\pgfsetstrokecolor{currentstroke4}%
\pgfsetdash{}{0pt}%
\pgfpathmoveto{\pgfqpoint{0.800616in}{1.005790in}}%
\pgfpathlineto{\pgfqpoint{0.831598in}{0.618596in}}%
\pgfpathlineto{\pgfqpoint{0.862580in}{0.695434in}}%
\pgfpathlineto{\pgfqpoint{0.893562in}{0.796368in}}%
\pgfpathlineto{\pgfqpoint{0.924545in}{0.898074in}}%
\pgfpathlineto{\pgfqpoint{0.955527in}{0.954664in}}%
\pgfpathlineto{\pgfqpoint{0.986509in}{0.962348in}}%
\pgfpathlineto{\pgfqpoint{1.017491in}{1.044394in}}%
\pgfpathlineto{\pgfqpoint{1.048474in}{1.093320in}}%
\pgfpathlineto{\pgfqpoint{1.079456in}{0.998984in}}%
\pgfpathlineto{\pgfqpoint{1.110438in}{1.029194in}}%
\pgfpathlineto{\pgfqpoint{1.141420in}{0.976001in}}%
\pgfpathlineto{\pgfqpoint{1.172402in}{1.045172in}}%
\pgfpathlineto{\pgfqpoint{1.203385in}{0.983592in}}%
\pgfpathlineto{\pgfqpoint{1.234367in}{1.073233in}}%
\pgfpathlineto{\pgfqpoint{1.265349in}{1.081050in}}%
\pgfpathlineto{\pgfqpoint{1.296331in}{1.105044in}}%
\pgfpathlineto{\pgfqpoint{1.327314in}{1.067900in}}%
\pgfpathlineto{\pgfqpoint{1.358296in}{1.060192in}}%
\pgfpathlineto{\pgfqpoint{1.389278in}{1.057755in}}%
\pgfpathlineto{\pgfqpoint{1.420260in}{1.149015in}}%
\pgfpathlineto{\pgfqpoint{1.451243in}{1.168613in}}%
\pgfpathlineto{\pgfqpoint{1.482225in}{1.096912in}}%
\pgfpathlineto{\pgfqpoint{1.513207in}{1.162667in}}%
\pgfpathlineto{\pgfqpoint{1.544189in}{1.121673in}}%
\pgfpathlineto{\pgfqpoint{1.575172in}{1.216879in}}%
\pgfpathlineto{\pgfqpoint{1.606154in}{1.203302in}}%
\pgfpathlineto{\pgfqpoint{1.637136in}{1.196818in}}%
\pgfpathlineto{\pgfqpoint{1.668118in}{1.188952in}}%
\pgfpathlineto{\pgfqpoint{1.699101in}{1.324339in}}%
\pgfpathlineto{\pgfqpoint{1.730083in}{1.269800in}}%
\pgfpathlineto{\pgfqpoint{1.761065in}{1.285518in}}%
\pgfpathlineto{\pgfqpoint{1.792047in}{1.328185in}}%
\pgfpathlineto{\pgfqpoint{1.823030in}{1.275107in}}%
\pgfpathlineto{\pgfqpoint{1.854012in}{1.343024in}}%
\pgfpathlineto{\pgfqpoint{1.884994in}{1.312350in}}%
\pgfpathlineto{\pgfqpoint{1.915976in}{1.322354in}}%
\pgfpathlineto{\pgfqpoint{1.946959in}{1.308894in}}%
\pgfpathlineto{\pgfqpoint{1.977941in}{1.371835in}}%
\pgfpathlineto{\pgfqpoint{2.008923in}{1.376511in}}%
\pgfpathlineto{\pgfqpoint{2.039905in}{1.419812in}}%
\pgfpathlineto{\pgfqpoint{2.070888in}{1.395636in}}%
\pgfpathlineto{\pgfqpoint{2.101870in}{1.430245in}}%
\pgfpathlineto{\pgfqpoint{2.132852in}{1.409183in}}%
\pgfpathlineto{\pgfqpoint{2.163834in}{1.477654in}}%
\pgfpathlineto{\pgfqpoint{2.194817in}{1.432128in}}%
\pgfpathlineto{\pgfqpoint{2.225799in}{1.484316in}}%
\pgfpathlineto{\pgfqpoint{2.256781in}{1.510185in}}%
\pgfpathlineto{\pgfqpoint{2.287763in}{1.493986in}}%
\pgfpathlineto{\pgfqpoint{2.318745in}{1.573766in}}%
\pgfpathlineto{\pgfqpoint{2.349728in}{1.500744in}}%
\pgfpathlineto{\pgfqpoint{2.380710in}{1.590172in}}%
\pgfpathlineto{\pgfqpoint{2.411692in}{1.588574in}}%
\pgfpathlineto{\pgfqpoint{2.442674in}{1.555527in}}%
\pgfpathlineto{\pgfqpoint{2.473657in}{1.622760in}}%
\pgfpathlineto{\pgfqpoint{2.504639in}{1.604986in}}%
\pgfpathlineto{\pgfqpoint{2.535621in}{1.609687in}}%
\pgfpathlineto{\pgfqpoint{2.566603in}{1.596665in}}%
\pgfpathlineto{\pgfqpoint{2.597586in}{1.670138in}}%
\pgfpathlineto{\pgfqpoint{2.628568in}{1.668859in}}%
\pgfpathlineto{\pgfqpoint{2.659550in}{1.643855in}}%
\pgfpathlineto{\pgfqpoint{2.690532in}{1.673362in}}%
\pgfpathlineto{\pgfqpoint{2.721515in}{1.658589in}}%
\pgfpathlineto{\pgfqpoint{2.752497in}{1.705630in}}%
\pgfpathlineto{\pgfqpoint{2.783479in}{1.734191in}}%
\pgfpathlineto{\pgfqpoint{2.814461in}{1.759244in}}%
\pgfpathlineto{\pgfqpoint{2.845444in}{1.697377in}}%
\pgfpathlineto{\pgfqpoint{2.876426in}{1.750388in}}%
\pgfpathlineto{\pgfqpoint{2.907408in}{1.779174in}}%
\pgfpathlineto{\pgfqpoint{2.938390in}{1.799147in}}%
\pgfpathlineto{\pgfqpoint{2.969373in}{1.805843in}}%
\pgfpathlineto{\pgfqpoint{3.000355in}{1.783682in}}%
\pgfpathlineto{\pgfqpoint{3.031337in}{1.800500in}}%
\pgfpathlineto{\pgfqpoint{3.062319in}{1.800119in}}%
\pgfpathlineto{\pgfqpoint{3.093302in}{1.912493in}}%
\pgfpathlineto{\pgfqpoint{3.124284in}{1.886240in}}%
\pgfpathlineto{\pgfqpoint{3.155266in}{1.879652in}}%
\pgfpathlineto{\pgfqpoint{3.186248in}{1.894156in}}%
\pgfpathlineto{\pgfqpoint{3.217231in}{1.923313in}}%
\pgfpathlineto{\pgfqpoint{3.248213in}{1.991048in}}%
\pgfpathlineto{\pgfqpoint{3.279195in}{1.912410in}}%
\pgfpathlineto{\pgfqpoint{3.310177in}{1.941681in}}%
\pgfpathlineto{\pgfqpoint{3.341159in}{2.018573in}}%
\pgfpathlineto{\pgfqpoint{3.372142in}{1.957215in}}%
\pgfpathlineto{\pgfqpoint{3.403124in}{1.992278in}}%
\pgfpathlineto{\pgfqpoint{3.434106in}{1.993416in}}%
\pgfpathlineto{\pgfqpoint{3.465088in}{2.049153in}}%
\pgfpathlineto{\pgfqpoint{3.496071in}{2.060018in}}%
\pgfpathlineto{\pgfqpoint{3.527053in}{2.033505in}}%
\pgfpathlineto{\pgfqpoint{3.558035in}{2.128052in}}%
\pgfpathlineto{\pgfqpoint{3.589017in}{2.074241in}}%
\pgfpathlineto{\pgfqpoint{3.620000in}{2.101114in}}%
\pgfpathlineto{\pgfqpoint{3.650982in}{2.129259in}}%
\pgfpathlineto{\pgfqpoint{3.681964in}{2.127229in}}%
\pgfpathlineto{\pgfqpoint{3.712946in}{2.107945in}}%
\pgfpathlineto{\pgfqpoint{3.743929in}{2.120329in}}%
\pgfpathlineto{\pgfqpoint{3.774911in}{2.163294in}}%
\pgfpathlineto{\pgfqpoint{3.805893in}{2.210195in}}%
\pgfpathlineto{\pgfqpoint{3.836875in}{2.187593in}}%
\pgfpathlineto{\pgfqpoint{3.867858in}{2.203769in}}%
\pgfpathlineto{\pgfqpoint{3.898840in}{2.245686in}}%
\pgfpathlineto{\pgfqpoint{3.929822in}{2.193235in}}%
\pgfpathlineto{\pgfqpoint{3.960804in}{2.173724in}}%
\pgfpathlineto{\pgfqpoint{3.991787in}{2.211325in}}%
\pgfpathlineto{\pgfqpoint{4.022769in}{2.244933in}}%
\pgfpathlineto{\pgfqpoint{4.053751in}{2.250791in}}%
\pgfpathlineto{\pgfqpoint{4.084733in}{2.236233in}}%
\pgfpathlineto{\pgfqpoint{4.115716in}{2.210161in}}%
\pgfpathlineto{\pgfqpoint{4.146698in}{2.251107in}}%
\pgfpathlineto{\pgfqpoint{4.177680in}{2.313877in}}%
\pgfpathlineto{\pgfqpoint{4.208662in}{2.292584in}}%
\pgfpathlineto{\pgfqpoint{4.239645in}{2.331722in}}%
\pgfpathlineto{\pgfqpoint{4.270627in}{2.368698in}}%
\pgfpathlineto{\pgfqpoint{4.301609in}{2.361160in}}%
\pgfpathlineto{\pgfqpoint{4.332591in}{2.349683in}}%
\pgfpathlineto{\pgfqpoint{4.363573in}{2.340017in}}%
\pgfpathlineto{\pgfqpoint{4.394556in}{2.392274in}}%
\pgfpathlineto{\pgfqpoint{4.425538in}{2.362648in}}%
\pgfpathlineto{\pgfqpoint{4.456520in}{2.351361in}}%
\pgfpathlineto{\pgfqpoint{4.487502in}{2.349001in}}%
\pgfpathlineto{\pgfqpoint{4.518485in}{2.422037in}}%
\pgfpathlineto{\pgfqpoint{4.549467in}{2.403483in}}%
\pgfpathlineto{\pgfqpoint{4.580449in}{2.401854in}}%
\pgfpathlineto{\pgfqpoint{4.642414in}{2.419855in}}%
\pgfpathlineto{\pgfqpoint{4.704378in}{2.342975in}}%
\pgfusepath{stroke}%
\end{pgfscope}%
\begin{pgfscope}%
\pgfpathrectangle{\pgfqpoint{0.588387in}{0.521603in}}{\pgfqpoint{4.669024in}{2.010285in}}%
\pgfusepath{clip}%
\pgfsetrectcap%
\pgfsetroundjoin%
\pgfsetlinewidth{1.505625pt}%
\pgfsetstrokecolor{currentstroke5}%
\pgfsetdash{}{0pt}%
\pgfpathmoveto{\pgfqpoint{0.800616in}{0.998890in}}%
\pgfpathlineto{\pgfqpoint{0.831598in}{0.616571in}}%
\pgfpathlineto{\pgfqpoint{0.862580in}{0.710808in}}%
\pgfpathlineto{\pgfqpoint{0.893562in}{0.811516in}}%
\pgfpathlineto{\pgfqpoint{0.924545in}{0.935744in}}%
\pgfpathlineto{\pgfqpoint{0.955527in}{0.959563in}}%
\pgfpathlineto{\pgfqpoint{0.986509in}{0.944193in}}%
\pgfpathlineto{\pgfqpoint{1.017491in}{1.025016in}}%
\pgfpathlineto{\pgfqpoint{1.048474in}{1.024726in}}%
\pgfpathlineto{\pgfqpoint{1.079456in}{0.977980in}}%
\pgfpathlineto{\pgfqpoint{1.110438in}{0.989550in}}%
\pgfpathlineto{\pgfqpoint{1.141420in}{0.880082in}}%
\pgfpathlineto{\pgfqpoint{1.172402in}{0.981910in}}%
\pgfpathlineto{\pgfqpoint{1.203385in}{0.956082in}}%
\pgfpathlineto{\pgfqpoint{1.234367in}{1.027250in}}%
\pgfpathlineto{\pgfqpoint{1.265349in}{1.023190in}}%
\pgfpathlineto{\pgfqpoint{1.296331in}{1.072331in}}%
\pgfpathlineto{\pgfqpoint{1.327314in}{1.016227in}}%
\pgfpathlineto{\pgfqpoint{1.358296in}{1.017771in}}%
\pgfpathlineto{\pgfqpoint{1.389278in}{1.018215in}}%
\pgfpathlineto{\pgfqpoint{1.420260in}{1.058925in}}%
\pgfpathlineto{\pgfqpoint{1.451243in}{1.117386in}}%
\pgfpathlineto{\pgfqpoint{1.482225in}{1.032505in}}%
\pgfpathlineto{\pgfqpoint{1.513207in}{1.087020in}}%
\pgfpathlineto{\pgfqpoint{1.544189in}{1.088697in}}%
\pgfpathlineto{\pgfqpoint{1.575172in}{1.136993in}}%
\pgfpathlineto{\pgfqpoint{1.606154in}{1.114602in}}%
\pgfpathlineto{\pgfqpoint{1.637136in}{1.126095in}}%
\pgfpathlineto{\pgfqpoint{1.668118in}{1.147325in}}%
\pgfpathlineto{\pgfqpoint{1.699101in}{1.227297in}}%
\pgfpathlineto{\pgfqpoint{1.730083in}{1.202857in}}%
\pgfpathlineto{\pgfqpoint{1.761065in}{1.236936in}}%
\pgfpathlineto{\pgfqpoint{1.792047in}{1.272486in}}%
\pgfpathlineto{\pgfqpoint{1.823030in}{1.198071in}}%
\pgfpathlineto{\pgfqpoint{1.854012in}{1.283791in}}%
\pgfpathlineto{\pgfqpoint{1.884994in}{1.267310in}}%
\pgfpathlineto{\pgfqpoint{1.915976in}{1.280881in}}%
\pgfpathlineto{\pgfqpoint{1.946959in}{1.216213in}}%
\pgfpathlineto{\pgfqpoint{1.977941in}{1.313817in}}%
\pgfpathlineto{\pgfqpoint{2.008923in}{1.318386in}}%
\pgfpathlineto{\pgfqpoint{2.039905in}{1.369209in}}%
\pgfpathlineto{\pgfqpoint{2.070888in}{1.331319in}}%
\pgfpathlineto{\pgfqpoint{2.101870in}{1.355959in}}%
\pgfpathlineto{\pgfqpoint{2.132852in}{1.340320in}}%
\pgfpathlineto{\pgfqpoint{2.163834in}{1.399411in}}%
\pgfpathlineto{\pgfqpoint{2.194817in}{1.413594in}}%
\pgfpathlineto{\pgfqpoint{2.225799in}{1.403841in}}%
\pgfpathlineto{\pgfqpoint{2.256781in}{1.460769in}}%
\pgfpathlineto{\pgfqpoint{2.287763in}{1.456032in}}%
\pgfpathlineto{\pgfqpoint{2.318745in}{1.466611in}}%
\pgfpathlineto{\pgfqpoint{2.349728in}{1.446904in}}%
\pgfpathlineto{\pgfqpoint{2.380710in}{1.551837in}}%
\pgfpathlineto{\pgfqpoint{2.411692in}{1.536442in}}%
\pgfpathlineto{\pgfqpoint{2.442674in}{1.509191in}}%
\pgfpathlineto{\pgfqpoint{2.473657in}{1.603056in}}%
\pgfpathlineto{\pgfqpoint{2.504639in}{1.576744in}}%
\pgfpathlineto{\pgfqpoint{2.535621in}{1.593372in}}%
\pgfpathlineto{\pgfqpoint{2.566603in}{1.571567in}}%
\pgfpathlineto{\pgfqpoint{2.597586in}{1.601928in}}%
\pgfpathlineto{\pgfqpoint{2.628568in}{1.648495in}}%
\pgfpathlineto{\pgfqpoint{2.659550in}{1.609402in}}%
\pgfpathlineto{\pgfqpoint{2.690532in}{1.651893in}}%
\pgfpathlineto{\pgfqpoint{2.721515in}{1.618835in}}%
\pgfpathlineto{\pgfqpoint{2.752497in}{1.699210in}}%
\pgfpathlineto{\pgfqpoint{2.783479in}{1.754568in}}%
\pgfpathlineto{\pgfqpoint{2.814461in}{1.728787in}}%
\pgfpathlineto{\pgfqpoint{2.845444in}{1.655294in}}%
\pgfpathlineto{\pgfqpoint{2.876426in}{1.693116in}}%
\pgfpathlineto{\pgfqpoint{2.907408in}{1.751705in}}%
\pgfpathlineto{\pgfqpoint{2.938390in}{1.783621in}}%
\pgfpathlineto{\pgfqpoint{2.969373in}{1.752960in}}%
\pgfpathlineto{\pgfqpoint{3.000355in}{1.748885in}}%
\pgfpathlineto{\pgfqpoint{3.031337in}{1.714640in}}%
\pgfpathlineto{\pgfqpoint{3.062319in}{1.783719in}}%
\pgfpathlineto{\pgfqpoint{3.093302in}{1.898946in}}%
\pgfpathlineto{\pgfqpoint{3.124284in}{1.898477in}}%
\pgfpathlineto{\pgfqpoint{3.155266in}{1.898162in}}%
\pgfpathlineto{\pgfqpoint{3.186248in}{1.875056in}}%
\pgfpathlineto{\pgfqpoint{3.217231in}{1.921723in}}%
\pgfpathlineto{\pgfqpoint{3.248213in}{2.013579in}}%
\pgfpathlineto{\pgfqpoint{3.279195in}{1.928690in}}%
\pgfpathlineto{\pgfqpoint{3.310177in}{1.938455in}}%
\pgfpathlineto{\pgfqpoint{3.341159in}{2.041313in}}%
\pgfpathlineto{\pgfqpoint{3.372142in}{1.945819in}}%
\pgfpathlineto{\pgfqpoint{3.403124in}{1.987272in}}%
\pgfpathlineto{\pgfqpoint{3.434106in}{2.000886in}}%
\pgfpathlineto{\pgfqpoint{3.465088in}{2.054493in}}%
\pgfpathlineto{\pgfqpoint{3.496071in}{2.112473in}}%
\pgfpathlineto{\pgfqpoint{3.527053in}{2.029991in}}%
\pgfpathlineto{\pgfqpoint{3.558035in}{2.144582in}}%
\pgfpathlineto{\pgfqpoint{3.589017in}{2.031125in}}%
\pgfpathlineto{\pgfqpoint{3.620000in}{2.123629in}}%
\pgfpathlineto{\pgfqpoint{3.650982in}{2.168778in}}%
\pgfpathlineto{\pgfqpoint{3.681964in}{2.088643in}}%
\pgfpathlineto{\pgfqpoint{3.712946in}{2.147281in}}%
\pgfpathlineto{\pgfqpoint{3.774911in}{2.177823in}}%
\pgfpathlineto{\pgfqpoint{3.805893in}{2.359075in}}%
\pgfpathlineto{\pgfqpoint{3.836875in}{2.210978in}}%
\pgfpathlineto{\pgfqpoint{3.867858in}{2.234027in}}%
\pgfpathlineto{\pgfqpoint{3.898840in}{2.259117in}}%
\pgfpathlineto{\pgfqpoint{4.053751in}{2.266566in}}%
\pgfpathlineto{\pgfqpoint{4.177680in}{2.334780in}}%
\pgfpathlineto{\pgfqpoint{4.394556in}{2.405366in}}%
\pgfpathlineto{\pgfqpoint{4.518485in}{2.409888in}}%
\pgfusepath{stroke}%
\end{pgfscope}%
\begin{pgfscope}%
\pgfpathrectangle{\pgfqpoint{0.588387in}{0.521603in}}{\pgfqpoint{4.669024in}{2.010285in}}%
\pgfusepath{clip}%
\pgfsetrectcap%
\pgfsetroundjoin%
\pgfsetlinewidth{1.505625pt}%
\pgfsetstrokecolor{currentstroke6}%
\pgfsetdash{}{0pt}%
\pgfpathmoveto{\pgfqpoint{0.800616in}{1.001144in}}%
\pgfpathlineto{\pgfqpoint{0.831598in}{0.615078in}}%
\pgfpathlineto{\pgfqpoint{0.862580in}{0.683389in}}%
\pgfpathlineto{\pgfqpoint{0.893562in}{0.782188in}}%
\pgfpathlineto{\pgfqpoint{0.924545in}{0.895010in}}%
\pgfpathlineto{\pgfqpoint{0.955527in}{0.944540in}}%
\pgfpathlineto{\pgfqpoint{0.986509in}{0.946691in}}%
\pgfpathlineto{\pgfqpoint{1.017491in}{1.016260in}}%
\pgfpathlineto{\pgfqpoint{1.048474in}{1.032133in}}%
\pgfpathlineto{\pgfqpoint{1.079456in}{0.959323in}}%
\pgfpathlineto{\pgfqpoint{1.110438in}{0.963747in}}%
\pgfpathlineto{\pgfqpoint{1.141420in}{0.881193in}}%
\pgfpathlineto{\pgfqpoint{1.172402in}{0.975399in}}%
\pgfpathlineto{\pgfqpoint{1.203385in}{0.937430in}}%
\pgfpathlineto{\pgfqpoint{1.234367in}{1.013958in}}%
\pgfpathlineto{\pgfqpoint{1.265349in}{1.021905in}}%
\pgfpathlineto{\pgfqpoint{1.296331in}{1.074638in}}%
\pgfpathlineto{\pgfqpoint{1.327314in}{1.027643in}}%
\pgfpathlineto{\pgfqpoint{1.358296in}{1.017132in}}%
\pgfpathlineto{\pgfqpoint{1.389278in}{1.006528in}}%
\pgfpathlineto{\pgfqpoint{1.420260in}{1.049441in}}%
\pgfpathlineto{\pgfqpoint{1.451243in}{1.116492in}}%
\pgfpathlineto{\pgfqpoint{1.482225in}{1.027703in}}%
\pgfpathlineto{\pgfqpoint{1.513207in}{1.098460in}}%
\pgfpathlineto{\pgfqpoint{1.544189in}{1.082261in}}%
\pgfpathlineto{\pgfqpoint{1.575172in}{1.135898in}}%
\pgfpathlineto{\pgfqpoint{1.606154in}{1.113073in}}%
\pgfpathlineto{\pgfqpoint{1.637136in}{1.124222in}}%
\pgfpathlineto{\pgfqpoint{1.668118in}{1.138595in}}%
\pgfpathlineto{\pgfqpoint{1.699101in}{1.247857in}}%
\pgfpathlineto{\pgfqpoint{1.730083in}{1.201635in}}%
\pgfpathlineto{\pgfqpoint{1.761065in}{1.233816in}}%
\pgfpathlineto{\pgfqpoint{1.792047in}{1.281715in}}%
\pgfpathlineto{\pgfqpoint{1.823030in}{1.183476in}}%
\pgfpathlineto{\pgfqpoint{1.854012in}{1.290661in}}%
\pgfpathlineto{\pgfqpoint{1.884994in}{1.278100in}}%
\pgfpathlineto{\pgfqpoint{1.915976in}{1.267305in}}%
\pgfpathlineto{\pgfqpoint{1.946959in}{1.213423in}}%
\pgfpathlineto{\pgfqpoint{1.977941in}{1.294652in}}%
\pgfpathlineto{\pgfqpoint{2.008923in}{1.323185in}}%
\pgfpathlineto{\pgfqpoint{2.039905in}{1.372877in}}%
\pgfpathlineto{\pgfqpoint{2.070888in}{1.343589in}}%
\pgfpathlineto{\pgfqpoint{2.101870in}{1.345707in}}%
\pgfpathlineto{\pgfqpoint{2.132852in}{1.335617in}}%
\pgfpathlineto{\pgfqpoint{2.163834in}{1.383001in}}%
\pgfpathlineto{\pgfqpoint{2.194817in}{1.400320in}}%
\pgfpathlineto{\pgfqpoint{2.225799in}{1.394698in}}%
\pgfpathlineto{\pgfqpoint{2.256781in}{1.462234in}}%
\pgfpathlineto{\pgfqpoint{2.287763in}{1.450758in}}%
\pgfpathlineto{\pgfqpoint{2.318745in}{1.449541in}}%
\pgfpathlineto{\pgfqpoint{2.349728in}{1.440065in}}%
\pgfpathlineto{\pgfqpoint{2.380710in}{1.565046in}}%
\pgfpathlineto{\pgfqpoint{2.411692in}{1.527321in}}%
\pgfpathlineto{\pgfqpoint{2.442674in}{1.494996in}}%
\pgfpathlineto{\pgfqpoint{2.473657in}{1.601658in}}%
\pgfpathlineto{\pgfqpoint{2.504639in}{1.576578in}}%
\pgfpathlineto{\pgfqpoint{2.535621in}{1.579498in}}%
\pgfpathlineto{\pgfqpoint{2.566603in}{1.571125in}}%
\pgfpathlineto{\pgfqpoint{2.597586in}{1.584742in}}%
\pgfpathlineto{\pgfqpoint{2.628568in}{1.641162in}}%
\pgfpathlineto{\pgfqpoint{2.659550in}{1.596256in}}%
\pgfpathlineto{\pgfqpoint{2.690532in}{1.659770in}}%
\pgfpathlineto{\pgfqpoint{2.721515in}{1.606101in}}%
\pgfpathlineto{\pgfqpoint{2.752497in}{1.681371in}}%
\pgfpathlineto{\pgfqpoint{2.783479in}{1.717801in}}%
\pgfpathlineto{\pgfqpoint{2.814461in}{1.705555in}}%
\pgfpathlineto{\pgfqpoint{2.845444in}{1.645519in}}%
\pgfpathlineto{\pgfqpoint{2.876426in}{1.697338in}}%
\pgfpathlineto{\pgfqpoint{2.907408in}{1.737406in}}%
\pgfpathlineto{\pgfqpoint{2.938390in}{1.780234in}}%
\pgfpathlineto{\pgfqpoint{2.969373in}{1.744555in}}%
\pgfpathlineto{\pgfqpoint{3.000355in}{1.748811in}}%
\pgfpathlineto{\pgfqpoint{3.031337in}{1.707204in}}%
\pgfpathlineto{\pgfqpoint{3.062319in}{1.787346in}}%
\pgfpathlineto{\pgfqpoint{3.093302in}{1.880749in}}%
\pgfpathlineto{\pgfqpoint{3.124284in}{1.885952in}}%
\pgfpathlineto{\pgfqpoint{3.155266in}{1.879178in}}%
\pgfpathlineto{\pgfqpoint{3.186248in}{1.863809in}}%
\pgfpathlineto{\pgfqpoint{3.217231in}{1.922424in}}%
\pgfpathlineto{\pgfqpoint{3.248213in}{2.005463in}}%
\pgfpathlineto{\pgfqpoint{3.279195in}{1.932912in}}%
\pgfpathlineto{\pgfqpoint{3.310177in}{1.923263in}}%
\pgfpathlineto{\pgfqpoint{3.341159in}{2.017487in}}%
\pgfpathlineto{\pgfqpoint{3.372142in}{1.944922in}}%
\pgfpathlineto{\pgfqpoint{3.403124in}{1.989276in}}%
\pgfpathlineto{\pgfqpoint{3.434106in}{1.970661in}}%
\pgfpathlineto{\pgfqpoint{3.465088in}{2.039441in}}%
\pgfpathlineto{\pgfqpoint{3.496071in}{2.066468in}}%
\pgfpathlineto{\pgfqpoint{3.527053in}{2.046336in}}%
\pgfpathlineto{\pgfqpoint{3.558035in}{2.121054in}}%
\pgfpathlineto{\pgfqpoint{3.589017in}{2.029336in}}%
\pgfpathlineto{\pgfqpoint{3.620000in}{2.116537in}}%
\pgfpathlineto{\pgfqpoint{3.650982in}{2.198197in}}%
\pgfpathlineto{\pgfqpoint{3.681964in}{2.089341in}}%
\pgfpathlineto{\pgfqpoint{3.712946in}{2.119523in}}%
\pgfpathlineto{\pgfqpoint{3.774911in}{2.187827in}}%
\pgfpathlineto{\pgfqpoint{3.805893in}{2.331532in}}%
\pgfpathlineto{\pgfqpoint{3.836875in}{2.193130in}}%
\pgfpathlineto{\pgfqpoint{3.867858in}{2.227409in}}%
\pgfpathlineto{\pgfqpoint{3.898840in}{2.269959in}}%
\pgfpathlineto{\pgfqpoint{4.053751in}{2.295505in}}%
\pgfpathlineto{\pgfqpoint{4.177680in}{2.320947in}}%
\pgfpathlineto{\pgfqpoint{4.394556in}{2.362469in}}%
\pgfpathlineto{\pgfqpoint{4.518485in}{2.440512in}}%
\pgfusepath{stroke}%
\end{pgfscope}%
\begin{pgfscope}%
\pgfpathrectangle{\pgfqpoint{0.588387in}{0.521603in}}{\pgfqpoint{4.669024in}{2.010285in}}%
\pgfusepath{clip}%
\pgfsetrectcap%
\pgfsetroundjoin%
\pgfsetlinewidth{1.505625pt}%
\pgfsetstrokecolor{currentstroke7}%
\pgfsetdash{}{0pt}%
\pgfpathmoveto{\pgfqpoint{0.800616in}{0.990713in}}%
\pgfpathlineto{\pgfqpoint{0.831598in}{0.613777in}}%
\pgfpathlineto{\pgfqpoint{0.862580in}{0.703430in}}%
\pgfpathlineto{\pgfqpoint{0.893562in}{0.797352in}}%
\pgfpathlineto{\pgfqpoint{0.924545in}{0.913090in}}%
\pgfpathlineto{\pgfqpoint{0.955527in}{0.950544in}}%
\pgfpathlineto{\pgfqpoint{0.986509in}{0.959270in}}%
\pgfpathlineto{\pgfqpoint{1.017491in}{1.063315in}}%
\pgfpathlineto{\pgfqpoint{1.048474in}{1.136126in}}%
\pgfpathlineto{\pgfqpoint{1.079456in}{1.048495in}}%
\pgfpathlineto{\pgfqpoint{1.110438in}{1.087448in}}%
\pgfpathlineto{\pgfqpoint{1.141420in}{1.061110in}}%
\pgfpathlineto{\pgfqpoint{1.172402in}{1.083055in}}%
\pgfpathlineto{\pgfqpoint{1.203385in}{1.045211in}}%
\pgfpathlineto{\pgfqpoint{1.234367in}{1.120849in}}%
\pgfpathlineto{\pgfqpoint{1.265349in}{1.143015in}}%
\pgfpathlineto{\pgfqpoint{1.296331in}{1.149681in}}%
\pgfpathlineto{\pgfqpoint{1.327314in}{1.118492in}}%
\pgfpathlineto{\pgfqpoint{1.358296in}{1.099833in}}%
\pgfpathlineto{\pgfqpoint{1.389278in}{1.131454in}}%
\pgfpathlineto{\pgfqpoint{1.420260in}{1.275669in}}%
\pgfpathlineto{\pgfqpoint{1.451243in}{1.238484in}}%
\pgfpathlineto{\pgfqpoint{1.482225in}{1.145275in}}%
\pgfpathlineto{\pgfqpoint{1.513207in}{1.211045in}}%
\pgfpathlineto{\pgfqpoint{1.544189in}{1.192888in}}%
\pgfpathlineto{\pgfqpoint{1.575172in}{1.318228in}}%
\pgfpathlineto{\pgfqpoint{1.606154in}{1.283613in}}%
\pgfpathlineto{\pgfqpoint{1.637136in}{1.285466in}}%
\pgfpathlineto{\pgfqpoint{1.668118in}{1.262872in}}%
\pgfpathlineto{\pgfqpoint{1.699101in}{1.381557in}}%
\pgfpathlineto{\pgfqpoint{1.730083in}{1.335564in}}%
\pgfpathlineto{\pgfqpoint{1.761065in}{1.362004in}}%
\pgfpathlineto{\pgfqpoint{1.792047in}{1.383488in}}%
\pgfpathlineto{\pgfqpoint{1.823030in}{1.371620in}}%
\pgfpathlineto{\pgfqpoint{1.854012in}{1.420443in}}%
\pgfpathlineto{\pgfqpoint{1.884994in}{1.366014in}}%
\pgfpathlineto{\pgfqpoint{1.915976in}{1.371167in}}%
\pgfpathlineto{\pgfqpoint{1.946959in}{1.394952in}}%
\pgfpathlineto{\pgfqpoint{1.977941in}{1.423327in}}%
\pgfpathlineto{\pgfqpoint{2.008923in}{1.473343in}}%
\pgfpathlineto{\pgfqpoint{2.039905in}{1.499874in}}%
\pgfpathlineto{\pgfqpoint{2.070888in}{1.457305in}}%
\pgfpathlineto{\pgfqpoint{2.101870in}{1.514059in}}%
\pgfpathlineto{\pgfqpoint{2.132852in}{1.493101in}}%
\pgfpathlineto{\pgfqpoint{2.163834in}{1.578191in}}%
\pgfpathlineto{\pgfqpoint{2.194817in}{1.491846in}}%
\pgfpathlineto{\pgfqpoint{2.225799in}{1.552228in}}%
\pgfpathlineto{\pgfqpoint{2.256781in}{1.562068in}}%
\pgfpathlineto{\pgfqpoint{2.287763in}{1.538949in}}%
\pgfpathlineto{\pgfqpoint{2.318745in}{1.657504in}}%
\pgfpathlineto{\pgfqpoint{2.349728in}{1.586534in}}%
\pgfpathlineto{\pgfqpoint{2.380710in}{1.625885in}}%
\pgfpathlineto{\pgfqpoint{2.411692in}{1.633020in}}%
\pgfpathlineto{\pgfqpoint{2.442674in}{1.615568in}}%
\pgfpathlineto{\pgfqpoint{2.473657in}{1.646410in}}%
\pgfpathlineto{\pgfqpoint{2.504639in}{1.653763in}}%
\pgfpathlineto{\pgfqpoint{2.535621in}{1.639374in}}%
\pgfpathlineto{\pgfqpoint{2.566603in}{1.663457in}}%
\pgfpathlineto{\pgfqpoint{2.597586in}{1.745915in}}%
\pgfpathlineto{\pgfqpoint{2.628568in}{1.701155in}}%
\pgfpathlineto{\pgfqpoint{2.659550in}{1.708036in}}%
\pgfpathlineto{\pgfqpoint{2.690532in}{1.744540in}}%
\pgfpathlineto{\pgfqpoint{2.721515in}{1.665656in}}%
\pgfpathlineto{\pgfqpoint{2.752497in}{1.729535in}}%
\pgfpathlineto{\pgfqpoint{2.783479in}{1.831178in}}%
\pgfpathlineto{\pgfqpoint{2.814461in}{1.799783in}}%
\pgfpathlineto{\pgfqpoint{2.845444in}{1.733523in}}%
\pgfpathlineto{\pgfqpoint{2.876426in}{1.773486in}}%
\pgfpathlineto{\pgfqpoint{2.907408in}{1.853323in}}%
\pgfpathlineto{\pgfqpoint{2.938390in}{1.803296in}}%
\pgfpathlineto{\pgfqpoint{2.969373in}{1.852778in}}%
\pgfpathlineto{\pgfqpoint{3.000355in}{1.834251in}}%
\pgfpathlineto{\pgfqpoint{3.031337in}{1.873727in}}%
\pgfpathlineto{\pgfqpoint{3.062319in}{1.865046in}}%
\pgfpathlineto{\pgfqpoint{3.093302in}{1.895213in}}%
\pgfpathlineto{\pgfqpoint{3.155266in}{1.894232in}}%
\pgfpathlineto{\pgfqpoint{3.186248in}{1.898904in}}%
\pgfpathlineto{\pgfqpoint{3.217231in}{1.927081in}}%
\pgfpathlineto{\pgfqpoint{3.248213in}{1.997704in}}%
\pgfpathlineto{\pgfqpoint{3.279195in}{1.935664in}}%
\pgfpathlineto{\pgfqpoint{3.310177in}{1.974762in}}%
\pgfpathlineto{\pgfqpoint{3.341159in}{2.030320in}}%
\pgfpathlineto{\pgfqpoint{3.372142in}{1.958915in}}%
\pgfpathlineto{\pgfqpoint{3.403124in}{1.986253in}}%
\pgfpathlineto{\pgfqpoint{3.434106in}{2.003072in}}%
\pgfpathlineto{\pgfqpoint{3.465088in}{2.043295in}}%
\pgfpathlineto{\pgfqpoint{3.496071in}{2.032903in}}%
\pgfpathlineto{\pgfqpoint{3.527053in}{2.043305in}}%
\pgfpathlineto{\pgfqpoint{3.558035in}{2.056437in}}%
\pgfpathlineto{\pgfqpoint{3.589017in}{2.078702in}}%
\pgfpathlineto{\pgfqpoint{3.620000in}{2.064780in}}%
\pgfpathlineto{\pgfqpoint{3.650982in}{2.094674in}}%
\pgfpathlineto{\pgfqpoint{3.681964in}{2.129033in}}%
\pgfpathlineto{\pgfqpoint{3.712946in}{2.104539in}}%
\pgfpathlineto{\pgfqpoint{3.743929in}{2.142837in}}%
\pgfpathlineto{\pgfqpoint{3.774911in}{2.128237in}}%
\pgfpathlineto{\pgfqpoint{3.805893in}{2.170888in}}%
\pgfpathlineto{\pgfqpoint{3.836875in}{2.155363in}}%
\pgfpathlineto{\pgfqpoint{3.867858in}{2.164807in}}%
\pgfpathlineto{\pgfqpoint{3.898840in}{2.225992in}}%
\pgfpathlineto{\pgfqpoint{3.929822in}{2.188299in}}%
\pgfpathlineto{\pgfqpoint{3.960804in}{2.188068in}}%
\pgfpathlineto{\pgfqpoint{3.991787in}{2.203032in}}%
\pgfpathlineto{\pgfqpoint{4.022769in}{2.224216in}}%
\pgfpathlineto{\pgfqpoint{4.053751in}{2.240102in}}%
\pgfpathlineto{\pgfqpoint{4.084733in}{2.214741in}}%
\pgfpathlineto{\pgfqpoint{4.115716in}{2.267519in}}%
\pgfpathlineto{\pgfqpoint{4.146698in}{2.254813in}}%
\pgfpathlineto{\pgfqpoint{4.177680in}{2.279596in}}%
\pgfpathlineto{\pgfqpoint{4.208662in}{2.272439in}}%
\pgfpathlineto{\pgfqpoint{4.239645in}{2.293938in}}%
\pgfpathlineto{\pgfqpoint{4.270627in}{2.309381in}}%
\pgfpathlineto{\pgfqpoint{4.301609in}{2.258224in}}%
\pgfpathlineto{\pgfqpoint{4.332591in}{2.345203in}}%
\pgfpathlineto{\pgfqpoint{4.363573in}{2.333838in}}%
\pgfpathlineto{\pgfqpoint{4.394556in}{2.347607in}}%
\pgfpathlineto{\pgfqpoint{4.425538in}{2.358505in}}%
\pgfpathlineto{\pgfqpoint{4.456520in}{2.331622in}}%
\pgfpathlineto{\pgfqpoint{4.487502in}{2.324609in}}%
\pgfpathlineto{\pgfqpoint{4.518485in}{2.367176in}}%
\pgfpathlineto{\pgfqpoint{4.549467in}{2.381160in}}%
\pgfpathlineto{\pgfqpoint{4.580449in}{2.367688in}}%
\pgfpathlineto{\pgfqpoint{4.611431in}{2.369039in}}%
\pgfpathlineto{\pgfqpoint{4.642414in}{2.391786in}}%
\pgfpathlineto{\pgfqpoint{4.673396in}{2.391334in}}%
\pgfpathlineto{\pgfqpoint{4.704378in}{2.395990in}}%
\pgfpathlineto{\pgfqpoint{4.735360in}{2.413580in}}%
\pgfpathlineto{\pgfqpoint{4.766343in}{2.400878in}}%
\pgfpathlineto{\pgfqpoint{4.797325in}{2.413757in}}%
\pgfpathlineto{\pgfqpoint{4.828307in}{2.407302in}}%
\pgfpathlineto{\pgfqpoint{4.859289in}{2.387328in}}%
\pgfpathlineto{\pgfqpoint{4.890272in}{2.397144in}}%
\pgfpathlineto{\pgfqpoint{4.921254in}{2.369039in}}%
\pgfpathlineto{\pgfqpoint{4.952236in}{2.419993in}}%
\pgfpathlineto{\pgfqpoint{4.983218in}{2.437447in}}%
\pgfpathlineto{\pgfqpoint{5.014201in}{2.435036in}}%
\pgfpathlineto{\pgfqpoint{5.045183in}{2.430384in}}%
\pgfpathlineto{\pgfqpoint{5.045183in}{2.430384in}}%
\pgfusepath{stroke}%
\end{pgfscope}%
\begin{pgfscope}%
\pgfpathrectangle{\pgfqpoint{0.588387in}{0.521603in}}{\pgfqpoint{4.669024in}{2.010285in}}%
\pgfusepath{clip}%
\pgfsetrectcap%
\pgfsetroundjoin%
\pgfsetlinewidth{1.505625pt}%
\definecolor{currentstroke}{rgb}{0.498039,0.498039,0.498039}%
\pgfsetstrokecolor{currentstroke}%
\pgfsetdash{}{0pt}%
\pgfpathmoveto{\pgfqpoint{0.800616in}{1.028687in}}%
\pgfpathlineto{\pgfqpoint{0.831598in}{0.612980in}}%
\pgfpathlineto{\pgfqpoint{0.862580in}{0.681696in}}%
\pgfpathlineto{\pgfqpoint{0.893562in}{0.778601in}}%
\pgfpathlineto{\pgfqpoint{0.924545in}{0.896793in}}%
\pgfpathlineto{\pgfqpoint{0.955527in}{0.944618in}}%
\pgfpathlineto{\pgfqpoint{0.986509in}{0.953604in}}%
\pgfpathlineto{\pgfqpoint{1.017491in}{1.040022in}}%
\pgfpathlineto{\pgfqpoint{1.048474in}{1.093796in}}%
\pgfpathlineto{\pgfqpoint{1.079456in}{1.000333in}}%
\pgfpathlineto{\pgfqpoint{1.110438in}{1.025971in}}%
\pgfpathlineto{\pgfqpoint{1.141420in}{0.970816in}}%
\pgfpathlineto{\pgfqpoint{1.172402in}{1.021619in}}%
\pgfpathlineto{\pgfqpoint{1.203385in}{0.977769in}}%
\pgfpathlineto{\pgfqpoint{1.234367in}{1.074187in}}%
\pgfpathlineto{\pgfqpoint{1.265349in}{1.078773in}}%
\pgfpathlineto{\pgfqpoint{1.296331in}{1.101797in}}%
\pgfpathlineto{\pgfqpoint{1.327314in}{1.059651in}}%
\pgfpathlineto{\pgfqpoint{1.358296in}{1.057031in}}%
\pgfpathlineto{\pgfqpoint{1.389278in}{1.058027in}}%
\pgfpathlineto{\pgfqpoint{1.420260in}{1.138955in}}%
\pgfpathlineto{\pgfqpoint{1.451243in}{1.169979in}}%
\pgfpathlineto{\pgfqpoint{1.482225in}{1.079795in}}%
\pgfpathlineto{\pgfqpoint{1.513207in}{1.144035in}}%
\pgfpathlineto{\pgfqpoint{1.544189in}{1.113722in}}%
\pgfpathlineto{\pgfqpoint{1.575172in}{1.198482in}}%
\pgfpathlineto{\pgfqpoint{1.606154in}{1.193783in}}%
\pgfpathlineto{\pgfqpoint{1.637136in}{1.179400in}}%
\pgfpathlineto{\pgfqpoint{1.668118in}{1.184820in}}%
\pgfpathlineto{\pgfqpoint{1.699101in}{1.301231in}}%
\pgfpathlineto{\pgfqpoint{1.730083in}{1.258039in}}%
\pgfpathlineto{\pgfqpoint{1.761065in}{1.276716in}}%
\pgfpathlineto{\pgfqpoint{1.792047in}{1.314627in}}%
\pgfpathlineto{\pgfqpoint{1.823030in}{1.264487in}}%
\pgfpathlineto{\pgfqpoint{1.854012in}{1.351472in}}%
\pgfpathlineto{\pgfqpoint{1.884994in}{1.286117in}}%
\pgfpathlineto{\pgfqpoint{1.915976in}{1.307831in}}%
\pgfpathlineto{\pgfqpoint{1.946959in}{1.302996in}}%
\pgfpathlineto{\pgfqpoint{1.977941in}{1.357774in}}%
\pgfpathlineto{\pgfqpoint{2.008923in}{1.367538in}}%
\pgfpathlineto{\pgfqpoint{2.039905in}{1.402391in}}%
\pgfpathlineto{\pgfqpoint{2.070888in}{1.379683in}}%
\pgfpathlineto{\pgfqpoint{2.101870in}{1.415884in}}%
\pgfpathlineto{\pgfqpoint{2.132852in}{1.383784in}}%
\pgfpathlineto{\pgfqpoint{2.163834in}{1.461848in}}%
\pgfpathlineto{\pgfqpoint{2.194817in}{1.428131in}}%
\pgfpathlineto{\pgfqpoint{2.225799in}{1.464192in}}%
\pgfpathlineto{\pgfqpoint{2.256781in}{1.495554in}}%
\pgfpathlineto{\pgfqpoint{2.287763in}{1.461346in}}%
\pgfpathlineto{\pgfqpoint{2.318745in}{1.551095in}}%
\pgfpathlineto{\pgfqpoint{2.349728in}{1.497595in}}%
\pgfpathlineto{\pgfqpoint{2.380710in}{1.574966in}}%
\pgfpathlineto{\pgfqpoint{2.411692in}{1.558882in}}%
\pgfpathlineto{\pgfqpoint{2.442674in}{1.524802in}}%
\pgfpathlineto{\pgfqpoint{2.473657in}{1.579790in}}%
\pgfpathlineto{\pgfqpoint{2.504639in}{1.585205in}}%
\pgfpathlineto{\pgfqpoint{2.535621in}{1.586819in}}%
\pgfpathlineto{\pgfqpoint{2.566603in}{1.560234in}}%
\pgfpathlineto{\pgfqpoint{2.597586in}{1.644853in}}%
\pgfpathlineto{\pgfqpoint{2.628568in}{1.639282in}}%
\pgfpathlineto{\pgfqpoint{2.659550in}{1.607609in}}%
\pgfpathlineto{\pgfqpoint{2.690532in}{1.643841in}}%
\pgfpathlineto{\pgfqpoint{2.721515in}{1.620158in}}%
\pgfpathlineto{\pgfqpoint{2.752497in}{1.669537in}}%
\pgfpathlineto{\pgfqpoint{2.783479in}{1.706205in}}%
\pgfpathlineto{\pgfqpoint{2.814461in}{1.731253in}}%
\pgfpathlineto{\pgfqpoint{2.845444in}{1.671613in}}%
\pgfpathlineto{\pgfqpoint{2.876426in}{1.721963in}}%
\pgfpathlineto{\pgfqpoint{2.907408in}{1.756394in}}%
\pgfpathlineto{\pgfqpoint{2.938390in}{1.767073in}}%
\pgfpathlineto{\pgfqpoint{2.969373in}{1.782419in}}%
\pgfpathlineto{\pgfqpoint{3.000355in}{1.789619in}}%
\pgfpathlineto{\pgfqpoint{3.031337in}{1.778894in}}%
\pgfpathlineto{\pgfqpoint{3.062319in}{1.773396in}}%
\pgfpathlineto{\pgfqpoint{3.093302in}{1.867411in}}%
\pgfpathlineto{\pgfqpoint{3.124284in}{1.836685in}}%
\pgfpathlineto{\pgfqpoint{3.155266in}{1.855891in}}%
\pgfpathlineto{\pgfqpoint{3.186248in}{1.866349in}}%
\pgfpathlineto{\pgfqpoint{3.217231in}{1.899500in}}%
\pgfpathlineto{\pgfqpoint{3.248213in}{1.960457in}}%
\pgfpathlineto{\pgfqpoint{3.279195in}{1.881033in}}%
\pgfpathlineto{\pgfqpoint{3.310177in}{1.910875in}}%
\pgfpathlineto{\pgfqpoint{3.341159in}{1.967183in}}%
\pgfpathlineto{\pgfqpoint{3.372142in}{1.932918in}}%
\pgfpathlineto{\pgfqpoint{3.403124in}{1.947687in}}%
\pgfpathlineto{\pgfqpoint{3.434106in}{1.992095in}}%
\pgfpathlineto{\pgfqpoint{3.465088in}{2.009320in}}%
\pgfpathlineto{\pgfqpoint{3.496071in}{2.037629in}}%
\pgfpathlineto{\pgfqpoint{3.527053in}{1.979388in}}%
\pgfpathlineto{\pgfqpoint{3.558035in}{2.060856in}}%
\pgfpathlineto{\pgfqpoint{3.589017in}{2.035020in}}%
\pgfpathlineto{\pgfqpoint{3.620000in}{2.067796in}}%
\pgfpathlineto{\pgfqpoint{3.650982in}{2.105999in}}%
\pgfpathlineto{\pgfqpoint{3.681964in}{2.087746in}}%
\pgfpathlineto{\pgfqpoint{3.712946in}{2.085177in}}%
\pgfpathlineto{\pgfqpoint{3.743929in}{2.104188in}}%
\pgfpathlineto{\pgfqpoint{3.774911in}{2.129282in}}%
\pgfpathlineto{\pgfqpoint{3.805893in}{2.184666in}}%
\pgfpathlineto{\pgfqpoint{3.836875in}{2.160340in}}%
\pgfpathlineto{\pgfqpoint{3.867858in}{2.165039in}}%
\pgfpathlineto{\pgfqpoint{3.898840in}{2.215720in}}%
\pgfpathlineto{\pgfqpoint{3.929822in}{2.157629in}}%
\pgfpathlineto{\pgfqpoint{3.960804in}{2.144212in}}%
\pgfpathlineto{\pgfqpoint{3.991787in}{2.151927in}}%
\pgfpathlineto{\pgfqpoint{4.022769in}{2.209905in}}%
\pgfpathlineto{\pgfqpoint{4.053751in}{2.229119in}}%
\pgfpathlineto{\pgfqpoint{4.084733in}{2.205459in}}%
\pgfpathlineto{\pgfqpoint{4.115716in}{2.272936in}}%
\pgfpathlineto{\pgfqpoint{4.146698in}{2.233788in}}%
\pgfpathlineto{\pgfqpoint{4.177680in}{2.269886in}}%
\pgfpathlineto{\pgfqpoint{4.208662in}{2.269982in}}%
\pgfpathlineto{\pgfqpoint{4.239645in}{2.303380in}}%
\pgfpathlineto{\pgfqpoint{4.270627in}{2.318212in}}%
\pgfpathlineto{\pgfqpoint{4.301609in}{2.255977in}}%
\pgfpathlineto{\pgfqpoint{4.332591in}{2.315938in}}%
\pgfpathlineto{\pgfqpoint{4.363573in}{2.315407in}}%
\pgfpathlineto{\pgfqpoint{4.394556in}{2.350523in}}%
\pgfpathlineto{\pgfqpoint{4.425538in}{2.325593in}}%
\pgfpathlineto{\pgfqpoint{4.456520in}{2.321749in}}%
\pgfpathlineto{\pgfqpoint{4.487502in}{2.366070in}}%
\pgfpathlineto{\pgfqpoint{4.518485in}{2.359501in}}%
\pgfpathlineto{\pgfqpoint{4.549467in}{2.386483in}}%
\pgfpathlineto{\pgfqpoint{4.580449in}{2.364004in}}%
\pgfpathlineto{\pgfqpoint{4.611431in}{2.390474in}}%
\pgfpathlineto{\pgfqpoint{4.642414in}{2.367798in}}%
\pgfpathlineto{\pgfqpoint{4.673396in}{2.373652in}}%
\pgfpathlineto{\pgfqpoint{4.704378in}{2.382907in}}%
\pgfpathlineto{\pgfqpoint{4.735360in}{2.430655in}}%
\pgfpathlineto{\pgfqpoint{4.797325in}{2.407349in}}%
\pgfpathlineto{\pgfqpoint{4.859289in}{2.388793in}}%
\pgfpathlineto{\pgfqpoint{4.952236in}{2.436109in}}%
\pgfpathlineto{\pgfqpoint{4.983218in}{2.437714in}}%
\pgfpathlineto{\pgfqpoint{4.983218in}{2.437714in}}%
\pgfusepath{stroke}%
\end{pgfscope}%
\begin{pgfscope}%
\pgfsetrectcap%
\pgfsetmiterjoin%
\pgfsetlinewidth{0.803000pt}%
\definecolor{currentstroke}{rgb}{0.000000,0.000000,0.000000}%
\pgfsetstrokecolor{currentstroke}%
\pgfsetdash{}{0pt}%
\pgfpathmoveto{\pgfqpoint{0.588387in}{0.521603in}}%
\pgfpathlineto{\pgfqpoint{0.588387in}{2.531888in}}%
\pgfusepath{stroke}%
\end{pgfscope}%
\begin{pgfscope}%
\pgfsetrectcap%
\pgfsetmiterjoin%
\pgfsetlinewidth{0.803000pt}%
\definecolor{currentstroke}{rgb}{0.000000,0.000000,0.000000}%
\pgfsetstrokecolor{currentstroke}%
\pgfsetdash{}{0pt}%
\pgfpathmoveto{\pgfqpoint{5.257411in}{0.521603in}}%
\pgfpathlineto{\pgfqpoint{5.257411in}{2.531888in}}%
\pgfusepath{stroke}%
\end{pgfscope}%
\begin{pgfscope}%
\pgfsetrectcap%
\pgfsetmiterjoin%
\pgfsetlinewidth{0.803000pt}%
\definecolor{currentstroke}{rgb}{0.000000,0.000000,0.000000}%
\pgfsetstrokecolor{currentstroke}%
\pgfsetdash{}{0pt}%
\pgfpathmoveto{\pgfqpoint{0.588387in}{0.521603in}}%
\pgfpathlineto{\pgfqpoint{5.257411in}{0.521603in}}%
\pgfusepath{stroke}%
\end{pgfscope}%
\begin{pgfscope}%
\pgfsetrectcap%
\pgfsetmiterjoin%
\pgfsetlinewidth{0.803000pt}%
\definecolor{currentstroke}{rgb}{0.000000,0.000000,0.000000}%
\pgfsetstrokecolor{currentstroke}%
\pgfsetdash{}{0pt}%
\pgfpathmoveto{\pgfqpoint{0.588387in}{2.531888in}}%
\pgfpathlineto{\pgfqpoint{5.257411in}{2.531888in}}%
\pgfusepath{stroke}%
\end{pgfscope}%
\begin{pgfscope}%
\definecolor{textcolor}{rgb}{0.000000,0.000000,0.000000}%
\pgfsetstrokecolor{textcolor}%
\pgfsetfillcolor{textcolor}%
\pgftext[x=2.922899in,y=2.615222in,,base]{\color{textcolor}{\rmfamily\fontsize{12.000000}{14.400000}\selectfont\catcode`\^=\active\def^{\ifmmode\sp\else\^{}\fi}\catcode`\%=\active\def%{\%}Mean}}%
\end{pgfscope}%
\begin{pgfscope}%
\pgfsetbuttcap%
\pgfsetmiterjoin%
\definecolor{currentfill}{rgb}{1.000000,1.000000,1.000000}%
\pgfsetfillcolor{currentfill}%
\pgfsetfillopacity{0.800000}%
\pgfsetlinewidth{1.003750pt}%
\definecolor{currentstroke}{rgb}{0.800000,0.800000,0.800000}%
\pgfsetstrokecolor{currentstroke}%
\pgfsetstrokeopacity{0.800000}%
\pgfsetdash{}{0pt}%
\pgfpathmoveto{\pgfqpoint{5.344911in}{0.943243in}}%
\pgfpathlineto{\pgfqpoint{8.259376in}{0.943243in}}%
\pgfpathquadraticcurveto{\pgfqpoint{8.284376in}{0.943243in}}{\pgfqpoint{8.284376in}{0.968243in}}%
\pgfpathlineto{\pgfqpoint{8.284376in}{2.444388in}}%
\pgfpathquadraticcurveto{\pgfqpoint{8.284376in}{2.469388in}}{\pgfqpoint{8.259376in}{2.469388in}}%
\pgfpathlineto{\pgfqpoint{5.344911in}{2.469388in}}%
\pgfpathquadraticcurveto{\pgfqpoint{5.319911in}{2.469388in}}{\pgfqpoint{5.319911in}{2.444388in}}%
\pgfpathlineto{\pgfqpoint{5.319911in}{0.968243in}}%
\pgfpathquadraticcurveto{\pgfqpoint{5.319911in}{0.943243in}}{\pgfqpoint{5.344911in}{0.943243in}}%
\pgfpathlineto{\pgfqpoint{5.344911in}{0.943243in}}%
\pgfpathclose%
\pgfusepath{stroke,fill}%
\end{pgfscope}%
\begin{pgfscope}%
\pgfsetrectcap%
\pgfsetroundjoin%
\pgfsetlinewidth{1.505625pt}%
\pgfsetstrokecolor{currentstroke1}%
\pgfsetdash{}{0pt}%
\pgfpathmoveto{\pgfqpoint{5.369911in}{2.368168in}}%
\pgfpathlineto{\pgfqpoint{5.494911in}{2.368168in}}%
\pgfpathlineto{\pgfqpoint{5.619911in}{2.368168in}}%
\pgfusepath{stroke}%
\end{pgfscope}%
\begin{pgfscope}%
\definecolor{textcolor}{rgb}{0.000000,0.000000,0.000000}%
\pgfsetstrokecolor{textcolor}%
\pgfsetfillcolor{textcolor}%
\pgftext[x=5.719911in,y=2.324418in,left,base]{\color{textcolor}{\rmfamily\fontsize{9.000000}{10.800000}\selectfont\catcode`\^=\active\def^{\ifmmode\sp\else\^{}\fi}\catcode`\%=\active\def%{\%}\CyclesMatchChunks{} \& \MergeLinear{}}}%
\end{pgfscope}%
\begin{pgfscope}%
\pgfsetrectcap%
\pgfsetroundjoin%
\pgfsetlinewidth{1.505625pt}%
\pgfsetstrokecolor{currentstroke2}%
\pgfsetdash{}{0pt}%
\pgfpathmoveto{\pgfqpoint{5.369911in}{2.181217in}}%
\pgfpathlineto{\pgfqpoint{5.494911in}{2.181217in}}%
\pgfpathlineto{\pgfqpoint{5.619911in}{2.181217in}}%
\pgfusepath{stroke}%
\end{pgfscope}%
\begin{pgfscope}%
\definecolor{textcolor}{rgb}{0.000000,0.000000,0.000000}%
\pgfsetstrokecolor{textcolor}%
\pgfsetfillcolor{textcolor}%
\pgftext[x=5.719911in,y=2.137467in,left,base]{\color{textcolor}{\rmfamily\fontsize{9.000000}{10.800000}\selectfont\catcode`\^=\active\def^{\ifmmode\sp\else\^{}\fi}\catcode`\%=\active\def%{\%}\CyclesMatchChunks{} \& \SharedVertices{}}}%
\end{pgfscope}%
\begin{pgfscope}%
\pgfsetrectcap%
\pgfsetroundjoin%
\pgfsetlinewidth{1.505625pt}%
\pgfsetstrokecolor{currentstroke3}%
\pgfsetdash{}{0pt}%
\pgfpathmoveto{\pgfqpoint{5.369911in}{1.994267in}}%
\pgfpathlineto{\pgfqpoint{5.494911in}{1.994267in}}%
\pgfpathlineto{\pgfqpoint{5.619911in}{1.994267in}}%
\pgfusepath{stroke}%
\end{pgfscope}%
\begin{pgfscope}%
\definecolor{textcolor}{rgb}{0.000000,0.000000,0.000000}%
\pgfsetstrokecolor{textcolor}%
\pgfsetfillcolor{textcolor}%
\pgftext[x=5.719911in,y=1.950517in,left,base]{\color{textcolor}{\rmfamily\fontsize{9.000000}{10.800000}\selectfont\catcode`\^=\active\def^{\ifmmode\sp\else\^{}\fi}\catcode`\%=\active\def%{\%}\Neighbors{} \& \MergeLinear{}}}%
\end{pgfscope}%
\begin{pgfscope}%
\pgfsetrectcap%
\pgfsetroundjoin%
\pgfsetlinewidth{1.505625pt}%
\pgfsetstrokecolor{currentstroke4}%
\pgfsetdash{}{0pt}%
\pgfpathmoveto{\pgfqpoint{5.369911in}{1.810795in}}%
\pgfpathlineto{\pgfqpoint{5.494911in}{1.810795in}}%
\pgfpathlineto{\pgfqpoint{5.619911in}{1.810795in}}%
\pgfusepath{stroke}%
\end{pgfscope}%
\begin{pgfscope}%
\definecolor{textcolor}{rgb}{0.000000,0.000000,0.000000}%
\pgfsetstrokecolor{textcolor}%
\pgfsetfillcolor{textcolor}%
\pgftext[x=5.719911in,y=1.767045in,left,base]{\color{textcolor}{\rmfamily\fontsize{9.000000}{10.800000}\selectfont\catcode`\^=\active\def^{\ifmmode\sp\else\^{}\fi}\catcode`\%=\active\def%{\%}\Neighbors{} \& \SharedVertices{}}}%
\end{pgfscope}%
\begin{pgfscope}%
\pgfsetrectcap%
\pgfsetroundjoin%
\pgfsetlinewidth{1.505625pt}%
\pgfsetstrokecolor{currentstroke5}%
\pgfsetdash{}{0pt}%
\pgfpathmoveto{\pgfqpoint{5.369911in}{1.623845in}}%
\pgfpathlineto{\pgfqpoint{5.494911in}{1.623845in}}%
\pgfpathlineto{\pgfqpoint{5.619911in}{1.623845in}}%
\pgfusepath{stroke}%
\end{pgfscope}%
\begin{pgfscope}%
\definecolor{textcolor}{rgb}{0.000000,0.000000,0.000000}%
\pgfsetstrokecolor{textcolor}%
\pgfsetfillcolor{textcolor}%
\pgftext[x=5.719911in,y=1.580095in,left,base]{\color{textcolor}{\rmfamily\fontsize{9.000000}{10.800000}\selectfont\catcode`\^=\active\def^{\ifmmode\sp\else\^{}\fi}\catcode`\%=\active\def%{\%}\NeighborsDegree{} \& \MergeLinear{}}}%
\end{pgfscope}%
\begin{pgfscope}%
\pgfsetrectcap%
\pgfsetroundjoin%
\pgfsetlinewidth{1.505625pt}%
\pgfsetstrokecolor{currentstroke6}%
\pgfsetdash{}{0pt}%
\pgfpathmoveto{\pgfqpoint{5.369911in}{1.436894in}}%
\pgfpathlineto{\pgfqpoint{5.494911in}{1.436894in}}%
\pgfpathlineto{\pgfqpoint{5.619911in}{1.436894in}}%
\pgfusepath{stroke}%
\end{pgfscope}%
\begin{pgfscope}%
\definecolor{textcolor}{rgb}{0.000000,0.000000,0.000000}%
\pgfsetstrokecolor{textcolor}%
\pgfsetfillcolor{textcolor}%
\pgftext[x=5.719911in,y=1.393144in,left,base]{\color{textcolor}{\rmfamily\fontsize{9.000000}{10.800000}\selectfont\catcode`\^=\active\def^{\ifmmode\sp\else\^{}\fi}\catcode`\%=\active\def%{\%}\NeighborsDegree{} \& \SharedVertices{}}}%
\end{pgfscope}%
\begin{pgfscope}%
\pgfsetrectcap%
\pgfsetroundjoin%
\pgfsetlinewidth{1.505625pt}%
\pgfsetstrokecolor{currentstroke7}%
\pgfsetdash{}{0pt}%
\pgfpathmoveto{\pgfqpoint{5.369911in}{1.249944in}}%
\pgfpathlineto{\pgfqpoint{5.494911in}{1.249944in}}%
\pgfpathlineto{\pgfqpoint{5.619911in}{1.249944in}}%
\pgfusepath{stroke}%
\end{pgfscope}%
\begin{pgfscope}%
\definecolor{textcolor}{rgb}{0.000000,0.000000,0.000000}%
\pgfsetstrokecolor{textcolor}%
\pgfsetfillcolor{textcolor}%
\pgftext[x=5.719911in,y=1.206194in,left,base]{\color{textcolor}{\rmfamily\fontsize{9.000000}{10.800000}\selectfont\catcode`\^=\active\def^{\ifmmode\sp\else\^{}\fi}\catcode`\%=\active\def%{\%}\None{} \& \MergeLinear{}}}%
\end{pgfscope}%
\begin{pgfscope}%
\pgfsetrectcap%
\pgfsetroundjoin%
\pgfsetlinewidth{1.505625pt}%
\definecolor{currentstroke}{rgb}{0.498039,0.498039,0.498039}%
\pgfsetstrokecolor{currentstroke}%
\pgfsetdash{}{0pt}%
\pgfpathmoveto{\pgfqpoint{5.369911in}{1.066472in}}%
\pgfpathlineto{\pgfqpoint{5.494911in}{1.066472in}}%
\pgfpathlineto{\pgfqpoint{5.619911in}{1.066472in}}%
\pgfusepath{stroke}%
\end{pgfscope}%
\begin{pgfscope}%
\definecolor{textcolor}{rgb}{0.000000,0.000000,0.000000}%
\pgfsetstrokecolor{textcolor}%
\pgfsetfillcolor{textcolor}%
\pgftext[x=5.719911in,y=1.022722in,left,base]{\color{textcolor}{\rmfamily\fontsize{9.000000}{10.800000}\selectfont\catcode`\^=\active\def^{\ifmmode\sp\else\^{}\fi}\catcode`\%=\active\def%{\%}\None{} \& \SharedVertices{}}}%
\end{pgfscope}%
\end{pgfpicture}%
\makeatother%
\endgroup%
}
	\caption[Mean runtime for graphs with no NAC-coloring]{
		Mean running time to find all NAC-colorings for graphs with no NAC-coloring.}%
	\label{fig:graph_no_nac_coloring_first_runtime}
\end{figure}%
\begin{figure}[thbp]
	\centering
	\scalebox{\BenchFigureScale}{%% Creator: Matplotlib, PGF backend
%%
%% To include the figure in your LaTeX document, write
%%   \input{<filename>.pgf}
%%
%% Make sure the required packages are loaded in your preamble
%%   \usepackage{pgf}
%%
%% Also ensure that all the required font packages are loaded; for instance,
%% the lmodern package is sometimes necessary when using math font.
%%   \usepackage{lmodern}
%%
%% Figures using additional raster images can only be included by \input if
%% they are in the same directory as the main LaTeX file. For loading figures
%% from other directories you can use the `import` package
%%   \usepackage{import}
%%
%% and then include the figures with
%%   \import{<path to file>}{<filename>.pgf}
%%
%% Matplotlib used the following preamble
%%   \def\mathdefault#1{#1}
%%   \everymath=\expandafter{\the\everymath\displaystyle}
%%   \IfFileExists{scrextend.sty}{
%%     \usepackage[fontsize=10.000000pt]{scrextend}
%%   }{
%%     \renewcommand{\normalsize}{\fontsize{10.000000}{12.000000}\selectfont}
%%     \normalsize
%%   }
%%   
%%   \ifdefined\pdftexversion\else  % non-pdftex case.
%%     \usepackage{fontspec}
%%     \setmainfont{DejaVuSans.ttf}[Path=\detokenize{/home/petr/Projects/PyRigi/.venv/lib/python3.12/site-packages/matplotlib/mpl-data/fonts/ttf/}]
%%     \setsansfont{DejaVuSans.ttf}[Path=\detokenize{/home/petr/Projects/PyRigi/.venv/lib/python3.12/site-packages/matplotlib/mpl-data/fonts/ttf/}]
%%     \setmonofont{DejaVuSansMono.ttf}[Path=\detokenize{/home/petr/Projects/PyRigi/.venv/lib/python3.12/site-packages/matplotlib/mpl-data/fonts/ttf/}]
%%   \fi
%%   \makeatletter\@ifpackageloaded{under\Score{}}{}{\usepackage[strings]{under\Score{}}}\makeatother
%%
\begingroup%
\makeatletter%
\begin{pgfpicture}%
\pgfpathrectangle{\pgfpointorigin}{\pgfqpoint{8.384376in}{2.841849in}}%
\pgfusepath{use as bounding box, clip}%
\begin{pgfscope}%
\pgfsetbuttcap%
\pgfsetmiterjoin%
\definecolor{currentfill}{rgb}{1.000000,1.000000,1.000000}%
\pgfsetfillcolor{currentfill}%
\pgfsetlinewidth{0.000000pt}%
\definecolor{currentstroke}{rgb}{1.000000,1.000000,1.000000}%
\pgfsetstrokecolor{currentstroke}%
\pgfsetdash{}{0pt}%
\pgfpathmoveto{\pgfqpoint{0.000000in}{0.000000in}}%
\pgfpathlineto{\pgfqpoint{8.384376in}{0.000000in}}%
\pgfpathlineto{\pgfqpoint{8.384376in}{2.841849in}}%
\pgfpathlineto{\pgfqpoint{0.000000in}{2.841849in}}%
\pgfpathlineto{\pgfqpoint{0.000000in}{0.000000in}}%
\pgfpathclose%
\pgfusepath{fill}%
\end{pgfscope}%
\begin{pgfscope}%
\pgfsetbuttcap%
\pgfsetmiterjoin%
\definecolor{currentfill}{rgb}{1.000000,1.000000,1.000000}%
\pgfsetfillcolor{currentfill}%
\pgfsetlinewidth{0.000000pt}%
\definecolor{currentstroke}{rgb}{0.000000,0.000000,0.000000}%
\pgfsetstrokecolor{currentstroke}%
\pgfsetstrokeopacity{0.000000}%
\pgfsetdash{}{0pt}%
\pgfpathmoveto{\pgfqpoint{0.588387in}{0.521603in}}%
\pgfpathlineto{\pgfqpoint{4.248423in}{0.521603in}}%
\pgfpathlineto{\pgfqpoint{4.248423in}{2.741849in}}%
\pgfpathlineto{\pgfqpoint{0.588387in}{2.741849in}}%
\pgfpathlineto{\pgfqpoint{0.588387in}{0.521603in}}%
\pgfpathclose%
\pgfusepath{fill}%
\end{pgfscope}%
\begin{pgfscope}%
\pgfsetbuttcap%
\pgfsetroundjoin%
\definecolor{currentfill}{rgb}{0.000000,0.000000,0.000000}%
\pgfsetfillcolor{currentfill}%
\pgfsetlinewidth{0.803000pt}%
\definecolor{currentstroke}{rgb}{0.000000,0.000000,0.000000}%
\pgfsetstrokecolor{currentstroke}%
\pgfsetdash{}{0pt}%
\pgfsys@defobject{currentmarker}{\pgfqpoint{0.000000in}{-0.048611in}}{\pgfqpoint{0.000000in}{0.000000in}}{%
\pgfpathmoveto{\pgfqpoint{0.000000in}{0.000000in}}%
\pgfpathlineto{\pgfqpoint{0.000000in}{-0.048611in}}%
\pgfusepath{stroke,fill}%
}%
\begin{pgfscope}%
\pgfsys@transformshift{0.918775in}{0.521603in}%
\pgfsys@useobject{currentmarker}{}%
\end{pgfscope}%
\end{pgfscope}%
\begin{pgfscope}%
\definecolor{textcolor}{rgb}{0.000000,0.000000,0.000000}%
\pgfsetstrokecolor{textcolor}%
\pgfsetfillcolor{textcolor}%
\pgftext[x=0.918775in,y=0.424381in,,top]{\color{textcolor}{\rmfamily\fontsize{10.000000}{12.000000}\selectfont\catcode`\^=\active\def^{\ifmmode\sp\else\^{}\fi}\catcode`\%=\active\def%{\%}$\mathdefault{20}$}}%
\end{pgfscope}%
\begin{pgfscope}%
\pgfsetbuttcap%
\pgfsetroundjoin%
\definecolor{currentfill}{rgb}{0.000000,0.000000,0.000000}%
\pgfsetfillcolor{currentfill}%
\pgfsetlinewidth{0.803000pt}%
\definecolor{currentstroke}{rgb}{0.000000,0.000000,0.000000}%
\pgfsetstrokecolor{currentstroke}%
\pgfsetdash{}{0pt}%
\pgfsys@defobject{currentmarker}{\pgfqpoint{0.000000in}{-0.048611in}}{\pgfqpoint{0.000000in}{0.000000in}}{%
\pgfpathmoveto{\pgfqpoint{0.000000in}{0.000000in}}%
\pgfpathlineto{\pgfqpoint{0.000000in}{-0.048611in}}%
\pgfusepath{stroke,fill}%
}%
\begin{pgfscope}%
\pgfsys@transformshift{1.387409in}{0.521603in}%
\pgfsys@useobject{currentmarker}{}%
\end{pgfscope}%
\end{pgfscope}%
\begin{pgfscope}%
\definecolor{textcolor}{rgb}{0.000000,0.000000,0.000000}%
\pgfsetstrokecolor{textcolor}%
\pgfsetfillcolor{textcolor}%
\pgftext[x=1.387409in,y=0.424381in,,top]{\color{textcolor}{\rmfamily\fontsize{10.000000}{12.000000}\selectfont\catcode`\^=\active\def^{\ifmmode\sp\else\^{}\fi}\catcode`\%=\active\def%{\%}$\mathdefault{40}$}}%
\end{pgfscope}%
\begin{pgfscope}%
\pgfsetbuttcap%
\pgfsetroundjoin%
\definecolor{currentfill}{rgb}{0.000000,0.000000,0.000000}%
\pgfsetfillcolor{currentfill}%
\pgfsetlinewidth{0.803000pt}%
\definecolor{currentstroke}{rgb}{0.000000,0.000000,0.000000}%
\pgfsetstrokecolor{currentstroke}%
\pgfsetdash{}{0pt}%
\pgfsys@defobject{currentmarker}{\pgfqpoint{0.000000in}{-0.048611in}}{\pgfqpoint{0.000000in}{0.000000in}}{%
\pgfpathmoveto{\pgfqpoint{0.000000in}{0.000000in}}%
\pgfpathlineto{\pgfqpoint{0.000000in}{-0.048611in}}%
\pgfusepath{stroke,fill}%
}%
\begin{pgfscope}%
\pgfsys@transformshift{1.856044in}{0.521603in}%
\pgfsys@useobject{currentmarker}{}%
\end{pgfscope}%
\end{pgfscope}%
\begin{pgfscope}%
\definecolor{textcolor}{rgb}{0.000000,0.000000,0.000000}%
\pgfsetstrokecolor{textcolor}%
\pgfsetfillcolor{textcolor}%
\pgftext[x=1.856044in,y=0.424381in,,top]{\color{textcolor}{\rmfamily\fontsize{10.000000}{12.000000}\selectfont\catcode`\^=\active\def^{\ifmmode\sp\else\^{}\fi}\catcode`\%=\active\def%{\%}$\mathdefault{60}$}}%
\end{pgfscope}%
\begin{pgfscope}%
\pgfsetbuttcap%
\pgfsetroundjoin%
\definecolor{currentfill}{rgb}{0.000000,0.000000,0.000000}%
\pgfsetfillcolor{currentfill}%
\pgfsetlinewidth{0.803000pt}%
\definecolor{currentstroke}{rgb}{0.000000,0.000000,0.000000}%
\pgfsetstrokecolor{currentstroke}%
\pgfsetdash{}{0pt}%
\pgfsys@defobject{currentmarker}{\pgfqpoint{0.000000in}{-0.048611in}}{\pgfqpoint{0.000000in}{0.000000in}}{%
\pgfpathmoveto{\pgfqpoint{0.000000in}{0.000000in}}%
\pgfpathlineto{\pgfqpoint{0.000000in}{-0.048611in}}%
\pgfusepath{stroke,fill}%
}%
\begin{pgfscope}%
\pgfsys@transformshift{2.324678in}{0.521603in}%
\pgfsys@useobject{currentmarker}{}%
\end{pgfscope}%
\end{pgfscope}%
\begin{pgfscope}%
\definecolor{textcolor}{rgb}{0.000000,0.000000,0.000000}%
\pgfsetstrokecolor{textcolor}%
\pgfsetfillcolor{textcolor}%
\pgftext[x=2.324678in,y=0.424381in,,top]{\color{textcolor}{\rmfamily\fontsize{10.000000}{12.000000}\selectfont\catcode`\^=\active\def^{\ifmmode\sp\else\^{}\fi}\catcode`\%=\active\def%{\%}$\mathdefault{80}$}}%
\end{pgfscope}%
\begin{pgfscope}%
\pgfsetbuttcap%
\pgfsetroundjoin%
\definecolor{currentfill}{rgb}{0.000000,0.000000,0.000000}%
\pgfsetfillcolor{currentfill}%
\pgfsetlinewidth{0.803000pt}%
\definecolor{currentstroke}{rgb}{0.000000,0.000000,0.000000}%
\pgfsetstrokecolor{currentstroke}%
\pgfsetdash{}{0pt}%
\pgfsys@defobject{currentmarker}{\pgfqpoint{0.000000in}{-0.048611in}}{\pgfqpoint{0.000000in}{0.000000in}}{%
\pgfpathmoveto{\pgfqpoint{0.000000in}{0.000000in}}%
\pgfpathlineto{\pgfqpoint{0.000000in}{-0.048611in}}%
\pgfusepath{stroke,fill}%
}%
\begin{pgfscope}%
\pgfsys@transformshift{2.793313in}{0.521603in}%
\pgfsys@useobject{currentmarker}{}%
\end{pgfscope}%
\end{pgfscope}%
\begin{pgfscope}%
\definecolor{textcolor}{rgb}{0.000000,0.000000,0.000000}%
\pgfsetstrokecolor{textcolor}%
\pgfsetfillcolor{textcolor}%
\pgftext[x=2.793313in,y=0.424381in,,top]{\color{textcolor}{\rmfamily\fontsize{10.000000}{12.000000}\selectfont\catcode`\^=\active\def^{\ifmmode\sp\else\^{}\fi}\catcode`\%=\active\def%{\%}$\mathdefault{100}$}}%
\end{pgfscope}%
\begin{pgfscope}%
\pgfsetbuttcap%
\pgfsetroundjoin%
\definecolor{currentfill}{rgb}{0.000000,0.000000,0.000000}%
\pgfsetfillcolor{currentfill}%
\pgfsetlinewidth{0.803000pt}%
\definecolor{currentstroke}{rgb}{0.000000,0.000000,0.000000}%
\pgfsetstrokecolor{currentstroke}%
\pgfsetdash{}{0pt}%
\pgfsys@defobject{currentmarker}{\pgfqpoint{0.000000in}{-0.048611in}}{\pgfqpoint{0.000000in}{0.000000in}}{%
\pgfpathmoveto{\pgfqpoint{0.000000in}{0.000000in}}%
\pgfpathlineto{\pgfqpoint{0.000000in}{-0.048611in}}%
\pgfusepath{stroke,fill}%
}%
\begin{pgfscope}%
\pgfsys@transformshift{3.261947in}{0.521603in}%
\pgfsys@useobject{currentmarker}{}%
\end{pgfscope}%
\end{pgfscope}%
\begin{pgfscope}%
\definecolor{textcolor}{rgb}{0.000000,0.000000,0.000000}%
\pgfsetstrokecolor{textcolor}%
\pgfsetfillcolor{textcolor}%
\pgftext[x=3.261947in,y=0.424381in,,top]{\color{textcolor}{\rmfamily\fontsize{10.000000}{12.000000}\selectfont\catcode`\^=\active\def^{\ifmmode\sp\else\^{}\fi}\catcode`\%=\active\def%{\%}$\mathdefault{120}$}}%
\end{pgfscope}%
\begin{pgfscope}%
\pgfsetbuttcap%
\pgfsetroundjoin%
\definecolor{currentfill}{rgb}{0.000000,0.000000,0.000000}%
\pgfsetfillcolor{currentfill}%
\pgfsetlinewidth{0.803000pt}%
\definecolor{currentstroke}{rgb}{0.000000,0.000000,0.000000}%
\pgfsetstrokecolor{currentstroke}%
\pgfsetdash{}{0pt}%
\pgfsys@defobject{currentmarker}{\pgfqpoint{0.000000in}{-0.048611in}}{\pgfqpoint{0.000000in}{0.000000in}}{%
\pgfpathmoveto{\pgfqpoint{0.000000in}{0.000000in}}%
\pgfpathlineto{\pgfqpoint{0.000000in}{-0.048611in}}%
\pgfusepath{stroke,fill}%
}%
\begin{pgfscope}%
\pgfsys@transformshift{3.730582in}{0.521603in}%
\pgfsys@useobject{currentmarker}{}%
\end{pgfscope}%
\end{pgfscope}%
\begin{pgfscope}%
\definecolor{textcolor}{rgb}{0.000000,0.000000,0.000000}%
\pgfsetstrokecolor{textcolor}%
\pgfsetfillcolor{textcolor}%
\pgftext[x=3.730582in,y=0.424381in,,top]{\color{textcolor}{\rmfamily\fontsize{10.000000}{12.000000}\selectfont\catcode`\^=\active\def^{\ifmmode\sp\else\^{}\fi}\catcode`\%=\active\def%{\%}$\mathdefault{140}$}}%
\end{pgfscope}%
\begin{pgfscope}%
\pgfsetbuttcap%
\pgfsetroundjoin%
\definecolor{currentfill}{rgb}{0.000000,0.000000,0.000000}%
\pgfsetfillcolor{currentfill}%
\pgfsetlinewidth{0.803000pt}%
\definecolor{currentstroke}{rgb}{0.000000,0.000000,0.000000}%
\pgfsetstrokecolor{currentstroke}%
\pgfsetdash{}{0pt}%
\pgfsys@defobject{currentmarker}{\pgfqpoint{0.000000in}{-0.048611in}}{\pgfqpoint{0.000000in}{0.000000in}}{%
\pgfpathmoveto{\pgfqpoint{0.000000in}{0.000000in}}%
\pgfpathlineto{\pgfqpoint{0.000000in}{-0.048611in}}%
\pgfusepath{stroke,fill}%
}%
\begin{pgfscope}%
\pgfsys@transformshift{4.199216in}{0.521603in}%
\pgfsys@useobject{currentmarker}{}%
\end{pgfscope}%
\end{pgfscope}%
\begin{pgfscope}%
\definecolor{textcolor}{rgb}{0.000000,0.000000,0.000000}%
\pgfsetstrokecolor{textcolor}%
\pgfsetfillcolor{textcolor}%
\pgftext[x=4.199216in,y=0.424381in,,top]{\color{textcolor}{\rmfamily\fontsize{10.000000}{12.000000}\selectfont\catcode`\^=\active\def^{\ifmmode\sp\else\^{}\fi}\catcode`\%=\active\def%{\%}$\mathdefault{160}$}}%
\end{pgfscope}%
\begin{pgfscope}%
\definecolor{textcolor}{rgb}{0.000000,0.000000,0.000000}%
\pgfsetstrokecolor{textcolor}%
\pgfsetfillcolor{textcolor}%
\pgftext[x=2.418405in,y=0.234413in,,top]{\color{textcolor}{\rmfamily\fontsize{10.000000}{12.000000}\selectfont\catcode`\^=\active\def^{\ifmmode\sp\else\^{}\fi}\catcode`\%=\active\def%{\%}$\triangle$-connected components}}%
\end{pgfscope}%
\begin{pgfscope}%
\pgfsetbuttcap%
\pgfsetroundjoin%
\definecolor{currentfill}{rgb}{0.000000,0.000000,0.000000}%
\pgfsetfillcolor{currentfill}%
\pgfsetlinewidth{0.803000pt}%
\definecolor{currentstroke}{rgb}{0.000000,0.000000,0.000000}%
\pgfsetstrokecolor{currentstroke}%
\pgfsetdash{}{0pt}%
\pgfsys@defobject{currentmarker}{\pgfqpoint{-0.048611in}{0.000000in}}{\pgfqpoint{-0.000000in}{0.000000in}}{%
\pgfpathmoveto{\pgfqpoint{-0.000000in}{0.000000in}}%
\pgfpathlineto{\pgfqpoint{-0.048611in}{0.000000in}}%
\pgfusepath{stroke,fill}%
}%
\begin{pgfscope}%
\pgfsys@transformshift{0.588387in}{0.617054in}%
\pgfsys@useobject{currentmarker}{}%
\end{pgfscope}%
\end{pgfscope}%
\begin{pgfscope}%
\definecolor{textcolor}{rgb}{0.000000,0.000000,0.000000}%
\pgfsetstrokecolor{textcolor}%
\pgfsetfillcolor{textcolor}%
\pgftext[x=0.289968in, y=0.564293in, left, base]{\color{textcolor}{\rmfamily\fontsize{10.000000}{12.000000}\selectfont\catcode`\^=\active\def^{\ifmmode\sp\else\^{}\fi}\catcode`\%=\active\def%{\%}$\mathdefault{10^{2}}$}}%
\end{pgfscope}%
\begin{pgfscope}%
\pgfsetbuttcap%
\pgfsetroundjoin%
\definecolor{currentfill}{rgb}{0.000000,0.000000,0.000000}%
\pgfsetfillcolor{currentfill}%
\pgfsetlinewidth{0.803000pt}%
\definecolor{currentstroke}{rgb}{0.000000,0.000000,0.000000}%
\pgfsetstrokecolor{currentstroke}%
\pgfsetdash{}{0pt}%
\pgfsys@defobject{currentmarker}{\pgfqpoint{-0.048611in}{0.000000in}}{\pgfqpoint{-0.000000in}{0.000000in}}{%
\pgfpathmoveto{\pgfqpoint{-0.000000in}{0.000000in}}%
\pgfpathlineto{\pgfqpoint{-0.048611in}{0.000000in}}%
\pgfusepath{stroke,fill}%
}%
\begin{pgfscope}%
\pgfsys@transformshift{0.588387in}{2.274401in}%
\pgfsys@useobject{currentmarker}{}%
\end{pgfscope}%
\end{pgfscope}%
\begin{pgfscope}%
\definecolor{textcolor}{rgb}{0.000000,0.000000,0.000000}%
\pgfsetstrokecolor{textcolor}%
\pgfsetfillcolor{textcolor}%
\pgftext[x=0.289968in, y=2.221640in, left, base]{\color{textcolor}{\rmfamily\fontsize{10.000000}{12.000000}\selectfont\catcode`\^=\active\def^{\ifmmode\sp\else\^{}\fi}\catcode`\%=\active\def%{\%}$\mathdefault{10^{3}}$}}%
\end{pgfscope}%
\begin{pgfscope}%
\pgfsetbuttcap%
\pgfsetroundjoin%
\definecolor{currentfill}{rgb}{0.000000,0.000000,0.000000}%
\pgfsetfillcolor{currentfill}%
\pgfsetlinewidth{0.602250pt}%
\definecolor{currentstroke}{rgb}{0.000000,0.000000,0.000000}%
\pgfsetstrokecolor{currentstroke}%
\pgfsetdash{}{0pt}%
\pgfsys@defobject{currentmarker}{\pgfqpoint{-0.027778in}{0.000000in}}{\pgfqpoint{-0.000000in}{0.000000in}}{%
\pgfpathmoveto{\pgfqpoint{-0.000000in}{0.000000in}}%
\pgfpathlineto{\pgfqpoint{-0.027778in}{0.000000in}}%
\pgfusepath{stroke,fill}%
}%
\begin{pgfscope}%
\pgfsys@transformshift{0.588387in}{0.541218in}%
\pgfsys@useobject{currentmarker}{}%
\end{pgfscope}%
\end{pgfscope}%
\begin{pgfscope}%
\pgfsetbuttcap%
\pgfsetroundjoin%
\definecolor{currentfill}{rgb}{0.000000,0.000000,0.000000}%
\pgfsetfillcolor{currentfill}%
\pgfsetlinewidth{0.602250pt}%
\definecolor{currentstroke}{rgb}{0.000000,0.000000,0.000000}%
\pgfsetstrokecolor{currentstroke}%
\pgfsetdash{}{0pt}%
\pgfsys@defobject{currentmarker}{\pgfqpoint{-0.027778in}{0.000000in}}{\pgfqpoint{-0.000000in}{0.000000in}}{%
\pgfpathmoveto{\pgfqpoint{-0.000000in}{0.000000in}}%
\pgfpathlineto{\pgfqpoint{-0.027778in}{0.000000in}}%
\pgfusepath{stroke,fill}%
}%
\begin{pgfscope}%
\pgfsys@transformshift{0.588387in}{1.115965in}%
\pgfsys@useobject{currentmarker}{}%
\end{pgfscope}%
\end{pgfscope}%
\begin{pgfscope}%
\pgfsetbuttcap%
\pgfsetroundjoin%
\definecolor{currentfill}{rgb}{0.000000,0.000000,0.000000}%
\pgfsetfillcolor{currentfill}%
\pgfsetlinewidth{0.602250pt}%
\definecolor{currentstroke}{rgb}{0.000000,0.000000,0.000000}%
\pgfsetstrokecolor{currentstroke}%
\pgfsetdash{}{0pt}%
\pgfsys@defobject{currentmarker}{\pgfqpoint{-0.027778in}{0.000000in}}{\pgfqpoint{-0.000000in}{0.000000in}}{%
\pgfpathmoveto{\pgfqpoint{-0.000000in}{0.000000in}}%
\pgfpathlineto{\pgfqpoint{-0.027778in}{0.000000in}}%
\pgfusepath{stroke,fill}%
}%
\begin{pgfscope}%
\pgfsys@transformshift{0.588387in}{1.407810in}%
\pgfsys@useobject{currentmarker}{}%
\end{pgfscope}%
\end{pgfscope}%
\begin{pgfscope}%
\pgfsetbuttcap%
\pgfsetroundjoin%
\definecolor{currentfill}{rgb}{0.000000,0.000000,0.000000}%
\pgfsetfillcolor{currentfill}%
\pgfsetlinewidth{0.602250pt}%
\definecolor{currentstroke}{rgb}{0.000000,0.000000,0.000000}%
\pgfsetstrokecolor{currentstroke}%
\pgfsetdash{}{0pt}%
\pgfsys@defobject{currentmarker}{\pgfqpoint{-0.027778in}{0.000000in}}{\pgfqpoint{-0.000000in}{0.000000in}}{%
\pgfpathmoveto{\pgfqpoint{-0.000000in}{0.000000in}}%
\pgfpathlineto{\pgfqpoint{-0.027778in}{0.000000in}}%
\pgfusepath{stroke,fill}%
}%
\begin{pgfscope}%
\pgfsys@transformshift{0.588387in}{1.614876in}%
\pgfsys@useobject{currentmarker}{}%
\end{pgfscope}%
\end{pgfscope}%
\begin{pgfscope}%
\pgfsetbuttcap%
\pgfsetroundjoin%
\definecolor{currentfill}{rgb}{0.000000,0.000000,0.000000}%
\pgfsetfillcolor{currentfill}%
\pgfsetlinewidth{0.602250pt}%
\definecolor{currentstroke}{rgb}{0.000000,0.000000,0.000000}%
\pgfsetstrokecolor{currentstroke}%
\pgfsetdash{}{0pt}%
\pgfsys@defobject{currentmarker}{\pgfqpoint{-0.027778in}{0.000000in}}{\pgfqpoint{-0.000000in}{0.000000in}}{%
\pgfpathmoveto{\pgfqpoint{-0.000000in}{0.000000in}}%
\pgfpathlineto{\pgfqpoint{-0.027778in}{0.000000in}}%
\pgfusepath{stroke,fill}%
}%
\begin{pgfscope}%
\pgfsys@transformshift{0.588387in}{1.775490in}%
\pgfsys@useobject{currentmarker}{}%
\end{pgfscope}%
\end{pgfscope}%
\begin{pgfscope}%
\pgfsetbuttcap%
\pgfsetroundjoin%
\definecolor{currentfill}{rgb}{0.000000,0.000000,0.000000}%
\pgfsetfillcolor{currentfill}%
\pgfsetlinewidth{0.602250pt}%
\definecolor{currentstroke}{rgb}{0.000000,0.000000,0.000000}%
\pgfsetstrokecolor{currentstroke}%
\pgfsetdash{}{0pt}%
\pgfsys@defobject{currentmarker}{\pgfqpoint{-0.027778in}{0.000000in}}{\pgfqpoint{-0.000000in}{0.000000in}}{%
\pgfpathmoveto{\pgfqpoint{-0.000000in}{0.000000in}}%
\pgfpathlineto{\pgfqpoint{-0.027778in}{0.000000in}}%
\pgfusepath{stroke,fill}%
}%
\begin{pgfscope}%
\pgfsys@transformshift{0.588387in}{1.906721in}%
\pgfsys@useobject{currentmarker}{}%
\end{pgfscope}%
\end{pgfscope}%
\begin{pgfscope}%
\pgfsetbuttcap%
\pgfsetroundjoin%
\definecolor{currentfill}{rgb}{0.000000,0.000000,0.000000}%
\pgfsetfillcolor{currentfill}%
\pgfsetlinewidth{0.602250pt}%
\definecolor{currentstroke}{rgb}{0.000000,0.000000,0.000000}%
\pgfsetstrokecolor{currentstroke}%
\pgfsetdash{}{0pt}%
\pgfsys@defobject{currentmarker}{\pgfqpoint{-0.027778in}{0.000000in}}{\pgfqpoint{-0.000000in}{0.000000in}}{%
\pgfpathmoveto{\pgfqpoint{-0.000000in}{0.000000in}}%
\pgfpathlineto{\pgfqpoint{-0.027778in}{0.000000in}}%
\pgfusepath{stroke,fill}%
}%
\begin{pgfscope}%
\pgfsys@transformshift{0.588387in}{2.017675in}%
\pgfsys@useobject{currentmarker}{}%
\end{pgfscope}%
\end{pgfscope}%
\begin{pgfscope}%
\pgfsetbuttcap%
\pgfsetroundjoin%
\definecolor{currentfill}{rgb}{0.000000,0.000000,0.000000}%
\pgfsetfillcolor{currentfill}%
\pgfsetlinewidth{0.602250pt}%
\definecolor{currentstroke}{rgb}{0.000000,0.000000,0.000000}%
\pgfsetstrokecolor{currentstroke}%
\pgfsetdash{}{0pt}%
\pgfsys@defobject{currentmarker}{\pgfqpoint{-0.027778in}{0.000000in}}{\pgfqpoint{-0.000000in}{0.000000in}}{%
\pgfpathmoveto{\pgfqpoint{-0.000000in}{0.000000in}}%
\pgfpathlineto{\pgfqpoint{-0.027778in}{0.000000in}}%
\pgfusepath{stroke,fill}%
}%
\begin{pgfscope}%
\pgfsys@transformshift{0.588387in}{2.113788in}%
\pgfsys@useobject{currentmarker}{}%
\end{pgfscope}%
\end{pgfscope}%
\begin{pgfscope}%
\pgfsetbuttcap%
\pgfsetroundjoin%
\definecolor{currentfill}{rgb}{0.000000,0.000000,0.000000}%
\pgfsetfillcolor{currentfill}%
\pgfsetlinewidth{0.602250pt}%
\definecolor{currentstroke}{rgb}{0.000000,0.000000,0.000000}%
\pgfsetstrokecolor{currentstroke}%
\pgfsetdash{}{0pt}%
\pgfsys@defobject{currentmarker}{\pgfqpoint{-0.027778in}{0.000000in}}{\pgfqpoint{-0.000000in}{0.000000in}}{%
\pgfpathmoveto{\pgfqpoint{-0.000000in}{0.000000in}}%
\pgfpathlineto{\pgfqpoint{-0.027778in}{0.000000in}}%
\pgfusepath{stroke,fill}%
}%
\begin{pgfscope}%
\pgfsys@transformshift{0.588387in}{2.198565in}%
\pgfsys@useobject{currentmarker}{}%
\end{pgfscope}%
\end{pgfscope}%
\begin{pgfscope}%
\definecolor{textcolor}{rgb}{0.000000,0.000000,0.000000}%
\pgfsetstrokecolor{textcolor}%
\pgfsetfillcolor{textcolor}%
\pgftext[x=0.234413in,y=1.631726in,,bottom,rotate=90.000000]{\color{textcolor}{\rmfamily\fontsize{10.000000}{12.000000}\selectfont\catcode`\^=\active\def^{\ifmmode\sp\else\^{}\fi}\catcode`\%=\active\def%{\%}Checks [call]}}%
\end{pgfscope}%
\begin{pgfscope}%
\pgfpathrectangle{\pgfqpoint{0.588387in}{0.521603in}}{\pgfqpoint{3.660036in}{2.220246in}}%
\pgfusepath{clip}%
\pgfsetrectcap%
\pgfsetroundjoin%
\pgfsetlinewidth{1.505625pt}%
\pgfsetstrokecolor{currentstroke1}%
\pgfsetdash{}{0pt}%
\pgfpathmoveto{\pgfqpoint{0.754752in}{0.830864in}}%
\pgfpathlineto{\pgfqpoint{0.778184in}{1.131821in}}%
\pgfpathlineto{\pgfqpoint{0.801616in}{0.696192in}}%
\pgfpathlineto{\pgfqpoint{0.825048in}{0.680605in}}%
\pgfpathlineto{\pgfqpoint{0.848479in}{0.847420in}}%
\pgfpathlineto{\pgfqpoint{0.871911in}{1.015654in}}%
\pgfpathlineto{\pgfqpoint{0.895343in}{1.123583in}}%
\pgfpathlineto{\pgfqpoint{0.918775in}{0.826297in}}%
\pgfpathlineto{\pgfqpoint{0.942206in}{0.864990in}}%
\pgfpathlineto{\pgfqpoint{0.965638in}{1.044881in}}%
\pgfpathlineto{\pgfqpoint{0.989070in}{1.171189in}}%
\pgfpathlineto{\pgfqpoint{1.012501in}{1.191170in}}%
\pgfpathlineto{\pgfqpoint{1.035933in}{1.009794in}}%
\pgfpathlineto{\pgfqpoint{1.059365in}{1.091039in}}%
\pgfpathlineto{\pgfqpoint{1.082797in}{1.208475in}}%
\pgfpathlineto{\pgfqpoint{1.106228in}{1.225004in}}%
\pgfpathlineto{\pgfqpoint{1.129660in}{1.305678in}}%
\pgfpathlineto{\pgfqpoint{1.153092in}{1.182593in}}%
\pgfpathlineto{\pgfqpoint{1.176524in}{1.249649in}}%
\pgfpathlineto{\pgfqpoint{1.199955in}{1.264848in}}%
\pgfpathlineto{\pgfqpoint{1.223387in}{1.335519in}}%
\pgfpathlineto{\pgfqpoint{1.246819in}{1.421759in}}%
\pgfpathlineto{\pgfqpoint{1.270250in}{1.323882in}}%
\pgfpathlineto{\pgfqpoint{1.293682in}{1.293649in}}%
\pgfpathlineto{\pgfqpoint{1.317114in}{1.381994in}}%
\pgfpathlineto{\pgfqpoint{1.340546in}{1.430670in}}%
\pgfpathlineto{\pgfqpoint{1.363977in}{1.517966in}}%
\pgfpathlineto{\pgfqpoint{1.387409in}{1.370780in}}%
\pgfpathlineto{\pgfqpoint{1.410841in}{1.404603in}}%
\pgfpathlineto{\pgfqpoint{1.434273in}{1.438987in}}%
\pgfpathlineto{\pgfqpoint{1.457704in}{1.535883in}}%
\pgfpathlineto{\pgfqpoint{1.481136in}{1.548900in}}%
\pgfpathlineto{\pgfqpoint{1.504568in}{1.451033in}}%
\pgfpathlineto{\pgfqpoint{1.527999in}{1.521119in}}%
\pgfpathlineto{\pgfqpoint{1.551431in}{1.591205in}}%
\pgfpathlineto{\pgfqpoint{1.574863in}{1.592010in}}%
\pgfpathlineto{\pgfqpoint{1.598295in}{1.598000in}}%
\pgfpathlineto{\pgfqpoint{1.621726in}{1.543439in}}%
\pgfpathlineto{\pgfqpoint{1.645158in}{1.574563in}}%
\pgfpathlineto{\pgfqpoint{1.668590in}{1.567735in}}%
\pgfpathlineto{\pgfqpoint{1.692021in}{1.608924in}}%
\pgfpathlineto{\pgfqpoint{1.715453in}{1.737117in}}%
\pgfpathlineto{\pgfqpoint{1.738885in}{1.618256in}}%
\pgfpathlineto{\pgfqpoint{1.762317in}{1.586262in}}%
\pgfpathlineto{\pgfqpoint{1.785748in}{1.620612in}}%
\pgfpathlineto{\pgfqpoint{1.809180in}{1.704966in}}%
\pgfpathlineto{\pgfqpoint{1.832612in}{1.719475in}}%
\pgfpathlineto{\pgfqpoint{1.856044in}{1.638168in}}%
\pgfpathlineto{\pgfqpoint{1.879475in}{1.697834in}}%
\pgfpathlineto{\pgfqpoint{1.902907in}{1.673851in}}%
\pgfpathlineto{\pgfqpoint{1.926339in}{1.750095in}}%
\pgfpathlineto{\pgfqpoint{1.949770in}{1.715572in}}%
\pgfpathlineto{\pgfqpoint{1.973202in}{1.663576in}}%
\pgfpathlineto{\pgfqpoint{1.996634in}{1.729289in}}%
\pgfpathlineto{\pgfqpoint{2.020066in}{1.803720in}}%
\pgfpathlineto{\pgfqpoint{2.043497in}{1.783127in}}%
\pgfpathlineto{\pgfqpoint{2.066929in}{1.794433in}}%
\pgfpathlineto{\pgfqpoint{2.090361in}{1.794001in}}%
\pgfpathlineto{\pgfqpoint{2.113793in}{1.761927in}}%
\pgfpathlineto{\pgfqpoint{2.137224in}{1.816143in}}%
\pgfpathlineto{\pgfqpoint{2.160656in}{1.816241in}}%
\pgfpathlineto{\pgfqpoint{2.184088in}{1.864563in}}%
\pgfpathlineto{\pgfqpoint{2.207519in}{1.861720in}}%
\pgfpathlineto{\pgfqpoint{2.230951in}{1.825946in}}%
\pgfpathlineto{\pgfqpoint{2.254383in}{1.832932in}}%
\pgfpathlineto{\pgfqpoint{2.277815in}{1.898101in}}%
\pgfpathlineto{\pgfqpoint{2.301246in}{1.905760in}}%
\pgfpathlineto{\pgfqpoint{2.324678in}{1.871693in}}%
\pgfpathlineto{\pgfqpoint{2.348110in}{1.833546in}}%
\pgfpathlineto{\pgfqpoint{2.371542in}{1.953281in}}%
\pgfpathlineto{\pgfqpoint{2.394973in}{1.895522in}}%
\pgfpathlineto{\pgfqpoint{2.418405in}{1.878448in}}%
\pgfpathlineto{\pgfqpoint{2.441837in}{1.844615in}}%
\pgfpathlineto{\pgfqpoint{2.465268in}{1.922357in}}%
\pgfpathlineto{\pgfqpoint{2.488700in}{1.882009in}}%
\pgfpathlineto{\pgfqpoint{2.512132in}{1.912624in}}%
\pgfpathlineto{\pgfqpoint{2.535564in}{2.030973in}}%
\pgfpathlineto{\pgfqpoint{2.558995in}{1.990858in}}%
\pgfpathlineto{\pgfqpoint{2.582427in}{2.011389in}}%
\pgfpathlineto{\pgfqpoint{2.605859in}{1.947151in}}%
\pgfpathlineto{\pgfqpoint{2.629291in}{1.953736in}}%
\pgfpathlineto{\pgfqpoint{2.652722in}{2.024657in}}%
\pgfpathlineto{\pgfqpoint{2.676154in}{1.973138in}}%
\pgfpathlineto{\pgfqpoint{2.699586in}{1.944120in}}%
\pgfpathlineto{\pgfqpoint{2.723017in}{1.939197in}}%
\pgfpathlineto{\pgfqpoint{2.746449in}{2.009646in}}%
\pgfpathlineto{\pgfqpoint{2.769881in}{2.048335in}}%
\pgfpathlineto{\pgfqpoint{2.793313in}{1.978708in}}%
\pgfpathlineto{\pgfqpoint{2.816744in}{2.094333in}}%
\pgfpathlineto{\pgfqpoint{2.840176in}{1.966545in}}%
\pgfpathlineto{\pgfqpoint{2.863608in}{1.986147in}}%
\pgfpathlineto{\pgfqpoint{2.887039in}{2.073655in}}%
\pgfpathlineto{\pgfqpoint{2.910471in}{1.987633in}}%
\pgfpathlineto{\pgfqpoint{2.933903in}{2.050439in}}%
\pgfpathlineto{\pgfqpoint{2.957335in}{2.085154in}}%
\pgfpathlineto{\pgfqpoint{2.980766in}{1.944120in}}%
\pgfpathlineto{\pgfqpoint{3.004198in}{2.085154in}}%
\pgfpathlineto{\pgfqpoint{3.027630in}{1.979672in}}%
\pgfpathlineto{\pgfqpoint{3.051062in}{2.047880in}}%
\pgfpathlineto{\pgfqpoint{3.074493in}{2.117378in}}%
\pgfpathlineto{\pgfqpoint{3.097925in}{2.125010in}}%
\pgfpathlineto{\pgfqpoint{3.121357in}{2.007318in}}%
\pgfpathlineto{\pgfqpoint{3.144788in}{2.023410in}}%
\pgfpathlineto{\pgfqpoint{3.168220in}{1.991711in}}%
\pgfpathlineto{\pgfqpoint{3.191652in}{2.008013in}}%
\pgfpathlineto{\pgfqpoint{3.215084in}{2.309976in}}%
\pgfpathlineto{\pgfqpoint{3.238515in}{2.050635in}}%
\pgfpathlineto{\pgfqpoint{3.261947in}{2.021776in}}%
\pgfpathlineto{\pgfqpoint{3.285379in}{2.029909in}}%
\pgfpathlineto{\pgfqpoint{3.308811in}{2.173612in}}%
\pgfpathlineto{\pgfqpoint{3.332242in}{2.074157in}}%
\pgfpathlineto{\pgfqpoint{3.355674in}{2.049062in}}%
\pgfpathlineto{\pgfqpoint{3.379106in}{2.053771in}}%
\pgfpathlineto{\pgfqpoint{3.402537in}{2.084405in}}%
\pgfpathlineto{\pgfqpoint{3.425969in}{2.096794in}}%
\pgfpathlineto{\pgfqpoint{3.449401in}{2.069251in}}%
\pgfpathlineto{\pgfqpoint{3.472833in}{2.073070in}}%
\pgfpathlineto{\pgfqpoint{3.496264in}{2.099246in}}%
\pgfpathlineto{\pgfqpoint{3.519696in}{2.117378in}}%
\pgfpathlineto{\pgfqpoint{3.589991in}{2.238995in}}%
\pgfpathlineto{\pgfqpoint{3.707150in}{2.162487in}}%
\pgfusepath{stroke}%
\end{pgfscope}%
\begin{pgfscope}%
\pgfpathrectangle{\pgfqpoint{0.588387in}{0.521603in}}{\pgfqpoint{3.660036in}{2.220246in}}%
\pgfusepath{clip}%
\pgfsetrectcap%
\pgfsetroundjoin%
\pgfsetlinewidth{1.505625pt}%
\pgfsetstrokecolor{currentstroke2}%
\pgfsetdash{}{0pt}%
\pgfpathmoveto{\pgfqpoint{0.754752in}{0.842213in}}%
\pgfpathlineto{\pgfqpoint{0.778184in}{1.131821in}}%
\pgfpathlineto{\pgfqpoint{0.801616in}{0.696320in}}%
\pgfpathlineto{\pgfqpoint{0.825048in}{0.685490in}}%
\pgfpathlineto{\pgfqpoint{0.871911in}{1.015911in}}%
\pgfpathlineto{\pgfqpoint{0.895343in}{1.121475in}}%
\pgfpathlineto{\pgfqpoint{0.918775in}{0.826393in}}%
\pgfpathlineto{\pgfqpoint{0.942206in}{0.866935in}}%
\pgfpathlineto{\pgfqpoint{0.965638in}{1.047363in}}%
\pgfpathlineto{\pgfqpoint{0.989070in}{1.175280in}}%
\pgfpathlineto{\pgfqpoint{1.012501in}{1.198592in}}%
\pgfpathlineto{\pgfqpoint{1.035933in}{1.013580in}}%
\pgfpathlineto{\pgfqpoint{1.059365in}{1.092519in}}%
\pgfpathlineto{\pgfqpoint{1.082797in}{1.208475in}}%
\pgfpathlineto{\pgfqpoint{1.106228in}{1.222794in}}%
\pgfpathlineto{\pgfqpoint{1.129660in}{1.308221in}}%
\pgfpathlineto{\pgfqpoint{1.153092in}{1.184274in}}%
\pgfpathlineto{\pgfqpoint{1.176524in}{1.250645in}}%
\pgfpathlineto{\pgfqpoint{1.199955in}{1.268391in}}%
\pgfpathlineto{\pgfqpoint{1.223387in}{1.337033in}}%
\pgfpathlineto{\pgfqpoint{1.246819in}{1.419707in}}%
\pgfpathlineto{\pgfqpoint{1.270250in}{1.339016in}}%
\pgfpathlineto{\pgfqpoint{1.293682in}{1.289086in}}%
\pgfpathlineto{\pgfqpoint{1.317114in}{1.394183in}}%
\pgfpathlineto{\pgfqpoint{1.340546in}{1.436776in}}%
\pgfpathlineto{\pgfqpoint{1.363977in}{1.515946in}}%
\pgfpathlineto{\pgfqpoint{1.387409in}{1.373849in}}%
\pgfpathlineto{\pgfqpoint{1.434273in}{1.438987in}}%
\pgfpathlineto{\pgfqpoint{1.457704in}{1.545194in}}%
\pgfpathlineto{\pgfqpoint{1.481136in}{1.557822in}}%
\pgfpathlineto{\pgfqpoint{1.504568in}{1.451033in}}%
\pgfpathlineto{\pgfqpoint{1.551431in}{1.598172in}}%
\pgfpathlineto{\pgfqpoint{1.574863in}{1.586326in}}%
\pgfpathlineto{\pgfqpoint{1.598295in}{1.603128in}}%
\pgfpathlineto{\pgfqpoint{1.621726in}{1.560771in}}%
\pgfpathlineto{\pgfqpoint{1.645158in}{1.584855in}}%
\pgfpathlineto{\pgfqpoint{1.668590in}{1.581237in}}%
\pgfpathlineto{\pgfqpoint{1.692021in}{1.608355in}}%
\pgfpathlineto{\pgfqpoint{1.715453in}{1.727435in}}%
\pgfpathlineto{\pgfqpoint{1.738885in}{1.614876in}}%
\pgfpathlineto{\pgfqpoint{1.762317in}{1.595801in}}%
\pgfpathlineto{\pgfqpoint{1.785748in}{1.611027in}}%
\pgfpathlineto{\pgfqpoint{1.809180in}{1.704401in}}%
\pgfpathlineto{\pgfqpoint{1.832612in}{1.724722in}}%
\pgfpathlineto{\pgfqpoint{1.856044in}{1.650527in}}%
\pgfpathlineto{\pgfqpoint{1.879475in}{1.695717in}}%
\pgfpathlineto{\pgfqpoint{1.902907in}{1.686492in}}%
\pgfpathlineto{\pgfqpoint{1.926339in}{1.750095in}}%
\pgfpathlineto{\pgfqpoint{1.949770in}{1.721706in}}%
\pgfpathlineto{\pgfqpoint{1.973202in}{1.663576in}}%
\pgfpathlineto{\pgfqpoint{1.996634in}{1.729990in}}%
\pgfpathlineto{\pgfqpoint{2.020066in}{1.780681in}}%
\pgfpathlineto{\pgfqpoint{2.043497in}{1.797418in}}%
\pgfpathlineto{\pgfqpoint{2.066929in}{1.818697in}}%
\pgfpathlineto{\pgfqpoint{2.090361in}{1.820558in}}%
\pgfpathlineto{\pgfqpoint{2.113793in}{1.763880in}}%
\pgfpathlineto{\pgfqpoint{2.137224in}{1.802262in}}%
\pgfpathlineto{\pgfqpoint{2.160656in}{1.834209in}}%
\pgfpathlineto{\pgfqpoint{2.184088in}{1.869969in}}%
\pgfpathlineto{\pgfqpoint{2.207519in}{1.857061in}}%
\pgfpathlineto{\pgfqpoint{2.230951in}{1.815548in}}%
\pgfpathlineto{\pgfqpoint{2.254383in}{1.850706in}}%
\pgfpathlineto{\pgfqpoint{2.277815in}{1.898216in}}%
\pgfpathlineto{\pgfqpoint{2.301246in}{1.905760in}}%
\pgfpathlineto{\pgfqpoint{2.324678in}{1.881698in}}%
\pgfpathlineto{\pgfqpoint{2.348110in}{1.849308in}}%
\pgfpathlineto{\pgfqpoint{2.371542in}{1.961798in}}%
\pgfpathlineto{\pgfqpoint{2.394973in}{1.889339in}}%
\pgfpathlineto{\pgfqpoint{2.418405in}{1.895978in}}%
\pgfpathlineto{\pgfqpoint{2.441837in}{1.857061in}}%
\pgfpathlineto{\pgfqpoint{2.465268in}{1.921251in}}%
\pgfpathlineto{\pgfqpoint{2.488700in}{1.882009in}}%
\pgfpathlineto{\pgfqpoint{2.512132in}{1.915976in}}%
\pgfpathlineto{\pgfqpoint{2.535564in}{2.014638in}}%
\pgfpathlineto{\pgfqpoint{2.558995in}{1.940466in}}%
\pgfpathlineto{\pgfqpoint{2.582427in}{1.996074in}}%
\pgfpathlineto{\pgfqpoint{2.605859in}{1.931611in}}%
\pgfpathlineto{\pgfqpoint{2.652722in}{2.025615in}}%
\pgfpathlineto{\pgfqpoint{2.676154in}{2.011479in}}%
\pgfpathlineto{\pgfqpoint{2.699586in}{1.942462in}}%
\pgfpathlineto{\pgfqpoint{2.723017in}{1.953866in}}%
\pgfpathlineto{\pgfqpoint{2.746449in}{2.019366in}}%
\pgfpathlineto{\pgfqpoint{2.769881in}{2.041021in}}%
\pgfpathlineto{\pgfqpoint{2.793313in}{2.034388in}}%
\pgfpathlineto{\pgfqpoint{2.816744in}{2.091039in}}%
\pgfpathlineto{\pgfqpoint{2.840176in}{1.968309in}}%
\pgfpathlineto{\pgfqpoint{2.863608in}{1.986147in}}%
\pgfpathlineto{\pgfqpoint{2.887039in}{2.094996in}}%
\pgfpathlineto{\pgfqpoint{2.910471in}{1.992891in}}%
\pgfpathlineto{\pgfqpoint{2.933903in}{2.062906in}}%
\pgfpathlineto{\pgfqpoint{2.957335in}{2.085154in}}%
\pgfpathlineto{\pgfqpoint{2.980766in}{1.944120in}}%
\pgfpathlineto{\pgfqpoint{3.004198in}{2.024226in}}%
\pgfpathlineto{\pgfqpoint{3.027630in}{2.177136in}}%
\pgfpathlineto{\pgfqpoint{3.051062in}{2.045905in}}%
\pgfpathlineto{\pgfqpoint{3.074493in}{2.113788in}}%
\pgfpathlineto{\pgfqpoint{3.097925in}{2.135936in}}%
\pgfpathlineto{\pgfqpoint{3.121357in}{2.047880in}}%
\pgfpathlineto{\pgfqpoint{3.144788in}{2.023410in}}%
\pgfpathlineto{\pgfqpoint{3.168220in}{1.991711in}}%
\pgfpathlineto{\pgfqpoint{3.191652in}{2.011479in}}%
\pgfpathlineto{\pgfqpoint{3.215084in}{2.443682in}}%
\pgfpathlineto{\pgfqpoint{3.238515in}{2.050635in}}%
\pgfpathlineto{\pgfqpoint{3.261947in}{2.021776in}}%
\pgfpathlineto{\pgfqpoint{3.285379in}{2.029909in}}%
\pgfpathlineto{\pgfqpoint{3.308811in}{2.278707in}}%
\pgfpathlineto{\pgfqpoint{3.332242in}{2.078287in}}%
\pgfpathlineto{\pgfqpoint{3.355674in}{2.041939in}}%
\pgfpathlineto{\pgfqpoint{3.379106in}{2.095564in}}%
\pgfpathlineto{\pgfqpoint{3.402537in}{2.099246in}}%
\pgfpathlineto{\pgfqpoint{3.425969in}{2.106554in}}%
\pgfpathlineto{\pgfqpoint{3.449401in}{2.076868in}}%
\pgfpathlineto{\pgfqpoint{3.472833in}{2.130682in}}%
\pgfpathlineto{\pgfqpoint{3.496264in}{2.094641in}}%
\pgfpathlineto{\pgfqpoint{3.519696in}{2.118570in}}%
\pgfpathlineto{\pgfqpoint{3.543128in}{2.128041in}}%
\pgfpathlineto{\pgfqpoint{3.566560in}{2.477498in}}%
\pgfpathlineto{\pgfqpoint{3.589991in}{2.135063in}}%
\pgfpathlineto{\pgfqpoint{3.613423in}{2.142018in}}%
\pgfpathlineto{\pgfqpoint{3.636855in}{2.106554in}}%
\pgfpathlineto{\pgfqpoint{3.660286in}{2.120950in}}%
\pgfpathlineto{\pgfqpoint{3.683718in}{2.169182in}}%
\pgfpathlineto{\pgfqpoint{3.707150in}{2.162487in}}%
\pgfpathlineto{\pgfqpoint{3.777445in}{2.148906in}}%
\pgfpathlineto{\pgfqpoint{3.824309in}{2.155728in}}%
\pgfpathlineto{\pgfqpoint{3.824309in}{2.155728in}}%
\pgfusepath{stroke}%
\end{pgfscope}%
\begin{pgfscope}%
\pgfpathrectangle{\pgfqpoint{0.588387in}{0.521603in}}{\pgfqpoint{3.660036in}{2.220246in}}%
\pgfusepath{clip}%
\pgfsetrectcap%
\pgfsetroundjoin%
\pgfsetlinewidth{1.505625pt}%
\pgfsetstrokecolor{currentstroke3}%
\pgfsetdash{}{0pt}%
\pgfpathmoveto{\pgfqpoint{0.754752in}{0.722227in}}%
\pgfpathlineto{\pgfqpoint{0.778184in}{1.031008in}}%
\pgfpathlineto{\pgfqpoint{0.801616in}{0.705776in}}%
\pgfpathlineto{\pgfqpoint{0.825048in}{0.622893in}}%
\pgfpathlineto{\pgfqpoint{0.848479in}{0.759775in}}%
\pgfpathlineto{\pgfqpoint{0.871911in}{0.929663in}}%
\pgfpathlineto{\pgfqpoint{0.895343in}{1.055849in}}%
\pgfpathlineto{\pgfqpoint{0.918775in}{0.776936in}}%
\pgfpathlineto{\pgfqpoint{0.942206in}{0.833505in}}%
\pgfpathlineto{\pgfqpoint{0.965638in}{0.993930in}}%
\pgfpathlineto{\pgfqpoint{0.989070in}{1.131226in}}%
\pgfpathlineto{\pgfqpoint{1.012501in}{1.124064in}}%
\pgfpathlineto{\pgfqpoint{1.035933in}{0.971227in}}%
\pgfpathlineto{\pgfqpoint{1.059365in}{1.056521in}}%
\pgfpathlineto{\pgfqpoint{1.082797in}{1.173921in}}%
\pgfpathlineto{\pgfqpoint{1.106228in}{1.160461in}}%
\pgfpathlineto{\pgfqpoint{1.129660in}{1.233253in}}%
\pgfpathlineto{\pgfqpoint{1.153092in}{1.153001in}}%
\pgfpathlineto{\pgfqpoint{1.176524in}{1.218926in}}%
\pgfpathlineto{\pgfqpoint{1.199955in}{1.213112in}}%
\pgfpathlineto{\pgfqpoint{1.223387in}{1.301445in}}%
\pgfpathlineto{\pgfqpoint{1.246819in}{1.362667in}}%
\pgfpathlineto{\pgfqpoint{1.270250in}{1.300674in}}%
\pgfpathlineto{\pgfqpoint{1.293682in}{1.257469in}}%
\pgfpathlineto{\pgfqpoint{1.340546in}{1.411374in}}%
\pgfpathlineto{\pgfqpoint{1.363977in}{1.479998in}}%
\pgfpathlineto{\pgfqpoint{1.387409in}{1.338690in}}%
\pgfpathlineto{\pgfqpoint{1.410841in}{1.364828in}}%
\pgfpathlineto{\pgfqpoint{1.434273in}{1.431987in}}%
\pgfpathlineto{\pgfqpoint{1.457704in}{1.517894in}}%
\pgfpathlineto{\pgfqpoint{1.481136in}{1.508426in}}%
\pgfpathlineto{\pgfqpoint{1.504568in}{1.436955in}}%
\pgfpathlineto{\pgfqpoint{1.527999in}{1.478509in}}%
\pgfpathlineto{\pgfqpoint{1.551431in}{1.543487in}}%
\pgfpathlineto{\pgfqpoint{1.574863in}{1.529303in}}%
\pgfpathlineto{\pgfqpoint{1.598295in}{1.561953in}}%
\pgfpathlineto{\pgfqpoint{1.621726in}{1.537673in}}%
\pgfpathlineto{\pgfqpoint{1.645158in}{1.567810in}}%
\pgfpathlineto{\pgfqpoint{1.668590in}{1.564476in}}%
\pgfpathlineto{\pgfqpoint{1.692021in}{1.596827in}}%
\pgfpathlineto{\pgfqpoint{1.715453in}{1.664433in}}%
\pgfpathlineto{\pgfqpoint{1.738885in}{1.622476in}}%
\pgfpathlineto{\pgfqpoint{1.762317in}{1.594474in}}%
\pgfpathlineto{\pgfqpoint{1.785748in}{1.631737in}}%
\pgfpathlineto{\pgfqpoint{1.809180in}{1.694839in}}%
\pgfpathlineto{\pgfqpoint{1.832612in}{1.745553in}}%
\pgfpathlineto{\pgfqpoint{1.856044in}{1.643107in}}%
\pgfpathlineto{\pgfqpoint{1.879475in}{1.670196in}}%
\pgfpathlineto{\pgfqpoint{1.926339in}{1.757689in}}%
\pgfpathlineto{\pgfqpoint{1.949770in}{1.750490in}}%
\pgfpathlineto{\pgfqpoint{1.973202in}{1.698287in}}%
\pgfpathlineto{\pgfqpoint{1.996634in}{1.751629in}}%
\pgfpathlineto{\pgfqpoint{2.020066in}{1.810850in}}%
\pgfpathlineto{\pgfqpoint{2.043497in}{1.779025in}}%
\pgfpathlineto{\pgfqpoint{2.066929in}{1.804977in}}%
\pgfpathlineto{\pgfqpoint{2.090361in}{1.782221in}}%
\pgfpathlineto{\pgfqpoint{2.113793in}{1.802373in}}%
\pgfpathlineto{\pgfqpoint{2.137224in}{1.799941in}}%
\pgfpathlineto{\pgfqpoint{2.160656in}{1.835012in}}%
\pgfpathlineto{\pgfqpoint{2.184088in}{1.887952in}}%
\pgfpathlineto{\pgfqpoint{2.207519in}{1.845456in}}%
\pgfpathlineto{\pgfqpoint{2.230951in}{1.825029in}}%
\pgfpathlineto{\pgfqpoint{2.254383in}{1.867595in}}%
\pgfpathlineto{\pgfqpoint{2.277815in}{1.900294in}}%
\pgfpathlineto{\pgfqpoint{2.301246in}{1.924447in}}%
\pgfpathlineto{\pgfqpoint{2.324678in}{1.881761in}}%
\pgfpathlineto{\pgfqpoint{2.348110in}{1.881229in}}%
\pgfpathlineto{\pgfqpoint{2.371542in}{1.915013in}}%
\pgfpathlineto{\pgfqpoint{2.394973in}{1.951265in}}%
\pgfpathlineto{\pgfqpoint{2.418405in}{1.963646in}}%
\pgfpathlineto{\pgfqpoint{2.441837in}{1.916517in}}%
\pgfpathlineto{\pgfqpoint{2.465268in}{1.933841in}}%
\pgfpathlineto{\pgfqpoint{2.488700in}{1.978788in}}%
\pgfpathlineto{\pgfqpoint{2.512132in}{1.976742in}}%
\pgfpathlineto{\pgfqpoint{2.535564in}{1.996281in}}%
\pgfpathlineto{\pgfqpoint{2.558995in}{1.975234in}}%
\pgfpathlineto{\pgfqpoint{2.582427in}{2.024647in}}%
\pgfpathlineto{\pgfqpoint{2.605859in}{1.983379in}}%
\pgfpathlineto{\pgfqpoint{2.629291in}{2.007454in}}%
\pgfpathlineto{\pgfqpoint{2.652722in}{2.070391in}}%
\pgfpathlineto{\pgfqpoint{2.676154in}{2.060654in}}%
\pgfpathlineto{\pgfqpoint{2.699586in}{1.995175in}}%
\pgfpathlineto{\pgfqpoint{2.723017in}{2.026362in}}%
\pgfpathlineto{\pgfqpoint{2.746449in}{2.062249in}}%
\pgfpathlineto{\pgfqpoint{2.769881in}{2.079704in}}%
\pgfpathlineto{\pgfqpoint{2.793313in}{2.080256in}}%
\pgfpathlineto{\pgfqpoint{2.816744in}{2.087352in}}%
\pgfpathlineto{\pgfqpoint{2.840176in}{2.060979in}}%
\pgfpathlineto{\pgfqpoint{2.863608in}{2.107256in}}%
\pgfpathlineto{\pgfqpoint{2.887039in}{2.127643in}}%
\pgfpathlineto{\pgfqpoint{2.910471in}{2.057339in}}%
\pgfpathlineto{\pgfqpoint{2.933903in}{2.083842in}}%
\pgfpathlineto{\pgfqpoint{2.957335in}{2.098853in}}%
\pgfpathlineto{\pgfqpoint{2.980766in}{2.099678in}}%
\pgfpathlineto{\pgfqpoint{3.004198in}{2.122844in}}%
\pgfpathlineto{\pgfqpoint{3.027630in}{2.129048in}}%
\pgfpathlineto{\pgfqpoint{3.051062in}{2.148398in}}%
\pgfpathlineto{\pgfqpoint{3.074493in}{2.157787in}}%
\pgfpathlineto{\pgfqpoint{3.097925in}{2.153688in}}%
\pgfpathlineto{\pgfqpoint{3.121357in}{2.206857in}}%
\pgfpathlineto{\pgfqpoint{3.144788in}{2.145832in}}%
\pgfpathlineto{\pgfqpoint{3.168220in}{2.137057in}}%
\pgfpathlineto{\pgfqpoint{3.215084in}{2.290909in}}%
\pgfpathlineto{\pgfqpoint{3.238515in}{2.208925in}}%
\pgfpathlineto{\pgfqpoint{3.261947in}{2.175556in}}%
\pgfpathlineto{\pgfqpoint{3.285379in}{2.168670in}}%
\pgfpathlineto{\pgfqpoint{3.308811in}{2.203030in}}%
\pgfpathlineto{\pgfqpoint{3.332242in}{2.222915in}}%
\pgfpathlineto{\pgfqpoint{3.355674in}{2.245129in}}%
\pgfpathlineto{\pgfqpoint{3.402537in}{2.219970in}}%
\pgfpathlineto{\pgfqpoint{3.425969in}{2.268837in}}%
\pgfpathlineto{\pgfqpoint{3.449401in}{2.268438in}}%
\pgfpathlineto{\pgfqpoint{3.472833in}{2.247157in}}%
\pgfpathlineto{\pgfqpoint{3.496264in}{2.234749in}}%
\pgfpathlineto{\pgfqpoint{3.519696in}{2.261327in}}%
\pgfpathlineto{\pgfqpoint{3.543128in}{2.298776in}}%
\pgfpathlineto{\pgfqpoint{3.566560in}{2.351244in}}%
\pgfpathlineto{\pgfqpoint{3.589991in}{2.400656in}}%
\pgfpathlineto{\pgfqpoint{3.613423in}{2.280672in}}%
\pgfpathlineto{\pgfqpoint{3.636855in}{2.262792in}}%
\pgfpathlineto{\pgfqpoint{3.660286in}{2.310718in}}%
\pgfpathlineto{\pgfqpoint{3.683718in}{2.402476in}}%
\pgfpathlineto{\pgfqpoint{3.707150in}{2.396882in}}%
\pgfpathlineto{\pgfqpoint{3.730582in}{2.353397in}}%
\pgfpathlineto{\pgfqpoint{3.754013in}{2.465456in}}%
\pgfpathlineto{\pgfqpoint{3.777445in}{2.428071in}}%
\pgfpathlineto{\pgfqpoint{3.800877in}{2.511426in}}%
\pgfpathlineto{\pgfqpoint{3.824309in}{2.477136in}}%
\pgfpathlineto{\pgfqpoint{3.847740in}{2.480207in}}%
\pgfpathlineto{\pgfqpoint{3.871172in}{2.345616in}}%
\pgfpathlineto{\pgfqpoint{3.894604in}{2.524765in}}%
\pgfpathlineto{\pgfqpoint{3.918035in}{2.504140in}}%
\pgfpathlineto{\pgfqpoint{3.964899in}{2.497307in}}%
\pgfpathlineto{\pgfqpoint{4.011762in}{2.532853in}}%
\pgfpathlineto{\pgfqpoint{4.011762in}{2.532853in}}%
\pgfusepath{stroke}%
\end{pgfscope}%
\begin{pgfscope}%
\pgfpathrectangle{\pgfqpoint{0.588387in}{0.521603in}}{\pgfqpoint{3.660036in}{2.220246in}}%
\pgfusepath{clip}%
\pgfsetrectcap%
\pgfsetroundjoin%
\pgfsetlinewidth{1.505625pt}%
\pgfsetstrokecolor{currentstroke4}%
\pgfsetdash{}{0pt}%
\pgfpathmoveto{\pgfqpoint{0.754752in}{0.830635in}}%
\pgfpathlineto{\pgfqpoint{0.778184in}{1.139005in}}%
\pgfpathlineto{\pgfqpoint{0.801616in}{0.701688in}}%
\pgfpathlineto{\pgfqpoint{0.825048in}{0.677761in}}%
\pgfpathlineto{\pgfqpoint{0.848479in}{0.841081in}}%
\pgfpathlineto{\pgfqpoint{0.871911in}{1.013748in}}%
\pgfpathlineto{\pgfqpoint{0.895343in}{1.127363in}}%
\pgfpathlineto{\pgfqpoint{0.918775in}{0.821442in}}%
\pgfpathlineto{\pgfqpoint{0.942206in}{0.862287in}}%
\pgfpathlineto{\pgfqpoint{0.965638in}{1.043426in}}%
\pgfpathlineto{\pgfqpoint{0.989070in}{1.177447in}}%
\pgfpathlineto{\pgfqpoint{1.012501in}{1.187144in}}%
\pgfpathlineto{\pgfqpoint{1.035933in}{1.005888in}}%
\pgfpathlineto{\pgfqpoint{1.059365in}{1.085790in}}%
\pgfpathlineto{\pgfqpoint{1.082797in}{1.205720in}}%
\pgfpathlineto{\pgfqpoint{1.106228in}{1.220340in}}%
\pgfpathlineto{\pgfqpoint{1.129660in}{1.304635in}}%
\pgfpathlineto{\pgfqpoint{1.153092in}{1.180740in}}%
\pgfpathlineto{\pgfqpoint{1.176524in}{1.245570in}}%
\pgfpathlineto{\pgfqpoint{1.199955in}{1.262503in}}%
\pgfpathlineto{\pgfqpoint{1.223387in}{1.341682in}}%
\pgfpathlineto{\pgfqpoint{1.246819in}{1.404435in}}%
\pgfpathlineto{\pgfqpoint{1.270250in}{1.324470in}}%
\pgfpathlineto{\pgfqpoint{1.293682in}{1.286793in}}%
\pgfpathlineto{\pgfqpoint{1.317114in}{1.388275in}}%
\pgfpathlineto{\pgfqpoint{1.340546in}{1.438000in}}%
\pgfpathlineto{\pgfqpoint{1.363977in}{1.505035in}}%
\pgfpathlineto{\pgfqpoint{1.387409in}{1.359434in}}%
\pgfpathlineto{\pgfqpoint{1.410841in}{1.392137in}}%
\pgfpathlineto{\pgfqpoint{1.434273in}{1.436163in}}%
\pgfpathlineto{\pgfqpoint{1.457704in}{1.544653in}}%
\pgfpathlineto{\pgfqpoint{1.481136in}{1.541655in}}%
\pgfpathlineto{\pgfqpoint{1.504568in}{1.448533in}}%
\pgfpathlineto{\pgfqpoint{1.527999in}{1.490891in}}%
\pgfpathlineto{\pgfqpoint{1.551431in}{1.558227in}}%
\pgfpathlineto{\pgfqpoint{1.574863in}{1.574279in}}%
\pgfpathlineto{\pgfqpoint{1.598295in}{1.588048in}}%
\pgfpathlineto{\pgfqpoint{1.621726in}{1.540349in}}%
\pgfpathlineto{\pgfqpoint{1.645158in}{1.568872in}}%
\pgfpathlineto{\pgfqpoint{1.668590in}{1.580383in}}%
\pgfpathlineto{\pgfqpoint{1.692021in}{1.602636in}}%
\pgfpathlineto{\pgfqpoint{1.715453in}{1.686336in}}%
\pgfpathlineto{\pgfqpoint{1.738885in}{1.610204in}}%
\pgfpathlineto{\pgfqpoint{1.762317in}{1.582121in}}%
\pgfpathlineto{\pgfqpoint{1.785748in}{1.604242in}}%
\pgfpathlineto{\pgfqpoint{1.809180in}{1.712477in}}%
\pgfpathlineto{\pgfqpoint{1.832612in}{1.731738in}}%
\pgfpathlineto{\pgfqpoint{1.856044in}{1.643107in}}%
\pgfpathlineto{\pgfqpoint{1.879475in}{1.680647in}}%
\pgfpathlineto{\pgfqpoint{1.902907in}{1.686242in}}%
\pgfpathlineto{\pgfqpoint{1.926339in}{1.747106in}}%
\pgfpathlineto{\pgfqpoint{1.949770in}{1.732770in}}%
\pgfpathlineto{\pgfqpoint{1.973202in}{1.661705in}}%
\pgfpathlineto{\pgfqpoint{1.996634in}{1.748332in}}%
\pgfpathlineto{\pgfqpoint{2.020066in}{1.793631in}}%
\pgfpathlineto{\pgfqpoint{2.043497in}{1.779891in}}%
\pgfpathlineto{\pgfqpoint{2.066929in}{1.790684in}}%
\pgfpathlineto{\pgfqpoint{2.090361in}{1.793857in}}%
\pgfpathlineto{\pgfqpoint{2.113793in}{1.754308in}}%
\pgfpathlineto{\pgfqpoint{2.137224in}{1.799629in}}%
\pgfpathlineto{\pgfqpoint{2.160656in}{1.815561in}}%
\pgfpathlineto{\pgfqpoint{2.184088in}{1.866764in}}%
\pgfpathlineto{\pgfqpoint{2.207519in}{1.847060in}}%
\pgfpathlineto{\pgfqpoint{2.230951in}{1.805846in}}%
\pgfpathlineto{\pgfqpoint{2.254383in}{1.828831in}}%
\pgfpathlineto{\pgfqpoint{2.277815in}{1.892412in}}%
\pgfpathlineto{\pgfqpoint{2.301246in}{1.897059in}}%
\pgfpathlineto{\pgfqpoint{2.324678in}{1.840155in}}%
\pgfpathlineto{\pgfqpoint{2.348110in}{1.829105in}}%
\pgfpathlineto{\pgfqpoint{2.371542in}{1.906264in}}%
\pgfpathlineto{\pgfqpoint{2.394973in}{1.889339in}}%
\pgfpathlineto{\pgfqpoint{2.418405in}{1.881767in}}%
\pgfpathlineto{\pgfqpoint{2.441837in}{1.850865in}}%
\pgfpathlineto{\pgfqpoint{2.465268in}{1.919589in}}%
\pgfpathlineto{\pgfqpoint{2.488700in}{1.898881in}}%
\pgfpathlineto{\pgfqpoint{2.512132in}{1.937930in}}%
\pgfpathlineto{\pgfqpoint{2.535564in}{1.983995in}}%
\pgfpathlineto{\pgfqpoint{2.558995in}{1.934028in}}%
\pgfpathlineto{\pgfqpoint{2.582427in}{1.987174in}}%
\pgfpathlineto{\pgfqpoint{2.605859in}{1.926960in}}%
\pgfpathlineto{\pgfqpoint{2.629291in}{1.955420in}}%
\pgfpathlineto{\pgfqpoint{2.652722in}{2.066657in}}%
\pgfpathlineto{\pgfqpoint{2.676154in}{2.091864in}}%
\pgfpathlineto{\pgfqpoint{2.699586in}{1.915393in}}%
\pgfpathlineto{\pgfqpoint{2.723017in}{1.928355in}}%
\pgfpathlineto{\pgfqpoint{2.746449in}{2.059273in}}%
\pgfpathlineto{\pgfqpoint{2.769881in}{2.038566in}}%
\pgfpathlineto{\pgfqpoint{2.793313in}{2.000564in}}%
\pgfpathlineto{\pgfqpoint{2.816744in}{2.109778in}}%
\pgfpathlineto{\pgfqpoint{2.840176in}{1.961226in}}%
\pgfpathlineto{\pgfqpoint{2.863608in}{1.990267in}}%
\pgfpathlineto{\pgfqpoint{2.887039in}{2.072190in}}%
\pgfpathlineto{\pgfqpoint{2.910471in}{1.979005in}}%
\pgfpathlineto{\pgfqpoint{2.957335in}{2.068293in}}%
\pgfpathlineto{\pgfqpoint{2.980766in}{1.927996in}}%
\pgfpathlineto{\pgfqpoint{3.004198in}{2.001032in}}%
\pgfpathlineto{\pgfqpoint{3.051062in}{2.055725in}}%
\pgfpathlineto{\pgfqpoint{3.074493in}{2.099017in}}%
\pgfpathlineto{\pgfqpoint{3.097925in}{2.106554in}}%
\pgfpathlineto{\pgfqpoint{3.121357in}{2.020753in}}%
\pgfpathlineto{\pgfqpoint{3.144788in}{2.023410in}}%
\pgfpathlineto{\pgfqpoint{3.168220in}{1.990004in}}%
\pgfpathlineto{\pgfqpoint{3.191652in}{2.015615in}}%
\pgfpathlineto{\pgfqpoint{3.215084in}{2.341131in}}%
\pgfpathlineto{\pgfqpoint{3.238515in}{2.049062in}}%
\pgfpathlineto{\pgfqpoint{3.261947in}{2.009401in}}%
\pgfpathlineto{\pgfqpoint{3.285379in}{2.016303in}}%
\pgfpathlineto{\pgfqpoint{3.308811in}{2.148906in}}%
\pgfpathlineto{\pgfqpoint{3.332242in}{2.079029in}}%
\pgfpathlineto{\pgfqpoint{3.355674in}{2.054051in}}%
\pgfpathlineto{\pgfqpoint{3.379106in}{2.049849in}}%
\pgfpathlineto{\pgfqpoint{3.402537in}{2.069251in}}%
\pgfpathlineto{\pgfqpoint{3.425969in}{2.086900in}}%
\pgfpathlineto{\pgfqpoint{3.449401in}{2.057673in}}%
\pgfpathlineto{\pgfqpoint{3.472833in}{2.150047in}}%
\pgfpathlineto{\pgfqpoint{3.496264in}{2.091864in}}%
\pgfpathlineto{\pgfqpoint{3.519696in}{2.108369in}}%
\pgfpathlineto{\pgfqpoint{3.543128in}{2.106554in}}%
\pgfpathlineto{\pgfqpoint{3.566560in}{2.102909in}}%
\pgfpathlineto{\pgfqpoint{3.589991in}{2.184024in}}%
\pgfpathlineto{\pgfqpoint{3.613423in}{2.135063in}}%
\pgfpathlineto{\pgfqpoint{3.636855in}{2.110180in}}%
\pgfpathlineto{\pgfqpoint{3.660286in}{2.116183in}}%
\pgfpathlineto{\pgfqpoint{3.707150in}{2.152325in}}%
\pgfpathlineto{\pgfqpoint{3.730582in}{2.125685in}}%
\pgfpathlineto{\pgfqpoint{3.777445in}{2.162487in}}%
\pgfpathlineto{\pgfqpoint{3.824309in}{2.155728in}}%
\pgfpathlineto{\pgfqpoint{3.918035in}{2.162487in}}%
\pgfpathlineto{\pgfqpoint{3.918035in}{2.162487in}}%
\pgfusepath{stroke}%
\end{pgfscope}%
\begin{pgfscope}%
\pgfpathrectangle{\pgfqpoint{0.588387in}{0.521603in}}{\pgfqpoint{3.660036in}{2.220246in}}%
\pgfusepath{clip}%
\pgfsetrectcap%
\pgfsetroundjoin%
\pgfsetlinewidth{1.505625pt}%
\pgfsetstrokecolor{currentstroke5}%
\pgfsetdash{}{0pt}%
\pgfpathmoveto{\pgfqpoint{0.754752in}{0.819448in}}%
\pgfpathlineto{\pgfqpoint{0.778184in}{1.122976in}}%
\pgfpathlineto{\pgfqpoint{0.801616in}{0.697089in}}%
\pgfpathlineto{\pgfqpoint{0.825048in}{0.682630in}}%
\pgfpathlineto{\pgfqpoint{0.848479in}{0.851381in}}%
\pgfpathlineto{\pgfqpoint{0.871911in}{1.014335in}}%
\pgfpathlineto{\pgfqpoint{0.895343in}{1.135282in}}%
\pgfpathlineto{\pgfqpoint{0.918775in}{0.830392in}}%
\pgfpathlineto{\pgfqpoint{0.942206in}{0.898188in}}%
\pgfpathlineto{\pgfqpoint{0.965638in}{1.059232in}}%
\pgfpathlineto{\pgfqpoint{0.989070in}{1.199583in}}%
\pgfpathlineto{\pgfqpoint{1.012501in}{1.243153in}}%
\pgfpathlineto{\pgfqpoint{1.035933in}{1.023141in}}%
\pgfpathlineto{\pgfqpoint{1.059365in}{1.100215in}}%
\pgfpathlineto{\pgfqpoint{1.082797in}{1.221780in}}%
\pgfpathlineto{\pgfqpoint{1.106228in}{1.251357in}}%
\pgfpathlineto{\pgfqpoint{1.129660in}{1.347915in}}%
\pgfpathlineto{\pgfqpoint{1.153092in}{1.194363in}}%
\pgfpathlineto{\pgfqpoint{1.176524in}{1.252942in}}%
\pgfpathlineto{\pgfqpoint{1.199955in}{1.283686in}}%
\pgfpathlineto{\pgfqpoint{1.223387in}{1.378794in}}%
\pgfpathlineto{\pgfqpoint{1.246819in}{1.451559in}}%
\pgfpathlineto{\pgfqpoint{1.270250in}{1.331204in}}%
\pgfpathlineto{\pgfqpoint{1.293682in}{1.315216in}}%
\pgfpathlineto{\pgfqpoint{1.317114in}{1.402516in}}%
\pgfpathlineto{\pgfqpoint{1.340546in}{1.482262in}}%
\pgfpathlineto{\pgfqpoint{1.363977in}{1.558559in}}%
\pgfpathlineto{\pgfqpoint{1.387409in}{1.403934in}}%
\pgfpathlineto{\pgfqpoint{1.410841in}{1.421556in}}%
\pgfpathlineto{\pgfqpoint{1.434273in}{1.491641in}}%
\pgfpathlineto{\pgfqpoint{1.457704in}{1.582400in}}%
\pgfpathlineto{\pgfqpoint{1.481136in}{1.612422in}}%
\pgfpathlineto{\pgfqpoint{1.504568in}{1.487362in}}%
\pgfpathlineto{\pgfqpoint{1.527999in}{1.524253in}}%
\pgfpathlineto{\pgfqpoint{1.551431in}{1.587797in}}%
\pgfpathlineto{\pgfqpoint{1.574863in}{1.603186in}}%
\pgfpathlineto{\pgfqpoint{1.598295in}{1.650680in}}%
\pgfpathlineto{\pgfqpoint{1.621726in}{1.581091in}}%
\pgfpathlineto{\pgfqpoint{1.645158in}{1.644327in}}%
\pgfpathlineto{\pgfqpoint{1.668590in}{1.624351in}}%
\pgfpathlineto{\pgfqpoint{1.692021in}{1.670972in}}%
\pgfpathlineto{\pgfqpoint{1.715453in}{1.759120in}}%
\pgfpathlineto{\pgfqpoint{1.738885in}{1.664564in}}%
\pgfpathlineto{\pgfqpoint{1.762317in}{1.637495in}}%
\pgfpathlineto{\pgfqpoint{1.785748in}{1.690712in}}%
\pgfpathlineto{\pgfqpoint{1.809180in}{1.766372in}}%
\pgfpathlineto{\pgfqpoint{1.832612in}{1.827321in}}%
\pgfpathlineto{\pgfqpoint{1.856044in}{1.706658in}}%
\pgfpathlineto{\pgfqpoint{1.879475in}{1.730341in}}%
\pgfpathlineto{\pgfqpoint{1.902907in}{1.791623in}}%
\pgfpathlineto{\pgfqpoint{1.926339in}{1.807344in}}%
\pgfpathlineto{\pgfqpoint{1.949770in}{1.842199in}}%
\pgfpathlineto{\pgfqpoint{1.973202in}{1.744606in}}%
\pgfpathlineto{\pgfqpoint{1.996634in}{1.794316in}}%
\pgfpathlineto{\pgfqpoint{2.020066in}{1.932640in}}%
\pgfpathlineto{\pgfqpoint{2.043497in}{1.849072in}}%
\pgfpathlineto{\pgfqpoint{2.066929in}{1.858987in}}%
\pgfpathlineto{\pgfqpoint{2.090361in}{1.841733in}}%
\pgfpathlineto{\pgfqpoint{2.113793in}{1.840418in}}%
\pgfpathlineto{\pgfqpoint{2.137224in}{1.840531in}}%
\pgfpathlineto{\pgfqpoint{2.160656in}{1.914714in}}%
\pgfpathlineto{\pgfqpoint{2.184088in}{1.918146in}}%
\pgfpathlineto{\pgfqpoint{2.207519in}{1.866954in}}%
\pgfpathlineto{\pgfqpoint{2.230951in}{1.896451in}}%
\pgfpathlineto{\pgfqpoint{2.254383in}{1.912694in}}%
\pgfpathlineto{\pgfqpoint{2.277815in}{1.984303in}}%
\pgfpathlineto{\pgfqpoint{2.301246in}{1.979491in}}%
\pgfpathlineto{\pgfqpoint{2.324678in}{1.944689in}}%
\pgfpathlineto{\pgfqpoint{2.348110in}{1.914674in}}%
\pgfpathlineto{\pgfqpoint{2.371542in}{2.054330in}}%
\pgfpathlineto{\pgfqpoint{2.394973in}{2.016903in}}%
\pgfpathlineto{\pgfqpoint{2.418405in}{1.986147in}}%
\pgfpathlineto{\pgfqpoint{2.441837in}{1.932061in}}%
\pgfpathlineto{\pgfqpoint{2.465268in}{2.016028in}}%
\pgfpathlineto{\pgfqpoint{2.488700in}{1.979672in}}%
\pgfpathlineto{\pgfqpoint{2.512132in}{2.112887in}}%
\pgfpathlineto{\pgfqpoint{2.535564in}{2.037523in}}%
\pgfpathlineto{\pgfqpoint{2.558995in}{2.006274in}}%
\pgfpathlineto{\pgfqpoint{2.582427in}{2.140863in}}%
\pgfpathlineto{\pgfqpoint{2.605859in}{2.062520in}}%
\pgfpathlineto{\pgfqpoint{2.629291in}{2.102909in}}%
\pgfpathlineto{\pgfqpoint{2.652722in}{2.202075in}}%
\pgfpathlineto{\pgfqpoint{2.676154in}{2.185654in}}%
\pgfpathlineto{\pgfqpoint{2.699586in}{2.049357in}}%
\pgfpathlineto{\pgfqpoint{2.723017in}{2.063485in}}%
\pgfpathlineto{\pgfqpoint{2.746449in}{2.252972in}}%
\pgfpathlineto{\pgfqpoint{2.769881in}{2.200960in}}%
\pgfpathlineto{\pgfqpoint{2.793313in}{2.065412in}}%
\pgfpathlineto{\pgfqpoint{2.816744in}{2.186739in}}%
\pgfpathlineto{\pgfqpoint{2.840176in}{2.073070in}}%
\pgfpathlineto{\pgfqpoint{2.863608in}{2.131561in}}%
\pgfpathlineto{\pgfqpoint{2.887039in}{2.200362in}}%
\pgfpathlineto{\pgfqpoint{2.910471in}{2.104278in}}%
\pgfpathlineto{\pgfqpoint{2.933903in}{2.113788in}}%
\pgfpathlineto{\pgfqpoint{2.957335in}{2.196162in}}%
\pgfpathlineto{\pgfqpoint{3.004198in}{2.388638in}}%
\pgfpathlineto{\pgfqpoint{3.027630in}{2.598157in}}%
\pgfpathlineto{\pgfqpoint{3.051062in}{2.210463in}}%
\pgfpathlineto{\pgfqpoint{3.074493in}{2.329796in}}%
\pgfpathlineto{\pgfqpoint{3.097925in}{2.309519in}}%
\pgfpathlineto{\pgfqpoint{3.215084in}{2.571981in}}%
\pgfpathlineto{\pgfqpoint{3.308811in}{2.242013in}}%
\pgfpathlineto{\pgfqpoint{3.472833in}{2.424575in}}%
\pgfpathlineto{\pgfqpoint{3.566560in}{2.452085in}}%
\pgfusepath{stroke}%
\end{pgfscope}%
\begin{pgfscope}%
\pgfpathrectangle{\pgfqpoint{0.588387in}{0.521603in}}{\pgfqpoint{3.660036in}{2.220246in}}%
\pgfusepath{clip}%
\pgfsetrectcap%
\pgfsetroundjoin%
\pgfsetlinewidth{1.505625pt}%
\pgfsetstrokecolor{currentstroke6}%
\pgfsetdash{}{0pt}%
\pgfpathmoveto{\pgfqpoint{0.754752in}{0.808740in}}%
\pgfpathlineto{\pgfqpoint{0.778184in}{1.131751in}}%
\pgfpathlineto{\pgfqpoint{0.801616in}{0.697066in}}%
\pgfpathlineto{\pgfqpoint{0.825048in}{0.682834in}}%
\pgfpathlineto{\pgfqpoint{0.848479in}{0.851938in}}%
\pgfpathlineto{\pgfqpoint{0.871911in}{1.014335in}}%
\pgfpathlineto{\pgfqpoint{0.895343in}{1.134721in}}%
\pgfpathlineto{\pgfqpoint{0.918775in}{0.832950in}}%
\pgfpathlineto{\pgfqpoint{0.942206in}{0.898542in}}%
\pgfpathlineto{\pgfqpoint{0.965638in}{1.062293in}}%
\pgfpathlineto{\pgfqpoint{0.989070in}{1.202842in}}%
\pgfpathlineto{\pgfqpoint{1.012501in}{1.243549in}}%
\pgfpathlineto{\pgfqpoint{1.035933in}{1.024069in}}%
\pgfpathlineto{\pgfqpoint{1.059365in}{1.100215in}}%
\pgfpathlineto{\pgfqpoint{1.082797in}{1.220955in}}%
\pgfpathlineto{\pgfqpoint{1.106228in}{1.253083in}}%
\pgfpathlineto{\pgfqpoint{1.129660in}{1.345961in}}%
\pgfpathlineto{\pgfqpoint{1.153092in}{1.194363in}}%
\pgfpathlineto{\pgfqpoint{1.176524in}{1.254757in}}%
\pgfpathlineto{\pgfqpoint{1.199955in}{1.285980in}}%
\pgfpathlineto{\pgfqpoint{1.223387in}{1.375259in}}%
\pgfpathlineto{\pgfqpoint{1.246819in}{1.451903in}}%
\pgfpathlineto{\pgfqpoint{1.270250in}{1.336467in}}%
\pgfpathlineto{\pgfqpoint{1.293682in}{1.316379in}}%
\pgfpathlineto{\pgfqpoint{1.317114in}{1.402168in}}%
\pgfpathlineto{\pgfqpoint{1.340546in}{1.480182in}}%
\pgfpathlineto{\pgfqpoint{1.363977in}{1.548379in}}%
\pgfpathlineto{\pgfqpoint{1.387409in}{1.398823in}}%
\pgfpathlineto{\pgfqpoint{1.410841in}{1.421339in}}%
\pgfpathlineto{\pgfqpoint{1.434273in}{1.491641in}}%
\pgfpathlineto{\pgfqpoint{1.457704in}{1.579441in}}%
\pgfpathlineto{\pgfqpoint{1.481136in}{1.597673in}}%
\pgfpathlineto{\pgfqpoint{1.504568in}{1.493068in}}%
\pgfpathlineto{\pgfqpoint{1.527999in}{1.523019in}}%
\pgfpathlineto{\pgfqpoint{1.551431in}{1.586262in}}%
\pgfpathlineto{\pgfqpoint{1.574863in}{1.608046in}}%
\pgfpathlineto{\pgfqpoint{1.598295in}{1.646283in}}%
\pgfpathlineto{\pgfqpoint{1.621726in}{1.574808in}}%
\pgfpathlineto{\pgfqpoint{1.645158in}{1.612756in}}%
\pgfpathlineto{\pgfqpoint{1.668590in}{1.623996in}}%
\pgfpathlineto{\pgfqpoint{1.692021in}{1.669569in}}%
\pgfpathlineto{\pgfqpoint{1.715453in}{1.739757in}}%
\pgfpathlineto{\pgfqpoint{1.738885in}{1.651002in}}%
\pgfpathlineto{\pgfqpoint{1.762317in}{1.638567in}}%
\pgfpathlineto{\pgfqpoint{1.785748in}{1.690712in}}%
\pgfpathlineto{\pgfqpoint{1.809180in}{1.771839in}}%
\pgfpathlineto{\pgfqpoint{1.832612in}{1.813117in}}%
\pgfpathlineto{\pgfqpoint{1.856044in}{1.704542in}}%
\pgfpathlineto{\pgfqpoint{1.879475in}{1.718686in}}%
\pgfpathlineto{\pgfqpoint{1.902907in}{1.785024in}}%
\pgfpathlineto{\pgfqpoint{1.926339in}{1.807344in}}%
\pgfpathlineto{\pgfqpoint{1.949770in}{1.828065in}}%
\pgfpathlineto{\pgfqpoint{1.973202in}{1.744606in}}%
\pgfpathlineto{\pgfqpoint{1.996634in}{1.785852in}}%
\pgfpathlineto{\pgfqpoint{2.020066in}{1.887090in}}%
\pgfpathlineto{\pgfqpoint{2.043497in}{1.842901in}}%
\pgfpathlineto{\pgfqpoint{2.066929in}{1.858987in}}%
\pgfpathlineto{\pgfqpoint{2.090361in}{1.836109in}}%
\pgfpathlineto{\pgfqpoint{2.113793in}{1.840418in}}%
\pgfpathlineto{\pgfqpoint{2.137224in}{1.840531in}}%
\pgfpathlineto{\pgfqpoint{2.160656in}{1.900577in}}%
\pgfpathlineto{\pgfqpoint{2.184088in}{1.920033in}}%
\pgfpathlineto{\pgfqpoint{2.207519in}{1.864414in}}%
\pgfpathlineto{\pgfqpoint{2.230951in}{1.863459in}}%
\pgfpathlineto{\pgfqpoint{2.254383in}{1.893402in}}%
\pgfpathlineto{\pgfqpoint{2.277815in}{1.968907in}}%
\pgfpathlineto{\pgfqpoint{2.301246in}{1.975141in}}%
\pgfpathlineto{\pgfqpoint{2.324678in}{1.915662in}}%
\pgfpathlineto{\pgfqpoint{2.348110in}{1.914674in}}%
\pgfpathlineto{\pgfqpoint{2.371542in}{2.002834in}}%
\pgfpathlineto{\pgfqpoint{2.394973in}{2.008621in}}%
\pgfpathlineto{\pgfqpoint{2.418405in}{1.986147in}}%
\pgfpathlineto{\pgfqpoint{2.441837in}{1.932061in}}%
\pgfpathlineto{\pgfqpoint{2.465268in}{1.995963in}}%
\pgfpathlineto{\pgfqpoint{2.488700in}{1.979672in}}%
\pgfpathlineto{\pgfqpoint{2.512132in}{2.100164in}}%
\pgfpathlineto{\pgfqpoint{2.535564in}{2.030631in}}%
\pgfpathlineto{\pgfqpoint{2.558995in}{1.991072in}}%
\pgfpathlineto{\pgfqpoint{2.582427in}{2.142018in}}%
\pgfpathlineto{\pgfqpoint{2.605859in}{2.024837in}}%
\pgfpathlineto{\pgfqpoint{2.629291in}{2.078760in}}%
\pgfpathlineto{\pgfqpoint{2.652722in}{2.200322in}}%
\pgfpathlineto{\pgfqpoint{2.676154in}{2.214385in}}%
\pgfpathlineto{\pgfqpoint{2.699586in}{2.031424in}}%
\pgfpathlineto{\pgfqpoint{2.723017in}{2.063485in}}%
\pgfpathlineto{\pgfqpoint{2.746449in}{2.172783in}}%
\pgfpathlineto{\pgfqpoint{2.769881in}{2.179384in}}%
\pgfpathlineto{\pgfqpoint{2.793313in}{2.088144in}}%
\pgfpathlineto{\pgfqpoint{2.816744in}{2.191062in}}%
\pgfpathlineto{\pgfqpoint{2.840176in}{2.073070in}}%
\pgfpathlineto{\pgfqpoint{2.863608in}{2.131561in}}%
\pgfpathlineto{\pgfqpoint{2.887039in}{2.212034in}}%
\pgfpathlineto{\pgfqpoint{2.910471in}{2.132876in}}%
\pgfpathlineto{\pgfqpoint{2.933903in}{2.113788in}}%
\pgfpathlineto{\pgfqpoint{2.957335in}{2.158834in}}%
\pgfpathlineto{\pgfqpoint{3.004198in}{2.175816in}}%
\pgfpathlineto{\pgfqpoint{3.027630in}{2.271516in}}%
\pgfpathlineto{\pgfqpoint{3.051062in}{2.165005in}}%
\pgfpathlineto{\pgfqpoint{3.074493in}{2.264253in}}%
\pgfpathlineto{\pgfqpoint{3.097925in}{2.248011in}}%
\pgfpathlineto{\pgfqpoint{3.215084in}{2.346919in}}%
\pgfpathlineto{\pgfqpoint{3.308811in}{2.242013in}}%
\pgfpathlineto{\pgfqpoint{3.472833in}{2.271516in}}%
\pgfusepath{stroke}%
\end{pgfscope}%
\begin{pgfscope}%
\pgfpathrectangle{\pgfqpoint{0.588387in}{0.521603in}}{\pgfqpoint{3.660036in}{2.220246in}}%
\pgfusepath{clip}%
\pgfsetrectcap%
\pgfsetroundjoin%
\pgfsetlinewidth{1.505625pt}%
\pgfsetstrokecolor{currentstroke7}%
\pgfsetdash{}{0pt}%
\pgfpathmoveto{\pgfqpoint{0.754752in}{0.740045in}}%
\pgfpathlineto{\pgfqpoint{0.778184in}{1.036119in}}%
\pgfpathlineto{\pgfqpoint{0.801616in}{0.707609in}}%
\pgfpathlineto{\pgfqpoint{0.825048in}{0.622524in}}%
\pgfpathlineto{\pgfqpoint{0.848479in}{0.769049in}}%
\pgfpathlineto{\pgfqpoint{0.871911in}{0.931633in}}%
\pgfpathlineto{\pgfqpoint{0.895343in}{1.055643in}}%
\pgfpathlineto{\pgfqpoint{0.918775in}{0.780636in}}%
\pgfpathlineto{\pgfqpoint{0.942206in}{0.834470in}}%
\pgfpathlineto{\pgfqpoint{0.965638in}{0.998423in}}%
\pgfpathlineto{\pgfqpoint{0.989070in}{1.137967in}}%
\pgfpathlineto{\pgfqpoint{1.012501in}{1.140643in}}%
\pgfpathlineto{\pgfqpoint{1.035933in}{0.978029in}}%
\pgfpathlineto{\pgfqpoint{1.059365in}{1.063999in}}%
\pgfpathlineto{\pgfqpoint{1.082797in}{1.179370in}}%
\pgfpathlineto{\pgfqpoint{1.106228in}{1.164664in}}%
\pgfpathlineto{\pgfqpoint{1.129660in}{1.237096in}}%
\pgfpathlineto{\pgfqpoint{1.153092in}{1.158592in}}%
\pgfpathlineto{\pgfqpoint{1.176524in}{1.223873in}}%
\pgfpathlineto{\pgfqpoint{1.199955in}{1.228768in}}%
\pgfpathlineto{\pgfqpoint{1.223387in}{1.291602in}}%
\pgfpathlineto{\pgfqpoint{1.246819in}{1.387499in}}%
\pgfpathlineto{\pgfqpoint{1.270250in}{1.320770in}}%
\pgfpathlineto{\pgfqpoint{1.293682in}{1.265554in}}%
\pgfpathlineto{\pgfqpoint{1.317114in}{1.370072in}}%
\pgfpathlineto{\pgfqpoint{1.340546in}{1.411763in}}%
\pgfpathlineto{\pgfqpoint{1.363977in}{1.488439in}}%
\pgfpathlineto{\pgfqpoint{1.387409in}{1.386514in}}%
\pgfpathlineto{\pgfqpoint{1.410841in}{1.376012in}}%
\pgfpathlineto{\pgfqpoint{1.434273in}{1.434712in}}%
\pgfpathlineto{\pgfqpoint{1.457704in}{1.534736in}}%
\pgfpathlineto{\pgfqpoint{1.481136in}{1.529028in}}%
\pgfpathlineto{\pgfqpoint{1.504568in}{1.480250in}}%
\pgfpathlineto{\pgfqpoint{1.527999in}{1.533211in}}%
\pgfpathlineto{\pgfqpoint{1.551431in}{1.541497in}}%
\pgfpathlineto{\pgfqpoint{1.574863in}{1.563369in}}%
\pgfpathlineto{\pgfqpoint{1.598295in}{1.568260in}}%
\pgfpathlineto{\pgfqpoint{1.621726in}{1.564188in}}%
\pgfpathlineto{\pgfqpoint{1.645158in}{1.595317in}}%
\pgfpathlineto{\pgfqpoint{1.668590in}{1.582948in}}%
\pgfpathlineto{\pgfqpoint{1.692021in}{1.604205in}}%
\pgfpathlineto{\pgfqpoint{1.715453in}{1.693688in}}%
\pgfpathlineto{\pgfqpoint{1.738885in}{1.625904in}}%
\pgfpathlineto{\pgfqpoint{1.762317in}{1.614573in}}%
\pgfpathlineto{\pgfqpoint{1.785748in}{1.636787in}}%
\pgfpathlineto{\pgfqpoint{1.809180in}{1.713942in}}%
\pgfpathlineto{\pgfqpoint{1.832612in}{1.751304in}}%
\pgfpathlineto{\pgfqpoint{1.856044in}{1.671736in}}%
\pgfpathlineto{\pgfqpoint{1.879475in}{1.680089in}}%
\pgfpathlineto{\pgfqpoint{1.902907in}{1.750204in}}%
\pgfpathlineto{\pgfqpoint{1.926339in}{1.768048in}}%
\pgfpathlineto{\pgfqpoint{1.949770in}{1.755900in}}%
\pgfpathlineto{\pgfqpoint{1.973202in}{1.713008in}}%
\pgfpathlineto{\pgfqpoint{1.996634in}{1.759049in}}%
\pgfpathlineto{\pgfqpoint{2.020066in}{1.855244in}}%
\pgfpathlineto{\pgfqpoint{2.043497in}{1.806950in}}%
\pgfpathlineto{\pgfqpoint{2.066929in}{1.820633in}}%
\pgfpathlineto{\pgfqpoint{2.090361in}{1.803591in}}%
\pgfpathlineto{\pgfqpoint{2.113793in}{1.808014in}}%
\pgfpathlineto{\pgfqpoint{2.137224in}{1.824389in}}%
\pgfpathlineto{\pgfqpoint{2.160656in}{1.859627in}}%
\pgfpathlineto{\pgfqpoint{2.184088in}{1.906599in}}%
\pgfpathlineto{\pgfqpoint{2.207519in}{1.877773in}}%
\pgfpathlineto{\pgfqpoint{2.230951in}{1.881854in}}%
\pgfpathlineto{\pgfqpoint{2.254383in}{1.872643in}}%
\pgfpathlineto{\pgfqpoint{2.277815in}{1.921170in}}%
\pgfpathlineto{\pgfqpoint{2.301246in}{1.943756in}}%
\pgfpathlineto{\pgfqpoint{2.324678in}{1.896086in}}%
\pgfpathlineto{\pgfqpoint{2.348110in}{1.895397in}}%
\pgfpathlineto{\pgfqpoint{2.371542in}{2.157666in}}%
\pgfpathlineto{\pgfqpoint{2.394973in}{1.963019in}}%
\pgfpathlineto{\pgfqpoint{2.418405in}{1.981762in}}%
\pgfpathlineto{\pgfqpoint{2.441837in}{1.948158in}}%
\pgfpathlineto{\pgfqpoint{2.465268in}{1.946045in}}%
\pgfpathlineto{\pgfqpoint{2.488700in}{1.990194in}}%
\pgfpathlineto{\pgfqpoint{2.512132in}{2.007173in}}%
\pgfpathlineto{\pgfqpoint{2.535564in}{2.026224in}}%
\pgfpathlineto{\pgfqpoint{2.558995in}{2.010335in}}%
\pgfpathlineto{\pgfqpoint{2.582427in}{2.022884in}}%
\pgfpathlineto{\pgfqpoint{2.605859in}{1.981263in}}%
\pgfpathlineto{\pgfqpoint{2.629291in}{2.020708in}}%
\pgfpathlineto{\pgfqpoint{2.652722in}{2.086635in}}%
\pgfpathlineto{\pgfqpoint{2.676154in}{2.102387in}}%
\pgfpathlineto{\pgfqpoint{2.699586in}{2.051533in}}%
\pgfpathlineto{\pgfqpoint{2.723017in}{2.077556in}}%
\pgfpathlineto{\pgfqpoint{2.746449in}{2.083805in}}%
\pgfpathlineto{\pgfqpoint{2.769881in}{2.087805in}}%
\pgfpathlineto{\pgfqpoint{2.793313in}{2.106303in}}%
\pgfpathlineto{\pgfqpoint{2.816744in}{2.117703in}}%
\pgfpathlineto{\pgfqpoint{2.840176in}{2.058968in}}%
\pgfpathlineto{\pgfqpoint{2.863608in}{2.112478in}}%
\pgfpathlineto{\pgfqpoint{2.887039in}{2.129406in}}%
\pgfpathlineto{\pgfqpoint{2.910471in}{2.117684in}}%
\pgfpathlineto{\pgfqpoint{2.933903in}{2.108006in}}%
\pgfpathlineto{\pgfqpoint{2.957335in}{2.133064in}}%
\pgfpathlineto{\pgfqpoint{2.980766in}{2.107408in}}%
\pgfpathlineto{\pgfqpoint{3.004198in}{2.196602in}}%
\pgfpathlineto{\pgfqpoint{3.027630in}{2.194134in}}%
\pgfpathlineto{\pgfqpoint{3.051062in}{2.206226in}}%
\pgfpathlineto{\pgfqpoint{3.074493in}{2.198108in}}%
\pgfpathlineto{\pgfqpoint{3.097925in}{2.184400in}}%
\pgfpathlineto{\pgfqpoint{3.121357in}{2.174400in}}%
\pgfpathlineto{\pgfqpoint{3.144788in}{2.168480in}}%
\pgfpathlineto{\pgfqpoint{3.168220in}{2.144978in}}%
\pgfpathlineto{\pgfqpoint{3.191652in}{2.239667in}}%
\pgfpathlineto{\pgfqpoint{3.215084in}{2.497511in}}%
\pgfpathlineto{\pgfqpoint{3.238515in}{2.218249in}}%
\pgfpathlineto{\pgfqpoint{3.261947in}{2.207434in}}%
\pgfpathlineto{\pgfqpoint{3.285379in}{2.178352in}}%
\pgfpathlineto{\pgfqpoint{3.308811in}{2.238131in}}%
\pgfpathlineto{\pgfqpoint{3.332242in}{2.238188in}}%
\pgfpathlineto{\pgfqpoint{3.355674in}{2.353486in}}%
\pgfpathlineto{\pgfqpoint{3.379106in}{2.196964in}}%
\pgfpathlineto{\pgfqpoint{3.402537in}{2.227308in}}%
\pgfpathlineto{\pgfqpoint{3.425969in}{2.254229in}}%
\pgfpathlineto{\pgfqpoint{3.449401in}{2.262060in}}%
\pgfpathlineto{\pgfqpoint{3.472833in}{2.267564in}}%
\pgfpathlineto{\pgfqpoint{3.496264in}{2.263766in}}%
\pgfpathlineto{\pgfqpoint{3.519696in}{2.277911in}}%
\pgfpathlineto{\pgfqpoint{3.543128in}{2.298593in}}%
\pgfpathlineto{\pgfqpoint{3.566560in}{2.424575in}}%
\pgfpathlineto{\pgfqpoint{3.589991in}{2.370937in}}%
\pgfpathlineto{\pgfqpoint{3.613423in}{2.277848in}}%
\pgfpathlineto{\pgfqpoint{3.636855in}{2.287370in}}%
\pgfpathlineto{\pgfqpoint{3.660286in}{2.280656in}}%
\pgfpathlineto{\pgfqpoint{3.683718in}{2.410415in}}%
\pgfpathlineto{\pgfqpoint{3.707150in}{2.412200in}}%
\pgfpathlineto{\pgfqpoint{3.730582in}{2.392558in}}%
\pgfpathlineto{\pgfqpoint{3.754013in}{2.408027in}}%
\pgfpathlineto{\pgfqpoint{3.777445in}{2.353397in}}%
\pgfpathlineto{\pgfqpoint{3.800877in}{2.484341in}}%
\pgfpathlineto{\pgfqpoint{3.824309in}{2.436165in}}%
\pgfpathlineto{\pgfqpoint{3.847740in}{2.400817in}}%
\pgfpathlineto{\pgfqpoint{3.871172in}{2.361095in}}%
\pgfpathlineto{\pgfqpoint{3.894604in}{2.435398in}}%
\pgfpathlineto{\pgfqpoint{3.918035in}{2.552683in}}%
\pgfpathlineto{\pgfqpoint{3.941467in}{2.575779in}}%
\pgfpathlineto{\pgfqpoint{3.964899in}{2.532853in}}%
\pgfpathlineto{\pgfqpoint{3.988331in}{2.528820in}}%
\pgfpathlineto{\pgfqpoint{4.011762in}{2.640929in}}%
\pgfpathlineto{\pgfqpoint{4.035194in}{2.575779in}}%
\pgfpathlineto{\pgfqpoint{4.082057in}{2.544816in}}%
\pgfpathlineto{\pgfqpoint{4.082057in}{2.544816in}}%
\pgfusepath{stroke}%
\end{pgfscope}%
\begin{pgfscope}%
\pgfpathrectangle{\pgfqpoint{0.588387in}{0.521603in}}{\pgfqpoint{3.660036in}{2.220246in}}%
\pgfusepath{clip}%
\pgfsetrectcap%
\pgfsetroundjoin%
\pgfsetlinewidth{1.505625pt}%
\definecolor{currentstroke}{rgb}{0.498039,0.498039,0.498039}%
\pgfsetstrokecolor{currentstroke}%
\pgfsetdash{}{0pt}%
\pgfpathmoveto{\pgfqpoint{0.754752in}{0.827711in}}%
\pgfpathlineto{\pgfqpoint{0.778184in}{1.131304in}}%
\pgfpathlineto{\pgfqpoint{0.801616in}{0.707708in}}%
\pgfpathlineto{\pgfqpoint{0.825048in}{0.681736in}}%
\pgfpathlineto{\pgfqpoint{0.848479in}{0.859341in}}%
\pgfpathlineto{\pgfqpoint{0.871911in}{1.015508in}}%
\pgfpathlineto{\pgfqpoint{0.895343in}{1.129132in}}%
\pgfpathlineto{\pgfqpoint{0.918775in}{0.824744in}}%
\pgfpathlineto{\pgfqpoint{0.942206in}{0.864311in}}%
\pgfpathlineto{\pgfqpoint{0.965638in}{1.046801in}}%
\pgfpathlineto{\pgfqpoint{0.989070in}{1.188128in}}%
\pgfpathlineto{\pgfqpoint{1.012501in}{1.205349in}}%
\pgfpathlineto{\pgfqpoint{1.035933in}{1.009379in}}%
\pgfpathlineto{\pgfqpoint{1.059365in}{1.095469in}}%
\pgfpathlineto{\pgfqpoint{1.082797in}{1.210423in}}%
\pgfpathlineto{\pgfqpoint{1.106228in}{1.226421in}}%
\pgfpathlineto{\pgfqpoint{1.129660in}{1.311560in}}%
\pgfpathlineto{\pgfqpoint{1.153092in}{1.187748in}}%
\pgfpathlineto{\pgfqpoint{1.176524in}{1.249857in}}%
\pgfpathlineto{\pgfqpoint{1.199955in}{1.276857in}}%
\pgfpathlineto{\pgfqpoint{1.223387in}{1.327389in}}%
\pgfpathlineto{\pgfqpoint{1.246819in}{1.430061in}}%
\pgfpathlineto{\pgfqpoint{1.270250in}{1.341316in}}%
\pgfpathlineto{\pgfqpoint{1.293682in}{1.303937in}}%
\pgfpathlineto{\pgfqpoint{1.317114in}{1.408979in}}%
\pgfpathlineto{\pgfqpoint{1.340546in}{1.441711in}}%
\pgfpathlineto{\pgfqpoint{1.363977in}{1.512779in}}%
\pgfpathlineto{\pgfqpoint{1.387409in}{1.379730in}}%
\pgfpathlineto{\pgfqpoint{1.410841in}{1.406825in}}%
\pgfpathlineto{\pgfqpoint{1.434273in}{1.439721in}}%
\pgfpathlineto{\pgfqpoint{1.457704in}{1.554623in}}%
\pgfpathlineto{\pgfqpoint{1.481136in}{1.539867in}}%
\pgfpathlineto{\pgfqpoint{1.504568in}{1.453726in}}%
\pgfpathlineto{\pgfqpoint{1.527999in}{1.529093in}}%
\pgfpathlineto{\pgfqpoint{1.551431in}{1.556699in}}%
\pgfpathlineto{\pgfqpoint{1.574863in}{1.597297in}}%
\pgfpathlineto{\pgfqpoint{1.598295in}{1.599168in}}%
\pgfpathlineto{\pgfqpoint{1.621726in}{1.554321in}}%
\pgfpathlineto{\pgfqpoint{1.645158in}{1.586769in}}%
\pgfpathlineto{\pgfqpoint{1.668590in}{1.613517in}}%
\pgfpathlineto{\pgfqpoint{1.692021in}{1.616286in}}%
\pgfpathlineto{\pgfqpoint{1.715453in}{1.716777in}}%
\pgfpathlineto{\pgfqpoint{1.738885in}{1.616568in}}%
\pgfpathlineto{\pgfqpoint{1.762317in}{1.599204in}}%
\pgfpathlineto{\pgfqpoint{1.785748in}{1.610545in}}%
\pgfpathlineto{\pgfqpoint{1.809180in}{1.707783in}}%
\pgfpathlineto{\pgfqpoint{1.832612in}{1.751317in}}%
\pgfpathlineto{\pgfqpoint{1.856044in}{1.685511in}}%
\pgfpathlineto{\pgfqpoint{1.879475in}{1.694107in}}%
\pgfpathlineto{\pgfqpoint{1.902907in}{1.693476in}}%
\pgfpathlineto{\pgfqpoint{1.926339in}{1.759969in}}%
\pgfpathlineto{\pgfqpoint{1.949770in}{1.745357in}}%
\pgfpathlineto{\pgfqpoint{1.973202in}{1.672120in}}%
\pgfpathlineto{\pgfqpoint{1.996634in}{1.756317in}}%
\pgfpathlineto{\pgfqpoint{2.020066in}{1.847201in}}%
\pgfpathlineto{\pgfqpoint{2.043497in}{1.803720in}}%
\pgfpathlineto{\pgfqpoint{2.066929in}{1.809602in}}%
\pgfpathlineto{\pgfqpoint{2.090361in}{1.802868in}}%
\pgfpathlineto{\pgfqpoint{2.113793in}{1.762904in}}%
\pgfpathlineto{\pgfqpoint{2.137224in}{1.807788in}}%
\pgfpathlineto{\pgfqpoint{2.160656in}{1.822842in}}%
\pgfpathlineto{\pgfqpoint{2.184088in}{1.908638in}}%
\pgfpathlineto{\pgfqpoint{2.207519in}{1.860326in}}%
\pgfpathlineto{\pgfqpoint{2.230951in}{1.823050in}}%
\pgfpathlineto{\pgfqpoint{2.254383in}{1.836197in}}%
\pgfpathlineto{\pgfqpoint{2.277815in}{1.929658in}}%
\pgfpathlineto{\pgfqpoint{2.301246in}{1.918146in}}%
\pgfpathlineto{\pgfqpoint{2.324678in}{1.862184in}}%
\pgfpathlineto{\pgfqpoint{2.348110in}{1.835314in}}%
\pgfpathlineto{\pgfqpoint{2.371542in}{1.944554in}}%
\pgfpathlineto{\pgfqpoint{2.418405in}{1.891635in}}%
\pgfpathlineto{\pgfqpoint{2.441837in}{1.850865in}}%
\pgfpathlineto{\pgfqpoint{2.465268in}{1.931141in}}%
\pgfpathlineto{\pgfqpoint{2.488700in}{1.894623in}}%
\pgfpathlineto{\pgfqpoint{2.512132in}{1.951054in}}%
\pgfpathlineto{\pgfqpoint{2.535564in}{1.999035in}}%
\pgfpathlineto{\pgfqpoint{2.558995in}{1.958552in}}%
\pgfpathlineto{\pgfqpoint{2.582427in}{1.988665in}}%
\pgfpathlineto{\pgfqpoint{2.605859in}{1.926701in}}%
\pgfpathlineto{\pgfqpoint{2.629291in}{1.968749in}}%
\pgfpathlineto{\pgfqpoint{2.652722in}{2.032284in}}%
\pgfpathlineto{\pgfqpoint{2.676154in}{2.078760in}}%
\pgfpathlineto{\pgfqpoint{2.699586in}{1.974132in}}%
\pgfpathlineto{\pgfqpoint{2.723017in}{1.946918in}}%
\pgfpathlineto{\pgfqpoint{2.746449in}{2.032284in}}%
\pgfpathlineto{\pgfqpoint{2.769881in}{2.036720in}}%
\pgfpathlineto{\pgfqpoint{2.793313in}{2.022684in}}%
\pgfpathlineto{\pgfqpoint{2.816744in}{2.120157in}}%
\pgfpathlineto{\pgfqpoint{2.840176in}{1.968309in}}%
\pgfpathlineto{\pgfqpoint{2.863608in}{1.997461in}}%
\pgfpathlineto{\pgfqpoint{2.887039in}{2.100375in}}%
\pgfpathlineto{\pgfqpoint{2.910471in}{2.034869in}}%
\pgfpathlineto{\pgfqpoint{2.933903in}{2.052205in}}%
\pgfpathlineto{\pgfqpoint{2.957335in}{2.104369in}}%
\pgfpathlineto{\pgfqpoint{2.980766in}{1.944120in}}%
\pgfpathlineto{\pgfqpoint{3.004198in}{2.018497in}}%
\pgfpathlineto{\pgfqpoint{3.027630in}{2.062842in}}%
\pgfpathlineto{\pgfqpoint{3.074493in}{2.102909in}}%
\pgfpathlineto{\pgfqpoint{3.097925in}{2.124504in}}%
\pgfpathlineto{\pgfqpoint{3.121357in}{2.007318in}}%
\pgfpathlineto{\pgfqpoint{3.144788in}{2.029909in}}%
\pgfpathlineto{\pgfqpoint{3.168220in}{1.996810in}}%
\pgfpathlineto{\pgfqpoint{3.191652in}{2.029909in}}%
\pgfpathlineto{\pgfqpoint{3.215084in}{2.430559in}}%
\pgfpathlineto{\pgfqpoint{3.238515in}{2.053771in}}%
\pgfpathlineto{\pgfqpoint{3.261947in}{2.021776in}}%
\pgfpathlineto{\pgfqpoint{3.285379in}{2.029909in}}%
\pgfpathlineto{\pgfqpoint{3.308811in}{2.162487in}}%
\pgfpathlineto{\pgfqpoint{3.332242in}{2.080646in}}%
\pgfpathlineto{\pgfqpoint{3.355674in}{2.108973in}}%
\pgfpathlineto{\pgfqpoint{3.379106in}{2.053771in}}%
\pgfpathlineto{\pgfqpoint{3.425969in}{2.099246in}}%
\pgfpathlineto{\pgfqpoint{3.449401in}{2.069251in}}%
\pgfpathlineto{\pgfqpoint{3.472833in}{2.151472in}}%
\pgfpathlineto{\pgfqpoint{3.496264in}{2.102909in}}%
\pgfpathlineto{\pgfqpoint{3.519696in}{2.120950in}}%
\pgfpathlineto{\pgfqpoint{3.543128in}{2.099246in}}%
\pgfpathlineto{\pgfqpoint{3.566560in}{2.331571in}}%
\pgfpathlineto{\pgfqpoint{3.589991in}{2.165842in}}%
\pgfpathlineto{\pgfqpoint{3.613423in}{2.142018in}}%
\pgfpathlineto{\pgfqpoint{3.636855in}{2.113788in}}%
\pgfpathlineto{\pgfqpoint{3.660286in}{2.120950in}}%
\pgfpathlineto{\pgfqpoint{3.707150in}{2.162487in}}%
\pgfpathlineto{\pgfqpoint{3.730582in}{2.135063in}}%
\pgfpathlineto{\pgfqpoint{3.777445in}{2.162487in}}%
\pgfpathlineto{\pgfqpoint{3.824309in}{2.155728in}}%
\pgfpathlineto{\pgfqpoint{3.894604in}{2.201757in}}%
\pgfpathlineto{\pgfqpoint{3.918035in}{2.175816in}}%
\pgfpathlineto{\pgfqpoint{3.918035in}{2.175816in}}%
\pgfusepath{stroke}%
\end{pgfscope}%
\begin{pgfscope}%
\pgfsetrectcap%
\pgfsetmiterjoin%
\pgfsetlinewidth{0.803000pt}%
\definecolor{currentstroke}{rgb}{0.000000,0.000000,0.000000}%
\pgfsetstrokecolor{currentstroke}%
\pgfsetdash{}{0pt}%
\pgfpathmoveto{\pgfqpoint{0.588387in}{0.521603in}}%
\pgfpathlineto{\pgfqpoint{0.588387in}{2.741849in}}%
\pgfusepath{stroke}%
\end{pgfscope}%
\begin{pgfscope}%
\pgfsetrectcap%
\pgfsetmiterjoin%
\pgfsetlinewidth{0.803000pt}%
\definecolor{currentstroke}{rgb}{0.000000,0.000000,0.000000}%
\pgfsetstrokecolor{currentstroke}%
\pgfsetdash{}{0pt}%
\pgfpathmoveto{\pgfqpoint{4.248423in}{0.521603in}}%
\pgfpathlineto{\pgfqpoint{4.248423in}{2.741849in}}%
\pgfusepath{stroke}%
\end{pgfscope}%
\begin{pgfscope}%
\pgfsetrectcap%
\pgfsetmiterjoin%
\pgfsetlinewidth{0.803000pt}%
\definecolor{currentstroke}{rgb}{0.000000,0.000000,0.000000}%
\pgfsetstrokecolor{currentstroke}%
\pgfsetdash{}{0pt}%
\pgfpathmoveto{\pgfqpoint{0.588387in}{0.521603in}}%
\pgfpathlineto{\pgfqpoint{4.248423in}{0.521603in}}%
\pgfusepath{stroke}%
\end{pgfscope}%
\begin{pgfscope}%
\pgfsetrectcap%
\pgfsetmiterjoin%
\pgfsetlinewidth{0.803000pt}%
\definecolor{currentstroke}{rgb}{0.000000,0.000000,0.000000}%
\pgfsetstrokecolor{currentstroke}%
\pgfsetdash{}{0pt}%
\pgfpathmoveto{\pgfqpoint{0.588387in}{2.741849in}}%
\pgfpathlineto{\pgfqpoint{4.248423in}{2.741849in}}%
\pgfusepath{stroke}%
\end{pgfscope}%
\begin{pgfscope}%
\pgfsetbuttcap%
\pgfsetmiterjoin%
\definecolor{currentfill}{rgb}{1.000000,1.000000,1.000000}%
\pgfsetfillcolor{currentfill}%
\pgfsetfillopacity{0.800000}%
\pgfsetlinewidth{1.003750pt}%
\definecolor{currentstroke}{rgb}{0.800000,0.800000,0.800000}%
\pgfsetstrokecolor{currentstroke}%
\pgfsetstrokeopacity{0.800000}%
\pgfsetdash{}{0pt}%
\pgfpathmoveto{\pgfqpoint{4.365089in}{0.623654in}}%
\pgfpathlineto{\pgfqpoint{8.251043in}{0.623654in}}%
\pgfpathquadraticcurveto{\pgfqpoint{8.284376in}{0.623654in}}{\pgfqpoint{8.284376in}{0.656988in}}%
\pgfpathlineto{\pgfqpoint{8.284376in}{2.625183in}}%
\pgfpathquadraticcurveto{\pgfqpoint{8.284376in}{2.658516in}}{\pgfqpoint{8.251043in}{2.658516in}}%
\pgfpathlineto{\pgfqpoint{4.365089in}{2.658516in}}%
\pgfpathquadraticcurveto{\pgfqpoint{4.331756in}{2.658516in}}{\pgfqpoint{4.331756in}{2.625183in}}%
\pgfpathlineto{\pgfqpoint{4.331756in}{0.656988in}}%
\pgfpathquadraticcurveto{\pgfqpoint{4.331756in}{0.623654in}}{\pgfqpoint{4.365089in}{0.623654in}}%
\pgfpathlineto{\pgfqpoint{4.365089in}{0.623654in}}%
\pgfpathclose%
\pgfusepath{stroke,fill}%
\end{pgfscope}%
\begin{pgfscope}%
\pgfsetrectcap%
\pgfsetroundjoin%
\pgfsetlinewidth{1.505625pt}%
\pgfsetstrokecolor{currentstroke1}%
\pgfsetdash{}{0pt}%
\pgfpathmoveto{\pgfqpoint{4.398423in}{2.523555in}}%
\pgfpathlineto{\pgfqpoint{4.565089in}{2.523555in}}%
\pgfpathlineto{\pgfqpoint{4.731756in}{2.523555in}}%
\pgfusepath{stroke}%
\end{pgfscope}%
\begin{pgfscope}%
\definecolor{textcolor}{rgb}{0.000000,0.000000,0.000000}%
\pgfsetstrokecolor{textcolor}%
\pgfsetfillcolor{textcolor}%
\pgftext[x=4.865089in,y=2.465222in,left,base]{\color{textcolor}{\rmfamily\fontsize{12.000000}{14.400000}\selectfont\catcode`\^=\active\def^{\ifmmode\sp\else\^{}\fi}\catcode`\%=\active\def%{\%}\CyclesMatchChunks{} \& \MergeLinear{}}}%
\end{pgfscope}%
\begin{pgfscope}%
\pgfsetrectcap%
\pgfsetroundjoin%
\pgfsetlinewidth{1.505625pt}%
\pgfsetstrokecolor{currentstroke2}%
\pgfsetdash{}{0pt}%
\pgfpathmoveto{\pgfqpoint{4.398423in}{2.274288in}}%
\pgfpathlineto{\pgfqpoint{4.565089in}{2.274288in}}%
\pgfpathlineto{\pgfqpoint{4.731756in}{2.274288in}}%
\pgfusepath{stroke}%
\end{pgfscope}%
\begin{pgfscope}%
\definecolor{textcolor}{rgb}{0.000000,0.000000,0.000000}%
\pgfsetstrokecolor{textcolor}%
\pgfsetfillcolor{textcolor}%
\pgftext[x=4.865089in,y=2.215954in,left,base]{\color{textcolor}{\rmfamily\fontsize{12.000000}{14.400000}\selectfont\catcode`\^=\active\def^{\ifmmode\sp\else\^{}\fi}\catcode`\%=\active\def%{\%}\CyclesMatchChunks{} \& \SharedVertices{}}}%
\end{pgfscope}%
\begin{pgfscope}%
\pgfsetrectcap%
\pgfsetroundjoin%
\pgfsetlinewidth{1.505625pt}%
\pgfsetstrokecolor{currentstroke3}%
\pgfsetdash{}{0pt}%
\pgfpathmoveto{\pgfqpoint{4.398423in}{2.025020in}}%
\pgfpathlineto{\pgfqpoint{4.565089in}{2.025020in}}%
\pgfpathlineto{\pgfqpoint{4.731756in}{2.025020in}}%
\pgfusepath{stroke}%
\end{pgfscope}%
\begin{pgfscope}%
\definecolor{textcolor}{rgb}{0.000000,0.000000,0.000000}%
\pgfsetstrokecolor{textcolor}%
\pgfsetfillcolor{textcolor}%
\pgftext[x=4.865089in,y=1.966687in,left,base]{\color{textcolor}{\rmfamily\fontsize{12.000000}{14.400000}\selectfont\catcode`\^=\active\def^{\ifmmode\sp\else\^{}\fi}\catcode`\%=\active\def%{\%}\Neighbors{} \& \MergeLinear{}}}%
\end{pgfscope}%
\begin{pgfscope}%
\pgfsetrectcap%
\pgfsetroundjoin%
\pgfsetlinewidth{1.505625pt}%
\pgfsetstrokecolor{currentstroke4}%
\pgfsetdash{}{0pt}%
\pgfpathmoveto{\pgfqpoint{4.398423in}{1.780391in}}%
\pgfpathlineto{\pgfqpoint{4.565089in}{1.780391in}}%
\pgfpathlineto{\pgfqpoint{4.731756in}{1.780391in}}%
\pgfusepath{stroke}%
\end{pgfscope}%
\begin{pgfscope}%
\definecolor{textcolor}{rgb}{0.000000,0.000000,0.000000}%
\pgfsetstrokecolor{textcolor}%
\pgfsetfillcolor{textcolor}%
\pgftext[x=4.865089in,y=1.722058in,left,base]{\color{textcolor}{\rmfamily\fontsize{12.000000}{14.400000}\selectfont\catcode`\^=\active\def^{\ifmmode\sp\else\^{}\fi}\catcode`\%=\active\def%{\%}\Neighbors{} \& \SharedVertices{}}}%
\end{pgfscope}%
\begin{pgfscope}%
\pgfsetrectcap%
\pgfsetroundjoin%
\pgfsetlinewidth{1.505625pt}%
\pgfsetstrokecolor{currentstroke5}%
\pgfsetdash{}{0pt}%
\pgfpathmoveto{\pgfqpoint{4.398423in}{1.531124in}}%
\pgfpathlineto{\pgfqpoint{4.565089in}{1.531124in}}%
\pgfpathlineto{\pgfqpoint{4.731756in}{1.531124in}}%
\pgfusepath{stroke}%
\end{pgfscope}%
\begin{pgfscope}%
\definecolor{textcolor}{rgb}{0.000000,0.000000,0.000000}%
\pgfsetstrokecolor{textcolor}%
\pgfsetfillcolor{textcolor}%
\pgftext[x=4.865089in,y=1.472791in,left,base]{\color{textcolor}{\rmfamily\fontsize{12.000000}{14.400000}\selectfont\catcode`\^=\active\def^{\ifmmode\sp\else\^{}\fi}\catcode`\%=\active\def%{\%}\NeighborsDegree{} \& \MergeLinear{}}}%
\end{pgfscope}%
\begin{pgfscope}%
\pgfsetrectcap%
\pgfsetroundjoin%
\pgfsetlinewidth{1.505625pt}%
\pgfsetstrokecolor{currentstroke6}%
\pgfsetdash{}{0pt}%
\pgfpathmoveto{\pgfqpoint{4.398423in}{1.281857in}}%
\pgfpathlineto{\pgfqpoint{4.565089in}{1.281857in}}%
\pgfpathlineto{\pgfqpoint{4.731756in}{1.281857in}}%
\pgfusepath{stroke}%
\end{pgfscope}%
\begin{pgfscope}%
\definecolor{textcolor}{rgb}{0.000000,0.000000,0.000000}%
\pgfsetstrokecolor{textcolor}%
\pgfsetfillcolor{textcolor}%
\pgftext[x=4.865089in,y=1.223523in,left,base]{\color{textcolor}{\rmfamily\fontsize{12.000000}{14.400000}\selectfont\catcode`\^=\active\def^{\ifmmode\sp\else\^{}\fi}\catcode`\%=\active\def%{\%}\NeighborsDegree{} \& \SharedVertices{}}}%
\end{pgfscope}%
\begin{pgfscope}%
\pgfsetrectcap%
\pgfsetroundjoin%
\pgfsetlinewidth{1.505625pt}%
\pgfsetstrokecolor{currentstroke7}%
\pgfsetdash{}{0pt}%
\pgfpathmoveto{\pgfqpoint{4.398423in}{1.032589in}}%
\pgfpathlineto{\pgfqpoint{4.565089in}{1.032589in}}%
\pgfpathlineto{\pgfqpoint{4.731756in}{1.032589in}}%
\pgfusepath{stroke}%
\end{pgfscope}%
\begin{pgfscope}%
\definecolor{textcolor}{rgb}{0.000000,0.000000,0.000000}%
\pgfsetstrokecolor{textcolor}%
\pgfsetfillcolor{textcolor}%
\pgftext[x=4.865089in,y=0.974256in,left,base]{\color{textcolor}{\rmfamily\fontsize{12.000000}{14.400000}\selectfont\catcode`\^=\active\def^{\ifmmode\sp\else\^{}\fi}\catcode`\%=\active\def%{\%}\None{} \& \MergeLinear{}}}%
\end{pgfscope}%
\begin{pgfscope}%
\pgfsetrectcap%
\pgfsetroundjoin%
\pgfsetlinewidth{1.505625pt}%
\definecolor{currentstroke}{rgb}{0.498039,0.498039,0.498039}%
\pgfsetstrokecolor{currentstroke}%
\pgfsetdash{}{0pt}%
\pgfpathmoveto{\pgfqpoint{4.398423in}{0.787961in}}%
\pgfpathlineto{\pgfqpoint{4.565089in}{0.787961in}}%
\pgfpathlineto{\pgfqpoint{4.731756in}{0.787961in}}%
\pgfusepath{stroke}%
\end{pgfscope}%
\begin{pgfscope}%
\definecolor{textcolor}{rgb}{0.000000,0.000000,0.000000}%
\pgfsetstrokecolor{textcolor}%
\pgfsetfillcolor{textcolor}%
\pgftext[x=4.865089in,y=0.729627in,left,base]{\color{textcolor}{\rmfamily\fontsize{12.000000}{14.400000}\selectfont\catcode`\^=\active\def^{\ifmmode\sp\else\^{}\fi}\catcode`\%=\active\def%{\%}\None{} \& \SharedVertices{}}}%
\end{pgfscope}%
\end{pgfpicture}%
\makeatother%
\endgroup%
}
	\caption[Checks performed for graphs with no NAC-coloring]{
		The number of checks performed to find all NAC-colorings for graphs with no NAC-coloring.}%
	\label{fig:graph_no_nac_coloring_first_checks}
\end{figure}%

In \Subgraphs{} algorithm description, an important parameter was the size of subgraphs \( k \).
Majority of the benchmarks in the previous section were run with \( k = 4 \).
%
We show the impact of	\( k \) on runtime and number of checks.
Note that you see averages over all the strategies used for benchmarking
graphs with no NAC-colorings.
%
From graphs in \Cref{fig:graph_no_nac_coloring_first_runtime_subgraph_size,fig:graph_no_nac_coloring_first_checks_subgraph_size},
it can be seen that the number of checks is reduced for smaller \( k \).
On the other hand, the runtime improves slightly for larger \( k \)
and becomes negligible for larger graphs.

\begin{figure}[thbp]
	\centering
	\scalebox{\BenchFigureScale}{%% Creator: Matplotlib, PGF backend
%%
%% To include the figure in your LaTeX document, write
%%   \input{<filename>.pgf}
%%
%% Make sure the required packages are loaded in your preamble
%%   \usepackage{pgf}
%%
%% Also ensure that all the required font packages are loaded; for instance,
%% the lmodern package is sometimes necessary when using math font.
%%   \usepackage{lmodern}
%%
%% Figures using additional raster images can only be included by \input if
%% they are in the same directory as the main LaTeX file. For loading figures
%% from other directories you can use the `import` package
%%   \usepackage{import}
%%
%% and then include the figures with
%%   \import{<path to file>}{<filename>.pgf}
%%
%% Matplotlib used the following preamble
%%   \def\mathdefault#1{#1}
%%   \everymath=\expandafter{\the\everymath\displaystyle}
%%   \IfFileExists{scrextend.sty}{
%%     \usepackage[fontsize=10.000000pt]{scrextend}
%%   }{
%%     \renewcommand{\normalsize}{\fontsize{10.000000}{12.000000}\selectfont}
%%     \normalsize
%%   }
%%   
%%   \ifdefined\pdftexversion\else  % non-pdftex case.
%%     \usepackage{fontspec}
%%     \setmainfont{DejaVuSans.ttf}[Path=\detokenize{/home/petr/Projects/PyRigi/.venv/lib/python3.12/site-packages/matplotlib/mpl-data/fonts/ttf/}]
%%     \setsansfont{DejaVuSans.ttf}[Path=\detokenize{/home/petr/Projects/PyRigi/.venv/lib/python3.12/site-packages/matplotlib/mpl-data/fonts/ttf/}]
%%     \setmonofont{DejaVuSansMono.ttf}[Path=\detokenize{/home/petr/Projects/PyRigi/.venv/lib/python3.12/site-packages/matplotlib/mpl-data/fonts/ttf/}]
%%   \fi
%%   \makeatletter\@ifpackageloaded{under\Score{}}{}{\usepackage[strings]{under\Score{}}}\makeatother
%%
\begingroup%
\makeatletter%
\begin{pgfpicture}%
\pgfpathrectangle{\pgfpointorigin}{\pgfqpoint{8.384376in}{2.841849in}}%
\pgfusepath{use as bounding box, clip}%
\begin{pgfscope}%
\pgfsetbuttcap%
\pgfsetmiterjoin%
\definecolor{currentfill}{rgb}{1.000000,1.000000,1.000000}%
\pgfsetfillcolor{currentfill}%
\pgfsetlinewidth{0.000000pt}%
\definecolor{currentstroke}{rgb}{1.000000,1.000000,1.000000}%
\pgfsetstrokecolor{currentstroke}%
\pgfsetdash{}{0pt}%
\pgfpathmoveto{\pgfqpoint{0.000000in}{0.000000in}}%
\pgfpathlineto{\pgfqpoint{8.384376in}{0.000000in}}%
\pgfpathlineto{\pgfqpoint{8.384376in}{2.841849in}}%
\pgfpathlineto{\pgfqpoint{0.000000in}{2.841849in}}%
\pgfpathlineto{\pgfqpoint{0.000000in}{0.000000in}}%
\pgfpathclose%
\pgfusepath{fill}%
\end{pgfscope}%
\begin{pgfscope}%
\pgfsetbuttcap%
\pgfsetmiterjoin%
\definecolor{currentfill}{rgb}{1.000000,1.000000,1.000000}%
\pgfsetfillcolor{currentfill}%
\pgfsetlinewidth{0.000000pt}%
\definecolor{currentstroke}{rgb}{0.000000,0.000000,0.000000}%
\pgfsetstrokecolor{currentstroke}%
\pgfsetstrokeopacity{0.000000}%
\pgfsetdash{}{0pt}%
\pgfpathmoveto{\pgfqpoint{0.588387in}{0.521603in}}%
\pgfpathlineto{\pgfqpoint{7.692348in}{0.521603in}}%
\pgfpathlineto{\pgfqpoint{7.692348in}{2.741849in}}%
\pgfpathlineto{\pgfqpoint{0.588387in}{2.741849in}}%
\pgfpathlineto{\pgfqpoint{0.588387in}{0.521603in}}%
\pgfpathclose%
\pgfusepath{fill}%
\end{pgfscope}%
\begin{pgfscope}%
\pgfsetbuttcap%
\pgfsetroundjoin%
\definecolor{currentfill}{rgb}{0.000000,0.000000,0.000000}%
\pgfsetfillcolor{currentfill}%
\pgfsetlinewidth{0.803000pt}%
\definecolor{currentstroke}{rgb}{0.000000,0.000000,0.000000}%
\pgfsetstrokecolor{currentstroke}%
\pgfsetdash{}{0pt}%
\pgfsys@defobject{currentmarker}{\pgfqpoint{0.000000in}{-0.048611in}}{\pgfqpoint{0.000000in}{0.000000in}}{%
\pgfpathmoveto{\pgfqpoint{0.000000in}{0.000000in}}%
\pgfpathlineto{\pgfqpoint{0.000000in}{-0.048611in}}%
\pgfusepath{stroke,fill}%
}%
\begin{pgfscope}%
\pgfsys@transformshift{1.200465in}{0.521603in}%
\pgfsys@useobject{currentmarker}{}%
\end{pgfscope}%
\end{pgfscope}%
\begin{pgfscope}%
\definecolor{textcolor}{rgb}{0.000000,0.000000,0.000000}%
\pgfsetstrokecolor{textcolor}%
\pgfsetfillcolor{textcolor}%
\pgftext[x=1.200465in,y=0.424381in,,top]{\color{textcolor}{\rmfamily\fontsize{10.000000}{12.000000}\selectfont\catcode`\^=\active\def^{\ifmmode\sp\else\^{}\fi}\catcode`\%=\active\def%{\%}$\mathdefault{16}$}}%
\end{pgfscope}%
\begin{pgfscope}%
\pgfsetbuttcap%
\pgfsetroundjoin%
\definecolor{currentfill}{rgb}{0.000000,0.000000,0.000000}%
\pgfsetfillcolor{currentfill}%
\pgfsetlinewidth{0.803000pt}%
\definecolor{currentstroke}{rgb}{0.000000,0.000000,0.000000}%
\pgfsetstrokecolor{currentstroke}%
\pgfsetdash{}{0pt}%
\pgfsys@defobject{currentmarker}{\pgfqpoint{0.000000in}{-0.048611in}}{\pgfqpoint{0.000000in}{0.000000in}}{%
\pgfpathmoveto{\pgfqpoint{0.000000in}{0.000000in}}%
\pgfpathlineto{\pgfqpoint{0.000000in}{-0.048611in}}%
\pgfusepath{stroke,fill}%
}%
\begin{pgfscope}%
\pgfsys@transformshift{1.971587in}{0.521603in}%
\pgfsys@useobject{currentmarker}{}%
\end{pgfscope}%
\end{pgfscope}%
\begin{pgfscope}%
\definecolor{textcolor}{rgb}{0.000000,0.000000,0.000000}%
\pgfsetstrokecolor{textcolor}%
\pgfsetfillcolor{textcolor}%
\pgftext[x=1.971587in,y=0.424381in,,top]{\color{textcolor}{\rmfamily\fontsize{10.000000}{12.000000}\selectfont\catcode`\^=\active\def^{\ifmmode\sp\else\^{}\fi}\catcode`\%=\active\def%{\%}$\mathdefault{24}$}}%
\end{pgfscope}%
\begin{pgfscope}%
\pgfsetbuttcap%
\pgfsetroundjoin%
\definecolor{currentfill}{rgb}{0.000000,0.000000,0.000000}%
\pgfsetfillcolor{currentfill}%
\pgfsetlinewidth{0.803000pt}%
\definecolor{currentstroke}{rgb}{0.000000,0.000000,0.000000}%
\pgfsetstrokecolor{currentstroke}%
\pgfsetdash{}{0pt}%
\pgfsys@defobject{currentmarker}{\pgfqpoint{0.000000in}{-0.048611in}}{\pgfqpoint{0.000000in}{0.000000in}}{%
\pgfpathmoveto{\pgfqpoint{0.000000in}{0.000000in}}%
\pgfpathlineto{\pgfqpoint{0.000000in}{-0.048611in}}%
\pgfusepath{stroke,fill}%
}%
\begin{pgfscope}%
\pgfsys@transformshift{2.742709in}{0.521603in}%
\pgfsys@useobject{currentmarker}{}%
\end{pgfscope}%
\end{pgfscope}%
\begin{pgfscope}%
\definecolor{textcolor}{rgb}{0.000000,0.000000,0.000000}%
\pgfsetstrokecolor{textcolor}%
\pgfsetfillcolor{textcolor}%
\pgftext[x=2.742709in,y=0.424381in,,top]{\color{textcolor}{\rmfamily\fontsize{10.000000}{12.000000}\selectfont\catcode`\^=\active\def^{\ifmmode\sp\else\^{}\fi}\catcode`\%=\active\def%{\%}$\mathdefault{32}$}}%
\end{pgfscope}%
\begin{pgfscope}%
\pgfsetbuttcap%
\pgfsetroundjoin%
\definecolor{currentfill}{rgb}{0.000000,0.000000,0.000000}%
\pgfsetfillcolor{currentfill}%
\pgfsetlinewidth{0.803000pt}%
\definecolor{currentstroke}{rgb}{0.000000,0.000000,0.000000}%
\pgfsetstrokecolor{currentstroke}%
\pgfsetdash{}{0pt}%
\pgfsys@defobject{currentmarker}{\pgfqpoint{0.000000in}{-0.048611in}}{\pgfqpoint{0.000000in}{0.000000in}}{%
\pgfpathmoveto{\pgfqpoint{0.000000in}{0.000000in}}%
\pgfpathlineto{\pgfqpoint{0.000000in}{-0.048611in}}%
\pgfusepath{stroke,fill}%
}%
\begin{pgfscope}%
\pgfsys@transformshift{3.513831in}{0.521603in}%
\pgfsys@useobject{currentmarker}{}%
\end{pgfscope}%
\end{pgfscope}%
\begin{pgfscope}%
\definecolor{textcolor}{rgb}{0.000000,0.000000,0.000000}%
\pgfsetstrokecolor{textcolor}%
\pgfsetfillcolor{textcolor}%
\pgftext[x=3.513831in,y=0.424381in,,top]{\color{textcolor}{\rmfamily\fontsize{10.000000}{12.000000}\selectfont\catcode`\^=\active\def^{\ifmmode\sp\else\^{}\fi}\catcode`\%=\active\def%{\%}$\mathdefault{40}$}}%
\end{pgfscope}%
\begin{pgfscope}%
\pgfsetbuttcap%
\pgfsetroundjoin%
\definecolor{currentfill}{rgb}{0.000000,0.000000,0.000000}%
\pgfsetfillcolor{currentfill}%
\pgfsetlinewidth{0.803000pt}%
\definecolor{currentstroke}{rgb}{0.000000,0.000000,0.000000}%
\pgfsetstrokecolor{currentstroke}%
\pgfsetdash{}{0pt}%
\pgfsys@defobject{currentmarker}{\pgfqpoint{0.000000in}{-0.048611in}}{\pgfqpoint{0.000000in}{0.000000in}}{%
\pgfpathmoveto{\pgfqpoint{0.000000in}{0.000000in}}%
\pgfpathlineto{\pgfqpoint{0.000000in}{-0.048611in}}%
\pgfusepath{stroke,fill}%
}%
\begin{pgfscope}%
\pgfsys@transformshift{4.284953in}{0.521603in}%
\pgfsys@useobject{currentmarker}{}%
\end{pgfscope}%
\end{pgfscope}%
\begin{pgfscope}%
\definecolor{textcolor}{rgb}{0.000000,0.000000,0.000000}%
\pgfsetstrokecolor{textcolor}%
\pgfsetfillcolor{textcolor}%
\pgftext[x=4.284953in,y=0.424381in,,top]{\color{textcolor}{\rmfamily\fontsize{10.000000}{12.000000}\selectfont\catcode`\^=\active\def^{\ifmmode\sp\else\^{}\fi}\catcode`\%=\active\def%{\%}$\mathdefault{48}$}}%
\end{pgfscope}%
\begin{pgfscope}%
\pgfsetbuttcap%
\pgfsetroundjoin%
\definecolor{currentfill}{rgb}{0.000000,0.000000,0.000000}%
\pgfsetfillcolor{currentfill}%
\pgfsetlinewidth{0.803000pt}%
\definecolor{currentstroke}{rgb}{0.000000,0.000000,0.000000}%
\pgfsetstrokecolor{currentstroke}%
\pgfsetdash{}{0pt}%
\pgfsys@defobject{currentmarker}{\pgfqpoint{0.000000in}{-0.048611in}}{\pgfqpoint{0.000000in}{0.000000in}}{%
\pgfpathmoveto{\pgfqpoint{0.000000in}{0.000000in}}%
\pgfpathlineto{\pgfqpoint{0.000000in}{-0.048611in}}%
\pgfusepath{stroke,fill}%
}%
\begin{pgfscope}%
\pgfsys@transformshift{5.056075in}{0.521603in}%
\pgfsys@useobject{currentmarker}{}%
\end{pgfscope}%
\end{pgfscope}%
\begin{pgfscope}%
\definecolor{textcolor}{rgb}{0.000000,0.000000,0.000000}%
\pgfsetstrokecolor{textcolor}%
\pgfsetfillcolor{textcolor}%
\pgftext[x=5.056075in,y=0.424381in,,top]{\color{textcolor}{\rmfamily\fontsize{10.000000}{12.000000}\selectfont\catcode`\^=\active\def^{\ifmmode\sp\else\^{}\fi}\catcode`\%=\active\def%{\%}$\mathdefault{56}$}}%
\end{pgfscope}%
\begin{pgfscope}%
\pgfsetbuttcap%
\pgfsetroundjoin%
\definecolor{currentfill}{rgb}{0.000000,0.000000,0.000000}%
\pgfsetfillcolor{currentfill}%
\pgfsetlinewidth{0.803000pt}%
\definecolor{currentstroke}{rgb}{0.000000,0.000000,0.000000}%
\pgfsetstrokecolor{currentstroke}%
\pgfsetdash{}{0pt}%
\pgfsys@defobject{currentmarker}{\pgfqpoint{0.000000in}{-0.048611in}}{\pgfqpoint{0.000000in}{0.000000in}}{%
\pgfpathmoveto{\pgfqpoint{0.000000in}{0.000000in}}%
\pgfpathlineto{\pgfqpoint{0.000000in}{-0.048611in}}%
\pgfusepath{stroke,fill}%
}%
\begin{pgfscope}%
\pgfsys@transformshift{5.827197in}{0.521603in}%
\pgfsys@useobject{currentmarker}{}%
\end{pgfscope}%
\end{pgfscope}%
\begin{pgfscope}%
\definecolor{textcolor}{rgb}{0.000000,0.000000,0.000000}%
\pgfsetstrokecolor{textcolor}%
\pgfsetfillcolor{textcolor}%
\pgftext[x=5.827197in,y=0.424381in,,top]{\color{textcolor}{\rmfamily\fontsize{10.000000}{12.000000}\selectfont\catcode`\^=\active\def^{\ifmmode\sp\else\^{}\fi}\catcode`\%=\active\def%{\%}$\mathdefault{64}$}}%
\end{pgfscope}%
\begin{pgfscope}%
\pgfsetbuttcap%
\pgfsetroundjoin%
\definecolor{currentfill}{rgb}{0.000000,0.000000,0.000000}%
\pgfsetfillcolor{currentfill}%
\pgfsetlinewidth{0.803000pt}%
\definecolor{currentstroke}{rgb}{0.000000,0.000000,0.000000}%
\pgfsetstrokecolor{currentstroke}%
\pgfsetdash{}{0pt}%
\pgfsys@defobject{currentmarker}{\pgfqpoint{0.000000in}{-0.048611in}}{\pgfqpoint{0.000000in}{0.000000in}}{%
\pgfpathmoveto{\pgfqpoint{0.000000in}{0.000000in}}%
\pgfpathlineto{\pgfqpoint{0.000000in}{-0.048611in}}%
\pgfusepath{stroke,fill}%
}%
\begin{pgfscope}%
\pgfsys@transformshift{6.598319in}{0.521603in}%
\pgfsys@useobject{currentmarker}{}%
\end{pgfscope}%
\end{pgfscope}%
\begin{pgfscope}%
\definecolor{textcolor}{rgb}{0.000000,0.000000,0.000000}%
\pgfsetstrokecolor{textcolor}%
\pgfsetfillcolor{textcolor}%
\pgftext[x=6.598319in,y=0.424381in,,top]{\color{textcolor}{\rmfamily\fontsize{10.000000}{12.000000}\selectfont\catcode`\^=\active\def^{\ifmmode\sp\else\^{}\fi}\catcode`\%=\active\def%{\%}$\mathdefault{72}$}}%
\end{pgfscope}%
\begin{pgfscope}%
\pgfsetbuttcap%
\pgfsetroundjoin%
\definecolor{currentfill}{rgb}{0.000000,0.000000,0.000000}%
\pgfsetfillcolor{currentfill}%
\pgfsetlinewidth{0.803000pt}%
\definecolor{currentstroke}{rgb}{0.000000,0.000000,0.000000}%
\pgfsetstrokecolor{currentstroke}%
\pgfsetdash{}{0pt}%
\pgfsys@defobject{currentmarker}{\pgfqpoint{0.000000in}{-0.048611in}}{\pgfqpoint{0.000000in}{0.000000in}}{%
\pgfpathmoveto{\pgfqpoint{0.000000in}{0.000000in}}%
\pgfpathlineto{\pgfqpoint{0.000000in}{-0.048611in}}%
\pgfusepath{stroke,fill}%
}%
\begin{pgfscope}%
\pgfsys@transformshift{7.369440in}{0.521603in}%
\pgfsys@useobject{currentmarker}{}%
\end{pgfscope}%
\end{pgfscope}%
\begin{pgfscope}%
\definecolor{textcolor}{rgb}{0.000000,0.000000,0.000000}%
\pgfsetstrokecolor{textcolor}%
\pgfsetfillcolor{textcolor}%
\pgftext[x=7.369440in,y=0.424381in,,top]{\color{textcolor}{\rmfamily\fontsize{10.000000}{12.000000}\selectfont\catcode`\^=\active\def^{\ifmmode\sp\else\^{}\fi}\catcode`\%=\active\def%{\%}$\mathdefault{80}$}}%
\end{pgfscope}%
\begin{pgfscope}%
\definecolor{textcolor}{rgb}{0.000000,0.000000,0.000000}%
\pgfsetstrokecolor{textcolor}%
\pgfsetfillcolor{textcolor}%
\pgftext[x=4.140367in,y=0.234413in,,top]{\color{textcolor}{\rmfamily\fontsize{10.000000}{12.000000}\selectfont\catcode`\^=\active\def^{\ifmmode\sp\else\^{}\fi}\catcode`\%=\active\def%{\%}Triangle components}}%
\end{pgfscope}%
\begin{pgfscope}%
\pgfsetbuttcap%
\pgfsetroundjoin%
\definecolor{currentfill}{rgb}{0.000000,0.000000,0.000000}%
\pgfsetfillcolor{currentfill}%
\pgfsetlinewidth{0.803000pt}%
\definecolor{currentstroke}{rgb}{0.000000,0.000000,0.000000}%
\pgfsetstrokecolor{currentstroke}%
\pgfsetdash{}{0pt}%
\pgfsys@defobject{currentmarker}{\pgfqpoint{-0.048611in}{0.000000in}}{\pgfqpoint{-0.000000in}{0.000000in}}{%
\pgfpathmoveto{\pgfqpoint{-0.000000in}{0.000000in}}%
\pgfpathlineto{\pgfqpoint{-0.048611in}{0.000000in}}%
\pgfusepath{stroke,fill}%
}%
\begin{pgfscope}%
\pgfsys@transformshift{0.588387in}{0.566055in}%
\pgfsys@useobject{currentmarker}{}%
\end{pgfscope}%
\end{pgfscope}%
\begin{pgfscope}%
\definecolor{textcolor}{rgb}{0.000000,0.000000,0.000000}%
\pgfsetstrokecolor{textcolor}%
\pgfsetfillcolor{textcolor}%
\pgftext[x=0.289968in, y=0.513294in, left, base]{\color{textcolor}{\rmfamily\fontsize{10.000000}{12.000000}\selectfont\catcode`\^=\active\def^{\ifmmode\sp\else\^{}\fi}\catcode`\%=\active\def%{\%}$\mathdefault{10^{2}}$}}%
\end{pgfscope}%
\begin{pgfscope}%
\pgfsetbuttcap%
\pgfsetroundjoin%
\definecolor{currentfill}{rgb}{0.000000,0.000000,0.000000}%
\pgfsetfillcolor{currentfill}%
\pgfsetlinewidth{0.803000pt}%
\definecolor{currentstroke}{rgb}{0.000000,0.000000,0.000000}%
\pgfsetstrokecolor{currentstroke}%
\pgfsetdash{}{0pt}%
\pgfsys@defobject{currentmarker}{\pgfqpoint{-0.048611in}{0.000000in}}{\pgfqpoint{-0.000000in}{0.000000in}}{%
\pgfpathmoveto{\pgfqpoint{-0.000000in}{0.000000in}}%
\pgfpathlineto{\pgfqpoint{-0.048611in}{0.000000in}}%
\pgfusepath{stroke,fill}%
}%
\begin{pgfscope}%
\pgfsys@transformshift{0.588387in}{2.285770in}%
\pgfsys@useobject{currentmarker}{}%
\end{pgfscope}%
\end{pgfscope}%
\begin{pgfscope}%
\definecolor{textcolor}{rgb}{0.000000,0.000000,0.000000}%
\pgfsetstrokecolor{textcolor}%
\pgfsetfillcolor{textcolor}%
\pgftext[x=0.289968in, y=2.233008in, left, base]{\color{textcolor}{\rmfamily\fontsize{10.000000}{12.000000}\selectfont\catcode`\^=\active\def^{\ifmmode\sp\else\^{}\fi}\catcode`\%=\active\def%{\%}$\mathdefault{10^{3}}$}}%
\end{pgfscope}%
\begin{pgfscope}%
\pgfsetbuttcap%
\pgfsetroundjoin%
\definecolor{currentfill}{rgb}{0.000000,0.000000,0.000000}%
\pgfsetfillcolor{currentfill}%
\pgfsetlinewidth{0.602250pt}%
\definecolor{currentstroke}{rgb}{0.000000,0.000000,0.000000}%
\pgfsetstrokecolor{currentstroke}%
\pgfsetdash{}{0pt}%
\pgfsys@defobject{currentmarker}{\pgfqpoint{-0.027778in}{0.000000in}}{\pgfqpoint{-0.000000in}{0.000000in}}{%
\pgfpathmoveto{\pgfqpoint{-0.000000in}{0.000000in}}%
\pgfpathlineto{\pgfqpoint{-0.027778in}{0.000000in}}%
\pgfusepath{stroke,fill}%
}%
\begin{pgfscope}%
\pgfsys@transformshift{0.588387in}{1.083741in}%
\pgfsys@useobject{currentmarker}{}%
\end{pgfscope}%
\end{pgfscope}%
\begin{pgfscope}%
\pgfsetbuttcap%
\pgfsetroundjoin%
\definecolor{currentfill}{rgb}{0.000000,0.000000,0.000000}%
\pgfsetfillcolor{currentfill}%
\pgfsetlinewidth{0.602250pt}%
\definecolor{currentstroke}{rgb}{0.000000,0.000000,0.000000}%
\pgfsetstrokecolor{currentstroke}%
\pgfsetdash{}{0pt}%
\pgfsys@defobject{currentmarker}{\pgfqpoint{-0.027778in}{0.000000in}}{\pgfqpoint{-0.000000in}{0.000000in}}{%
\pgfpathmoveto{\pgfqpoint{-0.000000in}{0.000000in}}%
\pgfpathlineto{\pgfqpoint{-0.027778in}{0.000000in}}%
\pgfusepath{stroke,fill}%
}%
\begin{pgfscope}%
\pgfsys@transformshift{0.588387in}{1.386567in}%
\pgfsys@useobject{currentmarker}{}%
\end{pgfscope}%
\end{pgfscope}%
\begin{pgfscope}%
\pgfsetbuttcap%
\pgfsetroundjoin%
\definecolor{currentfill}{rgb}{0.000000,0.000000,0.000000}%
\pgfsetfillcolor{currentfill}%
\pgfsetlinewidth{0.602250pt}%
\definecolor{currentstroke}{rgb}{0.000000,0.000000,0.000000}%
\pgfsetstrokecolor{currentstroke}%
\pgfsetdash{}{0pt}%
\pgfsys@defobject{currentmarker}{\pgfqpoint{-0.027778in}{0.000000in}}{\pgfqpoint{-0.000000in}{0.000000in}}{%
\pgfpathmoveto{\pgfqpoint{-0.000000in}{0.000000in}}%
\pgfpathlineto{\pgfqpoint{-0.027778in}{0.000000in}}%
\pgfusepath{stroke,fill}%
}%
\begin{pgfscope}%
\pgfsys@transformshift{0.588387in}{1.601426in}%
\pgfsys@useobject{currentmarker}{}%
\end{pgfscope}%
\end{pgfscope}%
\begin{pgfscope}%
\pgfsetbuttcap%
\pgfsetroundjoin%
\definecolor{currentfill}{rgb}{0.000000,0.000000,0.000000}%
\pgfsetfillcolor{currentfill}%
\pgfsetlinewidth{0.602250pt}%
\definecolor{currentstroke}{rgb}{0.000000,0.000000,0.000000}%
\pgfsetstrokecolor{currentstroke}%
\pgfsetdash{}{0pt}%
\pgfsys@defobject{currentmarker}{\pgfqpoint{-0.027778in}{0.000000in}}{\pgfqpoint{-0.000000in}{0.000000in}}{%
\pgfpathmoveto{\pgfqpoint{-0.000000in}{0.000000in}}%
\pgfpathlineto{\pgfqpoint{-0.027778in}{0.000000in}}%
\pgfusepath{stroke,fill}%
}%
\begin{pgfscope}%
\pgfsys@transformshift{0.588387in}{1.768084in}%
\pgfsys@useobject{currentmarker}{}%
\end{pgfscope}%
\end{pgfscope}%
\begin{pgfscope}%
\pgfsetbuttcap%
\pgfsetroundjoin%
\definecolor{currentfill}{rgb}{0.000000,0.000000,0.000000}%
\pgfsetfillcolor{currentfill}%
\pgfsetlinewidth{0.602250pt}%
\definecolor{currentstroke}{rgb}{0.000000,0.000000,0.000000}%
\pgfsetstrokecolor{currentstroke}%
\pgfsetdash{}{0pt}%
\pgfsys@defobject{currentmarker}{\pgfqpoint{-0.027778in}{0.000000in}}{\pgfqpoint{-0.000000in}{0.000000in}}{%
\pgfpathmoveto{\pgfqpoint{-0.000000in}{0.000000in}}%
\pgfpathlineto{\pgfqpoint{-0.027778in}{0.000000in}}%
\pgfusepath{stroke,fill}%
}%
\begin{pgfscope}%
\pgfsys@transformshift{0.588387in}{1.904253in}%
\pgfsys@useobject{currentmarker}{}%
\end{pgfscope}%
\end{pgfscope}%
\begin{pgfscope}%
\pgfsetbuttcap%
\pgfsetroundjoin%
\definecolor{currentfill}{rgb}{0.000000,0.000000,0.000000}%
\pgfsetfillcolor{currentfill}%
\pgfsetlinewidth{0.602250pt}%
\definecolor{currentstroke}{rgb}{0.000000,0.000000,0.000000}%
\pgfsetstrokecolor{currentstroke}%
\pgfsetdash{}{0pt}%
\pgfsys@defobject{currentmarker}{\pgfqpoint{-0.027778in}{0.000000in}}{\pgfqpoint{-0.000000in}{0.000000in}}{%
\pgfpathmoveto{\pgfqpoint{-0.000000in}{0.000000in}}%
\pgfpathlineto{\pgfqpoint{-0.027778in}{0.000000in}}%
\pgfusepath{stroke,fill}%
}%
\begin{pgfscope}%
\pgfsys@transformshift{0.588387in}{2.019382in}%
\pgfsys@useobject{currentmarker}{}%
\end{pgfscope}%
\end{pgfscope}%
\begin{pgfscope}%
\pgfsetbuttcap%
\pgfsetroundjoin%
\definecolor{currentfill}{rgb}{0.000000,0.000000,0.000000}%
\pgfsetfillcolor{currentfill}%
\pgfsetlinewidth{0.602250pt}%
\definecolor{currentstroke}{rgb}{0.000000,0.000000,0.000000}%
\pgfsetstrokecolor{currentstroke}%
\pgfsetdash{}{0pt}%
\pgfsys@defobject{currentmarker}{\pgfqpoint{-0.027778in}{0.000000in}}{\pgfqpoint{-0.000000in}{0.000000in}}{%
\pgfpathmoveto{\pgfqpoint{-0.000000in}{0.000000in}}%
\pgfpathlineto{\pgfqpoint{-0.027778in}{0.000000in}}%
\pgfusepath{stroke,fill}%
}%
\begin{pgfscope}%
\pgfsys@transformshift{0.588387in}{2.119112in}%
\pgfsys@useobject{currentmarker}{}%
\end{pgfscope}%
\end{pgfscope}%
\begin{pgfscope}%
\pgfsetbuttcap%
\pgfsetroundjoin%
\definecolor{currentfill}{rgb}{0.000000,0.000000,0.000000}%
\pgfsetfillcolor{currentfill}%
\pgfsetlinewidth{0.602250pt}%
\definecolor{currentstroke}{rgb}{0.000000,0.000000,0.000000}%
\pgfsetstrokecolor{currentstroke}%
\pgfsetdash{}{0pt}%
\pgfsys@defobject{currentmarker}{\pgfqpoint{-0.027778in}{0.000000in}}{\pgfqpoint{-0.000000in}{0.000000in}}{%
\pgfpathmoveto{\pgfqpoint{-0.000000in}{0.000000in}}%
\pgfpathlineto{\pgfqpoint{-0.027778in}{0.000000in}}%
\pgfusepath{stroke,fill}%
}%
\begin{pgfscope}%
\pgfsys@transformshift{0.588387in}{2.207080in}%
\pgfsys@useobject{currentmarker}{}%
\end{pgfscope}%
\end{pgfscope}%
\begin{pgfscope}%
\definecolor{textcolor}{rgb}{0.000000,0.000000,0.000000}%
\pgfsetstrokecolor{textcolor}%
\pgfsetfillcolor{textcolor}%
\pgftext[x=0.234413in,y=1.631726in,,bottom,rotate=90.000000]{\color{textcolor}{\rmfamily\fontsize{10.000000}{12.000000}\selectfont\catcode`\^=\active\def^{\ifmmode\sp\else\^{}\fi}\catcode`\%=\active\def%{\%}Time [ms]}}%
\end{pgfscope}%
\begin{pgfscope}%
\pgfpathrectangle{\pgfqpoint{0.588387in}{0.521603in}}{\pgfqpoint{7.103961in}{2.220246in}}%
\pgfusepath{clip}%
\pgfsetrectcap%
\pgfsetroundjoin%
\pgfsetlinewidth{1.505625pt}%
\pgfsetstrokecolor{currentstroke1}%
\pgfsetdash{}{0pt}%
\pgfpathmoveto{\pgfqpoint{0.911295in}{0.627672in}}%
\pgfpathlineto{\pgfqpoint{1.007685in}{0.716963in}}%
\pgfpathlineto{\pgfqpoint{1.104075in}{0.779647in}}%
\pgfpathlineto{\pgfqpoint{1.200465in}{0.862644in}}%
\pgfpathlineto{\pgfqpoint{1.296855in}{0.897036in}}%
\pgfpathlineto{\pgfqpoint{1.393246in}{0.925354in}}%
\pgfpathlineto{\pgfqpoint{1.489636in}{0.896776in}}%
\pgfpathlineto{\pgfqpoint{1.586026in}{0.914595in}}%
\pgfpathlineto{\pgfqpoint{1.682416in}{0.907565in}}%
\pgfpathlineto{\pgfqpoint{1.778807in}{0.866470in}}%
\pgfpathlineto{\pgfqpoint{1.875197in}{0.887986in}}%
\pgfpathlineto{\pgfqpoint{1.971587in}{0.905565in}}%
\pgfpathlineto{\pgfqpoint{2.067977in}{0.873759in}}%
\pgfpathlineto{\pgfqpoint{2.164368in}{0.930435in}}%
\pgfpathlineto{\pgfqpoint{2.260758in}{0.942224in}}%
\pgfpathlineto{\pgfqpoint{2.357148in}{1.001527in}}%
\pgfpathlineto{\pgfqpoint{2.453538in}{1.004364in}}%
\pgfpathlineto{\pgfqpoint{2.549929in}{1.036659in}}%
\pgfpathlineto{\pgfqpoint{2.646319in}{1.067083in}}%
\pgfpathlineto{\pgfqpoint{2.742709in}{1.071488in}}%
\pgfpathlineto{\pgfqpoint{2.839099in}{1.087297in}}%
\pgfpathlineto{\pgfqpoint{2.935490in}{1.148348in}}%
\pgfpathlineto{\pgfqpoint{3.031880in}{1.198726in}}%
\pgfpathlineto{\pgfqpoint{3.128270in}{1.219945in}}%
\pgfpathlineto{\pgfqpoint{3.224660in}{1.230570in}}%
\pgfpathlineto{\pgfqpoint{3.321050in}{1.333140in}}%
\pgfpathlineto{\pgfqpoint{3.417441in}{1.303696in}}%
\pgfpathlineto{\pgfqpoint{3.513831in}{1.364559in}}%
\pgfpathlineto{\pgfqpoint{3.610221in}{1.351608in}}%
\pgfpathlineto{\pgfqpoint{3.706611in}{1.373302in}}%
\pgfpathlineto{\pgfqpoint{3.803002in}{1.392192in}}%
\pgfpathlineto{\pgfqpoint{3.899392in}{1.478673in}}%
\pgfpathlineto{\pgfqpoint{3.995782in}{1.488131in}}%
\pgfpathlineto{\pgfqpoint{4.092172in}{1.494595in}}%
\pgfpathlineto{\pgfqpoint{4.188563in}{1.572961in}}%
\pgfpathlineto{\pgfqpoint{4.284953in}{1.569559in}}%
\pgfpathlineto{\pgfqpoint{4.381343in}{1.483061in}}%
\pgfpathlineto{\pgfqpoint{4.477733in}{1.658581in}}%
\pgfpathlineto{\pgfqpoint{4.574124in}{1.664235in}}%
\pgfpathlineto{\pgfqpoint{4.670514in}{1.849941in}}%
\pgfpathlineto{\pgfqpoint{4.766904in}{1.651303in}}%
\pgfpathlineto{\pgfqpoint{4.863294in}{2.057725in}}%
\pgfpathlineto{\pgfqpoint{4.959684in}{1.842148in}}%
\pgfpathlineto{\pgfqpoint{5.056075in}{1.902955in}}%
\pgfpathlineto{\pgfqpoint{5.152465in}{1.933546in}}%
\pgfpathlineto{\pgfqpoint{5.248855in}{1.873805in}}%
\pgfpathlineto{\pgfqpoint{5.441636in}{2.147282in}}%
\pgfpathlineto{\pgfqpoint{5.538026in}{1.883267in}}%
\pgfpathlineto{\pgfqpoint{5.634416in}{2.201980in}}%
\pgfpathlineto{\pgfqpoint{5.827197in}{2.411440in}}%
\pgfpathlineto{\pgfqpoint{5.923587in}{2.396780in}}%
\pgfpathlineto{\pgfqpoint{6.019977in}{2.146326in}}%
\pgfpathlineto{\pgfqpoint{6.116367in}{2.492769in}}%
\pgfpathlineto{\pgfqpoint{6.212758in}{2.179933in}}%
\pgfpathlineto{\pgfqpoint{6.405538in}{2.266142in}}%
\pgfpathlineto{\pgfqpoint{6.598319in}{2.637381in}}%
\pgfpathlineto{\pgfqpoint{6.887489in}{2.378535in}}%
\pgfpathlineto{\pgfqpoint{7.369440in}{2.640929in}}%
\pgfusepath{stroke}%
\end{pgfscope}%
\begin{pgfscope}%
\pgfpathrectangle{\pgfqpoint{0.588387in}{0.521603in}}{\pgfqpoint{7.103961in}{2.220246in}}%
\pgfusepath{clip}%
\pgfsetrectcap%
\pgfsetroundjoin%
\pgfsetlinewidth{1.505625pt}%
\pgfsetstrokecolor{currentstroke2}%
\pgfsetdash{}{0pt}%
\pgfpathmoveto{\pgfqpoint{0.911295in}{0.622524in}}%
\pgfpathlineto{\pgfqpoint{1.007685in}{0.748635in}}%
\pgfpathlineto{\pgfqpoint{1.104075in}{0.821323in}}%
\pgfpathlineto{\pgfqpoint{1.200465in}{0.862482in}}%
\pgfpathlineto{\pgfqpoint{1.296855in}{0.890851in}}%
\pgfpathlineto{\pgfqpoint{1.393246in}{0.950999in}}%
\pgfpathlineto{\pgfqpoint{1.489636in}{0.918876in}}%
\pgfpathlineto{\pgfqpoint{1.586026in}{0.906775in}}%
\pgfpathlineto{\pgfqpoint{1.682416in}{0.912984in}}%
\pgfpathlineto{\pgfqpoint{1.778807in}{0.871058in}}%
\pgfpathlineto{\pgfqpoint{1.875197in}{0.901326in}}%
\pgfpathlineto{\pgfqpoint{1.971587in}{0.897730in}}%
\pgfpathlineto{\pgfqpoint{2.067977in}{0.855432in}}%
\pgfpathlineto{\pgfqpoint{2.164368in}{0.886323in}}%
\pgfpathlineto{\pgfqpoint{2.260758in}{0.925999in}}%
\pgfpathlineto{\pgfqpoint{2.357148in}{0.939629in}}%
\pgfpathlineto{\pgfqpoint{2.453538in}{0.961896in}}%
\pgfpathlineto{\pgfqpoint{2.549929in}{0.969303in}}%
\pgfpathlineto{\pgfqpoint{2.646319in}{1.034593in}}%
\pgfpathlineto{\pgfqpoint{2.742709in}{1.024894in}}%
\pgfpathlineto{\pgfqpoint{2.839099in}{1.022830in}}%
\pgfpathlineto{\pgfqpoint{2.935490in}{1.105264in}}%
\pgfpathlineto{\pgfqpoint{3.031880in}{1.138222in}}%
\pgfpathlineto{\pgfqpoint{3.128270in}{1.131220in}}%
\pgfpathlineto{\pgfqpoint{3.224660in}{1.142878in}}%
\pgfpathlineto{\pgfqpoint{3.321050in}{1.293424in}}%
\pgfpathlineto{\pgfqpoint{3.417441in}{1.236528in}}%
\pgfpathlineto{\pgfqpoint{3.513831in}{1.251891in}}%
\pgfpathlineto{\pgfqpoint{3.610221in}{1.257411in}}%
\pgfpathlineto{\pgfqpoint{3.706611in}{1.311465in}}%
\pgfpathlineto{\pgfqpoint{3.803002in}{1.345520in}}%
\pgfpathlineto{\pgfqpoint{3.899392in}{1.339361in}}%
\pgfpathlineto{\pgfqpoint{3.995782in}{1.371682in}}%
\pgfpathlineto{\pgfqpoint{4.092172in}{1.398055in}}%
\pgfpathlineto{\pgfqpoint{4.188563in}{1.506518in}}%
\pgfpathlineto{\pgfqpoint{4.284953in}{1.457625in}}%
\pgfpathlineto{\pgfqpoint{4.381343in}{1.403713in}}%
\pgfpathlineto{\pgfqpoint{4.477733in}{1.534946in}}%
\pgfpathlineto{\pgfqpoint{4.574124in}{1.555164in}}%
\pgfpathlineto{\pgfqpoint{4.670514in}{1.724726in}}%
\pgfpathlineto{\pgfqpoint{4.766904in}{1.605151in}}%
\pgfpathlineto{\pgfqpoint{4.863294in}{1.848713in}}%
\pgfpathlineto{\pgfqpoint{4.959684in}{1.725169in}}%
\pgfpathlineto{\pgfqpoint{5.056075in}{1.759823in}}%
\pgfpathlineto{\pgfqpoint{5.152465in}{1.808638in}}%
\pgfpathlineto{\pgfqpoint{5.248855in}{1.740546in}}%
\pgfpathlineto{\pgfqpoint{5.441636in}{2.029240in}}%
\pgfpathlineto{\pgfqpoint{5.538026in}{1.743330in}}%
\pgfpathlineto{\pgfqpoint{5.634416in}{2.009718in}}%
\pgfpathlineto{\pgfqpoint{5.827197in}{2.250677in}}%
\pgfpathlineto{\pgfqpoint{5.923587in}{2.227462in}}%
\pgfpathlineto{\pgfqpoint{6.019977in}{2.000856in}}%
\pgfpathlineto{\pgfqpoint{6.116367in}{2.298506in}}%
\pgfpathlineto{\pgfqpoint{6.212758in}{2.026920in}}%
\pgfpathlineto{\pgfqpoint{6.405538in}{2.074239in}}%
\pgfpathlineto{\pgfqpoint{6.598319in}{2.480339in}}%
\pgfpathlineto{\pgfqpoint{6.887489in}{2.212866in}}%
\pgfpathlineto{\pgfqpoint{7.369440in}{2.473400in}}%
\pgfusepath{stroke}%
\end{pgfscope}%
\begin{pgfscope}%
\pgfpathrectangle{\pgfqpoint{0.588387in}{0.521603in}}{\pgfqpoint{7.103961in}{2.220246in}}%
\pgfusepath{clip}%
\pgfsetrectcap%
\pgfsetroundjoin%
\pgfsetlinewidth{1.505625pt}%
\pgfsetstrokecolor{currentstroke3}%
\pgfsetdash{}{0pt}%
\pgfpathmoveto{\pgfqpoint{0.911295in}{0.689895in}}%
\pgfpathlineto{\pgfqpoint{1.007685in}{0.796328in}}%
\pgfpathlineto{\pgfqpoint{1.104075in}{0.772888in}}%
\pgfpathlineto{\pgfqpoint{1.200465in}{0.841509in}}%
\pgfpathlineto{\pgfqpoint{1.296855in}{0.877182in}}%
\pgfpathlineto{\pgfqpoint{1.393246in}{0.959394in}}%
\pgfpathlineto{\pgfqpoint{1.489636in}{0.916184in}}%
\pgfpathlineto{\pgfqpoint{1.586026in}{0.869314in}}%
\pgfpathlineto{\pgfqpoint{1.682416in}{0.885536in}}%
\pgfpathlineto{\pgfqpoint{1.778807in}{0.856569in}}%
\pgfpathlineto{\pgfqpoint{1.875197in}{0.866021in}}%
\pgfpathlineto{\pgfqpoint{1.971587in}{0.918266in}}%
\pgfpathlineto{\pgfqpoint{2.067977in}{0.821436in}}%
\pgfpathlineto{\pgfqpoint{2.164368in}{0.841036in}}%
\pgfpathlineto{\pgfqpoint{2.260758in}{0.891687in}}%
\pgfpathlineto{\pgfqpoint{2.357148in}{0.933552in}}%
\pgfpathlineto{\pgfqpoint{2.453538in}{0.966341in}}%
\pgfpathlineto{\pgfqpoint{2.549929in}{0.917841in}}%
\pgfpathlineto{\pgfqpoint{2.646319in}{0.976135in}}%
\pgfpathlineto{\pgfqpoint{2.742709in}{1.002773in}}%
\pgfpathlineto{\pgfqpoint{2.839099in}{0.993212in}}%
\pgfpathlineto{\pgfqpoint{2.935490in}{1.079109in}}%
\pgfpathlineto{\pgfqpoint{3.031880in}{1.089590in}}%
\pgfpathlineto{\pgfqpoint{3.128270in}{1.098874in}}%
\pgfpathlineto{\pgfqpoint{3.224660in}{1.106548in}}%
\pgfpathlineto{\pgfqpoint{3.321050in}{1.249179in}}%
\pgfpathlineto{\pgfqpoint{3.417441in}{1.207213in}}%
\pgfpathlineto{\pgfqpoint{3.513831in}{1.182548in}}%
\pgfpathlineto{\pgfqpoint{3.610221in}{1.210200in}}%
\pgfpathlineto{\pgfqpoint{3.706611in}{1.285556in}}%
\pgfpathlineto{\pgfqpoint{3.803002in}{1.270587in}}%
\pgfpathlineto{\pgfqpoint{3.899392in}{1.331094in}}%
\pgfpathlineto{\pgfqpoint{3.995782in}{1.342944in}}%
\pgfpathlineto{\pgfqpoint{4.092172in}{1.344925in}}%
\pgfpathlineto{\pgfqpoint{4.188563in}{1.493314in}}%
\pgfpathlineto{\pgfqpoint{4.284953in}{1.411905in}}%
\pgfpathlineto{\pgfqpoint{4.381343in}{1.390396in}}%
\pgfpathlineto{\pgfqpoint{4.477733in}{1.473428in}}%
\pgfpathlineto{\pgfqpoint{4.574124in}{1.496343in}}%
\pgfpathlineto{\pgfqpoint{4.670514in}{1.695249in}}%
\pgfpathlineto{\pgfqpoint{4.766904in}{1.503029in}}%
\pgfpathlineto{\pgfqpoint{4.863294in}{1.818790in}}%
\pgfpathlineto{\pgfqpoint{4.959684in}{1.637777in}}%
\pgfpathlineto{\pgfqpoint{5.056075in}{1.667643in}}%
\pgfpathlineto{\pgfqpoint{5.152465in}{1.773294in}}%
\pgfpathlineto{\pgfqpoint{5.248855in}{1.687161in}}%
\pgfpathlineto{\pgfqpoint{5.441636in}{1.945125in}}%
\pgfpathlineto{\pgfqpoint{5.538026in}{1.631952in}}%
\pgfpathlineto{\pgfqpoint{5.634416in}{1.911839in}}%
\pgfpathlineto{\pgfqpoint{5.827197in}{2.121442in}}%
\pgfpathlineto{\pgfqpoint{5.923587in}{2.148180in}}%
\pgfpathlineto{\pgfqpoint{6.019977in}{1.909602in}}%
\pgfpathlineto{\pgfqpoint{6.116367in}{2.192203in}}%
\pgfpathlineto{\pgfqpoint{6.212758in}{1.946008in}}%
\pgfpathlineto{\pgfqpoint{6.405538in}{1.991667in}}%
\pgfpathlineto{\pgfqpoint{6.598319in}{2.392098in}}%
\pgfpathlineto{\pgfqpoint{6.887489in}{2.104737in}}%
\pgfpathlineto{\pgfqpoint{7.369440in}{2.351673in}}%
\pgfusepath{stroke}%
\end{pgfscope}%
\begin{pgfscope}%
\pgfpathrectangle{\pgfqpoint{0.588387in}{0.521603in}}{\pgfqpoint{7.103961in}{2.220246in}}%
\pgfusepath{clip}%
\pgfsetrectcap%
\pgfsetroundjoin%
\pgfsetlinewidth{1.505625pt}%
\pgfsetstrokecolor{currentstroke4}%
\pgfsetdash{}{0pt}%
\pgfpathmoveto{\pgfqpoint{0.911295in}{0.660977in}}%
\pgfpathlineto{\pgfqpoint{1.007685in}{0.803543in}}%
\pgfpathlineto{\pgfqpoint{1.104075in}{1.001028in}}%
\pgfpathlineto{\pgfqpoint{1.200465in}{1.160946in}}%
\pgfpathlineto{\pgfqpoint{1.296855in}{1.223041in}}%
\pgfpathlineto{\pgfqpoint{1.393246in}{1.006619in}}%
\pgfpathlineto{\pgfqpoint{1.489636in}{0.978099in}}%
\pgfpathlineto{\pgfqpoint{1.586026in}{1.031442in}}%
\pgfpathlineto{\pgfqpoint{1.682416in}{1.061200in}}%
\pgfpathlineto{\pgfqpoint{1.778807in}{1.062754in}}%
\pgfpathlineto{\pgfqpoint{1.875197in}{1.142241in}}%
\pgfpathlineto{\pgfqpoint{1.971587in}{0.960193in}}%
\pgfpathlineto{\pgfqpoint{2.067977in}{0.957976in}}%
\pgfpathlineto{\pgfqpoint{2.164368in}{1.001218in}}%
\pgfpathlineto{\pgfqpoint{2.260758in}{1.076375in}}%
\pgfpathlineto{\pgfqpoint{2.357148in}{1.105351in}}%
\pgfpathlineto{\pgfqpoint{2.453538in}{1.146619in}}%
\pgfpathlineto{\pgfqpoint{2.549929in}{1.021262in}}%
\pgfpathlineto{\pgfqpoint{2.646319in}{1.086967in}}%
\pgfpathlineto{\pgfqpoint{2.742709in}{1.117274in}}%
\pgfpathlineto{\pgfqpoint{2.839099in}{1.098515in}}%
\pgfpathlineto{\pgfqpoint{2.935490in}{1.207039in}}%
\pgfpathlineto{\pgfqpoint{3.031880in}{1.240931in}}%
\pgfpathlineto{\pgfqpoint{3.128270in}{1.155433in}}%
\pgfpathlineto{\pgfqpoint{3.224660in}{1.175097in}}%
\pgfpathlineto{\pgfqpoint{3.321050in}{1.300185in}}%
\pgfpathlineto{\pgfqpoint{3.417441in}{1.272163in}}%
\pgfpathlineto{\pgfqpoint{3.513831in}{1.275626in}}%
\pgfpathlineto{\pgfqpoint{3.610221in}{1.304213in}}%
\pgfpathlineto{\pgfqpoint{3.706611in}{1.329835in}}%
\pgfpathlineto{\pgfqpoint{3.803002in}{1.295325in}}%
\pgfpathlineto{\pgfqpoint{3.899392in}{1.362427in}}%
\pgfpathlineto{\pgfqpoint{3.995782in}{1.387213in}}%
\pgfpathlineto{\pgfqpoint{4.092172in}{1.389215in}}%
\pgfpathlineto{\pgfqpoint{4.188563in}{1.637807in}}%
\pgfpathlineto{\pgfqpoint{4.284953in}{1.431059in}}%
\pgfpathlineto{\pgfqpoint{4.381343in}{1.427808in}}%
\pgfpathlineto{\pgfqpoint{4.477733in}{1.485761in}}%
\pgfpathlineto{\pgfqpoint{4.574124in}{1.575180in}}%
\pgfpathlineto{\pgfqpoint{4.670514in}{1.649654in}}%
\pgfpathlineto{\pgfqpoint{4.766904in}{1.602626in}}%
\pgfpathlineto{\pgfqpoint{4.863294in}{1.619367in}}%
\pgfpathlineto{\pgfqpoint{4.959684in}{1.648335in}}%
\pgfpathlineto{\pgfqpoint{5.056075in}{1.668354in}}%
\pgfpathlineto{\pgfqpoint{5.152465in}{1.744344in}}%
\pgfpathlineto{\pgfqpoint{5.248855in}{1.715666in}}%
\pgfpathlineto{\pgfqpoint{5.345245in}{1.729971in}}%
\pgfpathlineto{\pgfqpoint{5.441636in}{1.914672in}}%
\pgfpathlineto{\pgfqpoint{5.538026in}{1.603198in}}%
\pgfpathlineto{\pgfqpoint{5.634416in}{1.900822in}}%
\pgfpathlineto{\pgfqpoint{5.730806in}{1.891700in}}%
\pgfpathlineto{\pgfqpoint{5.827197in}{2.015210in}}%
\pgfpathlineto{\pgfqpoint{5.923587in}{2.064849in}}%
\pgfpathlineto{\pgfqpoint{6.019977in}{1.898874in}}%
\pgfpathlineto{\pgfqpoint{6.116367in}{2.122093in}}%
\pgfpathlineto{\pgfqpoint{6.212758in}{1.912224in}}%
\pgfpathlineto{\pgfqpoint{6.309148in}{1.960579in}}%
\pgfpathlineto{\pgfqpoint{6.405538in}{1.991588in}}%
\pgfpathlineto{\pgfqpoint{6.598319in}{2.288639in}}%
\pgfpathlineto{\pgfqpoint{6.791099in}{2.162411in}}%
\pgfpathlineto{\pgfqpoint{6.887489in}{2.094757in}}%
\pgfpathlineto{\pgfqpoint{7.080270in}{2.445675in}}%
\pgfpathlineto{\pgfqpoint{7.369440in}{2.343292in}}%
\pgfusepath{stroke}%
\end{pgfscope}%
\begin{pgfscope}%
\pgfpathrectangle{\pgfqpoint{0.588387in}{0.521603in}}{\pgfqpoint{7.103961in}{2.220246in}}%
\pgfusepath{clip}%
\pgfsetrectcap%
\pgfsetroundjoin%
\pgfsetlinewidth{1.505625pt}%
\pgfsetstrokecolor{currentstroke5}%
\pgfsetdash{}{0pt}%
\pgfpathmoveto{\pgfqpoint{0.911295in}{2.039269in}}%
\pgfpathlineto{\pgfqpoint{1.007685in}{0.801865in}}%
\pgfpathlineto{\pgfqpoint{1.104075in}{0.924122in}}%
\pgfpathlineto{\pgfqpoint{1.200465in}{1.033001in}}%
\pgfpathlineto{\pgfqpoint{1.296855in}{1.121377in}}%
\pgfpathlineto{\pgfqpoint{1.393246in}{1.298890in}}%
\pgfpathlineto{\pgfqpoint{1.489636in}{1.369344in}}%
\pgfpathlineto{\pgfqpoint{1.586026in}{1.524834in}}%
\pgfpathlineto{\pgfqpoint{1.682416in}{0.974821in}}%
\pgfpathlineto{\pgfqpoint{1.778807in}{0.943209in}}%
\pgfpathlineto{\pgfqpoint{1.875197in}{1.013391in}}%
\pgfpathlineto{\pgfqpoint{1.971587in}{1.125246in}}%
\pgfpathlineto{\pgfqpoint{2.067977in}{1.135859in}}%
\pgfpathlineto{\pgfqpoint{2.164368in}{1.199592in}}%
\pgfpathlineto{\pgfqpoint{2.260758in}{1.301292in}}%
\pgfpathlineto{\pgfqpoint{2.357148in}{1.022842in}}%
\pgfpathlineto{\pgfqpoint{2.453538in}{1.063503in}}%
\pgfpathlineto{\pgfqpoint{2.549929in}{1.112539in}}%
\pgfpathlineto{\pgfqpoint{2.646319in}{1.163211in}}%
\pgfpathlineto{\pgfqpoint{2.742709in}{1.233768in}}%
\pgfpathlineto{\pgfqpoint{2.839099in}{1.137694in}}%
\pgfpathlineto{\pgfqpoint{2.935490in}{1.325466in}}%
\pgfpathlineto{\pgfqpoint{3.031880in}{1.184656in}}%
\pgfpathlineto{\pgfqpoint{3.128270in}{1.180163in}}%
\pgfpathlineto{\pgfqpoint{3.224660in}{1.198583in}}%
\pgfpathlineto{\pgfqpoint{3.321050in}{1.378312in}}%
\pgfpathlineto{\pgfqpoint{3.417441in}{1.328770in}}%
\pgfpathlineto{\pgfqpoint{3.513831in}{1.324088in}}%
\pgfpathlineto{\pgfqpoint{3.610221in}{1.377993in}}%
\pgfpathlineto{\pgfqpoint{3.706611in}{1.322980in}}%
\pgfpathlineto{\pgfqpoint{3.803002in}{1.333738in}}%
\pgfpathlineto{\pgfqpoint{3.899392in}{1.359184in}}%
\pgfpathlineto{\pgfqpoint{3.995782in}{1.437383in}}%
\pgfpathlineto{\pgfqpoint{4.092172in}{1.393850in}}%
\pgfpathlineto{\pgfqpoint{4.188563in}{1.634078in}}%
\pgfpathlineto{\pgfqpoint{4.284953in}{1.527674in}}%
\pgfpathlineto{\pgfqpoint{4.381343in}{1.455201in}}%
\pgfpathlineto{\pgfqpoint{4.477733in}{1.498861in}}%
\pgfpathlineto{\pgfqpoint{4.574124in}{1.563881in}}%
\pgfpathlineto{\pgfqpoint{4.670514in}{1.720281in}}%
\pgfpathlineto{\pgfqpoint{4.766904in}{1.684398in}}%
\pgfpathlineto{\pgfqpoint{4.863294in}{1.753947in}}%
\pgfpathlineto{\pgfqpoint{4.959684in}{1.708801in}}%
\pgfpathlineto{\pgfqpoint{5.056075in}{1.675786in}}%
\pgfpathlineto{\pgfqpoint{5.152465in}{1.791247in}}%
\pgfpathlineto{\pgfqpoint{5.248855in}{1.686734in}}%
\pgfpathlineto{\pgfqpoint{5.441636in}{1.997458in}}%
\pgfpathlineto{\pgfqpoint{5.538026in}{1.632960in}}%
\pgfpathlineto{\pgfqpoint{5.634416in}{1.879577in}}%
\pgfpathlineto{\pgfqpoint{5.827197in}{2.015907in}}%
\pgfpathlineto{\pgfqpoint{5.923587in}{2.060002in}}%
\pgfpathlineto{\pgfqpoint{6.019977in}{1.848713in}}%
\pgfpathlineto{\pgfqpoint{6.116367in}{2.107587in}}%
\pgfpathlineto{\pgfqpoint{6.212758in}{1.968188in}}%
\pgfpathlineto{\pgfqpoint{6.405538in}{1.956721in}}%
\pgfpathlineto{\pgfqpoint{6.598319in}{2.219323in}}%
\pgfpathlineto{\pgfqpoint{6.887489in}{1.982196in}}%
\pgfpathlineto{\pgfqpoint{7.369440in}{2.321408in}}%
\pgfusepath{stroke}%
\end{pgfscope}%
\begin{pgfscope}%
\pgfpathrectangle{\pgfqpoint{0.588387in}{0.521603in}}{\pgfqpoint{7.103961in}{2.220246in}}%
\pgfusepath{clip}%
\pgfsetrectcap%
\pgfsetroundjoin%
\pgfsetlinewidth{1.505625pt}%
\pgfsetstrokecolor{currentstroke6}%
\pgfsetdash{}{0pt}%
\pgfpathmoveto{\pgfqpoint{0.911295in}{2.074655in}}%
\pgfpathlineto{\pgfqpoint{1.007685in}{2.237534in}}%
\pgfpathlineto{\pgfqpoint{1.104075in}{2.507991in}}%
\pgfpathlineto{\pgfqpoint{1.200465in}{1.155017in}}%
\pgfpathlineto{\pgfqpoint{1.296855in}{1.254459in}}%
\pgfpathlineto{\pgfqpoint{1.393246in}{1.428645in}}%
\pgfpathlineto{\pgfqpoint{1.489636in}{1.487286in}}%
\pgfpathlineto{\pgfqpoint{1.586026in}{1.642403in}}%
\pgfpathlineto{\pgfqpoint{1.682416in}{1.809143in}}%
\pgfpathlineto{\pgfqpoint{1.778807in}{1.853991in}}%
\pgfpathlineto{\pgfqpoint{1.875197in}{2.108712in}}%
\pgfpathlineto{\pgfqpoint{1.971587in}{1.201064in}}%
\pgfpathlineto{\pgfqpoint{2.067977in}{1.246913in}}%
\pgfpathlineto{\pgfqpoint{2.164368in}{1.301647in}}%
\pgfpathlineto{\pgfqpoint{2.260758in}{1.428830in}}%
\pgfpathlineto{\pgfqpoint{2.357148in}{1.569217in}}%
\pgfpathlineto{\pgfqpoint{2.453538in}{1.670172in}}%
\pgfpathlineto{\pgfqpoint{2.549929in}{1.698689in}}%
\pgfpathlineto{\pgfqpoint{2.646319in}{1.879924in}}%
\pgfpathlineto{\pgfqpoint{2.742709in}{1.315012in}}%
\pgfpathlineto{\pgfqpoint{2.839099in}{1.300232in}}%
\pgfpathlineto{\pgfqpoint{2.935490in}{1.430300in}}%
\pgfpathlineto{\pgfqpoint{3.031880in}{1.524112in}}%
\pgfpathlineto{\pgfqpoint{3.128270in}{1.573446in}}%
\pgfpathlineto{\pgfqpoint{3.224660in}{1.586290in}}%
\pgfpathlineto{\pgfqpoint{3.321050in}{1.792897in}}%
\pgfpathlineto{\pgfqpoint{3.417441in}{1.790603in}}%
\pgfpathlineto{\pgfqpoint{3.513831in}{1.412769in}}%
\pgfpathlineto{\pgfqpoint{3.610221in}{1.478673in}}%
\pgfpathlineto{\pgfqpoint{3.706611in}{1.662140in}}%
\pgfpathlineto{\pgfqpoint{3.803002in}{1.590883in}}%
\pgfpathlineto{\pgfqpoint{3.899392in}{1.549731in}}%
\pgfpathlineto{\pgfqpoint{3.995782in}{1.667073in}}%
\pgfpathlineto{\pgfqpoint{4.092172in}{1.719883in}}%
\pgfpathlineto{\pgfqpoint{4.188563in}{1.983223in}}%
\pgfpathlineto{\pgfqpoint{4.284953in}{1.594706in}}%
\pgfpathlineto{\pgfqpoint{4.381343in}{1.636679in}}%
\pgfpathlineto{\pgfqpoint{4.477733in}{1.657175in}}%
\pgfpathlineto{\pgfqpoint{4.574124in}{1.727610in}}%
\pgfpathlineto{\pgfqpoint{4.670514in}{1.862113in}}%
\pgfpathlineto{\pgfqpoint{4.766904in}{1.895803in}}%
\pgfpathlineto{\pgfqpoint{4.863294in}{1.798811in}}%
\pgfpathlineto{\pgfqpoint{4.959684in}{1.888020in}}%
\pgfpathlineto{\pgfqpoint{5.056075in}{1.807954in}}%
\pgfpathlineto{\pgfqpoint{5.152465in}{1.872792in}}%
\pgfpathlineto{\pgfqpoint{5.248855in}{1.860272in}}%
\pgfpathlineto{\pgfqpoint{5.441636in}{2.253724in}}%
\pgfpathlineto{\pgfqpoint{5.538026in}{1.769576in}}%
\pgfpathlineto{\pgfqpoint{5.634416in}{2.100682in}}%
\pgfpathlineto{\pgfqpoint{5.827197in}{2.069415in}}%
\pgfpathlineto{\pgfqpoint{5.923587in}{2.023373in}}%
\pgfpathlineto{\pgfqpoint{6.019977in}{1.961588in}}%
\pgfpathlineto{\pgfqpoint{6.116367in}{2.170734in}}%
\pgfpathlineto{\pgfqpoint{6.212758in}{2.095159in}}%
\pgfpathlineto{\pgfqpoint{6.405538in}{2.121326in}}%
\pgfpathlineto{\pgfqpoint{6.598319in}{2.257224in}}%
\pgfpathlineto{\pgfqpoint{6.887489in}{2.150869in}}%
\pgfpathlineto{\pgfqpoint{7.369440in}{2.359833in}}%
\pgfusepath{stroke}%
\end{pgfscope}%
\begin{pgfscope}%
\pgfsetrectcap%
\pgfsetmiterjoin%
\pgfsetlinewidth{0.803000pt}%
\definecolor{currentstroke}{rgb}{0.000000,0.000000,0.000000}%
\pgfsetstrokecolor{currentstroke}%
\pgfsetdash{}{0pt}%
\pgfpathmoveto{\pgfqpoint{0.588387in}{0.521603in}}%
\pgfpathlineto{\pgfqpoint{0.588387in}{2.741849in}}%
\pgfusepath{stroke}%
\end{pgfscope}%
\begin{pgfscope}%
\pgfsetrectcap%
\pgfsetmiterjoin%
\pgfsetlinewidth{0.803000pt}%
\definecolor{currentstroke}{rgb}{0.000000,0.000000,0.000000}%
\pgfsetstrokecolor{currentstroke}%
\pgfsetdash{}{0pt}%
\pgfpathmoveto{\pgfqpoint{7.692348in}{0.521603in}}%
\pgfpathlineto{\pgfqpoint{7.692348in}{2.741849in}}%
\pgfusepath{stroke}%
\end{pgfscope}%
\begin{pgfscope}%
\pgfsetrectcap%
\pgfsetmiterjoin%
\pgfsetlinewidth{0.803000pt}%
\definecolor{currentstroke}{rgb}{0.000000,0.000000,0.000000}%
\pgfsetstrokecolor{currentstroke}%
\pgfsetdash{}{0pt}%
\pgfpathmoveto{\pgfqpoint{0.588387in}{0.521603in}}%
\pgfpathlineto{\pgfqpoint{7.692348in}{0.521603in}}%
\pgfusepath{stroke}%
\end{pgfscope}%
\begin{pgfscope}%
\pgfsetrectcap%
\pgfsetmiterjoin%
\pgfsetlinewidth{0.803000pt}%
\definecolor{currentstroke}{rgb}{0.000000,0.000000,0.000000}%
\pgfsetstrokecolor{currentstroke}%
\pgfsetdash{}{0pt}%
\pgfpathmoveto{\pgfqpoint{0.588387in}{2.741849in}}%
\pgfpathlineto{\pgfqpoint{7.692348in}{2.741849in}}%
\pgfusepath{stroke}%
\end{pgfscope}%
\begin{pgfscope}%
\pgfsetbuttcap%
\pgfsetmiterjoin%
\definecolor{currentfill}{rgb}{1.000000,1.000000,1.000000}%
\pgfsetfillcolor{currentfill}%
\pgfsetfillopacity{0.800000}%
\pgfsetlinewidth{1.003750pt}%
\definecolor{currentstroke}{rgb}{0.800000,0.800000,0.800000}%
\pgfsetstrokecolor{currentstroke}%
\pgfsetstrokeopacity{0.800000}%
\pgfsetdash{}{0pt}%
\pgfpathmoveto{\pgfqpoint{7.779848in}{1.541020in}}%
\pgfpathlineto{\pgfqpoint{8.259376in}{1.541020in}}%
\pgfpathquadraticcurveto{\pgfqpoint{8.284376in}{1.541020in}}{\pgfqpoint{8.284376in}{1.566020in}}%
\pgfpathlineto{\pgfqpoint{8.284376in}{2.654349in}}%
\pgfpathquadraticcurveto{\pgfqpoint{8.284376in}{2.679349in}}{\pgfqpoint{8.259376in}{2.679349in}}%
\pgfpathlineto{\pgfqpoint{7.779848in}{2.679349in}}%
\pgfpathquadraticcurveto{\pgfqpoint{7.754848in}{2.679349in}}{\pgfqpoint{7.754848in}{2.654349in}}%
\pgfpathlineto{\pgfqpoint{7.754848in}{1.566020in}}%
\pgfpathquadraticcurveto{\pgfqpoint{7.754848in}{1.541020in}}{\pgfqpoint{7.779848in}{1.541020in}}%
\pgfpathlineto{\pgfqpoint{7.779848in}{1.541020in}}%
\pgfpathclose%
\pgfusepath{stroke,fill}%
\end{pgfscope}%
\begin{pgfscope}%
\pgfsetrectcap%
\pgfsetroundjoin%
\pgfsetlinewidth{1.505625pt}%
\pgfsetstrokecolor{currentstroke1}%
\pgfsetdash{}{0pt}%
\pgfpathmoveto{\pgfqpoint{7.804848in}{2.578129in}}%
\pgfpathlineto{\pgfqpoint{7.929848in}{2.578129in}}%
\pgfpathlineto{\pgfqpoint{8.054848in}{2.578129in}}%
\pgfusepath{stroke}%
\end{pgfscope}%
\begin{pgfscope}%
\definecolor{textcolor}{rgb}{0.000000,0.000000,0.000000}%
\pgfsetstrokecolor{textcolor}%
\pgfsetfillcolor{textcolor}%
\pgftext[x=8.154848in,y=2.534379in,left,base]{\color{textcolor}{\rmfamily\fontsize{9.000000}{10.800000}\selectfont\catcode`\^=\active\def^{\ifmmode\sp\else\^{}\fi}\catcode`\%=\active\def%{\%}3}}%
\end{pgfscope}%
\begin{pgfscope}%
\pgfsetrectcap%
\pgfsetroundjoin%
\pgfsetlinewidth{1.505625pt}%
\pgfsetstrokecolor{currentstroke2}%
\pgfsetdash{}{0pt}%
\pgfpathmoveto{\pgfqpoint{7.804848in}{2.394657in}}%
\pgfpathlineto{\pgfqpoint{7.929848in}{2.394657in}}%
\pgfpathlineto{\pgfqpoint{8.054848in}{2.394657in}}%
\pgfusepath{stroke}%
\end{pgfscope}%
\begin{pgfscope}%
\definecolor{textcolor}{rgb}{0.000000,0.000000,0.000000}%
\pgfsetstrokecolor{textcolor}%
\pgfsetfillcolor{textcolor}%
\pgftext[x=8.154848in,y=2.350907in,left,base]{\color{textcolor}{\rmfamily\fontsize{9.000000}{10.800000}\selectfont\catcode`\^=\active\def^{\ifmmode\sp\else\^{}\fi}\catcode`\%=\active\def%{\%}4}}%
\end{pgfscope}%
\begin{pgfscope}%
\pgfsetrectcap%
\pgfsetroundjoin%
\pgfsetlinewidth{1.505625pt}%
\pgfsetstrokecolor{currentstroke3}%
\pgfsetdash{}{0pt}%
\pgfpathmoveto{\pgfqpoint{7.804848in}{2.211185in}}%
\pgfpathlineto{\pgfqpoint{7.929848in}{2.211185in}}%
\pgfpathlineto{\pgfqpoint{8.054848in}{2.211185in}}%
\pgfusepath{stroke}%
\end{pgfscope}%
\begin{pgfscope}%
\definecolor{textcolor}{rgb}{0.000000,0.000000,0.000000}%
\pgfsetstrokecolor{textcolor}%
\pgfsetfillcolor{textcolor}%
\pgftext[x=8.154848in,y=2.167435in,left,base]{\color{textcolor}{\rmfamily\fontsize{9.000000}{10.800000}\selectfont\catcode`\^=\active\def^{\ifmmode\sp\else\^{}\fi}\catcode`\%=\active\def%{\%}5}}%
\end{pgfscope}%
\begin{pgfscope}%
\pgfsetrectcap%
\pgfsetroundjoin%
\pgfsetlinewidth{1.505625pt}%
\pgfsetstrokecolor{currentstroke4}%
\pgfsetdash{}{0pt}%
\pgfpathmoveto{\pgfqpoint{7.804848in}{2.027714in}}%
\pgfpathlineto{\pgfqpoint{7.929848in}{2.027714in}}%
\pgfpathlineto{\pgfqpoint{8.054848in}{2.027714in}}%
\pgfusepath{stroke}%
\end{pgfscope}%
\begin{pgfscope}%
\definecolor{textcolor}{rgb}{0.000000,0.000000,0.000000}%
\pgfsetstrokecolor{textcolor}%
\pgfsetfillcolor{textcolor}%
\pgftext[x=8.154848in,y=1.983964in,left,base]{\color{textcolor}{\rmfamily\fontsize{9.000000}{10.800000}\selectfont\catcode`\^=\active\def^{\ifmmode\sp\else\^{}\fi}\catcode`\%=\active\def%{\%}6}}%
\end{pgfscope}%
\begin{pgfscope}%
\pgfsetrectcap%
\pgfsetroundjoin%
\pgfsetlinewidth{1.505625pt}%
\pgfsetstrokecolor{currentstroke5}%
\pgfsetdash{}{0pt}%
\pgfpathmoveto{\pgfqpoint{7.804848in}{1.844242in}}%
\pgfpathlineto{\pgfqpoint{7.929848in}{1.844242in}}%
\pgfpathlineto{\pgfqpoint{8.054848in}{1.844242in}}%
\pgfusepath{stroke}%
\end{pgfscope}%
\begin{pgfscope}%
\definecolor{textcolor}{rgb}{0.000000,0.000000,0.000000}%
\pgfsetstrokecolor{textcolor}%
\pgfsetfillcolor{textcolor}%
\pgftext[x=8.154848in,y=1.800492in,left,base]{\color{textcolor}{\rmfamily\fontsize{9.000000}{10.800000}\selectfont\catcode`\^=\active\def^{\ifmmode\sp\else\^{}\fi}\catcode`\%=\active\def%{\%}7}}%
\end{pgfscope}%
\begin{pgfscope}%
\pgfsetrectcap%
\pgfsetroundjoin%
\pgfsetlinewidth{1.505625pt}%
\pgfsetstrokecolor{currentstroke6}%
\pgfsetdash{}{0pt}%
\pgfpathmoveto{\pgfqpoint{7.804848in}{1.660771in}}%
\pgfpathlineto{\pgfqpoint{7.929848in}{1.660771in}}%
\pgfpathlineto{\pgfqpoint{8.054848in}{1.660771in}}%
\pgfusepath{stroke}%
\end{pgfscope}%
\begin{pgfscope}%
\definecolor{textcolor}{rgb}{0.000000,0.000000,0.000000}%
\pgfsetstrokecolor{textcolor}%
\pgfsetfillcolor{textcolor}%
\pgftext[x=8.154848in,y=1.617021in,left,base]{\color{textcolor}{\rmfamily\fontsize{9.000000}{10.800000}\selectfont\catcode`\^=\active\def^{\ifmmode\sp\else\^{}\fi}\catcode`\%=\active\def%{\%}8}}%
\end{pgfscope}%
\end{pgfpicture}%
\makeatother%
\endgroup%
}
	\caption[Mean runtime for graphs with no NAC-coloring]{
		Mean running time to find all NAC-colorings for graphs with no NAC-coloring for different subgraph sizes \( k \).}%
	\label{fig:graph_no_nac_coloring_first_runtime_subgraph_size}
\end{figure}%
\begin{figure}[thbp]
	\centering
	\scalebox{\BenchFigureScale}{%% Creator: Matplotlib, PGF backend
%%
%% To include the figure in your LaTeX document, write
%%   \input{<filename>.pgf}
%%
%% Make sure the required packages are loaded in your preamble
%%   \usepackage{pgf}
%%
%% Also ensure that all the required font packages are loaded; for instance,
%% the lmodern package is sometimes necessary when using math font.
%%   \usepackage{lmodern}
%%
%% Figures using additional raster images can only be included by \input if
%% they are in the same directory as the main LaTeX file. For loading figures
%% from other directories you can use the `import` package
%%   \usepackage{import}
%%
%% and then include the figures with
%%   \import{<path to file>}{<filename>.pgf}
%%
%% Matplotlib used the following preamble
%%   \def\mathdefault#1{#1}
%%   \everymath=\expandafter{\the\everymath\displaystyle}
%%   \IfFileExists{scrextend.sty}{
%%     \usepackage[fontsize=10.000000pt]{scrextend}
%%   }{
%%     \renewcommand{\normalsize}{\fontsize{10.000000}{12.000000}\selectfont}
%%     \normalsize
%%   }
%%   
%%   \ifdefined\pdftexversion\else  % non-pdftex case.
%%     \usepackage{fontspec}
%%     \setmainfont{DejaVuSans.ttf}[Path=\detokenize{/home/petr/Projects/PyRigi/.venv/lib/python3.12/site-packages/matplotlib/mpl-data/fonts/ttf/}]
%%     \setsansfont{DejaVuSans.ttf}[Path=\detokenize{/home/petr/Projects/PyRigi/.venv/lib/python3.12/site-packages/matplotlib/mpl-data/fonts/ttf/}]
%%     \setmonofont{DejaVuSansMono.ttf}[Path=\detokenize{/home/petr/Projects/PyRigi/.venv/lib/python3.12/site-packages/matplotlib/mpl-data/fonts/ttf/}]
%%   \fi
%%   \makeatletter\@ifpackageloaded{underscore}{}{\usepackage[strings]{underscore}}\makeatother
%%
\begingroup%
\makeatletter%
\begin{pgfpicture}%
\pgfpathrectangle{\pgfpointorigin}{\pgfqpoint{8.384376in}{2.841849in}}%
\pgfusepath{use as bounding box, clip}%
\begin{pgfscope}%
\pgfsetbuttcap%
\pgfsetmiterjoin%
\definecolor{currentfill}{rgb}{1.000000,1.000000,1.000000}%
\pgfsetfillcolor{currentfill}%
\pgfsetlinewidth{0.000000pt}%
\definecolor{currentstroke}{rgb}{1.000000,1.000000,1.000000}%
\pgfsetstrokecolor{currentstroke}%
\pgfsetdash{}{0pt}%
\pgfpathmoveto{\pgfqpoint{0.000000in}{0.000000in}}%
\pgfpathlineto{\pgfqpoint{8.384376in}{0.000000in}}%
\pgfpathlineto{\pgfqpoint{8.384376in}{2.841849in}}%
\pgfpathlineto{\pgfqpoint{0.000000in}{2.841849in}}%
\pgfpathlineto{\pgfqpoint{0.000000in}{0.000000in}}%
\pgfpathclose%
\pgfusepath{fill}%
\end{pgfscope}%
\begin{pgfscope}%
\pgfsetbuttcap%
\pgfsetmiterjoin%
\definecolor{currentfill}{rgb}{1.000000,1.000000,1.000000}%
\pgfsetfillcolor{currentfill}%
\pgfsetlinewidth{0.000000pt}%
\definecolor{currentstroke}{rgb}{0.000000,0.000000,0.000000}%
\pgfsetstrokecolor{currentstroke}%
\pgfsetstrokeopacity{0.000000}%
\pgfsetdash{}{0pt}%
\pgfpathmoveto{\pgfqpoint{0.588387in}{0.521603in}}%
\pgfpathlineto{\pgfqpoint{7.692348in}{0.521603in}}%
\pgfpathlineto{\pgfqpoint{7.692348in}{2.531888in}}%
\pgfpathlineto{\pgfqpoint{0.588387in}{2.531888in}}%
\pgfpathlineto{\pgfqpoint{0.588387in}{0.521603in}}%
\pgfpathclose%
\pgfusepath{fill}%
\end{pgfscope}%
\begin{pgfscope}%
\pgfsetbuttcap%
\pgfsetroundjoin%
\definecolor{currentfill}{rgb}{0.000000,0.000000,0.000000}%
\pgfsetfillcolor{currentfill}%
\pgfsetlinewidth{0.803000pt}%
\definecolor{currentstroke}{rgb}{0.000000,0.000000,0.000000}%
\pgfsetstrokecolor{currentstroke}%
\pgfsetdash{}{0pt}%
\pgfsys@defobject{currentmarker}{\pgfqpoint{0.000000in}{-0.048611in}}{\pgfqpoint{0.000000in}{0.000000in}}{%
\pgfpathmoveto{\pgfqpoint{0.000000in}{0.000000in}}%
\pgfpathlineto{\pgfqpoint{0.000000in}{-0.048611in}}%
\pgfusepath{stroke,fill}%
}%
\begin{pgfscope}%
\pgfsys@transformshift{1.241273in}{0.521603in}%
\pgfsys@useobject{currentmarker}{}%
\end{pgfscope}%
\end{pgfscope}%
\begin{pgfscope}%
\definecolor{textcolor}{rgb}{0.000000,0.000000,0.000000}%
\pgfsetstrokecolor{textcolor}%
\pgfsetfillcolor{textcolor}%
\pgftext[x=1.241273in,y=0.424381in,,top]{\color{textcolor}{\rmfamily\fontsize{10.000000}{12.000000}\selectfont\catcode`\^=\active\def^{\ifmmode\sp\else\^{}\fi}\catcode`\%=\active\def%{\%}$\mathdefault{20}$}}%
\end{pgfscope}%
\begin{pgfscope}%
\pgfsetbuttcap%
\pgfsetroundjoin%
\definecolor{currentfill}{rgb}{0.000000,0.000000,0.000000}%
\pgfsetfillcolor{currentfill}%
\pgfsetlinewidth{0.803000pt}%
\definecolor{currentstroke}{rgb}{0.000000,0.000000,0.000000}%
\pgfsetstrokecolor{currentstroke}%
\pgfsetdash{}{0pt}%
\pgfsys@defobject{currentmarker}{\pgfqpoint{0.000000in}{-0.048611in}}{\pgfqpoint{0.000000in}{0.000000in}}{%
\pgfpathmoveto{\pgfqpoint{0.000000in}{0.000000in}}%
\pgfpathlineto{\pgfqpoint{0.000000in}{-0.048611in}}%
\pgfusepath{stroke,fill}%
}%
\begin{pgfscope}%
\pgfsys@transformshift{2.184068in}{0.521603in}%
\pgfsys@useobject{currentmarker}{}%
\end{pgfscope}%
\end{pgfscope}%
\begin{pgfscope}%
\definecolor{textcolor}{rgb}{0.000000,0.000000,0.000000}%
\pgfsetstrokecolor{textcolor}%
\pgfsetfillcolor{textcolor}%
\pgftext[x=2.184068in,y=0.424381in,,top]{\color{textcolor}{\rmfamily\fontsize{10.000000}{12.000000}\selectfont\catcode`\^=\active\def^{\ifmmode\sp\else\^{}\fi}\catcode`\%=\active\def%{\%}$\mathdefault{40}$}}%
\end{pgfscope}%
\begin{pgfscope}%
\pgfsetbuttcap%
\pgfsetroundjoin%
\definecolor{currentfill}{rgb}{0.000000,0.000000,0.000000}%
\pgfsetfillcolor{currentfill}%
\pgfsetlinewidth{0.803000pt}%
\definecolor{currentstroke}{rgb}{0.000000,0.000000,0.000000}%
\pgfsetstrokecolor{currentstroke}%
\pgfsetdash{}{0pt}%
\pgfsys@defobject{currentmarker}{\pgfqpoint{0.000000in}{-0.048611in}}{\pgfqpoint{0.000000in}{0.000000in}}{%
\pgfpathmoveto{\pgfqpoint{0.000000in}{0.000000in}}%
\pgfpathlineto{\pgfqpoint{0.000000in}{-0.048611in}}%
\pgfusepath{stroke,fill}%
}%
\begin{pgfscope}%
\pgfsys@transformshift{3.126863in}{0.521603in}%
\pgfsys@useobject{currentmarker}{}%
\end{pgfscope}%
\end{pgfscope}%
\begin{pgfscope}%
\definecolor{textcolor}{rgb}{0.000000,0.000000,0.000000}%
\pgfsetstrokecolor{textcolor}%
\pgfsetfillcolor{textcolor}%
\pgftext[x=3.126863in,y=0.424381in,,top]{\color{textcolor}{\rmfamily\fontsize{10.000000}{12.000000}\selectfont\catcode`\^=\active\def^{\ifmmode\sp\else\^{}\fi}\catcode`\%=\active\def%{\%}$\mathdefault{60}$}}%
\end{pgfscope}%
\begin{pgfscope}%
\pgfsetbuttcap%
\pgfsetroundjoin%
\definecolor{currentfill}{rgb}{0.000000,0.000000,0.000000}%
\pgfsetfillcolor{currentfill}%
\pgfsetlinewidth{0.803000pt}%
\definecolor{currentstroke}{rgb}{0.000000,0.000000,0.000000}%
\pgfsetstrokecolor{currentstroke}%
\pgfsetdash{}{0pt}%
\pgfsys@defobject{currentmarker}{\pgfqpoint{0.000000in}{-0.048611in}}{\pgfqpoint{0.000000in}{0.000000in}}{%
\pgfpathmoveto{\pgfqpoint{0.000000in}{0.000000in}}%
\pgfpathlineto{\pgfqpoint{0.000000in}{-0.048611in}}%
\pgfusepath{stroke,fill}%
}%
\begin{pgfscope}%
\pgfsys@transformshift{4.069658in}{0.521603in}%
\pgfsys@useobject{currentmarker}{}%
\end{pgfscope}%
\end{pgfscope}%
\begin{pgfscope}%
\definecolor{textcolor}{rgb}{0.000000,0.000000,0.000000}%
\pgfsetstrokecolor{textcolor}%
\pgfsetfillcolor{textcolor}%
\pgftext[x=4.069658in,y=0.424381in,,top]{\color{textcolor}{\rmfamily\fontsize{10.000000}{12.000000}\selectfont\catcode`\^=\active\def^{\ifmmode\sp\else\^{}\fi}\catcode`\%=\active\def%{\%}$\mathdefault{80}$}}%
\end{pgfscope}%
\begin{pgfscope}%
\pgfsetbuttcap%
\pgfsetroundjoin%
\definecolor{currentfill}{rgb}{0.000000,0.000000,0.000000}%
\pgfsetfillcolor{currentfill}%
\pgfsetlinewidth{0.803000pt}%
\definecolor{currentstroke}{rgb}{0.000000,0.000000,0.000000}%
\pgfsetstrokecolor{currentstroke}%
\pgfsetdash{}{0pt}%
\pgfsys@defobject{currentmarker}{\pgfqpoint{0.000000in}{-0.048611in}}{\pgfqpoint{0.000000in}{0.000000in}}{%
\pgfpathmoveto{\pgfqpoint{0.000000in}{0.000000in}}%
\pgfpathlineto{\pgfqpoint{0.000000in}{-0.048611in}}%
\pgfusepath{stroke,fill}%
}%
\begin{pgfscope}%
\pgfsys@transformshift{5.012453in}{0.521603in}%
\pgfsys@useobject{currentmarker}{}%
\end{pgfscope}%
\end{pgfscope}%
\begin{pgfscope}%
\definecolor{textcolor}{rgb}{0.000000,0.000000,0.000000}%
\pgfsetstrokecolor{textcolor}%
\pgfsetfillcolor{textcolor}%
\pgftext[x=5.012453in,y=0.424381in,,top]{\color{textcolor}{\rmfamily\fontsize{10.000000}{12.000000}\selectfont\catcode`\^=\active\def^{\ifmmode\sp\else\^{}\fi}\catcode`\%=\active\def%{\%}$\mathdefault{100}$}}%
\end{pgfscope}%
\begin{pgfscope}%
\pgfsetbuttcap%
\pgfsetroundjoin%
\definecolor{currentfill}{rgb}{0.000000,0.000000,0.000000}%
\pgfsetfillcolor{currentfill}%
\pgfsetlinewidth{0.803000pt}%
\definecolor{currentstroke}{rgb}{0.000000,0.000000,0.000000}%
\pgfsetstrokecolor{currentstroke}%
\pgfsetdash{}{0pt}%
\pgfsys@defobject{currentmarker}{\pgfqpoint{0.000000in}{-0.048611in}}{\pgfqpoint{0.000000in}{0.000000in}}{%
\pgfpathmoveto{\pgfqpoint{0.000000in}{0.000000in}}%
\pgfpathlineto{\pgfqpoint{0.000000in}{-0.048611in}}%
\pgfusepath{stroke,fill}%
}%
\begin{pgfscope}%
\pgfsys@transformshift{5.955248in}{0.521603in}%
\pgfsys@useobject{currentmarker}{}%
\end{pgfscope}%
\end{pgfscope}%
\begin{pgfscope}%
\definecolor{textcolor}{rgb}{0.000000,0.000000,0.000000}%
\pgfsetstrokecolor{textcolor}%
\pgfsetfillcolor{textcolor}%
\pgftext[x=5.955248in,y=0.424381in,,top]{\color{textcolor}{\rmfamily\fontsize{10.000000}{12.000000}\selectfont\catcode`\^=\active\def^{\ifmmode\sp\else\^{}\fi}\catcode`\%=\active\def%{\%}$\mathdefault{120}$}}%
\end{pgfscope}%
\begin{pgfscope}%
\pgfsetbuttcap%
\pgfsetroundjoin%
\definecolor{currentfill}{rgb}{0.000000,0.000000,0.000000}%
\pgfsetfillcolor{currentfill}%
\pgfsetlinewidth{0.803000pt}%
\definecolor{currentstroke}{rgb}{0.000000,0.000000,0.000000}%
\pgfsetstrokecolor{currentstroke}%
\pgfsetdash{}{0pt}%
\pgfsys@defobject{currentmarker}{\pgfqpoint{0.000000in}{-0.048611in}}{\pgfqpoint{0.000000in}{0.000000in}}{%
\pgfpathmoveto{\pgfqpoint{0.000000in}{0.000000in}}%
\pgfpathlineto{\pgfqpoint{0.000000in}{-0.048611in}}%
\pgfusepath{stroke,fill}%
}%
\begin{pgfscope}%
\pgfsys@transformshift{6.898043in}{0.521603in}%
\pgfsys@useobject{currentmarker}{}%
\end{pgfscope}%
\end{pgfscope}%
\begin{pgfscope}%
\definecolor{textcolor}{rgb}{0.000000,0.000000,0.000000}%
\pgfsetstrokecolor{textcolor}%
\pgfsetfillcolor{textcolor}%
\pgftext[x=6.898043in,y=0.424381in,,top]{\color{textcolor}{\rmfamily\fontsize{10.000000}{12.000000}\selectfont\catcode`\^=\active\def^{\ifmmode\sp\else\^{}\fi}\catcode`\%=\active\def%{\%}$\mathdefault{140}$}}%
\end{pgfscope}%
\begin{pgfscope}%
\definecolor{textcolor}{rgb}{0.000000,0.000000,0.000000}%
\pgfsetstrokecolor{textcolor}%
\pgfsetfillcolor{textcolor}%
\pgftext[x=4.140367in,y=0.234413in,,top]{\color{textcolor}{\rmfamily\fontsize{10.000000}{12.000000}\selectfont\catcode`\^=\active\def^{\ifmmode\sp\else\^{}\fi}\catcode`\%=\active\def%{\%}Triangle components}}%
\end{pgfscope}%
\begin{pgfscope}%
\pgfsetbuttcap%
\pgfsetroundjoin%
\definecolor{currentfill}{rgb}{0.000000,0.000000,0.000000}%
\pgfsetfillcolor{currentfill}%
\pgfsetlinewidth{0.803000pt}%
\definecolor{currentstroke}{rgb}{0.000000,0.000000,0.000000}%
\pgfsetstrokecolor{currentstroke}%
\pgfsetdash{}{0pt}%
\pgfsys@defobject{currentmarker}{\pgfqpoint{-0.048611in}{0.000000in}}{\pgfqpoint{-0.000000in}{0.000000in}}{%
\pgfpathmoveto{\pgfqpoint{-0.000000in}{0.000000in}}%
\pgfpathlineto{\pgfqpoint{-0.048611in}{0.000000in}}%
\pgfusepath{stroke,fill}%
}%
\begin{pgfscope}%
\pgfsys@transformshift{0.588387in}{0.943926in}%
\pgfsys@useobject{currentmarker}{}%
\end{pgfscope}%
\end{pgfscope}%
\begin{pgfscope}%
\definecolor{textcolor}{rgb}{0.000000,0.000000,0.000000}%
\pgfsetstrokecolor{textcolor}%
\pgfsetfillcolor{textcolor}%
\pgftext[x=0.289968in, y=0.891164in, left, base]{\color{textcolor}{\rmfamily\fontsize{10.000000}{12.000000}\selectfont\catcode`\^=\active\def^{\ifmmode\sp\else\^{}\fi}\catcode`\%=\active\def%{\%}$\mathdefault{10^{2}}$}}%
\end{pgfscope}%
\begin{pgfscope}%
\pgfsetbuttcap%
\pgfsetroundjoin%
\definecolor{currentfill}{rgb}{0.000000,0.000000,0.000000}%
\pgfsetfillcolor{currentfill}%
\pgfsetlinewidth{0.803000pt}%
\definecolor{currentstroke}{rgb}{0.000000,0.000000,0.000000}%
\pgfsetstrokecolor{currentstroke}%
\pgfsetdash{}{0pt}%
\pgfsys@defobject{currentmarker}{\pgfqpoint{-0.048611in}{0.000000in}}{\pgfqpoint{-0.000000in}{0.000000in}}{%
\pgfpathmoveto{\pgfqpoint{-0.000000in}{0.000000in}}%
\pgfpathlineto{\pgfqpoint{-0.048611in}{0.000000in}}%
\pgfusepath{stroke,fill}%
}%
\begin{pgfscope}%
\pgfsys@transformshift{0.588387in}{1.872122in}%
\pgfsys@useobject{currentmarker}{}%
\end{pgfscope}%
\end{pgfscope}%
\begin{pgfscope}%
\definecolor{textcolor}{rgb}{0.000000,0.000000,0.000000}%
\pgfsetstrokecolor{textcolor}%
\pgfsetfillcolor{textcolor}%
\pgftext[x=0.289968in, y=1.819360in, left, base]{\color{textcolor}{\rmfamily\fontsize{10.000000}{12.000000}\selectfont\catcode`\^=\active\def^{\ifmmode\sp\else\^{}\fi}\catcode`\%=\active\def%{\%}$\mathdefault{10^{3}}$}}%
\end{pgfscope}%
\begin{pgfscope}%
\pgfsetbuttcap%
\pgfsetroundjoin%
\definecolor{currentfill}{rgb}{0.000000,0.000000,0.000000}%
\pgfsetfillcolor{currentfill}%
\pgfsetlinewidth{0.602250pt}%
\definecolor{currentstroke}{rgb}{0.000000,0.000000,0.000000}%
\pgfsetstrokecolor{currentstroke}%
\pgfsetdash{}{0pt}%
\pgfsys@defobject{currentmarker}{\pgfqpoint{-0.027778in}{0.000000in}}{\pgfqpoint{-0.000000in}{0.000000in}}{%
\pgfpathmoveto{\pgfqpoint{-0.000000in}{0.000000in}}%
\pgfpathlineto{\pgfqpoint{-0.027778in}{0.000000in}}%
\pgfusepath{stroke,fill}%
}%
\begin{pgfscope}%
\pgfsys@transformshift{0.588387in}{0.574559in}%
\pgfsys@useobject{currentmarker}{}%
\end{pgfscope}%
\end{pgfscope}%
\begin{pgfscope}%
\pgfsetbuttcap%
\pgfsetroundjoin%
\definecolor{currentfill}{rgb}{0.000000,0.000000,0.000000}%
\pgfsetfillcolor{currentfill}%
\pgfsetlinewidth{0.602250pt}%
\definecolor{currentstroke}{rgb}{0.000000,0.000000,0.000000}%
\pgfsetstrokecolor{currentstroke}%
\pgfsetdash{}{0pt}%
\pgfsys@defobject{currentmarker}{\pgfqpoint{-0.027778in}{0.000000in}}{\pgfqpoint{-0.000000in}{0.000000in}}{%
\pgfpathmoveto{\pgfqpoint{-0.000000in}{0.000000in}}%
\pgfpathlineto{\pgfqpoint{-0.027778in}{0.000000in}}%
\pgfusepath{stroke,fill}%
}%
\begin{pgfscope}%
\pgfsys@transformshift{0.588387in}{0.664511in}%
\pgfsys@useobject{currentmarker}{}%
\end{pgfscope}%
\end{pgfscope}%
\begin{pgfscope}%
\pgfsetbuttcap%
\pgfsetroundjoin%
\definecolor{currentfill}{rgb}{0.000000,0.000000,0.000000}%
\pgfsetfillcolor{currentfill}%
\pgfsetlinewidth{0.602250pt}%
\definecolor{currentstroke}{rgb}{0.000000,0.000000,0.000000}%
\pgfsetstrokecolor{currentstroke}%
\pgfsetdash{}{0pt}%
\pgfsys@defobject{currentmarker}{\pgfqpoint{-0.027778in}{0.000000in}}{\pgfqpoint{-0.000000in}{0.000000in}}{%
\pgfpathmoveto{\pgfqpoint{-0.000000in}{0.000000in}}%
\pgfpathlineto{\pgfqpoint{-0.027778in}{0.000000in}}%
\pgfusepath{stroke,fill}%
}%
\begin{pgfscope}%
\pgfsys@transformshift{0.588387in}{0.738007in}%
\pgfsys@useobject{currentmarker}{}%
\end{pgfscope}%
\end{pgfscope}%
\begin{pgfscope}%
\pgfsetbuttcap%
\pgfsetroundjoin%
\definecolor{currentfill}{rgb}{0.000000,0.000000,0.000000}%
\pgfsetfillcolor{currentfill}%
\pgfsetlinewidth{0.602250pt}%
\definecolor{currentstroke}{rgb}{0.000000,0.000000,0.000000}%
\pgfsetstrokecolor{currentstroke}%
\pgfsetdash{}{0pt}%
\pgfsys@defobject{currentmarker}{\pgfqpoint{-0.027778in}{0.000000in}}{\pgfqpoint{-0.000000in}{0.000000in}}{%
\pgfpathmoveto{\pgfqpoint{-0.000000in}{0.000000in}}%
\pgfpathlineto{\pgfqpoint{-0.027778in}{0.000000in}}%
\pgfusepath{stroke,fill}%
}%
\begin{pgfscope}%
\pgfsys@transformshift{0.588387in}{0.800146in}%
\pgfsys@useobject{currentmarker}{}%
\end{pgfscope}%
\end{pgfscope}%
\begin{pgfscope}%
\pgfsetbuttcap%
\pgfsetroundjoin%
\definecolor{currentfill}{rgb}{0.000000,0.000000,0.000000}%
\pgfsetfillcolor{currentfill}%
\pgfsetlinewidth{0.602250pt}%
\definecolor{currentstroke}{rgb}{0.000000,0.000000,0.000000}%
\pgfsetstrokecolor{currentstroke}%
\pgfsetdash{}{0pt}%
\pgfsys@defobject{currentmarker}{\pgfqpoint{-0.027778in}{0.000000in}}{\pgfqpoint{-0.000000in}{0.000000in}}{%
\pgfpathmoveto{\pgfqpoint{-0.000000in}{0.000000in}}%
\pgfpathlineto{\pgfqpoint{-0.027778in}{0.000000in}}%
\pgfusepath{stroke,fill}%
}%
\begin{pgfscope}%
\pgfsys@transformshift{0.588387in}{0.853974in}%
\pgfsys@useobject{currentmarker}{}%
\end{pgfscope}%
\end{pgfscope}%
\begin{pgfscope}%
\pgfsetbuttcap%
\pgfsetroundjoin%
\definecolor{currentfill}{rgb}{0.000000,0.000000,0.000000}%
\pgfsetfillcolor{currentfill}%
\pgfsetlinewidth{0.602250pt}%
\definecolor{currentstroke}{rgb}{0.000000,0.000000,0.000000}%
\pgfsetstrokecolor{currentstroke}%
\pgfsetdash{}{0pt}%
\pgfsys@defobject{currentmarker}{\pgfqpoint{-0.027778in}{0.000000in}}{\pgfqpoint{-0.000000in}{0.000000in}}{%
\pgfpathmoveto{\pgfqpoint{-0.000000in}{0.000000in}}%
\pgfpathlineto{\pgfqpoint{-0.027778in}{0.000000in}}%
\pgfusepath{stroke,fill}%
}%
\begin{pgfscope}%
\pgfsys@transformshift{0.588387in}{0.901454in}%
\pgfsys@useobject{currentmarker}{}%
\end{pgfscope}%
\end{pgfscope}%
\begin{pgfscope}%
\pgfsetbuttcap%
\pgfsetroundjoin%
\definecolor{currentfill}{rgb}{0.000000,0.000000,0.000000}%
\pgfsetfillcolor{currentfill}%
\pgfsetlinewidth{0.602250pt}%
\definecolor{currentstroke}{rgb}{0.000000,0.000000,0.000000}%
\pgfsetstrokecolor{currentstroke}%
\pgfsetdash{}{0pt}%
\pgfsys@defobject{currentmarker}{\pgfqpoint{-0.027778in}{0.000000in}}{\pgfqpoint{-0.000000in}{0.000000in}}{%
\pgfpathmoveto{\pgfqpoint{-0.000000in}{0.000000in}}%
\pgfpathlineto{\pgfqpoint{-0.027778in}{0.000000in}}%
\pgfusepath{stroke,fill}%
}%
\begin{pgfscope}%
\pgfsys@transformshift{0.588387in}{1.223341in}%
\pgfsys@useobject{currentmarker}{}%
\end{pgfscope}%
\end{pgfscope}%
\begin{pgfscope}%
\pgfsetbuttcap%
\pgfsetroundjoin%
\definecolor{currentfill}{rgb}{0.000000,0.000000,0.000000}%
\pgfsetfillcolor{currentfill}%
\pgfsetlinewidth{0.602250pt}%
\definecolor{currentstroke}{rgb}{0.000000,0.000000,0.000000}%
\pgfsetstrokecolor{currentstroke}%
\pgfsetdash{}{0pt}%
\pgfsys@defobject{currentmarker}{\pgfqpoint{-0.027778in}{0.000000in}}{\pgfqpoint{-0.000000in}{0.000000in}}{%
\pgfpathmoveto{\pgfqpoint{-0.000000in}{0.000000in}}%
\pgfpathlineto{\pgfqpoint{-0.027778in}{0.000000in}}%
\pgfusepath{stroke,fill}%
}%
\begin{pgfscope}%
\pgfsys@transformshift{0.588387in}{1.386788in}%
\pgfsys@useobject{currentmarker}{}%
\end{pgfscope}%
\end{pgfscope}%
\begin{pgfscope}%
\pgfsetbuttcap%
\pgfsetroundjoin%
\definecolor{currentfill}{rgb}{0.000000,0.000000,0.000000}%
\pgfsetfillcolor{currentfill}%
\pgfsetlinewidth{0.602250pt}%
\definecolor{currentstroke}{rgb}{0.000000,0.000000,0.000000}%
\pgfsetstrokecolor{currentstroke}%
\pgfsetdash{}{0pt}%
\pgfsys@defobject{currentmarker}{\pgfqpoint{-0.027778in}{0.000000in}}{\pgfqpoint{-0.000000in}{0.000000in}}{%
\pgfpathmoveto{\pgfqpoint{-0.000000in}{0.000000in}}%
\pgfpathlineto{\pgfqpoint{-0.027778in}{0.000000in}}%
\pgfusepath{stroke,fill}%
}%
\begin{pgfscope}%
\pgfsys@transformshift{0.588387in}{1.502755in}%
\pgfsys@useobject{currentmarker}{}%
\end{pgfscope}%
\end{pgfscope}%
\begin{pgfscope}%
\pgfsetbuttcap%
\pgfsetroundjoin%
\definecolor{currentfill}{rgb}{0.000000,0.000000,0.000000}%
\pgfsetfillcolor{currentfill}%
\pgfsetlinewidth{0.602250pt}%
\definecolor{currentstroke}{rgb}{0.000000,0.000000,0.000000}%
\pgfsetstrokecolor{currentstroke}%
\pgfsetdash{}{0pt}%
\pgfsys@defobject{currentmarker}{\pgfqpoint{-0.027778in}{0.000000in}}{\pgfqpoint{-0.000000in}{0.000000in}}{%
\pgfpathmoveto{\pgfqpoint{-0.000000in}{0.000000in}}%
\pgfpathlineto{\pgfqpoint{-0.027778in}{0.000000in}}%
\pgfusepath{stroke,fill}%
}%
\begin{pgfscope}%
\pgfsys@transformshift{0.588387in}{1.592707in}%
\pgfsys@useobject{currentmarker}{}%
\end{pgfscope}%
\end{pgfscope}%
\begin{pgfscope}%
\pgfsetbuttcap%
\pgfsetroundjoin%
\definecolor{currentfill}{rgb}{0.000000,0.000000,0.000000}%
\pgfsetfillcolor{currentfill}%
\pgfsetlinewidth{0.602250pt}%
\definecolor{currentstroke}{rgb}{0.000000,0.000000,0.000000}%
\pgfsetstrokecolor{currentstroke}%
\pgfsetdash{}{0pt}%
\pgfsys@defobject{currentmarker}{\pgfqpoint{-0.027778in}{0.000000in}}{\pgfqpoint{-0.000000in}{0.000000in}}{%
\pgfpathmoveto{\pgfqpoint{-0.000000in}{0.000000in}}%
\pgfpathlineto{\pgfqpoint{-0.027778in}{0.000000in}}%
\pgfusepath{stroke,fill}%
}%
\begin{pgfscope}%
\pgfsys@transformshift{0.588387in}{1.666203in}%
\pgfsys@useobject{currentmarker}{}%
\end{pgfscope}%
\end{pgfscope}%
\begin{pgfscope}%
\pgfsetbuttcap%
\pgfsetroundjoin%
\definecolor{currentfill}{rgb}{0.000000,0.000000,0.000000}%
\pgfsetfillcolor{currentfill}%
\pgfsetlinewidth{0.602250pt}%
\definecolor{currentstroke}{rgb}{0.000000,0.000000,0.000000}%
\pgfsetstrokecolor{currentstroke}%
\pgfsetdash{}{0pt}%
\pgfsys@defobject{currentmarker}{\pgfqpoint{-0.027778in}{0.000000in}}{\pgfqpoint{-0.000000in}{0.000000in}}{%
\pgfpathmoveto{\pgfqpoint{-0.000000in}{0.000000in}}%
\pgfpathlineto{\pgfqpoint{-0.027778in}{0.000000in}}%
\pgfusepath{stroke,fill}%
}%
\begin{pgfscope}%
\pgfsys@transformshift{0.588387in}{1.728342in}%
\pgfsys@useobject{currentmarker}{}%
\end{pgfscope}%
\end{pgfscope}%
\begin{pgfscope}%
\pgfsetbuttcap%
\pgfsetroundjoin%
\definecolor{currentfill}{rgb}{0.000000,0.000000,0.000000}%
\pgfsetfillcolor{currentfill}%
\pgfsetlinewidth{0.602250pt}%
\definecolor{currentstroke}{rgb}{0.000000,0.000000,0.000000}%
\pgfsetstrokecolor{currentstroke}%
\pgfsetdash{}{0pt}%
\pgfsys@defobject{currentmarker}{\pgfqpoint{-0.027778in}{0.000000in}}{\pgfqpoint{-0.000000in}{0.000000in}}{%
\pgfpathmoveto{\pgfqpoint{-0.000000in}{0.000000in}}%
\pgfpathlineto{\pgfqpoint{-0.027778in}{0.000000in}}%
\pgfusepath{stroke,fill}%
}%
\begin{pgfscope}%
\pgfsys@transformshift{0.588387in}{1.782170in}%
\pgfsys@useobject{currentmarker}{}%
\end{pgfscope}%
\end{pgfscope}%
\begin{pgfscope}%
\pgfsetbuttcap%
\pgfsetroundjoin%
\definecolor{currentfill}{rgb}{0.000000,0.000000,0.000000}%
\pgfsetfillcolor{currentfill}%
\pgfsetlinewidth{0.602250pt}%
\definecolor{currentstroke}{rgb}{0.000000,0.000000,0.000000}%
\pgfsetstrokecolor{currentstroke}%
\pgfsetdash{}{0pt}%
\pgfsys@defobject{currentmarker}{\pgfqpoint{-0.027778in}{0.000000in}}{\pgfqpoint{-0.000000in}{0.000000in}}{%
\pgfpathmoveto{\pgfqpoint{-0.000000in}{0.000000in}}%
\pgfpathlineto{\pgfqpoint{-0.027778in}{0.000000in}}%
\pgfusepath{stroke,fill}%
}%
\begin{pgfscope}%
\pgfsys@transformshift{0.588387in}{1.829650in}%
\pgfsys@useobject{currentmarker}{}%
\end{pgfscope}%
\end{pgfscope}%
\begin{pgfscope}%
\pgfsetbuttcap%
\pgfsetroundjoin%
\definecolor{currentfill}{rgb}{0.000000,0.000000,0.000000}%
\pgfsetfillcolor{currentfill}%
\pgfsetlinewidth{0.602250pt}%
\definecolor{currentstroke}{rgb}{0.000000,0.000000,0.000000}%
\pgfsetstrokecolor{currentstroke}%
\pgfsetdash{}{0pt}%
\pgfsys@defobject{currentmarker}{\pgfqpoint{-0.027778in}{0.000000in}}{\pgfqpoint{-0.000000in}{0.000000in}}{%
\pgfpathmoveto{\pgfqpoint{-0.000000in}{0.000000in}}%
\pgfpathlineto{\pgfqpoint{-0.027778in}{0.000000in}}%
\pgfusepath{stroke,fill}%
}%
\begin{pgfscope}%
\pgfsys@transformshift{0.588387in}{2.151536in}%
\pgfsys@useobject{currentmarker}{}%
\end{pgfscope}%
\end{pgfscope}%
\begin{pgfscope}%
\pgfsetbuttcap%
\pgfsetroundjoin%
\definecolor{currentfill}{rgb}{0.000000,0.000000,0.000000}%
\pgfsetfillcolor{currentfill}%
\pgfsetlinewidth{0.602250pt}%
\definecolor{currentstroke}{rgb}{0.000000,0.000000,0.000000}%
\pgfsetstrokecolor{currentstroke}%
\pgfsetdash{}{0pt}%
\pgfsys@defobject{currentmarker}{\pgfqpoint{-0.027778in}{0.000000in}}{\pgfqpoint{-0.000000in}{0.000000in}}{%
\pgfpathmoveto{\pgfqpoint{-0.000000in}{0.000000in}}%
\pgfpathlineto{\pgfqpoint{-0.027778in}{0.000000in}}%
\pgfusepath{stroke,fill}%
}%
\begin{pgfscope}%
\pgfsys@transformshift{0.588387in}{2.314984in}%
\pgfsys@useobject{currentmarker}{}%
\end{pgfscope}%
\end{pgfscope}%
\begin{pgfscope}%
\pgfsetbuttcap%
\pgfsetroundjoin%
\definecolor{currentfill}{rgb}{0.000000,0.000000,0.000000}%
\pgfsetfillcolor{currentfill}%
\pgfsetlinewidth{0.602250pt}%
\definecolor{currentstroke}{rgb}{0.000000,0.000000,0.000000}%
\pgfsetstrokecolor{currentstroke}%
\pgfsetdash{}{0pt}%
\pgfsys@defobject{currentmarker}{\pgfqpoint{-0.027778in}{0.000000in}}{\pgfqpoint{-0.000000in}{0.000000in}}{%
\pgfpathmoveto{\pgfqpoint{-0.000000in}{0.000000in}}%
\pgfpathlineto{\pgfqpoint{-0.027778in}{0.000000in}}%
\pgfusepath{stroke,fill}%
}%
\begin{pgfscope}%
\pgfsys@transformshift{0.588387in}{2.430951in}%
\pgfsys@useobject{currentmarker}{}%
\end{pgfscope}%
\end{pgfscope}%
\begin{pgfscope}%
\pgfsetbuttcap%
\pgfsetroundjoin%
\definecolor{currentfill}{rgb}{0.000000,0.000000,0.000000}%
\pgfsetfillcolor{currentfill}%
\pgfsetlinewidth{0.602250pt}%
\definecolor{currentstroke}{rgb}{0.000000,0.000000,0.000000}%
\pgfsetstrokecolor{currentstroke}%
\pgfsetdash{}{0pt}%
\pgfsys@defobject{currentmarker}{\pgfqpoint{-0.027778in}{0.000000in}}{\pgfqpoint{-0.000000in}{0.000000in}}{%
\pgfpathmoveto{\pgfqpoint{-0.000000in}{0.000000in}}%
\pgfpathlineto{\pgfqpoint{-0.027778in}{0.000000in}}%
\pgfusepath{stroke,fill}%
}%
\begin{pgfscope}%
\pgfsys@transformshift{0.588387in}{2.520903in}%
\pgfsys@useobject{currentmarker}{}%
\end{pgfscope}%
\end{pgfscope}%
\begin{pgfscope}%
\definecolor{textcolor}{rgb}{0.000000,0.000000,0.000000}%
\pgfsetstrokecolor{textcolor}%
\pgfsetfillcolor{textcolor}%
\pgftext[x=0.234413in,y=1.526746in,,bottom,rotate=90.000000]{\color{textcolor}{\rmfamily\fontsize{10.000000}{12.000000}\selectfont\catcode`\^=\active\def^{\ifmmode\sp\else\^{}\fi}\catcode`\%=\active\def%{\%}Checks [call]}}%
\end{pgfscope}%
\begin{pgfscope}%
\pgfpathrectangle{\pgfqpoint{0.588387in}{0.521603in}}{\pgfqpoint{7.103961in}{2.010285in}}%
\pgfusepath{clip}%
\pgfsetrectcap%
\pgfsetroundjoin%
\pgfsetlinewidth{1.505625pt}%
\pgfsetstrokecolor{currentstroke1}%
\pgfsetdash{}{0pt}%
\pgfpathmoveto{\pgfqpoint{0.911295in}{0.612980in}}%
\pgfpathlineto{\pgfqpoint{0.958434in}{0.710195in}}%
\pgfpathlineto{\pgfqpoint{1.005574in}{0.683015in}}%
\pgfpathlineto{\pgfqpoint{1.052714in}{0.711936in}}%
\pgfpathlineto{\pgfqpoint{1.099854in}{0.788257in}}%
\pgfpathlineto{\pgfqpoint{1.146993in}{0.764023in}}%
\pgfpathlineto{\pgfqpoint{1.194133in}{0.789079in}}%
\pgfpathlineto{\pgfqpoint{1.241273in}{0.854868in}}%
\pgfpathlineto{\pgfqpoint{1.288413in}{0.834403in}}%
\pgfpathlineto{\pgfqpoint{1.335552in}{0.854741in}}%
\pgfpathlineto{\pgfqpoint{1.382692in}{0.913934in}}%
\pgfpathlineto{\pgfqpoint{1.429832in}{0.896514in}}%
\pgfpathlineto{\pgfqpoint{1.476972in}{0.911891in}}%
\pgfpathlineto{\pgfqpoint{1.524111in}{0.961087in}}%
\pgfpathlineto{\pgfqpoint{1.571251in}{0.946997in}}%
\pgfpathlineto{\pgfqpoint{1.618391in}{0.961271in}}%
\pgfpathlineto{\pgfqpoint{1.665531in}{1.006685in}}%
\pgfpathlineto{\pgfqpoint{1.712670in}{0.990081in}}%
\pgfpathlineto{\pgfqpoint{1.759810in}{1.005201in}}%
\pgfpathlineto{\pgfqpoint{1.806950in}{1.047633in}}%
\pgfpathlineto{\pgfqpoint{1.854090in}{1.040583in}}%
\pgfpathlineto{\pgfqpoint{1.901229in}{1.047403in}}%
\pgfpathlineto{\pgfqpoint{1.948369in}{1.085711in}}%
\pgfpathlineto{\pgfqpoint{1.995509in}{1.072256in}}%
\pgfpathlineto{\pgfqpoint{2.042649in}{1.089346in}}%
\pgfpathlineto{\pgfqpoint{2.089788in}{1.120663in}}%
\pgfpathlineto{\pgfqpoint{2.136928in}{1.104752in}}%
\pgfpathlineto{\pgfqpoint{2.184068in}{1.125681in}}%
\pgfpathlineto{\pgfqpoint{2.231208in}{1.150650in}}%
\pgfpathlineto{\pgfqpoint{2.278347in}{1.133389in}}%
\pgfpathlineto{\pgfqpoint{2.325487in}{1.152457in}}%
\pgfpathlineto{\pgfqpoint{2.372627in}{1.182714in}}%
\pgfpathlineto{\pgfqpoint{2.419767in}{1.175458in}}%
\pgfpathlineto{\pgfqpoint{2.466906in}{1.183992in}}%
\pgfpathlineto{\pgfqpoint{2.514046in}{1.203333in}}%
\pgfpathlineto{\pgfqpoint{2.561186in}{1.201929in}}%
\pgfpathlineto{\pgfqpoint{2.608326in}{1.199597in}}%
\pgfpathlineto{\pgfqpoint{2.655465in}{1.238763in}}%
\pgfpathlineto{\pgfqpoint{2.702605in}{1.227234in}}%
\pgfpathlineto{\pgfqpoint{2.749745in}{1.236038in}}%
\pgfpathlineto{\pgfqpoint{2.796885in}{1.246729in}}%
\pgfpathlineto{\pgfqpoint{2.844024in}{1.258285in}}%
\pgfpathlineto{\pgfqpoint{2.891164in}{1.255570in}}%
\pgfpathlineto{\pgfqpoint{2.938304in}{1.277110in}}%
\pgfpathlineto{\pgfqpoint{2.985444in}{1.265812in}}%
\pgfpathlineto{\pgfqpoint{3.032583in}{1.278338in}}%
\pgfpathlineto{\pgfqpoint{3.079723in}{1.305148in}}%
\pgfpathlineto{\pgfqpoint{3.126863in}{1.294816in}}%
\pgfpathlineto{\pgfqpoint{3.174003in}{1.300867in}}%
\pgfpathlineto{\pgfqpoint{3.221142in}{1.327030in}}%
\pgfpathlineto{\pgfqpoint{3.268282in}{1.304325in}}%
\pgfpathlineto{\pgfqpoint{3.315422in}{1.328413in}}%
\pgfpathlineto{\pgfqpoint{3.362562in}{1.340650in}}%
\pgfpathlineto{\pgfqpoint{3.409701in}{1.332191in}}%
\pgfpathlineto{\pgfqpoint{3.456841in}{1.367724in}}%
\pgfpathlineto{\pgfqpoint{3.503981in}{1.358911in}}%
\pgfpathlineto{\pgfqpoint{3.551121in}{1.345657in}}%
\pgfpathlineto{\pgfqpoint{3.598260in}{1.366865in}}%
\pgfpathlineto{\pgfqpoint{3.645400in}{1.364126in}}%
\pgfpathlineto{\pgfqpoint{3.692540in}{1.383705in}}%
\pgfpathlineto{\pgfqpoint{3.739680in}{1.383821in}}%
\pgfpathlineto{\pgfqpoint{3.786819in}{1.395297in}}%
\pgfpathlineto{\pgfqpoint{3.833959in}{1.387459in}}%
\pgfpathlineto{\pgfqpoint{3.881099in}{1.402598in}}%
\pgfpathlineto{\pgfqpoint{3.928239in}{1.407414in}}%
\pgfpathlineto{\pgfqpoint{3.975378in}{1.411542in}}%
\pgfpathlineto{\pgfqpoint{4.022518in}{1.402383in}}%
\pgfpathlineto{\pgfqpoint{4.069658in}{1.436352in}}%
\pgfpathlineto{\pgfqpoint{4.116798in}{1.418226in}}%
\pgfpathlineto{\pgfqpoint{4.163937in}{1.481472in}}%
\pgfpathlineto{\pgfqpoint{4.211077in}{1.441664in}}%
\pgfpathlineto{\pgfqpoint{4.258217in}{1.423985in}}%
\pgfpathlineto{\pgfqpoint{4.305357in}{1.460283in}}%
\pgfpathlineto{\pgfqpoint{4.352496in}{1.454872in}}%
\pgfpathlineto{\pgfqpoint{4.399636in}{1.434864in}}%
\pgfpathlineto{\pgfqpoint{4.446776in}{1.488132in}}%
\pgfpathlineto{\pgfqpoint{4.493916in}{1.468046in}}%
\pgfpathlineto{\pgfqpoint{4.541055in}{1.484195in}}%
\pgfpathlineto{\pgfqpoint{4.588195in}{1.540412in}}%
\pgfpathlineto{\pgfqpoint{4.635335in}{1.505967in}}%
\pgfpathlineto{\pgfqpoint{4.682475in}{1.487348in}}%
\pgfpathlineto{\pgfqpoint{4.729614in}{1.505367in}}%
\pgfpathlineto{\pgfqpoint{4.776754in}{1.571605in}}%
\pgfpathlineto{\pgfqpoint{4.823894in}{1.534245in}}%
\pgfpathlineto{\pgfqpoint{4.871034in}{1.497685in}}%
\pgfpathlineto{\pgfqpoint{4.918173in}{1.532533in}}%
\pgfpathlineto{\pgfqpoint{4.965313in}{1.525292in}}%
\pgfpathlineto{\pgfqpoint{5.012453in}{1.552468in}}%
\pgfpathlineto{\pgfqpoint{5.059593in}{1.558510in}}%
\pgfpathlineto{\pgfqpoint{5.106732in}{1.502755in}}%
\pgfpathlineto{\pgfqpoint{5.153872in}{1.506766in}}%
\pgfpathlineto{\pgfqpoint{5.201012in}{1.543308in}}%
\pgfpathlineto{\pgfqpoint{5.248152in}{1.561889in}}%
\pgfpathlineto{\pgfqpoint{5.295291in}{1.541176in}}%
\pgfpathlineto{\pgfqpoint{5.342431in}{1.541176in}}%
\pgfpathlineto{\pgfqpoint{5.436711in}{1.595462in}}%
\pgfpathlineto{\pgfqpoint{5.530990in}{1.588073in}}%
\pgfpathlineto{\pgfqpoint{5.578130in}{1.573361in}}%
\pgfpathlineto{\pgfqpoint{5.625270in}{1.575579in}}%
\pgfpathlineto{\pgfqpoint{5.860968in}{1.626706in}}%
\pgfusepath{stroke}%
\end{pgfscope}%
\begin{pgfscope}%
\pgfpathrectangle{\pgfqpoint{0.588387in}{0.521603in}}{\pgfqpoint{7.103961in}{2.010285in}}%
\pgfusepath{clip}%
\pgfsetrectcap%
\pgfsetroundjoin%
\pgfsetlinewidth{1.505625pt}%
\pgfsetstrokecolor{currentstroke2}%
\pgfsetdash{}{0pt}%
\pgfpathmoveto{\pgfqpoint{0.911295in}{0.761873in}}%
\pgfpathlineto{\pgfqpoint{0.958434in}{0.891873in}}%
\pgfpathlineto{\pgfqpoint{1.005574in}{0.991297in}}%
\pgfpathlineto{\pgfqpoint{1.052714in}{0.855959in}}%
\pgfpathlineto{\pgfqpoint{1.099854in}{0.892437in}}%
\pgfpathlineto{\pgfqpoint{1.146993in}{0.989666in}}%
\pgfpathlineto{\pgfqpoint{1.194133in}{1.068546in}}%
\pgfpathlineto{\pgfqpoint{1.241273in}{0.960744in}}%
\pgfpathlineto{\pgfqpoint{1.288413in}{0.990080in}}%
\pgfpathlineto{\pgfqpoint{1.335552in}{1.067422in}}%
\pgfpathlineto{\pgfqpoint{1.382692in}{1.136676in}}%
\pgfpathlineto{\pgfqpoint{1.429832in}{1.047201in}}%
\pgfpathlineto{\pgfqpoint{1.476972in}{1.068520in}}%
\pgfpathlineto{\pgfqpoint{1.524111in}{1.134055in}}%
\pgfpathlineto{\pgfqpoint{1.571251in}{1.189823in}}%
\pgfpathlineto{\pgfqpoint{1.618391in}{1.113199in}}%
\pgfpathlineto{\pgfqpoint{1.665531in}{1.134351in}}%
\pgfpathlineto{\pgfqpoint{1.712670in}{1.190087in}}%
\pgfpathlineto{\pgfqpoint{1.759810in}{1.239261in}}%
\pgfpathlineto{\pgfqpoint{1.806950in}{1.175397in}}%
\pgfpathlineto{\pgfqpoint{1.854090in}{1.192868in}}%
\pgfpathlineto{\pgfqpoint{1.901229in}{1.242311in}}%
\pgfpathlineto{\pgfqpoint{1.948369in}{1.286485in}}%
\pgfpathlineto{\pgfqpoint{1.995509in}{1.225837in}}%
\pgfpathlineto{\pgfqpoint{2.042649in}{1.247206in}}%
\pgfpathlineto{\pgfqpoint{2.089788in}{1.287308in}}%
\pgfpathlineto{\pgfqpoint{2.136928in}{1.325580in}}%
\pgfpathlineto{\pgfqpoint{2.184068in}{1.279135in}}%
\pgfpathlineto{\pgfqpoint{2.231208in}{1.288398in}}%
\pgfpathlineto{\pgfqpoint{2.278347in}{1.322512in}}%
\pgfpathlineto{\pgfqpoint{2.325487in}{1.363554in}}%
\pgfpathlineto{\pgfqpoint{2.372627in}{1.318462in}}%
\pgfpathlineto{\pgfqpoint{2.419767in}{1.330689in}}%
\pgfpathlineto{\pgfqpoint{2.466906in}{1.368658in}}%
\pgfpathlineto{\pgfqpoint{2.514046in}{1.394485in}}%
\pgfpathlineto{\pgfqpoint{2.561186in}{1.355832in}}%
\pgfpathlineto{\pgfqpoint{2.608326in}{1.359394in}}%
\pgfpathlineto{\pgfqpoint{2.655465in}{1.400791in}}%
\pgfpathlineto{\pgfqpoint{2.702605in}{1.428290in}}%
\pgfpathlineto{\pgfqpoint{2.749745in}{1.387980in}}%
\pgfpathlineto{\pgfqpoint{2.796885in}{1.392476in}}%
\pgfpathlineto{\pgfqpoint{2.844024in}{1.427871in}}%
\pgfpathlineto{\pgfqpoint{2.891164in}{1.452720in}}%
\pgfpathlineto{\pgfqpoint{2.938304in}{1.417713in}}%
\pgfpathlineto{\pgfqpoint{2.985444in}{1.423985in}}%
\pgfpathlineto{\pgfqpoint{3.032583in}{1.461672in}}%
\pgfpathlineto{\pgfqpoint{3.079723in}{1.485649in}}%
\pgfpathlineto{\pgfqpoint{3.126863in}{1.451306in}}%
\pgfpathlineto{\pgfqpoint{3.174003in}{1.457841in}}%
\pgfpathlineto{\pgfqpoint{3.221142in}{1.487832in}}%
\pgfpathlineto{\pgfqpoint{3.268282in}{1.502570in}}%
\pgfpathlineto{\pgfqpoint{3.315422in}{1.474814in}}%
\pgfpathlineto{\pgfqpoint{3.362562in}{1.478963in}}%
\pgfpathlineto{\pgfqpoint{3.409701in}{1.509340in}}%
\pgfpathlineto{\pgfqpoint{3.456841in}{1.550551in}}%
\pgfpathlineto{\pgfqpoint{3.503981in}{1.501491in}}%
\pgfpathlineto{\pgfqpoint{3.551121in}{1.505993in}}%
\pgfpathlineto{\pgfqpoint{3.598260in}{1.535547in}}%
\pgfpathlineto{\pgfqpoint{3.645400in}{1.547123in}}%
\pgfpathlineto{\pgfqpoint{3.692540in}{1.521793in}}%
\pgfpathlineto{\pgfqpoint{3.739680in}{1.535298in}}%
\pgfpathlineto{\pgfqpoint{3.786819in}{1.556704in}}%
\pgfpathlineto{\pgfqpoint{3.833959in}{1.576734in}}%
\pgfpathlineto{\pgfqpoint{3.881099in}{1.553238in}}%
\pgfpathlineto{\pgfqpoint{3.928239in}{1.550736in}}%
\pgfpathlineto{\pgfqpoint{3.975378in}{1.580869in}}%
\pgfpathlineto{\pgfqpoint{4.022518in}{1.590355in}}%
\pgfpathlineto{\pgfqpoint{4.069658in}{1.584867in}}%
\pgfpathlineto{\pgfqpoint{4.116798in}{1.571845in}}%
\pgfpathlineto{\pgfqpoint{4.163937in}{1.638733in}}%
\pgfpathlineto{\pgfqpoint{4.211077in}{1.611281in}}%
\pgfpathlineto{\pgfqpoint{4.258217in}{1.591947in}}%
\pgfpathlineto{\pgfqpoint{4.305357in}{1.596319in}}%
\pgfpathlineto{\pgfqpoint{4.399636in}{1.634730in}}%
\pgfpathlineto{\pgfqpoint{4.446776in}{1.619851in}}%
\pgfpathlineto{\pgfqpoint{4.493916in}{1.618919in}}%
\pgfpathlineto{\pgfqpoint{4.588195in}{1.664960in}}%
\pgfpathlineto{\pgfqpoint{4.635335in}{1.627122in}}%
\pgfpathlineto{\pgfqpoint{4.682475in}{1.632744in}}%
\pgfpathlineto{\pgfqpoint{4.729614in}{1.663093in}}%
\pgfpathlineto{\pgfqpoint{4.776754in}{1.698988in}}%
\pgfpathlineto{\pgfqpoint{4.823894in}{1.647742in}}%
\pgfpathlineto{\pgfqpoint{4.871034in}{1.657214in}}%
\pgfpathlineto{\pgfqpoint{4.918173in}{1.675021in}}%
\pgfpathlineto{\pgfqpoint{4.965313in}{1.679471in}}%
\pgfpathlineto{\pgfqpoint{5.012453in}{1.681252in}}%
\pgfpathlineto{\pgfqpoint{5.059593in}{1.697653in}}%
\pgfpathlineto{\pgfqpoint{5.106732in}{1.677448in}}%
\pgfpathlineto{\pgfqpoint{5.153872in}{1.695368in}}%
\pgfpathlineto{\pgfqpoint{5.201012in}{1.687816in}}%
\pgfpathlineto{\pgfqpoint{5.248152in}{1.687582in}}%
\pgfpathlineto{\pgfqpoint{5.342431in}{1.709411in}}%
\pgfpathlineto{\pgfqpoint{5.389571in}{1.685129in}}%
\pgfpathlineto{\pgfqpoint{5.436711in}{1.705374in}}%
\pgfpathlineto{\pgfqpoint{5.483850in}{1.708832in}}%
\pgfpathlineto{\pgfqpoint{5.530990in}{1.743390in}}%
\pgfpathlineto{\pgfqpoint{5.578130in}{1.722829in}}%
\pgfpathlineto{\pgfqpoint{5.625270in}{1.717066in}}%
\pgfpathlineto{\pgfqpoint{5.672409in}{1.729290in}}%
\pgfpathlineto{\pgfqpoint{5.719549in}{1.735786in}}%
\pgfpathlineto{\pgfqpoint{5.766689in}{1.714383in}}%
\pgfpathlineto{\pgfqpoint{5.813829in}{1.740320in}}%
\pgfpathlineto{\pgfqpoint{5.860968in}{1.771361in}}%
\pgfpathlineto{\pgfqpoint{5.908108in}{1.747509in}}%
\pgfpathlineto{\pgfqpoint{5.955248in}{1.735647in}}%
\pgfpathlineto{\pgfqpoint{6.002388in}{1.733573in}}%
\pgfpathlineto{\pgfqpoint{6.049527in}{1.752916in}}%
\pgfpathlineto{\pgfqpoint{6.096667in}{1.762162in}}%
\pgfpathlineto{\pgfqpoint{6.143807in}{1.757495in}}%
\pgfpathlineto{\pgfqpoint{6.190947in}{1.755347in}}%
\pgfpathlineto{\pgfqpoint{6.238086in}{1.761270in}}%
\pgfpathlineto{\pgfqpoint{6.285226in}{1.773300in}}%
\pgfpathlineto{\pgfqpoint{6.332366in}{1.756252in}}%
\pgfpathlineto{\pgfqpoint{6.379506in}{1.763924in}}%
\pgfpathlineto{\pgfqpoint{6.426645in}{1.773634in}}%
\pgfpathlineto{\pgfqpoint{6.473785in}{1.789164in}}%
\pgfpathlineto{\pgfqpoint{6.520925in}{1.791993in}}%
\pgfpathlineto{\pgfqpoint{6.568065in}{1.806609in}}%
\pgfpathlineto{\pgfqpoint{6.615204in}{1.823366in}}%
\pgfpathlineto{\pgfqpoint{6.662344in}{1.796568in}}%
\pgfpathlineto{\pgfqpoint{6.709484in}{1.780928in}}%
\pgfpathlineto{\pgfqpoint{6.756624in}{1.788713in}}%
\pgfpathlineto{\pgfqpoint{6.850903in}{1.808312in}}%
\pgfpathlineto{\pgfqpoint{6.898043in}{1.794086in}}%
\pgfpathlineto{\pgfqpoint{6.992322in}{1.807934in}}%
\pgfpathlineto{\pgfqpoint{7.086602in}{1.805659in}}%
\pgfpathlineto{\pgfqpoint{7.133742in}{1.809444in}}%
\pgfpathlineto{\pgfqpoint{7.228021in}{1.831437in}}%
\pgfpathlineto{\pgfqpoint{7.275161in}{1.816909in}}%
\pgfpathlineto{\pgfqpoint{7.275161in}{1.816909in}}%
\pgfusepath{stroke}%
\end{pgfscope}%
\begin{pgfscope}%
\pgfpathrectangle{\pgfqpoint{0.588387in}{0.521603in}}{\pgfqpoint{7.103961in}{2.010285in}}%
\pgfusepath{clip}%
\pgfsetrectcap%
\pgfsetroundjoin%
\pgfsetlinewidth{1.505625pt}%
\pgfsetstrokecolor{currentstroke3}%
\pgfsetdash{}{0pt}%
\pgfpathmoveto{\pgfqpoint{0.911295in}{1.128642in}}%
\pgfpathlineto{\pgfqpoint{0.958434in}{1.320804in}}%
\pgfpathlineto{\pgfqpoint{1.005574in}{0.991288in}}%
\pgfpathlineto{\pgfqpoint{1.052714in}{1.045019in}}%
\pgfpathlineto{\pgfqpoint{1.099854in}{1.171100in}}%
\pgfpathlineto{\pgfqpoint{1.146993in}{1.268564in}}%
\pgfpathlineto{\pgfqpoint{1.194133in}{1.323932in}}%
\pgfpathlineto{\pgfqpoint{1.241273in}{1.134455in}}%
\pgfpathlineto{\pgfqpoint{1.288413in}{1.172942in}}%
\pgfpathlineto{\pgfqpoint{1.335552in}{1.268457in}}%
\pgfpathlineto{\pgfqpoint{1.382692in}{1.350604in}}%
\pgfpathlineto{\pgfqpoint{1.429832in}{1.415837in}}%
\pgfpathlineto{\pgfqpoint{1.476972in}{1.240020in}}%
\pgfpathlineto{\pgfqpoint{1.524111in}{1.268891in}}%
\pgfpathlineto{\pgfqpoint{1.571251in}{1.347142in}}%
\pgfpathlineto{\pgfqpoint{1.618391in}{1.411653in}}%
\pgfpathlineto{\pgfqpoint{1.665531in}{1.469355in}}%
\pgfpathlineto{\pgfqpoint{1.712670in}{1.322401in}}%
\pgfpathlineto{\pgfqpoint{1.759810in}{1.348566in}}%
\pgfpathlineto{\pgfqpoint{1.854090in}{1.479087in}}%
\pgfpathlineto{\pgfqpoint{1.901229in}{1.522307in}}%
\pgfpathlineto{\pgfqpoint{1.948369in}{1.400596in}}%
\pgfpathlineto{\pgfqpoint{1.995509in}{1.414724in}}%
\pgfpathlineto{\pgfqpoint{2.042649in}{1.479837in}}%
\pgfpathlineto{\pgfqpoint{2.089788in}{1.524427in}}%
\pgfpathlineto{\pgfqpoint{2.136928in}{1.567171in}}%
\pgfpathlineto{\pgfqpoint{2.184068in}{1.463824in}}%
\pgfpathlineto{\pgfqpoint{2.231208in}{1.477763in}}%
\pgfpathlineto{\pgfqpoint{2.278347in}{1.517494in}}%
\pgfpathlineto{\pgfqpoint{2.325487in}{1.575084in}}%
\pgfpathlineto{\pgfqpoint{2.372627in}{1.612907in}}%
\pgfpathlineto{\pgfqpoint{2.419767in}{1.521126in}}%
\pgfpathlineto{\pgfqpoint{2.466906in}{1.536060in}}%
\pgfpathlineto{\pgfqpoint{2.514046in}{1.571759in}}%
\pgfpathlineto{\pgfqpoint{2.561186in}{1.604672in}}%
\pgfpathlineto{\pgfqpoint{2.608326in}{1.635892in}}%
\pgfpathlineto{\pgfqpoint{2.655465in}{1.560511in}}%
\pgfpathlineto{\pgfqpoint{2.702605in}{1.570852in}}%
\pgfpathlineto{\pgfqpoint{2.749745in}{1.606475in}}%
\pgfpathlineto{\pgfqpoint{2.796885in}{1.635030in}}%
\pgfpathlineto{\pgfqpoint{2.844024in}{1.691664in}}%
\pgfpathlineto{\pgfqpoint{2.891164in}{1.590575in}}%
\pgfpathlineto{\pgfqpoint{2.938304in}{1.604290in}}%
\pgfpathlineto{\pgfqpoint{2.985444in}{1.636199in}}%
\pgfpathlineto{\pgfqpoint{3.032583in}{1.679706in}}%
\pgfpathlineto{\pgfqpoint{3.079723in}{1.709941in}}%
\pgfpathlineto{\pgfqpoint{3.126863in}{1.638454in}}%
\pgfpathlineto{\pgfqpoint{3.174003in}{1.646306in}}%
\pgfpathlineto{\pgfqpoint{3.221142in}{1.683244in}}%
\pgfpathlineto{\pgfqpoint{3.268282in}{1.700072in}}%
\pgfpathlineto{\pgfqpoint{3.315422in}{1.731871in}}%
\pgfpathlineto{\pgfqpoint{3.362562in}{1.660028in}}%
\pgfpathlineto{\pgfqpoint{3.409701in}{1.679664in}}%
\pgfpathlineto{\pgfqpoint{3.456841in}{1.729595in}}%
\pgfpathlineto{\pgfqpoint{3.503981in}{1.734567in}}%
\pgfpathlineto{\pgfqpoint{3.551121in}{1.755403in}}%
\pgfpathlineto{\pgfqpoint{3.598260in}{1.701347in}}%
\pgfpathlineto{\pgfqpoint{3.645400in}{1.700144in}}%
\pgfpathlineto{\pgfqpoint{3.692540in}{1.729099in}}%
\pgfpathlineto{\pgfqpoint{3.739680in}{1.763109in}}%
\pgfpathlineto{\pgfqpoint{3.786819in}{1.787633in}}%
\pgfpathlineto{\pgfqpoint{3.833959in}{1.726943in}}%
\pgfpathlineto{\pgfqpoint{3.928239in}{1.755529in}}%
\pgfpathlineto{\pgfqpoint{3.975378in}{1.794021in}}%
\pgfpathlineto{\pgfqpoint{4.022518in}{1.802637in}}%
\pgfpathlineto{\pgfqpoint{4.069658in}{1.766178in}}%
\pgfpathlineto{\pgfqpoint{4.116798in}{1.758255in}}%
\pgfpathlineto{\pgfqpoint{4.163937in}{1.832277in}}%
\pgfpathlineto{\pgfqpoint{4.211077in}{1.806525in}}%
\pgfpathlineto{\pgfqpoint{4.258217in}{1.829439in}}%
\pgfpathlineto{\pgfqpoint{4.305357in}{1.777048in}}%
\pgfpathlineto{\pgfqpoint{4.352496in}{1.788332in}}%
\pgfpathlineto{\pgfqpoint{4.399636in}{1.804450in}}%
\pgfpathlineto{\pgfqpoint{4.446776in}{1.847877in}}%
\pgfpathlineto{\pgfqpoint{4.493916in}{1.853150in}}%
\pgfpathlineto{\pgfqpoint{4.541055in}{1.815781in}}%
\pgfpathlineto{\pgfqpoint{4.588195in}{1.842211in}}%
\pgfpathlineto{\pgfqpoint{4.635335in}{1.828412in}}%
\pgfpathlineto{\pgfqpoint{4.682475in}{1.851723in}}%
\pgfpathlineto{\pgfqpoint{4.729614in}{1.898618in}}%
\pgfpathlineto{\pgfqpoint{4.776754in}{1.858104in}}%
\pgfpathlineto{\pgfqpoint{4.823894in}{1.837563in}}%
\pgfpathlineto{\pgfqpoint{4.871034in}{1.856848in}}%
\pgfpathlineto{\pgfqpoint{4.918173in}{1.899633in}}%
\pgfpathlineto{\pgfqpoint{4.965313in}{1.892250in}}%
\pgfpathlineto{\pgfqpoint{5.012453in}{1.869075in}}%
\pgfpathlineto{\pgfqpoint{5.059593in}{1.886944in}}%
\pgfpathlineto{\pgfqpoint{5.106732in}{1.863978in}}%
\pgfpathlineto{\pgfqpoint{5.201012in}{1.921513in}}%
\pgfpathlineto{\pgfqpoint{5.248152in}{1.877417in}}%
\pgfpathlineto{\pgfqpoint{5.295291in}{1.874651in}}%
\pgfpathlineto{\pgfqpoint{5.342431in}{1.887422in}}%
\pgfpathlineto{\pgfqpoint{5.389571in}{1.903643in}}%
\pgfpathlineto{\pgfqpoint{5.436711in}{1.950980in}}%
\pgfpathlineto{\pgfqpoint{5.483850in}{1.910142in}}%
\pgfpathlineto{\pgfqpoint{5.530990in}{1.901477in}}%
\pgfpathlineto{\pgfqpoint{5.578130in}{1.935553in}}%
\pgfpathlineto{\pgfqpoint{5.625270in}{1.939056in}}%
\pgfpathlineto{\pgfqpoint{5.672409in}{1.956422in}}%
\pgfpathlineto{\pgfqpoint{5.719549in}{1.901596in}}%
\pgfpathlineto{\pgfqpoint{5.766689in}{1.903891in}}%
\pgfpathlineto{\pgfqpoint{5.813829in}{1.957300in}}%
\pgfpathlineto{\pgfqpoint{5.860968in}{2.077993in}}%
\pgfpathlineto{\pgfqpoint{5.908108in}{1.951743in}}%
\pgfpathlineto{\pgfqpoint{5.955248in}{1.921790in}}%
\pgfpathlineto{\pgfqpoint{6.002388in}{1.919530in}}%
\pgfpathlineto{\pgfqpoint{6.049527in}{1.949776in}}%
\pgfpathlineto{\pgfqpoint{6.096667in}{1.951785in}}%
\pgfpathlineto{\pgfqpoint{6.143807in}{1.996636in}}%
\pgfpathlineto{\pgfqpoint{6.190947in}{1.948185in}}%
\pgfpathlineto{\pgfqpoint{6.238086in}{1.935480in}}%
\pgfpathlineto{\pgfqpoint{6.285226in}{1.960098in}}%
\pgfpathlineto{\pgfqpoint{6.332366in}{1.973340in}}%
\pgfpathlineto{\pgfqpoint{6.379506in}{1.982231in}}%
\pgfpathlineto{\pgfqpoint{6.426645in}{1.956807in}}%
\pgfpathlineto{\pgfqpoint{6.473785in}{1.959578in}}%
\pgfpathlineto{\pgfqpoint{6.520925in}{1.972499in}}%
\pgfpathlineto{\pgfqpoint{6.568065in}{2.044429in}}%
\pgfpathlineto{\pgfqpoint{6.615204in}{2.028562in}}%
\pgfpathlineto{\pgfqpoint{6.662344in}{1.963039in}}%
\pgfpathlineto{\pgfqpoint{6.709484in}{1.969388in}}%
\pgfpathlineto{\pgfqpoint{6.756624in}{1.984877in}}%
\pgfpathlineto{\pgfqpoint{6.803763in}{2.008989in}}%
\pgfpathlineto{\pgfqpoint{6.850903in}{2.009162in}}%
\pgfpathlineto{\pgfqpoint{6.898043in}{1.983766in}}%
\pgfpathlineto{\pgfqpoint{6.945183in}{1.984851in}}%
\pgfpathlineto{\pgfqpoint{7.039462in}{2.013049in}}%
\pgfpathlineto{\pgfqpoint{7.086602in}{2.040908in}}%
\pgfpathlineto{\pgfqpoint{7.133742in}{1.991301in}}%
\pgfpathlineto{\pgfqpoint{7.180881in}{1.993694in}}%
\pgfpathlineto{\pgfqpoint{7.228021in}{2.014609in}}%
\pgfpathlineto{\pgfqpoint{7.322301in}{2.040908in}}%
\pgfpathlineto{\pgfqpoint{7.369440in}{2.023567in}}%
\pgfpathlineto{\pgfqpoint{7.369440in}{2.023567in}}%
\pgfusepath{stroke}%
\end{pgfscope}%
\begin{pgfscope}%
\pgfpathrectangle{\pgfqpoint{0.588387in}{0.521603in}}{\pgfqpoint{7.103961in}{2.010285in}}%
\pgfusepath{clip}%
\pgfsetrectcap%
\pgfsetroundjoin%
\pgfsetlinewidth{1.505625pt}%
\pgfsetstrokecolor{currentstroke4}%
\pgfsetdash{}{0pt}%
\pgfpathmoveto{\pgfqpoint{0.911295in}{1.128864in}}%
\pgfpathlineto{\pgfqpoint{0.958434in}{1.321120in}}%
\pgfpathlineto{\pgfqpoint{1.005574in}{1.413162in}}%
\pgfpathlineto{\pgfqpoint{1.052714in}{1.602918in}}%
\pgfpathlineto{\pgfqpoint{1.099854in}{1.689937in}}%
\pgfpathlineto{\pgfqpoint{1.146993in}{1.268845in}}%
\pgfpathlineto{\pgfqpoint{1.194133in}{1.323318in}}%
\pgfpathlineto{\pgfqpoint{1.241273in}{1.451086in}}%
\pgfpathlineto{\pgfqpoint{1.288413in}{1.547738in}}%
\pgfpathlineto{\pgfqpoint{1.335552in}{1.600203in}}%
\pgfpathlineto{\pgfqpoint{1.382692in}{1.732210in}}%
\pgfpathlineto{\pgfqpoint{1.429832in}{1.414552in}}%
\pgfpathlineto{\pgfqpoint{1.476972in}{1.450873in}}%
\pgfpathlineto{\pgfqpoint{1.524111in}{1.548118in}}%
\pgfpathlineto{\pgfqpoint{1.571251in}{1.626706in}}%
\pgfpathlineto{\pgfqpoint{1.618391in}{1.692003in}}%
\pgfpathlineto{\pgfqpoint{1.665531in}{1.731222in}}%
\pgfpathlineto{\pgfqpoint{1.712670in}{1.518151in}}%
\pgfpathlineto{\pgfqpoint{1.759810in}{1.549012in}}%
\pgfpathlineto{\pgfqpoint{1.806950in}{1.627678in}}%
\pgfpathlineto{\pgfqpoint{1.854090in}{1.697867in}}%
\pgfpathlineto{\pgfqpoint{1.901229in}{1.749138in}}%
\pgfpathlineto{\pgfqpoint{1.948369in}{1.802442in}}%
\pgfpathlineto{\pgfqpoint{1.995509in}{1.602813in}}%
\pgfpathlineto{\pgfqpoint{2.042649in}{1.635440in}}%
\pgfpathlineto{\pgfqpoint{2.089788in}{1.699462in}}%
\pgfpathlineto{\pgfqpoint{2.136928in}{1.757616in}}%
\pgfpathlineto{\pgfqpoint{2.184068in}{1.807705in}}%
\pgfpathlineto{\pgfqpoint{2.231208in}{1.846072in}}%
\pgfpathlineto{\pgfqpoint{2.278347in}{1.670634in}}%
\pgfpathlineto{\pgfqpoint{2.325487in}{1.703934in}}%
\pgfpathlineto{\pgfqpoint{2.372627in}{1.765098in}}%
\pgfpathlineto{\pgfqpoint{2.419767in}{1.810817in}}%
\pgfpathlineto{\pgfqpoint{2.466906in}{1.854610in}}%
\pgfpathlineto{\pgfqpoint{2.514046in}{1.884210in}}%
\pgfpathlineto{\pgfqpoint{2.561186in}{1.738410in}}%
\pgfpathlineto{\pgfqpoint{2.608326in}{1.752096in}}%
\pgfpathlineto{\pgfqpoint{2.655465in}{1.812685in}}%
\pgfpathlineto{\pgfqpoint{2.702605in}{1.852421in}}%
\pgfpathlineto{\pgfqpoint{2.749745in}{1.890508in}}%
\pgfpathlineto{\pgfqpoint{2.796885in}{1.914668in}}%
\pgfpathlineto{\pgfqpoint{2.844024in}{1.801735in}}%
\pgfpathlineto{\pgfqpoint{2.891164in}{1.807868in}}%
\pgfpathlineto{\pgfqpoint{2.938304in}{1.848532in}}%
\pgfpathlineto{\pgfqpoint{2.985444in}{1.885298in}}%
\pgfpathlineto{\pgfqpoint{3.032583in}{1.931978in}}%
\pgfpathlineto{\pgfqpoint{3.079723in}{1.963877in}}%
\pgfpathlineto{\pgfqpoint{3.126863in}{1.870524in}}%
\pgfpathlineto{\pgfqpoint{3.174003in}{1.856991in}}%
\pgfpathlineto{\pgfqpoint{3.221142in}{1.917846in}}%
\pgfpathlineto{\pgfqpoint{3.268282in}{1.911975in}}%
\pgfpathlineto{\pgfqpoint{3.315422in}{1.958796in}}%
\pgfpathlineto{\pgfqpoint{3.362562in}{1.976725in}}%
\pgfpathlineto{\pgfqpoint{3.409701in}{1.879032in}}%
\pgfpathlineto{\pgfqpoint{3.456841in}{1.939133in}}%
\pgfpathlineto{\pgfqpoint{3.503981in}{1.925966in}}%
\pgfpathlineto{\pgfqpoint{3.551121in}{1.944309in}}%
\pgfpathlineto{\pgfqpoint{3.598260in}{1.994169in}}%
\pgfpathlineto{\pgfqpoint{3.645400in}{2.002143in}}%
\pgfpathlineto{\pgfqpoint{3.692540in}{1.914398in}}%
\pgfpathlineto{\pgfqpoint{3.739680in}{1.932777in}}%
\pgfpathlineto{\pgfqpoint{3.786819in}{1.963993in}}%
\pgfpathlineto{\pgfqpoint{3.833959in}{1.988054in}}%
\pgfpathlineto{\pgfqpoint{3.881099in}{2.031744in}}%
\pgfpathlineto{\pgfqpoint{3.928239in}{2.129945in}}%
\pgfpathlineto{\pgfqpoint{3.975378in}{1.959452in}}%
\pgfpathlineto{\pgfqpoint{4.022518in}{1.949108in}}%
\pgfpathlineto{\pgfqpoint{4.069658in}{2.013309in}}%
\pgfpathlineto{\pgfqpoint{4.116798in}{2.009645in}}%
\pgfpathlineto{\pgfqpoint{4.163937in}{2.145124in}}%
\pgfpathlineto{\pgfqpoint{4.211077in}{2.061081in}}%
\pgfpathlineto{\pgfqpoint{4.258217in}{1.957272in}}%
\pgfpathlineto{\pgfqpoint{4.305357in}{1.974562in}}%
\pgfpathlineto{\pgfqpoint{4.352496in}{2.027640in}}%
\pgfpathlineto{\pgfqpoint{4.399636in}{2.015552in}}%
\pgfpathlineto{\pgfqpoint{4.446776in}{2.085654in}}%
\pgfpathlineto{\pgfqpoint{4.493916in}{2.091551in}}%
\pgfpathlineto{\pgfqpoint{4.541055in}{2.027209in}}%
\pgfpathlineto{\pgfqpoint{4.588195in}{2.116099in}}%
\pgfpathlineto{\pgfqpoint{4.635335in}{2.060576in}}%
\pgfpathlineto{\pgfqpoint{4.682475in}{2.074722in}}%
\pgfpathlineto{\pgfqpoint{4.729614in}{2.113438in}}%
\pgfpathlineto{\pgfqpoint{4.776754in}{2.197400in}}%
\pgfpathlineto{\pgfqpoint{4.823894in}{2.062801in}}%
\pgfpathlineto{\pgfqpoint{4.871034in}{2.046178in}}%
\pgfpathlineto{\pgfqpoint{4.918173in}{2.111270in}}%
\pgfpathlineto{\pgfqpoint{4.965313in}{2.093906in}}%
\pgfpathlineto{\pgfqpoint{5.012453in}{2.179892in}}%
\pgfpathlineto{\pgfqpoint{5.059593in}{2.182163in}}%
\pgfpathlineto{\pgfqpoint{5.106732in}{2.047745in}}%
\pgfpathlineto{\pgfqpoint{5.153872in}{2.063596in}}%
\pgfpathlineto{\pgfqpoint{5.201012in}{2.103894in}}%
\pgfpathlineto{\pgfqpoint{5.248152in}{2.173628in}}%
\pgfpathlineto{\pgfqpoint{5.295291in}{2.151416in}}%
\pgfpathlineto{\pgfqpoint{5.342431in}{2.159251in}}%
\pgfpathlineto{\pgfqpoint{5.436711in}{2.187261in}}%
\pgfpathlineto{\pgfqpoint{5.483850in}{2.297070in}}%
\pgfpathlineto{\pgfqpoint{5.530990in}{2.137905in}}%
\pgfpathlineto{\pgfqpoint{5.578130in}{2.245244in}}%
\pgfpathlineto{\pgfqpoint{5.625270in}{2.199464in}}%
\pgfpathlineto{\pgfqpoint{5.860968in}{2.402469in}}%
\pgfpathlineto{\pgfqpoint{6.049527in}{2.184143in}}%
\pgfpathlineto{\pgfqpoint{6.379506in}{2.182187in}}%
\pgfpathlineto{\pgfqpoint{6.568065in}{2.307327in}}%
\pgfusepath{stroke}%
\end{pgfscope}%
\begin{pgfscope}%
\pgfpathrectangle{\pgfqpoint{0.588387in}{0.521603in}}{\pgfqpoint{7.103961in}{2.010285in}}%
\pgfusepath{clip}%
\pgfsetrectcap%
\pgfsetroundjoin%
\pgfsetlinewidth{1.505625pt}%
\pgfsetstrokecolor{currentstroke5}%
\pgfsetdash{}{0pt}%
\pgfpathmoveto{\pgfqpoint{0.911295in}{2.440512in}}%
\pgfpathlineto{\pgfqpoint{0.958434in}{1.320808in}}%
\pgfpathlineto{\pgfqpoint{1.005574in}{1.414116in}}%
\pgfpathlineto{\pgfqpoint{1.052714in}{1.603755in}}%
\pgfpathlineto{\pgfqpoint{1.099854in}{1.689318in}}%
\pgfpathlineto{\pgfqpoint{1.146993in}{1.878868in}}%
\pgfpathlineto{\pgfqpoint{1.194133in}{1.971825in}}%
\pgfpathlineto{\pgfqpoint{1.241273in}{2.157907in}}%
\pgfpathlineto{\pgfqpoint{1.288413in}{1.548677in}}%
\pgfpathlineto{\pgfqpoint{1.335552in}{1.599235in}}%
\pgfpathlineto{\pgfqpoint{1.382692in}{1.734239in}}%
\pgfpathlineto{\pgfqpoint{1.429832in}{1.824410in}}%
\pgfpathlineto{\pgfqpoint{1.476972in}{1.877562in}}%
\pgfpathlineto{\pgfqpoint{1.524111in}{2.000288in}}%
\pgfpathlineto{\pgfqpoint{1.571251in}{2.106634in}}%
\pgfpathlineto{\pgfqpoint{1.618391in}{1.691745in}}%
\pgfpathlineto{\pgfqpoint{1.665531in}{1.732501in}}%
\pgfpathlineto{\pgfqpoint{1.712670in}{1.822806in}}%
\pgfpathlineto{\pgfqpoint{1.759810in}{1.901890in}}%
\pgfpathlineto{\pgfqpoint{1.806950in}{1.968749in}}%
\pgfpathlineto{\pgfqpoint{1.854090in}{2.021250in}}%
\pgfpathlineto{\pgfqpoint{1.901229in}{2.113111in}}%
\pgfpathlineto{\pgfqpoint{1.948369in}{1.817114in}}%
\pgfpathlineto{\pgfqpoint{1.995509in}{1.828737in}}%
\pgfpathlineto{\pgfqpoint{2.042649in}{1.934651in}}%
\pgfpathlineto{\pgfqpoint{2.089788in}{1.979989in}}%
\pgfpathlineto{\pgfqpoint{2.136928in}{2.035226in}}%
\pgfpathlineto{\pgfqpoint{2.184068in}{2.113769in}}%
\pgfpathlineto{\pgfqpoint{2.231208in}{2.109217in}}%
\pgfpathlineto{\pgfqpoint{2.278347in}{1.874557in}}%
\pgfpathlineto{\pgfqpoint{2.325487in}{1.930995in}}%
\pgfpathlineto{\pgfqpoint{2.372627in}{2.023844in}}%
\pgfpathlineto{\pgfqpoint{2.419767in}{2.183506in}}%
\pgfpathlineto{\pgfqpoint{2.466906in}{2.119126in}}%
\pgfpathlineto{\pgfqpoint{2.514046in}{2.153045in}}%
\pgfpathlineto{\pgfqpoint{2.561186in}{2.149437in}}%
\pgfpathlineto{\pgfqpoint{2.608326in}{1.934952in}}%
\pgfpathlineto{\pgfqpoint{2.655465in}{2.000452in}}%
\pgfpathlineto{\pgfqpoint{2.702605in}{2.067145in}}%
\pgfpathlineto{\pgfqpoint{2.749745in}{2.120022in}}%
\pgfpathlineto{\pgfqpoint{2.796885in}{2.113962in}}%
\pgfpathlineto{\pgfqpoint{2.844024in}{2.307798in}}%
\pgfpathlineto{\pgfqpoint{2.891164in}{2.206904in}}%
\pgfpathlineto{\pgfqpoint{2.938304in}{2.027790in}}%
\pgfpathlineto{\pgfqpoint{2.985444in}{2.008907in}}%
\pgfpathlineto{\pgfqpoint{3.032583in}{2.087500in}}%
\pgfpathlineto{\pgfqpoint{3.126863in}{2.309845in}}%
\pgfpathlineto{\pgfqpoint{3.174003in}{2.196860in}}%
\pgfpathlineto{\pgfqpoint{3.221142in}{2.301318in}}%
\pgfpathlineto{\pgfqpoint{3.315422in}{2.160111in}}%
\pgfpathlineto{\pgfqpoint{3.362562in}{2.114404in}}%
\pgfpathlineto{\pgfqpoint{3.409701in}{2.140227in}}%
\pgfpathlineto{\pgfqpoint{3.456841in}{2.352915in}}%
\pgfpathlineto{\pgfqpoint{3.503981in}{2.213446in}}%
\pgfpathlineto{\pgfqpoint{3.598260in}{2.090530in}}%
\pgfpathlineto{\pgfqpoint{3.692540in}{2.146262in}}%
\pgfpathlineto{\pgfqpoint{3.833959in}{2.205767in}}%
\pgfpathlineto{\pgfqpoint{4.069658in}{2.271765in}}%
\pgfusepath{stroke}%
\end{pgfscope}%
\begin{pgfscope}%
\pgfsetrectcap%
\pgfsetmiterjoin%
\pgfsetlinewidth{0.803000pt}%
\definecolor{currentstroke}{rgb}{0.000000,0.000000,0.000000}%
\pgfsetstrokecolor{currentstroke}%
\pgfsetdash{}{0pt}%
\pgfpathmoveto{\pgfqpoint{0.588387in}{0.521603in}}%
\pgfpathlineto{\pgfqpoint{0.588387in}{2.531888in}}%
\pgfusepath{stroke}%
\end{pgfscope}%
\begin{pgfscope}%
\pgfsetrectcap%
\pgfsetmiterjoin%
\pgfsetlinewidth{0.803000pt}%
\definecolor{currentstroke}{rgb}{0.000000,0.000000,0.000000}%
\pgfsetstrokecolor{currentstroke}%
\pgfsetdash{}{0pt}%
\pgfpathmoveto{\pgfqpoint{7.692348in}{0.521603in}}%
\pgfpathlineto{\pgfqpoint{7.692348in}{2.531888in}}%
\pgfusepath{stroke}%
\end{pgfscope}%
\begin{pgfscope}%
\pgfsetrectcap%
\pgfsetmiterjoin%
\pgfsetlinewidth{0.803000pt}%
\definecolor{currentstroke}{rgb}{0.000000,0.000000,0.000000}%
\pgfsetstrokecolor{currentstroke}%
\pgfsetdash{}{0pt}%
\pgfpathmoveto{\pgfqpoint{0.588387in}{0.521603in}}%
\pgfpathlineto{\pgfqpoint{7.692348in}{0.521603in}}%
\pgfusepath{stroke}%
\end{pgfscope}%
\begin{pgfscope}%
\pgfsetrectcap%
\pgfsetmiterjoin%
\pgfsetlinewidth{0.803000pt}%
\definecolor{currentstroke}{rgb}{0.000000,0.000000,0.000000}%
\pgfsetstrokecolor{currentstroke}%
\pgfsetdash{}{0pt}%
\pgfpathmoveto{\pgfqpoint{0.588387in}{2.531888in}}%
\pgfpathlineto{\pgfqpoint{7.692348in}{2.531888in}}%
\pgfusepath{stroke}%
\end{pgfscope}%
\begin{pgfscope}%
\definecolor{textcolor}{rgb}{0.000000,0.000000,0.000000}%
\pgfsetstrokecolor{textcolor}%
\pgfsetfillcolor{textcolor}%
\pgftext[x=4.140367in,y=2.615222in,,base]{\color{textcolor}{\rmfamily\fontsize{12.000000}{14.400000}\selectfont\catcode`\^=\active\def^{\ifmmode\sp\else\^{}\fi}\catcode`\%=\active\def%{\%}Mean}}%
\end{pgfscope}%
\begin{pgfscope}%
\pgfsetbuttcap%
\pgfsetmiterjoin%
\definecolor{currentfill}{rgb}{1.000000,1.000000,1.000000}%
\pgfsetfillcolor{currentfill}%
\pgfsetfillopacity{0.800000}%
\pgfsetlinewidth{1.003750pt}%
\definecolor{currentstroke}{rgb}{0.800000,0.800000,0.800000}%
\pgfsetstrokecolor{currentstroke}%
\pgfsetstrokeopacity{0.800000}%
\pgfsetdash{}{0pt}%
\pgfpathmoveto{\pgfqpoint{7.779848in}{1.514531in}}%
\pgfpathlineto{\pgfqpoint{8.259376in}{1.514531in}}%
\pgfpathquadraticcurveto{\pgfqpoint{8.284376in}{1.514531in}}{\pgfqpoint{8.284376in}{1.539531in}}%
\pgfpathlineto{\pgfqpoint{8.284376in}{2.444388in}}%
\pgfpathquadraticcurveto{\pgfqpoint{8.284376in}{2.469388in}}{\pgfqpoint{8.259376in}{2.469388in}}%
\pgfpathlineto{\pgfqpoint{7.779848in}{2.469388in}}%
\pgfpathquadraticcurveto{\pgfqpoint{7.754848in}{2.469388in}}{\pgfqpoint{7.754848in}{2.444388in}}%
\pgfpathlineto{\pgfqpoint{7.754848in}{1.539531in}}%
\pgfpathquadraticcurveto{\pgfqpoint{7.754848in}{1.514531in}}{\pgfqpoint{7.779848in}{1.514531in}}%
\pgfpathlineto{\pgfqpoint{7.779848in}{1.514531in}}%
\pgfpathclose%
\pgfusepath{stroke,fill}%
\end{pgfscope}%
\begin{pgfscope}%
\pgfsetrectcap%
\pgfsetroundjoin%
\pgfsetlinewidth{1.505625pt}%
\pgfsetstrokecolor{currentstroke1}%
\pgfsetdash{}{0pt}%
\pgfpathmoveto{\pgfqpoint{7.804848in}{2.368168in}}%
\pgfpathlineto{\pgfqpoint{7.929848in}{2.368168in}}%
\pgfpathlineto{\pgfqpoint{8.054848in}{2.368168in}}%
\pgfusepath{stroke}%
\end{pgfscope}%
\begin{pgfscope}%
\definecolor{textcolor}{rgb}{0.000000,0.000000,0.000000}%
\pgfsetstrokecolor{textcolor}%
\pgfsetfillcolor{textcolor}%
\pgftext[x=8.154848in,y=2.324418in,left,base]{\color{textcolor}{\rmfamily\fontsize{9.000000}{10.800000}\selectfont\catcode`\^=\active\def^{\ifmmode\sp\else\^{}\fi}\catcode`\%=\active\def%{\%}3}}%
\end{pgfscope}%
\begin{pgfscope}%
\pgfsetrectcap%
\pgfsetroundjoin%
\pgfsetlinewidth{1.505625pt}%
\pgfsetstrokecolor{currentstroke2}%
\pgfsetdash{}{0pt}%
\pgfpathmoveto{\pgfqpoint{7.804848in}{2.184696in}}%
\pgfpathlineto{\pgfqpoint{7.929848in}{2.184696in}}%
\pgfpathlineto{\pgfqpoint{8.054848in}{2.184696in}}%
\pgfusepath{stroke}%
\end{pgfscope}%
\begin{pgfscope}%
\definecolor{textcolor}{rgb}{0.000000,0.000000,0.000000}%
\pgfsetstrokecolor{textcolor}%
\pgfsetfillcolor{textcolor}%
\pgftext[x=8.154848in,y=2.140946in,left,base]{\color{textcolor}{\rmfamily\fontsize{9.000000}{10.800000}\selectfont\catcode`\^=\active\def^{\ifmmode\sp\else\^{}\fi}\catcode`\%=\active\def%{\%}4}}%
\end{pgfscope}%
\begin{pgfscope}%
\pgfsetrectcap%
\pgfsetroundjoin%
\pgfsetlinewidth{1.505625pt}%
\pgfsetstrokecolor{currentstroke3}%
\pgfsetdash{}{0pt}%
\pgfpathmoveto{\pgfqpoint{7.804848in}{2.001225in}}%
\pgfpathlineto{\pgfqpoint{7.929848in}{2.001225in}}%
\pgfpathlineto{\pgfqpoint{8.054848in}{2.001225in}}%
\pgfusepath{stroke}%
\end{pgfscope}%
\begin{pgfscope}%
\definecolor{textcolor}{rgb}{0.000000,0.000000,0.000000}%
\pgfsetstrokecolor{textcolor}%
\pgfsetfillcolor{textcolor}%
\pgftext[x=8.154848in,y=1.957475in,left,base]{\color{textcolor}{\rmfamily\fontsize{9.000000}{10.800000}\selectfont\catcode`\^=\active\def^{\ifmmode\sp\else\^{}\fi}\catcode`\%=\active\def%{\%}5}}%
\end{pgfscope}%
\begin{pgfscope}%
\pgfsetrectcap%
\pgfsetroundjoin%
\pgfsetlinewidth{1.505625pt}%
\pgfsetstrokecolor{currentstroke4}%
\pgfsetdash{}{0pt}%
\pgfpathmoveto{\pgfqpoint{7.804848in}{1.817753in}}%
\pgfpathlineto{\pgfqpoint{7.929848in}{1.817753in}}%
\pgfpathlineto{\pgfqpoint{8.054848in}{1.817753in}}%
\pgfusepath{stroke}%
\end{pgfscope}%
\begin{pgfscope}%
\definecolor{textcolor}{rgb}{0.000000,0.000000,0.000000}%
\pgfsetstrokecolor{textcolor}%
\pgfsetfillcolor{textcolor}%
\pgftext[x=8.154848in,y=1.774003in,left,base]{\color{textcolor}{\rmfamily\fontsize{9.000000}{10.800000}\selectfont\catcode`\^=\active\def^{\ifmmode\sp\else\^{}\fi}\catcode`\%=\active\def%{\%}6}}%
\end{pgfscope}%
\begin{pgfscope}%
\pgfsetrectcap%
\pgfsetroundjoin%
\pgfsetlinewidth{1.505625pt}%
\pgfsetstrokecolor{currentstroke5}%
\pgfsetdash{}{0pt}%
\pgfpathmoveto{\pgfqpoint{7.804848in}{1.634281in}}%
\pgfpathlineto{\pgfqpoint{7.929848in}{1.634281in}}%
\pgfpathlineto{\pgfqpoint{8.054848in}{1.634281in}}%
\pgfusepath{stroke}%
\end{pgfscope}%
\begin{pgfscope}%
\definecolor{textcolor}{rgb}{0.000000,0.000000,0.000000}%
\pgfsetstrokecolor{textcolor}%
\pgfsetfillcolor{textcolor}%
\pgftext[x=8.154848in,y=1.590531in,left,base]{\color{textcolor}{\rmfamily\fontsize{9.000000}{10.800000}\selectfont\catcode`\^=\active\def^{\ifmmode\sp\else\^{}\fi}\catcode`\%=\active\def%{\%}7}}%
\end{pgfscope}%
\end{pgfpicture}%
\makeatother%
\endgroup%
}
	\caption[Checks performed for graphs with no NAC-coloring]{
		The number of checks performed to find all NAC-colorings for graphs with no NAC-coloring for different subgraph sizes \( k \).}%
	\label{fig:graph_no_nac_coloring_first_checks_subgraph_size}
\end{figure}%


\todo[inline]{Failing -> Other}

\subsection{Failing strategies}%
\label{sec:failing_strategies}

In this section, we show the performance of other strategies described in \Cref{chapter:alg}.
We do not show these strategies in previous figures as they would influence
the scale and would make figures and legends unreadable and unclear.

Some of these strategies perform as well as
than our preferred strategies for some graph classes,
but fail for others and therefore are not universal enough to use in a library.
%
In the following figures, we fixed
split strategy to \Neighbors{} or merging strategy to \MergeLinear{}
as they perform the best as shown in \Cref{sec:bench_graph_classes}.

First, we show in \Cref{fig:graph_mimimally_rigid_failing_merging_first_runtime,fig:graph_no_nac_coloring_generated_rigid_failing_merging_first_runtime}
how strategies like \Log{} and \PromisingCycles{} fail on minimally rigid graphs.
We also shown in \Cref{fig:graph_mimimally_rigid_failing_split_first_runtime},
\KernighanLin{} and \Cuts{} perform worse.
%
Graphs with no three nor four cycles behave similarly for these strategies.
%
\begin{figure}[thbp]
	\centering
	\scalebox{\BenchFigureScale}{%% Creator: Matplotlib, PGF backend
%%
%% To include the figure in your LaTeX document, write
%%   \input{<filename>.pgf}
%%
%% Make sure the required packages are loaded in your preamble
%%   \usepackage{pgf}
%%
%% Also ensure that all the required font packages are loaded; for instance,
%% the lmodern package is sometimes necessary when using math font.
%%   \usepackage{lmodern}
%%
%% Figures using additional raster images can only be included by \input if
%% they are in the same directory as the main LaTeX file. For loading figures
%% from other directories you can use the `import` package
%%   \usepackage{import}
%%
%% and then include the figures with
%%   \import{<path to file>}{<filename>.pgf}
%%
%% Matplotlib used the following preamble
%%   \def\mathdefault#1{#1}
%%   \everymath=\expandafter{\the\everymath\displaystyle}
%%   \IfFileExists{scrextend.sty}{
%%     \usepackage[fontsize=10.000000pt]{scrextend}
%%   }{
%%     \renewcommand{\normalsize}{\fontsize{10.000000}{12.000000}\selectfont}
%%     \normalsize
%%   }
%%   
%%   \ifdefined\pdftexversion\else  % non-pdftex case.
%%     \usepackage{fontspec}
%%     \setmainfont{DejaVuSans.ttf}[Path=\detokenize{/home/petr/Projects/PyRigi/.venv/lib/python3.12/site-packages/matplotlib/mpl-data/fonts/ttf/}]
%%     \setsansfont{DejaVuSans.ttf}[Path=\detokenize{/home/petr/Projects/PyRigi/.venv/lib/python3.12/site-packages/matplotlib/mpl-data/fonts/ttf/}]
%%     \setmonofont{DejaVuSansMono.ttf}[Path=\detokenize{/home/petr/Projects/PyRigi/.venv/lib/python3.12/site-packages/matplotlib/mpl-data/fonts/ttf/}]
%%   \fi
%%   \makeatletter\@ifpackageloaded{under\Score{}}{}{\usepackage[strings]{under\Score{}}}\makeatother
%%
\begingroup%
\makeatletter%
\begin{pgfpicture}%
\pgfpathrectangle{\pgfpointorigin}{\pgfqpoint{8.384376in}{2.841849in}}%
\pgfusepath{use as bounding box, clip}%
\begin{pgfscope}%
\pgfsetbuttcap%
\pgfsetmiterjoin%
\definecolor{currentfill}{rgb}{1.000000,1.000000,1.000000}%
\pgfsetfillcolor{currentfill}%
\pgfsetlinewidth{0.000000pt}%
\definecolor{currentstroke}{rgb}{1.000000,1.000000,1.000000}%
\pgfsetstrokecolor{currentstroke}%
\pgfsetdash{}{0pt}%
\pgfpathmoveto{\pgfqpoint{0.000000in}{0.000000in}}%
\pgfpathlineto{\pgfqpoint{8.384376in}{0.000000in}}%
\pgfpathlineto{\pgfqpoint{8.384376in}{2.841849in}}%
\pgfpathlineto{\pgfqpoint{0.000000in}{2.841849in}}%
\pgfpathlineto{\pgfqpoint{0.000000in}{0.000000in}}%
\pgfpathclose%
\pgfusepath{fill}%
\end{pgfscope}%
\begin{pgfscope}%
\pgfsetbuttcap%
\pgfsetmiterjoin%
\definecolor{currentfill}{rgb}{1.000000,1.000000,1.000000}%
\pgfsetfillcolor{currentfill}%
\pgfsetlinewidth{0.000000pt}%
\definecolor{currentstroke}{rgb}{0.000000,0.000000,0.000000}%
\pgfsetstrokecolor{currentstroke}%
\pgfsetstrokeopacity{0.000000}%
\pgfsetdash{}{0pt}%
\pgfpathmoveto{\pgfqpoint{0.588387in}{0.521603in}}%
\pgfpathlineto{\pgfqpoint{5.903102in}{0.521603in}}%
\pgfpathlineto{\pgfqpoint{5.903102in}{2.741849in}}%
\pgfpathlineto{\pgfqpoint{0.588387in}{2.741849in}}%
\pgfpathlineto{\pgfqpoint{0.588387in}{0.521603in}}%
\pgfpathclose%
\pgfusepath{fill}%
\end{pgfscope}%
\begin{pgfscope}%
\pgfsetbuttcap%
\pgfsetroundjoin%
\definecolor{currentfill}{rgb}{0.000000,0.000000,0.000000}%
\pgfsetfillcolor{currentfill}%
\pgfsetlinewidth{0.803000pt}%
\definecolor{currentstroke}{rgb}{0.000000,0.000000,0.000000}%
\pgfsetstrokecolor{currentstroke}%
\pgfsetdash{}{0pt}%
\pgfsys@defobject{currentmarker}{\pgfqpoint{0.000000in}{-0.048611in}}{\pgfqpoint{0.000000in}{0.000000in}}{%
\pgfpathmoveto{\pgfqpoint{0.000000in}{0.000000in}}%
\pgfpathlineto{\pgfqpoint{0.000000in}{-0.048611in}}%
\pgfusepath{stroke,fill}%
}%
\begin{pgfscope}%
\pgfsys@transformshift{1.027172in}{0.521603in}%
\pgfsys@useobject{currentmarker}{}%
\end{pgfscope}%
\end{pgfscope}%
\begin{pgfscope}%
\definecolor{textcolor}{rgb}{0.000000,0.000000,0.000000}%
\pgfsetstrokecolor{textcolor}%
\pgfsetfillcolor{textcolor}%
\pgftext[x=1.027172in,y=0.424381in,,top]{\color{textcolor}{\rmfamily\fontsize{10.000000}{12.000000}\selectfont\catcode`\^=\active\def^{\ifmmode\sp\else\^{}\fi}\catcode`\%=\active\def%{\%}$\mathdefault{12}$}}%
\end{pgfscope}%
\begin{pgfscope}%
\pgfsetbuttcap%
\pgfsetroundjoin%
\definecolor{currentfill}{rgb}{0.000000,0.000000,0.000000}%
\pgfsetfillcolor{currentfill}%
\pgfsetlinewidth{0.803000pt}%
\definecolor{currentstroke}{rgb}{0.000000,0.000000,0.000000}%
\pgfsetstrokecolor{currentstroke}%
\pgfsetdash{}{0pt}%
\pgfsys@defobject{currentmarker}{\pgfqpoint{0.000000in}{-0.048611in}}{\pgfqpoint{0.000000in}{0.000000in}}{%
\pgfpathmoveto{\pgfqpoint{0.000000in}{0.000000in}}%
\pgfpathlineto{\pgfqpoint{0.000000in}{-0.048611in}}%
\pgfusepath{stroke,fill}%
}%
\begin{pgfscope}%
\pgfsys@transformshift{1.618791in}{0.521603in}%
\pgfsys@useobject{currentmarker}{}%
\end{pgfscope}%
\end{pgfscope}%
\begin{pgfscope}%
\definecolor{textcolor}{rgb}{0.000000,0.000000,0.000000}%
\pgfsetstrokecolor{textcolor}%
\pgfsetfillcolor{textcolor}%
\pgftext[x=1.618791in,y=0.424381in,,top]{\color{textcolor}{\rmfamily\fontsize{10.000000}{12.000000}\selectfont\catcode`\^=\active\def^{\ifmmode\sp\else\^{}\fi}\catcode`\%=\active\def%{\%}$\mathdefault{18}$}}%
\end{pgfscope}%
\begin{pgfscope}%
\pgfsetbuttcap%
\pgfsetroundjoin%
\definecolor{currentfill}{rgb}{0.000000,0.000000,0.000000}%
\pgfsetfillcolor{currentfill}%
\pgfsetlinewidth{0.803000pt}%
\definecolor{currentstroke}{rgb}{0.000000,0.000000,0.000000}%
\pgfsetstrokecolor{currentstroke}%
\pgfsetdash{}{0pt}%
\pgfsys@defobject{currentmarker}{\pgfqpoint{0.000000in}{-0.048611in}}{\pgfqpoint{0.000000in}{0.000000in}}{%
\pgfpathmoveto{\pgfqpoint{0.000000in}{0.000000in}}%
\pgfpathlineto{\pgfqpoint{0.000000in}{-0.048611in}}%
\pgfusepath{stroke,fill}%
}%
\begin{pgfscope}%
\pgfsys@transformshift{2.210411in}{0.521603in}%
\pgfsys@useobject{currentmarker}{}%
\end{pgfscope}%
\end{pgfscope}%
\begin{pgfscope}%
\definecolor{textcolor}{rgb}{0.000000,0.000000,0.000000}%
\pgfsetstrokecolor{textcolor}%
\pgfsetfillcolor{textcolor}%
\pgftext[x=2.210411in,y=0.424381in,,top]{\color{textcolor}{\rmfamily\fontsize{10.000000}{12.000000}\selectfont\catcode`\^=\active\def^{\ifmmode\sp\else\^{}\fi}\catcode`\%=\active\def%{\%}$\mathdefault{24}$}}%
\end{pgfscope}%
\begin{pgfscope}%
\pgfsetbuttcap%
\pgfsetroundjoin%
\definecolor{currentfill}{rgb}{0.000000,0.000000,0.000000}%
\pgfsetfillcolor{currentfill}%
\pgfsetlinewidth{0.803000pt}%
\definecolor{currentstroke}{rgb}{0.000000,0.000000,0.000000}%
\pgfsetstrokecolor{currentstroke}%
\pgfsetdash{}{0pt}%
\pgfsys@defobject{currentmarker}{\pgfqpoint{0.000000in}{-0.048611in}}{\pgfqpoint{0.000000in}{0.000000in}}{%
\pgfpathmoveto{\pgfqpoint{0.000000in}{0.000000in}}%
\pgfpathlineto{\pgfqpoint{0.000000in}{-0.048611in}}%
\pgfusepath{stroke,fill}%
}%
\begin{pgfscope}%
\pgfsys@transformshift{2.802030in}{0.521603in}%
\pgfsys@useobject{currentmarker}{}%
\end{pgfscope}%
\end{pgfscope}%
\begin{pgfscope}%
\definecolor{textcolor}{rgb}{0.000000,0.000000,0.000000}%
\pgfsetstrokecolor{textcolor}%
\pgfsetfillcolor{textcolor}%
\pgftext[x=2.802030in,y=0.424381in,,top]{\color{textcolor}{\rmfamily\fontsize{10.000000}{12.000000}\selectfont\catcode`\^=\active\def^{\ifmmode\sp\else\^{}\fi}\catcode`\%=\active\def%{\%}$\mathdefault{30}$}}%
\end{pgfscope}%
\begin{pgfscope}%
\pgfsetbuttcap%
\pgfsetroundjoin%
\definecolor{currentfill}{rgb}{0.000000,0.000000,0.000000}%
\pgfsetfillcolor{currentfill}%
\pgfsetlinewidth{0.803000pt}%
\definecolor{currentstroke}{rgb}{0.000000,0.000000,0.000000}%
\pgfsetstrokecolor{currentstroke}%
\pgfsetdash{}{0pt}%
\pgfsys@defobject{currentmarker}{\pgfqpoint{0.000000in}{-0.048611in}}{\pgfqpoint{0.000000in}{0.000000in}}{%
\pgfpathmoveto{\pgfqpoint{0.000000in}{0.000000in}}%
\pgfpathlineto{\pgfqpoint{0.000000in}{-0.048611in}}%
\pgfusepath{stroke,fill}%
}%
\begin{pgfscope}%
\pgfsys@transformshift{3.393649in}{0.521603in}%
\pgfsys@useobject{currentmarker}{}%
\end{pgfscope}%
\end{pgfscope}%
\begin{pgfscope}%
\definecolor{textcolor}{rgb}{0.000000,0.000000,0.000000}%
\pgfsetstrokecolor{textcolor}%
\pgfsetfillcolor{textcolor}%
\pgftext[x=3.393649in,y=0.424381in,,top]{\color{textcolor}{\rmfamily\fontsize{10.000000}{12.000000}\selectfont\catcode`\^=\active\def^{\ifmmode\sp\else\^{}\fi}\catcode`\%=\active\def%{\%}$\mathdefault{36}$}}%
\end{pgfscope}%
\begin{pgfscope}%
\pgfsetbuttcap%
\pgfsetroundjoin%
\definecolor{currentfill}{rgb}{0.000000,0.000000,0.000000}%
\pgfsetfillcolor{currentfill}%
\pgfsetlinewidth{0.803000pt}%
\definecolor{currentstroke}{rgb}{0.000000,0.000000,0.000000}%
\pgfsetstrokecolor{currentstroke}%
\pgfsetdash{}{0pt}%
\pgfsys@defobject{currentmarker}{\pgfqpoint{0.000000in}{-0.048611in}}{\pgfqpoint{0.000000in}{0.000000in}}{%
\pgfpathmoveto{\pgfqpoint{0.000000in}{0.000000in}}%
\pgfpathlineto{\pgfqpoint{0.000000in}{-0.048611in}}%
\pgfusepath{stroke,fill}%
}%
\begin{pgfscope}%
\pgfsys@transformshift{3.985269in}{0.521603in}%
\pgfsys@useobject{currentmarker}{}%
\end{pgfscope}%
\end{pgfscope}%
\begin{pgfscope}%
\definecolor{textcolor}{rgb}{0.000000,0.000000,0.000000}%
\pgfsetstrokecolor{textcolor}%
\pgfsetfillcolor{textcolor}%
\pgftext[x=3.985269in,y=0.424381in,,top]{\color{textcolor}{\rmfamily\fontsize{10.000000}{12.000000}\selectfont\catcode`\^=\active\def^{\ifmmode\sp\else\^{}\fi}\catcode`\%=\active\def%{\%}$\mathdefault{42}$}}%
\end{pgfscope}%
\begin{pgfscope}%
\pgfsetbuttcap%
\pgfsetroundjoin%
\definecolor{currentfill}{rgb}{0.000000,0.000000,0.000000}%
\pgfsetfillcolor{currentfill}%
\pgfsetlinewidth{0.803000pt}%
\definecolor{currentstroke}{rgb}{0.000000,0.000000,0.000000}%
\pgfsetstrokecolor{currentstroke}%
\pgfsetdash{}{0pt}%
\pgfsys@defobject{currentmarker}{\pgfqpoint{0.000000in}{-0.048611in}}{\pgfqpoint{0.000000in}{0.000000in}}{%
\pgfpathmoveto{\pgfqpoint{0.000000in}{0.000000in}}%
\pgfpathlineto{\pgfqpoint{0.000000in}{-0.048611in}}%
\pgfusepath{stroke,fill}%
}%
\begin{pgfscope}%
\pgfsys@transformshift{4.576888in}{0.521603in}%
\pgfsys@useobject{currentmarker}{}%
\end{pgfscope}%
\end{pgfscope}%
\begin{pgfscope}%
\definecolor{textcolor}{rgb}{0.000000,0.000000,0.000000}%
\pgfsetstrokecolor{textcolor}%
\pgfsetfillcolor{textcolor}%
\pgftext[x=4.576888in,y=0.424381in,,top]{\color{textcolor}{\rmfamily\fontsize{10.000000}{12.000000}\selectfont\catcode`\^=\active\def^{\ifmmode\sp\else\^{}\fi}\catcode`\%=\active\def%{\%}$\mathdefault{48}$}}%
\end{pgfscope}%
\begin{pgfscope}%
\pgfsetbuttcap%
\pgfsetroundjoin%
\definecolor{currentfill}{rgb}{0.000000,0.000000,0.000000}%
\pgfsetfillcolor{currentfill}%
\pgfsetlinewidth{0.803000pt}%
\definecolor{currentstroke}{rgb}{0.000000,0.000000,0.000000}%
\pgfsetstrokecolor{currentstroke}%
\pgfsetdash{}{0pt}%
\pgfsys@defobject{currentmarker}{\pgfqpoint{0.000000in}{-0.048611in}}{\pgfqpoint{0.000000in}{0.000000in}}{%
\pgfpathmoveto{\pgfqpoint{0.000000in}{0.000000in}}%
\pgfpathlineto{\pgfqpoint{0.000000in}{-0.048611in}}%
\pgfusepath{stroke,fill}%
}%
\begin{pgfscope}%
\pgfsys@transformshift{5.168508in}{0.521603in}%
\pgfsys@useobject{currentmarker}{}%
\end{pgfscope}%
\end{pgfscope}%
\begin{pgfscope}%
\definecolor{textcolor}{rgb}{0.000000,0.000000,0.000000}%
\pgfsetstrokecolor{textcolor}%
\pgfsetfillcolor{textcolor}%
\pgftext[x=5.168508in,y=0.424381in,,top]{\color{textcolor}{\rmfamily\fontsize{10.000000}{12.000000}\selectfont\catcode`\^=\active\def^{\ifmmode\sp\else\^{}\fi}\catcode`\%=\active\def%{\%}$\mathdefault{54}$}}%
\end{pgfscope}%
\begin{pgfscope}%
\pgfsetbuttcap%
\pgfsetroundjoin%
\definecolor{currentfill}{rgb}{0.000000,0.000000,0.000000}%
\pgfsetfillcolor{currentfill}%
\pgfsetlinewidth{0.803000pt}%
\definecolor{currentstroke}{rgb}{0.000000,0.000000,0.000000}%
\pgfsetstrokecolor{currentstroke}%
\pgfsetdash{}{0pt}%
\pgfsys@defobject{currentmarker}{\pgfqpoint{0.000000in}{-0.048611in}}{\pgfqpoint{0.000000in}{0.000000in}}{%
\pgfpathmoveto{\pgfqpoint{0.000000in}{0.000000in}}%
\pgfpathlineto{\pgfqpoint{0.000000in}{-0.048611in}}%
\pgfusepath{stroke,fill}%
}%
\begin{pgfscope}%
\pgfsys@transformshift{5.760127in}{0.521603in}%
\pgfsys@useobject{currentmarker}{}%
\end{pgfscope}%
\end{pgfscope}%
\begin{pgfscope}%
\definecolor{textcolor}{rgb}{0.000000,0.000000,0.000000}%
\pgfsetstrokecolor{textcolor}%
\pgfsetfillcolor{textcolor}%
\pgftext[x=5.760127in,y=0.424381in,,top]{\color{textcolor}{\rmfamily\fontsize{10.000000}{12.000000}\selectfont\catcode`\^=\active\def^{\ifmmode\sp\else\^{}\fi}\catcode`\%=\active\def%{\%}$\mathdefault{60}$}}%
\end{pgfscope}%
\begin{pgfscope}%
\definecolor{textcolor}{rgb}{0.000000,0.000000,0.000000}%
\pgfsetstrokecolor{textcolor}%
\pgfsetfillcolor{textcolor}%
\pgftext[x=3.245745in,y=0.234413in,,top]{\color{textcolor}{\rmfamily\fontsize{10.000000}{12.000000}\selectfont\catcode`\^=\active\def^{\ifmmode\sp\else\^{}\fi}\catcode`\%=\active\def%{\%}Vertices}}%
\end{pgfscope}%
\begin{pgfscope}%
\pgfsetbuttcap%
\pgfsetroundjoin%
\definecolor{currentfill}{rgb}{0.000000,0.000000,0.000000}%
\pgfsetfillcolor{currentfill}%
\pgfsetlinewidth{0.803000pt}%
\definecolor{currentstroke}{rgb}{0.000000,0.000000,0.000000}%
\pgfsetstrokecolor{currentstroke}%
\pgfsetdash{}{0pt}%
\pgfsys@defobject{currentmarker}{\pgfqpoint{-0.048611in}{0.000000in}}{\pgfqpoint{-0.000000in}{0.000000in}}{%
\pgfpathmoveto{\pgfqpoint{-0.000000in}{0.000000in}}%
\pgfpathlineto{\pgfqpoint{-0.048611in}{0.000000in}}%
\pgfusepath{stroke,fill}%
}%
\begin{pgfscope}%
\pgfsys@transformshift{0.588387in}{0.988735in}%
\pgfsys@useobject{currentmarker}{}%
\end{pgfscope}%
\end{pgfscope}%
\begin{pgfscope}%
\definecolor{textcolor}{rgb}{0.000000,0.000000,0.000000}%
\pgfsetstrokecolor{textcolor}%
\pgfsetfillcolor{textcolor}%
\pgftext[x=0.289968in, y=0.935973in, left, base]{\color{textcolor}{\rmfamily\fontsize{10.000000}{12.000000}\selectfont\catcode`\^=\active\def^{\ifmmode\sp\else\^{}\fi}\catcode`\%=\active\def%{\%}$\mathdefault{10^{1}}$}}%
\end{pgfscope}%
\begin{pgfscope}%
\pgfsetbuttcap%
\pgfsetroundjoin%
\definecolor{currentfill}{rgb}{0.000000,0.000000,0.000000}%
\pgfsetfillcolor{currentfill}%
\pgfsetlinewidth{0.803000pt}%
\definecolor{currentstroke}{rgb}{0.000000,0.000000,0.000000}%
\pgfsetstrokecolor{currentstroke}%
\pgfsetdash{}{0pt}%
\pgfsys@defobject{currentmarker}{\pgfqpoint{-0.048611in}{0.000000in}}{\pgfqpoint{-0.000000in}{0.000000in}}{%
\pgfpathmoveto{\pgfqpoint{-0.000000in}{0.000000in}}%
\pgfpathlineto{\pgfqpoint{-0.048611in}{0.000000in}}%
\pgfusepath{stroke,fill}%
}%
\begin{pgfscope}%
\pgfsys@transformshift{0.588387in}{1.755187in}%
\pgfsys@useobject{currentmarker}{}%
\end{pgfscope}%
\end{pgfscope}%
\begin{pgfscope}%
\definecolor{textcolor}{rgb}{0.000000,0.000000,0.000000}%
\pgfsetstrokecolor{textcolor}%
\pgfsetfillcolor{textcolor}%
\pgftext[x=0.289968in, y=1.702425in, left, base]{\color{textcolor}{\rmfamily\fontsize{10.000000}{12.000000}\selectfont\catcode`\^=\active\def^{\ifmmode\sp\else\^{}\fi}\catcode`\%=\active\def%{\%}$\mathdefault{10^{2}}$}}%
\end{pgfscope}%
\begin{pgfscope}%
\pgfsetbuttcap%
\pgfsetroundjoin%
\definecolor{currentfill}{rgb}{0.000000,0.000000,0.000000}%
\pgfsetfillcolor{currentfill}%
\pgfsetlinewidth{0.803000pt}%
\definecolor{currentstroke}{rgb}{0.000000,0.000000,0.000000}%
\pgfsetstrokecolor{currentstroke}%
\pgfsetdash{}{0pt}%
\pgfsys@defobject{currentmarker}{\pgfqpoint{-0.048611in}{0.000000in}}{\pgfqpoint{-0.000000in}{0.000000in}}{%
\pgfpathmoveto{\pgfqpoint{-0.000000in}{0.000000in}}%
\pgfpathlineto{\pgfqpoint{-0.048611in}{0.000000in}}%
\pgfusepath{stroke,fill}%
}%
\begin{pgfscope}%
\pgfsys@transformshift{0.588387in}{2.521639in}%
\pgfsys@useobject{currentmarker}{}%
\end{pgfscope}%
\end{pgfscope}%
\begin{pgfscope}%
\definecolor{textcolor}{rgb}{0.000000,0.000000,0.000000}%
\pgfsetstrokecolor{textcolor}%
\pgfsetfillcolor{textcolor}%
\pgftext[x=0.289968in, y=2.468877in, left, base]{\color{textcolor}{\rmfamily\fontsize{10.000000}{12.000000}\selectfont\catcode`\^=\active\def^{\ifmmode\sp\else\^{}\fi}\catcode`\%=\active\def%{\%}$\mathdefault{10^{3}}$}}%
\end{pgfscope}%
\begin{pgfscope}%
\pgfsetbuttcap%
\pgfsetroundjoin%
\definecolor{currentfill}{rgb}{0.000000,0.000000,0.000000}%
\pgfsetfillcolor{currentfill}%
\pgfsetlinewidth{0.602250pt}%
\definecolor{currentstroke}{rgb}{0.000000,0.000000,0.000000}%
\pgfsetstrokecolor{currentstroke}%
\pgfsetdash{}{0pt}%
\pgfsys@defobject{currentmarker}{\pgfqpoint{-0.027778in}{0.000000in}}{\pgfqpoint{-0.000000in}{0.000000in}}{%
\pgfpathmoveto{\pgfqpoint{-0.000000in}{0.000000in}}%
\pgfpathlineto{\pgfqpoint{-0.027778in}{0.000000in}}%
\pgfusepath{stroke,fill}%
}%
\begin{pgfscope}%
\pgfsys@transformshift{0.588387in}{0.587973in}%
\pgfsys@useobject{currentmarker}{}%
\end{pgfscope}%
\end{pgfscope}%
\begin{pgfscope}%
\pgfsetbuttcap%
\pgfsetroundjoin%
\definecolor{currentfill}{rgb}{0.000000,0.000000,0.000000}%
\pgfsetfillcolor{currentfill}%
\pgfsetlinewidth{0.602250pt}%
\definecolor{currentstroke}{rgb}{0.000000,0.000000,0.000000}%
\pgfsetstrokecolor{currentstroke}%
\pgfsetdash{}{0pt}%
\pgfsys@defobject{currentmarker}{\pgfqpoint{-0.027778in}{0.000000in}}{\pgfqpoint{-0.000000in}{0.000000in}}{%
\pgfpathmoveto{\pgfqpoint{-0.000000in}{0.000000in}}%
\pgfpathlineto{\pgfqpoint{-0.027778in}{0.000000in}}%
\pgfusepath{stroke,fill}%
}%
\begin{pgfscope}%
\pgfsys@transformshift{0.588387in}{0.683733in}%
\pgfsys@useobject{currentmarker}{}%
\end{pgfscope}%
\end{pgfscope}%
\begin{pgfscope}%
\pgfsetbuttcap%
\pgfsetroundjoin%
\definecolor{currentfill}{rgb}{0.000000,0.000000,0.000000}%
\pgfsetfillcolor{currentfill}%
\pgfsetlinewidth{0.602250pt}%
\definecolor{currentstroke}{rgb}{0.000000,0.000000,0.000000}%
\pgfsetstrokecolor{currentstroke}%
\pgfsetdash{}{0pt}%
\pgfsys@defobject{currentmarker}{\pgfqpoint{-0.027778in}{0.000000in}}{\pgfqpoint{-0.000000in}{0.000000in}}{%
\pgfpathmoveto{\pgfqpoint{-0.000000in}{0.000000in}}%
\pgfpathlineto{\pgfqpoint{-0.027778in}{0.000000in}}%
\pgfusepath{stroke,fill}%
}%
\begin{pgfscope}%
\pgfsys@transformshift{0.588387in}{0.758010in}%
\pgfsys@useobject{currentmarker}{}%
\end{pgfscope}%
\end{pgfscope}%
\begin{pgfscope}%
\pgfsetbuttcap%
\pgfsetroundjoin%
\definecolor{currentfill}{rgb}{0.000000,0.000000,0.000000}%
\pgfsetfillcolor{currentfill}%
\pgfsetlinewidth{0.602250pt}%
\definecolor{currentstroke}{rgb}{0.000000,0.000000,0.000000}%
\pgfsetstrokecolor{currentstroke}%
\pgfsetdash{}{0pt}%
\pgfsys@defobject{currentmarker}{\pgfqpoint{-0.027778in}{0.000000in}}{\pgfqpoint{-0.000000in}{0.000000in}}{%
\pgfpathmoveto{\pgfqpoint{-0.000000in}{0.000000in}}%
\pgfpathlineto{\pgfqpoint{-0.027778in}{0.000000in}}%
\pgfusepath{stroke,fill}%
}%
\begin{pgfscope}%
\pgfsys@transformshift{0.588387in}{0.818698in}%
\pgfsys@useobject{currentmarker}{}%
\end{pgfscope}%
\end{pgfscope}%
\begin{pgfscope}%
\pgfsetbuttcap%
\pgfsetroundjoin%
\definecolor{currentfill}{rgb}{0.000000,0.000000,0.000000}%
\pgfsetfillcolor{currentfill}%
\pgfsetlinewidth{0.602250pt}%
\definecolor{currentstroke}{rgb}{0.000000,0.000000,0.000000}%
\pgfsetstrokecolor{currentstroke}%
\pgfsetdash{}{0pt}%
\pgfsys@defobject{currentmarker}{\pgfqpoint{-0.027778in}{0.000000in}}{\pgfqpoint{-0.000000in}{0.000000in}}{%
\pgfpathmoveto{\pgfqpoint{-0.000000in}{0.000000in}}%
\pgfpathlineto{\pgfqpoint{-0.027778in}{0.000000in}}%
\pgfusepath{stroke,fill}%
}%
\begin{pgfscope}%
\pgfsys@transformshift{0.588387in}{0.870010in}%
\pgfsys@useobject{currentmarker}{}%
\end{pgfscope}%
\end{pgfscope}%
\begin{pgfscope}%
\pgfsetbuttcap%
\pgfsetroundjoin%
\definecolor{currentfill}{rgb}{0.000000,0.000000,0.000000}%
\pgfsetfillcolor{currentfill}%
\pgfsetlinewidth{0.602250pt}%
\definecolor{currentstroke}{rgb}{0.000000,0.000000,0.000000}%
\pgfsetstrokecolor{currentstroke}%
\pgfsetdash{}{0pt}%
\pgfsys@defobject{currentmarker}{\pgfqpoint{-0.027778in}{0.000000in}}{\pgfqpoint{-0.000000in}{0.000000in}}{%
\pgfpathmoveto{\pgfqpoint{-0.000000in}{0.000000in}}%
\pgfpathlineto{\pgfqpoint{-0.027778in}{0.000000in}}%
\pgfusepath{stroke,fill}%
}%
\begin{pgfscope}%
\pgfsys@transformshift{0.588387in}{0.914458in}%
\pgfsys@useobject{currentmarker}{}%
\end{pgfscope}%
\end{pgfscope}%
\begin{pgfscope}%
\pgfsetbuttcap%
\pgfsetroundjoin%
\definecolor{currentfill}{rgb}{0.000000,0.000000,0.000000}%
\pgfsetfillcolor{currentfill}%
\pgfsetlinewidth{0.602250pt}%
\definecolor{currentstroke}{rgb}{0.000000,0.000000,0.000000}%
\pgfsetstrokecolor{currentstroke}%
\pgfsetdash{}{0pt}%
\pgfsys@defobject{currentmarker}{\pgfqpoint{-0.027778in}{0.000000in}}{\pgfqpoint{-0.000000in}{0.000000in}}{%
\pgfpathmoveto{\pgfqpoint{-0.000000in}{0.000000in}}%
\pgfpathlineto{\pgfqpoint{-0.027778in}{0.000000in}}%
\pgfusepath{stroke,fill}%
}%
\begin{pgfscope}%
\pgfsys@transformshift{0.588387in}{0.953664in}%
\pgfsys@useobject{currentmarker}{}%
\end{pgfscope}%
\end{pgfscope}%
\begin{pgfscope}%
\pgfsetbuttcap%
\pgfsetroundjoin%
\definecolor{currentfill}{rgb}{0.000000,0.000000,0.000000}%
\pgfsetfillcolor{currentfill}%
\pgfsetlinewidth{0.602250pt}%
\definecolor{currentstroke}{rgb}{0.000000,0.000000,0.000000}%
\pgfsetstrokecolor{currentstroke}%
\pgfsetdash{}{0pt}%
\pgfsys@defobject{currentmarker}{\pgfqpoint{-0.027778in}{0.000000in}}{\pgfqpoint{-0.000000in}{0.000000in}}{%
\pgfpathmoveto{\pgfqpoint{-0.000000in}{0.000000in}}%
\pgfpathlineto{\pgfqpoint{-0.027778in}{0.000000in}}%
\pgfusepath{stroke,fill}%
}%
\begin{pgfscope}%
\pgfsys@transformshift{0.588387in}{1.219460in}%
\pgfsys@useobject{currentmarker}{}%
\end{pgfscope}%
\end{pgfscope}%
\begin{pgfscope}%
\pgfsetbuttcap%
\pgfsetroundjoin%
\definecolor{currentfill}{rgb}{0.000000,0.000000,0.000000}%
\pgfsetfillcolor{currentfill}%
\pgfsetlinewidth{0.602250pt}%
\definecolor{currentstroke}{rgb}{0.000000,0.000000,0.000000}%
\pgfsetstrokecolor{currentstroke}%
\pgfsetdash{}{0pt}%
\pgfsys@defobject{currentmarker}{\pgfqpoint{-0.027778in}{0.000000in}}{\pgfqpoint{-0.000000in}{0.000000in}}{%
\pgfpathmoveto{\pgfqpoint{-0.000000in}{0.000000in}}%
\pgfpathlineto{\pgfqpoint{-0.027778in}{0.000000in}}%
\pgfusepath{stroke,fill}%
}%
\begin{pgfscope}%
\pgfsys@transformshift{0.588387in}{1.354425in}%
\pgfsys@useobject{currentmarker}{}%
\end{pgfscope}%
\end{pgfscope}%
\begin{pgfscope}%
\pgfsetbuttcap%
\pgfsetroundjoin%
\definecolor{currentfill}{rgb}{0.000000,0.000000,0.000000}%
\pgfsetfillcolor{currentfill}%
\pgfsetlinewidth{0.602250pt}%
\definecolor{currentstroke}{rgb}{0.000000,0.000000,0.000000}%
\pgfsetstrokecolor{currentstroke}%
\pgfsetdash{}{0pt}%
\pgfsys@defobject{currentmarker}{\pgfqpoint{-0.027778in}{0.000000in}}{\pgfqpoint{-0.000000in}{0.000000in}}{%
\pgfpathmoveto{\pgfqpoint{-0.000000in}{0.000000in}}%
\pgfpathlineto{\pgfqpoint{-0.027778in}{0.000000in}}%
\pgfusepath{stroke,fill}%
}%
\begin{pgfscope}%
\pgfsys@transformshift{0.588387in}{1.450185in}%
\pgfsys@useobject{currentmarker}{}%
\end{pgfscope}%
\end{pgfscope}%
\begin{pgfscope}%
\pgfsetbuttcap%
\pgfsetroundjoin%
\definecolor{currentfill}{rgb}{0.000000,0.000000,0.000000}%
\pgfsetfillcolor{currentfill}%
\pgfsetlinewidth{0.602250pt}%
\definecolor{currentstroke}{rgb}{0.000000,0.000000,0.000000}%
\pgfsetstrokecolor{currentstroke}%
\pgfsetdash{}{0pt}%
\pgfsys@defobject{currentmarker}{\pgfqpoint{-0.027778in}{0.000000in}}{\pgfqpoint{-0.000000in}{0.000000in}}{%
\pgfpathmoveto{\pgfqpoint{-0.000000in}{0.000000in}}%
\pgfpathlineto{\pgfqpoint{-0.027778in}{0.000000in}}%
\pgfusepath{stroke,fill}%
}%
\begin{pgfscope}%
\pgfsys@transformshift{0.588387in}{1.524462in}%
\pgfsys@useobject{currentmarker}{}%
\end{pgfscope}%
\end{pgfscope}%
\begin{pgfscope}%
\pgfsetbuttcap%
\pgfsetroundjoin%
\definecolor{currentfill}{rgb}{0.000000,0.000000,0.000000}%
\pgfsetfillcolor{currentfill}%
\pgfsetlinewidth{0.602250pt}%
\definecolor{currentstroke}{rgb}{0.000000,0.000000,0.000000}%
\pgfsetstrokecolor{currentstroke}%
\pgfsetdash{}{0pt}%
\pgfsys@defobject{currentmarker}{\pgfqpoint{-0.027778in}{0.000000in}}{\pgfqpoint{-0.000000in}{0.000000in}}{%
\pgfpathmoveto{\pgfqpoint{-0.000000in}{0.000000in}}%
\pgfpathlineto{\pgfqpoint{-0.027778in}{0.000000in}}%
\pgfusepath{stroke,fill}%
}%
\begin{pgfscope}%
\pgfsys@transformshift{0.588387in}{1.585150in}%
\pgfsys@useobject{currentmarker}{}%
\end{pgfscope}%
\end{pgfscope}%
\begin{pgfscope}%
\pgfsetbuttcap%
\pgfsetroundjoin%
\definecolor{currentfill}{rgb}{0.000000,0.000000,0.000000}%
\pgfsetfillcolor{currentfill}%
\pgfsetlinewidth{0.602250pt}%
\definecolor{currentstroke}{rgb}{0.000000,0.000000,0.000000}%
\pgfsetstrokecolor{currentstroke}%
\pgfsetdash{}{0pt}%
\pgfsys@defobject{currentmarker}{\pgfqpoint{-0.027778in}{0.000000in}}{\pgfqpoint{-0.000000in}{0.000000in}}{%
\pgfpathmoveto{\pgfqpoint{-0.000000in}{0.000000in}}%
\pgfpathlineto{\pgfqpoint{-0.027778in}{0.000000in}}%
\pgfusepath{stroke,fill}%
}%
\begin{pgfscope}%
\pgfsys@transformshift{0.588387in}{1.636462in}%
\pgfsys@useobject{currentmarker}{}%
\end{pgfscope}%
\end{pgfscope}%
\begin{pgfscope}%
\pgfsetbuttcap%
\pgfsetroundjoin%
\definecolor{currentfill}{rgb}{0.000000,0.000000,0.000000}%
\pgfsetfillcolor{currentfill}%
\pgfsetlinewidth{0.602250pt}%
\definecolor{currentstroke}{rgb}{0.000000,0.000000,0.000000}%
\pgfsetstrokecolor{currentstroke}%
\pgfsetdash{}{0pt}%
\pgfsys@defobject{currentmarker}{\pgfqpoint{-0.027778in}{0.000000in}}{\pgfqpoint{-0.000000in}{0.000000in}}{%
\pgfpathmoveto{\pgfqpoint{-0.000000in}{0.000000in}}%
\pgfpathlineto{\pgfqpoint{-0.027778in}{0.000000in}}%
\pgfusepath{stroke,fill}%
}%
\begin{pgfscope}%
\pgfsys@transformshift{0.588387in}{1.680910in}%
\pgfsys@useobject{currentmarker}{}%
\end{pgfscope}%
\end{pgfscope}%
\begin{pgfscope}%
\pgfsetbuttcap%
\pgfsetroundjoin%
\definecolor{currentfill}{rgb}{0.000000,0.000000,0.000000}%
\pgfsetfillcolor{currentfill}%
\pgfsetlinewidth{0.602250pt}%
\definecolor{currentstroke}{rgb}{0.000000,0.000000,0.000000}%
\pgfsetstrokecolor{currentstroke}%
\pgfsetdash{}{0pt}%
\pgfsys@defobject{currentmarker}{\pgfqpoint{-0.027778in}{0.000000in}}{\pgfqpoint{-0.000000in}{0.000000in}}{%
\pgfpathmoveto{\pgfqpoint{-0.000000in}{0.000000in}}%
\pgfpathlineto{\pgfqpoint{-0.027778in}{0.000000in}}%
\pgfusepath{stroke,fill}%
}%
\begin{pgfscope}%
\pgfsys@transformshift{0.588387in}{1.720116in}%
\pgfsys@useobject{currentmarker}{}%
\end{pgfscope}%
\end{pgfscope}%
\begin{pgfscope}%
\pgfsetbuttcap%
\pgfsetroundjoin%
\definecolor{currentfill}{rgb}{0.000000,0.000000,0.000000}%
\pgfsetfillcolor{currentfill}%
\pgfsetlinewidth{0.602250pt}%
\definecolor{currentstroke}{rgb}{0.000000,0.000000,0.000000}%
\pgfsetstrokecolor{currentstroke}%
\pgfsetdash{}{0pt}%
\pgfsys@defobject{currentmarker}{\pgfqpoint{-0.027778in}{0.000000in}}{\pgfqpoint{-0.000000in}{0.000000in}}{%
\pgfpathmoveto{\pgfqpoint{-0.000000in}{0.000000in}}%
\pgfpathlineto{\pgfqpoint{-0.027778in}{0.000000in}}%
\pgfusepath{stroke,fill}%
}%
\begin{pgfscope}%
\pgfsys@transformshift{0.588387in}{1.985912in}%
\pgfsys@useobject{currentmarker}{}%
\end{pgfscope}%
\end{pgfscope}%
\begin{pgfscope}%
\pgfsetbuttcap%
\pgfsetroundjoin%
\definecolor{currentfill}{rgb}{0.000000,0.000000,0.000000}%
\pgfsetfillcolor{currentfill}%
\pgfsetlinewidth{0.602250pt}%
\definecolor{currentstroke}{rgb}{0.000000,0.000000,0.000000}%
\pgfsetstrokecolor{currentstroke}%
\pgfsetdash{}{0pt}%
\pgfsys@defobject{currentmarker}{\pgfqpoint{-0.027778in}{0.000000in}}{\pgfqpoint{-0.000000in}{0.000000in}}{%
\pgfpathmoveto{\pgfqpoint{-0.000000in}{0.000000in}}%
\pgfpathlineto{\pgfqpoint{-0.027778in}{0.000000in}}%
\pgfusepath{stroke,fill}%
}%
\begin{pgfscope}%
\pgfsys@transformshift{0.588387in}{2.120877in}%
\pgfsys@useobject{currentmarker}{}%
\end{pgfscope}%
\end{pgfscope}%
\begin{pgfscope}%
\pgfsetbuttcap%
\pgfsetroundjoin%
\definecolor{currentfill}{rgb}{0.000000,0.000000,0.000000}%
\pgfsetfillcolor{currentfill}%
\pgfsetlinewidth{0.602250pt}%
\definecolor{currentstroke}{rgb}{0.000000,0.000000,0.000000}%
\pgfsetstrokecolor{currentstroke}%
\pgfsetdash{}{0pt}%
\pgfsys@defobject{currentmarker}{\pgfqpoint{-0.027778in}{0.000000in}}{\pgfqpoint{-0.000000in}{0.000000in}}{%
\pgfpathmoveto{\pgfqpoint{-0.000000in}{0.000000in}}%
\pgfpathlineto{\pgfqpoint{-0.027778in}{0.000000in}}%
\pgfusepath{stroke,fill}%
}%
\begin{pgfscope}%
\pgfsys@transformshift{0.588387in}{2.216637in}%
\pgfsys@useobject{currentmarker}{}%
\end{pgfscope}%
\end{pgfscope}%
\begin{pgfscope}%
\pgfsetbuttcap%
\pgfsetroundjoin%
\definecolor{currentfill}{rgb}{0.000000,0.000000,0.000000}%
\pgfsetfillcolor{currentfill}%
\pgfsetlinewidth{0.602250pt}%
\definecolor{currentstroke}{rgb}{0.000000,0.000000,0.000000}%
\pgfsetstrokecolor{currentstroke}%
\pgfsetdash{}{0pt}%
\pgfsys@defobject{currentmarker}{\pgfqpoint{-0.027778in}{0.000000in}}{\pgfqpoint{-0.000000in}{0.000000in}}{%
\pgfpathmoveto{\pgfqpoint{-0.000000in}{0.000000in}}%
\pgfpathlineto{\pgfqpoint{-0.027778in}{0.000000in}}%
\pgfusepath{stroke,fill}%
}%
\begin{pgfscope}%
\pgfsys@transformshift{0.588387in}{2.290914in}%
\pgfsys@useobject{currentmarker}{}%
\end{pgfscope}%
\end{pgfscope}%
\begin{pgfscope}%
\pgfsetbuttcap%
\pgfsetroundjoin%
\definecolor{currentfill}{rgb}{0.000000,0.000000,0.000000}%
\pgfsetfillcolor{currentfill}%
\pgfsetlinewidth{0.602250pt}%
\definecolor{currentstroke}{rgb}{0.000000,0.000000,0.000000}%
\pgfsetstrokecolor{currentstroke}%
\pgfsetdash{}{0pt}%
\pgfsys@defobject{currentmarker}{\pgfqpoint{-0.027778in}{0.000000in}}{\pgfqpoint{-0.000000in}{0.000000in}}{%
\pgfpathmoveto{\pgfqpoint{-0.000000in}{0.000000in}}%
\pgfpathlineto{\pgfqpoint{-0.027778in}{0.000000in}}%
\pgfusepath{stroke,fill}%
}%
\begin{pgfscope}%
\pgfsys@transformshift{0.588387in}{2.351602in}%
\pgfsys@useobject{currentmarker}{}%
\end{pgfscope}%
\end{pgfscope}%
\begin{pgfscope}%
\pgfsetbuttcap%
\pgfsetroundjoin%
\definecolor{currentfill}{rgb}{0.000000,0.000000,0.000000}%
\pgfsetfillcolor{currentfill}%
\pgfsetlinewidth{0.602250pt}%
\definecolor{currentstroke}{rgb}{0.000000,0.000000,0.000000}%
\pgfsetstrokecolor{currentstroke}%
\pgfsetdash{}{0pt}%
\pgfsys@defobject{currentmarker}{\pgfqpoint{-0.027778in}{0.000000in}}{\pgfqpoint{-0.000000in}{0.000000in}}{%
\pgfpathmoveto{\pgfqpoint{-0.000000in}{0.000000in}}%
\pgfpathlineto{\pgfqpoint{-0.027778in}{0.000000in}}%
\pgfusepath{stroke,fill}%
}%
\begin{pgfscope}%
\pgfsys@transformshift{0.588387in}{2.402914in}%
\pgfsys@useobject{currentmarker}{}%
\end{pgfscope}%
\end{pgfscope}%
\begin{pgfscope}%
\pgfsetbuttcap%
\pgfsetroundjoin%
\definecolor{currentfill}{rgb}{0.000000,0.000000,0.000000}%
\pgfsetfillcolor{currentfill}%
\pgfsetlinewidth{0.602250pt}%
\definecolor{currentstroke}{rgb}{0.000000,0.000000,0.000000}%
\pgfsetstrokecolor{currentstroke}%
\pgfsetdash{}{0pt}%
\pgfsys@defobject{currentmarker}{\pgfqpoint{-0.027778in}{0.000000in}}{\pgfqpoint{-0.000000in}{0.000000in}}{%
\pgfpathmoveto{\pgfqpoint{-0.000000in}{0.000000in}}%
\pgfpathlineto{\pgfqpoint{-0.027778in}{0.000000in}}%
\pgfusepath{stroke,fill}%
}%
\begin{pgfscope}%
\pgfsys@transformshift{0.588387in}{2.447362in}%
\pgfsys@useobject{currentmarker}{}%
\end{pgfscope}%
\end{pgfscope}%
\begin{pgfscope}%
\pgfsetbuttcap%
\pgfsetroundjoin%
\definecolor{currentfill}{rgb}{0.000000,0.000000,0.000000}%
\pgfsetfillcolor{currentfill}%
\pgfsetlinewidth{0.602250pt}%
\definecolor{currentstroke}{rgb}{0.000000,0.000000,0.000000}%
\pgfsetstrokecolor{currentstroke}%
\pgfsetdash{}{0pt}%
\pgfsys@defobject{currentmarker}{\pgfqpoint{-0.027778in}{0.000000in}}{\pgfqpoint{-0.000000in}{0.000000in}}{%
\pgfpathmoveto{\pgfqpoint{-0.000000in}{0.000000in}}%
\pgfpathlineto{\pgfqpoint{-0.027778in}{0.000000in}}%
\pgfusepath{stroke,fill}%
}%
\begin{pgfscope}%
\pgfsys@transformshift{0.588387in}{2.486568in}%
\pgfsys@useobject{currentmarker}{}%
\end{pgfscope}%
\end{pgfscope}%
\begin{pgfscope}%
\definecolor{textcolor}{rgb}{0.000000,0.000000,0.000000}%
\pgfsetstrokecolor{textcolor}%
\pgfsetfillcolor{textcolor}%
\pgftext[x=0.234413in,y=1.631726in,,bottom,rotate=90.000000]{\color{textcolor}{\rmfamily\fontsize{10.000000}{12.000000}\selectfont\catcode`\^=\active\def^{\ifmmode\sp\else\^{}\fi}\catcode`\%=\active\def%{\%}Time [ms]}}%
\end{pgfscope}%
\begin{pgfscope}%
\pgfpathrectangle{\pgfqpoint{0.588387in}{0.521603in}}{\pgfqpoint{5.314715in}{2.220246in}}%
\pgfusepath{clip}%
\pgfsetrectcap%
\pgfsetroundjoin%
\pgfsetlinewidth{1.505625pt}%
\pgfsetstrokecolor{currentstroke1}%
\pgfsetdash{}{0pt}%
\pgfpathmoveto{\pgfqpoint{0.829965in}{0.626408in}}%
\pgfpathlineto{\pgfqpoint{0.928568in}{0.663634in}}%
\pgfpathlineto{\pgfqpoint{1.027172in}{0.740388in}}%
\pgfpathlineto{\pgfqpoint{1.125775in}{0.818698in}}%
\pgfpathlineto{\pgfqpoint{1.224378in}{0.904741in}}%
\pgfpathlineto{\pgfqpoint{1.322981in}{0.946302in}}%
\pgfpathlineto{\pgfqpoint{1.421585in}{1.020460in}}%
\pgfpathlineto{\pgfqpoint{1.520188in}{1.060715in}}%
\pgfpathlineto{\pgfqpoint{1.618791in}{1.081031in}}%
\pgfpathlineto{\pgfqpoint{1.717394in}{1.174417in}}%
\pgfpathlineto{\pgfqpoint{1.815998in}{1.234708in}}%
\pgfpathlineto{\pgfqpoint{1.914601in}{1.255790in}}%
\pgfpathlineto{\pgfqpoint{2.013204in}{1.277319in}}%
\pgfpathlineto{\pgfqpoint{2.111807in}{1.330157in}}%
\pgfpathlineto{\pgfqpoint{2.210411in}{1.351290in}}%
\pgfpathlineto{\pgfqpoint{2.309014in}{1.397918in}}%
\pgfpathlineto{\pgfqpoint{2.407617in}{1.401550in}}%
\pgfpathlineto{\pgfqpoint{2.506220in}{1.449273in}}%
\pgfpathlineto{\pgfqpoint{2.604824in}{1.476828in}}%
\pgfpathlineto{\pgfqpoint{2.703427in}{1.497723in}}%
\pgfpathlineto{\pgfqpoint{2.802030in}{1.546204in}}%
\pgfpathlineto{\pgfqpoint{2.900633in}{1.557320in}}%
\pgfpathlineto{\pgfqpoint{2.999237in}{1.579026in}}%
\pgfpathlineto{\pgfqpoint{3.097840in}{1.613709in}}%
\pgfpathlineto{\pgfqpoint{3.196443in}{1.637204in}}%
\pgfpathlineto{\pgfqpoint{3.295046in}{1.649717in}}%
\pgfpathlineto{\pgfqpoint{3.393649in}{1.675271in}}%
\pgfpathlineto{\pgfqpoint{3.492253in}{1.696779in}}%
\pgfpathlineto{\pgfqpoint{3.590856in}{1.709667in}}%
\pgfpathlineto{\pgfqpoint{3.689459in}{1.729461in}}%
\pgfpathlineto{\pgfqpoint{3.788062in}{1.748462in}}%
\pgfpathlineto{\pgfqpoint{3.886666in}{1.774190in}}%
\pgfpathlineto{\pgfqpoint{3.985269in}{1.759732in}}%
\pgfpathlineto{\pgfqpoint{4.083872in}{1.832776in}}%
\pgfpathlineto{\pgfqpoint{4.182475in}{1.848073in}}%
\pgfpathlineto{\pgfqpoint{4.281079in}{1.858912in}}%
\pgfpathlineto{\pgfqpoint{4.379682in}{1.853846in}}%
\pgfpathlineto{\pgfqpoint{4.478285in}{1.872201in}}%
\pgfpathlineto{\pgfqpoint{4.576888in}{1.891674in}}%
\pgfpathlineto{\pgfqpoint{4.675492in}{1.890983in}}%
\pgfpathlineto{\pgfqpoint{4.774095in}{1.918328in}}%
\pgfpathlineto{\pgfqpoint{4.872698in}{1.900798in}}%
\pgfpathlineto{\pgfqpoint{4.971301in}{1.943597in}}%
\pgfpathlineto{\pgfqpoint{5.069905in}{1.999367in}}%
\pgfpathlineto{\pgfqpoint{5.168508in}{1.981514in}}%
\pgfpathlineto{\pgfqpoint{5.267111in}{2.003339in}}%
\pgfpathlineto{\pgfqpoint{5.365714in}{2.005111in}}%
\pgfpathlineto{\pgfqpoint{5.464318in}{2.018770in}}%
\pgfpathlineto{\pgfqpoint{5.562921in}{2.036032in}}%
\pgfpathlineto{\pgfqpoint{5.661524in}{2.026041in}}%
\pgfusepath{stroke}%
\end{pgfscope}%
\begin{pgfscope}%
\pgfpathrectangle{\pgfqpoint{0.588387in}{0.521603in}}{\pgfqpoint{5.314715in}{2.220246in}}%
\pgfusepath{clip}%
\pgfsetrectcap%
\pgfsetroundjoin%
\pgfsetlinewidth{1.505625pt}%
\pgfsetstrokecolor{currentstroke2}%
\pgfsetdash{}{0pt}%
\pgfpathmoveto{\pgfqpoint{0.829965in}{0.671145in}}%
\pgfpathlineto{\pgfqpoint{0.928568in}{0.751709in}}%
\pgfpathlineto{\pgfqpoint{1.027172in}{0.908555in}}%
\pgfpathlineto{\pgfqpoint{1.125775in}{1.055862in}}%
\pgfpathlineto{\pgfqpoint{1.224378in}{1.257372in}}%
\pgfpathlineto{\pgfqpoint{1.322981in}{1.424800in}}%
\pgfpathlineto{\pgfqpoint{1.421585in}{1.614844in}}%
\pgfpathlineto{\pgfqpoint{1.520188in}{1.819775in}}%
\pgfpathlineto{\pgfqpoint{1.618791in}{1.863808in}}%
\pgfpathlineto{\pgfqpoint{1.717394in}{2.167446in}}%
\pgfpathlineto{\pgfqpoint{1.815998in}{2.430536in}}%
\pgfpathlineto{\pgfqpoint{1.914601in}{2.468037in}}%
\pgfpathlineto{\pgfqpoint{2.013204in}{2.523646in}}%
\pgfpathlineto{\pgfqpoint{2.111807in}{2.553829in}}%
\pgfpathlineto{\pgfqpoint{2.210411in}{2.527250in}}%
\pgfpathlineto{\pgfqpoint{2.309014in}{2.570971in}}%
\pgfpathlineto{\pgfqpoint{2.407617in}{2.476259in}}%
\pgfpathlineto{\pgfqpoint{2.506220in}{2.560489in}}%
\pgfpathlineto{\pgfqpoint{2.604824in}{2.572703in}}%
\pgfpathlineto{\pgfqpoint{2.703427in}{2.527577in}}%
\pgfpathlineto{\pgfqpoint{2.999237in}{2.640929in}}%
\pgfusepath{stroke}%
\end{pgfscope}%
\begin{pgfscope}%
\pgfpathrectangle{\pgfqpoint{0.588387in}{0.521603in}}{\pgfqpoint{5.314715in}{2.220246in}}%
\pgfusepath{clip}%
\pgfsetrectcap%
\pgfsetroundjoin%
\pgfsetlinewidth{1.505625pt}%
\pgfsetstrokecolor{currentstroke3}%
\pgfsetdash{}{0pt}%
\pgfpathmoveto{\pgfqpoint{0.829965in}{0.622524in}}%
\pgfpathlineto{\pgfqpoint{0.928568in}{0.666385in}}%
\pgfpathlineto{\pgfqpoint{1.027172in}{0.784012in}}%
\pgfpathlineto{\pgfqpoint{1.125775in}{0.840506in}}%
\pgfpathlineto{\pgfqpoint{1.224378in}{0.956586in}}%
\pgfpathlineto{\pgfqpoint{1.322981in}{0.985045in}}%
\pgfpathlineto{\pgfqpoint{1.421585in}{1.070168in}}%
\pgfpathlineto{\pgfqpoint{1.520188in}{1.132427in}}%
\pgfpathlineto{\pgfqpoint{1.618791in}{1.231964in}}%
\pgfpathlineto{\pgfqpoint{1.717394in}{1.418635in}}%
\pgfpathlineto{\pgfqpoint{1.815998in}{1.332017in}}%
\pgfpathlineto{\pgfqpoint{1.914601in}{1.539678in}}%
\pgfpathlineto{\pgfqpoint{2.013204in}{1.654959in}}%
\pgfpathlineto{\pgfqpoint{2.111807in}{1.718552in}}%
\pgfpathlineto{\pgfqpoint{2.210411in}{1.808193in}}%
\pgfpathlineto{\pgfqpoint{2.309014in}{1.734092in}}%
\pgfpathlineto{\pgfqpoint{2.407617in}{1.773501in}}%
\pgfpathlineto{\pgfqpoint{2.506220in}{1.587326in}}%
\pgfpathlineto{\pgfqpoint{2.604824in}{1.646732in}}%
\pgfpathlineto{\pgfqpoint{2.703427in}{1.653286in}}%
\pgfpathlineto{\pgfqpoint{2.802030in}{1.939317in}}%
\pgfpathlineto{\pgfqpoint{2.900633in}{1.758605in}}%
\pgfpathlineto{\pgfqpoint{2.999237in}{1.936996in}}%
\pgfpathlineto{\pgfqpoint{3.097840in}{1.691868in}}%
\pgfpathlineto{\pgfqpoint{3.196443in}{1.887628in}}%
\pgfpathlineto{\pgfqpoint{3.295046in}{1.982510in}}%
\pgfpathlineto{\pgfqpoint{3.393649in}{2.006874in}}%
\pgfpathlineto{\pgfqpoint{3.492253in}{1.938567in}}%
\pgfpathlineto{\pgfqpoint{3.590856in}{2.124048in}}%
\pgfpathlineto{\pgfqpoint{3.689459in}{1.948924in}}%
\pgfpathlineto{\pgfqpoint{3.788062in}{2.082253in}}%
\pgfpathlineto{\pgfqpoint{3.886666in}{1.890493in}}%
\pgfpathlineto{\pgfqpoint{3.985269in}{2.105512in}}%
\pgfpathlineto{\pgfqpoint{4.083872in}{2.022519in}}%
\pgfpathlineto{\pgfqpoint{4.182475in}{2.116058in}}%
\pgfpathlineto{\pgfqpoint{4.281079in}{2.166691in}}%
\pgfpathlineto{\pgfqpoint{4.379682in}{2.209062in}}%
\pgfpathlineto{\pgfqpoint{4.478285in}{2.101680in}}%
\pgfpathlineto{\pgfqpoint{4.576888in}{2.064071in}}%
\pgfpathlineto{\pgfqpoint{4.675492in}{2.091233in}}%
\pgfpathlineto{\pgfqpoint{4.774095in}{2.058854in}}%
\pgfpathlineto{\pgfqpoint{4.872698in}{1.980458in}}%
\pgfpathlineto{\pgfqpoint{4.971301in}{2.008433in}}%
\pgfpathlineto{\pgfqpoint{5.069905in}{2.076302in}}%
\pgfpathlineto{\pgfqpoint{5.168508in}{2.014161in}}%
\pgfpathlineto{\pgfqpoint{5.267111in}{2.156613in}}%
\pgfpathlineto{\pgfqpoint{5.365714in}{2.168363in}}%
\pgfpathlineto{\pgfqpoint{5.464318in}{2.281558in}}%
\pgfpathlineto{\pgfqpoint{5.562921in}{2.173138in}}%
\pgfpathlineto{\pgfqpoint{5.661524in}{2.221182in}}%
\pgfusepath{stroke}%
\end{pgfscope}%
\begin{pgfscope}%
\pgfsetrectcap%
\pgfsetmiterjoin%
\pgfsetlinewidth{0.803000pt}%
\definecolor{currentstroke}{rgb}{0.000000,0.000000,0.000000}%
\pgfsetstrokecolor{currentstroke}%
\pgfsetdash{}{0pt}%
\pgfpathmoveto{\pgfqpoint{0.588387in}{0.521603in}}%
\pgfpathlineto{\pgfqpoint{0.588387in}{2.741849in}}%
\pgfusepath{stroke}%
\end{pgfscope}%
\begin{pgfscope}%
\pgfsetrectcap%
\pgfsetmiterjoin%
\pgfsetlinewidth{0.803000pt}%
\definecolor{currentstroke}{rgb}{0.000000,0.000000,0.000000}%
\pgfsetstrokecolor{currentstroke}%
\pgfsetdash{}{0pt}%
\pgfpathmoveto{\pgfqpoint{5.903102in}{0.521603in}}%
\pgfpathlineto{\pgfqpoint{5.903102in}{2.741849in}}%
\pgfusepath{stroke}%
\end{pgfscope}%
\begin{pgfscope}%
\pgfsetrectcap%
\pgfsetmiterjoin%
\pgfsetlinewidth{0.803000pt}%
\definecolor{currentstroke}{rgb}{0.000000,0.000000,0.000000}%
\pgfsetstrokecolor{currentstroke}%
\pgfsetdash{}{0pt}%
\pgfpathmoveto{\pgfqpoint{0.588387in}{0.521603in}}%
\pgfpathlineto{\pgfqpoint{5.903102in}{0.521603in}}%
\pgfusepath{stroke}%
\end{pgfscope}%
\begin{pgfscope}%
\pgfsetrectcap%
\pgfsetmiterjoin%
\pgfsetlinewidth{0.803000pt}%
\definecolor{currentstroke}{rgb}{0.000000,0.000000,0.000000}%
\pgfsetstrokecolor{currentstroke}%
\pgfsetdash{}{0pt}%
\pgfpathmoveto{\pgfqpoint{0.588387in}{2.741849in}}%
\pgfpathlineto{\pgfqpoint{5.903102in}{2.741849in}}%
\pgfusepath{stroke}%
\end{pgfscope}%
\begin{pgfscope}%
\pgfsetbuttcap%
\pgfsetmiterjoin%
\definecolor{currentfill}{rgb}{1.000000,1.000000,1.000000}%
\pgfsetfillcolor{currentfill}%
\pgfsetfillopacity{0.800000}%
\pgfsetlinewidth{1.003750pt}%
\definecolor{currentstroke}{rgb}{0.800000,0.800000,0.800000}%
\pgfsetstrokecolor{currentstroke}%
\pgfsetstrokeopacity{0.800000}%
\pgfsetdash{}{0pt}%
\pgfpathmoveto{\pgfqpoint{5.990602in}{2.084477in}}%
\pgfpathlineto{\pgfqpoint{8.259376in}{2.084477in}}%
\pgfpathquadraticcurveto{\pgfqpoint{8.284376in}{2.084477in}}{\pgfqpoint{8.284376in}{2.109477in}}%
\pgfpathlineto{\pgfqpoint{8.284376in}{2.654349in}}%
\pgfpathquadraticcurveto{\pgfqpoint{8.284376in}{2.679349in}}{\pgfqpoint{8.259376in}{2.679349in}}%
\pgfpathlineto{\pgfqpoint{5.990602in}{2.679349in}}%
\pgfpathquadraticcurveto{\pgfqpoint{5.965602in}{2.679349in}}{\pgfqpoint{5.965602in}{2.654349in}}%
\pgfpathlineto{\pgfqpoint{5.965602in}{2.109477in}}%
\pgfpathquadraticcurveto{\pgfqpoint{5.965602in}{2.084477in}}{\pgfqpoint{5.990602in}{2.084477in}}%
\pgfpathlineto{\pgfqpoint{5.990602in}{2.084477in}}%
\pgfpathclose%
\pgfusepath{stroke,fill}%
\end{pgfscope}%
\begin{pgfscope}%
\pgfsetrectcap%
\pgfsetroundjoin%
\pgfsetlinewidth{1.505625pt}%
\pgfsetstrokecolor{currentstroke1}%
\pgfsetdash{}{0pt}%
\pgfpathmoveto{\pgfqpoint{6.015602in}{2.578129in}}%
\pgfpathlineto{\pgfqpoint{6.140602in}{2.578129in}}%
\pgfpathlineto{\pgfqpoint{6.265602in}{2.578129in}}%
\pgfusepath{stroke}%
\end{pgfscope}%
\begin{pgfscope}%
\definecolor{textcolor}{rgb}{0.000000,0.000000,0.000000}%
\pgfsetstrokecolor{textcolor}%
\pgfsetfillcolor{textcolor}%
\pgftext[x=6.365602in,y=2.534379in,left,base]{\color{textcolor}{\rmfamily\fontsize{9.000000}{10.800000}\selectfont\catcode`\^=\active\def^{\ifmmode\sp\else\^{}\fi}\catcode`\%=\active\def%{\%}\Neighbors{} \& \MergeLinear{}}}%
\end{pgfscope}%
\begin{pgfscope}%
\pgfsetrectcap%
\pgfsetroundjoin%
\pgfsetlinewidth{1.505625pt}%
\pgfsetstrokecolor{currentstroke2}%
\pgfsetdash{}{0pt}%
\pgfpathmoveto{\pgfqpoint{6.015602in}{2.394657in}}%
\pgfpathlineto{\pgfqpoint{6.140602in}{2.394657in}}%
\pgfpathlineto{\pgfqpoint{6.265602in}{2.394657in}}%
\pgfusepath{stroke}%
\end{pgfscope}%
\begin{pgfscope}%
\definecolor{textcolor}{rgb}{0.000000,0.000000,0.000000}%
\pgfsetstrokecolor{textcolor}%
\pgfsetfillcolor{textcolor}%
\pgftext[x=6.365602in,y=2.350907in,left,base]{\color{textcolor}{\rmfamily\fontsize{9.000000}{10.800000}\selectfont\catcode`\^=\active\def^{\ifmmode\sp\else\^{}\fi}\catcode`\%=\active\def%{\%}\Neighbors{} \& \PromisingCycles{}}}%
\end{pgfscope}%
\begin{pgfscope}%
\pgfsetrectcap%
\pgfsetroundjoin%
\pgfsetlinewidth{1.505625pt}%
\pgfsetstrokecolor{currentstroke3}%
\pgfsetdash{}{0pt}%
\pgfpathmoveto{\pgfqpoint{6.015602in}{2.207707in}}%
\pgfpathlineto{\pgfqpoint{6.140602in}{2.207707in}}%
\pgfpathlineto{\pgfqpoint{6.265602in}{2.207707in}}%
\pgfusepath{stroke}%
\end{pgfscope}%
\begin{pgfscope}%
\definecolor{textcolor}{rgb}{0.000000,0.000000,0.000000}%
\pgfsetstrokecolor{textcolor}%
\pgfsetfillcolor{textcolor}%
\pgftext[x=6.365602in,y=2.163957in,left,base]{\color{textcolor}{\rmfamily\fontsize{9.000000}{10.800000}\selectfont\catcode`\^=\active\def^{\ifmmode\sp\else\^{}\fi}\catcode`\%=\active\def%{\%}\Neighbors{} \& \SharedVertices{}}}%
\end{pgfscope}%
\end{pgfpicture}%
\makeatother%
\endgroup%
}
	\caption[Failing merging strategies for minimally rigid graphs]{
		Mean running time to find all NAC-colorings for minimally rigid graphs with failing merging strategies.}%
	\label{fig:graph_mimimally_rigid_failing_merging_first_runtime}
\end{figure}%
\begin{figure}[thbp]
	\centering
	\scalebox{\BenchFigureScale}{%% Creator: Matplotlib, PGF backend
%%
%% To include the figure in your LaTeX document, write
%%   \input{<filename>.pgf}
%%
%% Make sure the required packages are loaded in your preamble
%%   \usepackage{pgf}
%%
%% Also ensure that all the required font packages are loaded; for instance,
%% the lmodern package is sometimes necessary when using math font.
%%   \usepackage{lmodern}
%%
%% Figures using additional raster images can only be included by \input if
%% they are in the same directory as the main LaTeX file. For loading figures
%% from other directories you can use the `import` package
%%   \usepackage{import}
%%
%% and then include the figures with
%%   \import{<path to file>}{<filename>.pgf}
%%
%% Matplotlib used the following preamble
%%   \def\mathdefault#1{#1}
%%   \everymath=\expandafter{\the\everymath\displaystyle}
%%   \IfFileExists{scrextend.sty}{
%%     \usepackage[fontsize=10.000000pt]{scrextend}
%%   }{
%%     \renewcommand{\normalsize}{\fontsize{10.000000}{12.000000}\selectfont}
%%     \normalsize
%%   }
%%   
%%   \ifdefined\pdftexversion\else  % non-pdftex case.
%%     \usepackage{fontspec}
%%     \setmainfont{DejaVuSans.ttf}[Path=\detokenize{/home/petr/Projects/PyRigi/.venv/lib/python3.12/site-packages/matplotlib/mpl-data/fonts/ttf/}]
%%     \setsansfont{DejaVuSans.ttf}[Path=\detokenize{/home/petr/Projects/PyRigi/.venv/lib/python3.12/site-packages/matplotlib/mpl-data/fonts/ttf/}]
%%     \setmonofont{DejaVuSansMono.ttf}[Path=\detokenize{/home/petr/Projects/PyRigi/.venv/lib/python3.12/site-packages/matplotlib/mpl-data/fonts/ttf/}]
%%   \fi
%%   \makeatletter\@ifpackageloaded{under\Score{}}{}{\usepackage[strings]{under\Score{}}}\makeatother
%%
\begingroup%
\makeatletter%
\begin{pgfpicture}%
\pgfpathrectangle{\pgfpointorigin}{\pgfqpoint{8.384376in}{2.841849in}}%
\pgfusepath{use as bounding box, clip}%
\begin{pgfscope}%
\pgfsetbuttcap%
\pgfsetmiterjoin%
\definecolor{currentfill}{rgb}{1.000000,1.000000,1.000000}%
\pgfsetfillcolor{currentfill}%
\pgfsetlinewidth{0.000000pt}%
\definecolor{currentstroke}{rgb}{1.000000,1.000000,1.000000}%
\pgfsetstrokecolor{currentstroke}%
\pgfsetdash{}{0pt}%
\pgfpathmoveto{\pgfqpoint{0.000000in}{0.000000in}}%
\pgfpathlineto{\pgfqpoint{8.384376in}{0.000000in}}%
\pgfpathlineto{\pgfqpoint{8.384376in}{2.841849in}}%
\pgfpathlineto{\pgfqpoint{0.000000in}{2.841849in}}%
\pgfpathlineto{\pgfqpoint{0.000000in}{0.000000in}}%
\pgfpathclose%
\pgfusepath{fill}%
\end{pgfscope}%
\begin{pgfscope}%
\pgfsetbuttcap%
\pgfsetmiterjoin%
\definecolor{currentfill}{rgb}{1.000000,1.000000,1.000000}%
\pgfsetfillcolor{currentfill}%
\pgfsetlinewidth{0.000000pt}%
\definecolor{currentstroke}{rgb}{0.000000,0.000000,0.000000}%
\pgfsetstrokecolor{currentstroke}%
\pgfsetstrokeopacity{0.000000}%
\pgfsetdash{}{0pt}%
\pgfpathmoveto{\pgfqpoint{0.588387in}{0.521603in}}%
\pgfpathlineto{\pgfqpoint{5.090464in}{0.521603in}}%
\pgfpathlineto{\pgfqpoint{5.090464in}{2.741849in}}%
\pgfpathlineto{\pgfqpoint{0.588387in}{2.741849in}}%
\pgfpathlineto{\pgfqpoint{0.588387in}{0.521603in}}%
\pgfpathclose%
\pgfusepath{fill}%
\end{pgfscope}%
\begin{pgfscope}%
\pgfsetbuttcap%
\pgfsetroundjoin%
\definecolor{currentfill}{rgb}{0.000000,0.000000,0.000000}%
\pgfsetfillcolor{currentfill}%
\pgfsetlinewidth{0.803000pt}%
\definecolor{currentstroke}{rgb}{0.000000,0.000000,0.000000}%
\pgfsetstrokecolor{currentstroke}%
\pgfsetdash{}{0pt}%
\pgfsys@defobject{currentmarker}{\pgfqpoint{0.000000in}{-0.048611in}}{\pgfqpoint{0.000000in}{0.000000in}}{%
\pgfpathmoveto{\pgfqpoint{0.000000in}{0.000000in}}%
\pgfpathlineto{\pgfqpoint{0.000000in}{-0.048611in}}%
\pgfusepath{stroke,fill}%
}%
\begin{pgfscope}%
\pgfsys@transformshift{0.960080in}{0.521603in}%
\pgfsys@useobject{currentmarker}{}%
\end{pgfscope}%
\end{pgfscope}%
\begin{pgfscope}%
\definecolor{textcolor}{rgb}{0.000000,0.000000,0.000000}%
\pgfsetstrokecolor{textcolor}%
\pgfsetfillcolor{textcolor}%
\pgftext[x=0.960080in,y=0.424381in,,top]{\color{textcolor}{\rmfamily\fontsize{10.000000}{12.000000}\selectfont\catcode`\^=\active\def^{\ifmmode\sp\else\^{}\fi}\catcode`\%=\active\def%{\%}$\mathdefault{12}$}}%
\end{pgfscope}%
\begin{pgfscope}%
\pgfsetbuttcap%
\pgfsetroundjoin%
\definecolor{currentfill}{rgb}{0.000000,0.000000,0.000000}%
\pgfsetfillcolor{currentfill}%
\pgfsetlinewidth{0.803000pt}%
\definecolor{currentstroke}{rgb}{0.000000,0.000000,0.000000}%
\pgfsetstrokecolor{currentstroke}%
\pgfsetdash{}{0pt}%
\pgfsys@defobject{currentmarker}{\pgfqpoint{0.000000in}{-0.048611in}}{\pgfqpoint{0.000000in}{0.000000in}}{%
\pgfpathmoveto{\pgfqpoint{0.000000in}{0.000000in}}%
\pgfpathlineto{\pgfqpoint{0.000000in}{-0.048611in}}%
\pgfusepath{stroke,fill}%
}%
\begin{pgfscope}%
\pgfsys@transformshift{1.461239in}{0.521603in}%
\pgfsys@useobject{currentmarker}{}%
\end{pgfscope}%
\end{pgfscope}%
\begin{pgfscope}%
\definecolor{textcolor}{rgb}{0.000000,0.000000,0.000000}%
\pgfsetstrokecolor{textcolor}%
\pgfsetfillcolor{textcolor}%
\pgftext[x=1.461239in,y=0.424381in,,top]{\color{textcolor}{\rmfamily\fontsize{10.000000}{12.000000}\selectfont\catcode`\^=\active\def^{\ifmmode\sp\else\^{}\fi}\catcode`\%=\active\def%{\%}$\mathdefault{18}$}}%
\end{pgfscope}%
\begin{pgfscope}%
\pgfsetbuttcap%
\pgfsetroundjoin%
\definecolor{currentfill}{rgb}{0.000000,0.000000,0.000000}%
\pgfsetfillcolor{currentfill}%
\pgfsetlinewidth{0.803000pt}%
\definecolor{currentstroke}{rgb}{0.000000,0.000000,0.000000}%
\pgfsetstrokecolor{currentstroke}%
\pgfsetdash{}{0pt}%
\pgfsys@defobject{currentmarker}{\pgfqpoint{0.000000in}{-0.048611in}}{\pgfqpoint{0.000000in}{0.000000in}}{%
\pgfpathmoveto{\pgfqpoint{0.000000in}{0.000000in}}%
\pgfpathlineto{\pgfqpoint{0.000000in}{-0.048611in}}%
\pgfusepath{stroke,fill}%
}%
\begin{pgfscope}%
\pgfsys@transformshift{1.962398in}{0.521603in}%
\pgfsys@useobject{currentmarker}{}%
\end{pgfscope}%
\end{pgfscope}%
\begin{pgfscope}%
\definecolor{textcolor}{rgb}{0.000000,0.000000,0.000000}%
\pgfsetstrokecolor{textcolor}%
\pgfsetfillcolor{textcolor}%
\pgftext[x=1.962398in,y=0.424381in,,top]{\color{textcolor}{\rmfamily\fontsize{10.000000}{12.000000}\selectfont\catcode`\^=\active\def^{\ifmmode\sp\else\^{}\fi}\catcode`\%=\active\def%{\%}$\mathdefault{24}$}}%
\end{pgfscope}%
\begin{pgfscope}%
\pgfsetbuttcap%
\pgfsetroundjoin%
\definecolor{currentfill}{rgb}{0.000000,0.000000,0.000000}%
\pgfsetfillcolor{currentfill}%
\pgfsetlinewidth{0.803000pt}%
\definecolor{currentstroke}{rgb}{0.000000,0.000000,0.000000}%
\pgfsetstrokecolor{currentstroke}%
\pgfsetdash{}{0pt}%
\pgfsys@defobject{currentmarker}{\pgfqpoint{0.000000in}{-0.048611in}}{\pgfqpoint{0.000000in}{0.000000in}}{%
\pgfpathmoveto{\pgfqpoint{0.000000in}{0.000000in}}%
\pgfpathlineto{\pgfqpoint{0.000000in}{-0.048611in}}%
\pgfusepath{stroke,fill}%
}%
\begin{pgfscope}%
\pgfsys@transformshift{2.463556in}{0.521603in}%
\pgfsys@useobject{currentmarker}{}%
\end{pgfscope}%
\end{pgfscope}%
\begin{pgfscope}%
\definecolor{textcolor}{rgb}{0.000000,0.000000,0.000000}%
\pgfsetstrokecolor{textcolor}%
\pgfsetfillcolor{textcolor}%
\pgftext[x=2.463556in,y=0.424381in,,top]{\color{textcolor}{\rmfamily\fontsize{10.000000}{12.000000}\selectfont\catcode`\^=\active\def^{\ifmmode\sp\else\^{}\fi}\catcode`\%=\active\def%{\%}$\mathdefault{30}$}}%
\end{pgfscope}%
\begin{pgfscope}%
\pgfsetbuttcap%
\pgfsetroundjoin%
\definecolor{currentfill}{rgb}{0.000000,0.000000,0.000000}%
\pgfsetfillcolor{currentfill}%
\pgfsetlinewidth{0.803000pt}%
\definecolor{currentstroke}{rgb}{0.000000,0.000000,0.000000}%
\pgfsetstrokecolor{currentstroke}%
\pgfsetdash{}{0pt}%
\pgfsys@defobject{currentmarker}{\pgfqpoint{0.000000in}{-0.048611in}}{\pgfqpoint{0.000000in}{0.000000in}}{%
\pgfpathmoveto{\pgfqpoint{0.000000in}{0.000000in}}%
\pgfpathlineto{\pgfqpoint{0.000000in}{-0.048611in}}%
\pgfusepath{stroke,fill}%
}%
\begin{pgfscope}%
\pgfsys@transformshift{2.964715in}{0.521603in}%
\pgfsys@useobject{currentmarker}{}%
\end{pgfscope}%
\end{pgfscope}%
\begin{pgfscope}%
\definecolor{textcolor}{rgb}{0.000000,0.000000,0.000000}%
\pgfsetstrokecolor{textcolor}%
\pgfsetfillcolor{textcolor}%
\pgftext[x=2.964715in,y=0.424381in,,top]{\color{textcolor}{\rmfamily\fontsize{10.000000}{12.000000}\selectfont\catcode`\^=\active\def^{\ifmmode\sp\else\^{}\fi}\catcode`\%=\active\def%{\%}$\mathdefault{36}$}}%
\end{pgfscope}%
\begin{pgfscope}%
\pgfsetbuttcap%
\pgfsetroundjoin%
\definecolor{currentfill}{rgb}{0.000000,0.000000,0.000000}%
\pgfsetfillcolor{currentfill}%
\pgfsetlinewidth{0.803000pt}%
\definecolor{currentstroke}{rgb}{0.000000,0.000000,0.000000}%
\pgfsetstrokecolor{currentstroke}%
\pgfsetdash{}{0pt}%
\pgfsys@defobject{currentmarker}{\pgfqpoint{0.000000in}{-0.048611in}}{\pgfqpoint{0.000000in}{0.000000in}}{%
\pgfpathmoveto{\pgfqpoint{0.000000in}{0.000000in}}%
\pgfpathlineto{\pgfqpoint{0.000000in}{-0.048611in}}%
\pgfusepath{stroke,fill}%
}%
\begin{pgfscope}%
\pgfsys@transformshift{3.465874in}{0.521603in}%
\pgfsys@useobject{currentmarker}{}%
\end{pgfscope}%
\end{pgfscope}%
\begin{pgfscope}%
\definecolor{textcolor}{rgb}{0.000000,0.000000,0.000000}%
\pgfsetstrokecolor{textcolor}%
\pgfsetfillcolor{textcolor}%
\pgftext[x=3.465874in,y=0.424381in,,top]{\color{textcolor}{\rmfamily\fontsize{10.000000}{12.000000}\selectfont\catcode`\^=\active\def^{\ifmmode\sp\else\^{}\fi}\catcode`\%=\active\def%{\%}$\mathdefault{42}$}}%
\end{pgfscope}%
\begin{pgfscope}%
\pgfsetbuttcap%
\pgfsetroundjoin%
\definecolor{currentfill}{rgb}{0.000000,0.000000,0.000000}%
\pgfsetfillcolor{currentfill}%
\pgfsetlinewidth{0.803000pt}%
\definecolor{currentstroke}{rgb}{0.000000,0.000000,0.000000}%
\pgfsetstrokecolor{currentstroke}%
\pgfsetdash{}{0pt}%
\pgfsys@defobject{currentmarker}{\pgfqpoint{0.000000in}{-0.048611in}}{\pgfqpoint{0.000000in}{0.000000in}}{%
\pgfpathmoveto{\pgfqpoint{0.000000in}{0.000000in}}%
\pgfpathlineto{\pgfqpoint{0.000000in}{-0.048611in}}%
\pgfusepath{stroke,fill}%
}%
\begin{pgfscope}%
\pgfsys@transformshift{3.967033in}{0.521603in}%
\pgfsys@useobject{currentmarker}{}%
\end{pgfscope}%
\end{pgfscope}%
\begin{pgfscope}%
\definecolor{textcolor}{rgb}{0.000000,0.000000,0.000000}%
\pgfsetstrokecolor{textcolor}%
\pgfsetfillcolor{textcolor}%
\pgftext[x=3.967033in,y=0.424381in,,top]{\color{textcolor}{\rmfamily\fontsize{10.000000}{12.000000}\selectfont\catcode`\^=\active\def^{\ifmmode\sp\else\^{}\fi}\catcode`\%=\active\def%{\%}$\mathdefault{48}$}}%
\end{pgfscope}%
\begin{pgfscope}%
\pgfsetbuttcap%
\pgfsetroundjoin%
\definecolor{currentfill}{rgb}{0.000000,0.000000,0.000000}%
\pgfsetfillcolor{currentfill}%
\pgfsetlinewidth{0.803000pt}%
\definecolor{currentstroke}{rgb}{0.000000,0.000000,0.000000}%
\pgfsetstrokecolor{currentstroke}%
\pgfsetdash{}{0pt}%
\pgfsys@defobject{currentmarker}{\pgfqpoint{0.000000in}{-0.048611in}}{\pgfqpoint{0.000000in}{0.000000in}}{%
\pgfpathmoveto{\pgfqpoint{0.000000in}{0.000000in}}%
\pgfpathlineto{\pgfqpoint{0.000000in}{-0.048611in}}%
\pgfusepath{stroke,fill}%
}%
\begin{pgfscope}%
\pgfsys@transformshift{4.468192in}{0.521603in}%
\pgfsys@useobject{currentmarker}{}%
\end{pgfscope}%
\end{pgfscope}%
\begin{pgfscope}%
\definecolor{textcolor}{rgb}{0.000000,0.000000,0.000000}%
\pgfsetstrokecolor{textcolor}%
\pgfsetfillcolor{textcolor}%
\pgftext[x=4.468192in,y=0.424381in,,top]{\color{textcolor}{\rmfamily\fontsize{10.000000}{12.000000}\selectfont\catcode`\^=\active\def^{\ifmmode\sp\else\^{}\fi}\catcode`\%=\active\def%{\%}$\mathdefault{54}$}}%
\end{pgfscope}%
\begin{pgfscope}%
\pgfsetbuttcap%
\pgfsetroundjoin%
\definecolor{currentfill}{rgb}{0.000000,0.000000,0.000000}%
\pgfsetfillcolor{currentfill}%
\pgfsetlinewidth{0.803000pt}%
\definecolor{currentstroke}{rgb}{0.000000,0.000000,0.000000}%
\pgfsetstrokecolor{currentstroke}%
\pgfsetdash{}{0pt}%
\pgfsys@defobject{currentmarker}{\pgfqpoint{0.000000in}{-0.048611in}}{\pgfqpoint{0.000000in}{0.000000in}}{%
\pgfpathmoveto{\pgfqpoint{0.000000in}{0.000000in}}%
\pgfpathlineto{\pgfqpoint{0.000000in}{-0.048611in}}%
\pgfusepath{stroke,fill}%
}%
\begin{pgfscope}%
\pgfsys@transformshift{4.969350in}{0.521603in}%
\pgfsys@useobject{currentmarker}{}%
\end{pgfscope}%
\end{pgfscope}%
\begin{pgfscope}%
\definecolor{textcolor}{rgb}{0.000000,0.000000,0.000000}%
\pgfsetstrokecolor{textcolor}%
\pgfsetfillcolor{textcolor}%
\pgftext[x=4.969350in,y=0.424381in,,top]{\color{textcolor}{\rmfamily\fontsize{10.000000}{12.000000}\selectfont\catcode`\^=\active\def^{\ifmmode\sp\else\^{}\fi}\catcode`\%=\active\def%{\%}$\mathdefault{60}$}}%
\end{pgfscope}%
\begin{pgfscope}%
\definecolor{textcolor}{rgb}{0.000000,0.000000,0.000000}%
\pgfsetstrokecolor{textcolor}%
\pgfsetfillcolor{textcolor}%
\pgftext[x=2.839425in,y=0.234413in,,top]{\color{textcolor}{\rmfamily\fontsize{10.000000}{12.000000}\selectfont\catcode`\^=\active\def^{\ifmmode\sp\else\^{}\fi}\catcode`\%=\active\def%{\%}Vertices}}%
\end{pgfscope}%
\begin{pgfscope}%
\pgfsetbuttcap%
\pgfsetroundjoin%
\definecolor{currentfill}{rgb}{0.000000,0.000000,0.000000}%
\pgfsetfillcolor{currentfill}%
\pgfsetlinewidth{0.803000pt}%
\definecolor{currentstroke}{rgb}{0.000000,0.000000,0.000000}%
\pgfsetstrokecolor{currentstroke}%
\pgfsetdash{}{0pt}%
\pgfsys@defobject{currentmarker}{\pgfqpoint{-0.048611in}{0.000000in}}{\pgfqpoint{-0.000000in}{0.000000in}}{%
\pgfpathmoveto{\pgfqpoint{-0.000000in}{0.000000in}}%
\pgfpathlineto{\pgfqpoint{-0.048611in}{0.000000in}}%
\pgfusepath{stroke,fill}%
}%
\begin{pgfscope}%
\pgfsys@transformshift{0.588387in}{1.000513in}%
\pgfsys@useobject{currentmarker}{}%
\end{pgfscope}%
\end{pgfscope}%
\begin{pgfscope}%
\definecolor{textcolor}{rgb}{0.000000,0.000000,0.000000}%
\pgfsetstrokecolor{textcolor}%
\pgfsetfillcolor{textcolor}%
\pgftext[x=0.289968in, y=0.947752in, left, base]{\color{textcolor}{\rmfamily\fontsize{10.000000}{12.000000}\selectfont\catcode`\^=\active\def^{\ifmmode\sp\else\^{}\fi}\catcode`\%=\active\def%{\%}$\mathdefault{10^{1}}$}}%
\end{pgfscope}%
\begin{pgfscope}%
\pgfsetbuttcap%
\pgfsetroundjoin%
\definecolor{currentfill}{rgb}{0.000000,0.000000,0.000000}%
\pgfsetfillcolor{currentfill}%
\pgfsetlinewidth{0.803000pt}%
\definecolor{currentstroke}{rgb}{0.000000,0.000000,0.000000}%
\pgfsetstrokecolor{currentstroke}%
\pgfsetdash{}{0pt}%
\pgfsys@defobject{currentmarker}{\pgfqpoint{-0.048611in}{0.000000in}}{\pgfqpoint{-0.000000in}{0.000000in}}{%
\pgfpathmoveto{\pgfqpoint{-0.000000in}{0.000000in}}%
\pgfpathlineto{\pgfqpoint{-0.048611in}{0.000000in}}%
\pgfusepath{stroke,fill}%
}%
\begin{pgfscope}%
\pgfsys@transformshift{0.588387in}{1.821346in}%
\pgfsys@useobject{currentmarker}{}%
\end{pgfscope}%
\end{pgfscope}%
\begin{pgfscope}%
\definecolor{textcolor}{rgb}{0.000000,0.000000,0.000000}%
\pgfsetstrokecolor{textcolor}%
\pgfsetfillcolor{textcolor}%
\pgftext[x=0.289968in, y=1.768585in, left, base]{\color{textcolor}{\rmfamily\fontsize{10.000000}{12.000000}\selectfont\catcode`\^=\active\def^{\ifmmode\sp\else\^{}\fi}\catcode`\%=\active\def%{\%}$\mathdefault{10^{2}}$}}%
\end{pgfscope}%
\begin{pgfscope}%
\pgfsetbuttcap%
\pgfsetroundjoin%
\definecolor{currentfill}{rgb}{0.000000,0.000000,0.000000}%
\pgfsetfillcolor{currentfill}%
\pgfsetlinewidth{0.803000pt}%
\definecolor{currentstroke}{rgb}{0.000000,0.000000,0.000000}%
\pgfsetstrokecolor{currentstroke}%
\pgfsetdash{}{0pt}%
\pgfsys@defobject{currentmarker}{\pgfqpoint{-0.048611in}{0.000000in}}{\pgfqpoint{-0.000000in}{0.000000in}}{%
\pgfpathmoveto{\pgfqpoint{-0.000000in}{0.000000in}}%
\pgfpathlineto{\pgfqpoint{-0.048611in}{0.000000in}}%
\pgfusepath{stroke,fill}%
}%
\begin{pgfscope}%
\pgfsys@transformshift{0.588387in}{2.642179in}%
\pgfsys@useobject{currentmarker}{}%
\end{pgfscope}%
\end{pgfscope}%
\begin{pgfscope}%
\definecolor{textcolor}{rgb}{0.000000,0.000000,0.000000}%
\pgfsetstrokecolor{textcolor}%
\pgfsetfillcolor{textcolor}%
\pgftext[x=0.289968in, y=2.589417in, left, base]{\color{textcolor}{\rmfamily\fontsize{10.000000}{12.000000}\selectfont\catcode`\^=\active\def^{\ifmmode\sp\else\^{}\fi}\catcode`\%=\active\def%{\%}$\mathdefault{10^{3}}$}}%
\end{pgfscope}%
\begin{pgfscope}%
\pgfsetbuttcap%
\pgfsetroundjoin%
\definecolor{currentfill}{rgb}{0.000000,0.000000,0.000000}%
\pgfsetfillcolor{currentfill}%
\pgfsetlinewidth{0.602250pt}%
\definecolor{currentstroke}{rgb}{0.000000,0.000000,0.000000}%
\pgfsetstrokecolor{currentstroke}%
\pgfsetdash{}{0pt}%
\pgfsys@defobject{currentmarker}{\pgfqpoint{-0.027778in}{0.000000in}}{\pgfqpoint{-0.000000in}{0.000000in}}{%
\pgfpathmoveto{\pgfqpoint{-0.000000in}{0.000000in}}%
\pgfpathlineto{\pgfqpoint{-0.027778in}{0.000000in}}%
\pgfusepath{stroke,fill}%
}%
\begin{pgfscope}%
\pgfsys@transformshift{0.588387in}{0.571317in}%
\pgfsys@useobject{currentmarker}{}%
\end{pgfscope}%
\end{pgfscope}%
\begin{pgfscope}%
\pgfsetbuttcap%
\pgfsetroundjoin%
\definecolor{currentfill}{rgb}{0.000000,0.000000,0.000000}%
\pgfsetfillcolor{currentfill}%
\pgfsetlinewidth{0.602250pt}%
\definecolor{currentstroke}{rgb}{0.000000,0.000000,0.000000}%
\pgfsetstrokecolor{currentstroke}%
\pgfsetdash{}{0pt}%
\pgfsys@defobject{currentmarker}{\pgfqpoint{-0.027778in}{0.000000in}}{\pgfqpoint{-0.000000in}{0.000000in}}{%
\pgfpathmoveto{\pgfqpoint{-0.000000in}{0.000000in}}%
\pgfpathlineto{\pgfqpoint{-0.027778in}{0.000000in}}%
\pgfusepath{stroke,fill}%
}%
\begin{pgfscope}%
\pgfsys@transformshift{0.588387in}{0.673871in}%
\pgfsys@useobject{currentmarker}{}%
\end{pgfscope}%
\end{pgfscope}%
\begin{pgfscope}%
\pgfsetbuttcap%
\pgfsetroundjoin%
\definecolor{currentfill}{rgb}{0.000000,0.000000,0.000000}%
\pgfsetfillcolor{currentfill}%
\pgfsetlinewidth{0.602250pt}%
\definecolor{currentstroke}{rgb}{0.000000,0.000000,0.000000}%
\pgfsetstrokecolor{currentstroke}%
\pgfsetdash{}{0pt}%
\pgfsys@defobject{currentmarker}{\pgfqpoint{-0.027778in}{0.000000in}}{\pgfqpoint{-0.000000in}{0.000000in}}{%
\pgfpathmoveto{\pgfqpoint{-0.000000in}{0.000000in}}%
\pgfpathlineto{\pgfqpoint{-0.027778in}{0.000000in}}%
\pgfusepath{stroke,fill}%
}%
\begin{pgfscope}%
\pgfsys@transformshift{0.588387in}{0.753418in}%
\pgfsys@useobject{currentmarker}{}%
\end{pgfscope}%
\end{pgfscope}%
\begin{pgfscope}%
\pgfsetbuttcap%
\pgfsetroundjoin%
\definecolor{currentfill}{rgb}{0.000000,0.000000,0.000000}%
\pgfsetfillcolor{currentfill}%
\pgfsetlinewidth{0.602250pt}%
\definecolor{currentstroke}{rgb}{0.000000,0.000000,0.000000}%
\pgfsetstrokecolor{currentstroke}%
\pgfsetdash{}{0pt}%
\pgfsys@defobject{currentmarker}{\pgfqpoint{-0.027778in}{0.000000in}}{\pgfqpoint{-0.000000in}{0.000000in}}{%
\pgfpathmoveto{\pgfqpoint{-0.000000in}{0.000000in}}%
\pgfpathlineto{\pgfqpoint{-0.027778in}{0.000000in}}%
\pgfusepath{stroke,fill}%
}%
\begin{pgfscope}%
\pgfsys@transformshift{0.588387in}{0.818413in}%
\pgfsys@useobject{currentmarker}{}%
\end{pgfscope}%
\end{pgfscope}%
\begin{pgfscope}%
\pgfsetbuttcap%
\pgfsetroundjoin%
\definecolor{currentfill}{rgb}{0.000000,0.000000,0.000000}%
\pgfsetfillcolor{currentfill}%
\pgfsetlinewidth{0.602250pt}%
\definecolor{currentstroke}{rgb}{0.000000,0.000000,0.000000}%
\pgfsetstrokecolor{currentstroke}%
\pgfsetdash{}{0pt}%
\pgfsys@defobject{currentmarker}{\pgfqpoint{-0.027778in}{0.000000in}}{\pgfqpoint{-0.000000in}{0.000000in}}{%
\pgfpathmoveto{\pgfqpoint{-0.000000in}{0.000000in}}%
\pgfpathlineto{\pgfqpoint{-0.027778in}{0.000000in}}%
\pgfusepath{stroke,fill}%
}%
\begin{pgfscope}%
\pgfsys@transformshift{0.588387in}{0.873365in}%
\pgfsys@useobject{currentmarker}{}%
\end{pgfscope}%
\end{pgfscope}%
\begin{pgfscope}%
\pgfsetbuttcap%
\pgfsetroundjoin%
\definecolor{currentfill}{rgb}{0.000000,0.000000,0.000000}%
\pgfsetfillcolor{currentfill}%
\pgfsetlinewidth{0.602250pt}%
\definecolor{currentstroke}{rgb}{0.000000,0.000000,0.000000}%
\pgfsetstrokecolor{currentstroke}%
\pgfsetdash{}{0pt}%
\pgfsys@defobject{currentmarker}{\pgfqpoint{-0.027778in}{0.000000in}}{\pgfqpoint{-0.000000in}{0.000000in}}{%
\pgfpathmoveto{\pgfqpoint{-0.000000in}{0.000000in}}%
\pgfpathlineto{\pgfqpoint{-0.027778in}{0.000000in}}%
\pgfusepath{stroke,fill}%
}%
\begin{pgfscope}%
\pgfsys@transformshift{0.588387in}{0.920966in}%
\pgfsys@useobject{currentmarker}{}%
\end{pgfscope}%
\end{pgfscope}%
\begin{pgfscope}%
\pgfsetbuttcap%
\pgfsetroundjoin%
\definecolor{currentfill}{rgb}{0.000000,0.000000,0.000000}%
\pgfsetfillcolor{currentfill}%
\pgfsetlinewidth{0.602250pt}%
\definecolor{currentstroke}{rgb}{0.000000,0.000000,0.000000}%
\pgfsetstrokecolor{currentstroke}%
\pgfsetdash{}{0pt}%
\pgfsys@defobject{currentmarker}{\pgfqpoint{-0.027778in}{0.000000in}}{\pgfqpoint{-0.000000in}{0.000000in}}{%
\pgfpathmoveto{\pgfqpoint{-0.000000in}{0.000000in}}%
\pgfpathlineto{\pgfqpoint{-0.027778in}{0.000000in}}%
\pgfusepath{stroke,fill}%
}%
\begin{pgfscope}%
\pgfsys@transformshift{0.588387in}{0.962954in}%
\pgfsys@useobject{currentmarker}{}%
\end{pgfscope}%
\end{pgfscope}%
\begin{pgfscope}%
\pgfsetbuttcap%
\pgfsetroundjoin%
\definecolor{currentfill}{rgb}{0.000000,0.000000,0.000000}%
\pgfsetfillcolor{currentfill}%
\pgfsetlinewidth{0.602250pt}%
\definecolor{currentstroke}{rgb}{0.000000,0.000000,0.000000}%
\pgfsetstrokecolor{currentstroke}%
\pgfsetdash{}{0pt}%
\pgfsys@defobject{currentmarker}{\pgfqpoint{-0.027778in}{0.000000in}}{\pgfqpoint{-0.000000in}{0.000000in}}{%
\pgfpathmoveto{\pgfqpoint{-0.000000in}{0.000000in}}%
\pgfpathlineto{\pgfqpoint{-0.027778in}{0.000000in}}%
\pgfusepath{stroke,fill}%
}%
\begin{pgfscope}%
\pgfsys@transformshift{0.588387in}{1.247609in}%
\pgfsys@useobject{currentmarker}{}%
\end{pgfscope}%
\end{pgfscope}%
\begin{pgfscope}%
\pgfsetbuttcap%
\pgfsetroundjoin%
\definecolor{currentfill}{rgb}{0.000000,0.000000,0.000000}%
\pgfsetfillcolor{currentfill}%
\pgfsetlinewidth{0.602250pt}%
\definecolor{currentstroke}{rgb}{0.000000,0.000000,0.000000}%
\pgfsetstrokecolor{currentstroke}%
\pgfsetdash{}{0pt}%
\pgfsys@defobject{currentmarker}{\pgfqpoint{-0.027778in}{0.000000in}}{\pgfqpoint{-0.000000in}{0.000000in}}{%
\pgfpathmoveto{\pgfqpoint{-0.000000in}{0.000000in}}%
\pgfpathlineto{\pgfqpoint{-0.027778in}{0.000000in}}%
\pgfusepath{stroke,fill}%
}%
\begin{pgfscope}%
\pgfsys@transformshift{0.588387in}{1.392150in}%
\pgfsys@useobject{currentmarker}{}%
\end{pgfscope}%
\end{pgfscope}%
\begin{pgfscope}%
\pgfsetbuttcap%
\pgfsetroundjoin%
\definecolor{currentfill}{rgb}{0.000000,0.000000,0.000000}%
\pgfsetfillcolor{currentfill}%
\pgfsetlinewidth{0.602250pt}%
\definecolor{currentstroke}{rgb}{0.000000,0.000000,0.000000}%
\pgfsetstrokecolor{currentstroke}%
\pgfsetdash{}{0pt}%
\pgfsys@defobject{currentmarker}{\pgfqpoint{-0.027778in}{0.000000in}}{\pgfqpoint{-0.000000in}{0.000000in}}{%
\pgfpathmoveto{\pgfqpoint{-0.000000in}{0.000000in}}%
\pgfpathlineto{\pgfqpoint{-0.027778in}{0.000000in}}%
\pgfusepath{stroke,fill}%
}%
\begin{pgfscope}%
\pgfsys@transformshift{0.588387in}{1.494704in}%
\pgfsys@useobject{currentmarker}{}%
\end{pgfscope}%
\end{pgfscope}%
\begin{pgfscope}%
\pgfsetbuttcap%
\pgfsetroundjoin%
\definecolor{currentfill}{rgb}{0.000000,0.000000,0.000000}%
\pgfsetfillcolor{currentfill}%
\pgfsetlinewidth{0.602250pt}%
\definecolor{currentstroke}{rgb}{0.000000,0.000000,0.000000}%
\pgfsetstrokecolor{currentstroke}%
\pgfsetdash{}{0pt}%
\pgfsys@defobject{currentmarker}{\pgfqpoint{-0.027778in}{0.000000in}}{\pgfqpoint{-0.000000in}{0.000000in}}{%
\pgfpathmoveto{\pgfqpoint{-0.000000in}{0.000000in}}%
\pgfpathlineto{\pgfqpoint{-0.027778in}{0.000000in}}%
\pgfusepath{stroke,fill}%
}%
\begin{pgfscope}%
\pgfsys@transformshift{0.588387in}{1.574251in}%
\pgfsys@useobject{currentmarker}{}%
\end{pgfscope}%
\end{pgfscope}%
\begin{pgfscope}%
\pgfsetbuttcap%
\pgfsetroundjoin%
\definecolor{currentfill}{rgb}{0.000000,0.000000,0.000000}%
\pgfsetfillcolor{currentfill}%
\pgfsetlinewidth{0.602250pt}%
\definecolor{currentstroke}{rgb}{0.000000,0.000000,0.000000}%
\pgfsetstrokecolor{currentstroke}%
\pgfsetdash{}{0pt}%
\pgfsys@defobject{currentmarker}{\pgfqpoint{-0.027778in}{0.000000in}}{\pgfqpoint{-0.000000in}{0.000000in}}{%
\pgfpathmoveto{\pgfqpoint{-0.000000in}{0.000000in}}%
\pgfpathlineto{\pgfqpoint{-0.027778in}{0.000000in}}%
\pgfusepath{stroke,fill}%
}%
\begin{pgfscope}%
\pgfsys@transformshift{0.588387in}{1.639245in}%
\pgfsys@useobject{currentmarker}{}%
\end{pgfscope}%
\end{pgfscope}%
\begin{pgfscope}%
\pgfsetbuttcap%
\pgfsetroundjoin%
\definecolor{currentfill}{rgb}{0.000000,0.000000,0.000000}%
\pgfsetfillcolor{currentfill}%
\pgfsetlinewidth{0.602250pt}%
\definecolor{currentstroke}{rgb}{0.000000,0.000000,0.000000}%
\pgfsetstrokecolor{currentstroke}%
\pgfsetdash{}{0pt}%
\pgfsys@defobject{currentmarker}{\pgfqpoint{-0.027778in}{0.000000in}}{\pgfqpoint{-0.000000in}{0.000000in}}{%
\pgfpathmoveto{\pgfqpoint{-0.000000in}{0.000000in}}%
\pgfpathlineto{\pgfqpoint{-0.027778in}{0.000000in}}%
\pgfusepath{stroke,fill}%
}%
\begin{pgfscope}%
\pgfsys@transformshift{0.588387in}{1.694198in}%
\pgfsys@useobject{currentmarker}{}%
\end{pgfscope}%
\end{pgfscope}%
\begin{pgfscope}%
\pgfsetbuttcap%
\pgfsetroundjoin%
\definecolor{currentfill}{rgb}{0.000000,0.000000,0.000000}%
\pgfsetfillcolor{currentfill}%
\pgfsetlinewidth{0.602250pt}%
\definecolor{currentstroke}{rgb}{0.000000,0.000000,0.000000}%
\pgfsetstrokecolor{currentstroke}%
\pgfsetdash{}{0pt}%
\pgfsys@defobject{currentmarker}{\pgfqpoint{-0.027778in}{0.000000in}}{\pgfqpoint{-0.000000in}{0.000000in}}{%
\pgfpathmoveto{\pgfqpoint{-0.000000in}{0.000000in}}%
\pgfpathlineto{\pgfqpoint{-0.027778in}{0.000000in}}%
\pgfusepath{stroke,fill}%
}%
\begin{pgfscope}%
\pgfsys@transformshift{0.588387in}{1.741799in}%
\pgfsys@useobject{currentmarker}{}%
\end{pgfscope}%
\end{pgfscope}%
\begin{pgfscope}%
\pgfsetbuttcap%
\pgfsetroundjoin%
\definecolor{currentfill}{rgb}{0.000000,0.000000,0.000000}%
\pgfsetfillcolor{currentfill}%
\pgfsetlinewidth{0.602250pt}%
\definecolor{currentstroke}{rgb}{0.000000,0.000000,0.000000}%
\pgfsetstrokecolor{currentstroke}%
\pgfsetdash{}{0pt}%
\pgfsys@defobject{currentmarker}{\pgfqpoint{-0.027778in}{0.000000in}}{\pgfqpoint{-0.000000in}{0.000000in}}{%
\pgfpathmoveto{\pgfqpoint{-0.000000in}{0.000000in}}%
\pgfpathlineto{\pgfqpoint{-0.027778in}{0.000000in}}%
\pgfusepath{stroke,fill}%
}%
\begin{pgfscope}%
\pgfsys@transformshift{0.588387in}{1.783787in}%
\pgfsys@useobject{currentmarker}{}%
\end{pgfscope}%
\end{pgfscope}%
\begin{pgfscope}%
\pgfsetbuttcap%
\pgfsetroundjoin%
\definecolor{currentfill}{rgb}{0.000000,0.000000,0.000000}%
\pgfsetfillcolor{currentfill}%
\pgfsetlinewidth{0.602250pt}%
\definecolor{currentstroke}{rgb}{0.000000,0.000000,0.000000}%
\pgfsetstrokecolor{currentstroke}%
\pgfsetdash{}{0pt}%
\pgfsys@defobject{currentmarker}{\pgfqpoint{-0.027778in}{0.000000in}}{\pgfqpoint{-0.000000in}{0.000000in}}{%
\pgfpathmoveto{\pgfqpoint{-0.000000in}{0.000000in}}%
\pgfpathlineto{\pgfqpoint{-0.027778in}{0.000000in}}%
\pgfusepath{stroke,fill}%
}%
\begin{pgfscope}%
\pgfsys@transformshift{0.588387in}{2.068441in}%
\pgfsys@useobject{currentmarker}{}%
\end{pgfscope}%
\end{pgfscope}%
\begin{pgfscope}%
\pgfsetbuttcap%
\pgfsetroundjoin%
\definecolor{currentfill}{rgb}{0.000000,0.000000,0.000000}%
\pgfsetfillcolor{currentfill}%
\pgfsetlinewidth{0.602250pt}%
\definecolor{currentstroke}{rgb}{0.000000,0.000000,0.000000}%
\pgfsetstrokecolor{currentstroke}%
\pgfsetdash{}{0pt}%
\pgfsys@defobject{currentmarker}{\pgfqpoint{-0.027778in}{0.000000in}}{\pgfqpoint{-0.000000in}{0.000000in}}{%
\pgfpathmoveto{\pgfqpoint{-0.000000in}{0.000000in}}%
\pgfpathlineto{\pgfqpoint{-0.027778in}{0.000000in}}%
\pgfusepath{stroke,fill}%
}%
\begin{pgfscope}%
\pgfsys@transformshift{0.588387in}{2.212983in}%
\pgfsys@useobject{currentmarker}{}%
\end{pgfscope}%
\end{pgfscope}%
\begin{pgfscope}%
\pgfsetbuttcap%
\pgfsetroundjoin%
\definecolor{currentfill}{rgb}{0.000000,0.000000,0.000000}%
\pgfsetfillcolor{currentfill}%
\pgfsetlinewidth{0.602250pt}%
\definecolor{currentstroke}{rgb}{0.000000,0.000000,0.000000}%
\pgfsetstrokecolor{currentstroke}%
\pgfsetdash{}{0pt}%
\pgfsys@defobject{currentmarker}{\pgfqpoint{-0.027778in}{0.000000in}}{\pgfqpoint{-0.000000in}{0.000000in}}{%
\pgfpathmoveto{\pgfqpoint{-0.000000in}{0.000000in}}%
\pgfpathlineto{\pgfqpoint{-0.027778in}{0.000000in}}%
\pgfusepath{stroke,fill}%
}%
\begin{pgfscope}%
\pgfsys@transformshift{0.588387in}{2.315537in}%
\pgfsys@useobject{currentmarker}{}%
\end{pgfscope}%
\end{pgfscope}%
\begin{pgfscope}%
\pgfsetbuttcap%
\pgfsetroundjoin%
\definecolor{currentfill}{rgb}{0.000000,0.000000,0.000000}%
\pgfsetfillcolor{currentfill}%
\pgfsetlinewidth{0.602250pt}%
\definecolor{currentstroke}{rgb}{0.000000,0.000000,0.000000}%
\pgfsetstrokecolor{currentstroke}%
\pgfsetdash{}{0pt}%
\pgfsys@defobject{currentmarker}{\pgfqpoint{-0.027778in}{0.000000in}}{\pgfqpoint{-0.000000in}{0.000000in}}{%
\pgfpathmoveto{\pgfqpoint{-0.000000in}{0.000000in}}%
\pgfpathlineto{\pgfqpoint{-0.027778in}{0.000000in}}%
\pgfusepath{stroke,fill}%
}%
\begin{pgfscope}%
\pgfsys@transformshift{0.588387in}{2.395084in}%
\pgfsys@useobject{currentmarker}{}%
\end{pgfscope}%
\end{pgfscope}%
\begin{pgfscope}%
\pgfsetbuttcap%
\pgfsetroundjoin%
\definecolor{currentfill}{rgb}{0.000000,0.000000,0.000000}%
\pgfsetfillcolor{currentfill}%
\pgfsetlinewidth{0.602250pt}%
\definecolor{currentstroke}{rgb}{0.000000,0.000000,0.000000}%
\pgfsetstrokecolor{currentstroke}%
\pgfsetdash{}{0pt}%
\pgfsys@defobject{currentmarker}{\pgfqpoint{-0.027778in}{0.000000in}}{\pgfqpoint{-0.000000in}{0.000000in}}{%
\pgfpathmoveto{\pgfqpoint{-0.000000in}{0.000000in}}%
\pgfpathlineto{\pgfqpoint{-0.027778in}{0.000000in}}%
\pgfusepath{stroke,fill}%
}%
\begin{pgfscope}%
\pgfsys@transformshift{0.588387in}{2.460078in}%
\pgfsys@useobject{currentmarker}{}%
\end{pgfscope}%
\end{pgfscope}%
\begin{pgfscope}%
\pgfsetbuttcap%
\pgfsetroundjoin%
\definecolor{currentfill}{rgb}{0.000000,0.000000,0.000000}%
\pgfsetfillcolor{currentfill}%
\pgfsetlinewidth{0.602250pt}%
\definecolor{currentstroke}{rgb}{0.000000,0.000000,0.000000}%
\pgfsetstrokecolor{currentstroke}%
\pgfsetdash{}{0pt}%
\pgfsys@defobject{currentmarker}{\pgfqpoint{-0.027778in}{0.000000in}}{\pgfqpoint{-0.000000in}{0.000000in}}{%
\pgfpathmoveto{\pgfqpoint{-0.000000in}{0.000000in}}%
\pgfpathlineto{\pgfqpoint{-0.027778in}{0.000000in}}%
\pgfusepath{stroke,fill}%
}%
\begin{pgfscope}%
\pgfsys@transformshift{0.588387in}{2.515030in}%
\pgfsys@useobject{currentmarker}{}%
\end{pgfscope}%
\end{pgfscope}%
\begin{pgfscope}%
\pgfsetbuttcap%
\pgfsetroundjoin%
\definecolor{currentfill}{rgb}{0.000000,0.000000,0.000000}%
\pgfsetfillcolor{currentfill}%
\pgfsetlinewidth{0.602250pt}%
\definecolor{currentstroke}{rgb}{0.000000,0.000000,0.000000}%
\pgfsetstrokecolor{currentstroke}%
\pgfsetdash{}{0pt}%
\pgfsys@defobject{currentmarker}{\pgfqpoint{-0.027778in}{0.000000in}}{\pgfqpoint{-0.000000in}{0.000000in}}{%
\pgfpathmoveto{\pgfqpoint{-0.000000in}{0.000000in}}%
\pgfpathlineto{\pgfqpoint{-0.027778in}{0.000000in}}%
\pgfusepath{stroke,fill}%
}%
\begin{pgfscope}%
\pgfsys@transformshift{0.588387in}{2.562632in}%
\pgfsys@useobject{currentmarker}{}%
\end{pgfscope}%
\end{pgfscope}%
\begin{pgfscope}%
\pgfsetbuttcap%
\pgfsetroundjoin%
\definecolor{currentfill}{rgb}{0.000000,0.000000,0.000000}%
\pgfsetfillcolor{currentfill}%
\pgfsetlinewidth{0.602250pt}%
\definecolor{currentstroke}{rgb}{0.000000,0.000000,0.000000}%
\pgfsetstrokecolor{currentstroke}%
\pgfsetdash{}{0pt}%
\pgfsys@defobject{currentmarker}{\pgfqpoint{-0.027778in}{0.000000in}}{\pgfqpoint{-0.000000in}{0.000000in}}{%
\pgfpathmoveto{\pgfqpoint{-0.000000in}{0.000000in}}%
\pgfpathlineto{\pgfqpoint{-0.027778in}{0.000000in}}%
\pgfusepath{stroke,fill}%
}%
\begin{pgfscope}%
\pgfsys@transformshift{0.588387in}{2.604620in}%
\pgfsys@useobject{currentmarker}{}%
\end{pgfscope}%
\end{pgfscope}%
\begin{pgfscope}%
\definecolor{textcolor}{rgb}{0.000000,0.000000,0.000000}%
\pgfsetstrokecolor{textcolor}%
\pgfsetfillcolor{textcolor}%
\pgftext[x=0.234413in,y=1.631726in,,bottom,rotate=90.000000]{\color{textcolor}{\rmfamily\fontsize{10.000000}{12.000000}\selectfont\catcode`\^=\active\def^{\ifmmode\sp\else\^{}\fi}\catcode`\%=\active\def%{\%}Time [ms]}}%
\end{pgfscope}%
\begin{pgfscope}%
\pgfpathrectangle{\pgfqpoint{0.588387in}{0.521603in}}{\pgfqpoint{4.502076in}{2.220246in}}%
\pgfusepath{clip}%
\pgfsetrectcap%
\pgfsetroundjoin%
\pgfsetlinewidth{1.505625pt}%
\pgfsetstrokecolor{currentstroke1}%
\pgfsetdash{}{0pt}%
\pgfpathmoveto{\pgfqpoint{0.793027in}{1.163340in}}%
\pgfpathlineto{\pgfqpoint{0.876554in}{1.316595in}}%
\pgfpathlineto{\pgfqpoint{0.960080in}{1.639924in}}%
\pgfpathlineto{\pgfqpoint{1.043606in}{1.609152in}}%
\pgfpathlineto{\pgfqpoint{1.127133in}{1.898987in}}%
\pgfpathlineto{\pgfqpoint{1.210659in}{1.876957in}}%
\pgfpathlineto{\pgfqpoint{1.294186in}{1.882203in}}%
\pgfpathlineto{\pgfqpoint{1.377712in}{1.960427in}}%
\pgfpathlineto{\pgfqpoint{1.461239in}{2.011941in}}%
\pgfpathlineto{\pgfqpoint{1.544765in}{2.173596in}}%
\pgfpathlineto{\pgfqpoint{1.628292in}{2.189780in}}%
\pgfpathlineto{\pgfqpoint{1.711818in}{2.289826in}}%
\pgfpathlineto{\pgfqpoint{1.795345in}{2.344773in}}%
\pgfpathlineto{\pgfqpoint{1.878871in}{2.233821in}}%
\pgfpathlineto{\pgfqpoint{1.962398in}{2.310288in}}%
\pgfpathlineto{\pgfqpoint{2.045924in}{2.362698in}}%
\pgfpathlineto{\pgfqpoint{2.129451in}{2.336008in}}%
\pgfpathlineto{\pgfqpoint{2.212977in}{2.167682in}}%
\pgfpathlineto{\pgfqpoint{2.296503in}{2.393710in}}%
\pgfpathlineto{\pgfqpoint{2.380030in}{2.410123in}}%
\pgfpathlineto{\pgfqpoint{2.463556in}{2.369826in}}%
\pgfpathlineto{\pgfqpoint{2.547083in}{2.517567in}}%
\pgfpathlineto{\pgfqpoint{2.630609in}{2.378670in}}%
\pgfpathlineto{\pgfqpoint{2.714136in}{2.606426in}}%
\pgfpathlineto{\pgfqpoint{2.797662in}{2.520923in}}%
\pgfpathlineto{\pgfqpoint{2.881189in}{2.555203in}}%
\pgfpathlineto{\pgfqpoint{2.964715in}{2.486273in}}%
\pgfpathlineto{\pgfqpoint{3.048242in}{2.573817in}}%
\pgfpathlineto{\pgfqpoint{3.131768in}{2.640929in}}%
\pgfpathlineto{\pgfqpoint{3.215295in}{2.494055in}}%
\pgfpathlineto{\pgfqpoint{3.298821in}{2.513243in}}%
\pgfpathlineto{\pgfqpoint{3.465874in}{2.629109in}}%
\pgfpathlineto{\pgfqpoint{4.050559in}{2.546218in}}%
\pgfusepath{stroke}%
\end{pgfscope}%
\begin{pgfscope}%
\pgfpathrectangle{\pgfqpoint{0.588387in}{0.521603in}}{\pgfqpoint{4.502076in}{2.220246in}}%
\pgfusepath{clip}%
\pgfsetrectcap%
\pgfsetroundjoin%
\pgfsetlinewidth{1.505625pt}%
\pgfsetstrokecolor{currentstroke2}%
\pgfsetdash{}{0pt}%
\pgfpathmoveto{\pgfqpoint{0.793027in}{0.627530in}}%
\pgfpathlineto{\pgfqpoint{0.876554in}{0.670373in}}%
\pgfpathlineto{\pgfqpoint{0.960080in}{0.761677in}}%
\pgfpathlineto{\pgfqpoint{1.043606in}{0.846518in}}%
\pgfpathlineto{\pgfqpoint{1.127133in}{0.931421in}}%
\pgfpathlineto{\pgfqpoint{1.210659in}{0.975818in}}%
\pgfpathlineto{\pgfqpoint{1.294186in}{1.050946in}}%
\pgfpathlineto{\pgfqpoint{1.377712in}{1.097453in}}%
\pgfpathlineto{\pgfqpoint{1.461239in}{1.116660in}}%
\pgfpathlineto{\pgfqpoint{1.544765in}{1.197612in}}%
\pgfpathlineto{\pgfqpoint{1.628292in}{1.259388in}}%
\pgfpathlineto{\pgfqpoint{1.711818in}{1.299674in}}%
\pgfpathlineto{\pgfqpoint{1.795345in}{1.329155in}}%
\pgfpathlineto{\pgfqpoint{1.878871in}{1.377366in}}%
\pgfpathlineto{\pgfqpoint{1.962398in}{1.411307in}}%
\pgfpathlineto{\pgfqpoint{2.045924in}{1.440030in}}%
\pgfpathlineto{\pgfqpoint{2.129451in}{1.461756in}}%
\pgfpathlineto{\pgfqpoint{2.212977in}{1.496232in}}%
\pgfpathlineto{\pgfqpoint{2.296503in}{1.535179in}}%
\pgfpathlineto{\pgfqpoint{2.380030in}{1.561204in}}%
\pgfpathlineto{\pgfqpoint{2.463556in}{1.609844in}}%
\pgfpathlineto{\pgfqpoint{2.547083in}{1.609642in}}%
\pgfpathlineto{\pgfqpoint{2.630609in}{1.644407in}}%
\pgfpathlineto{\pgfqpoint{2.714136in}{1.683536in}}%
\pgfpathlineto{\pgfqpoint{2.797662in}{1.713405in}}%
\pgfpathlineto{\pgfqpoint{2.881189in}{1.722486in}}%
\pgfpathlineto{\pgfqpoint{2.964715in}{1.768098in}}%
\pgfpathlineto{\pgfqpoint{3.048242in}{1.775015in}}%
\pgfpathlineto{\pgfqpoint{3.131768in}{1.793795in}}%
\pgfpathlineto{\pgfqpoint{3.215295in}{1.792831in}}%
\pgfpathlineto{\pgfqpoint{3.298821in}{1.822014in}}%
\pgfpathlineto{\pgfqpoint{3.382347in}{1.845465in}}%
\pgfpathlineto{\pgfqpoint{3.465874in}{1.847334in}}%
\pgfpathlineto{\pgfqpoint{3.549400in}{1.936002in}}%
\pgfpathlineto{\pgfqpoint{3.632927in}{1.939538in}}%
\pgfpathlineto{\pgfqpoint{3.716453in}{1.954570in}}%
\pgfpathlineto{\pgfqpoint{3.799980in}{1.949007in}}%
\pgfpathlineto{\pgfqpoint{3.883506in}{1.967961in}}%
\pgfpathlineto{\pgfqpoint{3.967033in}{1.980155in}}%
\pgfpathlineto{\pgfqpoint{4.050559in}{1.988894in}}%
\pgfpathlineto{\pgfqpoint{4.134086in}{2.021857in}}%
\pgfpathlineto{\pgfqpoint{4.217612in}{2.005491in}}%
\pgfpathlineto{\pgfqpoint{4.301139in}{2.032118in}}%
\pgfpathlineto{\pgfqpoint{4.384665in}{2.109040in}}%
\pgfpathlineto{\pgfqpoint{4.468192in}{2.091727in}}%
\pgfpathlineto{\pgfqpoint{4.551718in}{2.104438in}}%
\pgfpathlineto{\pgfqpoint{4.635244in}{2.090472in}}%
\pgfpathlineto{\pgfqpoint{4.718771in}{2.128952in}}%
\pgfpathlineto{\pgfqpoint{4.802297in}{2.142057in}}%
\pgfpathlineto{\pgfqpoint{4.885824in}{2.134363in}}%
\pgfusepath{stroke}%
\end{pgfscope}%
\begin{pgfscope}%
\pgfpathrectangle{\pgfqpoint{0.588387in}{0.521603in}}{\pgfqpoint{4.502076in}{2.220246in}}%
\pgfusepath{clip}%
\pgfsetrectcap%
\pgfsetroundjoin%
\pgfsetlinewidth{1.505625pt}%
\pgfsetstrokecolor{currentstroke3}%
\pgfsetdash{}{0pt}%
\pgfpathmoveto{\pgfqpoint{0.793027in}{0.639656in}}%
\pgfpathlineto{\pgfqpoint{0.876554in}{0.666669in}}%
\pgfpathlineto{\pgfqpoint{0.960080in}{0.747804in}}%
\pgfpathlineto{\pgfqpoint{1.043606in}{0.826218in}}%
\pgfpathlineto{\pgfqpoint{1.127133in}{0.924836in}}%
\pgfpathlineto{\pgfqpoint{1.210659in}{0.954432in}}%
\pgfpathlineto{\pgfqpoint{1.294186in}{1.028053in}}%
\pgfpathlineto{\pgfqpoint{1.377712in}{1.081617in}}%
\pgfpathlineto{\pgfqpoint{1.461239in}{1.101252in}}%
\pgfpathlineto{\pgfqpoint{1.544765in}{1.185057in}}%
\pgfpathlineto{\pgfqpoint{1.628292in}{1.238583in}}%
\pgfpathlineto{\pgfqpoint{1.711818in}{1.273390in}}%
\pgfpathlineto{\pgfqpoint{1.795345in}{1.302003in}}%
\pgfpathlineto{\pgfqpoint{1.878871in}{1.347927in}}%
\pgfpathlineto{\pgfqpoint{1.962398in}{1.374646in}}%
\pgfpathlineto{\pgfqpoint{2.045924in}{1.414111in}}%
\pgfpathlineto{\pgfqpoint{2.129451in}{1.430655in}}%
\pgfpathlineto{\pgfqpoint{2.212977in}{1.481223in}}%
\pgfpathlineto{\pgfqpoint{2.296503in}{1.507486in}}%
\pgfpathlineto{\pgfqpoint{2.380030in}{1.538420in}}%
\pgfpathlineto{\pgfqpoint{2.463556in}{1.593970in}}%
\pgfpathlineto{\pgfqpoint{2.547083in}{1.596491in}}%
\pgfpathlineto{\pgfqpoint{2.630609in}{1.637197in}}%
\pgfpathlineto{\pgfqpoint{2.714136in}{1.679745in}}%
\pgfpathlineto{\pgfqpoint{2.797662in}{1.689552in}}%
\pgfpathlineto{\pgfqpoint{2.881189in}{1.717453in}}%
\pgfpathlineto{\pgfqpoint{2.964715in}{1.728897in}}%
\pgfpathlineto{\pgfqpoint{3.048242in}{1.754251in}}%
\pgfpathlineto{\pgfqpoint{3.131768in}{1.764066in}}%
\pgfpathlineto{\pgfqpoint{3.215295in}{1.772980in}}%
\pgfpathlineto{\pgfqpoint{3.298821in}{1.793554in}}%
\pgfpathlineto{\pgfqpoint{3.382347in}{1.891502in}}%
\pgfpathlineto{\pgfqpoint{3.465874in}{1.822014in}}%
\pgfpathlineto{\pgfqpoint{3.549400in}{1.877125in}}%
\pgfpathlineto{\pgfqpoint{3.632927in}{1.892050in}}%
\pgfpathlineto{\pgfqpoint{3.716453in}{1.898030in}}%
\pgfpathlineto{\pgfqpoint{3.799980in}{1.906025in}}%
\pgfpathlineto{\pgfqpoint{3.883506in}{1.914532in}}%
\pgfpathlineto{\pgfqpoint{3.967033in}{1.936809in}}%
\pgfpathlineto{\pgfqpoint{4.050559in}{1.937292in}}%
\pgfpathlineto{\pgfqpoint{4.134086in}{1.981294in}}%
\pgfpathlineto{\pgfqpoint{4.217612in}{1.980155in}}%
\pgfpathlineto{\pgfqpoint{4.301139in}{1.984974in}}%
\pgfpathlineto{\pgfqpoint{4.384665in}{2.076373in}}%
\pgfpathlineto{\pgfqpoint{4.468192in}{2.035555in}}%
\pgfpathlineto{\pgfqpoint{4.551718in}{2.073529in}}%
\pgfpathlineto{\pgfqpoint{4.635244in}{2.087317in}}%
\pgfpathlineto{\pgfqpoint{4.718771in}{2.108443in}}%
\pgfpathlineto{\pgfqpoint{4.802297in}{2.120004in}}%
\pgfpathlineto{\pgfqpoint{4.885824in}{2.132506in}}%
\pgfusepath{stroke}%
\end{pgfscope}%
\begin{pgfscope}%
\pgfpathrectangle{\pgfqpoint{0.588387in}{0.521603in}}{\pgfqpoint{4.502076in}{2.220246in}}%
\pgfusepath{clip}%
\pgfsetrectcap%
\pgfsetroundjoin%
\pgfsetlinewidth{1.505625pt}%
\pgfsetstrokecolor{currentstroke4}%
\pgfsetdash{}{0pt}%
\pgfpathmoveto{\pgfqpoint{0.793027in}{0.622524in}}%
\pgfpathlineto{\pgfqpoint{0.876554in}{0.652347in}}%
\pgfpathlineto{\pgfqpoint{0.960080in}{0.734546in}}%
\pgfpathlineto{\pgfqpoint{1.043606in}{0.818413in}}%
\pgfpathlineto{\pgfqpoint{1.127133in}{0.910560in}}%
\pgfpathlineto{\pgfqpoint{1.210659in}{0.955071in}}%
\pgfpathlineto{\pgfqpoint{1.294186in}{1.034490in}}%
\pgfpathlineto{\pgfqpoint{1.377712in}{1.077601in}}%
\pgfpathlineto{\pgfqpoint{1.461239in}{1.099358in}}%
\pgfpathlineto{\pgfqpoint{1.544765in}{1.198891in}}%
\pgfpathlineto{\pgfqpoint{1.628292in}{1.263939in}}%
\pgfpathlineto{\pgfqpoint{1.711818in}{1.286517in}}%
\pgfpathlineto{\pgfqpoint{1.795345in}{1.309573in}}%
\pgfpathlineto{\pgfqpoint{1.878871in}{1.366160in}}%
\pgfpathlineto{\pgfqpoint{1.962398in}{1.388792in}}%
\pgfpathlineto{\pgfqpoint{2.045924in}{1.438729in}}%
\pgfpathlineto{\pgfqpoint{2.129451in}{1.442618in}}%
\pgfpathlineto{\pgfqpoint{2.212977in}{1.493728in}}%
\pgfpathlineto{\pgfqpoint{2.296503in}{1.523238in}}%
\pgfpathlineto{\pgfqpoint{2.380030in}{1.545615in}}%
\pgfpathlineto{\pgfqpoint{2.463556in}{1.597536in}}%
\pgfpathlineto{\pgfqpoint{2.547083in}{1.609441in}}%
\pgfpathlineto{\pgfqpoint{2.630609in}{1.632687in}}%
\pgfpathlineto{\pgfqpoint{2.714136in}{1.669830in}}%
\pgfpathlineto{\pgfqpoint{2.797662in}{1.694992in}}%
\pgfpathlineto{\pgfqpoint{2.881189in}{1.708393in}}%
\pgfpathlineto{\pgfqpoint{2.964715in}{1.735761in}}%
\pgfpathlineto{\pgfqpoint{3.048242in}{1.758794in}}%
\pgfpathlineto{\pgfqpoint{3.131768in}{1.772597in}}%
\pgfpathlineto{\pgfqpoint{3.215295in}{1.793795in}}%
\pgfpathlineto{\pgfqpoint{3.298821in}{1.814144in}}%
\pgfpathlineto{\pgfqpoint{3.382347in}{1.841697in}}%
\pgfpathlineto{\pgfqpoint{3.465874in}{1.826214in}}%
\pgfpathlineto{\pgfqpoint{3.549400in}{1.904440in}}%
\pgfpathlineto{\pgfqpoint{3.632927in}{1.920823in}}%
\pgfpathlineto{\pgfqpoint{3.716453in}{1.932431in}}%
\pgfpathlineto{\pgfqpoint{3.799980in}{1.927006in}}%
\pgfpathlineto{\pgfqpoint{3.883506in}{1.946663in}}%
\pgfpathlineto{\pgfqpoint{3.967033in}{1.967518in}}%
\pgfpathlineto{\pgfqpoint{4.050559in}{1.966778in}}%
\pgfpathlineto{\pgfqpoint{4.134086in}{1.996063in}}%
\pgfpathlineto{\pgfqpoint{4.217612in}{1.977289in}}%
\pgfpathlineto{\pgfqpoint{4.301139in}{2.023124in}}%
\pgfpathlineto{\pgfqpoint{4.384665in}{2.082851in}}%
\pgfpathlineto{\pgfqpoint{4.468192in}{2.063732in}}%
\pgfpathlineto{\pgfqpoint{4.551718in}{2.087105in}}%
\pgfpathlineto{\pgfqpoint{4.635244in}{2.089003in}}%
\pgfpathlineto{\pgfqpoint{4.718771in}{2.103631in}}%
\pgfpathlineto{\pgfqpoint{4.802297in}{2.122118in}}%
\pgfpathlineto{\pgfqpoint{4.885824in}{2.111418in}}%
\pgfusepath{stroke}%
\end{pgfscope}%
\begin{pgfscope}%
\pgfpathrectangle{\pgfqpoint{0.588387in}{0.521603in}}{\pgfqpoint{4.502076in}{2.220246in}}%
\pgfusepath{clip}%
\pgfsetrectcap%
\pgfsetroundjoin%
\pgfsetlinewidth{1.505625pt}%
\pgfsetstrokecolor{currentstroke5}%
\pgfsetdash{}{0pt}%
\pgfpathmoveto{\pgfqpoint{0.793027in}{0.625855in}}%
\pgfpathlineto{\pgfqpoint{0.876554in}{0.650121in}}%
\pgfpathlineto{\pgfqpoint{0.960080in}{0.744393in}}%
\pgfpathlineto{\pgfqpoint{1.043606in}{0.817017in}}%
\pgfpathlineto{\pgfqpoint{1.127133in}{0.914183in}}%
\pgfpathlineto{\pgfqpoint{1.210659in}{0.958876in}}%
\pgfpathlineto{\pgfqpoint{1.294186in}{1.029090in}}%
\pgfpathlineto{\pgfqpoint{1.377712in}{1.077377in}}%
\pgfpathlineto{\pgfqpoint{1.461239in}{1.101042in}}%
\pgfpathlineto{\pgfqpoint{1.544765in}{1.180043in}}%
\pgfpathlineto{\pgfqpoint{1.628292in}{1.243688in}}%
\pgfpathlineto{\pgfqpoint{1.711818in}{1.274458in}}%
\pgfpathlineto{\pgfqpoint{1.795345in}{1.309807in}}%
\pgfpathlineto{\pgfqpoint{1.878871in}{1.358897in}}%
\pgfpathlineto{\pgfqpoint{1.962398in}{1.394371in}}%
\pgfpathlineto{\pgfqpoint{2.045924in}{1.418104in}}%
\pgfpathlineto{\pgfqpoint{2.129451in}{1.452159in}}%
\pgfpathlineto{\pgfqpoint{2.212977in}{1.480789in}}%
\pgfpathlineto{\pgfqpoint{2.296503in}{1.529194in}}%
\pgfpathlineto{\pgfqpoint{2.380030in}{1.555848in}}%
\pgfpathlineto{\pgfqpoint{2.463556in}{1.588874in}}%
\pgfpathlineto{\pgfqpoint{2.547083in}{1.599410in}}%
\pgfpathlineto{\pgfqpoint{2.630609in}{1.623492in}}%
\pgfpathlineto{\pgfqpoint{2.714136in}{1.666406in}}%
\pgfpathlineto{\pgfqpoint{2.797662in}{1.676749in}}%
\pgfpathlineto{\pgfqpoint{2.881189in}{1.701444in}}%
\pgfpathlineto{\pgfqpoint{2.964715in}{1.723074in}}%
\pgfpathlineto{\pgfqpoint{3.048242in}{1.763935in}}%
\pgfpathlineto{\pgfqpoint{3.131768in}{1.777670in}}%
\pgfpathlineto{\pgfqpoint{3.215295in}{1.770033in}}%
\pgfpathlineto{\pgfqpoint{3.298821in}{1.807489in}}%
\pgfpathlineto{\pgfqpoint{3.382347in}{1.836611in}}%
\pgfpathlineto{\pgfqpoint{3.465874in}{1.816410in}}%
\pgfpathlineto{\pgfqpoint{3.549400in}{1.893873in}}%
\pgfpathlineto{\pgfqpoint{3.632927in}{1.916243in}}%
\pgfpathlineto{\pgfqpoint{3.716453in}{1.920317in}}%
\pgfpathlineto{\pgfqpoint{3.799980in}{1.914188in}}%
\pgfpathlineto{\pgfqpoint{3.883506in}{1.941611in}}%
\pgfpathlineto{\pgfqpoint{3.967033in}{1.958989in}}%
\pgfpathlineto{\pgfqpoint{4.050559in}{1.954416in}}%
\pgfpathlineto{\pgfqpoint{4.134086in}{1.996881in}}%
\pgfpathlineto{\pgfqpoint{4.217612in}{1.962305in}}%
\pgfpathlineto{\pgfqpoint{4.301139in}{2.017000in}}%
\pgfpathlineto{\pgfqpoint{4.384665in}{2.075501in}}%
\pgfpathlineto{\pgfqpoint{4.468192in}{2.055279in}}%
\pgfpathlineto{\pgfqpoint{4.551718in}{2.064408in}}%
\pgfpathlineto{\pgfqpoint{4.635244in}{2.082637in}}%
\pgfpathlineto{\pgfqpoint{4.718771in}{2.100590in}}%
\pgfpathlineto{\pgfqpoint{4.802297in}{2.163168in}}%
\pgfpathlineto{\pgfqpoint{4.885824in}{2.103631in}}%
\pgfusepath{stroke}%
\end{pgfscope}%
\begin{pgfscope}%
\pgfpathrectangle{\pgfqpoint{0.588387in}{0.521603in}}{\pgfqpoint{4.502076in}{2.220246in}}%
\pgfusepath{clip}%
\pgfsetrectcap%
\pgfsetroundjoin%
\pgfsetlinewidth{1.505625pt}%
\pgfsetstrokecolor{currentstroke6}%
\pgfsetdash{}{0pt}%
\pgfpathmoveto{\pgfqpoint{0.793027in}{0.623101in}}%
\pgfpathlineto{\pgfqpoint{0.876554in}{0.650293in}}%
\pgfpathlineto{\pgfqpoint{0.960080in}{0.723209in}}%
\pgfpathlineto{\pgfqpoint{1.043606in}{0.791423in}}%
\pgfpathlineto{\pgfqpoint{1.127133in}{0.874978in}}%
\pgfpathlineto{\pgfqpoint{1.210659in}{0.917795in}}%
\pgfpathlineto{\pgfqpoint{1.294186in}{0.980896in}}%
\pgfpathlineto{\pgfqpoint{1.377712in}{1.030416in}}%
\pgfpathlineto{\pgfqpoint{1.461239in}{1.046685in}}%
\pgfpathlineto{\pgfqpoint{1.544765in}{1.122839in}}%
\pgfpathlineto{\pgfqpoint{1.628292in}{1.185057in}}%
\pgfpathlineto{\pgfqpoint{1.711818in}{1.226589in}}%
\pgfpathlineto{\pgfqpoint{1.795345in}{1.252587in}}%
\pgfpathlineto{\pgfqpoint{1.878871in}{1.303435in}}%
\pgfpathlineto{\pgfqpoint{1.962398in}{1.319271in}}%
\pgfpathlineto{\pgfqpoint{2.045924in}{1.341565in}}%
\pgfpathlineto{\pgfqpoint{2.129451in}{1.366160in}}%
\pgfpathlineto{\pgfqpoint{2.212977in}{1.405631in}}%
\pgfpathlineto{\pgfqpoint{2.296503in}{1.437600in}}%
\pgfpathlineto{\pgfqpoint{2.380030in}{1.473918in}}%
\pgfpathlineto{\pgfqpoint{2.463556in}{1.505674in}}%
\pgfpathlineto{\pgfqpoint{2.547083in}{1.537433in}}%
\pgfpathlineto{\pgfqpoint{2.630609in}{1.546940in}}%
\pgfpathlineto{\pgfqpoint{2.714136in}{1.580216in}}%
\pgfpathlineto{\pgfqpoint{2.797662in}{1.591644in}}%
\pgfpathlineto{\pgfqpoint{2.881189in}{1.622715in}}%
\pgfpathlineto{\pgfqpoint{2.964715in}{1.657168in}}%
\pgfpathlineto{\pgfqpoint{3.048242in}{1.667265in}}%
\pgfpathlineto{\pgfqpoint{3.131768in}{1.689391in}}%
\pgfpathlineto{\pgfqpoint{3.215295in}{1.699568in}}%
\pgfpathlineto{\pgfqpoint{3.298821in}{1.716408in}}%
\pgfpathlineto{\pgfqpoint{3.382347in}{1.741242in}}%
\pgfpathlineto{\pgfqpoint{3.465874in}{1.754117in}}%
\pgfpathlineto{\pgfqpoint{3.549400in}{1.812777in}}%
\pgfpathlineto{\pgfqpoint{3.632927in}{1.816184in}}%
\pgfpathlineto{\pgfqpoint{3.716453in}{1.827312in}}%
\pgfpathlineto{\pgfqpoint{3.799980in}{1.866489in}}%
\pgfpathlineto{\pgfqpoint{3.883506in}{1.845049in}}%
\pgfpathlineto{\pgfqpoint{3.967033in}{1.869226in}}%
\pgfpathlineto{\pgfqpoint{4.050559in}{1.874639in}}%
\pgfpathlineto{\pgfqpoint{4.134086in}{1.907603in}}%
\pgfpathlineto{\pgfqpoint{4.217612in}{1.914532in}}%
\pgfpathlineto{\pgfqpoint{4.301139in}{1.950095in}}%
\pgfpathlineto{\pgfqpoint{4.384665in}{1.995517in}}%
\pgfpathlineto{\pgfqpoint{4.468192in}{1.999594in}}%
\pgfpathlineto{\pgfqpoint{4.551718in}{1.993048in}}%
\pgfpathlineto{\pgfqpoint{4.635244in}{1.994147in}}%
\pgfpathlineto{\pgfqpoint{4.718771in}{2.001481in}}%
\pgfpathlineto{\pgfqpoint{4.802297in}{2.001481in}}%
\pgfpathlineto{\pgfqpoint{4.885824in}{2.009982in}}%
\pgfusepath{stroke}%
\end{pgfscope}%
\begin{pgfscope}%
\pgfsetrectcap%
\pgfsetmiterjoin%
\pgfsetlinewidth{0.803000pt}%
\definecolor{currentstroke}{rgb}{0.000000,0.000000,0.000000}%
\pgfsetstrokecolor{currentstroke}%
\pgfsetdash{}{0pt}%
\pgfpathmoveto{\pgfqpoint{0.588387in}{0.521603in}}%
\pgfpathlineto{\pgfqpoint{0.588387in}{2.741849in}}%
\pgfusepath{stroke}%
\end{pgfscope}%
\begin{pgfscope}%
\pgfsetrectcap%
\pgfsetmiterjoin%
\pgfsetlinewidth{0.803000pt}%
\definecolor{currentstroke}{rgb}{0.000000,0.000000,0.000000}%
\pgfsetstrokecolor{currentstroke}%
\pgfsetdash{}{0pt}%
\pgfpathmoveto{\pgfqpoint{5.090464in}{0.521603in}}%
\pgfpathlineto{\pgfqpoint{5.090464in}{2.741849in}}%
\pgfusepath{stroke}%
\end{pgfscope}%
\begin{pgfscope}%
\pgfsetrectcap%
\pgfsetmiterjoin%
\pgfsetlinewidth{0.803000pt}%
\definecolor{currentstroke}{rgb}{0.000000,0.000000,0.000000}%
\pgfsetstrokecolor{currentstroke}%
\pgfsetdash{}{0pt}%
\pgfpathmoveto{\pgfqpoint{0.588387in}{0.521603in}}%
\pgfpathlineto{\pgfqpoint{5.090464in}{0.521603in}}%
\pgfusepath{stroke}%
\end{pgfscope}%
\begin{pgfscope}%
\pgfsetrectcap%
\pgfsetmiterjoin%
\pgfsetlinewidth{0.803000pt}%
\definecolor{currentstroke}{rgb}{0.000000,0.000000,0.000000}%
\pgfsetstrokecolor{currentstroke}%
\pgfsetdash{}{0pt}%
\pgfpathmoveto{\pgfqpoint{0.588387in}{2.741849in}}%
\pgfpathlineto{\pgfqpoint{5.090464in}{2.741849in}}%
\pgfusepath{stroke}%
\end{pgfscope}%
\begin{pgfscope}%
\pgfsetbuttcap%
\pgfsetmiterjoin%
\definecolor{currentfill}{rgb}{1.000000,1.000000,1.000000}%
\pgfsetfillcolor{currentfill}%
\pgfsetfillopacity{0.800000}%
\pgfsetlinewidth{1.003750pt}%
\definecolor{currentstroke}{rgb}{0.800000,0.800000,0.800000}%
\pgfsetstrokecolor{currentstroke}%
\pgfsetstrokeopacity{0.800000}%
\pgfsetdash{}{0pt}%
\pgfpathmoveto{\pgfqpoint{5.207130in}{1.126828in}}%
\pgfpathlineto{\pgfqpoint{8.251043in}{1.126828in}}%
\pgfpathquadraticcurveto{\pgfqpoint{8.284376in}{1.126828in}}{\pgfqpoint{8.284376in}{1.160161in}}%
\pgfpathlineto{\pgfqpoint{8.284376in}{2.625183in}}%
\pgfpathquadraticcurveto{\pgfqpoint{8.284376in}{2.658516in}}{\pgfqpoint{8.251043in}{2.658516in}}%
\pgfpathlineto{\pgfqpoint{5.207130in}{2.658516in}}%
\pgfpathquadraticcurveto{\pgfqpoint{5.173797in}{2.658516in}}{\pgfqpoint{5.173797in}{2.625183in}}%
\pgfpathlineto{\pgfqpoint{5.173797in}{1.160161in}}%
\pgfpathquadraticcurveto{\pgfqpoint{5.173797in}{1.126828in}}{\pgfqpoint{5.207130in}{1.126828in}}%
\pgfpathlineto{\pgfqpoint{5.207130in}{1.126828in}}%
\pgfpathclose%
\pgfusepath{stroke,fill}%
\end{pgfscope}%
\begin{pgfscope}%
\pgfsetrectcap%
\pgfsetroundjoin%
\pgfsetlinewidth{1.505625pt}%
\pgfsetstrokecolor{currentstroke2}%
\pgfsetdash{}{0pt}%
\pgfpathmoveto{\pgfqpoint{5.240464in}{2.523555in}}%
\pgfpathlineto{\pgfqpoint{5.407130in}{2.523555in}}%
\pgfpathlineto{\pgfqpoint{5.573797in}{2.523555in}}%
\pgfusepath{stroke}%
\end{pgfscope}%
\begin{pgfscope}%
\definecolor{textcolor}{rgb}{0.000000,0.000000,0.000000}%
\pgfsetstrokecolor{textcolor}%
\pgfsetfillcolor{textcolor}%
\pgftext[x=5.707130in,y=2.465222in,left,base]{\color{textcolor}{\rmfamily\fontsize{12.000000}{14.400000}\selectfont\catcode`\^=\active\def^{\ifmmode\sp\else\^{}\fi}\catcode`\%=\active\def%{\%}\CyclesMatchChunks{} \& \MergeLinear{}}}%
\end{pgfscope}%
\begin{pgfscope}%
\pgfsetrectcap%
\pgfsetroundjoin%
\pgfsetlinewidth{1.505625pt}%
\pgfsetstrokecolor{currentstroke4}%
\pgfsetdash{}{0pt}%
\pgfpathmoveto{\pgfqpoint{5.240464in}{2.274288in}}%
\pgfpathlineto{\pgfqpoint{5.407130in}{2.274288in}}%
\pgfpathlineto{\pgfqpoint{5.573797in}{2.274288in}}%
\pgfusepath{stroke}%
\end{pgfscope}%
\begin{pgfscope}%
\definecolor{textcolor}{rgb}{0.000000,0.000000,0.000000}%
\pgfsetstrokecolor{textcolor}%
\pgfsetfillcolor{textcolor}%
\pgftext[x=5.707130in,y=2.215954in,left,base]{\color{textcolor}{\rmfamily\fontsize{12.000000}{14.400000}\selectfont\catcode`\^=\active\def^{\ifmmode\sp\else\^{}\fi}\catcode`\%=\active\def%{\%}\Neighbors{} \& \MergeLinear{}}}%
\end{pgfscope}%
\begin{pgfscope}%
\pgfsetrectcap%
\pgfsetroundjoin%
\pgfsetlinewidth{1.505625pt}%
\pgfsetstrokecolor{currentstroke5}%
\pgfsetdash{}{0pt}%
\pgfpathmoveto{\pgfqpoint{5.240464in}{2.029659in}}%
\pgfpathlineto{\pgfqpoint{5.407130in}{2.029659in}}%
\pgfpathlineto{\pgfqpoint{5.573797in}{2.029659in}}%
\pgfusepath{stroke}%
\end{pgfscope}%
\begin{pgfscope}%
\definecolor{textcolor}{rgb}{0.000000,0.000000,0.000000}%
\pgfsetstrokecolor{textcolor}%
\pgfsetfillcolor{textcolor}%
\pgftext[x=5.707130in,y=1.971325in,left,base]{\color{textcolor}{\rmfamily\fontsize{12.000000}{14.400000}\selectfont\catcode`\^=\active\def^{\ifmmode\sp\else\^{}\fi}\catcode`\%=\active\def%{\%}\NeighborsDegree{} \& \MergeLinear{}}}%
\end{pgfscope}%
\begin{pgfscope}%
\pgfsetrectcap%
\pgfsetroundjoin%
\pgfsetlinewidth{1.505625pt}%
\pgfsetstrokecolor{currentstroke6}%
\pgfsetdash{}{0pt}%
\pgfpathmoveto{\pgfqpoint{5.240464in}{1.780391in}}%
\pgfpathlineto{\pgfqpoint{5.407130in}{1.780391in}}%
\pgfpathlineto{\pgfqpoint{5.573797in}{1.780391in}}%
\pgfusepath{stroke}%
\end{pgfscope}%
\begin{pgfscope}%
\definecolor{textcolor}{rgb}{0.000000,0.000000,0.000000}%
\pgfsetstrokecolor{textcolor}%
\pgfsetfillcolor{textcolor}%
\pgftext[x=5.707130in,y=1.722058in,left,base]{\color{textcolor}{\rmfamily\fontsize{12.000000}{14.400000}\selectfont\catcode`\^=\active\def^{\ifmmode\sp\else\^{}\fi}\catcode`\%=\active\def%{\%}\None{} \& \MergeLinear{}}}%
\end{pgfscope}%
\begin{pgfscope}%
\pgfsetrectcap%
\pgfsetroundjoin%
\pgfsetlinewidth{1.505625pt}%
\pgfsetstrokecolor{currentstroke1}%
\pgfsetdash{}{0pt}%
\pgfpathmoveto{\pgfqpoint{5.240464in}{1.535763in}}%
\pgfpathlineto{\pgfqpoint{5.407130in}{1.535763in}}%
\pgfpathlineto{\pgfqpoint{5.573797in}{1.535763in}}%
\pgfusepath{stroke}%
\end{pgfscope}%
\begin{pgfscope}%
\definecolor{textcolor}{rgb}{0.000000,0.000000,0.000000}%
\pgfsetstrokecolor{textcolor}%
\pgfsetfillcolor{textcolor}%
\pgftext[x=5.707130in,y=1.477429in,left,base]{\color{textcolor}{\rmfamily\fontsize{12.000000}{14.400000}\selectfont\catcode`\^=\active\def^{\ifmmode\sp\else\^{}\fi}\catcode`\%=\active\def%{\%}\Cuts{} \& \MergeLinear{}}}%
\end{pgfscope}%
\begin{pgfscope}%
\pgfsetrectcap%
\pgfsetroundjoin%
\pgfsetlinewidth{1.505625pt}%
\pgfsetstrokecolor{currentstroke3}%
\pgfsetdash{}{0pt}%
\pgfpathmoveto{\pgfqpoint{5.240464in}{1.291134in}}%
\pgfpathlineto{\pgfqpoint{5.407130in}{1.291134in}}%
\pgfpathlineto{\pgfqpoint{5.573797in}{1.291134in}}%
\pgfusepath{stroke}%
\end{pgfscope}%
\begin{pgfscope}%
\definecolor{textcolor}{rgb}{0.000000,0.000000,0.000000}%
\pgfsetstrokecolor{textcolor}%
\pgfsetfillcolor{textcolor}%
\pgftext[x=5.707130in,y=1.232801in,left,base]{\color{textcolor}{\rmfamily\fontsize{12.000000}{14.400000}\selectfont\catcode`\^=\active\def^{\ifmmode\sp\else\^{}\fi}\catcode`\%=\active\def%{\%}\KernighanLin{} \& \MergeLinear{}}}%
\end{pgfscope}%
\end{pgfpicture}%
\makeatother%
\endgroup%
}
	\caption[Failing splitting strategies for minimally rigid graphs]{
		Mean running time to find all NAC-colorings for minimally rigid graphs with failing splitting strategies.}%
	\label{fig:graph_mimimally_rigid_failing_split_first_runtime}
\end{figure}%


As shown in \Cref{fig:graph_no_nac_coloring_generated_rigid_failing_merging_first_runtime},
strategies \Log{}, \SortedBits{} and \MinMax{} fail,
\PromisingCycles{} perform well on the other hand.
%
It can be seen that strategies \SortedSize{} and \Score{} perform
as well as our preferred strategies on graphs with no NAC-coloring.
But as they must list all the NAC-colorings on each subgraph,
they are not suitable for cases where only one NAC-coloring is requested.
%
In \Cref{fig:graph_no_nac_coloring_generated_rigid_failing_split_first_runtime}
it can be seen that performance of \KernighanLin{} and \Cuts{} is worse.
%
\begin{figure}[thbp]
	\centering
	\scalebox{\BenchFigureScale}{%% Creator: Matplotlib, PGF backend
%%
%% To include the figure in your LaTeX document, write
%%   \input{<filename>.pgf}
%%
%% Make sure the required packages are loaded in your preamble
%%   \usepackage{pgf}
%%
%% Also ensure that all the required font packages are loaded; for instance,
%% the lmodern package is sometimes necessary when using math font.
%%   \usepackage{lmodern}
%%
%% Figures using additional raster images can only be included by \input if
%% they are in the same directory as the main LaTeX file. For loading figures
%% from other directories you can use the `import` package
%%   \usepackage{import}
%%
%% and then include the figures with
%%   \import{<path to file>}{<filename>.pgf}
%%
%% Matplotlib used the following preamble
%%   \def\mathdefault#1{#1}
%%   \everymath=\expandafter{\the\everymath\displaystyle}
%%   \IfFileExists{scrextend.sty}{
%%     \usepackage[fontsize=10.000000pt]{scrextend}
%%   }{
%%     \renewcommand{\normalsize}{\fontsize{10.000000}{12.000000}\selectfont}
%%     \normalsize
%%   }
%%   
%%   \ifdefined\pdftexversion\else  % non-pdftex case.
%%     \usepackage{fontspec}
%%     \setmainfont{DejaVuSans.ttf}[Path=\detokenize{/home/petr/Projects/PyRigi/.venv/lib/python3.12/site-packages/matplotlib/mpl-data/fonts/ttf/}]
%%     \setsansfont{DejaVuSans.ttf}[Path=\detokenize{/home/petr/Projects/PyRigi/.venv/lib/python3.12/site-packages/matplotlib/mpl-data/fonts/ttf/}]
%%     \setmonofont{DejaVuSansMono.ttf}[Path=\detokenize{/home/petr/Projects/PyRigi/.venv/lib/python3.12/site-packages/matplotlib/mpl-data/fonts/ttf/}]
%%   \fi
%%   \makeatletter\@ifpackageloaded{under\Score{}}{}{\usepackage[strings]{under\Score{}}}\makeatother
%%
\begingroup%
\makeatletter%
\begin{pgfpicture}%
\pgfpathrectangle{\pgfpointorigin}{\pgfqpoint{8.384376in}{2.841849in}}%
\pgfusepath{use as bounding box, clip}%
\begin{pgfscope}%
\pgfsetbuttcap%
\pgfsetmiterjoin%
\definecolor{currentfill}{rgb}{1.000000,1.000000,1.000000}%
\pgfsetfillcolor{currentfill}%
\pgfsetlinewidth{0.000000pt}%
\definecolor{currentstroke}{rgb}{1.000000,1.000000,1.000000}%
\pgfsetstrokecolor{currentstroke}%
\pgfsetdash{}{0pt}%
\pgfpathmoveto{\pgfqpoint{0.000000in}{0.000000in}}%
\pgfpathlineto{\pgfqpoint{8.384376in}{0.000000in}}%
\pgfpathlineto{\pgfqpoint{8.384376in}{2.841849in}}%
\pgfpathlineto{\pgfqpoint{0.000000in}{2.841849in}}%
\pgfpathlineto{\pgfqpoint{0.000000in}{0.000000in}}%
\pgfpathclose%
\pgfusepath{fill}%
\end{pgfscope}%
\begin{pgfscope}%
\pgfsetbuttcap%
\pgfsetmiterjoin%
\definecolor{currentfill}{rgb}{1.000000,1.000000,1.000000}%
\pgfsetfillcolor{currentfill}%
\pgfsetlinewidth{0.000000pt}%
\definecolor{currentstroke}{rgb}{0.000000,0.000000,0.000000}%
\pgfsetstrokecolor{currentstroke}%
\pgfsetstrokeopacity{0.000000}%
\pgfsetdash{}{0pt}%
\pgfpathmoveto{\pgfqpoint{0.588387in}{0.521603in}}%
\pgfpathlineto{\pgfqpoint{5.109344in}{0.521603in}}%
\pgfpathlineto{\pgfqpoint{5.109344in}{2.741849in}}%
\pgfpathlineto{\pgfqpoint{0.588387in}{2.741849in}}%
\pgfpathlineto{\pgfqpoint{0.588387in}{0.521603in}}%
\pgfpathclose%
\pgfusepath{fill}%
\end{pgfscope}%
\begin{pgfscope}%
\pgfsetbuttcap%
\pgfsetroundjoin%
\definecolor{currentfill}{rgb}{0.000000,0.000000,0.000000}%
\pgfsetfillcolor{currentfill}%
\pgfsetlinewidth{0.803000pt}%
\definecolor{currentstroke}{rgb}{0.000000,0.000000,0.000000}%
\pgfsetstrokecolor{currentstroke}%
\pgfsetdash{}{0pt}%
\pgfsys@defobject{currentmarker}{\pgfqpoint{0.000000in}{-0.048611in}}{\pgfqpoint{0.000000in}{0.000000in}}{%
\pgfpathmoveto{\pgfqpoint{0.000000in}{0.000000in}}%
\pgfpathlineto{\pgfqpoint{0.000000in}{-0.048611in}}%
\pgfusepath{stroke,fill}%
}%
\begin{pgfscope}%
\pgfsys@transformshift{0.653773in}{0.521603in}%
\pgfsys@useobject{currentmarker}{}%
\end{pgfscope}%
\end{pgfscope}%
\begin{pgfscope}%
\definecolor{textcolor}{rgb}{0.000000,0.000000,0.000000}%
\pgfsetstrokecolor{textcolor}%
\pgfsetfillcolor{textcolor}%
\pgftext[x=0.653773in,y=0.424381in,,top]{\color{textcolor}{\rmfamily\fontsize{10.000000}{12.000000}\selectfont\catcode`\^=\active\def^{\ifmmode\sp\else\^{}\fi}\catcode`\%=\active\def%{\%}$\mathdefault{10}$}}%
\end{pgfscope}%
\begin{pgfscope}%
\pgfsetbuttcap%
\pgfsetroundjoin%
\definecolor{currentfill}{rgb}{0.000000,0.000000,0.000000}%
\pgfsetfillcolor{currentfill}%
\pgfsetlinewidth{0.803000pt}%
\definecolor{currentstroke}{rgb}{0.000000,0.000000,0.000000}%
\pgfsetstrokecolor{currentstroke}%
\pgfsetdash{}{0pt}%
\pgfsys@defobject{currentmarker}{\pgfqpoint{0.000000in}{-0.048611in}}{\pgfqpoint{0.000000in}{0.000000in}}{%
\pgfpathmoveto{\pgfqpoint{0.000000in}{0.000000in}}%
\pgfpathlineto{\pgfqpoint{0.000000in}{-0.048611in}}%
\pgfusepath{stroke,fill}%
}%
\begin{pgfscope}%
\pgfsys@transformshift{1.120814in}{0.521603in}%
\pgfsys@useobject{currentmarker}{}%
\end{pgfscope}%
\end{pgfscope}%
\begin{pgfscope}%
\definecolor{textcolor}{rgb}{0.000000,0.000000,0.000000}%
\pgfsetstrokecolor{textcolor}%
\pgfsetfillcolor{textcolor}%
\pgftext[x=1.120814in,y=0.424381in,,top]{\color{textcolor}{\rmfamily\fontsize{10.000000}{12.000000}\selectfont\catcode`\^=\active\def^{\ifmmode\sp\else\^{}\fi}\catcode`\%=\active\def%{\%}$\mathdefault{20}$}}%
\end{pgfscope}%
\begin{pgfscope}%
\pgfsetbuttcap%
\pgfsetroundjoin%
\definecolor{currentfill}{rgb}{0.000000,0.000000,0.000000}%
\pgfsetfillcolor{currentfill}%
\pgfsetlinewidth{0.803000pt}%
\definecolor{currentstroke}{rgb}{0.000000,0.000000,0.000000}%
\pgfsetstrokecolor{currentstroke}%
\pgfsetdash{}{0pt}%
\pgfsys@defobject{currentmarker}{\pgfqpoint{0.000000in}{-0.048611in}}{\pgfqpoint{0.000000in}{0.000000in}}{%
\pgfpathmoveto{\pgfqpoint{0.000000in}{0.000000in}}%
\pgfpathlineto{\pgfqpoint{0.000000in}{-0.048611in}}%
\pgfusepath{stroke,fill}%
}%
\begin{pgfscope}%
\pgfsys@transformshift{1.587855in}{0.521603in}%
\pgfsys@useobject{currentmarker}{}%
\end{pgfscope}%
\end{pgfscope}%
\begin{pgfscope}%
\definecolor{textcolor}{rgb}{0.000000,0.000000,0.000000}%
\pgfsetstrokecolor{textcolor}%
\pgfsetfillcolor{textcolor}%
\pgftext[x=1.587855in,y=0.424381in,,top]{\color{textcolor}{\rmfamily\fontsize{10.000000}{12.000000}\selectfont\catcode`\^=\active\def^{\ifmmode\sp\else\^{}\fi}\catcode`\%=\active\def%{\%}$\mathdefault{30}$}}%
\end{pgfscope}%
\begin{pgfscope}%
\pgfsetbuttcap%
\pgfsetroundjoin%
\definecolor{currentfill}{rgb}{0.000000,0.000000,0.000000}%
\pgfsetfillcolor{currentfill}%
\pgfsetlinewidth{0.803000pt}%
\definecolor{currentstroke}{rgb}{0.000000,0.000000,0.000000}%
\pgfsetstrokecolor{currentstroke}%
\pgfsetdash{}{0pt}%
\pgfsys@defobject{currentmarker}{\pgfqpoint{0.000000in}{-0.048611in}}{\pgfqpoint{0.000000in}{0.000000in}}{%
\pgfpathmoveto{\pgfqpoint{0.000000in}{0.000000in}}%
\pgfpathlineto{\pgfqpoint{0.000000in}{-0.048611in}}%
\pgfusepath{stroke,fill}%
}%
\begin{pgfscope}%
\pgfsys@transformshift{2.054896in}{0.521603in}%
\pgfsys@useobject{currentmarker}{}%
\end{pgfscope}%
\end{pgfscope}%
\begin{pgfscope}%
\definecolor{textcolor}{rgb}{0.000000,0.000000,0.000000}%
\pgfsetstrokecolor{textcolor}%
\pgfsetfillcolor{textcolor}%
\pgftext[x=2.054896in,y=0.424381in,,top]{\color{textcolor}{\rmfamily\fontsize{10.000000}{12.000000}\selectfont\catcode`\^=\active\def^{\ifmmode\sp\else\^{}\fi}\catcode`\%=\active\def%{\%}$\mathdefault{40}$}}%
\end{pgfscope}%
\begin{pgfscope}%
\pgfsetbuttcap%
\pgfsetroundjoin%
\definecolor{currentfill}{rgb}{0.000000,0.000000,0.000000}%
\pgfsetfillcolor{currentfill}%
\pgfsetlinewidth{0.803000pt}%
\definecolor{currentstroke}{rgb}{0.000000,0.000000,0.000000}%
\pgfsetstrokecolor{currentstroke}%
\pgfsetdash{}{0pt}%
\pgfsys@defobject{currentmarker}{\pgfqpoint{0.000000in}{-0.048611in}}{\pgfqpoint{0.000000in}{0.000000in}}{%
\pgfpathmoveto{\pgfqpoint{0.000000in}{0.000000in}}%
\pgfpathlineto{\pgfqpoint{0.000000in}{-0.048611in}}%
\pgfusepath{stroke,fill}%
}%
\begin{pgfscope}%
\pgfsys@transformshift{2.521937in}{0.521603in}%
\pgfsys@useobject{currentmarker}{}%
\end{pgfscope}%
\end{pgfscope}%
\begin{pgfscope}%
\definecolor{textcolor}{rgb}{0.000000,0.000000,0.000000}%
\pgfsetstrokecolor{textcolor}%
\pgfsetfillcolor{textcolor}%
\pgftext[x=2.521937in,y=0.424381in,,top]{\color{textcolor}{\rmfamily\fontsize{10.000000}{12.000000}\selectfont\catcode`\^=\active\def^{\ifmmode\sp\else\^{}\fi}\catcode`\%=\active\def%{\%}$\mathdefault{50}$}}%
\end{pgfscope}%
\begin{pgfscope}%
\pgfsetbuttcap%
\pgfsetroundjoin%
\definecolor{currentfill}{rgb}{0.000000,0.000000,0.000000}%
\pgfsetfillcolor{currentfill}%
\pgfsetlinewidth{0.803000pt}%
\definecolor{currentstroke}{rgb}{0.000000,0.000000,0.000000}%
\pgfsetstrokecolor{currentstroke}%
\pgfsetdash{}{0pt}%
\pgfsys@defobject{currentmarker}{\pgfqpoint{0.000000in}{-0.048611in}}{\pgfqpoint{0.000000in}{0.000000in}}{%
\pgfpathmoveto{\pgfqpoint{0.000000in}{0.000000in}}%
\pgfpathlineto{\pgfqpoint{0.000000in}{-0.048611in}}%
\pgfusepath{stroke,fill}%
}%
\begin{pgfscope}%
\pgfsys@transformshift{2.988978in}{0.521603in}%
\pgfsys@useobject{currentmarker}{}%
\end{pgfscope}%
\end{pgfscope}%
\begin{pgfscope}%
\definecolor{textcolor}{rgb}{0.000000,0.000000,0.000000}%
\pgfsetstrokecolor{textcolor}%
\pgfsetfillcolor{textcolor}%
\pgftext[x=2.988978in,y=0.424381in,,top]{\color{textcolor}{\rmfamily\fontsize{10.000000}{12.000000}\selectfont\catcode`\^=\active\def^{\ifmmode\sp\else\^{}\fi}\catcode`\%=\active\def%{\%}$\mathdefault{60}$}}%
\end{pgfscope}%
\begin{pgfscope}%
\pgfsetbuttcap%
\pgfsetroundjoin%
\definecolor{currentfill}{rgb}{0.000000,0.000000,0.000000}%
\pgfsetfillcolor{currentfill}%
\pgfsetlinewidth{0.803000pt}%
\definecolor{currentstroke}{rgb}{0.000000,0.000000,0.000000}%
\pgfsetstrokecolor{currentstroke}%
\pgfsetdash{}{0pt}%
\pgfsys@defobject{currentmarker}{\pgfqpoint{0.000000in}{-0.048611in}}{\pgfqpoint{0.000000in}{0.000000in}}{%
\pgfpathmoveto{\pgfqpoint{0.000000in}{0.000000in}}%
\pgfpathlineto{\pgfqpoint{0.000000in}{-0.048611in}}%
\pgfusepath{stroke,fill}%
}%
\begin{pgfscope}%
\pgfsys@transformshift{3.456019in}{0.521603in}%
\pgfsys@useobject{currentmarker}{}%
\end{pgfscope}%
\end{pgfscope}%
\begin{pgfscope}%
\definecolor{textcolor}{rgb}{0.000000,0.000000,0.000000}%
\pgfsetstrokecolor{textcolor}%
\pgfsetfillcolor{textcolor}%
\pgftext[x=3.456019in,y=0.424381in,,top]{\color{textcolor}{\rmfamily\fontsize{10.000000}{12.000000}\selectfont\catcode`\^=\active\def^{\ifmmode\sp\else\^{}\fi}\catcode`\%=\active\def%{\%}$\mathdefault{70}$}}%
\end{pgfscope}%
\begin{pgfscope}%
\pgfsetbuttcap%
\pgfsetroundjoin%
\definecolor{currentfill}{rgb}{0.000000,0.000000,0.000000}%
\pgfsetfillcolor{currentfill}%
\pgfsetlinewidth{0.803000pt}%
\definecolor{currentstroke}{rgb}{0.000000,0.000000,0.000000}%
\pgfsetstrokecolor{currentstroke}%
\pgfsetdash{}{0pt}%
\pgfsys@defobject{currentmarker}{\pgfqpoint{0.000000in}{-0.048611in}}{\pgfqpoint{0.000000in}{0.000000in}}{%
\pgfpathmoveto{\pgfqpoint{0.000000in}{0.000000in}}%
\pgfpathlineto{\pgfqpoint{0.000000in}{-0.048611in}}%
\pgfusepath{stroke,fill}%
}%
\begin{pgfscope}%
\pgfsys@transformshift{3.923060in}{0.521603in}%
\pgfsys@useobject{currentmarker}{}%
\end{pgfscope}%
\end{pgfscope}%
\begin{pgfscope}%
\definecolor{textcolor}{rgb}{0.000000,0.000000,0.000000}%
\pgfsetstrokecolor{textcolor}%
\pgfsetfillcolor{textcolor}%
\pgftext[x=3.923060in,y=0.424381in,,top]{\color{textcolor}{\rmfamily\fontsize{10.000000}{12.000000}\selectfont\catcode`\^=\active\def^{\ifmmode\sp\else\^{}\fi}\catcode`\%=\active\def%{\%}$\mathdefault{80}$}}%
\end{pgfscope}%
\begin{pgfscope}%
\pgfsetbuttcap%
\pgfsetroundjoin%
\definecolor{currentfill}{rgb}{0.000000,0.000000,0.000000}%
\pgfsetfillcolor{currentfill}%
\pgfsetlinewidth{0.803000pt}%
\definecolor{currentstroke}{rgb}{0.000000,0.000000,0.000000}%
\pgfsetstrokecolor{currentstroke}%
\pgfsetdash{}{0pt}%
\pgfsys@defobject{currentmarker}{\pgfqpoint{0.000000in}{-0.048611in}}{\pgfqpoint{0.000000in}{0.000000in}}{%
\pgfpathmoveto{\pgfqpoint{0.000000in}{0.000000in}}%
\pgfpathlineto{\pgfqpoint{0.000000in}{-0.048611in}}%
\pgfusepath{stroke,fill}%
}%
\begin{pgfscope}%
\pgfsys@transformshift{4.390101in}{0.521603in}%
\pgfsys@useobject{currentmarker}{}%
\end{pgfscope}%
\end{pgfscope}%
\begin{pgfscope}%
\definecolor{textcolor}{rgb}{0.000000,0.000000,0.000000}%
\pgfsetstrokecolor{textcolor}%
\pgfsetfillcolor{textcolor}%
\pgftext[x=4.390101in,y=0.424381in,,top]{\color{textcolor}{\rmfamily\fontsize{10.000000}{12.000000}\selectfont\catcode`\^=\active\def^{\ifmmode\sp\else\^{}\fi}\catcode`\%=\active\def%{\%}$\mathdefault{90}$}}%
\end{pgfscope}%
\begin{pgfscope}%
\pgfsetbuttcap%
\pgfsetroundjoin%
\definecolor{currentfill}{rgb}{0.000000,0.000000,0.000000}%
\pgfsetfillcolor{currentfill}%
\pgfsetlinewidth{0.803000pt}%
\definecolor{currentstroke}{rgb}{0.000000,0.000000,0.000000}%
\pgfsetstrokecolor{currentstroke}%
\pgfsetdash{}{0pt}%
\pgfsys@defobject{currentmarker}{\pgfqpoint{0.000000in}{-0.048611in}}{\pgfqpoint{0.000000in}{0.000000in}}{%
\pgfpathmoveto{\pgfqpoint{0.000000in}{0.000000in}}%
\pgfpathlineto{\pgfqpoint{0.000000in}{-0.048611in}}%
\pgfusepath{stroke,fill}%
}%
\begin{pgfscope}%
\pgfsys@transformshift{4.857142in}{0.521603in}%
\pgfsys@useobject{currentmarker}{}%
\end{pgfscope}%
\end{pgfscope}%
\begin{pgfscope}%
\definecolor{textcolor}{rgb}{0.000000,0.000000,0.000000}%
\pgfsetstrokecolor{textcolor}%
\pgfsetfillcolor{textcolor}%
\pgftext[x=4.857142in,y=0.424381in,,top]{\color{textcolor}{\rmfamily\fontsize{10.000000}{12.000000}\selectfont\catcode`\^=\active\def^{\ifmmode\sp\else\^{}\fi}\catcode`\%=\active\def%{\%}$\mathdefault{100}$}}%
\end{pgfscope}%
\begin{pgfscope}%
\definecolor{textcolor}{rgb}{0.000000,0.000000,0.000000}%
\pgfsetstrokecolor{textcolor}%
\pgfsetfillcolor{textcolor}%
\pgftext[x=2.848866in,y=0.234413in,,top]{\color{textcolor}{\rmfamily\fontsize{10.000000}{12.000000}\selectfont\catcode`\^=\active\def^{\ifmmode\sp\else\^{}\fi}\catcode`\%=\active\def%{\%}$\triangle$-connected components}}%
\end{pgfscope}%
\begin{pgfscope}%
\pgfsetbuttcap%
\pgfsetroundjoin%
\definecolor{currentfill}{rgb}{0.000000,0.000000,0.000000}%
\pgfsetfillcolor{currentfill}%
\pgfsetlinewidth{0.803000pt}%
\definecolor{currentstroke}{rgb}{0.000000,0.000000,0.000000}%
\pgfsetstrokecolor{currentstroke}%
\pgfsetdash{}{0pt}%
\pgfsys@defobject{currentmarker}{\pgfqpoint{-0.048611in}{0.000000in}}{\pgfqpoint{-0.000000in}{0.000000in}}{%
\pgfpathmoveto{\pgfqpoint{-0.000000in}{0.000000in}}%
\pgfpathlineto{\pgfqpoint{-0.048611in}{0.000000in}}%
\pgfusepath{stroke,fill}%
}%
\begin{pgfscope}%
\pgfsys@transformshift{0.588387in}{0.606626in}%
\pgfsys@useobject{currentmarker}{}%
\end{pgfscope}%
\end{pgfscope}%
\begin{pgfscope}%
\definecolor{textcolor}{rgb}{0.000000,0.000000,0.000000}%
\pgfsetstrokecolor{textcolor}%
\pgfsetfillcolor{textcolor}%
\pgftext[x=0.289968in, y=0.553865in, left, base]{\color{textcolor}{\rmfamily\fontsize{10.000000}{12.000000}\selectfont\catcode`\^=\active\def^{\ifmmode\sp\else\^{}\fi}\catcode`\%=\active\def%{\%}$\mathdefault{10^{2}}$}}%
\end{pgfscope}%
\begin{pgfscope}%
\pgfsetbuttcap%
\pgfsetroundjoin%
\definecolor{currentfill}{rgb}{0.000000,0.000000,0.000000}%
\pgfsetfillcolor{currentfill}%
\pgfsetlinewidth{0.803000pt}%
\definecolor{currentstroke}{rgb}{0.000000,0.000000,0.000000}%
\pgfsetstrokecolor{currentstroke}%
\pgfsetdash{}{0pt}%
\pgfsys@defobject{currentmarker}{\pgfqpoint{-0.048611in}{0.000000in}}{\pgfqpoint{-0.000000in}{0.000000in}}{%
\pgfpathmoveto{\pgfqpoint{-0.000000in}{0.000000in}}%
\pgfpathlineto{\pgfqpoint{-0.048611in}{0.000000in}}%
\pgfusepath{stroke,fill}%
}%
\begin{pgfscope}%
\pgfsys@transformshift{0.588387in}{1.984442in}%
\pgfsys@useobject{currentmarker}{}%
\end{pgfscope}%
\end{pgfscope}%
\begin{pgfscope}%
\definecolor{textcolor}{rgb}{0.000000,0.000000,0.000000}%
\pgfsetstrokecolor{textcolor}%
\pgfsetfillcolor{textcolor}%
\pgftext[x=0.289968in, y=1.931681in, left, base]{\color{textcolor}{\rmfamily\fontsize{10.000000}{12.000000}\selectfont\catcode`\^=\active\def^{\ifmmode\sp\else\^{}\fi}\catcode`\%=\active\def%{\%}$\mathdefault{10^{3}}$}}%
\end{pgfscope}%
\begin{pgfscope}%
\pgfsetbuttcap%
\pgfsetroundjoin%
\definecolor{currentfill}{rgb}{0.000000,0.000000,0.000000}%
\pgfsetfillcolor{currentfill}%
\pgfsetlinewidth{0.602250pt}%
\definecolor{currentstroke}{rgb}{0.000000,0.000000,0.000000}%
\pgfsetstrokecolor{currentstroke}%
\pgfsetdash{}{0pt}%
\pgfsys@defobject{currentmarker}{\pgfqpoint{-0.027778in}{0.000000in}}{\pgfqpoint{-0.000000in}{0.000000in}}{%
\pgfpathmoveto{\pgfqpoint{-0.000000in}{0.000000in}}%
\pgfpathlineto{\pgfqpoint{-0.027778in}{0.000000in}}%
\pgfusepath{stroke,fill}%
}%
\begin{pgfscope}%
\pgfsys@transformshift{0.588387in}{0.543581in}%
\pgfsys@useobject{currentmarker}{}%
\end{pgfscope}%
\end{pgfscope}%
\begin{pgfscope}%
\pgfsetbuttcap%
\pgfsetroundjoin%
\definecolor{currentfill}{rgb}{0.000000,0.000000,0.000000}%
\pgfsetfillcolor{currentfill}%
\pgfsetlinewidth{0.602250pt}%
\definecolor{currentstroke}{rgb}{0.000000,0.000000,0.000000}%
\pgfsetstrokecolor{currentstroke}%
\pgfsetdash{}{0pt}%
\pgfsys@defobject{currentmarker}{\pgfqpoint{-0.027778in}{0.000000in}}{\pgfqpoint{-0.000000in}{0.000000in}}{%
\pgfpathmoveto{\pgfqpoint{-0.000000in}{0.000000in}}%
\pgfpathlineto{\pgfqpoint{-0.027778in}{0.000000in}}%
\pgfusepath{stroke,fill}%
}%
\begin{pgfscope}%
\pgfsys@transformshift{0.588387in}{1.021390in}%
\pgfsys@useobject{currentmarker}{}%
\end{pgfscope}%
\end{pgfscope}%
\begin{pgfscope}%
\pgfsetbuttcap%
\pgfsetroundjoin%
\definecolor{currentfill}{rgb}{0.000000,0.000000,0.000000}%
\pgfsetfillcolor{currentfill}%
\pgfsetlinewidth{0.602250pt}%
\definecolor{currentstroke}{rgb}{0.000000,0.000000,0.000000}%
\pgfsetstrokecolor{currentstroke}%
\pgfsetdash{}{0pt}%
\pgfsys@defobject{currentmarker}{\pgfqpoint{-0.027778in}{0.000000in}}{\pgfqpoint{-0.000000in}{0.000000in}}{%
\pgfpathmoveto{\pgfqpoint{-0.000000in}{0.000000in}}%
\pgfpathlineto{\pgfqpoint{-0.027778in}{0.000000in}}%
\pgfusepath{stroke,fill}%
}%
\begin{pgfscope}%
\pgfsys@transformshift{0.588387in}{1.264012in}%
\pgfsys@useobject{currentmarker}{}%
\end{pgfscope}%
\end{pgfscope}%
\begin{pgfscope}%
\pgfsetbuttcap%
\pgfsetroundjoin%
\definecolor{currentfill}{rgb}{0.000000,0.000000,0.000000}%
\pgfsetfillcolor{currentfill}%
\pgfsetlinewidth{0.602250pt}%
\definecolor{currentstroke}{rgb}{0.000000,0.000000,0.000000}%
\pgfsetstrokecolor{currentstroke}%
\pgfsetdash{}{0pt}%
\pgfsys@defobject{currentmarker}{\pgfqpoint{-0.027778in}{0.000000in}}{\pgfqpoint{-0.000000in}{0.000000in}}{%
\pgfpathmoveto{\pgfqpoint{-0.000000in}{0.000000in}}%
\pgfpathlineto{\pgfqpoint{-0.027778in}{0.000000in}}%
\pgfusepath{stroke,fill}%
}%
\begin{pgfscope}%
\pgfsys@transformshift{0.588387in}{1.436154in}%
\pgfsys@useobject{currentmarker}{}%
\end{pgfscope}%
\end{pgfscope}%
\begin{pgfscope}%
\pgfsetbuttcap%
\pgfsetroundjoin%
\definecolor{currentfill}{rgb}{0.000000,0.000000,0.000000}%
\pgfsetfillcolor{currentfill}%
\pgfsetlinewidth{0.602250pt}%
\definecolor{currentstroke}{rgb}{0.000000,0.000000,0.000000}%
\pgfsetstrokecolor{currentstroke}%
\pgfsetdash{}{0pt}%
\pgfsys@defobject{currentmarker}{\pgfqpoint{-0.027778in}{0.000000in}}{\pgfqpoint{-0.000000in}{0.000000in}}{%
\pgfpathmoveto{\pgfqpoint{-0.000000in}{0.000000in}}%
\pgfpathlineto{\pgfqpoint{-0.027778in}{0.000000in}}%
\pgfusepath{stroke,fill}%
}%
\begin{pgfscope}%
\pgfsys@transformshift{0.588387in}{1.569678in}%
\pgfsys@useobject{currentmarker}{}%
\end{pgfscope}%
\end{pgfscope}%
\begin{pgfscope}%
\pgfsetbuttcap%
\pgfsetroundjoin%
\definecolor{currentfill}{rgb}{0.000000,0.000000,0.000000}%
\pgfsetfillcolor{currentfill}%
\pgfsetlinewidth{0.602250pt}%
\definecolor{currentstroke}{rgb}{0.000000,0.000000,0.000000}%
\pgfsetstrokecolor{currentstroke}%
\pgfsetdash{}{0pt}%
\pgfsys@defobject{currentmarker}{\pgfqpoint{-0.027778in}{0.000000in}}{\pgfqpoint{-0.000000in}{0.000000in}}{%
\pgfpathmoveto{\pgfqpoint{-0.000000in}{0.000000in}}%
\pgfpathlineto{\pgfqpoint{-0.027778in}{0.000000in}}%
\pgfusepath{stroke,fill}%
}%
\begin{pgfscope}%
\pgfsys@transformshift{0.588387in}{1.678775in}%
\pgfsys@useobject{currentmarker}{}%
\end{pgfscope}%
\end{pgfscope}%
\begin{pgfscope}%
\pgfsetbuttcap%
\pgfsetroundjoin%
\definecolor{currentfill}{rgb}{0.000000,0.000000,0.000000}%
\pgfsetfillcolor{currentfill}%
\pgfsetlinewidth{0.602250pt}%
\definecolor{currentstroke}{rgb}{0.000000,0.000000,0.000000}%
\pgfsetstrokecolor{currentstroke}%
\pgfsetdash{}{0pt}%
\pgfsys@defobject{currentmarker}{\pgfqpoint{-0.027778in}{0.000000in}}{\pgfqpoint{-0.000000in}{0.000000in}}{%
\pgfpathmoveto{\pgfqpoint{-0.000000in}{0.000000in}}%
\pgfpathlineto{\pgfqpoint{-0.027778in}{0.000000in}}%
\pgfusepath{stroke,fill}%
}%
\begin{pgfscope}%
\pgfsys@transformshift{0.588387in}{1.771016in}%
\pgfsys@useobject{currentmarker}{}%
\end{pgfscope}%
\end{pgfscope}%
\begin{pgfscope}%
\pgfsetbuttcap%
\pgfsetroundjoin%
\definecolor{currentfill}{rgb}{0.000000,0.000000,0.000000}%
\pgfsetfillcolor{currentfill}%
\pgfsetlinewidth{0.602250pt}%
\definecolor{currentstroke}{rgb}{0.000000,0.000000,0.000000}%
\pgfsetstrokecolor{currentstroke}%
\pgfsetdash{}{0pt}%
\pgfsys@defobject{currentmarker}{\pgfqpoint{-0.027778in}{0.000000in}}{\pgfqpoint{-0.000000in}{0.000000in}}{%
\pgfpathmoveto{\pgfqpoint{-0.000000in}{0.000000in}}%
\pgfpathlineto{\pgfqpoint{-0.027778in}{0.000000in}}%
\pgfusepath{stroke,fill}%
}%
\begin{pgfscope}%
\pgfsys@transformshift{0.588387in}{1.850918in}%
\pgfsys@useobject{currentmarker}{}%
\end{pgfscope}%
\end{pgfscope}%
\begin{pgfscope}%
\pgfsetbuttcap%
\pgfsetroundjoin%
\definecolor{currentfill}{rgb}{0.000000,0.000000,0.000000}%
\pgfsetfillcolor{currentfill}%
\pgfsetlinewidth{0.602250pt}%
\definecolor{currentstroke}{rgb}{0.000000,0.000000,0.000000}%
\pgfsetstrokecolor{currentstroke}%
\pgfsetdash{}{0pt}%
\pgfsys@defobject{currentmarker}{\pgfqpoint{-0.027778in}{0.000000in}}{\pgfqpoint{-0.000000in}{0.000000in}}{%
\pgfpathmoveto{\pgfqpoint{-0.000000in}{0.000000in}}%
\pgfpathlineto{\pgfqpoint{-0.027778in}{0.000000in}}%
\pgfusepath{stroke,fill}%
}%
\begin{pgfscope}%
\pgfsys@transformshift{0.588387in}{1.921397in}%
\pgfsys@useobject{currentmarker}{}%
\end{pgfscope}%
\end{pgfscope}%
\begin{pgfscope}%
\pgfsetbuttcap%
\pgfsetroundjoin%
\definecolor{currentfill}{rgb}{0.000000,0.000000,0.000000}%
\pgfsetfillcolor{currentfill}%
\pgfsetlinewidth{0.602250pt}%
\definecolor{currentstroke}{rgb}{0.000000,0.000000,0.000000}%
\pgfsetstrokecolor{currentstroke}%
\pgfsetdash{}{0pt}%
\pgfsys@defobject{currentmarker}{\pgfqpoint{-0.027778in}{0.000000in}}{\pgfqpoint{-0.000000in}{0.000000in}}{%
\pgfpathmoveto{\pgfqpoint{-0.000000in}{0.000000in}}%
\pgfpathlineto{\pgfqpoint{-0.027778in}{0.000000in}}%
\pgfusepath{stroke,fill}%
}%
\begin{pgfscope}%
\pgfsys@transformshift{0.588387in}{2.399206in}%
\pgfsys@useobject{currentmarker}{}%
\end{pgfscope}%
\end{pgfscope}%
\begin{pgfscope}%
\pgfsetbuttcap%
\pgfsetroundjoin%
\definecolor{currentfill}{rgb}{0.000000,0.000000,0.000000}%
\pgfsetfillcolor{currentfill}%
\pgfsetlinewidth{0.602250pt}%
\definecolor{currentstroke}{rgb}{0.000000,0.000000,0.000000}%
\pgfsetstrokecolor{currentstroke}%
\pgfsetdash{}{0pt}%
\pgfsys@defobject{currentmarker}{\pgfqpoint{-0.027778in}{0.000000in}}{\pgfqpoint{-0.000000in}{0.000000in}}{%
\pgfpathmoveto{\pgfqpoint{-0.000000in}{0.000000in}}%
\pgfpathlineto{\pgfqpoint{-0.027778in}{0.000000in}}%
\pgfusepath{stroke,fill}%
}%
\begin{pgfscope}%
\pgfsys@transformshift{0.588387in}{2.641827in}%
\pgfsys@useobject{currentmarker}{}%
\end{pgfscope}%
\end{pgfscope}%
\begin{pgfscope}%
\definecolor{textcolor}{rgb}{0.000000,0.000000,0.000000}%
\pgfsetstrokecolor{textcolor}%
\pgfsetfillcolor{textcolor}%
\pgftext[x=0.234413in,y=1.631726in,,bottom,rotate=90.000000]{\color{textcolor}{\rmfamily\fontsize{10.000000}{12.000000}\selectfont\catcode`\^=\active\def^{\ifmmode\sp\else\^{}\fi}\catcode`\%=\active\def%{\%}Time [ms]}}%
\end{pgfscope}%
\begin{pgfscope}%
\pgfpathrectangle{\pgfqpoint{0.588387in}{0.521603in}}{\pgfqpoint{4.520957in}{2.220246in}}%
\pgfusepath{clip}%
\pgfsetrectcap%
\pgfsetroundjoin%
\pgfsetlinewidth{1.505625pt}%
\pgfsetstrokecolor{currentstroke1}%
\pgfsetdash{}{0pt}%
\pgfpathmoveto{\pgfqpoint{0.793885in}{0.649902in}}%
\pgfpathlineto{\pgfqpoint{0.840589in}{0.745861in}}%
\pgfpathlineto{\pgfqpoint{0.887293in}{0.773848in}}%
\pgfpathlineto{\pgfqpoint{0.933998in}{0.857813in}}%
\pgfpathlineto{\pgfqpoint{0.980702in}{0.968031in}}%
\pgfpathlineto{\pgfqpoint{1.027406in}{0.999854in}}%
\pgfpathlineto{\pgfqpoint{1.074110in}{0.972700in}}%
\pgfpathlineto{\pgfqpoint{1.120814in}{0.985654in}}%
\pgfpathlineto{\pgfqpoint{1.167518in}{1.013802in}}%
\pgfpathlineto{\pgfqpoint{1.214222in}{0.959498in}}%
\pgfpathlineto{\pgfqpoint{1.260926in}{1.001193in}}%
\pgfpathlineto{\pgfqpoint{1.307630in}{0.968236in}}%
\pgfpathlineto{\pgfqpoint{1.354334in}{0.928526in}}%
\pgfpathlineto{\pgfqpoint{1.401039in}{0.954263in}}%
\pgfpathlineto{\pgfqpoint{1.447743in}{1.038676in}}%
\pgfpathlineto{\pgfqpoint{1.494447in}{1.025861in}}%
\pgfpathlineto{\pgfqpoint{1.541151in}{1.042888in}}%
\pgfpathlineto{\pgfqpoint{1.587855in}{1.032843in}}%
\pgfpathlineto{\pgfqpoint{1.634559in}{1.117759in}}%
\pgfpathlineto{\pgfqpoint{1.681263in}{1.058000in}}%
\pgfpathlineto{\pgfqpoint{1.727967in}{1.117465in}}%
\pgfpathlineto{\pgfqpoint{1.774671in}{1.131982in}}%
\pgfpathlineto{\pgfqpoint{1.821375in}{1.097934in}}%
\pgfpathlineto{\pgfqpoint{1.868080in}{1.166764in}}%
\pgfpathlineto{\pgfqpoint{1.914784in}{1.141980in}}%
\pgfpathlineto{\pgfqpoint{1.961488in}{1.247209in}}%
\pgfpathlineto{\pgfqpoint{2.008192in}{1.188957in}}%
\pgfpathlineto{\pgfqpoint{2.054896in}{1.232923in}}%
\pgfpathlineto{\pgfqpoint{2.101600in}{1.198746in}}%
\pgfpathlineto{\pgfqpoint{2.148304in}{1.306107in}}%
\pgfpathlineto{\pgfqpoint{2.195008in}{1.294607in}}%
\pgfpathlineto{\pgfqpoint{2.241712in}{1.351008in}}%
\pgfpathlineto{\pgfqpoint{2.288416in}{1.340457in}}%
\pgfpathlineto{\pgfqpoint{2.335121in}{1.300796in}}%
\pgfpathlineto{\pgfqpoint{2.381825in}{1.365346in}}%
\pgfpathlineto{\pgfqpoint{2.428529in}{1.353870in}}%
\pgfpathlineto{\pgfqpoint{2.475233in}{1.367216in}}%
\pgfpathlineto{\pgfqpoint{2.521937in}{1.379899in}}%
\pgfpathlineto{\pgfqpoint{2.568641in}{1.402238in}}%
\pgfpathlineto{\pgfqpoint{2.615345in}{1.478872in}}%
\pgfpathlineto{\pgfqpoint{2.662049in}{1.465824in}}%
\pgfpathlineto{\pgfqpoint{2.708753in}{1.488010in}}%
\pgfpathlineto{\pgfqpoint{2.755457in}{1.509630in}}%
\pgfpathlineto{\pgfqpoint{2.802162in}{1.491759in}}%
\pgfpathlineto{\pgfqpoint{2.848866in}{1.626855in}}%
\pgfpathlineto{\pgfqpoint{2.895570in}{1.582588in}}%
\pgfpathlineto{\pgfqpoint{2.942274in}{1.627012in}}%
\pgfpathlineto{\pgfqpoint{2.988978in}{1.618310in}}%
\pgfpathlineto{\pgfqpoint{3.035682in}{1.658886in}}%
\pgfpathlineto{\pgfqpoint{3.082386in}{1.730647in}}%
\pgfpathlineto{\pgfqpoint{3.129090in}{1.685798in}}%
\pgfpathlineto{\pgfqpoint{3.175794in}{1.909195in}}%
\pgfpathlineto{\pgfqpoint{3.222498in}{1.822975in}}%
\pgfpathlineto{\pgfqpoint{3.269202in}{1.742180in}}%
\pgfpathlineto{\pgfqpoint{3.315907in}{1.927899in}}%
\pgfpathlineto{\pgfqpoint{3.362611in}{1.770217in}}%
\pgfpathlineto{\pgfqpoint{3.409315in}{1.713642in}}%
\pgfpathlineto{\pgfqpoint{3.456019in}{1.811546in}}%
\pgfpathlineto{\pgfqpoint{3.502723in}{1.730647in}}%
\pgfpathlineto{\pgfqpoint{3.549427in}{2.015628in}}%
\pgfpathlineto{\pgfqpoint{3.596131in}{1.861789in}}%
\pgfpathlineto{\pgfqpoint{3.642835in}{1.968474in}}%
\pgfpathlineto{\pgfqpoint{3.689539in}{1.826231in}}%
\pgfpathlineto{\pgfqpoint{3.736243in}{1.972608in}}%
\pgfpathlineto{\pgfqpoint{3.829652in}{1.963330in}}%
\pgfpathlineto{\pgfqpoint{3.876356in}{1.866908in}}%
\pgfpathlineto{\pgfqpoint{3.923060in}{1.993056in}}%
\pgfpathlineto{\pgfqpoint{4.016468in}{2.265058in}}%
\pgfpathlineto{\pgfqpoint{4.343397in}{2.265993in}}%
\pgfpathlineto{\pgfqpoint{4.436805in}{2.373531in}}%
\pgfpathlineto{\pgfqpoint{4.483509in}{2.639896in}}%
\pgfpathlineto{\pgfqpoint{4.530213in}{2.430013in}}%
\pgfpathlineto{\pgfqpoint{4.763734in}{2.519092in}}%
\pgfpathlineto{\pgfqpoint{4.810438in}{2.414759in}}%
\pgfpathlineto{\pgfqpoint{4.903846in}{2.323053in}}%
\pgfusepath{stroke}%
\end{pgfscope}%
\begin{pgfscope}%
\pgfpathrectangle{\pgfqpoint{0.588387in}{0.521603in}}{\pgfqpoint{4.520957in}{2.220246in}}%
\pgfusepath{clip}%
\pgfsetrectcap%
\pgfsetroundjoin%
\pgfsetlinewidth{1.505625pt}%
\pgfsetstrokecolor{currentstroke2}%
\pgfsetdash{}{0pt}%
\pgfpathmoveto{\pgfqpoint{0.793885in}{0.622524in}}%
\pgfpathlineto{\pgfqpoint{0.840589in}{0.733923in}}%
\pgfpathlineto{\pgfqpoint{0.887293in}{0.796134in}}%
\pgfpathlineto{\pgfqpoint{0.933998in}{0.951684in}}%
\pgfpathlineto{\pgfqpoint{0.980702in}{1.036818in}}%
\pgfpathlineto{\pgfqpoint{1.027406in}{1.050289in}}%
\pgfpathlineto{\pgfqpoint{1.074110in}{1.070971in}}%
\pgfpathlineto{\pgfqpoint{1.120814in}{1.147411in}}%
\pgfpathlineto{\pgfqpoint{1.167518in}{1.152932in}}%
\pgfpathlineto{\pgfqpoint{1.214222in}{1.089871in}}%
\pgfpathlineto{\pgfqpoint{1.260926in}{1.232027in}}%
\pgfpathlineto{\pgfqpoint{1.307630in}{1.258770in}}%
\pgfpathlineto{\pgfqpoint{1.354334in}{1.096175in}}%
\pgfpathlineto{\pgfqpoint{1.401039in}{1.037345in}}%
\pgfpathlineto{\pgfqpoint{1.447743in}{1.168430in}}%
\pgfpathlineto{\pgfqpoint{1.494447in}{1.298478in}}%
\pgfpathlineto{\pgfqpoint{1.541151in}{1.351962in}}%
\pgfpathlineto{\pgfqpoint{1.587855in}{1.203292in}}%
\pgfpathlineto{\pgfqpoint{1.634559in}{1.270295in}}%
\pgfpathlineto{\pgfqpoint{1.681263in}{1.762380in}}%
\pgfpathlineto{\pgfqpoint{1.727967in}{1.847566in}}%
\pgfpathlineto{\pgfqpoint{1.774671in}{1.667781in}}%
\pgfpathlineto{\pgfqpoint{1.821375in}{2.015060in}}%
\pgfpathlineto{\pgfqpoint{1.868080in}{2.128220in}}%
\pgfpathlineto{\pgfqpoint{1.914784in}{1.997800in}}%
\pgfpathlineto{\pgfqpoint{1.961488in}{1.966799in}}%
\pgfpathlineto{\pgfqpoint{2.008192in}{2.101848in}}%
\pgfpathlineto{\pgfqpoint{2.054896in}{2.473895in}}%
\pgfpathlineto{\pgfqpoint{2.101600in}{2.301410in}}%
\pgfpathlineto{\pgfqpoint{2.148304in}{2.293629in}}%
\pgfpathlineto{\pgfqpoint{2.195008in}{2.072531in}}%
\pgfpathlineto{\pgfqpoint{2.241712in}{2.262682in}}%
\pgfpathlineto{\pgfqpoint{2.288416in}{2.247937in}}%
\pgfpathlineto{\pgfqpoint{2.335121in}{2.359842in}}%
\pgfpathlineto{\pgfqpoint{2.381825in}{2.280275in}}%
\pgfpathlineto{\pgfqpoint{2.428529in}{2.330986in}}%
\pgfpathlineto{\pgfqpoint{2.475233in}{2.251891in}}%
\pgfpathlineto{\pgfqpoint{2.521937in}{2.365531in}}%
\pgfpathlineto{\pgfqpoint{2.568641in}{2.331488in}}%
\pgfpathlineto{\pgfqpoint{2.615345in}{2.491404in}}%
\pgfpathlineto{\pgfqpoint{2.662049in}{2.490848in}}%
\pgfpathlineto{\pgfqpoint{2.708753in}{2.467286in}}%
\pgfpathlineto{\pgfqpoint{2.755457in}{2.425354in}}%
\pgfpathlineto{\pgfqpoint{2.802162in}{2.499175in}}%
\pgfpathlineto{\pgfqpoint{2.848866in}{2.512527in}}%
\pgfpathlineto{\pgfqpoint{2.895570in}{2.464074in}}%
\pgfpathlineto{\pgfqpoint{2.942274in}{2.555815in}}%
\pgfusepath{stroke}%
\end{pgfscope}%
\begin{pgfscope}%
\pgfpathrectangle{\pgfqpoint{0.588387in}{0.521603in}}{\pgfqpoint{4.520957in}{2.220246in}}%
\pgfusepath{clip}%
\pgfsetrectcap%
\pgfsetroundjoin%
\pgfsetlinewidth{1.505625pt}%
\pgfsetstrokecolor{currentstroke3}%
\pgfsetdash{}{0pt}%
\pgfpathmoveto{\pgfqpoint{0.793885in}{0.690457in}}%
\pgfpathlineto{\pgfqpoint{0.840589in}{0.785354in}}%
\pgfpathlineto{\pgfqpoint{0.887293in}{0.713652in}}%
\pgfpathlineto{\pgfqpoint{0.933998in}{0.696225in}}%
\pgfpathlineto{\pgfqpoint{0.980702in}{1.072142in}}%
\pgfpathlineto{\pgfqpoint{1.027406in}{0.798486in}}%
\pgfpathlineto{\pgfqpoint{1.074110in}{0.782993in}}%
\pgfpathlineto{\pgfqpoint{1.120814in}{0.866935in}}%
\pgfpathlineto{\pgfqpoint{1.167518in}{0.918108in}}%
\pgfpathlineto{\pgfqpoint{1.214222in}{0.897144in}}%
\pgfpathlineto{\pgfqpoint{1.260926in}{0.920612in}}%
\pgfpathlineto{\pgfqpoint{1.307630in}{1.142954in}}%
\pgfpathlineto{\pgfqpoint{1.354334in}{1.987039in}}%
\pgfpathlineto{\pgfqpoint{1.401039in}{0.789958in}}%
\pgfpathlineto{\pgfqpoint{1.447743in}{0.799609in}}%
\pgfpathlineto{\pgfqpoint{1.494447in}{1.140072in}}%
\pgfpathlineto{\pgfqpoint{1.541151in}{0.910735in}}%
\pgfpathlineto{\pgfqpoint{1.587855in}{0.908090in}}%
\pgfpathlineto{\pgfqpoint{1.634559in}{0.912511in}}%
\pgfpathlineto{\pgfqpoint{1.681263in}{0.987828in}}%
\pgfpathlineto{\pgfqpoint{1.727967in}{0.968508in}}%
\pgfpathlineto{\pgfqpoint{1.774671in}{1.030931in}}%
\pgfpathlineto{\pgfqpoint{1.821375in}{2.223262in}}%
\pgfpathlineto{\pgfqpoint{1.868080in}{2.458138in}}%
\pgfpathlineto{\pgfqpoint{1.914784in}{2.380157in}}%
\pgfpathlineto{\pgfqpoint{2.101600in}{2.151703in}}%
\pgfpathlineto{\pgfqpoint{2.288416in}{1.346774in}}%
\pgfpathlineto{\pgfqpoint{3.035682in}{2.535357in}}%
\pgfusepath{stroke}%
\end{pgfscope}%
\begin{pgfscope}%
\pgfpathrectangle{\pgfqpoint{0.588387in}{0.521603in}}{\pgfqpoint{4.520957in}{2.220246in}}%
\pgfusepath{clip}%
\pgfsetrectcap%
\pgfsetroundjoin%
\pgfsetlinewidth{1.505625pt}%
\pgfsetstrokecolor{currentstroke4}%
\pgfsetdash{}{0pt}%
\pgfpathmoveto{\pgfqpoint{0.793885in}{0.675466in}}%
\pgfpathlineto{\pgfqpoint{0.840589in}{0.773222in}}%
\pgfpathlineto{\pgfqpoint{0.887293in}{0.798309in}}%
\pgfpathlineto{\pgfqpoint{0.933998in}{0.879503in}}%
\pgfpathlineto{\pgfqpoint{0.980702in}{1.000327in}}%
\pgfpathlineto{\pgfqpoint{1.027406in}{1.026382in}}%
\pgfpathlineto{\pgfqpoint{1.074110in}{1.009323in}}%
\pgfpathlineto{\pgfqpoint{1.120814in}{1.028521in}}%
\pgfpathlineto{\pgfqpoint{1.167518in}{1.051379in}}%
\pgfpathlineto{\pgfqpoint{1.214222in}{1.007223in}}%
\pgfpathlineto{\pgfqpoint{1.260926in}{1.048355in}}%
\pgfpathlineto{\pgfqpoint{1.307630in}{1.011604in}}%
\pgfpathlineto{\pgfqpoint{1.354334in}{0.972350in}}%
\pgfpathlineto{\pgfqpoint{1.401039in}{0.985685in}}%
\pgfpathlineto{\pgfqpoint{1.447743in}{1.079448in}}%
\pgfpathlineto{\pgfqpoint{1.494447in}{1.065187in}}%
\pgfpathlineto{\pgfqpoint{1.541151in}{1.081774in}}%
\pgfpathlineto{\pgfqpoint{1.587855in}{1.089443in}}%
\pgfpathlineto{\pgfqpoint{1.634559in}{1.133123in}}%
\pgfpathlineto{\pgfqpoint{1.681263in}{1.090838in}}%
\pgfpathlineto{\pgfqpoint{1.727967in}{1.134711in}}%
\pgfpathlineto{\pgfqpoint{1.774671in}{1.174846in}}%
\pgfpathlineto{\pgfqpoint{1.821375in}{1.109126in}}%
\pgfpathlineto{\pgfqpoint{1.868080in}{1.193061in}}%
\pgfpathlineto{\pgfqpoint{1.914784in}{1.171948in}}%
\pgfpathlineto{\pgfqpoint{1.961488in}{1.207030in}}%
\pgfpathlineto{\pgfqpoint{2.008192in}{1.211283in}}%
\pgfpathlineto{\pgfqpoint{2.054896in}{1.252020in}}%
\pgfpathlineto{\pgfqpoint{2.101600in}{1.205993in}}%
\pgfpathlineto{\pgfqpoint{2.148304in}{1.332863in}}%
\pgfpathlineto{\pgfqpoint{2.195008in}{1.307144in}}%
\pgfpathlineto{\pgfqpoint{2.241712in}{1.356323in}}%
\pgfpathlineto{\pgfqpoint{2.288416in}{1.356563in}}%
\pgfpathlineto{\pgfqpoint{2.335121in}{1.314479in}}%
\pgfpathlineto{\pgfqpoint{2.381825in}{1.375183in}}%
\pgfpathlineto{\pgfqpoint{2.428529in}{1.374195in}}%
\pgfpathlineto{\pgfqpoint{2.475233in}{1.394737in}}%
\pgfpathlineto{\pgfqpoint{2.521937in}{1.416848in}}%
\pgfpathlineto{\pgfqpoint{2.568641in}{1.409092in}}%
\pgfpathlineto{\pgfqpoint{2.615345in}{1.515542in}}%
\pgfpathlineto{\pgfqpoint{2.662049in}{1.490460in}}%
\pgfpathlineto{\pgfqpoint{2.708753in}{1.489776in}}%
\pgfpathlineto{\pgfqpoint{2.755457in}{1.500694in}}%
\pgfpathlineto{\pgfqpoint{2.802162in}{1.509854in}}%
\pgfpathlineto{\pgfqpoint{2.848866in}{1.611504in}}%
\pgfpathlineto{\pgfqpoint{2.895570in}{1.600174in}}%
\pgfpathlineto{\pgfqpoint{2.942274in}{1.611560in}}%
\pgfpathlineto{\pgfqpoint{2.988978in}{1.647558in}}%
\pgfpathlineto{\pgfqpoint{3.035682in}{1.653959in}}%
\pgfpathlineto{\pgfqpoint{3.082386in}{1.725289in}}%
\pgfpathlineto{\pgfqpoint{3.129090in}{1.686086in}}%
\pgfpathlineto{\pgfqpoint{3.175794in}{1.850170in}}%
\pgfpathlineto{\pgfqpoint{3.222498in}{1.809300in}}%
\pgfpathlineto{\pgfqpoint{3.269202in}{1.762059in}}%
\pgfpathlineto{\pgfqpoint{3.315907in}{1.874567in}}%
\pgfpathlineto{\pgfqpoint{3.362611in}{1.816196in}}%
\pgfpathlineto{\pgfqpoint{3.409315in}{1.756742in}}%
\pgfpathlineto{\pgfqpoint{3.456019in}{1.789118in}}%
\pgfpathlineto{\pgfqpoint{3.502723in}{1.742568in}}%
\pgfpathlineto{\pgfqpoint{3.549427in}{1.978126in}}%
\pgfpathlineto{\pgfqpoint{3.596131in}{1.932594in}}%
\pgfpathlineto{\pgfqpoint{3.642835in}{1.963743in}}%
\pgfpathlineto{\pgfqpoint{3.689539in}{1.902828in}}%
\pgfpathlineto{\pgfqpoint{3.736243in}{1.942784in}}%
\pgfpathlineto{\pgfqpoint{3.829652in}{2.026324in}}%
\pgfpathlineto{\pgfqpoint{3.876356in}{1.995998in}}%
\pgfpathlineto{\pgfqpoint{3.923060in}{2.033120in}}%
\pgfpathlineto{\pgfqpoint{4.016468in}{2.153284in}}%
\pgfpathlineto{\pgfqpoint{4.343397in}{2.256448in}}%
\pgfpathlineto{\pgfqpoint{4.436805in}{2.289690in}}%
\pgfpathlineto{\pgfqpoint{4.483509in}{2.617192in}}%
\pgfpathlineto{\pgfqpoint{4.530213in}{2.386812in}}%
\pgfpathlineto{\pgfqpoint{4.763734in}{2.441506in}}%
\pgfpathlineto{\pgfqpoint{4.810438in}{2.391982in}}%
\pgfpathlineto{\pgfqpoint{4.903846in}{2.294699in}}%
\pgfusepath{stroke}%
\end{pgfscope}%
\begin{pgfscope}%
\pgfpathrectangle{\pgfqpoint{0.588387in}{0.521603in}}{\pgfqpoint{4.520957in}{2.220246in}}%
\pgfusepath{clip}%
\pgfsetrectcap%
\pgfsetroundjoin%
\pgfsetlinewidth{1.505625pt}%
\pgfsetstrokecolor{currentstroke5}%
\pgfsetdash{}{0pt}%
\pgfpathmoveto{\pgfqpoint{0.793885in}{0.664703in}}%
\pgfpathlineto{\pgfqpoint{0.840589in}{0.774419in}}%
\pgfpathlineto{\pgfqpoint{0.887293in}{0.786408in}}%
\pgfpathlineto{\pgfqpoint{0.933998in}{0.870286in}}%
\pgfpathlineto{\pgfqpoint{0.980702in}{0.991605in}}%
\pgfpathlineto{\pgfqpoint{1.027406in}{1.012445in}}%
\pgfpathlineto{\pgfqpoint{1.074110in}{0.999512in}}%
\pgfpathlineto{\pgfqpoint{1.120814in}{1.007930in}}%
\pgfpathlineto{\pgfqpoint{1.167518in}{1.031177in}}%
\pgfpathlineto{\pgfqpoint{1.214222in}{0.982348in}}%
\pgfpathlineto{\pgfqpoint{1.260926in}{1.030484in}}%
\pgfpathlineto{\pgfqpoint{1.307630in}{0.996746in}}%
\pgfpathlineto{\pgfqpoint{1.354334in}{0.948879in}}%
\pgfpathlineto{\pgfqpoint{1.401039in}{0.981650in}}%
\pgfpathlineto{\pgfqpoint{1.447743in}{1.057510in}}%
\pgfpathlineto{\pgfqpoint{1.494447in}{1.044751in}}%
\pgfpathlineto{\pgfqpoint{1.541151in}{1.052035in}}%
\pgfpathlineto{\pgfqpoint{1.587855in}{1.052882in}}%
\pgfpathlineto{\pgfqpoint{1.634559in}{1.111091in}}%
\pgfpathlineto{\pgfqpoint{1.681263in}{1.064735in}}%
\pgfpathlineto{\pgfqpoint{1.727967in}{1.121697in}}%
\pgfpathlineto{\pgfqpoint{1.774671in}{1.132775in}}%
\pgfpathlineto{\pgfqpoint{1.821375in}{1.081474in}}%
\pgfpathlineto{\pgfqpoint{1.868080in}{1.159321in}}%
\pgfpathlineto{\pgfqpoint{1.914784in}{1.140608in}}%
\pgfpathlineto{\pgfqpoint{1.961488in}{1.177664in}}%
\pgfpathlineto{\pgfqpoint{2.008192in}{1.152203in}}%
\pgfpathlineto{\pgfqpoint{2.054896in}{1.245100in}}%
\pgfpathlineto{\pgfqpoint{2.101600in}{1.177808in}}%
\pgfpathlineto{\pgfqpoint{2.148304in}{1.301304in}}%
\pgfpathlineto{\pgfqpoint{2.195008in}{1.269130in}}%
\pgfpathlineto{\pgfqpoint{2.241712in}{1.319682in}}%
\pgfpathlineto{\pgfqpoint{2.288416in}{1.315828in}}%
\pgfpathlineto{\pgfqpoint{2.335121in}{1.257191in}}%
\pgfpathlineto{\pgfqpoint{2.381825in}{1.301226in}}%
\pgfpathlineto{\pgfqpoint{2.428529in}{1.334286in}}%
\pgfpathlineto{\pgfqpoint{2.475233in}{1.347570in}}%
\pgfpathlineto{\pgfqpoint{2.521937in}{1.368660in}}%
\pgfpathlineto{\pgfqpoint{2.568641in}{1.360086in}}%
\pgfpathlineto{\pgfqpoint{2.615345in}{1.456088in}}%
\pgfpathlineto{\pgfqpoint{2.662049in}{1.452387in}}%
\pgfpathlineto{\pgfqpoint{2.708753in}{1.420877in}}%
\pgfpathlineto{\pgfqpoint{2.755457in}{1.423132in}}%
\pgfpathlineto{\pgfqpoint{2.802162in}{1.465044in}}%
\pgfpathlineto{\pgfqpoint{2.848866in}{1.597390in}}%
\pgfpathlineto{\pgfqpoint{2.895570in}{1.492019in}}%
\pgfpathlineto{\pgfqpoint{2.942274in}{1.573456in}}%
\pgfpathlineto{\pgfqpoint{2.988978in}{1.590552in}}%
\pgfpathlineto{\pgfqpoint{3.035682in}{1.613510in}}%
\pgfpathlineto{\pgfqpoint{3.082386in}{1.738521in}}%
\pgfpathlineto{\pgfqpoint{3.129090in}{1.619596in}}%
\pgfpathlineto{\pgfqpoint{3.175794in}{1.721354in}}%
\pgfpathlineto{\pgfqpoint{3.222498in}{1.780556in}}%
\pgfpathlineto{\pgfqpoint{3.269202in}{1.737165in}}%
\pgfpathlineto{\pgfqpoint{3.315907in}{1.797354in}}%
\pgfpathlineto{\pgfqpoint{3.362611in}{1.735535in}}%
\pgfpathlineto{\pgfqpoint{3.409315in}{1.640689in}}%
\pgfpathlineto{\pgfqpoint{3.456019in}{1.721354in}}%
\pgfpathlineto{\pgfqpoint{3.502723in}{1.690136in}}%
\pgfpathlineto{\pgfqpoint{3.549427in}{1.898526in}}%
\pgfpathlineto{\pgfqpoint{3.596131in}{1.838320in}}%
\pgfpathlineto{\pgfqpoint{3.642835in}{1.879400in}}%
\pgfpathlineto{\pgfqpoint{3.689539in}{1.852039in}}%
\pgfpathlineto{\pgfqpoint{3.736243in}{1.901455in}}%
\pgfpathlineto{\pgfqpoint{3.829652in}{1.906929in}}%
\pgfpathlineto{\pgfqpoint{3.876356in}{1.748367in}}%
\pgfpathlineto{\pgfqpoint{3.923060in}{1.899044in}}%
\pgfpathlineto{\pgfqpoint{4.016468in}{2.057574in}}%
\pgfpathlineto{\pgfqpoint{4.343397in}{2.105144in}}%
\pgfpathlineto{\pgfqpoint{4.436805in}{2.065465in}}%
\pgfpathlineto{\pgfqpoint{4.483509in}{2.382520in}}%
\pgfpathlineto{\pgfqpoint{4.530213in}{2.209043in}}%
\pgfpathlineto{\pgfqpoint{4.763734in}{2.195740in}}%
\pgfpathlineto{\pgfqpoint{4.810438in}{2.263996in}}%
\pgfpathlineto{\pgfqpoint{4.903846in}{2.124867in}}%
\pgfusepath{stroke}%
\end{pgfscope}%
\begin{pgfscope}%
\pgfpathrectangle{\pgfqpoint{0.588387in}{0.521603in}}{\pgfqpoint{4.520957in}{2.220246in}}%
\pgfusepath{clip}%
\pgfsetrectcap%
\pgfsetroundjoin%
\pgfsetlinewidth{1.505625pt}%
\pgfsetstrokecolor{currentstroke6}%
\pgfsetdash{}{0pt}%
\pgfpathmoveto{\pgfqpoint{0.793885in}{0.652962in}}%
\pgfpathlineto{\pgfqpoint{0.840589in}{0.755487in}}%
\pgfpathlineto{\pgfqpoint{0.887293in}{0.768572in}}%
\pgfpathlineto{\pgfqpoint{0.933998in}{0.867925in}}%
\pgfpathlineto{\pgfqpoint{0.980702in}{0.968381in}}%
\pgfpathlineto{\pgfqpoint{1.027406in}{1.011617in}}%
\pgfpathlineto{\pgfqpoint{1.074110in}{0.988964in}}%
\pgfpathlineto{\pgfqpoint{1.120814in}{0.993423in}}%
\pgfpathlineto{\pgfqpoint{1.167518in}{1.031492in}}%
\pgfpathlineto{\pgfqpoint{1.214222in}{0.968730in}}%
\pgfpathlineto{\pgfqpoint{1.260926in}{0.995933in}}%
\pgfpathlineto{\pgfqpoint{1.307630in}{0.987329in}}%
\pgfpathlineto{\pgfqpoint{1.354334in}{0.936130in}}%
\pgfpathlineto{\pgfqpoint{1.401039in}{0.958432in}}%
\pgfpathlineto{\pgfqpoint{1.447743in}{1.036548in}}%
\pgfpathlineto{\pgfqpoint{1.494447in}{1.036007in}}%
\pgfpathlineto{\pgfqpoint{1.541151in}{1.070753in}}%
\pgfpathlineto{\pgfqpoint{1.587855in}{1.050949in}}%
\pgfpathlineto{\pgfqpoint{1.634559in}{1.106060in}}%
\pgfpathlineto{\pgfqpoint{1.681263in}{1.062214in}}%
\pgfpathlineto{\pgfqpoint{1.727967in}{1.124441in}}%
\pgfpathlineto{\pgfqpoint{1.774671in}{1.127408in}}%
\pgfpathlineto{\pgfqpoint{1.821375in}{1.109263in}}%
\pgfpathlineto{\pgfqpoint{1.868080in}{1.180725in}}%
\pgfpathlineto{\pgfqpoint{1.914784in}{1.136540in}}%
\pgfpathlineto{\pgfqpoint{1.961488in}{1.206632in}}%
\pgfpathlineto{\pgfqpoint{2.008192in}{1.181713in}}%
\pgfpathlineto{\pgfqpoint{2.054896in}{1.219558in}}%
\pgfpathlineto{\pgfqpoint{2.101600in}{1.188100in}}%
\pgfpathlineto{\pgfqpoint{2.148304in}{1.321478in}}%
\pgfpathlineto{\pgfqpoint{2.195008in}{1.302372in}}%
\pgfpathlineto{\pgfqpoint{2.241712in}{1.354883in}}%
\pgfpathlineto{\pgfqpoint{2.288416in}{1.329342in}}%
\pgfpathlineto{\pgfqpoint{2.335121in}{1.316631in}}%
\pgfpathlineto{\pgfqpoint{2.381825in}{1.348865in}}%
\pgfpathlineto{\pgfqpoint{2.428529in}{1.342978in}}%
\pgfpathlineto{\pgfqpoint{2.475233in}{1.367053in}}%
\pgfpathlineto{\pgfqpoint{2.521937in}{1.388303in}}%
\pgfpathlineto{\pgfqpoint{2.568641in}{1.399428in}}%
\pgfpathlineto{\pgfqpoint{2.615345in}{1.479429in}}%
\pgfpathlineto{\pgfqpoint{2.662049in}{1.469962in}}%
\pgfpathlineto{\pgfqpoint{2.708753in}{1.481246in}}%
\pgfpathlineto{\pgfqpoint{2.755457in}{1.482744in}}%
\pgfpathlineto{\pgfqpoint{2.802162in}{1.478633in}}%
\pgfpathlineto{\pgfqpoint{2.848866in}{1.594756in}}%
\pgfpathlineto{\pgfqpoint{2.895570in}{1.547118in}}%
\pgfpathlineto{\pgfqpoint{2.942274in}{1.622342in}}%
\pgfpathlineto{\pgfqpoint{2.988978in}{1.581674in}}%
\pgfpathlineto{\pgfqpoint{3.035682in}{1.650527in}}%
\pgfpathlineto{\pgfqpoint{3.082386in}{1.672006in}}%
\pgfpathlineto{\pgfqpoint{3.129090in}{1.664903in}}%
\pgfpathlineto{\pgfqpoint{3.175794in}{1.838065in}}%
\pgfpathlineto{\pgfqpoint{3.222498in}{1.818056in}}%
\pgfpathlineto{\pgfqpoint{3.269202in}{1.732443in}}%
\pgfpathlineto{\pgfqpoint{3.315907in}{1.874746in}}%
\pgfpathlineto{\pgfqpoint{3.362611in}{1.762753in}}%
\pgfpathlineto{\pgfqpoint{3.409315in}{1.700323in}}%
\pgfpathlineto{\pgfqpoint{3.456019in}{1.786694in}}%
\pgfpathlineto{\pgfqpoint{3.502723in}{1.722977in}}%
\pgfpathlineto{\pgfqpoint{3.549427in}{1.961882in}}%
\pgfpathlineto{\pgfqpoint{3.596131in}{1.852412in}}%
\pgfpathlineto{\pgfqpoint{3.642835in}{1.939407in}}%
\pgfpathlineto{\pgfqpoint{3.689539in}{1.869150in}}%
\pgfpathlineto{\pgfqpoint{3.736243in}{1.974563in}}%
\pgfpathlineto{\pgfqpoint{3.829652in}{2.050650in}}%
\pgfpathlineto{\pgfqpoint{3.876356in}{1.829599in}}%
\pgfpathlineto{\pgfqpoint{3.923060in}{1.994970in}}%
\pgfpathlineto{\pgfqpoint{4.016468in}{2.200764in}}%
\pgfpathlineto{\pgfqpoint{4.343397in}{2.238815in}}%
\pgfpathlineto{\pgfqpoint{4.436805in}{2.328802in}}%
\pgfpathlineto{\pgfqpoint{4.483509in}{2.640929in}}%
\pgfpathlineto{\pgfqpoint{4.530213in}{2.370400in}}%
\pgfpathlineto{\pgfqpoint{4.763734in}{2.386048in}}%
\pgfpathlineto{\pgfqpoint{4.810438in}{2.485303in}}%
\pgfpathlineto{\pgfqpoint{4.903846in}{2.268666in}}%
\pgfusepath{stroke}%
\end{pgfscope}%
\begin{pgfscope}%
\pgfpathrectangle{\pgfqpoint{0.588387in}{0.521603in}}{\pgfqpoint{4.520957in}{2.220246in}}%
\pgfusepath{clip}%
\pgfsetrectcap%
\pgfsetroundjoin%
\pgfsetlinewidth{1.505625pt}%
\pgfsetstrokecolor{currentstroke7}%
\pgfsetdash{}{0pt}%
\pgfpathmoveto{\pgfqpoint{0.793885in}{0.651968in}}%
\pgfpathlineto{\pgfqpoint{0.840589in}{0.756492in}}%
\pgfpathlineto{\pgfqpoint{0.887293in}{0.760274in}}%
\pgfpathlineto{\pgfqpoint{0.933998in}{0.840487in}}%
\pgfpathlineto{\pgfqpoint{0.980702in}{0.956202in}}%
\pgfpathlineto{\pgfqpoint{1.027406in}{0.991998in}}%
\pgfpathlineto{\pgfqpoint{1.074110in}{0.974456in}}%
\pgfpathlineto{\pgfqpoint{1.120814in}{0.979262in}}%
\pgfpathlineto{\pgfqpoint{1.167518in}{1.000840in}}%
\pgfpathlineto{\pgfqpoint{1.214222in}{0.946885in}}%
\pgfpathlineto{\pgfqpoint{1.260926in}{0.989436in}}%
\pgfpathlineto{\pgfqpoint{1.307630in}{0.978486in}}%
\pgfpathlineto{\pgfqpoint{1.354334in}{0.927512in}}%
\pgfpathlineto{\pgfqpoint{1.401039in}{0.938873in}}%
\pgfpathlineto{\pgfqpoint{1.447743in}{1.028837in}}%
\pgfpathlineto{\pgfqpoint{1.494447in}{1.090741in}}%
\pgfpathlineto{\pgfqpoint{1.541151in}{1.039989in}}%
\pgfpathlineto{\pgfqpoint{1.587855in}{1.022408in}}%
\pgfpathlineto{\pgfqpoint{1.634559in}{1.084177in}}%
\pgfpathlineto{\pgfqpoint{1.681263in}{1.052367in}}%
\pgfpathlineto{\pgfqpoint{1.727967in}{1.107560in}}%
\pgfpathlineto{\pgfqpoint{1.774671in}{1.118996in}}%
\pgfpathlineto{\pgfqpoint{1.821375in}{1.075483in}}%
\pgfpathlineto{\pgfqpoint{1.868080in}{1.167683in}}%
\pgfpathlineto{\pgfqpoint{1.914784in}{1.123600in}}%
\pgfpathlineto{\pgfqpoint{1.961488in}{1.174671in}}%
\pgfpathlineto{\pgfqpoint{2.008192in}{1.141195in}}%
\pgfpathlineto{\pgfqpoint{2.054896in}{1.198421in}}%
\pgfpathlineto{\pgfqpoint{2.101600in}{1.177675in}}%
\pgfpathlineto{\pgfqpoint{2.148304in}{1.277813in}}%
\pgfpathlineto{\pgfqpoint{2.195008in}{1.256328in}}%
\pgfpathlineto{\pgfqpoint{2.241712in}{1.328611in}}%
\pgfpathlineto{\pgfqpoint{2.288416in}{1.314623in}}%
\pgfpathlineto{\pgfqpoint{2.335121in}{1.252169in}}%
\pgfpathlineto{\pgfqpoint{2.381825in}{1.315363in}}%
\pgfpathlineto{\pgfqpoint{2.428529in}{1.324980in}}%
\pgfpathlineto{\pgfqpoint{2.475233in}{1.371305in}}%
\pgfpathlineto{\pgfqpoint{2.521937in}{1.363379in}}%
\pgfpathlineto{\pgfqpoint{2.568641in}{1.405757in}}%
\pgfpathlineto{\pgfqpoint{2.615345in}{1.511533in}}%
\pgfpathlineto{\pgfqpoint{2.662049in}{1.441738in}}%
\pgfpathlineto{\pgfqpoint{2.708753in}{1.431649in}}%
\pgfpathlineto{\pgfqpoint{2.755457in}{1.451092in}}%
\pgfpathlineto{\pgfqpoint{2.802162in}{1.488309in}}%
\pgfpathlineto{\pgfqpoint{2.848866in}{1.572305in}}%
\pgfpathlineto{\pgfqpoint{2.895570in}{1.498986in}}%
\pgfpathlineto{\pgfqpoint{2.942274in}{1.568080in}}%
\pgfpathlineto{\pgfqpoint{2.988978in}{1.572216in}}%
\pgfpathlineto{\pgfqpoint{3.035682in}{1.607299in}}%
\pgfpathlineto{\pgfqpoint{3.082386in}{1.726671in}}%
\pgfpathlineto{\pgfqpoint{3.129090in}{1.626574in}}%
\pgfpathlineto{\pgfqpoint{3.175794in}{1.790981in}}%
\pgfpathlineto{\pgfqpoint{3.222498in}{1.809100in}}%
\pgfpathlineto{\pgfqpoint{3.269202in}{1.729060in}}%
\pgfpathlineto{\pgfqpoint{3.315907in}{1.857242in}}%
\pgfpathlineto{\pgfqpoint{3.362611in}{1.733718in}}%
\pgfpathlineto{\pgfqpoint{3.409315in}{1.672762in}}%
\pgfpathlineto{\pgfqpoint{3.456019in}{1.773858in}}%
\pgfpathlineto{\pgfqpoint{3.502723in}{1.733991in}}%
\pgfpathlineto{\pgfqpoint{3.549427in}{1.854274in}}%
\pgfpathlineto{\pgfqpoint{3.596131in}{1.815218in}}%
\pgfpathlineto{\pgfqpoint{3.642835in}{1.860195in}}%
\pgfpathlineto{\pgfqpoint{3.689539in}{1.824735in}}%
\pgfpathlineto{\pgfqpoint{3.736243in}{1.909308in}}%
\pgfpathlineto{\pgfqpoint{3.829652in}{1.923388in}}%
\pgfpathlineto{\pgfqpoint{3.876356in}{1.813097in}}%
\pgfpathlineto{\pgfqpoint{3.923060in}{1.931287in}}%
\pgfpathlineto{\pgfqpoint{4.016468in}{2.214771in}}%
\pgfpathlineto{\pgfqpoint{4.343397in}{2.153848in}}%
\pgfpathlineto{\pgfqpoint{4.436805in}{2.336493in}}%
\pgfpathlineto{\pgfqpoint{4.483509in}{2.570300in}}%
\pgfpathlineto{\pgfqpoint{4.530213in}{2.293986in}}%
\pgfpathlineto{\pgfqpoint{4.763734in}{2.353040in}}%
\pgfpathlineto{\pgfqpoint{4.810438in}{2.321352in}}%
\pgfpathlineto{\pgfqpoint{4.903846in}{2.298605in}}%
\pgfusepath{stroke}%
\end{pgfscope}%
\begin{pgfscope}%
\pgfpathrectangle{\pgfqpoint{0.588387in}{0.521603in}}{\pgfqpoint{4.520957in}{2.220246in}}%
\pgfusepath{clip}%
\pgfsetrectcap%
\pgfsetroundjoin%
\pgfsetlinewidth{1.505625pt}%
\definecolor{currentstroke}{rgb}{0.498039,0.498039,0.498039}%
\pgfsetstrokecolor{currentstroke}%
\pgfsetdash{}{0pt}%
\pgfpathmoveto{\pgfqpoint{0.840589in}{0.710716in}}%
\pgfpathlineto{\pgfqpoint{0.887293in}{1.576224in}}%
\pgfpathlineto{\pgfqpoint{0.933998in}{1.089204in}}%
\pgfpathlineto{\pgfqpoint{1.027406in}{1.536777in}}%
\pgfpathlineto{\pgfqpoint{1.074110in}{1.189781in}}%
\pgfpathlineto{\pgfqpoint{1.120814in}{2.583418in}}%
\pgfusepath{stroke}%
\end{pgfscope}%
\begin{pgfscope}%
\pgfsetrectcap%
\pgfsetmiterjoin%
\pgfsetlinewidth{0.803000pt}%
\definecolor{currentstroke}{rgb}{0.000000,0.000000,0.000000}%
\pgfsetstrokecolor{currentstroke}%
\pgfsetdash{}{0pt}%
\pgfpathmoveto{\pgfqpoint{0.588387in}{0.521603in}}%
\pgfpathlineto{\pgfqpoint{0.588387in}{2.741849in}}%
\pgfusepath{stroke}%
\end{pgfscope}%
\begin{pgfscope}%
\pgfsetrectcap%
\pgfsetmiterjoin%
\pgfsetlinewidth{0.803000pt}%
\definecolor{currentstroke}{rgb}{0.000000,0.000000,0.000000}%
\pgfsetstrokecolor{currentstroke}%
\pgfsetdash{}{0pt}%
\pgfpathmoveto{\pgfqpoint{5.109344in}{0.521603in}}%
\pgfpathlineto{\pgfqpoint{5.109344in}{2.741849in}}%
\pgfusepath{stroke}%
\end{pgfscope}%
\begin{pgfscope}%
\pgfsetrectcap%
\pgfsetmiterjoin%
\pgfsetlinewidth{0.803000pt}%
\definecolor{currentstroke}{rgb}{0.000000,0.000000,0.000000}%
\pgfsetstrokecolor{currentstroke}%
\pgfsetdash{}{0pt}%
\pgfpathmoveto{\pgfqpoint{0.588387in}{0.521603in}}%
\pgfpathlineto{\pgfqpoint{5.109344in}{0.521603in}}%
\pgfusepath{stroke}%
\end{pgfscope}%
\begin{pgfscope}%
\pgfsetrectcap%
\pgfsetmiterjoin%
\pgfsetlinewidth{0.803000pt}%
\definecolor{currentstroke}{rgb}{0.000000,0.000000,0.000000}%
\pgfsetstrokecolor{currentstroke}%
\pgfsetdash{}{0pt}%
\pgfpathmoveto{\pgfqpoint{0.588387in}{2.741849in}}%
\pgfpathlineto{\pgfqpoint{5.109344in}{2.741849in}}%
\pgfusepath{stroke}%
\end{pgfscope}%
\begin{pgfscope}%
\pgfsetbuttcap%
\pgfsetmiterjoin%
\definecolor{currentfill}{rgb}{1.000000,1.000000,1.000000}%
\pgfsetfillcolor{currentfill}%
\pgfsetfillopacity{0.800000}%
\pgfsetlinewidth{1.003750pt}%
\definecolor{currentstroke}{rgb}{0.800000,0.800000,0.800000}%
\pgfsetstrokecolor{currentstroke}%
\pgfsetstrokeopacity{0.800000}%
\pgfsetdash{}{0pt}%
\pgfpathmoveto{\pgfqpoint{5.226011in}{0.628293in}}%
\pgfpathlineto{\pgfqpoint{8.251043in}{0.628293in}}%
\pgfpathquadraticcurveto{\pgfqpoint{8.284376in}{0.628293in}}{\pgfqpoint{8.284376in}{0.661626in}}%
\pgfpathlineto{\pgfqpoint{8.284376in}{2.625183in}}%
\pgfpathquadraticcurveto{\pgfqpoint{8.284376in}{2.658516in}}{\pgfqpoint{8.251043in}{2.658516in}}%
\pgfpathlineto{\pgfqpoint{5.226011in}{2.658516in}}%
\pgfpathquadraticcurveto{\pgfqpoint{5.192677in}{2.658516in}}{\pgfqpoint{5.192677in}{2.625183in}}%
\pgfpathlineto{\pgfqpoint{5.192677in}{0.661626in}}%
\pgfpathquadraticcurveto{\pgfqpoint{5.192677in}{0.628293in}}{\pgfqpoint{5.226011in}{0.628293in}}%
\pgfpathlineto{\pgfqpoint{5.226011in}{0.628293in}}%
\pgfpathclose%
\pgfusepath{stroke,fill}%
\end{pgfscope}%
\begin{pgfscope}%
\pgfsetrectcap%
\pgfsetroundjoin%
\pgfsetlinewidth{1.505625pt}%
\pgfsetstrokecolor{currentstroke1}%
\pgfsetdash{}{0pt}%
\pgfpathmoveto{\pgfqpoint{5.259344in}{2.523555in}}%
\pgfpathlineto{\pgfqpoint{5.426011in}{2.523555in}}%
\pgfpathlineto{\pgfqpoint{5.592677in}{2.523555in}}%
\pgfusepath{stroke}%
\end{pgfscope}%
\begin{pgfscope}%
\definecolor{textcolor}{rgb}{0.000000,0.000000,0.000000}%
\pgfsetstrokecolor{textcolor}%
\pgfsetfillcolor{textcolor}%
\pgftext[x=5.726011in,y=2.465222in,left,base]{\color{textcolor}{\rmfamily\fontsize{12.000000}{14.400000}\selectfont\catcode`\^=\active\def^{\ifmmode\sp\else\^{}\fi}\catcode`\%=\active\def%{\%}\Neighbors{} \& \MergeLinear{}}}%
\end{pgfscope}%
\begin{pgfscope}%
\pgfsetrectcap%
\pgfsetroundjoin%
\pgfsetlinewidth{1.505625pt}%
\pgfsetstrokecolor{currentstroke2}%
\pgfsetdash{}{0pt}%
\pgfpathmoveto{\pgfqpoint{5.259344in}{2.278926in}}%
\pgfpathlineto{\pgfqpoint{5.426011in}{2.278926in}}%
\pgfpathlineto{\pgfqpoint{5.592677in}{2.278926in}}%
\pgfusepath{stroke}%
\end{pgfscope}%
\begin{pgfscope}%
\definecolor{textcolor}{rgb}{0.000000,0.000000,0.000000}%
\pgfsetstrokecolor{textcolor}%
\pgfsetfillcolor{textcolor}%
\pgftext[x=5.726011in,y=2.220593in,left,base]{\color{textcolor}{\rmfamily\fontsize{12.000000}{14.400000}\selectfont\catcode`\^=\active\def^{\ifmmode\sp\else\^{}\fi}\catcode`\%=\active\def%{\%}\Neighbors{} \& \Log{}}}%
\end{pgfscope}%
\begin{pgfscope}%
\pgfsetrectcap%
\pgfsetroundjoin%
\pgfsetlinewidth{1.505625pt}%
\pgfsetstrokecolor{currentstroke3}%
\pgfsetdash{}{0pt}%
\pgfpathmoveto{\pgfqpoint{5.259344in}{2.034297in}}%
\pgfpathlineto{\pgfqpoint{5.426011in}{2.034297in}}%
\pgfpathlineto{\pgfqpoint{5.592677in}{2.034297in}}%
\pgfusepath{stroke}%
\end{pgfscope}%
\begin{pgfscope}%
\definecolor{textcolor}{rgb}{0.000000,0.000000,0.000000}%
\pgfsetstrokecolor{textcolor}%
\pgfsetfillcolor{textcolor}%
\pgftext[x=5.726011in,y=1.975964in,left,base]{\color{textcolor}{\rmfamily\fontsize{12.000000}{14.400000}\selectfont\catcode`\^=\active\def^{\ifmmode\sp\else\^{}\fi}\catcode`\%=\active\def%{\%}\Neighbors{} \& \MinMax{}}}%
\end{pgfscope}%
\begin{pgfscope}%
\pgfsetrectcap%
\pgfsetroundjoin%
\pgfsetlinewidth{1.505625pt}%
\pgfsetstrokecolor{currentstroke4}%
\pgfsetdash{}{0pt}%
\pgfpathmoveto{\pgfqpoint{5.259344in}{1.785030in}}%
\pgfpathlineto{\pgfqpoint{5.426011in}{1.785030in}}%
\pgfpathlineto{\pgfqpoint{5.592677in}{1.785030in}}%
\pgfusepath{stroke}%
\end{pgfscope}%
\begin{pgfscope}%
\definecolor{textcolor}{rgb}{0.000000,0.000000,0.000000}%
\pgfsetstrokecolor{textcolor}%
\pgfsetfillcolor{textcolor}%
\pgftext[x=5.726011in,y=1.726697in,left,base]{\color{textcolor}{\rmfamily\fontsize{12.000000}{14.400000}\selectfont\catcode`\^=\active\def^{\ifmmode\sp\else\^{}\fi}\catcode`\%=\active\def%{\%}\Neighbors{} \& \PromisingCycles{}}}%
\end{pgfscope}%
\begin{pgfscope}%
\pgfsetrectcap%
\pgfsetroundjoin%
\pgfsetlinewidth{1.505625pt}%
\pgfsetstrokecolor{currentstroke5}%
\pgfsetdash{}{0pt}%
\pgfpathmoveto{\pgfqpoint{5.259344in}{1.535763in}}%
\pgfpathlineto{\pgfqpoint{5.426011in}{1.535763in}}%
\pgfpathlineto{\pgfqpoint{5.592677in}{1.535763in}}%
\pgfusepath{stroke}%
\end{pgfscope}%
\begin{pgfscope}%
\definecolor{textcolor}{rgb}{0.000000,0.000000,0.000000}%
\pgfsetstrokecolor{textcolor}%
\pgfsetfillcolor{textcolor}%
\pgftext[x=5.726011in,y=1.477429in,left,base]{\color{textcolor}{\rmfamily\fontsize{12.000000}{14.400000}\selectfont\catcode`\^=\active\def^{\ifmmode\sp\else\^{}\fi}\catcode`\%=\active\def%{\%}\Neighbors{} \& \Score{}}}%
\end{pgfscope}%
\begin{pgfscope}%
\pgfsetrectcap%
\pgfsetroundjoin%
\pgfsetlinewidth{1.505625pt}%
\pgfsetstrokecolor{currentstroke6}%
\pgfsetdash{}{0pt}%
\pgfpathmoveto{\pgfqpoint{5.259344in}{1.291134in}}%
\pgfpathlineto{\pgfqpoint{5.426011in}{1.291134in}}%
\pgfpathlineto{\pgfqpoint{5.592677in}{1.291134in}}%
\pgfusepath{stroke}%
\end{pgfscope}%
\begin{pgfscope}%
\definecolor{textcolor}{rgb}{0.000000,0.000000,0.000000}%
\pgfsetstrokecolor{textcolor}%
\pgfsetfillcolor{textcolor}%
\pgftext[x=5.726011in,y=1.232801in,left,base]{\color{textcolor}{\rmfamily\fontsize{12.000000}{14.400000}\selectfont\catcode`\^=\active\def^{\ifmmode\sp\else\^{}\fi}\catcode`\%=\active\def%{\%}\Neighbors{} \& \SharedVertices{}}}%
\end{pgfscope}%
\begin{pgfscope}%
\pgfsetrectcap%
\pgfsetroundjoin%
\pgfsetlinewidth{1.505625pt}%
\definecolor{currentstroke}{rgb}{0.498039,0.498039,0.498039}%
\pgfsetstrokecolor{currentstroke}%
\pgfsetdash{}{0pt}%
\pgfpathmoveto{\pgfqpoint{5.259344in}{1.041867in}}%
\pgfpathlineto{\pgfqpoint{5.426011in}{1.041867in}}%
\pgfpathlineto{\pgfqpoint{5.592677in}{1.041867in}}%
\pgfusepath{stroke}%
\end{pgfscope}%
\begin{pgfscope}%
\definecolor{textcolor}{rgb}{0.000000,0.000000,0.000000}%
\pgfsetstrokecolor{textcolor}%
\pgfsetfillcolor{textcolor}%
\pgftext[x=5.726011in,y=0.983533in,left,base]{\color{textcolor}{\rmfamily\fontsize{12.000000}{14.400000}\selectfont\catcode`\^=\active\def^{\ifmmode\sp\else\^{}\fi}\catcode`\%=\active\def%{\%}\Neighbors{} \& \SortedBits{}}}%
\end{pgfscope}%
\begin{pgfscope}%
\pgfsetrectcap%
\pgfsetroundjoin%
\pgfsetlinewidth{1.505625pt}%
\pgfsetstrokecolor{currentstroke7}%
\pgfsetdash{}{0pt}%
\pgfpathmoveto{\pgfqpoint{5.259344in}{0.792599in}}%
\pgfpathlineto{\pgfqpoint{5.426011in}{0.792599in}}%
\pgfpathlineto{\pgfqpoint{5.592677in}{0.792599in}}%
\pgfusepath{stroke}%
\end{pgfscope}%
\begin{pgfscope}%
\definecolor{textcolor}{rgb}{0.000000,0.000000,0.000000}%
\pgfsetstrokecolor{textcolor}%
\pgfsetfillcolor{textcolor}%
\pgftext[x=5.726011in,y=0.734266in,left,base]{\color{textcolor}{\rmfamily\fontsize{12.000000}{14.400000}\selectfont\catcode`\^=\active\def^{\ifmmode\sp\else\^{}\fi}\catcode`\%=\active\def%{\%}\Neighbors{} \& \SortedSize{}}}%
\end{pgfscope}%
\end{pgfpicture}%
\makeatother%
\endgroup%
}
	\caption[Failing merging strategies for graphs with no NAC-coloring]{
		Mean running time to find all NAC-colorings for graphs with no NAC-coloring with failing merging strategies.}%
	\label{fig:graph_no_nac_coloring_generated_rigid_failing_merging_first_runtime}
\end{figure}%
\begin{figure}[thbp]
	\centering
	\scalebox{\BenchFigureScale}{%% Creator: Matplotlib, PGF backend
%%
%% To include the figure in your LaTeX document, write
%%   \input{<filename>.pgf}
%%
%% Make sure the required packages are loaded in your preamble
%%   \usepackage{pgf}
%%
%% Also ensure that all the required font packages are loaded; for instance,
%% the lmodern package is sometimes necessary when using math font.
%%   \usepackage{lmodern}
%%
%% Figures using additional raster images can only be included by \input if
%% they are in the same directory as the main LaTeX file. For loading figures
%% from other directories you can use the `import` package
%%   \usepackage{import}
%%
%% and then include the figures with
%%   \import{<path to file>}{<filename>.pgf}
%%
%% Matplotlib used the following preamble
%%   \def\mathdefault#1{#1}
%%   \everymath=\expandafter{\the\everymath\displaystyle}
%%   \IfFileExists{scrextend.sty}{
%%     \usepackage[fontsize=10.000000pt]{scrextend}
%%   }{
%%     \renewcommand{\normalsize}{\fontsize{10.000000}{12.000000}\selectfont}
%%     \normalsize
%%   }
%%   
%%   \ifdefined\pdftexversion\else  % non-pdftex case.
%%     \usepackage{fontspec}
%%     \setmainfont{DejaVuSans.ttf}[Path=\detokenize{/home/petr/Projects/PyRigi/.venv/lib/python3.12/site-packages/matplotlib/mpl-data/fonts/ttf/}]
%%     \setsansfont{DejaVuSans.ttf}[Path=\detokenize{/home/petr/Projects/PyRigi/.venv/lib/python3.12/site-packages/matplotlib/mpl-data/fonts/ttf/}]
%%     \setmonofont{DejaVuSansMono.ttf}[Path=\detokenize{/home/petr/Projects/PyRigi/.venv/lib/python3.12/site-packages/matplotlib/mpl-data/fonts/ttf/}]
%%   \fi
%%   \makeatletter\@ifpackageloaded{under\Score{}}{}{\usepackage[strings]{under\Score{}}}\makeatother
%%
\begingroup%
\makeatletter%
\begin{pgfpicture}%
\pgfpathrectangle{\pgfpointorigin}{\pgfqpoint{8.384376in}{2.841849in}}%
\pgfusepath{use as bounding box, clip}%
\begin{pgfscope}%
\pgfsetbuttcap%
\pgfsetmiterjoin%
\definecolor{currentfill}{rgb}{1.000000,1.000000,1.000000}%
\pgfsetfillcolor{currentfill}%
\pgfsetlinewidth{0.000000pt}%
\definecolor{currentstroke}{rgb}{1.000000,1.000000,1.000000}%
\pgfsetstrokecolor{currentstroke}%
\pgfsetdash{}{0pt}%
\pgfpathmoveto{\pgfqpoint{0.000000in}{0.000000in}}%
\pgfpathlineto{\pgfqpoint{8.384376in}{0.000000in}}%
\pgfpathlineto{\pgfqpoint{8.384376in}{2.841849in}}%
\pgfpathlineto{\pgfqpoint{0.000000in}{2.841849in}}%
\pgfpathlineto{\pgfqpoint{0.000000in}{0.000000in}}%
\pgfpathclose%
\pgfusepath{fill}%
\end{pgfscope}%
\begin{pgfscope}%
\pgfsetbuttcap%
\pgfsetmiterjoin%
\definecolor{currentfill}{rgb}{1.000000,1.000000,1.000000}%
\pgfsetfillcolor{currentfill}%
\pgfsetlinewidth{0.000000pt}%
\definecolor{currentstroke}{rgb}{0.000000,0.000000,0.000000}%
\pgfsetstrokecolor{currentstroke}%
\pgfsetstrokeopacity{0.000000}%
\pgfsetdash{}{0pt}%
\pgfpathmoveto{\pgfqpoint{0.588387in}{0.521603in}}%
\pgfpathlineto{\pgfqpoint{5.888942in}{0.521603in}}%
\pgfpathlineto{\pgfqpoint{5.888942in}{2.741849in}}%
\pgfpathlineto{\pgfqpoint{0.588387in}{2.741849in}}%
\pgfpathlineto{\pgfqpoint{0.588387in}{0.521603in}}%
\pgfpathclose%
\pgfusepath{fill}%
\end{pgfscope}%
\begin{pgfscope}%
\pgfsetbuttcap%
\pgfsetroundjoin%
\definecolor{currentfill}{rgb}{0.000000,0.000000,0.000000}%
\pgfsetfillcolor{currentfill}%
\pgfsetlinewidth{0.803000pt}%
\definecolor{currentstroke}{rgb}{0.000000,0.000000,0.000000}%
\pgfsetstrokecolor{currentstroke}%
\pgfsetdash{}{0pt}%
\pgfsys@defobject{currentmarker}{\pgfqpoint{0.000000in}{-0.048611in}}{\pgfqpoint{0.000000in}{0.000000in}}{%
\pgfpathmoveto{\pgfqpoint{0.000000in}{0.000000in}}%
\pgfpathlineto{\pgfqpoint{0.000000in}{-0.048611in}}%
\pgfusepath{stroke,fill}%
}%
\begin{pgfscope}%
\pgfsys@transformshift{0.665048in}{0.521603in}%
\pgfsys@useobject{currentmarker}{}%
\end{pgfscope}%
\end{pgfscope}%
\begin{pgfscope}%
\definecolor{textcolor}{rgb}{0.000000,0.000000,0.000000}%
\pgfsetstrokecolor{textcolor}%
\pgfsetfillcolor{textcolor}%
\pgftext[x=0.665048in,y=0.424381in,,top]{\color{textcolor}{\rmfamily\fontsize{10.000000}{12.000000}\selectfont\catcode`\^=\active\def^{\ifmmode\sp\else\^{}\fi}\catcode`\%=\active\def%{\%}$\mathdefault{10}$}}%
\end{pgfscope}%
\begin{pgfscope}%
\pgfsetbuttcap%
\pgfsetroundjoin%
\definecolor{currentfill}{rgb}{0.000000,0.000000,0.000000}%
\pgfsetfillcolor{currentfill}%
\pgfsetlinewidth{0.803000pt}%
\definecolor{currentstroke}{rgb}{0.000000,0.000000,0.000000}%
\pgfsetstrokecolor{currentstroke}%
\pgfsetdash{}{0pt}%
\pgfsys@defobject{currentmarker}{\pgfqpoint{0.000000in}{-0.048611in}}{\pgfqpoint{0.000000in}{0.000000in}}{%
\pgfpathmoveto{\pgfqpoint{0.000000in}{0.000000in}}%
\pgfpathlineto{\pgfqpoint{0.000000in}{-0.048611in}}%
\pgfusepath{stroke,fill}%
}%
\begin{pgfscope}%
\pgfsys@transformshift{1.212626in}{0.521603in}%
\pgfsys@useobject{currentmarker}{}%
\end{pgfscope}%
\end{pgfscope}%
\begin{pgfscope}%
\definecolor{textcolor}{rgb}{0.000000,0.000000,0.000000}%
\pgfsetstrokecolor{textcolor}%
\pgfsetfillcolor{textcolor}%
\pgftext[x=1.212626in,y=0.424381in,,top]{\color{textcolor}{\rmfamily\fontsize{10.000000}{12.000000}\selectfont\catcode`\^=\active\def^{\ifmmode\sp\else\^{}\fi}\catcode`\%=\active\def%{\%}$\mathdefault{20}$}}%
\end{pgfscope}%
\begin{pgfscope}%
\pgfsetbuttcap%
\pgfsetroundjoin%
\definecolor{currentfill}{rgb}{0.000000,0.000000,0.000000}%
\pgfsetfillcolor{currentfill}%
\pgfsetlinewidth{0.803000pt}%
\definecolor{currentstroke}{rgb}{0.000000,0.000000,0.000000}%
\pgfsetstrokecolor{currentstroke}%
\pgfsetdash{}{0pt}%
\pgfsys@defobject{currentmarker}{\pgfqpoint{0.000000in}{-0.048611in}}{\pgfqpoint{0.000000in}{0.000000in}}{%
\pgfpathmoveto{\pgfqpoint{0.000000in}{0.000000in}}%
\pgfpathlineto{\pgfqpoint{0.000000in}{-0.048611in}}%
\pgfusepath{stroke,fill}%
}%
\begin{pgfscope}%
\pgfsys@transformshift{1.760204in}{0.521603in}%
\pgfsys@useobject{currentmarker}{}%
\end{pgfscope}%
\end{pgfscope}%
\begin{pgfscope}%
\definecolor{textcolor}{rgb}{0.000000,0.000000,0.000000}%
\pgfsetstrokecolor{textcolor}%
\pgfsetfillcolor{textcolor}%
\pgftext[x=1.760204in,y=0.424381in,,top]{\color{textcolor}{\rmfamily\fontsize{10.000000}{12.000000}\selectfont\catcode`\^=\active\def^{\ifmmode\sp\else\^{}\fi}\catcode`\%=\active\def%{\%}$\mathdefault{30}$}}%
\end{pgfscope}%
\begin{pgfscope}%
\pgfsetbuttcap%
\pgfsetroundjoin%
\definecolor{currentfill}{rgb}{0.000000,0.000000,0.000000}%
\pgfsetfillcolor{currentfill}%
\pgfsetlinewidth{0.803000pt}%
\definecolor{currentstroke}{rgb}{0.000000,0.000000,0.000000}%
\pgfsetstrokecolor{currentstroke}%
\pgfsetdash{}{0pt}%
\pgfsys@defobject{currentmarker}{\pgfqpoint{0.000000in}{-0.048611in}}{\pgfqpoint{0.000000in}{0.000000in}}{%
\pgfpathmoveto{\pgfqpoint{0.000000in}{0.000000in}}%
\pgfpathlineto{\pgfqpoint{0.000000in}{-0.048611in}}%
\pgfusepath{stroke,fill}%
}%
\begin{pgfscope}%
\pgfsys@transformshift{2.307782in}{0.521603in}%
\pgfsys@useobject{currentmarker}{}%
\end{pgfscope}%
\end{pgfscope}%
\begin{pgfscope}%
\definecolor{textcolor}{rgb}{0.000000,0.000000,0.000000}%
\pgfsetstrokecolor{textcolor}%
\pgfsetfillcolor{textcolor}%
\pgftext[x=2.307782in,y=0.424381in,,top]{\color{textcolor}{\rmfamily\fontsize{10.000000}{12.000000}\selectfont\catcode`\^=\active\def^{\ifmmode\sp\else\^{}\fi}\catcode`\%=\active\def%{\%}$\mathdefault{40}$}}%
\end{pgfscope}%
\begin{pgfscope}%
\pgfsetbuttcap%
\pgfsetroundjoin%
\definecolor{currentfill}{rgb}{0.000000,0.000000,0.000000}%
\pgfsetfillcolor{currentfill}%
\pgfsetlinewidth{0.803000pt}%
\definecolor{currentstroke}{rgb}{0.000000,0.000000,0.000000}%
\pgfsetstrokecolor{currentstroke}%
\pgfsetdash{}{0pt}%
\pgfsys@defobject{currentmarker}{\pgfqpoint{0.000000in}{-0.048611in}}{\pgfqpoint{0.000000in}{0.000000in}}{%
\pgfpathmoveto{\pgfqpoint{0.000000in}{0.000000in}}%
\pgfpathlineto{\pgfqpoint{0.000000in}{-0.048611in}}%
\pgfusepath{stroke,fill}%
}%
\begin{pgfscope}%
\pgfsys@transformshift{2.855360in}{0.521603in}%
\pgfsys@useobject{currentmarker}{}%
\end{pgfscope}%
\end{pgfscope}%
\begin{pgfscope}%
\definecolor{textcolor}{rgb}{0.000000,0.000000,0.000000}%
\pgfsetstrokecolor{textcolor}%
\pgfsetfillcolor{textcolor}%
\pgftext[x=2.855360in,y=0.424381in,,top]{\color{textcolor}{\rmfamily\fontsize{10.000000}{12.000000}\selectfont\catcode`\^=\active\def^{\ifmmode\sp\else\^{}\fi}\catcode`\%=\active\def%{\%}$\mathdefault{50}$}}%
\end{pgfscope}%
\begin{pgfscope}%
\pgfsetbuttcap%
\pgfsetroundjoin%
\definecolor{currentfill}{rgb}{0.000000,0.000000,0.000000}%
\pgfsetfillcolor{currentfill}%
\pgfsetlinewidth{0.803000pt}%
\definecolor{currentstroke}{rgb}{0.000000,0.000000,0.000000}%
\pgfsetstrokecolor{currentstroke}%
\pgfsetdash{}{0pt}%
\pgfsys@defobject{currentmarker}{\pgfqpoint{0.000000in}{-0.048611in}}{\pgfqpoint{0.000000in}{0.000000in}}{%
\pgfpathmoveto{\pgfqpoint{0.000000in}{0.000000in}}%
\pgfpathlineto{\pgfqpoint{0.000000in}{-0.048611in}}%
\pgfusepath{stroke,fill}%
}%
\begin{pgfscope}%
\pgfsys@transformshift{3.402938in}{0.521603in}%
\pgfsys@useobject{currentmarker}{}%
\end{pgfscope}%
\end{pgfscope}%
\begin{pgfscope}%
\definecolor{textcolor}{rgb}{0.000000,0.000000,0.000000}%
\pgfsetstrokecolor{textcolor}%
\pgfsetfillcolor{textcolor}%
\pgftext[x=3.402938in,y=0.424381in,,top]{\color{textcolor}{\rmfamily\fontsize{10.000000}{12.000000}\selectfont\catcode`\^=\active\def^{\ifmmode\sp\else\^{}\fi}\catcode`\%=\active\def%{\%}$\mathdefault{60}$}}%
\end{pgfscope}%
\begin{pgfscope}%
\pgfsetbuttcap%
\pgfsetroundjoin%
\definecolor{currentfill}{rgb}{0.000000,0.000000,0.000000}%
\pgfsetfillcolor{currentfill}%
\pgfsetlinewidth{0.803000pt}%
\definecolor{currentstroke}{rgb}{0.000000,0.000000,0.000000}%
\pgfsetstrokecolor{currentstroke}%
\pgfsetdash{}{0pt}%
\pgfsys@defobject{currentmarker}{\pgfqpoint{0.000000in}{-0.048611in}}{\pgfqpoint{0.000000in}{0.000000in}}{%
\pgfpathmoveto{\pgfqpoint{0.000000in}{0.000000in}}%
\pgfpathlineto{\pgfqpoint{0.000000in}{-0.048611in}}%
\pgfusepath{stroke,fill}%
}%
\begin{pgfscope}%
\pgfsys@transformshift{3.950516in}{0.521603in}%
\pgfsys@useobject{currentmarker}{}%
\end{pgfscope}%
\end{pgfscope}%
\begin{pgfscope}%
\definecolor{textcolor}{rgb}{0.000000,0.000000,0.000000}%
\pgfsetstrokecolor{textcolor}%
\pgfsetfillcolor{textcolor}%
\pgftext[x=3.950516in,y=0.424381in,,top]{\color{textcolor}{\rmfamily\fontsize{10.000000}{12.000000}\selectfont\catcode`\^=\active\def^{\ifmmode\sp\else\^{}\fi}\catcode`\%=\active\def%{\%}$\mathdefault{70}$}}%
\end{pgfscope}%
\begin{pgfscope}%
\pgfsetbuttcap%
\pgfsetroundjoin%
\definecolor{currentfill}{rgb}{0.000000,0.000000,0.000000}%
\pgfsetfillcolor{currentfill}%
\pgfsetlinewidth{0.803000pt}%
\definecolor{currentstroke}{rgb}{0.000000,0.000000,0.000000}%
\pgfsetstrokecolor{currentstroke}%
\pgfsetdash{}{0pt}%
\pgfsys@defobject{currentmarker}{\pgfqpoint{0.000000in}{-0.048611in}}{\pgfqpoint{0.000000in}{0.000000in}}{%
\pgfpathmoveto{\pgfqpoint{0.000000in}{0.000000in}}%
\pgfpathlineto{\pgfqpoint{0.000000in}{-0.048611in}}%
\pgfusepath{stroke,fill}%
}%
\begin{pgfscope}%
\pgfsys@transformshift{4.498094in}{0.521603in}%
\pgfsys@useobject{currentmarker}{}%
\end{pgfscope}%
\end{pgfscope}%
\begin{pgfscope}%
\definecolor{textcolor}{rgb}{0.000000,0.000000,0.000000}%
\pgfsetstrokecolor{textcolor}%
\pgfsetfillcolor{textcolor}%
\pgftext[x=4.498094in,y=0.424381in,,top]{\color{textcolor}{\rmfamily\fontsize{10.000000}{12.000000}\selectfont\catcode`\^=\active\def^{\ifmmode\sp\else\^{}\fi}\catcode`\%=\active\def%{\%}$\mathdefault{80}$}}%
\end{pgfscope}%
\begin{pgfscope}%
\pgfsetbuttcap%
\pgfsetroundjoin%
\definecolor{currentfill}{rgb}{0.000000,0.000000,0.000000}%
\pgfsetfillcolor{currentfill}%
\pgfsetlinewidth{0.803000pt}%
\definecolor{currentstroke}{rgb}{0.000000,0.000000,0.000000}%
\pgfsetstrokecolor{currentstroke}%
\pgfsetdash{}{0pt}%
\pgfsys@defobject{currentmarker}{\pgfqpoint{0.000000in}{-0.048611in}}{\pgfqpoint{0.000000in}{0.000000in}}{%
\pgfpathmoveto{\pgfqpoint{0.000000in}{0.000000in}}%
\pgfpathlineto{\pgfqpoint{0.000000in}{-0.048611in}}%
\pgfusepath{stroke,fill}%
}%
\begin{pgfscope}%
\pgfsys@transformshift{5.045672in}{0.521603in}%
\pgfsys@useobject{currentmarker}{}%
\end{pgfscope}%
\end{pgfscope}%
\begin{pgfscope}%
\definecolor{textcolor}{rgb}{0.000000,0.000000,0.000000}%
\pgfsetstrokecolor{textcolor}%
\pgfsetfillcolor{textcolor}%
\pgftext[x=5.045672in,y=0.424381in,,top]{\color{textcolor}{\rmfamily\fontsize{10.000000}{12.000000}\selectfont\catcode`\^=\active\def^{\ifmmode\sp\else\^{}\fi}\catcode`\%=\active\def%{\%}$\mathdefault{90}$}}%
\end{pgfscope}%
\begin{pgfscope}%
\pgfsetbuttcap%
\pgfsetroundjoin%
\definecolor{currentfill}{rgb}{0.000000,0.000000,0.000000}%
\pgfsetfillcolor{currentfill}%
\pgfsetlinewidth{0.803000pt}%
\definecolor{currentstroke}{rgb}{0.000000,0.000000,0.000000}%
\pgfsetstrokecolor{currentstroke}%
\pgfsetdash{}{0pt}%
\pgfsys@defobject{currentmarker}{\pgfqpoint{0.000000in}{-0.048611in}}{\pgfqpoint{0.000000in}{0.000000in}}{%
\pgfpathmoveto{\pgfqpoint{0.000000in}{0.000000in}}%
\pgfpathlineto{\pgfqpoint{0.000000in}{-0.048611in}}%
\pgfusepath{stroke,fill}%
}%
\begin{pgfscope}%
\pgfsys@transformshift{5.593250in}{0.521603in}%
\pgfsys@useobject{currentmarker}{}%
\end{pgfscope}%
\end{pgfscope}%
\begin{pgfscope}%
\definecolor{textcolor}{rgb}{0.000000,0.000000,0.000000}%
\pgfsetstrokecolor{textcolor}%
\pgfsetfillcolor{textcolor}%
\pgftext[x=5.593250in,y=0.424381in,,top]{\color{textcolor}{\rmfamily\fontsize{10.000000}{12.000000}\selectfont\catcode`\^=\active\def^{\ifmmode\sp\else\^{}\fi}\catcode`\%=\active\def%{\%}$\mathdefault{100}$}}%
\end{pgfscope}%
\begin{pgfscope}%
\definecolor{textcolor}{rgb}{0.000000,0.000000,0.000000}%
\pgfsetstrokecolor{textcolor}%
\pgfsetfillcolor{textcolor}%
\pgftext[x=3.238665in,y=0.234413in,,top]{\color{textcolor}{\rmfamily\fontsize{10.000000}{12.000000}\selectfont\catcode`\^=\active\def^{\ifmmode\sp\else\^{}\fi}\catcode`\%=\active\def%{\%}Triangle components}}%
\end{pgfscope}%
\begin{pgfscope}%
\pgfsetbuttcap%
\pgfsetroundjoin%
\definecolor{currentfill}{rgb}{0.000000,0.000000,0.000000}%
\pgfsetfillcolor{currentfill}%
\pgfsetlinewidth{0.803000pt}%
\definecolor{currentstroke}{rgb}{0.000000,0.000000,0.000000}%
\pgfsetstrokecolor{currentstroke}%
\pgfsetdash{}{0pt}%
\pgfsys@defobject{currentmarker}{\pgfqpoint{-0.048611in}{0.000000in}}{\pgfqpoint{-0.000000in}{0.000000in}}{%
\pgfpathmoveto{\pgfqpoint{-0.000000in}{0.000000in}}%
\pgfpathlineto{\pgfqpoint{-0.048611in}{0.000000in}}%
\pgfusepath{stroke,fill}%
}%
\begin{pgfscope}%
\pgfsys@transformshift{0.588387in}{0.585015in}%
\pgfsys@useobject{currentmarker}{}%
\end{pgfscope}%
\end{pgfscope}%
\begin{pgfscope}%
\definecolor{textcolor}{rgb}{0.000000,0.000000,0.000000}%
\pgfsetstrokecolor{textcolor}%
\pgfsetfillcolor{textcolor}%
\pgftext[x=0.289968in, y=0.532254in, left, base]{\color{textcolor}{\rmfamily\fontsize{10.000000}{12.000000}\selectfont\catcode`\^=\active\def^{\ifmmode\sp\else\^{}\fi}\catcode`\%=\active\def%{\%}$\mathdefault{10^{2}}$}}%
\end{pgfscope}%
\begin{pgfscope}%
\pgfsetbuttcap%
\pgfsetroundjoin%
\definecolor{currentfill}{rgb}{0.000000,0.000000,0.000000}%
\pgfsetfillcolor{currentfill}%
\pgfsetlinewidth{0.803000pt}%
\definecolor{currentstroke}{rgb}{0.000000,0.000000,0.000000}%
\pgfsetstrokecolor{currentstroke}%
\pgfsetdash{}{0pt}%
\pgfsys@defobject{currentmarker}{\pgfqpoint{-0.048611in}{0.000000in}}{\pgfqpoint{-0.000000in}{0.000000in}}{%
\pgfpathmoveto{\pgfqpoint{-0.000000in}{0.000000in}}%
\pgfpathlineto{\pgfqpoint{-0.048611in}{0.000000in}}%
\pgfusepath{stroke,fill}%
}%
\begin{pgfscope}%
\pgfsys@transformshift{0.588387in}{1.779228in}%
\pgfsys@useobject{currentmarker}{}%
\end{pgfscope}%
\end{pgfscope}%
\begin{pgfscope}%
\definecolor{textcolor}{rgb}{0.000000,0.000000,0.000000}%
\pgfsetstrokecolor{textcolor}%
\pgfsetfillcolor{textcolor}%
\pgftext[x=0.289968in, y=1.726467in, left, base]{\color{textcolor}{\rmfamily\fontsize{10.000000}{12.000000}\selectfont\catcode`\^=\active\def^{\ifmmode\sp\else\^{}\fi}\catcode`\%=\active\def%{\%}$\mathdefault{10^{3}}$}}%
\end{pgfscope}%
\begin{pgfscope}%
\pgfsetbuttcap%
\pgfsetroundjoin%
\definecolor{currentfill}{rgb}{0.000000,0.000000,0.000000}%
\pgfsetfillcolor{currentfill}%
\pgfsetlinewidth{0.602250pt}%
\definecolor{currentstroke}{rgb}{0.000000,0.000000,0.000000}%
\pgfsetstrokecolor{currentstroke}%
\pgfsetdash{}{0pt}%
\pgfsys@defobject{currentmarker}{\pgfqpoint{-0.027778in}{0.000000in}}{\pgfqpoint{-0.000000in}{0.000000in}}{%
\pgfpathmoveto{\pgfqpoint{-0.000000in}{0.000000in}}%
\pgfpathlineto{\pgfqpoint{-0.027778in}{0.000000in}}%
\pgfusepath{stroke,fill}%
}%
\begin{pgfscope}%
\pgfsys@transformshift{0.588387in}{0.530371in}%
\pgfsys@useobject{currentmarker}{}%
\end{pgfscope}%
\end{pgfscope}%
\begin{pgfscope}%
\pgfsetbuttcap%
\pgfsetroundjoin%
\definecolor{currentfill}{rgb}{0.000000,0.000000,0.000000}%
\pgfsetfillcolor{currentfill}%
\pgfsetlinewidth{0.602250pt}%
\definecolor{currentstroke}{rgb}{0.000000,0.000000,0.000000}%
\pgfsetstrokecolor{currentstroke}%
\pgfsetdash{}{0pt}%
\pgfsys@defobject{currentmarker}{\pgfqpoint{-0.027778in}{0.000000in}}{\pgfqpoint{-0.000000in}{0.000000in}}{%
\pgfpathmoveto{\pgfqpoint{-0.000000in}{0.000000in}}%
\pgfpathlineto{\pgfqpoint{-0.027778in}{0.000000in}}%
\pgfusepath{stroke,fill}%
}%
\begin{pgfscope}%
\pgfsys@transformshift{0.588387in}{0.944509in}%
\pgfsys@useobject{currentmarker}{}%
\end{pgfscope}%
\end{pgfscope}%
\begin{pgfscope}%
\pgfsetbuttcap%
\pgfsetroundjoin%
\definecolor{currentfill}{rgb}{0.000000,0.000000,0.000000}%
\pgfsetfillcolor{currentfill}%
\pgfsetlinewidth{0.602250pt}%
\definecolor{currentstroke}{rgb}{0.000000,0.000000,0.000000}%
\pgfsetstrokecolor{currentstroke}%
\pgfsetdash{}{0pt}%
\pgfsys@defobject{currentmarker}{\pgfqpoint{-0.027778in}{0.000000in}}{\pgfqpoint{-0.000000in}{0.000000in}}{%
\pgfpathmoveto{\pgfqpoint{-0.000000in}{0.000000in}}%
\pgfpathlineto{\pgfqpoint{-0.027778in}{0.000000in}}%
\pgfusepath{stroke,fill}%
}%
\begin{pgfscope}%
\pgfsys@transformshift{0.588387in}{1.154800in}%
\pgfsys@useobject{currentmarker}{}%
\end{pgfscope}%
\end{pgfscope}%
\begin{pgfscope}%
\pgfsetbuttcap%
\pgfsetroundjoin%
\definecolor{currentfill}{rgb}{0.000000,0.000000,0.000000}%
\pgfsetfillcolor{currentfill}%
\pgfsetlinewidth{0.602250pt}%
\definecolor{currentstroke}{rgb}{0.000000,0.000000,0.000000}%
\pgfsetstrokecolor{currentstroke}%
\pgfsetdash{}{0pt}%
\pgfsys@defobject{currentmarker}{\pgfqpoint{-0.027778in}{0.000000in}}{\pgfqpoint{-0.000000in}{0.000000in}}{%
\pgfpathmoveto{\pgfqpoint{-0.000000in}{0.000000in}}%
\pgfpathlineto{\pgfqpoint{-0.027778in}{0.000000in}}%
\pgfusepath{stroke,fill}%
}%
\begin{pgfscope}%
\pgfsys@transformshift{0.588387in}{1.304003in}%
\pgfsys@useobject{currentmarker}{}%
\end{pgfscope}%
\end{pgfscope}%
\begin{pgfscope}%
\pgfsetbuttcap%
\pgfsetroundjoin%
\definecolor{currentfill}{rgb}{0.000000,0.000000,0.000000}%
\pgfsetfillcolor{currentfill}%
\pgfsetlinewidth{0.602250pt}%
\definecolor{currentstroke}{rgb}{0.000000,0.000000,0.000000}%
\pgfsetstrokecolor{currentstroke}%
\pgfsetdash{}{0pt}%
\pgfsys@defobject{currentmarker}{\pgfqpoint{-0.027778in}{0.000000in}}{\pgfqpoint{-0.000000in}{0.000000in}}{%
\pgfpathmoveto{\pgfqpoint{-0.000000in}{0.000000in}}%
\pgfpathlineto{\pgfqpoint{-0.027778in}{0.000000in}}%
\pgfusepath{stroke,fill}%
}%
\begin{pgfscope}%
\pgfsys@transformshift{0.588387in}{1.419734in}%
\pgfsys@useobject{currentmarker}{}%
\end{pgfscope}%
\end{pgfscope}%
\begin{pgfscope}%
\pgfsetbuttcap%
\pgfsetroundjoin%
\definecolor{currentfill}{rgb}{0.000000,0.000000,0.000000}%
\pgfsetfillcolor{currentfill}%
\pgfsetlinewidth{0.602250pt}%
\definecolor{currentstroke}{rgb}{0.000000,0.000000,0.000000}%
\pgfsetstrokecolor{currentstroke}%
\pgfsetdash{}{0pt}%
\pgfsys@defobject{currentmarker}{\pgfqpoint{-0.027778in}{0.000000in}}{\pgfqpoint{-0.000000in}{0.000000in}}{%
\pgfpathmoveto{\pgfqpoint{-0.000000in}{0.000000in}}%
\pgfpathlineto{\pgfqpoint{-0.027778in}{0.000000in}}%
\pgfusepath{stroke,fill}%
}%
\begin{pgfscope}%
\pgfsys@transformshift{0.588387in}{1.514294in}%
\pgfsys@useobject{currentmarker}{}%
\end{pgfscope}%
\end{pgfscope}%
\begin{pgfscope}%
\pgfsetbuttcap%
\pgfsetroundjoin%
\definecolor{currentfill}{rgb}{0.000000,0.000000,0.000000}%
\pgfsetfillcolor{currentfill}%
\pgfsetlinewidth{0.602250pt}%
\definecolor{currentstroke}{rgb}{0.000000,0.000000,0.000000}%
\pgfsetstrokecolor{currentstroke}%
\pgfsetdash{}{0pt}%
\pgfsys@defobject{currentmarker}{\pgfqpoint{-0.027778in}{0.000000in}}{\pgfqpoint{-0.000000in}{0.000000in}}{%
\pgfpathmoveto{\pgfqpoint{-0.000000in}{0.000000in}}%
\pgfpathlineto{\pgfqpoint{-0.027778in}{0.000000in}}%
\pgfusepath{stroke,fill}%
}%
\begin{pgfscope}%
\pgfsys@transformshift{0.588387in}{1.594242in}%
\pgfsys@useobject{currentmarker}{}%
\end{pgfscope}%
\end{pgfscope}%
\begin{pgfscope}%
\pgfsetbuttcap%
\pgfsetroundjoin%
\definecolor{currentfill}{rgb}{0.000000,0.000000,0.000000}%
\pgfsetfillcolor{currentfill}%
\pgfsetlinewidth{0.602250pt}%
\definecolor{currentstroke}{rgb}{0.000000,0.000000,0.000000}%
\pgfsetstrokecolor{currentstroke}%
\pgfsetdash{}{0pt}%
\pgfsys@defobject{currentmarker}{\pgfqpoint{-0.027778in}{0.000000in}}{\pgfqpoint{-0.000000in}{0.000000in}}{%
\pgfpathmoveto{\pgfqpoint{-0.000000in}{0.000000in}}%
\pgfpathlineto{\pgfqpoint{-0.027778in}{0.000000in}}%
\pgfusepath{stroke,fill}%
}%
\begin{pgfscope}%
\pgfsys@transformshift{0.588387in}{1.663497in}%
\pgfsys@useobject{currentmarker}{}%
\end{pgfscope}%
\end{pgfscope}%
\begin{pgfscope}%
\pgfsetbuttcap%
\pgfsetroundjoin%
\definecolor{currentfill}{rgb}{0.000000,0.000000,0.000000}%
\pgfsetfillcolor{currentfill}%
\pgfsetlinewidth{0.602250pt}%
\definecolor{currentstroke}{rgb}{0.000000,0.000000,0.000000}%
\pgfsetstrokecolor{currentstroke}%
\pgfsetdash{}{0pt}%
\pgfsys@defobject{currentmarker}{\pgfqpoint{-0.027778in}{0.000000in}}{\pgfqpoint{-0.000000in}{0.000000in}}{%
\pgfpathmoveto{\pgfqpoint{-0.000000in}{0.000000in}}%
\pgfpathlineto{\pgfqpoint{-0.027778in}{0.000000in}}%
\pgfusepath{stroke,fill}%
}%
\begin{pgfscope}%
\pgfsys@transformshift{0.588387in}{1.724584in}%
\pgfsys@useobject{currentmarker}{}%
\end{pgfscope}%
\end{pgfscope}%
\begin{pgfscope}%
\pgfsetbuttcap%
\pgfsetroundjoin%
\definecolor{currentfill}{rgb}{0.000000,0.000000,0.000000}%
\pgfsetfillcolor{currentfill}%
\pgfsetlinewidth{0.602250pt}%
\definecolor{currentstroke}{rgb}{0.000000,0.000000,0.000000}%
\pgfsetstrokecolor{currentstroke}%
\pgfsetdash{}{0pt}%
\pgfsys@defobject{currentmarker}{\pgfqpoint{-0.027778in}{0.000000in}}{\pgfqpoint{-0.000000in}{0.000000in}}{%
\pgfpathmoveto{\pgfqpoint{-0.000000in}{0.000000in}}%
\pgfpathlineto{\pgfqpoint{-0.027778in}{0.000000in}}%
\pgfusepath{stroke,fill}%
}%
\begin{pgfscope}%
\pgfsys@transformshift{0.588387in}{2.138722in}%
\pgfsys@useobject{currentmarker}{}%
\end{pgfscope}%
\end{pgfscope}%
\begin{pgfscope}%
\pgfsetbuttcap%
\pgfsetroundjoin%
\definecolor{currentfill}{rgb}{0.000000,0.000000,0.000000}%
\pgfsetfillcolor{currentfill}%
\pgfsetlinewidth{0.602250pt}%
\definecolor{currentstroke}{rgb}{0.000000,0.000000,0.000000}%
\pgfsetstrokecolor{currentstroke}%
\pgfsetdash{}{0pt}%
\pgfsys@defobject{currentmarker}{\pgfqpoint{-0.027778in}{0.000000in}}{\pgfqpoint{-0.000000in}{0.000000in}}{%
\pgfpathmoveto{\pgfqpoint{-0.000000in}{0.000000in}}%
\pgfpathlineto{\pgfqpoint{-0.027778in}{0.000000in}}%
\pgfusepath{stroke,fill}%
}%
\begin{pgfscope}%
\pgfsys@transformshift{0.588387in}{2.349013in}%
\pgfsys@useobject{currentmarker}{}%
\end{pgfscope}%
\end{pgfscope}%
\begin{pgfscope}%
\pgfsetbuttcap%
\pgfsetroundjoin%
\definecolor{currentfill}{rgb}{0.000000,0.000000,0.000000}%
\pgfsetfillcolor{currentfill}%
\pgfsetlinewidth{0.602250pt}%
\definecolor{currentstroke}{rgb}{0.000000,0.000000,0.000000}%
\pgfsetstrokecolor{currentstroke}%
\pgfsetdash{}{0pt}%
\pgfsys@defobject{currentmarker}{\pgfqpoint{-0.027778in}{0.000000in}}{\pgfqpoint{-0.000000in}{0.000000in}}{%
\pgfpathmoveto{\pgfqpoint{-0.000000in}{0.000000in}}%
\pgfpathlineto{\pgfqpoint{-0.027778in}{0.000000in}}%
\pgfusepath{stroke,fill}%
}%
\begin{pgfscope}%
\pgfsys@transformshift{0.588387in}{2.498216in}%
\pgfsys@useobject{currentmarker}{}%
\end{pgfscope}%
\end{pgfscope}%
\begin{pgfscope}%
\pgfsetbuttcap%
\pgfsetroundjoin%
\definecolor{currentfill}{rgb}{0.000000,0.000000,0.000000}%
\pgfsetfillcolor{currentfill}%
\pgfsetlinewidth{0.602250pt}%
\definecolor{currentstroke}{rgb}{0.000000,0.000000,0.000000}%
\pgfsetstrokecolor{currentstroke}%
\pgfsetdash{}{0pt}%
\pgfsys@defobject{currentmarker}{\pgfqpoint{-0.027778in}{0.000000in}}{\pgfqpoint{-0.000000in}{0.000000in}}{%
\pgfpathmoveto{\pgfqpoint{-0.000000in}{0.000000in}}%
\pgfpathlineto{\pgfqpoint{-0.027778in}{0.000000in}}%
\pgfusepath{stroke,fill}%
}%
\begin{pgfscope}%
\pgfsys@transformshift{0.588387in}{2.613948in}%
\pgfsys@useobject{currentmarker}{}%
\end{pgfscope}%
\end{pgfscope}%
\begin{pgfscope}%
\pgfsetbuttcap%
\pgfsetroundjoin%
\definecolor{currentfill}{rgb}{0.000000,0.000000,0.000000}%
\pgfsetfillcolor{currentfill}%
\pgfsetlinewidth{0.602250pt}%
\definecolor{currentstroke}{rgb}{0.000000,0.000000,0.000000}%
\pgfsetstrokecolor{currentstroke}%
\pgfsetdash{}{0pt}%
\pgfsys@defobject{currentmarker}{\pgfqpoint{-0.027778in}{0.000000in}}{\pgfqpoint{-0.000000in}{0.000000in}}{%
\pgfpathmoveto{\pgfqpoint{-0.000000in}{0.000000in}}%
\pgfpathlineto{\pgfqpoint{-0.027778in}{0.000000in}}%
\pgfusepath{stroke,fill}%
}%
\begin{pgfscope}%
\pgfsys@transformshift{0.588387in}{2.708507in}%
\pgfsys@useobject{currentmarker}{}%
\end{pgfscope}%
\end{pgfscope}%
\begin{pgfscope}%
\definecolor{textcolor}{rgb}{0.000000,0.000000,0.000000}%
\pgfsetstrokecolor{textcolor}%
\pgfsetfillcolor{textcolor}%
\pgftext[x=0.234413in,y=1.631726in,,bottom,rotate=90.000000]{\color{textcolor}{\rmfamily\fontsize{10.000000}{12.000000}\selectfont\catcode`\^=\active\def^{\ifmmode\sp\else\^{}\fi}\catcode`\%=\active\def%{\%}Time [ms]}}%
\end{pgfscope}%
\begin{pgfscope}%
\pgfpathrectangle{\pgfqpoint{0.588387in}{0.521603in}}{\pgfqpoint{5.300555in}{2.220246in}}%
\pgfusepath{clip}%
\pgfsetrectcap%
\pgfsetroundjoin%
\pgfsetlinewidth{1.505625pt}%
\pgfsetstrokecolor{currentstroke1}%
\pgfsetdash{}{0pt}%
\pgfpathmoveto{\pgfqpoint{0.829322in}{0.643970in}}%
\pgfpathlineto{\pgfqpoint{0.884079in}{0.709519in}}%
\pgfpathlineto{\pgfqpoint{0.938837in}{0.800567in}}%
\pgfpathlineto{\pgfqpoint{0.993595in}{0.806106in}}%
\pgfpathlineto{\pgfqpoint{1.048353in}{0.886143in}}%
\pgfpathlineto{\pgfqpoint{1.103111in}{0.931237in}}%
\pgfpathlineto{\pgfqpoint{1.157868in}{0.913201in}}%
\pgfpathlineto{\pgfqpoint{1.212626in}{0.937149in}}%
\pgfpathlineto{\pgfqpoint{1.267384in}{0.977475in}}%
\pgfpathlineto{\pgfqpoint{1.322142in}{0.980098in}}%
\pgfpathlineto{\pgfqpoint{1.376899in}{1.002687in}}%
\pgfpathlineto{\pgfqpoint{1.431657in}{1.104162in}}%
\pgfpathlineto{\pgfqpoint{1.486415in}{1.056810in}}%
\pgfpathlineto{\pgfqpoint{1.541173in}{1.143786in}}%
\pgfpathlineto{\pgfqpoint{1.595931in}{1.183489in}}%
\pgfpathlineto{\pgfqpoint{1.650688in}{1.212804in}}%
\pgfpathlineto{\pgfqpoint{1.705446in}{1.243424in}}%
\pgfpathlineto{\pgfqpoint{1.760204in}{1.340329in}}%
\pgfpathlineto{\pgfqpoint{1.814962in}{1.375831in}}%
\pgfpathlineto{\pgfqpoint{1.869720in}{1.374360in}}%
\pgfpathlineto{\pgfqpoint{1.924477in}{1.538980in}}%
\pgfpathlineto{\pgfqpoint{1.979235in}{1.529519in}}%
\pgfpathlineto{\pgfqpoint{2.033993in}{1.562821in}}%
\pgfpathlineto{\pgfqpoint{2.088751in}{1.536479in}}%
\pgfpathlineto{\pgfqpoint{2.143509in}{1.666371in}}%
\pgfpathlineto{\pgfqpoint{2.198266in}{1.656434in}}%
\pgfpathlineto{\pgfqpoint{2.253024in}{1.773623in}}%
\pgfpathlineto{\pgfqpoint{2.307782in}{1.796611in}}%
\pgfpathlineto{\pgfqpoint{2.362540in}{1.881297in}}%
\pgfpathlineto{\pgfqpoint{2.417298in}{1.815368in}}%
\pgfpathlineto{\pgfqpoint{2.472055in}{1.914189in}}%
\pgfpathlineto{\pgfqpoint{2.526813in}{2.004388in}}%
\pgfpathlineto{\pgfqpoint{2.581571in}{1.870427in}}%
\pgfpathlineto{\pgfqpoint{2.636329in}{2.087259in}}%
\pgfpathlineto{\pgfqpoint{2.691087in}{2.025094in}}%
\pgfpathlineto{\pgfqpoint{2.745844in}{2.222867in}}%
\pgfpathlineto{\pgfqpoint{2.800602in}{2.347830in}}%
\pgfpathlineto{\pgfqpoint{2.855360in}{2.264724in}}%
\pgfpathlineto{\pgfqpoint{2.910118in}{2.351772in}}%
\pgfpathlineto{\pgfqpoint{2.964876in}{2.329549in}}%
\pgfpathlineto{\pgfqpoint{3.019633in}{2.099549in}}%
\pgfpathlineto{\pgfqpoint{3.074391in}{2.420822in}}%
\pgfpathlineto{\pgfqpoint{3.129149in}{2.352717in}}%
\pgfpathlineto{\pgfqpoint{3.183907in}{2.421975in}}%
\pgfpathlineto{\pgfqpoint{3.238665in}{2.309606in}}%
\pgfpathlineto{\pgfqpoint{3.293422in}{2.402697in}}%
\pgfpathlineto{\pgfqpoint{3.402938in}{2.559476in}}%
\pgfpathlineto{\pgfqpoint{3.621969in}{2.640929in}}%
\pgfusepath{stroke}%
\end{pgfscope}%
\begin{pgfscope}%
\pgfpathrectangle{\pgfqpoint{0.588387in}{0.521603in}}{\pgfqpoint{5.300555in}{2.220246in}}%
\pgfusepath{clip}%
\pgfsetrectcap%
\pgfsetroundjoin%
\pgfsetlinewidth{1.505625pt}%
\pgfsetstrokecolor{currentstroke2}%
\pgfsetdash{}{0pt}%
\pgfpathmoveto{\pgfqpoint{0.829322in}{0.654370in}}%
\pgfpathlineto{\pgfqpoint{0.884079in}{0.719560in}}%
\pgfpathlineto{\pgfqpoint{0.938837in}{0.718026in}}%
\pgfpathlineto{\pgfqpoint{0.993595in}{0.805773in}}%
\pgfpathlineto{\pgfqpoint{1.048353in}{0.891346in}}%
\pgfpathlineto{\pgfqpoint{1.103111in}{0.933898in}}%
\pgfpathlineto{\pgfqpoint{1.157868in}{0.906626in}}%
\pgfpathlineto{\pgfqpoint{1.212626in}{0.910678in}}%
\pgfpathlineto{\pgfqpoint{1.267384in}{0.950581in}}%
\pgfpathlineto{\pgfqpoint{1.322142in}{0.889884in}}%
\pgfpathlineto{\pgfqpoint{1.376899in}{0.916242in}}%
\pgfpathlineto{\pgfqpoint{1.431657in}{0.903755in}}%
\pgfpathlineto{\pgfqpoint{1.486415in}{0.868963in}}%
\pgfpathlineto{\pgfqpoint{1.541173in}{0.887839in}}%
\pgfpathlineto{\pgfqpoint{1.595931in}{0.954518in}}%
\pgfpathlineto{\pgfqpoint{1.650688in}{0.960660in}}%
\pgfpathlineto{\pgfqpoint{1.705446in}{0.978958in}}%
\pgfpathlineto{\pgfqpoint{1.760204in}{0.954563in}}%
\pgfpathlineto{\pgfqpoint{1.814962in}{1.035370in}}%
\pgfpathlineto{\pgfqpoint{1.869720in}{0.987476in}}%
\pgfpathlineto{\pgfqpoint{1.924477in}{1.048195in}}%
\pgfpathlineto{\pgfqpoint{1.979235in}{1.060962in}}%
\pgfpathlineto{\pgfqpoint{2.033993in}{1.029094in}}%
\pgfpathlineto{\pgfqpoint{2.088751in}{1.090883in}}%
\pgfpathlineto{\pgfqpoint{2.143509in}{1.059438in}}%
\pgfpathlineto{\pgfqpoint{2.198266in}{1.114714in}}%
\pgfpathlineto{\pgfqpoint{2.253024in}{1.113137in}}%
\pgfpathlineto{\pgfqpoint{2.307782in}{1.145643in}}%
\pgfpathlineto{\pgfqpoint{2.362540in}{1.116064in}}%
\pgfpathlineto{\pgfqpoint{2.417298in}{1.186584in}}%
\pgfpathlineto{\pgfqpoint{2.472055in}{1.207042in}}%
\pgfpathlineto{\pgfqpoint{2.526813in}{1.256115in}}%
\pgfpathlineto{\pgfqpoint{2.581571in}{1.237576in}}%
\pgfpathlineto{\pgfqpoint{2.636329in}{1.216236in}}%
\pgfpathlineto{\pgfqpoint{2.691087in}{1.257265in}}%
\pgfpathlineto{\pgfqpoint{2.745844in}{1.281237in}}%
\pgfpathlineto{\pgfqpoint{2.800602in}{1.274378in}}%
\pgfpathlineto{\pgfqpoint{2.855360in}{1.309826in}}%
\pgfpathlineto{\pgfqpoint{2.910118in}{1.315702in}}%
\pgfpathlineto{\pgfqpoint{2.964876in}{1.399858in}}%
\pgfpathlineto{\pgfqpoint{3.019633in}{1.370441in}}%
\pgfpathlineto{\pgfqpoint{3.074391in}{1.405498in}}%
\pgfpathlineto{\pgfqpoint{3.129149in}{1.394767in}}%
\pgfpathlineto{\pgfqpoint{3.183907in}{1.399719in}}%
\pgfpathlineto{\pgfqpoint{3.238665in}{1.500986in}}%
\pgfpathlineto{\pgfqpoint{3.293422in}{1.509767in}}%
\pgfpathlineto{\pgfqpoint{3.348180in}{1.495144in}}%
\pgfpathlineto{\pgfqpoint{3.402938in}{1.487805in}}%
\pgfpathlineto{\pgfqpoint{3.457696in}{1.547316in}}%
\pgfpathlineto{\pgfqpoint{3.512454in}{1.690651in}}%
\pgfpathlineto{\pgfqpoint{3.567211in}{1.569683in}}%
\pgfpathlineto{\pgfqpoint{3.621969in}{1.685395in}}%
\pgfpathlineto{\pgfqpoint{3.676727in}{1.698588in}}%
\pgfpathlineto{\pgfqpoint{3.731485in}{1.667887in}}%
\pgfpathlineto{\pgfqpoint{3.786243in}{1.751533in}}%
\pgfpathlineto{\pgfqpoint{3.841000in}{1.666793in}}%
\pgfpathlineto{\pgfqpoint{3.895758in}{1.560177in}}%
\pgfpathlineto{\pgfqpoint{3.950516in}{1.690138in}}%
\pgfpathlineto{\pgfqpoint{4.005274in}{1.572298in}}%
\pgfpathlineto{\pgfqpoint{4.060032in}{1.814077in}}%
\pgfpathlineto{\pgfqpoint{4.114789in}{1.725448in}}%
\pgfpathlineto{\pgfqpoint{4.169547in}{1.825586in}}%
\pgfpathlineto{\pgfqpoint{4.224305in}{1.735137in}}%
\pgfpathlineto{\pgfqpoint{4.279063in}{1.852165in}}%
\pgfpathlineto{\pgfqpoint{4.388578in}{1.800317in}}%
\pgfpathlineto{\pgfqpoint{4.443336in}{1.689727in}}%
\pgfpathlineto{\pgfqpoint{4.498094in}{1.867045in}}%
\pgfpathlineto{\pgfqpoint{4.607609in}{2.049991in}}%
\pgfpathlineto{\pgfqpoint{4.990914in}{2.068281in}}%
\pgfpathlineto{\pgfqpoint{5.100430in}{2.170165in}}%
\pgfpathlineto{\pgfqpoint{5.155187in}{2.420295in}}%
\pgfpathlineto{\pgfqpoint{5.209945in}{2.214580in}}%
\pgfpathlineto{\pgfqpoint{5.483734in}{2.295137in}}%
\pgfpathlineto{\pgfqpoint{5.538492in}{2.291286in}}%
\pgfpathlineto{\pgfqpoint{5.648008in}{2.164521in}}%
\pgfusepath{stroke}%
\end{pgfscope}%
\begin{pgfscope}%
\pgfpathrectangle{\pgfqpoint{0.588387in}{0.521603in}}{\pgfqpoint{5.300555in}{2.220246in}}%
\pgfusepath{clip}%
\pgfsetrectcap%
\pgfsetroundjoin%
\pgfsetlinewidth{1.505625pt}%
\pgfsetstrokecolor{currentstroke3}%
\pgfsetdash{}{0pt}%
\pgfpathmoveto{\pgfqpoint{0.829322in}{0.785370in}}%
\pgfpathlineto{\pgfqpoint{0.884079in}{0.821481in}}%
\pgfpathlineto{\pgfqpoint{0.938837in}{1.187190in}}%
\pgfpathlineto{\pgfqpoint{0.993595in}{1.418741in}}%
\pgfpathlineto{\pgfqpoint{1.048353in}{1.504184in}}%
\pgfpathlineto{\pgfqpoint{1.103111in}{1.673162in}}%
\pgfpathlineto{\pgfqpoint{1.157868in}{1.732291in}}%
\pgfpathlineto{\pgfqpoint{1.212626in}{1.702152in}}%
\pgfpathlineto{\pgfqpoint{1.267384in}{1.926123in}}%
\pgfpathlineto{\pgfqpoint{1.322142in}{1.855267in}}%
\pgfpathlineto{\pgfqpoint{1.376899in}{1.781513in}}%
\pgfpathlineto{\pgfqpoint{1.431657in}{1.850108in}}%
\pgfpathlineto{\pgfqpoint{1.486415in}{2.071202in}}%
\pgfpathlineto{\pgfqpoint{1.541173in}{1.897256in}}%
\pgfpathlineto{\pgfqpoint{1.595931in}{2.115955in}}%
\pgfpathlineto{\pgfqpoint{1.650688in}{1.923071in}}%
\pgfpathlineto{\pgfqpoint{1.705446in}{2.265588in}}%
\pgfpathlineto{\pgfqpoint{1.760204in}{1.868357in}}%
\pgfpathlineto{\pgfqpoint{1.814962in}{2.403242in}}%
\pgfpathlineto{\pgfqpoint{1.869720in}{1.992510in}}%
\pgfpathlineto{\pgfqpoint{1.924477in}{1.862869in}}%
\pgfpathlineto{\pgfqpoint{1.979235in}{1.218493in}}%
\pgfpathlineto{\pgfqpoint{2.033993in}{2.056717in}}%
\pgfpathlineto{\pgfqpoint{2.088751in}{1.800865in}}%
\pgfpathlineto{\pgfqpoint{2.143509in}{0.864777in}}%
\pgfpathlineto{\pgfqpoint{2.198266in}{1.494321in}}%
\pgfpathlineto{\pgfqpoint{2.307782in}{1.359994in}}%
\pgfpathlineto{\pgfqpoint{2.417298in}{1.703512in}}%
\pgfpathlineto{\pgfqpoint{2.526813in}{1.776525in}}%
\pgfpathlineto{\pgfqpoint{2.636329in}{2.396923in}}%
\pgfpathlineto{\pgfqpoint{2.745844in}{1.419216in}}%
\pgfpathlineto{\pgfqpoint{2.855360in}{2.015399in}}%
\pgfpathlineto{\pgfqpoint{2.964876in}{2.149374in}}%
\pgfpathlineto{\pgfqpoint{3.074391in}{2.542198in}}%
\pgfpathlineto{\pgfqpoint{3.183907in}{2.277481in}}%
\pgfpathlineto{\pgfqpoint{3.348180in}{2.622792in}}%
\pgfpathlineto{\pgfqpoint{3.402938in}{1.588281in}}%
\pgfpathlineto{\pgfqpoint{3.621969in}{2.354516in}}%
\pgfpathlineto{\pgfqpoint{3.731485in}{2.429998in}}%
\pgfpathlineto{\pgfqpoint{3.841000in}{1.690651in}}%
\pgfpathlineto{\pgfqpoint{4.279063in}{1.837310in}}%
\pgfpathlineto{\pgfqpoint{4.498094in}{1.875657in}}%
\pgfusepath{stroke}%
\end{pgfscope}%
\begin{pgfscope}%
\pgfpathrectangle{\pgfqpoint{0.588387in}{0.521603in}}{\pgfqpoint{5.300555in}{2.220246in}}%
\pgfusepath{clip}%
\pgfsetrectcap%
\pgfsetroundjoin%
\pgfsetlinewidth{1.505625pt}%
\pgfsetstrokecolor{currentstroke4}%
\pgfsetdash{}{0pt}%
\pgfpathmoveto{\pgfqpoint{0.829322in}{0.622524in}}%
\pgfpathlineto{\pgfqpoint{0.884079in}{0.705696in}}%
\pgfpathlineto{\pgfqpoint{0.938837in}{0.729954in}}%
\pgfpathlineto{\pgfqpoint{0.993595in}{0.802729in}}%
\pgfpathlineto{\pgfqpoint{1.048353in}{0.898260in}}%
\pgfpathlineto{\pgfqpoint{1.103111in}{0.925843in}}%
\pgfpathlineto{\pgfqpoint{1.157868in}{0.902307in}}%
\pgfpathlineto{\pgfqpoint{1.212626in}{0.913535in}}%
\pgfpathlineto{\pgfqpoint{1.267384in}{0.937932in}}%
\pgfpathlineto{\pgfqpoint{1.322142in}{0.890865in}}%
\pgfpathlineto{\pgfqpoint{1.376899in}{0.927003in}}%
\pgfpathlineto{\pgfqpoint{1.431657in}{0.898438in}}%
\pgfpathlineto{\pgfqpoint{1.486415in}{0.864020in}}%
\pgfpathlineto{\pgfqpoint{1.541173in}{0.886327in}}%
\pgfpathlineto{\pgfqpoint{1.595931in}{0.959492in}}%
\pgfpathlineto{\pgfqpoint{1.650688in}{0.948384in}}%
\pgfpathlineto{\pgfqpoint{1.705446in}{0.963143in}}%
\pgfpathlineto{\pgfqpoint{1.760204in}{0.954435in}}%
\pgfpathlineto{\pgfqpoint{1.814962in}{1.028037in}}%
\pgfpathlineto{\pgfqpoint{1.869720in}{0.976240in}}%
\pgfpathlineto{\pgfqpoint{1.924477in}{1.027781in}}%
\pgfpathlineto{\pgfqpoint{1.979235in}{1.040363in}}%
\pgfpathlineto{\pgfqpoint{2.033993in}{1.010853in}}%
\pgfpathlineto{\pgfqpoint{2.088751in}{1.070511in}}%
\pgfpathlineto{\pgfqpoint{2.143509in}{1.049030in}}%
\pgfpathlineto{\pgfqpoint{2.198266in}{1.140237in}}%
\pgfpathlineto{\pgfqpoint{2.253024in}{1.089746in}}%
\pgfpathlineto{\pgfqpoint{2.307782in}{1.127854in}}%
\pgfpathlineto{\pgfqpoint{2.362540in}{1.098231in}}%
\pgfpathlineto{\pgfqpoint{2.417298in}{1.191286in}}%
\pgfpathlineto{\pgfqpoint{2.472055in}{1.181318in}}%
\pgfpathlineto{\pgfqpoint{2.526813in}{1.230203in}}%
\pgfpathlineto{\pgfqpoint{2.581571in}{1.221059in}}%
\pgfpathlineto{\pgfqpoint{2.636329in}{1.186683in}}%
\pgfpathlineto{\pgfqpoint{2.691087in}{1.242631in}}%
\pgfpathlineto{\pgfqpoint{2.745844in}{1.232684in}}%
\pgfpathlineto{\pgfqpoint{2.800602in}{1.244252in}}%
\pgfpathlineto{\pgfqpoint{2.855360in}{1.255245in}}%
\pgfpathlineto{\pgfqpoint{2.910118in}{1.274606in}}%
\pgfpathlineto{\pgfqpoint{2.964876in}{1.341029in}}%
\pgfpathlineto{\pgfqpoint{3.019633in}{1.329719in}}%
\pgfpathlineto{\pgfqpoint{3.074391in}{1.348949in}}%
\pgfpathlineto{\pgfqpoint{3.129149in}{1.367688in}}%
\pgfpathlineto{\pgfqpoint{3.183907in}{1.352199in}}%
\pgfpathlineto{\pgfqpoint{3.238665in}{1.469292in}}%
\pgfpathlineto{\pgfqpoint{3.293422in}{1.430924in}}%
\pgfpathlineto{\pgfqpoint{3.348180in}{1.469428in}}%
\pgfpathlineto{\pgfqpoint{3.402938in}{1.461886in}}%
\pgfpathlineto{\pgfqpoint{3.457696in}{1.497055in}}%
\pgfpathlineto{\pgfqpoint{3.512454in}{1.559253in}}%
\pgfpathlineto{\pgfqpoint{3.567211in}{1.520381in}}%
\pgfpathlineto{\pgfqpoint{3.621969in}{1.714008in}}%
\pgfpathlineto{\pgfqpoint{3.676727in}{1.639277in}}%
\pgfpathlineto{\pgfqpoint{3.731485in}{1.569249in}}%
\pgfpathlineto{\pgfqpoint{3.786243in}{1.730220in}}%
\pgfpathlineto{\pgfqpoint{3.841000in}{1.593551in}}%
\pgfpathlineto{\pgfqpoint{3.895758in}{1.544514in}}%
\pgfpathlineto{\pgfqpoint{3.950516in}{1.629371in}}%
\pgfpathlineto{\pgfqpoint{4.005274in}{1.559253in}}%
\pgfpathlineto{\pgfqpoint{4.060032in}{1.806259in}}%
\pgfpathlineto{\pgfqpoint{4.114789in}{1.672920in}}%
\pgfpathlineto{\pgfqpoint{4.169547in}{1.765388in}}%
\pgfpathlineto{\pgfqpoint{4.224305in}{1.642100in}}%
\pgfpathlineto{\pgfqpoint{4.279063in}{1.768971in}}%
\pgfpathlineto{\pgfqpoint{4.388578in}{1.760930in}}%
\pgfpathlineto{\pgfqpoint{4.443336in}{1.677357in}}%
\pgfpathlineto{\pgfqpoint{4.498094in}{1.786695in}}%
\pgfpathlineto{\pgfqpoint{4.607609in}{2.022451in}}%
\pgfpathlineto{\pgfqpoint{4.990914in}{2.023261in}}%
\pgfpathlineto{\pgfqpoint{5.100430in}{2.116469in}}%
\pgfpathlineto{\pgfqpoint{5.155187in}{2.347339in}}%
\pgfpathlineto{\pgfqpoint{5.209945in}{2.165425in}}%
\pgfpathlineto{\pgfqpoint{5.483734in}{2.242633in}}%
\pgfpathlineto{\pgfqpoint{5.538492in}{2.152203in}}%
\pgfpathlineto{\pgfqpoint{5.648008in}{2.072718in}}%
\pgfusepath{stroke}%
\end{pgfscope}%
\begin{pgfscope}%
\pgfpathrectangle{\pgfqpoint{0.588387in}{0.521603in}}{\pgfqpoint{5.300555in}{2.220246in}}%
\pgfusepath{clip}%
\pgfsetrectcap%
\pgfsetroundjoin%
\pgfsetlinewidth{1.505625pt}%
\pgfsetstrokecolor{currentstroke5}%
\pgfsetdash{}{0pt}%
\pgfpathmoveto{\pgfqpoint{0.829322in}{0.632266in}}%
\pgfpathlineto{\pgfqpoint{0.884079in}{0.713285in}}%
\pgfpathlineto{\pgfqpoint{0.938837in}{0.718107in}}%
\pgfpathlineto{\pgfqpoint{0.993595in}{0.794642in}}%
\pgfpathlineto{\pgfqpoint{1.048353in}{0.887162in}}%
\pgfpathlineto{\pgfqpoint{1.103111in}{0.925160in}}%
\pgfpathlineto{\pgfqpoint{1.157868in}{0.909263in}}%
\pgfpathlineto{\pgfqpoint{1.212626in}{0.908148in}}%
\pgfpathlineto{\pgfqpoint{1.267384in}{0.942491in}}%
\pgfpathlineto{\pgfqpoint{1.322142in}{0.888232in}}%
\pgfpathlineto{\pgfqpoint{1.376899in}{0.911313in}}%
\pgfpathlineto{\pgfqpoint{1.431657in}{0.896274in}}%
\pgfpathlineto{\pgfqpoint{1.486415in}{0.872550in}}%
\pgfpathlineto{\pgfqpoint{1.541173in}{0.874942in}}%
\pgfpathlineto{\pgfqpoint{1.595931in}{0.944897in}}%
\pgfpathlineto{\pgfqpoint{1.650688in}{0.950479in}}%
\pgfpathlineto{\pgfqpoint{1.705446in}{0.967133in}}%
\pgfpathlineto{\pgfqpoint{1.760204in}{0.943405in}}%
\pgfpathlineto{\pgfqpoint{1.814962in}{1.011123in}}%
\pgfpathlineto{\pgfqpoint{1.869720in}{0.972979in}}%
\pgfpathlineto{\pgfqpoint{1.924477in}{1.018122in}}%
\pgfpathlineto{\pgfqpoint{1.979235in}{1.030925in}}%
\pgfpathlineto{\pgfqpoint{2.033993in}{1.009563in}}%
\pgfpathlineto{\pgfqpoint{2.088751in}{1.066226in}}%
\pgfpathlineto{\pgfqpoint{2.143509in}{1.036063in}}%
\pgfpathlineto{\pgfqpoint{2.198266in}{1.099289in}}%
\pgfpathlineto{\pgfqpoint{2.253024in}{1.080582in}}%
\pgfpathlineto{\pgfqpoint{2.307782in}{1.115377in}}%
\pgfpathlineto{\pgfqpoint{2.362540in}{1.092364in}}%
\pgfpathlineto{\pgfqpoint{2.417298in}{1.195802in}}%
\pgfpathlineto{\pgfqpoint{2.472055in}{1.173821in}}%
\pgfpathlineto{\pgfqpoint{2.526813in}{1.247338in}}%
\pgfpathlineto{\pgfqpoint{2.581571in}{1.200666in}}%
\pgfpathlineto{\pgfqpoint{2.636329in}{1.192576in}}%
\pgfpathlineto{\pgfqpoint{2.691087in}{1.240815in}}%
\pgfpathlineto{\pgfqpoint{2.745844in}{1.212694in}}%
\pgfpathlineto{\pgfqpoint{2.800602in}{1.252411in}}%
\pgfpathlineto{\pgfqpoint{2.855360in}{1.260303in}}%
\pgfpathlineto{\pgfqpoint{2.910118in}{1.275006in}}%
\pgfpathlineto{\pgfqpoint{2.964876in}{1.341451in}}%
\pgfpathlineto{\pgfqpoint{3.019633in}{1.377428in}}%
\pgfpathlineto{\pgfqpoint{3.074391in}{1.336571in}}%
\pgfpathlineto{\pgfqpoint{3.129149in}{1.372591in}}%
\pgfpathlineto{\pgfqpoint{3.183907in}{1.351326in}}%
\pgfpathlineto{\pgfqpoint{3.238665in}{1.430513in}}%
\pgfpathlineto{\pgfqpoint{3.293422in}{1.427164in}}%
\pgfpathlineto{\pgfqpoint{3.348180in}{1.465776in}}%
\pgfpathlineto{\pgfqpoint{3.402938in}{1.438202in}}%
\pgfpathlineto{\pgfqpoint{3.457696in}{1.493722in}}%
\pgfpathlineto{\pgfqpoint{3.512454in}{1.538774in}}%
\pgfpathlineto{\pgfqpoint{3.567211in}{1.532136in}}%
\pgfpathlineto{\pgfqpoint{3.621969in}{1.649701in}}%
\pgfpathlineto{\pgfqpoint{3.676727in}{1.630889in}}%
\pgfpathlineto{\pgfqpoint{3.731485in}{1.573379in}}%
\pgfpathlineto{\pgfqpoint{3.786243in}{1.702210in}}%
\pgfpathlineto{\pgfqpoint{3.841000in}{1.597860in}}%
\pgfpathlineto{\pgfqpoint{3.895758in}{1.558989in}}%
\pgfpathlineto{\pgfqpoint{3.950516in}{1.642325in}}%
\pgfpathlineto{\pgfqpoint{4.005274in}{1.555807in}}%
\pgfpathlineto{\pgfqpoint{4.060032in}{1.735701in}}%
\pgfpathlineto{\pgfqpoint{4.114789in}{1.659810in}}%
\pgfpathlineto{\pgfqpoint{4.169547in}{1.770336in}}%
\pgfpathlineto{\pgfqpoint{4.224305in}{1.645523in}}%
\pgfpathlineto{\pgfqpoint{4.279063in}{1.775455in}}%
\pgfpathlineto{\pgfqpoint{4.388578in}{1.752626in}}%
\pgfpathlineto{\pgfqpoint{4.443336in}{1.646696in}}%
\pgfpathlineto{\pgfqpoint{4.498094in}{1.763565in}}%
\pgfpathlineto{\pgfqpoint{4.607609in}{1.939655in}}%
\pgfpathlineto{\pgfqpoint{4.990914in}{2.015153in}}%
\pgfpathlineto{\pgfqpoint{5.100430in}{2.098849in}}%
\pgfpathlineto{\pgfqpoint{5.155187in}{2.260640in}}%
\pgfpathlineto{\pgfqpoint{5.209945in}{2.173692in}}%
\pgfpathlineto{\pgfqpoint{5.483734in}{2.142083in}}%
\pgfpathlineto{\pgfqpoint{5.538492in}{2.140922in}}%
\pgfpathlineto{\pgfqpoint{5.648008in}{2.073012in}}%
\pgfusepath{stroke}%
\end{pgfscope}%
\begin{pgfscope}%
\pgfpathrectangle{\pgfqpoint{0.588387in}{0.521603in}}{\pgfqpoint{5.300555in}{2.220246in}}%
\pgfusepath{clip}%
\pgfsetrectcap%
\pgfsetroundjoin%
\pgfsetlinewidth{1.505625pt}%
\pgfsetstrokecolor{currentstroke6}%
\pgfsetdash{}{0pt}%
\pgfpathmoveto{\pgfqpoint{0.829322in}{0.637971in}}%
\pgfpathlineto{\pgfqpoint{0.884079in}{0.709339in}}%
\pgfpathlineto{\pgfqpoint{0.938837in}{0.728483in}}%
\pgfpathlineto{\pgfqpoint{0.993595in}{0.784401in}}%
\pgfpathlineto{\pgfqpoint{1.048353in}{0.884537in}}%
\pgfpathlineto{\pgfqpoint{1.103111in}{0.908644in}}%
\pgfpathlineto{\pgfqpoint{1.157868in}{0.888167in}}%
\pgfpathlineto{\pgfqpoint{1.212626in}{0.897781in}}%
\pgfpathlineto{\pgfqpoint{1.267384in}{0.927381in}}%
\pgfpathlineto{\pgfqpoint{1.322142in}{0.891769in}}%
\pgfpathlineto{\pgfqpoint{1.376899in}{0.904557in}}%
\pgfpathlineto{\pgfqpoint{1.431657in}{0.887954in}}%
\pgfpathlineto{\pgfqpoint{1.486415in}{0.846665in}}%
\pgfpathlineto{\pgfqpoint{1.541173in}{0.871022in}}%
\pgfpathlineto{\pgfqpoint{1.595931in}{0.943228in}}%
\pgfpathlineto{\pgfqpoint{1.650688in}{0.932103in}}%
\pgfpathlineto{\pgfqpoint{1.705446in}{0.960288in}}%
\pgfpathlineto{\pgfqpoint{1.760204in}{0.934946in}}%
\pgfpathlineto{\pgfqpoint{1.814962in}{1.023456in}}%
\pgfpathlineto{\pgfqpoint{1.869720in}{0.962925in}}%
\pgfpathlineto{\pgfqpoint{1.924477in}{1.033240in}}%
\pgfpathlineto{\pgfqpoint{1.979235in}{1.032873in}}%
\pgfpathlineto{\pgfqpoint{2.033993in}{0.995482in}}%
\pgfpathlineto{\pgfqpoint{2.088751in}{1.042010in}}%
\pgfpathlineto{\pgfqpoint{2.143509in}{1.038503in}}%
\pgfpathlineto{\pgfqpoint{2.198266in}{1.074105in}}%
\pgfpathlineto{\pgfqpoint{2.253024in}{1.072760in}}%
\pgfpathlineto{\pgfqpoint{2.307782in}{1.109371in}}%
\pgfpathlineto{\pgfqpoint{2.362540in}{1.082154in}}%
\pgfpathlineto{\pgfqpoint{2.417298in}{1.209135in}}%
\pgfpathlineto{\pgfqpoint{2.472055in}{1.174151in}}%
\pgfpathlineto{\pgfqpoint{2.526813in}{1.193892in}}%
\pgfpathlineto{\pgfqpoint{2.581571in}{1.179668in}}%
\pgfpathlineto{\pgfqpoint{2.636329in}{1.171736in}}%
\pgfpathlineto{\pgfqpoint{2.691087in}{1.217084in}}%
\pgfpathlineto{\pgfqpoint{2.745844in}{1.218597in}}%
\pgfpathlineto{\pgfqpoint{2.800602in}{1.226492in}}%
\pgfpathlineto{\pgfqpoint{2.855360in}{1.239239in}}%
\pgfpathlineto{\pgfqpoint{2.910118in}{1.236105in}}%
\pgfpathlineto{\pgfqpoint{2.964876in}{1.316050in}}%
\pgfpathlineto{\pgfqpoint{3.019633in}{1.328209in}}%
\pgfpathlineto{\pgfqpoint{3.074391in}{1.330541in}}%
\pgfpathlineto{\pgfqpoint{3.129149in}{1.346624in}}%
\pgfpathlineto{\pgfqpoint{3.183907in}{1.335155in}}%
\pgfpathlineto{\pgfqpoint{3.238665in}{1.425306in}}%
\pgfpathlineto{\pgfqpoint{3.293422in}{1.371418in}}%
\pgfpathlineto{\pgfqpoint{3.348180in}{1.427683in}}%
\pgfpathlineto{\pgfqpoint{3.402938in}{1.474716in}}%
\pgfpathlineto{\pgfqpoint{3.457696in}{1.450632in}}%
\pgfpathlineto{\pgfqpoint{3.512454in}{1.548980in}}%
\pgfpathlineto{\pgfqpoint{3.567211in}{1.506236in}}%
\pgfpathlineto{\pgfqpoint{3.621969in}{1.604876in}}%
\pgfpathlineto{\pgfqpoint{3.676727in}{1.590400in}}%
\pgfpathlineto{\pgfqpoint{3.731485in}{1.577068in}}%
\pgfpathlineto{\pgfqpoint{3.786243in}{1.646808in}}%
\pgfpathlineto{\pgfqpoint{3.841000in}{1.576813in}}%
\pgfpathlineto{\pgfqpoint{3.895758in}{1.543970in}}%
\pgfpathlineto{\pgfqpoint{3.950516in}{1.620252in}}%
\pgfpathlineto{\pgfqpoint{4.005274in}{1.524847in}}%
\pgfpathlineto{\pgfqpoint{4.060032in}{1.746861in}}%
\pgfpathlineto{\pgfqpoint{4.114789in}{1.642025in}}%
\pgfpathlineto{\pgfqpoint{4.169547in}{1.761825in}}%
\pgfpathlineto{\pgfqpoint{4.224305in}{1.631176in}}%
\pgfpathlineto{\pgfqpoint{4.279063in}{1.727696in}}%
\pgfpathlineto{\pgfqpoint{4.388578in}{1.772617in}}%
\pgfpathlineto{\pgfqpoint{4.443336in}{1.601599in}}%
\pgfpathlineto{\pgfqpoint{4.498094in}{1.773492in}}%
\pgfpathlineto{\pgfqpoint{4.607609in}{2.007973in}}%
\pgfpathlineto{\pgfqpoint{4.990914in}{1.951633in}}%
\pgfpathlineto{\pgfqpoint{5.100430in}{2.141653in}}%
\pgfpathlineto{\pgfqpoint{5.155187in}{2.299466in}}%
\pgfpathlineto{\pgfqpoint{5.209945in}{2.141825in}}%
\pgfpathlineto{\pgfqpoint{5.483734in}{2.169921in}}%
\pgfpathlineto{\pgfqpoint{5.538492in}{2.177917in}}%
\pgfpathlineto{\pgfqpoint{5.648008in}{2.067885in}}%
\pgfusepath{stroke}%
\end{pgfscope}%
\begin{pgfscope}%
\pgfsetrectcap%
\pgfsetmiterjoin%
\pgfsetlinewidth{0.803000pt}%
\definecolor{currentstroke}{rgb}{0.000000,0.000000,0.000000}%
\pgfsetstrokecolor{currentstroke}%
\pgfsetdash{}{0pt}%
\pgfpathmoveto{\pgfqpoint{0.588387in}{0.521603in}}%
\pgfpathlineto{\pgfqpoint{0.588387in}{2.741849in}}%
\pgfusepath{stroke}%
\end{pgfscope}%
\begin{pgfscope}%
\pgfsetrectcap%
\pgfsetmiterjoin%
\pgfsetlinewidth{0.803000pt}%
\definecolor{currentstroke}{rgb}{0.000000,0.000000,0.000000}%
\pgfsetstrokecolor{currentstroke}%
\pgfsetdash{}{0pt}%
\pgfpathmoveto{\pgfqpoint{5.888942in}{0.521603in}}%
\pgfpathlineto{\pgfqpoint{5.888942in}{2.741849in}}%
\pgfusepath{stroke}%
\end{pgfscope}%
\begin{pgfscope}%
\pgfsetrectcap%
\pgfsetmiterjoin%
\pgfsetlinewidth{0.803000pt}%
\definecolor{currentstroke}{rgb}{0.000000,0.000000,0.000000}%
\pgfsetstrokecolor{currentstroke}%
\pgfsetdash{}{0pt}%
\pgfpathmoveto{\pgfqpoint{0.588387in}{0.521603in}}%
\pgfpathlineto{\pgfqpoint{5.888942in}{0.521603in}}%
\pgfusepath{stroke}%
\end{pgfscope}%
\begin{pgfscope}%
\pgfsetrectcap%
\pgfsetmiterjoin%
\pgfsetlinewidth{0.803000pt}%
\definecolor{currentstroke}{rgb}{0.000000,0.000000,0.000000}%
\pgfsetstrokecolor{currentstroke}%
\pgfsetdash{}{0pt}%
\pgfpathmoveto{\pgfqpoint{0.588387in}{2.741849in}}%
\pgfpathlineto{\pgfqpoint{5.888942in}{2.741849in}}%
\pgfusepath{stroke}%
\end{pgfscope}%
\begin{pgfscope}%
\pgfsetbuttcap%
\pgfsetmiterjoin%
\definecolor{currentfill}{rgb}{1.000000,1.000000,1.000000}%
\pgfsetfillcolor{currentfill}%
\pgfsetfillopacity{0.800000}%
\pgfsetlinewidth{1.003750pt}%
\definecolor{currentstroke}{rgb}{0.800000,0.800000,0.800000}%
\pgfsetstrokecolor{currentstroke}%
\pgfsetstrokeopacity{0.800000}%
\pgfsetdash{}{0pt}%
\pgfpathmoveto{\pgfqpoint{5.976442in}{1.530583in}}%
\pgfpathlineto{\pgfqpoint{8.259376in}{1.530583in}}%
\pgfpathquadraticcurveto{\pgfqpoint{8.284376in}{1.530583in}}{\pgfqpoint{8.284376in}{1.555583in}}%
\pgfpathlineto{\pgfqpoint{8.284376in}{2.654349in}}%
\pgfpathquadraticcurveto{\pgfqpoint{8.284376in}{2.679349in}}{\pgfqpoint{8.259376in}{2.679349in}}%
\pgfpathlineto{\pgfqpoint{5.976442in}{2.679349in}}%
\pgfpathquadraticcurveto{\pgfqpoint{5.951442in}{2.679349in}}{\pgfqpoint{5.951442in}{2.654349in}}%
\pgfpathlineto{\pgfqpoint{5.951442in}{1.555583in}}%
\pgfpathquadraticcurveto{\pgfqpoint{5.951442in}{1.530583in}}{\pgfqpoint{5.976442in}{1.530583in}}%
\pgfpathlineto{\pgfqpoint{5.976442in}{1.530583in}}%
\pgfpathclose%
\pgfusepath{stroke,fill}%
\end{pgfscope}%
\begin{pgfscope}%
\pgfsetrectcap%
\pgfsetroundjoin%
\pgfsetlinewidth{1.505625pt}%
\pgfsetstrokecolor{currentstroke2}%
\pgfsetdash{}{0pt}%
\pgfpathmoveto{\pgfqpoint{6.001442in}{2.578129in}}%
\pgfpathlineto{\pgfqpoint{6.126442in}{2.578129in}}%
\pgfpathlineto{\pgfqpoint{6.251442in}{2.578129in}}%
\pgfusepath{stroke}%
\end{pgfscope}%
\begin{pgfscope}%
\definecolor{textcolor}{rgb}{0.000000,0.000000,0.000000}%
\pgfsetstrokecolor{textcolor}%
\pgfsetfillcolor{textcolor}%
\pgftext[x=6.351442in,y=2.534379in,left,base]{\color{textcolor}{\rmfamily\fontsize{9.000000}{10.800000}\selectfont\catcode`\^=\active\def^{\ifmmode\sp\else\^{}\fi}\catcode`\%=\active\def%{\%}\CyclesMatchChunks{} \& \MergeLinear{}}}%
\end{pgfscope}%
\begin{pgfscope}%
\pgfsetrectcap%
\pgfsetroundjoin%
\pgfsetlinewidth{1.505625pt}%
\pgfsetstrokecolor{currentstroke4}%
\pgfsetdash{}{0pt}%
\pgfpathmoveto{\pgfqpoint{6.001442in}{2.391178in}}%
\pgfpathlineto{\pgfqpoint{6.126442in}{2.391178in}}%
\pgfpathlineto{\pgfqpoint{6.251442in}{2.391178in}}%
\pgfusepath{stroke}%
\end{pgfscope}%
\begin{pgfscope}%
\definecolor{textcolor}{rgb}{0.000000,0.000000,0.000000}%
\pgfsetstrokecolor{textcolor}%
\pgfsetfillcolor{textcolor}%
\pgftext[x=6.351442in,y=2.347428in,left,base]{\color{textcolor}{\rmfamily\fontsize{9.000000}{10.800000}\selectfont\catcode`\^=\active\def^{\ifmmode\sp\else\^{}\fi}\catcode`\%=\active\def%{\%}\Neighbors{} \& \MergeLinear{}}}%
\end{pgfscope}%
\begin{pgfscope}%
\pgfsetrectcap%
\pgfsetroundjoin%
\pgfsetlinewidth{1.505625pt}%
\pgfsetstrokecolor{currentstroke5}%
\pgfsetdash{}{0pt}%
\pgfpathmoveto{\pgfqpoint{6.001442in}{2.207707in}}%
\pgfpathlineto{\pgfqpoint{6.126442in}{2.207707in}}%
\pgfpathlineto{\pgfqpoint{6.251442in}{2.207707in}}%
\pgfusepath{stroke}%
\end{pgfscope}%
\begin{pgfscope}%
\definecolor{textcolor}{rgb}{0.000000,0.000000,0.000000}%
\pgfsetstrokecolor{textcolor}%
\pgfsetfillcolor{textcolor}%
\pgftext[x=6.351442in,y=2.163957in,left,base]{\color{textcolor}{\rmfamily\fontsize{9.000000}{10.800000}\selectfont\catcode`\^=\active\def^{\ifmmode\sp\else\^{}\fi}\catcode`\%=\active\def%{\%}\NeighborsDegree{} \& \MergeLinear{}}}%
\end{pgfscope}%
\begin{pgfscope}%
\pgfsetrectcap%
\pgfsetroundjoin%
\pgfsetlinewidth{1.505625pt}%
\pgfsetstrokecolor{currentstroke6}%
\pgfsetdash{}{0pt}%
\pgfpathmoveto{\pgfqpoint{6.001442in}{2.020756in}}%
\pgfpathlineto{\pgfqpoint{6.126442in}{2.020756in}}%
\pgfpathlineto{\pgfqpoint{6.251442in}{2.020756in}}%
\pgfusepath{stroke}%
\end{pgfscope}%
\begin{pgfscope}%
\definecolor{textcolor}{rgb}{0.000000,0.000000,0.000000}%
\pgfsetstrokecolor{textcolor}%
\pgfsetfillcolor{textcolor}%
\pgftext[x=6.351442in,y=1.977006in,left,base]{\color{textcolor}{\rmfamily\fontsize{9.000000}{10.800000}\selectfont\catcode`\^=\active\def^{\ifmmode\sp\else\^{}\fi}\catcode`\%=\active\def%{\%}\None{} \& \MergeLinear{}}}%
\end{pgfscope}%
\begin{pgfscope}%
\pgfsetrectcap%
\pgfsetroundjoin%
\pgfsetlinewidth{1.505625pt}%
\pgfsetstrokecolor{currentstroke1}%
\pgfsetdash{}{0pt}%
\pgfpathmoveto{\pgfqpoint{6.001442in}{1.837285in}}%
\pgfpathlineto{\pgfqpoint{6.126442in}{1.837285in}}%
\pgfpathlineto{\pgfqpoint{6.251442in}{1.837285in}}%
\pgfusepath{stroke}%
\end{pgfscope}%
\begin{pgfscope}%
\definecolor{textcolor}{rgb}{0.000000,0.000000,0.000000}%
\pgfsetstrokecolor{textcolor}%
\pgfsetfillcolor{textcolor}%
\pgftext[x=6.351442in,y=1.793535in,left,base]{\color{textcolor}{\rmfamily\fontsize{9.000000}{10.800000}\selectfont\catcode`\^=\active\def^{\ifmmode\sp\else\^{}\fi}\catcode`\%=\active\def%{\%}\Cuts{} \& \MergeLinear{}}}%
\end{pgfscope}%
\begin{pgfscope}%
\pgfsetrectcap%
\pgfsetroundjoin%
\pgfsetlinewidth{1.505625pt}%
\pgfsetstrokecolor{currentstroke3}%
\pgfsetdash{}{0pt}%
\pgfpathmoveto{\pgfqpoint{6.001442in}{1.653813in}}%
\pgfpathlineto{\pgfqpoint{6.126442in}{1.653813in}}%
\pgfpathlineto{\pgfqpoint{6.251442in}{1.653813in}}%
\pgfusepath{stroke}%
\end{pgfscope}%
\begin{pgfscope}%
\definecolor{textcolor}{rgb}{0.000000,0.000000,0.000000}%
\pgfsetstrokecolor{textcolor}%
\pgfsetfillcolor{textcolor}%
\pgftext[x=6.351442in,y=1.610063in,left,base]{\color{textcolor}{\rmfamily\fontsize{9.000000}{10.800000}\selectfont\catcode`\^=\active\def^{\ifmmode\sp\else\^{}\fi}\catcode`\%=\active\def%{\%}\KernighanLin{} \& \MergeLinear{}}}%
\end{pgfscope}%
\end{pgfpicture}%
\makeatother%
\endgroup%
}
	\caption[Failing splitting strategies for graphs with no NAC-coloring]{
		Mean running time to find all NAC-colorings for graphs with no NAC-coloring with failing splitting strategies.}%
	\label{fig:graph_no_nac_coloring_generated_rigid_failing_split_first_runtime}
\end{figure}%


Smart split described in \Cref{sec:smart_split}
did not improve the runtime.
We expected minor performance hit for smaller graphs because heuristic is run
multiple times, but gains for larger graphs where subgraphs merging order
should join subgraphs near to each other together. This is not the case.

\subsection{Final comparison}

Based on our benchmarks, most of which we presented in the previous sections,
we evaluate respective strategies and choose ones
that should be kept and merged into PyRigi.
For graph classes with a lot of NAC-colorings,
\NaiveCycles{} is usually the best choice
when we search for a single NAC-coloring.
%
The user of the library has to pay attention while using this strategy
as if there is just a few or no NAC-coloring and the graph does not trivially collapse
into few monochromatic classes, the runtime is huge.
Therefore, \NaiveCycles{} should be available, but the default.

First we evaluate strategies for splitting:
Strategy \None{} works well in most of the cases and for simple
instances it outperforms other strategies as it requires little to no overhead.
%
\CyclesMatchChunks{} performed well in the majority cases simillar to \None{},
but it is generally slightly outperformed by \Neighbors{}.
%
\Neighbors{} and \NeighborsDegree{} generally managed to reduce the number
of \IsNACColoring{} calls and for complex cases they often
perform slightly better then \None{}.
Note that the implementation of \Neighbors{} is slightly simpler than
the one of \NeighborsDegree{}.
%
\KernighanLin{} results are not consistent.
\Cuts{} perform even worse for even small instance.
This is probably because they operate on a graph of monochromatic
components that does not preserver all the properties
of the original graph well.
Never they were noticeably faster than \Neighbors{}.
Therefore, these algorithms cannot be used in general case.
%
Based on this evaluation, we keep algorithms \None{} and \Neighbors{}
as they perform the best across all graph classes and
as \Neighbors{} strategy has simpler implementation.

\MergeLinear{} strategy performed the best across the board.
It seems that the idea of merging one growing subgraph
with subgraphs of initial small size works the best
in a general case.
%
For listing all NAC-colorings, \SharedVertices{} performs
slightly better, but for a search for a single NAC-coloring for simple instances,
its runtime becomes less predictable.
%
From the general idea of the \Subgraphs{} algorithm,
it makes a lot of sense \Log{} strategy would work great.
This is not the case as \Log{} is slow for all the cases we tested.
%
\MinMax{} and \SortedBits{} strategies often perform badly
and if they do not, they do not outperform other strategies.
%
\PromisingCycles{} performs similarly well as \Neighbors{}
for graphs with no NAC-coloring,
but fails for simple instances which makes it not universal enough.
%
\SortedSize{} and \Score{} also perform well for graphs with no NAC-coloring,
but as explained in \Cref{sec:merging},
they are unsuitable for instances where only a single NAC-coloring
should be found as these strategies always list all NAC-colorings
on all subgraphs.
%
Based on this evaluation, we keep strategy \MergeLinear{} as a general good choice
and also preserve \SharedVertices{} for graphs that are not simple instances.

