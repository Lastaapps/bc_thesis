\chapter{Implementation \& Benchmarks}%
\label{chapter:benchmarks}

\begin{chapterabstract}

	In this chapter, we first describe the structure of the project
	and discuss some design choices.
	After that, we evaluate performance of the algorithms
	proposed in \Cref{chapter:alg}.
	First, we compare the high-level approaches with previous approaches and among each other,
	and then we compare heuristics with each other
	for different  graph classes.
	We show reduction both in runtime and in the number
	of \IsNACColoring{} checks performed.
	Lastly, we evaluate the strategies and
	recommend ones that should be merged into PyRigi.

\end{chapterabstract}

\section{Implementation}

\todo[inline]{Popsat základní API knihovny}

In this section we first describe the structure of the project containing
the code of the algorithm.
Next, we mention libraries, relation to PyRigi and
some worth mentioning implementation details.
The code is available as an attachment of this thesis,
nevertheless the most recent version can be found on GitHub~\cite{my_code}.

The code is written in Python~\cite{python}, minimal supported version is Python 3.12.
To set up the project, create a virtual environment and install packages
from \texttt{requirements.txt}. On NixOS, \texttt{shell.nix} can be used.
See \texttt{README.md} for additional instructions.
We go through the main folders and files of the project,
to see the code structure in~\Cref{chapter:attachments}.

The base directory contains tooling for running, visualizing and exporting benchmarks.
File \texttt{NAC\_playground.ipynb} presents a simple case
to show how the algorithm's API can be used.
File \texttt{NAC\_presentation.ipynb} shows how the benchmarks can be run and analyzed.

The code of the algorithm described in~\Cref{sec:stable_cuts_implementation}
with additional helper functions is implemented in directory \texttt{stablecut}.
Note that some changes were done when the code was merged into PyRigi.

The code of the algorithms described in~\Cref{chapter:alg}
is stored in directory \texttt{nac}.
%
Directory \texttt{nac/util} stores helper functions and classes
like an implementation of the \textsc{UnionFind} data structure.
%
File \texttt{check.py} implements \IsNACColoring{} check.
%
File \texttt{monochromatic\_classes.py} is used to find \trcon{} components
and monochromatic classes in a graph. With this, we can compare performance
between using monochromatic classes, \trcon{} components or just edges.
%
File \texttt{cycle\_detection.py} holds algorithms for finding cycles
used by \Cref{sec:small_cycles}
and some heuristics.
%
In \Cref{sec:polynomial_optimizations}
we presented checks that can in polynomial time
sometimes find a NAC-coloring or determine that there is none.
These checks are implemented in \texttt{existence.py} and
used mostly from \texttt{single.py} that is the entry-point
for a single NAC-coloring search.
%
General NAC-coloring searching is implemented in \texttt{search.py}
along with parameter parsing, graph vertices normalization and
optimizations like search for articulation vertices.
After that, the correct algorithm from \Naive{}, \NaiveCycles{} or \Subgraphs{}
is chosen and called.
%
These algorithms are implemented in \texttt{algorithms.py} alongside many helper functions.
Heuristics for \Subgraphs{} algorithm are stored in \texttt{strategies.py}.
%
Tests of both stable cuts and NAC-coloring parts are stored in directory \texttt{test}.

Common function parameters are:
\texttt{graph} repressing the subgraph where NAC-colorings should be found.
%
\texttt{comp\_graph} is a graph where vertices are some integer IDs of monochromatic classes
and edges exist if the classes are neighboring,
see \Cref{observ:monochromatic_classes_graph}.
%
Monochromatic class IDs also serve as indices into \texttt{component\_to\_edges}
that maps an ID of a monochromatic class to its edges.
%
NAC-colorings are represented as bit-masks where bit's offset correspond to a class ID\@.

In \texttt{graphs\_store} we store datasets used for benchmarking.
Graphs are either obtained from~\cite{extremal_graphs},
generated using Nauty~\cite{nauty} with a plugin~\cite{nauty_plugin}
or generated using NetworkX~\cite{networkx} and checks from PyRigi~\cite{pyrigi}.
Graph are mostly stored in Graph6 format~\cite{graph6}.
Code for reading graphs from the store can be found in \texttt{benchmarks/dataset.py},
code for generating some graph classes can be found in  \texttt{benchmarks/generators.py}.
In the \texttt{benchmark/precomputed} directory, there are all result of the benchmarks that
we use for algorithms evaluation.
Individual runs are stored in a compressed CSV file.

As \IsNACColoring{} is a core component of all our algorithms,
we tried to optimized it well.
%
In the base implementation of \IsNACColoring{},
subgraphs from \( \red \) and \( \blue \) edges are created.
To create such subgraphs in code, edges can be added to an empty graph
using NetworkX's function \texttt{add\_edges\_from}.
%
This is rather slow as creating new vertices in the empty graph causes noticeable overhead.
Therefore, we create a graph with no edges and the same vertices as the original graph,
cache it and reuse it for the checks.
Every time only edges are added, the check is run, and the edges are removed.
By doing this, the performance of \IsNACColoring{} is increased by roughly 40\%.
%
Another way how the performance could be increased is by reserving space in lists
when the final size is known.
To our knowledge, this is impossible in Python.

The code uses \texttt{Graph} class and related algorithms from NetworkX~\cite{networkx}
as the base of many operations. We use some utility functions from PyRigi~\cite{pyrigi}
related to (global) rigidity tests and rigidity components search.
Other than that, the code is not dependent on PyRigi.
%
Pytest~\cite{pytest} is used for testing and
Matplotlib~\cite{matplotlib} for visualizations.


\section{Benchmarks}

In this section we first set meaningful targets for our benchmarks,
then we compare the performance of our algorithms with the previous implementation
and show running time and internal search optimizations for various graph classes.

The main question regarding NAC-coloring search is whether a graph has a NAC-coloring.
We usually ask the algorithm to not only answer yes, but to also provide a certificate.
%
For flexible graphs, it is algorithmically quite simple to find a NAC-coloring,
so this question is more interesting for rigid graphs.
%
For flexible graphs, it is more interesting to ask for the number of NAC-colorings
of a graph.
Note, that for larger flexible graphs with around thirty vertices
the number of NAC-colorings is huge as it often grows exponentially.
This slows our algorithm down as just materializing and iterating exponential
number of NAC-colorings takes exponential time.
%
For such cases, the FPT algorithm described in \Cref{chapter:fpt}
could be a better fit as it does not materialize any NAC-colorings,
but implementing it is out of the scope of this thesis.

The graphs the algorithm works with have integer vertices (we relabel the otherwise).
We noticed that for synthetically generated graphs,
the algorithm performs slightly better compared to
the same algorithm run on the same graph with vertices randomly relabeled.
To counteract this, we tried to relabel the graphs using BFS,
but we reached no performance gains compared to the random relabeling.

Visualizations in the following sections were created by grouping data per dataset,
graph size and some specified attribute --- usually the strategies used.
%
We show mean of the running time in milliseconds or
the number of \IsNACColoring{} calls.
%
Graphs with median and third quartile can be seen in \texttt{NAC\_presentation.ipynb}.


\subsection{Improvement over previous solutions}

The benchmarks comparing our algorithm with the previous implementations
were run on Linux on a laptop with Intel i7 of the 11th generation
with CPython 3.12~\cite{cpython} and SageMath 10.4~\cite{sagemath}.
The remaining benchmarks were run a laptop with Intel i5 of the 6th generation
using CPython 3.12.
On more modern hardware, the running times could be significantly shorter.

\Cref{tab:all_min_rigid}
shows the time required for finding all the NAC-colorings
of all minimally rigid graphs with given vertex count.
%
They are generated using Nauty~\cite{nauty}
with a corresponding plugin~\cite{nauty_plugin}.
%
We show results of the implementation
in \flexrilog{}~\cite{flexrilog} using \trcon{} components
run in SageMath~\cite{sagemath}
and compare them to our implementation of the same \Naive{} algorithm
using $\triangle$-connected components
and monochromatic classes as described in \Cref{sec:NACvalid}.
Next column shows \NaiveCycles{} from \Cref{sec:small_cycles}
using monochromatic classes.
The last column is for the \NeighborsDegree{} (each initial subgraph has $k=4$ monochromatic classes)
with \MergeLinear{} merging strategy.
%
In every case, our algorithms are significantly faster than implementation in \flexrilog{}~\cite{flexrilog}.
Notice also huge advantage gained by using monochromatic classes instead of \trcon{} components.
%
\begin{table}[ht]
	\caption[Running times on graphs]{
		The time (in seconds) needed to find all NAC-colorings for all graphs with a given size. Run by us.
		\textsc{FRLG} stands for \flexrilog{}, \textsc{ND} for \NeighborsDegree{}.}%
	\label{tab:all_min_rigid}
	\vspace{0.3cm}
	\centering
	\begin{tabular}{ccccccc}
		\hline
		\,$|V(G)|$\, & \,\#graphs\, & \,FRLG\, & \,$\triangle$-comps.\, & \,monochr.\, & \,cycles\, & \,\textsc{ND}\, \\
		\hline
		% 5        & 3           & 0.007 s      & 0.002 s            & 0.001 s       & 0.001 s & 0.002 s          \\
		% 6        & 13          & 0.063 s      & 0.030 s            & 0.010 s       & 0.005 s & 0.007 s          \\
		% 7        & 70          & 0.57 s       & 0.052 s            & 0.047 s       & 0.029 s & 0.041 s          \\
		8            & 608          & 14       & 1.09                   & 0.97         & 0.36       & 0.49            \\
		9            & 7\,222       & 509      & 34                     & 29           & 5.8        & 8.6             \\
		10           & 110\,132     & 27k      & 1\,725                 & 1\,446       & 151        & 213             \\
		11           & 2\,039\,273  & -        & -                      & -            & 5\,440     & 6\,650          \\
		\hline
	\end{tabular}
\end{table}

\Cref{fig:graph_time_minimally_rigid}
shows timings to compute all NAC-colorings of minimally rigid graphs
depending on the strategy used.
We did not list all NAC-colorings for all minimally rigid graphs with twelve vertices and more
as there are too many such graphs (around 44 millions for twelve vertices).
%
We rather randomly generated dataset of minimally rigid graphs
using NetworkX~\cite{networkx} and PyRigi~\cite{pyrigi}.
%
It can be seen that for graphs up to around fourteen vertices the \NaiveCycles{} algorithm
is still faster than \Subgraphs{}.
For graphs with more than eighteen vertices,
the growing advantage of \Subgraphs{} is already clear.
\Subgraphs{} algorithm can list all NAC-colorings on graphs with 30 monochromatic classes
usually in a few seconds.
This corresponds to \( 2^{29} \) checks done by \NaiveCycles{}.

\begin{figure}[ht]
	\centering
	\scalebox{\BenchFigureScale}{%% Creator: Matplotlib, PGF backend
%%
%% To include the figure in your LaTeX document, write
%%   \input{<filename>.pgf}
%%
%% Make sure the required packages are loaded in your preamble
%%   \usepackage{pgf}
%%
%% Also ensure that all the required font packages are loaded; for instance,
%% the lmodern package is sometimes necessary when using math font.
%%   \usepackage{lmodern}
%%
%% Figures using additional raster images can only be included by \input if
%% they are in the same directory as the main LaTeX file. For loading figures
%% from other directories you can use the `import` package
%%   \usepackage{import}
%%
%% and then include the figures with
%%   \import{<path to file>}{<filename>.pgf}
%%
%% Matplotlib used the following preamble
%%   \def\mathdefault#1{#1}
%%   \everymath=\expandafter{\the\everymath\displaystyle}
%%   \IfFileExists{scrextend.sty}{
%%     \usepackage[fontsize=10.000000pt]{scrextend}
%%   }{
%%     \renewcommand{\normalsize}{\fontsize{10.000000}{12.000000}\selectfont}
%%     \normalsize
%%   }
%%   
%%   \ifdefined\pdftexversion\else  % non-pdftex case.
%%     \usepackage{fontspec}
%%     \setmainfont{DejaVuSans.ttf}[Path=\detokenize{/home/petr/Projects/PyRigi/.venv/lib/python3.12/site-packages/matplotlib/mpl-data/fonts/ttf/}]
%%     \setsansfont{DejaVuSans.ttf}[Path=\detokenize{/home/petr/Projects/PyRigi/.venv/lib/python3.12/site-packages/matplotlib/mpl-data/fonts/ttf/}]
%%     \setmonofont{DejaVuSansMono.ttf}[Path=\detokenize{/home/petr/Projects/PyRigi/.venv/lib/python3.12/site-packages/matplotlib/mpl-data/fonts/ttf/}]
%%   \fi
%%   \makeatletter\@ifpackageloaded{under\Score{}}{}{\usepackage[strings]{under\Score{}}}\makeatother
%%
\begingroup%
\makeatletter%
\begin{pgfpicture}%
\pgfpathrectangle{\pgfpointorigin}{\pgfqpoint{8.384376in}{2.841849in}}%
\pgfusepath{use as bounding box, clip}%
\begin{pgfscope}%
\pgfsetbuttcap%
\pgfsetmiterjoin%
\definecolor{currentfill}{rgb}{1.000000,1.000000,1.000000}%
\pgfsetfillcolor{currentfill}%
\pgfsetlinewidth{0.000000pt}%
\definecolor{currentstroke}{rgb}{1.000000,1.000000,1.000000}%
\pgfsetstrokecolor{currentstroke}%
\pgfsetdash{}{0pt}%
\pgfpathmoveto{\pgfqpoint{0.000000in}{0.000000in}}%
\pgfpathlineto{\pgfqpoint{8.384376in}{0.000000in}}%
\pgfpathlineto{\pgfqpoint{8.384376in}{2.841849in}}%
\pgfpathlineto{\pgfqpoint{0.000000in}{2.841849in}}%
\pgfpathlineto{\pgfqpoint{0.000000in}{0.000000in}}%
\pgfpathclose%
\pgfusepath{fill}%
\end{pgfscope}%
\begin{pgfscope}%
\pgfsetbuttcap%
\pgfsetmiterjoin%
\definecolor{currentfill}{rgb}{1.000000,1.000000,1.000000}%
\pgfsetfillcolor{currentfill}%
\pgfsetlinewidth{0.000000pt}%
\definecolor{currentstroke}{rgb}{0.000000,0.000000,0.000000}%
\pgfsetstrokecolor{currentstroke}%
\pgfsetstrokeopacity{0.000000}%
\pgfsetdash{}{0pt}%
\pgfpathmoveto{\pgfqpoint{0.588387in}{0.521603in}}%
\pgfpathlineto{\pgfqpoint{5.487514in}{0.521603in}}%
\pgfpathlineto{\pgfqpoint{5.487514in}{2.741849in}}%
\pgfpathlineto{\pgfqpoint{0.588387in}{2.741849in}}%
\pgfpathlineto{\pgfqpoint{0.588387in}{0.521603in}}%
\pgfpathclose%
\pgfusepath{fill}%
\end{pgfscope}%
\begin{pgfscope}%
\pgfsetbuttcap%
\pgfsetroundjoin%
\definecolor{currentfill}{rgb}{0.000000,0.000000,0.000000}%
\pgfsetfillcolor{currentfill}%
\pgfsetlinewidth{0.803000pt}%
\definecolor{currentstroke}{rgb}{0.000000,0.000000,0.000000}%
\pgfsetstrokecolor{currentstroke}%
\pgfsetdash{}{0pt}%
\pgfsys@defobject{currentmarker}{\pgfqpoint{0.000000in}{-0.048611in}}{\pgfqpoint{0.000000in}{0.000000in}}{%
\pgfpathmoveto{\pgfqpoint{0.000000in}{0.000000in}}%
\pgfpathlineto{\pgfqpoint{0.000000in}{-0.048611in}}%
\pgfusepath{stroke,fill}%
}%
\begin{pgfscope}%
\pgfsys@transformshift{1.098414in}{0.521603in}%
\pgfsys@useobject{currentmarker}{}%
\end{pgfscope}%
\end{pgfscope}%
\begin{pgfscope}%
\definecolor{textcolor}{rgb}{0.000000,0.000000,0.000000}%
\pgfsetstrokecolor{textcolor}%
\pgfsetfillcolor{textcolor}%
\pgftext[x=1.098414in,y=0.424381in,,top]{\color{textcolor}{\rmfamily\fontsize{10.000000}{12.000000}\selectfont\catcode`\^=\active\def^{\ifmmode\sp\else\^{}\fi}\catcode`\%=\active\def%{\%}$\mathdefault{4}$}}%
\end{pgfscope}%
\begin{pgfscope}%
\pgfsetbuttcap%
\pgfsetroundjoin%
\definecolor{currentfill}{rgb}{0.000000,0.000000,0.000000}%
\pgfsetfillcolor{currentfill}%
\pgfsetlinewidth{0.803000pt}%
\definecolor{currentstroke}{rgb}{0.000000,0.000000,0.000000}%
\pgfsetstrokecolor{currentstroke}%
\pgfsetdash{}{0pt}%
\pgfsys@defobject{currentmarker}{\pgfqpoint{0.000000in}{-0.048611in}}{\pgfqpoint{0.000000in}{0.000000in}}{%
\pgfpathmoveto{\pgfqpoint{0.000000in}{0.000000in}}%
\pgfpathlineto{\pgfqpoint{0.000000in}{-0.048611in}}%
\pgfusepath{stroke,fill}%
}%
\begin{pgfscope}%
\pgfsys@transformshift{1.673091in}{0.521603in}%
\pgfsys@useobject{currentmarker}{}%
\end{pgfscope}%
\end{pgfscope}%
\begin{pgfscope}%
\definecolor{textcolor}{rgb}{0.000000,0.000000,0.000000}%
\pgfsetstrokecolor{textcolor}%
\pgfsetfillcolor{textcolor}%
\pgftext[x=1.673091in,y=0.424381in,,top]{\color{textcolor}{\rmfamily\fontsize{10.000000}{12.000000}\selectfont\catcode`\^=\active\def^{\ifmmode\sp\else\^{}\fi}\catcode`\%=\active\def%{\%}$\mathdefault{8}$}}%
\end{pgfscope}%
\begin{pgfscope}%
\pgfsetbuttcap%
\pgfsetroundjoin%
\definecolor{currentfill}{rgb}{0.000000,0.000000,0.000000}%
\pgfsetfillcolor{currentfill}%
\pgfsetlinewidth{0.803000pt}%
\definecolor{currentstroke}{rgb}{0.000000,0.000000,0.000000}%
\pgfsetstrokecolor{currentstroke}%
\pgfsetdash{}{0pt}%
\pgfsys@defobject{currentmarker}{\pgfqpoint{0.000000in}{-0.048611in}}{\pgfqpoint{0.000000in}{0.000000in}}{%
\pgfpathmoveto{\pgfqpoint{0.000000in}{0.000000in}}%
\pgfpathlineto{\pgfqpoint{0.000000in}{-0.048611in}}%
\pgfusepath{stroke,fill}%
}%
\begin{pgfscope}%
\pgfsys@transformshift{2.247769in}{0.521603in}%
\pgfsys@useobject{currentmarker}{}%
\end{pgfscope}%
\end{pgfscope}%
\begin{pgfscope}%
\definecolor{textcolor}{rgb}{0.000000,0.000000,0.000000}%
\pgfsetstrokecolor{textcolor}%
\pgfsetfillcolor{textcolor}%
\pgftext[x=2.247769in,y=0.424381in,,top]{\color{textcolor}{\rmfamily\fontsize{10.000000}{12.000000}\selectfont\catcode`\^=\active\def^{\ifmmode\sp\else\^{}\fi}\catcode`\%=\active\def%{\%}$\mathdefault{12}$}}%
\end{pgfscope}%
\begin{pgfscope}%
\pgfsetbuttcap%
\pgfsetroundjoin%
\definecolor{currentfill}{rgb}{0.000000,0.000000,0.000000}%
\pgfsetfillcolor{currentfill}%
\pgfsetlinewidth{0.803000pt}%
\definecolor{currentstroke}{rgb}{0.000000,0.000000,0.000000}%
\pgfsetstrokecolor{currentstroke}%
\pgfsetdash{}{0pt}%
\pgfsys@defobject{currentmarker}{\pgfqpoint{0.000000in}{-0.048611in}}{\pgfqpoint{0.000000in}{0.000000in}}{%
\pgfpathmoveto{\pgfqpoint{0.000000in}{0.000000in}}%
\pgfpathlineto{\pgfqpoint{0.000000in}{-0.048611in}}%
\pgfusepath{stroke,fill}%
}%
\begin{pgfscope}%
\pgfsys@transformshift{2.822446in}{0.521603in}%
\pgfsys@useobject{currentmarker}{}%
\end{pgfscope}%
\end{pgfscope}%
\begin{pgfscope}%
\definecolor{textcolor}{rgb}{0.000000,0.000000,0.000000}%
\pgfsetstrokecolor{textcolor}%
\pgfsetfillcolor{textcolor}%
\pgftext[x=2.822446in,y=0.424381in,,top]{\color{textcolor}{\rmfamily\fontsize{10.000000}{12.000000}\selectfont\catcode`\^=\active\def^{\ifmmode\sp\else\^{}\fi}\catcode`\%=\active\def%{\%}$\mathdefault{16}$}}%
\end{pgfscope}%
\begin{pgfscope}%
\pgfsetbuttcap%
\pgfsetroundjoin%
\definecolor{currentfill}{rgb}{0.000000,0.000000,0.000000}%
\pgfsetfillcolor{currentfill}%
\pgfsetlinewidth{0.803000pt}%
\definecolor{currentstroke}{rgb}{0.000000,0.000000,0.000000}%
\pgfsetstrokecolor{currentstroke}%
\pgfsetdash{}{0pt}%
\pgfsys@defobject{currentmarker}{\pgfqpoint{0.000000in}{-0.048611in}}{\pgfqpoint{0.000000in}{0.000000in}}{%
\pgfpathmoveto{\pgfqpoint{0.000000in}{0.000000in}}%
\pgfpathlineto{\pgfqpoint{0.000000in}{-0.048611in}}%
\pgfusepath{stroke,fill}%
}%
\begin{pgfscope}%
\pgfsys@transformshift{3.397124in}{0.521603in}%
\pgfsys@useobject{currentmarker}{}%
\end{pgfscope}%
\end{pgfscope}%
\begin{pgfscope}%
\definecolor{textcolor}{rgb}{0.000000,0.000000,0.000000}%
\pgfsetstrokecolor{textcolor}%
\pgfsetfillcolor{textcolor}%
\pgftext[x=3.397124in,y=0.424381in,,top]{\color{textcolor}{\rmfamily\fontsize{10.000000}{12.000000}\selectfont\catcode`\^=\active\def^{\ifmmode\sp\else\^{}\fi}\catcode`\%=\active\def%{\%}$\mathdefault{20}$}}%
\end{pgfscope}%
\begin{pgfscope}%
\pgfsetbuttcap%
\pgfsetroundjoin%
\definecolor{currentfill}{rgb}{0.000000,0.000000,0.000000}%
\pgfsetfillcolor{currentfill}%
\pgfsetlinewidth{0.803000pt}%
\definecolor{currentstroke}{rgb}{0.000000,0.000000,0.000000}%
\pgfsetstrokecolor{currentstroke}%
\pgfsetdash{}{0pt}%
\pgfsys@defobject{currentmarker}{\pgfqpoint{0.000000in}{-0.048611in}}{\pgfqpoint{0.000000in}{0.000000in}}{%
\pgfpathmoveto{\pgfqpoint{0.000000in}{0.000000in}}%
\pgfpathlineto{\pgfqpoint{0.000000in}{-0.048611in}}%
\pgfusepath{stroke,fill}%
}%
\begin{pgfscope}%
\pgfsys@transformshift{3.971802in}{0.521603in}%
\pgfsys@useobject{currentmarker}{}%
\end{pgfscope}%
\end{pgfscope}%
\begin{pgfscope}%
\definecolor{textcolor}{rgb}{0.000000,0.000000,0.000000}%
\pgfsetstrokecolor{textcolor}%
\pgfsetfillcolor{textcolor}%
\pgftext[x=3.971802in,y=0.424381in,,top]{\color{textcolor}{\rmfamily\fontsize{10.000000}{12.000000}\selectfont\catcode`\^=\active\def^{\ifmmode\sp\else\^{}\fi}\catcode`\%=\active\def%{\%}$\mathdefault{24}$}}%
\end{pgfscope}%
\begin{pgfscope}%
\pgfsetbuttcap%
\pgfsetroundjoin%
\definecolor{currentfill}{rgb}{0.000000,0.000000,0.000000}%
\pgfsetfillcolor{currentfill}%
\pgfsetlinewidth{0.803000pt}%
\definecolor{currentstroke}{rgb}{0.000000,0.000000,0.000000}%
\pgfsetstrokecolor{currentstroke}%
\pgfsetdash{}{0pt}%
\pgfsys@defobject{currentmarker}{\pgfqpoint{0.000000in}{-0.048611in}}{\pgfqpoint{0.000000in}{0.000000in}}{%
\pgfpathmoveto{\pgfqpoint{0.000000in}{0.000000in}}%
\pgfpathlineto{\pgfqpoint{0.000000in}{-0.048611in}}%
\pgfusepath{stroke,fill}%
}%
\begin{pgfscope}%
\pgfsys@transformshift{4.546479in}{0.521603in}%
\pgfsys@useobject{currentmarker}{}%
\end{pgfscope}%
\end{pgfscope}%
\begin{pgfscope}%
\definecolor{textcolor}{rgb}{0.000000,0.000000,0.000000}%
\pgfsetstrokecolor{textcolor}%
\pgfsetfillcolor{textcolor}%
\pgftext[x=4.546479in,y=0.424381in,,top]{\color{textcolor}{\rmfamily\fontsize{10.000000}{12.000000}\selectfont\catcode`\^=\active\def^{\ifmmode\sp\else\^{}\fi}\catcode`\%=\active\def%{\%}$\mathdefault{28}$}}%
\end{pgfscope}%
\begin{pgfscope}%
\pgfsetbuttcap%
\pgfsetroundjoin%
\definecolor{currentfill}{rgb}{0.000000,0.000000,0.000000}%
\pgfsetfillcolor{currentfill}%
\pgfsetlinewidth{0.803000pt}%
\definecolor{currentstroke}{rgb}{0.000000,0.000000,0.000000}%
\pgfsetstrokecolor{currentstroke}%
\pgfsetdash{}{0pt}%
\pgfsys@defobject{currentmarker}{\pgfqpoint{0.000000in}{-0.048611in}}{\pgfqpoint{0.000000in}{0.000000in}}{%
\pgfpathmoveto{\pgfqpoint{0.000000in}{0.000000in}}%
\pgfpathlineto{\pgfqpoint{0.000000in}{-0.048611in}}%
\pgfusepath{stroke,fill}%
}%
\begin{pgfscope}%
\pgfsys@transformshift{5.121157in}{0.521603in}%
\pgfsys@useobject{currentmarker}{}%
\end{pgfscope}%
\end{pgfscope}%
\begin{pgfscope}%
\definecolor{textcolor}{rgb}{0.000000,0.000000,0.000000}%
\pgfsetstrokecolor{textcolor}%
\pgfsetfillcolor{textcolor}%
\pgftext[x=5.121157in,y=0.424381in,,top]{\color{textcolor}{\rmfamily\fontsize{10.000000}{12.000000}\selectfont\catcode`\^=\active\def^{\ifmmode\sp\else\^{}\fi}\catcode`\%=\active\def%{\%}$\mathdefault{32}$}}%
\end{pgfscope}%
\begin{pgfscope}%
\definecolor{textcolor}{rgb}{0.000000,0.000000,0.000000}%
\pgfsetstrokecolor{textcolor}%
\pgfsetfillcolor{textcolor}%
\pgftext[x=3.037950in,y=0.234413in,,top]{\color{textcolor}{\rmfamily\fontsize{10.000000}{12.000000}\selectfont\catcode`\^=\active\def^{\ifmmode\sp\else\^{}\fi}\catcode`\%=\active\def%{\%}Monochromatic classes}}%
\end{pgfscope}%
\begin{pgfscope}%
\pgfsetbuttcap%
\pgfsetroundjoin%
\definecolor{currentfill}{rgb}{0.000000,0.000000,0.000000}%
\pgfsetfillcolor{currentfill}%
\pgfsetlinewidth{0.803000pt}%
\definecolor{currentstroke}{rgb}{0.000000,0.000000,0.000000}%
\pgfsetstrokecolor{currentstroke}%
\pgfsetdash{}{0pt}%
\pgfsys@defobject{currentmarker}{\pgfqpoint{-0.048611in}{0.000000in}}{\pgfqpoint{-0.000000in}{0.000000in}}{%
\pgfpathmoveto{\pgfqpoint{-0.000000in}{0.000000in}}%
\pgfpathlineto{\pgfqpoint{-0.048611in}{0.000000in}}%
\pgfusepath{stroke,fill}%
}%
\begin{pgfscope}%
\pgfsys@transformshift{0.588387in}{0.924341in}%
\pgfsys@useobject{currentmarker}{}%
\end{pgfscope}%
\end{pgfscope}%
\begin{pgfscope}%
\definecolor{textcolor}{rgb}{0.000000,0.000000,0.000000}%
\pgfsetstrokecolor{textcolor}%
\pgfsetfillcolor{textcolor}%
\pgftext[x=0.289968in, y=0.871579in, left, base]{\color{textcolor}{\rmfamily\fontsize{10.000000}{12.000000}\selectfont\catcode`\^=\active\def^{\ifmmode\sp\else\^{}\fi}\catcode`\%=\active\def%{\%}$\mathdefault{10^{1}}$}}%
\end{pgfscope}%
\begin{pgfscope}%
\pgfsetbuttcap%
\pgfsetroundjoin%
\definecolor{currentfill}{rgb}{0.000000,0.000000,0.000000}%
\pgfsetfillcolor{currentfill}%
\pgfsetlinewidth{0.803000pt}%
\definecolor{currentstroke}{rgb}{0.000000,0.000000,0.000000}%
\pgfsetstrokecolor{currentstroke}%
\pgfsetdash{}{0pt}%
\pgfsys@defobject{currentmarker}{\pgfqpoint{-0.048611in}{0.000000in}}{\pgfqpoint{-0.000000in}{0.000000in}}{%
\pgfpathmoveto{\pgfqpoint{-0.000000in}{0.000000in}}%
\pgfpathlineto{\pgfqpoint{-0.048611in}{0.000000in}}%
\pgfusepath{stroke,fill}%
}%
\begin{pgfscope}%
\pgfsys@transformshift{0.588387in}{1.548212in}%
\pgfsys@useobject{currentmarker}{}%
\end{pgfscope}%
\end{pgfscope}%
\begin{pgfscope}%
\definecolor{textcolor}{rgb}{0.000000,0.000000,0.000000}%
\pgfsetstrokecolor{textcolor}%
\pgfsetfillcolor{textcolor}%
\pgftext[x=0.289968in, y=1.495450in, left, base]{\color{textcolor}{\rmfamily\fontsize{10.000000}{12.000000}\selectfont\catcode`\^=\active\def^{\ifmmode\sp\else\^{}\fi}\catcode`\%=\active\def%{\%}$\mathdefault{10^{2}}$}}%
\end{pgfscope}%
\begin{pgfscope}%
\pgfsetbuttcap%
\pgfsetroundjoin%
\definecolor{currentfill}{rgb}{0.000000,0.000000,0.000000}%
\pgfsetfillcolor{currentfill}%
\pgfsetlinewidth{0.803000pt}%
\definecolor{currentstroke}{rgb}{0.000000,0.000000,0.000000}%
\pgfsetstrokecolor{currentstroke}%
\pgfsetdash{}{0pt}%
\pgfsys@defobject{currentmarker}{\pgfqpoint{-0.048611in}{0.000000in}}{\pgfqpoint{-0.000000in}{0.000000in}}{%
\pgfpathmoveto{\pgfqpoint{-0.000000in}{0.000000in}}%
\pgfpathlineto{\pgfqpoint{-0.048611in}{0.000000in}}%
\pgfusepath{stroke,fill}%
}%
\begin{pgfscope}%
\pgfsys@transformshift{0.588387in}{2.172083in}%
\pgfsys@useobject{currentmarker}{}%
\end{pgfscope}%
\end{pgfscope}%
\begin{pgfscope}%
\definecolor{textcolor}{rgb}{0.000000,0.000000,0.000000}%
\pgfsetstrokecolor{textcolor}%
\pgfsetfillcolor{textcolor}%
\pgftext[x=0.289968in, y=2.119322in, left, base]{\color{textcolor}{\rmfamily\fontsize{10.000000}{12.000000}\selectfont\catcode`\^=\active\def^{\ifmmode\sp\else\^{}\fi}\catcode`\%=\active\def%{\%}$\mathdefault{10^{3}}$}}%
\end{pgfscope}%
\begin{pgfscope}%
\pgfsetbuttcap%
\pgfsetroundjoin%
\definecolor{currentfill}{rgb}{0.000000,0.000000,0.000000}%
\pgfsetfillcolor{currentfill}%
\pgfsetlinewidth{0.602250pt}%
\definecolor{currentstroke}{rgb}{0.000000,0.000000,0.000000}%
\pgfsetstrokecolor{currentstroke}%
\pgfsetdash{}{0pt}%
\pgfsys@defobject{currentmarker}{\pgfqpoint{-0.027778in}{0.000000in}}{\pgfqpoint{-0.000000in}{0.000000in}}{%
\pgfpathmoveto{\pgfqpoint{-0.000000in}{0.000000in}}%
\pgfpathlineto{\pgfqpoint{-0.027778in}{0.000000in}}%
\pgfusepath{stroke,fill}%
}%
\begin{pgfscope}%
\pgfsys@transformshift{0.588387in}{0.598132in}%
\pgfsys@useobject{currentmarker}{}%
\end{pgfscope}%
\end{pgfscope}%
\begin{pgfscope}%
\pgfsetbuttcap%
\pgfsetroundjoin%
\definecolor{currentfill}{rgb}{0.000000,0.000000,0.000000}%
\pgfsetfillcolor{currentfill}%
\pgfsetlinewidth{0.602250pt}%
\definecolor{currentstroke}{rgb}{0.000000,0.000000,0.000000}%
\pgfsetstrokecolor{currentstroke}%
\pgfsetdash{}{0pt}%
\pgfsys@defobject{currentmarker}{\pgfqpoint{-0.027778in}{0.000000in}}{\pgfqpoint{-0.000000in}{0.000000in}}{%
\pgfpathmoveto{\pgfqpoint{-0.000000in}{0.000000in}}%
\pgfpathlineto{\pgfqpoint{-0.027778in}{0.000000in}}%
\pgfusepath{stroke,fill}%
}%
\begin{pgfscope}%
\pgfsys@transformshift{0.588387in}{0.676077in}%
\pgfsys@useobject{currentmarker}{}%
\end{pgfscope}%
\end{pgfscope}%
\begin{pgfscope}%
\pgfsetbuttcap%
\pgfsetroundjoin%
\definecolor{currentfill}{rgb}{0.000000,0.000000,0.000000}%
\pgfsetfillcolor{currentfill}%
\pgfsetlinewidth{0.602250pt}%
\definecolor{currentstroke}{rgb}{0.000000,0.000000,0.000000}%
\pgfsetstrokecolor{currentstroke}%
\pgfsetdash{}{0pt}%
\pgfsys@defobject{currentmarker}{\pgfqpoint{-0.027778in}{0.000000in}}{\pgfqpoint{-0.000000in}{0.000000in}}{%
\pgfpathmoveto{\pgfqpoint{-0.000000in}{0.000000in}}%
\pgfpathlineto{\pgfqpoint{-0.027778in}{0.000000in}}%
\pgfusepath{stroke,fill}%
}%
\begin{pgfscope}%
\pgfsys@transformshift{0.588387in}{0.736537in}%
\pgfsys@useobject{currentmarker}{}%
\end{pgfscope}%
\end{pgfscope}%
\begin{pgfscope}%
\pgfsetbuttcap%
\pgfsetroundjoin%
\definecolor{currentfill}{rgb}{0.000000,0.000000,0.000000}%
\pgfsetfillcolor{currentfill}%
\pgfsetlinewidth{0.602250pt}%
\definecolor{currentstroke}{rgb}{0.000000,0.000000,0.000000}%
\pgfsetstrokecolor{currentstroke}%
\pgfsetdash{}{0pt}%
\pgfsys@defobject{currentmarker}{\pgfqpoint{-0.027778in}{0.000000in}}{\pgfqpoint{-0.000000in}{0.000000in}}{%
\pgfpathmoveto{\pgfqpoint{-0.000000in}{0.000000in}}%
\pgfpathlineto{\pgfqpoint{-0.027778in}{0.000000in}}%
\pgfusepath{stroke,fill}%
}%
\begin{pgfscope}%
\pgfsys@transformshift{0.588387in}{0.785936in}%
\pgfsys@useobject{currentmarker}{}%
\end{pgfscope}%
\end{pgfscope}%
\begin{pgfscope}%
\pgfsetbuttcap%
\pgfsetroundjoin%
\definecolor{currentfill}{rgb}{0.000000,0.000000,0.000000}%
\pgfsetfillcolor{currentfill}%
\pgfsetlinewidth{0.602250pt}%
\definecolor{currentstroke}{rgb}{0.000000,0.000000,0.000000}%
\pgfsetstrokecolor{currentstroke}%
\pgfsetdash{}{0pt}%
\pgfsys@defobject{currentmarker}{\pgfqpoint{-0.027778in}{0.000000in}}{\pgfqpoint{-0.000000in}{0.000000in}}{%
\pgfpathmoveto{\pgfqpoint{-0.000000in}{0.000000in}}%
\pgfpathlineto{\pgfqpoint{-0.027778in}{0.000000in}}%
\pgfusepath{stroke,fill}%
}%
\begin{pgfscope}%
\pgfsys@transformshift{0.588387in}{0.827702in}%
\pgfsys@useobject{currentmarker}{}%
\end{pgfscope}%
\end{pgfscope}%
\begin{pgfscope}%
\pgfsetbuttcap%
\pgfsetroundjoin%
\definecolor{currentfill}{rgb}{0.000000,0.000000,0.000000}%
\pgfsetfillcolor{currentfill}%
\pgfsetlinewidth{0.602250pt}%
\definecolor{currentstroke}{rgb}{0.000000,0.000000,0.000000}%
\pgfsetstrokecolor{currentstroke}%
\pgfsetdash{}{0pt}%
\pgfsys@defobject{currentmarker}{\pgfqpoint{-0.027778in}{0.000000in}}{\pgfqpoint{-0.000000in}{0.000000in}}{%
\pgfpathmoveto{\pgfqpoint{-0.000000in}{0.000000in}}%
\pgfpathlineto{\pgfqpoint{-0.027778in}{0.000000in}}%
\pgfusepath{stroke,fill}%
}%
\begin{pgfscope}%
\pgfsys@transformshift{0.588387in}{0.863881in}%
\pgfsys@useobject{currentmarker}{}%
\end{pgfscope}%
\end{pgfscope}%
\begin{pgfscope}%
\pgfsetbuttcap%
\pgfsetroundjoin%
\definecolor{currentfill}{rgb}{0.000000,0.000000,0.000000}%
\pgfsetfillcolor{currentfill}%
\pgfsetlinewidth{0.602250pt}%
\definecolor{currentstroke}{rgb}{0.000000,0.000000,0.000000}%
\pgfsetstrokecolor{currentstroke}%
\pgfsetdash{}{0pt}%
\pgfsys@defobject{currentmarker}{\pgfqpoint{-0.027778in}{0.000000in}}{\pgfqpoint{-0.000000in}{0.000000in}}{%
\pgfpathmoveto{\pgfqpoint{-0.000000in}{0.000000in}}%
\pgfpathlineto{\pgfqpoint{-0.027778in}{0.000000in}}%
\pgfusepath{stroke,fill}%
}%
\begin{pgfscope}%
\pgfsys@transformshift{0.588387in}{0.895794in}%
\pgfsys@useobject{currentmarker}{}%
\end{pgfscope}%
\end{pgfscope}%
\begin{pgfscope}%
\pgfsetbuttcap%
\pgfsetroundjoin%
\definecolor{currentfill}{rgb}{0.000000,0.000000,0.000000}%
\pgfsetfillcolor{currentfill}%
\pgfsetlinewidth{0.602250pt}%
\definecolor{currentstroke}{rgb}{0.000000,0.000000,0.000000}%
\pgfsetstrokecolor{currentstroke}%
\pgfsetdash{}{0pt}%
\pgfsys@defobject{currentmarker}{\pgfqpoint{-0.027778in}{0.000000in}}{\pgfqpoint{-0.000000in}{0.000000in}}{%
\pgfpathmoveto{\pgfqpoint{-0.000000in}{0.000000in}}%
\pgfpathlineto{\pgfqpoint{-0.027778in}{0.000000in}}%
\pgfusepath{stroke,fill}%
}%
\begin{pgfscope}%
\pgfsys@transformshift{0.588387in}{1.112145in}%
\pgfsys@useobject{currentmarker}{}%
\end{pgfscope}%
\end{pgfscope}%
\begin{pgfscope}%
\pgfsetbuttcap%
\pgfsetroundjoin%
\definecolor{currentfill}{rgb}{0.000000,0.000000,0.000000}%
\pgfsetfillcolor{currentfill}%
\pgfsetlinewidth{0.602250pt}%
\definecolor{currentstroke}{rgb}{0.000000,0.000000,0.000000}%
\pgfsetstrokecolor{currentstroke}%
\pgfsetdash{}{0pt}%
\pgfsys@defobject{currentmarker}{\pgfqpoint{-0.027778in}{0.000000in}}{\pgfqpoint{-0.000000in}{0.000000in}}{%
\pgfpathmoveto{\pgfqpoint{-0.000000in}{0.000000in}}%
\pgfpathlineto{\pgfqpoint{-0.027778in}{0.000000in}}%
\pgfusepath{stroke,fill}%
}%
\begin{pgfscope}%
\pgfsys@transformshift{0.588387in}{1.222003in}%
\pgfsys@useobject{currentmarker}{}%
\end{pgfscope}%
\end{pgfscope}%
\begin{pgfscope}%
\pgfsetbuttcap%
\pgfsetroundjoin%
\definecolor{currentfill}{rgb}{0.000000,0.000000,0.000000}%
\pgfsetfillcolor{currentfill}%
\pgfsetlinewidth{0.602250pt}%
\definecolor{currentstroke}{rgb}{0.000000,0.000000,0.000000}%
\pgfsetstrokecolor{currentstroke}%
\pgfsetdash{}{0pt}%
\pgfsys@defobject{currentmarker}{\pgfqpoint{-0.027778in}{0.000000in}}{\pgfqpoint{-0.000000in}{0.000000in}}{%
\pgfpathmoveto{\pgfqpoint{-0.000000in}{0.000000in}}%
\pgfpathlineto{\pgfqpoint{-0.027778in}{0.000000in}}%
\pgfusepath{stroke,fill}%
}%
\begin{pgfscope}%
\pgfsys@transformshift{0.588387in}{1.299949in}%
\pgfsys@useobject{currentmarker}{}%
\end{pgfscope}%
\end{pgfscope}%
\begin{pgfscope}%
\pgfsetbuttcap%
\pgfsetroundjoin%
\definecolor{currentfill}{rgb}{0.000000,0.000000,0.000000}%
\pgfsetfillcolor{currentfill}%
\pgfsetlinewidth{0.602250pt}%
\definecolor{currentstroke}{rgb}{0.000000,0.000000,0.000000}%
\pgfsetstrokecolor{currentstroke}%
\pgfsetdash{}{0pt}%
\pgfsys@defobject{currentmarker}{\pgfqpoint{-0.027778in}{0.000000in}}{\pgfqpoint{-0.000000in}{0.000000in}}{%
\pgfpathmoveto{\pgfqpoint{-0.000000in}{0.000000in}}%
\pgfpathlineto{\pgfqpoint{-0.027778in}{0.000000in}}%
\pgfusepath{stroke,fill}%
}%
\begin{pgfscope}%
\pgfsys@transformshift{0.588387in}{1.360408in}%
\pgfsys@useobject{currentmarker}{}%
\end{pgfscope}%
\end{pgfscope}%
\begin{pgfscope}%
\pgfsetbuttcap%
\pgfsetroundjoin%
\definecolor{currentfill}{rgb}{0.000000,0.000000,0.000000}%
\pgfsetfillcolor{currentfill}%
\pgfsetlinewidth{0.602250pt}%
\definecolor{currentstroke}{rgb}{0.000000,0.000000,0.000000}%
\pgfsetstrokecolor{currentstroke}%
\pgfsetdash{}{0pt}%
\pgfsys@defobject{currentmarker}{\pgfqpoint{-0.027778in}{0.000000in}}{\pgfqpoint{-0.000000in}{0.000000in}}{%
\pgfpathmoveto{\pgfqpoint{-0.000000in}{0.000000in}}%
\pgfpathlineto{\pgfqpoint{-0.027778in}{0.000000in}}%
\pgfusepath{stroke,fill}%
}%
\begin{pgfscope}%
\pgfsys@transformshift{0.588387in}{1.409807in}%
\pgfsys@useobject{currentmarker}{}%
\end{pgfscope}%
\end{pgfscope}%
\begin{pgfscope}%
\pgfsetbuttcap%
\pgfsetroundjoin%
\definecolor{currentfill}{rgb}{0.000000,0.000000,0.000000}%
\pgfsetfillcolor{currentfill}%
\pgfsetlinewidth{0.602250pt}%
\definecolor{currentstroke}{rgb}{0.000000,0.000000,0.000000}%
\pgfsetstrokecolor{currentstroke}%
\pgfsetdash{}{0pt}%
\pgfsys@defobject{currentmarker}{\pgfqpoint{-0.027778in}{0.000000in}}{\pgfqpoint{-0.000000in}{0.000000in}}{%
\pgfpathmoveto{\pgfqpoint{-0.000000in}{0.000000in}}%
\pgfpathlineto{\pgfqpoint{-0.027778in}{0.000000in}}%
\pgfusepath{stroke,fill}%
}%
\begin{pgfscope}%
\pgfsys@transformshift{0.588387in}{1.451573in}%
\pgfsys@useobject{currentmarker}{}%
\end{pgfscope}%
\end{pgfscope}%
\begin{pgfscope}%
\pgfsetbuttcap%
\pgfsetroundjoin%
\definecolor{currentfill}{rgb}{0.000000,0.000000,0.000000}%
\pgfsetfillcolor{currentfill}%
\pgfsetlinewidth{0.602250pt}%
\definecolor{currentstroke}{rgb}{0.000000,0.000000,0.000000}%
\pgfsetstrokecolor{currentstroke}%
\pgfsetdash{}{0pt}%
\pgfsys@defobject{currentmarker}{\pgfqpoint{-0.027778in}{0.000000in}}{\pgfqpoint{-0.000000in}{0.000000in}}{%
\pgfpathmoveto{\pgfqpoint{-0.000000in}{0.000000in}}%
\pgfpathlineto{\pgfqpoint{-0.027778in}{0.000000in}}%
\pgfusepath{stroke,fill}%
}%
\begin{pgfscope}%
\pgfsys@transformshift{0.588387in}{1.487753in}%
\pgfsys@useobject{currentmarker}{}%
\end{pgfscope}%
\end{pgfscope}%
\begin{pgfscope}%
\pgfsetbuttcap%
\pgfsetroundjoin%
\definecolor{currentfill}{rgb}{0.000000,0.000000,0.000000}%
\pgfsetfillcolor{currentfill}%
\pgfsetlinewidth{0.602250pt}%
\definecolor{currentstroke}{rgb}{0.000000,0.000000,0.000000}%
\pgfsetstrokecolor{currentstroke}%
\pgfsetdash{}{0pt}%
\pgfsys@defobject{currentmarker}{\pgfqpoint{-0.027778in}{0.000000in}}{\pgfqpoint{-0.000000in}{0.000000in}}{%
\pgfpathmoveto{\pgfqpoint{-0.000000in}{0.000000in}}%
\pgfpathlineto{\pgfqpoint{-0.027778in}{0.000000in}}%
\pgfusepath{stroke,fill}%
}%
\begin{pgfscope}%
\pgfsys@transformshift{0.588387in}{1.519665in}%
\pgfsys@useobject{currentmarker}{}%
\end{pgfscope}%
\end{pgfscope}%
\begin{pgfscope}%
\pgfsetbuttcap%
\pgfsetroundjoin%
\definecolor{currentfill}{rgb}{0.000000,0.000000,0.000000}%
\pgfsetfillcolor{currentfill}%
\pgfsetlinewidth{0.602250pt}%
\definecolor{currentstroke}{rgb}{0.000000,0.000000,0.000000}%
\pgfsetstrokecolor{currentstroke}%
\pgfsetdash{}{0pt}%
\pgfsys@defobject{currentmarker}{\pgfqpoint{-0.027778in}{0.000000in}}{\pgfqpoint{-0.000000in}{0.000000in}}{%
\pgfpathmoveto{\pgfqpoint{-0.000000in}{0.000000in}}%
\pgfpathlineto{\pgfqpoint{-0.027778in}{0.000000in}}%
\pgfusepath{stroke,fill}%
}%
\begin{pgfscope}%
\pgfsys@transformshift{0.588387in}{1.736016in}%
\pgfsys@useobject{currentmarker}{}%
\end{pgfscope}%
\end{pgfscope}%
\begin{pgfscope}%
\pgfsetbuttcap%
\pgfsetroundjoin%
\definecolor{currentfill}{rgb}{0.000000,0.000000,0.000000}%
\pgfsetfillcolor{currentfill}%
\pgfsetlinewidth{0.602250pt}%
\definecolor{currentstroke}{rgb}{0.000000,0.000000,0.000000}%
\pgfsetstrokecolor{currentstroke}%
\pgfsetdash{}{0pt}%
\pgfsys@defobject{currentmarker}{\pgfqpoint{-0.027778in}{0.000000in}}{\pgfqpoint{-0.000000in}{0.000000in}}{%
\pgfpathmoveto{\pgfqpoint{-0.000000in}{0.000000in}}%
\pgfpathlineto{\pgfqpoint{-0.027778in}{0.000000in}}%
\pgfusepath{stroke,fill}%
}%
\begin{pgfscope}%
\pgfsys@transformshift{0.588387in}{1.845874in}%
\pgfsys@useobject{currentmarker}{}%
\end{pgfscope}%
\end{pgfscope}%
\begin{pgfscope}%
\pgfsetbuttcap%
\pgfsetroundjoin%
\definecolor{currentfill}{rgb}{0.000000,0.000000,0.000000}%
\pgfsetfillcolor{currentfill}%
\pgfsetlinewidth{0.602250pt}%
\definecolor{currentstroke}{rgb}{0.000000,0.000000,0.000000}%
\pgfsetstrokecolor{currentstroke}%
\pgfsetdash{}{0pt}%
\pgfsys@defobject{currentmarker}{\pgfqpoint{-0.027778in}{0.000000in}}{\pgfqpoint{-0.000000in}{0.000000in}}{%
\pgfpathmoveto{\pgfqpoint{-0.000000in}{0.000000in}}%
\pgfpathlineto{\pgfqpoint{-0.027778in}{0.000000in}}%
\pgfusepath{stroke,fill}%
}%
\begin{pgfscope}%
\pgfsys@transformshift{0.588387in}{1.923820in}%
\pgfsys@useobject{currentmarker}{}%
\end{pgfscope}%
\end{pgfscope}%
\begin{pgfscope}%
\pgfsetbuttcap%
\pgfsetroundjoin%
\definecolor{currentfill}{rgb}{0.000000,0.000000,0.000000}%
\pgfsetfillcolor{currentfill}%
\pgfsetlinewidth{0.602250pt}%
\definecolor{currentstroke}{rgb}{0.000000,0.000000,0.000000}%
\pgfsetstrokecolor{currentstroke}%
\pgfsetdash{}{0pt}%
\pgfsys@defobject{currentmarker}{\pgfqpoint{-0.027778in}{0.000000in}}{\pgfqpoint{-0.000000in}{0.000000in}}{%
\pgfpathmoveto{\pgfqpoint{-0.000000in}{0.000000in}}%
\pgfpathlineto{\pgfqpoint{-0.027778in}{0.000000in}}%
\pgfusepath{stroke,fill}%
}%
\begin{pgfscope}%
\pgfsys@transformshift{0.588387in}{1.984279in}%
\pgfsys@useobject{currentmarker}{}%
\end{pgfscope}%
\end{pgfscope}%
\begin{pgfscope}%
\pgfsetbuttcap%
\pgfsetroundjoin%
\definecolor{currentfill}{rgb}{0.000000,0.000000,0.000000}%
\pgfsetfillcolor{currentfill}%
\pgfsetlinewidth{0.602250pt}%
\definecolor{currentstroke}{rgb}{0.000000,0.000000,0.000000}%
\pgfsetstrokecolor{currentstroke}%
\pgfsetdash{}{0pt}%
\pgfsys@defobject{currentmarker}{\pgfqpoint{-0.027778in}{0.000000in}}{\pgfqpoint{-0.000000in}{0.000000in}}{%
\pgfpathmoveto{\pgfqpoint{-0.000000in}{0.000000in}}%
\pgfpathlineto{\pgfqpoint{-0.027778in}{0.000000in}}%
\pgfusepath{stroke,fill}%
}%
\begin{pgfscope}%
\pgfsys@transformshift{0.588387in}{2.033678in}%
\pgfsys@useobject{currentmarker}{}%
\end{pgfscope}%
\end{pgfscope}%
\begin{pgfscope}%
\pgfsetbuttcap%
\pgfsetroundjoin%
\definecolor{currentfill}{rgb}{0.000000,0.000000,0.000000}%
\pgfsetfillcolor{currentfill}%
\pgfsetlinewidth{0.602250pt}%
\definecolor{currentstroke}{rgb}{0.000000,0.000000,0.000000}%
\pgfsetstrokecolor{currentstroke}%
\pgfsetdash{}{0pt}%
\pgfsys@defobject{currentmarker}{\pgfqpoint{-0.027778in}{0.000000in}}{\pgfqpoint{-0.000000in}{0.000000in}}{%
\pgfpathmoveto{\pgfqpoint{-0.000000in}{0.000000in}}%
\pgfpathlineto{\pgfqpoint{-0.027778in}{0.000000in}}%
\pgfusepath{stroke,fill}%
}%
\begin{pgfscope}%
\pgfsys@transformshift{0.588387in}{2.075444in}%
\pgfsys@useobject{currentmarker}{}%
\end{pgfscope}%
\end{pgfscope}%
\begin{pgfscope}%
\pgfsetbuttcap%
\pgfsetroundjoin%
\definecolor{currentfill}{rgb}{0.000000,0.000000,0.000000}%
\pgfsetfillcolor{currentfill}%
\pgfsetlinewidth{0.602250pt}%
\definecolor{currentstroke}{rgb}{0.000000,0.000000,0.000000}%
\pgfsetstrokecolor{currentstroke}%
\pgfsetdash{}{0pt}%
\pgfsys@defobject{currentmarker}{\pgfqpoint{-0.027778in}{0.000000in}}{\pgfqpoint{-0.000000in}{0.000000in}}{%
\pgfpathmoveto{\pgfqpoint{-0.000000in}{0.000000in}}%
\pgfpathlineto{\pgfqpoint{-0.027778in}{0.000000in}}%
\pgfusepath{stroke,fill}%
}%
\begin{pgfscope}%
\pgfsys@transformshift{0.588387in}{2.111624in}%
\pgfsys@useobject{currentmarker}{}%
\end{pgfscope}%
\end{pgfscope}%
\begin{pgfscope}%
\pgfsetbuttcap%
\pgfsetroundjoin%
\definecolor{currentfill}{rgb}{0.000000,0.000000,0.000000}%
\pgfsetfillcolor{currentfill}%
\pgfsetlinewidth{0.602250pt}%
\definecolor{currentstroke}{rgb}{0.000000,0.000000,0.000000}%
\pgfsetstrokecolor{currentstroke}%
\pgfsetdash{}{0pt}%
\pgfsys@defobject{currentmarker}{\pgfqpoint{-0.027778in}{0.000000in}}{\pgfqpoint{-0.000000in}{0.000000in}}{%
\pgfpathmoveto{\pgfqpoint{-0.000000in}{0.000000in}}%
\pgfpathlineto{\pgfqpoint{-0.027778in}{0.000000in}}%
\pgfusepath{stroke,fill}%
}%
\begin{pgfscope}%
\pgfsys@transformshift{0.588387in}{2.143537in}%
\pgfsys@useobject{currentmarker}{}%
\end{pgfscope}%
\end{pgfscope}%
\begin{pgfscope}%
\pgfsetbuttcap%
\pgfsetroundjoin%
\definecolor{currentfill}{rgb}{0.000000,0.000000,0.000000}%
\pgfsetfillcolor{currentfill}%
\pgfsetlinewidth{0.602250pt}%
\definecolor{currentstroke}{rgb}{0.000000,0.000000,0.000000}%
\pgfsetstrokecolor{currentstroke}%
\pgfsetdash{}{0pt}%
\pgfsys@defobject{currentmarker}{\pgfqpoint{-0.027778in}{0.000000in}}{\pgfqpoint{-0.000000in}{0.000000in}}{%
\pgfpathmoveto{\pgfqpoint{-0.000000in}{0.000000in}}%
\pgfpathlineto{\pgfqpoint{-0.027778in}{0.000000in}}%
\pgfusepath{stroke,fill}%
}%
\begin{pgfscope}%
\pgfsys@transformshift{0.588387in}{2.359887in}%
\pgfsys@useobject{currentmarker}{}%
\end{pgfscope}%
\end{pgfscope}%
\begin{pgfscope}%
\pgfsetbuttcap%
\pgfsetroundjoin%
\definecolor{currentfill}{rgb}{0.000000,0.000000,0.000000}%
\pgfsetfillcolor{currentfill}%
\pgfsetlinewidth{0.602250pt}%
\definecolor{currentstroke}{rgb}{0.000000,0.000000,0.000000}%
\pgfsetstrokecolor{currentstroke}%
\pgfsetdash{}{0pt}%
\pgfsys@defobject{currentmarker}{\pgfqpoint{-0.027778in}{0.000000in}}{\pgfqpoint{-0.000000in}{0.000000in}}{%
\pgfpathmoveto{\pgfqpoint{-0.000000in}{0.000000in}}%
\pgfpathlineto{\pgfqpoint{-0.027778in}{0.000000in}}%
\pgfusepath{stroke,fill}%
}%
\begin{pgfscope}%
\pgfsys@transformshift{0.588387in}{2.469746in}%
\pgfsys@useobject{currentmarker}{}%
\end{pgfscope}%
\end{pgfscope}%
\begin{pgfscope}%
\pgfsetbuttcap%
\pgfsetroundjoin%
\definecolor{currentfill}{rgb}{0.000000,0.000000,0.000000}%
\pgfsetfillcolor{currentfill}%
\pgfsetlinewidth{0.602250pt}%
\definecolor{currentstroke}{rgb}{0.000000,0.000000,0.000000}%
\pgfsetstrokecolor{currentstroke}%
\pgfsetdash{}{0pt}%
\pgfsys@defobject{currentmarker}{\pgfqpoint{-0.027778in}{0.000000in}}{\pgfqpoint{-0.000000in}{0.000000in}}{%
\pgfpathmoveto{\pgfqpoint{-0.000000in}{0.000000in}}%
\pgfpathlineto{\pgfqpoint{-0.027778in}{0.000000in}}%
\pgfusepath{stroke,fill}%
}%
\begin{pgfscope}%
\pgfsys@transformshift{0.588387in}{2.547691in}%
\pgfsys@useobject{currentmarker}{}%
\end{pgfscope}%
\end{pgfscope}%
\begin{pgfscope}%
\pgfsetbuttcap%
\pgfsetroundjoin%
\definecolor{currentfill}{rgb}{0.000000,0.000000,0.000000}%
\pgfsetfillcolor{currentfill}%
\pgfsetlinewidth{0.602250pt}%
\definecolor{currentstroke}{rgb}{0.000000,0.000000,0.000000}%
\pgfsetstrokecolor{currentstroke}%
\pgfsetdash{}{0pt}%
\pgfsys@defobject{currentmarker}{\pgfqpoint{-0.027778in}{0.000000in}}{\pgfqpoint{-0.000000in}{0.000000in}}{%
\pgfpathmoveto{\pgfqpoint{-0.000000in}{0.000000in}}%
\pgfpathlineto{\pgfqpoint{-0.027778in}{0.000000in}}%
\pgfusepath{stroke,fill}%
}%
\begin{pgfscope}%
\pgfsys@transformshift{0.588387in}{2.608151in}%
\pgfsys@useobject{currentmarker}{}%
\end{pgfscope}%
\end{pgfscope}%
\begin{pgfscope}%
\pgfsetbuttcap%
\pgfsetroundjoin%
\definecolor{currentfill}{rgb}{0.000000,0.000000,0.000000}%
\pgfsetfillcolor{currentfill}%
\pgfsetlinewidth{0.602250pt}%
\definecolor{currentstroke}{rgb}{0.000000,0.000000,0.000000}%
\pgfsetstrokecolor{currentstroke}%
\pgfsetdash{}{0pt}%
\pgfsys@defobject{currentmarker}{\pgfqpoint{-0.027778in}{0.000000in}}{\pgfqpoint{-0.000000in}{0.000000in}}{%
\pgfpathmoveto{\pgfqpoint{-0.000000in}{0.000000in}}%
\pgfpathlineto{\pgfqpoint{-0.027778in}{0.000000in}}%
\pgfusepath{stroke,fill}%
}%
\begin{pgfscope}%
\pgfsys@transformshift{0.588387in}{2.657550in}%
\pgfsys@useobject{currentmarker}{}%
\end{pgfscope}%
\end{pgfscope}%
\begin{pgfscope}%
\pgfsetbuttcap%
\pgfsetroundjoin%
\definecolor{currentfill}{rgb}{0.000000,0.000000,0.000000}%
\pgfsetfillcolor{currentfill}%
\pgfsetlinewidth{0.602250pt}%
\definecolor{currentstroke}{rgb}{0.000000,0.000000,0.000000}%
\pgfsetstrokecolor{currentstroke}%
\pgfsetdash{}{0pt}%
\pgfsys@defobject{currentmarker}{\pgfqpoint{-0.027778in}{0.000000in}}{\pgfqpoint{-0.000000in}{0.000000in}}{%
\pgfpathmoveto{\pgfqpoint{-0.000000in}{0.000000in}}%
\pgfpathlineto{\pgfqpoint{-0.027778in}{0.000000in}}%
\pgfusepath{stroke,fill}%
}%
\begin{pgfscope}%
\pgfsys@transformshift{0.588387in}{2.699316in}%
\pgfsys@useobject{currentmarker}{}%
\end{pgfscope}%
\end{pgfscope}%
\begin{pgfscope}%
\pgfsetbuttcap%
\pgfsetroundjoin%
\definecolor{currentfill}{rgb}{0.000000,0.000000,0.000000}%
\pgfsetfillcolor{currentfill}%
\pgfsetlinewidth{0.602250pt}%
\definecolor{currentstroke}{rgb}{0.000000,0.000000,0.000000}%
\pgfsetstrokecolor{currentstroke}%
\pgfsetdash{}{0pt}%
\pgfsys@defobject{currentmarker}{\pgfqpoint{-0.027778in}{0.000000in}}{\pgfqpoint{-0.000000in}{0.000000in}}{%
\pgfpathmoveto{\pgfqpoint{-0.000000in}{0.000000in}}%
\pgfpathlineto{\pgfqpoint{-0.027778in}{0.000000in}}%
\pgfusepath{stroke,fill}%
}%
\begin{pgfscope}%
\pgfsys@transformshift{0.588387in}{2.735495in}%
\pgfsys@useobject{currentmarker}{}%
\end{pgfscope}%
\end{pgfscope}%
\begin{pgfscope}%
\definecolor{textcolor}{rgb}{0.000000,0.000000,0.000000}%
\pgfsetstrokecolor{textcolor}%
\pgfsetfillcolor{textcolor}%
\pgftext[x=0.234413in,y=1.631726in,,bottom,rotate=90.000000]{\color{textcolor}{\rmfamily\fontsize{10.000000}{12.000000}\selectfont\catcode`\^=\active\def^{\ifmmode\sp\else\^{}\fi}\catcode`\%=\active\def%{\%}Time [ms]}}%
\end{pgfscope}%
\begin{pgfscope}%
\pgfpathrectangle{\pgfqpoint{0.588387in}{0.521603in}}{\pgfqpoint{4.899126in}{2.220246in}}%
\pgfusepath{clip}%
\pgfsetrectcap%
\pgfsetroundjoin%
\pgfsetlinewidth{1.505625pt}%
\pgfsetstrokecolor{currentstroke1}%
\pgfsetdash{}{0pt}%
\pgfpathmoveto{\pgfqpoint{0.811075in}{0.690726in}}%
\pgfpathlineto{\pgfqpoint{0.954744in}{0.714850in}}%
\pgfpathlineto{\pgfqpoint{1.098414in}{0.730772in}}%
\pgfpathlineto{\pgfqpoint{1.242083in}{0.676077in}}%
\pgfpathlineto{\pgfqpoint{1.385752in}{0.692503in}}%
\pgfpathlineto{\pgfqpoint{1.529422in}{0.639898in}}%
\pgfpathlineto{\pgfqpoint{1.673091in}{0.627220in}}%
\pgfpathlineto{\pgfqpoint{1.816761in}{0.714946in}}%
\pgfpathlineto{\pgfqpoint{1.960430in}{0.755063in}}%
\pgfpathlineto{\pgfqpoint{2.104099in}{0.821281in}}%
\pgfpathlineto{\pgfqpoint{2.247769in}{0.944635in}}%
\pgfpathlineto{\pgfqpoint{2.391438in}{1.031572in}}%
\pgfpathlineto{\pgfqpoint{2.535108in}{1.198937in}}%
\pgfpathlineto{\pgfqpoint{2.678777in}{1.279435in}}%
\pgfpathlineto{\pgfqpoint{2.822446in}{1.502609in}}%
\pgfpathlineto{\pgfqpoint{2.966116in}{1.574900in}}%
\pgfpathlineto{\pgfqpoint{3.109785in}{1.746184in}}%
\pgfpathlineto{\pgfqpoint{3.253455in}{1.868961in}}%
\pgfpathlineto{\pgfqpoint{3.397124in}{2.073765in}}%
\pgfpathlineto{\pgfqpoint{3.540793in}{2.215391in}}%
\pgfpathlineto{\pgfqpoint{3.684463in}{2.408094in}}%
\pgfpathlineto{\pgfqpoint{3.828132in}{2.556582in}}%
\pgfusepath{stroke}%
\end{pgfscope}%
\begin{pgfscope}%
\pgfpathrectangle{\pgfqpoint{0.588387in}{0.521603in}}{\pgfqpoint{4.899126in}{2.220246in}}%
\pgfusepath{clip}%
\pgfsetrectcap%
\pgfsetroundjoin%
\pgfsetlinewidth{1.505625pt}%
\pgfsetstrokecolor{currentstroke2}%
\pgfsetdash{}{0pt}%
\pgfpathmoveto{\pgfqpoint{0.811075in}{0.704624in}}%
\pgfpathlineto{\pgfqpoint{0.954744in}{0.719328in}}%
\pgfpathlineto{\pgfqpoint{1.098414in}{0.720594in}}%
\pgfpathlineto{\pgfqpoint{1.242083in}{0.645372in}}%
\pgfpathlineto{\pgfqpoint{1.385752in}{0.622524in}}%
\pgfpathlineto{\pgfqpoint{1.529422in}{0.647172in}}%
\pgfpathlineto{\pgfqpoint{1.673091in}{0.771672in}}%
\pgfpathlineto{\pgfqpoint{1.816761in}{0.827066in}}%
\pgfpathlineto{\pgfqpoint{1.960430in}{0.900386in}}%
\pgfpathlineto{\pgfqpoint{2.104099in}{0.940714in}}%
\pgfpathlineto{\pgfqpoint{2.247769in}{1.066928in}}%
\pgfpathlineto{\pgfqpoint{2.391438in}{1.141190in}}%
\pgfpathlineto{\pgfqpoint{2.535108in}{1.253569in}}%
\pgfpathlineto{\pgfqpoint{2.678777in}{1.291541in}}%
\pgfpathlineto{\pgfqpoint{2.822446in}{1.500679in}}%
\pgfpathlineto{\pgfqpoint{2.966116in}{1.488940in}}%
\pgfpathlineto{\pgfqpoint{3.109785in}{1.615763in}}%
\pgfpathlineto{\pgfqpoint{3.253455in}{1.622641in}}%
\pgfpathlineto{\pgfqpoint{3.397124in}{1.879442in}}%
\pgfpathlineto{\pgfqpoint{3.540793in}{1.878760in}}%
\pgfpathlineto{\pgfqpoint{3.684463in}{2.030727in}}%
\pgfpathlineto{\pgfqpoint{3.828132in}{1.994660in}}%
\pgfpathlineto{\pgfqpoint{3.971802in}{2.236454in}}%
\pgfpathlineto{\pgfqpoint{4.115471in}{2.216924in}}%
\pgfpathlineto{\pgfqpoint{4.259140in}{2.314960in}}%
\pgfpathlineto{\pgfqpoint{4.402810in}{2.422334in}}%
\pgfpathlineto{\pgfqpoint{4.546479in}{2.375334in}}%
\pgfpathlineto{\pgfqpoint{4.690149in}{2.496143in}}%
\pgfpathlineto{\pgfqpoint{4.977487in}{2.620181in}}%
\pgfusepath{stroke}%
\end{pgfscope}%
\begin{pgfscope}%
\pgfpathrectangle{\pgfqpoint{0.588387in}{0.521603in}}{\pgfqpoint{4.899126in}{2.220246in}}%
\pgfusepath{clip}%
\pgfsetrectcap%
\pgfsetroundjoin%
\pgfsetlinewidth{1.505625pt}%
\pgfsetstrokecolor{currentstroke3}%
\pgfsetdash{}{0pt}%
\pgfpathmoveto{\pgfqpoint{0.811075in}{0.697764in}}%
\pgfpathlineto{\pgfqpoint{0.954744in}{0.723734in}}%
\pgfpathlineto{\pgfqpoint{1.098414in}{0.720594in}}%
\pgfpathlineto{\pgfqpoint{1.242083in}{0.651049in}}%
\pgfpathlineto{\pgfqpoint{1.385752in}{0.626678in}}%
\pgfpathlineto{\pgfqpoint{1.529422in}{0.652502in}}%
\pgfpathlineto{\pgfqpoint{1.673091in}{0.773133in}}%
\pgfpathlineto{\pgfqpoint{1.816761in}{0.829599in}}%
\pgfpathlineto{\pgfqpoint{1.960430in}{0.899474in}}%
\pgfpathlineto{\pgfqpoint{2.104099in}{0.939877in}}%
\pgfpathlineto{\pgfqpoint{2.247769in}{1.085265in}}%
\pgfpathlineto{\pgfqpoint{2.391438in}{1.139745in}}%
\pgfpathlineto{\pgfqpoint{2.535108in}{1.252458in}}%
\pgfpathlineto{\pgfqpoint{2.678777in}{1.287237in}}%
\pgfpathlineto{\pgfqpoint{2.822446in}{1.489165in}}%
\pgfpathlineto{\pgfqpoint{2.966116in}{1.473662in}}%
\pgfpathlineto{\pgfqpoint{3.109785in}{1.611308in}}%
\pgfpathlineto{\pgfqpoint{3.253455in}{1.601649in}}%
\pgfpathlineto{\pgfqpoint{3.397124in}{1.807805in}}%
\pgfpathlineto{\pgfqpoint{3.540793in}{1.835485in}}%
\pgfpathlineto{\pgfqpoint{3.684463in}{1.991356in}}%
\pgfpathlineto{\pgfqpoint{3.828132in}{1.941114in}}%
\pgfpathlineto{\pgfqpoint{3.971802in}{2.209076in}}%
\pgfpathlineto{\pgfqpoint{4.115471in}{2.127180in}}%
\pgfpathlineto{\pgfqpoint{4.259140in}{2.251178in}}%
\pgfpathlineto{\pgfqpoint{4.402810in}{2.325477in}}%
\pgfpathlineto{\pgfqpoint{4.546479in}{2.411982in}}%
\pgfpathlineto{\pgfqpoint{4.690149in}{2.394642in}}%
\pgfpathlineto{\pgfqpoint{4.977487in}{2.629270in}}%
\pgfusepath{stroke}%
\end{pgfscope}%
\begin{pgfscope}%
\pgfpathrectangle{\pgfqpoint{0.588387in}{0.521603in}}{\pgfqpoint{4.899126in}{2.220246in}}%
\pgfusepath{clip}%
\pgfsetrectcap%
\pgfsetroundjoin%
\pgfsetlinewidth{1.505625pt}%
\pgfsetstrokecolor{currentstroke4}%
\pgfsetdash{}{0pt}%
\pgfpathmoveto{\pgfqpoint{0.811075in}{0.711314in}}%
\pgfpathlineto{\pgfqpoint{0.954744in}{0.728069in}}%
\pgfpathlineto{\pgfqpoint{1.098414in}{0.723317in}}%
\pgfpathlineto{\pgfqpoint{1.242083in}{0.649672in}}%
\pgfpathlineto{\pgfqpoint{1.385752in}{0.629608in}}%
\pgfpathlineto{\pgfqpoint{1.529422in}{0.645372in}}%
\pgfpathlineto{\pgfqpoint{1.673091in}{0.770204in}}%
\pgfpathlineto{\pgfqpoint{1.816761in}{0.833353in}}%
\pgfpathlineto{\pgfqpoint{1.960430in}{0.901295in}}%
\pgfpathlineto{\pgfqpoint{2.104099in}{0.945686in}}%
\pgfpathlineto{\pgfqpoint{2.247769in}{1.079384in}}%
\pgfpathlineto{\pgfqpoint{2.391438in}{1.151558in}}%
\pgfpathlineto{\pgfqpoint{2.535108in}{1.271661in}}%
\pgfpathlineto{\pgfqpoint{2.678777in}{1.301917in}}%
\pgfpathlineto{\pgfqpoint{2.822446in}{1.508434in}}%
\pgfpathlineto{\pgfqpoint{2.966116in}{1.517459in}}%
\pgfpathlineto{\pgfqpoint{3.109785in}{1.652323in}}%
\pgfpathlineto{\pgfqpoint{3.253455in}{1.663786in}}%
\pgfpathlineto{\pgfqpoint{3.397124in}{1.929239in}}%
\pgfpathlineto{\pgfqpoint{3.540793in}{1.921408in}}%
\pgfpathlineto{\pgfqpoint{3.684463in}{2.079038in}}%
\pgfpathlineto{\pgfqpoint{3.828132in}{2.054013in}}%
\pgfpathlineto{\pgfqpoint{3.971802in}{2.267630in}}%
\pgfpathlineto{\pgfqpoint{4.115471in}{2.264386in}}%
\pgfpathlineto{\pgfqpoint{4.259140in}{2.451023in}}%
\pgfpathlineto{\pgfqpoint{4.402810in}{2.423274in}}%
\pgfpathlineto{\pgfqpoint{4.546479in}{2.535050in}}%
\pgfpathlineto{\pgfqpoint{4.690149in}{2.517597in}}%
\pgfpathlineto{\pgfqpoint{4.977487in}{2.401382in}}%
\pgfusepath{stroke}%
\end{pgfscope}%
\begin{pgfscope}%
\pgfpathrectangle{\pgfqpoint{0.588387in}{0.521603in}}{\pgfqpoint{4.899126in}{2.220246in}}%
\pgfusepath{clip}%
\pgfsetrectcap%
\pgfsetroundjoin%
\pgfsetlinewidth{1.505625pt}%
\pgfsetstrokecolor{currentstroke5}%
\pgfsetdash{}{0pt}%
\pgfpathmoveto{\pgfqpoint{0.811075in}{0.697764in}}%
\pgfpathlineto{\pgfqpoint{0.954744in}{0.728069in}}%
\pgfpathlineto{\pgfqpoint{1.098414in}{0.731063in}}%
\pgfpathlineto{\pgfqpoint{1.242083in}{0.655384in}}%
\pgfpathlineto{\pgfqpoint{1.385752in}{0.624312in}}%
\pgfpathlineto{\pgfqpoint{1.529422in}{0.643559in}}%
\pgfpathlineto{\pgfqpoint{1.673091in}{0.769466in}}%
\pgfpathlineto{\pgfqpoint{1.816761in}{0.832107in}}%
\pgfpathlineto{\pgfqpoint{1.960430in}{0.896257in}}%
\pgfpathlineto{\pgfqpoint{2.104099in}{0.947731in}}%
\pgfpathlineto{\pgfqpoint{2.247769in}{1.085819in}}%
\pgfpathlineto{\pgfqpoint{2.391438in}{1.144532in}}%
\pgfpathlineto{\pgfqpoint{2.535108in}{1.256464in}}%
\pgfpathlineto{\pgfqpoint{2.678777in}{1.296350in}}%
\pgfpathlineto{\pgfqpoint{2.822446in}{1.490631in}}%
\pgfpathlineto{\pgfqpoint{2.966116in}{1.494532in}}%
\pgfpathlineto{\pgfqpoint{3.109785in}{1.621221in}}%
\pgfpathlineto{\pgfqpoint{3.253455in}{1.619772in}}%
\pgfpathlineto{\pgfqpoint{3.397124in}{1.815395in}}%
\pgfpathlineto{\pgfqpoint{3.540793in}{1.829509in}}%
\pgfpathlineto{\pgfqpoint{3.684463in}{2.035358in}}%
\pgfpathlineto{\pgfqpoint{3.828132in}{1.988541in}}%
\pgfpathlineto{\pgfqpoint{3.971802in}{2.225202in}}%
\pgfpathlineto{\pgfqpoint{4.115471in}{2.167549in}}%
\pgfpathlineto{\pgfqpoint{4.259140in}{2.329195in}}%
\pgfpathlineto{\pgfqpoint{4.402810in}{2.413779in}}%
\pgfpathlineto{\pgfqpoint{4.546479in}{2.421860in}}%
\pgfpathlineto{\pgfqpoint{4.690149in}{2.459194in}}%
\pgfpathlineto{\pgfqpoint{4.977487in}{2.452644in}}%
\pgfpathlineto{\pgfqpoint{5.264826in}{2.640929in}}%
\pgfusepath{stroke}%
\end{pgfscope}%
\begin{pgfscope}%
\pgfpathrectangle{\pgfqpoint{0.588387in}{0.521603in}}{\pgfqpoint{4.899126in}{2.220246in}}%
\pgfusepath{clip}%
\pgfsetrectcap%
\pgfsetroundjoin%
\pgfsetlinewidth{1.505625pt}%
\pgfsetstrokecolor{currentstroke6}%
\pgfsetdash{}{0pt}%
\pgfpathmoveto{\pgfqpoint{0.811075in}{0.697764in}}%
\pgfpathlineto{\pgfqpoint{0.954744in}{0.732336in}}%
\pgfpathlineto{\pgfqpoint{1.098414in}{0.744545in}}%
\pgfpathlineto{\pgfqpoint{1.242083in}{0.678122in}}%
\pgfpathlineto{\pgfqpoint{1.385752in}{0.635792in}}%
\pgfpathlineto{\pgfqpoint{1.529422in}{0.645642in}}%
\pgfpathlineto{\pgfqpoint{1.673091in}{0.778184in}}%
\pgfpathlineto{\pgfqpoint{1.816761in}{0.844314in}}%
\pgfpathlineto{\pgfqpoint{1.960430in}{0.908461in}}%
\pgfpathlineto{\pgfqpoint{2.104099in}{0.960465in}}%
\pgfpathlineto{\pgfqpoint{2.247769in}{1.122723in}}%
\pgfpathlineto{\pgfqpoint{2.391438in}{1.188929in}}%
\pgfpathlineto{\pgfqpoint{2.535108in}{1.317909in}}%
\pgfpathlineto{\pgfqpoint{2.678777in}{1.349269in}}%
\pgfpathlineto{\pgfqpoint{2.822446in}{1.574170in}}%
\pgfpathlineto{\pgfqpoint{2.966116in}{1.610743in}}%
\pgfpathlineto{\pgfqpoint{3.109785in}{1.753173in}}%
\pgfpathlineto{\pgfqpoint{3.253455in}{1.788703in}}%
\pgfpathlineto{\pgfqpoint{3.397124in}{1.990433in}}%
\pgfpathlineto{\pgfqpoint{3.540793in}{2.013896in}}%
\pgfpathlineto{\pgfqpoint{3.684463in}{2.196520in}}%
\pgfpathlineto{\pgfqpoint{3.828132in}{2.199252in}}%
\pgfpathlineto{\pgfqpoint{3.971802in}{2.451939in}}%
\pgfpathlineto{\pgfqpoint{4.115471in}{2.400124in}}%
\pgfpathlineto{\pgfqpoint{4.259140in}{2.493325in}}%
\pgfpathlineto{\pgfqpoint{4.402810in}{2.441910in}}%
\pgfpathlineto{\pgfqpoint{4.690149in}{2.485760in}}%
\pgfusepath{stroke}%
\end{pgfscope}%
\begin{pgfscope}%
\pgfpathrectangle{\pgfqpoint{0.588387in}{0.521603in}}{\pgfqpoint{4.899126in}{2.220246in}}%
\pgfusepath{clip}%
\pgfsetrectcap%
\pgfsetroundjoin%
\pgfsetlinewidth{1.505625pt}%
\pgfsetstrokecolor{currentstroke7}%
\pgfsetdash{}{0pt}%
\pgfpathmoveto{\pgfqpoint{0.811075in}{0.697764in}}%
\pgfpathlineto{\pgfqpoint{0.954744in}{0.728069in}}%
\pgfpathlineto{\pgfqpoint{1.098414in}{0.744960in}}%
\pgfpathlineto{\pgfqpoint{1.242083in}{0.662180in}}%
\pgfpathlineto{\pgfqpoint{1.385752in}{0.633207in}}%
\pgfpathlineto{\pgfqpoint{1.529422in}{0.663303in}}%
\pgfpathlineto{\pgfqpoint{1.673091in}{0.769466in}}%
\pgfpathlineto{\pgfqpoint{1.816761in}{0.837669in}}%
\pgfpathlineto{\pgfqpoint{1.960430in}{0.903105in}}%
\pgfpathlineto{\pgfqpoint{2.104099in}{0.957730in}}%
\pgfpathlineto{\pgfqpoint{2.247769in}{1.096126in}}%
\pgfpathlineto{\pgfqpoint{2.391438in}{1.164498in}}%
\pgfpathlineto{\pgfqpoint{2.535108in}{1.282711in}}%
\pgfpathlineto{\pgfqpoint{2.678777in}{1.332279in}}%
\pgfpathlineto{\pgfqpoint{2.822446in}{1.531500in}}%
\pgfpathlineto{\pgfqpoint{2.966116in}{1.563169in}}%
\pgfpathlineto{\pgfqpoint{3.109785in}{1.704780in}}%
\pgfpathlineto{\pgfqpoint{3.253455in}{1.731198in}}%
\pgfpathlineto{\pgfqpoint{3.397124in}{1.913635in}}%
\pgfpathlineto{\pgfqpoint{3.540793in}{1.919796in}}%
\pgfpathlineto{\pgfqpoint{3.684463in}{2.120315in}}%
\pgfpathlineto{\pgfqpoint{3.828132in}{2.119489in}}%
\pgfpathlineto{\pgfqpoint{3.971802in}{2.327937in}}%
\pgfpathlineto{\pgfqpoint{4.115471in}{2.322175in}}%
\pgfpathlineto{\pgfqpoint{4.259140in}{2.479285in}}%
\pgfpathlineto{\pgfqpoint{4.402810in}{2.427057in}}%
\pgfpathlineto{\pgfqpoint{4.546479in}{2.548199in}}%
\pgfpathlineto{\pgfqpoint{4.690149in}{2.527431in}}%
\pgfusepath{stroke}%
\end{pgfscope}%
\begin{pgfscope}%
\pgfsetrectcap%
\pgfsetmiterjoin%
\pgfsetlinewidth{0.803000pt}%
\definecolor{currentstroke}{rgb}{0.000000,0.000000,0.000000}%
\pgfsetstrokecolor{currentstroke}%
\pgfsetdash{}{0pt}%
\pgfpathmoveto{\pgfqpoint{0.588387in}{0.521603in}}%
\pgfpathlineto{\pgfqpoint{0.588387in}{2.741849in}}%
\pgfusepath{stroke}%
\end{pgfscope}%
\begin{pgfscope}%
\pgfsetrectcap%
\pgfsetmiterjoin%
\pgfsetlinewidth{0.803000pt}%
\definecolor{currentstroke}{rgb}{0.000000,0.000000,0.000000}%
\pgfsetstrokecolor{currentstroke}%
\pgfsetdash{}{0pt}%
\pgfpathmoveto{\pgfqpoint{5.487514in}{0.521603in}}%
\pgfpathlineto{\pgfqpoint{5.487514in}{2.741849in}}%
\pgfusepath{stroke}%
\end{pgfscope}%
\begin{pgfscope}%
\pgfsetrectcap%
\pgfsetmiterjoin%
\pgfsetlinewidth{0.803000pt}%
\definecolor{currentstroke}{rgb}{0.000000,0.000000,0.000000}%
\pgfsetstrokecolor{currentstroke}%
\pgfsetdash{}{0pt}%
\pgfpathmoveto{\pgfqpoint{0.588387in}{0.521603in}}%
\pgfpathlineto{\pgfqpoint{5.487514in}{0.521603in}}%
\pgfusepath{stroke}%
\end{pgfscope}%
\begin{pgfscope}%
\pgfsetrectcap%
\pgfsetmiterjoin%
\pgfsetlinewidth{0.803000pt}%
\definecolor{currentstroke}{rgb}{0.000000,0.000000,0.000000}%
\pgfsetstrokecolor{currentstroke}%
\pgfsetdash{}{0pt}%
\pgfpathmoveto{\pgfqpoint{0.588387in}{2.741849in}}%
\pgfpathlineto{\pgfqpoint{5.487514in}{2.741849in}}%
\pgfusepath{stroke}%
\end{pgfscope}%
\begin{pgfscope}%
\pgfsetbuttcap%
\pgfsetmiterjoin%
\definecolor{currentfill}{rgb}{1.000000,1.000000,1.000000}%
\pgfsetfillcolor{currentfill}%
\pgfsetfillopacity{0.800000}%
\pgfsetlinewidth{1.003750pt}%
\definecolor{currentstroke}{rgb}{0.800000,0.800000,0.800000}%
\pgfsetstrokecolor{currentstroke}%
\pgfsetstrokeopacity{0.800000}%
\pgfsetdash{}{0pt}%
\pgfpathmoveto{\pgfqpoint{5.575014in}{1.343633in}}%
\pgfpathlineto{\pgfqpoint{8.259376in}{1.343633in}}%
\pgfpathquadraticcurveto{\pgfqpoint{8.284376in}{1.343633in}}{\pgfqpoint{8.284376in}{1.368633in}}%
\pgfpathlineto{\pgfqpoint{8.284376in}{2.654349in}}%
\pgfpathquadraticcurveto{\pgfqpoint{8.284376in}{2.679349in}}{\pgfqpoint{8.259376in}{2.679349in}}%
\pgfpathlineto{\pgfqpoint{5.575014in}{2.679349in}}%
\pgfpathquadraticcurveto{\pgfqpoint{5.550014in}{2.679349in}}{\pgfqpoint{5.550014in}{2.654349in}}%
\pgfpathlineto{\pgfqpoint{5.550014in}{1.368633in}}%
\pgfpathquadraticcurveto{\pgfqpoint{5.550014in}{1.343633in}}{\pgfqpoint{5.575014in}{1.343633in}}%
\pgfpathlineto{\pgfqpoint{5.575014in}{1.343633in}}%
\pgfpathclose%
\pgfusepath{stroke,fill}%
\end{pgfscope}%
\begin{pgfscope}%
\pgfsetrectcap%
\pgfsetroundjoin%
\pgfsetlinewidth{1.505625pt}%
\pgfsetstrokecolor{currentstroke1}%
\pgfsetdash{}{0pt}%
\pgfpathmoveto{\pgfqpoint{5.600014in}{2.578129in}}%
\pgfpathlineto{\pgfqpoint{5.725014in}{2.578129in}}%
\pgfpathlineto{\pgfqpoint{5.850014in}{2.578129in}}%
\pgfusepath{stroke}%
\end{pgfscope}%
\begin{pgfscope}%
\definecolor{textcolor}{rgb}{0.000000,0.000000,0.000000}%
\pgfsetstrokecolor{textcolor}%
\pgfsetfillcolor{textcolor}%
\pgftext[x=5.950014in,y=2.534379in,left,base]{\color{textcolor}{\rmfamily\fontsize{9.000000}{10.800000}\selectfont\catcode`\^=\active\def^{\ifmmode\sp\else\^{}\fi}\catcode`\%=\active\def%{\%}\NaiveCycles{}}}%
\end{pgfscope}%
\begin{pgfscope}%
\pgfsetrectcap%
\pgfsetroundjoin%
\pgfsetlinewidth{1.505625pt}%
\pgfsetstrokecolor{currentstroke2}%
\pgfsetdash{}{0pt}%
\pgfpathmoveto{\pgfqpoint{5.600014in}{2.394657in}}%
\pgfpathlineto{\pgfqpoint{5.725014in}{2.394657in}}%
\pgfpathlineto{\pgfqpoint{5.850014in}{2.394657in}}%
\pgfusepath{stroke}%
\end{pgfscope}%
\begin{pgfscope}%
\definecolor{textcolor}{rgb}{0.000000,0.000000,0.000000}%
\pgfsetstrokecolor{textcolor}%
\pgfsetfillcolor{textcolor}%
\pgftext[x=5.950014in,y=2.350907in,left,base]{\color{textcolor}{\rmfamily\fontsize{9.000000}{10.800000}\selectfont\catcode`\^=\active\def^{\ifmmode\sp\else\^{}\fi}\catcode`\%=\active\def%{\%}\Neighbors{} \& \MergeLinear{}}}%
\end{pgfscope}%
\begin{pgfscope}%
\pgfsetrectcap%
\pgfsetroundjoin%
\pgfsetlinewidth{1.505625pt}%
\pgfsetstrokecolor{currentstroke3}%
\pgfsetdash{}{0pt}%
\pgfpathmoveto{\pgfqpoint{5.600014in}{2.211185in}}%
\pgfpathlineto{\pgfqpoint{5.725014in}{2.211185in}}%
\pgfpathlineto{\pgfqpoint{5.850014in}{2.211185in}}%
\pgfusepath{stroke}%
\end{pgfscope}%
\begin{pgfscope}%
\definecolor{textcolor}{rgb}{0.000000,0.000000,0.000000}%
\pgfsetstrokecolor{textcolor}%
\pgfsetfillcolor{textcolor}%
\pgftext[x=5.950014in,y=2.167435in,left,base]{\color{textcolor}{\rmfamily\fontsize{9.000000}{10.800000}\selectfont\catcode`\^=\active\def^{\ifmmode\sp\else\^{}\fi}\catcode`\%=\active\def%{\%}\Neighbors{} \& \SharedVertices{}}}%
\end{pgfscope}%
\begin{pgfscope}%
\pgfsetrectcap%
\pgfsetroundjoin%
\pgfsetlinewidth{1.505625pt}%
\pgfsetstrokecolor{currentstroke4}%
\pgfsetdash{}{0pt}%
\pgfpathmoveto{\pgfqpoint{5.600014in}{2.024235in}}%
\pgfpathlineto{\pgfqpoint{5.725014in}{2.024235in}}%
\pgfpathlineto{\pgfqpoint{5.850014in}{2.024235in}}%
\pgfusepath{stroke}%
\end{pgfscope}%
\begin{pgfscope}%
\definecolor{textcolor}{rgb}{0.000000,0.000000,0.000000}%
\pgfsetstrokecolor{textcolor}%
\pgfsetfillcolor{textcolor}%
\pgftext[x=5.950014in,y=1.980485in,left,base]{\color{textcolor}{\rmfamily\fontsize{9.000000}{10.800000}\selectfont\catcode`\^=\active\def^{\ifmmode\sp\else\^{}\fi}\catcode`\%=\active\def%{\%}\NeighborsDegree{} \& \MergeLinear{}}}%
\end{pgfscope}%
\begin{pgfscope}%
\pgfsetrectcap%
\pgfsetroundjoin%
\pgfsetlinewidth{1.505625pt}%
\pgfsetstrokecolor{currentstroke5}%
\pgfsetdash{}{0pt}%
\pgfpathmoveto{\pgfqpoint{5.600014in}{1.837285in}}%
\pgfpathlineto{\pgfqpoint{5.725014in}{1.837285in}}%
\pgfpathlineto{\pgfqpoint{5.850014in}{1.837285in}}%
\pgfusepath{stroke}%
\end{pgfscope}%
\begin{pgfscope}%
\definecolor{textcolor}{rgb}{0.000000,0.000000,0.000000}%
\pgfsetstrokecolor{textcolor}%
\pgfsetfillcolor{textcolor}%
\pgftext[x=5.950014in,y=1.793535in,left,base]{\color{textcolor}{\rmfamily\fontsize{9.000000}{10.800000}\selectfont\catcode`\^=\active\def^{\ifmmode\sp\else\^{}\fi}\catcode`\%=\active\def%{\%}\NeighborsDegree{} \& \SharedVertices{}}}%
\end{pgfscope}%
\begin{pgfscope}%
\pgfsetrectcap%
\pgfsetroundjoin%
\pgfsetlinewidth{1.505625pt}%
\pgfsetstrokecolor{currentstroke6}%
\pgfsetdash{}{0pt}%
\pgfpathmoveto{\pgfqpoint{5.600014in}{1.650334in}}%
\pgfpathlineto{\pgfqpoint{5.725014in}{1.650334in}}%
\pgfpathlineto{\pgfqpoint{5.850014in}{1.650334in}}%
\pgfusepath{stroke}%
\end{pgfscope}%
\begin{pgfscope}%
\definecolor{textcolor}{rgb}{0.000000,0.000000,0.000000}%
\pgfsetstrokecolor{textcolor}%
\pgfsetfillcolor{textcolor}%
\pgftext[x=5.950014in,y=1.606584in,left,base]{\color{textcolor}{\rmfamily\fontsize{9.000000}{10.800000}\selectfont\catcode`\^=\active\def^{\ifmmode\sp\else\^{}\fi}\catcode`\%=\active\def%{\%}\None{} \& \MergeLinear{}}}%
\end{pgfscope}%
\begin{pgfscope}%
\pgfsetrectcap%
\pgfsetroundjoin%
\pgfsetlinewidth{1.505625pt}%
\pgfsetstrokecolor{currentstroke7}%
\pgfsetdash{}{0pt}%
\pgfpathmoveto{\pgfqpoint{5.600014in}{1.466863in}}%
\pgfpathlineto{\pgfqpoint{5.725014in}{1.466863in}}%
\pgfpathlineto{\pgfqpoint{5.850014in}{1.466863in}}%
\pgfusepath{stroke}%
\end{pgfscope}%
\begin{pgfscope}%
\definecolor{textcolor}{rgb}{0.000000,0.000000,0.000000}%
\pgfsetstrokecolor{textcolor}%
\pgfsetfillcolor{textcolor}%
\pgftext[x=5.950014in,y=1.423113in,left,base]{\color{textcolor}{\rmfamily\fontsize{9.000000}{10.800000}\selectfont\catcode`\^=\active\def^{\ifmmode\sp\else\^{}\fi}\catcode`\%=\active\def%{\%}\None{} \& \SharedVertices{}}}%
\end{pgfscope}%
\end{pgfpicture}%
\makeatother%
\endgroup%
}
	\caption[Running time for minimally rigid graphs]{
		Mean running time to find all NAC-colorings for minimally rigid graphs.}%
	\label{fig:graph_time_minimally_rigid}
\end{figure}%

If we analyze the number of \IsNACColoring{} calls performed by \NaiveCycles{} and \Subgraphs{} algorithms
as shown in \Cref{fig:graph_count_minimally_rigid},
it can be seen that the number of \IsNACColoring{} calls is reduced already for graphs
with eleven vertices,
even though the \NaiveCycles{} algorithm is still faster for these graphs.
%
We think that the slowdown is caused by internal overhead
used for subgraphs splitting and merging.

\begin{figure}[ht]
	\centering
	\scalebox{\BenchFigureScale}{%% Creator: Matplotlib, PGF backend
%%
%% To include the figure in your LaTeX document, write
%%   \input{<filename>.pgf}
%%
%% Make sure the required packages are loaded in your preamble
%%   \usepackage{pgf}
%%
%% Also ensure that all the required font packages are loaded; for instance,
%% the lmodern package is sometimes necessary when using math font.
%%   \usepackage{lmodern}
%%
%% Figures using additional raster images can only be included by \input if
%% they are in the same directory as the main LaTeX file. For loading figures
%% from other directories you can use the `import` package
%%   \usepackage{import}
%%
%% and then include the figures with
%%   \import{<path to file>}{<filename>.pgf}
%%
%% Matplotlib used the following preamble
%%   \def\mathdefault#1{#1}
%%   \everymath=\expandafter{\the\everymath\displaystyle}
%%   \IfFileExists{scrextend.sty}{
%%     \usepackage[fontsize=10.000000pt]{scrextend}
%%   }{
%%     \renewcommand{\normalsize}{\fontsize{10.000000}{12.000000}\selectfont}
%%     \normalsize
%%   }
%%   
%%   \ifdefined\pdftexversion\else  % non-pdftex case.
%%     \usepackage{fontspec}
%%     \setmainfont{DejaVuSans.ttf}[Path=\detokenize{/home/petr/Projects/PyRigi/.venv/lib/python3.12/site-packages/matplotlib/mpl-data/fonts/ttf/}]
%%     \setsansfont{DejaVuSans.ttf}[Path=\detokenize{/home/petr/Projects/PyRigi/.venv/lib/python3.12/site-packages/matplotlib/mpl-data/fonts/ttf/}]
%%     \setmonofont{DejaVuSansMono.ttf}[Path=\detokenize{/home/petr/Projects/PyRigi/.venv/lib/python3.12/site-packages/matplotlib/mpl-data/fonts/ttf/}]
%%   \fi
%%   \makeatletter\@ifpackageloaded{under\Score{}}{}{\usepackage[strings]{under\Score{}}}\makeatother
%%
\begingroup%
\makeatletter%
\begin{pgfpicture}%
\pgfpathrectangle{\pgfpointorigin}{\pgfqpoint{8.384376in}{2.841849in}}%
\pgfusepath{use as bounding box, clip}%
\begin{pgfscope}%
\pgfsetbuttcap%
\pgfsetmiterjoin%
\definecolor{currentfill}{rgb}{1.000000,1.000000,1.000000}%
\pgfsetfillcolor{currentfill}%
\pgfsetlinewidth{0.000000pt}%
\definecolor{currentstroke}{rgb}{1.000000,1.000000,1.000000}%
\pgfsetstrokecolor{currentstroke}%
\pgfsetdash{}{0pt}%
\pgfpathmoveto{\pgfqpoint{0.000000in}{0.000000in}}%
\pgfpathlineto{\pgfqpoint{8.384376in}{0.000000in}}%
\pgfpathlineto{\pgfqpoint{8.384376in}{2.841849in}}%
\pgfpathlineto{\pgfqpoint{0.000000in}{2.841849in}}%
\pgfpathlineto{\pgfqpoint{0.000000in}{0.000000in}}%
\pgfpathclose%
\pgfusepath{fill}%
\end{pgfscope}%
\begin{pgfscope}%
\pgfsetbuttcap%
\pgfsetmiterjoin%
\definecolor{currentfill}{rgb}{1.000000,1.000000,1.000000}%
\pgfsetfillcolor{currentfill}%
\pgfsetlinewidth{0.000000pt}%
\definecolor{currentstroke}{rgb}{0.000000,0.000000,0.000000}%
\pgfsetstrokecolor{currentstroke}%
\pgfsetstrokeopacity{0.000000}%
\pgfsetdash{}{0pt}%
\pgfpathmoveto{\pgfqpoint{0.588387in}{0.521603in}}%
\pgfpathlineto{\pgfqpoint{5.487514in}{0.521603in}}%
\pgfpathlineto{\pgfqpoint{5.487514in}{2.741849in}}%
\pgfpathlineto{\pgfqpoint{0.588387in}{2.741849in}}%
\pgfpathlineto{\pgfqpoint{0.588387in}{0.521603in}}%
\pgfpathclose%
\pgfusepath{fill}%
\end{pgfscope}%
\begin{pgfscope}%
\pgfsetbuttcap%
\pgfsetroundjoin%
\definecolor{currentfill}{rgb}{0.000000,0.000000,0.000000}%
\pgfsetfillcolor{currentfill}%
\pgfsetlinewidth{0.803000pt}%
\definecolor{currentstroke}{rgb}{0.000000,0.000000,0.000000}%
\pgfsetstrokecolor{currentstroke}%
\pgfsetdash{}{0pt}%
\pgfsys@defobject{currentmarker}{\pgfqpoint{0.000000in}{-0.048611in}}{\pgfqpoint{0.000000in}{0.000000in}}{%
\pgfpathmoveto{\pgfqpoint{0.000000in}{0.000000in}}%
\pgfpathlineto{\pgfqpoint{0.000000in}{-0.048611in}}%
\pgfusepath{stroke,fill}%
}%
\begin{pgfscope}%
\pgfsys@transformshift{1.098414in}{0.521603in}%
\pgfsys@useobject{currentmarker}{}%
\end{pgfscope}%
\end{pgfscope}%
\begin{pgfscope}%
\definecolor{textcolor}{rgb}{0.000000,0.000000,0.000000}%
\pgfsetstrokecolor{textcolor}%
\pgfsetfillcolor{textcolor}%
\pgftext[x=1.098414in,y=0.424381in,,top]{\color{textcolor}{\rmfamily\fontsize{10.000000}{12.000000}\selectfont\catcode`\^=\active\def^{\ifmmode\sp\else\^{}\fi}\catcode`\%=\active\def%{\%}$\mathdefault{4}$}}%
\end{pgfscope}%
\begin{pgfscope}%
\pgfsetbuttcap%
\pgfsetroundjoin%
\definecolor{currentfill}{rgb}{0.000000,0.000000,0.000000}%
\pgfsetfillcolor{currentfill}%
\pgfsetlinewidth{0.803000pt}%
\definecolor{currentstroke}{rgb}{0.000000,0.000000,0.000000}%
\pgfsetstrokecolor{currentstroke}%
\pgfsetdash{}{0pt}%
\pgfsys@defobject{currentmarker}{\pgfqpoint{0.000000in}{-0.048611in}}{\pgfqpoint{0.000000in}{0.000000in}}{%
\pgfpathmoveto{\pgfqpoint{0.000000in}{0.000000in}}%
\pgfpathlineto{\pgfqpoint{0.000000in}{-0.048611in}}%
\pgfusepath{stroke,fill}%
}%
\begin{pgfscope}%
\pgfsys@transformshift{1.673091in}{0.521603in}%
\pgfsys@useobject{currentmarker}{}%
\end{pgfscope}%
\end{pgfscope}%
\begin{pgfscope}%
\definecolor{textcolor}{rgb}{0.000000,0.000000,0.000000}%
\pgfsetstrokecolor{textcolor}%
\pgfsetfillcolor{textcolor}%
\pgftext[x=1.673091in,y=0.424381in,,top]{\color{textcolor}{\rmfamily\fontsize{10.000000}{12.000000}\selectfont\catcode`\^=\active\def^{\ifmmode\sp\else\^{}\fi}\catcode`\%=\active\def%{\%}$\mathdefault{8}$}}%
\end{pgfscope}%
\begin{pgfscope}%
\pgfsetbuttcap%
\pgfsetroundjoin%
\definecolor{currentfill}{rgb}{0.000000,0.000000,0.000000}%
\pgfsetfillcolor{currentfill}%
\pgfsetlinewidth{0.803000pt}%
\definecolor{currentstroke}{rgb}{0.000000,0.000000,0.000000}%
\pgfsetstrokecolor{currentstroke}%
\pgfsetdash{}{0pt}%
\pgfsys@defobject{currentmarker}{\pgfqpoint{0.000000in}{-0.048611in}}{\pgfqpoint{0.000000in}{0.000000in}}{%
\pgfpathmoveto{\pgfqpoint{0.000000in}{0.000000in}}%
\pgfpathlineto{\pgfqpoint{0.000000in}{-0.048611in}}%
\pgfusepath{stroke,fill}%
}%
\begin{pgfscope}%
\pgfsys@transformshift{2.247769in}{0.521603in}%
\pgfsys@useobject{currentmarker}{}%
\end{pgfscope}%
\end{pgfscope}%
\begin{pgfscope}%
\definecolor{textcolor}{rgb}{0.000000,0.000000,0.000000}%
\pgfsetstrokecolor{textcolor}%
\pgfsetfillcolor{textcolor}%
\pgftext[x=2.247769in,y=0.424381in,,top]{\color{textcolor}{\rmfamily\fontsize{10.000000}{12.000000}\selectfont\catcode`\^=\active\def^{\ifmmode\sp\else\^{}\fi}\catcode`\%=\active\def%{\%}$\mathdefault{12}$}}%
\end{pgfscope}%
\begin{pgfscope}%
\pgfsetbuttcap%
\pgfsetroundjoin%
\definecolor{currentfill}{rgb}{0.000000,0.000000,0.000000}%
\pgfsetfillcolor{currentfill}%
\pgfsetlinewidth{0.803000pt}%
\definecolor{currentstroke}{rgb}{0.000000,0.000000,0.000000}%
\pgfsetstrokecolor{currentstroke}%
\pgfsetdash{}{0pt}%
\pgfsys@defobject{currentmarker}{\pgfqpoint{0.000000in}{-0.048611in}}{\pgfqpoint{0.000000in}{0.000000in}}{%
\pgfpathmoveto{\pgfqpoint{0.000000in}{0.000000in}}%
\pgfpathlineto{\pgfqpoint{0.000000in}{-0.048611in}}%
\pgfusepath{stroke,fill}%
}%
\begin{pgfscope}%
\pgfsys@transformshift{2.822446in}{0.521603in}%
\pgfsys@useobject{currentmarker}{}%
\end{pgfscope}%
\end{pgfscope}%
\begin{pgfscope}%
\definecolor{textcolor}{rgb}{0.000000,0.000000,0.000000}%
\pgfsetstrokecolor{textcolor}%
\pgfsetfillcolor{textcolor}%
\pgftext[x=2.822446in,y=0.424381in,,top]{\color{textcolor}{\rmfamily\fontsize{10.000000}{12.000000}\selectfont\catcode`\^=\active\def^{\ifmmode\sp\else\^{}\fi}\catcode`\%=\active\def%{\%}$\mathdefault{16}$}}%
\end{pgfscope}%
\begin{pgfscope}%
\pgfsetbuttcap%
\pgfsetroundjoin%
\definecolor{currentfill}{rgb}{0.000000,0.000000,0.000000}%
\pgfsetfillcolor{currentfill}%
\pgfsetlinewidth{0.803000pt}%
\definecolor{currentstroke}{rgb}{0.000000,0.000000,0.000000}%
\pgfsetstrokecolor{currentstroke}%
\pgfsetdash{}{0pt}%
\pgfsys@defobject{currentmarker}{\pgfqpoint{0.000000in}{-0.048611in}}{\pgfqpoint{0.000000in}{0.000000in}}{%
\pgfpathmoveto{\pgfqpoint{0.000000in}{0.000000in}}%
\pgfpathlineto{\pgfqpoint{0.000000in}{-0.048611in}}%
\pgfusepath{stroke,fill}%
}%
\begin{pgfscope}%
\pgfsys@transformshift{3.397124in}{0.521603in}%
\pgfsys@useobject{currentmarker}{}%
\end{pgfscope}%
\end{pgfscope}%
\begin{pgfscope}%
\definecolor{textcolor}{rgb}{0.000000,0.000000,0.000000}%
\pgfsetstrokecolor{textcolor}%
\pgfsetfillcolor{textcolor}%
\pgftext[x=3.397124in,y=0.424381in,,top]{\color{textcolor}{\rmfamily\fontsize{10.000000}{12.000000}\selectfont\catcode`\^=\active\def^{\ifmmode\sp\else\^{}\fi}\catcode`\%=\active\def%{\%}$\mathdefault{20}$}}%
\end{pgfscope}%
\begin{pgfscope}%
\pgfsetbuttcap%
\pgfsetroundjoin%
\definecolor{currentfill}{rgb}{0.000000,0.000000,0.000000}%
\pgfsetfillcolor{currentfill}%
\pgfsetlinewidth{0.803000pt}%
\definecolor{currentstroke}{rgb}{0.000000,0.000000,0.000000}%
\pgfsetstrokecolor{currentstroke}%
\pgfsetdash{}{0pt}%
\pgfsys@defobject{currentmarker}{\pgfqpoint{0.000000in}{-0.048611in}}{\pgfqpoint{0.000000in}{0.000000in}}{%
\pgfpathmoveto{\pgfqpoint{0.000000in}{0.000000in}}%
\pgfpathlineto{\pgfqpoint{0.000000in}{-0.048611in}}%
\pgfusepath{stroke,fill}%
}%
\begin{pgfscope}%
\pgfsys@transformshift{3.971802in}{0.521603in}%
\pgfsys@useobject{currentmarker}{}%
\end{pgfscope}%
\end{pgfscope}%
\begin{pgfscope}%
\definecolor{textcolor}{rgb}{0.000000,0.000000,0.000000}%
\pgfsetstrokecolor{textcolor}%
\pgfsetfillcolor{textcolor}%
\pgftext[x=3.971802in,y=0.424381in,,top]{\color{textcolor}{\rmfamily\fontsize{10.000000}{12.000000}\selectfont\catcode`\^=\active\def^{\ifmmode\sp\else\^{}\fi}\catcode`\%=\active\def%{\%}$\mathdefault{24}$}}%
\end{pgfscope}%
\begin{pgfscope}%
\pgfsetbuttcap%
\pgfsetroundjoin%
\definecolor{currentfill}{rgb}{0.000000,0.000000,0.000000}%
\pgfsetfillcolor{currentfill}%
\pgfsetlinewidth{0.803000pt}%
\definecolor{currentstroke}{rgb}{0.000000,0.000000,0.000000}%
\pgfsetstrokecolor{currentstroke}%
\pgfsetdash{}{0pt}%
\pgfsys@defobject{currentmarker}{\pgfqpoint{0.000000in}{-0.048611in}}{\pgfqpoint{0.000000in}{0.000000in}}{%
\pgfpathmoveto{\pgfqpoint{0.000000in}{0.000000in}}%
\pgfpathlineto{\pgfqpoint{0.000000in}{-0.048611in}}%
\pgfusepath{stroke,fill}%
}%
\begin{pgfscope}%
\pgfsys@transformshift{4.546479in}{0.521603in}%
\pgfsys@useobject{currentmarker}{}%
\end{pgfscope}%
\end{pgfscope}%
\begin{pgfscope}%
\definecolor{textcolor}{rgb}{0.000000,0.000000,0.000000}%
\pgfsetstrokecolor{textcolor}%
\pgfsetfillcolor{textcolor}%
\pgftext[x=4.546479in,y=0.424381in,,top]{\color{textcolor}{\rmfamily\fontsize{10.000000}{12.000000}\selectfont\catcode`\^=\active\def^{\ifmmode\sp\else\^{}\fi}\catcode`\%=\active\def%{\%}$\mathdefault{28}$}}%
\end{pgfscope}%
\begin{pgfscope}%
\pgfsetbuttcap%
\pgfsetroundjoin%
\definecolor{currentfill}{rgb}{0.000000,0.000000,0.000000}%
\pgfsetfillcolor{currentfill}%
\pgfsetlinewidth{0.803000pt}%
\definecolor{currentstroke}{rgb}{0.000000,0.000000,0.000000}%
\pgfsetstrokecolor{currentstroke}%
\pgfsetdash{}{0pt}%
\pgfsys@defobject{currentmarker}{\pgfqpoint{0.000000in}{-0.048611in}}{\pgfqpoint{0.000000in}{0.000000in}}{%
\pgfpathmoveto{\pgfqpoint{0.000000in}{0.000000in}}%
\pgfpathlineto{\pgfqpoint{0.000000in}{-0.048611in}}%
\pgfusepath{stroke,fill}%
}%
\begin{pgfscope}%
\pgfsys@transformshift{5.121157in}{0.521603in}%
\pgfsys@useobject{currentmarker}{}%
\end{pgfscope}%
\end{pgfscope}%
\begin{pgfscope}%
\definecolor{textcolor}{rgb}{0.000000,0.000000,0.000000}%
\pgfsetstrokecolor{textcolor}%
\pgfsetfillcolor{textcolor}%
\pgftext[x=5.121157in,y=0.424381in,,top]{\color{textcolor}{\rmfamily\fontsize{10.000000}{12.000000}\selectfont\catcode`\^=\active\def^{\ifmmode\sp\else\^{}\fi}\catcode`\%=\active\def%{\%}$\mathdefault{32}$}}%
\end{pgfscope}%
\begin{pgfscope}%
\definecolor{textcolor}{rgb}{0.000000,0.000000,0.000000}%
\pgfsetstrokecolor{textcolor}%
\pgfsetfillcolor{textcolor}%
\pgftext[x=3.037950in,y=0.234413in,,top]{\color{textcolor}{\rmfamily\fontsize{10.000000}{12.000000}\selectfont\catcode`\^=\active\def^{\ifmmode\sp\else\^{}\fi}\catcode`\%=\active\def%{\%}Monochromatic classes}}%
\end{pgfscope}%
\begin{pgfscope}%
\pgfsetbuttcap%
\pgfsetroundjoin%
\definecolor{currentfill}{rgb}{0.000000,0.000000,0.000000}%
\pgfsetfillcolor{currentfill}%
\pgfsetlinewidth{0.803000pt}%
\definecolor{currentstroke}{rgb}{0.000000,0.000000,0.000000}%
\pgfsetstrokecolor{currentstroke}%
\pgfsetdash{}{0pt}%
\pgfsys@defobject{currentmarker}{\pgfqpoint{-0.048611in}{0.000000in}}{\pgfqpoint{-0.000000in}{0.000000in}}{%
\pgfpathmoveto{\pgfqpoint{-0.000000in}{0.000000in}}%
\pgfpathlineto{\pgfqpoint{-0.048611in}{0.000000in}}%
\pgfusepath{stroke,fill}%
}%
\begin{pgfscope}%
\pgfsys@transformshift{0.588387in}{0.622524in}%
\pgfsys@useobject{currentmarker}{}%
\end{pgfscope}%
\end{pgfscope}%
\begin{pgfscope}%
\definecolor{textcolor}{rgb}{0.000000,0.000000,0.000000}%
\pgfsetstrokecolor{textcolor}%
\pgfsetfillcolor{textcolor}%
\pgftext[x=0.289968in, y=0.569762in, left, base]{\color{textcolor}{\rmfamily\fontsize{10.000000}{12.000000}\selectfont\catcode`\^=\active\def^{\ifmmode\sp\else\^{}\fi}\catcode`\%=\active\def%{\%}$\mathdefault{10^{0}}$}}%
\end{pgfscope}%
\begin{pgfscope}%
\pgfsetbuttcap%
\pgfsetroundjoin%
\definecolor{currentfill}{rgb}{0.000000,0.000000,0.000000}%
\pgfsetfillcolor{currentfill}%
\pgfsetlinewidth{0.803000pt}%
\definecolor{currentstroke}{rgb}{0.000000,0.000000,0.000000}%
\pgfsetstrokecolor{currentstroke}%
\pgfsetdash{}{0pt}%
\pgfsys@defobject{currentmarker}{\pgfqpoint{-0.048611in}{0.000000in}}{\pgfqpoint{-0.000000in}{0.000000in}}{%
\pgfpathmoveto{\pgfqpoint{-0.000000in}{0.000000in}}%
\pgfpathlineto{\pgfqpoint{-0.048611in}{0.000000in}}%
\pgfusepath{stroke,fill}%
}%
\begin{pgfscope}%
\pgfsys@transformshift{0.588387in}{0.927296in}%
\pgfsys@useobject{currentmarker}{}%
\end{pgfscope}%
\end{pgfscope}%
\begin{pgfscope}%
\definecolor{textcolor}{rgb}{0.000000,0.000000,0.000000}%
\pgfsetstrokecolor{textcolor}%
\pgfsetfillcolor{textcolor}%
\pgftext[x=0.289968in, y=0.874535in, left, base]{\color{textcolor}{\rmfamily\fontsize{10.000000}{12.000000}\selectfont\catcode`\^=\active\def^{\ifmmode\sp\else\^{}\fi}\catcode`\%=\active\def%{\%}$\mathdefault{10^{1}}$}}%
\end{pgfscope}%
\begin{pgfscope}%
\pgfsetbuttcap%
\pgfsetroundjoin%
\definecolor{currentfill}{rgb}{0.000000,0.000000,0.000000}%
\pgfsetfillcolor{currentfill}%
\pgfsetlinewidth{0.803000pt}%
\definecolor{currentstroke}{rgb}{0.000000,0.000000,0.000000}%
\pgfsetstrokecolor{currentstroke}%
\pgfsetdash{}{0pt}%
\pgfsys@defobject{currentmarker}{\pgfqpoint{-0.048611in}{0.000000in}}{\pgfqpoint{-0.000000in}{0.000000in}}{%
\pgfpathmoveto{\pgfqpoint{-0.000000in}{0.000000in}}%
\pgfpathlineto{\pgfqpoint{-0.048611in}{0.000000in}}%
\pgfusepath{stroke,fill}%
}%
\begin{pgfscope}%
\pgfsys@transformshift{0.588387in}{1.232069in}%
\pgfsys@useobject{currentmarker}{}%
\end{pgfscope}%
\end{pgfscope}%
\begin{pgfscope}%
\definecolor{textcolor}{rgb}{0.000000,0.000000,0.000000}%
\pgfsetstrokecolor{textcolor}%
\pgfsetfillcolor{textcolor}%
\pgftext[x=0.289968in, y=1.179307in, left, base]{\color{textcolor}{\rmfamily\fontsize{10.000000}{12.000000}\selectfont\catcode`\^=\active\def^{\ifmmode\sp\else\^{}\fi}\catcode`\%=\active\def%{\%}$\mathdefault{10^{2}}$}}%
\end{pgfscope}%
\begin{pgfscope}%
\pgfsetbuttcap%
\pgfsetroundjoin%
\definecolor{currentfill}{rgb}{0.000000,0.000000,0.000000}%
\pgfsetfillcolor{currentfill}%
\pgfsetlinewidth{0.803000pt}%
\definecolor{currentstroke}{rgb}{0.000000,0.000000,0.000000}%
\pgfsetstrokecolor{currentstroke}%
\pgfsetdash{}{0pt}%
\pgfsys@defobject{currentmarker}{\pgfqpoint{-0.048611in}{0.000000in}}{\pgfqpoint{-0.000000in}{0.000000in}}{%
\pgfpathmoveto{\pgfqpoint{-0.000000in}{0.000000in}}%
\pgfpathlineto{\pgfqpoint{-0.048611in}{0.000000in}}%
\pgfusepath{stroke,fill}%
}%
\begin{pgfscope}%
\pgfsys@transformshift{0.588387in}{1.536841in}%
\pgfsys@useobject{currentmarker}{}%
\end{pgfscope}%
\end{pgfscope}%
\begin{pgfscope}%
\definecolor{textcolor}{rgb}{0.000000,0.000000,0.000000}%
\pgfsetstrokecolor{textcolor}%
\pgfsetfillcolor{textcolor}%
\pgftext[x=0.289968in, y=1.484080in, left, base]{\color{textcolor}{\rmfamily\fontsize{10.000000}{12.000000}\selectfont\catcode`\^=\active\def^{\ifmmode\sp\else\^{}\fi}\catcode`\%=\active\def%{\%}$\mathdefault{10^{3}}$}}%
\end{pgfscope}%
\begin{pgfscope}%
\pgfsetbuttcap%
\pgfsetroundjoin%
\definecolor{currentfill}{rgb}{0.000000,0.000000,0.000000}%
\pgfsetfillcolor{currentfill}%
\pgfsetlinewidth{0.803000pt}%
\definecolor{currentstroke}{rgb}{0.000000,0.000000,0.000000}%
\pgfsetstrokecolor{currentstroke}%
\pgfsetdash{}{0pt}%
\pgfsys@defobject{currentmarker}{\pgfqpoint{-0.048611in}{0.000000in}}{\pgfqpoint{-0.000000in}{0.000000in}}{%
\pgfpathmoveto{\pgfqpoint{-0.000000in}{0.000000in}}%
\pgfpathlineto{\pgfqpoint{-0.048611in}{0.000000in}}%
\pgfusepath{stroke,fill}%
}%
\begin{pgfscope}%
\pgfsys@transformshift{0.588387in}{1.841614in}%
\pgfsys@useobject{currentmarker}{}%
\end{pgfscope}%
\end{pgfscope}%
\begin{pgfscope}%
\definecolor{textcolor}{rgb}{0.000000,0.000000,0.000000}%
\pgfsetstrokecolor{textcolor}%
\pgfsetfillcolor{textcolor}%
\pgftext[x=0.289968in, y=1.788853in, left, base]{\color{textcolor}{\rmfamily\fontsize{10.000000}{12.000000}\selectfont\catcode`\^=\active\def^{\ifmmode\sp\else\^{}\fi}\catcode`\%=\active\def%{\%}$\mathdefault{10^{4}}$}}%
\end{pgfscope}%
\begin{pgfscope}%
\pgfsetbuttcap%
\pgfsetroundjoin%
\definecolor{currentfill}{rgb}{0.000000,0.000000,0.000000}%
\pgfsetfillcolor{currentfill}%
\pgfsetlinewidth{0.803000pt}%
\definecolor{currentstroke}{rgb}{0.000000,0.000000,0.000000}%
\pgfsetstrokecolor{currentstroke}%
\pgfsetdash{}{0pt}%
\pgfsys@defobject{currentmarker}{\pgfqpoint{-0.048611in}{0.000000in}}{\pgfqpoint{-0.000000in}{0.000000in}}{%
\pgfpathmoveto{\pgfqpoint{-0.000000in}{0.000000in}}%
\pgfpathlineto{\pgfqpoint{-0.048611in}{0.000000in}}%
\pgfusepath{stroke,fill}%
}%
\begin{pgfscope}%
\pgfsys@transformshift{0.588387in}{2.146387in}%
\pgfsys@useobject{currentmarker}{}%
\end{pgfscope}%
\end{pgfscope}%
\begin{pgfscope}%
\definecolor{textcolor}{rgb}{0.000000,0.000000,0.000000}%
\pgfsetstrokecolor{textcolor}%
\pgfsetfillcolor{textcolor}%
\pgftext[x=0.289968in, y=2.093625in, left, base]{\color{textcolor}{\rmfamily\fontsize{10.000000}{12.000000}\selectfont\catcode`\^=\active\def^{\ifmmode\sp\else\^{}\fi}\catcode`\%=\active\def%{\%}$\mathdefault{10^{5}}$}}%
\end{pgfscope}%
\begin{pgfscope}%
\pgfsetbuttcap%
\pgfsetroundjoin%
\definecolor{currentfill}{rgb}{0.000000,0.000000,0.000000}%
\pgfsetfillcolor{currentfill}%
\pgfsetlinewidth{0.803000pt}%
\definecolor{currentstroke}{rgb}{0.000000,0.000000,0.000000}%
\pgfsetstrokecolor{currentstroke}%
\pgfsetdash{}{0pt}%
\pgfsys@defobject{currentmarker}{\pgfqpoint{-0.048611in}{0.000000in}}{\pgfqpoint{-0.000000in}{0.000000in}}{%
\pgfpathmoveto{\pgfqpoint{-0.000000in}{0.000000in}}%
\pgfpathlineto{\pgfqpoint{-0.048611in}{0.000000in}}%
\pgfusepath{stroke,fill}%
}%
\begin{pgfscope}%
\pgfsys@transformshift{0.588387in}{2.451159in}%
\pgfsys@useobject{currentmarker}{}%
\end{pgfscope}%
\end{pgfscope}%
\begin{pgfscope}%
\definecolor{textcolor}{rgb}{0.000000,0.000000,0.000000}%
\pgfsetstrokecolor{textcolor}%
\pgfsetfillcolor{textcolor}%
\pgftext[x=0.289968in, y=2.398398in, left, base]{\color{textcolor}{\rmfamily\fontsize{10.000000}{12.000000}\selectfont\catcode`\^=\active\def^{\ifmmode\sp\else\^{}\fi}\catcode`\%=\active\def%{\%}$\mathdefault{10^{6}}$}}%
\end{pgfscope}%
\begin{pgfscope}%
\pgfsetbuttcap%
\pgfsetroundjoin%
\definecolor{currentfill}{rgb}{0.000000,0.000000,0.000000}%
\pgfsetfillcolor{currentfill}%
\pgfsetlinewidth{0.602250pt}%
\definecolor{currentstroke}{rgb}{0.000000,0.000000,0.000000}%
\pgfsetstrokecolor{currentstroke}%
\pgfsetdash{}{0pt}%
\pgfsys@defobject{currentmarker}{\pgfqpoint{-0.027778in}{0.000000in}}{\pgfqpoint{-0.000000in}{0.000000in}}{%
\pgfpathmoveto{\pgfqpoint{-0.000000in}{0.000000in}}%
\pgfpathlineto{\pgfqpoint{-0.027778in}{0.000000in}}%
\pgfusepath{stroke,fill}%
}%
\begin{pgfscope}%
\pgfsys@transformshift{0.588387in}{0.530778in}%
\pgfsys@useobject{currentmarker}{}%
\end{pgfscope}%
\end{pgfscope}%
\begin{pgfscope}%
\pgfsetbuttcap%
\pgfsetroundjoin%
\definecolor{currentfill}{rgb}{0.000000,0.000000,0.000000}%
\pgfsetfillcolor{currentfill}%
\pgfsetlinewidth{0.602250pt}%
\definecolor{currentstroke}{rgb}{0.000000,0.000000,0.000000}%
\pgfsetstrokecolor{currentstroke}%
\pgfsetdash{}{0pt}%
\pgfsys@defobject{currentmarker}{\pgfqpoint{-0.027778in}{0.000000in}}{\pgfqpoint{-0.000000in}{0.000000in}}{%
\pgfpathmoveto{\pgfqpoint{-0.000000in}{0.000000in}}%
\pgfpathlineto{\pgfqpoint{-0.027778in}{0.000000in}}%
\pgfusepath{stroke,fill}%
}%
\begin{pgfscope}%
\pgfsys@transformshift{0.588387in}{0.554910in}%
\pgfsys@useobject{currentmarker}{}%
\end{pgfscope}%
\end{pgfscope}%
\begin{pgfscope}%
\pgfsetbuttcap%
\pgfsetroundjoin%
\definecolor{currentfill}{rgb}{0.000000,0.000000,0.000000}%
\pgfsetfillcolor{currentfill}%
\pgfsetlinewidth{0.602250pt}%
\definecolor{currentstroke}{rgb}{0.000000,0.000000,0.000000}%
\pgfsetstrokecolor{currentstroke}%
\pgfsetdash{}{0pt}%
\pgfsys@defobject{currentmarker}{\pgfqpoint{-0.027778in}{0.000000in}}{\pgfqpoint{-0.000000in}{0.000000in}}{%
\pgfpathmoveto{\pgfqpoint{-0.000000in}{0.000000in}}%
\pgfpathlineto{\pgfqpoint{-0.027778in}{0.000000in}}%
\pgfusepath{stroke,fill}%
}%
\begin{pgfscope}%
\pgfsys@transformshift{0.588387in}{0.575314in}%
\pgfsys@useobject{currentmarker}{}%
\end{pgfscope}%
\end{pgfscope}%
\begin{pgfscope}%
\pgfsetbuttcap%
\pgfsetroundjoin%
\definecolor{currentfill}{rgb}{0.000000,0.000000,0.000000}%
\pgfsetfillcolor{currentfill}%
\pgfsetlinewidth{0.602250pt}%
\definecolor{currentstroke}{rgb}{0.000000,0.000000,0.000000}%
\pgfsetstrokecolor{currentstroke}%
\pgfsetdash{}{0pt}%
\pgfsys@defobject{currentmarker}{\pgfqpoint{-0.027778in}{0.000000in}}{\pgfqpoint{-0.000000in}{0.000000in}}{%
\pgfpathmoveto{\pgfqpoint{-0.000000in}{0.000000in}}%
\pgfpathlineto{\pgfqpoint{-0.027778in}{0.000000in}}%
\pgfusepath{stroke,fill}%
}%
\begin{pgfscope}%
\pgfsys@transformshift{0.588387in}{0.592988in}%
\pgfsys@useobject{currentmarker}{}%
\end{pgfscope}%
\end{pgfscope}%
\begin{pgfscope}%
\pgfsetbuttcap%
\pgfsetroundjoin%
\definecolor{currentfill}{rgb}{0.000000,0.000000,0.000000}%
\pgfsetfillcolor{currentfill}%
\pgfsetlinewidth{0.602250pt}%
\definecolor{currentstroke}{rgb}{0.000000,0.000000,0.000000}%
\pgfsetstrokecolor{currentstroke}%
\pgfsetdash{}{0pt}%
\pgfsys@defobject{currentmarker}{\pgfqpoint{-0.027778in}{0.000000in}}{\pgfqpoint{-0.000000in}{0.000000in}}{%
\pgfpathmoveto{\pgfqpoint{-0.000000in}{0.000000in}}%
\pgfpathlineto{\pgfqpoint{-0.027778in}{0.000000in}}%
\pgfusepath{stroke,fill}%
}%
\begin{pgfscope}%
\pgfsys@transformshift{0.588387in}{0.608578in}%
\pgfsys@useobject{currentmarker}{}%
\end{pgfscope}%
\end{pgfscope}%
\begin{pgfscope}%
\pgfsetbuttcap%
\pgfsetroundjoin%
\definecolor{currentfill}{rgb}{0.000000,0.000000,0.000000}%
\pgfsetfillcolor{currentfill}%
\pgfsetlinewidth{0.602250pt}%
\definecolor{currentstroke}{rgb}{0.000000,0.000000,0.000000}%
\pgfsetstrokecolor{currentstroke}%
\pgfsetdash{}{0pt}%
\pgfsys@defobject{currentmarker}{\pgfqpoint{-0.027778in}{0.000000in}}{\pgfqpoint{-0.000000in}{0.000000in}}{%
\pgfpathmoveto{\pgfqpoint{-0.000000in}{0.000000in}}%
\pgfpathlineto{\pgfqpoint{-0.027778in}{0.000000in}}%
\pgfusepath{stroke,fill}%
}%
\begin{pgfscope}%
\pgfsys@transformshift{0.588387in}{0.714269in}%
\pgfsys@useobject{currentmarker}{}%
\end{pgfscope}%
\end{pgfscope}%
\begin{pgfscope}%
\pgfsetbuttcap%
\pgfsetroundjoin%
\definecolor{currentfill}{rgb}{0.000000,0.000000,0.000000}%
\pgfsetfillcolor{currentfill}%
\pgfsetlinewidth{0.602250pt}%
\definecolor{currentstroke}{rgb}{0.000000,0.000000,0.000000}%
\pgfsetstrokecolor{currentstroke}%
\pgfsetdash{}{0pt}%
\pgfsys@defobject{currentmarker}{\pgfqpoint{-0.027778in}{0.000000in}}{\pgfqpoint{-0.000000in}{0.000000in}}{%
\pgfpathmoveto{\pgfqpoint{-0.000000in}{0.000000in}}%
\pgfpathlineto{\pgfqpoint{-0.027778in}{0.000000in}}%
\pgfusepath{stroke,fill}%
}%
\begin{pgfscope}%
\pgfsys@transformshift{0.588387in}{0.767937in}%
\pgfsys@useobject{currentmarker}{}%
\end{pgfscope}%
\end{pgfscope}%
\begin{pgfscope}%
\pgfsetbuttcap%
\pgfsetroundjoin%
\definecolor{currentfill}{rgb}{0.000000,0.000000,0.000000}%
\pgfsetfillcolor{currentfill}%
\pgfsetlinewidth{0.602250pt}%
\definecolor{currentstroke}{rgb}{0.000000,0.000000,0.000000}%
\pgfsetstrokecolor{currentstroke}%
\pgfsetdash{}{0pt}%
\pgfsys@defobject{currentmarker}{\pgfqpoint{-0.027778in}{0.000000in}}{\pgfqpoint{-0.000000in}{0.000000in}}{%
\pgfpathmoveto{\pgfqpoint{-0.000000in}{0.000000in}}%
\pgfpathlineto{\pgfqpoint{-0.027778in}{0.000000in}}%
\pgfusepath{stroke,fill}%
}%
\begin{pgfscope}%
\pgfsys@transformshift{0.588387in}{0.806015in}%
\pgfsys@useobject{currentmarker}{}%
\end{pgfscope}%
\end{pgfscope}%
\begin{pgfscope}%
\pgfsetbuttcap%
\pgfsetroundjoin%
\definecolor{currentfill}{rgb}{0.000000,0.000000,0.000000}%
\pgfsetfillcolor{currentfill}%
\pgfsetlinewidth{0.602250pt}%
\definecolor{currentstroke}{rgb}{0.000000,0.000000,0.000000}%
\pgfsetstrokecolor{currentstroke}%
\pgfsetdash{}{0pt}%
\pgfsys@defobject{currentmarker}{\pgfqpoint{-0.027778in}{0.000000in}}{\pgfqpoint{-0.000000in}{0.000000in}}{%
\pgfpathmoveto{\pgfqpoint{-0.000000in}{0.000000in}}%
\pgfpathlineto{\pgfqpoint{-0.027778in}{0.000000in}}%
\pgfusepath{stroke,fill}%
}%
\begin{pgfscope}%
\pgfsys@transformshift{0.588387in}{0.835551in}%
\pgfsys@useobject{currentmarker}{}%
\end{pgfscope}%
\end{pgfscope}%
\begin{pgfscope}%
\pgfsetbuttcap%
\pgfsetroundjoin%
\definecolor{currentfill}{rgb}{0.000000,0.000000,0.000000}%
\pgfsetfillcolor{currentfill}%
\pgfsetlinewidth{0.602250pt}%
\definecolor{currentstroke}{rgb}{0.000000,0.000000,0.000000}%
\pgfsetstrokecolor{currentstroke}%
\pgfsetdash{}{0pt}%
\pgfsys@defobject{currentmarker}{\pgfqpoint{-0.027778in}{0.000000in}}{\pgfqpoint{-0.000000in}{0.000000in}}{%
\pgfpathmoveto{\pgfqpoint{-0.000000in}{0.000000in}}%
\pgfpathlineto{\pgfqpoint{-0.027778in}{0.000000in}}%
\pgfusepath{stroke,fill}%
}%
\begin{pgfscope}%
\pgfsys@transformshift{0.588387in}{0.859683in}%
\pgfsys@useobject{currentmarker}{}%
\end{pgfscope}%
\end{pgfscope}%
\begin{pgfscope}%
\pgfsetbuttcap%
\pgfsetroundjoin%
\definecolor{currentfill}{rgb}{0.000000,0.000000,0.000000}%
\pgfsetfillcolor{currentfill}%
\pgfsetlinewidth{0.602250pt}%
\definecolor{currentstroke}{rgb}{0.000000,0.000000,0.000000}%
\pgfsetstrokecolor{currentstroke}%
\pgfsetdash{}{0pt}%
\pgfsys@defobject{currentmarker}{\pgfqpoint{-0.027778in}{0.000000in}}{\pgfqpoint{-0.000000in}{0.000000in}}{%
\pgfpathmoveto{\pgfqpoint{-0.000000in}{0.000000in}}%
\pgfpathlineto{\pgfqpoint{-0.027778in}{0.000000in}}%
\pgfusepath{stroke,fill}%
}%
\begin{pgfscope}%
\pgfsys@transformshift{0.588387in}{0.880086in}%
\pgfsys@useobject{currentmarker}{}%
\end{pgfscope}%
\end{pgfscope}%
\begin{pgfscope}%
\pgfsetbuttcap%
\pgfsetroundjoin%
\definecolor{currentfill}{rgb}{0.000000,0.000000,0.000000}%
\pgfsetfillcolor{currentfill}%
\pgfsetlinewidth{0.602250pt}%
\definecolor{currentstroke}{rgb}{0.000000,0.000000,0.000000}%
\pgfsetstrokecolor{currentstroke}%
\pgfsetdash{}{0pt}%
\pgfsys@defobject{currentmarker}{\pgfqpoint{-0.027778in}{0.000000in}}{\pgfqpoint{-0.000000in}{0.000000in}}{%
\pgfpathmoveto{\pgfqpoint{-0.000000in}{0.000000in}}%
\pgfpathlineto{\pgfqpoint{-0.027778in}{0.000000in}}%
\pgfusepath{stroke,fill}%
}%
\begin{pgfscope}%
\pgfsys@transformshift{0.588387in}{0.897761in}%
\pgfsys@useobject{currentmarker}{}%
\end{pgfscope}%
\end{pgfscope}%
\begin{pgfscope}%
\pgfsetbuttcap%
\pgfsetroundjoin%
\definecolor{currentfill}{rgb}{0.000000,0.000000,0.000000}%
\pgfsetfillcolor{currentfill}%
\pgfsetlinewidth{0.602250pt}%
\definecolor{currentstroke}{rgb}{0.000000,0.000000,0.000000}%
\pgfsetstrokecolor{currentstroke}%
\pgfsetdash{}{0pt}%
\pgfsys@defobject{currentmarker}{\pgfqpoint{-0.027778in}{0.000000in}}{\pgfqpoint{-0.000000in}{0.000000in}}{%
\pgfpathmoveto{\pgfqpoint{-0.000000in}{0.000000in}}%
\pgfpathlineto{\pgfqpoint{-0.027778in}{0.000000in}}%
\pgfusepath{stroke,fill}%
}%
\begin{pgfscope}%
\pgfsys@transformshift{0.588387in}{0.913351in}%
\pgfsys@useobject{currentmarker}{}%
\end{pgfscope}%
\end{pgfscope}%
\begin{pgfscope}%
\pgfsetbuttcap%
\pgfsetroundjoin%
\definecolor{currentfill}{rgb}{0.000000,0.000000,0.000000}%
\pgfsetfillcolor{currentfill}%
\pgfsetlinewidth{0.602250pt}%
\definecolor{currentstroke}{rgb}{0.000000,0.000000,0.000000}%
\pgfsetstrokecolor{currentstroke}%
\pgfsetdash{}{0pt}%
\pgfsys@defobject{currentmarker}{\pgfqpoint{-0.027778in}{0.000000in}}{\pgfqpoint{-0.000000in}{0.000000in}}{%
\pgfpathmoveto{\pgfqpoint{-0.000000in}{0.000000in}}%
\pgfpathlineto{\pgfqpoint{-0.027778in}{0.000000in}}%
\pgfusepath{stroke,fill}%
}%
\begin{pgfscope}%
\pgfsys@transformshift{0.588387in}{1.019042in}%
\pgfsys@useobject{currentmarker}{}%
\end{pgfscope}%
\end{pgfscope}%
\begin{pgfscope}%
\pgfsetbuttcap%
\pgfsetroundjoin%
\definecolor{currentfill}{rgb}{0.000000,0.000000,0.000000}%
\pgfsetfillcolor{currentfill}%
\pgfsetlinewidth{0.602250pt}%
\definecolor{currentstroke}{rgb}{0.000000,0.000000,0.000000}%
\pgfsetstrokecolor{currentstroke}%
\pgfsetdash{}{0pt}%
\pgfsys@defobject{currentmarker}{\pgfqpoint{-0.027778in}{0.000000in}}{\pgfqpoint{-0.000000in}{0.000000in}}{%
\pgfpathmoveto{\pgfqpoint{-0.000000in}{0.000000in}}%
\pgfpathlineto{\pgfqpoint{-0.027778in}{0.000000in}}%
\pgfusepath{stroke,fill}%
}%
\begin{pgfscope}%
\pgfsys@transformshift{0.588387in}{1.072710in}%
\pgfsys@useobject{currentmarker}{}%
\end{pgfscope}%
\end{pgfscope}%
\begin{pgfscope}%
\pgfsetbuttcap%
\pgfsetroundjoin%
\definecolor{currentfill}{rgb}{0.000000,0.000000,0.000000}%
\pgfsetfillcolor{currentfill}%
\pgfsetlinewidth{0.602250pt}%
\definecolor{currentstroke}{rgb}{0.000000,0.000000,0.000000}%
\pgfsetstrokecolor{currentstroke}%
\pgfsetdash{}{0pt}%
\pgfsys@defobject{currentmarker}{\pgfqpoint{-0.027778in}{0.000000in}}{\pgfqpoint{-0.000000in}{0.000000in}}{%
\pgfpathmoveto{\pgfqpoint{-0.000000in}{0.000000in}}%
\pgfpathlineto{\pgfqpoint{-0.027778in}{0.000000in}}%
\pgfusepath{stroke,fill}%
}%
\begin{pgfscope}%
\pgfsys@transformshift{0.588387in}{1.110788in}%
\pgfsys@useobject{currentmarker}{}%
\end{pgfscope}%
\end{pgfscope}%
\begin{pgfscope}%
\pgfsetbuttcap%
\pgfsetroundjoin%
\definecolor{currentfill}{rgb}{0.000000,0.000000,0.000000}%
\pgfsetfillcolor{currentfill}%
\pgfsetlinewidth{0.602250pt}%
\definecolor{currentstroke}{rgb}{0.000000,0.000000,0.000000}%
\pgfsetstrokecolor{currentstroke}%
\pgfsetdash{}{0pt}%
\pgfsys@defobject{currentmarker}{\pgfqpoint{-0.027778in}{0.000000in}}{\pgfqpoint{-0.000000in}{0.000000in}}{%
\pgfpathmoveto{\pgfqpoint{-0.000000in}{0.000000in}}%
\pgfpathlineto{\pgfqpoint{-0.027778in}{0.000000in}}%
\pgfusepath{stroke,fill}%
}%
\begin{pgfscope}%
\pgfsys@transformshift{0.588387in}{1.140323in}%
\pgfsys@useobject{currentmarker}{}%
\end{pgfscope}%
\end{pgfscope}%
\begin{pgfscope}%
\pgfsetbuttcap%
\pgfsetroundjoin%
\definecolor{currentfill}{rgb}{0.000000,0.000000,0.000000}%
\pgfsetfillcolor{currentfill}%
\pgfsetlinewidth{0.602250pt}%
\definecolor{currentstroke}{rgb}{0.000000,0.000000,0.000000}%
\pgfsetstrokecolor{currentstroke}%
\pgfsetdash{}{0pt}%
\pgfsys@defobject{currentmarker}{\pgfqpoint{-0.027778in}{0.000000in}}{\pgfqpoint{-0.000000in}{0.000000in}}{%
\pgfpathmoveto{\pgfqpoint{-0.000000in}{0.000000in}}%
\pgfpathlineto{\pgfqpoint{-0.027778in}{0.000000in}}%
\pgfusepath{stroke,fill}%
}%
\begin{pgfscope}%
\pgfsys@transformshift{0.588387in}{1.164455in}%
\pgfsys@useobject{currentmarker}{}%
\end{pgfscope}%
\end{pgfscope}%
\begin{pgfscope}%
\pgfsetbuttcap%
\pgfsetroundjoin%
\definecolor{currentfill}{rgb}{0.000000,0.000000,0.000000}%
\pgfsetfillcolor{currentfill}%
\pgfsetlinewidth{0.602250pt}%
\definecolor{currentstroke}{rgb}{0.000000,0.000000,0.000000}%
\pgfsetstrokecolor{currentstroke}%
\pgfsetdash{}{0pt}%
\pgfsys@defobject{currentmarker}{\pgfqpoint{-0.027778in}{0.000000in}}{\pgfqpoint{-0.000000in}{0.000000in}}{%
\pgfpathmoveto{\pgfqpoint{-0.000000in}{0.000000in}}%
\pgfpathlineto{\pgfqpoint{-0.027778in}{0.000000in}}%
\pgfusepath{stroke,fill}%
}%
\begin{pgfscope}%
\pgfsys@transformshift{0.588387in}{1.184859in}%
\pgfsys@useobject{currentmarker}{}%
\end{pgfscope}%
\end{pgfscope}%
\begin{pgfscope}%
\pgfsetbuttcap%
\pgfsetroundjoin%
\definecolor{currentfill}{rgb}{0.000000,0.000000,0.000000}%
\pgfsetfillcolor{currentfill}%
\pgfsetlinewidth{0.602250pt}%
\definecolor{currentstroke}{rgb}{0.000000,0.000000,0.000000}%
\pgfsetstrokecolor{currentstroke}%
\pgfsetdash{}{0pt}%
\pgfsys@defobject{currentmarker}{\pgfqpoint{-0.027778in}{0.000000in}}{\pgfqpoint{-0.000000in}{0.000000in}}{%
\pgfpathmoveto{\pgfqpoint{-0.000000in}{0.000000in}}%
\pgfpathlineto{\pgfqpoint{-0.027778in}{0.000000in}}%
\pgfusepath{stroke,fill}%
}%
\begin{pgfscope}%
\pgfsys@transformshift{0.588387in}{1.202533in}%
\pgfsys@useobject{currentmarker}{}%
\end{pgfscope}%
\end{pgfscope}%
\begin{pgfscope}%
\pgfsetbuttcap%
\pgfsetroundjoin%
\definecolor{currentfill}{rgb}{0.000000,0.000000,0.000000}%
\pgfsetfillcolor{currentfill}%
\pgfsetlinewidth{0.602250pt}%
\definecolor{currentstroke}{rgb}{0.000000,0.000000,0.000000}%
\pgfsetstrokecolor{currentstroke}%
\pgfsetdash{}{0pt}%
\pgfsys@defobject{currentmarker}{\pgfqpoint{-0.027778in}{0.000000in}}{\pgfqpoint{-0.000000in}{0.000000in}}{%
\pgfpathmoveto{\pgfqpoint{-0.000000in}{0.000000in}}%
\pgfpathlineto{\pgfqpoint{-0.027778in}{0.000000in}}%
\pgfusepath{stroke,fill}%
}%
\begin{pgfscope}%
\pgfsys@transformshift{0.588387in}{1.218123in}%
\pgfsys@useobject{currentmarker}{}%
\end{pgfscope}%
\end{pgfscope}%
\begin{pgfscope}%
\pgfsetbuttcap%
\pgfsetroundjoin%
\definecolor{currentfill}{rgb}{0.000000,0.000000,0.000000}%
\pgfsetfillcolor{currentfill}%
\pgfsetlinewidth{0.602250pt}%
\definecolor{currentstroke}{rgb}{0.000000,0.000000,0.000000}%
\pgfsetstrokecolor{currentstroke}%
\pgfsetdash{}{0pt}%
\pgfsys@defobject{currentmarker}{\pgfqpoint{-0.027778in}{0.000000in}}{\pgfqpoint{-0.000000in}{0.000000in}}{%
\pgfpathmoveto{\pgfqpoint{-0.000000in}{0.000000in}}%
\pgfpathlineto{\pgfqpoint{-0.027778in}{0.000000in}}%
\pgfusepath{stroke,fill}%
}%
\begin{pgfscope}%
\pgfsys@transformshift{0.588387in}{1.323815in}%
\pgfsys@useobject{currentmarker}{}%
\end{pgfscope}%
\end{pgfscope}%
\begin{pgfscope}%
\pgfsetbuttcap%
\pgfsetroundjoin%
\definecolor{currentfill}{rgb}{0.000000,0.000000,0.000000}%
\pgfsetfillcolor{currentfill}%
\pgfsetlinewidth{0.602250pt}%
\definecolor{currentstroke}{rgb}{0.000000,0.000000,0.000000}%
\pgfsetstrokecolor{currentstroke}%
\pgfsetdash{}{0pt}%
\pgfsys@defobject{currentmarker}{\pgfqpoint{-0.027778in}{0.000000in}}{\pgfqpoint{-0.000000in}{0.000000in}}{%
\pgfpathmoveto{\pgfqpoint{-0.000000in}{0.000000in}}%
\pgfpathlineto{\pgfqpoint{-0.027778in}{0.000000in}}%
\pgfusepath{stroke,fill}%
}%
\begin{pgfscope}%
\pgfsys@transformshift{0.588387in}{1.377482in}%
\pgfsys@useobject{currentmarker}{}%
\end{pgfscope}%
\end{pgfscope}%
\begin{pgfscope}%
\pgfsetbuttcap%
\pgfsetroundjoin%
\definecolor{currentfill}{rgb}{0.000000,0.000000,0.000000}%
\pgfsetfillcolor{currentfill}%
\pgfsetlinewidth{0.602250pt}%
\definecolor{currentstroke}{rgb}{0.000000,0.000000,0.000000}%
\pgfsetstrokecolor{currentstroke}%
\pgfsetdash{}{0pt}%
\pgfsys@defobject{currentmarker}{\pgfqpoint{-0.027778in}{0.000000in}}{\pgfqpoint{-0.000000in}{0.000000in}}{%
\pgfpathmoveto{\pgfqpoint{-0.000000in}{0.000000in}}%
\pgfpathlineto{\pgfqpoint{-0.027778in}{0.000000in}}%
\pgfusepath{stroke,fill}%
}%
\begin{pgfscope}%
\pgfsys@transformshift{0.588387in}{1.415560in}%
\pgfsys@useobject{currentmarker}{}%
\end{pgfscope}%
\end{pgfscope}%
\begin{pgfscope}%
\pgfsetbuttcap%
\pgfsetroundjoin%
\definecolor{currentfill}{rgb}{0.000000,0.000000,0.000000}%
\pgfsetfillcolor{currentfill}%
\pgfsetlinewidth{0.602250pt}%
\definecolor{currentstroke}{rgb}{0.000000,0.000000,0.000000}%
\pgfsetstrokecolor{currentstroke}%
\pgfsetdash{}{0pt}%
\pgfsys@defobject{currentmarker}{\pgfqpoint{-0.027778in}{0.000000in}}{\pgfqpoint{-0.000000in}{0.000000in}}{%
\pgfpathmoveto{\pgfqpoint{-0.000000in}{0.000000in}}%
\pgfpathlineto{\pgfqpoint{-0.027778in}{0.000000in}}%
\pgfusepath{stroke,fill}%
}%
\begin{pgfscope}%
\pgfsys@transformshift{0.588387in}{1.445096in}%
\pgfsys@useobject{currentmarker}{}%
\end{pgfscope}%
\end{pgfscope}%
\begin{pgfscope}%
\pgfsetbuttcap%
\pgfsetroundjoin%
\definecolor{currentfill}{rgb}{0.000000,0.000000,0.000000}%
\pgfsetfillcolor{currentfill}%
\pgfsetlinewidth{0.602250pt}%
\definecolor{currentstroke}{rgb}{0.000000,0.000000,0.000000}%
\pgfsetstrokecolor{currentstroke}%
\pgfsetdash{}{0pt}%
\pgfsys@defobject{currentmarker}{\pgfqpoint{-0.027778in}{0.000000in}}{\pgfqpoint{-0.000000in}{0.000000in}}{%
\pgfpathmoveto{\pgfqpoint{-0.000000in}{0.000000in}}%
\pgfpathlineto{\pgfqpoint{-0.027778in}{0.000000in}}%
\pgfusepath{stroke,fill}%
}%
\begin{pgfscope}%
\pgfsys@transformshift{0.588387in}{1.469228in}%
\pgfsys@useobject{currentmarker}{}%
\end{pgfscope}%
\end{pgfscope}%
\begin{pgfscope}%
\pgfsetbuttcap%
\pgfsetroundjoin%
\definecolor{currentfill}{rgb}{0.000000,0.000000,0.000000}%
\pgfsetfillcolor{currentfill}%
\pgfsetlinewidth{0.602250pt}%
\definecolor{currentstroke}{rgb}{0.000000,0.000000,0.000000}%
\pgfsetstrokecolor{currentstroke}%
\pgfsetdash{}{0pt}%
\pgfsys@defobject{currentmarker}{\pgfqpoint{-0.027778in}{0.000000in}}{\pgfqpoint{-0.000000in}{0.000000in}}{%
\pgfpathmoveto{\pgfqpoint{-0.000000in}{0.000000in}}%
\pgfpathlineto{\pgfqpoint{-0.027778in}{0.000000in}}%
\pgfusepath{stroke,fill}%
}%
\begin{pgfscope}%
\pgfsys@transformshift{0.588387in}{1.489632in}%
\pgfsys@useobject{currentmarker}{}%
\end{pgfscope}%
\end{pgfscope}%
\begin{pgfscope}%
\pgfsetbuttcap%
\pgfsetroundjoin%
\definecolor{currentfill}{rgb}{0.000000,0.000000,0.000000}%
\pgfsetfillcolor{currentfill}%
\pgfsetlinewidth{0.602250pt}%
\definecolor{currentstroke}{rgb}{0.000000,0.000000,0.000000}%
\pgfsetstrokecolor{currentstroke}%
\pgfsetdash{}{0pt}%
\pgfsys@defobject{currentmarker}{\pgfqpoint{-0.027778in}{0.000000in}}{\pgfqpoint{-0.000000in}{0.000000in}}{%
\pgfpathmoveto{\pgfqpoint{-0.000000in}{0.000000in}}%
\pgfpathlineto{\pgfqpoint{-0.027778in}{0.000000in}}%
\pgfusepath{stroke,fill}%
}%
\begin{pgfscope}%
\pgfsys@transformshift{0.588387in}{1.507306in}%
\pgfsys@useobject{currentmarker}{}%
\end{pgfscope}%
\end{pgfscope}%
\begin{pgfscope}%
\pgfsetbuttcap%
\pgfsetroundjoin%
\definecolor{currentfill}{rgb}{0.000000,0.000000,0.000000}%
\pgfsetfillcolor{currentfill}%
\pgfsetlinewidth{0.602250pt}%
\definecolor{currentstroke}{rgb}{0.000000,0.000000,0.000000}%
\pgfsetstrokecolor{currentstroke}%
\pgfsetdash{}{0pt}%
\pgfsys@defobject{currentmarker}{\pgfqpoint{-0.027778in}{0.000000in}}{\pgfqpoint{-0.000000in}{0.000000in}}{%
\pgfpathmoveto{\pgfqpoint{-0.000000in}{0.000000in}}%
\pgfpathlineto{\pgfqpoint{-0.027778in}{0.000000in}}%
\pgfusepath{stroke,fill}%
}%
\begin{pgfscope}%
\pgfsys@transformshift{0.588387in}{1.522896in}%
\pgfsys@useobject{currentmarker}{}%
\end{pgfscope}%
\end{pgfscope}%
\begin{pgfscope}%
\pgfsetbuttcap%
\pgfsetroundjoin%
\definecolor{currentfill}{rgb}{0.000000,0.000000,0.000000}%
\pgfsetfillcolor{currentfill}%
\pgfsetlinewidth{0.602250pt}%
\definecolor{currentstroke}{rgb}{0.000000,0.000000,0.000000}%
\pgfsetstrokecolor{currentstroke}%
\pgfsetdash{}{0pt}%
\pgfsys@defobject{currentmarker}{\pgfqpoint{-0.027778in}{0.000000in}}{\pgfqpoint{-0.000000in}{0.000000in}}{%
\pgfpathmoveto{\pgfqpoint{-0.000000in}{0.000000in}}%
\pgfpathlineto{\pgfqpoint{-0.027778in}{0.000000in}}%
\pgfusepath{stroke,fill}%
}%
\begin{pgfscope}%
\pgfsys@transformshift{0.588387in}{1.628587in}%
\pgfsys@useobject{currentmarker}{}%
\end{pgfscope}%
\end{pgfscope}%
\begin{pgfscope}%
\pgfsetbuttcap%
\pgfsetroundjoin%
\definecolor{currentfill}{rgb}{0.000000,0.000000,0.000000}%
\pgfsetfillcolor{currentfill}%
\pgfsetlinewidth{0.602250pt}%
\definecolor{currentstroke}{rgb}{0.000000,0.000000,0.000000}%
\pgfsetstrokecolor{currentstroke}%
\pgfsetdash{}{0pt}%
\pgfsys@defobject{currentmarker}{\pgfqpoint{-0.027778in}{0.000000in}}{\pgfqpoint{-0.000000in}{0.000000in}}{%
\pgfpathmoveto{\pgfqpoint{-0.000000in}{0.000000in}}%
\pgfpathlineto{\pgfqpoint{-0.027778in}{0.000000in}}%
\pgfusepath{stroke,fill}%
}%
\begin{pgfscope}%
\pgfsys@transformshift{0.588387in}{1.682255in}%
\pgfsys@useobject{currentmarker}{}%
\end{pgfscope}%
\end{pgfscope}%
\begin{pgfscope}%
\pgfsetbuttcap%
\pgfsetroundjoin%
\definecolor{currentfill}{rgb}{0.000000,0.000000,0.000000}%
\pgfsetfillcolor{currentfill}%
\pgfsetlinewidth{0.602250pt}%
\definecolor{currentstroke}{rgb}{0.000000,0.000000,0.000000}%
\pgfsetstrokecolor{currentstroke}%
\pgfsetdash{}{0pt}%
\pgfsys@defobject{currentmarker}{\pgfqpoint{-0.027778in}{0.000000in}}{\pgfqpoint{-0.000000in}{0.000000in}}{%
\pgfpathmoveto{\pgfqpoint{-0.000000in}{0.000000in}}%
\pgfpathlineto{\pgfqpoint{-0.027778in}{0.000000in}}%
\pgfusepath{stroke,fill}%
}%
\begin{pgfscope}%
\pgfsys@transformshift{0.588387in}{1.720333in}%
\pgfsys@useobject{currentmarker}{}%
\end{pgfscope}%
\end{pgfscope}%
\begin{pgfscope}%
\pgfsetbuttcap%
\pgfsetroundjoin%
\definecolor{currentfill}{rgb}{0.000000,0.000000,0.000000}%
\pgfsetfillcolor{currentfill}%
\pgfsetlinewidth{0.602250pt}%
\definecolor{currentstroke}{rgb}{0.000000,0.000000,0.000000}%
\pgfsetstrokecolor{currentstroke}%
\pgfsetdash{}{0pt}%
\pgfsys@defobject{currentmarker}{\pgfqpoint{-0.027778in}{0.000000in}}{\pgfqpoint{-0.000000in}{0.000000in}}{%
\pgfpathmoveto{\pgfqpoint{-0.000000in}{0.000000in}}%
\pgfpathlineto{\pgfqpoint{-0.027778in}{0.000000in}}%
\pgfusepath{stroke,fill}%
}%
\begin{pgfscope}%
\pgfsys@transformshift{0.588387in}{1.749868in}%
\pgfsys@useobject{currentmarker}{}%
\end{pgfscope}%
\end{pgfscope}%
\begin{pgfscope}%
\pgfsetbuttcap%
\pgfsetroundjoin%
\definecolor{currentfill}{rgb}{0.000000,0.000000,0.000000}%
\pgfsetfillcolor{currentfill}%
\pgfsetlinewidth{0.602250pt}%
\definecolor{currentstroke}{rgb}{0.000000,0.000000,0.000000}%
\pgfsetstrokecolor{currentstroke}%
\pgfsetdash{}{0pt}%
\pgfsys@defobject{currentmarker}{\pgfqpoint{-0.027778in}{0.000000in}}{\pgfqpoint{-0.000000in}{0.000000in}}{%
\pgfpathmoveto{\pgfqpoint{-0.000000in}{0.000000in}}%
\pgfpathlineto{\pgfqpoint{-0.027778in}{0.000000in}}%
\pgfusepath{stroke,fill}%
}%
\begin{pgfscope}%
\pgfsys@transformshift{0.588387in}{1.774001in}%
\pgfsys@useobject{currentmarker}{}%
\end{pgfscope}%
\end{pgfscope}%
\begin{pgfscope}%
\pgfsetbuttcap%
\pgfsetroundjoin%
\definecolor{currentfill}{rgb}{0.000000,0.000000,0.000000}%
\pgfsetfillcolor{currentfill}%
\pgfsetlinewidth{0.602250pt}%
\definecolor{currentstroke}{rgb}{0.000000,0.000000,0.000000}%
\pgfsetstrokecolor{currentstroke}%
\pgfsetdash{}{0pt}%
\pgfsys@defobject{currentmarker}{\pgfqpoint{-0.027778in}{0.000000in}}{\pgfqpoint{-0.000000in}{0.000000in}}{%
\pgfpathmoveto{\pgfqpoint{-0.000000in}{0.000000in}}%
\pgfpathlineto{\pgfqpoint{-0.027778in}{0.000000in}}%
\pgfusepath{stroke,fill}%
}%
\begin{pgfscope}%
\pgfsys@transformshift{0.588387in}{1.794404in}%
\pgfsys@useobject{currentmarker}{}%
\end{pgfscope}%
\end{pgfscope}%
\begin{pgfscope}%
\pgfsetbuttcap%
\pgfsetroundjoin%
\definecolor{currentfill}{rgb}{0.000000,0.000000,0.000000}%
\pgfsetfillcolor{currentfill}%
\pgfsetlinewidth{0.602250pt}%
\definecolor{currentstroke}{rgb}{0.000000,0.000000,0.000000}%
\pgfsetstrokecolor{currentstroke}%
\pgfsetdash{}{0pt}%
\pgfsys@defobject{currentmarker}{\pgfqpoint{-0.027778in}{0.000000in}}{\pgfqpoint{-0.000000in}{0.000000in}}{%
\pgfpathmoveto{\pgfqpoint{-0.000000in}{0.000000in}}%
\pgfpathlineto{\pgfqpoint{-0.027778in}{0.000000in}}%
\pgfusepath{stroke,fill}%
}%
\begin{pgfscope}%
\pgfsys@transformshift{0.588387in}{1.812079in}%
\pgfsys@useobject{currentmarker}{}%
\end{pgfscope}%
\end{pgfscope}%
\begin{pgfscope}%
\pgfsetbuttcap%
\pgfsetroundjoin%
\definecolor{currentfill}{rgb}{0.000000,0.000000,0.000000}%
\pgfsetfillcolor{currentfill}%
\pgfsetlinewidth{0.602250pt}%
\definecolor{currentstroke}{rgb}{0.000000,0.000000,0.000000}%
\pgfsetstrokecolor{currentstroke}%
\pgfsetdash{}{0pt}%
\pgfsys@defobject{currentmarker}{\pgfqpoint{-0.027778in}{0.000000in}}{\pgfqpoint{-0.000000in}{0.000000in}}{%
\pgfpathmoveto{\pgfqpoint{-0.000000in}{0.000000in}}%
\pgfpathlineto{\pgfqpoint{-0.027778in}{0.000000in}}%
\pgfusepath{stroke,fill}%
}%
\begin{pgfscope}%
\pgfsys@transformshift{0.588387in}{1.827668in}%
\pgfsys@useobject{currentmarker}{}%
\end{pgfscope}%
\end{pgfscope}%
\begin{pgfscope}%
\pgfsetbuttcap%
\pgfsetroundjoin%
\definecolor{currentfill}{rgb}{0.000000,0.000000,0.000000}%
\pgfsetfillcolor{currentfill}%
\pgfsetlinewidth{0.602250pt}%
\definecolor{currentstroke}{rgb}{0.000000,0.000000,0.000000}%
\pgfsetstrokecolor{currentstroke}%
\pgfsetdash{}{0pt}%
\pgfsys@defobject{currentmarker}{\pgfqpoint{-0.027778in}{0.000000in}}{\pgfqpoint{-0.000000in}{0.000000in}}{%
\pgfpathmoveto{\pgfqpoint{-0.000000in}{0.000000in}}%
\pgfpathlineto{\pgfqpoint{-0.027778in}{0.000000in}}%
\pgfusepath{stroke,fill}%
}%
\begin{pgfscope}%
\pgfsys@transformshift{0.588387in}{1.933360in}%
\pgfsys@useobject{currentmarker}{}%
\end{pgfscope}%
\end{pgfscope}%
\begin{pgfscope}%
\pgfsetbuttcap%
\pgfsetroundjoin%
\definecolor{currentfill}{rgb}{0.000000,0.000000,0.000000}%
\pgfsetfillcolor{currentfill}%
\pgfsetlinewidth{0.602250pt}%
\definecolor{currentstroke}{rgb}{0.000000,0.000000,0.000000}%
\pgfsetstrokecolor{currentstroke}%
\pgfsetdash{}{0pt}%
\pgfsys@defobject{currentmarker}{\pgfqpoint{-0.027778in}{0.000000in}}{\pgfqpoint{-0.000000in}{0.000000in}}{%
\pgfpathmoveto{\pgfqpoint{-0.000000in}{0.000000in}}%
\pgfpathlineto{\pgfqpoint{-0.027778in}{0.000000in}}%
\pgfusepath{stroke,fill}%
}%
\begin{pgfscope}%
\pgfsys@transformshift{0.588387in}{1.987028in}%
\pgfsys@useobject{currentmarker}{}%
\end{pgfscope}%
\end{pgfscope}%
\begin{pgfscope}%
\pgfsetbuttcap%
\pgfsetroundjoin%
\definecolor{currentfill}{rgb}{0.000000,0.000000,0.000000}%
\pgfsetfillcolor{currentfill}%
\pgfsetlinewidth{0.602250pt}%
\definecolor{currentstroke}{rgb}{0.000000,0.000000,0.000000}%
\pgfsetstrokecolor{currentstroke}%
\pgfsetdash{}{0pt}%
\pgfsys@defobject{currentmarker}{\pgfqpoint{-0.027778in}{0.000000in}}{\pgfqpoint{-0.000000in}{0.000000in}}{%
\pgfpathmoveto{\pgfqpoint{-0.000000in}{0.000000in}}%
\pgfpathlineto{\pgfqpoint{-0.027778in}{0.000000in}}%
\pgfusepath{stroke,fill}%
}%
\begin{pgfscope}%
\pgfsys@transformshift{0.588387in}{2.025105in}%
\pgfsys@useobject{currentmarker}{}%
\end{pgfscope}%
\end{pgfscope}%
\begin{pgfscope}%
\pgfsetbuttcap%
\pgfsetroundjoin%
\definecolor{currentfill}{rgb}{0.000000,0.000000,0.000000}%
\pgfsetfillcolor{currentfill}%
\pgfsetlinewidth{0.602250pt}%
\definecolor{currentstroke}{rgb}{0.000000,0.000000,0.000000}%
\pgfsetstrokecolor{currentstroke}%
\pgfsetdash{}{0pt}%
\pgfsys@defobject{currentmarker}{\pgfqpoint{-0.027778in}{0.000000in}}{\pgfqpoint{-0.000000in}{0.000000in}}{%
\pgfpathmoveto{\pgfqpoint{-0.000000in}{0.000000in}}%
\pgfpathlineto{\pgfqpoint{-0.027778in}{0.000000in}}%
\pgfusepath{stroke,fill}%
}%
\begin{pgfscope}%
\pgfsys@transformshift{0.588387in}{2.054641in}%
\pgfsys@useobject{currentmarker}{}%
\end{pgfscope}%
\end{pgfscope}%
\begin{pgfscope}%
\pgfsetbuttcap%
\pgfsetroundjoin%
\definecolor{currentfill}{rgb}{0.000000,0.000000,0.000000}%
\pgfsetfillcolor{currentfill}%
\pgfsetlinewidth{0.602250pt}%
\definecolor{currentstroke}{rgb}{0.000000,0.000000,0.000000}%
\pgfsetstrokecolor{currentstroke}%
\pgfsetdash{}{0pt}%
\pgfsys@defobject{currentmarker}{\pgfqpoint{-0.027778in}{0.000000in}}{\pgfqpoint{-0.000000in}{0.000000in}}{%
\pgfpathmoveto{\pgfqpoint{-0.000000in}{0.000000in}}%
\pgfpathlineto{\pgfqpoint{-0.027778in}{0.000000in}}%
\pgfusepath{stroke,fill}%
}%
\begin{pgfscope}%
\pgfsys@transformshift{0.588387in}{2.078773in}%
\pgfsys@useobject{currentmarker}{}%
\end{pgfscope}%
\end{pgfscope}%
\begin{pgfscope}%
\pgfsetbuttcap%
\pgfsetroundjoin%
\definecolor{currentfill}{rgb}{0.000000,0.000000,0.000000}%
\pgfsetfillcolor{currentfill}%
\pgfsetlinewidth{0.602250pt}%
\definecolor{currentstroke}{rgb}{0.000000,0.000000,0.000000}%
\pgfsetstrokecolor{currentstroke}%
\pgfsetdash{}{0pt}%
\pgfsys@defobject{currentmarker}{\pgfqpoint{-0.027778in}{0.000000in}}{\pgfqpoint{-0.000000in}{0.000000in}}{%
\pgfpathmoveto{\pgfqpoint{-0.000000in}{0.000000in}}%
\pgfpathlineto{\pgfqpoint{-0.027778in}{0.000000in}}%
\pgfusepath{stroke,fill}%
}%
\begin{pgfscope}%
\pgfsys@transformshift{0.588387in}{2.099177in}%
\pgfsys@useobject{currentmarker}{}%
\end{pgfscope}%
\end{pgfscope}%
\begin{pgfscope}%
\pgfsetbuttcap%
\pgfsetroundjoin%
\definecolor{currentfill}{rgb}{0.000000,0.000000,0.000000}%
\pgfsetfillcolor{currentfill}%
\pgfsetlinewidth{0.602250pt}%
\definecolor{currentstroke}{rgb}{0.000000,0.000000,0.000000}%
\pgfsetstrokecolor{currentstroke}%
\pgfsetdash{}{0pt}%
\pgfsys@defobject{currentmarker}{\pgfqpoint{-0.027778in}{0.000000in}}{\pgfqpoint{-0.000000in}{0.000000in}}{%
\pgfpathmoveto{\pgfqpoint{-0.000000in}{0.000000in}}%
\pgfpathlineto{\pgfqpoint{-0.027778in}{0.000000in}}%
\pgfusepath{stroke,fill}%
}%
\begin{pgfscope}%
\pgfsys@transformshift{0.588387in}{2.116851in}%
\pgfsys@useobject{currentmarker}{}%
\end{pgfscope}%
\end{pgfscope}%
\begin{pgfscope}%
\pgfsetbuttcap%
\pgfsetroundjoin%
\definecolor{currentfill}{rgb}{0.000000,0.000000,0.000000}%
\pgfsetfillcolor{currentfill}%
\pgfsetlinewidth{0.602250pt}%
\definecolor{currentstroke}{rgb}{0.000000,0.000000,0.000000}%
\pgfsetstrokecolor{currentstroke}%
\pgfsetdash{}{0pt}%
\pgfsys@defobject{currentmarker}{\pgfqpoint{-0.027778in}{0.000000in}}{\pgfqpoint{-0.000000in}{0.000000in}}{%
\pgfpathmoveto{\pgfqpoint{-0.000000in}{0.000000in}}%
\pgfpathlineto{\pgfqpoint{-0.027778in}{0.000000in}}%
\pgfusepath{stroke,fill}%
}%
\begin{pgfscope}%
\pgfsys@transformshift{0.588387in}{2.132441in}%
\pgfsys@useobject{currentmarker}{}%
\end{pgfscope}%
\end{pgfscope}%
\begin{pgfscope}%
\pgfsetbuttcap%
\pgfsetroundjoin%
\definecolor{currentfill}{rgb}{0.000000,0.000000,0.000000}%
\pgfsetfillcolor{currentfill}%
\pgfsetlinewidth{0.602250pt}%
\definecolor{currentstroke}{rgb}{0.000000,0.000000,0.000000}%
\pgfsetstrokecolor{currentstroke}%
\pgfsetdash{}{0pt}%
\pgfsys@defobject{currentmarker}{\pgfqpoint{-0.027778in}{0.000000in}}{\pgfqpoint{-0.000000in}{0.000000in}}{%
\pgfpathmoveto{\pgfqpoint{-0.000000in}{0.000000in}}%
\pgfpathlineto{\pgfqpoint{-0.027778in}{0.000000in}}%
\pgfusepath{stroke,fill}%
}%
\begin{pgfscope}%
\pgfsys@transformshift{0.588387in}{2.238132in}%
\pgfsys@useobject{currentmarker}{}%
\end{pgfscope}%
\end{pgfscope}%
\begin{pgfscope}%
\pgfsetbuttcap%
\pgfsetroundjoin%
\definecolor{currentfill}{rgb}{0.000000,0.000000,0.000000}%
\pgfsetfillcolor{currentfill}%
\pgfsetlinewidth{0.602250pt}%
\definecolor{currentstroke}{rgb}{0.000000,0.000000,0.000000}%
\pgfsetstrokecolor{currentstroke}%
\pgfsetdash{}{0pt}%
\pgfsys@defobject{currentmarker}{\pgfqpoint{-0.027778in}{0.000000in}}{\pgfqpoint{-0.000000in}{0.000000in}}{%
\pgfpathmoveto{\pgfqpoint{-0.000000in}{0.000000in}}%
\pgfpathlineto{\pgfqpoint{-0.027778in}{0.000000in}}%
\pgfusepath{stroke,fill}%
}%
\begin{pgfscope}%
\pgfsys@transformshift{0.588387in}{2.291800in}%
\pgfsys@useobject{currentmarker}{}%
\end{pgfscope}%
\end{pgfscope}%
\begin{pgfscope}%
\pgfsetbuttcap%
\pgfsetroundjoin%
\definecolor{currentfill}{rgb}{0.000000,0.000000,0.000000}%
\pgfsetfillcolor{currentfill}%
\pgfsetlinewidth{0.602250pt}%
\definecolor{currentstroke}{rgb}{0.000000,0.000000,0.000000}%
\pgfsetstrokecolor{currentstroke}%
\pgfsetdash{}{0pt}%
\pgfsys@defobject{currentmarker}{\pgfqpoint{-0.027778in}{0.000000in}}{\pgfqpoint{-0.000000in}{0.000000in}}{%
\pgfpathmoveto{\pgfqpoint{-0.000000in}{0.000000in}}%
\pgfpathlineto{\pgfqpoint{-0.027778in}{0.000000in}}%
\pgfusepath{stroke,fill}%
}%
\begin{pgfscope}%
\pgfsys@transformshift{0.588387in}{2.329878in}%
\pgfsys@useobject{currentmarker}{}%
\end{pgfscope}%
\end{pgfscope}%
\begin{pgfscope}%
\pgfsetbuttcap%
\pgfsetroundjoin%
\definecolor{currentfill}{rgb}{0.000000,0.000000,0.000000}%
\pgfsetfillcolor{currentfill}%
\pgfsetlinewidth{0.602250pt}%
\definecolor{currentstroke}{rgb}{0.000000,0.000000,0.000000}%
\pgfsetstrokecolor{currentstroke}%
\pgfsetdash{}{0pt}%
\pgfsys@defobject{currentmarker}{\pgfqpoint{-0.027778in}{0.000000in}}{\pgfqpoint{-0.000000in}{0.000000in}}{%
\pgfpathmoveto{\pgfqpoint{-0.000000in}{0.000000in}}%
\pgfpathlineto{\pgfqpoint{-0.027778in}{0.000000in}}%
\pgfusepath{stroke,fill}%
}%
\begin{pgfscope}%
\pgfsys@transformshift{0.588387in}{2.359414in}%
\pgfsys@useobject{currentmarker}{}%
\end{pgfscope}%
\end{pgfscope}%
\begin{pgfscope}%
\pgfsetbuttcap%
\pgfsetroundjoin%
\definecolor{currentfill}{rgb}{0.000000,0.000000,0.000000}%
\pgfsetfillcolor{currentfill}%
\pgfsetlinewidth{0.602250pt}%
\definecolor{currentstroke}{rgb}{0.000000,0.000000,0.000000}%
\pgfsetstrokecolor{currentstroke}%
\pgfsetdash{}{0pt}%
\pgfsys@defobject{currentmarker}{\pgfqpoint{-0.027778in}{0.000000in}}{\pgfqpoint{-0.000000in}{0.000000in}}{%
\pgfpathmoveto{\pgfqpoint{-0.000000in}{0.000000in}}%
\pgfpathlineto{\pgfqpoint{-0.027778in}{0.000000in}}%
\pgfusepath{stroke,fill}%
}%
\begin{pgfscope}%
\pgfsys@transformshift{0.588387in}{2.383546in}%
\pgfsys@useobject{currentmarker}{}%
\end{pgfscope}%
\end{pgfscope}%
\begin{pgfscope}%
\pgfsetbuttcap%
\pgfsetroundjoin%
\definecolor{currentfill}{rgb}{0.000000,0.000000,0.000000}%
\pgfsetfillcolor{currentfill}%
\pgfsetlinewidth{0.602250pt}%
\definecolor{currentstroke}{rgb}{0.000000,0.000000,0.000000}%
\pgfsetstrokecolor{currentstroke}%
\pgfsetdash{}{0pt}%
\pgfsys@defobject{currentmarker}{\pgfqpoint{-0.027778in}{0.000000in}}{\pgfqpoint{-0.000000in}{0.000000in}}{%
\pgfpathmoveto{\pgfqpoint{-0.000000in}{0.000000in}}%
\pgfpathlineto{\pgfqpoint{-0.027778in}{0.000000in}}%
\pgfusepath{stroke,fill}%
}%
\begin{pgfscope}%
\pgfsys@transformshift{0.588387in}{2.403949in}%
\pgfsys@useobject{currentmarker}{}%
\end{pgfscope}%
\end{pgfscope}%
\begin{pgfscope}%
\pgfsetbuttcap%
\pgfsetroundjoin%
\definecolor{currentfill}{rgb}{0.000000,0.000000,0.000000}%
\pgfsetfillcolor{currentfill}%
\pgfsetlinewidth{0.602250pt}%
\definecolor{currentstroke}{rgb}{0.000000,0.000000,0.000000}%
\pgfsetstrokecolor{currentstroke}%
\pgfsetdash{}{0pt}%
\pgfsys@defobject{currentmarker}{\pgfqpoint{-0.027778in}{0.000000in}}{\pgfqpoint{-0.000000in}{0.000000in}}{%
\pgfpathmoveto{\pgfqpoint{-0.000000in}{0.000000in}}%
\pgfpathlineto{\pgfqpoint{-0.027778in}{0.000000in}}%
\pgfusepath{stroke,fill}%
}%
\begin{pgfscope}%
\pgfsys@transformshift{0.588387in}{2.421624in}%
\pgfsys@useobject{currentmarker}{}%
\end{pgfscope}%
\end{pgfscope}%
\begin{pgfscope}%
\pgfsetbuttcap%
\pgfsetroundjoin%
\definecolor{currentfill}{rgb}{0.000000,0.000000,0.000000}%
\pgfsetfillcolor{currentfill}%
\pgfsetlinewidth{0.602250pt}%
\definecolor{currentstroke}{rgb}{0.000000,0.000000,0.000000}%
\pgfsetstrokecolor{currentstroke}%
\pgfsetdash{}{0pt}%
\pgfsys@defobject{currentmarker}{\pgfqpoint{-0.027778in}{0.000000in}}{\pgfqpoint{-0.000000in}{0.000000in}}{%
\pgfpathmoveto{\pgfqpoint{-0.000000in}{0.000000in}}%
\pgfpathlineto{\pgfqpoint{-0.027778in}{0.000000in}}%
\pgfusepath{stroke,fill}%
}%
\begin{pgfscope}%
\pgfsys@transformshift{0.588387in}{2.437214in}%
\pgfsys@useobject{currentmarker}{}%
\end{pgfscope}%
\end{pgfscope}%
\begin{pgfscope}%
\pgfsetbuttcap%
\pgfsetroundjoin%
\definecolor{currentfill}{rgb}{0.000000,0.000000,0.000000}%
\pgfsetfillcolor{currentfill}%
\pgfsetlinewidth{0.602250pt}%
\definecolor{currentstroke}{rgb}{0.000000,0.000000,0.000000}%
\pgfsetstrokecolor{currentstroke}%
\pgfsetdash{}{0pt}%
\pgfsys@defobject{currentmarker}{\pgfqpoint{-0.027778in}{0.000000in}}{\pgfqpoint{-0.000000in}{0.000000in}}{%
\pgfpathmoveto{\pgfqpoint{-0.000000in}{0.000000in}}%
\pgfpathlineto{\pgfqpoint{-0.027778in}{0.000000in}}%
\pgfusepath{stroke,fill}%
}%
\begin{pgfscope}%
\pgfsys@transformshift{0.588387in}{2.542905in}%
\pgfsys@useobject{currentmarker}{}%
\end{pgfscope}%
\end{pgfscope}%
\begin{pgfscope}%
\pgfsetbuttcap%
\pgfsetroundjoin%
\definecolor{currentfill}{rgb}{0.000000,0.000000,0.000000}%
\pgfsetfillcolor{currentfill}%
\pgfsetlinewidth{0.602250pt}%
\definecolor{currentstroke}{rgb}{0.000000,0.000000,0.000000}%
\pgfsetstrokecolor{currentstroke}%
\pgfsetdash{}{0pt}%
\pgfsys@defobject{currentmarker}{\pgfqpoint{-0.027778in}{0.000000in}}{\pgfqpoint{-0.000000in}{0.000000in}}{%
\pgfpathmoveto{\pgfqpoint{-0.000000in}{0.000000in}}%
\pgfpathlineto{\pgfqpoint{-0.027778in}{0.000000in}}%
\pgfusepath{stroke,fill}%
}%
\begin{pgfscope}%
\pgfsys@transformshift{0.588387in}{2.596573in}%
\pgfsys@useobject{currentmarker}{}%
\end{pgfscope}%
\end{pgfscope}%
\begin{pgfscope}%
\pgfsetbuttcap%
\pgfsetroundjoin%
\definecolor{currentfill}{rgb}{0.000000,0.000000,0.000000}%
\pgfsetfillcolor{currentfill}%
\pgfsetlinewidth{0.602250pt}%
\definecolor{currentstroke}{rgb}{0.000000,0.000000,0.000000}%
\pgfsetstrokecolor{currentstroke}%
\pgfsetdash{}{0pt}%
\pgfsys@defobject{currentmarker}{\pgfqpoint{-0.027778in}{0.000000in}}{\pgfqpoint{-0.000000in}{0.000000in}}{%
\pgfpathmoveto{\pgfqpoint{-0.000000in}{0.000000in}}%
\pgfpathlineto{\pgfqpoint{-0.027778in}{0.000000in}}%
\pgfusepath{stroke,fill}%
}%
\begin{pgfscope}%
\pgfsys@transformshift{0.588387in}{2.634651in}%
\pgfsys@useobject{currentmarker}{}%
\end{pgfscope}%
\end{pgfscope}%
\begin{pgfscope}%
\pgfsetbuttcap%
\pgfsetroundjoin%
\definecolor{currentfill}{rgb}{0.000000,0.000000,0.000000}%
\pgfsetfillcolor{currentfill}%
\pgfsetlinewidth{0.602250pt}%
\definecolor{currentstroke}{rgb}{0.000000,0.000000,0.000000}%
\pgfsetstrokecolor{currentstroke}%
\pgfsetdash{}{0pt}%
\pgfsys@defobject{currentmarker}{\pgfqpoint{-0.027778in}{0.000000in}}{\pgfqpoint{-0.000000in}{0.000000in}}{%
\pgfpathmoveto{\pgfqpoint{-0.000000in}{0.000000in}}%
\pgfpathlineto{\pgfqpoint{-0.027778in}{0.000000in}}%
\pgfusepath{stroke,fill}%
}%
\begin{pgfscope}%
\pgfsys@transformshift{0.588387in}{2.664186in}%
\pgfsys@useobject{currentmarker}{}%
\end{pgfscope}%
\end{pgfscope}%
\begin{pgfscope}%
\pgfsetbuttcap%
\pgfsetroundjoin%
\definecolor{currentfill}{rgb}{0.000000,0.000000,0.000000}%
\pgfsetfillcolor{currentfill}%
\pgfsetlinewidth{0.602250pt}%
\definecolor{currentstroke}{rgb}{0.000000,0.000000,0.000000}%
\pgfsetstrokecolor{currentstroke}%
\pgfsetdash{}{0pt}%
\pgfsys@defobject{currentmarker}{\pgfqpoint{-0.027778in}{0.000000in}}{\pgfqpoint{-0.000000in}{0.000000in}}{%
\pgfpathmoveto{\pgfqpoint{-0.000000in}{0.000000in}}%
\pgfpathlineto{\pgfqpoint{-0.027778in}{0.000000in}}%
\pgfusepath{stroke,fill}%
}%
\begin{pgfscope}%
\pgfsys@transformshift{0.588387in}{2.688318in}%
\pgfsys@useobject{currentmarker}{}%
\end{pgfscope}%
\end{pgfscope}%
\begin{pgfscope}%
\pgfsetbuttcap%
\pgfsetroundjoin%
\definecolor{currentfill}{rgb}{0.000000,0.000000,0.000000}%
\pgfsetfillcolor{currentfill}%
\pgfsetlinewidth{0.602250pt}%
\definecolor{currentstroke}{rgb}{0.000000,0.000000,0.000000}%
\pgfsetstrokecolor{currentstroke}%
\pgfsetdash{}{0pt}%
\pgfsys@defobject{currentmarker}{\pgfqpoint{-0.027778in}{0.000000in}}{\pgfqpoint{-0.000000in}{0.000000in}}{%
\pgfpathmoveto{\pgfqpoint{-0.000000in}{0.000000in}}%
\pgfpathlineto{\pgfqpoint{-0.027778in}{0.000000in}}%
\pgfusepath{stroke,fill}%
}%
\begin{pgfscope}%
\pgfsys@transformshift{0.588387in}{2.708722in}%
\pgfsys@useobject{currentmarker}{}%
\end{pgfscope}%
\end{pgfscope}%
\begin{pgfscope}%
\pgfsetbuttcap%
\pgfsetroundjoin%
\definecolor{currentfill}{rgb}{0.000000,0.000000,0.000000}%
\pgfsetfillcolor{currentfill}%
\pgfsetlinewidth{0.602250pt}%
\definecolor{currentstroke}{rgb}{0.000000,0.000000,0.000000}%
\pgfsetstrokecolor{currentstroke}%
\pgfsetdash{}{0pt}%
\pgfsys@defobject{currentmarker}{\pgfqpoint{-0.027778in}{0.000000in}}{\pgfqpoint{-0.000000in}{0.000000in}}{%
\pgfpathmoveto{\pgfqpoint{-0.000000in}{0.000000in}}%
\pgfpathlineto{\pgfqpoint{-0.027778in}{0.000000in}}%
\pgfusepath{stroke,fill}%
}%
\begin{pgfscope}%
\pgfsys@transformshift{0.588387in}{2.726396in}%
\pgfsys@useobject{currentmarker}{}%
\end{pgfscope}%
\end{pgfscope}%
\begin{pgfscope}%
\definecolor{textcolor}{rgb}{0.000000,0.000000,0.000000}%
\pgfsetstrokecolor{textcolor}%
\pgfsetfillcolor{textcolor}%
\pgftext[x=0.234413in,y=1.631726in,,bottom,rotate=90.000000]{\color{textcolor}{\rmfamily\fontsize{10.000000}{12.000000}\selectfont\catcode`\^=\active\def^{\ifmmode\sp\else\^{}\fi}\catcode`\%=\active\def%{\%}Checks [call]}}%
\end{pgfscope}%
\begin{pgfscope}%
\pgfpathrectangle{\pgfqpoint{0.588387in}{0.521603in}}{\pgfqpoint{4.899126in}{2.220246in}}%
\pgfusepath{clip}%
\pgfsetrectcap%
\pgfsetroundjoin%
\pgfsetlinewidth{1.505625pt}%
\pgfsetstrokecolor{currentstroke1}%
\pgfsetdash{}{0pt}%
\pgfpathmoveto{\pgfqpoint{0.811075in}{0.622524in}}%
\pgfpathlineto{\pgfqpoint{0.954744in}{0.767937in}}%
\pgfpathlineto{\pgfqpoint{1.098414in}{0.880086in}}%
\pgfpathlineto{\pgfqpoint{1.242083in}{0.980964in}}%
\pgfpathlineto{\pgfqpoint{1.385752in}{1.077050in}}%
\pgfpathlineto{\pgfqpoint{1.529422in}{1.170913in}}%
\pgfpathlineto{\pgfqpoint{1.673091in}{1.263705in}}%
\pgfpathlineto{\pgfqpoint{1.816761in}{1.355971in}}%
\pgfpathlineto{\pgfqpoint{1.960430in}{1.447976in}}%
\pgfpathlineto{\pgfqpoint{2.104099in}{1.539851in}}%
\pgfpathlineto{\pgfqpoint{2.247769in}{1.631662in}}%
\pgfpathlineto{\pgfqpoint{2.391438in}{1.723440in}}%
\pgfpathlineto{\pgfqpoint{2.535108in}{1.815202in}}%
\pgfpathlineto{\pgfqpoint{2.678777in}{1.906955in}}%
\pgfpathlineto{\pgfqpoint{2.822446in}{1.998705in}}%
\pgfpathlineto{\pgfqpoint{2.966116in}{2.090453in}}%
\pgfpathlineto{\pgfqpoint{3.109785in}{2.182199in}}%
\pgfpathlineto{\pgfqpoint{3.253455in}{2.273946in}}%
\pgfpathlineto{\pgfqpoint{3.397124in}{2.365692in}}%
\pgfpathlineto{\pgfqpoint{3.540793in}{2.457437in}}%
\pgfpathlineto{\pgfqpoint{3.684463in}{2.549183in}}%
\pgfpathlineto{\pgfqpoint{3.828132in}{2.640929in}}%
\pgfusepath{stroke}%
\end{pgfscope}%
\begin{pgfscope}%
\pgfpathrectangle{\pgfqpoint{0.588387in}{0.521603in}}{\pgfqpoint{4.899126in}{2.220246in}}%
\pgfusepath{clip}%
\pgfsetrectcap%
\pgfsetroundjoin%
\pgfsetlinewidth{1.505625pt}%
\pgfsetstrokecolor{currentstroke2}%
\pgfsetdash{}{0pt}%
\pgfpathmoveto{\pgfqpoint{0.811075in}{0.714269in}}%
\pgfpathlineto{\pgfqpoint{0.954744in}{0.806015in}}%
\pgfpathlineto{\pgfqpoint{1.098414in}{0.897761in}}%
\pgfpathlineto{\pgfqpoint{1.242083in}{0.989506in}}%
\pgfpathlineto{\pgfqpoint{1.385752in}{1.081252in}}%
\pgfpathlineto{\pgfqpoint{1.529422in}{1.172998in}}%
\pgfpathlineto{\pgfqpoint{1.673091in}{1.235684in}}%
\pgfpathlineto{\pgfqpoint{1.816761in}{1.293517in}}%
\pgfpathlineto{\pgfqpoint{1.960430in}{1.379770in}}%
\pgfpathlineto{\pgfqpoint{2.104099in}{1.421334in}}%
\pgfpathlineto{\pgfqpoint{2.247769in}{1.485939in}}%
\pgfpathlineto{\pgfqpoint{2.391438in}{1.540957in}}%
\pgfpathlineto{\pgfqpoint{2.535108in}{1.588699in}}%
\pgfpathlineto{\pgfqpoint{2.678777in}{1.674853in}}%
\pgfpathlineto{\pgfqpoint{2.822446in}{1.715774in}}%
\pgfpathlineto{\pgfqpoint{2.966116in}{1.708422in}}%
\pgfpathlineto{\pgfqpoint{3.109785in}{1.778931in}}%
\pgfpathlineto{\pgfqpoint{3.253455in}{1.790648in}}%
\pgfpathlineto{\pgfqpoint{3.397124in}{1.938024in}}%
\pgfpathlineto{\pgfqpoint{3.540793in}{1.920925in}}%
\pgfpathlineto{\pgfqpoint{3.684463in}{2.000580in}}%
\pgfpathlineto{\pgfqpoint{3.828132in}{2.007648in}}%
\pgfpathlineto{\pgfqpoint{3.971802in}{2.094664in}}%
\pgfpathlineto{\pgfqpoint{4.115471in}{2.103440in}}%
\pgfpathlineto{\pgfqpoint{4.259140in}{2.133594in}}%
\pgfpathlineto{\pgfqpoint{4.402810in}{2.205353in}}%
\pgfpathlineto{\pgfqpoint{4.546479in}{2.140080in}}%
\pgfpathlineto{\pgfqpoint{4.690149in}{2.234055in}}%
\pgfpathlineto{\pgfqpoint{4.977487in}{2.304646in}}%
\pgfusepath{stroke}%
\end{pgfscope}%
\begin{pgfscope}%
\pgfpathrectangle{\pgfqpoint{0.588387in}{0.521603in}}{\pgfqpoint{4.899126in}{2.220246in}}%
\pgfusepath{clip}%
\pgfsetrectcap%
\pgfsetroundjoin%
\pgfsetlinewidth{1.505625pt}%
\pgfsetstrokecolor{currentstroke3}%
\pgfsetdash{}{0pt}%
\pgfpathmoveto{\pgfqpoint{0.811075in}{0.714269in}}%
\pgfpathlineto{\pgfqpoint{0.954744in}{0.806015in}}%
\pgfpathlineto{\pgfqpoint{1.098414in}{0.897761in}}%
\pgfpathlineto{\pgfqpoint{1.242083in}{0.989506in}}%
\pgfpathlineto{\pgfqpoint{1.385752in}{1.081252in}}%
\pgfpathlineto{\pgfqpoint{1.529422in}{1.172998in}}%
\pgfpathlineto{\pgfqpoint{1.673091in}{1.235684in}}%
\pgfpathlineto{\pgfqpoint{1.816761in}{1.293517in}}%
\pgfpathlineto{\pgfqpoint{1.960430in}{1.379770in}}%
\pgfpathlineto{\pgfqpoint{2.104099in}{1.421334in}}%
\pgfpathlineto{\pgfqpoint{2.247769in}{1.474859in}}%
\pgfpathlineto{\pgfqpoint{2.391438in}{1.511893in}}%
\pgfpathlineto{\pgfqpoint{2.535108in}{1.576086in}}%
\pgfpathlineto{\pgfqpoint{2.678777in}{1.631405in}}%
\pgfpathlineto{\pgfqpoint{2.822446in}{1.685316in}}%
\pgfpathlineto{\pgfqpoint{2.966116in}{1.690605in}}%
\pgfpathlineto{\pgfqpoint{3.109785in}{1.767153in}}%
\pgfpathlineto{\pgfqpoint{3.253455in}{1.780123in}}%
\pgfpathlineto{\pgfqpoint{3.397124in}{1.865090in}}%
\pgfpathlineto{\pgfqpoint{3.540793in}{1.891655in}}%
\pgfpathlineto{\pgfqpoint{3.684463in}{1.952611in}}%
\pgfpathlineto{\pgfqpoint{3.828132in}{1.971875in}}%
\pgfpathlineto{\pgfqpoint{3.971802in}{2.082076in}}%
\pgfpathlineto{\pgfqpoint{4.115471in}{2.022721in}}%
\pgfpathlineto{\pgfqpoint{4.259140in}{2.064199in}}%
\pgfpathlineto{\pgfqpoint{4.402810in}{2.123577in}}%
\pgfpathlineto{\pgfqpoint{4.546479in}{2.170227in}}%
\pgfpathlineto{\pgfqpoint{4.690149in}{2.142684in}}%
\pgfpathlineto{\pgfqpoint{4.977487in}{2.290090in}}%
\pgfusepath{stroke}%
\end{pgfscope}%
\begin{pgfscope}%
\pgfpathrectangle{\pgfqpoint{0.588387in}{0.521603in}}{\pgfqpoint{4.899126in}{2.220246in}}%
\pgfusepath{clip}%
\pgfsetrectcap%
\pgfsetroundjoin%
\pgfsetlinewidth{1.505625pt}%
\pgfsetstrokecolor{currentstroke4}%
\pgfsetdash{}{0pt}%
\pgfpathmoveto{\pgfqpoint{0.811075in}{0.714269in}}%
\pgfpathlineto{\pgfqpoint{0.954744in}{0.806015in}}%
\pgfpathlineto{\pgfqpoint{1.098414in}{0.897761in}}%
\pgfpathlineto{\pgfqpoint{1.242083in}{0.989506in}}%
\pgfpathlineto{\pgfqpoint{1.385752in}{1.081252in}}%
\pgfpathlineto{\pgfqpoint{1.529422in}{1.172998in}}%
\pgfpathlineto{\pgfqpoint{1.673091in}{1.240280in}}%
\pgfpathlineto{\pgfqpoint{1.816761in}{1.297812in}}%
\pgfpathlineto{\pgfqpoint{1.960430in}{1.382019in}}%
\pgfpathlineto{\pgfqpoint{2.104099in}{1.435765in}}%
\pgfpathlineto{\pgfqpoint{2.247769in}{1.509542in}}%
\pgfpathlineto{\pgfqpoint{2.391438in}{1.556138in}}%
\pgfpathlineto{\pgfqpoint{2.535108in}{1.628057in}}%
\pgfpathlineto{\pgfqpoint{2.678777in}{1.679085in}}%
\pgfpathlineto{\pgfqpoint{2.822446in}{1.731462in}}%
\pgfpathlineto{\pgfqpoint{2.966116in}{1.751998in}}%
\pgfpathlineto{\pgfqpoint{3.109785in}{1.831177in}}%
\pgfpathlineto{\pgfqpoint{3.253455in}{1.832720in}}%
\pgfpathlineto{\pgfqpoint{3.397124in}{1.966199in}}%
\pgfpathlineto{\pgfqpoint{3.540793in}{1.964782in}}%
\pgfpathlineto{\pgfqpoint{3.684463in}{2.031896in}}%
\pgfpathlineto{\pgfqpoint{3.828132in}{2.048123in}}%
\pgfpathlineto{\pgfqpoint{3.971802in}{2.134688in}}%
\pgfpathlineto{\pgfqpoint{4.115471in}{2.150671in}}%
\pgfpathlineto{\pgfqpoint{4.259140in}{2.224267in}}%
\pgfpathlineto{\pgfqpoint{4.402810in}{2.203115in}}%
\pgfpathlineto{\pgfqpoint{4.546479in}{2.274098in}}%
\pgfpathlineto{\pgfqpoint{4.690149in}{2.262308in}}%
\pgfpathlineto{\pgfqpoint{4.977487in}{2.148987in}}%
\pgfusepath{stroke}%
\end{pgfscope}%
\begin{pgfscope}%
\pgfpathrectangle{\pgfqpoint{0.588387in}{0.521603in}}{\pgfqpoint{4.899126in}{2.220246in}}%
\pgfusepath{clip}%
\pgfsetrectcap%
\pgfsetroundjoin%
\pgfsetlinewidth{1.505625pt}%
\pgfsetstrokecolor{currentstroke5}%
\pgfsetdash{}{0pt}%
\pgfpathmoveto{\pgfqpoint{0.811075in}{0.714269in}}%
\pgfpathlineto{\pgfqpoint{0.954744in}{0.806015in}}%
\pgfpathlineto{\pgfqpoint{1.098414in}{0.897761in}}%
\pgfpathlineto{\pgfqpoint{1.242083in}{0.989506in}}%
\pgfpathlineto{\pgfqpoint{1.385752in}{1.081252in}}%
\pgfpathlineto{\pgfqpoint{1.529422in}{1.172998in}}%
\pgfpathlineto{\pgfqpoint{1.673091in}{1.240280in}}%
\pgfpathlineto{\pgfqpoint{1.816761in}{1.297812in}}%
\pgfpathlineto{\pgfqpoint{1.960430in}{1.382019in}}%
\pgfpathlineto{\pgfqpoint{2.104099in}{1.435765in}}%
\pgfpathlineto{\pgfqpoint{2.247769in}{1.485448in}}%
\pgfpathlineto{\pgfqpoint{2.391438in}{1.522911in}}%
\pgfpathlineto{\pgfqpoint{2.535108in}{1.589090in}}%
\pgfpathlineto{\pgfqpoint{2.678777in}{1.629815in}}%
\pgfpathlineto{\pgfqpoint{2.822446in}{1.683152in}}%
\pgfpathlineto{\pgfqpoint{2.966116in}{1.729091in}}%
\pgfpathlineto{\pgfqpoint{3.109785in}{1.782473in}}%
\pgfpathlineto{\pgfqpoint{3.253455in}{1.799646in}}%
\pgfpathlineto{\pgfqpoint{3.397124in}{1.875716in}}%
\pgfpathlineto{\pgfqpoint{3.540793in}{1.882999in}}%
\pgfpathlineto{\pgfqpoint{3.684463in}{1.987682in}}%
\pgfpathlineto{\pgfqpoint{3.828132in}{1.999015in}}%
\pgfpathlineto{\pgfqpoint{3.971802in}{2.082665in}}%
\pgfpathlineto{\pgfqpoint{4.115471in}{2.058938in}}%
\pgfpathlineto{\pgfqpoint{4.259140in}{2.134929in}}%
\pgfpathlineto{\pgfqpoint{4.402810in}{2.155432in}}%
\pgfpathlineto{\pgfqpoint{4.546479in}{2.187865in}}%
\pgfpathlineto{\pgfqpoint{4.690149in}{2.221312in}}%
\pgfpathlineto{\pgfqpoint{4.977487in}{2.178132in}}%
\pgfpathlineto{\pgfqpoint{5.264826in}{2.210156in}}%
\pgfusepath{stroke}%
\end{pgfscope}%
\begin{pgfscope}%
\pgfpathrectangle{\pgfqpoint{0.588387in}{0.521603in}}{\pgfqpoint{4.899126in}{2.220246in}}%
\pgfusepath{clip}%
\pgfsetrectcap%
\pgfsetroundjoin%
\pgfsetlinewidth{1.505625pt}%
\pgfsetstrokecolor{currentstroke6}%
\pgfsetdash{}{0pt}%
\pgfpathmoveto{\pgfqpoint{0.811075in}{0.714269in}}%
\pgfpathlineto{\pgfqpoint{0.954744in}{0.806015in}}%
\pgfpathlineto{\pgfqpoint{1.098414in}{0.897761in}}%
\pgfpathlineto{\pgfqpoint{1.242083in}{0.989506in}}%
\pgfpathlineto{\pgfqpoint{1.385752in}{1.081252in}}%
\pgfpathlineto{\pgfqpoint{1.529422in}{1.172998in}}%
\pgfpathlineto{\pgfqpoint{1.673091in}{1.261523in}}%
\pgfpathlineto{\pgfqpoint{1.816761in}{1.350454in}}%
\pgfpathlineto{\pgfqpoint{1.960430in}{1.421513in}}%
\pgfpathlineto{\pgfqpoint{2.104099in}{1.482303in}}%
\pgfpathlineto{\pgfqpoint{2.247769in}{1.554682in}}%
\pgfpathlineto{\pgfqpoint{2.391438in}{1.602710in}}%
\pgfpathlineto{\pgfqpoint{2.535108in}{1.673377in}}%
\pgfpathlineto{\pgfqpoint{2.678777in}{1.739749in}}%
\pgfpathlineto{\pgfqpoint{2.822446in}{1.798122in}}%
\pgfpathlineto{\pgfqpoint{2.966116in}{1.833564in}}%
\pgfpathlineto{\pgfqpoint{3.109785in}{1.903431in}}%
\pgfpathlineto{\pgfqpoint{3.253455in}{1.950208in}}%
\pgfpathlineto{\pgfqpoint{3.397124in}{2.023355in}}%
\pgfpathlineto{\pgfqpoint{3.540793in}{2.035826in}}%
\pgfpathlineto{\pgfqpoint{3.684463in}{2.112486in}}%
\pgfpathlineto{\pgfqpoint{3.828132in}{2.138707in}}%
\pgfpathlineto{\pgfqpoint{3.971802in}{2.250798in}}%
\pgfpathlineto{\pgfqpoint{4.115471in}{2.230848in}}%
\pgfpathlineto{\pgfqpoint{4.259140in}{2.258756in}}%
\pgfpathlineto{\pgfqpoint{4.402810in}{2.237999in}}%
\pgfpathlineto{\pgfqpoint{4.690149in}{2.257960in}}%
\pgfusepath{stroke}%
\end{pgfscope}%
\begin{pgfscope}%
\pgfpathrectangle{\pgfqpoint{0.588387in}{0.521603in}}{\pgfqpoint{4.899126in}{2.220246in}}%
\pgfusepath{clip}%
\pgfsetrectcap%
\pgfsetroundjoin%
\pgfsetlinewidth{1.505625pt}%
\pgfsetstrokecolor{currentstroke7}%
\pgfsetdash{}{0pt}%
\pgfpathmoveto{\pgfqpoint{0.811075in}{0.714269in}}%
\pgfpathlineto{\pgfqpoint{0.954744in}{0.806015in}}%
\pgfpathlineto{\pgfqpoint{1.098414in}{0.897761in}}%
\pgfpathlineto{\pgfqpoint{1.242083in}{0.989506in}}%
\pgfpathlineto{\pgfqpoint{1.385752in}{1.081252in}}%
\pgfpathlineto{\pgfqpoint{1.529422in}{1.172998in}}%
\pgfpathlineto{\pgfqpoint{1.673091in}{1.261523in}}%
\pgfpathlineto{\pgfqpoint{1.816761in}{1.350454in}}%
\pgfpathlineto{\pgfqpoint{1.960430in}{1.421513in}}%
\pgfpathlineto{\pgfqpoint{2.104099in}{1.482303in}}%
\pgfpathlineto{\pgfqpoint{2.247769in}{1.524132in}}%
\pgfpathlineto{\pgfqpoint{2.391438in}{1.578866in}}%
\pgfpathlineto{\pgfqpoint{2.535108in}{1.656073in}}%
\pgfpathlineto{\pgfqpoint{2.678777in}{1.721234in}}%
\pgfpathlineto{\pgfqpoint{2.822446in}{1.754956in}}%
\pgfpathlineto{\pgfqpoint{2.966116in}{1.802361in}}%
\pgfpathlineto{\pgfqpoint{3.109785in}{1.874203in}}%
\pgfpathlineto{\pgfqpoint{3.253455in}{1.913225in}}%
\pgfpathlineto{\pgfqpoint{3.397124in}{1.964518in}}%
\pgfpathlineto{\pgfqpoint{3.540793in}{1.979917in}}%
\pgfpathlineto{\pgfqpoint{3.684463in}{2.071933in}}%
\pgfpathlineto{\pgfqpoint{3.828132in}{2.116274in}}%
\pgfpathlineto{\pgfqpoint{3.971802in}{2.170694in}}%
\pgfpathlineto{\pgfqpoint{4.115471in}{2.181030in}}%
\pgfpathlineto{\pgfqpoint{4.259140in}{2.306018in}}%
\pgfpathlineto{\pgfqpoint{4.402810in}{2.230342in}}%
\pgfpathlineto{\pgfqpoint{4.546479in}{2.303487in}}%
\pgfpathlineto{\pgfqpoint{4.690149in}{2.275557in}}%
\pgfusepath{stroke}%
\end{pgfscope}%
\begin{pgfscope}%
\pgfsetrectcap%
\pgfsetmiterjoin%
\pgfsetlinewidth{0.803000pt}%
\definecolor{currentstroke}{rgb}{0.000000,0.000000,0.000000}%
\pgfsetstrokecolor{currentstroke}%
\pgfsetdash{}{0pt}%
\pgfpathmoveto{\pgfqpoint{0.588387in}{0.521603in}}%
\pgfpathlineto{\pgfqpoint{0.588387in}{2.741849in}}%
\pgfusepath{stroke}%
\end{pgfscope}%
\begin{pgfscope}%
\pgfsetrectcap%
\pgfsetmiterjoin%
\pgfsetlinewidth{0.803000pt}%
\definecolor{currentstroke}{rgb}{0.000000,0.000000,0.000000}%
\pgfsetstrokecolor{currentstroke}%
\pgfsetdash{}{0pt}%
\pgfpathmoveto{\pgfqpoint{5.487514in}{0.521603in}}%
\pgfpathlineto{\pgfqpoint{5.487514in}{2.741849in}}%
\pgfusepath{stroke}%
\end{pgfscope}%
\begin{pgfscope}%
\pgfsetrectcap%
\pgfsetmiterjoin%
\pgfsetlinewidth{0.803000pt}%
\definecolor{currentstroke}{rgb}{0.000000,0.000000,0.000000}%
\pgfsetstrokecolor{currentstroke}%
\pgfsetdash{}{0pt}%
\pgfpathmoveto{\pgfqpoint{0.588387in}{0.521603in}}%
\pgfpathlineto{\pgfqpoint{5.487514in}{0.521603in}}%
\pgfusepath{stroke}%
\end{pgfscope}%
\begin{pgfscope}%
\pgfsetrectcap%
\pgfsetmiterjoin%
\pgfsetlinewidth{0.803000pt}%
\definecolor{currentstroke}{rgb}{0.000000,0.000000,0.000000}%
\pgfsetstrokecolor{currentstroke}%
\pgfsetdash{}{0pt}%
\pgfpathmoveto{\pgfqpoint{0.588387in}{2.741849in}}%
\pgfpathlineto{\pgfqpoint{5.487514in}{2.741849in}}%
\pgfusepath{stroke}%
\end{pgfscope}%
\begin{pgfscope}%
\pgfsetbuttcap%
\pgfsetmiterjoin%
\definecolor{currentfill}{rgb}{1.000000,1.000000,1.000000}%
\pgfsetfillcolor{currentfill}%
\pgfsetfillopacity{0.800000}%
\pgfsetlinewidth{1.003750pt}%
\definecolor{currentstroke}{rgb}{0.800000,0.800000,0.800000}%
\pgfsetstrokecolor{currentstroke}%
\pgfsetstrokeopacity{0.800000}%
\pgfsetdash{}{0pt}%
\pgfpathmoveto{\pgfqpoint{5.575014in}{1.343633in}}%
\pgfpathlineto{\pgfqpoint{8.259376in}{1.343633in}}%
\pgfpathquadraticcurveto{\pgfqpoint{8.284376in}{1.343633in}}{\pgfqpoint{8.284376in}{1.368633in}}%
\pgfpathlineto{\pgfqpoint{8.284376in}{2.654349in}}%
\pgfpathquadraticcurveto{\pgfqpoint{8.284376in}{2.679349in}}{\pgfqpoint{8.259376in}{2.679349in}}%
\pgfpathlineto{\pgfqpoint{5.575014in}{2.679349in}}%
\pgfpathquadraticcurveto{\pgfqpoint{5.550014in}{2.679349in}}{\pgfqpoint{5.550014in}{2.654349in}}%
\pgfpathlineto{\pgfqpoint{5.550014in}{1.368633in}}%
\pgfpathquadraticcurveto{\pgfqpoint{5.550014in}{1.343633in}}{\pgfqpoint{5.575014in}{1.343633in}}%
\pgfpathlineto{\pgfqpoint{5.575014in}{1.343633in}}%
\pgfpathclose%
\pgfusepath{stroke,fill}%
\end{pgfscope}%
\begin{pgfscope}%
\pgfsetrectcap%
\pgfsetroundjoin%
\pgfsetlinewidth{1.505625pt}%
\pgfsetstrokecolor{currentstroke1}%
\pgfsetdash{}{0pt}%
\pgfpathmoveto{\pgfqpoint{5.600014in}{2.578129in}}%
\pgfpathlineto{\pgfqpoint{5.725014in}{2.578129in}}%
\pgfpathlineto{\pgfqpoint{5.850014in}{2.578129in}}%
\pgfusepath{stroke}%
\end{pgfscope}%
\begin{pgfscope}%
\definecolor{textcolor}{rgb}{0.000000,0.000000,0.000000}%
\pgfsetstrokecolor{textcolor}%
\pgfsetfillcolor{textcolor}%
\pgftext[x=5.950014in,y=2.534379in,left,base]{\color{textcolor}{\rmfamily\fontsize{9.000000}{10.800000}\selectfont\catcode`\^=\active\def^{\ifmmode\sp\else\^{}\fi}\catcode`\%=\active\def%{\%}\NaiveCycles{}}}%
\end{pgfscope}%
\begin{pgfscope}%
\pgfsetrectcap%
\pgfsetroundjoin%
\pgfsetlinewidth{1.505625pt}%
\pgfsetstrokecolor{currentstroke2}%
\pgfsetdash{}{0pt}%
\pgfpathmoveto{\pgfqpoint{5.600014in}{2.394657in}}%
\pgfpathlineto{\pgfqpoint{5.725014in}{2.394657in}}%
\pgfpathlineto{\pgfqpoint{5.850014in}{2.394657in}}%
\pgfusepath{stroke}%
\end{pgfscope}%
\begin{pgfscope}%
\definecolor{textcolor}{rgb}{0.000000,0.000000,0.000000}%
\pgfsetstrokecolor{textcolor}%
\pgfsetfillcolor{textcolor}%
\pgftext[x=5.950014in,y=2.350907in,left,base]{\color{textcolor}{\rmfamily\fontsize{9.000000}{10.800000}\selectfont\catcode`\^=\active\def^{\ifmmode\sp\else\^{}\fi}\catcode`\%=\active\def%{\%}\Neighbors{} \& \MergeLinear{}}}%
\end{pgfscope}%
\begin{pgfscope}%
\pgfsetrectcap%
\pgfsetroundjoin%
\pgfsetlinewidth{1.505625pt}%
\pgfsetstrokecolor{currentstroke3}%
\pgfsetdash{}{0pt}%
\pgfpathmoveto{\pgfqpoint{5.600014in}{2.211185in}}%
\pgfpathlineto{\pgfqpoint{5.725014in}{2.211185in}}%
\pgfpathlineto{\pgfqpoint{5.850014in}{2.211185in}}%
\pgfusepath{stroke}%
\end{pgfscope}%
\begin{pgfscope}%
\definecolor{textcolor}{rgb}{0.000000,0.000000,0.000000}%
\pgfsetstrokecolor{textcolor}%
\pgfsetfillcolor{textcolor}%
\pgftext[x=5.950014in,y=2.167435in,left,base]{\color{textcolor}{\rmfamily\fontsize{9.000000}{10.800000}\selectfont\catcode`\^=\active\def^{\ifmmode\sp\else\^{}\fi}\catcode`\%=\active\def%{\%}\Neighbors{} \& \SharedVertices{}}}%
\end{pgfscope}%
\begin{pgfscope}%
\pgfsetrectcap%
\pgfsetroundjoin%
\pgfsetlinewidth{1.505625pt}%
\pgfsetstrokecolor{currentstroke4}%
\pgfsetdash{}{0pt}%
\pgfpathmoveto{\pgfqpoint{5.600014in}{2.024235in}}%
\pgfpathlineto{\pgfqpoint{5.725014in}{2.024235in}}%
\pgfpathlineto{\pgfqpoint{5.850014in}{2.024235in}}%
\pgfusepath{stroke}%
\end{pgfscope}%
\begin{pgfscope}%
\definecolor{textcolor}{rgb}{0.000000,0.000000,0.000000}%
\pgfsetstrokecolor{textcolor}%
\pgfsetfillcolor{textcolor}%
\pgftext[x=5.950014in,y=1.980485in,left,base]{\color{textcolor}{\rmfamily\fontsize{9.000000}{10.800000}\selectfont\catcode`\^=\active\def^{\ifmmode\sp\else\^{}\fi}\catcode`\%=\active\def%{\%}\NeighborsDegree{} \& \MergeLinear{}}}%
\end{pgfscope}%
\begin{pgfscope}%
\pgfsetrectcap%
\pgfsetroundjoin%
\pgfsetlinewidth{1.505625pt}%
\pgfsetstrokecolor{currentstroke5}%
\pgfsetdash{}{0pt}%
\pgfpathmoveto{\pgfqpoint{5.600014in}{1.837285in}}%
\pgfpathlineto{\pgfqpoint{5.725014in}{1.837285in}}%
\pgfpathlineto{\pgfqpoint{5.850014in}{1.837285in}}%
\pgfusepath{stroke}%
\end{pgfscope}%
\begin{pgfscope}%
\definecolor{textcolor}{rgb}{0.000000,0.000000,0.000000}%
\pgfsetstrokecolor{textcolor}%
\pgfsetfillcolor{textcolor}%
\pgftext[x=5.950014in,y=1.793535in,left,base]{\color{textcolor}{\rmfamily\fontsize{9.000000}{10.800000}\selectfont\catcode`\^=\active\def^{\ifmmode\sp\else\^{}\fi}\catcode`\%=\active\def%{\%}\NeighborsDegree{} \& \SharedVertices{}}}%
\end{pgfscope}%
\begin{pgfscope}%
\pgfsetrectcap%
\pgfsetroundjoin%
\pgfsetlinewidth{1.505625pt}%
\pgfsetstrokecolor{currentstroke6}%
\pgfsetdash{}{0pt}%
\pgfpathmoveto{\pgfqpoint{5.600014in}{1.650334in}}%
\pgfpathlineto{\pgfqpoint{5.725014in}{1.650334in}}%
\pgfpathlineto{\pgfqpoint{5.850014in}{1.650334in}}%
\pgfusepath{stroke}%
\end{pgfscope}%
\begin{pgfscope}%
\definecolor{textcolor}{rgb}{0.000000,0.000000,0.000000}%
\pgfsetstrokecolor{textcolor}%
\pgfsetfillcolor{textcolor}%
\pgftext[x=5.950014in,y=1.606584in,left,base]{\color{textcolor}{\rmfamily\fontsize{9.000000}{10.800000}\selectfont\catcode`\^=\active\def^{\ifmmode\sp\else\^{}\fi}\catcode`\%=\active\def%{\%}\None{} \& \MergeLinear{}}}%
\end{pgfscope}%
\begin{pgfscope}%
\pgfsetrectcap%
\pgfsetroundjoin%
\pgfsetlinewidth{1.505625pt}%
\pgfsetstrokecolor{currentstroke7}%
\pgfsetdash{}{0pt}%
\pgfpathmoveto{\pgfqpoint{5.600014in}{1.466863in}}%
\pgfpathlineto{\pgfqpoint{5.725014in}{1.466863in}}%
\pgfpathlineto{\pgfqpoint{5.850014in}{1.466863in}}%
\pgfusepath{stroke}%
\end{pgfscope}%
\begin{pgfscope}%
\definecolor{textcolor}{rgb}{0.000000,0.000000,0.000000}%
\pgfsetstrokecolor{textcolor}%
\pgfsetfillcolor{textcolor}%
\pgftext[x=5.950014in,y=1.423113in,left,base]{\color{textcolor}{\rmfamily\fontsize{9.000000}{10.800000}\selectfont\catcode`\^=\active\def^{\ifmmode\sp\else\^{}\fi}\catcode`\%=\active\def%{\%}\None{} \& \SharedVertices{}}}%
\end{pgfscope}%
\end{pgfpicture}%
\makeatother%
\endgroup%
}
	\caption[Checks performed for minimally rigid graphs]{
		The number of checks performed to find all NAC-colorings for minimally rigid graphs.}%
	\label{fig:graph_count_minimally_rigid}
\end{figure}%

In \Cref{fig:graph_summary}
we show the relation between the number of \IsNACColoring{} checks that
would \Naive{} algorithm perform compared to our solution.
%
The numbers of checks are similar for graphs with few monochromatic classes
when we consider \NaiveCycles{} using monochromatic classes and \Subgraphs{}.
That explains why the \NaiveCycles{} algorithm outperformed
the \NeighborsDegree{} \& \MergeLinear{} strategies in \Cref{tab:all_min_rigid}. This should improve quickly for larger graphs.
We can also see how the use of \CycleMask{} routine
reduces the number of more expensive \IsNACColoring{} calls,
since these are called only when the small cycles check \CycleMask{} fails to decide
(\CycleMask{} is called every time).

\begin{figure}[ht]
	\centering
	\scalebox{\BenchFigureScale}{%% Creator: Matplotlib, PGF backend
%%
%% To include the figure in your LaTeX document, write
%%   \input{<filename>.pgf}
%%
%% Make sure the required packages are loaded in your preamble
%%   \usepackage{pgf}
%%
%% Also ensure that all the required font packages are loaded; for instance,
%% the lmodern package is sometimes necessary when using math font.
%%   \usepackage{lmodern}
%%
%% Figures using additional raster images can only be included by \input if
%% they are in the same directory as the main LaTeX file. For loading figures
%% from other directories you can use the `import` package
%%   \usepackage{import}
%%
%% and then include the figures with
%%   \import{<path to file>}{<filename>.pgf}
%%
%% Matplotlib used the following preamble
%%   \def\mathdefault#1{#1}
%%   \everymath=\expandafter{\the\everymath\displaystyle}
%%   \IfFileExists{scrextend.sty}{
%%     \usepackage[fontsize=10.000000pt]{scrextend}
%%   }{
%%     \renewcommand{\normalsize}{\fontsize{10.000000}{12.000000}\selectfont}
%%     \normalsize
%%   }
%%   
%%   \ifdefined\pdftexversion\else  % non-pdftex case.
%%     \usepackage{fontspec}
%%     \setmainfont{DejaVuSans.ttf}[Path=\detokenize{/home/petr/Projects/PyRigi/.venv/lib/python3.12/site-packages/matplotlib/mpl-data/fonts/ttf/}]
%%     \setsansfont{DejaVuSans.ttf}[Path=\detokenize{/home/petr/Projects/PyRigi/.venv/lib/python3.12/site-packages/matplotlib/mpl-data/fonts/ttf/}]
%%     \setmonofont{DejaVuSansMono.ttf}[Path=\detokenize{/home/petr/Projects/PyRigi/.venv/lib/python3.12/site-packages/matplotlib/mpl-data/fonts/ttf/}]
%%   \fi
%%   \makeatletter\@ifpackageloaded{under\Score{}}{}{\usepackage[strings]{under\Score{}}}\makeatother
%%
\begingroup%
\makeatletter%
\begin{pgfpicture}%
\pgfpathrectangle{\pgfpointorigin}{\pgfqpoint{8.384376in}{2.841849in}}%
\pgfusepath{use as bounding box, clip}%
\begin{pgfscope}%
\pgfsetbuttcap%
\pgfsetmiterjoin%
\definecolor{currentfill}{rgb}{1.000000,1.000000,1.000000}%
\pgfsetfillcolor{currentfill}%
\pgfsetlinewidth{0.000000pt}%
\definecolor{currentstroke}{rgb}{1.000000,1.000000,1.000000}%
\pgfsetstrokecolor{currentstroke}%
\pgfsetdash{}{0pt}%
\pgfpathmoveto{\pgfqpoint{0.000000in}{0.000000in}}%
\pgfpathlineto{\pgfqpoint{8.384376in}{0.000000in}}%
\pgfpathlineto{\pgfqpoint{8.384376in}{2.841849in}}%
\pgfpathlineto{\pgfqpoint{0.000000in}{2.841849in}}%
\pgfpathlineto{\pgfqpoint{0.000000in}{0.000000in}}%
\pgfpathclose%
\pgfusepath{fill}%
\end{pgfscope}%
\begin{pgfscope}%
\pgfsetbuttcap%
\pgfsetmiterjoin%
\definecolor{currentfill}{rgb}{1.000000,1.000000,1.000000}%
\pgfsetfillcolor{currentfill}%
\pgfsetlinewidth{0.000000pt}%
\definecolor{currentstroke}{rgb}{0.000000,0.000000,0.000000}%
\pgfsetstrokecolor{currentstroke}%
\pgfsetstrokeopacity{0.000000}%
\pgfsetdash{}{0pt}%
\pgfpathmoveto{\pgfqpoint{0.643750in}{0.521603in}}%
\pgfpathlineto{\pgfqpoint{8.284376in}{0.521603in}}%
\pgfpathlineto{\pgfqpoint{8.284376in}{2.741849in}}%
\pgfpathlineto{\pgfqpoint{0.643750in}{2.741849in}}%
\pgfpathlineto{\pgfqpoint{0.643750in}{0.521603in}}%
\pgfpathclose%
\pgfusepath{fill}%
\end{pgfscope}%
\begin{pgfscope}%
\pgfsetbuttcap%
\pgfsetroundjoin%
\definecolor{currentfill}{rgb}{0.000000,0.000000,0.000000}%
\pgfsetfillcolor{currentfill}%
\pgfsetlinewidth{0.803000pt}%
\definecolor{currentstroke}{rgb}{0.000000,0.000000,0.000000}%
\pgfsetstrokecolor{currentstroke}%
\pgfsetdash{}{0pt}%
\pgfsys@defobject{currentmarker}{\pgfqpoint{0.000000in}{-0.048611in}}{\pgfqpoint{0.000000in}{0.000000in}}{%
\pgfpathmoveto{\pgfqpoint{0.000000in}{0.000000in}}%
\pgfpathlineto{\pgfqpoint{0.000000in}{-0.048611in}}%
\pgfusepath{stroke,fill}%
}%
\begin{pgfscope}%
\pgfsys@transformshift{1.425178in}{0.521603in}%
\pgfsys@useobject{currentmarker}{}%
\end{pgfscope}%
\end{pgfscope}%
\begin{pgfscope}%
\definecolor{textcolor}{rgb}{0.000000,0.000000,0.000000}%
\pgfsetstrokecolor{textcolor}%
\pgfsetfillcolor{textcolor}%
\pgftext[x=1.425178in,y=0.424381in,,top]{\color{textcolor}{\rmfamily\fontsize{10.000000}{12.000000}\selectfont\catcode`\^=\active\def^{\ifmmode\sp\else\^{}\fi}\catcode`\%=\active\def%{\%}$\mathdefault{4}$}}%
\end{pgfscope}%
\begin{pgfscope}%
\pgfsetbuttcap%
\pgfsetroundjoin%
\definecolor{currentfill}{rgb}{0.000000,0.000000,0.000000}%
\pgfsetfillcolor{currentfill}%
\pgfsetlinewidth{0.803000pt}%
\definecolor{currentstroke}{rgb}{0.000000,0.000000,0.000000}%
\pgfsetstrokecolor{currentstroke}%
\pgfsetdash{}{0pt}%
\pgfsys@defobject{currentmarker}{\pgfqpoint{0.000000in}{-0.048611in}}{\pgfqpoint{0.000000in}{0.000000in}}{%
\pgfpathmoveto{\pgfqpoint{0.000000in}{0.000000in}}%
\pgfpathlineto{\pgfqpoint{0.000000in}{-0.048611in}}%
\pgfusepath{stroke,fill}%
}%
\begin{pgfscope}%
\pgfsys@transformshift{2.293431in}{0.521603in}%
\pgfsys@useobject{currentmarker}{}%
\end{pgfscope}%
\end{pgfscope}%
\begin{pgfscope}%
\definecolor{textcolor}{rgb}{0.000000,0.000000,0.000000}%
\pgfsetstrokecolor{textcolor}%
\pgfsetfillcolor{textcolor}%
\pgftext[x=2.293431in,y=0.424381in,,top]{\color{textcolor}{\rmfamily\fontsize{10.000000}{12.000000}\selectfont\catcode`\^=\active\def^{\ifmmode\sp\else\^{}\fi}\catcode`\%=\active\def%{\%}$\mathdefault{8}$}}%
\end{pgfscope}%
\begin{pgfscope}%
\pgfsetbuttcap%
\pgfsetroundjoin%
\definecolor{currentfill}{rgb}{0.000000,0.000000,0.000000}%
\pgfsetfillcolor{currentfill}%
\pgfsetlinewidth{0.803000pt}%
\definecolor{currentstroke}{rgb}{0.000000,0.000000,0.000000}%
\pgfsetstrokecolor{currentstroke}%
\pgfsetdash{}{0pt}%
\pgfsys@defobject{currentmarker}{\pgfqpoint{0.000000in}{-0.048611in}}{\pgfqpoint{0.000000in}{0.000000in}}{%
\pgfpathmoveto{\pgfqpoint{0.000000in}{0.000000in}}%
\pgfpathlineto{\pgfqpoint{0.000000in}{-0.048611in}}%
\pgfusepath{stroke,fill}%
}%
\begin{pgfscope}%
\pgfsys@transformshift{3.161684in}{0.521603in}%
\pgfsys@useobject{currentmarker}{}%
\end{pgfscope}%
\end{pgfscope}%
\begin{pgfscope}%
\definecolor{textcolor}{rgb}{0.000000,0.000000,0.000000}%
\pgfsetstrokecolor{textcolor}%
\pgfsetfillcolor{textcolor}%
\pgftext[x=3.161684in,y=0.424381in,,top]{\color{textcolor}{\rmfamily\fontsize{10.000000}{12.000000}\selectfont\catcode`\^=\active\def^{\ifmmode\sp\else\^{}\fi}\catcode`\%=\active\def%{\%}$\mathdefault{12}$}}%
\end{pgfscope}%
\begin{pgfscope}%
\pgfsetbuttcap%
\pgfsetroundjoin%
\definecolor{currentfill}{rgb}{0.000000,0.000000,0.000000}%
\pgfsetfillcolor{currentfill}%
\pgfsetlinewidth{0.803000pt}%
\definecolor{currentstroke}{rgb}{0.000000,0.000000,0.000000}%
\pgfsetstrokecolor{currentstroke}%
\pgfsetdash{}{0pt}%
\pgfsys@defobject{currentmarker}{\pgfqpoint{0.000000in}{-0.048611in}}{\pgfqpoint{0.000000in}{0.000000in}}{%
\pgfpathmoveto{\pgfqpoint{0.000000in}{0.000000in}}%
\pgfpathlineto{\pgfqpoint{0.000000in}{-0.048611in}}%
\pgfusepath{stroke,fill}%
}%
\begin{pgfscope}%
\pgfsys@transformshift{4.029937in}{0.521603in}%
\pgfsys@useobject{currentmarker}{}%
\end{pgfscope}%
\end{pgfscope}%
\begin{pgfscope}%
\definecolor{textcolor}{rgb}{0.000000,0.000000,0.000000}%
\pgfsetstrokecolor{textcolor}%
\pgfsetfillcolor{textcolor}%
\pgftext[x=4.029937in,y=0.424381in,,top]{\color{textcolor}{\rmfamily\fontsize{10.000000}{12.000000}\selectfont\catcode`\^=\active\def^{\ifmmode\sp\else\^{}\fi}\catcode`\%=\active\def%{\%}$\mathdefault{16}$}}%
\end{pgfscope}%
\begin{pgfscope}%
\pgfsetbuttcap%
\pgfsetroundjoin%
\definecolor{currentfill}{rgb}{0.000000,0.000000,0.000000}%
\pgfsetfillcolor{currentfill}%
\pgfsetlinewidth{0.803000pt}%
\definecolor{currentstroke}{rgb}{0.000000,0.000000,0.000000}%
\pgfsetstrokecolor{currentstroke}%
\pgfsetdash{}{0pt}%
\pgfsys@defobject{currentmarker}{\pgfqpoint{0.000000in}{-0.048611in}}{\pgfqpoint{0.000000in}{0.000000in}}{%
\pgfpathmoveto{\pgfqpoint{0.000000in}{0.000000in}}%
\pgfpathlineto{\pgfqpoint{0.000000in}{-0.048611in}}%
\pgfusepath{stroke,fill}%
}%
\begin{pgfscope}%
\pgfsys@transformshift{4.898190in}{0.521603in}%
\pgfsys@useobject{currentmarker}{}%
\end{pgfscope}%
\end{pgfscope}%
\begin{pgfscope}%
\definecolor{textcolor}{rgb}{0.000000,0.000000,0.000000}%
\pgfsetstrokecolor{textcolor}%
\pgfsetfillcolor{textcolor}%
\pgftext[x=4.898190in,y=0.424381in,,top]{\color{textcolor}{\rmfamily\fontsize{10.000000}{12.000000}\selectfont\catcode`\^=\active\def^{\ifmmode\sp\else\^{}\fi}\catcode`\%=\active\def%{\%}$\mathdefault{20}$}}%
\end{pgfscope}%
\begin{pgfscope}%
\pgfsetbuttcap%
\pgfsetroundjoin%
\definecolor{currentfill}{rgb}{0.000000,0.000000,0.000000}%
\pgfsetfillcolor{currentfill}%
\pgfsetlinewidth{0.803000pt}%
\definecolor{currentstroke}{rgb}{0.000000,0.000000,0.000000}%
\pgfsetstrokecolor{currentstroke}%
\pgfsetdash{}{0pt}%
\pgfsys@defobject{currentmarker}{\pgfqpoint{0.000000in}{-0.048611in}}{\pgfqpoint{0.000000in}{0.000000in}}{%
\pgfpathmoveto{\pgfqpoint{0.000000in}{0.000000in}}%
\pgfpathlineto{\pgfqpoint{0.000000in}{-0.048611in}}%
\pgfusepath{stroke,fill}%
}%
\begin{pgfscope}%
\pgfsys@transformshift{5.766443in}{0.521603in}%
\pgfsys@useobject{currentmarker}{}%
\end{pgfscope}%
\end{pgfscope}%
\begin{pgfscope}%
\definecolor{textcolor}{rgb}{0.000000,0.000000,0.000000}%
\pgfsetstrokecolor{textcolor}%
\pgfsetfillcolor{textcolor}%
\pgftext[x=5.766443in,y=0.424381in,,top]{\color{textcolor}{\rmfamily\fontsize{10.000000}{12.000000}\selectfont\catcode`\^=\active\def^{\ifmmode\sp\else\^{}\fi}\catcode`\%=\active\def%{\%}$\mathdefault{24}$}}%
\end{pgfscope}%
\begin{pgfscope}%
\pgfsetbuttcap%
\pgfsetroundjoin%
\definecolor{currentfill}{rgb}{0.000000,0.000000,0.000000}%
\pgfsetfillcolor{currentfill}%
\pgfsetlinewidth{0.803000pt}%
\definecolor{currentstroke}{rgb}{0.000000,0.000000,0.000000}%
\pgfsetstrokecolor{currentstroke}%
\pgfsetdash{}{0pt}%
\pgfsys@defobject{currentmarker}{\pgfqpoint{0.000000in}{-0.048611in}}{\pgfqpoint{0.000000in}{0.000000in}}{%
\pgfpathmoveto{\pgfqpoint{0.000000in}{0.000000in}}%
\pgfpathlineto{\pgfqpoint{0.000000in}{-0.048611in}}%
\pgfusepath{stroke,fill}%
}%
\begin{pgfscope}%
\pgfsys@transformshift{6.634696in}{0.521603in}%
\pgfsys@useobject{currentmarker}{}%
\end{pgfscope}%
\end{pgfscope}%
\begin{pgfscope}%
\definecolor{textcolor}{rgb}{0.000000,0.000000,0.000000}%
\pgfsetstrokecolor{textcolor}%
\pgfsetfillcolor{textcolor}%
\pgftext[x=6.634696in,y=0.424381in,,top]{\color{textcolor}{\rmfamily\fontsize{10.000000}{12.000000}\selectfont\catcode`\^=\active\def^{\ifmmode\sp\else\^{}\fi}\catcode`\%=\active\def%{\%}$\mathdefault{28}$}}%
\end{pgfscope}%
\begin{pgfscope}%
\pgfsetbuttcap%
\pgfsetroundjoin%
\definecolor{currentfill}{rgb}{0.000000,0.000000,0.000000}%
\pgfsetfillcolor{currentfill}%
\pgfsetlinewidth{0.803000pt}%
\definecolor{currentstroke}{rgb}{0.000000,0.000000,0.000000}%
\pgfsetstrokecolor{currentstroke}%
\pgfsetdash{}{0pt}%
\pgfsys@defobject{currentmarker}{\pgfqpoint{0.000000in}{-0.048611in}}{\pgfqpoint{0.000000in}{0.000000in}}{%
\pgfpathmoveto{\pgfqpoint{0.000000in}{0.000000in}}%
\pgfpathlineto{\pgfqpoint{0.000000in}{-0.048611in}}%
\pgfusepath{stroke,fill}%
}%
\begin{pgfscope}%
\pgfsys@transformshift{7.502949in}{0.521603in}%
\pgfsys@useobject{currentmarker}{}%
\end{pgfscope}%
\end{pgfscope}%
\begin{pgfscope}%
\definecolor{textcolor}{rgb}{0.000000,0.000000,0.000000}%
\pgfsetstrokecolor{textcolor}%
\pgfsetfillcolor{textcolor}%
\pgftext[x=7.502949in,y=0.424381in,,top]{\color{textcolor}{\rmfamily\fontsize{10.000000}{12.000000}\selectfont\catcode`\^=\active\def^{\ifmmode\sp\else\^{}\fi}\catcode`\%=\active\def%{\%}$\mathdefault{32}$}}%
\end{pgfscope}%
\begin{pgfscope}%
\definecolor{textcolor}{rgb}{0.000000,0.000000,0.000000}%
\pgfsetstrokecolor{textcolor}%
\pgfsetfillcolor{textcolor}%
\pgftext[x=4.464063in,y=0.234413in,,top]{\color{textcolor}{\rmfamily\fontsize{10.000000}{12.000000}\selectfont\catcode`\^=\active\def^{\ifmmode\sp\else\^{}\fi}\catcode`\%=\active\def%{\%}Monochromatic classes}}%
\end{pgfscope}%
\begin{pgfscope}%
\pgfsetbuttcap%
\pgfsetroundjoin%
\definecolor{currentfill}{rgb}{0.000000,0.000000,0.000000}%
\pgfsetfillcolor{currentfill}%
\pgfsetlinewidth{0.803000pt}%
\definecolor{currentstroke}{rgb}{0.000000,0.000000,0.000000}%
\pgfsetstrokecolor{currentstroke}%
\pgfsetdash{}{0pt}%
\pgfsys@defobject{currentmarker}{\pgfqpoint{-0.048611in}{0.000000in}}{\pgfqpoint{-0.000000in}{0.000000in}}{%
\pgfpathmoveto{\pgfqpoint{-0.000000in}{0.000000in}}%
\pgfpathlineto{\pgfqpoint{-0.048611in}{0.000000in}}%
\pgfusepath{stroke,fill}%
}%
\begin{pgfscope}%
\pgfsys@transformshift{0.643750in}{0.838318in}%
\pgfsys@useobject{currentmarker}{}%
\end{pgfscope}%
\end{pgfscope}%
\begin{pgfscope}%
\definecolor{textcolor}{rgb}{0.000000,0.000000,0.000000}%
\pgfsetstrokecolor{textcolor}%
\pgfsetfillcolor{textcolor}%
\pgftext[x=0.345331in, y=0.785556in, left, base]{\color{textcolor}{\rmfamily\fontsize{10.000000}{12.000000}\selectfont\catcode`\^=\active\def^{\ifmmode\sp\else\^{}\fi}\catcode`\%=\active\def%{\%}$\mathdefault{10^{2}}$}}%
\end{pgfscope}%
\begin{pgfscope}%
\pgfsetbuttcap%
\pgfsetroundjoin%
\definecolor{currentfill}{rgb}{0.000000,0.000000,0.000000}%
\pgfsetfillcolor{currentfill}%
\pgfsetlinewidth{0.803000pt}%
\definecolor{currentstroke}{rgb}{0.000000,0.000000,0.000000}%
\pgfsetstrokecolor{currentstroke}%
\pgfsetdash{}{0pt}%
\pgfsys@defobject{currentmarker}{\pgfqpoint{-0.048611in}{0.000000in}}{\pgfqpoint{-0.000000in}{0.000000in}}{%
\pgfpathmoveto{\pgfqpoint{-0.000000in}{0.000000in}}%
\pgfpathlineto{\pgfqpoint{-0.048611in}{0.000000in}}%
\pgfusepath{stroke,fill}%
}%
\begin{pgfscope}%
\pgfsys@transformshift{0.643750in}{1.219362in}%
\pgfsys@useobject{currentmarker}{}%
\end{pgfscope}%
\end{pgfscope}%
\begin{pgfscope}%
\definecolor{textcolor}{rgb}{0.000000,0.000000,0.000000}%
\pgfsetstrokecolor{textcolor}%
\pgfsetfillcolor{textcolor}%
\pgftext[x=0.345331in, y=1.166601in, left, base]{\color{textcolor}{\rmfamily\fontsize{10.000000}{12.000000}\selectfont\catcode`\^=\active\def^{\ifmmode\sp\else\^{}\fi}\catcode`\%=\active\def%{\%}$\mathdefault{10^{5}}$}}%
\end{pgfscope}%
\begin{pgfscope}%
\pgfsetbuttcap%
\pgfsetroundjoin%
\definecolor{currentfill}{rgb}{0.000000,0.000000,0.000000}%
\pgfsetfillcolor{currentfill}%
\pgfsetlinewidth{0.803000pt}%
\definecolor{currentstroke}{rgb}{0.000000,0.000000,0.000000}%
\pgfsetstrokecolor{currentstroke}%
\pgfsetdash{}{0pt}%
\pgfsys@defobject{currentmarker}{\pgfqpoint{-0.048611in}{0.000000in}}{\pgfqpoint{-0.000000in}{0.000000in}}{%
\pgfpathmoveto{\pgfqpoint{-0.000000in}{0.000000in}}%
\pgfpathlineto{\pgfqpoint{-0.048611in}{0.000000in}}%
\pgfusepath{stroke,fill}%
}%
\begin{pgfscope}%
\pgfsys@transformshift{0.643750in}{1.600407in}%
\pgfsys@useobject{currentmarker}{}%
\end{pgfscope}%
\end{pgfscope}%
\begin{pgfscope}%
\definecolor{textcolor}{rgb}{0.000000,0.000000,0.000000}%
\pgfsetstrokecolor{textcolor}%
\pgfsetfillcolor{textcolor}%
\pgftext[x=0.345331in, y=1.547645in, left, base]{\color{textcolor}{\rmfamily\fontsize{10.000000}{12.000000}\selectfont\catcode`\^=\active\def^{\ifmmode\sp\else\^{}\fi}\catcode`\%=\active\def%{\%}$\mathdefault{10^{8}}$}}%
\end{pgfscope}%
\begin{pgfscope}%
\pgfsetbuttcap%
\pgfsetroundjoin%
\definecolor{currentfill}{rgb}{0.000000,0.000000,0.000000}%
\pgfsetfillcolor{currentfill}%
\pgfsetlinewidth{0.803000pt}%
\definecolor{currentstroke}{rgb}{0.000000,0.000000,0.000000}%
\pgfsetstrokecolor{currentstroke}%
\pgfsetdash{}{0pt}%
\pgfsys@defobject{currentmarker}{\pgfqpoint{-0.048611in}{0.000000in}}{\pgfqpoint{-0.000000in}{0.000000in}}{%
\pgfpathmoveto{\pgfqpoint{-0.000000in}{0.000000in}}%
\pgfpathlineto{\pgfqpoint{-0.048611in}{0.000000in}}%
\pgfusepath{stroke,fill}%
}%
\begin{pgfscope}%
\pgfsys@transformshift{0.643750in}{1.981451in}%
\pgfsys@useobject{currentmarker}{}%
\end{pgfscope}%
\end{pgfscope}%
\begin{pgfscope}%
\definecolor{textcolor}{rgb}{0.000000,0.000000,0.000000}%
\pgfsetstrokecolor{textcolor}%
\pgfsetfillcolor{textcolor}%
\pgftext[x=0.289968in, y=1.928689in, left, base]{\color{textcolor}{\rmfamily\fontsize{10.000000}{12.000000}\selectfont\catcode`\^=\active\def^{\ifmmode\sp\else\^{}\fi}\catcode`\%=\active\def%{\%}$\mathdefault{10^{11}}$}}%
\end{pgfscope}%
\begin{pgfscope}%
\pgfsetbuttcap%
\pgfsetroundjoin%
\definecolor{currentfill}{rgb}{0.000000,0.000000,0.000000}%
\pgfsetfillcolor{currentfill}%
\pgfsetlinewidth{0.803000pt}%
\definecolor{currentstroke}{rgb}{0.000000,0.000000,0.000000}%
\pgfsetstrokecolor{currentstroke}%
\pgfsetdash{}{0pt}%
\pgfsys@defobject{currentmarker}{\pgfqpoint{-0.048611in}{0.000000in}}{\pgfqpoint{-0.000000in}{0.000000in}}{%
\pgfpathmoveto{\pgfqpoint{-0.000000in}{0.000000in}}%
\pgfpathlineto{\pgfqpoint{-0.048611in}{0.000000in}}%
\pgfusepath{stroke,fill}%
}%
\begin{pgfscope}%
\pgfsys@transformshift{0.643750in}{2.362495in}%
\pgfsys@useobject{currentmarker}{}%
\end{pgfscope}%
\end{pgfscope}%
\begin{pgfscope}%
\definecolor{textcolor}{rgb}{0.000000,0.000000,0.000000}%
\pgfsetstrokecolor{textcolor}%
\pgfsetfillcolor{textcolor}%
\pgftext[x=0.289968in, y=2.309734in, left, base]{\color{textcolor}{\rmfamily\fontsize{10.000000}{12.000000}\selectfont\catcode`\^=\active\def^{\ifmmode\sp\else\^{}\fi}\catcode`\%=\active\def%{\%}$\mathdefault{10^{14}}$}}%
\end{pgfscope}%
\begin{pgfscope}%
\definecolor{textcolor}{rgb}{0.000000,0.000000,0.000000}%
\pgfsetstrokecolor{textcolor}%
\pgfsetfillcolor{textcolor}%
\pgftext[x=0.234413in,y=1.631726in,,bottom,rotate=90.000000]{\color{textcolor}{\rmfamily\fontsize{10.000000}{12.000000}\selectfont\catcode`\^=\active\def^{\ifmmode\sp\else\^{}\fi}\catcode`\%=\active\def%{\%}Checks [call]}}%
\end{pgfscope}%
\begin{pgfscope}%
\pgfpathrectangle{\pgfqpoint{0.643750in}{0.521603in}}{\pgfqpoint{7.640626in}{2.220246in}}%
\pgfusepath{clip}%
\pgfsetrectcap%
\pgfsetroundjoin%
\pgfsetlinewidth{1.505625pt}%
\pgfsetstrokecolor{currentstroke1}%
\pgfsetdash{}{0pt}%
\pgfpathmoveto{\pgfqpoint{0.991051in}{2.618289in}}%
\pgfpathlineto{\pgfqpoint{1.208115in}{2.582972in}}%
\pgfpathlineto{\pgfqpoint{1.425178in}{2.640929in}}%
\pgfpathlineto{\pgfqpoint{1.642241in}{2.450208in}}%
\pgfpathlineto{\pgfqpoint{1.859304in}{2.556654in}}%
\pgfpathlineto{\pgfqpoint{2.076368in}{2.569786in}}%
\pgfpathlineto{\pgfqpoint{2.293431in}{2.405350in}}%
\pgfpathlineto{\pgfqpoint{2.510494in}{2.463250in}}%
\pgfpathlineto{\pgfqpoint{2.727557in}{2.353674in}}%
\pgfpathlineto{\pgfqpoint{2.944621in}{2.103172in}}%
\pgfpathlineto{\pgfqpoint{3.161684in}{2.428318in}}%
\pgfpathlineto{\pgfqpoint{3.378747in}{1.966533in}}%
\pgfpathlineto{\pgfqpoint{3.595810in}{2.221514in}}%
\pgfpathlineto{\pgfqpoint{3.812874in}{2.144665in}}%
\pgfpathlineto{\pgfqpoint{4.029937in}{2.262275in}}%
\pgfpathlineto{\pgfqpoint{4.247000in}{2.137084in}}%
\pgfpathlineto{\pgfqpoint{4.464063in}{2.114293in}}%
\pgfpathlineto{\pgfqpoint{4.681127in}{1.815053in}}%
\pgfpathlineto{\pgfqpoint{4.898190in}{1.956060in}}%
\pgfpathlineto{\pgfqpoint{5.115253in}{2.179567in}}%
\pgfpathlineto{\pgfqpoint{5.332316in}{1.986927in}}%
\pgfpathlineto{\pgfqpoint{5.549380in}{2.189365in}}%
\pgfpathlineto{\pgfqpoint{5.766443in}{2.061992in}}%
\pgfpathlineto{\pgfqpoint{5.983506in}{1.890173in}}%
\pgfpathlineto{\pgfqpoint{6.200569in}{2.160996in}}%
\pgfpathlineto{\pgfqpoint{6.417633in}{2.073458in}}%
\pgfpathlineto{\pgfqpoint{6.634696in}{1.724291in}}%
\pgfpathlineto{\pgfqpoint{6.851759in}{1.857906in}}%
\pgfpathlineto{\pgfqpoint{7.068822in}{2.304875in}}%
\pgfpathlineto{\pgfqpoint{7.285886in}{2.203264in}}%
\pgfpathlineto{\pgfqpoint{7.720012in}{1.884287in}}%
\pgfpathlineto{\pgfqpoint{7.937075in}{1.846052in}}%
\pgfusepath{stroke}%
\end{pgfscope}%
\begin{pgfscope}%
\pgfpathrectangle{\pgfqpoint{0.643750in}{0.521603in}}{\pgfqpoint{7.640626in}{2.220246in}}%
\pgfusepath{clip}%
\pgfsetrectcap%
\pgfsetroundjoin%
\pgfsetlinewidth{1.505625pt}%
\pgfsetstrokecolor{currentstroke2}%
\pgfsetdash{}{0pt}%
\pgfpathmoveto{\pgfqpoint{0.991051in}{1.151352in}}%
\pgfpathlineto{\pgfqpoint{1.208115in}{1.225999in}}%
\pgfpathlineto{\pgfqpoint{1.425178in}{1.201836in}}%
\pgfpathlineto{\pgfqpoint{1.642241in}{1.250919in}}%
\pgfpathlineto{\pgfqpoint{1.859304in}{1.228600in}}%
\pgfpathlineto{\pgfqpoint{2.076368in}{1.279886in}}%
\pgfpathlineto{\pgfqpoint{2.293431in}{1.214487in}}%
\pgfpathlineto{\pgfqpoint{2.510494in}{1.287904in}}%
\pgfpathlineto{\pgfqpoint{2.727557in}{1.318506in}}%
\pgfpathlineto{\pgfqpoint{2.944621in}{1.245160in}}%
\pgfpathlineto{\pgfqpoint{3.161684in}{1.331327in}}%
\pgfpathlineto{\pgfqpoint{3.378747in}{1.197408in}}%
\pgfpathlineto{\pgfqpoint{3.595810in}{1.316669in}}%
\pgfpathlineto{\pgfqpoint{3.812874in}{1.440288in}}%
\pgfpathlineto{\pgfqpoint{4.029937in}{1.450348in}}%
\pgfpathlineto{\pgfqpoint{4.247000in}{1.347073in}}%
\pgfpathlineto{\pgfqpoint{4.464063in}{1.357126in}}%
\pgfpathlineto{\pgfqpoint{4.681127in}{1.350706in}}%
\pgfpathlineto{\pgfqpoint{4.898190in}{1.377639in}}%
\pgfpathlineto{\pgfqpoint{5.115253in}{1.431670in}}%
\pgfpathlineto{\pgfqpoint{5.332316in}{1.443325in}}%
\pgfpathlineto{\pgfqpoint{5.549380in}{1.444025in}}%
\pgfpathlineto{\pgfqpoint{5.766443in}{1.519739in}}%
\pgfpathlineto{\pgfqpoint{5.983506in}{1.513442in}}%
\pgfpathlineto{\pgfqpoint{6.200569in}{1.641319in}}%
\pgfpathlineto{\pgfqpoint{6.417633in}{1.602465in}}%
\pgfpathlineto{\pgfqpoint{6.634696in}{1.619914in}}%
\pgfpathlineto{\pgfqpoint{6.851759in}{1.654876in}}%
\pgfpathlineto{\pgfqpoint{7.068822in}{1.846052in}}%
\pgfpathlineto{\pgfqpoint{7.285886in}{1.739458in}}%
\pgfpathlineto{\pgfqpoint{7.720012in}{1.807817in}}%
\pgfpathlineto{\pgfqpoint{7.937075in}{1.846052in}}%
\pgfusepath{stroke}%
\end{pgfscope}%
\begin{pgfscope}%
\pgfpathrectangle{\pgfqpoint{0.643750in}{0.521603in}}{\pgfqpoint{7.640626in}{2.220246in}}%
\pgfusepath{clip}%
\pgfsetrectcap%
\pgfsetroundjoin%
\pgfsetlinewidth{1.505625pt}%
\pgfsetstrokecolor{currentstroke3}%
\pgfsetdash{}{0pt}%
\pgfpathmoveto{\pgfqpoint{0.991051in}{0.622524in}}%
\pgfpathlineto{\pgfqpoint{1.208115in}{0.660759in}}%
\pgfpathlineto{\pgfqpoint{1.425178in}{0.698994in}}%
\pgfpathlineto{\pgfqpoint{1.642241in}{0.737229in}}%
\pgfpathlineto{\pgfqpoint{1.859304in}{0.775465in}}%
\pgfpathlineto{\pgfqpoint{2.076368in}{0.813700in}}%
\pgfpathlineto{\pgfqpoint{2.293431in}{0.851935in}}%
\pgfpathlineto{\pgfqpoint{2.510494in}{0.890170in}}%
\pgfpathlineto{\pgfqpoint{2.727557in}{0.928406in}}%
\pgfpathlineto{\pgfqpoint{2.944621in}{0.966641in}}%
\pgfpathlineto{\pgfqpoint{3.161684in}{1.004876in}}%
\pgfpathlineto{\pgfqpoint{3.378747in}{1.043111in}}%
\pgfpathlineto{\pgfqpoint{3.595810in}{1.081347in}}%
\pgfpathlineto{\pgfqpoint{3.812874in}{1.119582in}}%
\pgfpathlineto{\pgfqpoint{4.029937in}{1.157817in}}%
\pgfpathlineto{\pgfqpoint{4.247000in}{1.196053in}}%
\pgfpathlineto{\pgfqpoint{4.464063in}{1.234288in}}%
\pgfpathlineto{\pgfqpoint{4.681127in}{1.272523in}}%
\pgfpathlineto{\pgfqpoint{4.898190in}{1.310758in}}%
\pgfpathlineto{\pgfqpoint{5.115253in}{1.348994in}}%
\pgfpathlineto{\pgfqpoint{5.332316in}{1.387229in}}%
\pgfpathlineto{\pgfqpoint{5.549380in}{1.425464in}}%
\pgfpathlineto{\pgfqpoint{5.766443in}{1.463699in}}%
\pgfpathlineto{\pgfqpoint{5.983506in}{1.501935in}}%
\pgfpathlineto{\pgfqpoint{6.200569in}{1.540170in}}%
\pgfpathlineto{\pgfqpoint{6.417633in}{1.578405in}}%
\pgfpathlineto{\pgfqpoint{6.634696in}{1.616640in}}%
\pgfpathlineto{\pgfqpoint{6.851759in}{1.654876in}}%
\pgfpathlineto{\pgfqpoint{7.068822in}{1.693111in}}%
\pgfpathlineto{\pgfqpoint{7.285886in}{1.731346in}}%
\pgfpathlineto{\pgfqpoint{7.720012in}{1.807817in}}%
\pgfpathlineto{\pgfqpoint{7.937075in}{1.846052in}}%
\pgfusepath{stroke}%
\end{pgfscope}%
\begin{pgfscope}%
\pgfpathrectangle{\pgfqpoint{0.643750in}{0.521603in}}{\pgfqpoint{7.640626in}{2.220246in}}%
\pgfusepath{clip}%
\pgfsetrectcap%
\pgfsetroundjoin%
\pgfsetlinewidth{1.505625pt}%
\pgfsetstrokecolor{currentstroke4}%
\pgfsetdash{}{0pt}%
\pgfpathmoveto{\pgfqpoint{0.991051in}{0.622524in}}%
\pgfpathlineto{\pgfqpoint{1.208115in}{0.660759in}}%
\pgfpathlineto{\pgfqpoint{1.425178in}{0.698994in}}%
\pgfpathlineto{\pgfqpoint{1.642241in}{0.737229in}}%
\pgfpathlineto{\pgfqpoint{1.859304in}{0.775239in}}%
\pgfpathlineto{\pgfqpoint{2.076368in}{0.813700in}}%
\pgfpathlineto{\pgfqpoint{2.293431in}{0.845197in}}%
\pgfpathlineto{\pgfqpoint{2.510494in}{0.875033in}}%
\pgfpathlineto{\pgfqpoint{2.727557in}{0.910532in}}%
\pgfpathlineto{\pgfqpoint{2.944621in}{0.935624in}}%
\pgfpathlineto{\pgfqpoint{3.161684in}{0.950745in}}%
\pgfpathlineto{\pgfqpoint{3.378747in}{0.972291in}}%
\pgfpathlineto{\pgfqpoint{3.595810in}{0.998560in}}%
\pgfpathlineto{\pgfqpoint{3.812874in}{1.029885in}}%
\pgfpathlineto{\pgfqpoint{4.029937in}{1.052894in}}%
\pgfpathlineto{\pgfqpoint{4.247000in}{1.062798in}}%
\pgfpathlineto{\pgfqpoint{4.464063in}{1.112928in}}%
\pgfpathlineto{\pgfqpoint{4.681127in}{1.099172in}}%
\pgfpathlineto{\pgfqpoint{4.898190in}{1.132661in}}%
\pgfpathlineto{\pgfqpoint{5.115253in}{1.146570in}}%
\pgfpathlineto{\pgfqpoint{5.332316in}{1.166922in}}%
\pgfpathlineto{\pgfqpoint{5.549380in}{1.206202in}}%
\pgfpathlineto{\pgfqpoint{5.766443in}{1.209798in}}%
\pgfpathlineto{\pgfqpoint{5.983506in}{1.216637in}}%
\pgfpathlineto{\pgfqpoint{6.200569in}{1.257050in}}%
\pgfpathlineto{\pgfqpoint{6.417633in}{1.232813in}}%
\pgfpathlineto{\pgfqpoint{6.634696in}{1.328293in}}%
\pgfpathlineto{\pgfqpoint{6.851759in}{1.256856in}}%
\pgfpathlineto{\pgfqpoint{7.068822in}{1.197566in}}%
\pgfpathlineto{\pgfqpoint{7.285886in}{1.315866in}}%
\pgfpathlineto{\pgfqpoint{7.720012in}{1.245938in}}%
\pgfpathlineto{\pgfqpoint{7.937075in}{1.335354in}}%
\pgfusepath{stroke}%
\end{pgfscope}%
\begin{pgfscope}%
\pgfpathrectangle{\pgfqpoint{0.643750in}{0.521603in}}{\pgfqpoint{7.640626in}{2.220246in}}%
\pgfusepath{clip}%
\pgfsetrectcap%
\pgfsetroundjoin%
\pgfsetlinewidth{1.505625pt}%
\pgfsetstrokecolor{currentstroke5}%
\pgfsetdash{}{0pt}%
\pgfpathmoveto{\pgfqpoint{0.991051in}{0.622524in}}%
\pgfpathlineto{\pgfqpoint{1.208115in}{0.653568in}}%
\pgfpathlineto{\pgfqpoint{1.425178in}{0.664465in}}%
\pgfpathlineto{\pgfqpoint{1.642241in}{0.681651in}}%
\pgfpathlineto{\pgfqpoint{1.859304in}{0.700034in}}%
\pgfpathlineto{\pgfqpoint{2.076368in}{0.712586in}}%
\pgfpathlineto{\pgfqpoint{2.293431in}{0.765953in}}%
\pgfpathlineto{\pgfqpoint{2.510494in}{0.779809in}}%
\pgfpathlineto{\pgfqpoint{2.727557in}{0.798189in}}%
\pgfpathlineto{\pgfqpoint{2.944621in}{0.818015in}}%
\pgfpathlineto{\pgfqpoint{3.161684in}{0.846318in}}%
\pgfpathlineto{\pgfqpoint{3.378747in}{0.864005in}}%
\pgfpathlineto{\pgfqpoint{3.595810in}{0.883651in}}%
\pgfpathlineto{\pgfqpoint{3.812874in}{0.897401in}}%
\pgfpathlineto{\pgfqpoint{4.029937in}{0.945573in}}%
\pgfpathlineto{\pgfqpoint{4.247000in}{0.948404in}}%
\pgfpathlineto{\pgfqpoint{4.464063in}{1.003491in}}%
\pgfpathlineto{\pgfqpoint{4.681127in}{0.979329in}}%
\pgfpathlineto{\pgfqpoint{4.898190in}{1.017992in}}%
\pgfpathlineto{\pgfqpoint{5.115253in}{1.028499in}}%
\pgfpathlineto{\pgfqpoint{5.332316in}{1.053422in}}%
\pgfpathlineto{\pgfqpoint{5.549380in}{1.074575in}}%
\pgfpathlineto{\pgfqpoint{5.766443in}{1.098611in}}%
\pgfpathlineto{\pgfqpoint{5.983506in}{1.102044in}}%
\pgfpathlineto{\pgfqpoint{6.200569in}{1.128364in}}%
\pgfpathlineto{\pgfqpoint{6.417633in}{1.120351in}}%
\pgfpathlineto{\pgfqpoint{6.634696in}{1.178232in}}%
\pgfpathlineto{\pgfqpoint{6.851759in}{1.147348in}}%
\pgfpathlineto{\pgfqpoint{7.068822in}{1.023951in}}%
\pgfpathlineto{\pgfqpoint{7.285886in}{1.174764in}}%
\pgfpathlineto{\pgfqpoint{7.720012in}{1.177433in}}%
\pgfpathlineto{\pgfqpoint{7.937075in}{1.189897in}}%
\pgfusepath{stroke}%
\end{pgfscope}%
\begin{pgfscope}%
\pgfsetrectcap%
\pgfsetmiterjoin%
\pgfsetlinewidth{0.803000pt}%
\definecolor{currentstroke}{rgb}{0.000000,0.000000,0.000000}%
\pgfsetstrokecolor{currentstroke}%
\pgfsetdash{}{0pt}%
\pgfpathmoveto{\pgfqpoint{0.643750in}{0.521603in}}%
\pgfpathlineto{\pgfqpoint{0.643750in}{2.741849in}}%
\pgfusepath{stroke}%
\end{pgfscope}%
\begin{pgfscope}%
\pgfsetrectcap%
\pgfsetmiterjoin%
\pgfsetlinewidth{0.803000pt}%
\definecolor{currentstroke}{rgb}{0.000000,0.000000,0.000000}%
\pgfsetstrokecolor{currentstroke}%
\pgfsetdash{}{0pt}%
\pgfpathmoveto{\pgfqpoint{8.284376in}{0.521603in}}%
\pgfpathlineto{\pgfqpoint{8.284376in}{2.741849in}}%
\pgfusepath{stroke}%
\end{pgfscope}%
\begin{pgfscope}%
\pgfsetrectcap%
\pgfsetmiterjoin%
\pgfsetlinewidth{0.803000pt}%
\definecolor{currentstroke}{rgb}{0.000000,0.000000,0.000000}%
\pgfsetstrokecolor{currentstroke}%
\pgfsetdash{}{0pt}%
\pgfpathmoveto{\pgfqpoint{0.643750in}{0.521603in}}%
\pgfpathlineto{\pgfqpoint{8.284376in}{0.521603in}}%
\pgfusepath{stroke}%
\end{pgfscope}%
\begin{pgfscope}%
\pgfsetrectcap%
\pgfsetmiterjoin%
\pgfsetlinewidth{0.803000pt}%
\definecolor{currentstroke}{rgb}{0.000000,0.000000,0.000000}%
\pgfsetstrokecolor{currentstroke}%
\pgfsetdash{}{0pt}%
\pgfpathmoveto{\pgfqpoint{0.643750in}{2.741849in}}%
\pgfpathlineto{\pgfqpoint{8.284376in}{2.741849in}}%
\pgfusepath{stroke}%
\end{pgfscope}%
\begin{pgfscope}%
\pgfsetbuttcap%
\pgfsetmiterjoin%
\definecolor{currentfill}{rgb}{1.000000,1.000000,1.000000}%
\pgfsetfillcolor{currentfill}%
\pgfsetfillopacity{0.800000}%
\pgfsetlinewidth{1.003750pt}%
\definecolor{currentstroke}{rgb}{0.800000,0.800000,0.800000}%
\pgfsetstrokecolor{currentstroke}%
\pgfsetstrokeopacity{0.800000}%
\pgfsetdash{}{0pt}%
\pgfpathmoveto{\pgfqpoint{0.760417in}{1.385372in}}%
\pgfpathlineto{\pgfqpoint{4.097823in}{1.385372in}}%
\pgfpathquadraticcurveto{\pgfqpoint{4.131156in}{1.385372in}}{\pgfqpoint{4.131156in}{1.418705in}}%
\pgfpathlineto{\pgfqpoint{4.131156in}{2.625183in}}%
\pgfpathquadraticcurveto{\pgfqpoint{4.131156in}{2.658516in}}{\pgfqpoint{4.097823in}{2.658516in}}%
\pgfpathlineto{\pgfqpoint{0.760417in}{2.658516in}}%
\pgfpathquadraticcurveto{\pgfqpoint{0.727083in}{2.658516in}}{\pgfqpoint{0.727083in}{2.625183in}}%
\pgfpathlineto{\pgfqpoint{0.727083in}{1.418705in}}%
\pgfpathquadraticcurveto{\pgfqpoint{0.727083in}{1.385372in}}{\pgfqpoint{0.760417in}{1.385372in}}%
\pgfpathlineto{\pgfqpoint{0.760417in}{1.385372in}}%
\pgfpathclose%
\pgfusepath{stroke,fill}%
\end{pgfscope}%
\begin{pgfscope}%
\pgfsetrectcap%
\pgfsetroundjoin%
\pgfsetlinewidth{1.505625pt}%
\pgfsetstrokecolor{currentstroke1}%
\pgfsetdash{}{0pt}%
\pgfpathmoveto{\pgfqpoint{0.793750in}{2.523555in}}%
\pgfpathlineto{\pgfqpoint{0.960417in}{2.523555in}}%
\pgfpathlineto{\pgfqpoint{1.127083in}{2.523555in}}%
\pgfusepath{stroke}%
\end{pgfscope}%
\begin{pgfscope}%
\definecolor{textcolor}{rgb}{0.000000,0.000000,0.000000}%
\pgfsetstrokecolor{textcolor}%
\pgfsetfillcolor{textcolor}%
\pgftext[x=1.260417in,y=2.465222in,left,base]{\color{textcolor}{\rmfamily\fontsize{12.000000}{14.400000}\selectfont\catcode`\^=\active\def^{\ifmmode\sp\else\^{}\fi}\catcode`\%=\active\def%{\%}Naive - Edges}}%
\end{pgfscope}%
\begin{pgfscope}%
\pgfsetrectcap%
\pgfsetroundjoin%
\pgfsetlinewidth{1.505625pt}%
\pgfsetstrokecolor{currentstroke2}%
\pgfsetdash{}{0pt}%
\pgfpathmoveto{\pgfqpoint{0.793750in}{2.278926in}}%
\pgfpathlineto{\pgfqpoint{0.960417in}{2.278926in}}%
\pgfpathlineto{\pgfqpoint{1.127083in}{2.278926in}}%
\pgfusepath{stroke}%
\end{pgfscope}%
\begin{pgfscope}%
\definecolor{textcolor}{rgb}{0.000000,0.000000,0.000000}%
\pgfsetstrokecolor{textcolor}%
\pgfsetfillcolor{textcolor}%
\pgftext[x=1.260417in,y=2.220593in,left,base]{\color{textcolor}{\rmfamily\fontsize{12.000000}{14.400000}\selectfont\catcode`\^=\active\def^{\ifmmode\sp\else\^{}\fi}\catcode`\%=\active\def%{\%}Naive - $\triangle$-connected components}}%
\end{pgfscope}%
\begin{pgfscope}%
\pgfsetrectcap%
\pgfsetroundjoin%
\pgfsetlinewidth{1.505625pt}%
\pgfsetstrokecolor{currentstroke3}%
\pgfsetdash{}{0pt}%
\pgfpathmoveto{\pgfqpoint{0.793750in}{2.034297in}}%
\pgfpathlineto{\pgfqpoint{0.960417in}{2.034297in}}%
\pgfpathlineto{\pgfqpoint{1.127083in}{2.034297in}}%
\pgfusepath{stroke}%
\end{pgfscope}%
\begin{pgfscope}%
\definecolor{textcolor}{rgb}{0.000000,0.000000,0.000000}%
\pgfsetstrokecolor{textcolor}%
\pgfsetfillcolor{textcolor}%
\pgftext[x=1.260417in,y=1.975964in,left,base]{\color{textcolor}{\rmfamily\fontsize{12.000000}{14.400000}\selectfont\catcode`\^=\active\def^{\ifmmode\sp\else\^{}\fi}\catcode`\%=\active\def%{\%}Naive - Monochromatic classes}}%
\end{pgfscope}%
\begin{pgfscope}%
\pgfsetrectcap%
\pgfsetroundjoin%
\pgfsetlinewidth{1.505625pt}%
\pgfsetstrokecolor{currentstroke4}%
\pgfsetdash{}{0pt}%
\pgfpathmoveto{\pgfqpoint{0.793750in}{1.789669in}}%
\pgfpathlineto{\pgfqpoint{0.960417in}{1.789669in}}%
\pgfpathlineto{\pgfqpoint{1.127083in}{1.789669in}}%
\pgfusepath{stroke}%
\end{pgfscope}%
\begin{pgfscope}%
\definecolor{textcolor}{rgb}{0.000000,0.000000,0.000000}%
\pgfsetstrokecolor{textcolor}%
\pgfsetfillcolor{textcolor}%
\pgftext[x=1.260417in,y=1.731335in,left,base]{\color{textcolor}{\rmfamily\fontsize{12.000000}{14.400000}\selectfont\catcode`\^=\active\def^{\ifmmode\sp\else\^{}\fi}\catcode`\%=\active\def%{\%}Subgraphs - \CycleMask{}}}%
\end{pgfscope}%
\begin{pgfscope}%
\pgfsetrectcap%
\pgfsetroundjoin%
\pgfsetlinewidth{1.505625pt}%
\pgfsetstrokecolor{currentstroke5}%
\pgfsetdash{}{0pt}%
\pgfpathmoveto{\pgfqpoint{0.793750in}{1.545040in}}%
\pgfpathlineto{\pgfqpoint{0.960417in}{1.545040in}}%
\pgfpathlineto{\pgfqpoint{1.127083in}{1.545040in}}%
\pgfusepath{stroke}%
\end{pgfscope}%
\begin{pgfscope}%
\definecolor{textcolor}{rgb}{0.000000,0.000000,0.000000}%
\pgfsetstrokecolor{textcolor}%
\pgfsetfillcolor{textcolor}%
\pgftext[x=1.260417in,y=1.486707in,left,base]{\color{textcolor}{\rmfamily\fontsize{12.000000}{14.400000}\selectfont\catcode`\^=\active\def^{\ifmmode\sp\else\^{}\fi}\catcode`\%=\active\def%{\%}Subgraphs - \IsNACColoring{}}}%
\end{pgfscope}%
\end{pgfpicture}%
\makeatother%
\endgroup%
}
	\caption[The number of \IsNACColoring{} calls]{
		The number of \IsNACColoring{} calls with respect to the number of monochromatic classes
		over all graphs used for benchmarking.}%
	\label{fig:graph_summary}
\end{figure}%



\subsection{Performance on specific graph classes}%
\label{sec:bench_graph_classes}

Each benchmark was run two or three times and the mean was taken.
The graphs are grouped either by the number of vertices
monochromatic classes or \trcon{} components, see respective \(x\)-axis.
Overall, over 350k \emph{benchmarks runs}
(a graph of a graph class run with a splitting strategy, merging strategy
and target subgraph size)
are presented (and 200k more were run for strategies not worth mentioning)
on over 12.8k graphs from multiple graph classes.
%
There are six splitting strategies, eight merging strategies, and usually two target subgraph sizes.
There is also one more for \NaiveCycles{}.
Therefore, the whole heuristic state space has ninety-seven combinations,
and thus we could have run up to 1.2 million benchmark runs on our dataset.
We omitted many benchmark runs for strategies where we saw the  performance is poor.
%
We tested mostly only graphs with up to one hundred vertices
as it is computationally hard to find larger graphs in the classes given
and to run enough benchmarks for them.
A NAC-coloring can be found for these graphs
in hundreds of milliseconds or seconds at most.

First, we only show strategies that performed generally well,
we show the others later in \Cref{sec:other_strategies}.
If a configuration did not finish in the given time limit,
we replace the runtime field with the limit of the benchmark, usually 5 seconds.
These runs are excluded from figures with the number of check calls.


\subsubsection*{Minimally rigid graphs}

In the previous section, we showed performance of the algorithm for listing
all NAC-colorings of minimally rigid graphs,
but did not compare the strategies among each other.
%
\Neighbors{} and \NeighborsDegree{} perform slightly better than \None{} and
\SharedVertices{} outperforms \MergeLinear{} slightly.
The same also holds for the number of \IsNACColoring{} checks.
Notice that the runtime grows exponentially, but with lower factor than \NaiveCycles{}.
The growth is expected as the number of NAC-coloring grows fast.

In \Cref{fig:graph_minimally_rigid_first_runtime,fig:graph_minimally_rigid_first_checks}
we focus on finding some NAC-coloring of minimally rigid graphs.
%
Minimally rigid graphs have large number of NAC-colorings,
therefore it is simple for both \NaiveCycles{}
and \Subgraphs{} algorithms to find some NAC-coloring,
and it can be seen that for larger graphs, the required runtime
does not grow significantly.
%
Note that minimally rigid graphs have no NAC-coloring if and only if they are formed from
a single \trcon{} component (resp. they are a 2-tree, recall \Cref{lemma:stable_cut_or_2_tree}).
Therefore, such instances do not worsen runtime performance as they are resolved instantly.
%
\NaiveCycles{} is faster as it has lower internal overhead.
The number of \IsNACColoring{} checks is also lower,
that is probably because \Subgraphs{} strategies do additional checks
while merging, which are not needed for \NaiveCycles{}.
\SharedVertices{} behaves unpredictably while \MergeLinear{} is consistent
for both the runtime and the number of \IsNACColoring{} checks.

\begin{figure}[thbp]
	\centering
	\scalebox{\BenchFigureScale}{%% Creator: Matplotlib, PGF backend
%%
%% To include the figure in your LaTeX document, write
%%   \input{<filename>.pgf}
%%
%% Make sure the required packages are loaded in your preamble
%%   \usepackage{pgf}
%%
%% Also ensure that all the required font packages are loaded; for instance,
%% the lmodern package is sometimes necessary when using math font.
%%   \usepackage{lmodern}
%%
%% Figures using additional raster images can only be included by \input if
%% they are in the same directory as the main LaTeX file. For loading figures
%% from other directories you can use the `import` package
%%   \usepackage{import}
%%
%% and then include the figures with
%%   \import{<path to file>}{<filename>.pgf}
%%
%% Matplotlib used the following preamble
%%   \def\mathdefault#1{#1}
%%   \everymath=\expandafter{\the\everymath\displaystyle}
%%   \IfFileExists{scrextend.sty}{
%%     \usepackage[fontsize=10.000000pt]{scrextend}
%%   }{
%%     \renewcommand{\normalsize}{\fontsize{10.000000}{12.000000}\selectfont}
%%     \normalsize
%%   }
%%   
%%   \ifdefined\pdftexversion\else  % non-pdftex case.
%%     \usepackage{fontspec}
%%     \setmainfont{DejaVuSans.ttf}[Path=\detokenize{/home/petr/Projects/PyRigi/.venv/lib/python3.12/site-packages/matplotlib/mpl-data/fonts/ttf/}]
%%     \setsansfont{DejaVuSans.ttf}[Path=\detokenize{/home/petr/Projects/PyRigi/.venv/lib/python3.12/site-packages/matplotlib/mpl-data/fonts/ttf/}]
%%     \setmonofont{DejaVuSansMono.ttf}[Path=\detokenize{/home/petr/Projects/PyRigi/.venv/lib/python3.12/site-packages/matplotlib/mpl-data/fonts/ttf/}]
%%   \fi
%%   \makeatletter\@ifpackageloaded{under\Score{}}{}{\usepackage[strings]{under\Score{}}}\makeatother
%%
\begingroup%
\makeatletter%
\begin{pgfpicture}%
\pgfpathrectangle{\pgfpointorigin}{\pgfqpoint{8.384376in}{2.841849in}}%
\pgfusepath{use as bounding box, clip}%
\begin{pgfscope}%
\pgfsetbuttcap%
\pgfsetmiterjoin%
\definecolor{currentfill}{rgb}{1.000000,1.000000,1.000000}%
\pgfsetfillcolor{currentfill}%
\pgfsetlinewidth{0.000000pt}%
\definecolor{currentstroke}{rgb}{1.000000,1.000000,1.000000}%
\pgfsetstrokecolor{currentstroke}%
\pgfsetdash{}{0pt}%
\pgfpathmoveto{\pgfqpoint{0.000000in}{0.000000in}}%
\pgfpathlineto{\pgfqpoint{8.384376in}{0.000000in}}%
\pgfpathlineto{\pgfqpoint{8.384376in}{2.841849in}}%
\pgfpathlineto{\pgfqpoint{0.000000in}{2.841849in}}%
\pgfpathlineto{\pgfqpoint{0.000000in}{0.000000in}}%
\pgfpathclose%
\pgfusepath{fill}%
\end{pgfscope}%
\begin{pgfscope}%
\pgfsetbuttcap%
\pgfsetmiterjoin%
\definecolor{currentfill}{rgb}{1.000000,1.000000,1.000000}%
\pgfsetfillcolor{currentfill}%
\pgfsetlinewidth{0.000000pt}%
\definecolor{currentstroke}{rgb}{0.000000,0.000000,0.000000}%
\pgfsetstrokecolor{currentstroke}%
\pgfsetstrokeopacity{0.000000}%
\pgfsetdash{}{0pt}%
\pgfpathmoveto{\pgfqpoint{0.588387in}{0.521603in}}%
\pgfpathlineto{\pgfqpoint{5.257411in}{0.521603in}}%
\pgfpathlineto{\pgfqpoint{5.257411in}{2.741849in}}%
\pgfpathlineto{\pgfqpoint{0.588387in}{2.741849in}}%
\pgfpathlineto{\pgfqpoint{0.588387in}{0.521603in}}%
\pgfpathclose%
\pgfusepath{fill}%
\end{pgfscope}%
\begin{pgfscope}%
\pgfsetbuttcap%
\pgfsetroundjoin%
\definecolor{currentfill}{rgb}{0.000000,0.000000,0.000000}%
\pgfsetfillcolor{currentfill}%
\pgfsetlinewidth{0.803000pt}%
\definecolor{currentstroke}{rgb}{0.000000,0.000000,0.000000}%
\pgfsetstrokecolor{currentstroke}%
\pgfsetdash{}{0pt}%
\pgfsys@defobject{currentmarker}{\pgfqpoint{0.000000in}{-0.048611in}}{\pgfqpoint{0.000000in}{0.000000in}}{%
\pgfpathmoveto{\pgfqpoint{0.000000in}{0.000000in}}%
\pgfpathlineto{\pgfqpoint{0.000000in}{-0.048611in}}%
\pgfusepath{stroke,fill}%
}%
\begin{pgfscope}%
\pgfsys@transformshift{0.718197in}{0.521603in}%
\pgfsys@useobject{currentmarker}{}%
\end{pgfscope}%
\end{pgfscope}%
\begin{pgfscope}%
\definecolor{textcolor}{rgb}{0.000000,0.000000,0.000000}%
\pgfsetstrokecolor{textcolor}%
\pgfsetfillcolor{textcolor}%
\pgftext[x=0.718197in,y=0.424381in,,top]{\color{textcolor}{\rmfamily\fontsize{10.000000}{12.000000}\selectfont\catcode`\^=\active\def^{\ifmmode\sp\else\^{}\fi}\catcode`\%=\active\def%{\%}$\mathdefault{0}$}}%
\end{pgfscope}%
\begin{pgfscope}%
\pgfsetbuttcap%
\pgfsetroundjoin%
\definecolor{currentfill}{rgb}{0.000000,0.000000,0.000000}%
\pgfsetfillcolor{currentfill}%
\pgfsetlinewidth{0.803000pt}%
\definecolor{currentstroke}{rgb}{0.000000,0.000000,0.000000}%
\pgfsetstrokecolor{currentstroke}%
\pgfsetdash{}{0pt}%
\pgfsys@defobject{currentmarker}{\pgfqpoint{0.000000in}{-0.048611in}}{\pgfqpoint{0.000000in}{0.000000in}}{%
\pgfpathmoveto{\pgfqpoint{0.000000in}{0.000000in}}%
\pgfpathlineto{\pgfqpoint{0.000000in}{-0.048611in}}%
\pgfusepath{stroke,fill}%
}%
\begin{pgfscope}%
\pgfsys@transformshift{1.336338in}{0.521603in}%
\pgfsys@useobject{currentmarker}{}%
\end{pgfscope}%
\end{pgfscope}%
\begin{pgfscope}%
\definecolor{textcolor}{rgb}{0.000000,0.000000,0.000000}%
\pgfsetstrokecolor{textcolor}%
\pgfsetfillcolor{textcolor}%
\pgftext[x=1.336338in,y=0.424381in,,top]{\color{textcolor}{\rmfamily\fontsize{10.000000}{12.000000}\selectfont\catcode`\^=\active\def^{\ifmmode\sp\else\^{}\fi}\catcode`\%=\active\def%{\%}$\mathdefault{15}$}}%
\end{pgfscope}%
\begin{pgfscope}%
\pgfsetbuttcap%
\pgfsetroundjoin%
\definecolor{currentfill}{rgb}{0.000000,0.000000,0.000000}%
\pgfsetfillcolor{currentfill}%
\pgfsetlinewidth{0.803000pt}%
\definecolor{currentstroke}{rgb}{0.000000,0.000000,0.000000}%
\pgfsetstrokecolor{currentstroke}%
\pgfsetdash{}{0pt}%
\pgfsys@defobject{currentmarker}{\pgfqpoint{0.000000in}{-0.048611in}}{\pgfqpoint{0.000000in}{0.000000in}}{%
\pgfpathmoveto{\pgfqpoint{0.000000in}{0.000000in}}%
\pgfpathlineto{\pgfqpoint{0.000000in}{-0.048611in}}%
\pgfusepath{stroke,fill}%
}%
\begin{pgfscope}%
\pgfsys@transformshift{1.954479in}{0.521603in}%
\pgfsys@useobject{currentmarker}{}%
\end{pgfscope}%
\end{pgfscope}%
\begin{pgfscope}%
\definecolor{textcolor}{rgb}{0.000000,0.000000,0.000000}%
\pgfsetstrokecolor{textcolor}%
\pgfsetfillcolor{textcolor}%
\pgftext[x=1.954479in,y=0.424381in,,top]{\color{textcolor}{\rmfamily\fontsize{10.000000}{12.000000}\selectfont\catcode`\^=\active\def^{\ifmmode\sp\else\^{}\fi}\catcode`\%=\active\def%{\%}$\mathdefault{30}$}}%
\end{pgfscope}%
\begin{pgfscope}%
\pgfsetbuttcap%
\pgfsetroundjoin%
\definecolor{currentfill}{rgb}{0.000000,0.000000,0.000000}%
\pgfsetfillcolor{currentfill}%
\pgfsetlinewidth{0.803000pt}%
\definecolor{currentstroke}{rgb}{0.000000,0.000000,0.000000}%
\pgfsetstrokecolor{currentstroke}%
\pgfsetdash{}{0pt}%
\pgfsys@defobject{currentmarker}{\pgfqpoint{0.000000in}{-0.048611in}}{\pgfqpoint{0.000000in}{0.000000in}}{%
\pgfpathmoveto{\pgfqpoint{0.000000in}{0.000000in}}%
\pgfpathlineto{\pgfqpoint{0.000000in}{-0.048611in}}%
\pgfusepath{stroke,fill}%
}%
\begin{pgfscope}%
\pgfsys@transformshift{2.572619in}{0.521603in}%
\pgfsys@useobject{currentmarker}{}%
\end{pgfscope}%
\end{pgfscope}%
\begin{pgfscope}%
\definecolor{textcolor}{rgb}{0.000000,0.000000,0.000000}%
\pgfsetstrokecolor{textcolor}%
\pgfsetfillcolor{textcolor}%
\pgftext[x=2.572619in,y=0.424381in,,top]{\color{textcolor}{\rmfamily\fontsize{10.000000}{12.000000}\selectfont\catcode`\^=\active\def^{\ifmmode\sp\else\^{}\fi}\catcode`\%=\active\def%{\%}$\mathdefault{45}$}}%
\end{pgfscope}%
\begin{pgfscope}%
\pgfsetbuttcap%
\pgfsetroundjoin%
\definecolor{currentfill}{rgb}{0.000000,0.000000,0.000000}%
\pgfsetfillcolor{currentfill}%
\pgfsetlinewidth{0.803000pt}%
\definecolor{currentstroke}{rgb}{0.000000,0.000000,0.000000}%
\pgfsetstrokecolor{currentstroke}%
\pgfsetdash{}{0pt}%
\pgfsys@defobject{currentmarker}{\pgfqpoint{0.000000in}{-0.048611in}}{\pgfqpoint{0.000000in}{0.000000in}}{%
\pgfpathmoveto{\pgfqpoint{0.000000in}{0.000000in}}%
\pgfpathlineto{\pgfqpoint{0.000000in}{-0.048611in}}%
\pgfusepath{stroke,fill}%
}%
\begin{pgfscope}%
\pgfsys@transformshift{3.190760in}{0.521603in}%
\pgfsys@useobject{currentmarker}{}%
\end{pgfscope}%
\end{pgfscope}%
\begin{pgfscope}%
\definecolor{textcolor}{rgb}{0.000000,0.000000,0.000000}%
\pgfsetstrokecolor{textcolor}%
\pgfsetfillcolor{textcolor}%
\pgftext[x=3.190760in,y=0.424381in,,top]{\color{textcolor}{\rmfamily\fontsize{10.000000}{12.000000}\selectfont\catcode`\^=\active\def^{\ifmmode\sp\else\^{}\fi}\catcode`\%=\active\def%{\%}$\mathdefault{60}$}}%
\end{pgfscope}%
\begin{pgfscope}%
\pgfsetbuttcap%
\pgfsetroundjoin%
\definecolor{currentfill}{rgb}{0.000000,0.000000,0.000000}%
\pgfsetfillcolor{currentfill}%
\pgfsetlinewidth{0.803000pt}%
\definecolor{currentstroke}{rgb}{0.000000,0.000000,0.000000}%
\pgfsetstrokecolor{currentstroke}%
\pgfsetdash{}{0pt}%
\pgfsys@defobject{currentmarker}{\pgfqpoint{0.000000in}{-0.048611in}}{\pgfqpoint{0.000000in}{0.000000in}}{%
\pgfpathmoveto{\pgfqpoint{0.000000in}{0.000000in}}%
\pgfpathlineto{\pgfqpoint{0.000000in}{-0.048611in}}%
\pgfusepath{stroke,fill}%
}%
\begin{pgfscope}%
\pgfsys@transformshift{3.808901in}{0.521603in}%
\pgfsys@useobject{currentmarker}{}%
\end{pgfscope}%
\end{pgfscope}%
\begin{pgfscope}%
\definecolor{textcolor}{rgb}{0.000000,0.000000,0.000000}%
\pgfsetstrokecolor{textcolor}%
\pgfsetfillcolor{textcolor}%
\pgftext[x=3.808901in,y=0.424381in,,top]{\color{textcolor}{\rmfamily\fontsize{10.000000}{12.000000}\selectfont\catcode`\^=\active\def^{\ifmmode\sp\else\^{}\fi}\catcode`\%=\active\def%{\%}$\mathdefault{75}$}}%
\end{pgfscope}%
\begin{pgfscope}%
\pgfsetbuttcap%
\pgfsetroundjoin%
\definecolor{currentfill}{rgb}{0.000000,0.000000,0.000000}%
\pgfsetfillcolor{currentfill}%
\pgfsetlinewidth{0.803000pt}%
\definecolor{currentstroke}{rgb}{0.000000,0.000000,0.000000}%
\pgfsetstrokecolor{currentstroke}%
\pgfsetdash{}{0pt}%
\pgfsys@defobject{currentmarker}{\pgfqpoint{0.000000in}{-0.048611in}}{\pgfqpoint{0.000000in}{0.000000in}}{%
\pgfpathmoveto{\pgfqpoint{0.000000in}{0.000000in}}%
\pgfpathlineto{\pgfqpoint{0.000000in}{-0.048611in}}%
\pgfusepath{stroke,fill}%
}%
\begin{pgfscope}%
\pgfsys@transformshift{4.427042in}{0.521603in}%
\pgfsys@useobject{currentmarker}{}%
\end{pgfscope}%
\end{pgfscope}%
\begin{pgfscope}%
\definecolor{textcolor}{rgb}{0.000000,0.000000,0.000000}%
\pgfsetstrokecolor{textcolor}%
\pgfsetfillcolor{textcolor}%
\pgftext[x=4.427042in,y=0.424381in,,top]{\color{textcolor}{\rmfamily\fontsize{10.000000}{12.000000}\selectfont\catcode`\^=\active\def^{\ifmmode\sp\else\^{}\fi}\catcode`\%=\active\def%{\%}$\mathdefault{90}$}}%
\end{pgfscope}%
\begin{pgfscope}%
\pgfsetbuttcap%
\pgfsetroundjoin%
\definecolor{currentfill}{rgb}{0.000000,0.000000,0.000000}%
\pgfsetfillcolor{currentfill}%
\pgfsetlinewidth{0.803000pt}%
\definecolor{currentstroke}{rgb}{0.000000,0.000000,0.000000}%
\pgfsetstrokecolor{currentstroke}%
\pgfsetdash{}{0pt}%
\pgfsys@defobject{currentmarker}{\pgfqpoint{0.000000in}{-0.048611in}}{\pgfqpoint{0.000000in}{0.000000in}}{%
\pgfpathmoveto{\pgfqpoint{0.000000in}{0.000000in}}%
\pgfpathlineto{\pgfqpoint{0.000000in}{-0.048611in}}%
\pgfusepath{stroke,fill}%
}%
\begin{pgfscope}%
\pgfsys@transformshift{5.045183in}{0.521603in}%
\pgfsys@useobject{currentmarker}{}%
\end{pgfscope}%
\end{pgfscope}%
\begin{pgfscope}%
\definecolor{textcolor}{rgb}{0.000000,0.000000,0.000000}%
\pgfsetstrokecolor{textcolor}%
\pgfsetfillcolor{textcolor}%
\pgftext[x=5.045183in,y=0.424381in,,top]{\color{textcolor}{\rmfamily\fontsize{10.000000}{12.000000}\selectfont\catcode`\^=\active\def^{\ifmmode\sp\else\^{}\fi}\catcode`\%=\active\def%{\%}$\mathdefault{105}$}}%
\end{pgfscope}%
\begin{pgfscope}%
\definecolor{textcolor}{rgb}{0.000000,0.000000,0.000000}%
\pgfsetstrokecolor{textcolor}%
\pgfsetfillcolor{textcolor}%
\pgftext[x=2.922899in,y=0.234413in,,top]{\color{textcolor}{\rmfamily\fontsize{10.000000}{12.000000}\selectfont\catcode`\^=\active\def^{\ifmmode\sp\else\^{}\fi}\catcode`\%=\active\def%{\%}Monochromatic classes}}%
\end{pgfscope}%
\begin{pgfscope}%
\pgfsetbuttcap%
\pgfsetroundjoin%
\definecolor{currentfill}{rgb}{0.000000,0.000000,0.000000}%
\pgfsetfillcolor{currentfill}%
\pgfsetlinewidth{0.803000pt}%
\definecolor{currentstroke}{rgb}{0.000000,0.000000,0.000000}%
\pgfsetstrokecolor{currentstroke}%
\pgfsetdash{}{0pt}%
\pgfsys@defobject{currentmarker}{\pgfqpoint{-0.048611in}{0.000000in}}{\pgfqpoint{-0.000000in}{0.000000in}}{%
\pgfpathmoveto{\pgfqpoint{-0.000000in}{0.000000in}}%
\pgfpathlineto{\pgfqpoint{-0.048611in}{0.000000in}}%
\pgfusepath{stroke,fill}%
}%
\begin{pgfscope}%
\pgfsys@transformshift{0.588387in}{1.046565in}%
\pgfsys@useobject{currentmarker}{}%
\end{pgfscope}%
\end{pgfscope}%
\begin{pgfscope}%
\definecolor{textcolor}{rgb}{0.000000,0.000000,0.000000}%
\pgfsetstrokecolor{textcolor}%
\pgfsetfillcolor{textcolor}%
\pgftext[x=0.289968in, y=0.993803in, left, base]{\color{textcolor}{\rmfamily\fontsize{10.000000}{12.000000}\selectfont\catcode`\^=\active\def^{\ifmmode\sp\else\^{}\fi}\catcode`\%=\active\def%{\%}$\mathdefault{10^{1}}$}}%
\end{pgfscope}%
\begin{pgfscope}%
\pgfsetbuttcap%
\pgfsetroundjoin%
\definecolor{currentfill}{rgb}{0.000000,0.000000,0.000000}%
\pgfsetfillcolor{currentfill}%
\pgfsetlinewidth{0.803000pt}%
\definecolor{currentstroke}{rgb}{0.000000,0.000000,0.000000}%
\pgfsetstrokecolor{currentstroke}%
\pgfsetdash{}{0pt}%
\pgfsys@defobject{currentmarker}{\pgfqpoint{-0.048611in}{0.000000in}}{\pgfqpoint{-0.000000in}{0.000000in}}{%
\pgfpathmoveto{\pgfqpoint{-0.000000in}{0.000000in}}%
\pgfpathlineto{\pgfqpoint{-0.048611in}{0.000000in}}%
\pgfusepath{stroke,fill}%
}%
\begin{pgfscope}%
\pgfsys@transformshift{0.588387in}{1.657004in}%
\pgfsys@useobject{currentmarker}{}%
\end{pgfscope}%
\end{pgfscope}%
\begin{pgfscope}%
\definecolor{textcolor}{rgb}{0.000000,0.000000,0.000000}%
\pgfsetstrokecolor{textcolor}%
\pgfsetfillcolor{textcolor}%
\pgftext[x=0.289968in, y=1.604243in, left, base]{\color{textcolor}{\rmfamily\fontsize{10.000000}{12.000000}\selectfont\catcode`\^=\active\def^{\ifmmode\sp\else\^{}\fi}\catcode`\%=\active\def%{\%}$\mathdefault{10^{2}}$}}%
\end{pgfscope}%
\begin{pgfscope}%
\pgfsetbuttcap%
\pgfsetroundjoin%
\definecolor{currentfill}{rgb}{0.000000,0.000000,0.000000}%
\pgfsetfillcolor{currentfill}%
\pgfsetlinewidth{0.803000pt}%
\definecolor{currentstroke}{rgb}{0.000000,0.000000,0.000000}%
\pgfsetstrokecolor{currentstroke}%
\pgfsetdash{}{0pt}%
\pgfsys@defobject{currentmarker}{\pgfqpoint{-0.048611in}{0.000000in}}{\pgfqpoint{-0.000000in}{0.000000in}}{%
\pgfpathmoveto{\pgfqpoint{-0.000000in}{0.000000in}}%
\pgfpathlineto{\pgfqpoint{-0.048611in}{0.000000in}}%
\pgfusepath{stroke,fill}%
}%
\begin{pgfscope}%
\pgfsys@transformshift{0.588387in}{2.267444in}%
\pgfsys@useobject{currentmarker}{}%
\end{pgfscope}%
\end{pgfscope}%
\begin{pgfscope}%
\definecolor{textcolor}{rgb}{0.000000,0.000000,0.000000}%
\pgfsetstrokecolor{textcolor}%
\pgfsetfillcolor{textcolor}%
\pgftext[x=0.289968in, y=2.214682in, left, base]{\color{textcolor}{\rmfamily\fontsize{10.000000}{12.000000}\selectfont\catcode`\^=\active\def^{\ifmmode\sp\else\^{}\fi}\catcode`\%=\active\def%{\%}$\mathdefault{10^{3}}$}}%
\end{pgfscope}%
\begin{pgfscope}%
\pgfsetbuttcap%
\pgfsetroundjoin%
\definecolor{currentfill}{rgb}{0.000000,0.000000,0.000000}%
\pgfsetfillcolor{currentfill}%
\pgfsetlinewidth{0.602250pt}%
\definecolor{currentstroke}{rgb}{0.000000,0.000000,0.000000}%
\pgfsetstrokecolor{currentstroke}%
\pgfsetdash{}{0pt}%
\pgfsys@defobject{currentmarker}{\pgfqpoint{-0.027778in}{0.000000in}}{\pgfqpoint{-0.000000in}{0.000000in}}{%
\pgfpathmoveto{\pgfqpoint{-0.000000in}{0.000000in}}%
\pgfpathlineto{\pgfqpoint{-0.027778in}{0.000000in}}%
\pgfusepath{stroke,fill}%
}%
\begin{pgfscope}%
\pgfsys@transformshift{0.588387in}{0.619886in}%
\pgfsys@useobject{currentmarker}{}%
\end{pgfscope}%
\end{pgfscope}%
\begin{pgfscope}%
\pgfsetbuttcap%
\pgfsetroundjoin%
\definecolor{currentfill}{rgb}{0.000000,0.000000,0.000000}%
\pgfsetfillcolor{currentfill}%
\pgfsetlinewidth{0.602250pt}%
\definecolor{currentstroke}{rgb}{0.000000,0.000000,0.000000}%
\pgfsetstrokecolor{currentstroke}%
\pgfsetdash{}{0pt}%
\pgfsys@defobject{currentmarker}{\pgfqpoint{-0.027778in}{0.000000in}}{\pgfqpoint{-0.000000in}{0.000000in}}{%
\pgfpathmoveto{\pgfqpoint{-0.000000in}{0.000000in}}%
\pgfpathlineto{\pgfqpoint{-0.027778in}{0.000000in}}%
\pgfusepath{stroke,fill}%
}%
\begin{pgfscope}%
\pgfsys@transformshift{0.588387in}{0.727379in}%
\pgfsys@useobject{currentmarker}{}%
\end{pgfscope}%
\end{pgfscope}%
\begin{pgfscope}%
\pgfsetbuttcap%
\pgfsetroundjoin%
\definecolor{currentfill}{rgb}{0.000000,0.000000,0.000000}%
\pgfsetfillcolor{currentfill}%
\pgfsetlinewidth{0.602250pt}%
\definecolor{currentstroke}{rgb}{0.000000,0.000000,0.000000}%
\pgfsetstrokecolor{currentstroke}%
\pgfsetdash{}{0pt}%
\pgfsys@defobject{currentmarker}{\pgfqpoint{-0.027778in}{0.000000in}}{\pgfqpoint{-0.000000in}{0.000000in}}{%
\pgfpathmoveto{\pgfqpoint{-0.000000in}{0.000000in}}%
\pgfpathlineto{\pgfqpoint{-0.027778in}{0.000000in}}%
\pgfusepath{stroke,fill}%
}%
\begin{pgfscope}%
\pgfsys@transformshift{0.588387in}{0.803646in}%
\pgfsys@useobject{currentmarker}{}%
\end{pgfscope}%
\end{pgfscope}%
\begin{pgfscope}%
\pgfsetbuttcap%
\pgfsetroundjoin%
\definecolor{currentfill}{rgb}{0.000000,0.000000,0.000000}%
\pgfsetfillcolor{currentfill}%
\pgfsetlinewidth{0.602250pt}%
\definecolor{currentstroke}{rgb}{0.000000,0.000000,0.000000}%
\pgfsetstrokecolor{currentstroke}%
\pgfsetdash{}{0pt}%
\pgfsys@defobject{currentmarker}{\pgfqpoint{-0.027778in}{0.000000in}}{\pgfqpoint{-0.000000in}{0.000000in}}{%
\pgfpathmoveto{\pgfqpoint{-0.000000in}{0.000000in}}%
\pgfpathlineto{\pgfqpoint{-0.027778in}{0.000000in}}%
\pgfusepath{stroke,fill}%
}%
\begin{pgfscope}%
\pgfsys@transformshift{0.588387in}{0.862804in}%
\pgfsys@useobject{currentmarker}{}%
\end{pgfscope}%
\end{pgfscope}%
\begin{pgfscope}%
\pgfsetbuttcap%
\pgfsetroundjoin%
\definecolor{currentfill}{rgb}{0.000000,0.000000,0.000000}%
\pgfsetfillcolor{currentfill}%
\pgfsetlinewidth{0.602250pt}%
\definecolor{currentstroke}{rgb}{0.000000,0.000000,0.000000}%
\pgfsetstrokecolor{currentstroke}%
\pgfsetdash{}{0pt}%
\pgfsys@defobject{currentmarker}{\pgfqpoint{-0.027778in}{0.000000in}}{\pgfqpoint{-0.000000in}{0.000000in}}{%
\pgfpathmoveto{\pgfqpoint{-0.000000in}{0.000000in}}%
\pgfpathlineto{\pgfqpoint{-0.027778in}{0.000000in}}%
\pgfusepath{stroke,fill}%
}%
\begin{pgfscope}%
\pgfsys@transformshift{0.588387in}{0.911139in}%
\pgfsys@useobject{currentmarker}{}%
\end{pgfscope}%
\end{pgfscope}%
\begin{pgfscope}%
\pgfsetbuttcap%
\pgfsetroundjoin%
\definecolor{currentfill}{rgb}{0.000000,0.000000,0.000000}%
\pgfsetfillcolor{currentfill}%
\pgfsetlinewidth{0.602250pt}%
\definecolor{currentstroke}{rgb}{0.000000,0.000000,0.000000}%
\pgfsetstrokecolor{currentstroke}%
\pgfsetdash{}{0pt}%
\pgfsys@defobject{currentmarker}{\pgfqpoint{-0.027778in}{0.000000in}}{\pgfqpoint{-0.000000in}{0.000000in}}{%
\pgfpathmoveto{\pgfqpoint{-0.000000in}{0.000000in}}%
\pgfpathlineto{\pgfqpoint{-0.027778in}{0.000000in}}%
\pgfusepath{stroke,fill}%
}%
\begin{pgfscope}%
\pgfsys@transformshift{0.588387in}{0.952006in}%
\pgfsys@useobject{currentmarker}{}%
\end{pgfscope}%
\end{pgfscope}%
\begin{pgfscope}%
\pgfsetbuttcap%
\pgfsetroundjoin%
\definecolor{currentfill}{rgb}{0.000000,0.000000,0.000000}%
\pgfsetfillcolor{currentfill}%
\pgfsetlinewidth{0.602250pt}%
\definecolor{currentstroke}{rgb}{0.000000,0.000000,0.000000}%
\pgfsetstrokecolor{currentstroke}%
\pgfsetdash{}{0pt}%
\pgfsys@defobject{currentmarker}{\pgfqpoint{-0.027778in}{0.000000in}}{\pgfqpoint{-0.000000in}{0.000000in}}{%
\pgfpathmoveto{\pgfqpoint{-0.000000in}{0.000000in}}%
\pgfpathlineto{\pgfqpoint{-0.027778in}{0.000000in}}%
\pgfusepath{stroke,fill}%
}%
\begin{pgfscope}%
\pgfsys@transformshift{0.588387in}{0.987407in}%
\pgfsys@useobject{currentmarker}{}%
\end{pgfscope}%
\end{pgfscope}%
\begin{pgfscope}%
\pgfsetbuttcap%
\pgfsetroundjoin%
\definecolor{currentfill}{rgb}{0.000000,0.000000,0.000000}%
\pgfsetfillcolor{currentfill}%
\pgfsetlinewidth{0.602250pt}%
\definecolor{currentstroke}{rgb}{0.000000,0.000000,0.000000}%
\pgfsetstrokecolor{currentstroke}%
\pgfsetdash{}{0pt}%
\pgfsys@defobject{currentmarker}{\pgfqpoint{-0.027778in}{0.000000in}}{\pgfqpoint{-0.000000in}{0.000000in}}{%
\pgfpathmoveto{\pgfqpoint{-0.000000in}{0.000000in}}%
\pgfpathlineto{\pgfqpoint{-0.027778in}{0.000000in}}%
\pgfusepath{stroke,fill}%
}%
\begin{pgfscope}%
\pgfsys@transformshift{0.588387in}{1.018632in}%
\pgfsys@useobject{currentmarker}{}%
\end{pgfscope}%
\end{pgfscope}%
\begin{pgfscope}%
\pgfsetbuttcap%
\pgfsetroundjoin%
\definecolor{currentfill}{rgb}{0.000000,0.000000,0.000000}%
\pgfsetfillcolor{currentfill}%
\pgfsetlinewidth{0.602250pt}%
\definecolor{currentstroke}{rgb}{0.000000,0.000000,0.000000}%
\pgfsetstrokecolor{currentstroke}%
\pgfsetdash{}{0pt}%
\pgfsys@defobject{currentmarker}{\pgfqpoint{-0.027778in}{0.000000in}}{\pgfqpoint{-0.000000in}{0.000000in}}{%
\pgfpathmoveto{\pgfqpoint{-0.000000in}{0.000000in}}%
\pgfpathlineto{\pgfqpoint{-0.027778in}{0.000000in}}%
\pgfusepath{stroke,fill}%
}%
\begin{pgfscope}%
\pgfsys@transformshift{0.588387in}{1.230325in}%
\pgfsys@useobject{currentmarker}{}%
\end{pgfscope}%
\end{pgfscope}%
\begin{pgfscope}%
\pgfsetbuttcap%
\pgfsetroundjoin%
\definecolor{currentfill}{rgb}{0.000000,0.000000,0.000000}%
\pgfsetfillcolor{currentfill}%
\pgfsetlinewidth{0.602250pt}%
\definecolor{currentstroke}{rgb}{0.000000,0.000000,0.000000}%
\pgfsetstrokecolor{currentstroke}%
\pgfsetdash{}{0pt}%
\pgfsys@defobject{currentmarker}{\pgfqpoint{-0.027778in}{0.000000in}}{\pgfqpoint{-0.000000in}{0.000000in}}{%
\pgfpathmoveto{\pgfqpoint{-0.000000in}{0.000000in}}%
\pgfpathlineto{\pgfqpoint{-0.027778in}{0.000000in}}%
\pgfusepath{stroke,fill}%
}%
\begin{pgfscope}%
\pgfsys@transformshift{0.588387in}{1.337818in}%
\pgfsys@useobject{currentmarker}{}%
\end{pgfscope}%
\end{pgfscope}%
\begin{pgfscope}%
\pgfsetbuttcap%
\pgfsetroundjoin%
\definecolor{currentfill}{rgb}{0.000000,0.000000,0.000000}%
\pgfsetfillcolor{currentfill}%
\pgfsetlinewidth{0.602250pt}%
\definecolor{currentstroke}{rgb}{0.000000,0.000000,0.000000}%
\pgfsetstrokecolor{currentstroke}%
\pgfsetdash{}{0pt}%
\pgfsys@defobject{currentmarker}{\pgfqpoint{-0.027778in}{0.000000in}}{\pgfqpoint{-0.000000in}{0.000000in}}{%
\pgfpathmoveto{\pgfqpoint{-0.000000in}{0.000000in}}%
\pgfpathlineto{\pgfqpoint{-0.027778in}{0.000000in}}%
\pgfusepath{stroke,fill}%
}%
\begin{pgfscope}%
\pgfsys@transformshift{0.588387in}{1.414086in}%
\pgfsys@useobject{currentmarker}{}%
\end{pgfscope}%
\end{pgfscope}%
\begin{pgfscope}%
\pgfsetbuttcap%
\pgfsetroundjoin%
\definecolor{currentfill}{rgb}{0.000000,0.000000,0.000000}%
\pgfsetfillcolor{currentfill}%
\pgfsetlinewidth{0.602250pt}%
\definecolor{currentstroke}{rgb}{0.000000,0.000000,0.000000}%
\pgfsetstrokecolor{currentstroke}%
\pgfsetdash{}{0pt}%
\pgfsys@defobject{currentmarker}{\pgfqpoint{-0.027778in}{0.000000in}}{\pgfqpoint{-0.000000in}{0.000000in}}{%
\pgfpathmoveto{\pgfqpoint{-0.000000in}{0.000000in}}%
\pgfpathlineto{\pgfqpoint{-0.027778in}{0.000000in}}%
\pgfusepath{stroke,fill}%
}%
\begin{pgfscope}%
\pgfsys@transformshift{0.588387in}{1.473244in}%
\pgfsys@useobject{currentmarker}{}%
\end{pgfscope}%
\end{pgfscope}%
\begin{pgfscope}%
\pgfsetbuttcap%
\pgfsetroundjoin%
\definecolor{currentfill}{rgb}{0.000000,0.000000,0.000000}%
\pgfsetfillcolor{currentfill}%
\pgfsetlinewidth{0.602250pt}%
\definecolor{currentstroke}{rgb}{0.000000,0.000000,0.000000}%
\pgfsetstrokecolor{currentstroke}%
\pgfsetdash{}{0pt}%
\pgfsys@defobject{currentmarker}{\pgfqpoint{-0.027778in}{0.000000in}}{\pgfqpoint{-0.000000in}{0.000000in}}{%
\pgfpathmoveto{\pgfqpoint{-0.000000in}{0.000000in}}%
\pgfpathlineto{\pgfqpoint{-0.027778in}{0.000000in}}%
\pgfusepath{stroke,fill}%
}%
\begin{pgfscope}%
\pgfsys@transformshift{0.588387in}{1.521579in}%
\pgfsys@useobject{currentmarker}{}%
\end{pgfscope}%
\end{pgfscope}%
\begin{pgfscope}%
\pgfsetbuttcap%
\pgfsetroundjoin%
\definecolor{currentfill}{rgb}{0.000000,0.000000,0.000000}%
\pgfsetfillcolor{currentfill}%
\pgfsetlinewidth{0.602250pt}%
\definecolor{currentstroke}{rgb}{0.000000,0.000000,0.000000}%
\pgfsetstrokecolor{currentstroke}%
\pgfsetdash{}{0pt}%
\pgfsys@defobject{currentmarker}{\pgfqpoint{-0.027778in}{0.000000in}}{\pgfqpoint{-0.000000in}{0.000000in}}{%
\pgfpathmoveto{\pgfqpoint{-0.000000in}{0.000000in}}%
\pgfpathlineto{\pgfqpoint{-0.027778in}{0.000000in}}%
\pgfusepath{stroke,fill}%
}%
\begin{pgfscope}%
\pgfsys@transformshift{0.588387in}{1.562446in}%
\pgfsys@useobject{currentmarker}{}%
\end{pgfscope}%
\end{pgfscope}%
\begin{pgfscope}%
\pgfsetbuttcap%
\pgfsetroundjoin%
\definecolor{currentfill}{rgb}{0.000000,0.000000,0.000000}%
\pgfsetfillcolor{currentfill}%
\pgfsetlinewidth{0.602250pt}%
\definecolor{currentstroke}{rgb}{0.000000,0.000000,0.000000}%
\pgfsetstrokecolor{currentstroke}%
\pgfsetdash{}{0pt}%
\pgfsys@defobject{currentmarker}{\pgfqpoint{-0.027778in}{0.000000in}}{\pgfqpoint{-0.000000in}{0.000000in}}{%
\pgfpathmoveto{\pgfqpoint{-0.000000in}{0.000000in}}%
\pgfpathlineto{\pgfqpoint{-0.027778in}{0.000000in}}%
\pgfusepath{stroke,fill}%
}%
\begin{pgfscope}%
\pgfsys@transformshift{0.588387in}{1.597847in}%
\pgfsys@useobject{currentmarker}{}%
\end{pgfscope}%
\end{pgfscope}%
\begin{pgfscope}%
\pgfsetbuttcap%
\pgfsetroundjoin%
\definecolor{currentfill}{rgb}{0.000000,0.000000,0.000000}%
\pgfsetfillcolor{currentfill}%
\pgfsetlinewidth{0.602250pt}%
\definecolor{currentstroke}{rgb}{0.000000,0.000000,0.000000}%
\pgfsetstrokecolor{currentstroke}%
\pgfsetdash{}{0pt}%
\pgfsys@defobject{currentmarker}{\pgfqpoint{-0.027778in}{0.000000in}}{\pgfqpoint{-0.000000in}{0.000000in}}{%
\pgfpathmoveto{\pgfqpoint{-0.000000in}{0.000000in}}%
\pgfpathlineto{\pgfqpoint{-0.027778in}{0.000000in}}%
\pgfusepath{stroke,fill}%
}%
\begin{pgfscope}%
\pgfsys@transformshift{0.588387in}{1.629072in}%
\pgfsys@useobject{currentmarker}{}%
\end{pgfscope}%
\end{pgfscope}%
\begin{pgfscope}%
\pgfsetbuttcap%
\pgfsetroundjoin%
\definecolor{currentfill}{rgb}{0.000000,0.000000,0.000000}%
\pgfsetfillcolor{currentfill}%
\pgfsetlinewidth{0.602250pt}%
\definecolor{currentstroke}{rgb}{0.000000,0.000000,0.000000}%
\pgfsetstrokecolor{currentstroke}%
\pgfsetdash{}{0pt}%
\pgfsys@defobject{currentmarker}{\pgfqpoint{-0.027778in}{0.000000in}}{\pgfqpoint{-0.000000in}{0.000000in}}{%
\pgfpathmoveto{\pgfqpoint{-0.000000in}{0.000000in}}%
\pgfpathlineto{\pgfqpoint{-0.027778in}{0.000000in}}%
\pgfusepath{stroke,fill}%
}%
\begin{pgfscope}%
\pgfsys@transformshift{0.588387in}{1.840765in}%
\pgfsys@useobject{currentmarker}{}%
\end{pgfscope}%
\end{pgfscope}%
\begin{pgfscope}%
\pgfsetbuttcap%
\pgfsetroundjoin%
\definecolor{currentfill}{rgb}{0.000000,0.000000,0.000000}%
\pgfsetfillcolor{currentfill}%
\pgfsetlinewidth{0.602250pt}%
\definecolor{currentstroke}{rgb}{0.000000,0.000000,0.000000}%
\pgfsetstrokecolor{currentstroke}%
\pgfsetdash{}{0pt}%
\pgfsys@defobject{currentmarker}{\pgfqpoint{-0.027778in}{0.000000in}}{\pgfqpoint{-0.000000in}{0.000000in}}{%
\pgfpathmoveto{\pgfqpoint{-0.000000in}{0.000000in}}%
\pgfpathlineto{\pgfqpoint{-0.027778in}{0.000000in}}%
\pgfusepath{stroke,fill}%
}%
\begin{pgfscope}%
\pgfsys@transformshift{0.588387in}{1.948258in}%
\pgfsys@useobject{currentmarker}{}%
\end{pgfscope}%
\end{pgfscope}%
\begin{pgfscope}%
\pgfsetbuttcap%
\pgfsetroundjoin%
\definecolor{currentfill}{rgb}{0.000000,0.000000,0.000000}%
\pgfsetfillcolor{currentfill}%
\pgfsetlinewidth{0.602250pt}%
\definecolor{currentstroke}{rgb}{0.000000,0.000000,0.000000}%
\pgfsetstrokecolor{currentstroke}%
\pgfsetdash{}{0pt}%
\pgfsys@defobject{currentmarker}{\pgfqpoint{-0.027778in}{0.000000in}}{\pgfqpoint{-0.000000in}{0.000000in}}{%
\pgfpathmoveto{\pgfqpoint{-0.000000in}{0.000000in}}%
\pgfpathlineto{\pgfqpoint{-0.027778in}{0.000000in}}%
\pgfusepath{stroke,fill}%
}%
\begin{pgfscope}%
\pgfsys@transformshift{0.588387in}{2.024526in}%
\pgfsys@useobject{currentmarker}{}%
\end{pgfscope}%
\end{pgfscope}%
\begin{pgfscope}%
\pgfsetbuttcap%
\pgfsetroundjoin%
\definecolor{currentfill}{rgb}{0.000000,0.000000,0.000000}%
\pgfsetfillcolor{currentfill}%
\pgfsetlinewidth{0.602250pt}%
\definecolor{currentstroke}{rgb}{0.000000,0.000000,0.000000}%
\pgfsetstrokecolor{currentstroke}%
\pgfsetdash{}{0pt}%
\pgfsys@defobject{currentmarker}{\pgfqpoint{-0.027778in}{0.000000in}}{\pgfqpoint{-0.000000in}{0.000000in}}{%
\pgfpathmoveto{\pgfqpoint{-0.000000in}{0.000000in}}%
\pgfpathlineto{\pgfqpoint{-0.027778in}{0.000000in}}%
\pgfusepath{stroke,fill}%
}%
\begin{pgfscope}%
\pgfsys@transformshift{0.588387in}{2.083683in}%
\pgfsys@useobject{currentmarker}{}%
\end{pgfscope}%
\end{pgfscope}%
\begin{pgfscope}%
\pgfsetbuttcap%
\pgfsetroundjoin%
\definecolor{currentfill}{rgb}{0.000000,0.000000,0.000000}%
\pgfsetfillcolor{currentfill}%
\pgfsetlinewidth{0.602250pt}%
\definecolor{currentstroke}{rgb}{0.000000,0.000000,0.000000}%
\pgfsetstrokecolor{currentstroke}%
\pgfsetdash{}{0pt}%
\pgfsys@defobject{currentmarker}{\pgfqpoint{-0.027778in}{0.000000in}}{\pgfqpoint{-0.000000in}{0.000000in}}{%
\pgfpathmoveto{\pgfqpoint{-0.000000in}{0.000000in}}%
\pgfpathlineto{\pgfqpoint{-0.027778in}{0.000000in}}%
\pgfusepath{stroke,fill}%
}%
\begin{pgfscope}%
\pgfsys@transformshift{0.588387in}{2.132019in}%
\pgfsys@useobject{currentmarker}{}%
\end{pgfscope}%
\end{pgfscope}%
\begin{pgfscope}%
\pgfsetbuttcap%
\pgfsetroundjoin%
\definecolor{currentfill}{rgb}{0.000000,0.000000,0.000000}%
\pgfsetfillcolor{currentfill}%
\pgfsetlinewidth{0.602250pt}%
\definecolor{currentstroke}{rgb}{0.000000,0.000000,0.000000}%
\pgfsetstrokecolor{currentstroke}%
\pgfsetdash{}{0pt}%
\pgfsys@defobject{currentmarker}{\pgfqpoint{-0.027778in}{0.000000in}}{\pgfqpoint{-0.000000in}{0.000000in}}{%
\pgfpathmoveto{\pgfqpoint{-0.000000in}{0.000000in}}%
\pgfpathlineto{\pgfqpoint{-0.027778in}{0.000000in}}%
\pgfusepath{stroke,fill}%
}%
\begin{pgfscope}%
\pgfsys@transformshift{0.588387in}{2.172886in}%
\pgfsys@useobject{currentmarker}{}%
\end{pgfscope}%
\end{pgfscope}%
\begin{pgfscope}%
\pgfsetbuttcap%
\pgfsetroundjoin%
\definecolor{currentfill}{rgb}{0.000000,0.000000,0.000000}%
\pgfsetfillcolor{currentfill}%
\pgfsetlinewidth{0.602250pt}%
\definecolor{currentstroke}{rgb}{0.000000,0.000000,0.000000}%
\pgfsetstrokecolor{currentstroke}%
\pgfsetdash{}{0pt}%
\pgfsys@defobject{currentmarker}{\pgfqpoint{-0.027778in}{0.000000in}}{\pgfqpoint{-0.000000in}{0.000000in}}{%
\pgfpathmoveto{\pgfqpoint{-0.000000in}{0.000000in}}%
\pgfpathlineto{\pgfqpoint{-0.027778in}{0.000000in}}%
\pgfusepath{stroke,fill}%
}%
\begin{pgfscope}%
\pgfsys@transformshift{0.588387in}{2.208286in}%
\pgfsys@useobject{currentmarker}{}%
\end{pgfscope}%
\end{pgfscope}%
\begin{pgfscope}%
\pgfsetbuttcap%
\pgfsetroundjoin%
\definecolor{currentfill}{rgb}{0.000000,0.000000,0.000000}%
\pgfsetfillcolor{currentfill}%
\pgfsetlinewidth{0.602250pt}%
\definecolor{currentstroke}{rgb}{0.000000,0.000000,0.000000}%
\pgfsetstrokecolor{currentstroke}%
\pgfsetdash{}{0pt}%
\pgfsys@defobject{currentmarker}{\pgfqpoint{-0.027778in}{0.000000in}}{\pgfqpoint{-0.000000in}{0.000000in}}{%
\pgfpathmoveto{\pgfqpoint{-0.000000in}{0.000000in}}%
\pgfpathlineto{\pgfqpoint{-0.027778in}{0.000000in}}%
\pgfusepath{stroke,fill}%
}%
\begin{pgfscope}%
\pgfsys@transformshift{0.588387in}{2.239512in}%
\pgfsys@useobject{currentmarker}{}%
\end{pgfscope}%
\end{pgfscope}%
\begin{pgfscope}%
\pgfsetbuttcap%
\pgfsetroundjoin%
\definecolor{currentfill}{rgb}{0.000000,0.000000,0.000000}%
\pgfsetfillcolor{currentfill}%
\pgfsetlinewidth{0.602250pt}%
\definecolor{currentstroke}{rgb}{0.000000,0.000000,0.000000}%
\pgfsetstrokecolor{currentstroke}%
\pgfsetdash{}{0pt}%
\pgfsys@defobject{currentmarker}{\pgfqpoint{-0.027778in}{0.000000in}}{\pgfqpoint{-0.000000in}{0.000000in}}{%
\pgfpathmoveto{\pgfqpoint{-0.000000in}{0.000000in}}%
\pgfpathlineto{\pgfqpoint{-0.027778in}{0.000000in}}%
\pgfusepath{stroke,fill}%
}%
\begin{pgfscope}%
\pgfsys@transformshift{0.588387in}{2.451205in}%
\pgfsys@useobject{currentmarker}{}%
\end{pgfscope}%
\end{pgfscope}%
\begin{pgfscope}%
\pgfsetbuttcap%
\pgfsetroundjoin%
\definecolor{currentfill}{rgb}{0.000000,0.000000,0.000000}%
\pgfsetfillcolor{currentfill}%
\pgfsetlinewidth{0.602250pt}%
\definecolor{currentstroke}{rgb}{0.000000,0.000000,0.000000}%
\pgfsetstrokecolor{currentstroke}%
\pgfsetdash{}{0pt}%
\pgfsys@defobject{currentmarker}{\pgfqpoint{-0.027778in}{0.000000in}}{\pgfqpoint{-0.000000in}{0.000000in}}{%
\pgfpathmoveto{\pgfqpoint{-0.000000in}{0.000000in}}%
\pgfpathlineto{\pgfqpoint{-0.027778in}{0.000000in}}%
\pgfusepath{stroke,fill}%
}%
\begin{pgfscope}%
\pgfsys@transformshift{0.588387in}{2.558698in}%
\pgfsys@useobject{currentmarker}{}%
\end{pgfscope}%
\end{pgfscope}%
\begin{pgfscope}%
\pgfsetbuttcap%
\pgfsetroundjoin%
\definecolor{currentfill}{rgb}{0.000000,0.000000,0.000000}%
\pgfsetfillcolor{currentfill}%
\pgfsetlinewidth{0.602250pt}%
\definecolor{currentstroke}{rgb}{0.000000,0.000000,0.000000}%
\pgfsetstrokecolor{currentstroke}%
\pgfsetdash{}{0pt}%
\pgfsys@defobject{currentmarker}{\pgfqpoint{-0.027778in}{0.000000in}}{\pgfqpoint{-0.000000in}{0.000000in}}{%
\pgfpathmoveto{\pgfqpoint{-0.000000in}{0.000000in}}%
\pgfpathlineto{\pgfqpoint{-0.027778in}{0.000000in}}%
\pgfusepath{stroke,fill}%
}%
\begin{pgfscope}%
\pgfsys@transformshift{0.588387in}{2.634965in}%
\pgfsys@useobject{currentmarker}{}%
\end{pgfscope}%
\end{pgfscope}%
\begin{pgfscope}%
\pgfsetbuttcap%
\pgfsetroundjoin%
\definecolor{currentfill}{rgb}{0.000000,0.000000,0.000000}%
\pgfsetfillcolor{currentfill}%
\pgfsetlinewidth{0.602250pt}%
\definecolor{currentstroke}{rgb}{0.000000,0.000000,0.000000}%
\pgfsetstrokecolor{currentstroke}%
\pgfsetdash{}{0pt}%
\pgfsys@defobject{currentmarker}{\pgfqpoint{-0.027778in}{0.000000in}}{\pgfqpoint{-0.000000in}{0.000000in}}{%
\pgfpathmoveto{\pgfqpoint{-0.000000in}{0.000000in}}%
\pgfpathlineto{\pgfqpoint{-0.027778in}{0.000000in}}%
\pgfusepath{stroke,fill}%
}%
\begin{pgfscope}%
\pgfsys@transformshift{0.588387in}{2.694123in}%
\pgfsys@useobject{currentmarker}{}%
\end{pgfscope}%
\end{pgfscope}%
\begin{pgfscope}%
\definecolor{textcolor}{rgb}{0.000000,0.000000,0.000000}%
\pgfsetstrokecolor{textcolor}%
\pgfsetfillcolor{textcolor}%
\pgftext[x=0.234413in,y=1.631726in,,bottom,rotate=90.000000]{\color{textcolor}{\rmfamily\fontsize{10.000000}{12.000000}\selectfont\catcode`\^=\active\def^{\ifmmode\sp\else\^{}\fi}\catcode`\%=\active\def%{\%}Time [ms]}}%
\end{pgfscope}%
\begin{pgfscope}%
\pgfpathrectangle{\pgfqpoint{0.588387in}{0.521603in}}{\pgfqpoint{4.669024in}{2.220246in}}%
\pgfusepath{clip}%
\pgfsetrectcap%
\pgfsetroundjoin%
\pgfsetlinewidth{1.505625pt}%
\pgfsetstrokecolor{currentstroke1}%
\pgfsetdash{}{0pt}%
\pgfpathmoveto{\pgfqpoint{0.800616in}{0.817980in}}%
\pgfpathlineto{\pgfqpoint{0.841825in}{0.841584in}}%
\pgfpathlineto{\pgfqpoint{0.883034in}{0.844513in}}%
\pgfpathlineto{\pgfqpoint{0.924244in}{0.780579in}}%
\pgfpathlineto{\pgfqpoint{0.965453in}{0.705357in}}%
\pgfpathlineto{\pgfqpoint{1.006663in}{0.698510in}}%
\pgfpathlineto{\pgfqpoint{1.047872in}{0.753264in}}%
\pgfpathlineto{\pgfqpoint{1.089081in}{0.779787in}}%
\pgfpathlineto{\pgfqpoint{1.130291in}{0.850277in}}%
\pgfpathlineto{\pgfqpoint{1.171500in}{0.845752in}}%
\pgfpathlineto{\pgfqpoint{1.212709in}{0.914392in}}%
\pgfpathlineto{\pgfqpoint{1.253919in}{0.931284in}}%
\pgfpathlineto{\pgfqpoint{1.295128in}{0.975929in}}%
\pgfpathlineto{\pgfqpoint{1.336338in}{0.993774in}}%
\pgfpathlineto{\pgfqpoint{1.377547in}{1.027066in}}%
\pgfpathlineto{\pgfqpoint{1.418756in}{1.045857in}}%
\pgfpathlineto{\pgfqpoint{1.459966in}{1.077790in}}%
\pgfpathlineto{\pgfqpoint{1.501175in}{1.097857in}}%
\pgfpathlineto{\pgfqpoint{1.542385in}{1.119419in}}%
\pgfpathlineto{\pgfqpoint{1.583594in}{1.134885in}}%
\pgfpathlineto{\pgfqpoint{1.624803in}{1.170061in}}%
\pgfpathlineto{\pgfqpoint{1.666013in}{1.180603in}}%
\pgfpathlineto{\pgfqpoint{1.707222in}{1.199617in}}%
\pgfpathlineto{\pgfqpoint{1.748432in}{1.220505in}}%
\pgfpathlineto{\pgfqpoint{1.789641in}{1.244441in}}%
\pgfpathlineto{\pgfqpoint{1.830850in}{1.253393in}}%
\pgfpathlineto{\pgfqpoint{1.872060in}{1.280163in}}%
\pgfpathlineto{\pgfqpoint{1.913269in}{1.283477in}}%
\pgfpathlineto{\pgfqpoint{1.954479in}{1.295166in}}%
\pgfpathlineto{\pgfqpoint{1.995688in}{1.324311in}}%
\pgfpathlineto{\pgfqpoint{2.036897in}{1.326535in}}%
\pgfpathlineto{\pgfqpoint{2.078107in}{1.344502in}}%
\pgfpathlineto{\pgfqpoint{2.119316in}{1.359039in}}%
\pgfpathlineto{\pgfqpoint{2.160525in}{1.368288in}}%
\pgfpathlineto{\pgfqpoint{2.201735in}{1.373954in}}%
\pgfpathlineto{\pgfqpoint{2.242944in}{1.404422in}}%
\pgfpathlineto{\pgfqpoint{2.284154in}{1.448823in}}%
\pgfpathlineto{\pgfqpoint{2.325363in}{1.418002in}}%
\pgfpathlineto{\pgfqpoint{2.366572in}{1.434138in}}%
\pgfpathlineto{\pgfqpoint{2.407782in}{1.450857in}}%
\pgfpathlineto{\pgfqpoint{2.448991in}{1.454263in}}%
\pgfpathlineto{\pgfqpoint{2.490201in}{1.471636in}}%
\pgfpathlineto{\pgfqpoint{2.531410in}{1.494260in}}%
\pgfpathlineto{\pgfqpoint{2.572619in}{1.489291in}}%
\pgfpathlineto{\pgfqpoint{2.613829in}{1.492132in}}%
\pgfpathlineto{\pgfqpoint{2.655038in}{1.517123in}}%
\pgfpathlineto{\pgfqpoint{2.696248in}{1.522021in}}%
\pgfpathlineto{\pgfqpoint{2.737457in}{1.535098in}}%
\pgfpathlineto{\pgfqpoint{2.778666in}{1.546272in}}%
\pgfpathlineto{\pgfqpoint{2.819876in}{1.559641in}}%
\pgfpathlineto{\pgfqpoint{2.861085in}{1.561090in}}%
\pgfpathlineto{\pgfqpoint{2.902295in}{1.568210in}}%
\pgfpathlineto{\pgfqpoint{2.943504in}{1.633164in}}%
\pgfpathlineto{\pgfqpoint{2.984713in}{1.588909in}}%
\pgfpathlineto{\pgfqpoint{3.025923in}{1.585987in}}%
\pgfpathlineto{\pgfqpoint{3.067132in}{1.624278in}}%
\pgfpathlineto{\pgfqpoint{3.108341in}{1.610781in}}%
\pgfpathlineto{\pgfqpoint{3.149551in}{1.648334in}}%
\pgfpathlineto{\pgfqpoint{3.190760in}{1.648587in}}%
\pgfpathlineto{\pgfqpoint{3.231970in}{1.631541in}}%
\pgfpathlineto{\pgfqpoint{3.273179in}{1.656474in}}%
\pgfpathlineto{\pgfqpoint{3.314388in}{1.642007in}}%
\pgfpathlineto{\pgfqpoint{3.355598in}{1.689406in}}%
\pgfpathlineto{\pgfqpoint{3.396807in}{1.674114in}}%
\pgfpathlineto{\pgfqpoint{3.438017in}{1.681064in}}%
\pgfpathlineto{\pgfqpoint{3.479226in}{1.689991in}}%
\pgfpathlineto{\pgfqpoint{3.520435in}{1.694057in}}%
\pgfpathlineto{\pgfqpoint{3.561645in}{1.700680in}}%
\pgfpathlineto{\pgfqpoint{3.602854in}{1.701632in}}%
\pgfpathlineto{\pgfqpoint{3.644064in}{1.719534in}}%
\pgfpathlineto{\pgfqpoint{3.685273in}{1.714033in}}%
\pgfpathlineto{\pgfqpoint{3.726482in}{1.735647in}}%
\pgfpathlineto{\pgfqpoint{3.767692in}{1.741430in}}%
\pgfpathlineto{\pgfqpoint{3.808901in}{1.748563in}}%
\pgfpathlineto{\pgfqpoint{3.891320in}{1.768931in}}%
\pgfpathlineto{\pgfqpoint{3.932529in}{1.756240in}}%
\pgfpathlineto{\pgfqpoint{3.973739in}{1.795330in}}%
\pgfpathlineto{\pgfqpoint{4.056157in}{1.772256in}}%
\pgfpathlineto{\pgfqpoint{4.097367in}{1.838101in}}%
\pgfpathlineto{\pgfqpoint{4.138576in}{1.804150in}}%
\pgfpathlineto{\pgfqpoint{4.179786in}{1.787344in}}%
\pgfpathlineto{\pgfqpoint{4.220995in}{1.805830in}}%
\pgfpathlineto{\pgfqpoint{4.262204in}{1.858702in}}%
\pgfpathlineto{\pgfqpoint{4.303414in}{1.824361in}}%
\pgfpathlineto{\pgfqpoint{4.385833in}{1.842464in}}%
\pgfpathlineto{\pgfqpoint{4.427042in}{1.863307in}}%
\pgfpathlineto{\pgfqpoint{4.468251in}{1.843028in}}%
\pgfpathlineto{\pgfqpoint{4.509461in}{1.864219in}}%
\pgfpathlineto{\pgfqpoint{4.550670in}{1.847957in}}%
\pgfpathlineto{\pgfqpoint{4.591880in}{1.845364in}}%
\pgfpathlineto{\pgfqpoint{4.633089in}{1.880493in}}%
\pgfpathlineto{\pgfqpoint{4.715508in}{1.880113in}}%
\pgfpathlineto{\pgfqpoint{4.756717in}{1.949580in}}%
\pgfpathlineto{\pgfqpoint{4.797926in}{1.909810in}}%
\pgfpathlineto{\pgfqpoint{4.880345in}{1.916369in}}%
\pgfpathlineto{\pgfqpoint{4.962764in}{1.930912in}}%
\pgfpathlineto{\pgfqpoint{5.003973in}{1.893482in}}%
\pgfpathlineto{\pgfqpoint{5.045183in}{1.912352in}}%
\pgfusepath{stroke}%
\end{pgfscope}%
\begin{pgfscope}%
\pgfpathrectangle{\pgfqpoint{0.588387in}{0.521603in}}{\pgfqpoint{4.669024in}{2.220246in}}%
\pgfusepath{clip}%
\pgfsetrectcap%
\pgfsetroundjoin%
\pgfsetlinewidth{1.505625pt}%
\pgfsetstrokecolor{currentstroke2}%
\pgfsetdash{}{0pt}%
\pgfpathmoveto{\pgfqpoint{0.800616in}{0.817980in}}%
\pgfpathlineto{\pgfqpoint{0.841825in}{0.832596in}}%
\pgfpathlineto{\pgfqpoint{0.883034in}{0.833465in}}%
\pgfpathlineto{\pgfqpoint{0.924244in}{0.764780in}}%
\pgfpathlineto{\pgfqpoint{0.965453in}{0.709499in}}%
\pgfpathlineto{\pgfqpoint{1.006663in}{0.696153in}}%
\pgfpathlineto{\pgfqpoint{1.047872in}{0.756937in}}%
\pgfpathlineto{\pgfqpoint{1.089081in}{0.783330in}}%
\pgfpathlineto{\pgfqpoint{1.130291in}{0.838025in}}%
\pgfpathlineto{\pgfqpoint{1.171500in}{0.847600in}}%
\pgfpathlineto{\pgfqpoint{1.212709in}{0.990458in}}%
\pgfpathlineto{\pgfqpoint{1.253919in}{0.988278in}}%
\pgfpathlineto{\pgfqpoint{1.295128in}{1.033675in}}%
\pgfpathlineto{\pgfqpoint{1.336338in}{1.083617in}}%
\pgfpathlineto{\pgfqpoint{1.377547in}{1.193959in}}%
\pgfpathlineto{\pgfqpoint{1.418756in}{1.172929in}}%
\pgfpathlineto{\pgfqpoint{1.459966in}{1.280038in}}%
\pgfpathlineto{\pgfqpoint{1.501175in}{1.384734in}}%
\pgfpathlineto{\pgfqpoint{1.542385in}{1.559256in}}%
\pgfpathlineto{\pgfqpoint{1.583594in}{1.454865in}}%
\pgfpathlineto{\pgfqpoint{1.624803in}{1.525671in}}%
\pgfpathlineto{\pgfqpoint{1.666013in}{1.604955in}}%
\pgfpathlineto{\pgfqpoint{1.707222in}{1.412370in}}%
\pgfpathlineto{\pgfqpoint{1.748432in}{1.684046in}}%
\pgfpathlineto{\pgfqpoint{1.789641in}{1.760042in}}%
\pgfpathlineto{\pgfqpoint{1.830850in}{1.977512in}}%
\pgfpathlineto{\pgfqpoint{1.872060in}{2.211966in}}%
\pgfpathlineto{\pgfqpoint{1.913269in}{2.262261in}}%
\pgfpathlineto{\pgfqpoint{1.954479in}{2.188021in}}%
\pgfpathlineto{\pgfqpoint{1.995688in}{2.081864in}}%
\pgfpathlineto{\pgfqpoint{2.036897in}{2.250369in}}%
\pgfpathlineto{\pgfqpoint{2.078107in}{2.111147in}}%
\pgfpathlineto{\pgfqpoint{2.119316in}{2.261671in}}%
\pgfpathlineto{\pgfqpoint{2.160525in}{2.175812in}}%
\pgfpathlineto{\pgfqpoint{2.201735in}{2.329405in}}%
\pgfpathlineto{\pgfqpoint{2.242944in}{2.309862in}}%
\pgfpathlineto{\pgfqpoint{2.284154in}{2.275212in}}%
\pgfpathlineto{\pgfqpoint{2.325363in}{2.255752in}}%
\pgfpathlineto{\pgfqpoint{2.366572in}{2.058431in}}%
\pgfpathlineto{\pgfqpoint{2.407782in}{2.350516in}}%
\pgfpathlineto{\pgfqpoint{2.448991in}{1.508403in}}%
\pgfpathlineto{\pgfqpoint{2.490201in}{2.415728in}}%
\pgfpathlineto{\pgfqpoint{2.531410in}{2.487203in}}%
\pgfpathlineto{\pgfqpoint{2.572619in}{2.465524in}}%
\pgfpathlineto{\pgfqpoint{2.613829in}{2.286313in}}%
\pgfpathlineto{\pgfqpoint{2.655038in}{2.281864in}}%
\pgfpathlineto{\pgfqpoint{2.696248in}{2.383955in}}%
\pgfpathlineto{\pgfqpoint{2.737457in}{2.416724in}}%
\pgfpathlineto{\pgfqpoint{2.778666in}{1.820710in}}%
\pgfpathlineto{\pgfqpoint{2.819876in}{2.257724in}}%
\pgfpathlineto{\pgfqpoint{2.861085in}{2.052576in}}%
\pgfpathlineto{\pgfqpoint{2.902295in}{2.275404in}}%
\pgfpathlineto{\pgfqpoint{2.943504in}{1.868193in}}%
\pgfpathlineto{\pgfqpoint{2.984713in}{2.365233in}}%
\pgfpathlineto{\pgfqpoint{3.025923in}{1.794147in}}%
\pgfpathlineto{\pgfqpoint{3.067132in}{2.461183in}}%
\pgfpathlineto{\pgfqpoint{3.108341in}{2.415417in}}%
\pgfpathlineto{\pgfqpoint{3.149551in}{2.393677in}}%
\pgfpathlineto{\pgfqpoint{3.190760in}{2.194382in}}%
\pgfpathlineto{\pgfqpoint{3.231970in}{2.383468in}}%
\pgfpathlineto{\pgfqpoint{3.273179in}{2.235714in}}%
\pgfpathlineto{\pgfqpoint{3.314388in}{2.336639in}}%
\pgfpathlineto{\pgfqpoint{3.355598in}{1.728591in}}%
\pgfpathlineto{\pgfqpoint{3.396807in}{2.399121in}}%
\pgfpathlineto{\pgfqpoint{3.438017in}{1.748094in}}%
\pgfpathlineto{\pgfqpoint{3.479226in}{2.259076in}}%
\pgfpathlineto{\pgfqpoint{3.520435in}{2.516753in}}%
\pgfpathlineto{\pgfqpoint{3.561645in}{2.468375in}}%
\pgfpathlineto{\pgfqpoint{3.602854in}{2.591412in}}%
\pgfpathlineto{\pgfqpoint{3.644064in}{2.311099in}}%
\pgfpathlineto{\pgfqpoint{3.685273in}{2.590342in}}%
\pgfpathlineto{\pgfqpoint{3.726482in}{2.466452in}}%
\pgfpathlineto{\pgfqpoint{3.767692in}{2.010927in}}%
\pgfpathlineto{\pgfqpoint{3.808901in}{1.859218in}}%
\pgfpathlineto{\pgfqpoint{3.891320in}{2.358931in}}%
\pgfpathlineto{\pgfqpoint{3.932529in}{2.326771in}}%
\pgfpathlineto{\pgfqpoint{3.973739in}{2.483850in}}%
\pgfpathlineto{\pgfqpoint{4.056157in}{2.475055in}}%
\pgfpathlineto{\pgfqpoint{4.097367in}{1.867235in}}%
\pgfpathlineto{\pgfqpoint{4.138576in}{2.482666in}}%
\pgfpathlineto{\pgfqpoint{4.179786in}{2.538507in}}%
\pgfpathlineto{\pgfqpoint{4.220995in}{2.530909in}}%
\pgfpathlineto{\pgfqpoint{4.303414in}{2.395994in}}%
\pgfpathlineto{\pgfqpoint{4.385833in}{2.532641in}}%
\pgfpathlineto{\pgfqpoint{4.427042in}{2.621419in}}%
\pgfpathlineto{\pgfqpoint{4.468251in}{2.554870in}}%
\pgfpathlineto{\pgfqpoint{4.509461in}{2.521473in}}%
\pgfpathlineto{\pgfqpoint{4.550670in}{2.382215in}}%
\pgfpathlineto{\pgfqpoint{4.591880in}{2.522235in}}%
\pgfpathlineto{\pgfqpoint{4.633089in}{2.373263in}}%
\pgfpathlineto{\pgfqpoint{4.715508in}{2.530945in}}%
\pgfpathlineto{\pgfqpoint{4.756717in}{2.524858in}}%
\pgfpathlineto{\pgfqpoint{4.797926in}{1.953074in}}%
\pgfpathlineto{\pgfqpoint{4.880345in}{2.076881in}}%
\pgfpathlineto{\pgfqpoint{4.962764in}{1.959504in}}%
\pgfusepath{stroke}%
\end{pgfscope}%
\begin{pgfscope}%
\pgfpathrectangle{\pgfqpoint{0.588387in}{0.521603in}}{\pgfqpoint{4.669024in}{2.220246in}}%
\pgfusepath{clip}%
\pgfsetrectcap%
\pgfsetroundjoin%
\pgfsetlinewidth{1.505625pt}%
\pgfsetstrokecolor{currentstroke3}%
\pgfsetdash{}{0pt}%
\pgfpathmoveto{\pgfqpoint{0.800616in}{0.817980in}}%
\pgfpathlineto{\pgfqpoint{0.841825in}{0.837128in}}%
\pgfpathlineto{\pgfqpoint{0.883034in}{0.823791in}}%
\pgfpathlineto{\pgfqpoint{0.924244in}{0.773140in}}%
\pgfpathlineto{\pgfqpoint{0.965453in}{0.782978in}}%
\pgfpathlineto{\pgfqpoint{1.006663in}{0.709815in}}%
\pgfpathlineto{\pgfqpoint{1.047872in}{0.622524in}}%
\pgfpathlineto{\pgfqpoint{1.089081in}{0.651737in}}%
\pgfpathlineto{\pgfqpoint{1.130291in}{0.645153in}}%
\pgfpathlineto{\pgfqpoint{1.171500in}{0.644758in}}%
\pgfpathlineto{\pgfqpoint{1.212709in}{0.666991in}}%
\pgfpathlineto{\pgfqpoint{1.253919in}{0.669092in}}%
\pgfpathlineto{\pgfqpoint{1.295128in}{0.736217in}}%
\pgfpathlineto{\pgfqpoint{1.336338in}{0.706976in}}%
\pgfpathlineto{\pgfqpoint{1.377547in}{0.736848in}}%
\pgfpathlineto{\pgfqpoint{1.418756in}{0.745591in}}%
\pgfpathlineto{\pgfqpoint{1.459966in}{0.774924in}}%
\pgfpathlineto{\pgfqpoint{1.501175in}{0.778251in}}%
\pgfpathlineto{\pgfqpoint{1.542385in}{0.812737in}}%
\pgfpathlineto{\pgfqpoint{1.583594in}{0.799765in}}%
\pgfpathlineto{\pgfqpoint{1.624803in}{0.834872in}}%
\pgfpathlineto{\pgfqpoint{1.666013in}{0.818586in}}%
\pgfpathlineto{\pgfqpoint{1.707222in}{0.852671in}}%
\pgfpathlineto{\pgfqpoint{1.748432in}{0.842786in}}%
\pgfpathlineto{\pgfqpoint{1.789641in}{0.866097in}}%
\pgfpathlineto{\pgfqpoint{1.830850in}{0.866395in}}%
\pgfpathlineto{\pgfqpoint{1.872060in}{0.883653in}}%
\pgfpathlineto{\pgfqpoint{1.913269in}{0.934796in}}%
\pgfpathlineto{\pgfqpoint{1.954479in}{0.905558in}}%
\pgfpathlineto{\pgfqpoint{1.995688in}{0.900534in}}%
\pgfpathlineto{\pgfqpoint{2.036897in}{0.961309in}}%
\pgfpathlineto{\pgfqpoint{2.078107in}{0.927543in}}%
\pgfpathlineto{\pgfqpoint{2.119316in}{0.955144in}}%
\pgfpathlineto{\pgfqpoint{2.160525in}{0.953459in}}%
\pgfpathlineto{\pgfqpoint{2.201735in}{0.992099in}}%
\pgfpathlineto{\pgfqpoint{2.242944in}{0.952866in}}%
\pgfpathlineto{\pgfqpoint{2.284154in}{0.957009in}}%
\pgfpathlineto{\pgfqpoint{2.325363in}{0.966909in}}%
\pgfpathlineto{\pgfqpoint{2.366572in}{0.996710in}}%
\pgfpathlineto{\pgfqpoint{2.407782in}{0.976686in}}%
\pgfpathlineto{\pgfqpoint{2.448991in}{0.984377in}}%
\pgfpathlineto{\pgfqpoint{2.490201in}{0.983525in}}%
\pgfpathlineto{\pgfqpoint{2.531410in}{1.018632in}}%
\pgfpathlineto{\pgfqpoint{2.572619in}{0.993953in}}%
\pgfpathlineto{\pgfqpoint{2.613829in}{1.016357in}}%
\pgfpathlineto{\pgfqpoint{2.655038in}{1.017681in}}%
\pgfpathlineto{\pgfqpoint{2.696248in}{1.259184in}}%
\pgfpathlineto{\pgfqpoint{2.737457in}{1.016166in}}%
\pgfpathlineto{\pgfqpoint{2.778666in}{1.034952in}}%
\pgfpathlineto{\pgfqpoint{2.819876in}{1.027775in}}%
\pgfpathlineto{\pgfqpoint{2.861085in}{1.055868in}}%
\pgfpathlineto{\pgfqpoint{2.902295in}{1.031891in}}%
\pgfpathlineto{\pgfqpoint{2.943504in}{1.066968in}}%
\pgfpathlineto{\pgfqpoint{2.984713in}{1.050000in}}%
\pgfpathlineto{\pgfqpoint{3.025923in}{1.071832in}}%
\pgfpathlineto{\pgfqpoint{3.067132in}{1.056633in}}%
\pgfpathlineto{\pgfqpoint{3.108341in}{1.084895in}}%
\pgfpathlineto{\pgfqpoint{3.149551in}{1.075991in}}%
\pgfpathlineto{\pgfqpoint{3.190760in}{1.162751in}}%
\pgfpathlineto{\pgfqpoint{3.231970in}{1.089022in}}%
\pgfpathlineto{\pgfqpoint{3.273179in}{1.103593in}}%
\pgfpathlineto{\pgfqpoint{3.314388in}{1.120614in}}%
\pgfpathlineto{\pgfqpoint{3.355598in}{1.145070in}}%
\pgfpathlineto{\pgfqpoint{3.396807in}{1.107835in}}%
\pgfpathlineto{\pgfqpoint{3.438017in}{1.116120in}}%
\pgfpathlineto{\pgfqpoint{3.479226in}{1.948258in}}%
\pgfpathlineto{\pgfqpoint{3.520435in}{1.126126in}}%
\pgfpathlineto{\pgfqpoint{3.561645in}{1.104757in}}%
\pgfpathlineto{\pgfqpoint{3.602854in}{1.129379in}}%
\pgfpathlineto{\pgfqpoint{3.644064in}{1.220422in}}%
\pgfpathlineto{\pgfqpoint{3.685273in}{1.234707in}}%
\pgfpathlineto{\pgfqpoint{3.726482in}{1.146373in}}%
\pgfpathlineto{\pgfqpoint{3.767692in}{1.158440in}}%
\pgfpathlineto{\pgfqpoint{3.808901in}{1.116120in}}%
\pgfpathlineto{\pgfqpoint{3.891320in}{1.159206in}}%
\pgfpathlineto{\pgfqpoint{3.932529in}{1.157569in}}%
\pgfpathlineto{\pgfqpoint{3.973739in}{1.171168in}}%
\pgfpathlineto{\pgfqpoint{4.056157in}{1.185818in}}%
\pgfpathlineto{\pgfqpoint{4.097367in}{1.187240in}}%
\pgfpathlineto{\pgfqpoint{4.138576in}{1.173368in}}%
\pgfpathlineto{\pgfqpoint{4.179786in}{1.154058in}}%
\pgfpathlineto{\pgfqpoint{4.220995in}{1.164717in}}%
\pgfpathlineto{\pgfqpoint{4.262204in}{1.202393in}}%
\pgfpathlineto{\pgfqpoint{4.303414in}{1.183312in}}%
\pgfpathlineto{\pgfqpoint{4.385833in}{1.220684in}}%
\pgfpathlineto{\pgfqpoint{4.427042in}{1.202393in}}%
\pgfpathlineto{\pgfqpoint{4.468251in}{1.224583in}}%
\pgfpathlineto{\pgfqpoint{4.509461in}{1.243260in}}%
\pgfpathlineto{\pgfqpoint{4.550670in}{1.230325in}}%
\pgfpathlineto{\pgfqpoint{4.591880in}{1.230325in}}%
\pgfpathlineto{\pgfqpoint{4.633089in}{1.357676in}}%
\pgfpathlineto{\pgfqpoint{4.715508in}{1.194925in}}%
\pgfpathlineto{\pgfqpoint{4.756717in}{1.261551in}}%
\pgfpathlineto{\pgfqpoint{4.797926in}{1.230325in}}%
\pgfpathlineto{\pgfqpoint{4.880345in}{1.216727in}}%
\pgfpathlineto{\pgfqpoint{4.962764in}{1.216727in}}%
\pgfpathlineto{\pgfqpoint{5.003973in}{1.216727in}}%
\pgfpathlineto{\pgfqpoint{5.045183in}{1.255593in}}%
\pgfusepath{stroke}%
\end{pgfscope}%
\begin{pgfscope}%
\pgfpathrectangle{\pgfqpoint{0.588387in}{0.521603in}}{\pgfqpoint{4.669024in}{2.220246in}}%
\pgfusepath{clip}%
\pgfsetrectcap%
\pgfsetroundjoin%
\pgfsetlinewidth{1.505625pt}%
\pgfsetstrokecolor{currentstroke4}%
\pgfsetdash{}{0pt}%
\pgfpathmoveto{\pgfqpoint{0.800616in}{0.824866in}}%
\pgfpathlineto{\pgfqpoint{0.841825in}{0.837128in}}%
\pgfpathlineto{\pgfqpoint{0.883034in}{0.833465in}}%
\pgfpathlineto{\pgfqpoint{0.924244in}{0.749181in}}%
\pgfpathlineto{\pgfqpoint{0.965453in}{0.719581in}}%
\pgfpathlineto{\pgfqpoint{1.006663in}{0.696153in}}%
\pgfpathlineto{\pgfqpoint{1.047872in}{0.745760in}}%
\pgfpathlineto{\pgfqpoint{1.089081in}{0.767624in}}%
\pgfpathlineto{\pgfqpoint{1.130291in}{0.830762in}}%
\pgfpathlineto{\pgfqpoint{1.171500in}{0.832446in}}%
\pgfpathlineto{\pgfqpoint{1.212709in}{0.899429in}}%
\pgfpathlineto{\pgfqpoint{1.253919in}{0.914605in}}%
\pgfpathlineto{\pgfqpoint{1.295128in}{0.962700in}}%
\pgfpathlineto{\pgfqpoint{1.336338in}{0.977561in}}%
\pgfpathlineto{\pgfqpoint{1.377547in}{1.015399in}}%
\pgfpathlineto{\pgfqpoint{1.418756in}{1.034635in}}%
\pgfpathlineto{\pgfqpoint{1.459966in}{1.059499in}}%
\pgfpathlineto{\pgfqpoint{1.501175in}{1.081167in}}%
\pgfpathlineto{\pgfqpoint{1.542385in}{1.116594in}}%
\pgfpathlineto{\pgfqpoint{1.583594in}{1.124752in}}%
\pgfpathlineto{\pgfqpoint{1.624803in}{1.152877in}}%
\pgfpathlineto{\pgfqpoint{1.666013in}{1.165831in}}%
\pgfpathlineto{\pgfqpoint{1.707222in}{1.193976in}}%
\pgfpathlineto{\pgfqpoint{1.748432in}{1.204615in}}%
\pgfpathlineto{\pgfqpoint{1.789641in}{1.227410in}}%
\pgfpathlineto{\pgfqpoint{1.830850in}{1.273335in}}%
\pgfpathlineto{\pgfqpoint{1.872060in}{1.281656in}}%
\pgfpathlineto{\pgfqpoint{1.913269in}{1.283260in}}%
\pgfpathlineto{\pgfqpoint{1.954479in}{1.290805in}}%
\pgfpathlineto{\pgfqpoint{1.995688in}{1.304833in}}%
\pgfpathlineto{\pgfqpoint{2.036897in}{1.324994in}}%
\pgfpathlineto{\pgfqpoint{2.078107in}{1.362915in}}%
\pgfpathlineto{\pgfqpoint{2.119316in}{1.352152in}}%
\pgfpathlineto{\pgfqpoint{2.160525in}{1.355247in}}%
\pgfpathlineto{\pgfqpoint{2.201735in}{1.373587in}}%
\pgfpathlineto{\pgfqpoint{2.242944in}{1.390140in}}%
\pgfpathlineto{\pgfqpoint{2.284154in}{1.397212in}}%
\pgfpathlineto{\pgfqpoint{2.325363in}{1.411230in}}%
\pgfpathlineto{\pgfqpoint{2.366572in}{1.425665in}}%
\pgfpathlineto{\pgfqpoint{2.407782in}{1.436992in}}%
\pgfpathlineto{\pgfqpoint{2.448991in}{1.449034in}}%
\pgfpathlineto{\pgfqpoint{2.490201in}{1.459840in}}%
\pgfpathlineto{\pgfqpoint{2.531410in}{1.479143in}}%
\pgfpathlineto{\pgfqpoint{2.572619in}{1.485065in}}%
\pgfpathlineto{\pgfqpoint{2.613829in}{1.490990in}}%
\pgfpathlineto{\pgfqpoint{2.655038in}{1.515232in}}%
\pgfpathlineto{\pgfqpoint{2.696248in}{1.518020in}}%
\pgfpathlineto{\pgfqpoint{2.737457in}{1.535447in}}%
\pgfpathlineto{\pgfqpoint{2.778666in}{1.582247in}}%
\pgfpathlineto{\pgfqpoint{2.819876in}{1.549790in}}%
\pgfpathlineto{\pgfqpoint{2.861085in}{1.555039in}}%
\pgfpathlineto{\pgfqpoint{2.902295in}{1.559222in}}%
\pgfpathlineto{\pgfqpoint{2.943504in}{1.591813in}}%
\pgfpathlineto{\pgfqpoint{2.984713in}{1.581810in}}%
\pgfpathlineto{\pgfqpoint{3.025923in}{1.580737in}}%
\pgfpathlineto{\pgfqpoint{3.067132in}{1.594944in}}%
\pgfpathlineto{\pgfqpoint{3.108341in}{1.598582in}}%
\pgfpathlineto{\pgfqpoint{3.149551in}{1.617555in}}%
\pgfpathlineto{\pgfqpoint{3.190760in}{1.623866in}}%
\pgfpathlineto{\pgfqpoint{3.231970in}{1.640897in}}%
\pgfpathlineto{\pgfqpoint{3.273179in}{1.634899in}}%
\pgfpathlineto{\pgfqpoint{3.314388in}{1.636495in}}%
\pgfpathlineto{\pgfqpoint{3.355598in}{1.671199in}}%
\pgfpathlineto{\pgfqpoint{3.396807in}{1.656474in}}%
\pgfpathlineto{\pgfqpoint{3.438017in}{1.657004in}}%
\pgfpathlineto{\pgfqpoint{3.479226in}{1.678326in}}%
\pgfpathlineto{\pgfqpoint{3.520435in}{1.688230in}}%
\pgfpathlineto{\pgfqpoint{3.561645in}{1.701088in}}%
\pgfpathlineto{\pgfqpoint{3.602854in}{1.696352in}}%
\pgfpathlineto{\pgfqpoint{3.644064in}{1.696961in}}%
\pgfpathlineto{\pgfqpoint{3.685273in}{1.693287in}}%
\pgfpathlineto{\pgfqpoint{3.726482in}{1.720787in}}%
\pgfpathlineto{\pgfqpoint{3.767692in}{1.722967in}}%
\pgfpathlineto{\pgfqpoint{3.808901in}{1.730272in}}%
\pgfpathlineto{\pgfqpoint{3.891320in}{1.753350in}}%
\pgfpathlineto{\pgfqpoint{3.932529in}{1.748094in}}%
\pgfpathlineto{\pgfqpoint{3.973739in}{1.763020in}}%
\pgfpathlineto{\pgfqpoint{4.056157in}{1.763693in}}%
\pgfpathlineto{\pgfqpoint{4.097367in}{1.809871in}}%
\pgfpathlineto{\pgfqpoint{4.138576in}{1.788369in}}%
\pgfpathlineto{\pgfqpoint{4.179786in}{1.778272in}}%
\pgfpathlineto{\pgfqpoint{4.220995in}{1.797559in}}%
\pgfpathlineto{\pgfqpoint{4.262204in}{1.832690in}}%
\pgfpathlineto{\pgfqpoint{4.303414in}{1.815398in}}%
\pgfpathlineto{\pgfqpoint{4.385833in}{1.832885in}}%
\pgfpathlineto{\pgfqpoint{4.427042in}{1.842087in}}%
\pgfpathlineto{\pgfqpoint{4.468251in}{1.835988in}}%
\pgfpathlineto{\pgfqpoint{4.509461in}{1.850525in}}%
\pgfpathlineto{\pgfqpoint{4.550670in}{1.851799in}}%
\pgfpathlineto{\pgfqpoint{4.591880in}{1.848601in}}%
\pgfpathlineto{\pgfqpoint{4.633089in}{1.861032in}}%
\pgfpathlineto{\pgfqpoint{4.715508in}{1.859321in}}%
\pgfpathlineto{\pgfqpoint{4.756717in}{1.925190in}}%
\pgfpathlineto{\pgfqpoint{4.797926in}{1.886045in}}%
\pgfpathlineto{\pgfqpoint{4.880345in}{1.891665in}}%
\pgfpathlineto{\pgfqpoint{4.962764in}{1.907244in}}%
\pgfpathlineto{\pgfqpoint{5.003973in}{1.875502in}}%
\pgfpathlineto{\pgfqpoint{5.045183in}{1.914870in}}%
\pgfusepath{stroke}%
\end{pgfscope}%
\begin{pgfscope}%
\pgfpathrectangle{\pgfqpoint{0.588387in}{0.521603in}}{\pgfqpoint{4.669024in}{2.220246in}}%
\pgfusepath{clip}%
\pgfsetrectcap%
\pgfsetroundjoin%
\pgfsetlinewidth{1.505625pt}%
\pgfsetstrokecolor{currentstroke5}%
\pgfsetdash{}{0pt}%
\pgfpathmoveto{\pgfqpoint{0.800616in}{0.824866in}}%
\pgfpathlineto{\pgfqpoint{0.841825in}{0.841584in}}%
\pgfpathlineto{\pgfqpoint{0.883034in}{0.836271in}}%
\pgfpathlineto{\pgfqpoint{0.924244in}{0.759181in}}%
\pgfpathlineto{\pgfqpoint{0.965453in}{0.713263in}}%
\pgfpathlineto{\pgfqpoint{1.006663in}{0.696153in}}%
\pgfpathlineto{\pgfqpoint{1.047872in}{0.740637in}}%
\pgfpathlineto{\pgfqpoint{1.089081in}{0.774989in}}%
\pgfpathlineto{\pgfqpoint{1.130291in}{0.831680in}}%
\pgfpathlineto{\pgfqpoint{1.171500in}{0.831470in}}%
\pgfpathlineto{\pgfqpoint{1.212709in}{0.944177in}}%
\pgfpathlineto{\pgfqpoint{1.253919in}{0.941848in}}%
\pgfpathlineto{\pgfqpoint{1.295128in}{0.996245in}}%
\pgfpathlineto{\pgfqpoint{1.336338in}{1.036814in}}%
\pgfpathlineto{\pgfqpoint{1.377547in}{1.094900in}}%
\pgfpathlineto{\pgfqpoint{1.418756in}{1.057302in}}%
\pgfpathlineto{\pgfqpoint{1.459966in}{1.090525in}}%
\pgfpathlineto{\pgfqpoint{1.501175in}{1.110772in}}%
\pgfpathlineto{\pgfqpoint{1.542385in}{1.238521in}}%
\pgfpathlineto{\pgfqpoint{1.583594in}{1.158945in}}%
\pgfpathlineto{\pgfqpoint{1.624803in}{1.190854in}}%
\pgfpathlineto{\pgfqpoint{1.666013in}{1.335632in}}%
\pgfpathlineto{\pgfqpoint{1.707222in}{1.471748in}}%
\pgfpathlineto{\pgfqpoint{1.748432in}{1.372297in}}%
\pgfpathlineto{\pgfqpoint{1.789641in}{1.274486in}}%
\pgfpathlineto{\pgfqpoint{1.830850in}{1.393255in}}%
\pgfpathlineto{\pgfqpoint{1.872060in}{1.956679in}}%
\pgfpathlineto{\pgfqpoint{1.913269in}{1.565234in}}%
\pgfpathlineto{\pgfqpoint{1.954479in}{1.533987in}}%
\pgfpathlineto{\pgfqpoint{1.995688in}{1.560733in}}%
\pgfpathlineto{\pgfqpoint{2.036897in}{2.092953in}}%
\pgfpathlineto{\pgfqpoint{2.078107in}{1.745278in}}%
\pgfpathlineto{\pgfqpoint{2.119316in}{1.910533in}}%
\pgfpathlineto{\pgfqpoint{2.160525in}{1.596184in}}%
\pgfpathlineto{\pgfqpoint{2.201735in}{2.093887in}}%
\pgfpathlineto{\pgfqpoint{2.242944in}{1.798670in}}%
\pgfpathlineto{\pgfqpoint{2.284154in}{1.705634in}}%
\pgfpathlineto{\pgfqpoint{2.325363in}{1.815901in}}%
\pgfpathlineto{\pgfqpoint{2.366572in}{1.580989in}}%
\pgfpathlineto{\pgfqpoint{2.407782in}{1.583054in}}%
\pgfpathlineto{\pgfqpoint{2.448991in}{1.538689in}}%
\pgfpathlineto{\pgfqpoint{2.490201in}{1.814370in}}%
\pgfpathlineto{\pgfqpoint{2.531410in}{1.547849in}}%
\pgfpathlineto{\pgfqpoint{2.572619in}{1.763151in}}%
\pgfpathlineto{\pgfqpoint{2.613829in}{1.536127in}}%
\pgfpathlineto{\pgfqpoint{2.655038in}{2.038233in}}%
\pgfpathlineto{\pgfqpoint{2.696248in}{2.206791in}}%
\pgfpathlineto{\pgfqpoint{2.737457in}{2.004811in}}%
\pgfpathlineto{\pgfqpoint{2.778666in}{1.595946in}}%
\pgfpathlineto{\pgfqpoint{2.819876in}{2.137952in}}%
\pgfpathlineto{\pgfqpoint{2.861085in}{2.089897in}}%
\pgfpathlineto{\pgfqpoint{2.902295in}{1.902601in}}%
\pgfpathlineto{\pgfqpoint{2.943504in}{1.676424in}}%
\pgfpathlineto{\pgfqpoint{2.984713in}{1.706490in}}%
\pgfpathlineto{\pgfqpoint{3.025923in}{2.514805in}}%
\pgfpathlineto{\pgfqpoint{3.067132in}{2.011377in}}%
\pgfpathlineto{\pgfqpoint{3.108341in}{2.166799in}}%
\pgfpathlineto{\pgfqpoint{3.149551in}{2.030058in}}%
\pgfpathlineto{\pgfqpoint{3.190760in}{1.987511in}}%
\pgfpathlineto{\pgfqpoint{3.231970in}{1.830451in}}%
\pgfpathlineto{\pgfqpoint{3.273179in}{2.295015in}}%
\pgfpathlineto{\pgfqpoint{3.314388in}{2.111986in}}%
\pgfpathlineto{\pgfqpoint{3.355598in}{1.777432in}}%
\pgfpathlineto{\pgfqpoint{3.396807in}{2.148049in}}%
\pgfpathlineto{\pgfqpoint{3.438017in}{1.720370in}}%
\pgfpathlineto{\pgfqpoint{3.479226in}{1.745139in}}%
\pgfpathlineto{\pgfqpoint{3.520435in}{1.711886in}}%
\pgfpathlineto{\pgfqpoint{3.561645in}{2.214215in}}%
\pgfpathlineto{\pgfqpoint{3.602854in}{1.826701in}}%
\pgfpathlineto{\pgfqpoint{3.644064in}{2.148935in}}%
\pgfpathlineto{\pgfqpoint{3.685273in}{2.230423in}}%
\pgfpathlineto{\pgfqpoint{3.726482in}{1.994177in}}%
\pgfpathlineto{\pgfqpoint{3.767692in}{1.829251in}}%
\pgfpathlineto{\pgfqpoint{3.808901in}{2.130208in}}%
\pgfpathlineto{\pgfqpoint{3.891320in}{2.106930in}}%
\pgfpathlineto{\pgfqpoint{3.932529in}{1.776589in}}%
\pgfpathlineto{\pgfqpoint{3.973739in}{2.289702in}}%
\pgfpathlineto{\pgfqpoint{4.056157in}{1.975273in}}%
\pgfpathlineto{\pgfqpoint{4.097367in}{1.822947in}}%
\pgfpathlineto{\pgfqpoint{4.138576in}{2.157153in}}%
\pgfpathlineto{\pgfqpoint{4.179786in}{1.979484in}}%
\pgfpathlineto{\pgfqpoint{4.220995in}{1.894899in}}%
\pgfpathlineto{\pgfqpoint{4.262204in}{1.880113in}}%
\pgfpathlineto{\pgfqpoint{4.303414in}{1.842087in}}%
\pgfpathlineto{\pgfqpoint{4.385833in}{1.985530in}}%
\pgfpathlineto{\pgfqpoint{4.427042in}{1.925431in}}%
\pgfpathlineto{\pgfqpoint{4.468251in}{2.244927in}}%
\pgfpathlineto{\pgfqpoint{4.509461in}{2.523802in}}%
\pgfpathlineto{\pgfqpoint{4.550670in}{2.360453in}}%
\pgfpathlineto{\pgfqpoint{4.591880in}{2.086059in}}%
\pgfpathlineto{\pgfqpoint{4.633089in}{2.393479in}}%
\pgfpathlineto{\pgfqpoint{4.715508in}{2.134875in}}%
\pgfpathlineto{\pgfqpoint{4.756717in}{2.538650in}}%
\pgfpathlineto{\pgfqpoint{4.797926in}{2.369536in}}%
\pgfpathlineto{\pgfqpoint{4.880345in}{2.292390in}}%
\pgfpathlineto{\pgfqpoint{4.962764in}{2.256898in}}%
\pgfpathlineto{\pgfqpoint{5.003973in}{2.025848in}}%
\pgfpathlineto{\pgfqpoint{5.045183in}{2.525860in}}%
\pgfusepath{stroke}%
\end{pgfscope}%
\begin{pgfscope}%
\pgfpathrectangle{\pgfqpoint{0.588387in}{0.521603in}}{\pgfqpoint{4.669024in}{2.220246in}}%
\pgfusepath{clip}%
\pgfsetrectcap%
\pgfsetroundjoin%
\pgfsetlinewidth{1.505625pt}%
\pgfsetstrokecolor{currentstroke6}%
\pgfsetdash{}{0pt}%
\pgfpathmoveto{\pgfqpoint{0.800616in}{0.838125in}}%
\pgfpathlineto{\pgfqpoint{0.841825in}{0.850277in}}%
\pgfpathlineto{\pgfqpoint{0.883034in}{0.836271in}}%
\pgfpathlineto{\pgfqpoint{0.924244in}{0.753792in}}%
\pgfpathlineto{\pgfqpoint{0.965453in}{0.719581in}}%
\pgfpathlineto{\pgfqpoint{1.006663in}{0.696153in}}%
\pgfpathlineto{\pgfqpoint{1.047872in}{0.738038in}}%
\pgfpathlineto{\pgfqpoint{1.089081in}{0.774989in}}%
\pgfpathlineto{\pgfqpoint{1.130291in}{0.832596in}}%
\pgfpathlineto{\pgfqpoint{1.171500in}{0.833419in}}%
\pgfpathlineto{\pgfqpoint{1.212709in}{0.899429in}}%
\pgfpathlineto{\pgfqpoint{1.253919in}{0.918592in}}%
\pgfpathlineto{\pgfqpoint{1.295128in}{0.972980in}}%
\pgfpathlineto{\pgfqpoint{1.336338in}{0.984165in}}%
\pgfpathlineto{\pgfqpoint{1.377547in}{1.012675in}}%
\pgfpathlineto{\pgfqpoint{1.418756in}{1.031287in}}%
\pgfpathlineto{\pgfqpoint{1.459966in}{1.062191in}}%
\pgfpathlineto{\pgfqpoint{1.501175in}{1.082061in}}%
\pgfpathlineto{\pgfqpoint{1.542385in}{1.111335in}}%
\pgfpathlineto{\pgfqpoint{1.583594in}{1.122447in}}%
\pgfpathlineto{\pgfqpoint{1.624803in}{1.154646in}}%
\pgfpathlineto{\pgfqpoint{1.666013in}{1.165831in}}%
\pgfpathlineto{\pgfqpoint{1.707222in}{1.192069in}}%
\pgfpathlineto{\pgfqpoint{1.748432in}{1.204394in}}%
\pgfpathlineto{\pgfqpoint{1.789641in}{1.226571in}}%
\pgfpathlineto{\pgfqpoint{1.830850in}{1.237752in}}%
\pgfpathlineto{\pgfqpoint{1.872060in}{1.259400in}}%
\pgfpathlineto{\pgfqpoint{1.913269in}{1.277776in}}%
\pgfpathlineto{\pgfqpoint{1.954479in}{1.285924in}}%
\pgfpathlineto{\pgfqpoint{1.995688in}{1.314656in}}%
\pgfpathlineto{\pgfqpoint{2.036897in}{1.318341in}}%
\pgfpathlineto{\pgfqpoint{2.078107in}{1.342108in}}%
\pgfpathlineto{\pgfqpoint{2.119316in}{1.348287in}}%
\pgfpathlineto{\pgfqpoint{2.160525in}{1.358725in}}%
\pgfpathlineto{\pgfqpoint{2.201735in}{1.369137in}}%
\pgfpathlineto{\pgfqpoint{2.242944in}{1.383462in}}%
\pgfpathlineto{\pgfqpoint{2.284154in}{1.397212in}}%
\pgfpathlineto{\pgfqpoint{2.325363in}{1.407859in}}%
\pgfpathlineto{\pgfqpoint{2.366572in}{1.443199in}}%
\pgfpathlineto{\pgfqpoint{2.407782in}{1.431297in}}%
\pgfpathlineto{\pgfqpoint{2.448991in}{1.444238in}}%
\pgfpathlineto{\pgfqpoint{2.490201in}{1.469393in}}%
\pgfpathlineto{\pgfqpoint{2.531410in}{1.482364in}}%
\pgfpathlineto{\pgfqpoint{2.572619in}{1.486279in}}%
\pgfpathlineto{\pgfqpoint{2.613829in}{1.489460in}}%
\pgfpathlineto{\pgfqpoint{2.655038in}{1.500371in}}%
\pgfpathlineto{\pgfqpoint{2.696248in}{1.515772in}}%
\pgfpathlineto{\pgfqpoint{2.737457in}{1.524266in}}%
\pgfpathlineto{\pgfqpoint{2.778666in}{1.563526in}}%
\pgfpathlineto{\pgfqpoint{2.819876in}{1.547902in}}%
\pgfpathlineto{\pgfqpoint{2.861085in}{1.553645in}}%
\pgfpathlineto{\pgfqpoint{2.902295in}{1.549613in}}%
\pgfpathlineto{\pgfqpoint{2.943504in}{1.576461in}}%
\pgfpathlineto{\pgfqpoint{2.984713in}{1.590987in}}%
\pgfpathlineto{\pgfqpoint{3.025923in}{1.575381in}}%
\pgfpathlineto{\pgfqpoint{3.067132in}{1.588711in}}%
\pgfpathlineto{\pgfqpoint{3.108341in}{1.604393in}}%
\pgfpathlineto{\pgfqpoint{3.149551in}{1.615136in}}%
\pgfpathlineto{\pgfqpoint{3.190760in}{1.615861in}}%
\pgfpathlineto{\pgfqpoint{3.231970in}{1.612437in}}%
\pgfpathlineto{\pgfqpoint{3.273179in}{1.628482in}}%
\pgfpathlineto{\pgfqpoint{3.314388in}{1.629726in}}%
\pgfpathlineto{\pgfqpoint{3.355598in}{1.655675in}}%
\pgfpathlineto{\pgfqpoint{3.396807in}{1.650383in}}%
\pgfpathlineto{\pgfqpoint{3.438017in}{1.678632in}}%
\pgfpathlineto{\pgfqpoint{3.479226in}{1.678020in}}%
\pgfpathlineto{\pgfqpoint{3.520435in}{1.672452in}}%
\pgfpathlineto{\pgfqpoint{3.561645in}{1.697596in}}%
\pgfpathlineto{\pgfqpoint{3.602854in}{1.680660in}}%
\pgfpathlineto{\pgfqpoint{3.644064in}{1.694824in}}%
\pgfpathlineto{\pgfqpoint{3.685273in}{1.690186in}}%
\pgfpathlineto{\pgfqpoint{3.726482in}{1.723001in}}%
\pgfpathlineto{\pgfqpoint{3.767692in}{1.713498in}}%
\pgfpathlineto{\pgfqpoint{3.808901in}{1.724855in}}%
\pgfpathlineto{\pgfqpoint{3.891320in}{1.746763in}}%
\pgfpathlineto{\pgfqpoint{3.932529in}{1.741237in}}%
\pgfpathlineto{\pgfqpoint{3.973739in}{1.752599in}}%
\pgfpathlineto{\pgfqpoint{4.056157in}{1.764658in}}%
\pgfpathlineto{\pgfqpoint{4.097367in}{1.789765in}}%
\pgfpathlineto{\pgfqpoint{4.138576in}{1.779166in}}%
\pgfpathlineto{\pgfqpoint{4.179786in}{1.766259in}}%
\pgfpathlineto{\pgfqpoint{4.220995in}{1.789518in}}%
\pgfpathlineto{\pgfqpoint{4.262204in}{1.821526in}}%
\pgfpathlineto{\pgfqpoint{4.303414in}{1.808751in}}%
\pgfpathlineto{\pgfqpoint{4.385833in}{1.823554in}}%
\pgfpathlineto{\pgfqpoint{4.427042in}{1.826818in}}%
\pgfpathlineto{\pgfqpoint{4.468251in}{1.833664in}}%
\pgfpathlineto{\pgfqpoint{4.509461in}{1.878393in}}%
\pgfpathlineto{\pgfqpoint{4.550670in}{1.839103in}}%
\pgfpathlineto{\pgfqpoint{4.591880in}{1.836758in}}%
\pgfpathlineto{\pgfqpoint{4.633089in}{1.854120in}}%
\pgfpathlineto{\pgfqpoint{4.715508in}{1.845581in}}%
\pgfpathlineto{\pgfqpoint{4.756717in}{1.890752in}}%
\pgfpathlineto{\pgfqpoint{4.797926in}{1.869920in}}%
\pgfpathlineto{\pgfqpoint{4.880345in}{1.898505in}}%
\pgfpathlineto{\pgfqpoint{4.962764in}{1.904131in}}%
\pgfpathlineto{\pgfqpoint{5.003973in}{1.887993in}}%
\pgfpathlineto{\pgfqpoint{5.045183in}{2.026177in}}%
\pgfusepath{stroke}%
\end{pgfscope}%
\begin{pgfscope}%
\pgfpathrectangle{\pgfqpoint{0.588387in}{0.521603in}}{\pgfqpoint{4.669024in}{2.220246in}}%
\pgfusepath{clip}%
\pgfsetrectcap%
\pgfsetroundjoin%
\pgfsetlinewidth{1.505625pt}%
\pgfsetstrokecolor{currentstroke7}%
\pgfsetdash{}{0pt}%
\pgfpathmoveto{\pgfqpoint{0.800616in}{0.824866in}}%
\pgfpathlineto{\pgfqpoint{0.841825in}{0.837128in}}%
\pgfpathlineto{\pgfqpoint{0.883034in}{0.843565in}}%
\pgfpathlineto{\pgfqpoint{0.924244in}{0.761477in}}%
\pgfpathlineto{\pgfqpoint{0.965453in}{0.711547in}}%
\pgfpathlineto{\pgfqpoint{1.006663in}{0.696153in}}%
\pgfpathlineto{\pgfqpoint{1.047872in}{0.740637in}}%
\pgfpathlineto{\pgfqpoint{1.089081in}{0.773775in}}%
\pgfpathlineto{\pgfqpoint{1.130291in}{0.831680in}}%
\pgfpathlineto{\pgfqpoint{1.171500in}{0.834389in}}%
\pgfpathlineto{\pgfqpoint{1.212709in}{0.949186in}}%
\pgfpathlineto{\pgfqpoint{1.253919in}{0.950507in}}%
\pgfpathlineto{\pgfqpoint{1.295128in}{1.021611in}}%
\pgfpathlineto{\pgfqpoint{1.336338in}{1.051670in}}%
\pgfpathlineto{\pgfqpoint{1.377547in}{1.143570in}}%
\pgfpathlineto{\pgfqpoint{1.418756in}{1.083156in}}%
\pgfpathlineto{\pgfqpoint{1.459966in}{1.116120in}}%
\pgfpathlineto{\pgfqpoint{1.501175in}{1.138844in}}%
\pgfpathlineto{\pgfqpoint{1.542385in}{1.308053in}}%
\pgfpathlineto{\pgfqpoint{1.583594in}{1.209990in}}%
\pgfpathlineto{\pgfqpoint{1.624803in}{1.167833in}}%
\pgfpathlineto{\pgfqpoint{1.666013in}{1.293903in}}%
\pgfpathlineto{\pgfqpoint{1.707222in}{1.537001in}}%
\pgfpathlineto{\pgfqpoint{1.748432in}{1.365027in}}%
\pgfpathlineto{\pgfqpoint{1.789641in}{1.454894in}}%
\pgfpathlineto{\pgfqpoint{1.830850in}{1.601586in}}%
\pgfpathlineto{\pgfqpoint{1.872060in}{1.560546in}}%
\pgfpathlineto{\pgfqpoint{1.913269in}{1.464073in}}%
\pgfpathlineto{\pgfqpoint{1.954479in}{2.109553in}}%
\pgfpathlineto{\pgfqpoint{1.995688in}{1.816105in}}%
\pgfpathlineto{\pgfqpoint{2.036897in}{2.078192in}}%
\pgfpathlineto{\pgfqpoint{2.078107in}{1.790175in}}%
\pgfpathlineto{\pgfqpoint{2.119316in}{1.721931in}}%
\pgfpathlineto{\pgfqpoint{2.160525in}{1.476386in}}%
\pgfpathlineto{\pgfqpoint{2.201735in}{2.144913in}}%
\pgfpathlineto{\pgfqpoint{2.242944in}{1.837003in}}%
\pgfpathlineto{\pgfqpoint{2.284154in}{1.929145in}}%
\pgfpathlineto{\pgfqpoint{2.325363in}{1.993401in}}%
\pgfpathlineto{\pgfqpoint{2.366572in}{2.199086in}}%
\pgfpathlineto{\pgfqpoint{2.407782in}{1.885711in}}%
\pgfpathlineto{\pgfqpoint{2.448991in}{1.877817in}}%
\pgfpathlineto{\pgfqpoint{2.490201in}{1.954293in}}%
\pgfpathlineto{\pgfqpoint{2.531410in}{2.224163in}}%
\pgfpathlineto{\pgfqpoint{2.572619in}{1.932605in}}%
\pgfpathlineto{\pgfqpoint{2.613829in}{1.561278in}}%
\pgfpathlineto{\pgfqpoint{2.655038in}{1.949793in}}%
\pgfpathlineto{\pgfqpoint{2.696248in}{2.234606in}}%
\pgfpathlineto{\pgfqpoint{2.737457in}{2.033895in}}%
\pgfpathlineto{\pgfqpoint{2.778666in}{1.621820in}}%
\pgfpathlineto{\pgfqpoint{2.819876in}{2.137223in}}%
\pgfpathlineto{\pgfqpoint{2.861085in}{2.115313in}}%
\pgfpathlineto{\pgfqpoint{2.902295in}{2.249430in}}%
\pgfpathlineto{\pgfqpoint{2.943504in}{2.331935in}}%
\pgfpathlineto{\pgfqpoint{2.984713in}{2.012228in}}%
\pgfpathlineto{\pgfqpoint{3.025923in}{1.742872in}}%
\pgfpathlineto{\pgfqpoint{3.067132in}{2.200850in}}%
\pgfpathlineto{\pgfqpoint{3.108341in}{1.998631in}}%
\pgfpathlineto{\pgfqpoint{3.149551in}{2.147648in}}%
\pgfpathlineto{\pgfqpoint{3.190760in}{2.201659in}}%
\pgfpathlineto{\pgfqpoint{3.231970in}{2.156863in}}%
\pgfpathlineto{\pgfqpoint{3.273179in}{2.326814in}}%
\pgfpathlineto{\pgfqpoint{3.314388in}{2.041704in}}%
\pgfpathlineto{\pgfqpoint{3.355598in}{2.516132in}}%
\pgfpathlineto{\pgfqpoint{3.396807in}{2.308190in}}%
\pgfpathlineto{\pgfqpoint{3.438017in}{1.847957in}}%
\pgfpathlineto{\pgfqpoint{3.479226in}{2.197744in}}%
\pgfpathlineto{\pgfqpoint{3.520435in}{1.792959in}}%
\pgfpathlineto{\pgfqpoint{3.561645in}{2.330050in}}%
\pgfpathlineto{\pgfqpoint{3.602854in}{1.923902in}}%
\pgfpathlineto{\pgfqpoint{3.644064in}{2.111318in}}%
\pgfpathlineto{\pgfqpoint{3.685273in}{1.931227in}}%
\pgfpathlineto{\pgfqpoint{3.726482in}{2.306517in}}%
\pgfpathlineto{\pgfqpoint{3.767692in}{2.060712in}}%
\pgfpathlineto{\pgfqpoint{3.808901in}{2.268216in}}%
\pgfpathlineto{\pgfqpoint{3.891320in}{2.131186in}}%
\pgfpathlineto{\pgfqpoint{3.932529in}{2.308023in}}%
\pgfpathlineto{\pgfqpoint{3.973739in}{1.810490in}}%
\pgfpathlineto{\pgfqpoint{4.056157in}{2.373649in}}%
\pgfpathlineto{\pgfqpoint{4.097367in}{1.887993in}}%
\pgfpathlineto{\pgfqpoint{4.138576in}{2.241039in}}%
\pgfpathlineto{\pgfqpoint{4.179786in}{2.362823in}}%
\pgfpathlineto{\pgfqpoint{4.220995in}{2.177134in}}%
\pgfpathlineto{\pgfqpoint{4.262204in}{1.867235in}}%
\pgfpathlineto{\pgfqpoint{4.303414in}{1.855900in}}%
\pgfpathlineto{\pgfqpoint{4.385833in}{1.891300in}}%
\pgfpathlineto{\pgfqpoint{4.427042in}{2.111947in}}%
\pgfpathlineto{\pgfqpoint{4.468251in}{2.070642in}}%
\pgfpathlineto{\pgfqpoint{4.509461in}{1.988746in}}%
\pgfpathlineto{\pgfqpoint{4.550670in}{2.115380in}}%
\pgfpathlineto{\pgfqpoint{4.591880in}{2.521676in}}%
\pgfpathlineto{\pgfqpoint{4.633089in}{2.272260in}}%
\pgfpathlineto{\pgfqpoint{4.715508in}{2.099381in}}%
\pgfpathlineto{\pgfqpoint{4.756717in}{1.982219in}}%
\pgfpathlineto{\pgfqpoint{4.797926in}{1.979091in}}%
\pgfpathlineto{\pgfqpoint{4.880345in}{2.456779in}}%
\pgfpathlineto{\pgfqpoint{4.962764in}{2.007416in}}%
\pgfpathlineto{\pgfqpoint{5.003973in}{1.995116in}}%
\pgfpathlineto{\pgfqpoint{5.045183in}{1.946929in}}%
\pgfusepath{stroke}%
\end{pgfscope}%
\begin{pgfscope}%
\pgfpathrectangle{\pgfqpoint{0.588387in}{0.521603in}}{\pgfqpoint{4.669024in}{2.220246in}}%
\pgfusepath{clip}%
\pgfsetrectcap%
\pgfsetroundjoin%
\pgfsetlinewidth{1.505625pt}%
\definecolor{currentstroke}{rgb}{0.498039,0.498039,0.498039}%
\pgfsetstrokecolor{currentstroke}%
\pgfsetdash{}{0pt}%
\pgfpathmoveto{\pgfqpoint{0.800616in}{0.824866in}}%
\pgfpathlineto{\pgfqpoint{0.841825in}{0.850277in}}%
\pgfpathlineto{\pgfqpoint{0.883034in}{0.851982in}}%
\pgfpathlineto{\pgfqpoint{0.924244in}{0.778379in}}%
\pgfpathlineto{\pgfqpoint{0.965453in}{0.720309in}}%
\pgfpathlineto{\pgfqpoint{1.006663in}{0.699447in}}%
\pgfpathlineto{\pgfqpoint{1.047872in}{0.745760in}}%
\pgfpathlineto{\pgfqpoint{1.089081in}{0.765123in}}%
\pgfpathlineto{\pgfqpoint{1.130291in}{0.820437in}}%
\pgfpathlineto{\pgfqpoint{1.171500in}{0.826533in}}%
\pgfpathlineto{\pgfqpoint{1.212709in}{0.876206in}}%
\pgfpathlineto{\pgfqpoint{1.253919in}{0.893098in}}%
\pgfpathlineto{\pgfqpoint{1.295128in}{0.944850in}}%
\pgfpathlineto{\pgfqpoint{1.336338in}{0.953056in}}%
\pgfpathlineto{\pgfqpoint{1.377547in}{0.980076in}}%
\pgfpathlineto{\pgfqpoint{1.418756in}{0.993953in}}%
\pgfpathlineto{\pgfqpoint{1.459966in}{1.031968in}}%
\pgfpathlineto{\pgfqpoint{1.501175in}{1.051114in}}%
\pgfpathlineto{\pgfqpoint{1.542385in}{1.072392in}}%
\pgfpathlineto{\pgfqpoint{1.583594in}{1.078474in}}%
\pgfpathlineto{\pgfqpoint{1.624803in}{1.111319in}}%
\pgfpathlineto{\pgfqpoint{1.666013in}{1.133285in}}%
\pgfpathlineto{\pgfqpoint{1.707222in}{1.143925in}}%
\pgfpathlineto{\pgfqpoint{1.748432in}{1.154860in}}%
\pgfpathlineto{\pgfqpoint{1.789641in}{1.188698in}}%
\pgfpathlineto{\pgfqpoint{1.830850in}{1.205720in}}%
\pgfpathlineto{\pgfqpoint{1.872060in}{1.219252in}}%
\pgfpathlineto{\pgfqpoint{1.913269in}{1.223341in}}%
\pgfpathlineto{\pgfqpoint{1.954479in}{1.242206in}}%
\pgfpathlineto{\pgfqpoint{1.995688in}{1.263966in}}%
\pgfpathlineto{\pgfqpoint{2.036897in}{1.273549in}}%
\pgfpathlineto{\pgfqpoint{2.078107in}{1.281466in}}%
\pgfpathlineto{\pgfqpoint{2.119316in}{1.301152in}}%
\pgfpathlineto{\pgfqpoint{2.160525in}{1.312517in}}%
\pgfpathlineto{\pgfqpoint{2.201735in}{1.313601in}}%
\pgfpathlineto{\pgfqpoint{2.242944in}{1.327163in}}%
\pgfpathlineto{\pgfqpoint{2.284154in}{1.343068in}}%
\pgfpathlineto{\pgfqpoint{2.325363in}{1.360008in}}%
\pgfpathlineto{\pgfqpoint{2.366572in}{1.372666in}}%
\pgfpathlineto{\pgfqpoint{2.407782in}{1.375339in}}%
\pgfpathlineto{\pgfqpoint{2.448991in}{1.387489in}}%
\pgfpathlineto{\pgfqpoint{2.490201in}{1.398206in}}%
\pgfpathlineto{\pgfqpoint{2.531410in}{1.411589in}}%
\pgfpathlineto{\pgfqpoint{2.572619in}{1.412757in}}%
\pgfpathlineto{\pgfqpoint{2.613829in}{1.466218in}}%
\pgfpathlineto{\pgfqpoint{2.655038in}{1.447771in}}%
\pgfpathlineto{\pgfqpoint{2.696248in}{1.458527in}}%
\pgfpathlineto{\pgfqpoint{2.737457in}{1.474566in}}%
\pgfpathlineto{\pgfqpoint{2.778666in}{1.469429in}}%
\pgfpathlineto{\pgfqpoint{2.819876in}{1.482566in}}%
\pgfpathlineto{\pgfqpoint{2.861085in}{1.491181in}}%
\pgfpathlineto{\pgfqpoint{2.902295in}{1.489460in}}%
\pgfpathlineto{\pgfqpoint{2.943504in}{1.509835in}}%
\pgfpathlineto{\pgfqpoint{2.984713in}{1.510196in}}%
\pgfpathlineto{\pgfqpoint{3.025923in}{1.510296in}}%
\pgfpathlineto{\pgfqpoint{3.067132in}{1.530823in}}%
\pgfpathlineto{\pgfqpoint{3.108341in}{1.542799in}}%
\pgfpathlineto{\pgfqpoint{3.149551in}{1.555607in}}%
\pgfpathlineto{\pgfqpoint{3.190760in}{1.551328in}}%
\pgfpathlineto{\pgfqpoint{3.231970in}{1.551250in}}%
\pgfpathlineto{\pgfqpoint{3.273179in}{1.566207in}}%
\pgfpathlineto{\pgfqpoint{3.314388in}{1.591512in}}%
\pgfpathlineto{\pgfqpoint{3.355598in}{1.627595in}}%
\pgfpathlineto{\pgfqpoint{3.396807in}{1.604608in}}%
\pgfpathlineto{\pgfqpoint{3.438017in}{1.599498in}}%
\pgfpathlineto{\pgfqpoint{3.479226in}{1.612942in}}%
\pgfpathlineto{\pgfqpoint{3.520435in}{1.604393in}}%
\pgfpathlineto{\pgfqpoint{3.561645in}{1.617020in}}%
\pgfpathlineto{\pgfqpoint{3.602854in}{1.623114in}}%
\pgfpathlineto{\pgfqpoint{3.644064in}{1.628679in}}%
\pgfpathlineto{\pgfqpoint{3.685273in}{1.614956in}}%
\pgfpathlineto{\pgfqpoint{3.726482in}{1.653087in}}%
\pgfpathlineto{\pgfqpoint{3.767692in}{1.642707in}}%
\pgfpathlineto{\pgfqpoint{3.808901in}{1.657225in}}%
\pgfpathlineto{\pgfqpoint{3.891320in}{1.686631in}}%
\pgfpathlineto{\pgfqpoint{3.932529in}{1.678388in}}%
\pgfpathlineto{\pgfqpoint{3.973739in}{1.686061in}}%
\pgfpathlineto{\pgfqpoint{4.056157in}{1.690470in}}%
\pgfpathlineto{\pgfqpoint{4.097367in}{1.724513in}}%
\pgfpathlineto{\pgfqpoint{4.138576in}{1.731676in}}%
\pgfpathlineto{\pgfqpoint{4.179786in}{1.703121in}}%
\pgfpathlineto{\pgfqpoint{4.220995in}{1.744746in}}%
\pgfpathlineto{\pgfqpoint{4.262204in}{1.736565in}}%
\pgfpathlineto{\pgfqpoint{4.303414in}{1.787749in}}%
\pgfpathlineto{\pgfqpoint{4.385833in}{1.780896in}}%
\pgfpathlineto{\pgfqpoint{4.427042in}{1.767572in}}%
\pgfpathlineto{\pgfqpoint{4.468251in}{1.792959in}}%
\pgfpathlineto{\pgfqpoint{4.509461in}{1.774895in}}%
\pgfpathlineto{\pgfqpoint{4.550670in}{1.750434in}}%
\pgfpathlineto{\pgfqpoint{4.591880in}{1.789765in}}%
\pgfpathlineto{\pgfqpoint{4.633089in}{1.783259in}}%
\pgfpathlineto{\pgfqpoint{4.715508in}{1.780777in}}%
\pgfpathlineto{\pgfqpoint{4.756717in}{1.820097in}}%
\pgfpathlineto{\pgfqpoint{4.797926in}{1.804985in}}%
\pgfpathlineto{\pgfqpoint{4.880345in}{1.801806in}}%
\pgfpathlineto{\pgfqpoint{4.962764in}{1.835409in}}%
\pgfpathlineto{\pgfqpoint{5.003973in}{1.809871in}}%
\pgfpathlineto{\pgfqpoint{5.045183in}{1.866033in}}%
\pgfusepath{stroke}%
\end{pgfscope}%
\begin{pgfscope}%
\pgfpathrectangle{\pgfqpoint{0.588387in}{0.521603in}}{\pgfqpoint{4.669024in}{2.220246in}}%
\pgfusepath{clip}%
\pgfsetrectcap%
\pgfsetroundjoin%
\pgfsetlinewidth{1.505625pt}%
\definecolor{currentstroke}{rgb}{0.737255,0.741176,0.133333}%
\pgfsetstrokecolor{currentstroke}%
\pgfsetdash{}{0pt}%
\pgfpathmoveto{\pgfqpoint{0.800616in}{0.824866in}}%
\pgfpathlineto{\pgfqpoint{0.841825in}{0.850277in}}%
\pgfpathlineto{\pgfqpoint{0.883034in}{0.854297in}}%
\pgfpathlineto{\pgfqpoint{0.924244in}{0.775714in}}%
\pgfpathlineto{\pgfqpoint{0.965453in}{0.723033in}}%
\pgfpathlineto{\pgfqpoint{1.006663in}{0.703040in}}%
\pgfpathlineto{\pgfqpoint{1.047872in}{0.731426in}}%
\pgfpathlineto{\pgfqpoint{1.089081in}{0.762598in}}%
\pgfpathlineto{\pgfqpoint{1.130291in}{0.814632in}}%
\pgfpathlineto{\pgfqpoint{1.171500in}{0.818437in}}%
\pgfpathlineto{\pgfqpoint{1.212709in}{0.907016in}}%
\pgfpathlineto{\pgfqpoint{1.253919in}{0.892475in}}%
\pgfpathlineto{\pgfqpoint{1.295128in}{1.003743in}}%
\pgfpathlineto{\pgfqpoint{1.336338in}{1.036814in}}%
\pgfpathlineto{\pgfqpoint{1.377547in}{1.188089in}}%
\pgfpathlineto{\pgfqpoint{1.418756in}{1.097829in}}%
\pgfpathlineto{\pgfqpoint{1.459966in}{1.150563in}}%
\pgfpathlineto{\pgfqpoint{1.501175in}{1.263695in}}%
\pgfpathlineto{\pgfqpoint{1.542385in}{1.493533in}}%
\pgfpathlineto{\pgfqpoint{1.583594in}{1.247916in}}%
\pgfpathlineto{\pgfqpoint{1.624803in}{1.580501in}}%
\pgfpathlineto{\pgfqpoint{1.666013in}{1.568470in}}%
\pgfpathlineto{\pgfqpoint{1.707222in}{1.694417in}}%
\pgfpathlineto{\pgfqpoint{1.748432in}{1.744046in}}%
\pgfpathlineto{\pgfqpoint{1.789641in}{1.758464in}}%
\pgfpathlineto{\pgfqpoint{1.830850in}{1.813301in}}%
\pgfpathlineto{\pgfqpoint{1.872060in}{2.240194in}}%
\pgfpathlineto{\pgfqpoint{1.913269in}{2.026375in}}%
\pgfpathlineto{\pgfqpoint{1.954479in}{2.065700in}}%
\pgfpathlineto{\pgfqpoint{1.995688in}{2.089900in}}%
\pgfpathlineto{\pgfqpoint{2.036897in}{2.251979in}}%
\pgfpathlineto{\pgfqpoint{2.078107in}{2.143899in}}%
\pgfpathlineto{\pgfqpoint{2.119316in}{2.368937in}}%
\pgfpathlineto{\pgfqpoint{2.160525in}{2.240175in}}%
\pgfpathlineto{\pgfqpoint{2.201735in}{2.433407in}}%
\pgfpathlineto{\pgfqpoint{2.242944in}{2.189916in}}%
\pgfpathlineto{\pgfqpoint{2.284154in}{2.369437in}}%
\pgfpathlineto{\pgfqpoint{2.325363in}{2.321730in}}%
\pgfpathlineto{\pgfqpoint{2.366572in}{2.325006in}}%
\pgfpathlineto{\pgfqpoint{2.407782in}{2.387531in}}%
\pgfpathlineto{\pgfqpoint{2.448991in}{2.487543in}}%
\pgfpathlineto{\pgfqpoint{2.490201in}{2.321869in}}%
\pgfpathlineto{\pgfqpoint{2.531410in}{2.432890in}}%
\pgfpathlineto{\pgfqpoint{2.572619in}{2.486193in}}%
\pgfpathlineto{\pgfqpoint{2.613829in}{2.414449in}}%
\pgfpathlineto{\pgfqpoint{2.655038in}{2.450961in}}%
\pgfpathlineto{\pgfqpoint{2.696248in}{2.475652in}}%
\pgfpathlineto{\pgfqpoint{2.737457in}{2.412906in}}%
\pgfpathlineto{\pgfqpoint{2.778666in}{2.455487in}}%
\pgfpathlineto{\pgfqpoint{2.819876in}{2.435830in}}%
\pgfpathlineto{\pgfqpoint{2.861085in}{2.592830in}}%
\pgfpathlineto{\pgfqpoint{2.902295in}{2.524962in}}%
\pgfpathlineto{\pgfqpoint{2.943504in}{2.560898in}}%
\pgfpathlineto{\pgfqpoint{2.984713in}{2.392472in}}%
\pgfpathlineto{\pgfqpoint{3.025923in}{2.516650in}}%
\pgfpathlineto{\pgfqpoint{3.067132in}{2.400795in}}%
\pgfpathlineto{\pgfqpoint{3.108341in}{2.212741in}}%
\pgfpathlineto{\pgfqpoint{3.149551in}{2.465453in}}%
\pgfpathlineto{\pgfqpoint{3.190760in}{2.573926in}}%
\pgfpathlineto{\pgfqpoint{3.231970in}{2.446226in}}%
\pgfpathlineto{\pgfqpoint{3.273179in}{2.565657in}}%
\pgfpathlineto{\pgfqpoint{3.314388in}{2.514907in}}%
\pgfpathlineto{\pgfqpoint{3.355598in}{2.640929in}}%
\pgfpathlineto{\pgfqpoint{3.396807in}{2.601472in}}%
\pgfpathlineto{\pgfqpoint{3.438017in}{2.515248in}}%
\pgfpathlineto{\pgfqpoint{3.479226in}{2.547484in}}%
\pgfpathlineto{\pgfqpoint{3.561645in}{2.581176in}}%
\pgfpathlineto{\pgfqpoint{3.644064in}{2.566820in}}%
\pgfpathlineto{\pgfqpoint{3.685273in}{2.589189in}}%
\pgfpathlineto{\pgfqpoint{3.726482in}{2.592290in}}%
\pgfpathlineto{\pgfqpoint{3.767692in}{2.435962in}}%
\pgfpathlineto{\pgfqpoint{3.808901in}{2.474487in}}%
\pgfpathlineto{\pgfqpoint{3.891320in}{2.583185in}}%
\pgfpathlineto{\pgfqpoint{3.932529in}{2.471166in}}%
\pgfpathlineto{\pgfqpoint{3.973739in}{2.542764in}}%
\pgfpathlineto{\pgfqpoint{4.056157in}{2.498019in}}%
\pgfpathlineto{\pgfqpoint{4.097367in}{2.611777in}}%
\pgfpathlineto{\pgfqpoint{4.138576in}{2.572663in}}%
\pgfpathlineto{\pgfqpoint{4.179786in}{2.517425in}}%
\pgfpathlineto{\pgfqpoint{4.220995in}{2.540858in}}%
\pgfpathlineto{\pgfqpoint{4.303414in}{2.621018in}}%
\pgfpathlineto{\pgfqpoint{4.385833in}{2.616438in}}%
\pgfpathlineto{\pgfqpoint{4.427042in}{2.591412in}}%
\pgfpathlineto{\pgfqpoint{4.468251in}{2.555906in}}%
\pgfpathlineto{\pgfqpoint{4.550670in}{2.371736in}}%
\pgfpathlineto{\pgfqpoint{4.591880in}{2.520403in}}%
\pgfpathlineto{\pgfqpoint{4.633089in}{2.434503in}}%
\pgfpathlineto{\pgfqpoint{4.715508in}{2.591477in}}%
\pgfpathlineto{\pgfqpoint{4.797926in}{2.362267in}}%
\pgfpathlineto{\pgfqpoint{4.880345in}{2.430082in}}%
\pgfpathlineto{\pgfqpoint{4.962764in}{1.868432in}}%
\pgfpathlineto{\pgfqpoint{5.003973in}{1.894567in}}%
\pgfusepath{stroke}%
\end{pgfscope}%
\begin{pgfscope}%
\pgfsetrectcap%
\pgfsetmiterjoin%
\pgfsetlinewidth{0.803000pt}%
\definecolor{currentstroke}{rgb}{0.000000,0.000000,0.000000}%
\pgfsetstrokecolor{currentstroke}%
\pgfsetdash{}{0pt}%
\pgfpathmoveto{\pgfqpoint{0.588387in}{0.521603in}}%
\pgfpathlineto{\pgfqpoint{0.588387in}{2.741849in}}%
\pgfusepath{stroke}%
\end{pgfscope}%
\begin{pgfscope}%
\pgfsetrectcap%
\pgfsetmiterjoin%
\pgfsetlinewidth{0.803000pt}%
\definecolor{currentstroke}{rgb}{0.000000,0.000000,0.000000}%
\pgfsetstrokecolor{currentstroke}%
\pgfsetdash{}{0pt}%
\pgfpathmoveto{\pgfqpoint{5.257411in}{0.521603in}}%
\pgfpathlineto{\pgfqpoint{5.257411in}{2.741849in}}%
\pgfusepath{stroke}%
\end{pgfscope}%
\begin{pgfscope}%
\pgfsetrectcap%
\pgfsetmiterjoin%
\pgfsetlinewidth{0.803000pt}%
\definecolor{currentstroke}{rgb}{0.000000,0.000000,0.000000}%
\pgfsetstrokecolor{currentstroke}%
\pgfsetdash{}{0pt}%
\pgfpathmoveto{\pgfqpoint{0.588387in}{0.521603in}}%
\pgfpathlineto{\pgfqpoint{5.257411in}{0.521603in}}%
\pgfusepath{stroke}%
\end{pgfscope}%
\begin{pgfscope}%
\pgfsetrectcap%
\pgfsetmiterjoin%
\pgfsetlinewidth{0.803000pt}%
\definecolor{currentstroke}{rgb}{0.000000,0.000000,0.000000}%
\pgfsetstrokecolor{currentstroke}%
\pgfsetdash{}{0pt}%
\pgfpathmoveto{\pgfqpoint{0.588387in}{2.741849in}}%
\pgfpathlineto{\pgfqpoint{5.257411in}{2.741849in}}%
\pgfusepath{stroke}%
\end{pgfscope}%
\begin{pgfscope}%
\pgfsetbuttcap%
\pgfsetmiterjoin%
\definecolor{currentfill}{rgb}{1.000000,1.000000,1.000000}%
\pgfsetfillcolor{currentfill}%
\pgfsetfillopacity{0.800000}%
\pgfsetlinewidth{1.003750pt}%
\definecolor{currentstroke}{rgb}{0.800000,0.800000,0.800000}%
\pgfsetstrokecolor{currentstroke}%
\pgfsetstrokeopacity{0.800000}%
\pgfsetdash{}{0pt}%
\pgfpathmoveto{\pgfqpoint{5.344911in}{0.969732in}}%
\pgfpathlineto{\pgfqpoint{8.259376in}{0.969732in}}%
\pgfpathquadraticcurveto{\pgfqpoint{8.284376in}{0.969732in}}{\pgfqpoint{8.284376in}{0.994732in}}%
\pgfpathlineto{\pgfqpoint{8.284376in}{2.654349in}}%
\pgfpathquadraticcurveto{\pgfqpoint{8.284376in}{2.679349in}}{\pgfqpoint{8.259376in}{2.679349in}}%
\pgfpathlineto{\pgfqpoint{5.344911in}{2.679349in}}%
\pgfpathquadraticcurveto{\pgfqpoint{5.319911in}{2.679349in}}{\pgfqpoint{5.319911in}{2.654349in}}%
\pgfpathlineto{\pgfqpoint{5.319911in}{0.994732in}}%
\pgfpathquadraticcurveto{\pgfqpoint{5.319911in}{0.969732in}}{\pgfqpoint{5.344911in}{0.969732in}}%
\pgfpathlineto{\pgfqpoint{5.344911in}{0.969732in}}%
\pgfpathclose%
\pgfusepath{stroke,fill}%
\end{pgfscope}%
\begin{pgfscope}%
\pgfsetrectcap%
\pgfsetroundjoin%
\pgfsetlinewidth{1.505625pt}%
\pgfsetstrokecolor{currentstroke3}%
\pgfsetdash{}{0pt}%
\pgfpathmoveto{\pgfqpoint{5.369911in}{2.578129in}}%
\pgfpathlineto{\pgfqpoint{5.494911in}{2.578129in}}%
\pgfpathlineto{\pgfqpoint{5.619911in}{2.578129in}}%
\pgfusepath{stroke}%
\end{pgfscope}%
\begin{pgfscope}%
\definecolor{textcolor}{rgb}{0.000000,0.000000,0.000000}%
\pgfsetstrokecolor{textcolor}%
\pgfsetfillcolor{textcolor}%
\pgftext[x=5.719911in,y=2.534379in,left,base]{\color{textcolor}{\rmfamily\fontsize{9.000000}{10.800000}\selectfont\catcode`\^=\active\def^{\ifmmode\sp\else\^{}\fi}\catcode`\%=\active\def%{\%}\NaiveCycles{}}}%
\end{pgfscope}%
\begin{pgfscope}%
\pgfsetrectcap%
\pgfsetroundjoin%
\pgfsetlinewidth{1.505625pt}%
\pgfsetstrokecolor{currentstroke1}%
\pgfsetdash{}{0pt}%
\pgfpathmoveto{\pgfqpoint{5.369911in}{2.394657in}}%
\pgfpathlineto{\pgfqpoint{5.494911in}{2.394657in}}%
\pgfpathlineto{\pgfqpoint{5.619911in}{2.394657in}}%
\pgfusepath{stroke}%
\end{pgfscope}%
\begin{pgfscope}%
\definecolor{textcolor}{rgb}{0.000000,0.000000,0.000000}%
\pgfsetstrokecolor{textcolor}%
\pgfsetfillcolor{textcolor}%
\pgftext[x=5.719911in,y=2.350907in,left,base]{\color{textcolor}{\rmfamily\fontsize{9.000000}{10.800000}\selectfont\catcode`\^=\active\def^{\ifmmode\sp\else\^{}\fi}\catcode`\%=\active\def%{\%}\CyclesMatchChunks{} \& \MergeLinear{}}}%
\end{pgfscope}%
\begin{pgfscope}%
\pgfsetrectcap%
\pgfsetroundjoin%
\pgfsetlinewidth{1.505625pt}%
\pgfsetstrokecolor{currentstroke2}%
\pgfsetdash{}{0pt}%
\pgfpathmoveto{\pgfqpoint{5.369911in}{2.207707in}}%
\pgfpathlineto{\pgfqpoint{5.494911in}{2.207707in}}%
\pgfpathlineto{\pgfqpoint{5.619911in}{2.207707in}}%
\pgfusepath{stroke}%
\end{pgfscope}%
\begin{pgfscope}%
\definecolor{textcolor}{rgb}{0.000000,0.000000,0.000000}%
\pgfsetstrokecolor{textcolor}%
\pgfsetfillcolor{textcolor}%
\pgftext[x=5.719911in,y=2.163957in,left,base]{\color{textcolor}{\rmfamily\fontsize{9.000000}{10.800000}\selectfont\catcode`\^=\active\def^{\ifmmode\sp\else\^{}\fi}\catcode`\%=\active\def%{\%}\CyclesMatchChunks{} \& \SharedVertices{}}}%
\end{pgfscope}%
\begin{pgfscope}%
\pgfsetrectcap%
\pgfsetroundjoin%
\pgfsetlinewidth{1.505625pt}%
\pgfsetstrokecolor{currentstroke4}%
\pgfsetdash{}{0pt}%
\pgfpathmoveto{\pgfqpoint{5.369911in}{2.020756in}}%
\pgfpathlineto{\pgfqpoint{5.494911in}{2.020756in}}%
\pgfpathlineto{\pgfqpoint{5.619911in}{2.020756in}}%
\pgfusepath{stroke}%
\end{pgfscope}%
\begin{pgfscope}%
\definecolor{textcolor}{rgb}{0.000000,0.000000,0.000000}%
\pgfsetstrokecolor{textcolor}%
\pgfsetfillcolor{textcolor}%
\pgftext[x=5.719911in,y=1.977006in,left,base]{\color{textcolor}{\rmfamily\fontsize{9.000000}{10.800000}\selectfont\catcode`\^=\active\def^{\ifmmode\sp\else\^{}\fi}\catcode`\%=\active\def%{\%}\Neighbors{} \& \MergeLinear{}}}%
\end{pgfscope}%
\begin{pgfscope}%
\pgfsetrectcap%
\pgfsetroundjoin%
\pgfsetlinewidth{1.505625pt}%
\pgfsetstrokecolor{currentstroke5}%
\pgfsetdash{}{0pt}%
\pgfpathmoveto{\pgfqpoint{5.369911in}{1.837285in}}%
\pgfpathlineto{\pgfqpoint{5.494911in}{1.837285in}}%
\pgfpathlineto{\pgfqpoint{5.619911in}{1.837285in}}%
\pgfusepath{stroke}%
\end{pgfscope}%
\begin{pgfscope}%
\definecolor{textcolor}{rgb}{0.000000,0.000000,0.000000}%
\pgfsetstrokecolor{textcolor}%
\pgfsetfillcolor{textcolor}%
\pgftext[x=5.719911in,y=1.793535in,left,base]{\color{textcolor}{\rmfamily\fontsize{9.000000}{10.800000}\selectfont\catcode`\^=\active\def^{\ifmmode\sp\else\^{}\fi}\catcode`\%=\active\def%{\%}\Neighbors{} \& \SharedVertices{}}}%
\end{pgfscope}%
\begin{pgfscope}%
\pgfsetrectcap%
\pgfsetroundjoin%
\pgfsetlinewidth{1.505625pt}%
\pgfsetstrokecolor{currentstroke6}%
\pgfsetdash{}{0pt}%
\pgfpathmoveto{\pgfqpoint{5.369911in}{1.650334in}}%
\pgfpathlineto{\pgfqpoint{5.494911in}{1.650334in}}%
\pgfpathlineto{\pgfqpoint{5.619911in}{1.650334in}}%
\pgfusepath{stroke}%
\end{pgfscope}%
\begin{pgfscope}%
\definecolor{textcolor}{rgb}{0.000000,0.000000,0.000000}%
\pgfsetstrokecolor{textcolor}%
\pgfsetfillcolor{textcolor}%
\pgftext[x=5.719911in,y=1.606584in,left,base]{\color{textcolor}{\rmfamily\fontsize{9.000000}{10.800000}\selectfont\catcode`\^=\active\def^{\ifmmode\sp\else\^{}\fi}\catcode`\%=\active\def%{\%}\NeighborsDegree{} \& \MergeLinear{}}}%
\end{pgfscope}%
\begin{pgfscope}%
\pgfsetrectcap%
\pgfsetroundjoin%
\pgfsetlinewidth{1.505625pt}%
\pgfsetstrokecolor{currentstroke7}%
\pgfsetdash{}{0pt}%
\pgfpathmoveto{\pgfqpoint{5.369911in}{1.463384in}}%
\pgfpathlineto{\pgfqpoint{5.494911in}{1.463384in}}%
\pgfpathlineto{\pgfqpoint{5.619911in}{1.463384in}}%
\pgfusepath{stroke}%
\end{pgfscope}%
\begin{pgfscope}%
\definecolor{textcolor}{rgb}{0.000000,0.000000,0.000000}%
\pgfsetstrokecolor{textcolor}%
\pgfsetfillcolor{textcolor}%
\pgftext[x=5.719911in,y=1.419634in,left,base]{\color{textcolor}{\rmfamily\fontsize{9.000000}{10.800000}\selectfont\catcode`\^=\active\def^{\ifmmode\sp\else\^{}\fi}\catcode`\%=\active\def%{\%}\NeighborsDegree{} \& \SharedVertices{}}}%
\end{pgfscope}%
\begin{pgfscope}%
\pgfsetrectcap%
\pgfsetroundjoin%
\pgfsetlinewidth{1.505625pt}%
\definecolor{currentstroke}{rgb}{0.498039,0.498039,0.498039}%
\pgfsetstrokecolor{currentstroke}%
\pgfsetdash{}{0pt}%
\pgfpathmoveto{\pgfqpoint{5.369911in}{1.276433in}}%
\pgfpathlineto{\pgfqpoint{5.494911in}{1.276433in}}%
\pgfpathlineto{\pgfqpoint{5.619911in}{1.276433in}}%
\pgfusepath{stroke}%
\end{pgfscope}%
\begin{pgfscope}%
\definecolor{textcolor}{rgb}{0.000000,0.000000,0.000000}%
\pgfsetstrokecolor{textcolor}%
\pgfsetfillcolor{textcolor}%
\pgftext[x=5.719911in,y=1.232683in,left,base]{\color{textcolor}{\rmfamily\fontsize{9.000000}{10.800000}\selectfont\catcode`\^=\active\def^{\ifmmode\sp\else\^{}\fi}\catcode`\%=\active\def%{\%}\None{} \& \MergeLinear{}}}%
\end{pgfscope}%
\begin{pgfscope}%
\pgfsetrectcap%
\pgfsetroundjoin%
\pgfsetlinewidth{1.505625pt}%
\definecolor{currentstroke}{rgb}{0.737255,0.741176,0.133333}%
\pgfsetstrokecolor{currentstroke}%
\pgfsetdash{}{0pt}%
\pgfpathmoveto{\pgfqpoint{5.369911in}{1.092962in}}%
\pgfpathlineto{\pgfqpoint{5.494911in}{1.092962in}}%
\pgfpathlineto{\pgfqpoint{5.619911in}{1.092962in}}%
\pgfusepath{stroke}%
\end{pgfscope}%
\begin{pgfscope}%
\definecolor{textcolor}{rgb}{0.000000,0.000000,0.000000}%
\pgfsetstrokecolor{textcolor}%
\pgfsetfillcolor{textcolor}%
\pgftext[x=5.719911in,y=1.049212in,left,base]{\color{textcolor}{\rmfamily\fontsize{9.000000}{10.800000}\selectfont\catcode`\^=\active\def^{\ifmmode\sp\else\^{}\fi}\catcode`\%=\active\def%{\%}\None{} \& \SharedVertices{}}}%
\end{pgfscope}%
\end{pgfpicture}%
\makeatother%
\endgroup%
}
	\caption[Mean runtime for minimally rigid graphs (some)]{
		Mean running time to find some NAC-coloring for minimally rigid graphs.}%
	\label{fig:graph_minimally_rigid_first_runtime}
\end{figure}%
\begin{figure}[thbp]
	\centering
	\scalebox{\BenchFigureScale}{%% Creator: Matplotlib, PGF backend
%%
%% To include the figure in your LaTeX document, write
%%   \input{<filename>.pgf}
%%
%% Make sure the required packages are loaded in your preamble
%%   \usepackage{pgf}
%%
%% Also ensure that all the required font packages are loaded; for instance,
%% the lmodern package is sometimes necessary when using math font.
%%   \usepackage{lmodern}
%%
%% Figures using additional raster images can only be included by \input if
%% they are in the same directory as the main LaTeX file. For loading figures
%% from other directories you can use the `import` package
%%   \usepackage{import}
%%
%% and then include the figures with
%%   \import{<path to file>}{<filename>.pgf}
%%
%% Matplotlib used the following preamble
%%   \def\mathdefault#1{#1}
%%   \everymath=\expandafter{\the\everymath\displaystyle}
%%   \IfFileExists{scrextend.sty}{
%%     \usepackage[fontsize=10.000000pt]{scrextend}
%%   }{
%%     \renewcommand{\normalsize}{\fontsize{10.000000}{12.000000}\selectfont}
%%     \normalsize
%%   }
%%   
%%   \ifdefined\pdftexversion\else  % non-pdftex case.
%%     \usepackage{fontspec}
%%     \setmainfont{DejaVuSans.ttf}[Path=\detokenize{/home/petr/Projects/PyRigi/.venv/lib/python3.12/site-packages/matplotlib/mpl-data/fonts/ttf/}]
%%     \setsansfont{DejaVuSans.ttf}[Path=\detokenize{/home/petr/Projects/PyRigi/.venv/lib/python3.12/site-packages/matplotlib/mpl-data/fonts/ttf/}]
%%     \setmonofont{DejaVuSansMono.ttf}[Path=\detokenize{/home/petr/Projects/PyRigi/.venv/lib/python3.12/site-packages/matplotlib/mpl-data/fonts/ttf/}]
%%   \fi
%%   \makeatletter\@ifpackageloaded{under\Score{}}{}{\usepackage[strings]{under\Score{}}}\makeatother
%%
\begingroup%
\makeatletter%
\begin{pgfpicture}%
\pgfpathrectangle{\pgfpointorigin}{\pgfqpoint{8.384376in}{2.841849in}}%
\pgfusepath{use as bounding box, clip}%
\begin{pgfscope}%
\pgfsetbuttcap%
\pgfsetmiterjoin%
\definecolor{currentfill}{rgb}{1.000000,1.000000,1.000000}%
\pgfsetfillcolor{currentfill}%
\pgfsetlinewidth{0.000000pt}%
\definecolor{currentstroke}{rgb}{1.000000,1.000000,1.000000}%
\pgfsetstrokecolor{currentstroke}%
\pgfsetdash{}{0pt}%
\pgfpathmoveto{\pgfqpoint{0.000000in}{0.000000in}}%
\pgfpathlineto{\pgfqpoint{8.384376in}{0.000000in}}%
\pgfpathlineto{\pgfqpoint{8.384376in}{2.841849in}}%
\pgfpathlineto{\pgfqpoint{0.000000in}{2.841849in}}%
\pgfpathlineto{\pgfqpoint{0.000000in}{0.000000in}}%
\pgfpathclose%
\pgfusepath{fill}%
\end{pgfscope}%
\begin{pgfscope}%
\pgfsetbuttcap%
\pgfsetmiterjoin%
\definecolor{currentfill}{rgb}{1.000000,1.000000,1.000000}%
\pgfsetfillcolor{currentfill}%
\pgfsetlinewidth{0.000000pt}%
\definecolor{currentstroke}{rgb}{0.000000,0.000000,0.000000}%
\pgfsetstrokecolor{currentstroke}%
\pgfsetstrokeopacity{0.000000}%
\pgfsetdash{}{0pt}%
\pgfpathmoveto{\pgfqpoint{0.588387in}{0.521603in}}%
\pgfpathlineto{\pgfqpoint{4.248423in}{0.521603in}}%
\pgfpathlineto{\pgfqpoint{4.248423in}{2.741849in}}%
\pgfpathlineto{\pgfqpoint{0.588387in}{2.741849in}}%
\pgfpathlineto{\pgfqpoint{0.588387in}{0.521603in}}%
\pgfpathclose%
\pgfusepath{fill}%
\end{pgfscope}%
\begin{pgfscope}%
\pgfsetbuttcap%
\pgfsetroundjoin%
\definecolor{currentfill}{rgb}{0.000000,0.000000,0.000000}%
\pgfsetfillcolor{currentfill}%
\pgfsetlinewidth{0.803000pt}%
\definecolor{currentstroke}{rgb}{0.000000,0.000000,0.000000}%
\pgfsetstrokecolor{currentstroke}%
\pgfsetdash{}{0pt}%
\pgfsys@defobject{currentmarker}{\pgfqpoint{0.000000in}{-0.048611in}}{\pgfqpoint{0.000000in}{0.000000in}}{%
\pgfpathmoveto{\pgfqpoint{0.000000in}{0.000000in}}%
\pgfpathlineto{\pgfqpoint{0.000000in}{-0.048611in}}%
\pgfusepath{stroke,fill}%
}%
\begin{pgfscope}%
\pgfsys@transformshift{0.690145in}{0.521603in}%
\pgfsys@useobject{currentmarker}{}%
\end{pgfscope}%
\end{pgfscope}%
\begin{pgfscope}%
\definecolor{textcolor}{rgb}{0.000000,0.000000,0.000000}%
\pgfsetstrokecolor{textcolor}%
\pgfsetfillcolor{textcolor}%
\pgftext[x=0.690145in,y=0.424381in,,top]{\color{textcolor}{\rmfamily\fontsize{10.000000}{12.000000}\selectfont\catcode`\^=\active\def^{\ifmmode\sp\else\^{}\fi}\catcode`\%=\active\def%{\%}$\mathdefault{0}$}}%
\end{pgfscope}%
\begin{pgfscope}%
\pgfsetbuttcap%
\pgfsetroundjoin%
\definecolor{currentfill}{rgb}{0.000000,0.000000,0.000000}%
\pgfsetfillcolor{currentfill}%
\pgfsetlinewidth{0.803000pt}%
\definecolor{currentstroke}{rgb}{0.000000,0.000000,0.000000}%
\pgfsetstrokecolor{currentstroke}%
\pgfsetdash{}{0pt}%
\pgfsys@defobject{currentmarker}{\pgfqpoint{0.000000in}{-0.048611in}}{\pgfqpoint{0.000000in}{0.000000in}}{%
\pgfpathmoveto{\pgfqpoint{0.000000in}{0.000000in}}%
\pgfpathlineto{\pgfqpoint{0.000000in}{-0.048611in}}%
\pgfusepath{stroke,fill}%
}%
\begin{pgfscope}%
\pgfsys@transformshift{1.174704in}{0.521603in}%
\pgfsys@useobject{currentmarker}{}%
\end{pgfscope}%
\end{pgfscope}%
\begin{pgfscope}%
\definecolor{textcolor}{rgb}{0.000000,0.000000,0.000000}%
\pgfsetstrokecolor{textcolor}%
\pgfsetfillcolor{textcolor}%
\pgftext[x=1.174704in,y=0.424381in,,top]{\color{textcolor}{\rmfamily\fontsize{10.000000}{12.000000}\selectfont\catcode`\^=\active\def^{\ifmmode\sp\else\^{}\fi}\catcode`\%=\active\def%{\%}$\mathdefault{15}$}}%
\end{pgfscope}%
\begin{pgfscope}%
\pgfsetbuttcap%
\pgfsetroundjoin%
\definecolor{currentfill}{rgb}{0.000000,0.000000,0.000000}%
\pgfsetfillcolor{currentfill}%
\pgfsetlinewidth{0.803000pt}%
\definecolor{currentstroke}{rgb}{0.000000,0.000000,0.000000}%
\pgfsetstrokecolor{currentstroke}%
\pgfsetdash{}{0pt}%
\pgfsys@defobject{currentmarker}{\pgfqpoint{0.000000in}{-0.048611in}}{\pgfqpoint{0.000000in}{0.000000in}}{%
\pgfpathmoveto{\pgfqpoint{0.000000in}{0.000000in}}%
\pgfpathlineto{\pgfqpoint{0.000000in}{-0.048611in}}%
\pgfusepath{stroke,fill}%
}%
\begin{pgfscope}%
\pgfsys@transformshift{1.659263in}{0.521603in}%
\pgfsys@useobject{currentmarker}{}%
\end{pgfscope}%
\end{pgfscope}%
\begin{pgfscope}%
\definecolor{textcolor}{rgb}{0.000000,0.000000,0.000000}%
\pgfsetstrokecolor{textcolor}%
\pgfsetfillcolor{textcolor}%
\pgftext[x=1.659263in,y=0.424381in,,top]{\color{textcolor}{\rmfamily\fontsize{10.000000}{12.000000}\selectfont\catcode`\^=\active\def^{\ifmmode\sp\else\^{}\fi}\catcode`\%=\active\def%{\%}$\mathdefault{30}$}}%
\end{pgfscope}%
\begin{pgfscope}%
\pgfsetbuttcap%
\pgfsetroundjoin%
\definecolor{currentfill}{rgb}{0.000000,0.000000,0.000000}%
\pgfsetfillcolor{currentfill}%
\pgfsetlinewidth{0.803000pt}%
\definecolor{currentstroke}{rgb}{0.000000,0.000000,0.000000}%
\pgfsetstrokecolor{currentstroke}%
\pgfsetdash{}{0pt}%
\pgfsys@defobject{currentmarker}{\pgfqpoint{0.000000in}{-0.048611in}}{\pgfqpoint{0.000000in}{0.000000in}}{%
\pgfpathmoveto{\pgfqpoint{0.000000in}{0.000000in}}%
\pgfpathlineto{\pgfqpoint{0.000000in}{-0.048611in}}%
\pgfusepath{stroke,fill}%
}%
\begin{pgfscope}%
\pgfsys@transformshift{2.143822in}{0.521603in}%
\pgfsys@useobject{currentmarker}{}%
\end{pgfscope}%
\end{pgfscope}%
\begin{pgfscope}%
\definecolor{textcolor}{rgb}{0.000000,0.000000,0.000000}%
\pgfsetstrokecolor{textcolor}%
\pgfsetfillcolor{textcolor}%
\pgftext[x=2.143822in,y=0.424381in,,top]{\color{textcolor}{\rmfamily\fontsize{10.000000}{12.000000}\selectfont\catcode`\^=\active\def^{\ifmmode\sp\else\^{}\fi}\catcode`\%=\active\def%{\%}$\mathdefault{45}$}}%
\end{pgfscope}%
\begin{pgfscope}%
\pgfsetbuttcap%
\pgfsetroundjoin%
\definecolor{currentfill}{rgb}{0.000000,0.000000,0.000000}%
\pgfsetfillcolor{currentfill}%
\pgfsetlinewidth{0.803000pt}%
\definecolor{currentstroke}{rgb}{0.000000,0.000000,0.000000}%
\pgfsetstrokecolor{currentstroke}%
\pgfsetdash{}{0pt}%
\pgfsys@defobject{currentmarker}{\pgfqpoint{0.000000in}{-0.048611in}}{\pgfqpoint{0.000000in}{0.000000in}}{%
\pgfpathmoveto{\pgfqpoint{0.000000in}{0.000000in}}%
\pgfpathlineto{\pgfqpoint{0.000000in}{-0.048611in}}%
\pgfusepath{stroke,fill}%
}%
\begin{pgfscope}%
\pgfsys@transformshift{2.628381in}{0.521603in}%
\pgfsys@useobject{currentmarker}{}%
\end{pgfscope}%
\end{pgfscope}%
\begin{pgfscope}%
\definecolor{textcolor}{rgb}{0.000000,0.000000,0.000000}%
\pgfsetstrokecolor{textcolor}%
\pgfsetfillcolor{textcolor}%
\pgftext[x=2.628381in,y=0.424381in,,top]{\color{textcolor}{\rmfamily\fontsize{10.000000}{12.000000}\selectfont\catcode`\^=\active\def^{\ifmmode\sp\else\^{}\fi}\catcode`\%=\active\def%{\%}$\mathdefault{60}$}}%
\end{pgfscope}%
\begin{pgfscope}%
\pgfsetbuttcap%
\pgfsetroundjoin%
\definecolor{currentfill}{rgb}{0.000000,0.000000,0.000000}%
\pgfsetfillcolor{currentfill}%
\pgfsetlinewidth{0.803000pt}%
\definecolor{currentstroke}{rgb}{0.000000,0.000000,0.000000}%
\pgfsetstrokecolor{currentstroke}%
\pgfsetdash{}{0pt}%
\pgfsys@defobject{currentmarker}{\pgfqpoint{0.000000in}{-0.048611in}}{\pgfqpoint{0.000000in}{0.000000in}}{%
\pgfpathmoveto{\pgfqpoint{0.000000in}{0.000000in}}%
\pgfpathlineto{\pgfqpoint{0.000000in}{-0.048611in}}%
\pgfusepath{stroke,fill}%
}%
\begin{pgfscope}%
\pgfsys@transformshift{3.112940in}{0.521603in}%
\pgfsys@useobject{currentmarker}{}%
\end{pgfscope}%
\end{pgfscope}%
\begin{pgfscope}%
\definecolor{textcolor}{rgb}{0.000000,0.000000,0.000000}%
\pgfsetstrokecolor{textcolor}%
\pgfsetfillcolor{textcolor}%
\pgftext[x=3.112940in,y=0.424381in,,top]{\color{textcolor}{\rmfamily\fontsize{10.000000}{12.000000}\selectfont\catcode`\^=\active\def^{\ifmmode\sp\else\^{}\fi}\catcode`\%=\active\def%{\%}$\mathdefault{75}$}}%
\end{pgfscope}%
\begin{pgfscope}%
\pgfsetbuttcap%
\pgfsetroundjoin%
\definecolor{currentfill}{rgb}{0.000000,0.000000,0.000000}%
\pgfsetfillcolor{currentfill}%
\pgfsetlinewidth{0.803000pt}%
\definecolor{currentstroke}{rgb}{0.000000,0.000000,0.000000}%
\pgfsetstrokecolor{currentstroke}%
\pgfsetdash{}{0pt}%
\pgfsys@defobject{currentmarker}{\pgfqpoint{0.000000in}{-0.048611in}}{\pgfqpoint{0.000000in}{0.000000in}}{%
\pgfpathmoveto{\pgfqpoint{0.000000in}{0.000000in}}%
\pgfpathlineto{\pgfqpoint{0.000000in}{-0.048611in}}%
\pgfusepath{stroke,fill}%
}%
\begin{pgfscope}%
\pgfsys@transformshift{3.597498in}{0.521603in}%
\pgfsys@useobject{currentmarker}{}%
\end{pgfscope}%
\end{pgfscope}%
\begin{pgfscope}%
\definecolor{textcolor}{rgb}{0.000000,0.000000,0.000000}%
\pgfsetstrokecolor{textcolor}%
\pgfsetfillcolor{textcolor}%
\pgftext[x=3.597498in,y=0.424381in,,top]{\color{textcolor}{\rmfamily\fontsize{10.000000}{12.000000}\selectfont\catcode`\^=\active\def^{\ifmmode\sp\else\^{}\fi}\catcode`\%=\active\def%{\%}$\mathdefault{90}$}}%
\end{pgfscope}%
\begin{pgfscope}%
\pgfsetbuttcap%
\pgfsetroundjoin%
\definecolor{currentfill}{rgb}{0.000000,0.000000,0.000000}%
\pgfsetfillcolor{currentfill}%
\pgfsetlinewidth{0.803000pt}%
\definecolor{currentstroke}{rgb}{0.000000,0.000000,0.000000}%
\pgfsetstrokecolor{currentstroke}%
\pgfsetdash{}{0pt}%
\pgfsys@defobject{currentmarker}{\pgfqpoint{0.000000in}{-0.048611in}}{\pgfqpoint{0.000000in}{0.000000in}}{%
\pgfpathmoveto{\pgfqpoint{0.000000in}{0.000000in}}%
\pgfpathlineto{\pgfqpoint{0.000000in}{-0.048611in}}%
\pgfusepath{stroke,fill}%
}%
\begin{pgfscope}%
\pgfsys@transformshift{4.082057in}{0.521603in}%
\pgfsys@useobject{currentmarker}{}%
\end{pgfscope}%
\end{pgfscope}%
\begin{pgfscope}%
\definecolor{textcolor}{rgb}{0.000000,0.000000,0.000000}%
\pgfsetstrokecolor{textcolor}%
\pgfsetfillcolor{textcolor}%
\pgftext[x=4.082057in,y=0.424381in,,top]{\color{textcolor}{\rmfamily\fontsize{10.000000}{12.000000}\selectfont\catcode`\^=\active\def^{\ifmmode\sp\else\^{}\fi}\catcode`\%=\active\def%{\%}$\mathdefault{105}$}}%
\end{pgfscope}%
\begin{pgfscope}%
\definecolor{textcolor}{rgb}{0.000000,0.000000,0.000000}%
\pgfsetstrokecolor{textcolor}%
\pgfsetfillcolor{textcolor}%
\pgftext[x=2.418405in,y=0.234413in,,top]{\color{textcolor}{\rmfamily\fontsize{10.000000}{12.000000}\selectfont\catcode`\^=\active\def^{\ifmmode\sp\else\^{}\fi}\catcode`\%=\active\def%{\%}Monochromatic classes}}%
\end{pgfscope}%
\begin{pgfscope}%
\pgfsetbuttcap%
\pgfsetroundjoin%
\definecolor{currentfill}{rgb}{0.000000,0.000000,0.000000}%
\pgfsetfillcolor{currentfill}%
\pgfsetlinewidth{0.803000pt}%
\definecolor{currentstroke}{rgb}{0.000000,0.000000,0.000000}%
\pgfsetstrokecolor{currentstroke}%
\pgfsetdash{}{0pt}%
\pgfsys@defobject{currentmarker}{\pgfqpoint{-0.048611in}{0.000000in}}{\pgfqpoint{-0.000000in}{0.000000in}}{%
\pgfpathmoveto{\pgfqpoint{-0.000000in}{0.000000in}}%
\pgfpathlineto{\pgfqpoint{-0.048611in}{0.000000in}}%
\pgfusepath{stroke,fill}%
}%
\begin{pgfscope}%
\pgfsys@transformshift{0.588387in}{0.622524in}%
\pgfsys@useobject{currentmarker}{}%
\end{pgfscope}%
\end{pgfscope}%
\begin{pgfscope}%
\definecolor{textcolor}{rgb}{0.000000,0.000000,0.000000}%
\pgfsetstrokecolor{textcolor}%
\pgfsetfillcolor{textcolor}%
\pgftext[x=0.289968in, y=0.569762in, left, base]{\color{textcolor}{\rmfamily\fontsize{10.000000}{12.000000}\selectfont\catcode`\^=\active\def^{\ifmmode\sp\else\^{}\fi}\catcode`\%=\active\def%{\%}$\mathdefault{10^{0}}$}}%
\end{pgfscope}%
\begin{pgfscope}%
\pgfsetbuttcap%
\pgfsetroundjoin%
\definecolor{currentfill}{rgb}{0.000000,0.000000,0.000000}%
\pgfsetfillcolor{currentfill}%
\pgfsetlinewidth{0.803000pt}%
\definecolor{currentstroke}{rgb}{0.000000,0.000000,0.000000}%
\pgfsetstrokecolor{currentstroke}%
\pgfsetdash{}{0pt}%
\pgfsys@defobject{currentmarker}{\pgfqpoint{-0.048611in}{0.000000in}}{\pgfqpoint{-0.000000in}{0.000000in}}{%
\pgfpathmoveto{\pgfqpoint{-0.000000in}{0.000000in}}%
\pgfpathlineto{\pgfqpoint{-0.048611in}{0.000000in}}%
\pgfusepath{stroke,fill}%
}%
\begin{pgfscope}%
\pgfsys@transformshift{0.588387in}{0.995960in}%
\pgfsys@useobject{currentmarker}{}%
\end{pgfscope}%
\end{pgfscope}%
\begin{pgfscope}%
\definecolor{textcolor}{rgb}{0.000000,0.000000,0.000000}%
\pgfsetstrokecolor{textcolor}%
\pgfsetfillcolor{textcolor}%
\pgftext[x=0.289968in, y=0.943198in, left, base]{\color{textcolor}{\rmfamily\fontsize{10.000000}{12.000000}\selectfont\catcode`\^=\active\def^{\ifmmode\sp\else\^{}\fi}\catcode`\%=\active\def%{\%}$\mathdefault{10^{1}}$}}%
\end{pgfscope}%
\begin{pgfscope}%
\pgfsetbuttcap%
\pgfsetroundjoin%
\definecolor{currentfill}{rgb}{0.000000,0.000000,0.000000}%
\pgfsetfillcolor{currentfill}%
\pgfsetlinewidth{0.803000pt}%
\definecolor{currentstroke}{rgb}{0.000000,0.000000,0.000000}%
\pgfsetstrokecolor{currentstroke}%
\pgfsetdash{}{0pt}%
\pgfsys@defobject{currentmarker}{\pgfqpoint{-0.048611in}{0.000000in}}{\pgfqpoint{-0.000000in}{0.000000in}}{%
\pgfpathmoveto{\pgfqpoint{-0.000000in}{0.000000in}}%
\pgfpathlineto{\pgfqpoint{-0.048611in}{0.000000in}}%
\pgfusepath{stroke,fill}%
}%
\begin{pgfscope}%
\pgfsys@transformshift{0.588387in}{1.369396in}%
\pgfsys@useobject{currentmarker}{}%
\end{pgfscope}%
\end{pgfscope}%
\begin{pgfscope}%
\definecolor{textcolor}{rgb}{0.000000,0.000000,0.000000}%
\pgfsetstrokecolor{textcolor}%
\pgfsetfillcolor{textcolor}%
\pgftext[x=0.289968in, y=1.316634in, left, base]{\color{textcolor}{\rmfamily\fontsize{10.000000}{12.000000}\selectfont\catcode`\^=\active\def^{\ifmmode\sp\else\^{}\fi}\catcode`\%=\active\def%{\%}$\mathdefault{10^{2}}$}}%
\end{pgfscope}%
\begin{pgfscope}%
\pgfsetbuttcap%
\pgfsetroundjoin%
\definecolor{currentfill}{rgb}{0.000000,0.000000,0.000000}%
\pgfsetfillcolor{currentfill}%
\pgfsetlinewidth{0.803000pt}%
\definecolor{currentstroke}{rgb}{0.000000,0.000000,0.000000}%
\pgfsetstrokecolor{currentstroke}%
\pgfsetdash{}{0pt}%
\pgfsys@defobject{currentmarker}{\pgfqpoint{-0.048611in}{0.000000in}}{\pgfqpoint{-0.000000in}{0.000000in}}{%
\pgfpathmoveto{\pgfqpoint{-0.000000in}{0.000000in}}%
\pgfpathlineto{\pgfqpoint{-0.048611in}{0.000000in}}%
\pgfusepath{stroke,fill}%
}%
\begin{pgfscope}%
\pgfsys@transformshift{0.588387in}{1.742832in}%
\pgfsys@useobject{currentmarker}{}%
\end{pgfscope}%
\end{pgfscope}%
\begin{pgfscope}%
\definecolor{textcolor}{rgb}{0.000000,0.000000,0.000000}%
\pgfsetstrokecolor{textcolor}%
\pgfsetfillcolor{textcolor}%
\pgftext[x=0.289968in, y=1.690071in, left, base]{\color{textcolor}{\rmfamily\fontsize{10.000000}{12.000000}\selectfont\catcode`\^=\active\def^{\ifmmode\sp\else\^{}\fi}\catcode`\%=\active\def%{\%}$\mathdefault{10^{3}}$}}%
\end{pgfscope}%
\begin{pgfscope}%
\pgfsetbuttcap%
\pgfsetroundjoin%
\definecolor{currentfill}{rgb}{0.000000,0.000000,0.000000}%
\pgfsetfillcolor{currentfill}%
\pgfsetlinewidth{0.803000pt}%
\definecolor{currentstroke}{rgb}{0.000000,0.000000,0.000000}%
\pgfsetstrokecolor{currentstroke}%
\pgfsetdash{}{0pt}%
\pgfsys@defobject{currentmarker}{\pgfqpoint{-0.048611in}{0.000000in}}{\pgfqpoint{-0.000000in}{0.000000in}}{%
\pgfpathmoveto{\pgfqpoint{-0.000000in}{0.000000in}}%
\pgfpathlineto{\pgfqpoint{-0.048611in}{0.000000in}}%
\pgfusepath{stroke,fill}%
}%
\begin{pgfscope}%
\pgfsys@transformshift{0.588387in}{2.116268in}%
\pgfsys@useobject{currentmarker}{}%
\end{pgfscope}%
\end{pgfscope}%
\begin{pgfscope}%
\definecolor{textcolor}{rgb}{0.000000,0.000000,0.000000}%
\pgfsetstrokecolor{textcolor}%
\pgfsetfillcolor{textcolor}%
\pgftext[x=0.289968in, y=2.063507in, left, base]{\color{textcolor}{\rmfamily\fontsize{10.000000}{12.000000}\selectfont\catcode`\^=\active\def^{\ifmmode\sp\else\^{}\fi}\catcode`\%=\active\def%{\%}$\mathdefault{10^{4}}$}}%
\end{pgfscope}%
\begin{pgfscope}%
\pgfsetbuttcap%
\pgfsetroundjoin%
\definecolor{currentfill}{rgb}{0.000000,0.000000,0.000000}%
\pgfsetfillcolor{currentfill}%
\pgfsetlinewidth{0.803000pt}%
\definecolor{currentstroke}{rgb}{0.000000,0.000000,0.000000}%
\pgfsetstrokecolor{currentstroke}%
\pgfsetdash{}{0pt}%
\pgfsys@defobject{currentmarker}{\pgfqpoint{-0.048611in}{0.000000in}}{\pgfqpoint{-0.000000in}{0.000000in}}{%
\pgfpathmoveto{\pgfqpoint{-0.000000in}{0.000000in}}%
\pgfpathlineto{\pgfqpoint{-0.048611in}{0.000000in}}%
\pgfusepath{stroke,fill}%
}%
\begin{pgfscope}%
\pgfsys@transformshift{0.588387in}{2.489705in}%
\pgfsys@useobject{currentmarker}{}%
\end{pgfscope}%
\end{pgfscope}%
\begin{pgfscope}%
\definecolor{textcolor}{rgb}{0.000000,0.000000,0.000000}%
\pgfsetstrokecolor{textcolor}%
\pgfsetfillcolor{textcolor}%
\pgftext[x=0.289968in, y=2.436943in, left, base]{\color{textcolor}{\rmfamily\fontsize{10.000000}{12.000000}\selectfont\catcode`\^=\active\def^{\ifmmode\sp\else\^{}\fi}\catcode`\%=\active\def%{\%}$\mathdefault{10^{5}}$}}%
\end{pgfscope}%
\begin{pgfscope}%
\pgfsetbuttcap%
\pgfsetroundjoin%
\definecolor{currentfill}{rgb}{0.000000,0.000000,0.000000}%
\pgfsetfillcolor{currentfill}%
\pgfsetlinewidth{0.602250pt}%
\definecolor{currentstroke}{rgb}{0.000000,0.000000,0.000000}%
\pgfsetstrokecolor{currentstroke}%
\pgfsetdash{}{0pt}%
\pgfsys@defobject{currentmarker}{\pgfqpoint{-0.027778in}{0.000000in}}{\pgfqpoint{-0.000000in}{0.000000in}}{%
\pgfpathmoveto{\pgfqpoint{-0.000000in}{0.000000in}}%
\pgfpathlineto{\pgfqpoint{-0.027778in}{0.000000in}}%
\pgfusepath{stroke,fill}%
}%
\begin{pgfscope}%
\pgfsys@transformshift{0.588387in}{0.539677in}%
\pgfsys@useobject{currentmarker}{}%
\end{pgfscope}%
\end{pgfscope}%
\begin{pgfscope}%
\pgfsetbuttcap%
\pgfsetroundjoin%
\definecolor{currentfill}{rgb}{0.000000,0.000000,0.000000}%
\pgfsetfillcolor{currentfill}%
\pgfsetlinewidth{0.602250pt}%
\definecolor{currentstroke}{rgb}{0.000000,0.000000,0.000000}%
\pgfsetstrokecolor{currentstroke}%
\pgfsetdash{}{0pt}%
\pgfsys@defobject{currentmarker}{\pgfqpoint{-0.027778in}{0.000000in}}{\pgfqpoint{-0.000000in}{0.000000in}}{%
\pgfpathmoveto{\pgfqpoint{-0.000000in}{0.000000in}}%
\pgfpathlineto{\pgfqpoint{-0.027778in}{0.000000in}}%
\pgfusepath{stroke,fill}%
}%
\begin{pgfscope}%
\pgfsys@transformshift{0.588387in}{0.564678in}%
\pgfsys@useobject{currentmarker}{}%
\end{pgfscope}%
\end{pgfscope}%
\begin{pgfscope}%
\pgfsetbuttcap%
\pgfsetroundjoin%
\definecolor{currentfill}{rgb}{0.000000,0.000000,0.000000}%
\pgfsetfillcolor{currentfill}%
\pgfsetlinewidth{0.602250pt}%
\definecolor{currentstroke}{rgb}{0.000000,0.000000,0.000000}%
\pgfsetstrokecolor{currentstroke}%
\pgfsetdash{}{0pt}%
\pgfsys@defobject{currentmarker}{\pgfqpoint{-0.027778in}{0.000000in}}{\pgfqpoint{-0.000000in}{0.000000in}}{%
\pgfpathmoveto{\pgfqpoint{-0.000000in}{0.000000in}}%
\pgfpathlineto{\pgfqpoint{-0.027778in}{0.000000in}}%
\pgfusepath{stroke,fill}%
}%
\begin{pgfscope}%
\pgfsys@transformshift{0.588387in}{0.586334in}%
\pgfsys@useobject{currentmarker}{}%
\end{pgfscope}%
\end{pgfscope}%
\begin{pgfscope}%
\pgfsetbuttcap%
\pgfsetroundjoin%
\definecolor{currentfill}{rgb}{0.000000,0.000000,0.000000}%
\pgfsetfillcolor{currentfill}%
\pgfsetlinewidth{0.602250pt}%
\definecolor{currentstroke}{rgb}{0.000000,0.000000,0.000000}%
\pgfsetstrokecolor{currentstroke}%
\pgfsetdash{}{0pt}%
\pgfsys@defobject{currentmarker}{\pgfqpoint{-0.027778in}{0.000000in}}{\pgfqpoint{-0.000000in}{0.000000in}}{%
\pgfpathmoveto{\pgfqpoint{-0.000000in}{0.000000in}}%
\pgfpathlineto{\pgfqpoint{-0.027778in}{0.000000in}}%
\pgfusepath{stroke,fill}%
}%
\begin{pgfscope}%
\pgfsys@transformshift{0.588387in}{0.605436in}%
\pgfsys@useobject{currentmarker}{}%
\end{pgfscope}%
\end{pgfscope}%
\begin{pgfscope}%
\pgfsetbuttcap%
\pgfsetroundjoin%
\definecolor{currentfill}{rgb}{0.000000,0.000000,0.000000}%
\pgfsetfillcolor{currentfill}%
\pgfsetlinewidth{0.602250pt}%
\definecolor{currentstroke}{rgb}{0.000000,0.000000,0.000000}%
\pgfsetstrokecolor{currentstroke}%
\pgfsetdash{}{0pt}%
\pgfsys@defobject{currentmarker}{\pgfqpoint{-0.027778in}{0.000000in}}{\pgfqpoint{-0.000000in}{0.000000in}}{%
\pgfpathmoveto{\pgfqpoint{-0.000000in}{0.000000in}}%
\pgfpathlineto{\pgfqpoint{-0.027778in}{0.000000in}}%
\pgfusepath{stroke,fill}%
}%
\begin{pgfscope}%
\pgfsys@transformshift{0.588387in}{0.734939in}%
\pgfsys@useobject{currentmarker}{}%
\end{pgfscope}%
\end{pgfscope}%
\begin{pgfscope}%
\pgfsetbuttcap%
\pgfsetroundjoin%
\definecolor{currentfill}{rgb}{0.000000,0.000000,0.000000}%
\pgfsetfillcolor{currentfill}%
\pgfsetlinewidth{0.602250pt}%
\definecolor{currentstroke}{rgb}{0.000000,0.000000,0.000000}%
\pgfsetstrokecolor{currentstroke}%
\pgfsetdash{}{0pt}%
\pgfsys@defobject{currentmarker}{\pgfqpoint{-0.027778in}{0.000000in}}{\pgfqpoint{-0.000000in}{0.000000in}}{%
\pgfpathmoveto{\pgfqpoint{-0.000000in}{0.000000in}}%
\pgfpathlineto{\pgfqpoint{-0.027778in}{0.000000in}}%
\pgfusepath{stroke,fill}%
}%
\begin{pgfscope}%
\pgfsys@transformshift{0.588387in}{0.800698in}%
\pgfsys@useobject{currentmarker}{}%
\end{pgfscope}%
\end{pgfscope}%
\begin{pgfscope}%
\pgfsetbuttcap%
\pgfsetroundjoin%
\definecolor{currentfill}{rgb}{0.000000,0.000000,0.000000}%
\pgfsetfillcolor{currentfill}%
\pgfsetlinewidth{0.602250pt}%
\definecolor{currentstroke}{rgb}{0.000000,0.000000,0.000000}%
\pgfsetstrokecolor{currentstroke}%
\pgfsetdash{}{0pt}%
\pgfsys@defobject{currentmarker}{\pgfqpoint{-0.027778in}{0.000000in}}{\pgfqpoint{-0.000000in}{0.000000in}}{%
\pgfpathmoveto{\pgfqpoint{-0.000000in}{0.000000in}}%
\pgfpathlineto{\pgfqpoint{-0.027778in}{0.000000in}}%
\pgfusepath{stroke,fill}%
}%
\begin{pgfscope}%
\pgfsys@transformshift{0.588387in}{0.847355in}%
\pgfsys@useobject{currentmarker}{}%
\end{pgfscope}%
\end{pgfscope}%
\begin{pgfscope}%
\pgfsetbuttcap%
\pgfsetroundjoin%
\definecolor{currentfill}{rgb}{0.000000,0.000000,0.000000}%
\pgfsetfillcolor{currentfill}%
\pgfsetlinewidth{0.602250pt}%
\definecolor{currentstroke}{rgb}{0.000000,0.000000,0.000000}%
\pgfsetstrokecolor{currentstroke}%
\pgfsetdash{}{0pt}%
\pgfsys@defobject{currentmarker}{\pgfqpoint{-0.027778in}{0.000000in}}{\pgfqpoint{-0.000000in}{0.000000in}}{%
\pgfpathmoveto{\pgfqpoint{-0.000000in}{0.000000in}}%
\pgfpathlineto{\pgfqpoint{-0.027778in}{0.000000in}}%
\pgfusepath{stroke,fill}%
}%
\begin{pgfscope}%
\pgfsys@transformshift{0.588387in}{0.883544in}%
\pgfsys@useobject{currentmarker}{}%
\end{pgfscope}%
\end{pgfscope}%
\begin{pgfscope}%
\pgfsetbuttcap%
\pgfsetroundjoin%
\definecolor{currentfill}{rgb}{0.000000,0.000000,0.000000}%
\pgfsetfillcolor{currentfill}%
\pgfsetlinewidth{0.602250pt}%
\definecolor{currentstroke}{rgb}{0.000000,0.000000,0.000000}%
\pgfsetstrokecolor{currentstroke}%
\pgfsetdash{}{0pt}%
\pgfsys@defobject{currentmarker}{\pgfqpoint{-0.027778in}{0.000000in}}{\pgfqpoint{-0.000000in}{0.000000in}}{%
\pgfpathmoveto{\pgfqpoint{-0.000000in}{0.000000in}}%
\pgfpathlineto{\pgfqpoint{-0.027778in}{0.000000in}}%
\pgfusepath{stroke,fill}%
}%
\begin{pgfscope}%
\pgfsys@transformshift{0.588387in}{0.913113in}%
\pgfsys@useobject{currentmarker}{}%
\end{pgfscope}%
\end{pgfscope}%
\begin{pgfscope}%
\pgfsetbuttcap%
\pgfsetroundjoin%
\definecolor{currentfill}{rgb}{0.000000,0.000000,0.000000}%
\pgfsetfillcolor{currentfill}%
\pgfsetlinewidth{0.602250pt}%
\definecolor{currentstroke}{rgb}{0.000000,0.000000,0.000000}%
\pgfsetstrokecolor{currentstroke}%
\pgfsetdash{}{0pt}%
\pgfsys@defobject{currentmarker}{\pgfqpoint{-0.027778in}{0.000000in}}{\pgfqpoint{-0.000000in}{0.000000in}}{%
\pgfpathmoveto{\pgfqpoint{-0.000000in}{0.000000in}}%
\pgfpathlineto{\pgfqpoint{-0.027778in}{0.000000in}}%
\pgfusepath{stroke,fill}%
}%
\begin{pgfscope}%
\pgfsys@transformshift{0.588387in}{0.938114in}%
\pgfsys@useobject{currentmarker}{}%
\end{pgfscope}%
\end{pgfscope}%
\begin{pgfscope}%
\pgfsetbuttcap%
\pgfsetroundjoin%
\definecolor{currentfill}{rgb}{0.000000,0.000000,0.000000}%
\pgfsetfillcolor{currentfill}%
\pgfsetlinewidth{0.602250pt}%
\definecolor{currentstroke}{rgb}{0.000000,0.000000,0.000000}%
\pgfsetstrokecolor{currentstroke}%
\pgfsetdash{}{0pt}%
\pgfsys@defobject{currentmarker}{\pgfqpoint{-0.027778in}{0.000000in}}{\pgfqpoint{-0.000000in}{0.000000in}}{%
\pgfpathmoveto{\pgfqpoint{-0.000000in}{0.000000in}}%
\pgfpathlineto{\pgfqpoint{-0.027778in}{0.000000in}}%
\pgfusepath{stroke,fill}%
}%
\begin{pgfscope}%
\pgfsys@transformshift{0.588387in}{0.959770in}%
\pgfsys@useobject{currentmarker}{}%
\end{pgfscope}%
\end{pgfscope}%
\begin{pgfscope}%
\pgfsetbuttcap%
\pgfsetroundjoin%
\definecolor{currentfill}{rgb}{0.000000,0.000000,0.000000}%
\pgfsetfillcolor{currentfill}%
\pgfsetlinewidth{0.602250pt}%
\definecolor{currentstroke}{rgb}{0.000000,0.000000,0.000000}%
\pgfsetstrokecolor{currentstroke}%
\pgfsetdash{}{0pt}%
\pgfsys@defobject{currentmarker}{\pgfqpoint{-0.027778in}{0.000000in}}{\pgfqpoint{-0.000000in}{0.000000in}}{%
\pgfpathmoveto{\pgfqpoint{-0.000000in}{0.000000in}}%
\pgfpathlineto{\pgfqpoint{-0.027778in}{0.000000in}}%
\pgfusepath{stroke,fill}%
}%
\begin{pgfscope}%
\pgfsys@transformshift{0.588387in}{0.978872in}%
\pgfsys@useobject{currentmarker}{}%
\end{pgfscope}%
\end{pgfscope}%
\begin{pgfscope}%
\pgfsetbuttcap%
\pgfsetroundjoin%
\definecolor{currentfill}{rgb}{0.000000,0.000000,0.000000}%
\pgfsetfillcolor{currentfill}%
\pgfsetlinewidth{0.602250pt}%
\definecolor{currentstroke}{rgb}{0.000000,0.000000,0.000000}%
\pgfsetstrokecolor{currentstroke}%
\pgfsetdash{}{0pt}%
\pgfsys@defobject{currentmarker}{\pgfqpoint{-0.027778in}{0.000000in}}{\pgfqpoint{-0.000000in}{0.000000in}}{%
\pgfpathmoveto{\pgfqpoint{-0.000000in}{0.000000in}}%
\pgfpathlineto{\pgfqpoint{-0.027778in}{0.000000in}}%
\pgfusepath{stroke,fill}%
}%
\begin{pgfscope}%
\pgfsys@transformshift{0.588387in}{1.108375in}%
\pgfsys@useobject{currentmarker}{}%
\end{pgfscope}%
\end{pgfscope}%
\begin{pgfscope}%
\pgfsetbuttcap%
\pgfsetroundjoin%
\definecolor{currentfill}{rgb}{0.000000,0.000000,0.000000}%
\pgfsetfillcolor{currentfill}%
\pgfsetlinewidth{0.602250pt}%
\definecolor{currentstroke}{rgb}{0.000000,0.000000,0.000000}%
\pgfsetstrokecolor{currentstroke}%
\pgfsetdash{}{0pt}%
\pgfsys@defobject{currentmarker}{\pgfqpoint{-0.027778in}{0.000000in}}{\pgfqpoint{-0.000000in}{0.000000in}}{%
\pgfpathmoveto{\pgfqpoint{-0.000000in}{0.000000in}}%
\pgfpathlineto{\pgfqpoint{-0.027778in}{0.000000in}}%
\pgfusepath{stroke,fill}%
}%
\begin{pgfscope}%
\pgfsys@transformshift{0.588387in}{1.174134in}%
\pgfsys@useobject{currentmarker}{}%
\end{pgfscope}%
\end{pgfscope}%
\begin{pgfscope}%
\pgfsetbuttcap%
\pgfsetroundjoin%
\definecolor{currentfill}{rgb}{0.000000,0.000000,0.000000}%
\pgfsetfillcolor{currentfill}%
\pgfsetlinewidth{0.602250pt}%
\definecolor{currentstroke}{rgb}{0.000000,0.000000,0.000000}%
\pgfsetstrokecolor{currentstroke}%
\pgfsetdash{}{0pt}%
\pgfsys@defobject{currentmarker}{\pgfqpoint{-0.027778in}{0.000000in}}{\pgfqpoint{-0.000000in}{0.000000in}}{%
\pgfpathmoveto{\pgfqpoint{-0.000000in}{0.000000in}}%
\pgfpathlineto{\pgfqpoint{-0.027778in}{0.000000in}}%
\pgfusepath{stroke,fill}%
}%
\begin{pgfscope}%
\pgfsys@transformshift{0.588387in}{1.220791in}%
\pgfsys@useobject{currentmarker}{}%
\end{pgfscope}%
\end{pgfscope}%
\begin{pgfscope}%
\pgfsetbuttcap%
\pgfsetroundjoin%
\definecolor{currentfill}{rgb}{0.000000,0.000000,0.000000}%
\pgfsetfillcolor{currentfill}%
\pgfsetlinewidth{0.602250pt}%
\definecolor{currentstroke}{rgb}{0.000000,0.000000,0.000000}%
\pgfsetstrokecolor{currentstroke}%
\pgfsetdash{}{0pt}%
\pgfsys@defobject{currentmarker}{\pgfqpoint{-0.027778in}{0.000000in}}{\pgfqpoint{-0.000000in}{0.000000in}}{%
\pgfpathmoveto{\pgfqpoint{-0.000000in}{0.000000in}}%
\pgfpathlineto{\pgfqpoint{-0.027778in}{0.000000in}}%
\pgfusepath{stroke,fill}%
}%
\begin{pgfscope}%
\pgfsys@transformshift{0.588387in}{1.256980in}%
\pgfsys@useobject{currentmarker}{}%
\end{pgfscope}%
\end{pgfscope}%
\begin{pgfscope}%
\pgfsetbuttcap%
\pgfsetroundjoin%
\definecolor{currentfill}{rgb}{0.000000,0.000000,0.000000}%
\pgfsetfillcolor{currentfill}%
\pgfsetlinewidth{0.602250pt}%
\definecolor{currentstroke}{rgb}{0.000000,0.000000,0.000000}%
\pgfsetstrokecolor{currentstroke}%
\pgfsetdash{}{0pt}%
\pgfsys@defobject{currentmarker}{\pgfqpoint{-0.027778in}{0.000000in}}{\pgfqpoint{-0.000000in}{0.000000in}}{%
\pgfpathmoveto{\pgfqpoint{-0.000000in}{0.000000in}}%
\pgfpathlineto{\pgfqpoint{-0.027778in}{0.000000in}}%
\pgfusepath{stroke,fill}%
}%
\begin{pgfscope}%
\pgfsys@transformshift{0.588387in}{1.286550in}%
\pgfsys@useobject{currentmarker}{}%
\end{pgfscope}%
\end{pgfscope}%
\begin{pgfscope}%
\pgfsetbuttcap%
\pgfsetroundjoin%
\definecolor{currentfill}{rgb}{0.000000,0.000000,0.000000}%
\pgfsetfillcolor{currentfill}%
\pgfsetlinewidth{0.602250pt}%
\definecolor{currentstroke}{rgb}{0.000000,0.000000,0.000000}%
\pgfsetstrokecolor{currentstroke}%
\pgfsetdash{}{0pt}%
\pgfsys@defobject{currentmarker}{\pgfqpoint{-0.027778in}{0.000000in}}{\pgfqpoint{-0.000000in}{0.000000in}}{%
\pgfpathmoveto{\pgfqpoint{-0.000000in}{0.000000in}}%
\pgfpathlineto{\pgfqpoint{-0.027778in}{0.000000in}}%
\pgfusepath{stroke,fill}%
}%
\begin{pgfscope}%
\pgfsys@transformshift{0.588387in}{1.311550in}%
\pgfsys@useobject{currentmarker}{}%
\end{pgfscope}%
\end{pgfscope}%
\begin{pgfscope}%
\pgfsetbuttcap%
\pgfsetroundjoin%
\definecolor{currentfill}{rgb}{0.000000,0.000000,0.000000}%
\pgfsetfillcolor{currentfill}%
\pgfsetlinewidth{0.602250pt}%
\definecolor{currentstroke}{rgb}{0.000000,0.000000,0.000000}%
\pgfsetstrokecolor{currentstroke}%
\pgfsetdash{}{0pt}%
\pgfsys@defobject{currentmarker}{\pgfqpoint{-0.027778in}{0.000000in}}{\pgfqpoint{-0.000000in}{0.000000in}}{%
\pgfpathmoveto{\pgfqpoint{-0.000000in}{0.000000in}}%
\pgfpathlineto{\pgfqpoint{-0.027778in}{0.000000in}}%
\pgfusepath{stroke,fill}%
}%
\begin{pgfscope}%
\pgfsys@transformshift{0.588387in}{1.333206in}%
\pgfsys@useobject{currentmarker}{}%
\end{pgfscope}%
\end{pgfscope}%
\begin{pgfscope}%
\pgfsetbuttcap%
\pgfsetroundjoin%
\definecolor{currentfill}{rgb}{0.000000,0.000000,0.000000}%
\pgfsetfillcolor{currentfill}%
\pgfsetlinewidth{0.602250pt}%
\definecolor{currentstroke}{rgb}{0.000000,0.000000,0.000000}%
\pgfsetstrokecolor{currentstroke}%
\pgfsetdash{}{0pt}%
\pgfsys@defobject{currentmarker}{\pgfqpoint{-0.027778in}{0.000000in}}{\pgfqpoint{-0.000000in}{0.000000in}}{%
\pgfpathmoveto{\pgfqpoint{-0.000000in}{0.000000in}}%
\pgfpathlineto{\pgfqpoint{-0.027778in}{0.000000in}}%
\pgfusepath{stroke,fill}%
}%
\begin{pgfscope}%
\pgfsys@transformshift{0.588387in}{1.352308in}%
\pgfsys@useobject{currentmarker}{}%
\end{pgfscope}%
\end{pgfscope}%
\begin{pgfscope}%
\pgfsetbuttcap%
\pgfsetroundjoin%
\definecolor{currentfill}{rgb}{0.000000,0.000000,0.000000}%
\pgfsetfillcolor{currentfill}%
\pgfsetlinewidth{0.602250pt}%
\definecolor{currentstroke}{rgb}{0.000000,0.000000,0.000000}%
\pgfsetstrokecolor{currentstroke}%
\pgfsetdash{}{0pt}%
\pgfsys@defobject{currentmarker}{\pgfqpoint{-0.027778in}{0.000000in}}{\pgfqpoint{-0.000000in}{0.000000in}}{%
\pgfpathmoveto{\pgfqpoint{-0.000000in}{0.000000in}}%
\pgfpathlineto{\pgfqpoint{-0.027778in}{0.000000in}}%
\pgfusepath{stroke,fill}%
}%
\begin{pgfscope}%
\pgfsys@transformshift{0.588387in}{1.481811in}%
\pgfsys@useobject{currentmarker}{}%
\end{pgfscope}%
\end{pgfscope}%
\begin{pgfscope}%
\pgfsetbuttcap%
\pgfsetroundjoin%
\definecolor{currentfill}{rgb}{0.000000,0.000000,0.000000}%
\pgfsetfillcolor{currentfill}%
\pgfsetlinewidth{0.602250pt}%
\definecolor{currentstroke}{rgb}{0.000000,0.000000,0.000000}%
\pgfsetstrokecolor{currentstroke}%
\pgfsetdash{}{0pt}%
\pgfsys@defobject{currentmarker}{\pgfqpoint{-0.027778in}{0.000000in}}{\pgfqpoint{-0.000000in}{0.000000in}}{%
\pgfpathmoveto{\pgfqpoint{-0.000000in}{0.000000in}}%
\pgfpathlineto{\pgfqpoint{-0.027778in}{0.000000in}}%
\pgfusepath{stroke,fill}%
}%
\begin{pgfscope}%
\pgfsys@transformshift{0.588387in}{1.547570in}%
\pgfsys@useobject{currentmarker}{}%
\end{pgfscope}%
\end{pgfscope}%
\begin{pgfscope}%
\pgfsetbuttcap%
\pgfsetroundjoin%
\definecolor{currentfill}{rgb}{0.000000,0.000000,0.000000}%
\pgfsetfillcolor{currentfill}%
\pgfsetlinewidth{0.602250pt}%
\definecolor{currentstroke}{rgb}{0.000000,0.000000,0.000000}%
\pgfsetstrokecolor{currentstroke}%
\pgfsetdash{}{0pt}%
\pgfsys@defobject{currentmarker}{\pgfqpoint{-0.027778in}{0.000000in}}{\pgfqpoint{-0.000000in}{0.000000in}}{%
\pgfpathmoveto{\pgfqpoint{-0.000000in}{0.000000in}}%
\pgfpathlineto{\pgfqpoint{-0.027778in}{0.000000in}}%
\pgfusepath{stroke,fill}%
}%
\begin{pgfscope}%
\pgfsys@transformshift{0.588387in}{1.594227in}%
\pgfsys@useobject{currentmarker}{}%
\end{pgfscope}%
\end{pgfscope}%
\begin{pgfscope}%
\pgfsetbuttcap%
\pgfsetroundjoin%
\definecolor{currentfill}{rgb}{0.000000,0.000000,0.000000}%
\pgfsetfillcolor{currentfill}%
\pgfsetlinewidth{0.602250pt}%
\definecolor{currentstroke}{rgb}{0.000000,0.000000,0.000000}%
\pgfsetstrokecolor{currentstroke}%
\pgfsetdash{}{0pt}%
\pgfsys@defobject{currentmarker}{\pgfqpoint{-0.027778in}{0.000000in}}{\pgfqpoint{-0.000000in}{0.000000in}}{%
\pgfpathmoveto{\pgfqpoint{-0.000000in}{0.000000in}}%
\pgfpathlineto{\pgfqpoint{-0.027778in}{0.000000in}}%
\pgfusepath{stroke,fill}%
}%
\begin{pgfscope}%
\pgfsys@transformshift{0.588387in}{1.630417in}%
\pgfsys@useobject{currentmarker}{}%
\end{pgfscope}%
\end{pgfscope}%
\begin{pgfscope}%
\pgfsetbuttcap%
\pgfsetroundjoin%
\definecolor{currentfill}{rgb}{0.000000,0.000000,0.000000}%
\pgfsetfillcolor{currentfill}%
\pgfsetlinewidth{0.602250pt}%
\definecolor{currentstroke}{rgb}{0.000000,0.000000,0.000000}%
\pgfsetstrokecolor{currentstroke}%
\pgfsetdash{}{0pt}%
\pgfsys@defobject{currentmarker}{\pgfqpoint{-0.027778in}{0.000000in}}{\pgfqpoint{-0.000000in}{0.000000in}}{%
\pgfpathmoveto{\pgfqpoint{-0.000000in}{0.000000in}}%
\pgfpathlineto{\pgfqpoint{-0.027778in}{0.000000in}}%
\pgfusepath{stroke,fill}%
}%
\begin{pgfscope}%
\pgfsys@transformshift{0.588387in}{1.659986in}%
\pgfsys@useobject{currentmarker}{}%
\end{pgfscope}%
\end{pgfscope}%
\begin{pgfscope}%
\pgfsetbuttcap%
\pgfsetroundjoin%
\definecolor{currentfill}{rgb}{0.000000,0.000000,0.000000}%
\pgfsetfillcolor{currentfill}%
\pgfsetlinewidth{0.602250pt}%
\definecolor{currentstroke}{rgb}{0.000000,0.000000,0.000000}%
\pgfsetstrokecolor{currentstroke}%
\pgfsetdash{}{0pt}%
\pgfsys@defobject{currentmarker}{\pgfqpoint{-0.027778in}{0.000000in}}{\pgfqpoint{-0.000000in}{0.000000in}}{%
\pgfpathmoveto{\pgfqpoint{-0.000000in}{0.000000in}}%
\pgfpathlineto{\pgfqpoint{-0.027778in}{0.000000in}}%
\pgfusepath{stroke,fill}%
}%
\begin{pgfscope}%
\pgfsys@transformshift{0.588387in}{1.684986in}%
\pgfsys@useobject{currentmarker}{}%
\end{pgfscope}%
\end{pgfscope}%
\begin{pgfscope}%
\pgfsetbuttcap%
\pgfsetroundjoin%
\definecolor{currentfill}{rgb}{0.000000,0.000000,0.000000}%
\pgfsetfillcolor{currentfill}%
\pgfsetlinewidth{0.602250pt}%
\definecolor{currentstroke}{rgb}{0.000000,0.000000,0.000000}%
\pgfsetstrokecolor{currentstroke}%
\pgfsetdash{}{0pt}%
\pgfsys@defobject{currentmarker}{\pgfqpoint{-0.027778in}{0.000000in}}{\pgfqpoint{-0.000000in}{0.000000in}}{%
\pgfpathmoveto{\pgfqpoint{-0.000000in}{0.000000in}}%
\pgfpathlineto{\pgfqpoint{-0.027778in}{0.000000in}}%
\pgfusepath{stroke,fill}%
}%
\begin{pgfscope}%
\pgfsys@transformshift{0.588387in}{1.706642in}%
\pgfsys@useobject{currentmarker}{}%
\end{pgfscope}%
\end{pgfscope}%
\begin{pgfscope}%
\pgfsetbuttcap%
\pgfsetroundjoin%
\definecolor{currentfill}{rgb}{0.000000,0.000000,0.000000}%
\pgfsetfillcolor{currentfill}%
\pgfsetlinewidth{0.602250pt}%
\definecolor{currentstroke}{rgb}{0.000000,0.000000,0.000000}%
\pgfsetstrokecolor{currentstroke}%
\pgfsetdash{}{0pt}%
\pgfsys@defobject{currentmarker}{\pgfqpoint{-0.027778in}{0.000000in}}{\pgfqpoint{-0.000000in}{0.000000in}}{%
\pgfpathmoveto{\pgfqpoint{-0.000000in}{0.000000in}}%
\pgfpathlineto{\pgfqpoint{-0.027778in}{0.000000in}}%
\pgfusepath{stroke,fill}%
}%
\begin{pgfscope}%
\pgfsys@transformshift{0.588387in}{1.725745in}%
\pgfsys@useobject{currentmarker}{}%
\end{pgfscope}%
\end{pgfscope}%
\begin{pgfscope}%
\pgfsetbuttcap%
\pgfsetroundjoin%
\definecolor{currentfill}{rgb}{0.000000,0.000000,0.000000}%
\pgfsetfillcolor{currentfill}%
\pgfsetlinewidth{0.602250pt}%
\definecolor{currentstroke}{rgb}{0.000000,0.000000,0.000000}%
\pgfsetstrokecolor{currentstroke}%
\pgfsetdash{}{0pt}%
\pgfsys@defobject{currentmarker}{\pgfqpoint{-0.027778in}{0.000000in}}{\pgfqpoint{-0.000000in}{0.000000in}}{%
\pgfpathmoveto{\pgfqpoint{-0.000000in}{0.000000in}}%
\pgfpathlineto{\pgfqpoint{-0.027778in}{0.000000in}}%
\pgfusepath{stroke,fill}%
}%
\begin{pgfscope}%
\pgfsys@transformshift{0.588387in}{1.855248in}%
\pgfsys@useobject{currentmarker}{}%
\end{pgfscope}%
\end{pgfscope}%
\begin{pgfscope}%
\pgfsetbuttcap%
\pgfsetroundjoin%
\definecolor{currentfill}{rgb}{0.000000,0.000000,0.000000}%
\pgfsetfillcolor{currentfill}%
\pgfsetlinewidth{0.602250pt}%
\definecolor{currentstroke}{rgb}{0.000000,0.000000,0.000000}%
\pgfsetstrokecolor{currentstroke}%
\pgfsetdash{}{0pt}%
\pgfsys@defobject{currentmarker}{\pgfqpoint{-0.027778in}{0.000000in}}{\pgfqpoint{-0.000000in}{0.000000in}}{%
\pgfpathmoveto{\pgfqpoint{-0.000000in}{0.000000in}}%
\pgfpathlineto{\pgfqpoint{-0.027778in}{0.000000in}}%
\pgfusepath{stroke,fill}%
}%
\begin{pgfscope}%
\pgfsys@transformshift{0.588387in}{1.921007in}%
\pgfsys@useobject{currentmarker}{}%
\end{pgfscope}%
\end{pgfscope}%
\begin{pgfscope}%
\pgfsetbuttcap%
\pgfsetroundjoin%
\definecolor{currentfill}{rgb}{0.000000,0.000000,0.000000}%
\pgfsetfillcolor{currentfill}%
\pgfsetlinewidth{0.602250pt}%
\definecolor{currentstroke}{rgb}{0.000000,0.000000,0.000000}%
\pgfsetstrokecolor{currentstroke}%
\pgfsetdash{}{0pt}%
\pgfsys@defobject{currentmarker}{\pgfqpoint{-0.027778in}{0.000000in}}{\pgfqpoint{-0.000000in}{0.000000in}}{%
\pgfpathmoveto{\pgfqpoint{-0.000000in}{0.000000in}}%
\pgfpathlineto{\pgfqpoint{-0.027778in}{0.000000in}}%
\pgfusepath{stroke,fill}%
}%
\begin{pgfscope}%
\pgfsys@transformshift{0.588387in}{1.967663in}%
\pgfsys@useobject{currentmarker}{}%
\end{pgfscope}%
\end{pgfscope}%
\begin{pgfscope}%
\pgfsetbuttcap%
\pgfsetroundjoin%
\definecolor{currentfill}{rgb}{0.000000,0.000000,0.000000}%
\pgfsetfillcolor{currentfill}%
\pgfsetlinewidth{0.602250pt}%
\definecolor{currentstroke}{rgb}{0.000000,0.000000,0.000000}%
\pgfsetstrokecolor{currentstroke}%
\pgfsetdash{}{0pt}%
\pgfsys@defobject{currentmarker}{\pgfqpoint{-0.027778in}{0.000000in}}{\pgfqpoint{-0.000000in}{0.000000in}}{%
\pgfpathmoveto{\pgfqpoint{-0.000000in}{0.000000in}}%
\pgfpathlineto{\pgfqpoint{-0.027778in}{0.000000in}}%
\pgfusepath{stroke,fill}%
}%
\begin{pgfscope}%
\pgfsys@transformshift{0.588387in}{2.003853in}%
\pgfsys@useobject{currentmarker}{}%
\end{pgfscope}%
\end{pgfscope}%
\begin{pgfscope}%
\pgfsetbuttcap%
\pgfsetroundjoin%
\definecolor{currentfill}{rgb}{0.000000,0.000000,0.000000}%
\pgfsetfillcolor{currentfill}%
\pgfsetlinewidth{0.602250pt}%
\definecolor{currentstroke}{rgb}{0.000000,0.000000,0.000000}%
\pgfsetstrokecolor{currentstroke}%
\pgfsetdash{}{0pt}%
\pgfsys@defobject{currentmarker}{\pgfqpoint{-0.027778in}{0.000000in}}{\pgfqpoint{-0.000000in}{0.000000in}}{%
\pgfpathmoveto{\pgfqpoint{-0.000000in}{0.000000in}}%
\pgfpathlineto{\pgfqpoint{-0.027778in}{0.000000in}}%
\pgfusepath{stroke,fill}%
}%
\begin{pgfscope}%
\pgfsys@transformshift{0.588387in}{2.033422in}%
\pgfsys@useobject{currentmarker}{}%
\end{pgfscope}%
\end{pgfscope}%
\begin{pgfscope}%
\pgfsetbuttcap%
\pgfsetroundjoin%
\definecolor{currentfill}{rgb}{0.000000,0.000000,0.000000}%
\pgfsetfillcolor{currentfill}%
\pgfsetlinewidth{0.602250pt}%
\definecolor{currentstroke}{rgb}{0.000000,0.000000,0.000000}%
\pgfsetstrokecolor{currentstroke}%
\pgfsetdash{}{0pt}%
\pgfsys@defobject{currentmarker}{\pgfqpoint{-0.027778in}{0.000000in}}{\pgfqpoint{-0.000000in}{0.000000in}}{%
\pgfpathmoveto{\pgfqpoint{-0.000000in}{0.000000in}}%
\pgfpathlineto{\pgfqpoint{-0.027778in}{0.000000in}}%
\pgfusepath{stroke,fill}%
}%
\begin{pgfscope}%
\pgfsys@transformshift{0.588387in}{2.058422in}%
\pgfsys@useobject{currentmarker}{}%
\end{pgfscope}%
\end{pgfscope}%
\begin{pgfscope}%
\pgfsetbuttcap%
\pgfsetroundjoin%
\definecolor{currentfill}{rgb}{0.000000,0.000000,0.000000}%
\pgfsetfillcolor{currentfill}%
\pgfsetlinewidth{0.602250pt}%
\definecolor{currentstroke}{rgb}{0.000000,0.000000,0.000000}%
\pgfsetstrokecolor{currentstroke}%
\pgfsetdash{}{0pt}%
\pgfsys@defobject{currentmarker}{\pgfqpoint{-0.027778in}{0.000000in}}{\pgfqpoint{-0.000000in}{0.000000in}}{%
\pgfpathmoveto{\pgfqpoint{-0.000000in}{0.000000in}}%
\pgfpathlineto{\pgfqpoint{-0.027778in}{0.000000in}}%
\pgfusepath{stroke,fill}%
}%
\begin{pgfscope}%
\pgfsys@transformshift{0.588387in}{2.080079in}%
\pgfsys@useobject{currentmarker}{}%
\end{pgfscope}%
\end{pgfscope}%
\begin{pgfscope}%
\pgfsetbuttcap%
\pgfsetroundjoin%
\definecolor{currentfill}{rgb}{0.000000,0.000000,0.000000}%
\pgfsetfillcolor{currentfill}%
\pgfsetlinewidth{0.602250pt}%
\definecolor{currentstroke}{rgb}{0.000000,0.000000,0.000000}%
\pgfsetstrokecolor{currentstroke}%
\pgfsetdash{}{0pt}%
\pgfsys@defobject{currentmarker}{\pgfqpoint{-0.027778in}{0.000000in}}{\pgfqpoint{-0.000000in}{0.000000in}}{%
\pgfpathmoveto{\pgfqpoint{-0.000000in}{0.000000in}}%
\pgfpathlineto{\pgfqpoint{-0.027778in}{0.000000in}}%
\pgfusepath{stroke,fill}%
}%
\begin{pgfscope}%
\pgfsys@transformshift{0.588387in}{2.099181in}%
\pgfsys@useobject{currentmarker}{}%
\end{pgfscope}%
\end{pgfscope}%
\begin{pgfscope}%
\pgfsetbuttcap%
\pgfsetroundjoin%
\definecolor{currentfill}{rgb}{0.000000,0.000000,0.000000}%
\pgfsetfillcolor{currentfill}%
\pgfsetlinewidth{0.602250pt}%
\definecolor{currentstroke}{rgb}{0.000000,0.000000,0.000000}%
\pgfsetstrokecolor{currentstroke}%
\pgfsetdash{}{0pt}%
\pgfsys@defobject{currentmarker}{\pgfqpoint{-0.027778in}{0.000000in}}{\pgfqpoint{-0.000000in}{0.000000in}}{%
\pgfpathmoveto{\pgfqpoint{-0.000000in}{0.000000in}}%
\pgfpathlineto{\pgfqpoint{-0.027778in}{0.000000in}}%
\pgfusepath{stroke,fill}%
}%
\begin{pgfscope}%
\pgfsys@transformshift{0.588387in}{2.228684in}%
\pgfsys@useobject{currentmarker}{}%
\end{pgfscope}%
\end{pgfscope}%
\begin{pgfscope}%
\pgfsetbuttcap%
\pgfsetroundjoin%
\definecolor{currentfill}{rgb}{0.000000,0.000000,0.000000}%
\pgfsetfillcolor{currentfill}%
\pgfsetlinewidth{0.602250pt}%
\definecolor{currentstroke}{rgb}{0.000000,0.000000,0.000000}%
\pgfsetstrokecolor{currentstroke}%
\pgfsetdash{}{0pt}%
\pgfsys@defobject{currentmarker}{\pgfqpoint{-0.027778in}{0.000000in}}{\pgfqpoint{-0.000000in}{0.000000in}}{%
\pgfpathmoveto{\pgfqpoint{-0.000000in}{0.000000in}}%
\pgfpathlineto{\pgfqpoint{-0.027778in}{0.000000in}}%
\pgfusepath{stroke,fill}%
}%
\begin{pgfscope}%
\pgfsys@transformshift{0.588387in}{2.294443in}%
\pgfsys@useobject{currentmarker}{}%
\end{pgfscope}%
\end{pgfscope}%
\begin{pgfscope}%
\pgfsetbuttcap%
\pgfsetroundjoin%
\definecolor{currentfill}{rgb}{0.000000,0.000000,0.000000}%
\pgfsetfillcolor{currentfill}%
\pgfsetlinewidth{0.602250pt}%
\definecolor{currentstroke}{rgb}{0.000000,0.000000,0.000000}%
\pgfsetstrokecolor{currentstroke}%
\pgfsetdash{}{0pt}%
\pgfsys@defobject{currentmarker}{\pgfqpoint{-0.027778in}{0.000000in}}{\pgfqpoint{-0.000000in}{0.000000in}}{%
\pgfpathmoveto{\pgfqpoint{-0.000000in}{0.000000in}}%
\pgfpathlineto{\pgfqpoint{-0.027778in}{0.000000in}}%
\pgfusepath{stroke,fill}%
}%
\begin{pgfscope}%
\pgfsys@transformshift{0.588387in}{2.341099in}%
\pgfsys@useobject{currentmarker}{}%
\end{pgfscope}%
\end{pgfscope}%
\begin{pgfscope}%
\pgfsetbuttcap%
\pgfsetroundjoin%
\definecolor{currentfill}{rgb}{0.000000,0.000000,0.000000}%
\pgfsetfillcolor{currentfill}%
\pgfsetlinewidth{0.602250pt}%
\definecolor{currentstroke}{rgb}{0.000000,0.000000,0.000000}%
\pgfsetstrokecolor{currentstroke}%
\pgfsetdash{}{0pt}%
\pgfsys@defobject{currentmarker}{\pgfqpoint{-0.027778in}{0.000000in}}{\pgfqpoint{-0.000000in}{0.000000in}}{%
\pgfpathmoveto{\pgfqpoint{-0.000000in}{0.000000in}}%
\pgfpathlineto{\pgfqpoint{-0.027778in}{0.000000in}}%
\pgfusepath{stroke,fill}%
}%
\begin{pgfscope}%
\pgfsys@transformshift{0.588387in}{2.377289in}%
\pgfsys@useobject{currentmarker}{}%
\end{pgfscope}%
\end{pgfscope}%
\begin{pgfscope}%
\pgfsetbuttcap%
\pgfsetroundjoin%
\definecolor{currentfill}{rgb}{0.000000,0.000000,0.000000}%
\pgfsetfillcolor{currentfill}%
\pgfsetlinewidth{0.602250pt}%
\definecolor{currentstroke}{rgb}{0.000000,0.000000,0.000000}%
\pgfsetstrokecolor{currentstroke}%
\pgfsetdash{}{0pt}%
\pgfsys@defobject{currentmarker}{\pgfqpoint{-0.027778in}{0.000000in}}{\pgfqpoint{-0.000000in}{0.000000in}}{%
\pgfpathmoveto{\pgfqpoint{-0.000000in}{0.000000in}}%
\pgfpathlineto{\pgfqpoint{-0.027778in}{0.000000in}}%
\pgfusepath{stroke,fill}%
}%
\begin{pgfscope}%
\pgfsys@transformshift{0.588387in}{2.406858in}%
\pgfsys@useobject{currentmarker}{}%
\end{pgfscope}%
\end{pgfscope}%
\begin{pgfscope}%
\pgfsetbuttcap%
\pgfsetroundjoin%
\definecolor{currentfill}{rgb}{0.000000,0.000000,0.000000}%
\pgfsetfillcolor{currentfill}%
\pgfsetlinewidth{0.602250pt}%
\definecolor{currentstroke}{rgb}{0.000000,0.000000,0.000000}%
\pgfsetstrokecolor{currentstroke}%
\pgfsetdash{}{0pt}%
\pgfsys@defobject{currentmarker}{\pgfqpoint{-0.027778in}{0.000000in}}{\pgfqpoint{-0.000000in}{0.000000in}}{%
\pgfpathmoveto{\pgfqpoint{-0.000000in}{0.000000in}}%
\pgfpathlineto{\pgfqpoint{-0.027778in}{0.000000in}}%
\pgfusepath{stroke,fill}%
}%
\begin{pgfscope}%
\pgfsys@transformshift{0.588387in}{2.431859in}%
\pgfsys@useobject{currentmarker}{}%
\end{pgfscope}%
\end{pgfscope}%
\begin{pgfscope}%
\pgfsetbuttcap%
\pgfsetroundjoin%
\definecolor{currentfill}{rgb}{0.000000,0.000000,0.000000}%
\pgfsetfillcolor{currentfill}%
\pgfsetlinewidth{0.602250pt}%
\definecolor{currentstroke}{rgb}{0.000000,0.000000,0.000000}%
\pgfsetstrokecolor{currentstroke}%
\pgfsetdash{}{0pt}%
\pgfsys@defobject{currentmarker}{\pgfqpoint{-0.027778in}{0.000000in}}{\pgfqpoint{-0.000000in}{0.000000in}}{%
\pgfpathmoveto{\pgfqpoint{-0.000000in}{0.000000in}}%
\pgfpathlineto{\pgfqpoint{-0.027778in}{0.000000in}}%
\pgfusepath{stroke,fill}%
}%
\begin{pgfscope}%
\pgfsys@transformshift{0.588387in}{2.453515in}%
\pgfsys@useobject{currentmarker}{}%
\end{pgfscope}%
\end{pgfscope}%
\begin{pgfscope}%
\pgfsetbuttcap%
\pgfsetroundjoin%
\definecolor{currentfill}{rgb}{0.000000,0.000000,0.000000}%
\pgfsetfillcolor{currentfill}%
\pgfsetlinewidth{0.602250pt}%
\definecolor{currentstroke}{rgb}{0.000000,0.000000,0.000000}%
\pgfsetstrokecolor{currentstroke}%
\pgfsetdash{}{0pt}%
\pgfsys@defobject{currentmarker}{\pgfqpoint{-0.027778in}{0.000000in}}{\pgfqpoint{-0.000000in}{0.000000in}}{%
\pgfpathmoveto{\pgfqpoint{-0.000000in}{0.000000in}}%
\pgfpathlineto{\pgfqpoint{-0.027778in}{0.000000in}}%
\pgfusepath{stroke,fill}%
}%
\begin{pgfscope}%
\pgfsys@transformshift{0.588387in}{2.472617in}%
\pgfsys@useobject{currentmarker}{}%
\end{pgfscope}%
\end{pgfscope}%
\begin{pgfscope}%
\pgfsetbuttcap%
\pgfsetroundjoin%
\definecolor{currentfill}{rgb}{0.000000,0.000000,0.000000}%
\pgfsetfillcolor{currentfill}%
\pgfsetlinewidth{0.602250pt}%
\definecolor{currentstroke}{rgb}{0.000000,0.000000,0.000000}%
\pgfsetstrokecolor{currentstroke}%
\pgfsetdash{}{0pt}%
\pgfsys@defobject{currentmarker}{\pgfqpoint{-0.027778in}{0.000000in}}{\pgfqpoint{-0.000000in}{0.000000in}}{%
\pgfpathmoveto{\pgfqpoint{-0.000000in}{0.000000in}}%
\pgfpathlineto{\pgfqpoint{-0.027778in}{0.000000in}}%
\pgfusepath{stroke,fill}%
}%
\begin{pgfscope}%
\pgfsys@transformshift{0.588387in}{2.602120in}%
\pgfsys@useobject{currentmarker}{}%
\end{pgfscope}%
\end{pgfscope}%
\begin{pgfscope}%
\pgfsetbuttcap%
\pgfsetroundjoin%
\definecolor{currentfill}{rgb}{0.000000,0.000000,0.000000}%
\pgfsetfillcolor{currentfill}%
\pgfsetlinewidth{0.602250pt}%
\definecolor{currentstroke}{rgb}{0.000000,0.000000,0.000000}%
\pgfsetstrokecolor{currentstroke}%
\pgfsetdash{}{0pt}%
\pgfsys@defobject{currentmarker}{\pgfqpoint{-0.027778in}{0.000000in}}{\pgfqpoint{-0.000000in}{0.000000in}}{%
\pgfpathmoveto{\pgfqpoint{-0.000000in}{0.000000in}}%
\pgfpathlineto{\pgfqpoint{-0.027778in}{0.000000in}}%
\pgfusepath{stroke,fill}%
}%
\begin{pgfscope}%
\pgfsys@transformshift{0.588387in}{2.667879in}%
\pgfsys@useobject{currentmarker}{}%
\end{pgfscope}%
\end{pgfscope}%
\begin{pgfscope}%
\pgfsetbuttcap%
\pgfsetroundjoin%
\definecolor{currentfill}{rgb}{0.000000,0.000000,0.000000}%
\pgfsetfillcolor{currentfill}%
\pgfsetlinewidth{0.602250pt}%
\definecolor{currentstroke}{rgb}{0.000000,0.000000,0.000000}%
\pgfsetstrokecolor{currentstroke}%
\pgfsetdash{}{0pt}%
\pgfsys@defobject{currentmarker}{\pgfqpoint{-0.027778in}{0.000000in}}{\pgfqpoint{-0.000000in}{0.000000in}}{%
\pgfpathmoveto{\pgfqpoint{-0.000000in}{0.000000in}}%
\pgfpathlineto{\pgfqpoint{-0.027778in}{0.000000in}}%
\pgfusepath{stroke,fill}%
}%
\begin{pgfscope}%
\pgfsys@transformshift{0.588387in}{2.714536in}%
\pgfsys@useobject{currentmarker}{}%
\end{pgfscope}%
\end{pgfscope}%
\begin{pgfscope}%
\definecolor{textcolor}{rgb}{0.000000,0.000000,0.000000}%
\pgfsetstrokecolor{textcolor}%
\pgfsetfillcolor{textcolor}%
\pgftext[x=0.234413in,y=1.631726in,,bottom,rotate=90.000000]{\color{textcolor}{\rmfamily\fontsize{10.000000}{12.000000}\selectfont\catcode`\^=\active\def^{\ifmmode\sp\else\^{}\fi}\catcode`\%=\active\def%{\%}Checks [call]}}%
\end{pgfscope}%
\begin{pgfscope}%
\pgfpathrectangle{\pgfqpoint{0.588387in}{0.521603in}}{\pgfqpoint{3.660036in}{2.220246in}}%
\pgfusepath{clip}%
\pgfsetrectcap%
\pgfsetroundjoin%
\pgfsetlinewidth{1.505625pt}%
\pgfsetstrokecolor{currentstroke1}%
\pgfsetdash{}{0pt}%
\pgfpathmoveto{\pgfqpoint{0.754752in}{0.734939in}}%
\pgfpathlineto{\pgfqpoint{0.787056in}{0.741060in}}%
\pgfpathlineto{\pgfqpoint{0.819360in}{0.813084in}}%
\pgfpathlineto{\pgfqpoint{0.851664in}{0.860336in}}%
\pgfpathlineto{\pgfqpoint{0.883968in}{0.897067in}}%
\pgfpathlineto{\pgfqpoint{0.916272in}{0.970959in}}%
\pgfpathlineto{\pgfqpoint{0.948576in}{1.066956in}}%
\pgfpathlineto{\pgfqpoint{0.980880in}{1.086795in}}%
\pgfpathlineto{\pgfqpoint{1.013184in}{1.147923in}}%
\pgfpathlineto{\pgfqpoint{1.045488in}{1.179195in}}%
\pgfpathlineto{\pgfqpoint{1.077792in}{1.190409in}}%
\pgfpathlineto{\pgfqpoint{1.110096in}{1.203466in}}%
\pgfpathlineto{\pgfqpoint{1.142400in}{1.211456in}}%
\pgfpathlineto{\pgfqpoint{1.174704in}{1.259971in}}%
\pgfpathlineto{\pgfqpoint{1.207008in}{1.226156in}}%
\pgfpathlineto{\pgfqpoint{1.239311in}{1.248639in}}%
\pgfpathlineto{\pgfqpoint{1.271615in}{1.271487in}}%
\pgfpathlineto{\pgfqpoint{1.303919in}{1.301771in}}%
\pgfpathlineto{\pgfqpoint{1.336223in}{1.262396in}}%
\pgfpathlineto{\pgfqpoint{1.368527in}{1.281051in}}%
\pgfpathlineto{\pgfqpoint{1.400831in}{1.297775in}}%
\pgfpathlineto{\pgfqpoint{1.433135in}{1.319671in}}%
\pgfpathlineto{\pgfqpoint{1.465439in}{1.301854in}}%
\pgfpathlineto{\pgfqpoint{1.497743in}{1.319897in}}%
\pgfpathlineto{\pgfqpoint{1.530047in}{1.324754in}}%
\pgfpathlineto{\pgfqpoint{1.562351in}{1.335357in}}%
\pgfpathlineto{\pgfqpoint{1.594655in}{1.324207in}}%
\pgfpathlineto{\pgfqpoint{1.626959in}{1.317688in}}%
\pgfpathlineto{\pgfqpoint{1.659263in}{1.338606in}}%
\pgfpathlineto{\pgfqpoint{1.691567in}{1.359192in}}%
\pgfpathlineto{\pgfqpoint{1.723870in}{1.334803in}}%
\pgfpathlineto{\pgfqpoint{1.756174in}{1.357701in}}%
\pgfpathlineto{\pgfqpoint{1.788478in}{1.360864in}}%
\pgfpathlineto{\pgfqpoint{1.820782in}{1.371349in}}%
\pgfpathlineto{\pgfqpoint{1.853086in}{1.355453in}}%
\pgfpathlineto{\pgfqpoint{1.885390in}{1.360104in}}%
\pgfpathlineto{\pgfqpoint{1.917694in}{1.371331in}}%
\pgfpathlineto{\pgfqpoint{1.949998in}{1.392130in}}%
\pgfpathlineto{\pgfqpoint{1.982302in}{1.363618in}}%
\pgfpathlineto{\pgfqpoint{2.014606in}{1.397175in}}%
\pgfpathlineto{\pgfqpoint{2.046910in}{1.382830in}}%
\pgfpathlineto{\pgfqpoint{2.079214in}{1.397671in}}%
\pgfpathlineto{\pgfqpoint{2.111518in}{1.395205in}}%
\pgfpathlineto{\pgfqpoint{2.143822in}{1.396404in}}%
\pgfpathlineto{\pgfqpoint{2.176125in}{1.399795in}}%
\pgfpathlineto{\pgfqpoint{2.208429in}{1.421708in}}%
\pgfpathlineto{\pgfqpoint{2.240733in}{1.400445in}}%
\pgfpathlineto{\pgfqpoint{2.273037in}{1.397911in}}%
\pgfpathlineto{\pgfqpoint{2.305341in}{1.410514in}}%
\pgfpathlineto{\pgfqpoint{2.337645in}{1.433726in}}%
\pgfpathlineto{\pgfqpoint{2.369949in}{1.411857in}}%
\pgfpathlineto{\pgfqpoint{2.402253in}{1.421767in}}%
\pgfpathlineto{\pgfqpoint{2.434557in}{1.442138in}}%
\pgfpathlineto{\pgfqpoint{2.466861in}{1.447854in}}%
\pgfpathlineto{\pgfqpoint{2.499165in}{1.429936in}}%
\pgfpathlineto{\pgfqpoint{2.531469in}{1.439185in}}%
\pgfpathlineto{\pgfqpoint{2.563773in}{1.435994in}}%
\pgfpathlineto{\pgfqpoint{2.596077in}{1.446457in}}%
\pgfpathlineto{\pgfqpoint{2.628381in}{1.456168in}}%
\pgfpathlineto{\pgfqpoint{2.660684in}{1.434699in}}%
\pgfpathlineto{\pgfqpoint{2.692988in}{1.456784in}}%
\pgfpathlineto{\pgfqpoint{2.725292in}{1.459691in}}%
\pgfpathlineto{\pgfqpoint{2.757596in}{1.446128in}}%
\pgfpathlineto{\pgfqpoint{2.789900in}{1.461631in}}%
\pgfpathlineto{\pgfqpoint{2.822204in}{1.452566in}}%
\pgfpathlineto{\pgfqpoint{2.854508in}{1.462341in}}%
\pgfpathlineto{\pgfqpoint{2.886812in}{1.483023in}}%
\pgfpathlineto{\pgfqpoint{2.919116in}{1.469963in}}%
\pgfpathlineto{\pgfqpoint{2.951420in}{1.455136in}}%
\pgfpathlineto{\pgfqpoint{2.983724in}{1.507505in}}%
\pgfpathlineto{\pgfqpoint{3.016028in}{1.464122in}}%
\pgfpathlineto{\pgfqpoint{3.048332in}{1.470854in}}%
\pgfpathlineto{\pgfqpoint{3.080636in}{1.466293in}}%
\pgfpathlineto{\pgfqpoint{3.112940in}{1.493603in}}%
\pgfpathlineto{\pgfqpoint{3.177547in}{1.482050in}}%
\pgfpathlineto{\pgfqpoint{3.209851in}{1.480345in}}%
\pgfpathlineto{\pgfqpoint{3.242155in}{1.499768in}}%
\pgfpathlineto{\pgfqpoint{3.306763in}{1.492647in}}%
\pgfpathlineto{\pgfqpoint{3.339067in}{1.493541in}}%
\pgfpathlineto{\pgfqpoint{3.371371in}{1.505369in}}%
\pgfpathlineto{\pgfqpoint{3.403675in}{1.511042in}}%
\pgfpathlineto{\pgfqpoint{3.435979in}{1.500414in}}%
\pgfpathlineto{\pgfqpoint{3.468283in}{1.533459in}}%
\pgfpathlineto{\pgfqpoint{3.500587in}{1.507966in}}%
\pgfpathlineto{\pgfqpoint{3.565195in}{1.514061in}}%
\pgfpathlineto{\pgfqpoint{3.597498in}{1.503949in}}%
\pgfpathlineto{\pgfqpoint{3.629802in}{1.519936in}}%
\pgfpathlineto{\pgfqpoint{3.662106in}{1.530483in}}%
\pgfpathlineto{\pgfqpoint{3.694410in}{1.499466in}}%
\pgfpathlineto{\pgfqpoint{3.726714in}{1.537535in}}%
\pgfpathlineto{\pgfqpoint{3.759018in}{1.523110in}}%
\pgfpathlineto{\pgfqpoint{3.823626in}{1.525088in}}%
\pgfpathlineto{\pgfqpoint{3.855930in}{1.535218in}}%
\pgfpathlineto{\pgfqpoint{3.888234in}{1.558796in}}%
\pgfpathlineto{\pgfqpoint{3.952842in}{1.530483in}}%
\pgfpathlineto{\pgfqpoint{4.017450in}{1.544845in}}%
\pgfpathlineto{\pgfqpoint{4.049754in}{1.510703in}}%
\pgfpathlineto{\pgfqpoint{4.082057in}{1.548916in}}%
\pgfusepath{stroke}%
\end{pgfscope}%
\begin{pgfscope}%
\pgfpathrectangle{\pgfqpoint{0.588387in}{0.521603in}}{\pgfqpoint{3.660036in}{2.220246in}}%
\pgfusepath{clip}%
\pgfsetrectcap%
\pgfsetroundjoin%
\pgfsetlinewidth{1.505625pt}%
\pgfsetstrokecolor{currentstroke2}%
\pgfsetdash{}{0pt}%
\pgfpathmoveto{\pgfqpoint{0.754752in}{0.734939in}}%
\pgfpathlineto{\pgfqpoint{0.787056in}{0.741060in}}%
\pgfpathlineto{\pgfqpoint{0.819360in}{0.808243in}}%
\pgfpathlineto{\pgfqpoint{0.851664in}{0.855735in}}%
\pgfpathlineto{\pgfqpoint{0.883968in}{0.897067in}}%
\pgfpathlineto{\pgfqpoint{0.916272in}{0.970959in}}%
\pgfpathlineto{\pgfqpoint{0.948576in}{1.066956in}}%
\pgfpathlineto{\pgfqpoint{0.980880in}{1.086795in}}%
\pgfpathlineto{\pgfqpoint{1.013184in}{1.147923in}}%
\pgfpathlineto{\pgfqpoint{1.045488in}{1.179195in}}%
\pgfpathlineto{\pgfqpoint{1.077792in}{1.317489in}}%
\pgfpathlineto{\pgfqpoint{1.110096in}{1.299925in}}%
\pgfpathlineto{\pgfqpoint{1.142400in}{1.313074in}}%
\pgfpathlineto{\pgfqpoint{1.174704in}{1.398269in}}%
\pgfpathlineto{\pgfqpoint{1.207008in}{1.523953in}}%
\pgfpathlineto{\pgfqpoint{1.239311in}{1.500394in}}%
\pgfpathlineto{\pgfqpoint{1.271615in}{1.619696in}}%
\pgfpathlineto{\pgfqpoint{1.303919in}{1.771848in}}%
\pgfpathlineto{\pgfqpoint{1.336223in}{1.854985in}}%
\pgfpathlineto{\pgfqpoint{1.368527in}{1.763753in}}%
\pgfpathlineto{\pgfqpoint{1.400831in}{1.766076in}}%
\pgfpathlineto{\pgfqpoint{1.433135in}{1.905646in}}%
\pgfpathlineto{\pgfqpoint{1.465439in}{1.748012in}}%
\pgfpathlineto{\pgfqpoint{1.497743in}{1.929434in}}%
\pgfpathlineto{\pgfqpoint{1.530047in}{1.913596in}}%
\pgfpathlineto{\pgfqpoint{1.562351in}{1.808943in}}%
\pgfpathlineto{\pgfqpoint{1.594655in}{2.087815in}}%
\pgfpathlineto{\pgfqpoint{1.626959in}{2.042391in}}%
\pgfpathlineto{\pgfqpoint{1.659263in}{1.912020in}}%
\pgfpathlineto{\pgfqpoint{1.691567in}{2.031089in}}%
\pgfpathlineto{\pgfqpoint{1.723870in}{1.853634in}}%
\pgfpathlineto{\pgfqpoint{1.756174in}{1.943016in}}%
\pgfpathlineto{\pgfqpoint{1.788478in}{1.993436in}}%
\pgfpathlineto{\pgfqpoint{1.820782in}{1.970594in}}%
\pgfpathlineto{\pgfqpoint{1.853086in}{2.080293in}}%
\pgfpathlineto{\pgfqpoint{1.885390in}{1.844661in}}%
\pgfpathlineto{\pgfqpoint{1.917694in}{2.047640in}}%
\pgfpathlineto{\pgfqpoint{1.949998in}{2.042989in}}%
\pgfpathlineto{\pgfqpoint{1.982302in}{1.703430in}}%
\pgfpathlineto{\pgfqpoint{2.014606in}{2.008057in}}%
\pgfpathlineto{\pgfqpoint{2.046910in}{1.488597in}}%
\pgfpathlineto{\pgfqpoint{2.079214in}{2.089075in}}%
\pgfpathlineto{\pgfqpoint{2.111518in}{2.001389in}}%
\pgfpathlineto{\pgfqpoint{2.143822in}{1.951099in}}%
\pgfpathlineto{\pgfqpoint{2.176125in}{1.957938in}}%
\pgfpathlineto{\pgfqpoint{2.208429in}{2.127265in}}%
\pgfpathlineto{\pgfqpoint{2.240733in}{1.490605in}}%
\pgfpathlineto{\pgfqpoint{2.273037in}{1.719413in}}%
\pgfpathlineto{\pgfqpoint{2.305341in}{1.745816in}}%
\pgfpathlineto{\pgfqpoint{2.337645in}{1.846100in}}%
\pgfpathlineto{\pgfqpoint{2.369949in}{1.558427in}}%
\pgfpathlineto{\pgfqpoint{2.402253in}{1.943259in}}%
\pgfpathlineto{\pgfqpoint{2.434557in}{1.786154in}}%
\pgfpathlineto{\pgfqpoint{2.466861in}{2.017991in}}%
\pgfpathlineto{\pgfqpoint{2.499165in}{1.712271in}}%
\pgfpathlineto{\pgfqpoint{2.531469in}{1.939267in}}%
\pgfpathlineto{\pgfqpoint{2.563773in}{1.664692in}}%
\pgfpathlineto{\pgfqpoint{2.596077in}{1.905610in}}%
\pgfpathlineto{\pgfqpoint{2.628381in}{1.652963in}}%
\pgfpathlineto{\pgfqpoint{2.660684in}{1.999616in}}%
\pgfpathlineto{\pgfqpoint{2.692988in}{2.177011in}}%
\pgfpathlineto{\pgfqpoint{2.725292in}{1.663625in}}%
\pgfpathlineto{\pgfqpoint{2.757596in}{1.541231in}}%
\pgfpathlineto{\pgfqpoint{2.789900in}{2.006511in}}%
\pgfpathlineto{\pgfqpoint{2.822204in}{1.718936in}}%
\pgfpathlineto{\pgfqpoint{2.854508in}{1.856117in}}%
\pgfpathlineto{\pgfqpoint{2.886812in}{1.468289in}}%
\pgfpathlineto{\pgfqpoint{2.919116in}{1.985888in}}%
\pgfpathlineto{\pgfqpoint{2.951420in}{1.648216in}}%
\pgfpathlineto{\pgfqpoint{2.983724in}{1.946712in}}%
\pgfpathlineto{\pgfqpoint{3.016028in}{1.550251in}}%
\pgfpathlineto{\pgfqpoint{3.048332in}{1.707720in}}%
\pgfpathlineto{\pgfqpoint{3.080636in}{1.853125in}}%
\pgfpathlineto{\pgfqpoint{3.112940in}{1.699811in}}%
\pgfpathlineto{\pgfqpoint{3.177547in}{1.781832in}}%
\pgfpathlineto{\pgfqpoint{3.209851in}{1.742710in}}%
\pgfpathlineto{\pgfqpoint{3.242155in}{1.824889in}}%
\pgfpathlineto{\pgfqpoint{3.306763in}{2.080289in}}%
\pgfpathlineto{\pgfqpoint{3.339067in}{1.568821in}}%
\pgfpathlineto{\pgfqpoint{3.371371in}{2.044766in}}%
\pgfpathlineto{\pgfqpoint{3.403675in}{1.884884in}}%
\pgfpathlineto{\pgfqpoint{3.435979in}{2.199078in}}%
\pgfpathlineto{\pgfqpoint{3.500587in}{1.847383in}}%
\pgfpathlineto{\pgfqpoint{3.565195in}{1.845586in}}%
\pgfpathlineto{\pgfqpoint{3.597498in}{1.503772in}}%
\pgfpathlineto{\pgfqpoint{3.629802in}{1.589426in}}%
\pgfpathlineto{\pgfqpoint{3.662106in}{1.571175in}}%
\pgfpathlineto{\pgfqpoint{3.694410in}{1.765264in}}%
\pgfpathlineto{\pgfqpoint{3.726714in}{1.583327in}}%
\pgfpathlineto{\pgfqpoint{3.759018in}{1.923444in}}%
\pgfpathlineto{\pgfqpoint{3.823626in}{1.775082in}}%
\pgfpathlineto{\pgfqpoint{3.855930in}{1.514725in}}%
\pgfpathlineto{\pgfqpoint{3.888234in}{1.646536in}}%
\pgfpathlineto{\pgfqpoint{3.952842in}{1.864953in}}%
\pgfpathlineto{\pgfqpoint{4.017450in}{1.644690in}}%
\pgfusepath{stroke}%
\end{pgfscope}%
\begin{pgfscope}%
\pgfpathrectangle{\pgfqpoint{0.588387in}{0.521603in}}{\pgfqpoint{3.660036in}{2.220246in}}%
\pgfusepath{clip}%
\pgfsetrectcap%
\pgfsetroundjoin%
\pgfsetlinewidth{1.505625pt}%
\pgfsetstrokecolor{currentstroke3}%
\pgfsetdash{}{0pt}%
\pgfpathmoveto{\pgfqpoint{0.754752in}{0.622524in}}%
\pgfpathlineto{\pgfqpoint{0.787056in}{0.634543in}}%
\pgfpathlineto{\pgfqpoint{0.819360in}{0.739152in}}%
\pgfpathlineto{\pgfqpoint{0.851664in}{0.798330in}}%
\pgfpathlineto{\pgfqpoint{0.883968in}{0.836888in}}%
\pgfpathlineto{\pgfqpoint{0.916272in}{0.934510in}}%
\pgfpathlineto{\pgfqpoint{0.948576in}{0.941779in}}%
\pgfpathlineto{\pgfqpoint{0.980880in}{0.988547in}}%
\pgfpathlineto{\pgfqpoint{1.013184in}{1.063044in}}%
\pgfpathlineto{\pgfqpoint{1.045488in}{1.149900in}}%
\pgfpathlineto{\pgfqpoint{1.077792in}{1.127210in}}%
\pgfpathlineto{\pgfqpoint{1.110096in}{1.195995in}}%
\pgfpathlineto{\pgfqpoint{1.142400in}{1.119528in}}%
\pgfpathlineto{\pgfqpoint{1.174704in}{1.217549in}}%
\pgfpathlineto{\pgfqpoint{1.207008in}{1.225226in}}%
\pgfpathlineto{\pgfqpoint{1.239311in}{1.337711in}}%
\pgfpathlineto{\pgfqpoint{1.271615in}{1.335471in}}%
\pgfpathlineto{\pgfqpoint{1.303919in}{1.440150in}}%
\pgfpathlineto{\pgfqpoint{1.336223in}{1.378169in}}%
\pgfpathlineto{\pgfqpoint{1.368527in}{1.433207in}}%
\pgfpathlineto{\pgfqpoint{1.400831in}{1.385246in}}%
\pgfpathlineto{\pgfqpoint{1.433135in}{1.444782in}}%
\pgfpathlineto{\pgfqpoint{1.465439in}{1.375659in}}%
\pgfpathlineto{\pgfqpoint{1.497743in}{1.566854in}}%
\pgfpathlineto{\pgfqpoint{1.530047in}{1.354101in}}%
\pgfpathlineto{\pgfqpoint{1.562351in}{1.450969in}}%
\pgfpathlineto{\pgfqpoint{1.594655in}{1.315709in}}%
\pgfpathlineto{\pgfqpoint{1.626959in}{1.792070in}}%
\pgfpathlineto{\pgfqpoint{1.659263in}{1.307444in}}%
\pgfpathlineto{\pgfqpoint{1.691567in}{1.449529in}}%
\pgfpathlineto{\pgfqpoint{1.723870in}{1.752588in}}%
\pgfpathlineto{\pgfqpoint{1.756174in}{1.579801in}}%
\pgfpathlineto{\pgfqpoint{1.788478in}{1.373467in}}%
\pgfpathlineto{\pgfqpoint{1.820782in}{1.637943in}}%
\pgfpathlineto{\pgfqpoint{1.853086in}{1.783801in}}%
\pgfpathlineto{\pgfqpoint{1.885390in}{1.390193in}}%
\pgfpathlineto{\pgfqpoint{1.917694in}{1.245791in}}%
\pgfpathlineto{\pgfqpoint{1.949998in}{1.569440in}}%
\pgfpathlineto{\pgfqpoint{1.982302in}{1.569564in}}%
\pgfpathlineto{\pgfqpoint{2.014606in}{1.521708in}}%
\pgfpathlineto{\pgfqpoint{2.046910in}{1.119761in}}%
\pgfpathlineto{\pgfqpoint{2.079214in}{1.262104in}}%
\pgfpathlineto{\pgfqpoint{2.111518in}{1.268710in}}%
\pgfpathlineto{\pgfqpoint{2.143822in}{1.433721in}}%
\pgfpathlineto{\pgfqpoint{2.176125in}{1.324559in}}%
\pgfpathlineto{\pgfqpoint{2.208429in}{1.436232in}}%
\pgfpathlineto{\pgfqpoint{2.240733in}{2.115618in}}%
\pgfpathlineto{\pgfqpoint{2.273037in}{1.101942in}}%
\pgfpathlineto{\pgfqpoint{2.305341in}{1.321499in}}%
\pgfpathlineto{\pgfqpoint{2.337645in}{1.238368in}}%
\pgfpathlineto{\pgfqpoint{2.369949in}{1.172582in}}%
\pgfpathlineto{\pgfqpoint{2.402253in}{1.291565in}}%
\pgfpathlineto{\pgfqpoint{2.434557in}{1.336418in}}%
\pgfpathlineto{\pgfqpoint{2.466861in}{1.460758in}}%
\pgfpathlineto{\pgfqpoint{2.499165in}{1.418367in}}%
\pgfpathlineto{\pgfqpoint{2.531469in}{1.256666in}}%
\pgfpathlineto{\pgfqpoint{2.563773in}{0.988587in}}%
\pgfpathlineto{\pgfqpoint{2.596077in}{1.399493in}}%
\pgfpathlineto{\pgfqpoint{2.628381in}{1.868544in}}%
\pgfpathlineto{\pgfqpoint{2.660684in}{1.412340in}}%
\pgfpathlineto{\pgfqpoint{2.692988in}{0.975228in}}%
\pgfpathlineto{\pgfqpoint{2.725292in}{1.806499in}}%
\pgfpathlineto{\pgfqpoint{2.757596in}{1.837935in}}%
\pgfpathlineto{\pgfqpoint{2.789900in}{1.138843in}}%
\pgfpathlineto{\pgfqpoint{2.822204in}{1.131042in}}%
\pgfpathlineto{\pgfqpoint{2.854508in}{2.640929in}}%
\pgfpathlineto{\pgfqpoint{2.886812in}{1.011417in}}%
\pgfpathlineto{\pgfqpoint{2.919116in}{1.054688in}}%
\pgfpathlineto{\pgfqpoint{2.951420in}{1.396697in}}%
\pgfpathlineto{\pgfqpoint{2.983724in}{1.994392in}}%
\pgfpathlineto{\pgfqpoint{3.016028in}{1.996623in}}%
\pgfpathlineto{\pgfqpoint{3.048332in}{1.449202in}}%
\pgfpathlineto{\pgfqpoint{3.080636in}{1.480386in}}%
\pgfpathlineto{\pgfqpoint{3.112940in}{1.123217in}}%
\pgfpathlineto{\pgfqpoint{3.177547in}{1.440288in}}%
\pgfpathlineto{\pgfqpoint{3.209851in}{1.460525in}}%
\pgfpathlineto{\pgfqpoint{3.242155in}{1.574375in}}%
\pgfpathlineto{\pgfqpoint{3.306763in}{1.078511in}}%
\pgfpathlineto{\pgfqpoint{3.339067in}{1.137944in}}%
\pgfpathlineto{\pgfqpoint{3.371371in}{1.546013in}}%
\pgfpathlineto{\pgfqpoint{3.403675in}{1.201435in}}%
\pgfpathlineto{\pgfqpoint{3.435979in}{1.200200in}}%
\pgfpathlineto{\pgfqpoint{3.468283in}{1.179452in}}%
\pgfpathlineto{\pgfqpoint{3.500587in}{1.305050in}}%
\pgfpathlineto{\pgfqpoint{3.565195in}{1.002766in}}%
\pgfpathlineto{\pgfqpoint{3.597498in}{1.309803in}}%
\pgfpathlineto{\pgfqpoint{3.629802in}{1.769341in}}%
\pgfpathlineto{\pgfqpoint{3.662106in}{1.179452in}}%
\pgfpathlineto{\pgfqpoint{3.694410in}{1.838363in}}%
\pgfpathlineto{\pgfqpoint{3.726714in}{1.331166in}}%
\pgfpathlineto{\pgfqpoint{3.759018in}{2.062693in}}%
\pgfpathlineto{\pgfqpoint{3.823626in}{1.236248in}}%
\pgfpathlineto{\pgfqpoint{3.855930in}{1.061719in}}%
\pgfpathlineto{\pgfqpoint{3.888234in}{1.514227in}}%
\pgfpathlineto{\pgfqpoint{3.952842in}{1.211043in}}%
\pgfpathlineto{\pgfqpoint{4.017450in}{1.061719in}}%
\pgfpathlineto{\pgfqpoint{4.049754in}{1.050529in}}%
\pgfpathlineto{\pgfqpoint{4.082057in}{1.273906in}}%
\pgfusepath{stroke}%
\end{pgfscope}%
\begin{pgfscope}%
\pgfpathrectangle{\pgfqpoint{0.588387in}{0.521603in}}{\pgfqpoint{3.660036in}{2.220246in}}%
\pgfusepath{clip}%
\pgfsetrectcap%
\pgfsetroundjoin%
\pgfsetlinewidth{1.505625pt}%
\pgfsetstrokecolor{currentstroke4}%
\pgfsetdash{}{0pt}%
\pgfpathmoveto{\pgfqpoint{0.754752in}{0.734939in}}%
\pgfpathlineto{\pgfqpoint{0.787056in}{0.734939in}}%
\pgfpathlineto{\pgfqpoint{0.819360in}{0.800698in}}%
\pgfpathlineto{\pgfqpoint{0.851664in}{0.855267in}}%
\pgfpathlineto{\pgfqpoint{0.883968in}{0.916601in}}%
\pgfpathlineto{\pgfqpoint{0.916272in}{0.953872in}}%
\pgfpathlineto{\pgfqpoint{0.948576in}{1.065500in}}%
\pgfpathlineto{\pgfqpoint{0.980880in}{1.075638in}}%
\pgfpathlineto{\pgfqpoint{1.013184in}{1.149867in}}%
\pgfpathlineto{\pgfqpoint{1.045488in}{1.150003in}}%
\pgfpathlineto{\pgfqpoint{1.077792in}{1.160132in}}%
\pgfpathlineto{\pgfqpoint{1.110096in}{1.163553in}}%
\pgfpathlineto{\pgfqpoint{1.142400in}{1.194271in}}%
\pgfpathlineto{\pgfqpoint{1.174704in}{1.249560in}}%
\pgfpathlineto{\pgfqpoint{1.207008in}{1.203949in}}%
\pgfpathlineto{\pgfqpoint{1.239311in}{1.204721in}}%
\pgfpathlineto{\pgfqpoint{1.271615in}{1.235192in}}%
\pgfpathlineto{\pgfqpoint{1.303919in}{1.269173in}}%
\pgfpathlineto{\pgfqpoint{1.336223in}{1.248065in}}%
\pgfpathlineto{\pgfqpoint{1.368527in}{1.250085in}}%
\pgfpathlineto{\pgfqpoint{1.400831in}{1.270957in}}%
\pgfpathlineto{\pgfqpoint{1.433135in}{1.296870in}}%
\pgfpathlineto{\pgfqpoint{1.465439in}{1.274726in}}%
\pgfpathlineto{\pgfqpoint{1.497743in}{1.284364in}}%
\pgfpathlineto{\pgfqpoint{1.530047in}{1.296541in}}%
\pgfpathlineto{\pgfqpoint{1.562351in}{1.323817in}}%
\pgfpathlineto{\pgfqpoint{1.594655in}{1.298735in}}%
\pgfpathlineto{\pgfqpoint{1.626959in}{1.299631in}}%
\pgfpathlineto{\pgfqpoint{1.659263in}{1.316868in}}%
\pgfpathlineto{\pgfqpoint{1.691567in}{1.335064in}}%
\pgfpathlineto{\pgfqpoint{1.723870in}{1.316774in}}%
\pgfpathlineto{\pgfqpoint{1.756174in}{1.320049in}}%
\pgfpathlineto{\pgfqpoint{1.788478in}{1.337458in}}%
\pgfpathlineto{\pgfqpoint{1.820782in}{1.350496in}}%
\pgfpathlineto{\pgfqpoint{1.853086in}{1.337305in}}%
\pgfpathlineto{\pgfqpoint{1.885390in}{1.343558in}}%
\pgfpathlineto{\pgfqpoint{1.917694in}{1.352068in}}%
\pgfpathlineto{\pgfqpoint{1.949998in}{1.368816in}}%
\pgfpathlineto{\pgfqpoint{1.982302in}{1.355621in}}%
\pgfpathlineto{\pgfqpoint{2.014606in}{1.358348in}}%
\pgfpathlineto{\pgfqpoint{2.046910in}{1.369691in}}%
\pgfpathlineto{\pgfqpoint{2.079214in}{1.382887in}}%
\pgfpathlineto{\pgfqpoint{2.111518in}{1.390824in}}%
\pgfpathlineto{\pgfqpoint{2.143822in}{1.365722in}}%
\pgfpathlineto{\pgfqpoint{2.176125in}{1.381993in}}%
\pgfpathlineto{\pgfqpoint{2.208429in}{1.395973in}}%
\pgfpathlineto{\pgfqpoint{2.240733in}{1.385148in}}%
\pgfpathlineto{\pgfqpoint{2.273037in}{1.397001in}}%
\pgfpathlineto{\pgfqpoint{2.305341in}{1.398191in}}%
\pgfpathlineto{\pgfqpoint{2.337645in}{1.405381in}}%
\pgfpathlineto{\pgfqpoint{2.369949in}{1.398191in}}%
\pgfpathlineto{\pgfqpoint{2.402253in}{1.398497in}}%
\pgfpathlineto{\pgfqpoint{2.434557in}{1.404544in}}%
\pgfpathlineto{\pgfqpoint{2.466861in}{1.425420in}}%
\pgfpathlineto{\pgfqpoint{2.499165in}{1.412880in}}%
\pgfpathlineto{\pgfqpoint{2.531469in}{1.411826in}}%
\pgfpathlineto{\pgfqpoint{2.563773in}{1.423708in}}%
\pgfpathlineto{\pgfqpoint{2.596077in}{1.428289in}}%
\pgfpathlineto{\pgfqpoint{2.628381in}{1.425837in}}%
\pgfpathlineto{\pgfqpoint{2.660684in}{1.434299in}}%
\pgfpathlineto{\pgfqpoint{2.692988in}{1.423269in}}%
\pgfpathlineto{\pgfqpoint{2.725292in}{1.437421in}}%
\pgfpathlineto{\pgfqpoint{2.757596in}{1.430215in}}%
\pgfpathlineto{\pgfqpoint{2.789900in}{1.427706in}}%
\pgfpathlineto{\pgfqpoint{2.822204in}{1.437836in}}%
\pgfpathlineto{\pgfqpoint{2.854508in}{1.443132in}}%
\pgfpathlineto{\pgfqpoint{2.886812in}{1.461998in}}%
\pgfpathlineto{\pgfqpoint{2.919116in}{1.440092in}}%
\pgfpathlineto{\pgfqpoint{2.951420in}{1.443582in}}%
\pgfpathlineto{\pgfqpoint{2.983724in}{1.451527in}}%
\pgfpathlineto{\pgfqpoint{3.016028in}{1.451592in}}%
\pgfpathlineto{\pgfqpoint{3.048332in}{1.450416in}}%
\pgfpathlineto{\pgfqpoint{3.080636in}{1.462227in}}%
\pgfpathlineto{\pgfqpoint{3.112940in}{1.466219in}}%
\pgfpathlineto{\pgfqpoint{3.177547in}{1.457240in}}%
\pgfpathlineto{\pgfqpoint{3.209851in}{1.467405in}}%
\pgfpathlineto{\pgfqpoint{3.242155in}{1.475261in}}%
\pgfpathlineto{\pgfqpoint{3.306763in}{1.463573in}}%
\pgfpathlineto{\pgfqpoint{3.339067in}{1.475191in}}%
\pgfpathlineto{\pgfqpoint{3.371371in}{1.480400in}}%
\pgfpathlineto{\pgfqpoint{3.403675in}{1.472637in}}%
\pgfpathlineto{\pgfqpoint{3.435979in}{1.475645in}}%
\pgfpathlineto{\pgfqpoint{3.468283in}{1.477705in}}%
\pgfpathlineto{\pgfqpoint{3.500587in}{1.485816in}}%
\pgfpathlineto{\pgfqpoint{3.565195in}{1.485363in}}%
\pgfpathlineto{\pgfqpoint{3.597498in}{1.485222in}}%
\pgfpathlineto{\pgfqpoint{3.629802in}{1.490933in}}%
\pgfpathlineto{\pgfqpoint{3.662106in}{1.499829in}}%
\pgfpathlineto{\pgfqpoint{3.694410in}{1.487977in}}%
\pgfpathlineto{\pgfqpoint{3.726714in}{1.488562in}}%
\pgfpathlineto{\pgfqpoint{3.759018in}{1.500272in}}%
\pgfpathlineto{\pgfqpoint{3.823626in}{1.493289in}}%
\pgfpathlineto{\pgfqpoint{3.855930in}{1.502349in}}%
\pgfpathlineto{\pgfqpoint{3.888234in}{1.503417in}}%
\pgfpathlineto{\pgfqpoint{3.952842in}{1.504125in}}%
\pgfpathlineto{\pgfqpoint{4.017450in}{1.521213in}}%
\pgfpathlineto{\pgfqpoint{4.049754in}{1.501633in}}%
\pgfpathlineto{\pgfqpoint{4.082057in}{1.512727in}}%
\pgfusepath{stroke}%
\end{pgfscope}%
\begin{pgfscope}%
\pgfpathrectangle{\pgfqpoint{0.588387in}{0.521603in}}{\pgfqpoint{3.660036in}{2.220246in}}%
\pgfusepath{clip}%
\pgfsetrectcap%
\pgfsetroundjoin%
\pgfsetlinewidth{1.505625pt}%
\pgfsetstrokecolor{currentstroke5}%
\pgfsetdash{}{0pt}%
\pgfpathmoveto{\pgfqpoint{0.754752in}{0.734939in}}%
\pgfpathlineto{\pgfqpoint{0.787056in}{0.734939in}}%
\pgfpathlineto{\pgfqpoint{0.819360in}{0.800698in}}%
\pgfpathlineto{\pgfqpoint{0.851664in}{0.850893in}}%
\pgfpathlineto{\pgfqpoint{0.883968in}{0.917849in}}%
\pgfpathlineto{\pgfqpoint{0.916272in}{0.953872in}}%
\pgfpathlineto{\pgfqpoint{0.948576in}{1.065500in}}%
\pgfpathlineto{\pgfqpoint{0.980880in}{1.075638in}}%
\pgfpathlineto{\pgfqpoint{1.013184in}{1.149867in}}%
\pgfpathlineto{\pgfqpoint{1.045488in}{1.150003in}}%
\pgfpathlineto{\pgfqpoint{1.077792in}{1.260251in}}%
\pgfpathlineto{\pgfqpoint{1.110096in}{1.213144in}}%
\pgfpathlineto{\pgfqpoint{1.142400in}{1.249153in}}%
\pgfpathlineto{\pgfqpoint{1.174704in}{1.390923in}}%
\pgfpathlineto{\pgfqpoint{1.207008in}{1.362067in}}%
\pgfpathlineto{\pgfqpoint{1.239311in}{1.321843in}}%
\pgfpathlineto{\pgfqpoint{1.271615in}{1.398021in}}%
\pgfpathlineto{\pgfqpoint{1.303919in}{1.366723in}}%
\pgfpathlineto{\pgfqpoint{1.336223in}{1.482056in}}%
\pgfpathlineto{\pgfqpoint{1.368527in}{1.343968in}}%
\pgfpathlineto{\pgfqpoint{1.400831in}{1.325668in}}%
\pgfpathlineto{\pgfqpoint{1.433135in}{1.649692in}}%
\pgfpathlineto{\pgfqpoint{1.465439in}{1.866701in}}%
\pgfpathlineto{\pgfqpoint{1.497743in}{1.639783in}}%
\pgfpathlineto{\pgfqpoint{1.530047in}{1.392107in}}%
\pgfpathlineto{\pgfqpoint{1.562351in}{1.738310in}}%
\pgfpathlineto{\pgfqpoint{1.594655in}{2.045317in}}%
\pgfpathlineto{\pgfqpoint{1.626959in}{1.625129in}}%
\pgfpathlineto{\pgfqpoint{1.659263in}{1.402640in}}%
\pgfpathlineto{\pgfqpoint{1.691567in}{1.664862in}}%
\pgfpathlineto{\pgfqpoint{1.723870in}{1.916262in}}%
\pgfpathlineto{\pgfqpoint{1.756174in}{2.027954in}}%
\pgfpathlineto{\pgfqpoint{1.788478in}{1.687891in}}%
\pgfpathlineto{\pgfqpoint{1.820782in}{1.872496in}}%
\pgfpathlineto{\pgfqpoint{1.853086in}{1.623814in}}%
\pgfpathlineto{\pgfqpoint{1.885390in}{1.656028in}}%
\pgfpathlineto{\pgfqpoint{1.917694in}{1.572045in}}%
\pgfpathlineto{\pgfqpoint{1.949998in}{1.673787in}}%
\pgfpathlineto{\pgfqpoint{1.982302in}{1.502655in}}%
\pgfpathlineto{\pgfqpoint{2.014606in}{1.595870in}}%
\pgfpathlineto{\pgfqpoint{2.046910in}{1.561862in}}%
\pgfpathlineto{\pgfqpoint{2.079214in}{1.740114in}}%
\pgfpathlineto{\pgfqpoint{2.111518in}{1.512223in}}%
\pgfpathlineto{\pgfqpoint{2.143822in}{1.929479in}}%
\pgfpathlineto{\pgfqpoint{2.176125in}{1.452192in}}%
\pgfpathlineto{\pgfqpoint{2.208429in}{1.679071in}}%
\pgfpathlineto{\pgfqpoint{2.240733in}{1.721895in}}%
\pgfpathlineto{\pgfqpoint{2.273037in}{2.015478in}}%
\pgfpathlineto{\pgfqpoint{2.305341in}{1.603185in}}%
\pgfpathlineto{\pgfqpoint{2.337645in}{1.499893in}}%
\pgfpathlineto{\pgfqpoint{2.369949in}{1.852993in}}%
\pgfpathlineto{\pgfqpoint{2.402253in}{1.871931in}}%
\pgfpathlineto{\pgfqpoint{2.434557in}{1.640111in}}%
\pgfpathlineto{\pgfqpoint{2.466861in}{1.591112in}}%
\pgfpathlineto{\pgfqpoint{2.499165in}{1.404936in}}%
\pgfpathlineto{\pgfqpoint{2.531469in}{1.864588in}}%
\pgfpathlineto{\pgfqpoint{2.563773in}{1.734236in}}%
\pgfpathlineto{\pgfqpoint{2.596077in}{1.579215in}}%
\pgfpathlineto{\pgfqpoint{2.628381in}{1.788120in}}%
\pgfpathlineto{\pgfqpoint{2.660684in}{1.980882in}}%
\pgfpathlineto{\pgfqpoint{2.692988in}{1.703469in}}%
\pgfpathlineto{\pgfqpoint{2.725292in}{1.925936in}}%
\pgfpathlineto{\pgfqpoint{2.757596in}{1.426266in}}%
\pgfpathlineto{\pgfqpoint{2.789900in}{1.771394in}}%
\pgfpathlineto{\pgfqpoint{2.822204in}{1.668925in}}%
\pgfpathlineto{\pgfqpoint{2.854508in}{1.568583in}}%
\pgfpathlineto{\pgfqpoint{2.886812in}{1.502706in}}%
\pgfpathlineto{\pgfqpoint{2.919116in}{1.941682in}}%
\pgfpathlineto{\pgfqpoint{2.951420in}{1.601236in}}%
\pgfpathlineto{\pgfqpoint{2.983724in}{1.687499in}}%
\pgfpathlineto{\pgfqpoint{3.016028in}{1.562864in}}%
\pgfpathlineto{\pgfqpoint{3.048332in}{1.922073in}}%
\pgfpathlineto{\pgfqpoint{3.080636in}{1.516699in}}%
\pgfpathlineto{\pgfqpoint{3.112940in}{1.782349in}}%
\pgfpathlineto{\pgfqpoint{3.177547in}{1.809474in}}%
\pgfpathlineto{\pgfqpoint{3.209851in}{1.522480in}}%
\pgfpathlineto{\pgfqpoint{3.242155in}{1.958600in}}%
\pgfpathlineto{\pgfqpoint{3.306763in}{1.838447in}}%
\pgfpathlineto{\pgfqpoint{3.339067in}{1.470911in}}%
\pgfpathlineto{\pgfqpoint{3.371371in}{1.639210in}}%
\pgfpathlineto{\pgfqpoint{3.403675in}{2.098621in}}%
\pgfpathlineto{\pgfqpoint{3.435979in}{1.718343in}}%
\pgfpathlineto{\pgfqpoint{3.468283in}{1.559048in}}%
\pgfpathlineto{\pgfqpoint{3.500587in}{1.530483in}}%
\pgfpathlineto{\pgfqpoint{3.565195in}{1.714003in}}%
\pgfpathlineto{\pgfqpoint{3.597498in}{1.709705in}}%
\pgfpathlineto{\pgfqpoint{3.629802in}{1.533753in}}%
\pgfpathlineto{\pgfqpoint{3.662106in}{1.483425in}}%
\pgfpathlineto{\pgfqpoint{3.694410in}{1.540762in}}%
\pgfpathlineto{\pgfqpoint{3.726714in}{2.088656in}}%
\pgfpathlineto{\pgfqpoint{3.759018in}{2.021893in}}%
\pgfpathlineto{\pgfqpoint{3.823626in}{2.025449in}}%
\pgfpathlineto{\pgfqpoint{3.855930in}{1.622098in}}%
\pgfpathlineto{\pgfqpoint{3.888234in}{1.624695in}}%
\pgfpathlineto{\pgfqpoint{3.952842in}{1.677688in}}%
\pgfpathlineto{\pgfqpoint{4.017450in}{2.043508in}}%
\pgfpathlineto{\pgfqpoint{4.049754in}{1.662667in}}%
\pgfpathlineto{\pgfqpoint{4.082057in}{1.577139in}}%
\pgfusepath{stroke}%
\end{pgfscope}%
\begin{pgfscope}%
\pgfpathrectangle{\pgfqpoint{0.588387in}{0.521603in}}{\pgfqpoint{3.660036in}{2.220246in}}%
\pgfusepath{clip}%
\pgfsetrectcap%
\pgfsetroundjoin%
\pgfsetlinewidth{1.505625pt}%
\pgfsetstrokecolor{currentstroke6}%
\pgfsetdash{}{0pt}%
\pgfpathmoveto{\pgfqpoint{0.754752in}{0.734939in}}%
\pgfpathlineto{\pgfqpoint{0.787056in}{0.734939in}}%
\pgfpathlineto{\pgfqpoint{0.819360in}{0.800698in}}%
\pgfpathlineto{\pgfqpoint{0.851664in}{0.856367in}}%
\pgfpathlineto{\pgfqpoint{0.883968in}{0.918886in}}%
\pgfpathlineto{\pgfqpoint{0.916272in}{0.938114in}}%
\pgfpathlineto{\pgfqpoint{0.948576in}{1.064522in}}%
\pgfpathlineto{\pgfqpoint{0.980880in}{1.088457in}}%
\pgfpathlineto{\pgfqpoint{1.013184in}{1.154344in}}%
\pgfpathlineto{\pgfqpoint{1.045488in}{1.148971in}}%
\pgfpathlineto{\pgfqpoint{1.077792in}{1.176322in}}%
\pgfpathlineto{\pgfqpoint{1.110096in}{1.193930in}}%
\pgfpathlineto{\pgfqpoint{1.142400in}{1.217177in}}%
\pgfpathlineto{\pgfqpoint{1.174704in}{1.271946in}}%
\pgfpathlineto{\pgfqpoint{1.207008in}{1.211694in}}%
\pgfpathlineto{\pgfqpoint{1.239311in}{1.230598in}}%
\pgfpathlineto{\pgfqpoint{1.271615in}{1.258135in}}%
\pgfpathlineto{\pgfqpoint{1.303919in}{1.280432in}}%
\pgfpathlineto{\pgfqpoint{1.336223in}{1.256073in}}%
\pgfpathlineto{\pgfqpoint{1.368527in}{1.279647in}}%
\pgfpathlineto{\pgfqpoint{1.400831in}{1.280304in}}%
\pgfpathlineto{\pgfqpoint{1.433135in}{1.311416in}}%
\pgfpathlineto{\pgfqpoint{1.465439in}{1.284595in}}%
\pgfpathlineto{\pgfqpoint{1.497743in}{1.305615in}}%
\pgfpathlineto{\pgfqpoint{1.530047in}{1.315272in}}%
\pgfpathlineto{\pgfqpoint{1.562351in}{1.325468in}}%
\pgfpathlineto{\pgfqpoint{1.594655in}{1.316221in}}%
\pgfpathlineto{\pgfqpoint{1.626959in}{1.312290in}}%
\pgfpathlineto{\pgfqpoint{1.659263in}{1.338279in}}%
\pgfpathlineto{\pgfqpoint{1.691567in}{1.364292in}}%
\pgfpathlineto{\pgfqpoint{1.723870in}{1.328144in}}%
\pgfpathlineto{\pgfqpoint{1.756174in}{1.337757in}}%
\pgfpathlineto{\pgfqpoint{1.788478in}{1.350572in}}%
\pgfpathlineto{\pgfqpoint{1.820782in}{1.358129in}}%
\pgfpathlineto{\pgfqpoint{1.853086in}{1.363338in}}%
\pgfpathlineto{\pgfqpoint{1.885390in}{1.378428in}}%
\pgfpathlineto{\pgfqpoint{1.917694in}{1.359706in}}%
\pgfpathlineto{\pgfqpoint{1.949998in}{1.398611in}}%
\pgfpathlineto{\pgfqpoint{1.982302in}{1.370665in}}%
\pgfpathlineto{\pgfqpoint{2.014606in}{1.369000in}}%
\pgfpathlineto{\pgfqpoint{2.046910in}{1.376182in}}%
\pgfpathlineto{\pgfqpoint{2.079214in}{1.401150in}}%
\pgfpathlineto{\pgfqpoint{2.111518in}{1.377309in}}%
\pgfpathlineto{\pgfqpoint{2.143822in}{1.385677in}}%
\pgfpathlineto{\pgfqpoint{2.176125in}{1.392496in}}%
\pgfpathlineto{\pgfqpoint{2.208429in}{1.407458in}}%
\pgfpathlineto{\pgfqpoint{2.240733in}{1.386175in}}%
\pgfpathlineto{\pgfqpoint{2.273037in}{1.399976in}}%
\pgfpathlineto{\pgfqpoint{2.305341in}{1.399736in}}%
\pgfpathlineto{\pgfqpoint{2.337645in}{1.427285in}}%
\pgfpathlineto{\pgfqpoint{2.369949in}{1.399351in}}%
\pgfpathlineto{\pgfqpoint{2.402253in}{1.407570in}}%
\pgfpathlineto{\pgfqpoint{2.434557in}{1.421397in}}%
\pgfpathlineto{\pgfqpoint{2.466861in}{1.450184in}}%
\pgfpathlineto{\pgfqpoint{2.499165in}{1.420452in}}%
\pgfpathlineto{\pgfqpoint{2.531469in}{1.417251in}}%
\pgfpathlineto{\pgfqpoint{2.563773in}{1.426393in}}%
\pgfpathlineto{\pgfqpoint{2.596077in}{1.432787in}}%
\pgfpathlineto{\pgfqpoint{2.628381in}{1.420896in}}%
\pgfpathlineto{\pgfqpoint{2.660684in}{1.429067in}}%
\pgfpathlineto{\pgfqpoint{2.692988in}{1.428984in}}%
\pgfpathlineto{\pgfqpoint{2.725292in}{1.445959in}}%
\pgfpathlineto{\pgfqpoint{2.757596in}{1.427970in}}%
\pgfpathlineto{\pgfqpoint{2.789900in}{1.433852in}}%
\pgfpathlineto{\pgfqpoint{2.822204in}{1.432429in}}%
\pgfpathlineto{\pgfqpoint{2.854508in}{1.464499in}}%
\pgfpathlineto{\pgfqpoint{2.886812in}{1.469168in}}%
\pgfpathlineto{\pgfqpoint{2.919116in}{1.446449in}}%
\pgfpathlineto{\pgfqpoint{2.951420in}{1.445284in}}%
\pgfpathlineto{\pgfqpoint{2.983724in}{1.464303in}}%
\pgfpathlineto{\pgfqpoint{3.016028in}{1.453856in}}%
\pgfpathlineto{\pgfqpoint{3.048332in}{1.454433in}}%
\pgfpathlineto{\pgfqpoint{3.080636in}{1.464499in}}%
\pgfpathlineto{\pgfqpoint{3.112940in}{1.472064in}}%
\pgfpathlineto{\pgfqpoint{3.177547in}{1.465781in}}%
\pgfpathlineto{\pgfqpoint{3.209851in}{1.470738in}}%
\pgfpathlineto{\pgfqpoint{3.242155in}{1.476662in}}%
\pgfpathlineto{\pgfqpoint{3.306763in}{1.471462in}}%
\pgfpathlineto{\pgfqpoint{3.339067in}{1.541512in}}%
\pgfpathlineto{\pgfqpoint{3.371371in}{1.488380in}}%
\pgfpathlineto{\pgfqpoint{3.403675in}{1.499102in}}%
\pgfpathlineto{\pgfqpoint{3.435979in}{1.482061in}}%
\pgfpathlineto{\pgfqpoint{3.468283in}{1.480181in}}%
\pgfpathlineto{\pgfqpoint{3.500587in}{1.493352in}}%
\pgfpathlineto{\pgfqpoint{3.565195in}{1.482851in}}%
\pgfpathlineto{\pgfqpoint{3.597498in}{1.489144in}}%
\pgfpathlineto{\pgfqpoint{3.629802in}{1.503772in}}%
\pgfpathlineto{\pgfqpoint{3.662106in}{1.486605in}}%
\pgfpathlineto{\pgfqpoint{3.694410in}{1.506406in}}%
\pgfpathlineto{\pgfqpoint{3.726714in}{1.492784in}}%
\pgfpathlineto{\pgfqpoint{3.759018in}{1.500914in}}%
\pgfpathlineto{\pgfqpoint{3.823626in}{1.510703in}}%
\pgfpathlineto{\pgfqpoint{3.855930in}{1.498005in}}%
\pgfpathlineto{\pgfqpoint{3.888234in}{1.516862in}}%
\pgfpathlineto{\pgfqpoint{3.952842in}{1.509341in}}%
\pgfpathlineto{\pgfqpoint{4.017450in}{1.532868in}}%
\pgfpathlineto{\pgfqpoint{4.049754in}{1.518001in}}%
\pgfpathlineto{\pgfqpoint{4.082057in}{1.508655in}}%
\pgfusepath{stroke}%
\end{pgfscope}%
\begin{pgfscope}%
\pgfpathrectangle{\pgfqpoint{0.588387in}{0.521603in}}{\pgfqpoint{3.660036in}{2.220246in}}%
\pgfusepath{clip}%
\pgfsetrectcap%
\pgfsetroundjoin%
\pgfsetlinewidth{1.505625pt}%
\pgfsetstrokecolor{currentstroke7}%
\pgfsetdash{}{0pt}%
\pgfpathmoveto{\pgfqpoint{0.754752in}{0.734939in}}%
\pgfpathlineto{\pgfqpoint{0.787056in}{0.734939in}}%
\pgfpathlineto{\pgfqpoint{0.819360in}{0.797972in}}%
\pgfpathlineto{\pgfqpoint{0.851664in}{0.855494in}}%
\pgfpathlineto{\pgfqpoint{0.883968in}{0.918886in}}%
\pgfpathlineto{\pgfqpoint{0.916272in}{0.938114in}}%
\pgfpathlineto{\pgfqpoint{0.948576in}{1.064522in}}%
\pgfpathlineto{\pgfqpoint{0.980880in}{1.088457in}}%
\pgfpathlineto{\pgfqpoint{1.013184in}{1.154344in}}%
\pgfpathlineto{\pgfqpoint{1.045488in}{1.148971in}}%
\pgfpathlineto{\pgfqpoint{1.077792in}{1.274014in}}%
\pgfpathlineto{\pgfqpoint{1.110096in}{1.272088in}}%
\pgfpathlineto{\pgfqpoint{1.142400in}{1.340233in}}%
\pgfpathlineto{\pgfqpoint{1.174704in}{1.402383in}}%
\pgfpathlineto{\pgfqpoint{1.207008in}{1.504080in}}%
\pgfpathlineto{\pgfqpoint{1.239311in}{1.349204in}}%
\pgfpathlineto{\pgfqpoint{1.271615in}{1.474314in}}%
\pgfpathlineto{\pgfqpoint{1.303919in}{1.400337in}}%
\pgfpathlineto{\pgfqpoint{1.336223in}{1.605701in}}%
\pgfpathlineto{\pgfqpoint{1.368527in}{1.378311in}}%
\pgfpathlineto{\pgfqpoint{1.400831in}{1.299365in}}%
\pgfpathlineto{\pgfqpoint{1.433135in}{1.637477in}}%
\pgfpathlineto{\pgfqpoint{1.465439in}{1.821441in}}%
\pgfpathlineto{\pgfqpoint{1.497743in}{1.632918in}}%
\pgfpathlineto{\pgfqpoint{1.530047in}{1.817466in}}%
\pgfpathlineto{\pgfqpoint{1.562351in}{1.814673in}}%
\pgfpathlineto{\pgfqpoint{1.594655in}{1.898525in}}%
\pgfpathlineto{\pgfqpoint{1.626959in}{1.711716in}}%
\pgfpathlineto{\pgfqpoint{1.659263in}{1.758623in}}%
\pgfpathlineto{\pgfqpoint{1.691567in}{2.049367in}}%
\pgfpathlineto{\pgfqpoint{1.723870in}{1.789780in}}%
\pgfpathlineto{\pgfqpoint{1.756174in}{1.970361in}}%
\pgfpathlineto{\pgfqpoint{1.788478in}{2.022192in}}%
\pgfpathlineto{\pgfqpoint{1.820782in}{1.716713in}}%
\pgfpathlineto{\pgfqpoint{1.853086in}{2.116866in}}%
\pgfpathlineto{\pgfqpoint{1.885390in}{1.817645in}}%
\pgfpathlineto{\pgfqpoint{1.917694in}{1.905632in}}%
\pgfpathlineto{\pgfqpoint{1.949998in}{1.751439in}}%
\pgfpathlineto{\pgfqpoint{1.982302in}{2.097272in}}%
\pgfpathlineto{\pgfqpoint{2.014606in}{1.808172in}}%
\pgfpathlineto{\pgfqpoint{2.046910in}{1.897531in}}%
\pgfpathlineto{\pgfqpoint{2.079214in}{1.632461in}}%
\pgfpathlineto{\pgfqpoint{2.111518in}{1.795068in}}%
\pgfpathlineto{\pgfqpoint{2.143822in}{1.700926in}}%
\pgfpathlineto{\pgfqpoint{2.176125in}{1.580534in}}%
\pgfpathlineto{\pgfqpoint{2.208429in}{1.803911in}}%
\pgfpathlineto{\pgfqpoint{2.240733in}{2.213446in}}%
\pgfpathlineto{\pgfqpoint{2.273037in}{1.843524in}}%
\pgfpathlineto{\pgfqpoint{2.305341in}{1.514156in}}%
\pgfpathlineto{\pgfqpoint{2.337645in}{1.541016in}}%
\pgfpathlineto{\pgfqpoint{2.369949in}{1.920331in}}%
\pgfpathlineto{\pgfqpoint{2.402253in}{1.951612in}}%
\pgfpathlineto{\pgfqpoint{2.434557in}{1.837121in}}%
\pgfpathlineto{\pgfqpoint{2.466861in}{1.684364in}}%
\pgfpathlineto{\pgfqpoint{2.499165in}{1.543603in}}%
\pgfpathlineto{\pgfqpoint{2.531469in}{1.788870in}}%
\pgfpathlineto{\pgfqpoint{2.563773in}{2.012505in}}%
\pgfpathlineto{\pgfqpoint{2.596077in}{1.936794in}}%
\pgfpathlineto{\pgfqpoint{2.628381in}{1.516137in}}%
\pgfpathlineto{\pgfqpoint{2.660684in}{1.754987in}}%
\pgfpathlineto{\pgfqpoint{2.692988in}{2.071877in}}%
\pgfpathlineto{\pgfqpoint{2.725292in}{1.661993in}}%
\pgfpathlineto{\pgfqpoint{2.757596in}{1.422803in}}%
\pgfpathlineto{\pgfqpoint{2.789900in}{1.928619in}}%
\pgfpathlineto{\pgfqpoint{2.822204in}{1.839103in}}%
\pgfpathlineto{\pgfqpoint{2.854508in}{1.737186in}}%
\pgfpathlineto{\pgfqpoint{2.886812in}{1.491262in}}%
\pgfpathlineto{\pgfqpoint{2.919116in}{1.992393in}}%
\pgfpathlineto{\pgfqpoint{2.951420in}{2.035410in}}%
\pgfpathlineto{\pgfqpoint{2.983724in}{1.883114in}}%
\pgfpathlineto{\pgfqpoint{3.016028in}{1.994174in}}%
\pgfpathlineto{\pgfqpoint{3.048332in}{2.003858in}}%
\pgfpathlineto{\pgfqpoint{3.080636in}{2.136205in}}%
\pgfpathlineto{\pgfqpoint{3.112940in}{1.716665in}}%
\pgfpathlineto{\pgfqpoint{3.177547in}{1.916911in}}%
\pgfpathlineto{\pgfqpoint{3.209851in}{1.564860in}}%
\pgfpathlineto{\pgfqpoint{3.242155in}{1.671736in}}%
\pgfpathlineto{\pgfqpoint{3.306763in}{1.519454in}}%
\pgfpathlineto{\pgfqpoint{3.339067in}{1.470042in}}%
\pgfpathlineto{\pgfqpoint{3.371371in}{1.803703in}}%
\pgfpathlineto{\pgfqpoint{3.403675in}{2.339817in}}%
\pgfpathlineto{\pgfqpoint{3.435979in}{1.649663in}}%
\pgfpathlineto{\pgfqpoint{3.468283in}{1.615121in}}%
\pgfpathlineto{\pgfqpoint{3.500587in}{1.695797in}}%
\pgfpathlineto{\pgfqpoint{3.565195in}{1.630833in}}%
\pgfpathlineto{\pgfqpoint{3.597498in}{1.933087in}}%
\pgfpathlineto{\pgfqpoint{3.629802in}{1.977239in}}%
\pgfpathlineto{\pgfqpoint{3.662106in}{2.044256in}}%
\pgfpathlineto{\pgfqpoint{3.694410in}{1.971183in}}%
\pgfpathlineto{\pgfqpoint{3.726714in}{1.621070in}}%
\pgfpathlineto{\pgfqpoint{3.759018in}{2.058796in}}%
\pgfpathlineto{\pgfqpoint{3.823626in}{2.044546in}}%
\pgfpathlineto{\pgfqpoint{3.855930in}{1.794361in}}%
\pgfpathlineto{\pgfqpoint{3.888234in}{1.858002in}}%
\pgfpathlineto{\pgfqpoint{3.952842in}{1.657054in}}%
\pgfpathlineto{\pgfqpoint{4.017450in}{1.630092in}}%
\pgfpathlineto{\pgfqpoint{4.049754in}{1.810526in}}%
\pgfpathlineto{\pgfqpoint{4.082057in}{1.577139in}}%
\pgfusepath{stroke}%
\end{pgfscope}%
\begin{pgfscope}%
\pgfpathrectangle{\pgfqpoint{0.588387in}{0.521603in}}{\pgfqpoint{3.660036in}{2.220246in}}%
\pgfusepath{clip}%
\pgfsetrectcap%
\pgfsetroundjoin%
\pgfsetlinewidth{1.505625pt}%
\definecolor{currentstroke}{rgb}{0.498039,0.498039,0.498039}%
\pgfsetstrokecolor{currentstroke}%
\pgfsetdash{}{0pt}%
\pgfpathmoveto{\pgfqpoint{0.754752in}{0.734939in}}%
\pgfpathlineto{\pgfqpoint{0.787056in}{0.741060in}}%
\pgfpathlineto{\pgfqpoint{0.819360in}{0.806016in}}%
\pgfpathlineto{\pgfqpoint{0.851664in}{0.838518in}}%
\pgfpathlineto{\pgfqpoint{0.883968in}{0.892672in}}%
\pgfpathlineto{\pgfqpoint{0.916272in}{0.948762in}}%
\pgfpathlineto{\pgfqpoint{0.948576in}{1.080098in}}%
\pgfpathlineto{\pgfqpoint{0.980880in}{1.144246in}}%
\pgfpathlineto{\pgfqpoint{1.013184in}{1.177183in}}%
\pgfpathlineto{\pgfqpoint{1.045488in}{1.214902in}}%
\pgfpathlineto{\pgfqpoint{1.077792in}{1.203202in}}%
\pgfpathlineto{\pgfqpoint{1.110096in}{1.262188in}}%
\pgfpathlineto{\pgfqpoint{1.142400in}{1.244530in}}%
\pgfpathlineto{\pgfqpoint{1.174704in}{1.300533in}}%
\pgfpathlineto{\pgfqpoint{1.207008in}{1.239762in}}%
\pgfpathlineto{\pgfqpoint{1.239311in}{1.271729in}}%
\pgfpathlineto{\pgfqpoint{1.271615in}{1.285241in}}%
\pgfpathlineto{\pgfqpoint{1.303919in}{1.327816in}}%
\pgfpathlineto{\pgfqpoint{1.336223in}{1.274076in}}%
\pgfpathlineto{\pgfqpoint{1.368527in}{1.301721in}}%
\pgfpathlineto{\pgfqpoint{1.400831in}{1.312628in}}%
\pgfpathlineto{\pgfqpoint{1.433135in}{1.366909in}}%
\pgfpathlineto{\pgfqpoint{1.465439in}{1.316961in}}%
\pgfpathlineto{\pgfqpoint{1.497743in}{1.334613in}}%
\pgfpathlineto{\pgfqpoint{1.530047in}{1.333459in}}%
\pgfpathlineto{\pgfqpoint{1.562351in}{1.365856in}}%
\pgfpathlineto{\pgfqpoint{1.594655in}{1.338732in}}%
\pgfpathlineto{\pgfqpoint{1.626959in}{1.341274in}}%
\pgfpathlineto{\pgfqpoint{1.659263in}{1.354471in}}%
\pgfpathlineto{\pgfqpoint{1.691567in}{1.377066in}}%
\pgfpathlineto{\pgfqpoint{1.723870in}{1.355579in}}%
\pgfpathlineto{\pgfqpoint{1.756174in}{1.377079in}}%
\pgfpathlineto{\pgfqpoint{1.788478in}{1.382689in}}%
\pgfpathlineto{\pgfqpoint{1.820782in}{1.392036in}}%
\pgfpathlineto{\pgfqpoint{1.853086in}{1.386112in}}%
\pgfpathlineto{\pgfqpoint{1.885390in}{1.376922in}}%
\pgfpathlineto{\pgfqpoint{1.917694in}{1.382078in}}%
\pgfpathlineto{\pgfqpoint{1.949998in}{1.412776in}}%
\pgfpathlineto{\pgfqpoint{1.982302in}{1.402970in}}%
\pgfpathlineto{\pgfqpoint{2.014606in}{1.403740in}}%
\pgfpathlineto{\pgfqpoint{2.046910in}{1.419156in}}%
\pgfpathlineto{\pgfqpoint{2.079214in}{1.435481in}}%
\pgfpathlineto{\pgfqpoint{2.111518in}{1.416407in}}%
\pgfpathlineto{\pgfqpoint{2.143822in}{1.415793in}}%
\pgfpathlineto{\pgfqpoint{2.176125in}{1.438448in}}%
\pgfpathlineto{\pgfqpoint{2.208429in}{1.452879in}}%
\pgfpathlineto{\pgfqpoint{2.240733in}{1.424544in}}%
\pgfpathlineto{\pgfqpoint{2.273037in}{1.429003in}}%
\pgfpathlineto{\pgfqpoint{2.305341in}{1.424791in}}%
\pgfpathlineto{\pgfqpoint{2.337645in}{1.447003in}}%
\pgfpathlineto{\pgfqpoint{2.369949in}{1.424791in}}%
\pgfpathlineto{\pgfqpoint{2.402253in}{1.423206in}}%
\pgfpathlineto{\pgfqpoint{2.434557in}{1.433634in}}%
\pgfpathlineto{\pgfqpoint{2.466861in}{1.442552in}}%
\pgfpathlineto{\pgfqpoint{2.499165in}{1.425980in}}%
\pgfpathlineto{\pgfqpoint{2.531469in}{1.446340in}}%
\pgfpathlineto{\pgfqpoint{2.563773in}{1.459846in}}%
\pgfpathlineto{\pgfqpoint{2.596077in}{1.477633in}}%
\pgfpathlineto{\pgfqpoint{2.628381in}{1.461194in}}%
\pgfpathlineto{\pgfqpoint{2.660684in}{1.462960in}}%
\pgfpathlineto{\pgfqpoint{2.692988in}{1.465981in}}%
\pgfpathlineto{\pgfqpoint{2.725292in}{1.475097in}}%
\pgfpathlineto{\pgfqpoint{2.757596in}{1.445622in}}%
\pgfpathlineto{\pgfqpoint{2.789900in}{1.474344in}}%
\pgfpathlineto{\pgfqpoint{2.822204in}{1.466961in}}%
\pgfpathlineto{\pgfqpoint{2.854508in}{1.465790in}}%
\pgfpathlineto{\pgfqpoint{2.886812in}{1.460618in}}%
\pgfpathlineto{\pgfqpoint{2.919116in}{1.469566in}}%
\pgfpathlineto{\pgfqpoint{2.951420in}{1.479360in}}%
\pgfpathlineto{\pgfqpoint{2.983724in}{1.491313in}}%
\pgfpathlineto{\pgfqpoint{3.016028in}{1.473208in}}%
\pgfpathlineto{\pgfqpoint{3.048332in}{1.481920in}}%
\pgfpathlineto{\pgfqpoint{3.080636in}{1.508998in}}%
\pgfpathlineto{\pgfqpoint{3.112940in}{1.482688in}}%
\pgfpathlineto{\pgfqpoint{3.177547in}{1.483236in}}%
\pgfpathlineto{\pgfqpoint{3.209851in}{1.476369in}}%
\pgfpathlineto{\pgfqpoint{3.242155in}{1.506232in}}%
\pgfpathlineto{\pgfqpoint{3.306763in}{1.496934in}}%
\pgfpathlineto{\pgfqpoint{3.339067in}{1.527451in}}%
\pgfpathlineto{\pgfqpoint{3.371371in}{1.514548in}}%
\pgfpathlineto{\pgfqpoint{3.403675in}{1.488172in}}%
\pgfpathlineto{\pgfqpoint{3.435979in}{1.504532in}}%
\pgfpathlineto{\pgfqpoint{3.468283in}{1.509341in}}%
\pgfpathlineto{\pgfqpoint{3.500587in}{1.503594in}}%
\pgfpathlineto{\pgfqpoint{3.565195in}{1.554894in}}%
\pgfpathlineto{\pgfqpoint{3.597498in}{1.503417in}}%
\pgfpathlineto{\pgfqpoint{3.629802in}{1.534047in}}%
\pgfpathlineto{\pgfqpoint{3.662106in}{1.489724in}}%
\pgfpathlineto{\pgfqpoint{3.694410in}{1.514559in}}%
\pgfpathlineto{\pgfqpoint{3.726714in}{1.544569in}}%
\pgfpathlineto{\pgfqpoint{3.759018in}{1.526565in}}%
\pgfpathlineto{\pgfqpoint{3.823626in}{1.530783in}}%
\pgfpathlineto{\pgfqpoint{3.855930in}{1.510364in}}%
\pgfpathlineto{\pgfqpoint{3.888234in}{1.542072in}}%
\pgfpathlineto{\pgfqpoint{3.952842in}{1.518864in}}%
\pgfpathlineto{\pgfqpoint{4.017450in}{1.560552in}}%
\pgfpathlineto{\pgfqpoint{4.049754in}{1.556510in}}%
\pgfpathlineto{\pgfqpoint{4.082057in}{1.558037in}}%
\pgfusepath{stroke}%
\end{pgfscope}%
\begin{pgfscope}%
\pgfpathrectangle{\pgfqpoint{0.588387in}{0.521603in}}{\pgfqpoint{3.660036in}{2.220246in}}%
\pgfusepath{clip}%
\pgfsetrectcap%
\pgfsetroundjoin%
\pgfsetlinewidth{1.505625pt}%
\definecolor{currentstroke}{rgb}{0.737255,0.741176,0.133333}%
\pgfsetstrokecolor{currentstroke}%
\pgfsetdash{}{0pt}%
\pgfpathmoveto{\pgfqpoint{0.754752in}{0.734939in}}%
\pgfpathlineto{\pgfqpoint{0.787056in}{0.741060in}}%
\pgfpathlineto{\pgfqpoint{0.819360in}{0.797827in}}%
\pgfpathlineto{\pgfqpoint{0.851664in}{0.841856in}}%
\pgfpathlineto{\pgfqpoint{0.883968in}{0.894258in}}%
\pgfpathlineto{\pgfqpoint{0.916272in}{0.949303in}}%
\pgfpathlineto{\pgfqpoint{0.948576in}{1.080098in}}%
\pgfpathlineto{\pgfqpoint{0.980880in}{1.144246in}}%
\pgfpathlineto{\pgfqpoint{1.013184in}{1.177183in}}%
\pgfpathlineto{\pgfqpoint{1.045488in}{1.214902in}}%
\pgfpathlineto{\pgfqpoint{1.077792in}{1.287249in}}%
\pgfpathlineto{\pgfqpoint{1.110096in}{1.276953in}}%
\pgfpathlineto{\pgfqpoint{1.142400in}{1.372230in}}%
\pgfpathlineto{\pgfqpoint{1.174704in}{1.441169in}}%
\pgfpathlineto{\pgfqpoint{1.207008in}{1.566275in}}%
\pgfpathlineto{\pgfqpoint{1.239311in}{1.462047in}}%
\pgfpathlineto{\pgfqpoint{1.271615in}{1.522671in}}%
\pgfpathlineto{\pgfqpoint{1.303919in}{1.707610in}}%
\pgfpathlineto{\pgfqpoint{1.336223in}{1.843612in}}%
\pgfpathlineto{\pgfqpoint{1.368527in}{1.617805in}}%
\pgfpathlineto{\pgfqpoint{1.400831in}{1.897000in}}%
\pgfpathlineto{\pgfqpoint{1.433135in}{1.909458in}}%
\pgfpathlineto{\pgfqpoint{1.465439in}{1.957997in}}%
\pgfpathlineto{\pgfqpoint{1.497743in}{2.012697in}}%
\pgfpathlineto{\pgfqpoint{1.530047in}{1.923858in}}%
\pgfpathlineto{\pgfqpoint{1.562351in}{2.066715in}}%
\pgfpathlineto{\pgfqpoint{1.594655in}{2.362602in}}%
\pgfpathlineto{\pgfqpoint{1.626959in}{2.198236in}}%
\pgfpathlineto{\pgfqpoint{1.659263in}{2.079235in}}%
\pgfpathlineto{\pgfqpoint{1.691567in}{2.184178in}}%
\pgfpathlineto{\pgfqpoint{1.723870in}{2.109633in}}%
\pgfpathlineto{\pgfqpoint{1.756174in}{2.189836in}}%
\pgfpathlineto{\pgfqpoint{1.788478in}{2.278080in}}%
\pgfpathlineto{\pgfqpoint{1.820782in}{2.062099in}}%
\pgfpathlineto{\pgfqpoint{1.853086in}{1.981788in}}%
\pgfpathlineto{\pgfqpoint{1.885390in}{1.906033in}}%
\pgfpathlineto{\pgfqpoint{1.917694in}{1.879204in}}%
\pgfpathlineto{\pgfqpoint{1.949998in}{2.095488in}}%
\pgfpathlineto{\pgfqpoint{1.982302in}{1.886977in}}%
\pgfpathlineto{\pgfqpoint{2.014606in}{1.879041in}}%
\pgfpathlineto{\pgfqpoint{2.046910in}{1.442552in}}%
\pgfpathlineto{\pgfqpoint{2.079214in}{1.886354in}}%
\pgfpathlineto{\pgfqpoint{2.111518in}{2.267826in}}%
\pgfpathlineto{\pgfqpoint{2.143822in}{2.089082in}}%
\pgfpathlineto{\pgfqpoint{2.176125in}{1.943134in}}%
\pgfpathlineto{\pgfqpoint{2.208429in}{1.929190in}}%
\pgfpathlineto{\pgfqpoint{2.240733in}{2.047531in}}%
\pgfpathlineto{\pgfqpoint{2.273037in}{2.057596in}}%
\pgfpathlineto{\pgfqpoint{2.305341in}{2.081849in}}%
\pgfpathlineto{\pgfqpoint{2.337645in}{1.590502in}}%
\pgfpathlineto{\pgfqpoint{2.369949in}{1.955466in}}%
\pgfpathlineto{\pgfqpoint{2.402253in}{1.847368in}}%
\pgfpathlineto{\pgfqpoint{2.434557in}{1.411322in}}%
\pgfpathlineto{\pgfqpoint{2.466861in}{1.654190in}}%
\pgfpathlineto{\pgfqpoint{2.499165in}{1.785445in}}%
\pgfpathlineto{\pgfqpoint{2.531469in}{1.958578in}}%
\pgfpathlineto{\pgfqpoint{2.563773in}{1.988793in}}%
\pgfpathlineto{\pgfqpoint{2.596077in}{2.113641in}}%
\pgfpathlineto{\pgfqpoint{2.628381in}{1.739335in}}%
\pgfpathlineto{\pgfqpoint{2.660684in}{1.734968in}}%
\pgfpathlineto{\pgfqpoint{2.692988in}{1.785133in}}%
\pgfpathlineto{\pgfqpoint{2.725292in}{2.072562in}}%
\pgfpathlineto{\pgfqpoint{2.757596in}{2.414232in}}%
\pgfpathlineto{\pgfqpoint{2.789900in}{1.959685in}}%
\pgfpathlineto{\pgfqpoint{2.822204in}{1.435155in}}%
\pgfpathlineto{\pgfqpoint{2.854508in}{1.637932in}}%
\pgfpathlineto{\pgfqpoint{2.919116in}{1.771112in}}%
\pgfpathlineto{\pgfqpoint{2.983724in}{1.781894in}}%
\pgfpathlineto{\pgfqpoint{3.016028in}{1.451592in}}%
\pgfpathlineto{\pgfqpoint{3.048332in}{1.826604in}}%
\pgfpathlineto{\pgfqpoint{3.080636in}{2.069943in}}%
\pgfpathlineto{\pgfqpoint{3.112940in}{1.721546in}}%
\pgfpathlineto{\pgfqpoint{3.177547in}{1.661287in}}%
\pgfpathlineto{\pgfqpoint{3.209851in}{1.748359in}}%
\pgfpathlineto{\pgfqpoint{3.242155in}{1.719387in}}%
\pgfpathlineto{\pgfqpoint{3.306763in}{1.707687in}}%
\pgfpathlineto{\pgfqpoint{3.339067in}{2.320067in}}%
\pgfpathlineto{\pgfqpoint{3.371371in}{1.836439in}}%
\pgfpathlineto{\pgfqpoint{3.403675in}{1.468289in}}%
\pgfpathlineto{\pgfqpoint{3.435979in}{1.718936in}}%
\pgfpathlineto{\pgfqpoint{3.500587in}{1.519293in}}%
\pgfpathlineto{\pgfqpoint{3.565195in}{1.848669in}}%
\pgfpathlineto{\pgfqpoint{3.597498in}{1.482620in}}%
\pgfpathlineto{\pgfqpoint{3.629802in}{1.761501in}}%
\pgfpathlineto{\pgfqpoint{3.694410in}{1.741148in}}%
\pgfpathlineto{\pgfqpoint{3.726714in}{1.566432in}}%
\pgfpathlineto{\pgfqpoint{3.759018in}{1.798249in}}%
\pgfpathlineto{\pgfqpoint{3.823626in}{1.537823in}}%
\pgfpathlineto{\pgfqpoint{3.888234in}{1.617363in}}%
\pgfpathlineto{\pgfqpoint{3.952842in}{1.695525in}}%
\pgfpathlineto{\pgfqpoint{4.017450in}{1.518001in}}%
\pgfpathlineto{\pgfqpoint{4.049754in}{1.630092in}}%
\pgfusepath{stroke}%
\end{pgfscope}%
\begin{pgfscope}%
\pgfsetrectcap%
\pgfsetmiterjoin%
\pgfsetlinewidth{0.803000pt}%
\definecolor{currentstroke}{rgb}{0.000000,0.000000,0.000000}%
\pgfsetstrokecolor{currentstroke}%
\pgfsetdash{}{0pt}%
\pgfpathmoveto{\pgfqpoint{0.588387in}{0.521603in}}%
\pgfpathlineto{\pgfqpoint{0.588387in}{2.741849in}}%
\pgfusepath{stroke}%
\end{pgfscope}%
\begin{pgfscope}%
\pgfsetrectcap%
\pgfsetmiterjoin%
\pgfsetlinewidth{0.803000pt}%
\definecolor{currentstroke}{rgb}{0.000000,0.000000,0.000000}%
\pgfsetstrokecolor{currentstroke}%
\pgfsetdash{}{0pt}%
\pgfpathmoveto{\pgfqpoint{4.248423in}{0.521603in}}%
\pgfpathlineto{\pgfqpoint{4.248423in}{2.741849in}}%
\pgfusepath{stroke}%
\end{pgfscope}%
\begin{pgfscope}%
\pgfsetrectcap%
\pgfsetmiterjoin%
\pgfsetlinewidth{0.803000pt}%
\definecolor{currentstroke}{rgb}{0.000000,0.000000,0.000000}%
\pgfsetstrokecolor{currentstroke}%
\pgfsetdash{}{0pt}%
\pgfpathmoveto{\pgfqpoint{0.588387in}{0.521603in}}%
\pgfpathlineto{\pgfqpoint{4.248423in}{0.521603in}}%
\pgfusepath{stroke}%
\end{pgfscope}%
\begin{pgfscope}%
\pgfsetrectcap%
\pgfsetmiterjoin%
\pgfsetlinewidth{0.803000pt}%
\definecolor{currentstroke}{rgb}{0.000000,0.000000,0.000000}%
\pgfsetstrokecolor{currentstroke}%
\pgfsetdash{}{0pt}%
\pgfpathmoveto{\pgfqpoint{0.588387in}{2.741849in}}%
\pgfpathlineto{\pgfqpoint{4.248423in}{2.741849in}}%
\pgfusepath{stroke}%
\end{pgfscope}%
\begin{pgfscope}%
\pgfsetbuttcap%
\pgfsetmiterjoin%
\definecolor{currentfill}{rgb}{1.000000,1.000000,1.000000}%
\pgfsetfillcolor{currentfill}%
\pgfsetfillopacity{0.800000}%
\pgfsetlinewidth{1.003750pt}%
\definecolor{currentstroke}{rgb}{0.800000,0.800000,0.800000}%
\pgfsetstrokecolor{currentstroke}%
\pgfsetstrokeopacity{0.800000}%
\pgfsetdash{}{0pt}%
\pgfpathmoveto{\pgfqpoint{4.365089in}{0.379025in}}%
\pgfpathlineto{\pgfqpoint{8.251043in}{0.379025in}}%
\pgfpathquadraticcurveto{\pgfqpoint{8.284376in}{0.379025in}}{\pgfqpoint{8.284376in}{0.412359in}}%
\pgfpathlineto{\pgfqpoint{8.284376in}{2.625183in}}%
\pgfpathquadraticcurveto{\pgfqpoint{8.284376in}{2.658516in}}{\pgfqpoint{8.251043in}{2.658516in}}%
\pgfpathlineto{\pgfqpoint{4.365089in}{2.658516in}}%
\pgfpathquadraticcurveto{\pgfqpoint{4.331756in}{2.658516in}}{\pgfqpoint{4.331756in}{2.625183in}}%
\pgfpathlineto{\pgfqpoint{4.331756in}{0.412359in}}%
\pgfpathquadraticcurveto{\pgfqpoint{4.331756in}{0.379025in}}{\pgfqpoint{4.365089in}{0.379025in}}%
\pgfpathlineto{\pgfqpoint{4.365089in}{0.379025in}}%
\pgfpathclose%
\pgfusepath{stroke,fill}%
\end{pgfscope}%
\begin{pgfscope}%
\pgfsetrectcap%
\pgfsetroundjoin%
\pgfsetlinewidth{1.505625pt}%
\pgfsetstrokecolor{currentstroke3}%
\pgfsetdash{}{0pt}%
\pgfpathmoveto{\pgfqpoint{4.398423in}{2.523555in}}%
\pgfpathlineto{\pgfqpoint{4.565089in}{2.523555in}}%
\pgfpathlineto{\pgfqpoint{4.731756in}{2.523555in}}%
\pgfusepath{stroke}%
\end{pgfscope}%
\begin{pgfscope}%
\definecolor{textcolor}{rgb}{0.000000,0.000000,0.000000}%
\pgfsetstrokecolor{textcolor}%
\pgfsetfillcolor{textcolor}%
\pgftext[x=4.865089in,y=2.465222in,left,base]{\color{textcolor}{\rmfamily\fontsize{12.000000}{14.400000}\selectfont\catcode`\^=\active\def^{\ifmmode\sp\else\^{}\fi}\catcode`\%=\active\def%{\%}\NaiveCycles{}}}%
\end{pgfscope}%
\begin{pgfscope}%
\pgfsetrectcap%
\pgfsetroundjoin%
\pgfsetlinewidth{1.505625pt}%
\pgfsetstrokecolor{currentstroke1}%
\pgfsetdash{}{0pt}%
\pgfpathmoveto{\pgfqpoint{4.398423in}{2.278926in}}%
\pgfpathlineto{\pgfqpoint{4.565089in}{2.278926in}}%
\pgfpathlineto{\pgfqpoint{4.731756in}{2.278926in}}%
\pgfusepath{stroke}%
\end{pgfscope}%
\begin{pgfscope}%
\definecolor{textcolor}{rgb}{0.000000,0.000000,0.000000}%
\pgfsetstrokecolor{textcolor}%
\pgfsetfillcolor{textcolor}%
\pgftext[x=4.865089in,y=2.220593in,left,base]{\color{textcolor}{\rmfamily\fontsize{12.000000}{14.400000}\selectfont\catcode`\^=\active\def^{\ifmmode\sp\else\^{}\fi}\catcode`\%=\active\def%{\%}\CyclesMatchChunks{} \& \MergeLinear{}}}%
\end{pgfscope}%
\begin{pgfscope}%
\pgfsetrectcap%
\pgfsetroundjoin%
\pgfsetlinewidth{1.505625pt}%
\pgfsetstrokecolor{currentstroke2}%
\pgfsetdash{}{0pt}%
\pgfpathmoveto{\pgfqpoint{4.398423in}{2.029659in}}%
\pgfpathlineto{\pgfqpoint{4.565089in}{2.029659in}}%
\pgfpathlineto{\pgfqpoint{4.731756in}{2.029659in}}%
\pgfusepath{stroke}%
\end{pgfscope}%
\begin{pgfscope}%
\definecolor{textcolor}{rgb}{0.000000,0.000000,0.000000}%
\pgfsetstrokecolor{textcolor}%
\pgfsetfillcolor{textcolor}%
\pgftext[x=4.865089in,y=1.971325in,left,base]{\color{textcolor}{\rmfamily\fontsize{12.000000}{14.400000}\selectfont\catcode`\^=\active\def^{\ifmmode\sp\else\^{}\fi}\catcode`\%=\active\def%{\%}\CyclesMatchChunks{} \& \SharedVertices{}}}%
\end{pgfscope}%
\begin{pgfscope}%
\pgfsetrectcap%
\pgfsetroundjoin%
\pgfsetlinewidth{1.505625pt}%
\pgfsetstrokecolor{currentstroke4}%
\pgfsetdash{}{0pt}%
\pgfpathmoveto{\pgfqpoint{4.398423in}{1.780391in}}%
\pgfpathlineto{\pgfqpoint{4.565089in}{1.780391in}}%
\pgfpathlineto{\pgfqpoint{4.731756in}{1.780391in}}%
\pgfusepath{stroke}%
\end{pgfscope}%
\begin{pgfscope}%
\definecolor{textcolor}{rgb}{0.000000,0.000000,0.000000}%
\pgfsetstrokecolor{textcolor}%
\pgfsetfillcolor{textcolor}%
\pgftext[x=4.865089in,y=1.722058in,left,base]{\color{textcolor}{\rmfamily\fontsize{12.000000}{14.400000}\selectfont\catcode`\^=\active\def^{\ifmmode\sp\else\^{}\fi}\catcode`\%=\active\def%{\%}\Neighbors{} \& \MergeLinear{}}}%
\end{pgfscope}%
\begin{pgfscope}%
\pgfsetrectcap%
\pgfsetroundjoin%
\pgfsetlinewidth{1.505625pt}%
\pgfsetstrokecolor{currentstroke5}%
\pgfsetdash{}{0pt}%
\pgfpathmoveto{\pgfqpoint{4.398423in}{1.535763in}}%
\pgfpathlineto{\pgfqpoint{4.565089in}{1.535763in}}%
\pgfpathlineto{\pgfqpoint{4.731756in}{1.535763in}}%
\pgfusepath{stroke}%
\end{pgfscope}%
\begin{pgfscope}%
\definecolor{textcolor}{rgb}{0.000000,0.000000,0.000000}%
\pgfsetstrokecolor{textcolor}%
\pgfsetfillcolor{textcolor}%
\pgftext[x=4.865089in,y=1.477429in,left,base]{\color{textcolor}{\rmfamily\fontsize{12.000000}{14.400000}\selectfont\catcode`\^=\active\def^{\ifmmode\sp\else\^{}\fi}\catcode`\%=\active\def%{\%}\Neighbors{} \& \SharedVertices{}}}%
\end{pgfscope}%
\begin{pgfscope}%
\pgfsetrectcap%
\pgfsetroundjoin%
\pgfsetlinewidth{1.505625pt}%
\pgfsetstrokecolor{currentstroke6}%
\pgfsetdash{}{0pt}%
\pgfpathmoveto{\pgfqpoint{4.398423in}{1.286495in}}%
\pgfpathlineto{\pgfqpoint{4.565089in}{1.286495in}}%
\pgfpathlineto{\pgfqpoint{4.731756in}{1.286495in}}%
\pgfusepath{stroke}%
\end{pgfscope}%
\begin{pgfscope}%
\definecolor{textcolor}{rgb}{0.000000,0.000000,0.000000}%
\pgfsetstrokecolor{textcolor}%
\pgfsetfillcolor{textcolor}%
\pgftext[x=4.865089in,y=1.228162in,left,base]{\color{textcolor}{\rmfamily\fontsize{12.000000}{14.400000}\selectfont\catcode`\^=\active\def^{\ifmmode\sp\else\^{}\fi}\catcode`\%=\active\def%{\%}\NeighborsDegree{} \& \MergeLinear{}}}%
\end{pgfscope}%
\begin{pgfscope}%
\pgfsetrectcap%
\pgfsetroundjoin%
\pgfsetlinewidth{1.505625pt}%
\pgfsetstrokecolor{currentstroke7}%
\pgfsetdash{}{0pt}%
\pgfpathmoveto{\pgfqpoint{4.398423in}{1.037228in}}%
\pgfpathlineto{\pgfqpoint{4.565089in}{1.037228in}}%
\pgfpathlineto{\pgfqpoint{4.731756in}{1.037228in}}%
\pgfusepath{stroke}%
\end{pgfscope}%
\begin{pgfscope}%
\definecolor{textcolor}{rgb}{0.000000,0.000000,0.000000}%
\pgfsetstrokecolor{textcolor}%
\pgfsetfillcolor{textcolor}%
\pgftext[x=4.865089in,y=0.978895in,left,base]{\color{textcolor}{\rmfamily\fontsize{12.000000}{14.400000}\selectfont\catcode`\^=\active\def^{\ifmmode\sp\else\^{}\fi}\catcode`\%=\active\def%{\%}\NeighborsDegree{} \& \SharedVertices{}}}%
\end{pgfscope}%
\begin{pgfscope}%
\pgfsetrectcap%
\pgfsetroundjoin%
\pgfsetlinewidth{1.505625pt}%
\definecolor{currentstroke}{rgb}{0.498039,0.498039,0.498039}%
\pgfsetstrokecolor{currentstroke}%
\pgfsetdash{}{0pt}%
\pgfpathmoveto{\pgfqpoint{4.398423in}{0.787961in}}%
\pgfpathlineto{\pgfqpoint{4.565089in}{0.787961in}}%
\pgfpathlineto{\pgfqpoint{4.731756in}{0.787961in}}%
\pgfusepath{stroke}%
\end{pgfscope}%
\begin{pgfscope}%
\definecolor{textcolor}{rgb}{0.000000,0.000000,0.000000}%
\pgfsetstrokecolor{textcolor}%
\pgfsetfillcolor{textcolor}%
\pgftext[x=4.865089in,y=0.729627in,left,base]{\color{textcolor}{\rmfamily\fontsize{12.000000}{14.400000}\selectfont\catcode`\^=\active\def^{\ifmmode\sp\else\^{}\fi}\catcode`\%=\active\def%{\%}\None{} \& \MergeLinear{}}}%
\end{pgfscope}%
\begin{pgfscope}%
\pgfsetrectcap%
\pgfsetroundjoin%
\pgfsetlinewidth{1.505625pt}%
\definecolor{currentstroke}{rgb}{0.737255,0.741176,0.133333}%
\pgfsetstrokecolor{currentstroke}%
\pgfsetdash{}{0pt}%
\pgfpathmoveto{\pgfqpoint{4.398423in}{0.543332in}}%
\pgfpathlineto{\pgfqpoint{4.565089in}{0.543332in}}%
\pgfpathlineto{\pgfqpoint{4.731756in}{0.543332in}}%
\pgfusepath{stroke}%
\end{pgfscope}%
\begin{pgfscope}%
\definecolor{textcolor}{rgb}{0.000000,0.000000,0.000000}%
\pgfsetstrokecolor{textcolor}%
\pgfsetfillcolor{textcolor}%
\pgftext[x=4.865089in,y=0.484999in,left,base]{\color{textcolor}{\rmfamily\fontsize{12.000000}{14.400000}\selectfont\catcode`\^=\active\def^{\ifmmode\sp\else\^{}\fi}\catcode`\%=\active\def%{\%}\None{} \& \SharedVertices{}}}%
\end{pgfscope}%
\end{pgfpicture}%
\makeatother%
\endgroup%
}
	\caption[Checks performed for minimally rigid graphs (some)]{
		The number of checks performed to find some NAC-coloring for minimally rigid graphs.}%
	\label{fig:graph_minimally_rigid_first_checks}
\end{figure}%



\subsubsection*{No three nor four cycle graphs}

From~\cite{extremal_graphs} we obtained all graphs with up to 52 vertices
that have no three nor four cycles. This class of graphs is interesting for us
as there cannot be any \trcon{} components.
These graphs have also many NAC-colorings.
Because of that, as seen in \Cref{fig:graph_count_no_3_nor_4_cycles_first_runtime},
\NaiveCycles{} is again faster for finding some NAC-coloring
for the similar reasons as for minimally rigid graphs.
%
Also notice that \SharedVertices{} performs worse and non-deterministically
for \CyclesMatchChunks{} and \None{}.
For \Neighbors{}, the performance is more stable, but still worse than \MergeLinear{}.

\begin{figure}[thbp]
	\centering
	\scalebox{\BenchFigureScale}{%% Creator: Matplotlib, PGF backend
%%
%% To include the figure in your LaTeX document, write
%%   \input{<filename>.pgf}
%%
%% Make sure the required packages are loaded in your preamble
%%   \usepackage{pgf}
%%
%% Also ensure that all the required font packages are loaded; for instance,
%% the lmodern package is sometimes necessary when using math font.
%%   \usepackage{lmodern}
%%
%% Figures using additional raster images can only be included by \input if
%% they are in the same directory as the main LaTeX file. For loading figures
%% from other directories you can use the `import` package
%%   \usepackage{import}
%%
%% and then include the figures with
%%   \import{<path to file>}{<filename>.pgf}
%%
%% Matplotlib used the following preamble
%%   \def\mathdefault#1{#1}
%%   \everymath=\expandafter{\the\everymath\displaystyle}
%%   \IfFileExists{scrextend.sty}{
%%     \usepackage[fontsize=10.000000pt]{scrextend}
%%   }{
%%     \renewcommand{\normalsize}{\fontsize{10.000000}{12.000000}\selectfont}
%%     \normalsize
%%   }
%%   
%%   \ifdefined\pdftexversion\else  % non-pdftex case.
%%     \usepackage{fontspec}
%%     \setmainfont{DejaVuSans.ttf}[Path=\detokenize{/home/petr/Projects/PyRigi/.venv/lib/python3.12/site-packages/matplotlib/mpl-data/fonts/ttf/}]
%%     \setsansfont{DejaVuSans.ttf}[Path=\detokenize{/home/petr/Projects/PyRigi/.venv/lib/python3.12/site-packages/matplotlib/mpl-data/fonts/ttf/}]
%%     \setmonofont{DejaVuSansMono.ttf}[Path=\detokenize{/home/petr/Projects/PyRigi/.venv/lib/python3.12/site-packages/matplotlib/mpl-data/fonts/ttf/}]
%%   \fi
%%   \makeatletter\@ifpackageloaded{underscore}{}{\usepackage[strings]{underscore}}\makeatother
%%
\begingroup%
\makeatletter%
\begin{pgfpicture}%
\pgfpathrectangle{\pgfpointorigin}{\pgfqpoint{8.384376in}{2.841849in}}%
\pgfusepath{use as bounding box, clip}%
\begin{pgfscope}%
\pgfsetbuttcap%
\pgfsetmiterjoin%
\definecolor{currentfill}{rgb}{1.000000,1.000000,1.000000}%
\pgfsetfillcolor{currentfill}%
\pgfsetlinewidth{0.000000pt}%
\definecolor{currentstroke}{rgb}{1.000000,1.000000,1.000000}%
\pgfsetstrokecolor{currentstroke}%
\pgfsetdash{}{0pt}%
\pgfpathmoveto{\pgfqpoint{0.000000in}{0.000000in}}%
\pgfpathlineto{\pgfqpoint{8.384376in}{0.000000in}}%
\pgfpathlineto{\pgfqpoint{8.384376in}{2.841849in}}%
\pgfpathlineto{\pgfqpoint{0.000000in}{2.841849in}}%
\pgfpathlineto{\pgfqpoint{0.000000in}{0.000000in}}%
\pgfpathclose%
\pgfusepath{fill}%
\end{pgfscope}%
\begin{pgfscope}%
\pgfsetbuttcap%
\pgfsetmiterjoin%
\definecolor{currentfill}{rgb}{1.000000,1.000000,1.000000}%
\pgfsetfillcolor{currentfill}%
\pgfsetlinewidth{0.000000pt}%
\definecolor{currentstroke}{rgb}{0.000000,0.000000,0.000000}%
\pgfsetstrokecolor{currentstroke}%
\pgfsetstrokeopacity{0.000000}%
\pgfsetdash{}{0pt}%
\pgfpathmoveto{\pgfqpoint{0.588387in}{0.521603in}}%
\pgfpathlineto{\pgfqpoint{5.257411in}{0.521603in}}%
\pgfpathlineto{\pgfqpoint{5.257411in}{2.531888in}}%
\pgfpathlineto{\pgfqpoint{0.588387in}{2.531888in}}%
\pgfpathlineto{\pgfqpoint{0.588387in}{0.521603in}}%
\pgfpathclose%
\pgfusepath{fill}%
\end{pgfscope}%
\begin{pgfscope}%
\pgfsetbuttcap%
\pgfsetroundjoin%
\definecolor{currentfill}{rgb}{0.000000,0.000000,0.000000}%
\pgfsetfillcolor{currentfill}%
\pgfsetlinewidth{0.803000pt}%
\definecolor{currentstroke}{rgb}{0.000000,0.000000,0.000000}%
\pgfsetstrokecolor{currentstroke}%
\pgfsetdash{}{0pt}%
\pgfsys@defobject{currentmarker}{\pgfqpoint{0.000000in}{-0.048611in}}{\pgfqpoint{0.000000in}{0.000000in}}{%
\pgfpathmoveto{\pgfqpoint{0.000000in}{0.000000in}}%
\pgfpathlineto{\pgfqpoint{0.000000in}{-0.048611in}}%
\pgfusepath{stroke,fill}%
}%
\begin{pgfscope}%
\pgfsys@transformshift{0.677940in}{0.521603in}%
\pgfsys@useobject{currentmarker}{}%
\end{pgfscope}%
\end{pgfscope}%
\begin{pgfscope}%
\definecolor{textcolor}{rgb}{0.000000,0.000000,0.000000}%
\pgfsetstrokecolor{textcolor}%
\pgfsetfillcolor{textcolor}%
\pgftext[x=0.677940in,y=0.424381in,,top]{\color{textcolor}{\rmfamily\fontsize{10.000000}{12.000000}\selectfont\catcode`\^=\active\def^{\ifmmode\sp\else\^{}\fi}\catcode`\%=\active\def%{\%}$\mathdefault{0}$}}%
\end{pgfscope}%
\begin{pgfscope}%
\pgfsetbuttcap%
\pgfsetroundjoin%
\definecolor{currentfill}{rgb}{0.000000,0.000000,0.000000}%
\pgfsetfillcolor{currentfill}%
\pgfsetlinewidth{0.803000pt}%
\definecolor{currentstroke}{rgb}{0.000000,0.000000,0.000000}%
\pgfsetstrokecolor{currentstroke}%
\pgfsetdash{}{0pt}%
\pgfsys@defobject{currentmarker}{\pgfqpoint{0.000000in}{-0.048611in}}{\pgfqpoint{0.000000in}{0.000000in}}{%
\pgfpathmoveto{\pgfqpoint{0.000000in}{0.000000in}}%
\pgfpathlineto{\pgfqpoint{0.000000in}{-0.048611in}}%
\pgfusepath{stroke,fill}%
}%
\begin{pgfscope}%
\pgfsys@transformshift{1.168642in}{0.521603in}%
\pgfsys@useobject{currentmarker}{}%
\end{pgfscope}%
\end{pgfscope}%
\begin{pgfscope}%
\definecolor{textcolor}{rgb}{0.000000,0.000000,0.000000}%
\pgfsetstrokecolor{textcolor}%
\pgfsetfillcolor{textcolor}%
\pgftext[x=1.168642in,y=0.424381in,,top]{\color{textcolor}{\rmfamily\fontsize{10.000000}{12.000000}\selectfont\catcode`\^=\active\def^{\ifmmode\sp\else\^{}\fi}\catcode`\%=\active\def%{\%}$\mathdefault{20}$}}%
\end{pgfscope}%
\begin{pgfscope}%
\pgfsetbuttcap%
\pgfsetroundjoin%
\definecolor{currentfill}{rgb}{0.000000,0.000000,0.000000}%
\pgfsetfillcolor{currentfill}%
\pgfsetlinewidth{0.803000pt}%
\definecolor{currentstroke}{rgb}{0.000000,0.000000,0.000000}%
\pgfsetstrokecolor{currentstroke}%
\pgfsetdash{}{0pt}%
\pgfsys@defobject{currentmarker}{\pgfqpoint{0.000000in}{-0.048611in}}{\pgfqpoint{0.000000in}{0.000000in}}{%
\pgfpathmoveto{\pgfqpoint{0.000000in}{0.000000in}}%
\pgfpathlineto{\pgfqpoint{0.000000in}{-0.048611in}}%
\pgfusepath{stroke,fill}%
}%
\begin{pgfscope}%
\pgfsys@transformshift{1.659343in}{0.521603in}%
\pgfsys@useobject{currentmarker}{}%
\end{pgfscope}%
\end{pgfscope}%
\begin{pgfscope}%
\definecolor{textcolor}{rgb}{0.000000,0.000000,0.000000}%
\pgfsetstrokecolor{textcolor}%
\pgfsetfillcolor{textcolor}%
\pgftext[x=1.659343in,y=0.424381in,,top]{\color{textcolor}{\rmfamily\fontsize{10.000000}{12.000000}\selectfont\catcode`\^=\active\def^{\ifmmode\sp\else\^{}\fi}\catcode`\%=\active\def%{\%}$\mathdefault{40}$}}%
\end{pgfscope}%
\begin{pgfscope}%
\pgfsetbuttcap%
\pgfsetroundjoin%
\definecolor{currentfill}{rgb}{0.000000,0.000000,0.000000}%
\pgfsetfillcolor{currentfill}%
\pgfsetlinewidth{0.803000pt}%
\definecolor{currentstroke}{rgb}{0.000000,0.000000,0.000000}%
\pgfsetstrokecolor{currentstroke}%
\pgfsetdash{}{0pt}%
\pgfsys@defobject{currentmarker}{\pgfqpoint{0.000000in}{-0.048611in}}{\pgfqpoint{0.000000in}{0.000000in}}{%
\pgfpathmoveto{\pgfqpoint{0.000000in}{0.000000in}}%
\pgfpathlineto{\pgfqpoint{0.000000in}{-0.048611in}}%
\pgfusepath{stroke,fill}%
}%
\begin{pgfscope}%
\pgfsys@transformshift{2.150044in}{0.521603in}%
\pgfsys@useobject{currentmarker}{}%
\end{pgfscope}%
\end{pgfscope}%
\begin{pgfscope}%
\definecolor{textcolor}{rgb}{0.000000,0.000000,0.000000}%
\pgfsetstrokecolor{textcolor}%
\pgfsetfillcolor{textcolor}%
\pgftext[x=2.150044in,y=0.424381in,,top]{\color{textcolor}{\rmfamily\fontsize{10.000000}{12.000000}\selectfont\catcode`\^=\active\def^{\ifmmode\sp\else\^{}\fi}\catcode`\%=\active\def%{\%}$\mathdefault{60}$}}%
\end{pgfscope}%
\begin{pgfscope}%
\pgfsetbuttcap%
\pgfsetroundjoin%
\definecolor{currentfill}{rgb}{0.000000,0.000000,0.000000}%
\pgfsetfillcolor{currentfill}%
\pgfsetlinewidth{0.803000pt}%
\definecolor{currentstroke}{rgb}{0.000000,0.000000,0.000000}%
\pgfsetstrokecolor{currentstroke}%
\pgfsetdash{}{0pt}%
\pgfsys@defobject{currentmarker}{\pgfqpoint{0.000000in}{-0.048611in}}{\pgfqpoint{0.000000in}{0.000000in}}{%
\pgfpathmoveto{\pgfqpoint{0.000000in}{0.000000in}}%
\pgfpathlineto{\pgfqpoint{0.000000in}{-0.048611in}}%
\pgfusepath{stroke,fill}%
}%
\begin{pgfscope}%
\pgfsys@transformshift{2.640746in}{0.521603in}%
\pgfsys@useobject{currentmarker}{}%
\end{pgfscope}%
\end{pgfscope}%
\begin{pgfscope}%
\definecolor{textcolor}{rgb}{0.000000,0.000000,0.000000}%
\pgfsetstrokecolor{textcolor}%
\pgfsetfillcolor{textcolor}%
\pgftext[x=2.640746in,y=0.424381in,,top]{\color{textcolor}{\rmfamily\fontsize{10.000000}{12.000000}\selectfont\catcode`\^=\active\def^{\ifmmode\sp\else\^{}\fi}\catcode`\%=\active\def%{\%}$\mathdefault{80}$}}%
\end{pgfscope}%
\begin{pgfscope}%
\pgfsetbuttcap%
\pgfsetroundjoin%
\definecolor{currentfill}{rgb}{0.000000,0.000000,0.000000}%
\pgfsetfillcolor{currentfill}%
\pgfsetlinewidth{0.803000pt}%
\definecolor{currentstroke}{rgb}{0.000000,0.000000,0.000000}%
\pgfsetstrokecolor{currentstroke}%
\pgfsetdash{}{0pt}%
\pgfsys@defobject{currentmarker}{\pgfqpoint{0.000000in}{-0.048611in}}{\pgfqpoint{0.000000in}{0.000000in}}{%
\pgfpathmoveto{\pgfqpoint{0.000000in}{0.000000in}}%
\pgfpathlineto{\pgfqpoint{0.000000in}{-0.048611in}}%
\pgfusepath{stroke,fill}%
}%
\begin{pgfscope}%
\pgfsys@transformshift{3.131447in}{0.521603in}%
\pgfsys@useobject{currentmarker}{}%
\end{pgfscope}%
\end{pgfscope}%
\begin{pgfscope}%
\definecolor{textcolor}{rgb}{0.000000,0.000000,0.000000}%
\pgfsetstrokecolor{textcolor}%
\pgfsetfillcolor{textcolor}%
\pgftext[x=3.131447in,y=0.424381in,,top]{\color{textcolor}{\rmfamily\fontsize{10.000000}{12.000000}\selectfont\catcode`\^=\active\def^{\ifmmode\sp\else\^{}\fi}\catcode`\%=\active\def%{\%}$\mathdefault{100}$}}%
\end{pgfscope}%
\begin{pgfscope}%
\pgfsetbuttcap%
\pgfsetroundjoin%
\definecolor{currentfill}{rgb}{0.000000,0.000000,0.000000}%
\pgfsetfillcolor{currentfill}%
\pgfsetlinewidth{0.803000pt}%
\definecolor{currentstroke}{rgb}{0.000000,0.000000,0.000000}%
\pgfsetstrokecolor{currentstroke}%
\pgfsetdash{}{0pt}%
\pgfsys@defobject{currentmarker}{\pgfqpoint{0.000000in}{-0.048611in}}{\pgfqpoint{0.000000in}{0.000000in}}{%
\pgfpathmoveto{\pgfqpoint{0.000000in}{0.000000in}}%
\pgfpathlineto{\pgfqpoint{0.000000in}{-0.048611in}}%
\pgfusepath{stroke,fill}%
}%
\begin{pgfscope}%
\pgfsys@transformshift{3.622149in}{0.521603in}%
\pgfsys@useobject{currentmarker}{}%
\end{pgfscope}%
\end{pgfscope}%
\begin{pgfscope}%
\definecolor{textcolor}{rgb}{0.000000,0.000000,0.000000}%
\pgfsetstrokecolor{textcolor}%
\pgfsetfillcolor{textcolor}%
\pgftext[x=3.622149in,y=0.424381in,,top]{\color{textcolor}{\rmfamily\fontsize{10.000000}{12.000000}\selectfont\catcode`\^=\active\def^{\ifmmode\sp\else\^{}\fi}\catcode`\%=\active\def%{\%}$\mathdefault{120}$}}%
\end{pgfscope}%
\begin{pgfscope}%
\pgfsetbuttcap%
\pgfsetroundjoin%
\definecolor{currentfill}{rgb}{0.000000,0.000000,0.000000}%
\pgfsetfillcolor{currentfill}%
\pgfsetlinewidth{0.803000pt}%
\definecolor{currentstroke}{rgb}{0.000000,0.000000,0.000000}%
\pgfsetstrokecolor{currentstroke}%
\pgfsetdash{}{0pt}%
\pgfsys@defobject{currentmarker}{\pgfqpoint{0.000000in}{-0.048611in}}{\pgfqpoint{0.000000in}{0.000000in}}{%
\pgfpathmoveto{\pgfqpoint{0.000000in}{0.000000in}}%
\pgfpathlineto{\pgfqpoint{0.000000in}{-0.048611in}}%
\pgfusepath{stroke,fill}%
}%
\begin{pgfscope}%
\pgfsys@transformshift{4.112850in}{0.521603in}%
\pgfsys@useobject{currentmarker}{}%
\end{pgfscope}%
\end{pgfscope}%
\begin{pgfscope}%
\definecolor{textcolor}{rgb}{0.000000,0.000000,0.000000}%
\pgfsetstrokecolor{textcolor}%
\pgfsetfillcolor{textcolor}%
\pgftext[x=4.112850in,y=0.424381in,,top]{\color{textcolor}{\rmfamily\fontsize{10.000000}{12.000000}\selectfont\catcode`\^=\active\def^{\ifmmode\sp\else\^{}\fi}\catcode`\%=\active\def%{\%}$\mathdefault{140}$}}%
\end{pgfscope}%
\begin{pgfscope}%
\pgfsetbuttcap%
\pgfsetroundjoin%
\definecolor{currentfill}{rgb}{0.000000,0.000000,0.000000}%
\pgfsetfillcolor{currentfill}%
\pgfsetlinewidth{0.803000pt}%
\definecolor{currentstroke}{rgb}{0.000000,0.000000,0.000000}%
\pgfsetstrokecolor{currentstroke}%
\pgfsetdash{}{0pt}%
\pgfsys@defobject{currentmarker}{\pgfqpoint{0.000000in}{-0.048611in}}{\pgfqpoint{0.000000in}{0.000000in}}{%
\pgfpathmoveto{\pgfqpoint{0.000000in}{0.000000in}}%
\pgfpathlineto{\pgfqpoint{0.000000in}{-0.048611in}}%
\pgfusepath{stroke,fill}%
}%
\begin{pgfscope}%
\pgfsys@transformshift{4.603552in}{0.521603in}%
\pgfsys@useobject{currentmarker}{}%
\end{pgfscope}%
\end{pgfscope}%
\begin{pgfscope}%
\definecolor{textcolor}{rgb}{0.000000,0.000000,0.000000}%
\pgfsetstrokecolor{textcolor}%
\pgfsetfillcolor{textcolor}%
\pgftext[x=4.603552in,y=0.424381in,,top]{\color{textcolor}{\rmfamily\fontsize{10.000000}{12.000000}\selectfont\catcode`\^=\active\def^{\ifmmode\sp\else\^{}\fi}\catcode`\%=\active\def%{\%}$\mathdefault{160}$}}%
\end{pgfscope}%
\begin{pgfscope}%
\pgfsetbuttcap%
\pgfsetroundjoin%
\definecolor{currentfill}{rgb}{0.000000,0.000000,0.000000}%
\pgfsetfillcolor{currentfill}%
\pgfsetlinewidth{0.803000pt}%
\definecolor{currentstroke}{rgb}{0.000000,0.000000,0.000000}%
\pgfsetstrokecolor{currentstroke}%
\pgfsetdash{}{0pt}%
\pgfsys@defobject{currentmarker}{\pgfqpoint{0.000000in}{-0.048611in}}{\pgfqpoint{0.000000in}{0.000000in}}{%
\pgfpathmoveto{\pgfqpoint{0.000000in}{0.000000in}}%
\pgfpathlineto{\pgfqpoint{0.000000in}{-0.048611in}}%
\pgfusepath{stroke,fill}%
}%
\begin{pgfscope}%
\pgfsys@transformshift{5.094253in}{0.521603in}%
\pgfsys@useobject{currentmarker}{}%
\end{pgfscope}%
\end{pgfscope}%
\begin{pgfscope}%
\definecolor{textcolor}{rgb}{0.000000,0.000000,0.000000}%
\pgfsetstrokecolor{textcolor}%
\pgfsetfillcolor{textcolor}%
\pgftext[x=5.094253in,y=0.424381in,,top]{\color{textcolor}{\rmfamily\fontsize{10.000000}{12.000000}\selectfont\catcode`\^=\active\def^{\ifmmode\sp\else\^{}\fi}\catcode`\%=\active\def%{\%}$\mathdefault{180}$}}%
\end{pgfscope}%
\begin{pgfscope}%
\definecolor{textcolor}{rgb}{0.000000,0.000000,0.000000}%
\pgfsetstrokecolor{textcolor}%
\pgfsetfillcolor{textcolor}%
\pgftext[x=2.922899in,y=0.234413in,,top]{\color{textcolor}{\rmfamily\fontsize{10.000000}{12.000000}\selectfont\catcode`\^=\active\def^{\ifmmode\sp\else\^{}\fi}\catcode`\%=\active\def%{\%}Monochromatic classes}}%
\end{pgfscope}%
\begin{pgfscope}%
\pgfsetbuttcap%
\pgfsetroundjoin%
\definecolor{currentfill}{rgb}{0.000000,0.000000,0.000000}%
\pgfsetfillcolor{currentfill}%
\pgfsetlinewidth{0.803000pt}%
\definecolor{currentstroke}{rgb}{0.000000,0.000000,0.000000}%
\pgfsetstrokecolor{currentstroke}%
\pgfsetdash{}{0pt}%
\pgfsys@defobject{currentmarker}{\pgfqpoint{-0.048611in}{0.000000in}}{\pgfqpoint{-0.000000in}{0.000000in}}{%
\pgfpathmoveto{\pgfqpoint{-0.000000in}{0.000000in}}%
\pgfpathlineto{\pgfqpoint{-0.048611in}{0.000000in}}%
\pgfusepath{stroke,fill}%
}%
\begin{pgfscope}%
\pgfsys@transformshift{0.588387in}{0.612980in}%
\pgfsys@useobject{currentmarker}{}%
\end{pgfscope}%
\end{pgfscope}%
\begin{pgfscope}%
\definecolor{textcolor}{rgb}{0.000000,0.000000,0.000000}%
\pgfsetstrokecolor{textcolor}%
\pgfsetfillcolor{textcolor}%
\pgftext[x=0.289968in, y=0.560218in, left, base]{\color{textcolor}{\rmfamily\fontsize{10.000000}{12.000000}\selectfont\catcode`\^=\active\def^{\ifmmode\sp\else\^{}\fi}\catcode`\%=\active\def%{\%}$\mathdefault{10^{0}}$}}%
\end{pgfscope}%
\begin{pgfscope}%
\pgfsetbuttcap%
\pgfsetroundjoin%
\definecolor{currentfill}{rgb}{0.000000,0.000000,0.000000}%
\pgfsetfillcolor{currentfill}%
\pgfsetlinewidth{0.803000pt}%
\definecolor{currentstroke}{rgb}{0.000000,0.000000,0.000000}%
\pgfsetstrokecolor{currentstroke}%
\pgfsetdash{}{0pt}%
\pgfsys@defobject{currentmarker}{\pgfqpoint{-0.048611in}{0.000000in}}{\pgfqpoint{-0.000000in}{0.000000in}}{%
\pgfpathmoveto{\pgfqpoint{-0.000000in}{0.000000in}}%
\pgfpathlineto{\pgfqpoint{-0.048611in}{0.000000in}}%
\pgfusepath{stroke,fill}%
}%
\begin{pgfscope}%
\pgfsys@transformshift{0.588387in}{1.107607in}%
\pgfsys@useobject{currentmarker}{}%
\end{pgfscope}%
\end{pgfscope}%
\begin{pgfscope}%
\definecolor{textcolor}{rgb}{0.000000,0.000000,0.000000}%
\pgfsetstrokecolor{textcolor}%
\pgfsetfillcolor{textcolor}%
\pgftext[x=0.289968in, y=1.054846in, left, base]{\color{textcolor}{\rmfamily\fontsize{10.000000}{12.000000}\selectfont\catcode`\^=\active\def^{\ifmmode\sp\else\^{}\fi}\catcode`\%=\active\def%{\%}$\mathdefault{10^{1}}$}}%
\end{pgfscope}%
\begin{pgfscope}%
\pgfsetbuttcap%
\pgfsetroundjoin%
\definecolor{currentfill}{rgb}{0.000000,0.000000,0.000000}%
\pgfsetfillcolor{currentfill}%
\pgfsetlinewidth{0.803000pt}%
\definecolor{currentstroke}{rgb}{0.000000,0.000000,0.000000}%
\pgfsetstrokecolor{currentstroke}%
\pgfsetdash{}{0pt}%
\pgfsys@defobject{currentmarker}{\pgfqpoint{-0.048611in}{0.000000in}}{\pgfqpoint{-0.000000in}{0.000000in}}{%
\pgfpathmoveto{\pgfqpoint{-0.000000in}{0.000000in}}%
\pgfpathlineto{\pgfqpoint{-0.048611in}{0.000000in}}%
\pgfusepath{stroke,fill}%
}%
\begin{pgfscope}%
\pgfsys@transformshift{0.588387in}{1.602235in}%
\pgfsys@useobject{currentmarker}{}%
\end{pgfscope}%
\end{pgfscope}%
\begin{pgfscope}%
\definecolor{textcolor}{rgb}{0.000000,0.000000,0.000000}%
\pgfsetstrokecolor{textcolor}%
\pgfsetfillcolor{textcolor}%
\pgftext[x=0.289968in, y=1.549473in, left, base]{\color{textcolor}{\rmfamily\fontsize{10.000000}{12.000000}\selectfont\catcode`\^=\active\def^{\ifmmode\sp\else\^{}\fi}\catcode`\%=\active\def%{\%}$\mathdefault{10^{2}}$}}%
\end{pgfscope}%
\begin{pgfscope}%
\pgfsetbuttcap%
\pgfsetroundjoin%
\definecolor{currentfill}{rgb}{0.000000,0.000000,0.000000}%
\pgfsetfillcolor{currentfill}%
\pgfsetlinewidth{0.803000pt}%
\definecolor{currentstroke}{rgb}{0.000000,0.000000,0.000000}%
\pgfsetstrokecolor{currentstroke}%
\pgfsetdash{}{0pt}%
\pgfsys@defobject{currentmarker}{\pgfqpoint{-0.048611in}{0.000000in}}{\pgfqpoint{-0.000000in}{0.000000in}}{%
\pgfpathmoveto{\pgfqpoint{-0.000000in}{0.000000in}}%
\pgfpathlineto{\pgfqpoint{-0.048611in}{0.000000in}}%
\pgfusepath{stroke,fill}%
}%
\begin{pgfscope}%
\pgfsys@transformshift{0.588387in}{2.096862in}%
\pgfsys@useobject{currentmarker}{}%
\end{pgfscope}%
\end{pgfscope}%
\begin{pgfscope}%
\definecolor{textcolor}{rgb}{0.000000,0.000000,0.000000}%
\pgfsetstrokecolor{textcolor}%
\pgfsetfillcolor{textcolor}%
\pgftext[x=0.289968in, y=2.044100in, left, base]{\color{textcolor}{\rmfamily\fontsize{10.000000}{12.000000}\selectfont\catcode`\^=\active\def^{\ifmmode\sp\else\^{}\fi}\catcode`\%=\active\def%{\%}$\mathdefault{10^{3}}$}}%
\end{pgfscope}%
\begin{pgfscope}%
\pgfsetbuttcap%
\pgfsetroundjoin%
\definecolor{currentfill}{rgb}{0.000000,0.000000,0.000000}%
\pgfsetfillcolor{currentfill}%
\pgfsetlinewidth{0.602250pt}%
\definecolor{currentstroke}{rgb}{0.000000,0.000000,0.000000}%
\pgfsetstrokecolor{currentstroke}%
\pgfsetdash{}{0pt}%
\pgfsys@defobject{currentmarker}{\pgfqpoint{-0.027778in}{0.000000in}}{\pgfqpoint{-0.000000in}{0.000000in}}{%
\pgfpathmoveto{\pgfqpoint{-0.000000in}{0.000000in}}%
\pgfpathlineto{\pgfqpoint{-0.027778in}{0.000000in}}%
\pgfusepath{stroke,fill}%
}%
\begin{pgfscope}%
\pgfsys@transformshift{0.588387in}{0.536361in}%
\pgfsys@useobject{currentmarker}{}%
\end{pgfscope}%
\end{pgfscope}%
\begin{pgfscope}%
\pgfsetbuttcap%
\pgfsetroundjoin%
\definecolor{currentfill}{rgb}{0.000000,0.000000,0.000000}%
\pgfsetfillcolor{currentfill}%
\pgfsetlinewidth{0.602250pt}%
\definecolor{currentstroke}{rgb}{0.000000,0.000000,0.000000}%
\pgfsetstrokecolor{currentstroke}%
\pgfsetdash{}{0pt}%
\pgfsys@defobject{currentmarker}{\pgfqpoint{-0.027778in}{0.000000in}}{\pgfqpoint{-0.000000in}{0.000000in}}{%
\pgfpathmoveto{\pgfqpoint{-0.000000in}{0.000000in}}%
\pgfpathlineto{\pgfqpoint{-0.027778in}{0.000000in}}%
\pgfusepath{stroke,fill}%
}%
\begin{pgfscope}%
\pgfsys@transformshift{0.588387in}{0.565046in}%
\pgfsys@useobject{currentmarker}{}%
\end{pgfscope}%
\end{pgfscope}%
\begin{pgfscope}%
\pgfsetbuttcap%
\pgfsetroundjoin%
\definecolor{currentfill}{rgb}{0.000000,0.000000,0.000000}%
\pgfsetfillcolor{currentfill}%
\pgfsetlinewidth{0.602250pt}%
\definecolor{currentstroke}{rgb}{0.000000,0.000000,0.000000}%
\pgfsetstrokecolor{currentstroke}%
\pgfsetdash{}{0pt}%
\pgfsys@defobject{currentmarker}{\pgfqpoint{-0.027778in}{0.000000in}}{\pgfqpoint{-0.000000in}{0.000000in}}{%
\pgfpathmoveto{\pgfqpoint{-0.000000in}{0.000000in}}%
\pgfpathlineto{\pgfqpoint{-0.027778in}{0.000000in}}%
\pgfusepath{stroke,fill}%
}%
\begin{pgfscope}%
\pgfsys@transformshift{0.588387in}{0.590347in}%
\pgfsys@useobject{currentmarker}{}%
\end{pgfscope}%
\end{pgfscope}%
\begin{pgfscope}%
\pgfsetbuttcap%
\pgfsetroundjoin%
\definecolor{currentfill}{rgb}{0.000000,0.000000,0.000000}%
\pgfsetfillcolor{currentfill}%
\pgfsetlinewidth{0.602250pt}%
\definecolor{currentstroke}{rgb}{0.000000,0.000000,0.000000}%
\pgfsetstrokecolor{currentstroke}%
\pgfsetdash{}{0pt}%
\pgfsys@defobject{currentmarker}{\pgfqpoint{-0.027778in}{0.000000in}}{\pgfqpoint{-0.000000in}{0.000000in}}{%
\pgfpathmoveto{\pgfqpoint{-0.000000in}{0.000000in}}%
\pgfpathlineto{\pgfqpoint{-0.027778in}{0.000000in}}%
\pgfusepath{stroke,fill}%
}%
\begin{pgfscope}%
\pgfsys@transformshift{0.588387in}{0.761878in}%
\pgfsys@useobject{currentmarker}{}%
\end{pgfscope}%
\end{pgfscope}%
\begin{pgfscope}%
\pgfsetbuttcap%
\pgfsetroundjoin%
\definecolor{currentfill}{rgb}{0.000000,0.000000,0.000000}%
\pgfsetfillcolor{currentfill}%
\pgfsetlinewidth{0.602250pt}%
\definecolor{currentstroke}{rgb}{0.000000,0.000000,0.000000}%
\pgfsetstrokecolor{currentstroke}%
\pgfsetdash{}{0pt}%
\pgfsys@defobject{currentmarker}{\pgfqpoint{-0.027778in}{0.000000in}}{\pgfqpoint{-0.000000in}{0.000000in}}{%
\pgfpathmoveto{\pgfqpoint{-0.000000in}{0.000000in}}%
\pgfpathlineto{\pgfqpoint{-0.027778in}{0.000000in}}%
\pgfusepath{stroke,fill}%
}%
\begin{pgfscope}%
\pgfsys@transformshift{0.588387in}{0.848977in}%
\pgfsys@useobject{currentmarker}{}%
\end{pgfscope}%
\end{pgfscope}%
\begin{pgfscope}%
\pgfsetbuttcap%
\pgfsetroundjoin%
\definecolor{currentfill}{rgb}{0.000000,0.000000,0.000000}%
\pgfsetfillcolor{currentfill}%
\pgfsetlinewidth{0.602250pt}%
\definecolor{currentstroke}{rgb}{0.000000,0.000000,0.000000}%
\pgfsetstrokecolor{currentstroke}%
\pgfsetdash{}{0pt}%
\pgfsys@defobject{currentmarker}{\pgfqpoint{-0.027778in}{0.000000in}}{\pgfqpoint{-0.000000in}{0.000000in}}{%
\pgfpathmoveto{\pgfqpoint{-0.000000in}{0.000000in}}%
\pgfpathlineto{\pgfqpoint{-0.027778in}{0.000000in}}%
\pgfusepath{stroke,fill}%
}%
\begin{pgfscope}%
\pgfsys@transformshift{0.588387in}{0.910775in}%
\pgfsys@useobject{currentmarker}{}%
\end{pgfscope}%
\end{pgfscope}%
\begin{pgfscope}%
\pgfsetbuttcap%
\pgfsetroundjoin%
\definecolor{currentfill}{rgb}{0.000000,0.000000,0.000000}%
\pgfsetfillcolor{currentfill}%
\pgfsetlinewidth{0.602250pt}%
\definecolor{currentstroke}{rgb}{0.000000,0.000000,0.000000}%
\pgfsetstrokecolor{currentstroke}%
\pgfsetdash{}{0pt}%
\pgfsys@defobject{currentmarker}{\pgfqpoint{-0.027778in}{0.000000in}}{\pgfqpoint{-0.000000in}{0.000000in}}{%
\pgfpathmoveto{\pgfqpoint{-0.000000in}{0.000000in}}%
\pgfpathlineto{\pgfqpoint{-0.027778in}{0.000000in}}%
\pgfusepath{stroke,fill}%
}%
\begin{pgfscope}%
\pgfsys@transformshift{0.588387in}{0.958710in}%
\pgfsys@useobject{currentmarker}{}%
\end{pgfscope}%
\end{pgfscope}%
\begin{pgfscope}%
\pgfsetbuttcap%
\pgfsetroundjoin%
\definecolor{currentfill}{rgb}{0.000000,0.000000,0.000000}%
\pgfsetfillcolor{currentfill}%
\pgfsetlinewidth{0.602250pt}%
\definecolor{currentstroke}{rgb}{0.000000,0.000000,0.000000}%
\pgfsetstrokecolor{currentstroke}%
\pgfsetdash{}{0pt}%
\pgfsys@defobject{currentmarker}{\pgfqpoint{-0.027778in}{0.000000in}}{\pgfqpoint{-0.000000in}{0.000000in}}{%
\pgfpathmoveto{\pgfqpoint{-0.000000in}{0.000000in}}%
\pgfpathlineto{\pgfqpoint{-0.027778in}{0.000000in}}%
\pgfusepath{stroke,fill}%
}%
\begin{pgfscope}%
\pgfsys@transformshift{0.588387in}{0.997875in}%
\pgfsys@useobject{currentmarker}{}%
\end{pgfscope}%
\end{pgfscope}%
\begin{pgfscope}%
\pgfsetbuttcap%
\pgfsetroundjoin%
\definecolor{currentfill}{rgb}{0.000000,0.000000,0.000000}%
\pgfsetfillcolor{currentfill}%
\pgfsetlinewidth{0.602250pt}%
\definecolor{currentstroke}{rgb}{0.000000,0.000000,0.000000}%
\pgfsetstrokecolor{currentstroke}%
\pgfsetdash{}{0pt}%
\pgfsys@defobject{currentmarker}{\pgfqpoint{-0.027778in}{0.000000in}}{\pgfqpoint{-0.000000in}{0.000000in}}{%
\pgfpathmoveto{\pgfqpoint{-0.000000in}{0.000000in}}%
\pgfpathlineto{\pgfqpoint{-0.027778in}{0.000000in}}%
\pgfusepath{stroke,fill}%
}%
\begin{pgfscope}%
\pgfsys@transformshift{0.588387in}{1.030989in}%
\pgfsys@useobject{currentmarker}{}%
\end{pgfscope}%
\end{pgfscope}%
\begin{pgfscope}%
\pgfsetbuttcap%
\pgfsetroundjoin%
\definecolor{currentfill}{rgb}{0.000000,0.000000,0.000000}%
\pgfsetfillcolor{currentfill}%
\pgfsetlinewidth{0.602250pt}%
\definecolor{currentstroke}{rgb}{0.000000,0.000000,0.000000}%
\pgfsetstrokecolor{currentstroke}%
\pgfsetdash{}{0pt}%
\pgfsys@defobject{currentmarker}{\pgfqpoint{-0.027778in}{0.000000in}}{\pgfqpoint{-0.000000in}{0.000000in}}{%
\pgfpathmoveto{\pgfqpoint{-0.000000in}{0.000000in}}%
\pgfpathlineto{\pgfqpoint{-0.027778in}{0.000000in}}%
\pgfusepath{stroke,fill}%
}%
\begin{pgfscope}%
\pgfsys@transformshift{0.588387in}{1.059673in}%
\pgfsys@useobject{currentmarker}{}%
\end{pgfscope}%
\end{pgfscope}%
\begin{pgfscope}%
\pgfsetbuttcap%
\pgfsetroundjoin%
\definecolor{currentfill}{rgb}{0.000000,0.000000,0.000000}%
\pgfsetfillcolor{currentfill}%
\pgfsetlinewidth{0.602250pt}%
\definecolor{currentstroke}{rgb}{0.000000,0.000000,0.000000}%
\pgfsetstrokecolor{currentstroke}%
\pgfsetdash{}{0pt}%
\pgfsys@defobject{currentmarker}{\pgfqpoint{-0.027778in}{0.000000in}}{\pgfqpoint{-0.000000in}{0.000000in}}{%
\pgfpathmoveto{\pgfqpoint{-0.000000in}{0.000000in}}%
\pgfpathlineto{\pgfqpoint{-0.027778in}{0.000000in}}%
\pgfusepath{stroke,fill}%
}%
\begin{pgfscope}%
\pgfsys@transformshift{0.588387in}{1.084974in}%
\pgfsys@useobject{currentmarker}{}%
\end{pgfscope}%
\end{pgfscope}%
\begin{pgfscope}%
\pgfsetbuttcap%
\pgfsetroundjoin%
\definecolor{currentfill}{rgb}{0.000000,0.000000,0.000000}%
\pgfsetfillcolor{currentfill}%
\pgfsetlinewidth{0.602250pt}%
\definecolor{currentstroke}{rgb}{0.000000,0.000000,0.000000}%
\pgfsetstrokecolor{currentstroke}%
\pgfsetdash{}{0pt}%
\pgfsys@defobject{currentmarker}{\pgfqpoint{-0.027778in}{0.000000in}}{\pgfqpoint{-0.000000in}{0.000000in}}{%
\pgfpathmoveto{\pgfqpoint{-0.000000in}{0.000000in}}%
\pgfpathlineto{\pgfqpoint{-0.027778in}{0.000000in}}%
\pgfusepath{stroke,fill}%
}%
\begin{pgfscope}%
\pgfsys@transformshift{0.588387in}{1.256505in}%
\pgfsys@useobject{currentmarker}{}%
\end{pgfscope}%
\end{pgfscope}%
\begin{pgfscope}%
\pgfsetbuttcap%
\pgfsetroundjoin%
\definecolor{currentfill}{rgb}{0.000000,0.000000,0.000000}%
\pgfsetfillcolor{currentfill}%
\pgfsetlinewidth{0.602250pt}%
\definecolor{currentstroke}{rgb}{0.000000,0.000000,0.000000}%
\pgfsetstrokecolor{currentstroke}%
\pgfsetdash{}{0pt}%
\pgfsys@defobject{currentmarker}{\pgfqpoint{-0.027778in}{0.000000in}}{\pgfqpoint{-0.000000in}{0.000000in}}{%
\pgfpathmoveto{\pgfqpoint{-0.000000in}{0.000000in}}%
\pgfpathlineto{\pgfqpoint{-0.027778in}{0.000000in}}%
\pgfusepath{stroke,fill}%
}%
\begin{pgfscope}%
\pgfsys@transformshift{0.588387in}{1.343605in}%
\pgfsys@useobject{currentmarker}{}%
\end{pgfscope}%
\end{pgfscope}%
\begin{pgfscope}%
\pgfsetbuttcap%
\pgfsetroundjoin%
\definecolor{currentfill}{rgb}{0.000000,0.000000,0.000000}%
\pgfsetfillcolor{currentfill}%
\pgfsetlinewidth{0.602250pt}%
\definecolor{currentstroke}{rgb}{0.000000,0.000000,0.000000}%
\pgfsetstrokecolor{currentstroke}%
\pgfsetdash{}{0pt}%
\pgfsys@defobject{currentmarker}{\pgfqpoint{-0.027778in}{0.000000in}}{\pgfqpoint{-0.000000in}{0.000000in}}{%
\pgfpathmoveto{\pgfqpoint{-0.000000in}{0.000000in}}%
\pgfpathlineto{\pgfqpoint{-0.027778in}{0.000000in}}%
\pgfusepath{stroke,fill}%
}%
\begin{pgfscope}%
\pgfsys@transformshift{0.588387in}{1.405403in}%
\pgfsys@useobject{currentmarker}{}%
\end{pgfscope}%
\end{pgfscope}%
\begin{pgfscope}%
\pgfsetbuttcap%
\pgfsetroundjoin%
\definecolor{currentfill}{rgb}{0.000000,0.000000,0.000000}%
\pgfsetfillcolor{currentfill}%
\pgfsetlinewidth{0.602250pt}%
\definecolor{currentstroke}{rgb}{0.000000,0.000000,0.000000}%
\pgfsetstrokecolor{currentstroke}%
\pgfsetdash{}{0pt}%
\pgfsys@defobject{currentmarker}{\pgfqpoint{-0.027778in}{0.000000in}}{\pgfqpoint{-0.000000in}{0.000000in}}{%
\pgfpathmoveto{\pgfqpoint{-0.000000in}{0.000000in}}%
\pgfpathlineto{\pgfqpoint{-0.027778in}{0.000000in}}%
\pgfusepath{stroke,fill}%
}%
\begin{pgfscope}%
\pgfsys@transformshift{0.588387in}{1.453337in}%
\pgfsys@useobject{currentmarker}{}%
\end{pgfscope}%
\end{pgfscope}%
\begin{pgfscope}%
\pgfsetbuttcap%
\pgfsetroundjoin%
\definecolor{currentfill}{rgb}{0.000000,0.000000,0.000000}%
\pgfsetfillcolor{currentfill}%
\pgfsetlinewidth{0.602250pt}%
\definecolor{currentstroke}{rgb}{0.000000,0.000000,0.000000}%
\pgfsetstrokecolor{currentstroke}%
\pgfsetdash{}{0pt}%
\pgfsys@defobject{currentmarker}{\pgfqpoint{-0.027778in}{0.000000in}}{\pgfqpoint{-0.000000in}{0.000000in}}{%
\pgfpathmoveto{\pgfqpoint{-0.000000in}{0.000000in}}%
\pgfpathlineto{\pgfqpoint{-0.027778in}{0.000000in}}%
\pgfusepath{stroke,fill}%
}%
\begin{pgfscope}%
\pgfsys@transformshift{0.588387in}{1.492502in}%
\pgfsys@useobject{currentmarker}{}%
\end{pgfscope}%
\end{pgfscope}%
\begin{pgfscope}%
\pgfsetbuttcap%
\pgfsetroundjoin%
\definecolor{currentfill}{rgb}{0.000000,0.000000,0.000000}%
\pgfsetfillcolor{currentfill}%
\pgfsetlinewidth{0.602250pt}%
\definecolor{currentstroke}{rgb}{0.000000,0.000000,0.000000}%
\pgfsetstrokecolor{currentstroke}%
\pgfsetdash{}{0pt}%
\pgfsys@defobject{currentmarker}{\pgfqpoint{-0.027778in}{0.000000in}}{\pgfqpoint{-0.000000in}{0.000000in}}{%
\pgfpathmoveto{\pgfqpoint{-0.000000in}{0.000000in}}%
\pgfpathlineto{\pgfqpoint{-0.027778in}{0.000000in}}%
\pgfusepath{stroke,fill}%
}%
\begin{pgfscope}%
\pgfsys@transformshift{0.588387in}{1.525616in}%
\pgfsys@useobject{currentmarker}{}%
\end{pgfscope}%
\end{pgfscope}%
\begin{pgfscope}%
\pgfsetbuttcap%
\pgfsetroundjoin%
\definecolor{currentfill}{rgb}{0.000000,0.000000,0.000000}%
\pgfsetfillcolor{currentfill}%
\pgfsetlinewidth{0.602250pt}%
\definecolor{currentstroke}{rgb}{0.000000,0.000000,0.000000}%
\pgfsetstrokecolor{currentstroke}%
\pgfsetdash{}{0pt}%
\pgfsys@defobject{currentmarker}{\pgfqpoint{-0.027778in}{0.000000in}}{\pgfqpoint{-0.000000in}{0.000000in}}{%
\pgfpathmoveto{\pgfqpoint{-0.000000in}{0.000000in}}%
\pgfpathlineto{\pgfqpoint{-0.027778in}{0.000000in}}%
\pgfusepath{stroke,fill}%
}%
\begin{pgfscope}%
\pgfsys@transformshift{0.588387in}{1.554300in}%
\pgfsys@useobject{currentmarker}{}%
\end{pgfscope}%
\end{pgfscope}%
\begin{pgfscope}%
\pgfsetbuttcap%
\pgfsetroundjoin%
\definecolor{currentfill}{rgb}{0.000000,0.000000,0.000000}%
\pgfsetfillcolor{currentfill}%
\pgfsetlinewidth{0.602250pt}%
\definecolor{currentstroke}{rgb}{0.000000,0.000000,0.000000}%
\pgfsetstrokecolor{currentstroke}%
\pgfsetdash{}{0pt}%
\pgfsys@defobject{currentmarker}{\pgfqpoint{-0.027778in}{0.000000in}}{\pgfqpoint{-0.000000in}{0.000000in}}{%
\pgfpathmoveto{\pgfqpoint{-0.000000in}{0.000000in}}%
\pgfpathlineto{\pgfqpoint{-0.027778in}{0.000000in}}%
\pgfusepath{stroke,fill}%
}%
\begin{pgfscope}%
\pgfsys@transformshift{0.588387in}{1.579602in}%
\pgfsys@useobject{currentmarker}{}%
\end{pgfscope}%
\end{pgfscope}%
\begin{pgfscope}%
\pgfsetbuttcap%
\pgfsetroundjoin%
\definecolor{currentfill}{rgb}{0.000000,0.000000,0.000000}%
\pgfsetfillcolor{currentfill}%
\pgfsetlinewidth{0.602250pt}%
\definecolor{currentstroke}{rgb}{0.000000,0.000000,0.000000}%
\pgfsetstrokecolor{currentstroke}%
\pgfsetdash{}{0pt}%
\pgfsys@defobject{currentmarker}{\pgfqpoint{-0.027778in}{0.000000in}}{\pgfqpoint{-0.000000in}{0.000000in}}{%
\pgfpathmoveto{\pgfqpoint{-0.000000in}{0.000000in}}%
\pgfpathlineto{\pgfqpoint{-0.027778in}{0.000000in}}%
\pgfusepath{stroke,fill}%
}%
\begin{pgfscope}%
\pgfsys@transformshift{0.588387in}{1.751132in}%
\pgfsys@useobject{currentmarker}{}%
\end{pgfscope}%
\end{pgfscope}%
\begin{pgfscope}%
\pgfsetbuttcap%
\pgfsetroundjoin%
\definecolor{currentfill}{rgb}{0.000000,0.000000,0.000000}%
\pgfsetfillcolor{currentfill}%
\pgfsetlinewidth{0.602250pt}%
\definecolor{currentstroke}{rgb}{0.000000,0.000000,0.000000}%
\pgfsetstrokecolor{currentstroke}%
\pgfsetdash{}{0pt}%
\pgfsys@defobject{currentmarker}{\pgfqpoint{-0.027778in}{0.000000in}}{\pgfqpoint{-0.000000in}{0.000000in}}{%
\pgfpathmoveto{\pgfqpoint{-0.000000in}{0.000000in}}%
\pgfpathlineto{\pgfqpoint{-0.027778in}{0.000000in}}%
\pgfusepath{stroke,fill}%
}%
\begin{pgfscope}%
\pgfsys@transformshift{0.588387in}{1.838232in}%
\pgfsys@useobject{currentmarker}{}%
\end{pgfscope}%
\end{pgfscope}%
\begin{pgfscope}%
\pgfsetbuttcap%
\pgfsetroundjoin%
\definecolor{currentfill}{rgb}{0.000000,0.000000,0.000000}%
\pgfsetfillcolor{currentfill}%
\pgfsetlinewidth{0.602250pt}%
\definecolor{currentstroke}{rgb}{0.000000,0.000000,0.000000}%
\pgfsetstrokecolor{currentstroke}%
\pgfsetdash{}{0pt}%
\pgfsys@defobject{currentmarker}{\pgfqpoint{-0.027778in}{0.000000in}}{\pgfqpoint{-0.000000in}{0.000000in}}{%
\pgfpathmoveto{\pgfqpoint{-0.000000in}{0.000000in}}%
\pgfpathlineto{\pgfqpoint{-0.027778in}{0.000000in}}%
\pgfusepath{stroke,fill}%
}%
\begin{pgfscope}%
\pgfsys@transformshift{0.588387in}{1.900030in}%
\pgfsys@useobject{currentmarker}{}%
\end{pgfscope}%
\end{pgfscope}%
\begin{pgfscope}%
\pgfsetbuttcap%
\pgfsetroundjoin%
\definecolor{currentfill}{rgb}{0.000000,0.000000,0.000000}%
\pgfsetfillcolor{currentfill}%
\pgfsetlinewidth{0.602250pt}%
\definecolor{currentstroke}{rgb}{0.000000,0.000000,0.000000}%
\pgfsetstrokecolor{currentstroke}%
\pgfsetdash{}{0pt}%
\pgfsys@defobject{currentmarker}{\pgfqpoint{-0.027778in}{0.000000in}}{\pgfqpoint{-0.000000in}{0.000000in}}{%
\pgfpathmoveto{\pgfqpoint{-0.000000in}{0.000000in}}%
\pgfpathlineto{\pgfqpoint{-0.027778in}{0.000000in}}%
\pgfusepath{stroke,fill}%
}%
\begin{pgfscope}%
\pgfsys@transformshift{0.588387in}{1.947964in}%
\pgfsys@useobject{currentmarker}{}%
\end{pgfscope}%
\end{pgfscope}%
\begin{pgfscope}%
\pgfsetbuttcap%
\pgfsetroundjoin%
\definecolor{currentfill}{rgb}{0.000000,0.000000,0.000000}%
\pgfsetfillcolor{currentfill}%
\pgfsetlinewidth{0.602250pt}%
\definecolor{currentstroke}{rgb}{0.000000,0.000000,0.000000}%
\pgfsetstrokecolor{currentstroke}%
\pgfsetdash{}{0pt}%
\pgfsys@defobject{currentmarker}{\pgfqpoint{-0.027778in}{0.000000in}}{\pgfqpoint{-0.000000in}{0.000000in}}{%
\pgfpathmoveto{\pgfqpoint{-0.000000in}{0.000000in}}%
\pgfpathlineto{\pgfqpoint{-0.027778in}{0.000000in}}%
\pgfusepath{stroke,fill}%
}%
\begin{pgfscope}%
\pgfsys@transformshift{0.588387in}{1.987130in}%
\pgfsys@useobject{currentmarker}{}%
\end{pgfscope}%
\end{pgfscope}%
\begin{pgfscope}%
\pgfsetbuttcap%
\pgfsetroundjoin%
\definecolor{currentfill}{rgb}{0.000000,0.000000,0.000000}%
\pgfsetfillcolor{currentfill}%
\pgfsetlinewidth{0.602250pt}%
\definecolor{currentstroke}{rgb}{0.000000,0.000000,0.000000}%
\pgfsetstrokecolor{currentstroke}%
\pgfsetdash{}{0pt}%
\pgfsys@defobject{currentmarker}{\pgfqpoint{-0.027778in}{0.000000in}}{\pgfqpoint{-0.000000in}{0.000000in}}{%
\pgfpathmoveto{\pgfqpoint{-0.000000in}{0.000000in}}%
\pgfpathlineto{\pgfqpoint{-0.027778in}{0.000000in}}%
\pgfusepath{stroke,fill}%
}%
\begin{pgfscope}%
\pgfsys@transformshift{0.588387in}{2.020243in}%
\pgfsys@useobject{currentmarker}{}%
\end{pgfscope}%
\end{pgfscope}%
\begin{pgfscope}%
\pgfsetbuttcap%
\pgfsetroundjoin%
\definecolor{currentfill}{rgb}{0.000000,0.000000,0.000000}%
\pgfsetfillcolor{currentfill}%
\pgfsetlinewidth{0.602250pt}%
\definecolor{currentstroke}{rgb}{0.000000,0.000000,0.000000}%
\pgfsetstrokecolor{currentstroke}%
\pgfsetdash{}{0pt}%
\pgfsys@defobject{currentmarker}{\pgfqpoint{-0.027778in}{0.000000in}}{\pgfqpoint{-0.000000in}{0.000000in}}{%
\pgfpathmoveto{\pgfqpoint{-0.000000in}{0.000000in}}%
\pgfpathlineto{\pgfqpoint{-0.027778in}{0.000000in}}%
\pgfusepath{stroke,fill}%
}%
\begin{pgfscope}%
\pgfsys@transformshift{0.588387in}{2.048928in}%
\pgfsys@useobject{currentmarker}{}%
\end{pgfscope}%
\end{pgfscope}%
\begin{pgfscope}%
\pgfsetbuttcap%
\pgfsetroundjoin%
\definecolor{currentfill}{rgb}{0.000000,0.000000,0.000000}%
\pgfsetfillcolor{currentfill}%
\pgfsetlinewidth{0.602250pt}%
\definecolor{currentstroke}{rgb}{0.000000,0.000000,0.000000}%
\pgfsetstrokecolor{currentstroke}%
\pgfsetdash{}{0pt}%
\pgfsys@defobject{currentmarker}{\pgfqpoint{-0.027778in}{0.000000in}}{\pgfqpoint{-0.000000in}{0.000000in}}{%
\pgfpathmoveto{\pgfqpoint{-0.000000in}{0.000000in}}%
\pgfpathlineto{\pgfqpoint{-0.027778in}{0.000000in}}%
\pgfusepath{stroke,fill}%
}%
\begin{pgfscope}%
\pgfsys@transformshift{0.588387in}{2.074229in}%
\pgfsys@useobject{currentmarker}{}%
\end{pgfscope}%
\end{pgfscope}%
\begin{pgfscope}%
\pgfsetbuttcap%
\pgfsetroundjoin%
\definecolor{currentfill}{rgb}{0.000000,0.000000,0.000000}%
\pgfsetfillcolor{currentfill}%
\pgfsetlinewidth{0.602250pt}%
\definecolor{currentstroke}{rgb}{0.000000,0.000000,0.000000}%
\pgfsetstrokecolor{currentstroke}%
\pgfsetdash{}{0pt}%
\pgfsys@defobject{currentmarker}{\pgfqpoint{-0.027778in}{0.000000in}}{\pgfqpoint{-0.000000in}{0.000000in}}{%
\pgfpathmoveto{\pgfqpoint{-0.000000in}{0.000000in}}%
\pgfpathlineto{\pgfqpoint{-0.027778in}{0.000000in}}%
\pgfusepath{stroke,fill}%
}%
\begin{pgfscope}%
\pgfsys@transformshift{0.588387in}{2.245760in}%
\pgfsys@useobject{currentmarker}{}%
\end{pgfscope}%
\end{pgfscope}%
\begin{pgfscope}%
\pgfsetbuttcap%
\pgfsetroundjoin%
\definecolor{currentfill}{rgb}{0.000000,0.000000,0.000000}%
\pgfsetfillcolor{currentfill}%
\pgfsetlinewidth{0.602250pt}%
\definecolor{currentstroke}{rgb}{0.000000,0.000000,0.000000}%
\pgfsetstrokecolor{currentstroke}%
\pgfsetdash{}{0pt}%
\pgfsys@defobject{currentmarker}{\pgfqpoint{-0.027778in}{0.000000in}}{\pgfqpoint{-0.000000in}{0.000000in}}{%
\pgfpathmoveto{\pgfqpoint{-0.000000in}{0.000000in}}%
\pgfpathlineto{\pgfqpoint{-0.027778in}{0.000000in}}%
\pgfusepath{stroke,fill}%
}%
\begin{pgfscope}%
\pgfsys@transformshift{0.588387in}{2.332859in}%
\pgfsys@useobject{currentmarker}{}%
\end{pgfscope}%
\end{pgfscope}%
\begin{pgfscope}%
\pgfsetbuttcap%
\pgfsetroundjoin%
\definecolor{currentfill}{rgb}{0.000000,0.000000,0.000000}%
\pgfsetfillcolor{currentfill}%
\pgfsetlinewidth{0.602250pt}%
\definecolor{currentstroke}{rgb}{0.000000,0.000000,0.000000}%
\pgfsetstrokecolor{currentstroke}%
\pgfsetdash{}{0pt}%
\pgfsys@defobject{currentmarker}{\pgfqpoint{-0.027778in}{0.000000in}}{\pgfqpoint{-0.000000in}{0.000000in}}{%
\pgfpathmoveto{\pgfqpoint{-0.000000in}{0.000000in}}%
\pgfpathlineto{\pgfqpoint{-0.027778in}{0.000000in}}%
\pgfusepath{stroke,fill}%
}%
\begin{pgfscope}%
\pgfsys@transformshift{0.588387in}{2.394657in}%
\pgfsys@useobject{currentmarker}{}%
\end{pgfscope}%
\end{pgfscope}%
\begin{pgfscope}%
\pgfsetbuttcap%
\pgfsetroundjoin%
\definecolor{currentfill}{rgb}{0.000000,0.000000,0.000000}%
\pgfsetfillcolor{currentfill}%
\pgfsetlinewidth{0.602250pt}%
\definecolor{currentstroke}{rgb}{0.000000,0.000000,0.000000}%
\pgfsetstrokecolor{currentstroke}%
\pgfsetdash{}{0pt}%
\pgfsys@defobject{currentmarker}{\pgfqpoint{-0.027778in}{0.000000in}}{\pgfqpoint{-0.000000in}{0.000000in}}{%
\pgfpathmoveto{\pgfqpoint{-0.000000in}{0.000000in}}%
\pgfpathlineto{\pgfqpoint{-0.027778in}{0.000000in}}%
\pgfusepath{stroke,fill}%
}%
\begin{pgfscope}%
\pgfsys@transformshift{0.588387in}{2.442592in}%
\pgfsys@useobject{currentmarker}{}%
\end{pgfscope}%
\end{pgfscope}%
\begin{pgfscope}%
\pgfsetbuttcap%
\pgfsetroundjoin%
\definecolor{currentfill}{rgb}{0.000000,0.000000,0.000000}%
\pgfsetfillcolor{currentfill}%
\pgfsetlinewidth{0.602250pt}%
\definecolor{currentstroke}{rgb}{0.000000,0.000000,0.000000}%
\pgfsetstrokecolor{currentstroke}%
\pgfsetdash{}{0pt}%
\pgfsys@defobject{currentmarker}{\pgfqpoint{-0.027778in}{0.000000in}}{\pgfqpoint{-0.000000in}{0.000000in}}{%
\pgfpathmoveto{\pgfqpoint{-0.000000in}{0.000000in}}%
\pgfpathlineto{\pgfqpoint{-0.027778in}{0.000000in}}%
\pgfusepath{stroke,fill}%
}%
\begin{pgfscope}%
\pgfsys@transformshift{0.588387in}{2.481757in}%
\pgfsys@useobject{currentmarker}{}%
\end{pgfscope}%
\end{pgfscope}%
\begin{pgfscope}%
\pgfsetbuttcap%
\pgfsetroundjoin%
\definecolor{currentfill}{rgb}{0.000000,0.000000,0.000000}%
\pgfsetfillcolor{currentfill}%
\pgfsetlinewidth{0.602250pt}%
\definecolor{currentstroke}{rgb}{0.000000,0.000000,0.000000}%
\pgfsetstrokecolor{currentstroke}%
\pgfsetdash{}{0pt}%
\pgfsys@defobject{currentmarker}{\pgfqpoint{-0.027778in}{0.000000in}}{\pgfqpoint{-0.000000in}{0.000000in}}{%
\pgfpathmoveto{\pgfqpoint{-0.000000in}{0.000000in}}%
\pgfpathlineto{\pgfqpoint{-0.027778in}{0.000000in}}%
\pgfusepath{stroke,fill}%
}%
\begin{pgfscope}%
\pgfsys@transformshift{0.588387in}{2.514871in}%
\pgfsys@useobject{currentmarker}{}%
\end{pgfscope}%
\end{pgfscope}%
\begin{pgfscope}%
\definecolor{textcolor}{rgb}{0.000000,0.000000,0.000000}%
\pgfsetstrokecolor{textcolor}%
\pgfsetfillcolor{textcolor}%
\pgftext[x=0.234413in,y=1.526746in,,bottom,rotate=90.000000]{\color{textcolor}{\rmfamily\fontsize{10.000000}{12.000000}\selectfont\catcode`\^=\active\def^{\ifmmode\sp\else\^{}\fi}\catcode`\%=\active\def%{\%}Time [ms]}}%
\end{pgfscope}%
\begin{pgfscope}%
\pgfpathrectangle{\pgfqpoint{0.588387in}{0.521603in}}{\pgfqpoint{4.669024in}{2.010285in}}%
\pgfusepath{clip}%
\pgfsetrectcap%
\pgfsetroundjoin%
\pgfsetlinewidth{1.505625pt}%
\pgfsetstrokecolor{currentstroke1}%
\pgfsetdash{}{0pt}%
\pgfpathmoveto{\pgfqpoint{0.800616in}{0.612980in}}%
\pgfpathlineto{\pgfqpoint{0.825151in}{0.612980in}}%
\pgfpathlineto{\pgfqpoint{0.874221in}{0.761878in}}%
\pgfpathlineto{\pgfqpoint{0.923291in}{0.848977in}}%
\pgfpathlineto{\pgfqpoint{0.972361in}{0.910775in}}%
\pgfpathlineto{\pgfqpoint{1.045966in}{0.997875in}}%
\pgfpathlineto{\pgfqpoint{1.070501in}{0.927970in}}%
\pgfpathlineto{\pgfqpoint{1.119572in}{1.055802in}}%
\pgfpathlineto{\pgfqpoint{1.193177in}{1.084974in}}%
\pgfpathlineto{\pgfqpoint{1.242247in}{1.146772in}}%
\pgfpathlineto{\pgfqpoint{1.315852in}{1.208571in}}%
\pgfpathlineto{\pgfqpoint{1.364922in}{1.229488in}}%
\pgfpathlineto{\pgfqpoint{1.438527in}{1.278369in}}%
\pgfpathlineto{\pgfqpoint{1.512133in}{1.319908in}}%
\pgfpathlineto{\pgfqpoint{1.610273in}{1.376718in}}%
\pgfpathlineto{\pgfqpoint{1.683878in}{1.420938in}}%
\pgfpathlineto{\pgfqpoint{1.757483in}{1.443071in}}%
\pgfpathlineto{\pgfqpoint{1.831089in}{1.492502in}}%
\pgfpathlineto{\pgfqpoint{1.904694in}{1.506845in}}%
\pgfpathlineto{\pgfqpoint{2.002834in}{1.548862in}}%
\pgfpathlineto{\pgfqpoint{2.076439in}{1.562208in}}%
\pgfpathlineto{\pgfqpoint{2.174580in}{1.596796in}}%
\pgfpathlineto{\pgfqpoint{2.272720in}{1.626579in}}%
\pgfpathlineto{\pgfqpoint{2.346325in}{1.652731in}}%
\pgfpathlineto{\pgfqpoint{2.444465in}{1.714954in}}%
\pgfpathlineto{\pgfqpoint{2.542606in}{1.712396in}}%
\pgfpathlineto{\pgfqpoint{2.640746in}{1.755912in}}%
\pgfpathlineto{\pgfqpoint{2.763421in}{1.772580in}}%
\pgfpathlineto{\pgfqpoint{2.812491in}{1.788048in}}%
\pgfpathlineto{\pgfqpoint{2.886097in}{1.804681in}}%
\pgfpathlineto{\pgfqpoint{3.008772in}{1.831541in}}%
\pgfpathlineto{\pgfqpoint{3.106912in}{1.857382in}}%
\pgfpathlineto{\pgfqpoint{3.229588in}{1.887309in}}%
\pgfpathlineto{\pgfqpoint{3.352263in}{1.904547in}}%
\pgfpathlineto{\pgfqpoint{3.474938in}{1.933297in}}%
\pgfpathlineto{\pgfqpoint{3.622149in}{1.963300in}}%
\pgfpathlineto{\pgfqpoint{3.720289in}{1.994173in}}%
\pgfpathlineto{\pgfqpoint{3.842964in}{2.021467in}}%
\pgfpathlineto{\pgfqpoint{3.965640in}{2.031308in}}%
\pgfpathlineto{\pgfqpoint{4.088315in}{2.064837in}}%
\pgfpathlineto{\pgfqpoint{4.235526in}{2.082884in}}%
\pgfpathlineto{\pgfqpoint{4.358201in}{2.110692in}}%
\pgfpathlineto{\pgfqpoint{4.505411in}{2.126885in}}%
\pgfpathlineto{\pgfqpoint{4.652622in}{2.157961in}}%
\pgfpathlineto{\pgfqpoint{4.799832in}{2.284118in}}%
\pgfpathlineto{\pgfqpoint{4.971578in}{2.204826in}}%
\pgfpathlineto{\pgfqpoint{4.996113in}{2.236607in}}%
\pgfpathlineto{\pgfqpoint{5.045183in}{2.222990in}}%
\pgfusepath{stroke}%
\end{pgfscope}%
\begin{pgfscope}%
\pgfpathrectangle{\pgfqpoint{0.588387in}{0.521603in}}{\pgfqpoint{4.669024in}{2.010285in}}%
\pgfusepath{clip}%
\pgfsetrectcap%
\pgfsetroundjoin%
\pgfsetlinewidth{1.505625pt}%
\pgfsetstrokecolor{currentstroke2}%
\pgfsetdash{}{0pt}%
\pgfpathmoveto{\pgfqpoint{0.800616in}{0.612980in}}%
\pgfpathlineto{\pgfqpoint{0.825151in}{0.612980in}}%
\pgfpathlineto{\pgfqpoint{0.874221in}{0.612980in}}%
\pgfpathlineto{\pgfqpoint{0.923291in}{0.848977in}}%
\pgfpathlineto{\pgfqpoint{0.972361in}{0.910775in}}%
\pgfpathlineto{\pgfqpoint{1.045966in}{0.997875in}}%
\pgfpathlineto{\pgfqpoint{1.070501in}{1.020508in}}%
\pgfpathlineto{\pgfqpoint{1.119572in}{1.538381in}}%
\pgfpathlineto{\pgfqpoint{1.193177in}{1.107607in}}%
\pgfpathlineto{\pgfqpoint{1.242247in}{2.057353in}}%
\pgfpathlineto{\pgfqpoint{1.315852in}{2.346185in}}%
\pgfpathlineto{\pgfqpoint{1.364922in}{1.891660in}}%
\pgfpathlineto{\pgfqpoint{1.438527in}{2.244860in}}%
\pgfpathlineto{\pgfqpoint{1.512133in}{2.231636in}}%
\pgfpathlineto{\pgfqpoint{1.831089in}{2.357310in}}%
\pgfpathlineto{\pgfqpoint{1.904694in}{2.404289in}}%
\pgfpathlineto{\pgfqpoint{2.076439in}{2.404164in}}%
\pgfpathlineto{\pgfqpoint{2.174580in}{2.404344in}}%
\pgfpathlineto{\pgfqpoint{2.346325in}{2.196883in}}%
\pgfpathlineto{\pgfqpoint{2.640746in}{2.315957in}}%
\pgfpathlineto{\pgfqpoint{2.812491in}{2.391492in}}%
\pgfpathlineto{\pgfqpoint{2.886097in}{2.416464in}}%
\pgfpathlineto{\pgfqpoint{3.008772in}{2.404072in}}%
\pgfpathlineto{\pgfqpoint{3.106912in}{2.420560in}}%
\pgfpathlineto{\pgfqpoint{3.229588in}{2.380089in}}%
\pgfpathlineto{\pgfqpoint{4.996113in}{2.366411in}}%
\pgfpathlineto{\pgfqpoint{5.045183in}{2.434149in}}%
\pgfusepath{stroke}%
\end{pgfscope}%
\begin{pgfscope}%
\pgfpathrectangle{\pgfqpoint{0.588387in}{0.521603in}}{\pgfqpoint{4.669024in}{2.010285in}}%
\pgfusepath{clip}%
\pgfsetrectcap%
\pgfsetroundjoin%
\pgfsetlinewidth{1.505625pt}%
\pgfsetstrokecolor{currentstroke3}%
\pgfsetdash{}{0pt}%
\pgfpathmoveto{\pgfqpoint{0.800616in}{0.722712in}}%
\pgfpathlineto{\pgfqpoint{0.825151in}{0.612980in}}%
\pgfpathlineto{\pgfqpoint{0.874221in}{0.674778in}}%
\pgfpathlineto{\pgfqpoint{0.923291in}{0.700079in}}%
\pgfpathlineto{\pgfqpoint{0.972361in}{0.910775in}}%
\pgfpathlineto{\pgfqpoint{1.045966in}{0.979184in}}%
\pgfpathlineto{\pgfqpoint{1.070501in}{0.991823in}}%
\pgfpathlineto{\pgfqpoint{1.119572in}{1.055802in}}%
\pgfpathlineto{\pgfqpoint{1.193177in}{1.128081in}}%
\pgfpathlineto{\pgfqpoint{1.242247in}{1.189879in}}%
\pgfpathlineto{\pgfqpoint{1.315852in}{1.208571in}}%
\pgfpathlineto{\pgfqpoint{1.364922in}{1.255363in}}%
\pgfpathlineto{\pgfqpoint{1.438527in}{1.282034in}}%
\pgfpathlineto{\pgfqpoint{1.512133in}{1.352641in}}%
\pgfpathlineto{\pgfqpoint{1.610273in}{1.376718in}}%
\pgfpathlineto{\pgfqpoint{1.683878in}{1.436976in}}%
\pgfpathlineto{\pgfqpoint{1.757483in}{1.448021in}}%
\pgfpathlineto{\pgfqpoint{1.831089in}{1.488081in}}%
\pgfpathlineto{\pgfqpoint{1.904694in}{1.503793in}}%
\pgfpathlineto{\pgfqpoint{2.002834in}{1.581975in}}%
\pgfpathlineto{\pgfqpoint{2.076439in}{1.563499in}}%
\pgfpathlineto{\pgfqpoint{2.174580in}{1.598260in}}%
\pgfpathlineto{\pgfqpoint{2.272720in}{1.631634in}}%
\pgfpathlineto{\pgfqpoint{2.346325in}{1.655124in}}%
\pgfpathlineto{\pgfqpoint{2.444465in}{1.685966in}}%
\pgfpathlineto{\pgfqpoint{2.542606in}{1.721217in}}%
\pgfpathlineto{\pgfqpoint{2.640746in}{1.742550in}}%
\pgfpathlineto{\pgfqpoint{2.763421in}{1.771281in}}%
\pgfpathlineto{\pgfqpoint{2.812491in}{1.781622in}}%
\pgfpathlineto{\pgfqpoint{2.886097in}{1.795438in}}%
\pgfpathlineto{\pgfqpoint{3.008772in}{1.820009in}}%
\pgfpathlineto{\pgfqpoint{3.106912in}{1.834311in}}%
\pgfpathlineto{\pgfqpoint{3.229588in}{1.857493in}}%
\pgfpathlineto{\pgfqpoint{3.352263in}{1.898683in}}%
\pgfpathlineto{\pgfqpoint{3.474938in}{1.904635in}}%
\pgfpathlineto{\pgfqpoint{3.622149in}{1.936644in}}%
\pgfpathlineto{\pgfqpoint{3.720289in}{1.961559in}}%
\pgfpathlineto{\pgfqpoint{3.842964in}{1.987010in}}%
\pgfpathlineto{\pgfqpoint{3.965640in}{2.044862in}}%
\pgfpathlineto{\pgfqpoint{4.088315in}{2.066576in}}%
\pgfpathlineto{\pgfqpoint{4.235526in}{2.092741in}}%
\pgfpathlineto{\pgfqpoint{4.358201in}{2.119183in}}%
\pgfpathlineto{\pgfqpoint{4.505411in}{2.146849in}}%
\pgfpathlineto{\pgfqpoint{4.652622in}{2.175192in}}%
\pgfpathlineto{\pgfqpoint{4.799832in}{2.199430in}}%
\pgfpathlineto{\pgfqpoint{4.971578in}{2.222529in}}%
\pgfpathlineto{\pgfqpoint{4.996113in}{2.193857in}}%
\pgfpathlineto{\pgfqpoint{5.045183in}{2.232669in}}%
\pgfusepath{stroke}%
\end{pgfscope}%
\begin{pgfscope}%
\pgfpathrectangle{\pgfqpoint{0.588387in}{0.521603in}}{\pgfqpoint{4.669024in}{2.010285in}}%
\pgfusepath{clip}%
\pgfsetrectcap%
\pgfsetroundjoin%
\pgfsetlinewidth{1.505625pt}%
\pgfsetstrokecolor{currentstroke4}%
\pgfsetdash{}{0pt}%
\pgfpathmoveto{\pgfqpoint{0.800616in}{0.612980in}}%
\pgfpathlineto{\pgfqpoint{0.825151in}{0.612980in}}%
\pgfpathlineto{\pgfqpoint{0.874221in}{0.700079in}}%
\pgfpathlineto{\pgfqpoint{0.923291in}{0.761878in}}%
\pgfpathlineto{\pgfqpoint{0.972361in}{0.958710in}}%
\pgfpathlineto{\pgfqpoint{1.045966in}{0.997875in}}%
\pgfpathlineto{\pgfqpoint{1.070501in}{1.009489in}}%
\pgfpathlineto{\pgfqpoint{1.119572in}{1.237255in}}%
\pgfpathlineto{\pgfqpoint{1.193177in}{1.442318in}}%
\pgfpathlineto{\pgfqpoint{1.242247in}{1.563499in}}%
\pgfpathlineto{\pgfqpoint{1.315852in}{1.266986in}}%
\pgfpathlineto{\pgfqpoint{1.364922in}{1.282023in}}%
\pgfpathlineto{\pgfqpoint{1.438527in}{1.351882in}}%
\pgfpathlineto{\pgfqpoint{1.512133in}{1.363427in}}%
\pgfpathlineto{\pgfqpoint{1.610273in}{2.009225in}}%
\pgfpathlineto{\pgfqpoint{1.683878in}{1.716221in}}%
\pgfpathlineto{\pgfqpoint{1.757483in}{1.507482in}}%
\pgfpathlineto{\pgfqpoint{1.831089in}{1.563927in}}%
\pgfpathlineto{\pgfqpoint{1.904694in}{1.560910in}}%
\pgfpathlineto{\pgfqpoint{2.002834in}{1.558291in}}%
\pgfpathlineto{\pgfqpoint{2.076439in}{1.583152in}}%
\pgfpathlineto{\pgfqpoint{2.174580in}{1.604903in}}%
\pgfpathlineto{\pgfqpoint{2.272720in}{1.648444in}}%
\pgfpathlineto{\pgfqpoint{2.346325in}{1.669272in}}%
\pgfpathlineto{\pgfqpoint{2.444465in}{1.734965in}}%
\pgfpathlineto{\pgfqpoint{2.542606in}{1.726702in}}%
\pgfpathlineto{\pgfqpoint{2.640746in}{1.910255in}}%
\pgfpathlineto{\pgfqpoint{2.763421in}{1.811989in}}%
\pgfpathlineto{\pgfqpoint{2.812491in}{1.962063in}}%
\pgfpathlineto{\pgfqpoint{2.886097in}{2.132048in}}%
\pgfpathlineto{\pgfqpoint{3.008772in}{1.914413in}}%
\pgfpathlineto{\pgfqpoint{3.106912in}{1.935717in}}%
\pgfpathlineto{\pgfqpoint{3.229588in}{1.963928in}}%
\pgfpathlineto{\pgfqpoint{3.352263in}{1.931913in}}%
\pgfpathlineto{\pgfqpoint{3.474938in}{1.937172in}}%
\pgfpathlineto{\pgfqpoint{3.622149in}{1.985151in}}%
\pgfpathlineto{\pgfqpoint{3.720289in}{1.992957in}}%
\pgfpathlineto{\pgfqpoint{3.842964in}{2.035779in}}%
\pgfpathlineto{\pgfqpoint{4.088315in}{2.102794in}}%
\pgfpathlineto{\pgfqpoint{4.358201in}{2.126136in}}%
\pgfpathlineto{\pgfqpoint{4.652622in}{2.178008in}}%
\pgfpathlineto{\pgfqpoint{4.971578in}{2.228896in}}%
\pgfpathlineto{\pgfqpoint{4.996113in}{2.242154in}}%
\pgfpathlineto{\pgfqpoint{5.045183in}{2.254068in}}%
\pgfusepath{stroke}%
\end{pgfscope}%
\begin{pgfscope}%
\pgfpathrectangle{\pgfqpoint{0.588387in}{0.521603in}}{\pgfqpoint{4.669024in}{2.010285in}}%
\pgfusepath{clip}%
\pgfsetrectcap%
\pgfsetroundjoin%
\pgfsetlinewidth{1.505625pt}%
\pgfsetstrokecolor{currentstroke5}%
\pgfsetdash{}{0pt}%
\pgfpathmoveto{\pgfqpoint{0.800616in}{0.612980in}}%
\pgfpathlineto{\pgfqpoint{0.825151in}{0.612980in}}%
\pgfpathlineto{\pgfqpoint{0.874221in}{0.612980in}}%
\pgfpathlineto{\pgfqpoint{0.923291in}{0.700079in}}%
\pgfpathlineto{\pgfqpoint{0.972361in}{0.910775in}}%
\pgfpathlineto{\pgfqpoint{1.045966in}{0.988732in}}%
\pgfpathlineto{\pgfqpoint{1.070501in}{0.969190in}}%
\pgfpathlineto{\pgfqpoint{1.119572in}{1.049862in}}%
\pgfpathlineto{\pgfqpoint{1.193177in}{1.132909in}}%
\pgfpathlineto{\pgfqpoint{1.242247in}{1.203476in}}%
\pgfpathlineto{\pgfqpoint{1.315852in}{1.201751in}}%
\pgfpathlineto{\pgfqpoint{1.364922in}{1.267218in}}%
\pgfpathlineto{\pgfqpoint{1.438527in}{1.281465in}}%
\pgfpathlineto{\pgfqpoint{1.512133in}{1.365592in}}%
\pgfpathlineto{\pgfqpoint{1.610273in}{1.391539in}}%
\pgfpathlineto{\pgfqpoint{1.683878in}{1.437748in}}%
\pgfpathlineto{\pgfqpoint{1.757483in}{1.443071in}}%
\pgfpathlineto{\pgfqpoint{1.831089in}{1.496639in}}%
\pgfpathlineto{\pgfqpoint{1.904694in}{1.501517in}}%
\pgfpathlineto{\pgfqpoint{2.002834in}{1.540437in}}%
\pgfpathlineto{\pgfqpoint{2.076439in}{1.565844in}}%
\pgfpathlineto{\pgfqpoint{2.174580in}{1.594024in}}%
\pgfpathlineto{\pgfqpoint{2.272720in}{1.631321in}}%
\pgfpathlineto{\pgfqpoint{2.346325in}{1.648444in}}%
\pgfpathlineto{\pgfqpoint{2.444465in}{1.684994in}}%
\pgfpathlineto{\pgfqpoint{2.542606in}{1.703198in}}%
\pgfpathlineto{\pgfqpoint{2.640746in}{1.718421in}}%
\pgfpathlineto{\pgfqpoint{2.763421in}{1.749515in}}%
\pgfpathlineto{\pgfqpoint{2.812491in}{1.761400in}}%
\pgfpathlineto{\pgfqpoint{2.886097in}{1.780153in}}%
\pgfpathlineto{\pgfqpoint{3.008772in}{1.811908in}}%
\pgfpathlineto{\pgfqpoint{3.106912in}{1.821468in}}%
\pgfpathlineto{\pgfqpoint{3.229588in}{1.838436in}}%
\pgfpathlineto{\pgfqpoint{3.352263in}{1.872875in}}%
\pgfpathlineto{\pgfqpoint{3.474938in}{1.887594in}}%
\pgfpathlineto{\pgfqpoint{3.622149in}{1.925093in}}%
\pgfpathlineto{\pgfqpoint{3.720289in}{1.944281in}}%
\pgfpathlineto{\pgfqpoint{3.842964in}{1.968243in}}%
\pgfusepath{stroke}%
\end{pgfscope}%
\begin{pgfscope}%
\pgfpathrectangle{\pgfqpoint{0.588387in}{0.521603in}}{\pgfqpoint{4.669024in}{2.010285in}}%
\pgfusepath{clip}%
\pgfsetrectcap%
\pgfsetroundjoin%
\pgfsetlinewidth{1.505625pt}%
\pgfsetstrokecolor{currentstroke6}%
\pgfsetdash{}{0pt}%
\pgfpathmoveto{\pgfqpoint{0.800616in}{0.700079in}}%
\pgfpathlineto{\pgfqpoint{0.825151in}{0.612980in}}%
\pgfpathlineto{\pgfqpoint{0.874221in}{0.674778in}}%
\pgfpathlineto{\pgfqpoint{0.923291in}{0.700079in}}%
\pgfpathlineto{\pgfqpoint{0.972361in}{0.882091in}}%
\pgfpathlineto{\pgfqpoint{1.045966in}{0.958710in}}%
\pgfpathlineto{\pgfqpoint{1.070501in}{0.958710in}}%
\pgfpathlineto{\pgfqpoint{1.119572in}{1.026559in}}%
\pgfpathlineto{\pgfqpoint{1.193177in}{1.102169in}}%
\pgfpathlineto{\pgfqpoint{1.242247in}{1.148259in}}%
\pgfpathlineto{\pgfqpoint{1.315852in}{1.184941in}}%
\pgfpathlineto{\pgfqpoint{1.364922in}{1.235493in}}%
\pgfpathlineto{\pgfqpoint{1.438527in}{1.254449in}}%
\pgfpathlineto{\pgfqpoint{1.512133in}{1.327072in}}%
\pgfpathlineto{\pgfqpoint{1.610273in}{1.350648in}}%
\pgfpathlineto{\pgfqpoint{1.683878in}{1.412446in}}%
\pgfpathlineto{\pgfqpoint{1.757483in}{1.422597in}}%
\pgfpathlineto{\pgfqpoint{1.831089in}{1.472068in}}%
\pgfpathlineto{\pgfqpoint{1.904694in}{1.479682in}}%
\pgfpathlineto{\pgfqpoint{2.002834in}{1.543282in}}%
\pgfpathlineto{\pgfqpoint{2.076439in}{1.544222in}}%
\pgfpathlineto{\pgfqpoint{2.174580in}{1.569419in}}%
\pgfpathlineto{\pgfqpoint{2.272720in}{1.621403in}}%
\pgfpathlineto{\pgfqpoint{2.346325in}{1.623359in}}%
\pgfpathlineto{\pgfqpoint{2.444465in}{1.650169in}}%
\pgfpathlineto{\pgfqpoint{2.542606in}{1.687416in}}%
\pgfpathlineto{\pgfqpoint{2.640746in}{1.716852in}}%
\pgfpathlineto{\pgfqpoint{2.763421in}{1.750774in}}%
\pgfpathlineto{\pgfqpoint{2.812491in}{1.773470in}}%
\pgfpathlineto{\pgfqpoint{2.886097in}{1.781349in}}%
\pgfpathlineto{\pgfqpoint{3.008772in}{1.801742in}}%
\pgfpathlineto{\pgfqpoint{3.106912in}{1.820356in}}%
\pgfpathlineto{\pgfqpoint{3.229588in}{1.833857in}}%
\pgfpathlineto{\pgfqpoint{3.352263in}{1.862789in}}%
\pgfpathlineto{\pgfqpoint{3.474938in}{1.904810in}}%
\pgfpathlineto{\pgfqpoint{3.622149in}{1.911022in}}%
\pgfpathlineto{\pgfqpoint{3.720289in}{1.940385in}}%
\pgfpathlineto{\pgfqpoint{3.842964in}{1.966608in}}%
\pgfpathlineto{\pgfqpoint{3.965640in}{2.027781in}}%
\pgfpathlineto{\pgfqpoint{4.088315in}{2.133053in}}%
\pgfpathlineto{\pgfqpoint{4.235526in}{2.084482in}}%
\pgfpathlineto{\pgfqpoint{4.358201in}{2.098999in}}%
\pgfpathlineto{\pgfqpoint{4.505411in}{2.168526in}}%
\pgfpathlineto{\pgfqpoint{4.652622in}{2.154375in}}%
\pgfpathlineto{\pgfqpoint{4.799832in}{2.181803in}}%
\pgfpathlineto{\pgfqpoint{4.971578in}{2.201948in}}%
\pgfpathlineto{\pgfqpoint{4.996113in}{2.181120in}}%
\pgfpathlineto{\pgfqpoint{5.045183in}{2.317788in}}%
\pgfusepath{stroke}%
\end{pgfscope}%
\begin{pgfscope}%
\pgfpathrectangle{\pgfqpoint{0.588387in}{0.521603in}}{\pgfqpoint{4.669024in}{2.010285in}}%
\pgfusepath{clip}%
\pgfsetrectcap%
\pgfsetroundjoin%
\pgfsetlinewidth{1.505625pt}%
\pgfsetstrokecolor{currentstroke7}%
\pgfsetdash{}{0pt}%
\pgfpathmoveto{\pgfqpoint{0.800616in}{0.612980in}}%
\pgfpathlineto{\pgfqpoint{0.825151in}{0.612980in}}%
\pgfpathlineto{\pgfqpoint{0.874221in}{0.612980in}}%
\pgfpathlineto{\pgfqpoint{0.923291in}{0.700079in}}%
\pgfpathlineto{\pgfqpoint{0.972361in}{0.958710in}}%
\pgfpathlineto{\pgfqpoint{1.045966in}{1.084974in}}%
\pgfpathlineto{\pgfqpoint{1.070501in}{1.223689in}}%
\pgfpathlineto{\pgfqpoint{1.119572in}{1.382770in}}%
\pgfpathlineto{\pgfqpoint{1.193177in}{1.580792in}}%
\pgfpathlineto{\pgfqpoint{1.242247in}{1.192909in}}%
\pgfpathlineto{\pgfqpoint{1.315852in}{1.905858in}}%
\pgfpathlineto{\pgfqpoint{1.364922in}{2.323640in}}%
\pgfpathlineto{\pgfqpoint{1.438527in}{2.229025in}}%
\pgfpathlineto{\pgfqpoint{1.512133in}{2.326903in}}%
\pgfpathlineto{\pgfqpoint{1.831089in}{2.408051in}}%
\pgfpathlineto{\pgfqpoint{2.812491in}{2.433936in}}%
\pgfpathlineto{\pgfqpoint{2.886097in}{2.439624in}}%
\pgfpathlineto{\pgfqpoint{3.106912in}{2.440512in}}%
\pgfpathlineto{\pgfqpoint{3.229588in}{2.427829in}}%
\pgfpathlineto{\pgfqpoint{4.996113in}{2.409557in}}%
\pgfusepath{stroke}%
\end{pgfscope}%
\begin{pgfscope}%
\pgfpathrectangle{\pgfqpoint{0.588387in}{0.521603in}}{\pgfqpoint{4.669024in}{2.010285in}}%
\pgfusepath{clip}%
\pgfsetrectcap%
\pgfsetroundjoin%
\pgfsetlinewidth{1.505625pt}%
\definecolor{currentstroke}{rgb}{0.498039,0.498039,0.498039}%
\pgfsetstrokecolor{currentstroke}%
\pgfsetdash{}{0pt}%
\pgfpathmoveto{\pgfqpoint{0.923291in}{0.612980in}}%
\pgfpathlineto{\pgfqpoint{0.972361in}{0.612980in}}%
\pgfpathlineto{\pgfqpoint{1.045966in}{0.761878in}}%
\pgfpathlineto{\pgfqpoint{1.070501in}{0.674778in}}%
\pgfpathlineto{\pgfqpoint{1.119572in}{0.761878in}}%
\pgfpathlineto{\pgfqpoint{1.193177in}{0.761878in}}%
\pgfpathlineto{\pgfqpoint{1.242247in}{0.848977in}}%
\pgfpathlineto{\pgfqpoint{1.315852in}{0.848977in}}%
\pgfpathlineto{\pgfqpoint{1.364922in}{0.864664in}}%
\pgfpathlineto{\pgfqpoint{1.438527in}{0.914577in}}%
\pgfpathlineto{\pgfqpoint{1.512133in}{0.949940in}}%
\pgfpathlineto{\pgfqpoint{1.610273in}{0.958710in}}%
\pgfpathlineto{\pgfqpoint{1.683878in}{1.030989in}}%
\pgfpathlineto{\pgfqpoint{1.757483in}{1.059673in}}%
\pgfpathlineto{\pgfqpoint{1.831089in}{1.059673in}}%
\pgfpathlineto{\pgfqpoint{1.904694in}{1.098198in}}%
\pgfpathlineto{\pgfqpoint{2.002834in}{1.128081in}}%
\pgfpathlineto{\pgfqpoint{2.076439in}{1.146772in}}%
\pgfpathlineto{\pgfqpoint{2.174580in}{1.179886in}}%
\pgfpathlineto{\pgfqpoint{2.272720in}{1.194707in}}%
\pgfpathlineto{\pgfqpoint{2.346325in}{1.215181in}}%
\pgfpathlineto{\pgfqpoint{2.444465in}{1.233872in}}%
\pgfpathlineto{\pgfqpoint{2.542606in}{1.266986in}}%
\pgfpathlineto{\pgfqpoint{2.640746in}{1.286528in}}%
\pgfpathlineto{\pgfqpoint{2.763421in}{1.312864in}}%
\pgfpathlineto{\pgfqpoint{2.812491in}{1.318973in}}%
\pgfpathlineto{\pgfqpoint{2.886097in}{1.328525in}}%
\pgfpathlineto{\pgfqpoint{3.008772in}{1.350648in}}%
\pgfpathlineto{\pgfqpoint{3.106912in}{1.369259in}}%
\pgfpathlineto{\pgfqpoint{3.229588in}{1.394384in}}%
\pgfpathlineto{\pgfqpoint{3.352263in}{1.415883in}}%
\pgfpathlineto{\pgfqpoint{3.474938in}{1.435425in}}%
\pgfpathlineto{\pgfqpoint{3.622149in}{1.457591in}}%
\pgfpathlineto{\pgfqpoint{3.720289in}{1.473811in}}%
\pgfpathlineto{\pgfqpoint{3.842964in}{1.492502in}}%
\pgfpathlineto{\pgfqpoint{3.965640in}{1.519389in}}%
\pgfpathlineto{\pgfqpoint{4.088315in}{1.533154in}}%
\pgfpathlineto{\pgfqpoint{4.235526in}{1.551598in}}%
\pgfpathlineto{\pgfqpoint{4.358201in}{1.571081in}}%
\pgfpathlineto{\pgfqpoint{4.505411in}{1.593465in}}%
\pgfpathlineto{\pgfqpoint{4.652622in}{1.610660in}}%
\pgfpathlineto{\pgfqpoint{4.799832in}{1.630381in}}%
\pgfpathlineto{\pgfqpoint{4.971578in}{1.650169in}}%
\pgfpathlineto{\pgfqpoint{4.996113in}{1.616482in}}%
\pgfpathlineto{\pgfqpoint{5.045183in}{1.653098in}}%
\pgfusepath{stroke}%
\end{pgfscope}%
\begin{pgfscope}%
\pgfsetrectcap%
\pgfsetmiterjoin%
\pgfsetlinewidth{0.803000pt}%
\definecolor{currentstroke}{rgb}{0.000000,0.000000,0.000000}%
\pgfsetstrokecolor{currentstroke}%
\pgfsetdash{}{0pt}%
\pgfpathmoveto{\pgfqpoint{0.588387in}{0.521603in}}%
\pgfpathlineto{\pgfqpoint{0.588387in}{2.531888in}}%
\pgfusepath{stroke}%
\end{pgfscope}%
\begin{pgfscope}%
\pgfsetrectcap%
\pgfsetmiterjoin%
\pgfsetlinewidth{0.803000pt}%
\definecolor{currentstroke}{rgb}{0.000000,0.000000,0.000000}%
\pgfsetstrokecolor{currentstroke}%
\pgfsetdash{}{0pt}%
\pgfpathmoveto{\pgfqpoint{5.257411in}{0.521603in}}%
\pgfpathlineto{\pgfqpoint{5.257411in}{2.531888in}}%
\pgfusepath{stroke}%
\end{pgfscope}%
\begin{pgfscope}%
\pgfsetrectcap%
\pgfsetmiterjoin%
\pgfsetlinewidth{0.803000pt}%
\definecolor{currentstroke}{rgb}{0.000000,0.000000,0.000000}%
\pgfsetstrokecolor{currentstroke}%
\pgfsetdash{}{0pt}%
\pgfpathmoveto{\pgfqpoint{0.588387in}{0.521603in}}%
\pgfpathlineto{\pgfqpoint{5.257411in}{0.521603in}}%
\pgfusepath{stroke}%
\end{pgfscope}%
\begin{pgfscope}%
\pgfsetrectcap%
\pgfsetmiterjoin%
\pgfsetlinewidth{0.803000pt}%
\definecolor{currentstroke}{rgb}{0.000000,0.000000,0.000000}%
\pgfsetstrokecolor{currentstroke}%
\pgfsetdash{}{0pt}%
\pgfpathmoveto{\pgfqpoint{0.588387in}{2.531888in}}%
\pgfpathlineto{\pgfqpoint{5.257411in}{2.531888in}}%
\pgfusepath{stroke}%
\end{pgfscope}%
\begin{pgfscope}%
\definecolor{textcolor}{rgb}{0.000000,0.000000,0.000000}%
\pgfsetstrokecolor{textcolor}%
\pgfsetfillcolor{textcolor}%
\pgftext[x=2.922899in,y=2.615222in,,base]{\color{textcolor}{\rmfamily\fontsize{12.000000}{14.400000}\selectfont\catcode`\^=\active\def^{\ifmmode\sp\else\^{}\fi}\catcode`\%=\active\def%{\%}Mean}}%
\end{pgfscope}%
\begin{pgfscope}%
\pgfsetbuttcap%
\pgfsetmiterjoin%
\definecolor{currentfill}{rgb}{1.000000,1.000000,1.000000}%
\pgfsetfillcolor{currentfill}%
\pgfsetfillopacity{0.800000}%
\pgfsetlinewidth{1.003750pt}%
\definecolor{currentstroke}{rgb}{0.800000,0.800000,0.800000}%
\pgfsetstrokecolor{currentstroke}%
\pgfsetstrokeopacity{0.800000}%
\pgfsetdash{}{0pt}%
\pgfpathmoveto{\pgfqpoint{5.344911in}{0.946722in}}%
\pgfpathlineto{\pgfqpoint{8.259376in}{0.946722in}}%
\pgfpathquadraticcurveto{\pgfqpoint{8.284376in}{0.946722in}}{\pgfqpoint{8.284376in}{0.971722in}}%
\pgfpathlineto{\pgfqpoint{8.284376in}{2.444388in}}%
\pgfpathquadraticcurveto{\pgfqpoint{8.284376in}{2.469388in}}{\pgfqpoint{8.259376in}{2.469388in}}%
\pgfpathlineto{\pgfqpoint{5.344911in}{2.469388in}}%
\pgfpathquadraticcurveto{\pgfqpoint{5.319911in}{2.469388in}}{\pgfqpoint{5.319911in}{2.444388in}}%
\pgfpathlineto{\pgfqpoint{5.319911in}{0.971722in}}%
\pgfpathquadraticcurveto{\pgfqpoint{5.319911in}{0.946722in}}{\pgfqpoint{5.344911in}{0.946722in}}%
\pgfpathlineto{\pgfqpoint{5.344911in}{0.946722in}}%
\pgfpathclose%
\pgfusepath{stroke,fill}%
\end{pgfscope}%
\begin{pgfscope}%
\pgfsetrectcap%
\pgfsetroundjoin%
\pgfsetlinewidth{1.505625pt}%
\definecolor{currentstroke}{rgb}{0.498039,0.498039,0.498039}%
\pgfsetstrokecolor{currentstroke}%
\pgfsetdash{}{0pt}%
\pgfpathmoveto{\pgfqpoint{5.369911in}{2.368168in}}%
\pgfpathlineto{\pgfqpoint{5.494911in}{2.368168in}}%
\pgfpathlineto{\pgfqpoint{5.619911in}{2.368168in}}%
\pgfusepath{stroke}%
\end{pgfscope}%
\begin{pgfscope}%
\definecolor{textcolor}{rgb}{0.000000,0.000000,0.000000}%
\pgfsetstrokecolor{textcolor}%
\pgfsetfillcolor{textcolor}%
\pgftext[x=5.719911in,y=2.324418in,left,base]{\color{textcolor}{\rmfamily\fontsize{9.000000}{10.800000}\selectfont\catcode`\^=\active\def^{\ifmmode\sp\else\^{}\fi}\catcode`\%=\active\def%{\%}\NaiveCycles{}}}%
\end{pgfscope}%
\begin{pgfscope}%
\pgfsetrectcap%
\pgfsetroundjoin%
\pgfsetlinewidth{1.505625pt}%
\pgfsetstrokecolor{currentstroke1}%
\pgfsetdash{}{0pt}%
\pgfpathmoveto{\pgfqpoint{5.369911in}{2.184696in}}%
\pgfpathlineto{\pgfqpoint{5.494911in}{2.184696in}}%
\pgfpathlineto{\pgfqpoint{5.619911in}{2.184696in}}%
\pgfusepath{stroke}%
\end{pgfscope}%
\begin{pgfscope}%
\definecolor{textcolor}{rgb}{0.000000,0.000000,0.000000}%
\pgfsetstrokecolor{textcolor}%
\pgfsetfillcolor{textcolor}%
\pgftext[x=5.719911in,y=2.140946in,left,base]{\color{textcolor}{\rmfamily\fontsize{9.000000}{10.800000}\selectfont\catcode`\^=\active\def^{\ifmmode\sp\else\^{}\fi}\catcode`\%=\active\def%{\%}\CyclesMatchChunks{} \& \MergeLinear{}}}%
\end{pgfscope}%
\begin{pgfscope}%
\pgfsetrectcap%
\pgfsetroundjoin%
\pgfsetlinewidth{1.505625pt}%
\pgfsetstrokecolor{currentstroke2}%
\pgfsetdash{}{0pt}%
\pgfpathmoveto{\pgfqpoint{5.369911in}{1.997746in}}%
\pgfpathlineto{\pgfqpoint{5.494911in}{1.997746in}}%
\pgfpathlineto{\pgfqpoint{5.619911in}{1.997746in}}%
\pgfusepath{stroke}%
\end{pgfscope}%
\begin{pgfscope}%
\definecolor{textcolor}{rgb}{0.000000,0.000000,0.000000}%
\pgfsetstrokecolor{textcolor}%
\pgfsetfillcolor{textcolor}%
\pgftext[x=5.719911in,y=1.953996in,left,base]{\color{textcolor}{\rmfamily\fontsize{9.000000}{10.800000}\selectfont\catcode`\^=\active\def^{\ifmmode\sp\else\^{}\fi}\catcode`\%=\active\def%{\%}\CyclesMatchChunks{} \& \SharedVertices{}}}%
\end{pgfscope}%
\begin{pgfscope}%
\pgfsetrectcap%
\pgfsetroundjoin%
\pgfsetlinewidth{1.505625pt}%
\pgfsetstrokecolor{currentstroke3}%
\pgfsetdash{}{0pt}%
\pgfpathmoveto{\pgfqpoint{5.369911in}{1.810795in}}%
\pgfpathlineto{\pgfqpoint{5.494911in}{1.810795in}}%
\pgfpathlineto{\pgfqpoint{5.619911in}{1.810795in}}%
\pgfusepath{stroke}%
\end{pgfscope}%
\begin{pgfscope}%
\definecolor{textcolor}{rgb}{0.000000,0.000000,0.000000}%
\pgfsetstrokecolor{textcolor}%
\pgfsetfillcolor{textcolor}%
\pgftext[x=5.719911in,y=1.767045in,left,base]{\color{textcolor}{\rmfamily\fontsize{9.000000}{10.800000}\selectfont\catcode`\^=\active\def^{\ifmmode\sp\else\^{}\fi}\catcode`\%=\active\def%{\%}\Neighbors{} \& \MergeLinear{}}}%
\end{pgfscope}%
\begin{pgfscope}%
\pgfsetrectcap%
\pgfsetroundjoin%
\pgfsetlinewidth{1.505625pt}%
\pgfsetstrokecolor{currentstroke4}%
\pgfsetdash{}{0pt}%
\pgfpathmoveto{\pgfqpoint{5.369911in}{1.627324in}}%
\pgfpathlineto{\pgfqpoint{5.494911in}{1.627324in}}%
\pgfpathlineto{\pgfqpoint{5.619911in}{1.627324in}}%
\pgfusepath{stroke}%
\end{pgfscope}%
\begin{pgfscope}%
\definecolor{textcolor}{rgb}{0.000000,0.000000,0.000000}%
\pgfsetstrokecolor{textcolor}%
\pgfsetfillcolor{textcolor}%
\pgftext[x=5.719911in,y=1.583574in,left,base]{\color{textcolor}{\rmfamily\fontsize{9.000000}{10.800000}\selectfont\catcode`\^=\active\def^{\ifmmode\sp\else\^{}\fi}\catcode`\%=\active\def%{\%}\Neighbors{} \& \SharedVertices{}}}%
\end{pgfscope}%
\begin{pgfscope}%
\pgfsetrectcap%
\pgfsetroundjoin%
\pgfsetlinewidth{1.505625pt}%
\pgfsetstrokecolor{currentstroke5}%
\pgfsetdash{}{0pt}%
\pgfpathmoveto{\pgfqpoint{5.369911in}{1.440373in}}%
\pgfpathlineto{\pgfqpoint{5.494911in}{1.440373in}}%
\pgfpathlineto{\pgfqpoint{5.619911in}{1.440373in}}%
\pgfusepath{stroke}%
\end{pgfscope}%
\begin{pgfscope}%
\definecolor{textcolor}{rgb}{0.000000,0.000000,0.000000}%
\pgfsetstrokecolor{textcolor}%
\pgfsetfillcolor{textcolor}%
\pgftext[x=5.719911in,y=1.396623in,left,base]{\color{textcolor}{\rmfamily\fontsize{9.000000}{10.800000}\selectfont\catcode`\^=\active\def^{\ifmmode\sp\else\^{}\fi}\catcode`\%=\active\def%{\%}\NeighborsDegree{} \& \MergeLinear{}}}%
\end{pgfscope}%
\begin{pgfscope}%
\pgfsetrectcap%
\pgfsetroundjoin%
\pgfsetlinewidth{1.505625pt}%
\pgfsetstrokecolor{currentstroke6}%
\pgfsetdash{}{0pt}%
\pgfpathmoveto{\pgfqpoint{5.369911in}{1.253423in}}%
\pgfpathlineto{\pgfqpoint{5.494911in}{1.253423in}}%
\pgfpathlineto{\pgfqpoint{5.619911in}{1.253423in}}%
\pgfusepath{stroke}%
\end{pgfscope}%
\begin{pgfscope}%
\definecolor{textcolor}{rgb}{0.000000,0.000000,0.000000}%
\pgfsetstrokecolor{textcolor}%
\pgfsetfillcolor{textcolor}%
\pgftext[x=5.719911in,y=1.209673in,left,base]{\color{textcolor}{\rmfamily\fontsize{9.000000}{10.800000}\selectfont\catcode`\^=\active\def^{\ifmmode\sp\else\^{}\fi}\catcode`\%=\active\def%{\%}\None{} \& \MergeLinear{}}}%
\end{pgfscope}%
\begin{pgfscope}%
\pgfsetrectcap%
\pgfsetroundjoin%
\pgfsetlinewidth{1.505625pt}%
\pgfsetstrokecolor{currentstroke7}%
\pgfsetdash{}{0pt}%
\pgfpathmoveto{\pgfqpoint{5.369911in}{1.069951in}}%
\pgfpathlineto{\pgfqpoint{5.494911in}{1.069951in}}%
\pgfpathlineto{\pgfqpoint{5.619911in}{1.069951in}}%
\pgfusepath{stroke}%
\end{pgfscope}%
\begin{pgfscope}%
\definecolor{textcolor}{rgb}{0.000000,0.000000,0.000000}%
\pgfsetstrokecolor{textcolor}%
\pgfsetfillcolor{textcolor}%
\pgftext[x=5.719911in,y=1.026201in,left,base]{\color{textcolor}{\rmfamily\fontsize{9.000000}{10.800000}\selectfont\catcode`\^=\active\def^{\ifmmode\sp\else\^{}\fi}\catcode`\%=\active\def%{\%}\None{} \& \SharedVertices{}}}%
\end{pgfscope}%
\end{pgfpicture}%
\makeatother%
\endgroup%
}
	\caption[Mean runtime for graphs with no 3 nor 4 cycles (some)]{
		Mean running time to find some NAC-coloring for graphs with no three nor four cycles.}%
	\label{fig:graph_count_no_3_nor_4_cycles_first_runtime}
\end{figure}%
% \begin{figure}[thbp]
% 	\centering
% 	\scalebox{\BenchFigureScale}{%% Creator: Matplotlib, PGF backend
%%
%% To include the figure in your LaTeX document, write
%%   \input{<filename>.pgf}
%%
%% Make sure the required packages are loaded in your preamble
%%   \usepackage{pgf}
%%
%% Also ensure that all the required font packages are loaded; for instance,
%% the lmodern package is sometimes necessary when using math font.
%%   \usepackage{lmodern}
%%
%% Figures using additional raster images can only be included by \input if
%% they are in the same directory as the main LaTeX file. For loading figures
%% from other directories you can use the `import` package
%%   \usepackage{import}
%%
%% and then include the figures with
%%   \import{<path to file>}{<filename>.pgf}
%%
%% Matplotlib used the following preamble
%%   \def\mathdefault#1{#1}
%%   \everymath=\expandafter{\the\everymath\displaystyle}
%%   \IfFileExists{scrextend.sty}{
%%     \usepackage[fontsize=10.000000pt]{scrextend}
%%   }{
%%     \renewcommand{\normalsize}{\fontsize{10.000000}{12.000000}\selectfont}
%%     \normalsize
%%   }
%%   
%%   \ifdefined\pdftexversion\else  % non-pdftex case.
%%     \usepackage{fontspec}
%%     \setmainfont{DejaVuSans.ttf}[Path=\detokenize{/home/petr/Projects/PyRigi/.venv/lib/python3.12/site-packages/matplotlib/mpl-data/fonts/ttf/}]
%%     \setsansfont{DejaVuSans.ttf}[Path=\detokenize{/home/petr/Projects/PyRigi/.venv/lib/python3.12/site-packages/matplotlib/mpl-data/fonts/ttf/}]
%%     \setmonofont{DejaVuSansMono.ttf}[Path=\detokenize{/home/petr/Projects/PyRigi/.venv/lib/python3.12/site-packages/matplotlib/mpl-data/fonts/ttf/}]
%%   \fi
%%   \makeatletter\@ifpackageloaded{underscore}{}{\usepackage[strings]{underscore}}\makeatother
%%
\begingroup%
\makeatletter%
\begin{pgfpicture}%
\pgfpathrectangle{\pgfpointorigin}{\pgfqpoint{8.384376in}{2.841849in}}%
\pgfusepath{use as bounding box, clip}%
\begin{pgfscope}%
\pgfsetbuttcap%
\pgfsetmiterjoin%
\definecolor{currentfill}{rgb}{1.000000,1.000000,1.000000}%
\pgfsetfillcolor{currentfill}%
\pgfsetlinewidth{0.000000pt}%
\definecolor{currentstroke}{rgb}{1.000000,1.000000,1.000000}%
\pgfsetstrokecolor{currentstroke}%
\pgfsetdash{}{0pt}%
\pgfpathmoveto{\pgfqpoint{0.000000in}{0.000000in}}%
\pgfpathlineto{\pgfqpoint{8.384376in}{0.000000in}}%
\pgfpathlineto{\pgfqpoint{8.384376in}{2.841849in}}%
\pgfpathlineto{\pgfqpoint{0.000000in}{2.841849in}}%
\pgfpathlineto{\pgfqpoint{0.000000in}{0.000000in}}%
\pgfpathclose%
\pgfusepath{fill}%
\end{pgfscope}%
\begin{pgfscope}%
\pgfsetbuttcap%
\pgfsetmiterjoin%
\definecolor{currentfill}{rgb}{1.000000,1.000000,1.000000}%
\pgfsetfillcolor{currentfill}%
\pgfsetlinewidth{0.000000pt}%
\definecolor{currentstroke}{rgb}{0.000000,0.000000,0.000000}%
\pgfsetstrokecolor{currentstroke}%
\pgfsetstrokeopacity{0.000000}%
\pgfsetdash{}{0pt}%
\pgfpathmoveto{\pgfqpoint{0.588387in}{0.521603in}}%
\pgfpathlineto{\pgfqpoint{5.257411in}{0.521603in}}%
\pgfpathlineto{\pgfqpoint{5.257411in}{2.531888in}}%
\pgfpathlineto{\pgfqpoint{0.588387in}{2.531888in}}%
\pgfpathlineto{\pgfqpoint{0.588387in}{0.521603in}}%
\pgfpathclose%
\pgfusepath{fill}%
\end{pgfscope}%
\begin{pgfscope}%
\pgfsetbuttcap%
\pgfsetroundjoin%
\definecolor{currentfill}{rgb}{0.000000,0.000000,0.000000}%
\pgfsetfillcolor{currentfill}%
\pgfsetlinewidth{0.803000pt}%
\definecolor{currentstroke}{rgb}{0.000000,0.000000,0.000000}%
\pgfsetstrokecolor{currentstroke}%
\pgfsetdash{}{0pt}%
\pgfsys@defobject{currentmarker}{\pgfqpoint{0.000000in}{-0.048611in}}{\pgfqpoint{0.000000in}{0.000000in}}{%
\pgfpathmoveto{\pgfqpoint{0.000000in}{0.000000in}}%
\pgfpathlineto{\pgfqpoint{0.000000in}{-0.048611in}}%
\pgfusepath{stroke,fill}%
}%
\begin{pgfscope}%
\pgfsys@transformshift{0.677940in}{0.521603in}%
\pgfsys@useobject{currentmarker}{}%
\end{pgfscope}%
\end{pgfscope}%
\begin{pgfscope}%
\definecolor{textcolor}{rgb}{0.000000,0.000000,0.000000}%
\pgfsetstrokecolor{textcolor}%
\pgfsetfillcolor{textcolor}%
\pgftext[x=0.677940in,y=0.424381in,,top]{\color{textcolor}{\rmfamily\fontsize{10.000000}{12.000000}\selectfont\catcode`\^=\active\def^{\ifmmode\sp\else\^{}\fi}\catcode`\%=\active\def%{\%}$\mathdefault{0}$}}%
\end{pgfscope}%
\begin{pgfscope}%
\pgfsetbuttcap%
\pgfsetroundjoin%
\definecolor{currentfill}{rgb}{0.000000,0.000000,0.000000}%
\pgfsetfillcolor{currentfill}%
\pgfsetlinewidth{0.803000pt}%
\definecolor{currentstroke}{rgb}{0.000000,0.000000,0.000000}%
\pgfsetstrokecolor{currentstroke}%
\pgfsetdash{}{0pt}%
\pgfsys@defobject{currentmarker}{\pgfqpoint{0.000000in}{-0.048611in}}{\pgfqpoint{0.000000in}{0.000000in}}{%
\pgfpathmoveto{\pgfqpoint{0.000000in}{0.000000in}}%
\pgfpathlineto{\pgfqpoint{0.000000in}{-0.048611in}}%
\pgfusepath{stroke,fill}%
}%
\begin{pgfscope}%
\pgfsys@transformshift{1.168642in}{0.521603in}%
\pgfsys@useobject{currentmarker}{}%
\end{pgfscope}%
\end{pgfscope}%
\begin{pgfscope}%
\definecolor{textcolor}{rgb}{0.000000,0.000000,0.000000}%
\pgfsetstrokecolor{textcolor}%
\pgfsetfillcolor{textcolor}%
\pgftext[x=1.168642in,y=0.424381in,,top]{\color{textcolor}{\rmfamily\fontsize{10.000000}{12.000000}\selectfont\catcode`\^=\active\def^{\ifmmode\sp\else\^{}\fi}\catcode`\%=\active\def%{\%}$\mathdefault{20}$}}%
\end{pgfscope}%
\begin{pgfscope}%
\pgfsetbuttcap%
\pgfsetroundjoin%
\definecolor{currentfill}{rgb}{0.000000,0.000000,0.000000}%
\pgfsetfillcolor{currentfill}%
\pgfsetlinewidth{0.803000pt}%
\definecolor{currentstroke}{rgb}{0.000000,0.000000,0.000000}%
\pgfsetstrokecolor{currentstroke}%
\pgfsetdash{}{0pt}%
\pgfsys@defobject{currentmarker}{\pgfqpoint{0.000000in}{-0.048611in}}{\pgfqpoint{0.000000in}{0.000000in}}{%
\pgfpathmoveto{\pgfqpoint{0.000000in}{0.000000in}}%
\pgfpathlineto{\pgfqpoint{0.000000in}{-0.048611in}}%
\pgfusepath{stroke,fill}%
}%
\begin{pgfscope}%
\pgfsys@transformshift{1.659343in}{0.521603in}%
\pgfsys@useobject{currentmarker}{}%
\end{pgfscope}%
\end{pgfscope}%
\begin{pgfscope}%
\definecolor{textcolor}{rgb}{0.000000,0.000000,0.000000}%
\pgfsetstrokecolor{textcolor}%
\pgfsetfillcolor{textcolor}%
\pgftext[x=1.659343in,y=0.424381in,,top]{\color{textcolor}{\rmfamily\fontsize{10.000000}{12.000000}\selectfont\catcode`\^=\active\def^{\ifmmode\sp\else\^{}\fi}\catcode`\%=\active\def%{\%}$\mathdefault{40}$}}%
\end{pgfscope}%
\begin{pgfscope}%
\pgfsetbuttcap%
\pgfsetroundjoin%
\definecolor{currentfill}{rgb}{0.000000,0.000000,0.000000}%
\pgfsetfillcolor{currentfill}%
\pgfsetlinewidth{0.803000pt}%
\definecolor{currentstroke}{rgb}{0.000000,0.000000,0.000000}%
\pgfsetstrokecolor{currentstroke}%
\pgfsetdash{}{0pt}%
\pgfsys@defobject{currentmarker}{\pgfqpoint{0.000000in}{-0.048611in}}{\pgfqpoint{0.000000in}{0.000000in}}{%
\pgfpathmoveto{\pgfqpoint{0.000000in}{0.000000in}}%
\pgfpathlineto{\pgfqpoint{0.000000in}{-0.048611in}}%
\pgfusepath{stroke,fill}%
}%
\begin{pgfscope}%
\pgfsys@transformshift{2.150044in}{0.521603in}%
\pgfsys@useobject{currentmarker}{}%
\end{pgfscope}%
\end{pgfscope}%
\begin{pgfscope}%
\definecolor{textcolor}{rgb}{0.000000,0.000000,0.000000}%
\pgfsetstrokecolor{textcolor}%
\pgfsetfillcolor{textcolor}%
\pgftext[x=2.150044in,y=0.424381in,,top]{\color{textcolor}{\rmfamily\fontsize{10.000000}{12.000000}\selectfont\catcode`\^=\active\def^{\ifmmode\sp\else\^{}\fi}\catcode`\%=\active\def%{\%}$\mathdefault{60}$}}%
\end{pgfscope}%
\begin{pgfscope}%
\pgfsetbuttcap%
\pgfsetroundjoin%
\definecolor{currentfill}{rgb}{0.000000,0.000000,0.000000}%
\pgfsetfillcolor{currentfill}%
\pgfsetlinewidth{0.803000pt}%
\definecolor{currentstroke}{rgb}{0.000000,0.000000,0.000000}%
\pgfsetstrokecolor{currentstroke}%
\pgfsetdash{}{0pt}%
\pgfsys@defobject{currentmarker}{\pgfqpoint{0.000000in}{-0.048611in}}{\pgfqpoint{0.000000in}{0.000000in}}{%
\pgfpathmoveto{\pgfqpoint{0.000000in}{0.000000in}}%
\pgfpathlineto{\pgfqpoint{0.000000in}{-0.048611in}}%
\pgfusepath{stroke,fill}%
}%
\begin{pgfscope}%
\pgfsys@transformshift{2.640746in}{0.521603in}%
\pgfsys@useobject{currentmarker}{}%
\end{pgfscope}%
\end{pgfscope}%
\begin{pgfscope}%
\definecolor{textcolor}{rgb}{0.000000,0.000000,0.000000}%
\pgfsetstrokecolor{textcolor}%
\pgfsetfillcolor{textcolor}%
\pgftext[x=2.640746in,y=0.424381in,,top]{\color{textcolor}{\rmfamily\fontsize{10.000000}{12.000000}\selectfont\catcode`\^=\active\def^{\ifmmode\sp\else\^{}\fi}\catcode`\%=\active\def%{\%}$\mathdefault{80}$}}%
\end{pgfscope}%
\begin{pgfscope}%
\pgfsetbuttcap%
\pgfsetroundjoin%
\definecolor{currentfill}{rgb}{0.000000,0.000000,0.000000}%
\pgfsetfillcolor{currentfill}%
\pgfsetlinewidth{0.803000pt}%
\definecolor{currentstroke}{rgb}{0.000000,0.000000,0.000000}%
\pgfsetstrokecolor{currentstroke}%
\pgfsetdash{}{0pt}%
\pgfsys@defobject{currentmarker}{\pgfqpoint{0.000000in}{-0.048611in}}{\pgfqpoint{0.000000in}{0.000000in}}{%
\pgfpathmoveto{\pgfqpoint{0.000000in}{0.000000in}}%
\pgfpathlineto{\pgfqpoint{0.000000in}{-0.048611in}}%
\pgfusepath{stroke,fill}%
}%
\begin{pgfscope}%
\pgfsys@transformshift{3.131447in}{0.521603in}%
\pgfsys@useobject{currentmarker}{}%
\end{pgfscope}%
\end{pgfscope}%
\begin{pgfscope}%
\definecolor{textcolor}{rgb}{0.000000,0.000000,0.000000}%
\pgfsetstrokecolor{textcolor}%
\pgfsetfillcolor{textcolor}%
\pgftext[x=3.131447in,y=0.424381in,,top]{\color{textcolor}{\rmfamily\fontsize{10.000000}{12.000000}\selectfont\catcode`\^=\active\def^{\ifmmode\sp\else\^{}\fi}\catcode`\%=\active\def%{\%}$\mathdefault{100}$}}%
\end{pgfscope}%
\begin{pgfscope}%
\pgfsetbuttcap%
\pgfsetroundjoin%
\definecolor{currentfill}{rgb}{0.000000,0.000000,0.000000}%
\pgfsetfillcolor{currentfill}%
\pgfsetlinewidth{0.803000pt}%
\definecolor{currentstroke}{rgb}{0.000000,0.000000,0.000000}%
\pgfsetstrokecolor{currentstroke}%
\pgfsetdash{}{0pt}%
\pgfsys@defobject{currentmarker}{\pgfqpoint{0.000000in}{-0.048611in}}{\pgfqpoint{0.000000in}{0.000000in}}{%
\pgfpathmoveto{\pgfqpoint{0.000000in}{0.000000in}}%
\pgfpathlineto{\pgfqpoint{0.000000in}{-0.048611in}}%
\pgfusepath{stroke,fill}%
}%
\begin{pgfscope}%
\pgfsys@transformshift{3.622149in}{0.521603in}%
\pgfsys@useobject{currentmarker}{}%
\end{pgfscope}%
\end{pgfscope}%
\begin{pgfscope}%
\definecolor{textcolor}{rgb}{0.000000,0.000000,0.000000}%
\pgfsetstrokecolor{textcolor}%
\pgfsetfillcolor{textcolor}%
\pgftext[x=3.622149in,y=0.424381in,,top]{\color{textcolor}{\rmfamily\fontsize{10.000000}{12.000000}\selectfont\catcode`\^=\active\def^{\ifmmode\sp\else\^{}\fi}\catcode`\%=\active\def%{\%}$\mathdefault{120}$}}%
\end{pgfscope}%
\begin{pgfscope}%
\pgfsetbuttcap%
\pgfsetroundjoin%
\definecolor{currentfill}{rgb}{0.000000,0.000000,0.000000}%
\pgfsetfillcolor{currentfill}%
\pgfsetlinewidth{0.803000pt}%
\definecolor{currentstroke}{rgb}{0.000000,0.000000,0.000000}%
\pgfsetstrokecolor{currentstroke}%
\pgfsetdash{}{0pt}%
\pgfsys@defobject{currentmarker}{\pgfqpoint{0.000000in}{-0.048611in}}{\pgfqpoint{0.000000in}{0.000000in}}{%
\pgfpathmoveto{\pgfqpoint{0.000000in}{0.000000in}}%
\pgfpathlineto{\pgfqpoint{0.000000in}{-0.048611in}}%
\pgfusepath{stroke,fill}%
}%
\begin{pgfscope}%
\pgfsys@transformshift{4.112850in}{0.521603in}%
\pgfsys@useobject{currentmarker}{}%
\end{pgfscope}%
\end{pgfscope}%
\begin{pgfscope}%
\definecolor{textcolor}{rgb}{0.000000,0.000000,0.000000}%
\pgfsetstrokecolor{textcolor}%
\pgfsetfillcolor{textcolor}%
\pgftext[x=4.112850in,y=0.424381in,,top]{\color{textcolor}{\rmfamily\fontsize{10.000000}{12.000000}\selectfont\catcode`\^=\active\def^{\ifmmode\sp\else\^{}\fi}\catcode`\%=\active\def%{\%}$\mathdefault{140}$}}%
\end{pgfscope}%
\begin{pgfscope}%
\pgfsetbuttcap%
\pgfsetroundjoin%
\definecolor{currentfill}{rgb}{0.000000,0.000000,0.000000}%
\pgfsetfillcolor{currentfill}%
\pgfsetlinewidth{0.803000pt}%
\definecolor{currentstroke}{rgb}{0.000000,0.000000,0.000000}%
\pgfsetstrokecolor{currentstroke}%
\pgfsetdash{}{0pt}%
\pgfsys@defobject{currentmarker}{\pgfqpoint{0.000000in}{-0.048611in}}{\pgfqpoint{0.000000in}{0.000000in}}{%
\pgfpathmoveto{\pgfqpoint{0.000000in}{0.000000in}}%
\pgfpathlineto{\pgfqpoint{0.000000in}{-0.048611in}}%
\pgfusepath{stroke,fill}%
}%
\begin{pgfscope}%
\pgfsys@transformshift{4.603552in}{0.521603in}%
\pgfsys@useobject{currentmarker}{}%
\end{pgfscope}%
\end{pgfscope}%
\begin{pgfscope}%
\definecolor{textcolor}{rgb}{0.000000,0.000000,0.000000}%
\pgfsetstrokecolor{textcolor}%
\pgfsetfillcolor{textcolor}%
\pgftext[x=4.603552in,y=0.424381in,,top]{\color{textcolor}{\rmfamily\fontsize{10.000000}{12.000000}\selectfont\catcode`\^=\active\def^{\ifmmode\sp\else\^{}\fi}\catcode`\%=\active\def%{\%}$\mathdefault{160}$}}%
\end{pgfscope}%
\begin{pgfscope}%
\pgfsetbuttcap%
\pgfsetroundjoin%
\definecolor{currentfill}{rgb}{0.000000,0.000000,0.000000}%
\pgfsetfillcolor{currentfill}%
\pgfsetlinewidth{0.803000pt}%
\definecolor{currentstroke}{rgb}{0.000000,0.000000,0.000000}%
\pgfsetstrokecolor{currentstroke}%
\pgfsetdash{}{0pt}%
\pgfsys@defobject{currentmarker}{\pgfqpoint{0.000000in}{-0.048611in}}{\pgfqpoint{0.000000in}{0.000000in}}{%
\pgfpathmoveto{\pgfqpoint{0.000000in}{0.000000in}}%
\pgfpathlineto{\pgfqpoint{0.000000in}{-0.048611in}}%
\pgfusepath{stroke,fill}%
}%
\begin{pgfscope}%
\pgfsys@transformshift{5.094253in}{0.521603in}%
\pgfsys@useobject{currentmarker}{}%
\end{pgfscope}%
\end{pgfscope}%
\begin{pgfscope}%
\definecolor{textcolor}{rgb}{0.000000,0.000000,0.000000}%
\pgfsetstrokecolor{textcolor}%
\pgfsetfillcolor{textcolor}%
\pgftext[x=5.094253in,y=0.424381in,,top]{\color{textcolor}{\rmfamily\fontsize{10.000000}{12.000000}\selectfont\catcode`\^=\active\def^{\ifmmode\sp\else\^{}\fi}\catcode`\%=\active\def%{\%}$\mathdefault{180}$}}%
\end{pgfscope}%
\begin{pgfscope}%
\definecolor{textcolor}{rgb}{0.000000,0.000000,0.000000}%
\pgfsetstrokecolor{textcolor}%
\pgfsetfillcolor{textcolor}%
\pgftext[x=2.922899in,y=0.234413in,,top]{\color{textcolor}{\rmfamily\fontsize{10.000000}{12.000000}\selectfont\catcode`\^=\active\def^{\ifmmode\sp\else\^{}\fi}\catcode`\%=\active\def%{\%}Monochromatic classes}}%
\end{pgfscope}%
\begin{pgfscope}%
\pgfsetbuttcap%
\pgfsetroundjoin%
\definecolor{currentfill}{rgb}{0.000000,0.000000,0.000000}%
\pgfsetfillcolor{currentfill}%
\pgfsetlinewidth{0.803000pt}%
\definecolor{currentstroke}{rgb}{0.000000,0.000000,0.000000}%
\pgfsetstrokecolor{currentstroke}%
\pgfsetdash{}{0pt}%
\pgfsys@defobject{currentmarker}{\pgfqpoint{-0.048611in}{0.000000in}}{\pgfqpoint{-0.000000in}{0.000000in}}{%
\pgfpathmoveto{\pgfqpoint{-0.000000in}{0.000000in}}%
\pgfpathlineto{\pgfqpoint{-0.048611in}{0.000000in}}%
\pgfusepath{stroke,fill}%
}%
\begin{pgfscope}%
\pgfsys@transformshift{0.588387in}{0.766887in}%
\pgfsys@useobject{currentmarker}{}%
\end{pgfscope}%
\end{pgfscope}%
\begin{pgfscope}%
\definecolor{textcolor}{rgb}{0.000000,0.000000,0.000000}%
\pgfsetstrokecolor{textcolor}%
\pgfsetfillcolor{textcolor}%
\pgftext[x=0.289968in, y=0.714125in, left, base]{\color{textcolor}{\rmfamily\fontsize{10.000000}{12.000000}\selectfont\catcode`\^=\active\def^{\ifmmode\sp\else\^{}\fi}\catcode`\%=\active\def%{\%}$\mathdefault{10^{1}}$}}%
\end{pgfscope}%
\begin{pgfscope}%
\pgfsetbuttcap%
\pgfsetroundjoin%
\definecolor{currentfill}{rgb}{0.000000,0.000000,0.000000}%
\pgfsetfillcolor{currentfill}%
\pgfsetlinewidth{0.803000pt}%
\definecolor{currentstroke}{rgb}{0.000000,0.000000,0.000000}%
\pgfsetstrokecolor{currentstroke}%
\pgfsetdash{}{0pt}%
\pgfsys@defobject{currentmarker}{\pgfqpoint{-0.048611in}{0.000000in}}{\pgfqpoint{-0.000000in}{0.000000in}}{%
\pgfpathmoveto{\pgfqpoint{-0.000000in}{0.000000in}}%
\pgfpathlineto{\pgfqpoint{-0.048611in}{0.000000in}}%
\pgfusepath{stroke,fill}%
}%
\begin{pgfscope}%
\pgfsys@transformshift{0.588387in}{1.153646in}%
\pgfsys@useobject{currentmarker}{}%
\end{pgfscope}%
\end{pgfscope}%
\begin{pgfscope}%
\definecolor{textcolor}{rgb}{0.000000,0.000000,0.000000}%
\pgfsetstrokecolor{textcolor}%
\pgfsetfillcolor{textcolor}%
\pgftext[x=0.289968in, y=1.100885in, left, base]{\color{textcolor}{\rmfamily\fontsize{10.000000}{12.000000}\selectfont\catcode`\^=\active\def^{\ifmmode\sp\else\^{}\fi}\catcode`\%=\active\def%{\%}$\mathdefault{10^{2}}$}}%
\end{pgfscope}%
\begin{pgfscope}%
\pgfsetbuttcap%
\pgfsetroundjoin%
\definecolor{currentfill}{rgb}{0.000000,0.000000,0.000000}%
\pgfsetfillcolor{currentfill}%
\pgfsetlinewidth{0.803000pt}%
\definecolor{currentstroke}{rgb}{0.000000,0.000000,0.000000}%
\pgfsetstrokecolor{currentstroke}%
\pgfsetdash{}{0pt}%
\pgfsys@defobject{currentmarker}{\pgfqpoint{-0.048611in}{0.000000in}}{\pgfqpoint{-0.000000in}{0.000000in}}{%
\pgfpathmoveto{\pgfqpoint{-0.000000in}{0.000000in}}%
\pgfpathlineto{\pgfqpoint{-0.048611in}{0.000000in}}%
\pgfusepath{stroke,fill}%
}%
\begin{pgfscope}%
\pgfsys@transformshift{0.588387in}{1.540406in}%
\pgfsys@useobject{currentmarker}{}%
\end{pgfscope}%
\end{pgfscope}%
\begin{pgfscope}%
\definecolor{textcolor}{rgb}{0.000000,0.000000,0.000000}%
\pgfsetstrokecolor{textcolor}%
\pgfsetfillcolor{textcolor}%
\pgftext[x=0.289968in, y=1.487644in, left, base]{\color{textcolor}{\rmfamily\fontsize{10.000000}{12.000000}\selectfont\catcode`\^=\active\def^{\ifmmode\sp\else\^{}\fi}\catcode`\%=\active\def%{\%}$\mathdefault{10^{3}}$}}%
\end{pgfscope}%
\begin{pgfscope}%
\pgfsetbuttcap%
\pgfsetroundjoin%
\definecolor{currentfill}{rgb}{0.000000,0.000000,0.000000}%
\pgfsetfillcolor{currentfill}%
\pgfsetlinewidth{0.803000pt}%
\definecolor{currentstroke}{rgb}{0.000000,0.000000,0.000000}%
\pgfsetstrokecolor{currentstroke}%
\pgfsetdash{}{0pt}%
\pgfsys@defobject{currentmarker}{\pgfqpoint{-0.048611in}{0.000000in}}{\pgfqpoint{-0.000000in}{0.000000in}}{%
\pgfpathmoveto{\pgfqpoint{-0.000000in}{0.000000in}}%
\pgfpathlineto{\pgfqpoint{-0.048611in}{0.000000in}}%
\pgfusepath{stroke,fill}%
}%
\begin{pgfscope}%
\pgfsys@transformshift{0.588387in}{1.927165in}%
\pgfsys@useobject{currentmarker}{}%
\end{pgfscope}%
\end{pgfscope}%
\begin{pgfscope}%
\definecolor{textcolor}{rgb}{0.000000,0.000000,0.000000}%
\pgfsetstrokecolor{textcolor}%
\pgfsetfillcolor{textcolor}%
\pgftext[x=0.289968in, y=1.874404in, left, base]{\color{textcolor}{\rmfamily\fontsize{10.000000}{12.000000}\selectfont\catcode`\^=\active\def^{\ifmmode\sp\else\^{}\fi}\catcode`\%=\active\def%{\%}$\mathdefault{10^{4}}$}}%
\end{pgfscope}%
\begin{pgfscope}%
\pgfsetbuttcap%
\pgfsetroundjoin%
\definecolor{currentfill}{rgb}{0.000000,0.000000,0.000000}%
\pgfsetfillcolor{currentfill}%
\pgfsetlinewidth{0.803000pt}%
\definecolor{currentstroke}{rgb}{0.000000,0.000000,0.000000}%
\pgfsetstrokecolor{currentstroke}%
\pgfsetdash{}{0pt}%
\pgfsys@defobject{currentmarker}{\pgfqpoint{-0.048611in}{0.000000in}}{\pgfqpoint{-0.000000in}{0.000000in}}{%
\pgfpathmoveto{\pgfqpoint{-0.000000in}{0.000000in}}%
\pgfpathlineto{\pgfqpoint{-0.048611in}{0.000000in}}%
\pgfusepath{stroke,fill}%
}%
\begin{pgfscope}%
\pgfsys@transformshift{0.588387in}{2.313925in}%
\pgfsys@useobject{currentmarker}{}%
\end{pgfscope}%
\end{pgfscope}%
\begin{pgfscope}%
\definecolor{textcolor}{rgb}{0.000000,0.000000,0.000000}%
\pgfsetstrokecolor{textcolor}%
\pgfsetfillcolor{textcolor}%
\pgftext[x=0.289968in, y=2.261163in, left, base]{\color{textcolor}{\rmfamily\fontsize{10.000000}{12.000000}\selectfont\catcode`\^=\active\def^{\ifmmode\sp\else\^{}\fi}\catcode`\%=\active\def%{\%}$\mathdefault{10^{5}}$}}%
\end{pgfscope}%
\begin{pgfscope}%
\pgfsetbuttcap%
\pgfsetroundjoin%
\definecolor{currentfill}{rgb}{0.000000,0.000000,0.000000}%
\pgfsetfillcolor{currentfill}%
\pgfsetlinewidth{0.602250pt}%
\definecolor{currentstroke}{rgb}{0.000000,0.000000,0.000000}%
\pgfsetstrokecolor{currentstroke}%
\pgfsetdash{}{0pt}%
\pgfsys@defobject{currentmarker}{\pgfqpoint{-0.027778in}{0.000000in}}{\pgfqpoint{-0.000000in}{0.000000in}}{%
\pgfpathmoveto{\pgfqpoint{-0.000000in}{0.000000in}}%
\pgfpathlineto{\pgfqpoint{-0.027778in}{0.000000in}}%
\pgfusepath{stroke,fill}%
}%
\begin{pgfscope}%
\pgfsys@transformshift{0.588387in}{0.564659in}%
\pgfsys@useobject{currentmarker}{}%
\end{pgfscope}%
\end{pgfscope}%
\begin{pgfscope}%
\pgfsetbuttcap%
\pgfsetroundjoin%
\definecolor{currentfill}{rgb}{0.000000,0.000000,0.000000}%
\pgfsetfillcolor{currentfill}%
\pgfsetlinewidth{0.602250pt}%
\definecolor{currentstroke}{rgb}{0.000000,0.000000,0.000000}%
\pgfsetstrokecolor{currentstroke}%
\pgfsetdash{}{0pt}%
\pgfsys@defobject{currentmarker}{\pgfqpoint{-0.027778in}{0.000000in}}{\pgfqpoint{-0.000000in}{0.000000in}}{%
\pgfpathmoveto{\pgfqpoint{-0.000000in}{0.000000in}}%
\pgfpathlineto{\pgfqpoint{-0.027778in}{0.000000in}}%
\pgfusepath{stroke,fill}%
}%
\begin{pgfscope}%
\pgfsys@transformshift{0.588387in}{0.612980in}%
\pgfsys@useobject{currentmarker}{}%
\end{pgfscope}%
\end{pgfscope}%
\begin{pgfscope}%
\pgfsetbuttcap%
\pgfsetroundjoin%
\definecolor{currentfill}{rgb}{0.000000,0.000000,0.000000}%
\pgfsetfillcolor{currentfill}%
\pgfsetlinewidth{0.602250pt}%
\definecolor{currentstroke}{rgb}{0.000000,0.000000,0.000000}%
\pgfsetstrokecolor{currentstroke}%
\pgfsetdash{}{0pt}%
\pgfsys@defobject{currentmarker}{\pgfqpoint{-0.027778in}{0.000000in}}{\pgfqpoint{-0.000000in}{0.000000in}}{%
\pgfpathmoveto{\pgfqpoint{-0.000000in}{0.000000in}}%
\pgfpathlineto{\pgfqpoint{-0.027778in}{0.000000in}}%
\pgfusepath{stroke,fill}%
}%
\begin{pgfscope}%
\pgfsys@transformshift{0.588387in}{0.650461in}%
\pgfsys@useobject{currentmarker}{}%
\end{pgfscope}%
\end{pgfscope}%
\begin{pgfscope}%
\pgfsetbuttcap%
\pgfsetroundjoin%
\definecolor{currentfill}{rgb}{0.000000,0.000000,0.000000}%
\pgfsetfillcolor{currentfill}%
\pgfsetlinewidth{0.602250pt}%
\definecolor{currentstroke}{rgb}{0.000000,0.000000,0.000000}%
\pgfsetstrokecolor{currentstroke}%
\pgfsetdash{}{0pt}%
\pgfsys@defobject{currentmarker}{\pgfqpoint{-0.027778in}{0.000000in}}{\pgfqpoint{-0.000000in}{0.000000in}}{%
\pgfpathmoveto{\pgfqpoint{-0.000000in}{0.000000in}}%
\pgfpathlineto{\pgfqpoint{-0.027778in}{0.000000in}}%
\pgfusepath{stroke,fill}%
}%
\begin{pgfscope}%
\pgfsys@transformshift{0.588387in}{0.681085in}%
\pgfsys@useobject{currentmarker}{}%
\end{pgfscope}%
\end{pgfscope}%
\begin{pgfscope}%
\pgfsetbuttcap%
\pgfsetroundjoin%
\definecolor{currentfill}{rgb}{0.000000,0.000000,0.000000}%
\pgfsetfillcolor{currentfill}%
\pgfsetlinewidth{0.602250pt}%
\definecolor{currentstroke}{rgb}{0.000000,0.000000,0.000000}%
\pgfsetstrokecolor{currentstroke}%
\pgfsetdash{}{0pt}%
\pgfsys@defobject{currentmarker}{\pgfqpoint{-0.027778in}{0.000000in}}{\pgfqpoint{-0.000000in}{0.000000in}}{%
\pgfpathmoveto{\pgfqpoint{-0.000000in}{0.000000in}}%
\pgfpathlineto{\pgfqpoint{-0.027778in}{0.000000in}}%
\pgfusepath{stroke,fill}%
}%
\begin{pgfscope}%
\pgfsys@transformshift{0.588387in}{0.706977in}%
\pgfsys@useobject{currentmarker}{}%
\end{pgfscope}%
\end{pgfscope}%
\begin{pgfscope}%
\pgfsetbuttcap%
\pgfsetroundjoin%
\definecolor{currentfill}{rgb}{0.000000,0.000000,0.000000}%
\pgfsetfillcolor{currentfill}%
\pgfsetlinewidth{0.602250pt}%
\definecolor{currentstroke}{rgb}{0.000000,0.000000,0.000000}%
\pgfsetstrokecolor{currentstroke}%
\pgfsetdash{}{0pt}%
\pgfsys@defobject{currentmarker}{\pgfqpoint{-0.027778in}{0.000000in}}{\pgfqpoint{-0.000000in}{0.000000in}}{%
\pgfpathmoveto{\pgfqpoint{-0.000000in}{0.000000in}}%
\pgfpathlineto{\pgfqpoint{-0.027778in}{0.000000in}}%
\pgfusepath{stroke,fill}%
}%
\begin{pgfscope}%
\pgfsys@transformshift{0.588387in}{0.729406in}%
\pgfsys@useobject{currentmarker}{}%
\end{pgfscope}%
\end{pgfscope}%
\begin{pgfscope}%
\pgfsetbuttcap%
\pgfsetroundjoin%
\definecolor{currentfill}{rgb}{0.000000,0.000000,0.000000}%
\pgfsetfillcolor{currentfill}%
\pgfsetlinewidth{0.602250pt}%
\definecolor{currentstroke}{rgb}{0.000000,0.000000,0.000000}%
\pgfsetstrokecolor{currentstroke}%
\pgfsetdash{}{0pt}%
\pgfsys@defobject{currentmarker}{\pgfqpoint{-0.027778in}{0.000000in}}{\pgfqpoint{-0.000000in}{0.000000in}}{%
\pgfpathmoveto{\pgfqpoint{-0.000000in}{0.000000in}}%
\pgfpathlineto{\pgfqpoint{-0.027778in}{0.000000in}}%
\pgfusepath{stroke,fill}%
}%
\begin{pgfscope}%
\pgfsys@transformshift{0.588387in}{0.749190in}%
\pgfsys@useobject{currentmarker}{}%
\end{pgfscope}%
\end{pgfscope}%
\begin{pgfscope}%
\pgfsetbuttcap%
\pgfsetroundjoin%
\definecolor{currentfill}{rgb}{0.000000,0.000000,0.000000}%
\pgfsetfillcolor{currentfill}%
\pgfsetlinewidth{0.602250pt}%
\definecolor{currentstroke}{rgb}{0.000000,0.000000,0.000000}%
\pgfsetstrokecolor{currentstroke}%
\pgfsetdash{}{0pt}%
\pgfsys@defobject{currentmarker}{\pgfqpoint{-0.027778in}{0.000000in}}{\pgfqpoint{-0.000000in}{0.000000in}}{%
\pgfpathmoveto{\pgfqpoint{-0.000000in}{0.000000in}}%
\pgfpathlineto{\pgfqpoint{-0.027778in}{0.000000in}}%
\pgfusepath{stroke,fill}%
}%
\begin{pgfscope}%
\pgfsys@transformshift{0.588387in}{0.883313in}%
\pgfsys@useobject{currentmarker}{}%
\end{pgfscope}%
\end{pgfscope}%
\begin{pgfscope}%
\pgfsetbuttcap%
\pgfsetroundjoin%
\definecolor{currentfill}{rgb}{0.000000,0.000000,0.000000}%
\pgfsetfillcolor{currentfill}%
\pgfsetlinewidth{0.602250pt}%
\definecolor{currentstroke}{rgb}{0.000000,0.000000,0.000000}%
\pgfsetstrokecolor{currentstroke}%
\pgfsetdash{}{0pt}%
\pgfsys@defobject{currentmarker}{\pgfqpoint{-0.027778in}{0.000000in}}{\pgfqpoint{-0.000000in}{0.000000in}}{%
\pgfpathmoveto{\pgfqpoint{-0.000000in}{0.000000in}}%
\pgfpathlineto{\pgfqpoint{-0.027778in}{0.000000in}}%
\pgfusepath{stroke,fill}%
}%
\begin{pgfscope}%
\pgfsys@transformshift{0.588387in}{0.951418in}%
\pgfsys@useobject{currentmarker}{}%
\end{pgfscope}%
\end{pgfscope}%
\begin{pgfscope}%
\pgfsetbuttcap%
\pgfsetroundjoin%
\definecolor{currentfill}{rgb}{0.000000,0.000000,0.000000}%
\pgfsetfillcolor{currentfill}%
\pgfsetlinewidth{0.602250pt}%
\definecolor{currentstroke}{rgb}{0.000000,0.000000,0.000000}%
\pgfsetstrokecolor{currentstroke}%
\pgfsetdash{}{0pt}%
\pgfsys@defobject{currentmarker}{\pgfqpoint{-0.027778in}{0.000000in}}{\pgfqpoint{-0.000000in}{0.000000in}}{%
\pgfpathmoveto{\pgfqpoint{-0.000000in}{0.000000in}}%
\pgfpathlineto{\pgfqpoint{-0.027778in}{0.000000in}}%
\pgfusepath{stroke,fill}%
}%
\begin{pgfscope}%
\pgfsys@transformshift{0.588387in}{0.999739in}%
\pgfsys@useobject{currentmarker}{}%
\end{pgfscope}%
\end{pgfscope}%
\begin{pgfscope}%
\pgfsetbuttcap%
\pgfsetroundjoin%
\definecolor{currentfill}{rgb}{0.000000,0.000000,0.000000}%
\pgfsetfillcolor{currentfill}%
\pgfsetlinewidth{0.602250pt}%
\definecolor{currentstroke}{rgb}{0.000000,0.000000,0.000000}%
\pgfsetstrokecolor{currentstroke}%
\pgfsetdash{}{0pt}%
\pgfsys@defobject{currentmarker}{\pgfqpoint{-0.027778in}{0.000000in}}{\pgfqpoint{-0.000000in}{0.000000in}}{%
\pgfpathmoveto{\pgfqpoint{-0.000000in}{0.000000in}}%
\pgfpathlineto{\pgfqpoint{-0.027778in}{0.000000in}}%
\pgfusepath{stroke,fill}%
}%
\begin{pgfscope}%
\pgfsys@transformshift{0.588387in}{1.037220in}%
\pgfsys@useobject{currentmarker}{}%
\end{pgfscope}%
\end{pgfscope}%
\begin{pgfscope}%
\pgfsetbuttcap%
\pgfsetroundjoin%
\definecolor{currentfill}{rgb}{0.000000,0.000000,0.000000}%
\pgfsetfillcolor{currentfill}%
\pgfsetlinewidth{0.602250pt}%
\definecolor{currentstroke}{rgb}{0.000000,0.000000,0.000000}%
\pgfsetstrokecolor{currentstroke}%
\pgfsetdash{}{0pt}%
\pgfsys@defobject{currentmarker}{\pgfqpoint{-0.027778in}{0.000000in}}{\pgfqpoint{-0.000000in}{0.000000in}}{%
\pgfpathmoveto{\pgfqpoint{-0.000000in}{0.000000in}}%
\pgfpathlineto{\pgfqpoint{-0.027778in}{0.000000in}}%
\pgfusepath{stroke,fill}%
}%
\begin{pgfscope}%
\pgfsys@transformshift{0.588387in}{1.067844in}%
\pgfsys@useobject{currentmarker}{}%
\end{pgfscope}%
\end{pgfscope}%
\begin{pgfscope}%
\pgfsetbuttcap%
\pgfsetroundjoin%
\definecolor{currentfill}{rgb}{0.000000,0.000000,0.000000}%
\pgfsetfillcolor{currentfill}%
\pgfsetlinewidth{0.602250pt}%
\definecolor{currentstroke}{rgb}{0.000000,0.000000,0.000000}%
\pgfsetstrokecolor{currentstroke}%
\pgfsetdash{}{0pt}%
\pgfsys@defobject{currentmarker}{\pgfqpoint{-0.027778in}{0.000000in}}{\pgfqpoint{-0.000000in}{0.000000in}}{%
\pgfpathmoveto{\pgfqpoint{-0.000000in}{0.000000in}}%
\pgfpathlineto{\pgfqpoint{-0.027778in}{0.000000in}}%
\pgfusepath{stroke,fill}%
}%
\begin{pgfscope}%
\pgfsys@transformshift{0.588387in}{1.093737in}%
\pgfsys@useobject{currentmarker}{}%
\end{pgfscope}%
\end{pgfscope}%
\begin{pgfscope}%
\pgfsetbuttcap%
\pgfsetroundjoin%
\definecolor{currentfill}{rgb}{0.000000,0.000000,0.000000}%
\pgfsetfillcolor{currentfill}%
\pgfsetlinewidth{0.602250pt}%
\definecolor{currentstroke}{rgb}{0.000000,0.000000,0.000000}%
\pgfsetstrokecolor{currentstroke}%
\pgfsetdash{}{0pt}%
\pgfsys@defobject{currentmarker}{\pgfqpoint{-0.027778in}{0.000000in}}{\pgfqpoint{-0.000000in}{0.000000in}}{%
\pgfpathmoveto{\pgfqpoint{-0.000000in}{0.000000in}}%
\pgfpathlineto{\pgfqpoint{-0.027778in}{0.000000in}}%
\pgfusepath{stroke,fill}%
}%
\begin{pgfscope}%
\pgfsys@transformshift{0.588387in}{1.116166in}%
\pgfsys@useobject{currentmarker}{}%
\end{pgfscope}%
\end{pgfscope}%
\begin{pgfscope}%
\pgfsetbuttcap%
\pgfsetroundjoin%
\definecolor{currentfill}{rgb}{0.000000,0.000000,0.000000}%
\pgfsetfillcolor{currentfill}%
\pgfsetlinewidth{0.602250pt}%
\definecolor{currentstroke}{rgb}{0.000000,0.000000,0.000000}%
\pgfsetstrokecolor{currentstroke}%
\pgfsetdash{}{0pt}%
\pgfsys@defobject{currentmarker}{\pgfqpoint{-0.027778in}{0.000000in}}{\pgfqpoint{-0.000000in}{0.000000in}}{%
\pgfpathmoveto{\pgfqpoint{-0.000000in}{0.000000in}}%
\pgfpathlineto{\pgfqpoint{-0.027778in}{0.000000in}}%
\pgfusepath{stroke,fill}%
}%
\begin{pgfscope}%
\pgfsys@transformshift{0.588387in}{1.135949in}%
\pgfsys@useobject{currentmarker}{}%
\end{pgfscope}%
\end{pgfscope}%
\begin{pgfscope}%
\pgfsetbuttcap%
\pgfsetroundjoin%
\definecolor{currentfill}{rgb}{0.000000,0.000000,0.000000}%
\pgfsetfillcolor{currentfill}%
\pgfsetlinewidth{0.602250pt}%
\definecolor{currentstroke}{rgb}{0.000000,0.000000,0.000000}%
\pgfsetstrokecolor{currentstroke}%
\pgfsetdash{}{0pt}%
\pgfsys@defobject{currentmarker}{\pgfqpoint{-0.027778in}{0.000000in}}{\pgfqpoint{-0.000000in}{0.000000in}}{%
\pgfpathmoveto{\pgfqpoint{-0.000000in}{0.000000in}}%
\pgfpathlineto{\pgfqpoint{-0.027778in}{0.000000in}}%
\pgfusepath{stroke,fill}%
}%
\begin{pgfscope}%
\pgfsys@transformshift{0.588387in}{1.270073in}%
\pgfsys@useobject{currentmarker}{}%
\end{pgfscope}%
\end{pgfscope}%
\begin{pgfscope}%
\pgfsetbuttcap%
\pgfsetroundjoin%
\definecolor{currentfill}{rgb}{0.000000,0.000000,0.000000}%
\pgfsetfillcolor{currentfill}%
\pgfsetlinewidth{0.602250pt}%
\definecolor{currentstroke}{rgb}{0.000000,0.000000,0.000000}%
\pgfsetstrokecolor{currentstroke}%
\pgfsetdash{}{0pt}%
\pgfsys@defobject{currentmarker}{\pgfqpoint{-0.027778in}{0.000000in}}{\pgfqpoint{-0.000000in}{0.000000in}}{%
\pgfpathmoveto{\pgfqpoint{-0.000000in}{0.000000in}}%
\pgfpathlineto{\pgfqpoint{-0.027778in}{0.000000in}}%
\pgfusepath{stroke,fill}%
}%
\begin{pgfscope}%
\pgfsys@transformshift{0.588387in}{1.338178in}%
\pgfsys@useobject{currentmarker}{}%
\end{pgfscope}%
\end{pgfscope}%
\begin{pgfscope}%
\pgfsetbuttcap%
\pgfsetroundjoin%
\definecolor{currentfill}{rgb}{0.000000,0.000000,0.000000}%
\pgfsetfillcolor{currentfill}%
\pgfsetlinewidth{0.602250pt}%
\definecolor{currentstroke}{rgb}{0.000000,0.000000,0.000000}%
\pgfsetstrokecolor{currentstroke}%
\pgfsetdash{}{0pt}%
\pgfsys@defobject{currentmarker}{\pgfqpoint{-0.027778in}{0.000000in}}{\pgfqpoint{-0.000000in}{0.000000in}}{%
\pgfpathmoveto{\pgfqpoint{-0.000000in}{0.000000in}}%
\pgfpathlineto{\pgfqpoint{-0.027778in}{0.000000in}}%
\pgfusepath{stroke,fill}%
}%
\begin{pgfscope}%
\pgfsys@transformshift{0.588387in}{1.386499in}%
\pgfsys@useobject{currentmarker}{}%
\end{pgfscope}%
\end{pgfscope}%
\begin{pgfscope}%
\pgfsetbuttcap%
\pgfsetroundjoin%
\definecolor{currentfill}{rgb}{0.000000,0.000000,0.000000}%
\pgfsetfillcolor{currentfill}%
\pgfsetlinewidth{0.602250pt}%
\definecolor{currentstroke}{rgb}{0.000000,0.000000,0.000000}%
\pgfsetstrokecolor{currentstroke}%
\pgfsetdash{}{0pt}%
\pgfsys@defobject{currentmarker}{\pgfqpoint{-0.027778in}{0.000000in}}{\pgfqpoint{-0.000000in}{0.000000in}}{%
\pgfpathmoveto{\pgfqpoint{-0.000000in}{0.000000in}}%
\pgfpathlineto{\pgfqpoint{-0.027778in}{0.000000in}}%
\pgfusepath{stroke,fill}%
}%
\begin{pgfscope}%
\pgfsys@transformshift{0.588387in}{1.423980in}%
\pgfsys@useobject{currentmarker}{}%
\end{pgfscope}%
\end{pgfscope}%
\begin{pgfscope}%
\pgfsetbuttcap%
\pgfsetroundjoin%
\definecolor{currentfill}{rgb}{0.000000,0.000000,0.000000}%
\pgfsetfillcolor{currentfill}%
\pgfsetlinewidth{0.602250pt}%
\definecolor{currentstroke}{rgb}{0.000000,0.000000,0.000000}%
\pgfsetstrokecolor{currentstroke}%
\pgfsetdash{}{0pt}%
\pgfsys@defobject{currentmarker}{\pgfqpoint{-0.027778in}{0.000000in}}{\pgfqpoint{-0.000000in}{0.000000in}}{%
\pgfpathmoveto{\pgfqpoint{-0.000000in}{0.000000in}}%
\pgfpathlineto{\pgfqpoint{-0.027778in}{0.000000in}}%
\pgfusepath{stroke,fill}%
}%
\begin{pgfscope}%
\pgfsys@transformshift{0.588387in}{1.454604in}%
\pgfsys@useobject{currentmarker}{}%
\end{pgfscope}%
\end{pgfscope}%
\begin{pgfscope}%
\pgfsetbuttcap%
\pgfsetroundjoin%
\definecolor{currentfill}{rgb}{0.000000,0.000000,0.000000}%
\pgfsetfillcolor{currentfill}%
\pgfsetlinewidth{0.602250pt}%
\definecolor{currentstroke}{rgb}{0.000000,0.000000,0.000000}%
\pgfsetstrokecolor{currentstroke}%
\pgfsetdash{}{0pt}%
\pgfsys@defobject{currentmarker}{\pgfqpoint{-0.027778in}{0.000000in}}{\pgfqpoint{-0.000000in}{0.000000in}}{%
\pgfpathmoveto{\pgfqpoint{-0.000000in}{0.000000in}}%
\pgfpathlineto{\pgfqpoint{-0.027778in}{0.000000in}}%
\pgfusepath{stroke,fill}%
}%
\begin{pgfscope}%
\pgfsys@transformshift{0.588387in}{1.480496in}%
\pgfsys@useobject{currentmarker}{}%
\end{pgfscope}%
\end{pgfscope}%
\begin{pgfscope}%
\pgfsetbuttcap%
\pgfsetroundjoin%
\definecolor{currentfill}{rgb}{0.000000,0.000000,0.000000}%
\pgfsetfillcolor{currentfill}%
\pgfsetlinewidth{0.602250pt}%
\definecolor{currentstroke}{rgb}{0.000000,0.000000,0.000000}%
\pgfsetstrokecolor{currentstroke}%
\pgfsetdash{}{0pt}%
\pgfsys@defobject{currentmarker}{\pgfqpoint{-0.027778in}{0.000000in}}{\pgfqpoint{-0.000000in}{0.000000in}}{%
\pgfpathmoveto{\pgfqpoint{-0.000000in}{0.000000in}}%
\pgfpathlineto{\pgfqpoint{-0.027778in}{0.000000in}}%
\pgfusepath{stroke,fill}%
}%
\begin{pgfscope}%
\pgfsys@transformshift{0.588387in}{1.502925in}%
\pgfsys@useobject{currentmarker}{}%
\end{pgfscope}%
\end{pgfscope}%
\begin{pgfscope}%
\pgfsetbuttcap%
\pgfsetroundjoin%
\definecolor{currentfill}{rgb}{0.000000,0.000000,0.000000}%
\pgfsetfillcolor{currentfill}%
\pgfsetlinewidth{0.602250pt}%
\definecolor{currentstroke}{rgb}{0.000000,0.000000,0.000000}%
\pgfsetstrokecolor{currentstroke}%
\pgfsetdash{}{0pt}%
\pgfsys@defobject{currentmarker}{\pgfqpoint{-0.027778in}{0.000000in}}{\pgfqpoint{-0.000000in}{0.000000in}}{%
\pgfpathmoveto{\pgfqpoint{-0.000000in}{0.000000in}}%
\pgfpathlineto{\pgfqpoint{-0.027778in}{0.000000in}}%
\pgfusepath{stroke,fill}%
}%
\begin{pgfscope}%
\pgfsys@transformshift{0.588387in}{1.522709in}%
\pgfsys@useobject{currentmarker}{}%
\end{pgfscope}%
\end{pgfscope}%
\begin{pgfscope}%
\pgfsetbuttcap%
\pgfsetroundjoin%
\definecolor{currentfill}{rgb}{0.000000,0.000000,0.000000}%
\pgfsetfillcolor{currentfill}%
\pgfsetlinewidth{0.602250pt}%
\definecolor{currentstroke}{rgb}{0.000000,0.000000,0.000000}%
\pgfsetstrokecolor{currentstroke}%
\pgfsetdash{}{0pt}%
\pgfsys@defobject{currentmarker}{\pgfqpoint{-0.027778in}{0.000000in}}{\pgfqpoint{-0.000000in}{0.000000in}}{%
\pgfpathmoveto{\pgfqpoint{-0.000000in}{0.000000in}}%
\pgfpathlineto{\pgfqpoint{-0.027778in}{0.000000in}}%
\pgfusepath{stroke,fill}%
}%
\begin{pgfscope}%
\pgfsys@transformshift{0.588387in}{1.656832in}%
\pgfsys@useobject{currentmarker}{}%
\end{pgfscope}%
\end{pgfscope}%
\begin{pgfscope}%
\pgfsetbuttcap%
\pgfsetroundjoin%
\definecolor{currentfill}{rgb}{0.000000,0.000000,0.000000}%
\pgfsetfillcolor{currentfill}%
\pgfsetlinewidth{0.602250pt}%
\definecolor{currentstroke}{rgb}{0.000000,0.000000,0.000000}%
\pgfsetstrokecolor{currentstroke}%
\pgfsetdash{}{0pt}%
\pgfsys@defobject{currentmarker}{\pgfqpoint{-0.027778in}{0.000000in}}{\pgfqpoint{-0.000000in}{0.000000in}}{%
\pgfpathmoveto{\pgfqpoint{-0.000000in}{0.000000in}}%
\pgfpathlineto{\pgfqpoint{-0.027778in}{0.000000in}}%
\pgfusepath{stroke,fill}%
}%
\begin{pgfscope}%
\pgfsys@transformshift{0.588387in}{1.724937in}%
\pgfsys@useobject{currentmarker}{}%
\end{pgfscope}%
\end{pgfscope}%
\begin{pgfscope}%
\pgfsetbuttcap%
\pgfsetroundjoin%
\definecolor{currentfill}{rgb}{0.000000,0.000000,0.000000}%
\pgfsetfillcolor{currentfill}%
\pgfsetlinewidth{0.602250pt}%
\definecolor{currentstroke}{rgb}{0.000000,0.000000,0.000000}%
\pgfsetstrokecolor{currentstroke}%
\pgfsetdash{}{0pt}%
\pgfsys@defobject{currentmarker}{\pgfqpoint{-0.027778in}{0.000000in}}{\pgfqpoint{-0.000000in}{0.000000in}}{%
\pgfpathmoveto{\pgfqpoint{-0.000000in}{0.000000in}}%
\pgfpathlineto{\pgfqpoint{-0.027778in}{0.000000in}}%
\pgfusepath{stroke,fill}%
}%
\begin{pgfscope}%
\pgfsys@transformshift{0.588387in}{1.773258in}%
\pgfsys@useobject{currentmarker}{}%
\end{pgfscope}%
\end{pgfscope}%
\begin{pgfscope}%
\pgfsetbuttcap%
\pgfsetroundjoin%
\definecolor{currentfill}{rgb}{0.000000,0.000000,0.000000}%
\pgfsetfillcolor{currentfill}%
\pgfsetlinewidth{0.602250pt}%
\definecolor{currentstroke}{rgb}{0.000000,0.000000,0.000000}%
\pgfsetstrokecolor{currentstroke}%
\pgfsetdash{}{0pt}%
\pgfsys@defobject{currentmarker}{\pgfqpoint{-0.027778in}{0.000000in}}{\pgfqpoint{-0.000000in}{0.000000in}}{%
\pgfpathmoveto{\pgfqpoint{-0.000000in}{0.000000in}}%
\pgfpathlineto{\pgfqpoint{-0.027778in}{0.000000in}}%
\pgfusepath{stroke,fill}%
}%
\begin{pgfscope}%
\pgfsys@transformshift{0.588387in}{1.810739in}%
\pgfsys@useobject{currentmarker}{}%
\end{pgfscope}%
\end{pgfscope}%
\begin{pgfscope}%
\pgfsetbuttcap%
\pgfsetroundjoin%
\definecolor{currentfill}{rgb}{0.000000,0.000000,0.000000}%
\pgfsetfillcolor{currentfill}%
\pgfsetlinewidth{0.602250pt}%
\definecolor{currentstroke}{rgb}{0.000000,0.000000,0.000000}%
\pgfsetstrokecolor{currentstroke}%
\pgfsetdash{}{0pt}%
\pgfsys@defobject{currentmarker}{\pgfqpoint{-0.027778in}{0.000000in}}{\pgfqpoint{-0.000000in}{0.000000in}}{%
\pgfpathmoveto{\pgfqpoint{-0.000000in}{0.000000in}}%
\pgfpathlineto{\pgfqpoint{-0.027778in}{0.000000in}}%
\pgfusepath{stroke,fill}%
}%
\begin{pgfscope}%
\pgfsys@transformshift{0.588387in}{1.841363in}%
\pgfsys@useobject{currentmarker}{}%
\end{pgfscope}%
\end{pgfscope}%
\begin{pgfscope}%
\pgfsetbuttcap%
\pgfsetroundjoin%
\definecolor{currentfill}{rgb}{0.000000,0.000000,0.000000}%
\pgfsetfillcolor{currentfill}%
\pgfsetlinewidth{0.602250pt}%
\definecolor{currentstroke}{rgb}{0.000000,0.000000,0.000000}%
\pgfsetstrokecolor{currentstroke}%
\pgfsetdash{}{0pt}%
\pgfsys@defobject{currentmarker}{\pgfqpoint{-0.027778in}{0.000000in}}{\pgfqpoint{-0.000000in}{0.000000in}}{%
\pgfpathmoveto{\pgfqpoint{-0.000000in}{0.000000in}}%
\pgfpathlineto{\pgfqpoint{-0.027778in}{0.000000in}}%
\pgfusepath{stroke,fill}%
}%
\begin{pgfscope}%
\pgfsys@transformshift{0.588387in}{1.867256in}%
\pgfsys@useobject{currentmarker}{}%
\end{pgfscope}%
\end{pgfscope}%
\begin{pgfscope}%
\pgfsetbuttcap%
\pgfsetroundjoin%
\definecolor{currentfill}{rgb}{0.000000,0.000000,0.000000}%
\pgfsetfillcolor{currentfill}%
\pgfsetlinewidth{0.602250pt}%
\definecolor{currentstroke}{rgb}{0.000000,0.000000,0.000000}%
\pgfsetstrokecolor{currentstroke}%
\pgfsetdash{}{0pt}%
\pgfsys@defobject{currentmarker}{\pgfqpoint{-0.027778in}{0.000000in}}{\pgfqpoint{-0.000000in}{0.000000in}}{%
\pgfpathmoveto{\pgfqpoint{-0.000000in}{0.000000in}}%
\pgfpathlineto{\pgfqpoint{-0.027778in}{0.000000in}}%
\pgfusepath{stroke,fill}%
}%
\begin{pgfscope}%
\pgfsys@transformshift{0.588387in}{1.889685in}%
\pgfsys@useobject{currentmarker}{}%
\end{pgfscope}%
\end{pgfscope}%
\begin{pgfscope}%
\pgfsetbuttcap%
\pgfsetroundjoin%
\definecolor{currentfill}{rgb}{0.000000,0.000000,0.000000}%
\pgfsetfillcolor{currentfill}%
\pgfsetlinewidth{0.602250pt}%
\definecolor{currentstroke}{rgb}{0.000000,0.000000,0.000000}%
\pgfsetstrokecolor{currentstroke}%
\pgfsetdash{}{0pt}%
\pgfsys@defobject{currentmarker}{\pgfqpoint{-0.027778in}{0.000000in}}{\pgfqpoint{-0.000000in}{0.000000in}}{%
\pgfpathmoveto{\pgfqpoint{-0.000000in}{0.000000in}}%
\pgfpathlineto{\pgfqpoint{-0.027778in}{0.000000in}}%
\pgfusepath{stroke,fill}%
}%
\begin{pgfscope}%
\pgfsys@transformshift{0.588387in}{1.909468in}%
\pgfsys@useobject{currentmarker}{}%
\end{pgfscope}%
\end{pgfscope}%
\begin{pgfscope}%
\pgfsetbuttcap%
\pgfsetroundjoin%
\definecolor{currentfill}{rgb}{0.000000,0.000000,0.000000}%
\pgfsetfillcolor{currentfill}%
\pgfsetlinewidth{0.602250pt}%
\definecolor{currentstroke}{rgb}{0.000000,0.000000,0.000000}%
\pgfsetstrokecolor{currentstroke}%
\pgfsetdash{}{0pt}%
\pgfsys@defobject{currentmarker}{\pgfqpoint{-0.027778in}{0.000000in}}{\pgfqpoint{-0.000000in}{0.000000in}}{%
\pgfpathmoveto{\pgfqpoint{-0.000000in}{0.000000in}}%
\pgfpathlineto{\pgfqpoint{-0.027778in}{0.000000in}}%
\pgfusepath{stroke,fill}%
}%
\begin{pgfscope}%
\pgfsys@transformshift{0.588387in}{2.043592in}%
\pgfsys@useobject{currentmarker}{}%
\end{pgfscope}%
\end{pgfscope}%
\begin{pgfscope}%
\pgfsetbuttcap%
\pgfsetroundjoin%
\definecolor{currentfill}{rgb}{0.000000,0.000000,0.000000}%
\pgfsetfillcolor{currentfill}%
\pgfsetlinewidth{0.602250pt}%
\definecolor{currentstroke}{rgb}{0.000000,0.000000,0.000000}%
\pgfsetstrokecolor{currentstroke}%
\pgfsetdash{}{0pt}%
\pgfsys@defobject{currentmarker}{\pgfqpoint{-0.027778in}{0.000000in}}{\pgfqpoint{-0.000000in}{0.000000in}}{%
\pgfpathmoveto{\pgfqpoint{-0.000000in}{0.000000in}}%
\pgfpathlineto{\pgfqpoint{-0.027778in}{0.000000in}}%
\pgfusepath{stroke,fill}%
}%
\begin{pgfscope}%
\pgfsys@transformshift{0.588387in}{2.111697in}%
\pgfsys@useobject{currentmarker}{}%
\end{pgfscope}%
\end{pgfscope}%
\begin{pgfscope}%
\pgfsetbuttcap%
\pgfsetroundjoin%
\definecolor{currentfill}{rgb}{0.000000,0.000000,0.000000}%
\pgfsetfillcolor{currentfill}%
\pgfsetlinewidth{0.602250pt}%
\definecolor{currentstroke}{rgb}{0.000000,0.000000,0.000000}%
\pgfsetstrokecolor{currentstroke}%
\pgfsetdash{}{0pt}%
\pgfsys@defobject{currentmarker}{\pgfqpoint{-0.027778in}{0.000000in}}{\pgfqpoint{-0.000000in}{0.000000in}}{%
\pgfpathmoveto{\pgfqpoint{-0.000000in}{0.000000in}}%
\pgfpathlineto{\pgfqpoint{-0.027778in}{0.000000in}}%
\pgfusepath{stroke,fill}%
}%
\begin{pgfscope}%
\pgfsys@transformshift{0.588387in}{2.160018in}%
\pgfsys@useobject{currentmarker}{}%
\end{pgfscope}%
\end{pgfscope}%
\begin{pgfscope}%
\pgfsetbuttcap%
\pgfsetroundjoin%
\definecolor{currentfill}{rgb}{0.000000,0.000000,0.000000}%
\pgfsetfillcolor{currentfill}%
\pgfsetlinewidth{0.602250pt}%
\definecolor{currentstroke}{rgb}{0.000000,0.000000,0.000000}%
\pgfsetstrokecolor{currentstroke}%
\pgfsetdash{}{0pt}%
\pgfsys@defobject{currentmarker}{\pgfqpoint{-0.027778in}{0.000000in}}{\pgfqpoint{-0.000000in}{0.000000in}}{%
\pgfpathmoveto{\pgfqpoint{-0.000000in}{0.000000in}}%
\pgfpathlineto{\pgfqpoint{-0.027778in}{0.000000in}}%
\pgfusepath{stroke,fill}%
}%
\begin{pgfscope}%
\pgfsys@transformshift{0.588387in}{2.197499in}%
\pgfsys@useobject{currentmarker}{}%
\end{pgfscope}%
\end{pgfscope}%
\begin{pgfscope}%
\pgfsetbuttcap%
\pgfsetroundjoin%
\definecolor{currentfill}{rgb}{0.000000,0.000000,0.000000}%
\pgfsetfillcolor{currentfill}%
\pgfsetlinewidth{0.602250pt}%
\definecolor{currentstroke}{rgb}{0.000000,0.000000,0.000000}%
\pgfsetstrokecolor{currentstroke}%
\pgfsetdash{}{0pt}%
\pgfsys@defobject{currentmarker}{\pgfqpoint{-0.027778in}{0.000000in}}{\pgfqpoint{-0.000000in}{0.000000in}}{%
\pgfpathmoveto{\pgfqpoint{-0.000000in}{0.000000in}}%
\pgfpathlineto{\pgfqpoint{-0.027778in}{0.000000in}}%
\pgfusepath{stroke,fill}%
}%
\begin{pgfscope}%
\pgfsys@transformshift{0.588387in}{2.228123in}%
\pgfsys@useobject{currentmarker}{}%
\end{pgfscope}%
\end{pgfscope}%
\begin{pgfscope}%
\pgfsetbuttcap%
\pgfsetroundjoin%
\definecolor{currentfill}{rgb}{0.000000,0.000000,0.000000}%
\pgfsetfillcolor{currentfill}%
\pgfsetlinewidth{0.602250pt}%
\definecolor{currentstroke}{rgb}{0.000000,0.000000,0.000000}%
\pgfsetstrokecolor{currentstroke}%
\pgfsetdash{}{0pt}%
\pgfsys@defobject{currentmarker}{\pgfqpoint{-0.027778in}{0.000000in}}{\pgfqpoint{-0.000000in}{0.000000in}}{%
\pgfpathmoveto{\pgfqpoint{-0.000000in}{0.000000in}}%
\pgfpathlineto{\pgfqpoint{-0.027778in}{0.000000in}}%
\pgfusepath{stroke,fill}%
}%
\begin{pgfscope}%
\pgfsys@transformshift{0.588387in}{2.254015in}%
\pgfsys@useobject{currentmarker}{}%
\end{pgfscope}%
\end{pgfscope}%
\begin{pgfscope}%
\pgfsetbuttcap%
\pgfsetroundjoin%
\definecolor{currentfill}{rgb}{0.000000,0.000000,0.000000}%
\pgfsetfillcolor{currentfill}%
\pgfsetlinewidth{0.602250pt}%
\definecolor{currentstroke}{rgb}{0.000000,0.000000,0.000000}%
\pgfsetstrokecolor{currentstroke}%
\pgfsetdash{}{0pt}%
\pgfsys@defobject{currentmarker}{\pgfqpoint{-0.027778in}{0.000000in}}{\pgfqpoint{-0.000000in}{0.000000in}}{%
\pgfpathmoveto{\pgfqpoint{-0.000000in}{0.000000in}}%
\pgfpathlineto{\pgfqpoint{-0.027778in}{0.000000in}}%
\pgfusepath{stroke,fill}%
}%
\begin{pgfscope}%
\pgfsys@transformshift{0.588387in}{2.276444in}%
\pgfsys@useobject{currentmarker}{}%
\end{pgfscope}%
\end{pgfscope}%
\begin{pgfscope}%
\pgfsetbuttcap%
\pgfsetroundjoin%
\definecolor{currentfill}{rgb}{0.000000,0.000000,0.000000}%
\pgfsetfillcolor{currentfill}%
\pgfsetlinewidth{0.602250pt}%
\definecolor{currentstroke}{rgb}{0.000000,0.000000,0.000000}%
\pgfsetstrokecolor{currentstroke}%
\pgfsetdash{}{0pt}%
\pgfsys@defobject{currentmarker}{\pgfqpoint{-0.027778in}{0.000000in}}{\pgfqpoint{-0.000000in}{0.000000in}}{%
\pgfpathmoveto{\pgfqpoint{-0.000000in}{0.000000in}}%
\pgfpathlineto{\pgfqpoint{-0.027778in}{0.000000in}}%
\pgfusepath{stroke,fill}%
}%
\begin{pgfscope}%
\pgfsys@transformshift{0.588387in}{2.296228in}%
\pgfsys@useobject{currentmarker}{}%
\end{pgfscope}%
\end{pgfscope}%
\begin{pgfscope}%
\pgfsetbuttcap%
\pgfsetroundjoin%
\definecolor{currentfill}{rgb}{0.000000,0.000000,0.000000}%
\pgfsetfillcolor{currentfill}%
\pgfsetlinewidth{0.602250pt}%
\definecolor{currentstroke}{rgb}{0.000000,0.000000,0.000000}%
\pgfsetstrokecolor{currentstroke}%
\pgfsetdash{}{0pt}%
\pgfsys@defobject{currentmarker}{\pgfqpoint{-0.027778in}{0.000000in}}{\pgfqpoint{-0.000000in}{0.000000in}}{%
\pgfpathmoveto{\pgfqpoint{-0.000000in}{0.000000in}}%
\pgfpathlineto{\pgfqpoint{-0.027778in}{0.000000in}}%
\pgfusepath{stroke,fill}%
}%
\begin{pgfscope}%
\pgfsys@transformshift{0.588387in}{2.430351in}%
\pgfsys@useobject{currentmarker}{}%
\end{pgfscope}%
\end{pgfscope}%
\begin{pgfscope}%
\pgfsetbuttcap%
\pgfsetroundjoin%
\definecolor{currentfill}{rgb}{0.000000,0.000000,0.000000}%
\pgfsetfillcolor{currentfill}%
\pgfsetlinewidth{0.602250pt}%
\definecolor{currentstroke}{rgb}{0.000000,0.000000,0.000000}%
\pgfsetstrokecolor{currentstroke}%
\pgfsetdash{}{0pt}%
\pgfsys@defobject{currentmarker}{\pgfqpoint{-0.027778in}{0.000000in}}{\pgfqpoint{-0.000000in}{0.000000in}}{%
\pgfpathmoveto{\pgfqpoint{-0.000000in}{0.000000in}}%
\pgfpathlineto{\pgfqpoint{-0.027778in}{0.000000in}}%
\pgfusepath{stroke,fill}%
}%
\begin{pgfscope}%
\pgfsys@transformshift{0.588387in}{2.498456in}%
\pgfsys@useobject{currentmarker}{}%
\end{pgfscope}%
\end{pgfscope}%
\begin{pgfscope}%
\definecolor{textcolor}{rgb}{0.000000,0.000000,0.000000}%
\pgfsetstrokecolor{textcolor}%
\pgfsetfillcolor{textcolor}%
\pgftext[x=0.234413in,y=1.526746in,,bottom,rotate=90.000000]{\color{textcolor}{\rmfamily\fontsize{10.000000}{12.000000}\selectfont\catcode`\^=\active\def^{\ifmmode\sp\else\^{}\fi}\catcode`\%=\active\def%{\%}Checks [call]}}%
\end{pgfscope}%
\begin{pgfscope}%
\pgfpathrectangle{\pgfqpoint{0.588387in}{0.521603in}}{\pgfqpoint{4.669024in}{2.010285in}}%
\pgfusepath{clip}%
\pgfsetrectcap%
\pgfsetroundjoin%
\pgfsetlinewidth{1.505625pt}%
\pgfsetstrokecolor{currentstroke1}%
\pgfsetdash{}{0pt}%
\pgfpathmoveto{\pgfqpoint{0.800616in}{0.612980in}}%
\pgfpathlineto{\pgfqpoint{0.825151in}{0.612980in}}%
\pgfpathlineto{\pgfqpoint{0.874221in}{0.810956in}}%
\pgfpathlineto{\pgfqpoint{0.923291in}{0.913937in}}%
\pgfpathlineto{\pgfqpoint{0.972361in}{0.899322in}}%
\pgfpathlineto{\pgfqpoint{1.045966in}{1.086379in}}%
\pgfpathlineto{\pgfqpoint{1.070501in}{1.009922in}}%
\pgfpathlineto{\pgfqpoint{1.119572in}{1.042418in}}%
\pgfpathlineto{\pgfqpoint{1.193177in}{1.040546in}}%
\pgfpathlineto{\pgfqpoint{1.242247in}{1.126842in}}%
\pgfpathlineto{\pgfqpoint{1.315852in}{1.111913in}}%
\pgfpathlineto{\pgfqpoint{1.364922in}{1.138559in}}%
\pgfpathlineto{\pgfqpoint{1.438527in}{1.179504in}}%
\pgfpathlineto{\pgfqpoint{1.512133in}{1.194936in}}%
\pgfpathlineto{\pgfqpoint{1.610273in}{1.252376in}}%
\pgfpathlineto{\pgfqpoint{1.683878in}{1.290601in}}%
\pgfpathlineto{\pgfqpoint{1.757483in}{1.252376in}}%
\pgfpathlineto{\pgfqpoint{1.831089in}{1.287854in}}%
\pgfpathlineto{\pgfqpoint{1.904694in}{1.260189in}}%
\pgfpathlineto{\pgfqpoint{2.002834in}{1.234678in}}%
\pgfpathlineto{\pgfqpoint{2.076439in}{1.255305in}}%
\pgfpathlineto{\pgfqpoint{2.174580in}{1.262339in}}%
\pgfpathlineto{\pgfqpoint{2.272720in}{1.272573in}}%
\pgfpathlineto{\pgfqpoint{2.346325in}{1.310220in}}%
\pgfpathlineto{\pgfqpoint{2.444465in}{1.529297in}}%
\pgfpathlineto{\pgfqpoint{2.542606in}{1.308224in}}%
\pgfpathlineto{\pgfqpoint{2.640746in}{1.486159in}}%
\pgfpathlineto{\pgfqpoint{2.763421in}{1.371578in}}%
\pgfpathlineto{\pgfqpoint{2.812491in}{1.367278in}}%
\pgfpathlineto{\pgfqpoint{2.886097in}{1.392140in}}%
\pgfpathlineto{\pgfqpoint{3.008772in}{1.408581in}}%
\pgfpathlineto{\pgfqpoint{3.106912in}{1.436257in}}%
\pgfpathlineto{\pgfqpoint{3.229588in}{1.431626in}}%
\pgfpathlineto{\pgfqpoint{3.352263in}{1.403839in}}%
\pgfpathlineto{\pgfqpoint{3.474938in}{1.360186in}}%
\pgfpathlineto{\pgfqpoint{3.622149in}{1.434084in}}%
\pgfpathlineto{\pgfqpoint{3.720289in}{1.505012in}}%
\pgfpathlineto{\pgfqpoint{3.842964in}{1.400588in}}%
\pgfpathlineto{\pgfqpoint{3.965640in}{1.385234in}}%
\pgfpathlineto{\pgfqpoint{4.088315in}{1.610957in}}%
\pgfpathlineto{\pgfqpoint{4.235526in}{1.397470in}}%
\pgfpathlineto{\pgfqpoint{4.358201in}{1.651543in}}%
\pgfpathlineto{\pgfqpoint{4.505411in}{1.476858in}}%
\pgfpathlineto{\pgfqpoint{4.652622in}{1.582917in}}%
\pgfpathlineto{\pgfqpoint{4.799832in}{2.050800in}}%
\pgfpathlineto{\pgfqpoint{4.971578in}{1.435030in}}%
\pgfpathlineto{\pgfqpoint{4.996113in}{1.494398in}}%
\pgfpathlineto{\pgfqpoint{5.045183in}{1.679747in}}%
\pgfusepath{stroke}%
\end{pgfscope}%
\begin{pgfscope}%
\pgfpathrectangle{\pgfqpoint{0.588387in}{0.521603in}}{\pgfqpoint{4.669024in}{2.010285in}}%
\pgfusepath{clip}%
\pgfsetrectcap%
\pgfsetroundjoin%
\pgfsetlinewidth{1.505625pt}%
\pgfsetstrokecolor{currentstroke2}%
\pgfsetdash{}{0pt}%
\pgfpathmoveto{\pgfqpoint{0.800616in}{0.612980in}}%
\pgfpathlineto{\pgfqpoint{0.825151in}{0.612980in}}%
\pgfpathlineto{\pgfqpoint{0.874221in}{0.810956in}}%
\pgfpathlineto{\pgfqpoint{0.923291in}{0.913937in}}%
\pgfpathlineto{\pgfqpoint{0.972361in}{0.951418in}}%
\pgfpathlineto{\pgfqpoint{1.045966in}{1.030363in}}%
\pgfpathlineto{\pgfqpoint{1.070501in}{1.162640in}}%
\pgfpathlineto{\pgfqpoint{1.119572in}{1.639228in}}%
\pgfpathlineto{\pgfqpoint{1.193177in}{1.043808in}}%
\pgfpathlineto{\pgfqpoint{1.242247in}{2.103597in}}%
\pgfpathlineto{\pgfqpoint{1.315852in}{2.382649in}}%
\pgfpathlineto{\pgfqpoint{1.364922in}{1.656441in}}%
\pgfpathlineto{\pgfqpoint{1.438527in}{2.051734in}}%
\pgfpathlineto{\pgfqpoint{1.512133in}{1.831164in}}%
\pgfpathlineto{\pgfqpoint{1.831089in}{1.391056in}}%
\pgfpathlineto{\pgfqpoint{1.904694in}{2.105564in}}%
\pgfpathlineto{\pgfqpoint{2.076439in}{1.222867in}}%
\pgfpathlineto{\pgfqpoint{2.174580in}{2.105846in}}%
\pgfpathlineto{\pgfqpoint{2.346325in}{1.739652in}}%
\pgfpathlineto{\pgfqpoint{2.640746in}{1.800881in}}%
\pgfpathlineto{\pgfqpoint{2.812491in}{1.797656in}}%
\pgfpathlineto{\pgfqpoint{2.886097in}{1.899001in}}%
\pgfpathlineto{\pgfqpoint{3.008772in}{1.799988in}}%
\pgfpathlineto{\pgfqpoint{3.106912in}{1.739117in}}%
\pgfpathlineto{\pgfqpoint{3.229588in}{1.633634in}}%
\pgfpathlineto{\pgfqpoint{4.996113in}{1.776893in}}%
\pgfpathlineto{\pgfqpoint{5.045183in}{1.779769in}}%
\pgfusepath{stroke}%
\end{pgfscope}%
\begin{pgfscope}%
\pgfpathrectangle{\pgfqpoint{0.588387in}{0.521603in}}{\pgfqpoint{4.669024in}{2.010285in}}%
\pgfusepath{clip}%
\pgfsetrectcap%
\pgfsetroundjoin%
\pgfsetlinewidth{1.505625pt}%
\pgfsetstrokecolor{currentstroke3}%
\pgfsetdash{}{0pt}%
\pgfpathmoveto{\pgfqpoint{0.800616in}{0.612980in}}%
\pgfpathlineto{\pgfqpoint{0.825151in}{0.612980in}}%
\pgfpathlineto{\pgfqpoint{0.874221in}{0.742851in}}%
\pgfpathlineto{\pgfqpoint{0.923291in}{0.845832in}}%
\pgfpathlineto{\pgfqpoint{0.972361in}{0.992225in}}%
\pgfpathlineto{\pgfqpoint{1.045966in}{1.074701in}}%
\pgfpathlineto{\pgfqpoint{1.070501in}{1.138036in}}%
\pgfpathlineto{\pgfqpoint{1.119572in}{1.073255in}}%
\pgfpathlineto{\pgfqpoint{1.193177in}{1.174546in}}%
\pgfpathlineto{\pgfqpoint{1.242247in}{1.240453in}}%
\pgfpathlineto{\pgfqpoint{1.315852in}{1.174175in}}%
\pgfpathlineto{\pgfqpoint{1.364922in}{1.244235in}}%
\pgfpathlineto{\pgfqpoint{1.438527in}{1.202357in}}%
\pgfpathlineto{\pgfqpoint{1.512133in}{1.271096in}}%
\pgfpathlineto{\pgfqpoint{1.610273in}{1.261457in}}%
\pgfpathlineto{\pgfqpoint{1.683878in}{1.303014in}}%
\pgfpathlineto{\pgfqpoint{1.757483in}{1.285657in}}%
\pgfpathlineto{\pgfqpoint{1.831089in}{1.335165in}}%
\pgfpathlineto{\pgfqpoint{1.904694in}{1.299995in}}%
\pgfpathlineto{\pgfqpoint{2.002834in}{1.305752in}}%
\pgfpathlineto{\pgfqpoint{2.076439in}{1.334435in}}%
\pgfpathlineto{\pgfqpoint{2.174580in}{1.342871in}}%
\pgfpathlineto{\pgfqpoint{2.272720in}{1.382103in}}%
\pgfpathlineto{\pgfqpoint{2.346325in}{1.362906in}}%
\pgfpathlineto{\pgfqpoint{2.444465in}{1.358871in}}%
\pgfpathlineto{\pgfqpoint{2.542606in}{1.392142in}}%
\pgfpathlineto{\pgfqpoint{2.640746in}{1.406531in}}%
\pgfpathlineto{\pgfqpoint{2.763421in}{1.416070in}}%
\pgfpathlineto{\pgfqpoint{2.812491in}{1.419899in}}%
\pgfpathlineto{\pgfqpoint{2.886097in}{1.399700in}}%
\pgfpathlineto{\pgfqpoint{3.008772in}{1.436304in}}%
\pgfpathlineto{\pgfqpoint{3.106912in}{1.445261in}}%
\pgfpathlineto{\pgfqpoint{3.229588in}{1.435896in}}%
\pgfpathlineto{\pgfqpoint{3.352263in}{1.454277in}}%
\pgfpathlineto{\pgfqpoint{3.474938in}{1.453762in}}%
\pgfpathlineto{\pgfqpoint{3.622149in}{1.464303in}}%
\pgfpathlineto{\pgfqpoint{3.720289in}{1.486545in}}%
\pgfpathlineto{\pgfqpoint{3.842964in}{1.484839in}}%
\pgfpathlineto{\pgfqpoint{3.965640in}{1.374309in}}%
\pgfpathlineto{\pgfqpoint{4.088315in}{1.483705in}}%
\pgfpathlineto{\pgfqpoint{4.235526in}{1.370196in}}%
\pgfpathlineto{\pgfqpoint{4.358201in}{1.428782in}}%
\pgfpathlineto{\pgfqpoint{4.505411in}{1.466751in}}%
\pgfpathlineto{\pgfqpoint{4.652622in}{1.495411in}}%
\pgfpathlineto{\pgfqpoint{4.799832in}{1.477836in}}%
\pgfpathlineto{\pgfqpoint{4.971578in}{1.471121in}}%
\pgfpathlineto{\pgfqpoint{4.996113in}{1.507992in}}%
\pgfpathlineto{\pgfqpoint{5.045183in}{1.433943in}}%
\pgfusepath{stroke}%
\end{pgfscope}%
\begin{pgfscope}%
\pgfpathrectangle{\pgfqpoint{0.588387in}{0.521603in}}{\pgfqpoint{4.669024in}{2.010285in}}%
\pgfusepath{clip}%
\pgfsetrectcap%
\pgfsetroundjoin%
\pgfsetlinewidth{1.505625pt}%
\pgfsetstrokecolor{currentstroke4}%
\pgfsetdash{}{0pt}%
\pgfpathmoveto{\pgfqpoint{0.800616in}{0.612980in}}%
\pgfpathlineto{\pgfqpoint{0.825151in}{0.612980in}}%
\pgfpathlineto{\pgfqpoint{0.874221in}{0.766887in}}%
\pgfpathlineto{\pgfqpoint{0.923291in}{0.817295in}}%
\pgfpathlineto{\pgfqpoint{0.972361in}{0.982042in}}%
\pgfpathlineto{\pgfqpoint{1.045966in}{1.082576in}}%
\pgfpathlineto{\pgfqpoint{1.070501in}{1.129772in}}%
\pgfpathlineto{\pgfqpoint{1.119572in}{1.060904in}}%
\pgfpathlineto{\pgfqpoint{1.193177in}{1.746774in}}%
\pgfpathlineto{\pgfqpoint{1.242247in}{1.639193in}}%
\pgfpathlineto{\pgfqpoint{1.315852in}{1.296085in}}%
\pgfpathlineto{\pgfqpoint{1.364922in}{1.237297in}}%
\pgfpathlineto{\pgfqpoint{1.438527in}{1.278182in}}%
\pgfpathlineto{\pgfqpoint{1.512133in}{1.311537in}}%
\pgfpathlineto{\pgfqpoint{1.610273in}{2.039473in}}%
\pgfpathlineto{\pgfqpoint{1.683878in}{1.785484in}}%
\pgfpathlineto{\pgfqpoint{1.757483in}{1.440700in}}%
\pgfpathlineto{\pgfqpoint{1.831089in}{1.574539in}}%
\pgfpathlineto{\pgfqpoint{1.904694in}{1.286952in}}%
\pgfpathlineto{\pgfqpoint{2.002834in}{1.369268in}}%
\pgfpathlineto{\pgfqpoint{2.076439in}{1.319909in}}%
\pgfpathlineto{\pgfqpoint{2.174580in}{1.323868in}}%
\pgfpathlineto{\pgfqpoint{2.272720in}{1.348755in}}%
\pgfpathlineto{\pgfqpoint{2.346325in}{1.350845in}}%
\pgfpathlineto{\pgfqpoint{2.444465in}{1.343956in}}%
\pgfpathlineto{\pgfqpoint{2.542606in}{1.370427in}}%
\pgfpathlineto{\pgfqpoint{2.640746in}{1.497321in}}%
\pgfpathlineto{\pgfqpoint{2.763421in}{1.573669in}}%
\pgfpathlineto{\pgfqpoint{2.812491in}{1.589314in}}%
\pgfpathlineto{\pgfqpoint{2.886097in}{1.721517in}}%
\pgfpathlineto{\pgfqpoint{3.008772in}{1.623232in}}%
\pgfpathlineto{\pgfqpoint{3.106912in}{1.752741in}}%
\pgfpathlineto{\pgfqpoint{3.229588in}{1.704055in}}%
\pgfpathlineto{\pgfqpoint{3.352263in}{1.540490in}}%
\pgfpathlineto{\pgfqpoint{3.474938in}{1.628792in}}%
\pgfpathlineto{\pgfqpoint{3.622149in}{1.591062in}}%
\pgfpathlineto{\pgfqpoint{3.720289in}{1.630618in}}%
\pgfpathlineto{\pgfqpoint{3.842964in}{1.602585in}}%
\pgfpathlineto{\pgfqpoint{4.088315in}{1.711963in}}%
\pgfpathlineto{\pgfqpoint{4.358201in}{1.432015in}}%
\pgfpathlineto{\pgfqpoint{4.652622in}{1.495411in}}%
\pgfpathlineto{\pgfqpoint{4.971578in}{1.419210in}}%
\pgfpathlineto{\pgfqpoint{4.996113in}{1.911496in}}%
\pgfpathlineto{\pgfqpoint{5.045183in}{1.764997in}}%
\pgfusepath{stroke}%
\end{pgfscope}%
\begin{pgfscope}%
\pgfpathrectangle{\pgfqpoint{0.588387in}{0.521603in}}{\pgfqpoint{4.669024in}{2.010285in}}%
\pgfusepath{clip}%
\pgfsetrectcap%
\pgfsetroundjoin%
\pgfsetlinewidth{1.505625pt}%
\pgfsetstrokecolor{currentstroke5}%
\pgfsetdash{}{0pt}%
\pgfpathmoveto{\pgfqpoint{0.800616in}{0.612980in}}%
\pgfpathlineto{\pgfqpoint{0.825151in}{0.612980in}}%
\pgfpathlineto{\pgfqpoint{0.874221in}{0.706977in}}%
\pgfpathlineto{\pgfqpoint{0.923291in}{0.834992in}}%
\pgfpathlineto{\pgfqpoint{0.972361in}{0.967427in}}%
\pgfpathlineto{\pgfqpoint{1.045966in}{1.078685in}}%
\pgfpathlineto{\pgfqpoint{1.070501in}{1.132175in}}%
\pgfpathlineto{\pgfqpoint{1.119572in}{1.078497in}}%
\pgfpathlineto{\pgfqpoint{1.193177in}{1.150681in}}%
\pgfpathlineto{\pgfqpoint{1.242247in}{1.283097in}}%
\pgfpathlineto{\pgfqpoint{1.315852in}{1.213724in}}%
\pgfpathlineto{\pgfqpoint{1.364922in}{1.282911in}}%
\pgfpathlineto{\pgfqpoint{1.438527in}{1.225821in}}%
\pgfpathlineto{\pgfqpoint{1.512133in}{1.313516in}}%
\pgfpathlineto{\pgfqpoint{1.610273in}{1.292816in}}%
\pgfpathlineto{\pgfqpoint{1.683878in}{1.345034in}}%
\pgfpathlineto{\pgfqpoint{1.757483in}{1.314249in}}%
\pgfpathlineto{\pgfqpoint{1.831089in}{1.367553in}}%
\pgfpathlineto{\pgfqpoint{1.904694in}{1.341308in}}%
\pgfpathlineto{\pgfqpoint{2.002834in}{1.346373in}}%
\pgfpathlineto{\pgfqpoint{2.076439in}{1.378838in}}%
\pgfpathlineto{\pgfqpoint{2.174580in}{1.365860in}}%
\pgfpathlineto{\pgfqpoint{2.272720in}{1.423138in}}%
\pgfpathlineto{\pgfqpoint{2.346325in}{1.398744in}}%
\pgfpathlineto{\pgfqpoint{2.444465in}{1.390851in}}%
\pgfpathlineto{\pgfqpoint{2.542606in}{1.435971in}}%
\pgfpathlineto{\pgfqpoint{2.640746in}{1.434005in}}%
\pgfpathlineto{\pgfqpoint{2.763421in}{1.432015in}}%
\pgfpathlineto{\pgfqpoint{2.812491in}{1.449584in}}%
\pgfpathlineto{\pgfqpoint{2.886097in}{1.442762in}}%
\pgfpathlineto{\pgfqpoint{3.008772in}{1.476367in}}%
\pgfpathlineto{\pgfqpoint{3.106912in}{1.469963in}}%
\pgfpathlineto{\pgfqpoint{3.229588in}{1.479465in}}%
\pgfpathlineto{\pgfqpoint{3.352263in}{1.473076in}}%
\pgfpathlineto{\pgfqpoint{3.474938in}{1.481930in}}%
\pgfpathlineto{\pgfqpoint{3.622149in}{1.485694in}}%
\pgfpathlineto{\pgfqpoint{3.720289in}{1.514388in}}%
\pgfpathlineto{\pgfqpoint{3.842964in}{1.512712in}}%
\pgfusepath{stroke}%
\end{pgfscope}%
\begin{pgfscope}%
\pgfpathrectangle{\pgfqpoint{0.588387in}{0.521603in}}{\pgfqpoint{4.669024in}{2.010285in}}%
\pgfusepath{clip}%
\pgfsetrectcap%
\pgfsetroundjoin%
\pgfsetlinewidth{1.505625pt}%
\pgfsetstrokecolor{currentstroke6}%
\pgfsetdash{}{0pt}%
\pgfpathmoveto{\pgfqpoint{0.800616in}{0.612980in}}%
\pgfpathlineto{\pgfqpoint{0.825151in}{0.612980in}}%
\pgfpathlineto{\pgfqpoint{0.874221in}{0.761193in}}%
\pgfpathlineto{\pgfqpoint{0.923291in}{0.845832in}}%
\pgfpathlineto{\pgfqpoint{0.972361in}{0.982042in}}%
\pgfpathlineto{\pgfqpoint{1.045966in}{1.097298in}}%
\pgfpathlineto{\pgfqpoint{1.070501in}{1.146790in}}%
\pgfpathlineto{\pgfqpoint{1.119572in}{1.114053in}}%
\pgfpathlineto{\pgfqpoint{1.193177in}{1.215477in}}%
\pgfpathlineto{\pgfqpoint{1.242247in}{1.247243in}}%
\pgfpathlineto{\pgfqpoint{1.315852in}{1.217595in}}%
\pgfpathlineto{\pgfqpoint{1.364922in}{1.296074in}}%
\pgfpathlineto{\pgfqpoint{1.438527in}{1.233390in}}%
\pgfpathlineto{\pgfqpoint{1.512133in}{1.314012in}}%
\pgfpathlineto{\pgfqpoint{1.610273in}{1.283000in}}%
\pgfpathlineto{\pgfqpoint{1.683878in}{1.355033in}}%
\pgfpathlineto{\pgfqpoint{1.757483in}{1.336175in}}%
\pgfpathlineto{\pgfqpoint{1.831089in}{1.393803in}}%
\pgfpathlineto{\pgfqpoint{1.904694in}{1.368535in}}%
\pgfpathlineto{\pgfqpoint{2.002834in}{1.425983in}}%
\pgfpathlineto{\pgfqpoint{2.076439in}{1.478996in}}%
\pgfpathlineto{\pgfqpoint{2.174580in}{1.397011in}}%
\pgfpathlineto{\pgfqpoint{2.272720in}{1.584302in}}%
\pgfpathlineto{\pgfqpoint{2.346325in}{1.447138in}}%
\pgfpathlineto{\pgfqpoint{2.444465in}{1.504319in}}%
\pgfpathlineto{\pgfqpoint{2.542606in}{1.483036in}}%
\pgfpathlineto{\pgfqpoint{2.640746in}{1.441206in}}%
\pgfpathlineto{\pgfqpoint{2.763421in}{1.527734in}}%
\pgfpathlineto{\pgfqpoint{2.812491in}{1.615162in}}%
\pgfpathlineto{\pgfqpoint{2.886097in}{1.547434in}}%
\pgfpathlineto{\pgfqpoint{3.008772in}{1.495660in}}%
\pgfpathlineto{\pgfqpoint{3.106912in}{1.569014in}}%
\pgfpathlineto{\pgfqpoint{3.229588in}{1.500994in}}%
\pgfpathlineto{\pgfqpoint{3.352263in}{1.487353in}}%
\pgfpathlineto{\pgfqpoint{3.474938in}{1.691353in}}%
\pgfpathlineto{\pgfqpoint{3.622149in}{1.518424in}}%
\pgfpathlineto{\pgfqpoint{3.720289in}{1.543292in}}%
\pgfpathlineto{\pgfqpoint{3.842964in}{1.662610in}}%
\pgfpathlineto{\pgfqpoint{3.965640in}{1.490736in}}%
\pgfpathlineto{\pgfqpoint{4.088315in}{1.957432in}}%
\pgfpathlineto{\pgfqpoint{4.235526in}{1.602585in}}%
\pgfpathlineto{\pgfqpoint{4.358201in}{1.525393in}}%
\pgfpathlineto{\pgfqpoint{4.505411in}{1.971260in}}%
\pgfpathlineto{\pgfqpoint{4.652622in}{1.591311in}}%
\pgfpathlineto{\pgfqpoint{4.799832in}{1.673679in}}%
\pgfpathlineto{\pgfqpoint{4.971578in}{1.562562in}}%
\pgfpathlineto{\pgfqpoint{4.996113in}{1.688271in}}%
\pgfpathlineto{\pgfqpoint{5.045183in}{2.050426in}}%
\pgfusepath{stroke}%
\end{pgfscope}%
\begin{pgfscope}%
\pgfpathrectangle{\pgfqpoint{0.588387in}{0.521603in}}{\pgfqpoint{4.669024in}{2.010285in}}%
\pgfusepath{clip}%
\pgfsetrectcap%
\pgfsetroundjoin%
\pgfsetlinewidth{1.505625pt}%
\pgfsetstrokecolor{currentstroke7}%
\pgfsetdash{}{0pt}%
\pgfpathmoveto{\pgfqpoint{0.800616in}{0.612980in}}%
\pgfpathlineto{\pgfqpoint{0.825151in}{0.612980in}}%
\pgfpathlineto{\pgfqpoint{0.874221in}{0.775082in}}%
\pgfpathlineto{\pgfqpoint{0.923291in}{0.845832in}}%
\pgfpathlineto{\pgfqpoint{0.972361in}{1.150253in}}%
\pgfpathlineto{\pgfqpoint{1.045966in}{1.297874in}}%
\pgfpathlineto{\pgfqpoint{1.070501in}{1.464523in}}%
\pgfpathlineto{\pgfqpoint{1.119572in}{1.543332in}}%
\pgfpathlineto{\pgfqpoint{1.193177in}{1.723898in}}%
\pgfpathlineto{\pgfqpoint{1.242247in}{1.257204in}}%
\pgfpathlineto{\pgfqpoint{1.315852in}{2.161235in}}%
\pgfpathlineto{\pgfqpoint{1.364922in}{2.440512in}}%
\pgfpathlineto{\pgfqpoint{1.438527in}{2.321463in}}%
\pgfpathlineto{\pgfqpoint{1.512133in}{2.282906in}}%
\pgfpathlineto{\pgfqpoint{1.831089in}{1.870911in}}%
\pgfpathlineto{\pgfqpoint{2.812491in}{1.612713in}}%
\pgfpathlineto{\pgfqpoint{2.886097in}{1.643126in}}%
\pgfpathlineto{\pgfqpoint{3.106912in}{1.910269in}}%
\pgfpathlineto{\pgfqpoint{3.229588in}{1.414300in}}%
\pgfusepath{stroke}%
\end{pgfscope}%
\begin{pgfscope}%
\pgfpathrectangle{\pgfqpoint{0.588387in}{0.521603in}}{\pgfqpoint{4.669024in}{2.010285in}}%
\pgfusepath{clip}%
\pgfsetrectcap%
\pgfsetroundjoin%
\pgfsetlinewidth{1.505625pt}%
\definecolor{currentstroke}{rgb}{0.498039,0.498039,0.498039}%
\pgfsetstrokecolor{currentstroke}%
\pgfsetdash{}{0pt}%
\pgfpathmoveto{\pgfqpoint{0.923291in}{0.706977in}}%
\pgfpathlineto{\pgfqpoint{0.972361in}{0.706977in}}%
\pgfpathlineto{\pgfqpoint{1.045966in}{0.706977in}}%
\pgfpathlineto{\pgfqpoint{1.070501in}{0.706977in}}%
\pgfpathlineto{\pgfqpoint{1.119572in}{0.754439in}}%
\pgfpathlineto{\pgfqpoint{1.193177in}{0.706977in}}%
\pgfpathlineto{\pgfqpoint{1.242247in}{0.782896in}}%
\pgfpathlineto{\pgfqpoint{1.315852in}{0.706977in}}%
\pgfpathlineto{\pgfqpoint{1.364922in}{0.793650in}}%
\pgfpathlineto{\pgfqpoint{1.438527in}{0.791403in}}%
\pgfpathlineto{\pgfqpoint{1.512133in}{0.809224in}}%
\pgfpathlineto{\pgfqpoint{1.610273in}{0.834992in}}%
\pgfpathlineto{\pgfqpoint{1.683878in}{0.834992in}}%
\pgfpathlineto{\pgfqpoint{1.757483in}{0.956926in}}%
\pgfpathlineto{\pgfqpoint{1.831089in}{0.925215in}}%
\pgfpathlineto{\pgfqpoint{1.904694in}{0.944061in}}%
\pgfpathlineto{\pgfqpoint{2.002834in}{0.956926in}}%
\pgfpathlineto{\pgfqpoint{2.076439in}{0.956926in}}%
\pgfpathlineto{\pgfqpoint{2.174580in}{0.956926in}}%
\pgfpathlineto{\pgfqpoint{2.272720in}{0.956926in}}%
\pgfpathlineto{\pgfqpoint{2.346325in}{0.956926in}}%
\pgfpathlineto{\pgfqpoint{2.444465in}{0.956926in}}%
\pgfpathlineto{\pgfqpoint{2.542606in}{0.956926in}}%
\pgfpathlineto{\pgfqpoint{2.640746in}{0.956926in}}%
\pgfpathlineto{\pgfqpoint{2.763421in}{0.956926in}}%
\pgfpathlineto{\pgfqpoint{2.812491in}{1.006596in}}%
\pgfpathlineto{\pgfqpoint{2.886097in}{1.022675in}}%
\pgfpathlineto{\pgfqpoint{3.008772in}{1.037891in}}%
\pgfpathlineto{\pgfqpoint{3.106912in}{1.063751in}}%
\pgfpathlineto{\pgfqpoint{3.229588in}{1.076040in}}%
\pgfpathlineto{\pgfqpoint{3.352263in}{1.026827in}}%
\pgfpathlineto{\pgfqpoint{3.474938in}{1.076040in}}%
\pgfpathlineto{\pgfqpoint{3.622149in}{1.076040in}}%
\pgfpathlineto{\pgfqpoint{3.720289in}{1.076040in}}%
\pgfpathlineto{\pgfqpoint{3.842964in}{1.076040in}}%
\pgfpathlineto{\pgfqpoint{3.965640in}{1.076040in}}%
\pgfpathlineto{\pgfqpoint{4.088315in}{1.076040in}}%
\pgfpathlineto{\pgfqpoint{4.235526in}{1.076040in}}%
\pgfpathlineto{\pgfqpoint{4.358201in}{1.145031in}}%
\pgfpathlineto{\pgfqpoint{4.505411in}{1.193794in}}%
\pgfpathlineto{\pgfqpoint{4.652622in}{1.193794in}}%
\pgfpathlineto{\pgfqpoint{4.799832in}{1.193794in}}%
\pgfpathlineto{\pgfqpoint{4.971578in}{1.193794in}}%
\pgfpathlineto{\pgfqpoint{4.996113in}{1.193794in}}%
\pgfpathlineto{\pgfqpoint{5.045183in}{1.201616in}}%
\pgfusepath{stroke}%
\end{pgfscope}%
\begin{pgfscope}%
\pgfsetrectcap%
\pgfsetmiterjoin%
\pgfsetlinewidth{0.803000pt}%
\definecolor{currentstroke}{rgb}{0.000000,0.000000,0.000000}%
\pgfsetstrokecolor{currentstroke}%
\pgfsetdash{}{0pt}%
\pgfpathmoveto{\pgfqpoint{0.588387in}{0.521603in}}%
\pgfpathlineto{\pgfqpoint{0.588387in}{2.531888in}}%
\pgfusepath{stroke}%
\end{pgfscope}%
\begin{pgfscope}%
\pgfsetrectcap%
\pgfsetmiterjoin%
\pgfsetlinewidth{0.803000pt}%
\definecolor{currentstroke}{rgb}{0.000000,0.000000,0.000000}%
\pgfsetstrokecolor{currentstroke}%
\pgfsetdash{}{0pt}%
\pgfpathmoveto{\pgfqpoint{5.257411in}{0.521603in}}%
\pgfpathlineto{\pgfqpoint{5.257411in}{2.531888in}}%
\pgfusepath{stroke}%
\end{pgfscope}%
\begin{pgfscope}%
\pgfsetrectcap%
\pgfsetmiterjoin%
\pgfsetlinewidth{0.803000pt}%
\definecolor{currentstroke}{rgb}{0.000000,0.000000,0.000000}%
\pgfsetstrokecolor{currentstroke}%
\pgfsetdash{}{0pt}%
\pgfpathmoveto{\pgfqpoint{0.588387in}{0.521603in}}%
\pgfpathlineto{\pgfqpoint{5.257411in}{0.521603in}}%
\pgfusepath{stroke}%
\end{pgfscope}%
\begin{pgfscope}%
\pgfsetrectcap%
\pgfsetmiterjoin%
\pgfsetlinewidth{0.803000pt}%
\definecolor{currentstroke}{rgb}{0.000000,0.000000,0.000000}%
\pgfsetstrokecolor{currentstroke}%
\pgfsetdash{}{0pt}%
\pgfpathmoveto{\pgfqpoint{0.588387in}{2.531888in}}%
\pgfpathlineto{\pgfqpoint{5.257411in}{2.531888in}}%
\pgfusepath{stroke}%
\end{pgfscope}%
\begin{pgfscope}%
\definecolor{textcolor}{rgb}{0.000000,0.000000,0.000000}%
\pgfsetstrokecolor{textcolor}%
\pgfsetfillcolor{textcolor}%
\pgftext[x=2.922899in,y=2.615222in,,base]{\color{textcolor}{\rmfamily\fontsize{12.000000}{14.400000}\selectfont\catcode`\^=\active\def^{\ifmmode\sp\else\^{}\fi}\catcode`\%=\active\def%{\%}Mean}}%
\end{pgfscope}%
\begin{pgfscope}%
\pgfsetbuttcap%
\pgfsetmiterjoin%
\definecolor{currentfill}{rgb}{1.000000,1.000000,1.000000}%
\pgfsetfillcolor{currentfill}%
\pgfsetfillopacity{0.800000}%
\pgfsetlinewidth{1.003750pt}%
\definecolor{currentstroke}{rgb}{0.800000,0.800000,0.800000}%
\pgfsetstrokecolor{currentstroke}%
\pgfsetstrokeopacity{0.800000}%
\pgfsetdash{}{0pt}%
\pgfpathmoveto{\pgfqpoint{5.344911in}{0.946722in}}%
\pgfpathlineto{\pgfqpoint{8.259376in}{0.946722in}}%
\pgfpathquadraticcurveto{\pgfqpoint{8.284376in}{0.946722in}}{\pgfqpoint{8.284376in}{0.971722in}}%
\pgfpathlineto{\pgfqpoint{8.284376in}{2.444388in}}%
\pgfpathquadraticcurveto{\pgfqpoint{8.284376in}{2.469388in}}{\pgfqpoint{8.259376in}{2.469388in}}%
\pgfpathlineto{\pgfqpoint{5.344911in}{2.469388in}}%
\pgfpathquadraticcurveto{\pgfqpoint{5.319911in}{2.469388in}}{\pgfqpoint{5.319911in}{2.444388in}}%
\pgfpathlineto{\pgfqpoint{5.319911in}{0.971722in}}%
\pgfpathquadraticcurveto{\pgfqpoint{5.319911in}{0.946722in}}{\pgfqpoint{5.344911in}{0.946722in}}%
\pgfpathlineto{\pgfqpoint{5.344911in}{0.946722in}}%
\pgfpathclose%
\pgfusepath{stroke,fill}%
\end{pgfscope}%
\begin{pgfscope}%
\pgfsetrectcap%
\pgfsetroundjoin%
\pgfsetlinewidth{1.505625pt}%
\definecolor{currentstroke}{rgb}{0.498039,0.498039,0.498039}%
\pgfsetstrokecolor{currentstroke}%
\pgfsetdash{}{0pt}%
\pgfpathmoveto{\pgfqpoint{5.369911in}{2.368168in}}%
\pgfpathlineto{\pgfqpoint{5.494911in}{2.368168in}}%
\pgfpathlineto{\pgfqpoint{5.619911in}{2.368168in}}%
\pgfusepath{stroke}%
\end{pgfscope}%
\begin{pgfscope}%
\definecolor{textcolor}{rgb}{0.000000,0.000000,0.000000}%
\pgfsetstrokecolor{textcolor}%
\pgfsetfillcolor{textcolor}%
\pgftext[x=5.719911in,y=2.324418in,left,base]{\color{textcolor}{\rmfamily\fontsize{9.000000}{10.800000}\selectfont\catcode`\^=\active\def^{\ifmmode\sp\else\^{}\fi}\catcode`\%=\active\def%{\%}\NaiveCycles{}}}%
\end{pgfscope}%
\begin{pgfscope}%
\pgfsetrectcap%
\pgfsetroundjoin%
\pgfsetlinewidth{1.505625pt}%
\pgfsetstrokecolor{currentstroke1}%
\pgfsetdash{}{0pt}%
\pgfpathmoveto{\pgfqpoint{5.369911in}{2.184696in}}%
\pgfpathlineto{\pgfqpoint{5.494911in}{2.184696in}}%
\pgfpathlineto{\pgfqpoint{5.619911in}{2.184696in}}%
\pgfusepath{stroke}%
\end{pgfscope}%
\begin{pgfscope}%
\definecolor{textcolor}{rgb}{0.000000,0.000000,0.000000}%
\pgfsetstrokecolor{textcolor}%
\pgfsetfillcolor{textcolor}%
\pgftext[x=5.719911in,y=2.140946in,left,base]{\color{textcolor}{\rmfamily\fontsize{9.000000}{10.800000}\selectfont\catcode`\^=\active\def^{\ifmmode\sp\else\^{}\fi}\catcode`\%=\active\def%{\%}\CyclesMatchChunks{} \& \MergeLinear{}}}%
\end{pgfscope}%
\begin{pgfscope}%
\pgfsetrectcap%
\pgfsetroundjoin%
\pgfsetlinewidth{1.505625pt}%
\pgfsetstrokecolor{currentstroke2}%
\pgfsetdash{}{0pt}%
\pgfpathmoveto{\pgfqpoint{5.369911in}{1.997746in}}%
\pgfpathlineto{\pgfqpoint{5.494911in}{1.997746in}}%
\pgfpathlineto{\pgfqpoint{5.619911in}{1.997746in}}%
\pgfusepath{stroke}%
\end{pgfscope}%
\begin{pgfscope}%
\definecolor{textcolor}{rgb}{0.000000,0.000000,0.000000}%
\pgfsetstrokecolor{textcolor}%
\pgfsetfillcolor{textcolor}%
\pgftext[x=5.719911in,y=1.953996in,left,base]{\color{textcolor}{\rmfamily\fontsize{9.000000}{10.800000}\selectfont\catcode`\^=\active\def^{\ifmmode\sp\else\^{}\fi}\catcode`\%=\active\def%{\%}\CyclesMatchChunks{} \& \SharedVertices{}}}%
\end{pgfscope}%
\begin{pgfscope}%
\pgfsetrectcap%
\pgfsetroundjoin%
\pgfsetlinewidth{1.505625pt}%
\pgfsetstrokecolor{currentstroke3}%
\pgfsetdash{}{0pt}%
\pgfpathmoveto{\pgfqpoint{5.369911in}{1.810795in}}%
\pgfpathlineto{\pgfqpoint{5.494911in}{1.810795in}}%
\pgfpathlineto{\pgfqpoint{5.619911in}{1.810795in}}%
\pgfusepath{stroke}%
\end{pgfscope}%
\begin{pgfscope}%
\definecolor{textcolor}{rgb}{0.000000,0.000000,0.000000}%
\pgfsetstrokecolor{textcolor}%
\pgfsetfillcolor{textcolor}%
\pgftext[x=5.719911in,y=1.767045in,left,base]{\color{textcolor}{\rmfamily\fontsize{9.000000}{10.800000}\selectfont\catcode`\^=\active\def^{\ifmmode\sp\else\^{}\fi}\catcode`\%=\active\def%{\%}\Neighbors{} \& \MergeLinear{}}}%
\end{pgfscope}%
\begin{pgfscope}%
\pgfsetrectcap%
\pgfsetroundjoin%
\pgfsetlinewidth{1.505625pt}%
\pgfsetstrokecolor{currentstroke4}%
\pgfsetdash{}{0pt}%
\pgfpathmoveto{\pgfqpoint{5.369911in}{1.627324in}}%
\pgfpathlineto{\pgfqpoint{5.494911in}{1.627324in}}%
\pgfpathlineto{\pgfqpoint{5.619911in}{1.627324in}}%
\pgfusepath{stroke}%
\end{pgfscope}%
\begin{pgfscope}%
\definecolor{textcolor}{rgb}{0.000000,0.000000,0.000000}%
\pgfsetstrokecolor{textcolor}%
\pgfsetfillcolor{textcolor}%
\pgftext[x=5.719911in,y=1.583574in,left,base]{\color{textcolor}{\rmfamily\fontsize{9.000000}{10.800000}\selectfont\catcode`\^=\active\def^{\ifmmode\sp\else\^{}\fi}\catcode`\%=\active\def%{\%}\Neighbors{} \& \SharedVertices{}}}%
\end{pgfscope}%
\begin{pgfscope}%
\pgfsetrectcap%
\pgfsetroundjoin%
\pgfsetlinewidth{1.505625pt}%
\pgfsetstrokecolor{currentstroke5}%
\pgfsetdash{}{0pt}%
\pgfpathmoveto{\pgfqpoint{5.369911in}{1.440373in}}%
\pgfpathlineto{\pgfqpoint{5.494911in}{1.440373in}}%
\pgfpathlineto{\pgfqpoint{5.619911in}{1.440373in}}%
\pgfusepath{stroke}%
\end{pgfscope}%
\begin{pgfscope}%
\definecolor{textcolor}{rgb}{0.000000,0.000000,0.000000}%
\pgfsetstrokecolor{textcolor}%
\pgfsetfillcolor{textcolor}%
\pgftext[x=5.719911in,y=1.396623in,left,base]{\color{textcolor}{\rmfamily\fontsize{9.000000}{10.800000}\selectfont\catcode`\^=\active\def^{\ifmmode\sp\else\^{}\fi}\catcode`\%=\active\def%{\%}\NeighborsDegree{} \& \MergeLinear{}}}%
\end{pgfscope}%
\begin{pgfscope}%
\pgfsetrectcap%
\pgfsetroundjoin%
\pgfsetlinewidth{1.505625pt}%
\pgfsetstrokecolor{currentstroke6}%
\pgfsetdash{}{0pt}%
\pgfpathmoveto{\pgfqpoint{5.369911in}{1.253423in}}%
\pgfpathlineto{\pgfqpoint{5.494911in}{1.253423in}}%
\pgfpathlineto{\pgfqpoint{5.619911in}{1.253423in}}%
\pgfusepath{stroke}%
\end{pgfscope}%
\begin{pgfscope}%
\definecolor{textcolor}{rgb}{0.000000,0.000000,0.000000}%
\pgfsetstrokecolor{textcolor}%
\pgfsetfillcolor{textcolor}%
\pgftext[x=5.719911in,y=1.209673in,left,base]{\color{textcolor}{\rmfamily\fontsize{9.000000}{10.800000}\selectfont\catcode`\^=\active\def^{\ifmmode\sp\else\^{}\fi}\catcode`\%=\active\def%{\%}\None{} \& \MergeLinear{}}}%
\end{pgfscope}%
\begin{pgfscope}%
\pgfsetrectcap%
\pgfsetroundjoin%
\pgfsetlinewidth{1.505625pt}%
\pgfsetstrokecolor{currentstroke7}%
\pgfsetdash{}{0pt}%
\pgfpathmoveto{\pgfqpoint{5.369911in}{1.069951in}}%
\pgfpathlineto{\pgfqpoint{5.494911in}{1.069951in}}%
\pgfpathlineto{\pgfqpoint{5.619911in}{1.069951in}}%
\pgfusepath{stroke}%
\end{pgfscope}%
\begin{pgfscope}%
\definecolor{textcolor}{rgb}{0.000000,0.000000,0.000000}%
\pgfsetstrokecolor{textcolor}%
\pgfsetfillcolor{textcolor}%
\pgftext[x=5.719911in,y=1.026201in,left,base]{\color{textcolor}{\rmfamily\fontsize{9.000000}{10.800000}\selectfont\catcode`\^=\active\def^{\ifmmode\sp\else\^{}\fi}\catcode`\%=\active\def%{\%}\None{} \& \SharedVertices{}}}%
\end{pgfscope}%
\end{pgfpicture}%
\makeatother%
\endgroup%
}
% 	\caption[Checks performed for graphs with no 3 nor 4 cycles (some)]{
% 		The number of checks performed to find some NAC-coloring for graphs with no three nor four cycles.}%
% 	\label{fig:graph_count_no_3_nor_4_cycles_first_checks}
% \end{figure}%

For listing all NAC-colorings shown
in \Cref{fig:graph_count_no_3_nor_4_cycles_all_runtime}
shows that \NaiveCycles{} is almost never faster.
We can see that \None{} and \CyclesMatchChunks{} is also slower than
\Neighbors{} and \NeighborsDegree{}.
At around twenty-eight vertices five second time limit is reached for all algorithms.
\MergeLinear{} and \SharedVertices{} have no significant influence.
%
\begin{figure}[thbp]
	\centering
	\scalebox{\BenchFigureScale}{%% Creator: Matplotlib, PGF backend
%%
%% To include the figure in your LaTeX document, write
%%   \input{<filename>.pgf}
%%
%% Make sure the required packages are loaded in your preamble
%%   \usepackage{pgf}
%%
%% Also ensure that all the required font packages are loaded; for instance,
%% the lmodern package is sometimes necessary when using math font.
%%   \usepackage{lmodern}
%%
%% Figures using additional raster images can only be included by \input if
%% they are in the same directory as the main LaTeX file. For loading figures
%% from other directories you can use the `import` package
%%   \usepackage{import}
%%
%% and then include the figures with
%%   \import{<path to file>}{<filename>.pgf}
%%
%% Matplotlib used the following preamble
%%   \def\mathdefault#1{#1}
%%   \everymath=\expandafter{\the\everymath\displaystyle}
%%   \IfFileExists{scrextend.sty}{
%%     \usepackage[fontsize=10.000000pt]{scrextend}
%%   }{
%%     \renewcommand{\normalsize}{\fontsize{10.000000}{12.000000}\selectfont}
%%     \normalsize
%%   }
%%   
%%   \ifdefined\pdftexversion\else  % non-pdftex case.
%%     \usepackage{fontspec}
%%     \setmainfont{DejaVuSans.ttf}[Path=\detokenize{/home/petr/Projects/PyRigi/.venv/lib/python3.12/site-packages/matplotlib/mpl-data/fonts/ttf/}]
%%     \setsansfont{DejaVuSans.ttf}[Path=\detokenize{/home/petr/Projects/PyRigi/.venv/lib/python3.12/site-packages/matplotlib/mpl-data/fonts/ttf/}]
%%     \setmonofont{DejaVuSansMono.ttf}[Path=\detokenize{/home/petr/Projects/PyRigi/.venv/lib/python3.12/site-packages/matplotlib/mpl-data/fonts/ttf/}]
%%   \fi
%%   \makeatletter\@ifpackageloaded{under\Score{}}{}{\usepackage[strings]{under\Score{}}}\makeatother
%%
\begingroup%
\makeatletter%
\begin{pgfpicture}%
\pgfpathrectangle{\pgfpointorigin}{\pgfqpoint{8.384376in}{2.841860in}}%
\pgfusepath{use as bounding box, clip}%
\begin{pgfscope}%
\pgfsetbuttcap%
\pgfsetmiterjoin%
\definecolor{currentfill}{rgb}{1.000000,1.000000,1.000000}%
\pgfsetfillcolor{currentfill}%
\pgfsetlinewidth{0.000000pt}%
\definecolor{currentstroke}{rgb}{1.000000,1.000000,1.000000}%
\pgfsetstrokecolor{currentstroke}%
\pgfsetdash{}{0pt}%
\pgfpathmoveto{\pgfqpoint{0.000000in}{0.000000in}}%
\pgfpathlineto{\pgfqpoint{8.384376in}{0.000000in}}%
\pgfpathlineto{\pgfqpoint{8.384376in}{2.841860in}}%
\pgfpathlineto{\pgfqpoint{0.000000in}{2.841860in}}%
\pgfpathlineto{\pgfqpoint{0.000000in}{0.000000in}}%
\pgfpathclose%
\pgfusepath{fill}%
\end{pgfscope}%
\begin{pgfscope}%
\pgfsetbuttcap%
\pgfsetmiterjoin%
\definecolor{currentfill}{rgb}{1.000000,1.000000,1.000000}%
\pgfsetfillcolor{currentfill}%
\pgfsetlinewidth{0.000000pt}%
\definecolor{currentstroke}{rgb}{0.000000,0.000000,0.000000}%
\pgfsetstrokecolor{currentstroke}%
\pgfsetstrokeopacity{0.000000}%
\pgfsetdash{}{0pt}%
\pgfpathmoveto{\pgfqpoint{0.588387in}{0.521603in}}%
\pgfpathlineto{\pgfqpoint{4.248423in}{0.521603in}}%
\pgfpathlineto{\pgfqpoint{4.248423in}{2.741376in}}%
\pgfpathlineto{\pgfqpoint{0.588387in}{2.741376in}}%
\pgfpathlineto{\pgfqpoint{0.588387in}{0.521603in}}%
\pgfpathclose%
\pgfusepath{fill}%
\end{pgfscope}%
\begin{pgfscope}%
\pgfsetbuttcap%
\pgfsetroundjoin%
\definecolor{currentfill}{rgb}{0.000000,0.000000,0.000000}%
\pgfsetfillcolor{currentfill}%
\pgfsetlinewidth{0.803000pt}%
\definecolor{currentstroke}{rgb}{0.000000,0.000000,0.000000}%
\pgfsetstrokecolor{currentstroke}%
\pgfsetdash{}{0pt}%
\pgfsys@defobject{currentmarker}{\pgfqpoint{0.000000in}{-0.048611in}}{\pgfqpoint{0.000000in}{0.000000in}}{%
\pgfpathmoveto{\pgfqpoint{0.000000in}{0.000000in}}%
\pgfpathlineto{\pgfqpoint{0.000000in}{-0.048611in}}%
\pgfusepath{stroke,fill}%
}%
\begin{pgfscope}%
\pgfsys@transformshift{0.882726in}{0.521603in}%
\pgfsys@useobject{currentmarker}{}%
\end{pgfscope}%
\end{pgfscope}%
\begin{pgfscope}%
\definecolor{textcolor}{rgb}{0.000000,0.000000,0.000000}%
\pgfsetstrokecolor{textcolor}%
\pgfsetfillcolor{textcolor}%
\pgftext[x=0.882726in,y=0.424381in,,top]{\color{textcolor}{\rmfamily\fontsize{10.000000}{12.000000}\selectfont\catcode`\^=\active\def^{\ifmmode\sp\else\^{}\fi}\catcode`\%=\active\def%{\%}$\mathdefault{9}$}}%
\end{pgfscope}%
\begin{pgfscope}%
\pgfsetbuttcap%
\pgfsetroundjoin%
\definecolor{currentfill}{rgb}{0.000000,0.000000,0.000000}%
\pgfsetfillcolor{currentfill}%
\pgfsetlinewidth{0.803000pt}%
\definecolor{currentstroke}{rgb}{0.000000,0.000000,0.000000}%
\pgfsetstrokecolor{currentstroke}%
\pgfsetdash{}{0pt}%
\pgfsys@defobject{currentmarker}{\pgfqpoint{0.000000in}{-0.048611in}}{\pgfqpoint{0.000000in}{0.000000in}}{%
\pgfpathmoveto{\pgfqpoint{0.000000in}{0.000000in}}%
\pgfpathlineto{\pgfqpoint{0.000000in}{-0.048611in}}%
\pgfusepath{stroke,fill}%
}%
\begin{pgfscope}%
\pgfsys@transformshift{1.266646in}{0.521603in}%
\pgfsys@useobject{currentmarker}{}%
\end{pgfscope}%
\end{pgfscope}%
\begin{pgfscope}%
\definecolor{textcolor}{rgb}{0.000000,0.000000,0.000000}%
\pgfsetstrokecolor{textcolor}%
\pgfsetfillcolor{textcolor}%
\pgftext[x=1.266646in,y=0.424381in,,top]{\color{textcolor}{\rmfamily\fontsize{10.000000}{12.000000}\selectfont\catcode`\^=\active\def^{\ifmmode\sp\else\^{}\fi}\catcode`\%=\active\def%{\%}$\mathdefault{12}$}}%
\end{pgfscope}%
\begin{pgfscope}%
\pgfsetbuttcap%
\pgfsetroundjoin%
\definecolor{currentfill}{rgb}{0.000000,0.000000,0.000000}%
\pgfsetfillcolor{currentfill}%
\pgfsetlinewidth{0.803000pt}%
\definecolor{currentstroke}{rgb}{0.000000,0.000000,0.000000}%
\pgfsetstrokecolor{currentstroke}%
\pgfsetdash{}{0pt}%
\pgfsys@defobject{currentmarker}{\pgfqpoint{0.000000in}{-0.048611in}}{\pgfqpoint{0.000000in}{0.000000in}}{%
\pgfpathmoveto{\pgfqpoint{0.000000in}{0.000000in}}%
\pgfpathlineto{\pgfqpoint{0.000000in}{-0.048611in}}%
\pgfusepath{stroke,fill}%
}%
\begin{pgfscope}%
\pgfsys@transformshift{1.650565in}{0.521603in}%
\pgfsys@useobject{currentmarker}{}%
\end{pgfscope}%
\end{pgfscope}%
\begin{pgfscope}%
\definecolor{textcolor}{rgb}{0.000000,0.000000,0.000000}%
\pgfsetstrokecolor{textcolor}%
\pgfsetfillcolor{textcolor}%
\pgftext[x=1.650565in,y=0.424381in,,top]{\color{textcolor}{\rmfamily\fontsize{10.000000}{12.000000}\selectfont\catcode`\^=\active\def^{\ifmmode\sp\else\^{}\fi}\catcode`\%=\active\def%{\%}$\mathdefault{15}$}}%
\end{pgfscope}%
\begin{pgfscope}%
\pgfsetbuttcap%
\pgfsetroundjoin%
\definecolor{currentfill}{rgb}{0.000000,0.000000,0.000000}%
\pgfsetfillcolor{currentfill}%
\pgfsetlinewidth{0.803000pt}%
\definecolor{currentstroke}{rgb}{0.000000,0.000000,0.000000}%
\pgfsetstrokecolor{currentstroke}%
\pgfsetdash{}{0pt}%
\pgfsys@defobject{currentmarker}{\pgfqpoint{0.000000in}{-0.048611in}}{\pgfqpoint{0.000000in}{0.000000in}}{%
\pgfpathmoveto{\pgfqpoint{0.000000in}{0.000000in}}%
\pgfpathlineto{\pgfqpoint{0.000000in}{-0.048611in}}%
\pgfusepath{stroke,fill}%
}%
\begin{pgfscope}%
\pgfsys@transformshift{2.034485in}{0.521603in}%
\pgfsys@useobject{currentmarker}{}%
\end{pgfscope}%
\end{pgfscope}%
\begin{pgfscope}%
\definecolor{textcolor}{rgb}{0.000000,0.000000,0.000000}%
\pgfsetstrokecolor{textcolor}%
\pgfsetfillcolor{textcolor}%
\pgftext[x=2.034485in,y=0.424381in,,top]{\color{textcolor}{\rmfamily\fontsize{10.000000}{12.000000}\selectfont\catcode`\^=\active\def^{\ifmmode\sp\else\^{}\fi}\catcode`\%=\active\def%{\%}$\mathdefault{18}$}}%
\end{pgfscope}%
\begin{pgfscope}%
\pgfsetbuttcap%
\pgfsetroundjoin%
\definecolor{currentfill}{rgb}{0.000000,0.000000,0.000000}%
\pgfsetfillcolor{currentfill}%
\pgfsetlinewidth{0.803000pt}%
\definecolor{currentstroke}{rgb}{0.000000,0.000000,0.000000}%
\pgfsetstrokecolor{currentstroke}%
\pgfsetdash{}{0pt}%
\pgfsys@defobject{currentmarker}{\pgfqpoint{0.000000in}{-0.048611in}}{\pgfqpoint{0.000000in}{0.000000in}}{%
\pgfpathmoveto{\pgfqpoint{0.000000in}{0.000000in}}%
\pgfpathlineto{\pgfqpoint{0.000000in}{-0.048611in}}%
\pgfusepath{stroke,fill}%
}%
\begin{pgfscope}%
\pgfsys@transformshift{2.418405in}{0.521603in}%
\pgfsys@useobject{currentmarker}{}%
\end{pgfscope}%
\end{pgfscope}%
\begin{pgfscope}%
\definecolor{textcolor}{rgb}{0.000000,0.000000,0.000000}%
\pgfsetstrokecolor{textcolor}%
\pgfsetfillcolor{textcolor}%
\pgftext[x=2.418405in,y=0.424381in,,top]{\color{textcolor}{\rmfamily\fontsize{10.000000}{12.000000}\selectfont\catcode`\^=\active\def^{\ifmmode\sp\else\^{}\fi}\catcode`\%=\active\def%{\%}$\mathdefault{21}$}}%
\end{pgfscope}%
\begin{pgfscope}%
\pgfsetbuttcap%
\pgfsetroundjoin%
\definecolor{currentfill}{rgb}{0.000000,0.000000,0.000000}%
\pgfsetfillcolor{currentfill}%
\pgfsetlinewidth{0.803000pt}%
\definecolor{currentstroke}{rgb}{0.000000,0.000000,0.000000}%
\pgfsetstrokecolor{currentstroke}%
\pgfsetdash{}{0pt}%
\pgfsys@defobject{currentmarker}{\pgfqpoint{0.000000in}{-0.048611in}}{\pgfqpoint{0.000000in}{0.000000in}}{%
\pgfpathmoveto{\pgfqpoint{0.000000in}{0.000000in}}%
\pgfpathlineto{\pgfqpoint{0.000000in}{-0.048611in}}%
\pgfusepath{stroke,fill}%
}%
\begin{pgfscope}%
\pgfsys@transformshift{2.802325in}{0.521603in}%
\pgfsys@useobject{currentmarker}{}%
\end{pgfscope}%
\end{pgfscope}%
\begin{pgfscope}%
\definecolor{textcolor}{rgb}{0.000000,0.000000,0.000000}%
\pgfsetstrokecolor{textcolor}%
\pgfsetfillcolor{textcolor}%
\pgftext[x=2.802325in,y=0.424381in,,top]{\color{textcolor}{\rmfamily\fontsize{10.000000}{12.000000}\selectfont\catcode`\^=\active\def^{\ifmmode\sp\else\^{}\fi}\catcode`\%=\active\def%{\%}$\mathdefault{24}$}}%
\end{pgfscope}%
\begin{pgfscope}%
\pgfsetbuttcap%
\pgfsetroundjoin%
\definecolor{currentfill}{rgb}{0.000000,0.000000,0.000000}%
\pgfsetfillcolor{currentfill}%
\pgfsetlinewidth{0.803000pt}%
\definecolor{currentstroke}{rgb}{0.000000,0.000000,0.000000}%
\pgfsetstrokecolor{currentstroke}%
\pgfsetdash{}{0pt}%
\pgfsys@defobject{currentmarker}{\pgfqpoint{0.000000in}{-0.048611in}}{\pgfqpoint{0.000000in}{0.000000in}}{%
\pgfpathmoveto{\pgfqpoint{0.000000in}{0.000000in}}%
\pgfpathlineto{\pgfqpoint{0.000000in}{-0.048611in}}%
\pgfusepath{stroke,fill}%
}%
\begin{pgfscope}%
\pgfsys@transformshift{3.186245in}{0.521603in}%
\pgfsys@useobject{currentmarker}{}%
\end{pgfscope}%
\end{pgfscope}%
\begin{pgfscope}%
\definecolor{textcolor}{rgb}{0.000000,0.000000,0.000000}%
\pgfsetstrokecolor{textcolor}%
\pgfsetfillcolor{textcolor}%
\pgftext[x=3.186245in,y=0.424381in,,top]{\color{textcolor}{\rmfamily\fontsize{10.000000}{12.000000}\selectfont\catcode`\^=\active\def^{\ifmmode\sp\else\^{}\fi}\catcode`\%=\active\def%{\%}$\mathdefault{27}$}}%
\end{pgfscope}%
\begin{pgfscope}%
\pgfsetbuttcap%
\pgfsetroundjoin%
\definecolor{currentfill}{rgb}{0.000000,0.000000,0.000000}%
\pgfsetfillcolor{currentfill}%
\pgfsetlinewidth{0.803000pt}%
\definecolor{currentstroke}{rgb}{0.000000,0.000000,0.000000}%
\pgfsetstrokecolor{currentstroke}%
\pgfsetdash{}{0pt}%
\pgfsys@defobject{currentmarker}{\pgfqpoint{0.000000in}{-0.048611in}}{\pgfqpoint{0.000000in}{0.000000in}}{%
\pgfpathmoveto{\pgfqpoint{0.000000in}{0.000000in}}%
\pgfpathlineto{\pgfqpoint{0.000000in}{-0.048611in}}%
\pgfusepath{stroke,fill}%
}%
\begin{pgfscope}%
\pgfsys@transformshift{3.570164in}{0.521603in}%
\pgfsys@useobject{currentmarker}{}%
\end{pgfscope}%
\end{pgfscope}%
\begin{pgfscope}%
\definecolor{textcolor}{rgb}{0.000000,0.000000,0.000000}%
\pgfsetstrokecolor{textcolor}%
\pgfsetfillcolor{textcolor}%
\pgftext[x=3.570164in,y=0.424381in,,top]{\color{textcolor}{\rmfamily\fontsize{10.000000}{12.000000}\selectfont\catcode`\^=\active\def^{\ifmmode\sp\else\^{}\fi}\catcode`\%=\active\def%{\%}$\mathdefault{30}$}}%
\end{pgfscope}%
\begin{pgfscope}%
\pgfsetbuttcap%
\pgfsetroundjoin%
\definecolor{currentfill}{rgb}{0.000000,0.000000,0.000000}%
\pgfsetfillcolor{currentfill}%
\pgfsetlinewidth{0.803000pt}%
\definecolor{currentstroke}{rgb}{0.000000,0.000000,0.000000}%
\pgfsetstrokecolor{currentstroke}%
\pgfsetdash{}{0pt}%
\pgfsys@defobject{currentmarker}{\pgfqpoint{0.000000in}{-0.048611in}}{\pgfqpoint{0.000000in}{0.000000in}}{%
\pgfpathmoveto{\pgfqpoint{0.000000in}{0.000000in}}%
\pgfpathlineto{\pgfqpoint{0.000000in}{-0.048611in}}%
\pgfusepath{stroke,fill}%
}%
\begin{pgfscope}%
\pgfsys@transformshift{3.954084in}{0.521603in}%
\pgfsys@useobject{currentmarker}{}%
\end{pgfscope}%
\end{pgfscope}%
\begin{pgfscope}%
\definecolor{textcolor}{rgb}{0.000000,0.000000,0.000000}%
\pgfsetstrokecolor{textcolor}%
\pgfsetfillcolor{textcolor}%
\pgftext[x=3.954084in,y=0.424381in,,top]{\color{textcolor}{\rmfamily\fontsize{10.000000}{12.000000}\selectfont\catcode`\^=\active\def^{\ifmmode\sp\else\^{}\fi}\catcode`\%=\active\def%{\%}$\mathdefault{33}$}}%
\end{pgfscope}%
\begin{pgfscope}%
\definecolor{textcolor}{rgb}{0.000000,0.000000,0.000000}%
\pgfsetstrokecolor{textcolor}%
\pgfsetfillcolor{textcolor}%
\pgftext[x=2.418405in,y=0.234413in,,top]{\color{textcolor}{\rmfamily\fontsize{10.000000}{12.000000}\selectfont\catcode`\^=\active\def^{\ifmmode\sp\else\^{}\fi}\catcode`\%=\active\def%{\%}Monochromatic classes}}%
\end{pgfscope}%
\begin{pgfscope}%
\pgfsetbuttcap%
\pgfsetroundjoin%
\definecolor{currentfill}{rgb}{0.000000,0.000000,0.000000}%
\pgfsetfillcolor{currentfill}%
\pgfsetlinewidth{0.803000pt}%
\definecolor{currentstroke}{rgb}{0.000000,0.000000,0.000000}%
\pgfsetstrokecolor{currentstroke}%
\pgfsetdash{}{0pt}%
\pgfsys@defobject{currentmarker}{\pgfqpoint{-0.048611in}{0.000000in}}{\pgfqpoint{-0.000000in}{0.000000in}}{%
\pgfpathmoveto{\pgfqpoint{-0.000000in}{0.000000in}}%
\pgfpathlineto{\pgfqpoint{-0.048611in}{0.000000in}}%
\pgfusepath{stroke,fill}%
}%
\begin{pgfscope}%
\pgfsys@transformshift{0.588387in}{0.810961in}%
\pgfsys@useobject{currentmarker}{}%
\end{pgfscope}%
\end{pgfscope}%
\begin{pgfscope}%
\definecolor{textcolor}{rgb}{0.000000,0.000000,0.000000}%
\pgfsetstrokecolor{textcolor}%
\pgfsetfillcolor{textcolor}%
\pgftext[x=0.289968in, y=0.758199in, left, base]{\color{textcolor}{\rmfamily\fontsize{10.000000}{12.000000}\selectfont\catcode`\^=\active\def^{\ifmmode\sp\else\^{}\fi}\catcode`\%=\active\def%{\%}$\mathdefault{10^{1}}$}}%
\end{pgfscope}%
\begin{pgfscope}%
\pgfsetbuttcap%
\pgfsetroundjoin%
\definecolor{currentfill}{rgb}{0.000000,0.000000,0.000000}%
\pgfsetfillcolor{currentfill}%
\pgfsetlinewidth{0.803000pt}%
\definecolor{currentstroke}{rgb}{0.000000,0.000000,0.000000}%
\pgfsetstrokecolor{currentstroke}%
\pgfsetdash{}{0pt}%
\pgfsys@defobject{currentmarker}{\pgfqpoint{-0.048611in}{0.000000in}}{\pgfqpoint{-0.000000in}{0.000000in}}{%
\pgfpathmoveto{\pgfqpoint{-0.000000in}{0.000000in}}%
\pgfpathlineto{\pgfqpoint{-0.048611in}{0.000000in}}%
\pgfusepath{stroke,fill}%
}%
\begin{pgfscope}%
\pgfsys@transformshift{0.588387in}{1.437007in}%
\pgfsys@useobject{currentmarker}{}%
\end{pgfscope}%
\end{pgfscope}%
\begin{pgfscope}%
\definecolor{textcolor}{rgb}{0.000000,0.000000,0.000000}%
\pgfsetstrokecolor{textcolor}%
\pgfsetfillcolor{textcolor}%
\pgftext[x=0.289968in, y=1.384245in, left, base]{\color{textcolor}{\rmfamily\fontsize{10.000000}{12.000000}\selectfont\catcode`\^=\active\def^{\ifmmode\sp\else\^{}\fi}\catcode`\%=\active\def%{\%}$\mathdefault{10^{2}}$}}%
\end{pgfscope}%
\begin{pgfscope}%
\pgfsetbuttcap%
\pgfsetroundjoin%
\definecolor{currentfill}{rgb}{0.000000,0.000000,0.000000}%
\pgfsetfillcolor{currentfill}%
\pgfsetlinewidth{0.803000pt}%
\definecolor{currentstroke}{rgb}{0.000000,0.000000,0.000000}%
\pgfsetstrokecolor{currentstroke}%
\pgfsetdash{}{0pt}%
\pgfsys@defobject{currentmarker}{\pgfqpoint{-0.048611in}{0.000000in}}{\pgfqpoint{-0.000000in}{0.000000in}}{%
\pgfpathmoveto{\pgfqpoint{-0.000000in}{0.000000in}}%
\pgfpathlineto{\pgfqpoint{-0.048611in}{0.000000in}}%
\pgfusepath{stroke,fill}%
}%
\begin{pgfscope}%
\pgfsys@transformshift{0.588387in}{2.063053in}%
\pgfsys@useobject{currentmarker}{}%
\end{pgfscope}%
\end{pgfscope}%
\begin{pgfscope}%
\definecolor{textcolor}{rgb}{0.000000,0.000000,0.000000}%
\pgfsetstrokecolor{textcolor}%
\pgfsetfillcolor{textcolor}%
\pgftext[x=0.289968in, y=2.010291in, left, base]{\color{textcolor}{\rmfamily\fontsize{10.000000}{12.000000}\selectfont\catcode`\^=\active\def^{\ifmmode\sp\else\^{}\fi}\catcode`\%=\active\def%{\%}$\mathdefault{10^{3}}$}}%
\end{pgfscope}%
\begin{pgfscope}%
\pgfsetbuttcap%
\pgfsetroundjoin%
\definecolor{currentfill}{rgb}{0.000000,0.000000,0.000000}%
\pgfsetfillcolor{currentfill}%
\pgfsetlinewidth{0.803000pt}%
\definecolor{currentstroke}{rgb}{0.000000,0.000000,0.000000}%
\pgfsetstrokecolor{currentstroke}%
\pgfsetdash{}{0pt}%
\pgfsys@defobject{currentmarker}{\pgfqpoint{-0.048611in}{0.000000in}}{\pgfqpoint{-0.000000in}{0.000000in}}{%
\pgfpathmoveto{\pgfqpoint{-0.000000in}{0.000000in}}%
\pgfpathlineto{\pgfqpoint{-0.048611in}{0.000000in}}%
\pgfusepath{stroke,fill}%
}%
\begin{pgfscope}%
\pgfsys@transformshift{0.588387in}{2.689099in}%
\pgfsys@useobject{currentmarker}{}%
\end{pgfscope}%
\end{pgfscope}%
\begin{pgfscope}%
\definecolor{textcolor}{rgb}{0.000000,0.000000,0.000000}%
\pgfsetstrokecolor{textcolor}%
\pgfsetfillcolor{textcolor}%
\pgftext[x=0.289968in, y=2.636337in, left, base]{\color{textcolor}{\rmfamily\fontsize{10.000000}{12.000000}\selectfont\catcode`\^=\active\def^{\ifmmode\sp\else\^{}\fi}\catcode`\%=\active\def%{\%}$\mathdefault{10^{4}}$}}%
\end{pgfscope}%
\begin{pgfscope}%
\pgfsetbuttcap%
\pgfsetroundjoin%
\definecolor{currentfill}{rgb}{0.000000,0.000000,0.000000}%
\pgfsetfillcolor{currentfill}%
\pgfsetlinewidth{0.602250pt}%
\definecolor{currentstroke}{rgb}{0.000000,0.000000,0.000000}%
\pgfsetstrokecolor{currentstroke}%
\pgfsetdash{}{0pt}%
\pgfsys@defobject{currentmarker}{\pgfqpoint{-0.027778in}{0.000000in}}{\pgfqpoint{-0.000000in}{0.000000in}}{%
\pgfpathmoveto{\pgfqpoint{-0.000000in}{0.000000in}}%
\pgfpathlineto{\pgfqpoint{-0.027778in}{0.000000in}}%
\pgfusepath{stroke,fill}%
}%
\begin{pgfscope}%
\pgfsys@transformshift{0.588387in}{0.561832in}%
\pgfsys@useobject{currentmarker}{}%
\end{pgfscope}%
\end{pgfscope}%
\begin{pgfscope}%
\pgfsetbuttcap%
\pgfsetroundjoin%
\definecolor{currentfill}{rgb}{0.000000,0.000000,0.000000}%
\pgfsetfillcolor{currentfill}%
\pgfsetlinewidth{0.602250pt}%
\definecolor{currentstroke}{rgb}{0.000000,0.000000,0.000000}%
\pgfsetstrokecolor{currentstroke}%
\pgfsetdash{}{0pt}%
\pgfsys@defobject{currentmarker}{\pgfqpoint{-0.027778in}{0.000000in}}{\pgfqpoint{-0.000000in}{0.000000in}}{%
\pgfpathmoveto{\pgfqpoint{-0.000000in}{0.000000in}}%
\pgfpathlineto{\pgfqpoint{-0.027778in}{0.000000in}}%
\pgfusepath{stroke,fill}%
}%
\begin{pgfscope}%
\pgfsys@transformshift{0.588387in}{0.622502in}%
\pgfsys@useobject{currentmarker}{}%
\end{pgfscope}%
\end{pgfscope}%
\begin{pgfscope}%
\pgfsetbuttcap%
\pgfsetroundjoin%
\definecolor{currentfill}{rgb}{0.000000,0.000000,0.000000}%
\pgfsetfillcolor{currentfill}%
\pgfsetlinewidth{0.602250pt}%
\definecolor{currentstroke}{rgb}{0.000000,0.000000,0.000000}%
\pgfsetstrokecolor{currentstroke}%
\pgfsetdash{}{0pt}%
\pgfsys@defobject{currentmarker}{\pgfqpoint{-0.027778in}{0.000000in}}{\pgfqpoint{-0.000000in}{0.000000in}}{%
\pgfpathmoveto{\pgfqpoint{-0.000000in}{0.000000in}}%
\pgfpathlineto{\pgfqpoint{-0.027778in}{0.000000in}}%
\pgfusepath{stroke,fill}%
}%
\begin{pgfscope}%
\pgfsys@transformshift{0.588387in}{0.672073in}%
\pgfsys@useobject{currentmarker}{}%
\end{pgfscope}%
\end{pgfscope}%
\begin{pgfscope}%
\pgfsetbuttcap%
\pgfsetroundjoin%
\definecolor{currentfill}{rgb}{0.000000,0.000000,0.000000}%
\pgfsetfillcolor{currentfill}%
\pgfsetlinewidth{0.602250pt}%
\definecolor{currentstroke}{rgb}{0.000000,0.000000,0.000000}%
\pgfsetstrokecolor{currentstroke}%
\pgfsetdash{}{0pt}%
\pgfsys@defobject{currentmarker}{\pgfqpoint{-0.027778in}{0.000000in}}{\pgfqpoint{-0.000000in}{0.000000in}}{%
\pgfpathmoveto{\pgfqpoint{-0.000000in}{0.000000in}}%
\pgfpathlineto{\pgfqpoint{-0.027778in}{0.000000in}}%
\pgfusepath{stroke,fill}%
}%
\begin{pgfscope}%
\pgfsys@transformshift{0.588387in}{0.713985in}%
\pgfsys@useobject{currentmarker}{}%
\end{pgfscope}%
\end{pgfscope}%
\begin{pgfscope}%
\pgfsetbuttcap%
\pgfsetroundjoin%
\definecolor{currentfill}{rgb}{0.000000,0.000000,0.000000}%
\pgfsetfillcolor{currentfill}%
\pgfsetlinewidth{0.602250pt}%
\definecolor{currentstroke}{rgb}{0.000000,0.000000,0.000000}%
\pgfsetstrokecolor{currentstroke}%
\pgfsetdash{}{0pt}%
\pgfsys@defobject{currentmarker}{\pgfqpoint{-0.027778in}{0.000000in}}{\pgfqpoint{-0.000000in}{0.000000in}}{%
\pgfpathmoveto{\pgfqpoint{-0.000000in}{0.000000in}}%
\pgfpathlineto{\pgfqpoint{-0.027778in}{0.000000in}}%
\pgfusepath{stroke,fill}%
}%
\begin{pgfscope}%
\pgfsys@transformshift{0.588387in}{0.750291in}%
\pgfsys@useobject{currentmarker}{}%
\end{pgfscope}%
\end{pgfscope}%
\begin{pgfscope}%
\pgfsetbuttcap%
\pgfsetroundjoin%
\definecolor{currentfill}{rgb}{0.000000,0.000000,0.000000}%
\pgfsetfillcolor{currentfill}%
\pgfsetlinewidth{0.602250pt}%
\definecolor{currentstroke}{rgb}{0.000000,0.000000,0.000000}%
\pgfsetstrokecolor{currentstroke}%
\pgfsetdash{}{0pt}%
\pgfsys@defobject{currentmarker}{\pgfqpoint{-0.027778in}{0.000000in}}{\pgfqpoint{-0.000000in}{0.000000in}}{%
\pgfpathmoveto{\pgfqpoint{-0.000000in}{0.000000in}}%
\pgfpathlineto{\pgfqpoint{-0.027778in}{0.000000in}}%
\pgfusepath{stroke,fill}%
}%
\begin{pgfscope}%
\pgfsys@transformshift{0.588387in}{0.782314in}%
\pgfsys@useobject{currentmarker}{}%
\end{pgfscope}%
\end{pgfscope}%
\begin{pgfscope}%
\pgfsetbuttcap%
\pgfsetroundjoin%
\definecolor{currentfill}{rgb}{0.000000,0.000000,0.000000}%
\pgfsetfillcolor{currentfill}%
\pgfsetlinewidth{0.602250pt}%
\definecolor{currentstroke}{rgb}{0.000000,0.000000,0.000000}%
\pgfsetstrokecolor{currentstroke}%
\pgfsetdash{}{0pt}%
\pgfsys@defobject{currentmarker}{\pgfqpoint{-0.027778in}{0.000000in}}{\pgfqpoint{-0.000000in}{0.000000in}}{%
\pgfpathmoveto{\pgfqpoint{-0.000000in}{0.000000in}}%
\pgfpathlineto{\pgfqpoint{-0.027778in}{0.000000in}}%
\pgfusepath{stroke,fill}%
}%
\begin{pgfscope}%
\pgfsys@transformshift{0.588387in}{0.999419in}%
\pgfsys@useobject{currentmarker}{}%
\end{pgfscope}%
\end{pgfscope}%
\begin{pgfscope}%
\pgfsetbuttcap%
\pgfsetroundjoin%
\definecolor{currentfill}{rgb}{0.000000,0.000000,0.000000}%
\pgfsetfillcolor{currentfill}%
\pgfsetlinewidth{0.602250pt}%
\definecolor{currentstroke}{rgb}{0.000000,0.000000,0.000000}%
\pgfsetstrokecolor{currentstroke}%
\pgfsetdash{}{0pt}%
\pgfsys@defobject{currentmarker}{\pgfqpoint{-0.027778in}{0.000000in}}{\pgfqpoint{-0.000000in}{0.000000in}}{%
\pgfpathmoveto{\pgfqpoint{-0.000000in}{0.000000in}}%
\pgfpathlineto{\pgfqpoint{-0.027778in}{0.000000in}}%
\pgfusepath{stroke,fill}%
}%
\begin{pgfscope}%
\pgfsys@transformshift{0.588387in}{1.109661in}%
\pgfsys@useobject{currentmarker}{}%
\end{pgfscope}%
\end{pgfscope}%
\begin{pgfscope}%
\pgfsetbuttcap%
\pgfsetroundjoin%
\definecolor{currentfill}{rgb}{0.000000,0.000000,0.000000}%
\pgfsetfillcolor{currentfill}%
\pgfsetlinewidth{0.602250pt}%
\definecolor{currentstroke}{rgb}{0.000000,0.000000,0.000000}%
\pgfsetstrokecolor{currentstroke}%
\pgfsetdash{}{0pt}%
\pgfsys@defobject{currentmarker}{\pgfqpoint{-0.027778in}{0.000000in}}{\pgfqpoint{-0.000000in}{0.000000in}}{%
\pgfpathmoveto{\pgfqpoint{-0.000000in}{0.000000in}}%
\pgfpathlineto{\pgfqpoint{-0.027778in}{0.000000in}}%
\pgfusepath{stroke,fill}%
}%
\begin{pgfscope}%
\pgfsys@transformshift{0.588387in}{1.187878in}%
\pgfsys@useobject{currentmarker}{}%
\end{pgfscope}%
\end{pgfscope}%
\begin{pgfscope}%
\pgfsetbuttcap%
\pgfsetroundjoin%
\definecolor{currentfill}{rgb}{0.000000,0.000000,0.000000}%
\pgfsetfillcolor{currentfill}%
\pgfsetlinewidth{0.602250pt}%
\definecolor{currentstroke}{rgb}{0.000000,0.000000,0.000000}%
\pgfsetstrokecolor{currentstroke}%
\pgfsetdash{}{0pt}%
\pgfsys@defobject{currentmarker}{\pgfqpoint{-0.027778in}{0.000000in}}{\pgfqpoint{-0.000000in}{0.000000in}}{%
\pgfpathmoveto{\pgfqpoint{-0.000000in}{0.000000in}}%
\pgfpathlineto{\pgfqpoint{-0.027778in}{0.000000in}}%
\pgfusepath{stroke,fill}%
}%
\begin{pgfscope}%
\pgfsys@transformshift{0.588387in}{1.248548in}%
\pgfsys@useobject{currentmarker}{}%
\end{pgfscope}%
\end{pgfscope}%
\begin{pgfscope}%
\pgfsetbuttcap%
\pgfsetroundjoin%
\definecolor{currentfill}{rgb}{0.000000,0.000000,0.000000}%
\pgfsetfillcolor{currentfill}%
\pgfsetlinewidth{0.602250pt}%
\definecolor{currentstroke}{rgb}{0.000000,0.000000,0.000000}%
\pgfsetstrokecolor{currentstroke}%
\pgfsetdash{}{0pt}%
\pgfsys@defobject{currentmarker}{\pgfqpoint{-0.027778in}{0.000000in}}{\pgfqpoint{-0.000000in}{0.000000in}}{%
\pgfpathmoveto{\pgfqpoint{-0.000000in}{0.000000in}}%
\pgfpathlineto{\pgfqpoint{-0.027778in}{0.000000in}}%
\pgfusepath{stroke,fill}%
}%
\begin{pgfscope}%
\pgfsys@transformshift{0.588387in}{1.298119in}%
\pgfsys@useobject{currentmarker}{}%
\end{pgfscope}%
\end{pgfscope}%
\begin{pgfscope}%
\pgfsetbuttcap%
\pgfsetroundjoin%
\definecolor{currentfill}{rgb}{0.000000,0.000000,0.000000}%
\pgfsetfillcolor{currentfill}%
\pgfsetlinewidth{0.602250pt}%
\definecolor{currentstroke}{rgb}{0.000000,0.000000,0.000000}%
\pgfsetstrokecolor{currentstroke}%
\pgfsetdash{}{0pt}%
\pgfsys@defobject{currentmarker}{\pgfqpoint{-0.027778in}{0.000000in}}{\pgfqpoint{-0.000000in}{0.000000in}}{%
\pgfpathmoveto{\pgfqpoint{-0.000000in}{0.000000in}}%
\pgfpathlineto{\pgfqpoint{-0.027778in}{0.000000in}}%
\pgfusepath{stroke,fill}%
}%
\begin{pgfscope}%
\pgfsys@transformshift{0.588387in}{1.340031in}%
\pgfsys@useobject{currentmarker}{}%
\end{pgfscope}%
\end{pgfscope}%
\begin{pgfscope}%
\pgfsetbuttcap%
\pgfsetroundjoin%
\definecolor{currentfill}{rgb}{0.000000,0.000000,0.000000}%
\pgfsetfillcolor{currentfill}%
\pgfsetlinewidth{0.602250pt}%
\definecolor{currentstroke}{rgb}{0.000000,0.000000,0.000000}%
\pgfsetstrokecolor{currentstroke}%
\pgfsetdash{}{0pt}%
\pgfsys@defobject{currentmarker}{\pgfqpoint{-0.027778in}{0.000000in}}{\pgfqpoint{-0.000000in}{0.000000in}}{%
\pgfpathmoveto{\pgfqpoint{-0.000000in}{0.000000in}}%
\pgfpathlineto{\pgfqpoint{-0.027778in}{0.000000in}}%
\pgfusepath{stroke,fill}%
}%
\begin{pgfscope}%
\pgfsys@transformshift{0.588387in}{1.376337in}%
\pgfsys@useobject{currentmarker}{}%
\end{pgfscope}%
\end{pgfscope}%
\begin{pgfscope}%
\pgfsetbuttcap%
\pgfsetroundjoin%
\definecolor{currentfill}{rgb}{0.000000,0.000000,0.000000}%
\pgfsetfillcolor{currentfill}%
\pgfsetlinewidth{0.602250pt}%
\definecolor{currentstroke}{rgb}{0.000000,0.000000,0.000000}%
\pgfsetstrokecolor{currentstroke}%
\pgfsetdash{}{0pt}%
\pgfsys@defobject{currentmarker}{\pgfqpoint{-0.027778in}{0.000000in}}{\pgfqpoint{-0.000000in}{0.000000in}}{%
\pgfpathmoveto{\pgfqpoint{-0.000000in}{0.000000in}}%
\pgfpathlineto{\pgfqpoint{-0.027778in}{0.000000in}}%
\pgfusepath{stroke,fill}%
}%
\begin{pgfscope}%
\pgfsys@transformshift{0.588387in}{1.408360in}%
\pgfsys@useobject{currentmarker}{}%
\end{pgfscope}%
\end{pgfscope}%
\begin{pgfscope}%
\pgfsetbuttcap%
\pgfsetroundjoin%
\definecolor{currentfill}{rgb}{0.000000,0.000000,0.000000}%
\pgfsetfillcolor{currentfill}%
\pgfsetlinewidth{0.602250pt}%
\definecolor{currentstroke}{rgb}{0.000000,0.000000,0.000000}%
\pgfsetstrokecolor{currentstroke}%
\pgfsetdash{}{0pt}%
\pgfsys@defobject{currentmarker}{\pgfqpoint{-0.027778in}{0.000000in}}{\pgfqpoint{-0.000000in}{0.000000in}}{%
\pgfpathmoveto{\pgfqpoint{-0.000000in}{0.000000in}}%
\pgfpathlineto{\pgfqpoint{-0.027778in}{0.000000in}}%
\pgfusepath{stroke,fill}%
}%
\begin{pgfscope}%
\pgfsys@transformshift{0.588387in}{1.625465in}%
\pgfsys@useobject{currentmarker}{}%
\end{pgfscope}%
\end{pgfscope}%
\begin{pgfscope}%
\pgfsetbuttcap%
\pgfsetroundjoin%
\definecolor{currentfill}{rgb}{0.000000,0.000000,0.000000}%
\pgfsetfillcolor{currentfill}%
\pgfsetlinewidth{0.602250pt}%
\definecolor{currentstroke}{rgb}{0.000000,0.000000,0.000000}%
\pgfsetstrokecolor{currentstroke}%
\pgfsetdash{}{0pt}%
\pgfsys@defobject{currentmarker}{\pgfqpoint{-0.027778in}{0.000000in}}{\pgfqpoint{-0.000000in}{0.000000in}}{%
\pgfpathmoveto{\pgfqpoint{-0.000000in}{0.000000in}}%
\pgfpathlineto{\pgfqpoint{-0.027778in}{0.000000in}}%
\pgfusepath{stroke,fill}%
}%
\begin{pgfscope}%
\pgfsys@transformshift{0.588387in}{1.735707in}%
\pgfsys@useobject{currentmarker}{}%
\end{pgfscope}%
\end{pgfscope}%
\begin{pgfscope}%
\pgfsetbuttcap%
\pgfsetroundjoin%
\definecolor{currentfill}{rgb}{0.000000,0.000000,0.000000}%
\pgfsetfillcolor{currentfill}%
\pgfsetlinewidth{0.602250pt}%
\definecolor{currentstroke}{rgb}{0.000000,0.000000,0.000000}%
\pgfsetstrokecolor{currentstroke}%
\pgfsetdash{}{0pt}%
\pgfsys@defobject{currentmarker}{\pgfqpoint{-0.027778in}{0.000000in}}{\pgfqpoint{-0.000000in}{0.000000in}}{%
\pgfpathmoveto{\pgfqpoint{-0.000000in}{0.000000in}}%
\pgfpathlineto{\pgfqpoint{-0.027778in}{0.000000in}}%
\pgfusepath{stroke,fill}%
}%
\begin{pgfscope}%
\pgfsys@transformshift{0.588387in}{1.813924in}%
\pgfsys@useobject{currentmarker}{}%
\end{pgfscope}%
\end{pgfscope}%
\begin{pgfscope}%
\pgfsetbuttcap%
\pgfsetroundjoin%
\definecolor{currentfill}{rgb}{0.000000,0.000000,0.000000}%
\pgfsetfillcolor{currentfill}%
\pgfsetlinewidth{0.602250pt}%
\definecolor{currentstroke}{rgb}{0.000000,0.000000,0.000000}%
\pgfsetstrokecolor{currentstroke}%
\pgfsetdash{}{0pt}%
\pgfsys@defobject{currentmarker}{\pgfqpoint{-0.027778in}{0.000000in}}{\pgfqpoint{-0.000000in}{0.000000in}}{%
\pgfpathmoveto{\pgfqpoint{-0.000000in}{0.000000in}}%
\pgfpathlineto{\pgfqpoint{-0.027778in}{0.000000in}}%
\pgfusepath{stroke,fill}%
}%
\begin{pgfscope}%
\pgfsys@transformshift{0.588387in}{1.874594in}%
\pgfsys@useobject{currentmarker}{}%
\end{pgfscope}%
\end{pgfscope}%
\begin{pgfscope}%
\pgfsetbuttcap%
\pgfsetroundjoin%
\definecolor{currentfill}{rgb}{0.000000,0.000000,0.000000}%
\pgfsetfillcolor{currentfill}%
\pgfsetlinewidth{0.602250pt}%
\definecolor{currentstroke}{rgb}{0.000000,0.000000,0.000000}%
\pgfsetstrokecolor{currentstroke}%
\pgfsetdash{}{0pt}%
\pgfsys@defobject{currentmarker}{\pgfqpoint{-0.027778in}{0.000000in}}{\pgfqpoint{-0.000000in}{0.000000in}}{%
\pgfpathmoveto{\pgfqpoint{-0.000000in}{0.000000in}}%
\pgfpathlineto{\pgfqpoint{-0.027778in}{0.000000in}}%
\pgfusepath{stroke,fill}%
}%
\begin{pgfscope}%
\pgfsys@transformshift{0.588387in}{1.924165in}%
\pgfsys@useobject{currentmarker}{}%
\end{pgfscope}%
\end{pgfscope}%
\begin{pgfscope}%
\pgfsetbuttcap%
\pgfsetroundjoin%
\definecolor{currentfill}{rgb}{0.000000,0.000000,0.000000}%
\pgfsetfillcolor{currentfill}%
\pgfsetlinewidth{0.602250pt}%
\definecolor{currentstroke}{rgb}{0.000000,0.000000,0.000000}%
\pgfsetstrokecolor{currentstroke}%
\pgfsetdash{}{0pt}%
\pgfsys@defobject{currentmarker}{\pgfqpoint{-0.027778in}{0.000000in}}{\pgfqpoint{-0.000000in}{0.000000in}}{%
\pgfpathmoveto{\pgfqpoint{-0.000000in}{0.000000in}}%
\pgfpathlineto{\pgfqpoint{-0.027778in}{0.000000in}}%
\pgfusepath{stroke,fill}%
}%
\begin{pgfscope}%
\pgfsys@transformshift{0.588387in}{1.966077in}%
\pgfsys@useobject{currentmarker}{}%
\end{pgfscope}%
\end{pgfscope}%
\begin{pgfscope}%
\pgfsetbuttcap%
\pgfsetroundjoin%
\definecolor{currentfill}{rgb}{0.000000,0.000000,0.000000}%
\pgfsetfillcolor{currentfill}%
\pgfsetlinewidth{0.602250pt}%
\definecolor{currentstroke}{rgb}{0.000000,0.000000,0.000000}%
\pgfsetstrokecolor{currentstroke}%
\pgfsetdash{}{0pt}%
\pgfsys@defobject{currentmarker}{\pgfqpoint{-0.027778in}{0.000000in}}{\pgfqpoint{-0.000000in}{0.000000in}}{%
\pgfpathmoveto{\pgfqpoint{-0.000000in}{0.000000in}}%
\pgfpathlineto{\pgfqpoint{-0.027778in}{0.000000in}}%
\pgfusepath{stroke,fill}%
}%
\begin{pgfscope}%
\pgfsys@transformshift{0.588387in}{2.002383in}%
\pgfsys@useobject{currentmarker}{}%
\end{pgfscope}%
\end{pgfscope}%
\begin{pgfscope}%
\pgfsetbuttcap%
\pgfsetroundjoin%
\definecolor{currentfill}{rgb}{0.000000,0.000000,0.000000}%
\pgfsetfillcolor{currentfill}%
\pgfsetlinewidth{0.602250pt}%
\definecolor{currentstroke}{rgb}{0.000000,0.000000,0.000000}%
\pgfsetstrokecolor{currentstroke}%
\pgfsetdash{}{0pt}%
\pgfsys@defobject{currentmarker}{\pgfqpoint{-0.027778in}{0.000000in}}{\pgfqpoint{-0.000000in}{0.000000in}}{%
\pgfpathmoveto{\pgfqpoint{-0.000000in}{0.000000in}}%
\pgfpathlineto{\pgfqpoint{-0.027778in}{0.000000in}}%
\pgfusepath{stroke,fill}%
}%
\begin{pgfscope}%
\pgfsys@transformshift{0.588387in}{2.034407in}%
\pgfsys@useobject{currentmarker}{}%
\end{pgfscope}%
\end{pgfscope}%
\begin{pgfscope}%
\pgfsetbuttcap%
\pgfsetroundjoin%
\definecolor{currentfill}{rgb}{0.000000,0.000000,0.000000}%
\pgfsetfillcolor{currentfill}%
\pgfsetlinewidth{0.602250pt}%
\definecolor{currentstroke}{rgb}{0.000000,0.000000,0.000000}%
\pgfsetstrokecolor{currentstroke}%
\pgfsetdash{}{0pt}%
\pgfsys@defobject{currentmarker}{\pgfqpoint{-0.027778in}{0.000000in}}{\pgfqpoint{-0.000000in}{0.000000in}}{%
\pgfpathmoveto{\pgfqpoint{-0.000000in}{0.000000in}}%
\pgfpathlineto{\pgfqpoint{-0.027778in}{0.000000in}}%
\pgfusepath{stroke,fill}%
}%
\begin{pgfscope}%
\pgfsys@transformshift{0.588387in}{2.251511in}%
\pgfsys@useobject{currentmarker}{}%
\end{pgfscope}%
\end{pgfscope}%
\begin{pgfscope}%
\pgfsetbuttcap%
\pgfsetroundjoin%
\definecolor{currentfill}{rgb}{0.000000,0.000000,0.000000}%
\pgfsetfillcolor{currentfill}%
\pgfsetlinewidth{0.602250pt}%
\definecolor{currentstroke}{rgb}{0.000000,0.000000,0.000000}%
\pgfsetstrokecolor{currentstroke}%
\pgfsetdash{}{0pt}%
\pgfsys@defobject{currentmarker}{\pgfqpoint{-0.027778in}{0.000000in}}{\pgfqpoint{-0.000000in}{0.000000in}}{%
\pgfpathmoveto{\pgfqpoint{-0.000000in}{0.000000in}}%
\pgfpathlineto{\pgfqpoint{-0.027778in}{0.000000in}}%
\pgfusepath{stroke,fill}%
}%
\begin{pgfscope}%
\pgfsys@transformshift{0.588387in}{2.361753in}%
\pgfsys@useobject{currentmarker}{}%
\end{pgfscope}%
\end{pgfscope}%
\begin{pgfscope}%
\pgfsetbuttcap%
\pgfsetroundjoin%
\definecolor{currentfill}{rgb}{0.000000,0.000000,0.000000}%
\pgfsetfillcolor{currentfill}%
\pgfsetlinewidth{0.602250pt}%
\definecolor{currentstroke}{rgb}{0.000000,0.000000,0.000000}%
\pgfsetstrokecolor{currentstroke}%
\pgfsetdash{}{0pt}%
\pgfsys@defobject{currentmarker}{\pgfqpoint{-0.027778in}{0.000000in}}{\pgfqpoint{-0.000000in}{0.000000in}}{%
\pgfpathmoveto{\pgfqpoint{-0.000000in}{0.000000in}}%
\pgfpathlineto{\pgfqpoint{-0.027778in}{0.000000in}}%
\pgfusepath{stroke,fill}%
}%
\begin{pgfscope}%
\pgfsys@transformshift{0.588387in}{2.439970in}%
\pgfsys@useobject{currentmarker}{}%
\end{pgfscope}%
\end{pgfscope}%
\begin{pgfscope}%
\pgfsetbuttcap%
\pgfsetroundjoin%
\definecolor{currentfill}{rgb}{0.000000,0.000000,0.000000}%
\pgfsetfillcolor{currentfill}%
\pgfsetlinewidth{0.602250pt}%
\definecolor{currentstroke}{rgb}{0.000000,0.000000,0.000000}%
\pgfsetstrokecolor{currentstroke}%
\pgfsetdash{}{0pt}%
\pgfsys@defobject{currentmarker}{\pgfqpoint{-0.027778in}{0.000000in}}{\pgfqpoint{-0.000000in}{0.000000in}}{%
\pgfpathmoveto{\pgfqpoint{-0.000000in}{0.000000in}}%
\pgfpathlineto{\pgfqpoint{-0.027778in}{0.000000in}}%
\pgfusepath{stroke,fill}%
}%
\begin{pgfscope}%
\pgfsys@transformshift{0.588387in}{2.500640in}%
\pgfsys@useobject{currentmarker}{}%
\end{pgfscope}%
\end{pgfscope}%
\begin{pgfscope}%
\pgfsetbuttcap%
\pgfsetroundjoin%
\definecolor{currentfill}{rgb}{0.000000,0.000000,0.000000}%
\pgfsetfillcolor{currentfill}%
\pgfsetlinewidth{0.602250pt}%
\definecolor{currentstroke}{rgb}{0.000000,0.000000,0.000000}%
\pgfsetstrokecolor{currentstroke}%
\pgfsetdash{}{0pt}%
\pgfsys@defobject{currentmarker}{\pgfqpoint{-0.027778in}{0.000000in}}{\pgfqpoint{-0.000000in}{0.000000in}}{%
\pgfpathmoveto{\pgfqpoint{-0.000000in}{0.000000in}}%
\pgfpathlineto{\pgfqpoint{-0.027778in}{0.000000in}}%
\pgfusepath{stroke,fill}%
}%
\begin{pgfscope}%
\pgfsys@transformshift{0.588387in}{2.550211in}%
\pgfsys@useobject{currentmarker}{}%
\end{pgfscope}%
\end{pgfscope}%
\begin{pgfscope}%
\pgfsetbuttcap%
\pgfsetroundjoin%
\definecolor{currentfill}{rgb}{0.000000,0.000000,0.000000}%
\pgfsetfillcolor{currentfill}%
\pgfsetlinewidth{0.602250pt}%
\definecolor{currentstroke}{rgb}{0.000000,0.000000,0.000000}%
\pgfsetstrokecolor{currentstroke}%
\pgfsetdash{}{0pt}%
\pgfsys@defobject{currentmarker}{\pgfqpoint{-0.027778in}{0.000000in}}{\pgfqpoint{-0.000000in}{0.000000in}}{%
\pgfpathmoveto{\pgfqpoint{-0.000000in}{0.000000in}}%
\pgfpathlineto{\pgfqpoint{-0.027778in}{0.000000in}}%
\pgfusepath{stroke,fill}%
}%
\begin{pgfscope}%
\pgfsys@transformshift{0.588387in}{2.592123in}%
\pgfsys@useobject{currentmarker}{}%
\end{pgfscope}%
\end{pgfscope}%
\begin{pgfscope}%
\pgfsetbuttcap%
\pgfsetroundjoin%
\definecolor{currentfill}{rgb}{0.000000,0.000000,0.000000}%
\pgfsetfillcolor{currentfill}%
\pgfsetlinewidth{0.602250pt}%
\definecolor{currentstroke}{rgb}{0.000000,0.000000,0.000000}%
\pgfsetstrokecolor{currentstroke}%
\pgfsetdash{}{0pt}%
\pgfsys@defobject{currentmarker}{\pgfqpoint{-0.027778in}{0.000000in}}{\pgfqpoint{-0.000000in}{0.000000in}}{%
\pgfpathmoveto{\pgfqpoint{-0.000000in}{0.000000in}}%
\pgfpathlineto{\pgfqpoint{-0.027778in}{0.000000in}}%
\pgfusepath{stroke,fill}%
}%
\begin{pgfscope}%
\pgfsys@transformshift{0.588387in}{2.628429in}%
\pgfsys@useobject{currentmarker}{}%
\end{pgfscope}%
\end{pgfscope}%
\begin{pgfscope}%
\pgfsetbuttcap%
\pgfsetroundjoin%
\definecolor{currentfill}{rgb}{0.000000,0.000000,0.000000}%
\pgfsetfillcolor{currentfill}%
\pgfsetlinewidth{0.602250pt}%
\definecolor{currentstroke}{rgb}{0.000000,0.000000,0.000000}%
\pgfsetstrokecolor{currentstroke}%
\pgfsetdash{}{0pt}%
\pgfsys@defobject{currentmarker}{\pgfqpoint{-0.027778in}{0.000000in}}{\pgfqpoint{-0.000000in}{0.000000in}}{%
\pgfpathmoveto{\pgfqpoint{-0.000000in}{0.000000in}}%
\pgfpathlineto{\pgfqpoint{-0.027778in}{0.000000in}}%
\pgfusepath{stroke,fill}%
}%
\begin{pgfscope}%
\pgfsys@transformshift{0.588387in}{2.660453in}%
\pgfsys@useobject{currentmarker}{}%
\end{pgfscope}%
\end{pgfscope}%
\begin{pgfscope}%
\definecolor{textcolor}{rgb}{0.000000,0.000000,0.000000}%
\pgfsetstrokecolor{textcolor}%
\pgfsetfillcolor{textcolor}%
\pgftext[x=0.234413in,y=1.631490in,,bottom,rotate=90.000000]{\color{textcolor}{\rmfamily\fontsize{10.000000}{12.000000}\selectfont\catcode`\^=\active\def^{\ifmmode\sp\else\^{}\fi}\catcode`\%=\active\def%{\%}Time [ms]}}%
\end{pgfscope}%
\begin{pgfscope}%
\pgfpathrectangle{\pgfqpoint{0.588387in}{0.521603in}}{\pgfqpoint{3.660036in}{2.219773in}}%
\pgfusepath{clip}%
\pgfsetrectcap%
\pgfsetroundjoin%
\pgfsetlinewidth{1.505625pt}%
\pgfsetstrokecolor{currentstroke1}%
\pgfsetdash{}{0pt}%
\pgfpathmoveto{\pgfqpoint{0.754752in}{0.622502in}}%
\pgfpathlineto{\pgfqpoint{1.010699in}{0.836874in}}%
\pgfpathlineto{\pgfqpoint{1.266646in}{1.048990in}}%
\pgfpathlineto{\pgfqpoint{1.650565in}{1.259212in}}%
\pgfpathlineto{\pgfqpoint{1.778539in}{1.577656in}}%
\pgfpathlineto{\pgfqpoint{2.034485in}{1.813000in}}%
\pgfpathlineto{\pgfqpoint{2.418405in}{2.040235in}}%
\pgfpathlineto{\pgfqpoint{2.674352in}{2.329559in}}%
\pgfpathlineto{\pgfqpoint{3.058271in}{2.618319in}}%
\pgfpathlineto{\pgfqpoint{3.314218in}{2.599732in}}%
\pgfusepath{stroke}%
\end{pgfscope}%
\begin{pgfscope}%
\pgfpathrectangle{\pgfqpoint{0.588387in}{0.521603in}}{\pgfqpoint{3.660036in}{2.219773in}}%
\pgfusepath{clip}%
\pgfsetrectcap%
\pgfsetroundjoin%
\pgfsetlinewidth{1.505625pt}%
\pgfsetstrokecolor{currentstroke2}%
\pgfsetdash{}{0pt}%
\pgfpathmoveto{\pgfqpoint{0.754752in}{0.622502in}}%
\pgfpathlineto{\pgfqpoint{1.010699in}{0.810961in}}%
\pgfpathlineto{\pgfqpoint{1.266646in}{1.054597in}}%
\pgfpathlineto{\pgfqpoint{1.650565in}{1.307034in}}%
\pgfpathlineto{\pgfqpoint{1.778539in}{1.549953in}}%
\pgfpathlineto{\pgfqpoint{2.034485in}{1.816052in}}%
\pgfpathlineto{\pgfqpoint{2.418405in}{2.035311in}}%
\pgfpathlineto{\pgfqpoint{2.674352in}{2.253189in}}%
\pgfpathlineto{\pgfqpoint{3.058271in}{2.611858in}}%
\pgfpathlineto{\pgfqpoint{3.314218in}{2.620278in}}%
\pgfpathlineto{\pgfqpoint{3.698138in}{2.526646in}}%
\pgfusepath{stroke}%
\end{pgfscope}%
\begin{pgfscope}%
\pgfpathrectangle{\pgfqpoint{0.588387in}{0.521603in}}{\pgfqpoint{3.660036in}{2.219773in}}%
\pgfusepath{clip}%
\pgfsetrectcap%
\pgfsetroundjoin%
\pgfsetlinewidth{1.505625pt}%
\pgfsetstrokecolor{currentstroke3}%
\pgfsetdash{}{0pt}%
\pgfpathmoveto{\pgfqpoint{0.754752in}{0.672073in}}%
\pgfpathlineto{\pgfqpoint{1.010699in}{0.810961in}}%
\pgfpathlineto{\pgfqpoint{1.266646in}{1.031443in}}%
\pgfpathlineto{\pgfqpoint{1.650565in}{1.259212in}}%
\pgfpathlineto{\pgfqpoint{1.778539in}{1.517593in}}%
\pgfpathlineto{\pgfqpoint{2.034485in}{1.780708in}}%
\pgfpathlineto{\pgfqpoint{2.418405in}{1.892990in}}%
\pgfpathlineto{\pgfqpoint{2.674352in}{2.211668in}}%
\pgfpathlineto{\pgfqpoint{3.058271in}{2.119116in}}%
\pgfpathlineto{\pgfqpoint{3.314218in}{2.454711in}}%
\pgfpathlineto{\pgfqpoint{3.698138in}{2.579553in}}%
\pgfpathlineto{\pgfqpoint{4.082057in}{2.605148in}}%
\pgfusepath{stroke}%
\end{pgfscope}%
\begin{pgfscope}%
\pgfpathrectangle{\pgfqpoint{0.588387in}{0.521603in}}{\pgfqpoint{3.660036in}{2.219773in}}%
\pgfusepath{clip}%
\pgfsetrectcap%
\pgfsetroundjoin%
\pgfsetlinewidth{1.505625pt}%
\pgfsetstrokecolor{currentstroke4}%
\pgfsetdash{}{0pt}%
\pgfpathmoveto{\pgfqpoint{0.754752in}{0.622502in}}%
\pgfpathlineto{\pgfqpoint{1.010699in}{0.810961in}}%
\pgfpathlineto{\pgfqpoint{1.266646in}{1.054597in}}%
\pgfpathlineto{\pgfqpoint{1.650565in}{1.251254in}}%
\pgfpathlineto{\pgfqpoint{1.778539in}{1.524085in}}%
\pgfpathlineto{\pgfqpoint{2.034485in}{1.773425in}}%
\pgfpathlineto{\pgfqpoint{2.418405in}{1.883158in}}%
\pgfpathlineto{\pgfqpoint{2.674352in}{2.147378in}}%
\pgfpathlineto{\pgfqpoint{3.058271in}{2.102585in}}%
\pgfpathlineto{\pgfqpoint{3.314218in}{2.444862in}}%
\pgfpathlineto{\pgfqpoint{3.698138in}{2.561125in}}%
\pgfpathlineto{\pgfqpoint{4.082057in}{2.613114in}}%
\pgfusepath{stroke}%
\end{pgfscope}%
\begin{pgfscope}%
\pgfpathrectangle{\pgfqpoint{0.588387in}{0.521603in}}{\pgfqpoint{3.660036in}{2.219773in}}%
\pgfusepath{clip}%
\pgfsetrectcap%
\pgfsetroundjoin%
\pgfsetlinewidth{1.505625pt}%
\pgfsetstrokecolor{currentstroke5}%
\pgfsetdash{}{0pt}%
\pgfpathmoveto{\pgfqpoint{0.754752in}{0.622502in}}%
\pgfpathlineto{\pgfqpoint{1.010699in}{0.810961in}}%
\pgfpathlineto{\pgfqpoint{1.266646in}{1.043266in}}%
\pgfpathlineto{\pgfqpoint{1.650565in}{1.253932in}}%
\pgfpathlineto{\pgfqpoint{1.778539in}{1.531387in}}%
\pgfpathlineto{\pgfqpoint{2.034485in}{1.775278in}}%
\pgfpathlineto{\pgfqpoint{2.418405in}{2.002553in}}%
\pgfpathlineto{\pgfqpoint{2.674352in}{2.200502in}}%
\pgfpathlineto{\pgfqpoint{3.058271in}{2.187765in}}%
\pgfpathlineto{\pgfqpoint{3.314218in}{2.516349in}}%
\pgfpathlineto{\pgfqpoint{3.698138in}{2.588396in}}%
\pgfpathlineto{\pgfqpoint{4.082057in}{2.595573in}}%
\pgfusepath{stroke}%
\end{pgfscope}%
\begin{pgfscope}%
\pgfpathrectangle{\pgfqpoint{0.588387in}{0.521603in}}{\pgfqpoint{3.660036in}{2.219773in}}%
\pgfusepath{clip}%
\pgfsetrectcap%
\pgfsetroundjoin%
\pgfsetlinewidth{1.505625pt}%
\pgfsetstrokecolor{currentstroke6}%
\pgfsetdash{}{0pt}%
\pgfpathmoveto{\pgfqpoint{0.754752in}{0.622502in}}%
\pgfpathlineto{\pgfqpoint{1.010699in}{0.810961in}}%
\pgfpathlineto{\pgfqpoint{1.266646in}{1.048990in}}%
\pgfpathlineto{\pgfqpoint{1.650565in}{1.245816in}}%
\pgfpathlineto{\pgfqpoint{1.778539in}{1.529459in}}%
\pgfpathlineto{\pgfqpoint{2.034485in}{1.765825in}}%
\pgfpathlineto{\pgfqpoint{2.418405in}{1.896775in}}%
\pgfpathlineto{\pgfqpoint{2.674352in}{2.183478in}}%
\pgfpathlineto{\pgfqpoint{3.058271in}{2.105624in}}%
\pgfpathlineto{\pgfqpoint{3.314218in}{2.479310in}}%
\pgfpathlineto{\pgfqpoint{3.698138in}{2.573163in}}%
\pgfpathlineto{\pgfqpoint{4.082057in}{2.633702in}}%
\pgfusepath{stroke}%
\end{pgfscope}%
\begin{pgfscope}%
\pgfpathrectangle{\pgfqpoint{0.588387in}{0.521603in}}{\pgfqpoint{3.660036in}{2.219773in}}%
\pgfusepath{clip}%
\pgfsetrectcap%
\pgfsetroundjoin%
\pgfsetlinewidth{1.505625pt}%
\pgfsetstrokecolor{currentstroke7}%
\pgfsetdash{}{0pt}%
\pgfpathmoveto{\pgfqpoint{0.754752in}{0.672073in}}%
\pgfpathlineto{\pgfqpoint{1.010699in}{0.810961in}}%
\pgfpathlineto{\pgfqpoint{1.266646in}{1.048990in}}%
\pgfpathlineto{\pgfqpoint{1.650565in}{1.274462in}}%
\pgfpathlineto{\pgfqpoint{1.778539in}{1.548153in}}%
\pgfpathlineto{\pgfqpoint{2.034485in}{1.837666in}}%
\pgfpathlineto{\pgfqpoint{2.418405in}{2.007933in}}%
\pgfpathlineto{\pgfqpoint{2.674352in}{2.183565in}}%
\pgfpathlineto{\pgfqpoint{3.058271in}{2.425702in}}%
\pgfpathlineto{\pgfqpoint{3.314218in}{2.614459in}}%
\pgfpathlineto{\pgfqpoint{3.698138in}{2.640478in}}%
\pgfusepath{stroke}%
\end{pgfscope}%
\begin{pgfscope}%
\pgfpathrectangle{\pgfqpoint{0.588387in}{0.521603in}}{\pgfqpoint{3.660036in}{2.219773in}}%
\pgfusepath{clip}%
\pgfsetrectcap%
\pgfsetroundjoin%
\pgfsetlinewidth{1.505625pt}%
\definecolor{currentstroke}{rgb}{0.498039,0.498039,0.498039}%
\pgfsetstrokecolor{currentstroke}%
\pgfsetdash{}{0pt}%
\pgfpathmoveto{\pgfqpoint{1.010699in}{0.810961in}}%
\pgfpathlineto{\pgfqpoint{1.266646in}{1.031443in}}%
\pgfpathlineto{\pgfqpoint{1.650565in}{1.243055in}}%
\pgfpathlineto{\pgfqpoint{1.778539in}{1.540829in}}%
\pgfpathlineto{\pgfqpoint{2.034485in}{1.784088in}}%
\pgfpathlineto{\pgfqpoint{2.418405in}{1.964714in}}%
\pgfpathlineto{\pgfqpoint{2.674352in}{2.188837in}}%
\pgfpathlineto{\pgfqpoint{3.058271in}{2.415483in}}%
\pgfpathlineto{\pgfqpoint{3.314218in}{2.537460in}}%
\pgfpathlineto{\pgfqpoint{3.698138in}{2.637042in}}%
\pgfpathlineto{\pgfqpoint{4.082057in}{2.547890in}}%
\pgfusepath{stroke}%
\end{pgfscope}%
\begin{pgfscope}%
\pgfpathrectangle{\pgfqpoint{0.588387in}{0.521603in}}{\pgfqpoint{3.660036in}{2.219773in}}%
\pgfusepath{clip}%
\pgfsetrectcap%
\pgfsetroundjoin%
\pgfsetlinewidth{1.505625pt}%
\definecolor{currentstroke}{rgb}{0.737255,0.741176,0.133333}%
\pgfsetstrokecolor{currentstroke}%
\pgfsetdash{}{0pt}%
\pgfpathmoveto{\pgfqpoint{2.034485in}{1.825613in}}%
\pgfpathlineto{\pgfqpoint{2.418405in}{2.194722in}}%
\pgfpathlineto{\pgfqpoint{2.674352in}{2.565251in}}%
\pgfusepath{stroke}%
\end{pgfscope}%
\begin{pgfscope}%
\pgfsetrectcap%
\pgfsetmiterjoin%
\pgfsetlinewidth{0.803000pt}%
\definecolor{currentstroke}{rgb}{0.000000,0.000000,0.000000}%
\pgfsetstrokecolor{currentstroke}%
\pgfsetdash{}{0pt}%
\pgfpathmoveto{\pgfqpoint{0.588387in}{0.521603in}}%
\pgfpathlineto{\pgfqpoint{0.588387in}{2.741376in}}%
\pgfusepath{stroke}%
\end{pgfscope}%
\begin{pgfscope}%
\pgfsetrectcap%
\pgfsetmiterjoin%
\pgfsetlinewidth{0.803000pt}%
\definecolor{currentstroke}{rgb}{0.000000,0.000000,0.000000}%
\pgfsetstrokecolor{currentstroke}%
\pgfsetdash{}{0pt}%
\pgfpathmoveto{\pgfqpoint{4.248423in}{0.521603in}}%
\pgfpathlineto{\pgfqpoint{4.248423in}{2.741376in}}%
\pgfusepath{stroke}%
\end{pgfscope}%
\begin{pgfscope}%
\pgfsetrectcap%
\pgfsetmiterjoin%
\pgfsetlinewidth{0.803000pt}%
\definecolor{currentstroke}{rgb}{0.000000,0.000000,0.000000}%
\pgfsetstrokecolor{currentstroke}%
\pgfsetdash{}{0pt}%
\pgfpathmoveto{\pgfqpoint{0.588387in}{0.521603in}}%
\pgfpathlineto{\pgfqpoint{4.248423in}{0.521603in}}%
\pgfusepath{stroke}%
\end{pgfscope}%
\begin{pgfscope}%
\pgfsetrectcap%
\pgfsetmiterjoin%
\pgfsetlinewidth{0.803000pt}%
\definecolor{currentstroke}{rgb}{0.000000,0.000000,0.000000}%
\pgfsetstrokecolor{currentstroke}%
\pgfsetdash{}{0pt}%
\pgfpathmoveto{\pgfqpoint{0.588387in}{2.741376in}}%
\pgfpathlineto{\pgfqpoint{4.248423in}{2.741376in}}%
\pgfusepath{stroke}%
\end{pgfscope}%
\begin{pgfscope}%
\pgfsetbuttcap%
\pgfsetmiterjoin%
\definecolor{currentfill}{rgb}{1.000000,1.000000,1.000000}%
\pgfsetfillcolor{currentfill}%
\pgfsetfillopacity{0.800000}%
\pgfsetlinewidth{1.003750pt}%
\definecolor{currentstroke}{rgb}{0.800000,0.800000,0.800000}%
\pgfsetstrokecolor{currentstroke}%
\pgfsetstrokeopacity{0.800000}%
\pgfsetdash{}{0pt}%
\pgfpathmoveto{\pgfqpoint{4.365089in}{0.378553in}}%
\pgfpathlineto{\pgfqpoint{8.251043in}{0.378553in}}%
\pgfpathquadraticcurveto{\pgfqpoint{8.284376in}{0.378553in}}{\pgfqpoint{8.284376in}{0.411886in}}%
\pgfpathlineto{\pgfqpoint{8.284376in}{2.624710in}}%
\pgfpathquadraticcurveto{\pgfqpoint{8.284376in}{2.658043in}}{\pgfqpoint{8.251043in}{2.658043in}}%
\pgfpathlineto{\pgfqpoint{4.365089in}{2.658043in}}%
\pgfpathquadraticcurveto{\pgfqpoint{4.331756in}{2.658043in}}{\pgfqpoint{4.331756in}{2.624710in}}%
\pgfpathlineto{\pgfqpoint{4.331756in}{0.411886in}}%
\pgfpathquadraticcurveto{\pgfqpoint{4.331756in}{0.378553in}}{\pgfqpoint{4.365089in}{0.378553in}}%
\pgfpathlineto{\pgfqpoint{4.365089in}{0.378553in}}%
\pgfpathclose%
\pgfusepath{stroke,fill}%
\end{pgfscope}%
\begin{pgfscope}%
\pgfsetrectcap%
\pgfsetroundjoin%
\pgfsetlinewidth{1.505625pt}%
\definecolor{currentstroke}{rgb}{0.737255,0.741176,0.133333}%
\pgfsetstrokecolor{currentstroke}%
\pgfsetdash{}{0pt}%
\pgfpathmoveto{\pgfqpoint{4.398423in}{2.523082in}}%
\pgfpathlineto{\pgfqpoint{4.565089in}{2.523082in}}%
\pgfpathlineto{\pgfqpoint{4.731756in}{2.523082in}}%
\pgfusepath{stroke}%
\end{pgfscope}%
\begin{pgfscope}%
\definecolor{textcolor}{rgb}{0.000000,0.000000,0.000000}%
\pgfsetstrokecolor{textcolor}%
\pgfsetfillcolor{textcolor}%
\pgftext[x=4.865089in,y=2.464749in,left,base]{\color{textcolor}{\rmfamily\fontsize{12.000000}{14.400000}\selectfont\catcode`\^=\active\def^{\ifmmode\sp\else\^{}\fi}\catcode`\%=\active\def%{\%}\NaiveCycles{}}}%
\end{pgfscope}%
\begin{pgfscope}%
\pgfsetrectcap%
\pgfsetroundjoin%
\pgfsetlinewidth{1.505625pt}%
\pgfsetstrokecolor{currentstroke1}%
\pgfsetdash{}{0pt}%
\pgfpathmoveto{\pgfqpoint{4.398423in}{2.278453in}}%
\pgfpathlineto{\pgfqpoint{4.565089in}{2.278453in}}%
\pgfpathlineto{\pgfqpoint{4.731756in}{2.278453in}}%
\pgfusepath{stroke}%
\end{pgfscope}%
\begin{pgfscope}%
\definecolor{textcolor}{rgb}{0.000000,0.000000,0.000000}%
\pgfsetstrokecolor{textcolor}%
\pgfsetfillcolor{textcolor}%
\pgftext[x=4.865089in,y=2.220120in,left,base]{\color{textcolor}{\rmfamily\fontsize{12.000000}{14.400000}\selectfont\catcode`\^=\active\def^{\ifmmode\sp\else\^{}\fi}\catcode`\%=\active\def%{\%}\CyclesMatchChunks{} \& \MergeLinear{}}}%
\end{pgfscope}%
\begin{pgfscope}%
\pgfsetrectcap%
\pgfsetroundjoin%
\pgfsetlinewidth{1.505625pt}%
\pgfsetstrokecolor{currentstroke2}%
\pgfsetdash{}{0pt}%
\pgfpathmoveto{\pgfqpoint{4.398423in}{2.029186in}}%
\pgfpathlineto{\pgfqpoint{4.565089in}{2.029186in}}%
\pgfpathlineto{\pgfqpoint{4.731756in}{2.029186in}}%
\pgfusepath{stroke}%
\end{pgfscope}%
\begin{pgfscope}%
\definecolor{textcolor}{rgb}{0.000000,0.000000,0.000000}%
\pgfsetstrokecolor{textcolor}%
\pgfsetfillcolor{textcolor}%
\pgftext[x=4.865089in,y=1.970853in,left,base]{\color{textcolor}{\rmfamily\fontsize{12.000000}{14.400000}\selectfont\catcode`\^=\active\def^{\ifmmode\sp\else\^{}\fi}\catcode`\%=\active\def%{\%}\CyclesMatchChunks{} \& \SharedVertices{}}}%
\end{pgfscope}%
\begin{pgfscope}%
\pgfsetrectcap%
\pgfsetroundjoin%
\pgfsetlinewidth{1.505625pt}%
\pgfsetstrokecolor{currentstroke3}%
\pgfsetdash{}{0pt}%
\pgfpathmoveto{\pgfqpoint{4.398423in}{1.779919in}}%
\pgfpathlineto{\pgfqpoint{4.565089in}{1.779919in}}%
\pgfpathlineto{\pgfqpoint{4.731756in}{1.779919in}}%
\pgfusepath{stroke}%
\end{pgfscope}%
\begin{pgfscope}%
\definecolor{textcolor}{rgb}{0.000000,0.000000,0.000000}%
\pgfsetstrokecolor{textcolor}%
\pgfsetfillcolor{textcolor}%
\pgftext[x=4.865089in,y=1.721585in,left,base]{\color{textcolor}{\rmfamily\fontsize{12.000000}{14.400000}\selectfont\catcode`\^=\active\def^{\ifmmode\sp\else\^{}\fi}\catcode`\%=\active\def%{\%}\Neighbors{} \& \MergeLinear{}}}%
\end{pgfscope}%
\begin{pgfscope}%
\pgfsetrectcap%
\pgfsetroundjoin%
\pgfsetlinewidth{1.505625pt}%
\pgfsetstrokecolor{currentstroke4}%
\pgfsetdash{}{0pt}%
\pgfpathmoveto{\pgfqpoint{4.398423in}{1.535290in}}%
\pgfpathlineto{\pgfqpoint{4.565089in}{1.535290in}}%
\pgfpathlineto{\pgfqpoint{4.731756in}{1.535290in}}%
\pgfusepath{stroke}%
\end{pgfscope}%
\begin{pgfscope}%
\definecolor{textcolor}{rgb}{0.000000,0.000000,0.000000}%
\pgfsetstrokecolor{textcolor}%
\pgfsetfillcolor{textcolor}%
\pgftext[x=4.865089in,y=1.476957in,left,base]{\color{textcolor}{\rmfamily\fontsize{12.000000}{14.400000}\selectfont\catcode`\^=\active\def^{\ifmmode\sp\else\^{}\fi}\catcode`\%=\active\def%{\%}\Neighbors{} \& \SharedVertices{}}}%
\end{pgfscope}%
\begin{pgfscope}%
\pgfsetrectcap%
\pgfsetroundjoin%
\pgfsetlinewidth{1.505625pt}%
\pgfsetstrokecolor{currentstroke5}%
\pgfsetdash{}{0pt}%
\pgfpathmoveto{\pgfqpoint{4.398423in}{1.286023in}}%
\pgfpathlineto{\pgfqpoint{4.565089in}{1.286023in}}%
\pgfpathlineto{\pgfqpoint{4.731756in}{1.286023in}}%
\pgfusepath{stroke}%
\end{pgfscope}%
\begin{pgfscope}%
\definecolor{textcolor}{rgb}{0.000000,0.000000,0.000000}%
\pgfsetstrokecolor{textcolor}%
\pgfsetfillcolor{textcolor}%
\pgftext[x=4.865089in,y=1.227689in,left,base]{\color{textcolor}{\rmfamily\fontsize{12.000000}{14.400000}\selectfont\catcode`\^=\active\def^{\ifmmode\sp\else\^{}\fi}\catcode`\%=\active\def%{\%}\NeighborsDegree{} \& \MergeLinear{}}}%
\end{pgfscope}%
\begin{pgfscope}%
\pgfsetrectcap%
\pgfsetroundjoin%
\pgfsetlinewidth{1.505625pt}%
\pgfsetstrokecolor{currentstroke6}%
\pgfsetdash{}{0pt}%
\pgfpathmoveto{\pgfqpoint{4.398423in}{1.036755in}}%
\pgfpathlineto{\pgfqpoint{4.565089in}{1.036755in}}%
\pgfpathlineto{\pgfqpoint{4.731756in}{1.036755in}}%
\pgfusepath{stroke}%
\end{pgfscope}%
\begin{pgfscope}%
\definecolor{textcolor}{rgb}{0.000000,0.000000,0.000000}%
\pgfsetstrokecolor{textcolor}%
\pgfsetfillcolor{textcolor}%
\pgftext[x=4.865089in,y=0.978422in,left,base]{\color{textcolor}{\rmfamily\fontsize{12.000000}{14.400000}\selectfont\catcode`\^=\active\def^{\ifmmode\sp\else\^{}\fi}\catcode`\%=\active\def%{\%}\NeighborsDegree{} \& \SharedVertices{}}}%
\end{pgfscope}%
\begin{pgfscope}%
\pgfsetrectcap%
\pgfsetroundjoin%
\pgfsetlinewidth{1.505625pt}%
\pgfsetstrokecolor{currentstroke7}%
\pgfsetdash{}{0pt}%
\pgfpathmoveto{\pgfqpoint{4.398423in}{0.787488in}}%
\pgfpathlineto{\pgfqpoint{4.565089in}{0.787488in}}%
\pgfpathlineto{\pgfqpoint{4.731756in}{0.787488in}}%
\pgfusepath{stroke}%
\end{pgfscope}%
\begin{pgfscope}%
\definecolor{textcolor}{rgb}{0.000000,0.000000,0.000000}%
\pgfsetstrokecolor{textcolor}%
\pgfsetfillcolor{textcolor}%
\pgftext[x=4.865089in,y=0.729155in,left,base]{\color{textcolor}{\rmfamily\fontsize{12.000000}{14.400000}\selectfont\catcode`\^=\active\def^{\ifmmode\sp\else\^{}\fi}\catcode`\%=\active\def%{\%}\None{} \& \MergeLinear{}}}%
\end{pgfscope}%
\begin{pgfscope}%
\pgfsetrectcap%
\pgfsetroundjoin%
\pgfsetlinewidth{1.505625pt}%
\definecolor{currentstroke}{rgb}{0.498039,0.498039,0.498039}%
\pgfsetstrokecolor{currentstroke}%
\pgfsetdash{}{0pt}%
\pgfpathmoveto{\pgfqpoint{4.398423in}{0.542859in}}%
\pgfpathlineto{\pgfqpoint{4.565089in}{0.542859in}}%
\pgfpathlineto{\pgfqpoint{4.731756in}{0.542859in}}%
\pgfusepath{stroke}%
\end{pgfscope}%
\begin{pgfscope}%
\definecolor{textcolor}{rgb}{0.000000,0.000000,0.000000}%
\pgfsetstrokecolor{textcolor}%
\pgfsetfillcolor{textcolor}%
\pgftext[x=4.865089in,y=0.484526in,left,base]{\color{textcolor}{\rmfamily\fontsize{12.000000}{14.400000}\selectfont\catcode`\^=\active\def^{\ifmmode\sp\else\^{}\fi}\catcode`\%=\active\def%{\%}\None{} \& \SharedVertices{}}}%
\end{pgfscope}%
\end{pgfpicture}%
\makeatother%
\endgroup%
}
	\caption[Mean runtime for graphs with no 3 nor 4 cycles (all)]{
		Mean running time to find all NAC-colorings for graphs with no three nor four cycles.}%
	\label{fig:graph_count_no_3_nor_4_cycles_all_runtime}
\end{figure}%
% \begin{figure}[thbp]
% 	\centering
% 	\scalebox{\BenchFigureScale}{%% Creator: Matplotlib, PGF backend
%%
%% To include the figure in your LaTeX document, write
%%   \input{<filename>.pgf}
%%
%% Make sure the required packages are loaded in your preamble
%%   \usepackage{pgf}
%%
%% Also ensure that all the required font packages are loaded; for instance,
%% the lmodern package is sometimes necessary when using math font.
%%   \usepackage{lmodern}
%%
%% Figures using additional raster images can only be included by \input if
%% they are in the same directory as the main LaTeX file. For loading figures
%% from other directories you can use the `import` package
%%   \usepackage{import}
%%
%% and then include the figures with
%%   \import{<path to file>}{<filename>.pgf}
%%
%% Matplotlib used the following preamble
%%   \def\mathdefault#1{#1}
%%   \everymath=\expandafter{\the\everymath\displaystyle}
%%   \IfFileExists{scrextend.sty}{
%%     \usepackage[fontsize=10.000000pt]{scrextend}
%%   }{
%%     \renewcommand{\normalsize}{\fontsize{10.000000}{12.000000}\selectfont}
%%     \normalsize
%%   }
%%   
%%   \ifdefined\pdftexversion\else  % non-pdftex case.
%%     \usepackage{fontspec}
%%     \setmainfont{DejaVuSans.ttf}[Path=\detokenize{/home/petr/Projects/PyRigi/.venv/lib/python3.12/site-packages/matplotlib/mpl-data/fonts/ttf/}]
%%     \setsansfont{DejaVuSans.ttf}[Path=\detokenize{/home/petr/Projects/PyRigi/.venv/lib/python3.12/site-packages/matplotlib/mpl-data/fonts/ttf/}]
%%     \setmonofont{DejaVuSansMono.ttf}[Path=\detokenize{/home/petr/Projects/PyRigi/.venv/lib/python3.12/site-packages/matplotlib/mpl-data/fonts/ttf/}]
%%   \fi
%%   \makeatletter\@ifpackageloaded{underscore}{}{\usepackage[strings]{underscore}}\makeatother
%%
\begingroup%
\makeatletter%
\begin{pgfpicture}%
\pgfpathrectangle{\pgfpointorigin}{\pgfqpoint{8.384376in}{2.841849in}}%
\pgfusepath{use as bounding box, clip}%
\begin{pgfscope}%
\pgfsetbuttcap%
\pgfsetmiterjoin%
\definecolor{currentfill}{rgb}{1.000000,1.000000,1.000000}%
\pgfsetfillcolor{currentfill}%
\pgfsetlinewidth{0.000000pt}%
\definecolor{currentstroke}{rgb}{1.000000,1.000000,1.000000}%
\pgfsetstrokecolor{currentstroke}%
\pgfsetdash{}{0pt}%
\pgfpathmoveto{\pgfqpoint{0.000000in}{0.000000in}}%
\pgfpathlineto{\pgfqpoint{8.384376in}{0.000000in}}%
\pgfpathlineto{\pgfqpoint{8.384376in}{2.841849in}}%
\pgfpathlineto{\pgfqpoint{0.000000in}{2.841849in}}%
\pgfpathlineto{\pgfqpoint{0.000000in}{0.000000in}}%
\pgfpathclose%
\pgfusepath{fill}%
\end{pgfscope}%
\begin{pgfscope}%
\pgfsetbuttcap%
\pgfsetmiterjoin%
\definecolor{currentfill}{rgb}{1.000000,1.000000,1.000000}%
\pgfsetfillcolor{currentfill}%
\pgfsetlinewidth{0.000000pt}%
\definecolor{currentstroke}{rgb}{0.000000,0.000000,0.000000}%
\pgfsetstrokecolor{currentstroke}%
\pgfsetstrokeopacity{0.000000}%
\pgfsetdash{}{0pt}%
\pgfpathmoveto{\pgfqpoint{0.588387in}{0.521603in}}%
\pgfpathlineto{\pgfqpoint{6.119045in}{0.521603in}}%
\pgfpathlineto{\pgfqpoint{6.119045in}{2.531888in}}%
\pgfpathlineto{\pgfqpoint{0.588387in}{2.531888in}}%
\pgfpathlineto{\pgfqpoint{0.588387in}{0.521603in}}%
\pgfpathclose%
\pgfusepath{fill}%
\end{pgfscope}%
\begin{pgfscope}%
\pgfsetbuttcap%
\pgfsetroundjoin%
\definecolor{currentfill}{rgb}{0.000000,0.000000,0.000000}%
\pgfsetfillcolor{currentfill}%
\pgfsetlinewidth{0.803000pt}%
\definecolor{currentstroke}{rgb}{0.000000,0.000000,0.000000}%
\pgfsetstrokecolor{currentstroke}%
\pgfsetdash{}{0pt}%
\pgfsys@defobject{currentmarker}{\pgfqpoint{0.000000in}{-0.048611in}}{\pgfqpoint{0.000000in}{0.000000in}}{%
\pgfpathmoveto{\pgfqpoint{0.000000in}{0.000000in}}%
\pgfpathlineto{\pgfqpoint{0.000000in}{-0.048611in}}%
\pgfusepath{stroke,fill}%
}%
\begin{pgfscope}%
\pgfsys@transformshift{0.839781in}{0.521603in}%
\pgfsys@useobject{currentmarker}{}%
\end{pgfscope}%
\end{pgfscope}%
\begin{pgfscope}%
\definecolor{textcolor}{rgb}{0.000000,0.000000,0.000000}%
\pgfsetstrokecolor{textcolor}%
\pgfsetfillcolor{textcolor}%
\pgftext[x=0.839781in,y=0.424381in,,top]{\color{textcolor}{\rmfamily\fontsize{10.000000}{12.000000}\selectfont\catcode`\^=\active\def^{\ifmmode\sp\else\^{}\fi}\catcode`\%=\active\def%{\%}$\mathdefault{10}$}}%
\end{pgfscope}%
\begin{pgfscope}%
\pgfsetbuttcap%
\pgfsetroundjoin%
\definecolor{currentfill}{rgb}{0.000000,0.000000,0.000000}%
\pgfsetfillcolor{currentfill}%
\pgfsetlinewidth{0.803000pt}%
\definecolor{currentstroke}{rgb}{0.000000,0.000000,0.000000}%
\pgfsetstrokecolor{currentstroke}%
\pgfsetdash{}{0pt}%
\pgfsys@defobject{currentmarker}{\pgfqpoint{0.000000in}{-0.048611in}}{\pgfqpoint{0.000000in}{0.000000in}}{%
\pgfpathmoveto{\pgfqpoint{0.000000in}{0.000000in}}%
\pgfpathlineto{\pgfqpoint{0.000000in}{-0.048611in}}%
\pgfusepath{stroke,fill}%
}%
\begin{pgfscope}%
\pgfsys@transformshift{1.753939in}{0.521603in}%
\pgfsys@useobject{currentmarker}{}%
\end{pgfscope}%
\end{pgfscope}%
\begin{pgfscope}%
\definecolor{textcolor}{rgb}{0.000000,0.000000,0.000000}%
\pgfsetstrokecolor{textcolor}%
\pgfsetfillcolor{textcolor}%
\pgftext[x=1.753939in,y=0.424381in,,top]{\color{textcolor}{\rmfamily\fontsize{10.000000}{12.000000}\selectfont\catcode`\^=\active\def^{\ifmmode\sp\else\^{}\fi}\catcode`\%=\active\def%{\%}$\mathdefault{12}$}}%
\end{pgfscope}%
\begin{pgfscope}%
\pgfsetbuttcap%
\pgfsetroundjoin%
\definecolor{currentfill}{rgb}{0.000000,0.000000,0.000000}%
\pgfsetfillcolor{currentfill}%
\pgfsetlinewidth{0.803000pt}%
\definecolor{currentstroke}{rgb}{0.000000,0.000000,0.000000}%
\pgfsetstrokecolor{currentstroke}%
\pgfsetdash{}{0pt}%
\pgfsys@defobject{currentmarker}{\pgfqpoint{0.000000in}{-0.048611in}}{\pgfqpoint{0.000000in}{0.000000in}}{%
\pgfpathmoveto{\pgfqpoint{0.000000in}{0.000000in}}%
\pgfpathlineto{\pgfqpoint{0.000000in}{-0.048611in}}%
\pgfusepath{stroke,fill}%
}%
\begin{pgfscope}%
\pgfsys@transformshift{2.668097in}{0.521603in}%
\pgfsys@useobject{currentmarker}{}%
\end{pgfscope}%
\end{pgfscope}%
\begin{pgfscope}%
\definecolor{textcolor}{rgb}{0.000000,0.000000,0.000000}%
\pgfsetstrokecolor{textcolor}%
\pgfsetfillcolor{textcolor}%
\pgftext[x=2.668097in,y=0.424381in,,top]{\color{textcolor}{\rmfamily\fontsize{10.000000}{12.000000}\selectfont\catcode`\^=\active\def^{\ifmmode\sp\else\^{}\fi}\catcode`\%=\active\def%{\%}$\mathdefault{14}$}}%
\end{pgfscope}%
\begin{pgfscope}%
\pgfsetbuttcap%
\pgfsetroundjoin%
\definecolor{currentfill}{rgb}{0.000000,0.000000,0.000000}%
\pgfsetfillcolor{currentfill}%
\pgfsetlinewidth{0.803000pt}%
\definecolor{currentstroke}{rgb}{0.000000,0.000000,0.000000}%
\pgfsetstrokecolor{currentstroke}%
\pgfsetdash{}{0pt}%
\pgfsys@defobject{currentmarker}{\pgfqpoint{0.000000in}{-0.048611in}}{\pgfqpoint{0.000000in}{0.000000in}}{%
\pgfpathmoveto{\pgfqpoint{0.000000in}{0.000000in}}%
\pgfpathlineto{\pgfqpoint{0.000000in}{-0.048611in}}%
\pgfusepath{stroke,fill}%
}%
\begin{pgfscope}%
\pgfsys@transformshift{3.582255in}{0.521603in}%
\pgfsys@useobject{currentmarker}{}%
\end{pgfscope}%
\end{pgfscope}%
\begin{pgfscope}%
\definecolor{textcolor}{rgb}{0.000000,0.000000,0.000000}%
\pgfsetstrokecolor{textcolor}%
\pgfsetfillcolor{textcolor}%
\pgftext[x=3.582255in,y=0.424381in,,top]{\color{textcolor}{\rmfamily\fontsize{10.000000}{12.000000}\selectfont\catcode`\^=\active\def^{\ifmmode\sp\else\^{}\fi}\catcode`\%=\active\def%{\%}$\mathdefault{16}$}}%
\end{pgfscope}%
\begin{pgfscope}%
\pgfsetbuttcap%
\pgfsetroundjoin%
\definecolor{currentfill}{rgb}{0.000000,0.000000,0.000000}%
\pgfsetfillcolor{currentfill}%
\pgfsetlinewidth{0.803000pt}%
\definecolor{currentstroke}{rgb}{0.000000,0.000000,0.000000}%
\pgfsetstrokecolor{currentstroke}%
\pgfsetdash{}{0pt}%
\pgfsys@defobject{currentmarker}{\pgfqpoint{0.000000in}{-0.048611in}}{\pgfqpoint{0.000000in}{0.000000in}}{%
\pgfpathmoveto{\pgfqpoint{0.000000in}{0.000000in}}%
\pgfpathlineto{\pgfqpoint{0.000000in}{-0.048611in}}%
\pgfusepath{stroke,fill}%
}%
\begin{pgfscope}%
\pgfsys@transformshift{4.496414in}{0.521603in}%
\pgfsys@useobject{currentmarker}{}%
\end{pgfscope}%
\end{pgfscope}%
\begin{pgfscope}%
\definecolor{textcolor}{rgb}{0.000000,0.000000,0.000000}%
\pgfsetstrokecolor{textcolor}%
\pgfsetfillcolor{textcolor}%
\pgftext[x=4.496414in,y=0.424381in,,top]{\color{textcolor}{\rmfamily\fontsize{10.000000}{12.000000}\selectfont\catcode`\^=\active\def^{\ifmmode\sp\else\^{}\fi}\catcode`\%=\active\def%{\%}$\mathdefault{18}$}}%
\end{pgfscope}%
\begin{pgfscope}%
\pgfsetbuttcap%
\pgfsetroundjoin%
\definecolor{currentfill}{rgb}{0.000000,0.000000,0.000000}%
\pgfsetfillcolor{currentfill}%
\pgfsetlinewidth{0.803000pt}%
\definecolor{currentstroke}{rgb}{0.000000,0.000000,0.000000}%
\pgfsetstrokecolor{currentstroke}%
\pgfsetdash{}{0pt}%
\pgfsys@defobject{currentmarker}{\pgfqpoint{0.000000in}{-0.048611in}}{\pgfqpoint{0.000000in}{0.000000in}}{%
\pgfpathmoveto{\pgfqpoint{0.000000in}{0.000000in}}%
\pgfpathlineto{\pgfqpoint{0.000000in}{-0.048611in}}%
\pgfusepath{stroke,fill}%
}%
\begin{pgfscope}%
\pgfsys@transformshift{5.410572in}{0.521603in}%
\pgfsys@useobject{currentmarker}{}%
\end{pgfscope}%
\end{pgfscope}%
\begin{pgfscope}%
\definecolor{textcolor}{rgb}{0.000000,0.000000,0.000000}%
\pgfsetstrokecolor{textcolor}%
\pgfsetfillcolor{textcolor}%
\pgftext[x=5.410572in,y=0.424381in,,top]{\color{textcolor}{\rmfamily\fontsize{10.000000}{12.000000}\selectfont\catcode`\^=\active\def^{\ifmmode\sp\else\^{}\fi}\catcode`\%=\active\def%{\%}$\mathdefault{20}$}}%
\end{pgfscope}%
\begin{pgfscope}%
\definecolor{textcolor}{rgb}{0.000000,0.000000,0.000000}%
\pgfsetstrokecolor{textcolor}%
\pgfsetfillcolor{textcolor}%
\pgftext[x=3.353716in,y=0.234413in,,top]{\color{textcolor}{\rmfamily\fontsize{10.000000}{12.000000}\selectfont\catcode`\^=\active\def^{\ifmmode\sp\else\^{}\fi}\catcode`\%=\active\def%{\%}Monochromatic classes}}%
\end{pgfscope}%
\begin{pgfscope}%
\pgfsetbuttcap%
\pgfsetroundjoin%
\definecolor{currentfill}{rgb}{0.000000,0.000000,0.000000}%
\pgfsetfillcolor{currentfill}%
\pgfsetlinewidth{0.803000pt}%
\definecolor{currentstroke}{rgb}{0.000000,0.000000,0.000000}%
\pgfsetstrokecolor{currentstroke}%
\pgfsetdash{}{0pt}%
\pgfsys@defobject{currentmarker}{\pgfqpoint{-0.048611in}{0.000000in}}{\pgfqpoint{-0.000000in}{0.000000in}}{%
\pgfpathmoveto{\pgfqpoint{-0.000000in}{0.000000in}}%
\pgfpathlineto{\pgfqpoint{-0.048611in}{0.000000in}}%
\pgfusepath{stroke,fill}%
}%
\begin{pgfscope}%
\pgfsys@transformshift{0.588387in}{0.923970in}%
\pgfsys@useobject{currentmarker}{}%
\end{pgfscope}%
\end{pgfscope}%
\begin{pgfscope}%
\definecolor{textcolor}{rgb}{0.000000,0.000000,0.000000}%
\pgfsetstrokecolor{textcolor}%
\pgfsetfillcolor{textcolor}%
\pgftext[x=0.289968in, y=0.871208in, left, base]{\color{textcolor}{\rmfamily\fontsize{10.000000}{12.000000}\selectfont\catcode`\^=\active\def^{\ifmmode\sp\else\^{}\fi}\catcode`\%=\active\def%{\%}$\mathdefault{10^{3}}$}}%
\end{pgfscope}%
\begin{pgfscope}%
\pgfsetbuttcap%
\pgfsetroundjoin%
\definecolor{currentfill}{rgb}{0.000000,0.000000,0.000000}%
\pgfsetfillcolor{currentfill}%
\pgfsetlinewidth{0.803000pt}%
\definecolor{currentstroke}{rgb}{0.000000,0.000000,0.000000}%
\pgfsetstrokecolor{currentstroke}%
\pgfsetdash{}{0pt}%
\pgfsys@defobject{currentmarker}{\pgfqpoint{-0.048611in}{0.000000in}}{\pgfqpoint{-0.000000in}{0.000000in}}{%
\pgfpathmoveto{\pgfqpoint{-0.000000in}{0.000000in}}%
\pgfpathlineto{\pgfqpoint{-0.048611in}{0.000000in}}%
\pgfusepath{stroke,fill}%
}%
\begin{pgfscope}%
\pgfsys@transformshift{0.588387in}{1.672138in}%
\pgfsys@useobject{currentmarker}{}%
\end{pgfscope}%
\end{pgfscope}%
\begin{pgfscope}%
\definecolor{textcolor}{rgb}{0.000000,0.000000,0.000000}%
\pgfsetstrokecolor{textcolor}%
\pgfsetfillcolor{textcolor}%
\pgftext[x=0.289968in, y=1.619376in, left, base]{\color{textcolor}{\rmfamily\fontsize{10.000000}{12.000000}\selectfont\catcode`\^=\active\def^{\ifmmode\sp\else\^{}\fi}\catcode`\%=\active\def%{\%}$\mathdefault{10^{4}}$}}%
\end{pgfscope}%
\begin{pgfscope}%
\pgfsetbuttcap%
\pgfsetroundjoin%
\definecolor{currentfill}{rgb}{0.000000,0.000000,0.000000}%
\pgfsetfillcolor{currentfill}%
\pgfsetlinewidth{0.803000pt}%
\definecolor{currentstroke}{rgb}{0.000000,0.000000,0.000000}%
\pgfsetstrokecolor{currentstroke}%
\pgfsetdash{}{0pt}%
\pgfsys@defobject{currentmarker}{\pgfqpoint{-0.048611in}{0.000000in}}{\pgfqpoint{-0.000000in}{0.000000in}}{%
\pgfpathmoveto{\pgfqpoint{-0.000000in}{0.000000in}}%
\pgfpathlineto{\pgfqpoint{-0.048611in}{0.000000in}}%
\pgfusepath{stroke,fill}%
}%
\begin{pgfscope}%
\pgfsys@transformshift{0.588387in}{2.420306in}%
\pgfsys@useobject{currentmarker}{}%
\end{pgfscope}%
\end{pgfscope}%
\begin{pgfscope}%
\definecolor{textcolor}{rgb}{0.000000,0.000000,0.000000}%
\pgfsetstrokecolor{textcolor}%
\pgfsetfillcolor{textcolor}%
\pgftext[x=0.289968in, y=2.367544in, left, base]{\color{textcolor}{\rmfamily\fontsize{10.000000}{12.000000}\selectfont\catcode`\^=\active\def^{\ifmmode\sp\else\^{}\fi}\catcode`\%=\active\def%{\%}$\mathdefault{10^{5}}$}}%
\end{pgfscope}%
\begin{pgfscope}%
\pgfsetbuttcap%
\pgfsetroundjoin%
\definecolor{currentfill}{rgb}{0.000000,0.000000,0.000000}%
\pgfsetfillcolor{currentfill}%
\pgfsetlinewidth{0.602250pt}%
\definecolor{currentstroke}{rgb}{0.000000,0.000000,0.000000}%
\pgfsetstrokecolor{currentstroke}%
\pgfsetdash{}{0pt}%
\pgfsys@defobject{currentmarker}{\pgfqpoint{-0.027778in}{0.000000in}}{\pgfqpoint{-0.000000in}{0.000000in}}{%
\pgfpathmoveto{\pgfqpoint{-0.000000in}{0.000000in}}%
\pgfpathlineto{\pgfqpoint{-0.027778in}{0.000000in}}%
\pgfusepath{stroke,fill}%
}%
\begin{pgfscope}%
\pgfsys@transformshift{0.588387in}{0.532769in}%
\pgfsys@useobject{currentmarker}{}%
\end{pgfscope}%
\end{pgfscope}%
\begin{pgfscope}%
\pgfsetbuttcap%
\pgfsetroundjoin%
\definecolor{currentfill}{rgb}{0.000000,0.000000,0.000000}%
\pgfsetfillcolor{currentfill}%
\pgfsetlinewidth{0.602250pt}%
\definecolor{currentstroke}{rgb}{0.000000,0.000000,0.000000}%
\pgfsetstrokecolor{currentstroke}%
\pgfsetdash{}{0pt}%
\pgfsys@defobject{currentmarker}{\pgfqpoint{-0.027778in}{0.000000in}}{\pgfqpoint{-0.000000in}{0.000000in}}{%
\pgfpathmoveto{\pgfqpoint{-0.000000in}{0.000000in}}%
\pgfpathlineto{\pgfqpoint{-0.027778in}{0.000000in}}%
\pgfusepath{stroke,fill}%
}%
\begin{pgfscope}%
\pgfsys@transformshift{0.588387in}{0.626244in}%
\pgfsys@useobject{currentmarker}{}%
\end{pgfscope}%
\end{pgfscope}%
\begin{pgfscope}%
\pgfsetbuttcap%
\pgfsetroundjoin%
\definecolor{currentfill}{rgb}{0.000000,0.000000,0.000000}%
\pgfsetfillcolor{currentfill}%
\pgfsetlinewidth{0.602250pt}%
\definecolor{currentstroke}{rgb}{0.000000,0.000000,0.000000}%
\pgfsetstrokecolor{currentstroke}%
\pgfsetdash{}{0pt}%
\pgfsys@defobject{currentmarker}{\pgfqpoint{-0.027778in}{0.000000in}}{\pgfqpoint{-0.000000in}{0.000000in}}{%
\pgfpathmoveto{\pgfqpoint{-0.000000in}{0.000000in}}%
\pgfpathlineto{\pgfqpoint{-0.027778in}{0.000000in}}%
\pgfusepath{stroke,fill}%
}%
\begin{pgfscope}%
\pgfsys@transformshift{0.588387in}{0.698749in}%
\pgfsys@useobject{currentmarker}{}%
\end{pgfscope}%
\end{pgfscope}%
\begin{pgfscope}%
\pgfsetbuttcap%
\pgfsetroundjoin%
\definecolor{currentfill}{rgb}{0.000000,0.000000,0.000000}%
\pgfsetfillcolor{currentfill}%
\pgfsetlinewidth{0.602250pt}%
\definecolor{currentstroke}{rgb}{0.000000,0.000000,0.000000}%
\pgfsetstrokecolor{currentstroke}%
\pgfsetdash{}{0pt}%
\pgfsys@defobject{currentmarker}{\pgfqpoint{-0.027778in}{0.000000in}}{\pgfqpoint{-0.000000in}{0.000000in}}{%
\pgfpathmoveto{\pgfqpoint{-0.000000in}{0.000000in}}%
\pgfpathlineto{\pgfqpoint{-0.027778in}{0.000000in}}%
\pgfusepath{stroke,fill}%
}%
\begin{pgfscope}%
\pgfsys@transformshift{0.588387in}{0.757990in}%
\pgfsys@useobject{currentmarker}{}%
\end{pgfscope}%
\end{pgfscope}%
\begin{pgfscope}%
\pgfsetbuttcap%
\pgfsetroundjoin%
\definecolor{currentfill}{rgb}{0.000000,0.000000,0.000000}%
\pgfsetfillcolor{currentfill}%
\pgfsetlinewidth{0.602250pt}%
\definecolor{currentstroke}{rgb}{0.000000,0.000000,0.000000}%
\pgfsetstrokecolor{currentstroke}%
\pgfsetdash{}{0pt}%
\pgfsys@defobject{currentmarker}{\pgfqpoint{-0.027778in}{0.000000in}}{\pgfqpoint{-0.000000in}{0.000000in}}{%
\pgfpathmoveto{\pgfqpoint{-0.000000in}{0.000000in}}%
\pgfpathlineto{\pgfqpoint{-0.027778in}{0.000000in}}%
\pgfusepath{stroke,fill}%
}%
\begin{pgfscope}%
\pgfsys@transformshift{0.588387in}{0.808077in}%
\pgfsys@useobject{currentmarker}{}%
\end{pgfscope}%
\end{pgfscope}%
\begin{pgfscope}%
\pgfsetbuttcap%
\pgfsetroundjoin%
\definecolor{currentfill}{rgb}{0.000000,0.000000,0.000000}%
\pgfsetfillcolor{currentfill}%
\pgfsetlinewidth{0.602250pt}%
\definecolor{currentstroke}{rgb}{0.000000,0.000000,0.000000}%
\pgfsetstrokecolor{currentstroke}%
\pgfsetdash{}{0pt}%
\pgfsys@defobject{currentmarker}{\pgfqpoint{-0.027778in}{0.000000in}}{\pgfqpoint{-0.000000in}{0.000000in}}{%
\pgfpathmoveto{\pgfqpoint{-0.000000in}{0.000000in}}%
\pgfpathlineto{\pgfqpoint{-0.027778in}{0.000000in}}%
\pgfusepath{stroke,fill}%
}%
\begin{pgfscope}%
\pgfsys@transformshift{0.588387in}{0.851465in}%
\pgfsys@useobject{currentmarker}{}%
\end{pgfscope}%
\end{pgfscope}%
\begin{pgfscope}%
\pgfsetbuttcap%
\pgfsetroundjoin%
\definecolor{currentfill}{rgb}{0.000000,0.000000,0.000000}%
\pgfsetfillcolor{currentfill}%
\pgfsetlinewidth{0.602250pt}%
\definecolor{currentstroke}{rgb}{0.000000,0.000000,0.000000}%
\pgfsetstrokecolor{currentstroke}%
\pgfsetdash{}{0pt}%
\pgfsys@defobject{currentmarker}{\pgfqpoint{-0.027778in}{0.000000in}}{\pgfqpoint{-0.000000in}{0.000000in}}{%
\pgfpathmoveto{\pgfqpoint{-0.000000in}{0.000000in}}%
\pgfpathlineto{\pgfqpoint{-0.027778in}{0.000000in}}%
\pgfusepath{stroke,fill}%
}%
\begin{pgfscope}%
\pgfsys@transformshift{0.588387in}{0.889736in}%
\pgfsys@useobject{currentmarker}{}%
\end{pgfscope}%
\end{pgfscope}%
\begin{pgfscope}%
\pgfsetbuttcap%
\pgfsetroundjoin%
\definecolor{currentfill}{rgb}{0.000000,0.000000,0.000000}%
\pgfsetfillcolor{currentfill}%
\pgfsetlinewidth{0.602250pt}%
\definecolor{currentstroke}{rgb}{0.000000,0.000000,0.000000}%
\pgfsetstrokecolor{currentstroke}%
\pgfsetdash{}{0pt}%
\pgfsys@defobject{currentmarker}{\pgfqpoint{-0.027778in}{0.000000in}}{\pgfqpoint{-0.000000in}{0.000000in}}{%
\pgfpathmoveto{\pgfqpoint{-0.000000in}{0.000000in}}%
\pgfpathlineto{\pgfqpoint{-0.027778in}{0.000000in}}%
\pgfusepath{stroke,fill}%
}%
\begin{pgfscope}%
\pgfsys@transformshift{0.588387in}{1.149191in}%
\pgfsys@useobject{currentmarker}{}%
\end{pgfscope}%
\end{pgfscope}%
\begin{pgfscope}%
\pgfsetbuttcap%
\pgfsetroundjoin%
\definecolor{currentfill}{rgb}{0.000000,0.000000,0.000000}%
\pgfsetfillcolor{currentfill}%
\pgfsetlinewidth{0.602250pt}%
\definecolor{currentstroke}{rgb}{0.000000,0.000000,0.000000}%
\pgfsetstrokecolor{currentstroke}%
\pgfsetdash{}{0pt}%
\pgfsys@defobject{currentmarker}{\pgfqpoint{-0.027778in}{0.000000in}}{\pgfqpoint{-0.000000in}{0.000000in}}{%
\pgfpathmoveto{\pgfqpoint{-0.000000in}{0.000000in}}%
\pgfpathlineto{\pgfqpoint{-0.027778in}{0.000000in}}%
\pgfusepath{stroke,fill}%
}%
\begin{pgfscope}%
\pgfsys@transformshift{0.588387in}{1.280937in}%
\pgfsys@useobject{currentmarker}{}%
\end{pgfscope}%
\end{pgfscope}%
\begin{pgfscope}%
\pgfsetbuttcap%
\pgfsetroundjoin%
\definecolor{currentfill}{rgb}{0.000000,0.000000,0.000000}%
\pgfsetfillcolor{currentfill}%
\pgfsetlinewidth{0.602250pt}%
\definecolor{currentstroke}{rgb}{0.000000,0.000000,0.000000}%
\pgfsetstrokecolor{currentstroke}%
\pgfsetdash{}{0pt}%
\pgfsys@defobject{currentmarker}{\pgfqpoint{-0.027778in}{0.000000in}}{\pgfqpoint{-0.000000in}{0.000000in}}{%
\pgfpathmoveto{\pgfqpoint{-0.000000in}{0.000000in}}%
\pgfpathlineto{\pgfqpoint{-0.027778in}{0.000000in}}%
\pgfusepath{stroke,fill}%
}%
\begin{pgfscope}%
\pgfsys@transformshift{0.588387in}{1.374412in}%
\pgfsys@useobject{currentmarker}{}%
\end{pgfscope}%
\end{pgfscope}%
\begin{pgfscope}%
\pgfsetbuttcap%
\pgfsetroundjoin%
\definecolor{currentfill}{rgb}{0.000000,0.000000,0.000000}%
\pgfsetfillcolor{currentfill}%
\pgfsetlinewidth{0.602250pt}%
\definecolor{currentstroke}{rgb}{0.000000,0.000000,0.000000}%
\pgfsetstrokecolor{currentstroke}%
\pgfsetdash{}{0pt}%
\pgfsys@defobject{currentmarker}{\pgfqpoint{-0.027778in}{0.000000in}}{\pgfqpoint{-0.000000in}{0.000000in}}{%
\pgfpathmoveto{\pgfqpoint{-0.000000in}{0.000000in}}%
\pgfpathlineto{\pgfqpoint{-0.027778in}{0.000000in}}%
\pgfusepath{stroke,fill}%
}%
\begin{pgfscope}%
\pgfsys@transformshift{0.588387in}{1.446917in}%
\pgfsys@useobject{currentmarker}{}%
\end{pgfscope}%
\end{pgfscope}%
\begin{pgfscope}%
\pgfsetbuttcap%
\pgfsetroundjoin%
\definecolor{currentfill}{rgb}{0.000000,0.000000,0.000000}%
\pgfsetfillcolor{currentfill}%
\pgfsetlinewidth{0.602250pt}%
\definecolor{currentstroke}{rgb}{0.000000,0.000000,0.000000}%
\pgfsetstrokecolor{currentstroke}%
\pgfsetdash{}{0pt}%
\pgfsys@defobject{currentmarker}{\pgfqpoint{-0.027778in}{0.000000in}}{\pgfqpoint{-0.000000in}{0.000000in}}{%
\pgfpathmoveto{\pgfqpoint{-0.000000in}{0.000000in}}%
\pgfpathlineto{\pgfqpoint{-0.027778in}{0.000000in}}%
\pgfusepath{stroke,fill}%
}%
\begin{pgfscope}%
\pgfsys@transformshift{0.588387in}{1.506158in}%
\pgfsys@useobject{currentmarker}{}%
\end{pgfscope}%
\end{pgfscope}%
\begin{pgfscope}%
\pgfsetbuttcap%
\pgfsetroundjoin%
\definecolor{currentfill}{rgb}{0.000000,0.000000,0.000000}%
\pgfsetfillcolor{currentfill}%
\pgfsetlinewidth{0.602250pt}%
\definecolor{currentstroke}{rgb}{0.000000,0.000000,0.000000}%
\pgfsetstrokecolor{currentstroke}%
\pgfsetdash{}{0pt}%
\pgfsys@defobject{currentmarker}{\pgfqpoint{-0.027778in}{0.000000in}}{\pgfqpoint{-0.000000in}{0.000000in}}{%
\pgfpathmoveto{\pgfqpoint{-0.000000in}{0.000000in}}%
\pgfpathlineto{\pgfqpoint{-0.027778in}{0.000000in}}%
\pgfusepath{stroke,fill}%
}%
\begin{pgfscope}%
\pgfsys@transformshift{0.588387in}{1.556245in}%
\pgfsys@useobject{currentmarker}{}%
\end{pgfscope}%
\end{pgfscope}%
\begin{pgfscope}%
\pgfsetbuttcap%
\pgfsetroundjoin%
\definecolor{currentfill}{rgb}{0.000000,0.000000,0.000000}%
\pgfsetfillcolor{currentfill}%
\pgfsetlinewidth{0.602250pt}%
\definecolor{currentstroke}{rgb}{0.000000,0.000000,0.000000}%
\pgfsetstrokecolor{currentstroke}%
\pgfsetdash{}{0pt}%
\pgfsys@defobject{currentmarker}{\pgfqpoint{-0.027778in}{0.000000in}}{\pgfqpoint{-0.000000in}{0.000000in}}{%
\pgfpathmoveto{\pgfqpoint{-0.000000in}{0.000000in}}%
\pgfpathlineto{\pgfqpoint{-0.027778in}{0.000000in}}%
\pgfusepath{stroke,fill}%
}%
\begin{pgfscope}%
\pgfsys@transformshift{0.588387in}{1.599633in}%
\pgfsys@useobject{currentmarker}{}%
\end{pgfscope}%
\end{pgfscope}%
\begin{pgfscope}%
\pgfsetbuttcap%
\pgfsetroundjoin%
\definecolor{currentfill}{rgb}{0.000000,0.000000,0.000000}%
\pgfsetfillcolor{currentfill}%
\pgfsetlinewidth{0.602250pt}%
\definecolor{currentstroke}{rgb}{0.000000,0.000000,0.000000}%
\pgfsetstrokecolor{currentstroke}%
\pgfsetdash{}{0pt}%
\pgfsys@defobject{currentmarker}{\pgfqpoint{-0.027778in}{0.000000in}}{\pgfqpoint{-0.000000in}{0.000000in}}{%
\pgfpathmoveto{\pgfqpoint{-0.000000in}{0.000000in}}%
\pgfpathlineto{\pgfqpoint{-0.027778in}{0.000000in}}%
\pgfusepath{stroke,fill}%
}%
\begin{pgfscope}%
\pgfsys@transformshift{0.588387in}{1.637904in}%
\pgfsys@useobject{currentmarker}{}%
\end{pgfscope}%
\end{pgfscope}%
\begin{pgfscope}%
\pgfsetbuttcap%
\pgfsetroundjoin%
\definecolor{currentfill}{rgb}{0.000000,0.000000,0.000000}%
\pgfsetfillcolor{currentfill}%
\pgfsetlinewidth{0.602250pt}%
\definecolor{currentstroke}{rgb}{0.000000,0.000000,0.000000}%
\pgfsetstrokecolor{currentstroke}%
\pgfsetdash{}{0pt}%
\pgfsys@defobject{currentmarker}{\pgfqpoint{-0.027778in}{0.000000in}}{\pgfqpoint{-0.000000in}{0.000000in}}{%
\pgfpathmoveto{\pgfqpoint{-0.000000in}{0.000000in}}%
\pgfpathlineto{\pgfqpoint{-0.027778in}{0.000000in}}%
\pgfusepath{stroke,fill}%
}%
\begin{pgfscope}%
\pgfsys@transformshift{0.588387in}{1.897359in}%
\pgfsys@useobject{currentmarker}{}%
\end{pgfscope}%
\end{pgfscope}%
\begin{pgfscope}%
\pgfsetbuttcap%
\pgfsetroundjoin%
\definecolor{currentfill}{rgb}{0.000000,0.000000,0.000000}%
\pgfsetfillcolor{currentfill}%
\pgfsetlinewidth{0.602250pt}%
\definecolor{currentstroke}{rgb}{0.000000,0.000000,0.000000}%
\pgfsetstrokecolor{currentstroke}%
\pgfsetdash{}{0pt}%
\pgfsys@defobject{currentmarker}{\pgfqpoint{-0.027778in}{0.000000in}}{\pgfqpoint{-0.000000in}{0.000000in}}{%
\pgfpathmoveto{\pgfqpoint{-0.000000in}{0.000000in}}%
\pgfpathlineto{\pgfqpoint{-0.027778in}{0.000000in}}%
\pgfusepath{stroke,fill}%
}%
\begin{pgfscope}%
\pgfsys@transformshift{0.588387in}{2.029105in}%
\pgfsys@useobject{currentmarker}{}%
\end{pgfscope}%
\end{pgfscope}%
\begin{pgfscope}%
\pgfsetbuttcap%
\pgfsetroundjoin%
\definecolor{currentfill}{rgb}{0.000000,0.000000,0.000000}%
\pgfsetfillcolor{currentfill}%
\pgfsetlinewidth{0.602250pt}%
\definecolor{currentstroke}{rgb}{0.000000,0.000000,0.000000}%
\pgfsetstrokecolor{currentstroke}%
\pgfsetdash{}{0pt}%
\pgfsys@defobject{currentmarker}{\pgfqpoint{-0.027778in}{0.000000in}}{\pgfqpoint{-0.000000in}{0.000000in}}{%
\pgfpathmoveto{\pgfqpoint{-0.000000in}{0.000000in}}%
\pgfpathlineto{\pgfqpoint{-0.027778in}{0.000000in}}%
\pgfusepath{stroke,fill}%
}%
\begin{pgfscope}%
\pgfsys@transformshift{0.588387in}{2.122580in}%
\pgfsys@useobject{currentmarker}{}%
\end{pgfscope}%
\end{pgfscope}%
\begin{pgfscope}%
\pgfsetbuttcap%
\pgfsetroundjoin%
\definecolor{currentfill}{rgb}{0.000000,0.000000,0.000000}%
\pgfsetfillcolor{currentfill}%
\pgfsetlinewidth{0.602250pt}%
\definecolor{currentstroke}{rgb}{0.000000,0.000000,0.000000}%
\pgfsetstrokecolor{currentstroke}%
\pgfsetdash{}{0pt}%
\pgfsys@defobject{currentmarker}{\pgfqpoint{-0.027778in}{0.000000in}}{\pgfqpoint{-0.000000in}{0.000000in}}{%
\pgfpathmoveto{\pgfqpoint{-0.000000in}{0.000000in}}%
\pgfpathlineto{\pgfqpoint{-0.027778in}{0.000000in}}%
\pgfusepath{stroke,fill}%
}%
\begin{pgfscope}%
\pgfsys@transformshift{0.588387in}{2.195085in}%
\pgfsys@useobject{currentmarker}{}%
\end{pgfscope}%
\end{pgfscope}%
\begin{pgfscope}%
\pgfsetbuttcap%
\pgfsetroundjoin%
\definecolor{currentfill}{rgb}{0.000000,0.000000,0.000000}%
\pgfsetfillcolor{currentfill}%
\pgfsetlinewidth{0.602250pt}%
\definecolor{currentstroke}{rgb}{0.000000,0.000000,0.000000}%
\pgfsetstrokecolor{currentstroke}%
\pgfsetdash{}{0pt}%
\pgfsys@defobject{currentmarker}{\pgfqpoint{-0.027778in}{0.000000in}}{\pgfqpoint{-0.000000in}{0.000000in}}{%
\pgfpathmoveto{\pgfqpoint{-0.000000in}{0.000000in}}%
\pgfpathlineto{\pgfqpoint{-0.027778in}{0.000000in}}%
\pgfusepath{stroke,fill}%
}%
\begin{pgfscope}%
\pgfsys@transformshift{0.588387in}{2.254326in}%
\pgfsys@useobject{currentmarker}{}%
\end{pgfscope}%
\end{pgfscope}%
\begin{pgfscope}%
\pgfsetbuttcap%
\pgfsetroundjoin%
\definecolor{currentfill}{rgb}{0.000000,0.000000,0.000000}%
\pgfsetfillcolor{currentfill}%
\pgfsetlinewidth{0.602250pt}%
\definecolor{currentstroke}{rgb}{0.000000,0.000000,0.000000}%
\pgfsetstrokecolor{currentstroke}%
\pgfsetdash{}{0pt}%
\pgfsys@defobject{currentmarker}{\pgfqpoint{-0.027778in}{0.000000in}}{\pgfqpoint{-0.000000in}{0.000000in}}{%
\pgfpathmoveto{\pgfqpoint{-0.000000in}{0.000000in}}%
\pgfpathlineto{\pgfqpoint{-0.027778in}{0.000000in}}%
\pgfusepath{stroke,fill}%
}%
\begin{pgfscope}%
\pgfsys@transformshift{0.588387in}{2.304413in}%
\pgfsys@useobject{currentmarker}{}%
\end{pgfscope}%
\end{pgfscope}%
\begin{pgfscope}%
\pgfsetbuttcap%
\pgfsetroundjoin%
\definecolor{currentfill}{rgb}{0.000000,0.000000,0.000000}%
\pgfsetfillcolor{currentfill}%
\pgfsetlinewidth{0.602250pt}%
\definecolor{currentstroke}{rgb}{0.000000,0.000000,0.000000}%
\pgfsetstrokecolor{currentstroke}%
\pgfsetdash{}{0pt}%
\pgfsys@defobject{currentmarker}{\pgfqpoint{-0.027778in}{0.000000in}}{\pgfqpoint{-0.000000in}{0.000000in}}{%
\pgfpathmoveto{\pgfqpoint{-0.000000in}{0.000000in}}%
\pgfpathlineto{\pgfqpoint{-0.027778in}{0.000000in}}%
\pgfusepath{stroke,fill}%
}%
\begin{pgfscope}%
\pgfsys@transformshift{0.588387in}{2.347801in}%
\pgfsys@useobject{currentmarker}{}%
\end{pgfscope}%
\end{pgfscope}%
\begin{pgfscope}%
\pgfsetbuttcap%
\pgfsetroundjoin%
\definecolor{currentfill}{rgb}{0.000000,0.000000,0.000000}%
\pgfsetfillcolor{currentfill}%
\pgfsetlinewidth{0.602250pt}%
\definecolor{currentstroke}{rgb}{0.000000,0.000000,0.000000}%
\pgfsetstrokecolor{currentstroke}%
\pgfsetdash{}{0pt}%
\pgfsys@defobject{currentmarker}{\pgfqpoint{-0.027778in}{0.000000in}}{\pgfqpoint{-0.000000in}{0.000000in}}{%
\pgfpathmoveto{\pgfqpoint{-0.000000in}{0.000000in}}%
\pgfpathlineto{\pgfqpoint{-0.027778in}{0.000000in}}%
\pgfusepath{stroke,fill}%
}%
\begin{pgfscope}%
\pgfsys@transformshift{0.588387in}{2.386072in}%
\pgfsys@useobject{currentmarker}{}%
\end{pgfscope}%
\end{pgfscope}%
\begin{pgfscope}%
\definecolor{textcolor}{rgb}{0.000000,0.000000,0.000000}%
\pgfsetstrokecolor{textcolor}%
\pgfsetfillcolor{textcolor}%
\pgftext[x=0.234413in,y=1.526746in,,bottom,rotate=90.000000]{\color{textcolor}{\rmfamily\fontsize{10.000000}{12.000000}\selectfont\catcode`\^=\active\def^{\ifmmode\sp\else\^{}\fi}\catcode`\%=\active\def%{\%}Checks [call]}}%
\end{pgfscope}%
\begin{pgfscope}%
\pgfpathrectangle{\pgfqpoint{0.588387in}{0.521603in}}{\pgfqpoint{5.530657in}{2.010285in}}%
\pgfusepath{clip}%
\pgfsetrectcap%
\pgfsetroundjoin%
\pgfsetlinewidth{1.505625pt}%
\pgfsetstrokecolor{currentstroke1}%
\pgfsetdash{}{0pt}%
\pgfpathmoveto{\pgfqpoint{0.839781in}{0.674469in}}%
\pgfpathlineto{\pgfqpoint{1.753939in}{1.129086in}}%
\pgfpathlineto{\pgfqpoint{3.125176in}{1.508747in}}%
\pgfpathlineto{\pgfqpoint{3.582255in}{1.656803in}}%
\pgfpathlineto{\pgfqpoint{4.496414in}{1.974861in}}%
\pgfpathlineto{\pgfqpoint{5.867651in}{2.440512in}}%
\pgfusepath{stroke}%
\end{pgfscope}%
\begin{pgfscope}%
\pgfpathrectangle{\pgfqpoint{0.588387in}{0.521603in}}{\pgfqpoint{5.530657in}{2.010285in}}%
\pgfusepath{clip}%
\pgfsetrectcap%
\pgfsetroundjoin%
\pgfsetlinewidth{1.505625pt}%
\pgfsetstrokecolor{currentstroke2}%
\pgfsetdash{}{0pt}%
\pgfpathmoveto{\pgfqpoint{0.839781in}{0.674469in}}%
\pgfpathlineto{\pgfqpoint{1.753939in}{1.068460in}}%
\pgfpathlineto{\pgfqpoint{3.125176in}{1.508747in}}%
\pgfpathlineto{\pgfqpoint{3.582255in}{1.643357in}}%
\pgfpathlineto{\pgfqpoint{4.496414in}{1.951752in}}%
\pgfpathlineto{\pgfqpoint{5.867651in}{2.324954in}}%
\pgfusepath{stroke}%
\end{pgfscope}%
\begin{pgfscope}%
\pgfpathrectangle{\pgfqpoint{0.588387in}{0.521603in}}{\pgfqpoint{5.530657in}{2.010285in}}%
\pgfusepath{clip}%
\pgfsetrectcap%
\pgfsetroundjoin%
\pgfsetlinewidth{1.505625pt}%
\pgfsetstrokecolor{currentstroke3}%
\pgfsetdash{}{0pt}%
\pgfpathmoveto{\pgfqpoint{0.839781in}{0.612980in}}%
\pgfpathlineto{\pgfqpoint{1.753939in}{1.068460in}}%
\pgfpathlineto{\pgfqpoint{3.125176in}{1.611751in}}%
\pgfpathlineto{\pgfqpoint{3.582255in}{1.705705in}}%
\pgfpathlineto{\pgfqpoint{4.496414in}{2.041062in}}%
\pgfpathlineto{\pgfqpoint{5.867651in}{2.348515in}}%
\pgfusepath{stroke}%
\end{pgfscope}%
\begin{pgfscope}%
\pgfsetrectcap%
\pgfsetmiterjoin%
\pgfsetlinewidth{0.803000pt}%
\definecolor{currentstroke}{rgb}{0.000000,0.000000,0.000000}%
\pgfsetstrokecolor{currentstroke}%
\pgfsetdash{}{0pt}%
\pgfpathmoveto{\pgfqpoint{0.588387in}{0.521603in}}%
\pgfpathlineto{\pgfqpoint{0.588387in}{2.531888in}}%
\pgfusepath{stroke}%
\end{pgfscope}%
\begin{pgfscope}%
\pgfsetrectcap%
\pgfsetmiterjoin%
\pgfsetlinewidth{0.803000pt}%
\definecolor{currentstroke}{rgb}{0.000000,0.000000,0.000000}%
\pgfsetstrokecolor{currentstroke}%
\pgfsetdash{}{0pt}%
\pgfpathmoveto{\pgfqpoint{6.119045in}{0.521603in}}%
\pgfpathlineto{\pgfqpoint{6.119045in}{2.531888in}}%
\pgfusepath{stroke}%
\end{pgfscope}%
\begin{pgfscope}%
\pgfsetrectcap%
\pgfsetmiterjoin%
\pgfsetlinewidth{0.803000pt}%
\definecolor{currentstroke}{rgb}{0.000000,0.000000,0.000000}%
\pgfsetstrokecolor{currentstroke}%
\pgfsetdash{}{0pt}%
\pgfpathmoveto{\pgfqpoint{0.588387in}{0.521603in}}%
\pgfpathlineto{\pgfqpoint{6.119045in}{0.521603in}}%
\pgfusepath{stroke}%
\end{pgfscope}%
\begin{pgfscope}%
\pgfsetrectcap%
\pgfsetmiterjoin%
\pgfsetlinewidth{0.803000pt}%
\definecolor{currentstroke}{rgb}{0.000000,0.000000,0.000000}%
\pgfsetstrokecolor{currentstroke}%
\pgfsetdash{}{0pt}%
\pgfpathmoveto{\pgfqpoint{0.588387in}{2.531888in}}%
\pgfpathlineto{\pgfqpoint{6.119045in}{2.531888in}}%
\pgfusepath{stroke}%
\end{pgfscope}%
\begin{pgfscope}%
\definecolor{textcolor}{rgb}{0.000000,0.000000,0.000000}%
\pgfsetstrokecolor{textcolor}%
\pgfsetfillcolor{textcolor}%
\pgftext[x=3.353716in,y=2.615222in,,base]{\color{textcolor}{\rmfamily\fontsize{12.000000}{14.400000}\selectfont\catcode`\^=\active\def^{\ifmmode\sp\else\^{}\fi}\catcode`\%=\active\def%{\%}Mean}}%
\end{pgfscope}%
\begin{pgfscope}%
\pgfsetbuttcap%
\pgfsetmiterjoin%
\definecolor{currentfill}{rgb}{1.000000,1.000000,1.000000}%
\pgfsetfillcolor{currentfill}%
\pgfsetfillopacity{0.800000}%
\pgfsetlinewidth{1.003750pt}%
\definecolor{currentstroke}{rgb}{0.800000,0.800000,0.800000}%
\pgfsetstrokecolor{currentstroke}%
\pgfsetstrokeopacity{0.800000}%
\pgfsetdash{}{0pt}%
\pgfpathmoveto{\pgfqpoint{6.206545in}{1.877995in}}%
\pgfpathlineto{\pgfqpoint{8.259376in}{1.877995in}}%
\pgfpathquadraticcurveto{\pgfqpoint{8.284376in}{1.877995in}}{\pgfqpoint{8.284376in}{1.902995in}}%
\pgfpathlineto{\pgfqpoint{8.284376in}{2.444388in}}%
\pgfpathquadraticcurveto{\pgfqpoint{8.284376in}{2.469388in}}{\pgfqpoint{8.259376in}{2.469388in}}%
\pgfpathlineto{\pgfqpoint{6.206545in}{2.469388in}}%
\pgfpathquadraticcurveto{\pgfqpoint{6.181545in}{2.469388in}}{\pgfqpoint{6.181545in}{2.444388in}}%
\pgfpathlineto{\pgfqpoint{6.181545in}{1.902995in}}%
\pgfpathquadraticcurveto{\pgfqpoint{6.181545in}{1.877995in}}{\pgfqpoint{6.206545in}{1.877995in}}%
\pgfpathlineto{\pgfqpoint{6.206545in}{1.877995in}}%
\pgfpathclose%
\pgfusepath{stroke,fill}%
\end{pgfscope}%
\begin{pgfscope}%
\pgfsetrectcap%
\pgfsetroundjoin%
\pgfsetlinewidth{1.505625pt}%
\pgfsetstrokecolor{currentstroke1}%
\pgfsetdash{}{0pt}%
\pgfpathmoveto{\pgfqpoint{6.231545in}{2.368168in}}%
\pgfpathlineto{\pgfqpoint{6.356545in}{2.368168in}}%
\pgfpathlineto{\pgfqpoint{6.481545in}{2.368168in}}%
\pgfusepath{stroke}%
\end{pgfscope}%
\begin{pgfscope}%
\definecolor{textcolor}{rgb}{0.000000,0.000000,0.000000}%
\pgfsetstrokecolor{textcolor}%
\pgfsetfillcolor{textcolor}%
\pgftext[x=6.581545in,y=2.324418in,left,base]{\color{textcolor}{\rmfamily\fontsize{9.000000}{10.800000}\selectfont\catcode`\^=\active\def^{\ifmmode\sp\else\^{}\fi}\catcode`\%=\active\def%{\%}\Neighbors{} \& \MergeLinear{}}}%
\end{pgfscope}%
\begin{pgfscope}%
\pgfsetrectcap%
\pgfsetroundjoin%
\pgfsetlinewidth{1.505625pt}%
\pgfsetstrokecolor{currentstroke2}%
\pgfsetdash{}{0pt}%
\pgfpathmoveto{\pgfqpoint{6.231545in}{2.184696in}}%
\pgfpathlineto{\pgfqpoint{6.356545in}{2.184696in}}%
\pgfpathlineto{\pgfqpoint{6.481545in}{2.184696in}}%
\pgfusepath{stroke}%
\end{pgfscope}%
\begin{pgfscope}%
\definecolor{textcolor}{rgb}{0.000000,0.000000,0.000000}%
\pgfsetstrokecolor{textcolor}%
\pgfsetfillcolor{textcolor}%
\pgftext[x=6.581545in,y=2.140946in,left,base]{\color{textcolor}{\rmfamily\fontsize{9.000000}{10.800000}\selectfont\catcode`\^=\active\def^{\ifmmode\sp\else\^{}\fi}\catcode`\%=\active\def%{\%}\NeighborsDegree{} \& \MergeLinear{}}}%
\end{pgfscope}%
\begin{pgfscope}%
\pgfsetrectcap%
\pgfsetroundjoin%
\pgfsetlinewidth{1.505625pt}%
\pgfsetstrokecolor{currentstroke3}%
\pgfsetdash{}{0pt}%
\pgfpathmoveto{\pgfqpoint{6.231545in}{1.997746in}}%
\pgfpathlineto{\pgfqpoint{6.356545in}{1.997746in}}%
\pgfpathlineto{\pgfqpoint{6.481545in}{1.997746in}}%
\pgfusepath{stroke}%
\end{pgfscope}%
\begin{pgfscope}%
\definecolor{textcolor}{rgb}{0.000000,0.000000,0.000000}%
\pgfsetstrokecolor{textcolor}%
\pgfsetfillcolor{textcolor}%
\pgftext[x=6.581545in,y=1.953996in,left,base]{\color{textcolor}{\rmfamily\fontsize{9.000000}{10.800000}\selectfont\catcode`\^=\active\def^{\ifmmode\sp\else\^{}\fi}\catcode`\%=\active\def%{\%}\None{} \& \MergeLinear{}}}%
\end{pgfscope}%
\end{pgfpicture}%
\makeatother%
\endgroup%
}
% 	\caption[Checks performed for graphs with no 3 nor 4 cycles (all)]{
% 		The number of checks performed to find all NAC-colorings for graphs with no three nor four cycles.}%
% 	\label{fig:graph_count_no_3_nor_4_cycles_all_checks}
% \end{figure}%


\subsubsection*{Globally rigid graphs}

A random graph can be generated taking a graph with no edges on \( n \) vertices
and for each pair of vertices adding an edge with probability~\( p \).
This model is often referred to as Erdős-Rényi~\cite{random_gnp}.
%
John Haslegrave provided us with yet unpublished formula for \( p \)
such that a random graph generated with \( p \)
should probably have no or just a few NAC-colorings.
%
We instead used \( p' = 0.95\cdot p \) to generate random graphs that
should have small number of NAC-colorings.
For such random graphs, we checked if they are globally rigid using PyRigi~\cite{pyrigi}.
%
By doing this, we also randomly generated a dataset of globally rigid graphs
up to 57 vertices.
%
Note that there also exist globally rigid graphs
that are just one edge different from minimally rigid graphs.
These graphs are probably not included in our dataset.

The idea of monochromatic classes is so effective
that even large graphs collapse into just a few monochromatic classes.
Majority of the graphs in this dataset either has a NAC-coloring,
or only a single monochromatic class and therefore no NAC-coloring.
In our dataset, 75 percent of graphs have less than ten	monochromatic classes,
but only 10 percent of graphs have less than ten \trcon{} components
as you can see in \Cref{fig:monochrom_vs_triangle_globally_rigid}.
For minimally rigid graphs, this is not the case as expected, see
\Cref{fig:monochrom_vs_triangle_minimally_rigid}.
%
\begin{figure}[h!]
	\centering
	\begin{subfigure}{0.48\textwidth}
		\centering
		\scalebox{0.6}{%% Creator: Matplotlib, PGF backend
%%
%% To include the figure in your LaTeX document, write
%%   \input{<filename>.pgf}
%%
%% Make sure the required packages are loaded in your preamble
%%   \usepackage{pgf}
%%
%% Also ensure that all the required font packages are loaded; for instance,
%% the lmodern package is sometimes necessary when using math font.
%%   \usepackage{lmodern}
%%
%% Figures using additional raster images can only be included by \input if
%% they are in the same directory as the main LaTeX file. For loading figures
%% from other directories you can use the `import` package
%%   \usepackage{import}
%%
%% and then include the figures with
%%   \import{<path to file>}{<filename>.pgf}
%%
%% Matplotlib used the following preamble
%%   \def\mathdefault#1{#1}
%%   \everymath=\expandafter{\the\everymath\displaystyle}
%%   \IfFileExists{scrextend.sty}{
%%     \usepackage[fontsize=10.000000pt]{scrextend}
%%   }{
%%     \renewcommand{\normalsize}{\fontsize{10.000000}{12.000000}\selectfont}
%%     \normalsize
%%   }
%%   
%%   \ifdefined\pdftexversion\else  % non-pdftex case.
%%     \usepackage{fontspec}
%%     \setmainfont{DejaVuSans.ttf}[Path=\detokenize{/home/petr/Projects/PyRigi/.venv/lib/python3.12/site-packages/matplotlib/mpl-data/fonts/ttf/}]
%%     \setsansfont{DejaVuSans.ttf}[Path=\detokenize{/home/petr/Projects/PyRigi/.venv/lib/python3.12/site-packages/matplotlib/mpl-data/fonts/ttf/}]
%%     \setmonofont{DejaVuSansMono.ttf}[Path=\detokenize{/home/petr/Projects/PyRigi/.venv/lib/python3.12/site-packages/matplotlib/mpl-data/fonts/ttf/}]
%%   \fi
%%   \makeatletter\@ifpackageloaded{under\Score{}}{}{\usepackage[strings]{under\Score{}}}\makeatother
%%
\begingroup%
\makeatletter%
\begin{pgfpicture}%
\pgfpathrectangle{\pgfpointorigin}{\pgfqpoint{3.965986in}{2.317798in}}%
\pgfusepath{use as bounding box, clip}%
\begin{pgfscope}%
\pgfsetbuttcap%
\pgfsetmiterjoin%
\definecolor{currentfill}{rgb}{1.000000,1.000000,1.000000}%
\pgfsetfillcolor{currentfill}%
\pgfsetlinewidth{0.000000pt}%
\definecolor{currentstroke}{rgb}{1.000000,1.000000,1.000000}%
\pgfsetstrokecolor{currentstroke}%
\pgfsetdash{}{0pt}%
\pgfpathmoveto{\pgfqpoint{0.000000in}{0.000000in}}%
\pgfpathlineto{\pgfqpoint{3.965986in}{0.000000in}}%
\pgfpathlineto{\pgfqpoint{3.965986in}{2.317798in}}%
\pgfpathlineto{\pgfqpoint{0.000000in}{2.317798in}}%
\pgfpathlineto{\pgfqpoint{0.000000in}{0.000000in}}%
\pgfpathclose%
\pgfusepath{fill}%
\end{pgfscope}%
\begin{pgfscope}%
\pgfsetbuttcap%
\pgfsetmiterjoin%
\definecolor{currentfill}{rgb}{1.000000,1.000000,1.000000}%
\pgfsetfillcolor{currentfill}%
\pgfsetlinewidth{0.000000pt}%
\definecolor{currentstroke}{rgb}{0.000000,0.000000,0.000000}%
\pgfsetstrokecolor{currentstroke}%
\pgfsetstrokeopacity{0.000000}%
\pgfsetdash{}{0pt}%
\pgfpathmoveto{\pgfqpoint{0.664969in}{0.521603in}}%
\pgfpathlineto{\pgfqpoint{3.865986in}{0.521603in}}%
\pgfpathlineto{\pgfqpoint{3.865986in}{2.217798in}}%
\pgfpathlineto{\pgfqpoint{0.664969in}{2.217798in}}%
\pgfpathlineto{\pgfqpoint{0.664969in}{0.521603in}}%
\pgfpathclose%
\pgfusepath{fill}%
\end{pgfscope}%
\begin{pgfscope}%
\pgfpathrectangle{\pgfqpoint{0.664969in}{0.521603in}}{\pgfqpoint{3.201017in}{1.696195in}}%
\pgfusepath{clip}%
\pgfsetbuttcap%
\pgfsetmiterjoin%
\definecolor{currentfill}{rgb}{0.121569,0.466667,0.705882}%
\pgfsetfillcolor{currentfill}%
\pgfsetfillopacity{0.600000}%
\pgfsetlinewidth{0.000000pt}%
\definecolor{currentstroke}{rgb}{0.000000,0.000000,0.000000}%
\pgfsetstrokecolor{currentstroke}%
\pgfsetstrokeopacity{0.600000}%
\pgfsetdash{}{0pt}%
\pgfpathmoveto{\pgfqpoint{0.810470in}{0.521603in}}%
\pgfpathlineto{\pgfqpoint{0.870498in}{0.521603in}}%
\pgfpathlineto{\pgfqpoint{0.870498in}{2.137027in}}%
\pgfpathlineto{\pgfqpoint{0.810470in}{2.137027in}}%
\pgfpathlineto{\pgfqpoint{0.810470in}{0.521603in}}%
\pgfpathclose%
\pgfusepath{fill}%
\end{pgfscope}%
\begin{pgfscope}%
\pgfpathrectangle{\pgfqpoint{0.664969in}{0.521603in}}{\pgfqpoint{3.201017in}{1.696195in}}%
\pgfusepath{clip}%
\pgfsetbuttcap%
\pgfsetmiterjoin%
\definecolor{currentfill}{rgb}{0.121569,0.466667,0.705882}%
\pgfsetfillcolor{currentfill}%
\pgfsetfillopacity{0.600000}%
\pgfsetlinewidth{0.000000pt}%
\definecolor{currentstroke}{rgb}{0.000000,0.000000,0.000000}%
\pgfsetstrokecolor{currentstroke}%
\pgfsetstrokeopacity{0.600000}%
\pgfsetdash{}{0pt}%
\pgfpathmoveto{\pgfqpoint{0.870498in}{0.521603in}}%
\pgfpathlineto{\pgfqpoint{0.930526in}{0.521603in}}%
\pgfpathlineto{\pgfqpoint{0.930526in}{1.285691in}}%
\pgfpathlineto{\pgfqpoint{0.870498in}{1.285691in}}%
\pgfpathlineto{\pgfqpoint{0.870498in}{0.521603in}}%
\pgfpathclose%
\pgfusepath{fill}%
\end{pgfscope}%
\begin{pgfscope}%
\pgfpathrectangle{\pgfqpoint{0.664969in}{0.521603in}}{\pgfqpoint{3.201017in}{1.696195in}}%
\pgfusepath{clip}%
\pgfsetbuttcap%
\pgfsetmiterjoin%
\definecolor{currentfill}{rgb}{0.121569,0.466667,0.705882}%
\pgfsetfillcolor{currentfill}%
\pgfsetfillopacity{0.600000}%
\pgfsetlinewidth{0.000000pt}%
\definecolor{currentstroke}{rgb}{0.000000,0.000000,0.000000}%
\pgfsetstrokecolor{currentstroke}%
\pgfsetstrokeopacity{0.600000}%
\pgfsetdash{}{0pt}%
\pgfpathmoveto{\pgfqpoint{0.930526in}{0.521603in}}%
\pgfpathlineto{\pgfqpoint{0.990554in}{0.521603in}}%
\pgfpathlineto{\pgfqpoint{0.990554in}{1.072416in}}%
\pgfpathlineto{\pgfqpoint{0.930526in}{1.072416in}}%
\pgfpathlineto{\pgfqpoint{0.930526in}{0.521603in}}%
\pgfpathclose%
\pgfusepath{fill}%
\end{pgfscope}%
\begin{pgfscope}%
\pgfpathrectangle{\pgfqpoint{0.664969in}{0.521603in}}{\pgfqpoint{3.201017in}{1.696195in}}%
\pgfusepath{clip}%
\pgfsetbuttcap%
\pgfsetmiterjoin%
\definecolor{currentfill}{rgb}{0.121569,0.466667,0.705882}%
\pgfsetfillcolor{currentfill}%
\pgfsetfillopacity{0.600000}%
\pgfsetlinewidth{0.000000pt}%
\definecolor{currentstroke}{rgb}{0.000000,0.000000,0.000000}%
\pgfsetstrokecolor{currentstroke}%
\pgfsetstrokeopacity{0.600000}%
\pgfsetdash{}{0pt}%
\pgfpathmoveto{\pgfqpoint{0.990554in}{0.521603in}}%
\pgfpathlineto{\pgfqpoint{1.050582in}{0.521603in}}%
\pgfpathlineto{\pgfqpoint{1.050582in}{1.018657in}}%
\pgfpathlineto{\pgfqpoint{0.990554in}{1.018657in}}%
\pgfpathlineto{\pgfqpoint{0.990554in}{0.521603in}}%
\pgfpathclose%
\pgfusepath{fill}%
\end{pgfscope}%
\begin{pgfscope}%
\pgfpathrectangle{\pgfqpoint{0.664969in}{0.521603in}}{\pgfqpoint{3.201017in}{1.696195in}}%
\pgfusepath{clip}%
\pgfsetbuttcap%
\pgfsetmiterjoin%
\definecolor{currentfill}{rgb}{0.121569,0.466667,0.705882}%
\pgfsetfillcolor{currentfill}%
\pgfsetfillopacity{0.600000}%
\pgfsetlinewidth{0.000000pt}%
\definecolor{currentstroke}{rgb}{0.000000,0.000000,0.000000}%
\pgfsetstrokecolor{currentstroke}%
\pgfsetstrokeopacity{0.600000}%
\pgfsetdash{}{0pt}%
\pgfpathmoveto{\pgfqpoint{1.050582in}{0.521603in}}%
\pgfpathlineto{\pgfqpoint{1.110610in}{0.521603in}}%
\pgfpathlineto{\pgfqpoint{1.110610in}{0.750741in}}%
\pgfpathlineto{\pgfqpoint{1.050582in}{0.750741in}}%
\pgfpathlineto{\pgfqpoint{1.050582in}{0.521603in}}%
\pgfpathclose%
\pgfusepath{fill}%
\end{pgfscope}%
\begin{pgfscope}%
\pgfpathrectangle{\pgfqpoint{0.664969in}{0.521603in}}{\pgfqpoint{3.201017in}{1.696195in}}%
\pgfusepath{clip}%
\pgfsetbuttcap%
\pgfsetmiterjoin%
\definecolor{currentfill}{rgb}{0.121569,0.466667,0.705882}%
\pgfsetfillcolor{currentfill}%
\pgfsetfillopacity{0.600000}%
\pgfsetlinewidth{0.000000pt}%
\definecolor{currentstroke}{rgb}{0.000000,0.000000,0.000000}%
\pgfsetstrokecolor{currentstroke}%
\pgfsetstrokeopacity{0.600000}%
\pgfsetdash{}{0pt}%
\pgfpathmoveto{\pgfqpoint{1.110610in}{0.521603in}}%
\pgfpathlineto{\pgfqpoint{1.170638in}{0.521603in}}%
\pgfpathlineto{\pgfqpoint{1.170638in}{0.692576in}}%
\pgfpathlineto{\pgfqpoint{1.110610in}{0.692576in}}%
\pgfpathlineto{\pgfqpoint{1.110610in}{0.521603in}}%
\pgfpathclose%
\pgfusepath{fill}%
\end{pgfscope}%
\begin{pgfscope}%
\pgfpathrectangle{\pgfqpoint{0.664969in}{0.521603in}}{\pgfqpoint{3.201017in}{1.696195in}}%
\pgfusepath{clip}%
\pgfsetbuttcap%
\pgfsetmiterjoin%
\definecolor{currentfill}{rgb}{0.121569,0.466667,0.705882}%
\pgfsetfillcolor{currentfill}%
\pgfsetfillopacity{0.600000}%
\pgfsetlinewidth{0.000000pt}%
\definecolor{currentstroke}{rgb}{0.000000,0.000000,0.000000}%
\pgfsetstrokecolor{currentstroke}%
\pgfsetstrokeopacity{0.600000}%
\pgfsetdash{}{0pt}%
\pgfpathmoveto{\pgfqpoint{1.170638in}{0.521603in}}%
\pgfpathlineto{\pgfqpoint{1.230666in}{0.521603in}}%
\pgfpathlineto{\pgfqpoint{1.230666in}{0.661730in}}%
\pgfpathlineto{\pgfqpoint{1.170638in}{0.661730in}}%
\pgfpathlineto{\pgfqpoint{1.170638in}{0.521603in}}%
\pgfpathclose%
\pgfusepath{fill}%
\end{pgfscope}%
\begin{pgfscope}%
\pgfpathrectangle{\pgfqpoint{0.664969in}{0.521603in}}{\pgfqpoint{3.201017in}{1.696195in}}%
\pgfusepath{clip}%
\pgfsetbuttcap%
\pgfsetmiterjoin%
\definecolor{currentfill}{rgb}{0.121569,0.466667,0.705882}%
\pgfsetfillcolor{currentfill}%
\pgfsetfillopacity{0.600000}%
\pgfsetlinewidth{0.000000pt}%
\definecolor{currentstroke}{rgb}{0.000000,0.000000,0.000000}%
\pgfsetstrokecolor{currentstroke}%
\pgfsetstrokeopacity{0.600000}%
\pgfsetdash{}{0pt}%
\pgfpathmoveto{\pgfqpoint{1.230666in}{0.521603in}}%
\pgfpathlineto{\pgfqpoint{1.290694in}{0.521603in}}%
\pgfpathlineto{\pgfqpoint{1.290694in}{0.578888in}}%
\pgfpathlineto{\pgfqpoint{1.230666in}{0.578888in}}%
\pgfpathlineto{\pgfqpoint{1.230666in}{0.521603in}}%
\pgfpathclose%
\pgfusepath{fill}%
\end{pgfscope}%
\begin{pgfscope}%
\pgfpathrectangle{\pgfqpoint{0.664969in}{0.521603in}}{\pgfqpoint{3.201017in}{1.696195in}}%
\pgfusepath{clip}%
\pgfsetbuttcap%
\pgfsetmiterjoin%
\definecolor{currentfill}{rgb}{0.121569,0.466667,0.705882}%
\pgfsetfillcolor{currentfill}%
\pgfsetfillopacity{0.600000}%
\pgfsetlinewidth{0.000000pt}%
\definecolor{currentstroke}{rgb}{0.000000,0.000000,0.000000}%
\pgfsetstrokecolor{currentstroke}%
\pgfsetstrokeopacity{0.600000}%
\pgfsetdash{}{0pt}%
\pgfpathmoveto{\pgfqpoint{1.290694in}{0.521603in}}%
\pgfpathlineto{\pgfqpoint{1.350722in}{0.521603in}}%
\pgfpathlineto{\pgfqpoint{1.350722in}{0.567431in}}%
\pgfpathlineto{\pgfqpoint{1.290694in}{0.567431in}}%
\pgfpathlineto{\pgfqpoint{1.290694in}{0.521603in}}%
\pgfpathclose%
\pgfusepath{fill}%
\end{pgfscope}%
\begin{pgfscope}%
\pgfpathrectangle{\pgfqpoint{0.664969in}{0.521603in}}{\pgfqpoint{3.201017in}{1.696195in}}%
\pgfusepath{clip}%
\pgfsetbuttcap%
\pgfsetmiterjoin%
\definecolor{currentfill}{rgb}{0.121569,0.466667,0.705882}%
\pgfsetfillcolor{currentfill}%
\pgfsetfillopacity{0.600000}%
\pgfsetlinewidth{0.000000pt}%
\definecolor{currentstroke}{rgb}{0.000000,0.000000,0.000000}%
\pgfsetstrokecolor{currentstroke}%
\pgfsetstrokeopacity{0.600000}%
\pgfsetdash{}{0pt}%
\pgfpathmoveto{\pgfqpoint{1.350722in}{0.521603in}}%
\pgfpathlineto{\pgfqpoint{1.410750in}{0.521603in}}%
\pgfpathlineto{\pgfqpoint{1.410750in}{0.548924in}}%
\pgfpathlineto{\pgfqpoint{1.350722in}{0.548924in}}%
\pgfpathlineto{\pgfqpoint{1.350722in}{0.521603in}}%
\pgfpathclose%
\pgfusepath{fill}%
\end{pgfscope}%
\begin{pgfscope}%
\pgfpathrectangle{\pgfqpoint{0.664969in}{0.521603in}}{\pgfqpoint{3.201017in}{1.696195in}}%
\pgfusepath{clip}%
\pgfsetbuttcap%
\pgfsetmiterjoin%
\definecolor{currentfill}{rgb}{0.121569,0.466667,0.705882}%
\pgfsetfillcolor{currentfill}%
\pgfsetfillopacity{0.600000}%
\pgfsetlinewidth{0.000000pt}%
\definecolor{currentstroke}{rgb}{0.000000,0.000000,0.000000}%
\pgfsetstrokecolor{currentstroke}%
\pgfsetstrokeopacity{0.600000}%
\pgfsetdash{}{0pt}%
\pgfpathmoveto{\pgfqpoint{1.410750in}{0.521603in}}%
\pgfpathlineto{\pgfqpoint{1.470778in}{0.521603in}}%
\pgfpathlineto{\pgfqpoint{1.470778in}{0.535704in}}%
\pgfpathlineto{\pgfqpoint{1.410750in}{0.535704in}}%
\pgfpathlineto{\pgfqpoint{1.410750in}{0.521603in}}%
\pgfpathclose%
\pgfusepath{fill}%
\end{pgfscope}%
\begin{pgfscope}%
\pgfpathrectangle{\pgfqpoint{0.664969in}{0.521603in}}{\pgfqpoint{3.201017in}{1.696195in}}%
\pgfusepath{clip}%
\pgfsetbuttcap%
\pgfsetmiterjoin%
\definecolor{currentfill}{rgb}{0.121569,0.466667,0.705882}%
\pgfsetfillcolor{currentfill}%
\pgfsetfillopacity{0.600000}%
\pgfsetlinewidth{0.000000pt}%
\definecolor{currentstroke}{rgb}{0.000000,0.000000,0.000000}%
\pgfsetstrokecolor{currentstroke}%
\pgfsetstrokeopacity{0.600000}%
\pgfsetdash{}{0pt}%
\pgfpathmoveto{\pgfqpoint{1.470778in}{0.521603in}}%
\pgfpathlineto{\pgfqpoint{1.530806in}{0.521603in}}%
\pgfpathlineto{\pgfqpoint{1.530806in}{0.535704in}}%
\pgfpathlineto{\pgfqpoint{1.470778in}{0.535704in}}%
\pgfpathlineto{\pgfqpoint{1.470778in}{0.521603in}}%
\pgfpathclose%
\pgfusepath{fill}%
\end{pgfscope}%
\begin{pgfscope}%
\pgfpathrectangle{\pgfqpoint{0.664969in}{0.521603in}}{\pgfqpoint{3.201017in}{1.696195in}}%
\pgfusepath{clip}%
\pgfsetbuttcap%
\pgfsetmiterjoin%
\definecolor{currentfill}{rgb}{0.121569,0.466667,0.705882}%
\pgfsetfillcolor{currentfill}%
\pgfsetfillopacity{0.600000}%
\pgfsetlinewidth{0.000000pt}%
\definecolor{currentstroke}{rgb}{0.000000,0.000000,0.000000}%
\pgfsetstrokecolor{currentstroke}%
\pgfsetstrokeopacity{0.600000}%
\pgfsetdash{}{0pt}%
\pgfpathmoveto{\pgfqpoint{1.530806in}{0.521603in}}%
\pgfpathlineto{\pgfqpoint{1.590834in}{0.521603in}}%
\pgfpathlineto{\pgfqpoint{1.590834in}{0.541873in}}%
\pgfpathlineto{\pgfqpoint{1.530806in}{0.541873in}}%
\pgfpathlineto{\pgfqpoint{1.530806in}{0.521603in}}%
\pgfpathclose%
\pgfusepath{fill}%
\end{pgfscope}%
\begin{pgfscope}%
\pgfpathrectangle{\pgfqpoint{0.664969in}{0.521603in}}{\pgfqpoint{3.201017in}{1.696195in}}%
\pgfusepath{clip}%
\pgfsetbuttcap%
\pgfsetmiterjoin%
\definecolor{currentfill}{rgb}{0.121569,0.466667,0.705882}%
\pgfsetfillcolor{currentfill}%
\pgfsetfillopacity{0.600000}%
\pgfsetlinewidth{0.000000pt}%
\definecolor{currentstroke}{rgb}{0.000000,0.000000,0.000000}%
\pgfsetstrokecolor{currentstroke}%
\pgfsetstrokeopacity{0.600000}%
\pgfsetdash{}{0pt}%
\pgfpathmoveto{\pgfqpoint{1.590834in}{0.521603in}}%
\pgfpathlineto{\pgfqpoint{1.650862in}{0.521603in}}%
\pgfpathlineto{\pgfqpoint{1.650862in}{0.525129in}}%
\pgfpathlineto{\pgfqpoint{1.590834in}{0.525129in}}%
\pgfpathlineto{\pgfqpoint{1.590834in}{0.521603in}}%
\pgfpathclose%
\pgfusepath{fill}%
\end{pgfscope}%
\begin{pgfscope}%
\pgfpathrectangle{\pgfqpoint{0.664969in}{0.521603in}}{\pgfqpoint{3.201017in}{1.696195in}}%
\pgfusepath{clip}%
\pgfsetbuttcap%
\pgfsetmiterjoin%
\definecolor{currentfill}{rgb}{0.121569,0.466667,0.705882}%
\pgfsetfillcolor{currentfill}%
\pgfsetfillopacity{0.600000}%
\pgfsetlinewidth{0.000000pt}%
\definecolor{currentstroke}{rgb}{0.000000,0.000000,0.000000}%
\pgfsetstrokecolor{currentstroke}%
\pgfsetstrokeopacity{0.600000}%
\pgfsetdash{}{0pt}%
\pgfpathmoveto{\pgfqpoint{1.650862in}{0.521603in}}%
\pgfpathlineto{\pgfqpoint{1.710890in}{0.521603in}}%
\pgfpathlineto{\pgfqpoint{1.710890in}{0.527772in}}%
\pgfpathlineto{\pgfqpoint{1.650862in}{0.527772in}}%
\pgfpathlineto{\pgfqpoint{1.650862in}{0.521603in}}%
\pgfpathclose%
\pgfusepath{fill}%
\end{pgfscope}%
\begin{pgfscope}%
\pgfpathrectangle{\pgfqpoint{0.664969in}{0.521603in}}{\pgfqpoint{3.201017in}{1.696195in}}%
\pgfusepath{clip}%
\pgfsetbuttcap%
\pgfsetmiterjoin%
\definecolor{currentfill}{rgb}{0.121569,0.466667,0.705882}%
\pgfsetfillcolor{currentfill}%
\pgfsetfillopacity{0.600000}%
\pgfsetlinewidth{0.000000pt}%
\definecolor{currentstroke}{rgb}{0.000000,0.000000,0.000000}%
\pgfsetstrokecolor{currentstroke}%
\pgfsetstrokeopacity{0.600000}%
\pgfsetdash{}{0pt}%
\pgfpathmoveto{\pgfqpoint{1.710890in}{0.521603in}}%
\pgfpathlineto{\pgfqpoint{1.770918in}{0.521603in}}%
\pgfpathlineto{\pgfqpoint{1.770918in}{0.526891in}}%
\pgfpathlineto{\pgfqpoint{1.710890in}{0.526891in}}%
\pgfpathlineto{\pgfqpoint{1.710890in}{0.521603in}}%
\pgfpathclose%
\pgfusepath{fill}%
\end{pgfscope}%
\begin{pgfscope}%
\pgfpathrectangle{\pgfqpoint{0.664969in}{0.521603in}}{\pgfqpoint{3.201017in}{1.696195in}}%
\pgfusepath{clip}%
\pgfsetbuttcap%
\pgfsetmiterjoin%
\definecolor{currentfill}{rgb}{0.121569,0.466667,0.705882}%
\pgfsetfillcolor{currentfill}%
\pgfsetfillopacity{0.600000}%
\pgfsetlinewidth{0.000000pt}%
\definecolor{currentstroke}{rgb}{0.000000,0.000000,0.000000}%
\pgfsetstrokecolor{currentstroke}%
\pgfsetstrokeopacity{0.600000}%
\pgfsetdash{}{0pt}%
\pgfpathmoveto{\pgfqpoint{1.770918in}{0.521603in}}%
\pgfpathlineto{\pgfqpoint{1.830946in}{0.521603in}}%
\pgfpathlineto{\pgfqpoint{1.830946in}{0.526010in}}%
\pgfpathlineto{\pgfqpoint{1.770918in}{0.526010in}}%
\pgfpathlineto{\pgfqpoint{1.770918in}{0.521603in}}%
\pgfpathclose%
\pgfusepath{fill}%
\end{pgfscope}%
\begin{pgfscope}%
\pgfpathrectangle{\pgfqpoint{0.664969in}{0.521603in}}{\pgfqpoint{3.201017in}{1.696195in}}%
\pgfusepath{clip}%
\pgfsetbuttcap%
\pgfsetmiterjoin%
\definecolor{currentfill}{rgb}{0.121569,0.466667,0.705882}%
\pgfsetfillcolor{currentfill}%
\pgfsetfillopacity{0.600000}%
\pgfsetlinewidth{0.000000pt}%
\definecolor{currentstroke}{rgb}{0.000000,0.000000,0.000000}%
\pgfsetstrokecolor{currentstroke}%
\pgfsetstrokeopacity{0.600000}%
\pgfsetdash{}{0pt}%
\pgfpathmoveto{\pgfqpoint{1.830946in}{0.521603in}}%
\pgfpathlineto{\pgfqpoint{1.890974in}{0.521603in}}%
\pgfpathlineto{\pgfqpoint{1.890974in}{0.522485in}}%
\pgfpathlineto{\pgfqpoint{1.830946in}{0.522485in}}%
\pgfpathlineto{\pgfqpoint{1.830946in}{0.521603in}}%
\pgfpathclose%
\pgfusepath{fill}%
\end{pgfscope}%
\begin{pgfscope}%
\pgfpathrectangle{\pgfqpoint{0.664969in}{0.521603in}}{\pgfqpoint{3.201017in}{1.696195in}}%
\pgfusepath{clip}%
\pgfsetbuttcap%
\pgfsetmiterjoin%
\definecolor{currentfill}{rgb}{0.121569,0.466667,0.705882}%
\pgfsetfillcolor{currentfill}%
\pgfsetfillopacity{0.600000}%
\pgfsetlinewidth{0.000000pt}%
\definecolor{currentstroke}{rgb}{0.000000,0.000000,0.000000}%
\pgfsetstrokecolor{currentstroke}%
\pgfsetstrokeopacity{0.600000}%
\pgfsetdash{}{0pt}%
\pgfpathmoveto{\pgfqpoint{1.890974in}{0.521603in}}%
\pgfpathlineto{\pgfqpoint{1.951002in}{0.521603in}}%
\pgfpathlineto{\pgfqpoint{1.951002in}{0.525129in}}%
\pgfpathlineto{\pgfqpoint{1.890974in}{0.525129in}}%
\pgfpathlineto{\pgfqpoint{1.890974in}{0.521603in}}%
\pgfpathclose%
\pgfusepath{fill}%
\end{pgfscope}%
\begin{pgfscope}%
\pgfpathrectangle{\pgfqpoint{0.664969in}{0.521603in}}{\pgfqpoint{3.201017in}{1.696195in}}%
\pgfusepath{clip}%
\pgfsetbuttcap%
\pgfsetmiterjoin%
\definecolor{currentfill}{rgb}{0.121569,0.466667,0.705882}%
\pgfsetfillcolor{currentfill}%
\pgfsetfillopacity{0.600000}%
\pgfsetlinewidth{0.000000pt}%
\definecolor{currentstroke}{rgb}{0.000000,0.000000,0.000000}%
\pgfsetstrokecolor{currentstroke}%
\pgfsetstrokeopacity{0.600000}%
\pgfsetdash{}{0pt}%
\pgfpathmoveto{\pgfqpoint{1.951002in}{0.521603in}}%
\pgfpathlineto{\pgfqpoint{2.011031in}{0.521603in}}%
\pgfpathlineto{\pgfqpoint{2.011031in}{0.524247in}}%
\pgfpathlineto{\pgfqpoint{1.951002in}{0.524247in}}%
\pgfpathlineto{\pgfqpoint{1.951002in}{0.521603in}}%
\pgfpathclose%
\pgfusepath{fill}%
\end{pgfscope}%
\begin{pgfscope}%
\pgfpathrectangle{\pgfqpoint{0.664969in}{0.521603in}}{\pgfqpoint{3.201017in}{1.696195in}}%
\pgfusepath{clip}%
\pgfsetbuttcap%
\pgfsetmiterjoin%
\definecolor{currentfill}{rgb}{0.121569,0.466667,0.705882}%
\pgfsetfillcolor{currentfill}%
\pgfsetfillopacity{0.600000}%
\pgfsetlinewidth{0.000000pt}%
\definecolor{currentstroke}{rgb}{0.000000,0.000000,0.000000}%
\pgfsetstrokecolor{currentstroke}%
\pgfsetstrokeopacity{0.600000}%
\pgfsetdash{}{0pt}%
\pgfpathmoveto{\pgfqpoint{2.011031in}{0.521603in}}%
\pgfpathlineto{\pgfqpoint{2.071059in}{0.521603in}}%
\pgfpathlineto{\pgfqpoint{2.071059in}{0.523366in}}%
\pgfpathlineto{\pgfqpoint{2.011031in}{0.523366in}}%
\pgfpathlineto{\pgfqpoint{2.011031in}{0.521603in}}%
\pgfpathclose%
\pgfusepath{fill}%
\end{pgfscope}%
\begin{pgfscope}%
\pgfpathrectangle{\pgfqpoint{0.664969in}{0.521603in}}{\pgfqpoint{3.201017in}{1.696195in}}%
\pgfusepath{clip}%
\pgfsetbuttcap%
\pgfsetmiterjoin%
\definecolor{currentfill}{rgb}{0.121569,0.466667,0.705882}%
\pgfsetfillcolor{currentfill}%
\pgfsetfillopacity{0.600000}%
\pgfsetlinewidth{0.000000pt}%
\definecolor{currentstroke}{rgb}{0.000000,0.000000,0.000000}%
\pgfsetstrokecolor{currentstroke}%
\pgfsetstrokeopacity{0.600000}%
\pgfsetdash{}{0pt}%
\pgfpathmoveto{\pgfqpoint{2.071059in}{0.521603in}}%
\pgfpathlineto{\pgfqpoint{2.131087in}{0.521603in}}%
\pgfpathlineto{\pgfqpoint{2.131087in}{0.523366in}}%
\pgfpathlineto{\pgfqpoint{2.071059in}{0.523366in}}%
\pgfpathlineto{\pgfqpoint{2.071059in}{0.521603in}}%
\pgfpathclose%
\pgfusepath{fill}%
\end{pgfscope}%
\begin{pgfscope}%
\pgfpathrectangle{\pgfqpoint{0.664969in}{0.521603in}}{\pgfqpoint{3.201017in}{1.696195in}}%
\pgfusepath{clip}%
\pgfsetbuttcap%
\pgfsetmiterjoin%
\definecolor{currentfill}{rgb}{0.121569,0.466667,0.705882}%
\pgfsetfillcolor{currentfill}%
\pgfsetfillopacity{0.600000}%
\pgfsetlinewidth{0.000000pt}%
\definecolor{currentstroke}{rgb}{0.000000,0.000000,0.000000}%
\pgfsetstrokecolor{currentstroke}%
\pgfsetstrokeopacity{0.600000}%
\pgfsetdash{}{0pt}%
\pgfpathmoveto{\pgfqpoint{2.131087in}{0.521603in}}%
\pgfpathlineto{\pgfqpoint{2.191115in}{0.521603in}}%
\pgfpathlineto{\pgfqpoint{2.191115in}{0.522485in}}%
\pgfpathlineto{\pgfqpoint{2.131087in}{0.522485in}}%
\pgfpathlineto{\pgfqpoint{2.131087in}{0.521603in}}%
\pgfpathclose%
\pgfusepath{fill}%
\end{pgfscope}%
\begin{pgfscope}%
\pgfpathrectangle{\pgfqpoint{0.664969in}{0.521603in}}{\pgfqpoint{3.201017in}{1.696195in}}%
\pgfusepath{clip}%
\pgfsetbuttcap%
\pgfsetmiterjoin%
\definecolor{currentfill}{rgb}{0.121569,0.466667,0.705882}%
\pgfsetfillcolor{currentfill}%
\pgfsetfillopacity{0.600000}%
\pgfsetlinewidth{0.000000pt}%
\definecolor{currentstroke}{rgb}{0.000000,0.000000,0.000000}%
\pgfsetstrokecolor{currentstroke}%
\pgfsetstrokeopacity{0.600000}%
\pgfsetdash{}{0pt}%
\pgfpathmoveto{\pgfqpoint{2.191115in}{0.521603in}}%
\pgfpathlineto{\pgfqpoint{2.251143in}{0.521603in}}%
\pgfpathlineto{\pgfqpoint{2.251143in}{0.524247in}}%
\pgfpathlineto{\pgfqpoint{2.191115in}{0.524247in}}%
\pgfpathlineto{\pgfqpoint{2.191115in}{0.521603in}}%
\pgfpathclose%
\pgfusepath{fill}%
\end{pgfscope}%
\begin{pgfscope}%
\pgfpathrectangle{\pgfqpoint{0.664969in}{0.521603in}}{\pgfqpoint{3.201017in}{1.696195in}}%
\pgfusepath{clip}%
\pgfsetbuttcap%
\pgfsetmiterjoin%
\definecolor{currentfill}{rgb}{0.121569,0.466667,0.705882}%
\pgfsetfillcolor{currentfill}%
\pgfsetfillopacity{0.600000}%
\pgfsetlinewidth{0.000000pt}%
\definecolor{currentstroke}{rgb}{0.000000,0.000000,0.000000}%
\pgfsetstrokecolor{currentstroke}%
\pgfsetstrokeopacity{0.600000}%
\pgfsetdash{}{0pt}%
\pgfpathmoveto{\pgfqpoint{2.251143in}{0.521603in}}%
\pgfpathlineto{\pgfqpoint{2.311171in}{0.521603in}}%
\pgfpathlineto{\pgfqpoint{2.311171in}{0.522485in}}%
\pgfpathlineto{\pgfqpoint{2.251143in}{0.522485in}}%
\pgfpathlineto{\pgfqpoint{2.251143in}{0.521603in}}%
\pgfpathclose%
\pgfusepath{fill}%
\end{pgfscope}%
\begin{pgfscope}%
\pgfpathrectangle{\pgfqpoint{0.664969in}{0.521603in}}{\pgfqpoint{3.201017in}{1.696195in}}%
\pgfusepath{clip}%
\pgfsetbuttcap%
\pgfsetmiterjoin%
\definecolor{currentfill}{rgb}{0.121569,0.466667,0.705882}%
\pgfsetfillcolor{currentfill}%
\pgfsetfillopacity{0.600000}%
\pgfsetlinewidth{0.000000pt}%
\definecolor{currentstroke}{rgb}{0.000000,0.000000,0.000000}%
\pgfsetstrokecolor{currentstroke}%
\pgfsetstrokeopacity{0.600000}%
\pgfsetdash{}{0pt}%
\pgfpathmoveto{\pgfqpoint{2.311171in}{0.521603in}}%
\pgfpathlineto{\pgfqpoint{2.371199in}{0.521603in}}%
\pgfpathlineto{\pgfqpoint{2.371199in}{0.521603in}}%
\pgfpathlineto{\pgfqpoint{2.311171in}{0.521603in}}%
\pgfpathlineto{\pgfqpoint{2.311171in}{0.521603in}}%
\pgfpathclose%
\pgfusepath{fill}%
\end{pgfscope}%
\begin{pgfscope}%
\pgfpathrectangle{\pgfqpoint{0.664969in}{0.521603in}}{\pgfqpoint{3.201017in}{1.696195in}}%
\pgfusepath{clip}%
\pgfsetbuttcap%
\pgfsetmiterjoin%
\definecolor{currentfill}{rgb}{0.121569,0.466667,0.705882}%
\pgfsetfillcolor{currentfill}%
\pgfsetfillopacity{0.600000}%
\pgfsetlinewidth{0.000000pt}%
\definecolor{currentstroke}{rgb}{0.000000,0.000000,0.000000}%
\pgfsetstrokecolor{currentstroke}%
\pgfsetstrokeopacity{0.600000}%
\pgfsetdash{}{0pt}%
\pgfpathmoveto{\pgfqpoint{2.371199in}{0.521603in}}%
\pgfpathlineto{\pgfqpoint{2.431227in}{0.521603in}}%
\pgfpathlineto{\pgfqpoint{2.431227in}{0.524247in}}%
\pgfpathlineto{\pgfqpoint{2.371199in}{0.524247in}}%
\pgfpathlineto{\pgfqpoint{2.371199in}{0.521603in}}%
\pgfpathclose%
\pgfusepath{fill}%
\end{pgfscope}%
\begin{pgfscope}%
\pgfpathrectangle{\pgfqpoint{0.664969in}{0.521603in}}{\pgfqpoint{3.201017in}{1.696195in}}%
\pgfusepath{clip}%
\pgfsetbuttcap%
\pgfsetmiterjoin%
\definecolor{currentfill}{rgb}{0.121569,0.466667,0.705882}%
\pgfsetfillcolor{currentfill}%
\pgfsetfillopacity{0.600000}%
\pgfsetlinewidth{0.000000pt}%
\definecolor{currentstroke}{rgb}{0.000000,0.000000,0.000000}%
\pgfsetstrokecolor{currentstroke}%
\pgfsetstrokeopacity{0.600000}%
\pgfsetdash{}{0pt}%
\pgfpathmoveto{\pgfqpoint{2.431227in}{0.521603in}}%
\pgfpathlineto{\pgfqpoint{2.491255in}{0.521603in}}%
\pgfpathlineto{\pgfqpoint{2.491255in}{0.521603in}}%
\pgfpathlineto{\pgfqpoint{2.431227in}{0.521603in}}%
\pgfpathlineto{\pgfqpoint{2.431227in}{0.521603in}}%
\pgfpathclose%
\pgfusepath{fill}%
\end{pgfscope}%
\begin{pgfscope}%
\pgfpathrectangle{\pgfqpoint{0.664969in}{0.521603in}}{\pgfqpoint{3.201017in}{1.696195in}}%
\pgfusepath{clip}%
\pgfsetbuttcap%
\pgfsetmiterjoin%
\definecolor{currentfill}{rgb}{0.121569,0.466667,0.705882}%
\pgfsetfillcolor{currentfill}%
\pgfsetfillopacity{0.600000}%
\pgfsetlinewidth{0.000000pt}%
\definecolor{currentstroke}{rgb}{0.000000,0.000000,0.000000}%
\pgfsetstrokecolor{currentstroke}%
\pgfsetstrokeopacity{0.600000}%
\pgfsetdash{}{0pt}%
\pgfpathmoveto{\pgfqpoint{2.491255in}{0.521603in}}%
\pgfpathlineto{\pgfqpoint{2.551283in}{0.521603in}}%
\pgfpathlineto{\pgfqpoint{2.551283in}{0.522485in}}%
\pgfpathlineto{\pgfqpoint{2.491255in}{0.522485in}}%
\pgfpathlineto{\pgfqpoint{2.491255in}{0.521603in}}%
\pgfpathclose%
\pgfusepath{fill}%
\end{pgfscope}%
\begin{pgfscope}%
\pgfpathrectangle{\pgfqpoint{0.664969in}{0.521603in}}{\pgfqpoint{3.201017in}{1.696195in}}%
\pgfusepath{clip}%
\pgfsetbuttcap%
\pgfsetmiterjoin%
\definecolor{currentfill}{rgb}{1.000000,0.498039,0.054902}%
\pgfsetfillcolor{currentfill}%
\pgfsetfillopacity{0.600000}%
\pgfsetlinewidth{0.000000pt}%
\definecolor{currentstroke}{rgb}{0.000000,0.000000,0.000000}%
\pgfsetstrokecolor{currentstroke}%
\pgfsetstrokeopacity{0.600000}%
\pgfsetdash{}{0pt}%
\pgfpathmoveto{\pgfqpoint{0.810470in}{0.521603in}}%
\pgfpathlineto{\pgfqpoint{0.910815in}{0.521603in}}%
\pgfpathlineto{\pgfqpoint{0.910815in}{0.607089in}}%
\pgfpathlineto{\pgfqpoint{0.810470in}{0.607089in}}%
\pgfpathlineto{\pgfqpoint{0.810470in}{0.521603in}}%
\pgfpathclose%
\pgfusepath{fill}%
\end{pgfscope}%
\begin{pgfscope}%
\pgfpathrectangle{\pgfqpoint{0.664969in}{0.521603in}}{\pgfqpoint{3.201017in}{1.696195in}}%
\pgfusepath{clip}%
\pgfsetbuttcap%
\pgfsetmiterjoin%
\definecolor{currentfill}{rgb}{1.000000,0.498039,0.054902}%
\pgfsetfillcolor{currentfill}%
\pgfsetfillopacity{0.600000}%
\pgfsetlinewidth{0.000000pt}%
\definecolor{currentstroke}{rgb}{0.000000,0.000000,0.000000}%
\pgfsetstrokecolor{currentstroke}%
\pgfsetstrokeopacity{0.600000}%
\pgfsetdash{}{0pt}%
\pgfpathmoveto{\pgfqpoint{0.910815in}{0.521603in}}%
\pgfpathlineto{\pgfqpoint{1.011161in}{0.521603in}}%
\pgfpathlineto{\pgfqpoint{1.011161in}{0.834465in}}%
\pgfpathlineto{\pgfqpoint{0.910815in}{0.834465in}}%
\pgfpathlineto{\pgfqpoint{0.910815in}{0.521603in}}%
\pgfpathclose%
\pgfusepath{fill}%
\end{pgfscope}%
\begin{pgfscope}%
\pgfpathrectangle{\pgfqpoint{0.664969in}{0.521603in}}{\pgfqpoint{3.201017in}{1.696195in}}%
\pgfusepath{clip}%
\pgfsetbuttcap%
\pgfsetmiterjoin%
\definecolor{currentfill}{rgb}{1.000000,0.498039,0.054902}%
\pgfsetfillcolor{currentfill}%
\pgfsetfillopacity{0.600000}%
\pgfsetlinewidth{0.000000pt}%
\definecolor{currentstroke}{rgb}{0.000000,0.000000,0.000000}%
\pgfsetstrokecolor{currentstroke}%
\pgfsetstrokeopacity{0.600000}%
\pgfsetdash{}{0pt}%
\pgfpathmoveto{\pgfqpoint{1.011161in}{0.521603in}}%
\pgfpathlineto{\pgfqpoint{1.111506in}{0.521603in}}%
\pgfpathlineto{\pgfqpoint{1.111506in}{0.808907in}}%
\pgfpathlineto{\pgfqpoint{1.011161in}{0.808907in}}%
\pgfpathlineto{\pgfqpoint{1.011161in}{0.521603in}}%
\pgfpathclose%
\pgfusepath{fill}%
\end{pgfscope}%
\begin{pgfscope}%
\pgfpathrectangle{\pgfqpoint{0.664969in}{0.521603in}}{\pgfqpoint{3.201017in}{1.696195in}}%
\pgfusepath{clip}%
\pgfsetbuttcap%
\pgfsetmiterjoin%
\definecolor{currentfill}{rgb}{1.000000,0.498039,0.054902}%
\pgfsetfillcolor{currentfill}%
\pgfsetfillopacity{0.600000}%
\pgfsetlinewidth{0.000000pt}%
\definecolor{currentstroke}{rgb}{0.000000,0.000000,0.000000}%
\pgfsetstrokecolor{currentstroke}%
\pgfsetstrokeopacity{0.600000}%
\pgfsetdash{}{0pt}%
\pgfpathmoveto{\pgfqpoint{1.111506in}{0.521603in}}%
\pgfpathlineto{\pgfqpoint{1.211851in}{0.521603in}}%
\pgfpathlineto{\pgfqpoint{1.211851in}{0.783350in}}%
\pgfpathlineto{\pgfqpoint{1.111506in}{0.783350in}}%
\pgfpathlineto{\pgfqpoint{1.111506in}{0.521603in}}%
\pgfpathclose%
\pgfusepath{fill}%
\end{pgfscope}%
\begin{pgfscope}%
\pgfpathrectangle{\pgfqpoint{0.664969in}{0.521603in}}{\pgfqpoint{3.201017in}{1.696195in}}%
\pgfusepath{clip}%
\pgfsetbuttcap%
\pgfsetmiterjoin%
\definecolor{currentfill}{rgb}{1.000000,0.498039,0.054902}%
\pgfsetfillcolor{currentfill}%
\pgfsetfillopacity{0.600000}%
\pgfsetlinewidth{0.000000pt}%
\definecolor{currentstroke}{rgb}{0.000000,0.000000,0.000000}%
\pgfsetstrokecolor{currentstroke}%
\pgfsetstrokeopacity{0.600000}%
\pgfsetdash{}{0pt}%
\pgfpathmoveto{\pgfqpoint{1.211851in}{0.521603in}}%
\pgfpathlineto{\pgfqpoint{1.312197in}{0.521603in}}%
\pgfpathlineto{\pgfqpoint{1.312197in}{0.808907in}}%
\pgfpathlineto{\pgfqpoint{1.211851in}{0.808907in}}%
\pgfpathlineto{\pgfqpoint{1.211851in}{0.521603in}}%
\pgfpathclose%
\pgfusepath{fill}%
\end{pgfscope}%
\begin{pgfscope}%
\pgfpathrectangle{\pgfqpoint{0.664969in}{0.521603in}}{\pgfqpoint{3.201017in}{1.696195in}}%
\pgfusepath{clip}%
\pgfsetbuttcap%
\pgfsetmiterjoin%
\definecolor{currentfill}{rgb}{1.000000,0.498039,0.054902}%
\pgfsetfillcolor{currentfill}%
\pgfsetfillopacity{0.600000}%
\pgfsetlinewidth{0.000000pt}%
\definecolor{currentstroke}{rgb}{0.000000,0.000000,0.000000}%
\pgfsetstrokecolor{currentstroke}%
\pgfsetstrokeopacity{0.600000}%
\pgfsetdash{}{0pt}%
\pgfpathmoveto{\pgfqpoint{1.312197in}{0.521603in}}%
\pgfpathlineto{\pgfqpoint{1.412542in}{0.521603in}}%
\pgfpathlineto{\pgfqpoint{1.412542in}{0.770130in}}%
\pgfpathlineto{\pgfqpoint{1.312197in}{0.770130in}}%
\pgfpathlineto{\pgfqpoint{1.312197in}{0.521603in}}%
\pgfpathclose%
\pgfusepath{fill}%
\end{pgfscope}%
\begin{pgfscope}%
\pgfpathrectangle{\pgfqpoint{0.664969in}{0.521603in}}{\pgfqpoint{3.201017in}{1.696195in}}%
\pgfusepath{clip}%
\pgfsetbuttcap%
\pgfsetmiterjoin%
\definecolor{currentfill}{rgb}{1.000000,0.498039,0.054902}%
\pgfsetfillcolor{currentfill}%
\pgfsetfillopacity{0.600000}%
\pgfsetlinewidth{0.000000pt}%
\definecolor{currentstroke}{rgb}{0.000000,0.000000,0.000000}%
\pgfsetstrokecolor{currentstroke}%
\pgfsetstrokeopacity{0.600000}%
\pgfsetdash{}{0pt}%
\pgfpathmoveto{\pgfqpoint{1.412542in}{0.521603in}}%
\pgfpathlineto{\pgfqpoint{1.512887in}{0.521603in}}%
\pgfpathlineto{\pgfqpoint{1.512887in}{0.735759in}}%
\pgfpathlineto{\pgfqpoint{1.412542in}{0.735759in}}%
\pgfpathlineto{\pgfqpoint{1.412542in}{0.521603in}}%
\pgfpathclose%
\pgfusepath{fill}%
\end{pgfscope}%
\begin{pgfscope}%
\pgfpathrectangle{\pgfqpoint{0.664969in}{0.521603in}}{\pgfqpoint{3.201017in}{1.696195in}}%
\pgfusepath{clip}%
\pgfsetbuttcap%
\pgfsetmiterjoin%
\definecolor{currentfill}{rgb}{1.000000,0.498039,0.054902}%
\pgfsetfillcolor{currentfill}%
\pgfsetfillopacity{0.600000}%
\pgfsetlinewidth{0.000000pt}%
\definecolor{currentstroke}{rgb}{0.000000,0.000000,0.000000}%
\pgfsetstrokecolor{currentstroke}%
\pgfsetstrokeopacity{0.600000}%
\pgfsetdash{}{0pt}%
\pgfpathmoveto{\pgfqpoint{1.512887in}{0.521603in}}%
\pgfpathlineto{\pgfqpoint{1.613233in}{0.521603in}}%
\pgfpathlineto{\pgfqpoint{1.613233in}{0.698745in}}%
\pgfpathlineto{\pgfqpoint{1.512887in}{0.698745in}}%
\pgfpathlineto{\pgfqpoint{1.512887in}{0.521603in}}%
\pgfpathclose%
\pgfusepath{fill}%
\end{pgfscope}%
\begin{pgfscope}%
\pgfpathrectangle{\pgfqpoint{0.664969in}{0.521603in}}{\pgfqpoint{3.201017in}{1.696195in}}%
\pgfusepath{clip}%
\pgfsetbuttcap%
\pgfsetmiterjoin%
\definecolor{currentfill}{rgb}{1.000000,0.498039,0.054902}%
\pgfsetfillcolor{currentfill}%
\pgfsetfillopacity{0.600000}%
\pgfsetlinewidth{0.000000pt}%
\definecolor{currentstroke}{rgb}{0.000000,0.000000,0.000000}%
\pgfsetstrokecolor{currentstroke}%
\pgfsetstrokeopacity{0.600000}%
\pgfsetdash{}{0pt}%
\pgfpathmoveto{\pgfqpoint{1.613233in}{0.521603in}}%
\pgfpathlineto{\pgfqpoint{1.713578in}{0.521603in}}%
\pgfpathlineto{\pgfqpoint{1.713578in}{0.725184in}}%
\pgfpathlineto{\pgfqpoint{1.613233in}{0.725184in}}%
\pgfpathlineto{\pgfqpoint{1.613233in}{0.521603in}}%
\pgfpathclose%
\pgfusepath{fill}%
\end{pgfscope}%
\begin{pgfscope}%
\pgfpathrectangle{\pgfqpoint{0.664969in}{0.521603in}}{\pgfqpoint{3.201017in}{1.696195in}}%
\pgfusepath{clip}%
\pgfsetbuttcap%
\pgfsetmiterjoin%
\definecolor{currentfill}{rgb}{1.000000,0.498039,0.054902}%
\pgfsetfillcolor{currentfill}%
\pgfsetfillopacity{0.600000}%
\pgfsetlinewidth{0.000000pt}%
\definecolor{currentstroke}{rgb}{0.000000,0.000000,0.000000}%
\pgfsetstrokecolor{currentstroke}%
\pgfsetstrokeopacity{0.600000}%
\pgfsetdash{}{0pt}%
\pgfpathmoveto{\pgfqpoint{1.713578in}{0.521603in}}%
\pgfpathlineto{\pgfqpoint{1.813924in}{0.521603in}}%
\pgfpathlineto{\pgfqpoint{1.813924in}{0.723421in}}%
\pgfpathlineto{\pgfqpoint{1.713578in}{0.723421in}}%
\pgfpathlineto{\pgfqpoint{1.713578in}{0.521603in}}%
\pgfpathclose%
\pgfusepath{fill}%
\end{pgfscope}%
\begin{pgfscope}%
\pgfpathrectangle{\pgfqpoint{0.664969in}{0.521603in}}{\pgfqpoint{3.201017in}{1.696195in}}%
\pgfusepath{clip}%
\pgfsetbuttcap%
\pgfsetmiterjoin%
\definecolor{currentfill}{rgb}{1.000000,0.498039,0.054902}%
\pgfsetfillcolor{currentfill}%
\pgfsetfillopacity{0.600000}%
\pgfsetlinewidth{0.000000pt}%
\definecolor{currentstroke}{rgb}{0.000000,0.000000,0.000000}%
\pgfsetstrokecolor{currentstroke}%
\pgfsetstrokeopacity{0.600000}%
\pgfsetdash{}{0pt}%
\pgfpathmoveto{\pgfqpoint{1.813924in}{0.521603in}}%
\pgfpathlineto{\pgfqpoint{1.914269in}{0.521603in}}%
\pgfpathlineto{\pgfqpoint{1.914269in}{0.715489in}}%
\pgfpathlineto{\pgfqpoint{1.813924in}{0.715489in}}%
\pgfpathlineto{\pgfqpoint{1.813924in}{0.521603in}}%
\pgfpathclose%
\pgfusepath{fill}%
\end{pgfscope}%
\begin{pgfscope}%
\pgfpathrectangle{\pgfqpoint{0.664969in}{0.521603in}}{\pgfqpoint{3.201017in}{1.696195in}}%
\pgfusepath{clip}%
\pgfsetbuttcap%
\pgfsetmiterjoin%
\definecolor{currentfill}{rgb}{1.000000,0.498039,0.054902}%
\pgfsetfillcolor{currentfill}%
\pgfsetfillopacity{0.600000}%
\pgfsetlinewidth{0.000000pt}%
\definecolor{currentstroke}{rgb}{0.000000,0.000000,0.000000}%
\pgfsetstrokecolor{currentstroke}%
\pgfsetstrokeopacity{0.600000}%
\pgfsetdash{}{0pt}%
\pgfpathmoveto{\pgfqpoint{1.914269in}{0.521603in}}%
\pgfpathlineto{\pgfqpoint{2.014614in}{0.521603in}}%
\pgfpathlineto{\pgfqpoint{2.014614in}{0.676712in}}%
\pgfpathlineto{\pgfqpoint{1.914269in}{0.676712in}}%
\pgfpathlineto{\pgfqpoint{1.914269in}{0.521603in}}%
\pgfpathclose%
\pgfusepath{fill}%
\end{pgfscope}%
\begin{pgfscope}%
\pgfpathrectangle{\pgfqpoint{0.664969in}{0.521603in}}{\pgfqpoint{3.201017in}{1.696195in}}%
\pgfusepath{clip}%
\pgfsetbuttcap%
\pgfsetmiterjoin%
\definecolor{currentfill}{rgb}{1.000000,0.498039,0.054902}%
\pgfsetfillcolor{currentfill}%
\pgfsetfillopacity{0.600000}%
\pgfsetlinewidth{0.000000pt}%
\definecolor{currentstroke}{rgb}{0.000000,0.000000,0.000000}%
\pgfsetstrokecolor{currentstroke}%
\pgfsetstrokeopacity{0.600000}%
\pgfsetdash{}{0pt}%
\pgfpathmoveto{\pgfqpoint{2.014614in}{0.521603in}}%
\pgfpathlineto{\pgfqpoint{2.114960in}{0.521603in}}%
\pgfpathlineto{\pgfqpoint{2.114960in}{0.692576in}}%
\pgfpathlineto{\pgfqpoint{2.014614in}{0.692576in}}%
\pgfpathlineto{\pgfqpoint{2.014614in}{0.521603in}}%
\pgfpathclose%
\pgfusepath{fill}%
\end{pgfscope}%
\begin{pgfscope}%
\pgfpathrectangle{\pgfqpoint{0.664969in}{0.521603in}}{\pgfqpoint{3.201017in}{1.696195in}}%
\pgfusepath{clip}%
\pgfsetbuttcap%
\pgfsetmiterjoin%
\definecolor{currentfill}{rgb}{1.000000,0.498039,0.054902}%
\pgfsetfillcolor{currentfill}%
\pgfsetfillopacity{0.600000}%
\pgfsetlinewidth{0.000000pt}%
\definecolor{currentstroke}{rgb}{0.000000,0.000000,0.000000}%
\pgfsetstrokecolor{currentstroke}%
\pgfsetstrokeopacity{0.600000}%
\pgfsetdash{}{0pt}%
\pgfpathmoveto{\pgfqpoint{2.114960in}{0.521603in}}%
\pgfpathlineto{\pgfqpoint{2.215305in}{0.521603in}}%
\pgfpathlineto{\pgfqpoint{2.215305in}{0.691694in}}%
\pgfpathlineto{\pgfqpoint{2.114960in}{0.691694in}}%
\pgfpathlineto{\pgfqpoint{2.114960in}{0.521603in}}%
\pgfpathclose%
\pgfusepath{fill}%
\end{pgfscope}%
\begin{pgfscope}%
\pgfpathrectangle{\pgfqpoint{0.664969in}{0.521603in}}{\pgfqpoint{3.201017in}{1.696195in}}%
\pgfusepath{clip}%
\pgfsetbuttcap%
\pgfsetmiterjoin%
\definecolor{currentfill}{rgb}{1.000000,0.498039,0.054902}%
\pgfsetfillcolor{currentfill}%
\pgfsetfillopacity{0.600000}%
\pgfsetlinewidth{0.000000pt}%
\definecolor{currentstroke}{rgb}{0.000000,0.000000,0.000000}%
\pgfsetstrokecolor{currentstroke}%
\pgfsetstrokeopacity{0.600000}%
\pgfsetdash{}{0pt}%
\pgfpathmoveto{\pgfqpoint{2.215305in}{0.521603in}}%
\pgfpathlineto{\pgfqpoint{2.315650in}{0.521603in}}%
\pgfpathlineto{\pgfqpoint{2.315650in}{0.661730in}}%
\pgfpathlineto{\pgfqpoint{2.215305in}{0.661730in}}%
\pgfpathlineto{\pgfqpoint{2.215305in}{0.521603in}}%
\pgfpathclose%
\pgfusepath{fill}%
\end{pgfscope}%
\begin{pgfscope}%
\pgfpathrectangle{\pgfqpoint{0.664969in}{0.521603in}}{\pgfqpoint{3.201017in}{1.696195in}}%
\pgfusepath{clip}%
\pgfsetbuttcap%
\pgfsetmiterjoin%
\definecolor{currentfill}{rgb}{1.000000,0.498039,0.054902}%
\pgfsetfillcolor{currentfill}%
\pgfsetfillopacity{0.600000}%
\pgfsetlinewidth{0.000000pt}%
\definecolor{currentstroke}{rgb}{0.000000,0.000000,0.000000}%
\pgfsetstrokecolor{currentstroke}%
\pgfsetstrokeopacity{0.600000}%
\pgfsetdash{}{0pt}%
\pgfpathmoveto{\pgfqpoint{2.315650in}{0.521603in}}%
\pgfpathlineto{\pgfqpoint{2.415996in}{0.521603in}}%
\pgfpathlineto{\pgfqpoint{2.415996in}{0.693457in}}%
\pgfpathlineto{\pgfqpoint{2.315650in}{0.693457in}}%
\pgfpathlineto{\pgfqpoint{2.315650in}{0.521603in}}%
\pgfpathclose%
\pgfusepath{fill}%
\end{pgfscope}%
\begin{pgfscope}%
\pgfpathrectangle{\pgfqpoint{0.664969in}{0.521603in}}{\pgfqpoint{3.201017in}{1.696195in}}%
\pgfusepath{clip}%
\pgfsetbuttcap%
\pgfsetmiterjoin%
\definecolor{currentfill}{rgb}{1.000000,0.498039,0.054902}%
\pgfsetfillcolor{currentfill}%
\pgfsetfillopacity{0.600000}%
\pgfsetlinewidth{0.000000pt}%
\definecolor{currentstroke}{rgb}{0.000000,0.000000,0.000000}%
\pgfsetstrokecolor{currentstroke}%
\pgfsetstrokeopacity{0.600000}%
\pgfsetdash{}{0pt}%
\pgfpathmoveto{\pgfqpoint{2.415996in}{0.521603in}}%
\pgfpathlineto{\pgfqpoint{2.516341in}{0.521603in}}%
\pgfpathlineto{\pgfqpoint{2.516341in}{0.681119in}}%
\pgfpathlineto{\pgfqpoint{2.415996in}{0.681119in}}%
\pgfpathlineto{\pgfqpoint{2.415996in}{0.521603in}}%
\pgfpathclose%
\pgfusepath{fill}%
\end{pgfscope}%
\begin{pgfscope}%
\pgfpathrectangle{\pgfqpoint{0.664969in}{0.521603in}}{\pgfqpoint{3.201017in}{1.696195in}}%
\pgfusepath{clip}%
\pgfsetbuttcap%
\pgfsetmiterjoin%
\definecolor{currentfill}{rgb}{1.000000,0.498039,0.054902}%
\pgfsetfillcolor{currentfill}%
\pgfsetfillopacity{0.600000}%
\pgfsetlinewidth{0.000000pt}%
\definecolor{currentstroke}{rgb}{0.000000,0.000000,0.000000}%
\pgfsetstrokecolor{currentstroke}%
\pgfsetstrokeopacity{0.600000}%
\pgfsetdash{}{0pt}%
\pgfpathmoveto{\pgfqpoint{2.516341in}{0.521603in}}%
\pgfpathlineto{\pgfqpoint{2.616686in}{0.521603in}}%
\pgfpathlineto{\pgfqpoint{2.616686in}{0.674950in}}%
\pgfpathlineto{\pgfqpoint{2.516341in}{0.674950in}}%
\pgfpathlineto{\pgfqpoint{2.516341in}{0.521603in}}%
\pgfpathclose%
\pgfusepath{fill}%
\end{pgfscope}%
\begin{pgfscope}%
\pgfpathrectangle{\pgfqpoint{0.664969in}{0.521603in}}{\pgfqpoint{3.201017in}{1.696195in}}%
\pgfusepath{clip}%
\pgfsetbuttcap%
\pgfsetmiterjoin%
\definecolor{currentfill}{rgb}{1.000000,0.498039,0.054902}%
\pgfsetfillcolor{currentfill}%
\pgfsetfillopacity{0.600000}%
\pgfsetlinewidth{0.000000pt}%
\definecolor{currentstroke}{rgb}{0.000000,0.000000,0.000000}%
\pgfsetstrokecolor{currentstroke}%
\pgfsetstrokeopacity{0.600000}%
\pgfsetdash{}{0pt}%
\pgfpathmoveto{\pgfqpoint{2.616686in}{0.521603in}}%
\pgfpathlineto{\pgfqpoint{2.717032in}{0.521603in}}%
\pgfpathlineto{\pgfqpoint{2.717032in}{0.651154in}}%
\pgfpathlineto{\pgfqpoint{2.616686in}{0.651154in}}%
\pgfpathlineto{\pgfqpoint{2.616686in}{0.521603in}}%
\pgfpathclose%
\pgfusepath{fill}%
\end{pgfscope}%
\begin{pgfscope}%
\pgfpathrectangle{\pgfqpoint{0.664969in}{0.521603in}}{\pgfqpoint{3.201017in}{1.696195in}}%
\pgfusepath{clip}%
\pgfsetbuttcap%
\pgfsetmiterjoin%
\definecolor{currentfill}{rgb}{1.000000,0.498039,0.054902}%
\pgfsetfillcolor{currentfill}%
\pgfsetfillopacity{0.600000}%
\pgfsetlinewidth{0.000000pt}%
\definecolor{currentstroke}{rgb}{0.000000,0.000000,0.000000}%
\pgfsetstrokecolor{currentstroke}%
\pgfsetstrokeopacity{0.600000}%
\pgfsetdash{}{0pt}%
\pgfpathmoveto{\pgfqpoint{2.717032in}{0.521603in}}%
\pgfpathlineto{\pgfqpoint{2.817377in}{0.521603in}}%
\pgfpathlineto{\pgfqpoint{2.817377in}{0.641460in}}%
\pgfpathlineto{\pgfqpoint{2.717032in}{0.641460in}}%
\pgfpathlineto{\pgfqpoint{2.717032in}{0.521603in}}%
\pgfpathclose%
\pgfusepath{fill}%
\end{pgfscope}%
\begin{pgfscope}%
\pgfpathrectangle{\pgfqpoint{0.664969in}{0.521603in}}{\pgfqpoint{3.201017in}{1.696195in}}%
\pgfusepath{clip}%
\pgfsetbuttcap%
\pgfsetmiterjoin%
\definecolor{currentfill}{rgb}{1.000000,0.498039,0.054902}%
\pgfsetfillcolor{currentfill}%
\pgfsetfillopacity{0.600000}%
\pgfsetlinewidth{0.000000pt}%
\definecolor{currentstroke}{rgb}{0.000000,0.000000,0.000000}%
\pgfsetstrokecolor{currentstroke}%
\pgfsetstrokeopacity{0.600000}%
\pgfsetdash{}{0pt}%
\pgfpathmoveto{\pgfqpoint{2.817377in}{0.521603in}}%
\pgfpathlineto{\pgfqpoint{2.917723in}{0.521603in}}%
\pgfpathlineto{\pgfqpoint{2.917723in}{0.617665in}}%
\pgfpathlineto{\pgfqpoint{2.817377in}{0.617665in}}%
\pgfpathlineto{\pgfqpoint{2.817377in}{0.521603in}}%
\pgfpathclose%
\pgfusepath{fill}%
\end{pgfscope}%
\begin{pgfscope}%
\pgfpathrectangle{\pgfqpoint{0.664969in}{0.521603in}}{\pgfqpoint{3.201017in}{1.696195in}}%
\pgfusepath{clip}%
\pgfsetbuttcap%
\pgfsetmiterjoin%
\definecolor{currentfill}{rgb}{1.000000,0.498039,0.054902}%
\pgfsetfillcolor{currentfill}%
\pgfsetfillopacity{0.600000}%
\pgfsetlinewidth{0.000000pt}%
\definecolor{currentstroke}{rgb}{0.000000,0.000000,0.000000}%
\pgfsetstrokecolor{currentstroke}%
\pgfsetstrokeopacity{0.600000}%
\pgfsetdash{}{0pt}%
\pgfpathmoveto{\pgfqpoint{2.917723in}{0.521603in}}%
\pgfpathlineto{\pgfqpoint{3.018068in}{0.521603in}}%
\pgfpathlineto{\pgfqpoint{3.018068in}{0.592989in}}%
\pgfpathlineto{\pgfqpoint{2.917723in}{0.592989in}}%
\pgfpathlineto{\pgfqpoint{2.917723in}{0.521603in}}%
\pgfpathclose%
\pgfusepath{fill}%
\end{pgfscope}%
\begin{pgfscope}%
\pgfpathrectangle{\pgfqpoint{0.664969in}{0.521603in}}{\pgfqpoint{3.201017in}{1.696195in}}%
\pgfusepath{clip}%
\pgfsetbuttcap%
\pgfsetmiterjoin%
\definecolor{currentfill}{rgb}{1.000000,0.498039,0.054902}%
\pgfsetfillcolor{currentfill}%
\pgfsetfillopacity{0.600000}%
\pgfsetlinewidth{0.000000pt}%
\definecolor{currentstroke}{rgb}{0.000000,0.000000,0.000000}%
\pgfsetstrokecolor{currentstroke}%
\pgfsetstrokeopacity{0.600000}%
\pgfsetdash{}{0pt}%
\pgfpathmoveto{\pgfqpoint{3.018068in}{0.521603in}}%
\pgfpathlineto{\pgfqpoint{3.118413in}{0.521603in}}%
\pgfpathlineto{\pgfqpoint{3.118413in}{0.584176in}}%
\pgfpathlineto{\pgfqpoint{3.018068in}{0.584176in}}%
\pgfpathlineto{\pgfqpoint{3.018068in}{0.521603in}}%
\pgfpathclose%
\pgfusepath{fill}%
\end{pgfscope}%
\begin{pgfscope}%
\pgfpathrectangle{\pgfqpoint{0.664969in}{0.521603in}}{\pgfqpoint{3.201017in}{1.696195in}}%
\pgfusepath{clip}%
\pgfsetbuttcap%
\pgfsetmiterjoin%
\definecolor{currentfill}{rgb}{1.000000,0.498039,0.054902}%
\pgfsetfillcolor{currentfill}%
\pgfsetfillopacity{0.600000}%
\pgfsetlinewidth{0.000000pt}%
\definecolor{currentstroke}{rgb}{0.000000,0.000000,0.000000}%
\pgfsetstrokecolor{currentstroke}%
\pgfsetstrokeopacity{0.600000}%
\pgfsetdash{}{0pt}%
\pgfpathmoveto{\pgfqpoint{3.118413in}{0.521603in}}%
\pgfpathlineto{\pgfqpoint{3.218759in}{0.521603in}}%
\pgfpathlineto{\pgfqpoint{3.218759in}{0.562143in}}%
\pgfpathlineto{\pgfqpoint{3.118413in}{0.562143in}}%
\pgfpathlineto{\pgfqpoint{3.118413in}{0.521603in}}%
\pgfpathclose%
\pgfusepath{fill}%
\end{pgfscope}%
\begin{pgfscope}%
\pgfpathrectangle{\pgfqpoint{0.664969in}{0.521603in}}{\pgfqpoint{3.201017in}{1.696195in}}%
\pgfusepath{clip}%
\pgfsetbuttcap%
\pgfsetmiterjoin%
\definecolor{currentfill}{rgb}{1.000000,0.498039,0.054902}%
\pgfsetfillcolor{currentfill}%
\pgfsetfillopacity{0.600000}%
\pgfsetlinewidth{0.000000pt}%
\definecolor{currentstroke}{rgb}{0.000000,0.000000,0.000000}%
\pgfsetstrokecolor{currentstroke}%
\pgfsetstrokeopacity{0.600000}%
\pgfsetdash{}{0pt}%
\pgfpathmoveto{\pgfqpoint{3.218759in}{0.521603in}}%
\pgfpathlineto{\pgfqpoint{3.319104in}{0.521603in}}%
\pgfpathlineto{\pgfqpoint{3.319104in}{0.552449in}}%
\pgfpathlineto{\pgfqpoint{3.218759in}{0.552449in}}%
\pgfpathlineto{\pgfqpoint{3.218759in}{0.521603in}}%
\pgfpathclose%
\pgfusepath{fill}%
\end{pgfscope}%
\begin{pgfscope}%
\pgfpathrectangle{\pgfqpoint{0.664969in}{0.521603in}}{\pgfqpoint{3.201017in}{1.696195in}}%
\pgfusepath{clip}%
\pgfsetbuttcap%
\pgfsetmiterjoin%
\definecolor{currentfill}{rgb}{1.000000,0.498039,0.054902}%
\pgfsetfillcolor{currentfill}%
\pgfsetfillopacity{0.600000}%
\pgfsetlinewidth{0.000000pt}%
\definecolor{currentstroke}{rgb}{0.000000,0.000000,0.000000}%
\pgfsetstrokecolor{currentstroke}%
\pgfsetstrokeopacity{0.600000}%
\pgfsetdash{}{0pt}%
\pgfpathmoveto{\pgfqpoint{3.319104in}{0.521603in}}%
\pgfpathlineto{\pgfqpoint{3.419449in}{0.521603in}}%
\pgfpathlineto{\pgfqpoint{3.419449in}{0.537467in}}%
\pgfpathlineto{\pgfqpoint{3.319104in}{0.537467in}}%
\pgfpathlineto{\pgfqpoint{3.319104in}{0.521603in}}%
\pgfpathclose%
\pgfusepath{fill}%
\end{pgfscope}%
\begin{pgfscope}%
\pgfpathrectangle{\pgfqpoint{0.664969in}{0.521603in}}{\pgfqpoint{3.201017in}{1.696195in}}%
\pgfusepath{clip}%
\pgfsetbuttcap%
\pgfsetmiterjoin%
\definecolor{currentfill}{rgb}{1.000000,0.498039,0.054902}%
\pgfsetfillcolor{currentfill}%
\pgfsetfillopacity{0.600000}%
\pgfsetlinewidth{0.000000pt}%
\definecolor{currentstroke}{rgb}{0.000000,0.000000,0.000000}%
\pgfsetstrokecolor{currentstroke}%
\pgfsetstrokeopacity{0.600000}%
\pgfsetdash{}{0pt}%
\pgfpathmoveto{\pgfqpoint{3.419449in}{0.521603in}}%
\pgfpathlineto{\pgfqpoint{3.519795in}{0.521603in}}%
\pgfpathlineto{\pgfqpoint{3.519795in}{0.532179in}}%
\pgfpathlineto{\pgfqpoint{3.419449in}{0.532179in}}%
\pgfpathlineto{\pgfqpoint{3.419449in}{0.521603in}}%
\pgfpathclose%
\pgfusepath{fill}%
\end{pgfscope}%
\begin{pgfscope}%
\pgfpathrectangle{\pgfqpoint{0.664969in}{0.521603in}}{\pgfqpoint{3.201017in}{1.696195in}}%
\pgfusepath{clip}%
\pgfsetbuttcap%
\pgfsetmiterjoin%
\definecolor{currentfill}{rgb}{1.000000,0.498039,0.054902}%
\pgfsetfillcolor{currentfill}%
\pgfsetfillopacity{0.600000}%
\pgfsetlinewidth{0.000000pt}%
\definecolor{currentstroke}{rgb}{0.000000,0.000000,0.000000}%
\pgfsetstrokecolor{currentstroke}%
\pgfsetstrokeopacity{0.600000}%
\pgfsetdash{}{0pt}%
\pgfpathmoveto{\pgfqpoint{3.519795in}{0.521603in}}%
\pgfpathlineto{\pgfqpoint{3.620140in}{0.521603in}}%
\pgfpathlineto{\pgfqpoint{3.620140in}{0.530416in}}%
\pgfpathlineto{\pgfqpoint{3.519795in}{0.530416in}}%
\pgfpathlineto{\pgfqpoint{3.519795in}{0.521603in}}%
\pgfpathclose%
\pgfusepath{fill}%
\end{pgfscope}%
\begin{pgfscope}%
\pgfpathrectangle{\pgfqpoint{0.664969in}{0.521603in}}{\pgfqpoint{3.201017in}{1.696195in}}%
\pgfusepath{clip}%
\pgfsetbuttcap%
\pgfsetmiterjoin%
\definecolor{currentfill}{rgb}{1.000000,0.498039,0.054902}%
\pgfsetfillcolor{currentfill}%
\pgfsetfillopacity{0.600000}%
\pgfsetlinewidth{0.000000pt}%
\definecolor{currentstroke}{rgb}{0.000000,0.000000,0.000000}%
\pgfsetstrokecolor{currentstroke}%
\pgfsetstrokeopacity{0.600000}%
\pgfsetdash{}{0pt}%
\pgfpathmoveto{\pgfqpoint{3.620140in}{0.521603in}}%
\pgfpathlineto{\pgfqpoint{3.720485in}{0.521603in}}%
\pgfpathlineto{\pgfqpoint{3.720485in}{0.525129in}}%
\pgfpathlineto{\pgfqpoint{3.620140in}{0.525129in}}%
\pgfpathlineto{\pgfqpoint{3.620140in}{0.521603in}}%
\pgfpathclose%
\pgfusepath{fill}%
\end{pgfscope}%
\begin{pgfscope}%
\pgfsetbuttcap%
\pgfsetroundjoin%
\definecolor{currentfill}{rgb}{0.000000,0.000000,0.000000}%
\pgfsetfillcolor{currentfill}%
\pgfsetlinewidth{0.803000pt}%
\definecolor{currentstroke}{rgb}{0.000000,0.000000,0.000000}%
\pgfsetstrokecolor{currentstroke}%
\pgfsetdash{}{0pt}%
\pgfsys@defobject{currentmarker}{\pgfqpoint{0.000000in}{-0.048611in}}{\pgfqpoint{0.000000in}{0.000000in}}{%
\pgfpathmoveto{\pgfqpoint{0.000000in}{0.000000in}}%
\pgfpathlineto{\pgfqpoint{0.000000in}{-0.048611in}}%
\pgfusepath{stroke,fill}%
}%
\begin{pgfscope}%
\pgfsys@transformshift{0.758505in}{0.521603in}%
\pgfsys@useobject{currentmarker}{}%
\end{pgfscope}%
\end{pgfscope}%
\begin{pgfscope}%
\definecolor{textcolor}{rgb}{0.000000,0.000000,0.000000}%
\pgfsetstrokecolor{textcolor}%
\pgfsetfillcolor{textcolor}%
\pgftext[x=0.758505in,y=0.424381in,,top]{\color{textcolor}{\rmfamily\fontsize{10.000000}{12.000000}\selectfont\catcode`\^=\active\def^{\ifmmode\sp\else\^{}\fi}\catcode`\%=\active\def%{\%}$\mathdefault{0}$}}%
\end{pgfscope}%
\begin{pgfscope}%
\pgfsetbuttcap%
\pgfsetroundjoin%
\definecolor{currentfill}{rgb}{0.000000,0.000000,0.000000}%
\pgfsetfillcolor{currentfill}%
\pgfsetlinewidth{0.803000pt}%
\definecolor{currentstroke}{rgb}{0.000000,0.000000,0.000000}%
\pgfsetstrokecolor{currentstroke}%
\pgfsetdash{}{0pt}%
\pgfsys@defobject{currentmarker}{\pgfqpoint{0.000000in}{-0.048611in}}{\pgfqpoint{0.000000in}{0.000000in}}{%
\pgfpathmoveto{\pgfqpoint{0.000000in}{0.000000in}}%
\pgfpathlineto{\pgfqpoint{0.000000in}{-0.048611in}}%
\pgfusepath{stroke,fill}%
}%
\begin{pgfscope}%
\pgfsys@transformshift{1.278151in}{0.521603in}%
\pgfsys@useobject{currentmarker}{}%
\end{pgfscope}%
\end{pgfscope}%
\begin{pgfscope}%
\definecolor{textcolor}{rgb}{0.000000,0.000000,0.000000}%
\pgfsetstrokecolor{textcolor}%
\pgfsetfillcolor{textcolor}%
\pgftext[x=1.278151in,y=0.424381in,,top]{\color{textcolor}{\rmfamily\fontsize{10.000000}{12.000000}\selectfont\catcode`\^=\active\def^{\ifmmode\sp\else\^{}\fi}\catcode`\%=\active\def%{\%}$\mathdefault{20}$}}%
\end{pgfscope}%
\begin{pgfscope}%
\pgfsetbuttcap%
\pgfsetroundjoin%
\definecolor{currentfill}{rgb}{0.000000,0.000000,0.000000}%
\pgfsetfillcolor{currentfill}%
\pgfsetlinewidth{0.803000pt}%
\definecolor{currentstroke}{rgb}{0.000000,0.000000,0.000000}%
\pgfsetstrokecolor{currentstroke}%
\pgfsetdash{}{0pt}%
\pgfsys@defobject{currentmarker}{\pgfqpoint{0.000000in}{-0.048611in}}{\pgfqpoint{0.000000in}{0.000000in}}{%
\pgfpathmoveto{\pgfqpoint{0.000000in}{0.000000in}}%
\pgfpathlineto{\pgfqpoint{0.000000in}{-0.048611in}}%
\pgfusepath{stroke,fill}%
}%
\begin{pgfscope}%
\pgfsys@transformshift{1.797797in}{0.521603in}%
\pgfsys@useobject{currentmarker}{}%
\end{pgfscope}%
\end{pgfscope}%
\begin{pgfscope}%
\definecolor{textcolor}{rgb}{0.000000,0.000000,0.000000}%
\pgfsetstrokecolor{textcolor}%
\pgfsetfillcolor{textcolor}%
\pgftext[x=1.797797in,y=0.424381in,,top]{\color{textcolor}{\rmfamily\fontsize{10.000000}{12.000000}\selectfont\catcode`\^=\active\def^{\ifmmode\sp\else\^{}\fi}\catcode`\%=\active\def%{\%}$\mathdefault{40}$}}%
\end{pgfscope}%
\begin{pgfscope}%
\pgfsetbuttcap%
\pgfsetroundjoin%
\definecolor{currentfill}{rgb}{0.000000,0.000000,0.000000}%
\pgfsetfillcolor{currentfill}%
\pgfsetlinewidth{0.803000pt}%
\definecolor{currentstroke}{rgb}{0.000000,0.000000,0.000000}%
\pgfsetstrokecolor{currentstroke}%
\pgfsetdash{}{0pt}%
\pgfsys@defobject{currentmarker}{\pgfqpoint{0.000000in}{-0.048611in}}{\pgfqpoint{0.000000in}{0.000000in}}{%
\pgfpathmoveto{\pgfqpoint{0.000000in}{0.000000in}}%
\pgfpathlineto{\pgfqpoint{0.000000in}{-0.048611in}}%
\pgfusepath{stroke,fill}%
}%
\begin{pgfscope}%
\pgfsys@transformshift{2.317442in}{0.521603in}%
\pgfsys@useobject{currentmarker}{}%
\end{pgfscope}%
\end{pgfscope}%
\begin{pgfscope}%
\definecolor{textcolor}{rgb}{0.000000,0.000000,0.000000}%
\pgfsetstrokecolor{textcolor}%
\pgfsetfillcolor{textcolor}%
\pgftext[x=2.317442in,y=0.424381in,,top]{\color{textcolor}{\rmfamily\fontsize{10.000000}{12.000000}\selectfont\catcode`\^=\active\def^{\ifmmode\sp\else\^{}\fi}\catcode`\%=\active\def%{\%}$\mathdefault{60}$}}%
\end{pgfscope}%
\begin{pgfscope}%
\pgfsetbuttcap%
\pgfsetroundjoin%
\definecolor{currentfill}{rgb}{0.000000,0.000000,0.000000}%
\pgfsetfillcolor{currentfill}%
\pgfsetlinewidth{0.803000pt}%
\definecolor{currentstroke}{rgb}{0.000000,0.000000,0.000000}%
\pgfsetstrokecolor{currentstroke}%
\pgfsetdash{}{0pt}%
\pgfsys@defobject{currentmarker}{\pgfqpoint{0.000000in}{-0.048611in}}{\pgfqpoint{0.000000in}{0.000000in}}{%
\pgfpathmoveto{\pgfqpoint{0.000000in}{0.000000in}}%
\pgfpathlineto{\pgfqpoint{0.000000in}{-0.048611in}}%
\pgfusepath{stroke,fill}%
}%
\begin{pgfscope}%
\pgfsys@transformshift{2.837088in}{0.521603in}%
\pgfsys@useobject{currentmarker}{}%
\end{pgfscope}%
\end{pgfscope}%
\begin{pgfscope}%
\definecolor{textcolor}{rgb}{0.000000,0.000000,0.000000}%
\pgfsetstrokecolor{textcolor}%
\pgfsetfillcolor{textcolor}%
\pgftext[x=2.837088in,y=0.424381in,,top]{\color{textcolor}{\rmfamily\fontsize{10.000000}{12.000000}\selectfont\catcode`\^=\active\def^{\ifmmode\sp\else\^{}\fi}\catcode`\%=\active\def%{\%}$\mathdefault{80}$}}%
\end{pgfscope}%
\begin{pgfscope}%
\pgfsetbuttcap%
\pgfsetroundjoin%
\definecolor{currentfill}{rgb}{0.000000,0.000000,0.000000}%
\pgfsetfillcolor{currentfill}%
\pgfsetlinewidth{0.803000pt}%
\definecolor{currentstroke}{rgb}{0.000000,0.000000,0.000000}%
\pgfsetstrokecolor{currentstroke}%
\pgfsetdash{}{0pt}%
\pgfsys@defobject{currentmarker}{\pgfqpoint{0.000000in}{-0.048611in}}{\pgfqpoint{0.000000in}{0.000000in}}{%
\pgfpathmoveto{\pgfqpoint{0.000000in}{0.000000in}}%
\pgfpathlineto{\pgfqpoint{0.000000in}{-0.048611in}}%
\pgfusepath{stroke,fill}%
}%
\begin{pgfscope}%
\pgfsys@transformshift{3.356733in}{0.521603in}%
\pgfsys@useobject{currentmarker}{}%
\end{pgfscope}%
\end{pgfscope}%
\begin{pgfscope}%
\definecolor{textcolor}{rgb}{0.000000,0.000000,0.000000}%
\pgfsetstrokecolor{textcolor}%
\pgfsetfillcolor{textcolor}%
\pgftext[x=3.356733in,y=0.424381in,,top]{\color{textcolor}{\rmfamily\fontsize{10.000000}{12.000000}\selectfont\catcode`\^=\active\def^{\ifmmode\sp\else\^{}\fi}\catcode`\%=\active\def%{\%}$\mathdefault{100}$}}%
\end{pgfscope}%
\begin{pgfscope}%
\definecolor{textcolor}{rgb}{0.000000,0.000000,0.000000}%
\pgfsetstrokecolor{textcolor}%
\pgfsetfillcolor{textcolor}%
\pgftext[x=2.265478in,y=0.234413in,,top]{\color{textcolor}{\rmfamily\fontsize{10.000000}{12.000000}\selectfont\catcode`\^=\active\def^{\ifmmode\sp\else\^{}\fi}\catcode`\%=\active\def%{\%}Number of m. classes or t. components}}%
\end{pgfscope}%
\begin{pgfscope}%
\pgfsetbuttcap%
\pgfsetroundjoin%
\definecolor{currentfill}{rgb}{0.000000,0.000000,0.000000}%
\pgfsetfillcolor{currentfill}%
\pgfsetlinewidth{0.803000pt}%
\definecolor{currentstroke}{rgb}{0.000000,0.000000,0.000000}%
\pgfsetstrokecolor{currentstroke}%
\pgfsetdash{}{0pt}%
\pgfsys@defobject{currentmarker}{\pgfqpoint{-0.048611in}{0.000000in}}{\pgfqpoint{-0.000000in}{0.000000in}}{%
\pgfpathmoveto{\pgfqpoint{-0.000000in}{0.000000in}}%
\pgfpathlineto{\pgfqpoint{-0.048611in}{0.000000in}}%
\pgfusepath{stroke,fill}%
}%
\begin{pgfscope}%
\pgfsys@transformshift{0.664969in}{0.521603in}%
\pgfsys@useobject{currentmarker}{}%
\end{pgfscope}%
\end{pgfscope}%
\begin{pgfscope}%
\definecolor{textcolor}{rgb}{0.000000,0.000000,0.000000}%
\pgfsetstrokecolor{textcolor}%
\pgfsetfillcolor{textcolor}%
\pgftext[x=0.498302in, y=0.468842in, left, base]{\color{textcolor}{\rmfamily\fontsize{10.000000}{12.000000}\selectfont\catcode`\^=\active\def^{\ifmmode\sp\else\^{}\fi}\catcode`\%=\active\def%{\%}$\mathdefault{0}$}}%
\end{pgfscope}%
\begin{pgfscope}%
\pgfsetbuttcap%
\pgfsetroundjoin%
\definecolor{currentfill}{rgb}{0.000000,0.000000,0.000000}%
\pgfsetfillcolor{currentfill}%
\pgfsetlinewidth{0.803000pt}%
\definecolor{currentstroke}{rgb}{0.000000,0.000000,0.000000}%
\pgfsetstrokecolor{currentstroke}%
\pgfsetdash{}{0pt}%
\pgfsys@defobject{currentmarker}{\pgfqpoint{-0.048611in}{0.000000in}}{\pgfqpoint{-0.000000in}{0.000000in}}{%
\pgfpathmoveto{\pgfqpoint{-0.000000in}{0.000000in}}%
\pgfpathlineto{\pgfqpoint{-0.048611in}{0.000000in}}%
\pgfusepath{stroke,fill}%
}%
\begin{pgfscope}%
\pgfsys@transformshift{0.664969in}{0.962254in}%
\pgfsys@useobject{currentmarker}{}%
\end{pgfscope}%
\end{pgfscope}%
\begin{pgfscope}%
\definecolor{textcolor}{rgb}{0.000000,0.000000,0.000000}%
\pgfsetstrokecolor{textcolor}%
\pgfsetfillcolor{textcolor}%
\pgftext[x=0.359413in, y=0.909492in, left, base]{\color{textcolor}{\rmfamily\fontsize{10.000000}{12.000000}\selectfont\catcode`\^=\active\def^{\ifmmode\sp\else\^{}\fi}\catcode`\%=\active\def%{\%}$\mathdefault{500}$}}%
\end{pgfscope}%
\begin{pgfscope}%
\pgfsetbuttcap%
\pgfsetroundjoin%
\definecolor{currentfill}{rgb}{0.000000,0.000000,0.000000}%
\pgfsetfillcolor{currentfill}%
\pgfsetlinewidth{0.803000pt}%
\definecolor{currentstroke}{rgb}{0.000000,0.000000,0.000000}%
\pgfsetstrokecolor{currentstroke}%
\pgfsetdash{}{0pt}%
\pgfsys@defobject{currentmarker}{\pgfqpoint{-0.048611in}{0.000000in}}{\pgfqpoint{-0.000000in}{0.000000in}}{%
\pgfpathmoveto{\pgfqpoint{-0.000000in}{0.000000in}}%
\pgfpathlineto{\pgfqpoint{-0.048611in}{0.000000in}}%
\pgfusepath{stroke,fill}%
}%
\begin{pgfscope}%
\pgfsys@transformshift{0.664969in}{1.402904in}%
\pgfsys@useobject{currentmarker}{}%
\end{pgfscope}%
\end{pgfscope}%
\begin{pgfscope}%
\definecolor{textcolor}{rgb}{0.000000,0.000000,0.000000}%
\pgfsetstrokecolor{textcolor}%
\pgfsetfillcolor{textcolor}%
\pgftext[x=0.289968in, y=1.350142in, left, base]{\color{textcolor}{\rmfamily\fontsize{10.000000}{12.000000}\selectfont\catcode`\^=\active\def^{\ifmmode\sp\else\^{}\fi}\catcode`\%=\active\def%{\%}$\mathdefault{1000}$}}%
\end{pgfscope}%
\begin{pgfscope}%
\pgfsetbuttcap%
\pgfsetroundjoin%
\definecolor{currentfill}{rgb}{0.000000,0.000000,0.000000}%
\pgfsetfillcolor{currentfill}%
\pgfsetlinewidth{0.803000pt}%
\definecolor{currentstroke}{rgb}{0.000000,0.000000,0.000000}%
\pgfsetstrokecolor{currentstroke}%
\pgfsetdash{}{0pt}%
\pgfsys@defobject{currentmarker}{\pgfqpoint{-0.048611in}{0.000000in}}{\pgfqpoint{-0.000000in}{0.000000in}}{%
\pgfpathmoveto{\pgfqpoint{-0.000000in}{0.000000in}}%
\pgfpathlineto{\pgfqpoint{-0.048611in}{0.000000in}}%
\pgfusepath{stroke,fill}%
}%
\begin{pgfscope}%
\pgfsys@transformshift{0.664969in}{1.843554in}%
\pgfsys@useobject{currentmarker}{}%
\end{pgfscope}%
\end{pgfscope}%
\begin{pgfscope}%
\definecolor{textcolor}{rgb}{0.000000,0.000000,0.000000}%
\pgfsetstrokecolor{textcolor}%
\pgfsetfillcolor{textcolor}%
\pgftext[x=0.289968in, y=1.790792in, left, base]{\color{textcolor}{\rmfamily\fontsize{10.000000}{12.000000}\selectfont\catcode`\^=\active\def^{\ifmmode\sp\else\^{}\fi}\catcode`\%=\active\def%{\%}$\mathdefault{1500}$}}%
\end{pgfscope}%
\begin{pgfscope}%
\definecolor{textcolor}{rgb}{0.000000,0.000000,0.000000}%
\pgfsetstrokecolor{textcolor}%
\pgfsetfillcolor{textcolor}%
\pgftext[x=0.234413in,y=1.369701in,,bottom,rotate=90.000000]{\color{textcolor}{\rmfamily\fontsize{10.000000}{12.000000}\selectfont\catcode`\^=\active\def^{\ifmmode\sp\else\^{}\fi}\catcode`\%=\active\def%{\%}Number of graphs}}%
\end{pgfscope}%
\begin{pgfscope}%
\pgfsetrectcap%
\pgfsetmiterjoin%
\pgfsetlinewidth{0.803000pt}%
\definecolor{currentstroke}{rgb}{0.000000,0.000000,0.000000}%
\pgfsetstrokecolor{currentstroke}%
\pgfsetdash{}{0pt}%
\pgfpathmoveto{\pgfqpoint{0.664969in}{0.521603in}}%
\pgfpathlineto{\pgfqpoint{0.664969in}{2.217798in}}%
\pgfusepath{stroke}%
\end{pgfscope}%
\begin{pgfscope}%
\pgfsetrectcap%
\pgfsetmiterjoin%
\pgfsetlinewidth{0.803000pt}%
\definecolor{currentstroke}{rgb}{0.000000,0.000000,0.000000}%
\pgfsetstrokecolor{currentstroke}%
\pgfsetdash{}{0pt}%
\pgfpathmoveto{\pgfqpoint{3.865986in}{0.521603in}}%
\pgfpathlineto{\pgfqpoint{3.865986in}{2.217798in}}%
\pgfusepath{stroke}%
\end{pgfscope}%
\begin{pgfscope}%
\pgfsetrectcap%
\pgfsetmiterjoin%
\pgfsetlinewidth{0.803000pt}%
\definecolor{currentstroke}{rgb}{0.000000,0.000000,0.000000}%
\pgfsetstrokecolor{currentstroke}%
\pgfsetdash{}{0pt}%
\pgfpathmoveto{\pgfqpoint{0.664969in}{0.521603in}}%
\pgfpathlineto{\pgfqpoint{3.865986in}{0.521603in}}%
\pgfusepath{stroke}%
\end{pgfscope}%
\begin{pgfscope}%
\pgfsetrectcap%
\pgfsetmiterjoin%
\pgfsetlinewidth{0.803000pt}%
\definecolor{currentstroke}{rgb}{0.000000,0.000000,0.000000}%
\pgfsetstrokecolor{currentstroke}%
\pgfsetdash{}{0pt}%
\pgfpathmoveto{\pgfqpoint{0.664969in}{2.217798in}}%
\pgfpathlineto{\pgfqpoint{3.865986in}{2.217798in}}%
\pgfusepath{stroke}%
\end{pgfscope}%
\begin{pgfscope}%
\pgfsetbuttcap%
\pgfsetmiterjoin%
\definecolor{currentfill}{rgb}{1.000000,1.000000,1.000000}%
\pgfsetfillcolor{currentfill}%
\pgfsetfillopacity{0.800000}%
\pgfsetlinewidth{1.003750pt}%
\definecolor{currentstroke}{rgb}{0.800000,0.800000,0.800000}%
\pgfsetstrokecolor{currentstroke}%
\pgfsetstrokeopacity{0.800000}%
\pgfsetdash{}{0pt}%
\pgfpathmoveto{\pgfqpoint{1.375923in}{1.750855in}}%
\pgfpathlineto{\pgfqpoint{3.778486in}{1.750855in}}%
\pgfpathquadraticcurveto{\pgfqpoint{3.803486in}{1.750855in}}{\pgfqpoint{3.803486in}{1.775855in}}%
\pgfpathlineto{\pgfqpoint{3.803486in}{2.130298in}}%
\pgfpathquadraticcurveto{\pgfqpoint{3.803486in}{2.155298in}}{\pgfqpoint{3.778486in}{2.155298in}}%
\pgfpathlineto{\pgfqpoint{1.375923in}{2.155298in}}%
\pgfpathquadraticcurveto{\pgfqpoint{1.350923in}{2.155298in}}{\pgfqpoint{1.350923in}{2.130298in}}%
\pgfpathlineto{\pgfqpoint{1.350923in}{1.775855in}}%
\pgfpathquadraticcurveto{\pgfqpoint{1.350923in}{1.750855in}}{\pgfqpoint{1.375923in}{1.750855in}}%
\pgfpathlineto{\pgfqpoint{1.375923in}{1.750855in}}%
\pgfpathclose%
\pgfusepath{stroke,fill}%
\end{pgfscope}%
\begin{pgfscope}%
\pgfsetbuttcap%
\pgfsetmiterjoin%
\definecolor{currentfill}{rgb}{0.121569,0.466667,0.705882}%
\pgfsetfillcolor{currentfill}%
\pgfsetfillopacity{0.600000}%
\pgfsetlinewidth{0.000000pt}%
\definecolor{currentstroke}{rgb}{0.000000,0.000000,0.000000}%
\pgfsetstrokecolor{currentstroke}%
\pgfsetstrokeopacity{0.600000}%
\pgfsetdash{}{0pt}%
\pgfpathmoveto{\pgfqpoint{1.400923in}{2.010327in}}%
\pgfpathlineto{\pgfqpoint{1.650923in}{2.010327in}}%
\pgfpathlineto{\pgfqpoint{1.650923in}{2.097827in}}%
\pgfpathlineto{\pgfqpoint{1.400923in}{2.097827in}}%
\pgfpathlineto{\pgfqpoint{1.400923in}{2.010327in}}%
\pgfpathclose%
\pgfusepath{fill}%
\end{pgfscope}%
\begin{pgfscope}%
\definecolor{textcolor}{rgb}{0.000000,0.000000,0.000000}%
\pgfsetstrokecolor{textcolor}%
\pgfsetfillcolor{textcolor}%
\pgftext[x=1.750923in,y=2.010327in,left,base]{\color{textcolor}{\rmfamily\fontsize{9.000000}{10.800000}\selectfont\catcode`\^=\active\def^{\ifmmode\sp\else\^{}\fi}\catcode`\%=\active\def%{\%}Monochromatic classes}}%
\end{pgfscope}%
\begin{pgfscope}%
\pgfsetbuttcap%
\pgfsetmiterjoin%
\definecolor{currentfill}{rgb}{1.000000,0.498039,0.054902}%
\pgfsetfillcolor{currentfill}%
\pgfsetfillopacity{0.600000}%
\pgfsetlinewidth{0.000000pt}%
\definecolor{currentstroke}{rgb}{0.000000,0.000000,0.000000}%
\pgfsetstrokecolor{currentstroke}%
\pgfsetstrokeopacity{0.600000}%
\pgfsetdash{}{0pt}%
\pgfpathmoveto{\pgfqpoint{1.400923in}{1.826856in}}%
\pgfpathlineto{\pgfqpoint{1.650923in}{1.826856in}}%
\pgfpathlineto{\pgfqpoint{1.650923in}{1.914356in}}%
\pgfpathlineto{\pgfqpoint{1.400923in}{1.914356in}}%
\pgfpathlineto{\pgfqpoint{1.400923in}{1.826856in}}%
\pgfpathclose%
\pgfusepath{fill}%
\end{pgfscope}%
\begin{pgfscope}%
\definecolor{textcolor}{rgb}{0.000000,0.000000,0.000000}%
\pgfsetstrokecolor{textcolor}%
\pgfsetfillcolor{textcolor}%
\pgftext[x=1.750923in,y=1.826856in,left,base]{\color{textcolor}{\rmfamily\fontsize{9.000000}{10.800000}\selectfont\catcode`\^=\active\def^{\ifmmode\sp\else\^{}\fi}\catcode`\%=\active\def%{\%}Triangle-connected components}}%
\end{pgfscope}%
\end{pgfpicture}%
\makeatother%
\endgroup%
}
		\caption[Monoch. classes vs tr. con. components for globally rigid]{%
			\centering Globally rigid graphs}%
		\label{fig:monochrom_vs_triangle_globally_rigid}
	\end{subfigure}
	\hfill
	\begin{subfigure}{0.48\textwidth}
		\centering
		\scalebox{0.6}{%% Creator: Matplotlib, PGF backend
%%
%% To include the figure in your LaTeX document, write
%%   \input{<filename>.pgf}
%%
%% Make sure the required packages are loaded in your preamble
%%   \usepackage{pgf}
%%
%% Also ensure that all the required font packages are loaded; for instance,
%% the lmodern package is sometimes necessary when using math font.
%%   \usepackage{lmodern}
%%
%% Figures using additional raster images can only be included by \input if
%% they are in the same directory as the main LaTeX file. For loading figures
%% from other directories you can use the `import` package
%%   \usepackage{import}
%%
%% and then include the figures with
%%   \import{<path to file>}{<filename>.pgf}
%%
%% Matplotlib used the following preamble
%%   \def\mathdefault#1{#1}
%%   \everymath=\expandafter{\the\everymath\displaystyle}
%%   \IfFileExists{scrextend.sty}{
%%     \usepackage[fontsize=10.000000pt]{scrextend}
%%   }{
%%     \renewcommand{\normalsize}{\fontsize{10.000000}{12.000000}\selectfont}
%%     \normalsize
%%   }
%%   
%%   \ifdefined\pdftexversion\else  % non-pdftex case.
%%     \usepackage{fontspec}
%%     \setmainfont{DejaVuSans.ttf}[Path=\detokenize{/home/petr/Projects/PyRigi/.venv/lib/python3.12/site-packages/matplotlib/mpl-data/fonts/ttf/}]
%%     \setsansfont{DejaVuSans.ttf}[Path=\detokenize{/home/petr/Projects/PyRigi/.venv/lib/python3.12/site-packages/matplotlib/mpl-data/fonts/ttf/}]
%%     \setmonofont{DejaVuSansMono.ttf}[Path=\detokenize{/home/petr/Projects/PyRigi/.venv/lib/python3.12/site-packages/matplotlib/mpl-data/fonts/ttf/}]
%%   \fi
%%   \makeatletter\@ifpackageloaded{under\Score{}}{}{\usepackage[strings]{under\Score{}}}\makeatother
%%
\begingroup%
\makeatletter%
\begin{pgfpicture}%
\pgfpathrectangle{\pgfpointorigin}{\pgfqpoint{3.896542in}{2.317798in}}%
\pgfusepath{use as bounding box, clip}%
\begin{pgfscope}%
\pgfsetbuttcap%
\pgfsetmiterjoin%
\definecolor{currentfill}{rgb}{1.000000,1.000000,1.000000}%
\pgfsetfillcolor{currentfill}%
\pgfsetlinewidth{0.000000pt}%
\definecolor{currentstroke}{rgb}{1.000000,1.000000,1.000000}%
\pgfsetstrokecolor{currentstroke}%
\pgfsetdash{}{0pt}%
\pgfpathmoveto{\pgfqpoint{0.000000in}{0.000000in}}%
\pgfpathlineto{\pgfqpoint{3.896542in}{0.000000in}}%
\pgfpathlineto{\pgfqpoint{3.896542in}{2.317798in}}%
\pgfpathlineto{\pgfqpoint{0.000000in}{2.317798in}}%
\pgfpathlineto{\pgfqpoint{0.000000in}{0.000000in}}%
\pgfpathclose%
\pgfusepath{fill}%
\end{pgfscope}%
\begin{pgfscope}%
\pgfsetbuttcap%
\pgfsetmiterjoin%
\definecolor{currentfill}{rgb}{1.000000,1.000000,1.000000}%
\pgfsetfillcolor{currentfill}%
\pgfsetlinewidth{0.000000pt}%
\definecolor{currentstroke}{rgb}{0.000000,0.000000,0.000000}%
\pgfsetstrokecolor{currentstroke}%
\pgfsetstrokeopacity{0.000000}%
\pgfsetdash{}{0pt}%
\pgfpathmoveto{\pgfqpoint{0.595525in}{0.521603in}}%
\pgfpathlineto{\pgfqpoint{3.796542in}{0.521603in}}%
\pgfpathlineto{\pgfqpoint{3.796542in}{2.217798in}}%
\pgfpathlineto{\pgfqpoint{0.595525in}{2.217798in}}%
\pgfpathlineto{\pgfqpoint{0.595525in}{0.521603in}}%
\pgfpathclose%
\pgfusepath{fill}%
\end{pgfscope}%
\begin{pgfscope}%
\pgfpathrectangle{\pgfqpoint{0.595525in}{0.521603in}}{\pgfqpoint{3.201017in}{1.696195in}}%
\pgfusepath{clip}%
\pgfsetbuttcap%
\pgfsetmiterjoin%
\definecolor{currentfill}{rgb}{0.121569,0.466667,0.705882}%
\pgfsetfillcolor{currentfill}%
\pgfsetfillopacity{0.600000}%
\pgfsetlinewidth{0.000000pt}%
\definecolor{currentstroke}{rgb}{0.000000,0.000000,0.000000}%
\pgfsetstrokecolor{currentstroke}%
\pgfsetstrokeopacity{0.600000}%
\pgfsetdash{}{0pt}%
\pgfpathmoveto{\pgfqpoint{0.741025in}{0.521603in}}%
\pgfpathlineto{\pgfqpoint{0.852949in}{0.521603in}}%
\pgfpathlineto{\pgfqpoint{0.852949in}{0.973052in}}%
\pgfpathlineto{\pgfqpoint{0.741025in}{0.973052in}}%
\pgfpathlineto{\pgfqpoint{0.741025in}{0.521603in}}%
\pgfpathclose%
\pgfusepath{fill}%
\end{pgfscope}%
\begin{pgfscope}%
\pgfpathrectangle{\pgfqpoint{0.595525in}{0.521603in}}{\pgfqpoint{3.201017in}{1.696195in}}%
\pgfusepath{clip}%
\pgfsetbuttcap%
\pgfsetmiterjoin%
\definecolor{currentfill}{rgb}{0.121569,0.466667,0.705882}%
\pgfsetfillcolor{currentfill}%
\pgfsetfillopacity{0.600000}%
\pgfsetlinewidth{0.000000pt}%
\definecolor{currentstroke}{rgb}{0.000000,0.000000,0.000000}%
\pgfsetstrokecolor{currentstroke}%
\pgfsetstrokeopacity{0.600000}%
\pgfsetdash{}{0pt}%
\pgfpathmoveto{\pgfqpoint{0.852949in}{0.521603in}}%
\pgfpathlineto{\pgfqpoint{0.964873in}{0.521603in}}%
\pgfpathlineto{\pgfqpoint{0.964873in}{1.685579in}}%
\pgfpathlineto{\pgfqpoint{0.852949in}{1.685579in}}%
\pgfpathlineto{\pgfqpoint{0.852949in}{0.521603in}}%
\pgfpathclose%
\pgfusepath{fill}%
\end{pgfscope}%
\begin{pgfscope}%
\pgfpathrectangle{\pgfqpoint{0.595525in}{0.521603in}}{\pgfqpoint{3.201017in}{1.696195in}}%
\pgfusepath{clip}%
\pgfsetbuttcap%
\pgfsetmiterjoin%
\definecolor{currentfill}{rgb}{0.121569,0.466667,0.705882}%
\pgfsetfillcolor{currentfill}%
\pgfsetfillopacity{0.600000}%
\pgfsetlinewidth{0.000000pt}%
\definecolor{currentstroke}{rgb}{0.000000,0.000000,0.000000}%
\pgfsetstrokecolor{currentstroke}%
\pgfsetstrokeopacity{0.600000}%
\pgfsetdash{}{0pt}%
\pgfpathmoveto{\pgfqpoint{0.964873in}{0.521603in}}%
\pgfpathlineto{\pgfqpoint{1.076796in}{0.521603in}}%
\pgfpathlineto{\pgfqpoint{1.076796in}{1.914022in}}%
\pgfpathlineto{\pgfqpoint{0.964873in}{1.914022in}}%
\pgfpathlineto{\pgfqpoint{0.964873in}{0.521603in}}%
\pgfpathclose%
\pgfusepath{fill}%
\end{pgfscope}%
\begin{pgfscope}%
\pgfpathrectangle{\pgfqpoint{0.595525in}{0.521603in}}{\pgfqpoint{3.201017in}{1.696195in}}%
\pgfusepath{clip}%
\pgfsetbuttcap%
\pgfsetmiterjoin%
\definecolor{currentfill}{rgb}{0.121569,0.466667,0.705882}%
\pgfsetfillcolor{currentfill}%
\pgfsetfillopacity{0.600000}%
\pgfsetlinewidth{0.000000pt}%
\definecolor{currentstroke}{rgb}{0.000000,0.000000,0.000000}%
\pgfsetstrokecolor{currentstroke}%
\pgfsetstrokeopacity{0.600000}%
\pgfsetdash{}{0pt}%
\pgfpathmoveto{\pgfqpoint{1.076796in}{0.521603in}}%
\pgfpathlineto{\pgfqpoint{1.188720in}{0.521603in}}%
\pgfpathlineto{\pgfqpoint{1.188720in}{1.941218in}}%
\pgfpathlineto{\pgfqpoint{1.076796in}{1.941218in}}%
\pgfpathlineto{\pgfqpoint{1.076796in}{0.521603in}}%
\pgfpathclose%
\pgfusepath{fill}%
\end{pgfscope}%
\begin{pgfscope}%
\pgfpathrectangle{\pgfqpoint{0.595525in}{0.521603in}}{\pgfqpoint{3.201017in}{1.696195in}}%
\pgfusepath{clip}%
\pgfsetbuttcap%
\pgfsetmiterjoin%
\definecolor{currentfill}{rgb}{0.121569,0.466667,0.705882}%
\pgfsetfillcolor{currentfill}%
\pgfsetfillopacity{0.600000}%
\pgfsetlinewidth{0.000000pt}%
\definecolor{currentstroke}{rgb}{0.000000,0.000000,0.000000}%
\pgfsetstrokecolor{currentstroke}%
\pgfsetstrokeopacity{0.600000}%
\pgfsetdash{}{0pt}%
\pgfpathmoveto{\pgfqpoint{1.188720in}{0.521603in}}%
\pgfpathlineto{\pgfqpoint{1.300644in}{0.521603in}}%
\pgfpathlineto{\pgfqpoint{1.300644in}{2.093514in}}%
\pgfpathlineto{\pgfqpoint{1.188720in}{2.093514in}}%
\pgfpathlineto{\pgfqpoint{1.188720in}{0.521603in}}%
\pgfpathclose%
\pgfusepath{fill}%
\end{pgfscope}%
\begin{pgfscope}%
\pgfpathrectangle{\pgfqpoint{0.595525in}{0.521603in}}{\pgfqpoint{3.201017in}{1.696195in}}%
\pgfusepath{clip}%
\pgfsetbuttcap%
\pgfsetmiterjoin%
\definecolor{currentfill}{rgb}{0.121569,0.466667,0.705882}%
\pgfsetfillcolor{currentfill}%
\pgfsetfillopacity{0.600000}%
\pgfsetlinewidth{0.000000pt}%
\definecolor{currentstroke}{rgb}{0.000000,0.000000,0.000000}%
\pgfsetstrokecolor{currentstroke}%
\pgfsetstrokeopacity{0.600000}%
\pgfsetdash{}{0pt}%
\pgfpathmoveto{\pgfqpoint{1.300644in}{0.521603in}}%
\pgfpathlineto{\pgfqpoint{1.412567in}{0.521603in}}%
\pgfpathlineto{\pgfqpoint{1.412567in}{1.593113in}}%
\pgfpathlineto{\pgfqpoint{1.300644in}{1.593113in}}%
\pgfpathlineto{\pgfqpoint{1.300644in}{0.521603in}}%
\pgfpathclose%
\pgfusepath{fill}%
\end{pgfscope}%
\begin{pgfscope}%
\pgfpathrectangle{\pgfqpoint{0.595525in}{0.521603in}}{\pgfqpoint{3.201017in}{1.696195in}}%
\pgfusepath{clip}%
\pgfsetbuttcap%
\pgfsetmiterjoin%
\definecolor{currentfill}{rgb}{0.121569,0.466667,0.705882}%
\pgfsetfillcolor{currentfill}%
\pgfsetfillopacity{0.600000}%
\pgfsetlinewidth{0.000000pt}%
\definecolor{currentstroke}{rgb}{0.000000,0.000000,0.000000}%
\pgfsetstrokecolor{currentstroke}%
\pgfsetstrokeopacity{0.600000}%
\pgfsetdash{}{0pt}%
\pgfpathmoveto{\pgfqpoint{1.412567in}{0.521603in}}%
\pgfpathlineto{\pgfqpoint{1.524491in}{0.521603in}}%
\pgfpathlineto{\pgfqpoint{1.524491in}{1.326596in}}%
\pgfpathlineto{\pgfqpoint{1.412567in}{1.326596in}}%
\pgfpathlineto{\pgfqpoint{1.412567in}{0.521603in}}%
\pgfpathclose%
\pgfusepath{fill}%
\end{pgfscope}%
\begin{pgfscope}%
\pgfpathrectangle{\pgfqpoint{0.595525in}{0.521603in}}{\pgfqpoint{3.201017in}{1.696195in}}%
\pgfusepath{clip}%
\pgfsetbuttcap%
\pgfsetmiterjoin%
\definecolor{currentfill}{rgb}{0.121569,0.466667,0.705882}%
\pgfsetfillcolor{currentfill}%
\pgfsetfillopacity{0.600000}%
\pgfsetlinewidth{0.000000pt}%
\definecolor{currentstroke}{rgb}{0.000000,0.000000,0.000000}%
\pgfsetstrokecolor{currentstroke}%
\pgfsetstrokeopacity{0.600000}%
\pgfsetdash{}{0pt}%
\pgfpathmoveto{\pgfqpoint{1.524491in}{0.521603in}}%
\pgfpathlineto{\pgfqpoint{1.636415in}{0.521603in}}%
\pgfpathlineto{\pgfqpoint{1.636415in}{1.315717in}}%
\pgfpathlineto{\pgfqpoint{1.524491in}{1.315717in}}%
\pgfpathlineto{\pgfqpoint{1.524491in}{0.521603in}}%
\pgfpathclose%
\pgfusepath{fill}%
\end{pgfscope}%
\begin{pgfscope}%
\pgfpathrectangle{\pgfqpoint{0.595525in}{0.521603in}}{\pgfqpoint{3.201017in}{1.696195in}}%
\pgfusepath{clip}%
\pgfsetbuttcap%
\pgfsetmiterjoin%
\definecolor{currentfill}{rgb}{0.121569,0.466667,0.705882}%
\pgfsetfillcolor{currentfill}%
\pgfsetfillopacity{0.600000}%
\pgfsetlinewidth{0.000000pt}%
\definecolor{currentstroke}{rgb}{0.000000,0.000000,0.000000}%
\pgfsetstrokecolor{currentstroke}%
\pgfsetstrokeopacity{0.600000}%
\pgfsetdash{}{0pt}%
\pgfpathmoveto{\pgfqpoint{1.636415in}{0.521603in}}%
\pgfpathlineto{\pgfqpoint{1.748338in}{0.521603in}}%
\pgfpathlineto{\pgfqpoint{1.748338in}{1.288522in}}%
\pgfpathlineto{\pgfqpoint{1.636415in}{1.288522in}}%
\pgfpathlineto{\pgfqpoint{1.636415in}{0.521603in}}%
\pgfpathclose%
\pgfusepath{fill}%
\end{pgfscope}%
\begin{pgfscope}%
\pgfpathrectangle{\pgfqpoint{0.595525in}{0.521603in}}{\pgfqpoint{3.201017in}{1.696195in}}%
\pgfusepath{clip}%
\pgfsetbuttcap%
\pgfsetmiterjoin%
\definecolor{currentfill}{rgb}{0.121569,0.466667,0.705882}%
\pgfsetfillcolor{currentfill}%
\pgfsetfillopacity{0.600000}%
\pgfsetlinewidth{0.000000pt}%
\definecolor{currentstroke}{rgb}{0.000000,0.000000,0.000000}%
\pgfsetstrokecolor{currentstroke}%
\pgfsetstrokeopacity{0.600000}%
\pgfsetdash{}{0pt}%
\pgfpathmoveto{\pgfqpoint{1.748338in}{0.521603in}}%
\pgfpathlineto{\pgfqpoint{1.860262in}{0.521603in}}%
\pgfpathlineto{\pgfqpoint{1.860262in}{1.130787in}}%
\pgfpathlineto{\pgfqpoint{1.748338in}{1.130787in}}%
\pgfpathlineto{\pgfqpoint{1.748338in}{0.521603in}}%
\pgfpathclose%
\pgfusepath{fill}%
\end{pgfscope}%
\begin{pgfscope}%
\pgfpathrectangle{\pgfqpoint{0.595525in}{0.521603in}}{\pgfqpoint{3.201017in}{1.696195in}}%
\pgfusepath{clip}%
\pgfsetbuttcap%
\pgfsetmiterjoin%
\definecolor{currentfill}{rgb}{0.121569,0.466667,0.705882}%
\pgfsetfillcolor{currentfill}%
\pgfsetfillopacity{0.600000}%
\pgfsetlinewidth{0.000000pt}%
\definecolor{currentstroke}{rgb}{0.000000,0.000000,0.000000}%
\pgfsetstrokecolor{currentstroke}%
\pgfsetstrokeopacity{0.600000}%
\pgfsetdash{}{0pt}%
\pgfpathmoveto{\pgfqpoint{1.860262in}{0.521603in}}%
\pgfpathlineto{\pgfqpoint{1.972186in}{0.521603in}}%
\pgfpathlineto{\pgfqpoint{1.972186in}{0.994808in}}%
\pgfpathlineto{\pgfqpoint{1.860262in}{0.994808in}}%
\pgfpathlineto{\pgfqpoint{1.860262in}{0.521603in}}%
\pgfpathclose%
\pgfusepath{fill}%
\end{pgfscope}%
\begin{pgfscope}%
\pgfpathrectangle{\pgfqpoint{0.595525in}{0.521603in}}{\pgfqpoint{3.201017in}{1.696195in}}%
\pgfusepath{clip}%
\pgfsetbuttcap%
\pgfsetmiterjoin%
\definecolor{currentfill}{rgb}{0.121569,0.466667,0.705882}%
\pgfsetfillcolor{currentfill}%
\pgfsetfillopacity{0.600000}%
\pgfsetlinewidth{0.000000pt}%
\definecolor{currentstroke}{rgb}{0.000000,0.000000,0.000000}%
\pgfsetstrokecolor{currentstroke}%
\pgfsetstrokeopacity{0.600000}%
\pgfsetdash{}{0pt}%
\pgfpathmoveto{\pgfqpoint{1.972186in}{0.521603in}}%
\pgfpathlineto{\pgfqpoint{2.084109in}{0.521603in}}%
\pgfpathlineto{\pgfqpoint{2.084109in}{1.011126in}}%
\pgfpathlineto{\pgfqpoint{1.972186in}{1.011126in}}%
\pgfpathlineto{\pgfqpoint{1.972186in}{0.521603in}}%
\pgfpathclose%
\pgfusepath{fill}%
\end{pgfscope}%
\begin{pgfscope}%
\pgfpathrectangle{\pgfqpoint{0.595525in}{0.521603in}}{\pgfqpoint{3.201017in}{1.696195in}}%
\pgfusepath{clip}%
\pgfsetbuttcap%
\pgfsetmiterjoin%
\definecolor{currentfill}{rgb}{0.121569,0.466667,0.705882}%
\pgfsetfillcolor{currentfill}%
\pgfsetfillopacity{0.600000}%
\pgfsetlinewidth{0.000000pt}%
\definecolor{currentstroke}{rgb}{0.000000,0.000000,0.000000}%
\pgfsetstrokecolor{currentstroke}%
\pgfsetstrokeopacity{0.600000}%
\pgfsetdash{}{0pt}%
\pgfpathmoveto{\pgfqpoint{2.084109in}{0.521603in}}%
\pgfpathlineto{\pgfqpoint{2.196033in}{0.521603in}}%
\pgfpathlineto{\pgfqpoint{2.196033in}{0.880586in}}%
\pgfpathlineto{\pgfqpoint{2.084109in}{0.880586in}}%
\pgfpathlineto{\pgfqpoint{2.084109in}{0.521603in}}%
\pgfpathclose%
\pgfusepath{fill}%
\end{pgfscope}%
\begin{pgfscope}%
\pgfpathrectangle{\pgfqpoint{0.595525in}{0.521603in}}{\pgfqpoint{3.201017in}{1.696195in}}%
\pgfusepath{clip}%
\pgfsetbuttcap%
\pgfsetmiterjoin%
\definecolor{currentfill}{rgb}{0.121569,0.466667,0.705882}%
\pgfsetfillcolor{currentfill}%
\pgfsetfillopacity{0.600000}%
\pgfsetlinewidth{0.000000pt}%
\definecolor{currentstroke}{rgb}{0.000000,0.000000,0.000000}%
\pgfsetstrokecolor{currentstroke}%
\pgfsetstrokeopacity{0.600000}%
\pgfsetdash{}{0pt}%
\pgfpathmoveto{\pgfqpoint{2.196033in}{0.521603in}}%
\pgfpathlineto{\pgfqpoint{2.307957in}{0.521603in}}%
\pgfpathlineto{\pgfqpoint{2.307957in}{0.864269in}}%
\pgfpathlineto{\pgfqpoint{2.196033in}{0.864269in}}%
\pgfpathlineto{\pgfqpoint{2.196033in}{0.521603in}}%
\pgfpathclose%
\pgfusepath{fill}%
\end{pgfscope}%
\begin{pgfscope}%
\pgfpathrectangle{\pgfqpoint{0.595525in}{0.521603in}}{\pgfqpoint{3.201017in}{1.696195in}}%
\pgfusepath{clip}%
\pgfsetbuttcap%
\pgfsetmiterjoin%
\definecolor{currentfill}{rgb}{0.121569,0.466667,0.705882}%
\pgfsetfillcolor{currentfill}%
\pgfsetfillopacity{0.600000}%
\pgfsetlinewidth{0.000000pt}%
\definecolor{currentstroke}{rgb}{0.000000,0.000000,0.000000}%
\pgfsetstrokecolor{currentstroke}%
\pgfsetstrokeopacity{0.600000}%
\pgfsetdash{}{0pt}%
\pgfpathmoveto{\pgfqpoint{2.307957in}{0.521603in}}%
\pgfpathlineto{\pgfqpoint{2.419880in}{0.521603in}}%
\pgfpathlineto{\pgfqpoint{2.419880in}{0.842512in}}%
\pgfpathlineto{\pgfqpoint{2.307957in}{0.842512in}}%
\pgfpathlineto{\pgfqpoint{2.307957in}{0.521603in}}%
\pgfpathclose%
\pgfusepath{fill}%
\end{pgfscope}%
\begin{pgfscope}%
\pgfpathrectangle{\pgfqpoint{0.595525in}{0.521603in}}{\pgfqpoint{3.201017in}{1.696195in}}%
\pgfusepath{clip}%
\pgfsetbuttcap%
\pgfsetmiterjoin%
\definecolor{currentfill}{rgb}{0.121569,0.466667,0.705882}%
\pgfsetfillcolor{currentfill}%
\pgfsetfillopacity{0.600000}%
\pgfsetlinewidth{0.000000pt}%
\definecolor{currentstroke}{rgb}{0.000000,0.000000,0.000000}%
\pgfsetstrokecolor{currentstroke}%
\pgfsetstrokeopacity{0.600000}%
\pgfsetdash{}{0pt}%
\pgfpathmoveto{\pgfqpoint{2.419880in}{0.521603in}}%
\pgfpathlineto{\pgfqpoint{2.531804in}{0.521603in}}%
\pgfpathlineto{\pgfqpoint{2.531804in}{0.739169in}}%
\pgfpathlineto{\pgfqpoint{2.419880in}{0.739169in}}%
\pgfpathlineto{\pgfqpoint{2.419880in}{0.521603in}}%
\pgfpathclose%
\pgfusepath{fill}%
\end{pgfscope}%
\begin{pgfscope}%
\pgfpathrectangle{\pgfqpoint{0.595525in}{0.521603in}}{\pgfqpoint{3.201017in}{1.696195in}}%
\pgfusepath{clip}%
\pgfsetbuttcap%
\pgfsetmiterjoin%
\definecolor{currentfill}{rgb}{0.121569,0.466667,0.705882}%
\pgfsetfillcolor{currentfill}%
\pgfsetfillopacity{0.600000}%
\pgfsetlinewidth{0.000000pt}%
\definecolor{currentstroke}{rgb}{0.000000,0.000000,0.000000}%
\pgfsetstrokecolor{currentstroke}%
\pgfsetstrokeopacity{0.600000}%
\pgfsetdash{}{0pt}%
\pgfpathmoveto{\pgfqpoint{2.531804in}{0.521603in}}%
\pgfpathlineto{\pgfqpoint{2.643728in}{0.521603in}}%
\pgfpathlineto{\pgfqpoint{2.643728in}{0.690217in}}%
\pgfpathlineto{\pgfqpoint{2.531804in}{0.690217in}}%
\pgfpathlineto{\pgfqpoint{2.531804in}{0.521603in}}%
\pgfpathclose%
\pgfusepath{fill}%
\end{pgfscope}%
\begin{pgfscope}%
\pgfpathrectangle{\pgfqpoint{0.595525in}{0.521603in}}{\pgfqpoint{3.201017in}{1.696195in}}%
\pgfusepath{clip}%
\pgfsetbuttcap%
\pgfsetmiterjoin%
\definecolor{currentfill}{rgb}{0.121569,0.466667,0.705882}%
\pgfsetfillcolor{currentfill}%
\pgfsetfillopacity{0.600000}%
\pgfsetlinewidth{0.000000pt}%
\definecolor{currentstroke}{rgb}{0.000000,0.000000,0.000000}%
\pgfsetstrokecolor{currentstroke}%
\pgfsetstrokeopacity{0.600000}%
\pgfsetdash{}{0pt}%
\pgfpathmoveto{\pgfqpoint{2.643728in}{0.521603in}}%
\pgfpathlineto{\pgfqpoint{2.755651in}{0.521603in}}%
\pgfpathlineto{\pgfqpoint{2.755651in}{0.717412in}}%
\pgfpathlineto{\pgfqpoint{2.643728in}{0.717412in}}%
\pgfpathlineto{\pgfqpoint{2.643728in}{0.521603in}}%
\pgfpathclose%
\pgfusepath{fill}%
\end{pgfscope}%
\begin{pgfscope}%
\pgfpathrectangle{\pgfqpoint{0.595525in}{0.521603in}}{\pgfqpoint{3.201017in}{1.696195in}}%
\pgfusepath{clip}%
\pgfsetbuttcap%
\pgfsetmiterjoin%
\definecolor{currentfill}{rgb}{0.121569,0.466667,0.705882}%
\pgfsetfillcolor{currentfill}%
\pgfsetfillopacity{0.600000}%
\pgfsetlinewidth{0.000000pt}%
\definecolor{currentstroke}{rgb}{0.000000,0.000000,0.000000}%
\pgfsetstrokecolor{currentstroke}%
\pgfsetstrokeopacity{0.600000}%
\pgfsetdash{}{0pt}%
\pgfpathmoveto{\pgfqpoint{2.755651in}{0.521603in}}%
\pgfpathlineto{\pgfqpoint{2.867575in}{0.521603in}}%
\pgfpathlineto{\pgfqpoint{2.867575in}{0.701095in}}%
\pgfpathlineto{\pgfqpoint{2.755651in}{0.701095in}}%
\pgfpathlineto{\pgfqpoint{2.755651in}{0.521603in}}%
\pgfpathclose%
\pgfusepath{fill}%
\end{pgfscope}%
\begin{pgfscope}%
\pgfpathrectangle{\pgfqpoint{0.595525in}{0.521603in}}{\pgfqpoint{3.201017in}{1.696195in}}%
\pgfusepath{clip}%
\pgfsetbuttcap%
\pgfsetmiterjoin%
\definecolor{currentfill}{rgb}{0.121569,0.466667,0.705882}%
\pgfsetfillcolor{currentfill}%
\pgfsetfillopacity{0.600000}%
\pgfsetlinewidth{0.000000pt}%
\definecolor{currentstroke}{rgb}{0.000000,0.000000,0.000000}%
\pgfsetstrokecolor{currentstroke}%
\pgfsetstrokeopacity{0.600000}%
\pgfsetdash{}{0pt}%
\pgfpathmoveto{\pgfqpoint{2.867575in}{0.521603in}}%
\pgfpathlineto{\pgfqpoint{2.979499in}{0.521603in}}%
\pgfpathlineto{\pgfqpoint{2.979499in}{0.673899in}}%
\pgfpathlineto{\pgfqpoint{2.867575in}{0.673899in}}%
\pgfpathlineto{\pgfqpoint{2.867575in}{0.521603in}}%
\pgfpathclose%
\pgfusepath{fill}%
\end{pgfscope}%
\begin{pgfscope}%
\pgfpathrectangle{\pgfqpoint{0.595525in}{0.521603in}}{\pgfqpoint{3.201017in}{1.696195in}}%
\pgfusepath{clip}%
\pgfsetbuttcap%
\pgfsetmiterjoin%
\definecolor{currentfill}{rgb}{0.121569,0.466667,0.705882}%
\pgfsetfillcolor{currentfill}%
\pgfsetfillopacity{0.600000}%
\pgfsetlinewidth{0.000000pt}%
\definecolor{currentstroke}{rgb}{0.000000,0.000000,0.000000}%
\pgfsetstrokecolor{currentstroke}%
\pgfsetstrokeopacity{0.600000}%
\pgfsetdash{}{0pt}%
\pgfpathmoveto{\pgfqpoint{2.979499in}{0.521603in}}%
\pgfpathlineto{\pgfqpoint{3.091422in}{0.521603in}}%
\pgfpathlineto{\pgfqpoint{3.091422in}{0.690217in}}%
\pgfpathlineto{\pgfqpoint{2.979499in}{0.690217in}}%
\pgfpathlineto{\pgfqpoint{2.979499in}{0.521603in}}%
\pgfpathclose%
\pgfusepath{fill}%
\end{pgfscope}%
\begin{pgfscope}%
\pgfpathrectangle{\pgfqpoint{0.595525in}{0.521603in}}{\pgfqpoint{3.201017in}{1.696195in}}%
\pgfusepath{clip}%
\pgfsetbuttcap%
\pgfsetmiterjoin%
\definecolor{currentfill}{rgb}{0.121569,0.466667,0.705882}%
\pgfsetfillcolor{currentfill}%
\pgfsetfillopacity{0.600000}%
\pgfsetlinewidth{0.000000pt}%
\definecolor{currentstroke}{rgb}{0.000000,0.000000,0.000000}%
\pgfsetstrokecolor{currentstroke}%
\pgfsetstrokeopacity{0.600000}%
\pgfsetdash{}{0pt}%
\pgfpathmoveto{\pgfqpoint{3.091422in}{0.521603in}}%
\pgfpathlineto{\pgfqpoint{3.203346in}{0.521603in}}%
\pgfpathlineto{\pgfqpoint{3.203346in}{0.586873in}}%
\pgfpathlineto{\pgfqpoint{3.091422in}{0.586873in}}%
\pgfpathlineto{\pgfqpoint{3.091422in}{0.521603in}}%
\pgfpathclose%
\pgfusepath{fill}%
\end{pgfscope}%
\begin{pgfscope}%
\pgfpathrectangle{\pgfqpoint{0.595525in}{0.521603in}}{\pgfqpoint{3.201017in}{1.696195in}}%
\pgfusepath{clip}%
\pgfsetbuttcap%
\pgfsetmiterjoin%
\definecolor{currentfill}{rgb}{0.121569,0.466667,0.705882}%
\pgfsetfillcolor{currentfill}%
\pgfsetfillopacity{0.600000}%
\pgfsetlinewidth{0.000000pt}%
\definecolor{currentstroke}{rgb}{0.000000,0.000000,0.000000}%
\pgfsetstrokecolor{currentstroke}%
\pgfsetstrokeopacity{0.600000}%
\pgfsetdash{}{0pt}%
\pgfpathmoveto{\pgfqpoint{3.203346in}{0.521603in}}%
\pgfpathlineto{\pgfqpoint{3.315270in}{0.521603in}}%
\pgfpathlineto{\pgfqpoint{3.315270in}{0.614069in}}%
\pgfpathlineto{\pgfqpoint{3.203346in}{0.614069in}}%
\pgfpathlineto{\pgfqpoint{3.203346in}{0.521603in}}%
\pgfpathclose%
\pgfusepath{fill}%
\end{pgfscope}%
\begin{pgfscope}%
\pgfpathrectangle{\pgfqpoint{0.595525in}{0.521603in}}{\pgfqpoint{3.201017in}{1.696195in}}%
\pgfusepath{clip}%
\pgfsetbuttcap%
\pgfsetmiterjoin%
\definecolor{currentfill}{rgb}{0.121569,0.466667,0.705882}%
\pgfsetfillcolor{currentfill}%
\pgfsetfillopacity{0.600000}%
\pgfsetlinewidth{0.000000pt}%
\definecolor{currentstroke}{rgb}{0.000000,0.000000,0.000000}%
\pgfsetstrokecolor{currentstroke}%
\pgfsetstrokeopacity{0.600000}%
\pgfsetdash{}{0pt}%
\pgfpathmoveto{\pgfqpoint{3.315270in}{0.521603in}}%
\pgfpathlineto{\pgfqpoint{3.427193in}{0.521603in}}%
\pgfpathlineto{\pgfqpoint{3.427193in}{0.614069in}}%
\pgfpathlineto{\pgfqpoint{3.315270in}{0.614069in}}%
\pgfpathlineto{\pgfqpoint{3.315270in}{0.521603in}}%
\pgfpathclose%
\pgfusepath{fill}%
\end{pgfscope}%
\begin{pgfscope}%
\pgfpathrectangle{\pgfqpoint{0.595525in}{0.521603in}}{\pgfqpoint{3.201017in}{1.696195in}}%
\pgfusepath{clip}%
\pgfsetbuttcap%
\pgfsetmiterjoin%
\definecolor{currentfill}{rgb}{0.121569,0.466667,0.705882}%
\pgfsetfillcolor{currentfill}%
\pgfsetfillopacity{0.600000}%
\pgfsetlinewidth{0.000000pt}%
\definecolor{currentstroke}{rgb}{0.000000,0.000000,0.000000}%
\pgfsetstrokecolor{currentstroke}%
\pgfsetstrokeopacity{0.600000}%
\pgfsetdash{}{0pt}%
\pgfpathmoveto{\pgfqpoint{3.427193in}{0.521603in}}%
\pgfpathlineto{\pgfqpoint{3.539117in}{0.521603in}}%
\pgfpathlineto{\pgfqpoint{3.539117in}{0.586873in}}%
\pgfpathlineto{\pgfqpoint{3.427193in}{0.586873in}}%
\pgfpathlineto{\pgfqpoint{3.427193in}{0.521603in}}%
\pgfpathclose%
\pgfusepath{fill}%
\end{pgfscope}%
\begin{pgfscope}%
\pgfpathrectangle{\pgfqpoint{0.595525in}{0.521603in}}{\pgfqpoint{3.201017in}{1.696195in}}%
\pgfusepath{clip}%
\pgfsetbuttcap%
\pgfsetmiterjoin%
\definecolor{currentfill}{rgb}{0.121569,0.466667,0.705882}%
\pgfsetfillcolor{currentfill}%
\pgfsetfillopacity{0.600000}%
\pgfsetlinewidth{0.000000pt}%
\definecolor{currentstroke}{rgb}{0.000000,0.000000,0.000000}%
\pgfsetstrokecolor{currentstroke}%
\pgfsetstrokeopacity{0.600000}%
\pgfsetdash{}{0pt}%
\pgfpathmoveto{\pgfqpoint{3.539117in}{0.521603in}}%
\pgfpathlineto{\pgfqpoint{3.651041in}{0.521603in}}%
\pgfpathlineto{\pgfqpoint{3.651041in}{0.543360in}}%
\pgfpathlineto{\pgfqpoint{3.539117in}{0.543360in}}%
\pgfpathlineto{\pgfqpoint{3.539117in}{0.521603in}}%
\pgfpathclose%
\pgfusepath{fill}%
\end{pgfscope}%
\begin{pgfscope}%
\pgfpathrectangle{\pgfqpoint{0.595525in}{0.521603in}}{\pgfqpoint{3.201017in}{1.696195in}}%
\pgfusepath{clip}%
\pgfsetbuttcap%
\pgfsetmiterjoin%
\definecolor{currentfill}{rgb}{1.000000,0.498039,0.054902}%
\pgfsetfillcolor{currentfill}%
\pgfsetfillopacity{0.600000}%
\pgfsetlinewidth{0.000000pt}%
\definecolor{currentstroke}{rgb}{0.000000,0.000000,0.000000}%
\pgfsetstrokecolor{currentstroke}%
\pgfsetstrokeopacity{0.600000}%
\pgfsetdash{}{0pt}%
\pgfpathmoveto{\pgfqpoint{0.769278in}{0.521603in}}%
\pgfpathlineto{\pgfqpoint{0.880115in}{0.521603in}}%
\pgfpathlineto{\pgfqpoint{0.880115in}{0.755486in}}%
\pgfpathlineto{\pgfqpoint{0.769278in}{0.755486in}}%
\pgfpathlineto{\pgfqpoint{0.769278in}{0.521603in}}%
\pgfpathclose%
\pgfusepath{fill}%
\end{pgfscope}%
\begin{pgfscope}%
\pgfpathrectangle{\pgfqpoint{0.595525in}{0.521603in}}{\pgfqpoint{3.201017in}{1.696195in}}%
\pgfusepath{clip}%
\pgfsetbuttcap%
\pgfsetmiterjoin%
\definecolor{currentfill}{rgb}{1.000000,0.498039,0.054902}%
\pgfsetfillcolor{currentfill}%
\pgfsetfillopacity{0.600000}%
\pgfsetlinewidth{0.000000pt}%
\definecolor{currentstroke}{rgb}{0.000000,0.000000,0.000000}%
\pgfsetstrokecolor{currentstroke}%
\pgfsetstrokeopacity{0.600000}%
\pgfsetdash{}{0pt}%
\pgfpathmoveto{\pgfqpoint{0.880115in}{0.521603in}}%
\pgfpathlineto{\pgfqpoint{0.990952in}{0.521603in}}%
\pgfpathlineto{\pgfqpoint{0.990952in}{1.593113in}}%
\pgfpathlineto{\pgfqpoint{0.880115in}{1.593113in}}%
\pgfpathlineto{\pgfqpoint{0.880115in}{0.521603in}}%
\pgfpathclose%
\pgfusepath{fill}%
\end{pgfscope}%
\begin{pgfscope}%
\pgfpathrectangle{\pgfqpoint{0.595525in}{0.521603in}}{\pgfqpoint{3.201017in}{1.696195in}}%
\pgfusepath{clip}%
\pgfsetbuttcap%
\pgfsetmiterjoin%
\definecolor{currentfill}{rgb}{1.000000,0.498039,0.054902}%
\pgfsetfillcolor{currentfill}%
\pgfsetfillopacity{0.600000}%
\pgfsetlinewidth{0.000000pt}%
\definecolor{currentstroke}{rgb}{0.000000,0.000000,0.000000}%
\pgfsetstrokecolor{currentstroke}%
\pgfsetstrokeopacity{0.600000}%
\pgfsetdash{}{0pt}%
\pgfpathmoveto{\pgfqpoint{0.990952in}{0.521603in}}%
\pgfpathlineto{\pgfqpoint{1.101789in}{0.521603in}}%
\pgfpathlineto{\pgfqpoint{1.101789in}{2.006488in}}%
\pgfpathlineto{\pgfqpoint{0.990952in}{2.006488in}}%
\pgfpathlineto{\pgfqpoint{0.990952in}{0.521603in}}%
\pgfpathclose%
\pgfusepath{fill}%
\end{pgfscope}%
\begin{pgfscope}%
\pgfpathrectangle{\pgfqpoint{0.595525in}{0.521603in}}{\pgfqpoint{3.201017in}{1.696195in}}%
\pgfusepath{clip}%
\pgfsetbuttcap%
\pgfsetmiterjoin%
\definecolor{currentfill}{rgb}{1.000000,0.498039,0.054902}%
\pgfsetfillcolor{currentfill}%
\pgfsetfillopacity{0.600000}%
\pgfsetlinewidth{0.000000pt}%
\definecolor{currentstroke}{rgb}{0.000000,0.000000,0.000000}%
\pgfsetstrokecolor{currentstroke}%
\pgfsetstrokeopacity{0.600000}%
\pgfsetdash{}{0pt}%
\pgfpathmoveto{\pgfqpoint{1.101789in}{0.521603in}}%
\pgfpathlineto{\pgfqpoint{1.212626in}{0.521603in}}%
\pgfpathlineto{\pgfqpoint{1.212626in}{1.952096in}}%
\pgfpathlineto{\pgfqpoint{1.101789in}{1.952096in}}%
\pgfpathlineto{\pgfqpoint{1.101789in}{0.521603in}}%
\pgfpathclose%
\pgfusepath{fill}%
\end{pgfscope}%
\begin{pgfscope}%
\pgfpathrectangle{\pgfqpoint{0.595525in}{0.521603in}}{\pgfqpoint{3.201017in}{1.696195in}}%
\pgfusepath{clip}%
\pgfsetbuttcap%
\pgfsetmiterjoin%
\definecolor{currentfill}{rgb}{1.000000,0.498039,0.054902}%
\pgfsetfillcolor{currentfill}%
\pgfsetfillopacity{0.600000}%
\pgfsetlinewidth{0.000000pt}%
\definecolor{currentstroke}{rgb}{0.000000,0.000000,0.000000}%
\pgfsetstrokecolor{currentstroke}%
\pgfsetstrokeopacity{0.600000}%
\pgfsetdash{}{0pt}%
\pgfpathmoveto{\pgfqpoint{1.212626in}{0.521603in}}%
\pgfpathlineto{\pgfqpoint{1.323463in}{0.521603in}}%
\pgfpathlineto{\pgfqpoint{1.323463in}{2.137027in}}%
\pgfpathlineto{\pgfqpoint{1.212626in}{2.137027in}}%
\pgfpathlineto{\pgfqpoint{1.212626in}{0.521603in}}%
\pgfpathclose%
\pgfusepath{fill}%
\end{pgfscope}%
\begin{pgfscope}%
\pgfpathrectangle{\pgfqpoint{0.595525in}{0.521603in}}{\pgfqpoint{3.201017in}{1.696195in}}%
\pgfusepath{clip}%
\pgfsetbuttcap%
\pgfsetmiterjoin%
\definecolor{currentfill}{rgb}{1.000000,0.498039,0.054902}%
\pgfsetfillcolor{currentfill}%
\pgfsetfillopacity{0.600000}%
\pgfsetlinewidth{0.000000pt}%
\definecolor{currentstroke}{rgb}{0.000000,0.000000,0.000000}%
\pgfsetstrokecolor{currentstroke}%
\pgfsetstrokeopacity{0.600000}%
\pgfsetdash{}{0pt}%
\pgfpathmoveto{\pgfqpoint{1.323463in}{0.521603in}}%
\pgfpathlineto{\pgfqpoint{1.434300in}{0.521603in}}%
\pgfpathlineto{\pgfqpoint{1.434300in}{1.652944in}}%
\pgfpathlineto{\pgfqpoint{1.323463in}{1.652944in}}%
\pgfpathlineto{\pgfqpoint{1.323463in}{0.521603in}}%
\pgfpathclose%
\pgfusepath{fill}%
\end{pgfscope}%
\begin{pgfscope}%
\pgfpathrectangle{\pgfqpoint{0.595525in}{0.521603in}}{\pgfqpoint{3.201017in}{1.696195in}}%
\pgfusepath{clip}%
\pgfsetbuttcap%
\pgfsetmiterjoin%
\definecolor{currentfill}{rgb}{1.000000,0.498039,0.054902}%
\pgfsetfillcolor{currentfill}%
\pgfsetfillopacity{0.600000}%
\pgfsetlinewidth{0.000000pt}%
\definecolor{currentstroke}{rgb}{0.000000,0.000000,0.000000}%
\pgfsetstrokecolor{currentstroke}%
\pgfsetstrokeopacity{0.600000}%
\pgfsetdash{}{0pt}%
\pgfpathmoveto{\pgfqpoint{1.434300in}{0.521603in}}%
\pgfpathlineto{\pgfqpoint{1.545137in}{0.521603in}}%
\pgfpathlineto{\pgfqpoint{1.545137in}{1.391865in}}%
\pgfpathlineto{\pgfqpoint{1.434300in}{1.391865in}}%
\pgfpathlineto{\pgfqpoint{1.434300in}{0.521603in}}%
\pgfpathclose%
\pgfusepath{fill}%
\end{pgfscope}%
\begin{pgfscope}%
\pgfpathrectangle{\pgfqpoint{0.595525in}{0.521603in}}{\pgfqpoint{3.201017in}{1.696195in}}%
\pgfusepath{clip}%
\pgfsetbuttcap%
\pgfsetmiterjoin%
\definecolor{currentfill}{rgb}{1.000000,0.498039,0.054902}%
\pgfsetfillcolor{currentfill}%
\pgfsetfillopacity{0.600000}%
\pgfsetlinewidth{0.000000pt}%
\definecolor{currentstroke}{rgb}{0.000000,0.000000,0.000000}%
\pgfsetstrokecolor{currentstroke}%
\pgfsetstrokeopacity{0.600000}%
\pgfsetdash{}{0pt}%
\pgfpathmoveto{\pgfqpoint{1.545137in}{0.521603in}}%
\pgfpathlineto{\pgfqpoint{1.655974in}{0.521603in}}%
\pgfpathlineto{\pgfqpoint{1.655974in}{1.304839in}}%
\pgfpathlineto{\pgfqpoint{1.545137in}{1.304839in}}%
\pgfpathlineto{\pgfqpoint{1.545137in}{0.521603in}}%
\pgfpathclose%
\pgfusepath{fill}%
\end{pgfscope}%
\begin{pgfscope}%
\pgfpathrectangle{\pgfqpoint{0.595525in}{0.521603in}}{\pgfqpoint{3.201017in}{1.696195in}}%
\pgfusepath{clip}%
\pgfsetbuttcap%
\pgfsetmiterjoin%
\definecolor{currentfill}{rgb}{1.000000,0.498039,0.054902}%
\pgfsetfillcolor{currentfill}%
\pgfsetfillopacity{0.600000}%
\pgfsetlinewidth{0.000000pt}%
\definecolor{currentstroke}{rgb}{0.000000,0.000000,0.000000}%
\pgfsetstrokecolor{currentstroke}%
\pgfsetstrokeopacity{0.600000}%
\pgfsetdash{}{0pt}%
\pgfpathmoveto{\pgfqpoint{1.655974in}{0.521603in}}%
\pgfpathlineto{\pgfqpoint{1.766811in}{0.521603in}}%
\pgfpathlineto{\pgfqpoint{1.766811in}{1.299400in}}%
\pgfpathlineto{\pgfqpoint{1.655974in}{1.299400in}}%
\pgfpathlineto{\pgfqpoint{1.655974in}{0.521603in}}%
\pgfpathclose%
\pgfusepath{fill}%
\end{pgfscope}%
\begin{pgfscope}%
\pgfpathrectangle{\pgfqpoint{0.595525in}{0.521603in}}{\pgfqpoint{3.201017in}{1.696195in}}%
\pgfusepath{clip}%
\pgfsetbuttcap%
\pgfsetmiterjoin%
\definecolor{currentfill}{rgb}{1.000000,0.498039,0.054902}%
\pgfsetfillcolor{currentfill}%
\pgfsetfillopacity{0.600000}%
\pgfsetlinewidth{0.000000pt}%
\definecolor{currentstroke}{rgb}{0.000000,0.000000,0.000000}%
\pgfsetstrokecolor{currentstroke}%
\pgfsetstrokeopacity{0.600000}%
\pgfsetdash{}{0pt}%
\pgfpathmoveto{\pgfqpoint{1.766811in}{0.521603in}}%
\pgfpathlineto{\pgfqpoint{1.877648in}{0.521603in}}%
\pgfpathlineto{\pgfqpoint{1.877648in}{1.147104in}}%
\pgfpathlineto{\pgfqpoint{1.766811in}{1.147104in}}%
\pgfpathlineto{\pgfqpoint{1.766811in}{0.521603in}}%
\pgfpathclose%
\pgfusepath{fill}%
\end{pgfscope}%
\begin{pgfscope}%
\pgfpathrectangle{\pgfqpoint{0.595525in}{0.521603in}}{\pgfqpoint{3.201017in}{1.696195in}}%
\pgfusepath{clip}%
\pgfsetbuttcap%
\pgfsetmiterjoin%
\definecolor{currentfill}{rgb}{1.000000,0.498039,0.054902}%
\pgfsetfillcolor{currentfill}%
\pgfsetfillopacity{0.600000}%
\pgfsetlinewidth{0.000000pt}%
\definecolor{currentstroke}{rgb}{0.000000,0.000000,0.000000}%
\pgfsetstrokecolor{currentstroke}%
\pgfsetstrokeopacity{0.600000}%
\pgfsetdash{}{0pt}%
\pgfpathmoveto{\pgfqpoint{1.877648in}{0.521603in}}%
\pgfpathlineto{\pgfqpoint{1.988485in}{0.521603in}}%
\pgfpathlineto{\pgfqpoint{1.988485in}{1.016565in}}%
\pgfpathlineto{\pgfqpoint{1.877648in}{1.016565in}}%
\pgfpathlineto{\pgfqpoint{1.877648in}{0.521603in}}%
\pgfpathclose%
\pgfusepath{fill}%
\end{pgfscope}%
\begin{pgfscope}%
\pgfpathrectangle{\pgfqpoint{0.595525in}{0.521603in}}{\pgfqpoint{3.201017in}{1.696195in}}%
\pgfusepath{clip}%
\pgfsetbuttcap%
\pgfsetmiterjoin%
\definecolor{currentfill}{rgb}{1.000000,0.498039,0.054902}%
\pgfsetfillcolor{currentfill}%
\pgfsetfillopacity{0.600000}%
\pgfsetlinewidth{0.000000pt}%
\definecolor{currentstroke}{rgb}{0.000000,0.000000,0.000000}%
\pgfsetstrokecolor{currentstroke}%
\pgfsetstrokeopacity{0.600000}%
\pgfsetdash{}{0pt}%
\pgfpathmoveto{\pgfqpoint{1.988485in}{0.521603in}}%
\pgfpathlineto{\pgfqpoint{2.099322in}{0.521603in}}%
\pgfpathlineto{\pgfqpoint{2.099322in}{0.983930in}}%
\pgfpathlineto{\pgfqpoint{1.988485in}{0.983930in}}%
\pgfpathlineto{\pgfqpoint{1.988485in}{0.521603in}}%
\pgfpathclose%
\pgfusepath{fill}%
\end{pgfscope}%
\begin{pgfscope}%
\pgfpathrectangle{\pgfqpoint{0.595525in}{0.521603in}}{\pgfqpoint{3.201017in}{1.696195in}}%
\pgfusepath{clip}%
\pgfsetbuttcap%
\pgfsetmiterjoin%
\definecolor{currentfill}{rgb}{1.000000,0.498039,0.054902}%
\pgfsetfillcolor{currentfill}%
\pgfsetfillopacity{0.600000}%
\pgfsetlinewidth{0.000000pt}%
\definecolor{currentstroke}{rgb}{0.000000,0.000000,0.000000}%
\pgfsetstrokecolor{currentstroke}%
\pgfsetstrokeopacity{0.600000}%
\pgfsetdash{}{0pt}%
\pgfpathmoveto{\pgfqpoint{2.099322in}{0.521603in}}%
\pgfpathlineto{\pgfqpoint{2.210159in}{0.521603in}}%
\pgfpathlineto{\pgfqpoint{2.210159in}{0.891465in}}%
\pgfpathlineto{\pgfqpoint{2.099322in}{0.891465in}}%
\pgfpathlineto{\pgfqpoint{2.099322in}{0.521603in}}%
\pgfpathclose%
\pgfusepath{fill}%
\end{pgfscope}%
\begin{pgfscope}%
\pgfpathrectangle{\pgfqpoint{0.595525in}{0.521603in}}{\pgfqpoint{3.201017in}{1.696195in}}%
\pgfusepath{clip}%
\pgfsetbuttcap%
\pgfsetmiterjoin%
\definecolor{currentfill}{rgb}{1.000000,0.498039,0.054902}%
\pgfsetfillcolor{currentfill}%
\pgfsetfillopacity{0.600000}%
\pgfsetlinewidth{0.000000pt}%
\definecolor{currentstroke}{rgb}{0.000000,0.000000,0.000000}%
\pgfsetstrokecolor{currentstroke}%
\pgfsetstrokeopacity{0.600000}%
\pgfsetdash{}{0pt}%
\pgfpathmoveto{\pgfqpoint{2.210159in}{0.521603in}}%
\pgfpathlineto{\pgfqpoint{2.320996in}{0.521603in}}%
\pgfpathlineto{\pgfqpoint{2.320996in}{0.875147in}}%
\pgfpathlineto{\pgfqpoint{2.210159in}{0.875147in}}%
\pgfpathlineto{\pgfqpoint{2.210159in}{0.521603in}}%
\pgfpathclose%
\pgfusepath{fill}%
\end{pgfscope}%
\begin{pgfscope}%
\pgfpathrectangle{\pgfqpoint{0.595525in}{0.521603in}}{\pgfqpoint{3.201017in}{1.696195in}}%
\pgfusepath{clip}%
\pgfsetbuttcap%
\pgfsetmiterjoin%
\definecolor{currentfill}{rgb}{1.000000,0.498039,0.054902}%
\pgfsetfillcolor{currentfill}%
\pgfsetfillopacity{0.600000}%
\pgfsetlinewidth{0.000000pt}%
\definecolor{currentstroke}{rgb}{0.000000,0.000000,0.000000}%
\pgfsetstrokecolor{currentstroke}%
\pgfsetstrokeopacity{0.600000}%
\pgfsetdash{}{0pt}%
\pgfpathmoveto{\pgfqpoint{2.320996in}{0.521603in}}%
\pgfpathlineto{\pgfqpoint{2.431833in}{0.521603in}}%
\pgfpathlineto{\pgfqpoint{2.431833in}{0.847952in}}%
\pgfpathlineto{\pgfqpoint{2.320996in}{0.847952in}}%
\pgfpathlineto{\pgfqpoint{2.320996in}{0.521603in}}%
\pgfpathclose%
\pgfusepath{fill}%
\end{pgfscope}%
\begin{pgfscope}%
\pgfpathrectangle{\pgfqpoint{0.595525in}{0.521603in}}{\pgfqpoint{3.201017in}{1.696195in}}%
\pgfusepath{clip}%
\pgfsetbuttcap%
\pgfsetmiterjoin%
\definecolor{currentfill}{rgb}{1.000000,0.498039,0.054902}%
\pgfsetfillcolor{currentfill}%
\pgfsetfillopacity{0.600000}%
\pgfsetlinewidth{0.000000pt}%
\definecolor{currentstroke}{rgb}{0.000000,0.000000,0.000000}%
\pgfsetstrokecolor{currentstroke}%
\pgfsetstrokeopacity{0.600000}%
\pgfsetdash{}{0pt}%
\pgfpathmoveto{\pgfqpoint{2.431833in}{0.521603in}}%
\pgfpathlineto{\pgfqpoint{2.542670in}{0.521603in}}%
\pgfpathlineto{\pgfqpoint{2.542670in}{0.728291in}}%
\pgfpathlineto{\pgfqpoint{2.431833in}{0.728291in}}%
\pgfpathlineto{\pgfqpoint{2.431833in}{0.521603in}}%
\pgfpathclose%
\pgfusepath{fill}%
\end{pgfscope}%
\begin{pgfscope}%
\pgfpathrectangle{\pgfqpoint{0.595525in}{0.521603in}}{\pgfqpoint{3.201017in}{1.696195in}}%
\pgfusepath{clip}%
\pgfsetbuttcap%
\pgfsetmiterjoin%
\definecolor{currentfill}{rgb}{1.000000,0.498039,0.054902}%
\pgfsetfillcolor{currentfill}%
\pgfsetfillopacity{0.600000}%
\pgfsetlinewidth{0.000000pt}%
\definecolor{currentstroke}{rgb}{0.000000,0.000000,0.000000}%
\pgfsetstrokecolor{currentstroke}%
\pgfsetstrokeopacity{0.600000}%
\pgfsetdash{}{0pt}%
\pgfpathmoveto{\pgfqpoint{2.542670in}{0.521603in}}%
\pgfpathlineto{\pgfqpoint{2.653508in}{0.521603in}}%
\pgfpathlineto{\pgfqpoint{2.653508in}{0.701095in}}%
\pgfpathlineto{\pgfqpoint{2.542670in}{0.701095in}}%
\pgfpathlineto{\pgfqpoint{2.542670in}{0.521603in}}%
\pgfpathclose%
\pgfusepath{fill}%
\end{pgfscope}%
\begin{pgfscope}%
\pgfpathrectangle{\pgfqpoint{0.595525in}{0.521603in}}{\pgfqpoint{3.201017in}{1.696195in}}%
\pgfusepath{clip}%
\pgfsetbuttcap%
\pgfsetmiterjoin%
\definecolor{currentfill}{rgb}{1.000000,0.498039,0.054902}%
\pgfsetfillcolor{currentfill}%
\pgfsetfillopacity{0.600000}%
\pgfsetlinewidth{0.000000pt}%
\definecolor{currentstroke}{rgb}{0.000000,0.000000,0.000000}%
\pgfsetstrokecolor{currentstroke}%
\pgfsetstrokeopacity{0.600000}%
\pgfsetdash{}{0pt}%
\pgfpathmoveto{\pgfqpoint{2.653508in}{0.521603in}}%
\pgfpathlineto{\pgfqpoint{2.764345in}{0.521603in}}%
\pgfpathlineto{\pgfqpoint{2.764345in}{0.711973in}}%
\pgfpathlineto{\pgfqpoint{2.653508in}{0.711973in}}%
\pgfpathlineto{\pgfqpoint{2.653508in}{0.521603in}}%
\pgfpathclose%
\pgfusepath{fill}%
\end{pgfscope}%
\begin{pgfscope}%
\pgfpathrectangle{\pgfqpoint{0.595525in}{0.521603in}}{\pgfqpoint{3.201017in}{1.696195in}}%
\pgfusepath{clip}%
\pgfsetbuttcap%
\pgfsetmiterjoin%
\definecolor{currentfill}{rgb}{1.000000,0.498039,0.054902}%
\pgfsetfillcolor{currentfill}%
\pgfsetfillopacity{0.600000}%
\pgfsetlinewidth{0.000000pt}%
\definecolor{currentstroke}{rgb}{0.000000,0.000000,0.000000}%
\pgfsetstrokecolor{currentstroke}%
\pgfsetstrokeopacity{0.600000}%
\pgfsetdash{}{0pt}%
\pgfpathmoveto{\pgfqpoint{2.764345in}{0.521603in}}%
\pgfpathlineto{\pgfqpoint{2.875182in}{0.521603in}}%
\pgfpathlineto{\pgfqpoint{2.875182in}{0.695656in}}%
\pgfpathlineto{\pgfqpoint{2.764345in}{0.695656in}}%
\pgfpathlineto{\pgfqpoint{2.764345in}{0.521603in}}%
\pgfpathclose%
\pgfusepath{fill}%
\end{pgfscope}%
\begin{pgfscope}%
\pgfpathrectangle{\pgfqpoint{0.595525in}{0.521603in}}{\pgfqpoint{3.201017in}{1.696195in}}%
\pgfusepath{clip}%
\pgfsetbuttcap%
\pgfsetmiterjoin%
\definecolor{currentfill}{rgb}{1.000000,0.498039,0.054902}%
\pgfsetfillcolor{currentfill}%
\pgfsetfillopacity{0.600000}%
\pgfsetlinewidth{0.000000pt}%
\definecolor{currentstroke}{rgb}{0.000000,0.000000,0.000000}%
\pgfsetstrokecolor{currentstroke}%
\pgfsetstrokeopacity{0.600000}%
\pgfsetdash{}{0pt}%
\pgfpathmoveto{\pgfqpoint{2.875182in}{0.521603in}}%
\pgfpathlineto{\pgfqpoint{2.986019in}{0.521603in}}%
\pgfpathlineto{\pgfqpoint{2.986019in}{0.684777in}}%
\pgfpathlineto{\pgfqpoint{2.875182in}{0.684777in}}%
\pgfpathlineto{\pgfqpoint{2.875182in}{0.521603in}}%
\pgfpathclose%
\pgfusepath{fill}%
\end{pgfscope}%
\begin{pgfscope}%
\pgfpathrectangle{\pgfqpoint{0.595525in}{0.521603in}}{\pgfqpoint{3.201017in}{1.696195in}}%
\pgfusepath{clip}%
\pgfsetbuttcap%
\pgfsetmiterjoin%
\definecolor{currentfill}{rgb}{1.000000,0.498039,0.054902}%
\pgfsetfillcolor{currentfill}%
\pgfsetfillopacity{0.600000}%
\pgfsetlinewidth{0.000000pt}%
\definecolor{currentstroke}{rgb}{0.000000,0.000000,0.000000}%
\pgfsetstrokecolor{currentstroke}%
\pgfsetstrokeopacity{0.600000}%
\pgfsetdash{}{0pt}%
\pgfpathmoveto{\pgfqpoint{2.986019in}{0.521603in}}%
\pgfpathlineto{\pgfqpoint{3.096856in}{0.521603in}}%
\pgfpathlineto{\pgfqpoint{3.096856in}{0.684777in}}%
\pgfpathlineto{\pgfqpoint{2.986019in}{0.684777in}}%
\pgfpathlineto{\pgfqpoint{2.986019in}{0.521603in}}%
\pgfpathclose%
\pgfusepath{fill}%
\end{pgfscope}%
\begin{pgfscope}%
\pgfpathrectangle{\pgfqpoint{0.595525in}{0.521603in}}{\pgfqpoint{3.201017in}{1.696195in}}%
\pgfusepath{clip}%
\pgfsetbuttcap%
\pgfsetmiterjoin%
\definecolor{currentfill}{rgb}{1.000000,0.498039,0.054902}%
\pgfsetfillcolor{currentfill}%
\pgfsetfillopacity{0.600000}%
\pgfsetlinewidth{0.000000pt}%
\definecolor{currentstroke}{rgb}{0.000000,0.000000,0.000000}%
\pgfsetstrokecolor{currentstroke}%
\pgfsetstrokeopacity{0.600000}%
\pgfsetdash{}{0pt}%
\pgfpathmoveto{\pgfqpoint{3.096856in}{0.521603in}}%
\pgfpathlineto{\pgfqpoint{3.207693in}{0.521603in}}%
\pgfpathlineto{\pgfqpoint{3.207693in}{0.592312in}}%
\pgfpathlineto{\pgfqpoint{3.096856in}{0.592312in}}%
\pgfpathlineto{\pgfqpoint{3.096856in}{0.521603in}}%
\pgfpathclose%
\pgfusepath{fill}%
\end{pgfscope}%
\begin{pgfscope}%
\pgfpathrectangle{\pgfqpoint{0.595525in}{0.521603in}}{\pgfqpoint{3.201017in}{1.696195in}}%
\pgfusepath{clip}%
\pgfsetbuttcap%
\pgfsetmiterjoin%
\definecolor{currentfill}{rgb}{1.000000,0.498039,0.054902}%
\pgfsetfillcolor{currentfill}%
\pgfsetfillopacity{0.600000}%
\pgfsetlinewidth{0.000000pt}%
\definecolor{currentstroke}{rgb}{0.000000,0.000000,0.000000}%
\pgfsetstrokecolor{currentstroke}%
\pgfsetstrokeopacity{0.600000}%
\pgfsetdash{}{0pt}%
\pgfpathmoveto{\pgfqpoint{3.207693in}{0.521603in}}%
\pgfpathlineto{\pgfqpoint{3.318530in}{0.521603in}}%
\pgfpathlineto{\pgfqpoint{3.318530in}{0.614069in}}%
\pgfpathlineto{\pgfqpoint{3.207693in}{0.614069in}}%
\pgfpathlineto{\pgfqpoint{3.207693in}{0.521603in}}%
\pgfpathclose%
\pgfusepath{fill}%
\end{pgfscope}%
\begin{pgfscope}%
\pgfpathrectangle{\pgfqpoint{0.595525in}{0.521603in}}{\pgfqpoint{3.201017in}{1.696195in}}%
\pgfusepath{clip}%
\pgfsetbuttcap%
\pgfsetmiterjoin%
\definecolor{currentfill}{rgb}{1.000000,0.498039,0.054902}%
\pgfsetfillcolor{currentfill}%
\pgfsetfillopacity{0.600000}%
\pgfsetlinewidth{0.000000pt}%
\definecolor{currentstroke}{rgb}{0.000000,0.000000,0.000000}%
\pgfsetstrokecolor{currentstroke}%
\pgfsetstrokeopacity{0.600000}%
\pgfsetdash{}{0pt}%
\pgfpathmoveto{\pgfqpoint{3.318530in}{0.521603in}}%
\pgfpathlineto{\pgfqpoint{3.429367in}{0.521603in}}%
\pgfpathlineto{\pgfqpoint{3.429367in}{0.614069in}}%
\pgfpathlineto{\pgfqpoint{3.318530in}{0.614069in}}%
\pgfpathlineto{\pgfqpoint{3.318530in}{0.521603in}}%
\pgfpathclose%
\pgfusepath{fill}%
\end{pgfscope}%
\begin{pgfscope}%
\pgfpathrectangle{\pgfqpoint{0.595525in}{0.521603in}}{\pgfqpoint{3.201017in}{1.696195in}}%
\pgfusepath{clip}%
\pgfsetbuttcap%
\pgfsetmiterjoin%
\definecolor{currentfill}{rgb}{1.000000,0.498039,0.054902}%
\pgfsetfillcolor{currentfill}%
\pgfsetfillopacity{0.600000}%
\pgfsetlinewidth{0.000000pt}%
\definecolor{currentstroke}{rgb}{0.000000,0.000000,0.000000}%
\pgfsetstrokecolor{currentstroke}%
\pgfsetstrokeopacity{0.600000}%
\pgfsetdash{}{0pt}%
\pgfpathmoveto{\pgfqpoint{3.429367in}{0.521603in}}%
\pgfpathlineto{\pgfqpoint{3.540204in}{0.521603in}}%
\pgfpathlineto{\pgfqpoint{3.540204in}{0.586873in}}%
\pgfpathlineto{\pgfqpoint{3.429367in}{0.586873in}}%
\pgfpathlineto{\pgfqpoint{3.429367in}{0.521603in}}%
\pgfpathclose%
\pgfusepath{fill}%
\end{pgfscope}%
\begin{pgfscope}%
\pgfpathrectangle{\pgfqpoint{0.595525in}{0.521603in}}{\pgfqpoint{3.201017in}{1.696195in}}%
\pgfusepath{clip}%
\pgfsetbuttcap%
\pgfsetmiterjoin%
\definecolor{currentfill}{rgb}{1.000000,0.498039,0.054902}%
\pgfsetfillcolor{currentfill}%
\pgfsetfillopacity{0.600000}%
\pgfsetlinewidth{0.000000pt}%
\definecolor{currentstroke}{rgb}{0.000000,0.000000,0.000000}%
\pgfsetstrokecolor{currentstroke}%
\pgfsetstrokeopacity{0.600000}%
\pgfsetdash{}{0pt}%
\pgfpathmoveto{\pgfqpoint{3.540204in}{0.521603in}}%
\pgfpathlineto{\pgfqpoint{3.651041in}{0.521603in}}%
\pgfpathlineto{\pgfqpoint{3.651041in}{0.543360in}}%
\pgfpathlineto{\pgfqpoint{3.540204in}{0.543360in}}%
\pgfpathlineto{\pgfqpoint{3.540204in}{0.521603in}}%
\pgfpathclose%
\pgfusepath{fill}%
\end{pgfscope}%
\begin{pgfscope}%
\pgfsetbuttcap%
\pgfsetroundjoin%
\definecolor{currentfill}{rgb}{0.000000,0.000000,0.000000}%
\pgfsetfillcolor{currentfill}%
\pgfsetlinewidth{0.803000pt}%
\definecolor{currentstroke}{rgb}{0.000000,0.000000,0.000000}%
\pgfsetstrokecolor{currentstroke}%
\pgfsetdash{}{0pt}%
\pgfsys@defobject{currentmarker}{\pgfqpoint{0.000000in}{-0.048611in}}{\pgfqpoint{0.000000in}{0.000000in}}{%
\pgfpathmoveto{\pgfqpoint{0.000000in}{0.000000in}}%
\pgfpathlineto{\pgfqpoint{0.000000in}{-0.048611in}}%
\pgfusepath{stroke,fill}%
}%
\begin{pgfscope}%
\pgfsys@transformshift{0.684520in}{0.521603in}%
\pgfsys@useobject{currentmarker}{}%
\end{pgfscope}%
\end{pgfscope}%
\begin{pgfscope}%
\definecolor{textcolor}{rgb}{0.000000,0.000000,0.000000}%
\pgfsetstrokecolor{textcolor}%
\pgfsetfillcolor{textcolor}%
\pgftext[x=0.684520in,y=0.424381in,,top]{\color{textcolor}{\rmfamily\fontsize{10.000000}{12.000000}\selectfont\catcode`\^=\active\def^{\ifmmode\sp\else\^{}\fi}\catcode`\%=\active\def%{\%}$\mathdefault{0}$}}%
\end{pgfscope}%
\begin{pgfscope}%
\pgfsetbuttcap%
\pgfsetroundjoin%
\definecolor{currentfill}{rgb}{0.000000,0.000000,0.000000}%
\pgfsetfillcolor{currentfill}%
\pgfsetlinewidth{0.803000pt}%
\definecolor{currentstroke}{rgb}{0.000000,0.000000,0.000000}%
\pgfsetstrokecolor{currentstroke}%
\pgfsetdash{}{0pt}%
\pgfsys@defobject{currentmarker}{\pgfqpoint{0.000000in}{-0.048611in}}{\pgfqpoint{0.000000in}{0.000000in}}{%
\pgfpathmoveto{\pgfqpoint{0.000000in}{0.000000in}}%
\pgfpathlineto{\pgfqpoint{0.000000in}{-0.048611in}}%
\pgfusepath{stroke,fill}%
}%
\begin{pgfscope}%
\pgfsys@transformshift{1.249572in}{0.521603in}%
\pgfsys@useobject{currentmarker}{}%
\end{pgfscope}%
\end{pgfscope}%
\begin{pgfscope}%
\definecolor{textcolor}{rgb}{0.000000,0.000000,0.000000}%
\pgfsetstrokecolor{textcolor}%
\pgfsetfillcolor{textcolor}%
\pgftext[x=1.249572in,y=0.424381in,,top]{\color{textcolor}{\rmfamily\fontsize{10.000000}{12.000000}\selectfont\catcode`\^=\active\def^{\ifmmode\sp\else\^{}\fi}\catcode`\%=\active\def%{\%}$\mathdefault{20}$}}%
\end{pgfscope}%
\begin{pgfscope}%
\pgfsetbuttcap%
\pgfsetroundjoin%
\definecolor{currentfill}{rgb}{0.000000,0.000000,0.000000}%
\pgfsetfillcolor{currentfill}%
\pgfsetlinewidth{0.803000pt}%
\definecolor{currentstroke}{rgb}{0.000000,0.000000,0.000000}%
\pgfsetstrokecolor{currentstroke}%
\pgfsetdash{}{0pt}%
\pgfsys@defobject{currentmarker}{\pgfqpoint{0.000000in}{-0.048611in}}{\pgfqpoint{0.000000in}{0.000000in}}{%
\pgfpathmoveto{\pgfqpoint{0.000000in}{0.000000in}}%
\pgfpathlineto{\pgfqpoint{0.000000in}{-0.048611in}}%
\pgfusepath{stroke,fill}%
}%
\begin{pgfscope}%
\pgfsys@transformshift{1.814623in}{0.521603in}%
\pgfsys@useobject{currentmarker}{}%
\end{pgfscope}%
\end{pgfscope}%
\begin{pgfscope}%
\definecolor{textcolor}{rgb}{0.000000,0.000000,0.000000}%
\pgfsetstrokecolor{textcolor}%
\pgfsetfillcolor{textcolor}%
\pgftext[x=1.814623in,y=0.424381in,,top]{\color{textcolor}{\rmfamily\fontsize{10.000000}{12.000000}\selectfont\catcode`\^=\active\def^{\ifmmode\sp\else\^{}\fi}\catcode`\%=\active\def%{\%}$\mathdefault{40}$}}%
\end{pgfscope}%
\begin{pgfscope}%
\pgfsetbuttcap%
\pgfsetroundjoin%
\definecolor{currentfill}{rgb}{0.000000,0.000000,0.000000}%
\pgfsetfillcolor{currentfill}%
\pgfsetlinewidth{0.803000pt}%
\definecolor{currentstroke}{rgb}{0.000000,0.000000,0.000000}%
\pgfsetstrokecolor{currentstroke}%
\pgfsetdash{}{0pt}%
\pgfsys@defobject{currentmarker}{\pgfqpoint{0.000000in}{-0.048611in}}{\pgfqpoint{0.000000in}{0.000000in}}{%
\pgfpathmoveto{\pgfqpoint{0.000000in}{0.000000in}}%
\pgfpathlineto{\pgfqpoint{0.000000in}{-0.048611in}}%
\pgfusepath{stroke,fill}%
}%
\begin{pgfscope}%
\pgfsys@transformshift{2.379675in}{0.521603in}%
\pgfsys@useobject{currentmarker}{}%
\end{pgfscope}%
\end{pgfscope}%
\begin{pgfscope}%
\definecolor{textcolor}{rgb}{0.000000,0.000000,0.000000}%
\pgfsetstrokecolor{textcolor}%
\pgfsetfillcolor{textcolor}%
\pgftext[x=2.379675in,y=0.424381in,,top]{\color{textcolor}{\rmfamily\fontsize{10.000000}{12.000000}\selectfont\catcode`\^=\active\def^{\ifmmode\sp\else\^{}\fi}\catcode`\%=\active\def%{\%}$\mathdefault{60}$}}%
\end{pgfscope}%
\begin{pgfscope}%
\pgfsetbuttcap%
\pgfsetroundjoin%
\definecolor{currentfill}{rgb}{0.000000,0.000000,0.000000}%
\pgfsetfillcolor{currentfill}%
\pgfsetlinewidth{0.803000pt}%
\definecolor{currentstroke}{rgb}{0.000000,0.000000,0.000000}%
\pgfsetstrokecolor{currentstroke}%
\pgfsetdash{}{0pt}%
\pgfsys@defobject{currentmarker}{\pgfqpoint{0.000000in}{-0.048611in}}{\pgfqpoint{0.000000in}{0.000000in}}{%
\pgfpathmoveto{\pgfqpoint{0.000000in}{0.000000in}}%
\pgfpathlineto{\pgfqpoint{0.000000in}{-0.048611in}}%
\pgfusepath{stroke,fill}%
}%
\begin{pgfscope}%
\pgfsys@transformshift{2.944726in}{0.521603in}%
\pgfsys@useobject{currentmarker}{}%
\end{pgfscope}%
\end{pgfscope}%
\begin{pgfscope}%
\definecolor{textcolor}{rgb}{0.000000,0.000000,0.000000}%
\pgfsetstrokecolor{textcolor}%
\pgfsetfillcolor{textcolor}%
\pgftext[x=2.944726in,y=0.424381in,,top]{\color{textcolor}{\rmfamily\fontsize{10.000000}{12.000000}\selectfont\catcode`\^=\active\def^{\ifmmode\sp\else\^{}\fi}\catcode`\%=\active\def%{\%}$\mathdefault{80}$}}%
\end{pgfscope}%
\begin{pgfscope}%
\pgfsetbuttcap%
\pgfsetroundjoin%
\definecolor{currentfill}{rgb}{0.000000,0.000000,0.000000}%
\pgfsetfillcolor{currentfill}%
\pgfsetlinewidth{0.803000pt}%
\definecolor{currentstroke}{rgb}{0.000000,0.000000,0.000000}%
\pgfsetstrokecolor{currentstroke}%
\pgfsetdash{}{0pt}%
\pgfsys@defobject{currentmarker}{\pgfqpoint{0.000000in}{-0.048611in}}{\pgfqpoint{0.000000in}{0.000000in}}{%
\pgfpathmoveto{\pgfqpoint{0.000000in}{0.000000in}}%
\pgfpathlineto{\pgfqpoint{0.000000in}{-0.048611in}}%
\pgfusepath{stroke,fill}%
}%
\begin{pgfscope}%
\pgfsys@transformshift{3.509778in}{0.521603in}%
\pgfsys@useobject{currentmarker}{}%
\end{pgfscope}%
\end{pgfscope}%
\begin{pgfscope}%
\definecolor{textcolor}{rgb}{0.000000,0.000000,0.000000}%
\pgfsetstrokecolor{textcolor}%
\pgfsetfillcolor{textcolor}%
\pgftext[x=3.509778in,y=0.424381in,,top]{\color{textcolor}{\rmfamily\fontsize{10.000000}{12.000000}\selectfont\catcode`\^=\active\def^{\ifmmode\sp\else\^{}\fi}\catcode`\%=\active\def%{\%}$\mathdefault{100}$}}%
\end{pgfscope}%
\begin{pgfscope}%
\definecolor{textcolor}{rgb}{0.000000,0.000000,0.000000}%
\pgfsetstrokecolor{textcolor}%
\pgfsetfillcolor{textcolor}%
\pgftext[x=2.196033in,y=0.234413in,,top]{\color{textcolor}{\rmfamily\fontsize{10.000000}{12.000000}\selectfont\catcode`\^=\active\def^{\ifmmode\sp\else\^{}\fi}\catcode`\%=\active\def%{\%}Number of m. classes or $\triangle$-conn. components}}%
\end{pgfscope}%
\begin{pgfscope}%
\pgfsetbuttcap%
\pgfsetroundjoin%
\definecolor{currentfill}{rgb}{0.000000,0.000000,0.000000}%
\pgfsetfillcolor{currentfill}%
\pgfsetlinewidth{0.803000pt}%
\definecolor{currentstroke}{rgb}{0.000000,0.000000,0.000000}%
\pgfsetstrokecolor{currentstroke}%
\pgfsetdash{}{0pt}%
\pgfsys@defobject{currentmarker}{\pgfqpoint{-0.048611in}{0.000000in}}{\pgfqpoint{-0.000000in}{0.000000in}}{%
\pgfpathmoveto{\pgfqpoint{-0.000000in}{0.000000in}}%
\pgfpathlineto{\pgfqpoint{-0.048611in}{0.000000in}}%
\pgfusepath{stroke,fill}%
}%
\begin{pgfscope}%
\pgfsys@transformshift{0.595525in}{0.521603in}%
\pgfsys@useobject{currentmarker}{}%
\end{pgfscope}%
\end{pgfscope}%
\begin{pgfscope}%
\definecolor{textcolor}{rgb}{0.000000,0.000000,0.000000}%
\pgfsetstrokecolor{textcolor}%
\pgfsetfillcolor{textcolor}%
\pgftext[x=0.428858in, y=0.468842in, left, base]{\color{textcolor}{\rmfamily\fontsize{10.000000}{12.000000}\selectfont\catcode`\^=\active\def^{\ifmmode\sp\else\^{}\fi}\catcode`\%=\active\def%{\%}$\mathdefault{0}$}}%
\end{pgfscope}%
\begin{pgfscope}%
\pgfsetbuttcap%
\pgfsetroundjoin%
\definecolor{currentfill}{rgb}{0.000000,0.000000,0.000000}%
\pgfsetfillcolor{currentfill}%
\pgfsetlinewidth{0.803000pt}%
\definecolor{currentstroke}{rgb}{0.000000,0.000000,0.000000}%
\pgfsetstrokecolor{currentstroke}%
\pgfsetdash{}{0pt}%
\pgfsys@defobject{currentmarker}{\pgfqpoint{-0.048611in}{0.000000in}}{\pgfqpoint{-0.000000in}{0.000000in}}{%
\pgfpathmoveto{\pgfqpoint{-0.000000in}{0.000000in}}%
\pgfpathlineto{\pgfqpoint{-0.048611in}{0.000000in}}%
\pgfusepath{stroke,fill}%
}%
\begin{pgfscope}%
\pgfsys@transformshift{0.595525in}{1.065517in}%
\pgfsys@useobject{currentmarker}{}%
\end{pgfscope}%
\end{pgfscope}%
\begin{pgfscope}%
\definecolor{textcolor}{rgb}{0.000000,0.000000,0.000000}%
\pgfsetstrokecolor{textcolor}%
\pgfsetfillcolor{textcolor}%
\pgftext[x=0.289968in, y=1.012755in, left, base]{\color{textcolor}{\rmfamily\fontsize{10.000000}{12.000000}\selectfont\catcode`\^=\active\def^{\ifmmode\sp\else\^{}\fi}\catcode`\%=\active\def%{\%}$\mathdefault{100}$}}%
\end{pgfscope}%
\begin{pgfscope}%
\pgfsetbuttcap%
\pgfsetroundjoin%
\definecolor{currentfill}{rgb}{0.000000,0.000000,0.000000}%
\pgfsetfillcolor{currentfill}%
\pgfsetlinewidth{0.803000pt}%
\definecolor{currentstroke}{rgb}{0.000000,0.000000,0.000000}%
\pgfsetstrokecolor{currentstroke}%
\pgfsetdash{}{0pt}%
\pgfsys@defobject{currentmarker}{\pgfqpoint{-0.048611in}{0.000000in}}{\pgfqpoint{-0.000000in}{0.000000in}}{%
\pgfpathmoveto{\pgfqpoint{-0.000000in}{0.000000in}}%
\pgfpathlineto{\pgfqpoint{-0.048611in}{0.000000in}}%
\pgfusepath{stroke,fill}%
}%
\begin{pgfscope}%
\pgfsys@transformshift{0.595525in}{1.609431in}%
\pgfsys@useobject{currentmarker}{}%
\end{pgfscope}%
\end{pgfscope}%
\begin{pgfscope}%
\definecolor{textcolor}{rgb}{0.000000,0.000000,0.000000}%
\pgfsetstrokecolor{textcolor}%
\pgfsetfillcolor{textcolor}%
\pgftext[x=0.289968in, y=1.556669in, left, base]{\color{textcolor}{\rmfamily\fontsize{10.000000}{12.000000}\selectfont\catcode`\^=\active\def^{\ifmmode\sp\else\^{}\fi}\catcode`\%=\active\def%{\%}$\mathdefault{200}$}}%
\end{pgfscope}%
\begin{pgfscope}%
\pgfsetbuttcap%
\pgfsetroundjoin%
\definecolor{currentfill}{rgb}{0.000000,0.000000,0.000000}%
\pgfsetfillcolor{currentfill}%
\pgfsetlinewidth{0.803000pt}%
\definecolor{currentstroke}{rgb}{0.000000,0.000000,0.000000}%
\pgfsetstrokecolor{currentstroke}%
\pgfsetdash{}{0pt}%
\pgfsys@defobject{currentmarker}{\pgfqpoint{-0.048611in}{0.000000in}}{\pgfqpoint{-0.000000in}{0.000000in}}{%
\pgfpathmoveto{\pgfqpoint{-0.000000in}{0.000000in}}%
\pgfpathlineto{\pgfqpoint{-0.048611in}{0.000000in}}%
\pgfusepath{stroke,fill}%
}%
\begin{pgfscope}%
\pgfsys@transformshift{0.595525in}{2.153344in}%
\pgfsys@useobject{currentmarker}{}%
\end{pgfscope}%
\end{pgfscope}%
\begin{pgfscope}%
\definecolor{textcolor}{rgb}{0.000000,0.000000,0.000000}%
\pgfsetstrokecolor{textcolor}%
\pgfsetfillcolor{textcolor}%
\pgftext[x=0.289968in, y=2.100583in, left, base]{\color{textcolor}{\rmfamily\fontsize{10.000000}{12.000000}\selectfont\catcode`\^=\active\def^{\ifmmode\sp\else\^{}\fi}\catcode`\%=\active\def%{\%}$\mathdefault{300}$}}%
\end{pgfscope}%
\begin{pgfscope}%
\definecolor{textcolor}{rgb}{0.000000,0.000000,0.000000}%
\pgfsetstrokecolor{textcolor}%
\pgfsetfillcolor{textcolor}%
\pgftext[x=0.234413in,y=1.369701in,,bottom,rotate=90.000000]{\color{textcolor}{\rmfamily\fontsize{10.000000}{12.000000}\selectfont\catcode`\^=\active\def^{\ifmmode\sp\else\^{}\fi}\catcode`\%=\active\def%{\%}Number of graphs}}%
\end{pgfscope}%
\begin{pgfscope}%
\pgfsetrectcap%
\pgfsetmiterjoin%
\pgfsetlinewidth{0.803000pt}%
\definecolor{currentstroke}{rgb}{0.000000,0.000000,0.000000}%
\pgfsetstrokecolor{currentstroke}%
\pgfsetdash{}{0pt}%
\pgfpathmoveto{\pgfqpoint{0.595525in}{0.521603in}}%
\pgfpathlineto{\pgfqpoint{0.595525in}{2.217798in}}%
\pgfusepath{stroke}%
\end{pgfscope}%
\begin{pgfscope}%
\pgfsetrectcap%
\pgfsetmiterjoin%
\pgfsetlinewidth{0.803000pt}%
\definecolor{currentstroke}{rgb}{0.000000,0.000000,0.000000}%
\pgfsetstrokecolor{currentstroke}%
\pgfsetdash{}{0pt}%
\pgfpathmoveto{\pgfqpoint{3.796542in}{0.521603in}}%
\pgfpathlineto{\pgfqpoint{3.796542in}{2.217798in}}%
\pgfusepath{stroke}%
\end{pgfscope}%
\begin{pgfscope}%
\pgfsetrectcap%
\pgfsetmiterjoin%
\pgfsetlinewidth{0.803000pt}%
\definecolor{currentstroke}{rgb}{0.000000,0.000000,0.000000}%
\pgfsetstrokecolor{currentstroke}%
\pgfsetdash{}{0pt}%
\pgfpathmoveto{\pgfqpoint{0.595525in}{0.521603in}}%
\pgfpathlineto{\pgfqpoint{3.796542in}{0.521603in}}%
\pgfusepath{stroke}%
\end{pgfscope}%
\begin{pgfscope}%
\pgfsetrectcap%
\pgfsetmiterjoin%
\pgfsetlinewidth{0.803000pt}%
\definecolor{currentstroke}{rgb}{0.000000,0.000000,0.000000}%
\pgfsetstrokecolor{currentstroke}%
\pgfsetdash{}{0pt}%
\pgfpathmoveto{\pgfqpoint{0.595525in}{2.217798in}}%
\pgfpathlineto{\pgfqpoint{3.796542in}{2.217798in}}%
\pgfusepath{stroke}%
\end{pgfscope}%
\begin{pgfscope}%
\pgfsetbuttcap%
\pgfsetmiterjoin%
\definecolor{currentfill}{rgb}{1.000000,1.000000,1.000000}%
\pgfsetfillcolor{currentfill}%
\pgfsetfillopacity{0.800000}%
\pgfsetlinewidth{1.003750pt}%
\definecolor{currentstroke}{rgb}{0.800000,0.800000,0.800000}%
\pgfsetstrokecolor{currentstroke}%
\pgfsetstrokeopacity{0.800000}%
\pgfsetdash{}{0pt}%
\pgfpathmoveto{\pgfqpoint{0.982850in}{1.595207in}}%
\pgfpathlineto{\pgfqpoint{3.679875in}{1.595207in}}%
\pgfpathquadraticcurveto{\pgfqpoint{3.713208in}{1.595207in}}{\pgfqpoint{3.713208in}{1.628541in}}%
\pgfpathlineto{\pgfqpoint{3.713208in}{2.101131in}}%
\pgfpathquadraticcurveto{\pgfqpoint{3.713208in}{2.134465in}}{\pgfqpoint{3.679875in}{2.134465in}}%
\pgfpathlineto{\pgfqpoint{0.982850in}{2.134465in}}%
\pgfpathquadraticcurveto{\pgfqpoint{0.949516in}{2.134465in}}{\pgfqpoint{0.949516in}{2.101131in}}%
\pgfpathlineto{\pgfqpoint{0.949516in}{1.628541in}}%
\pgfpathquadraticcurveto{\pgfqpoint{0.949516in}{1.595207in}}{\pgfqpoint{0.982850in}{1.595207in}}%
\pgfpathlineto{\pgfqpoint{0.982850in}{1.595207in}}%
\pgfpathclose%
\pgfusepath{stroke,fill}%
\end{pgfscope}%
\begin{pgfscope}%
\pgfsetbuttcap%
\pgfsetmiterjoin%
\definecolor{currentfill}{rgb}{0.121569,0.466667,0.705882}%
\pgfsetfillcolor{currentfill}%
\pgfsetfillopacity{0.600000}%
\pgfsetlinewidth{0.000000pt}%
\definecolor{currentstroke}{rgb}{0.000000,0.000000,0.000000}%
\pgfsetstrokecolor{currentstroke}%
\pgfsetstrokeopacity{0.600000}%
\pgfsetdash{}{0pt}%
\pgfpathmoveto{\pgfqpoint{1.016183in}{1.941171in}}%
\pgfpathlineto{\pgfqpoint{1.349516in}{1.941171in}}%
\pgfpathlineto{\pgfqpoint{1.349516in}{2.057837in}}%
\pgfpathlineto{\pgfqpoint{1.016183in}{2.057837in}}%
\pgfpathlineto{\pgfqpoint{1.016183in}{1.941171in}}%
\pgfpathclose%
\pgfusepath{fill}%
\end{pgfscope}%
\begin{pgfscope}%
\definecolor{textcolor}{rgb}{0.000000,0.000000,0.000000}%
\pgfsetstrokecolor{textcolor}%
\pgfsetfillcolor{textcolor}%
\pgftext[x=1.482850in,y=1.941171in,left,base]{\color{textcolor}{\rmfamily\fontsize{12.000000}{14.400000}\selectfont\catcode`\^=\active\def^{\ifmmode\sp\else\^{}\fi}\catcode`\%=\active\def%{\%}Monochromatic classes}}%
\end{pgfscope}%
\begin{pgfscope}%
\pgfsetbuttcap%
\pgfsetmiterjoin%
\definecolor{currentfill}{rgb}{1.000000,0.498039,0.054902}%
\pgfsetfillcolor{currentfill}%
\pgfsetfillopacity{0.600000}%
\pgfsetlinewidth{0.000000pt}%
\definecolor{currentstroke}{rgb}{0.000000,0.000000,0.000000}%
\pgfsetstrokecolor{currentstroke}%
\pgfsetstrokeopacity{0.600000}%
\pgfsetdash{}{0pt}%
\pgfpathmoveto{\pgfqpoint{1.016183in}{1.696542in}}%
\pgfpathlineto{\pgfqpoint{1.349516in}{1.696542in}}%
\pgfpathlineto{\pgfqpoint{1.349516in}{1.813208in}}%
\pgfpathlineto{\pgfqpoint{1.016183in}{1.813208in}}%
\pgfpathlineto{\pgfqpoint{1.016183in}{1.696542in}}%
\pgfpathclose%
\pgfusepath{fill}%
\end{pgfscope}%
\begin{pgfscope}%
\definecolor{textcolor}{rgb}{0.000000,0.000000,0.000000}%
\pgfsetstrokecolor{textcolor}%
\pgfsetfillcolor{textcolor}%
\pgftext[x=1.482850in,y=1.696542in,left,base]{\color{textcolor}{\rmfamily\fontsize{12.000000}{14.400000}\selectfont\catcode`\^=\active\def^{\ifmmode\sp\else\^{}\fi}\catcode`\%=\active\def%{\%}$\triangle$-connected components}}%
\end{pgfscope}%
\end{pgfpicture}%
\makeatother%
\endgroup%
}
		\caption[Monoch. classes vs tr. con. components for minimally rigid]{%
			\centering Minimally rigid graphs}%
		\label{fig:monochrom_vs_triangle_minimally_rigid}
	\end{subfigure}
	\caption{Monochromatic classes vs \trcon{} components}%
	\label{fig:monochrom_vs_triangle}
\end{figure}

\NaiveCycles{} is no longer faster for finding a NAC-coloring
as shown in \Cref{fig:graph_globally_rigid_first_runtime}.
\None{} and \Neighbors{} strategies match the performance and
the other are not far behind.
%
\begin{figure}[thbp]
	\centering
	\scalebox{\BenchFigureScale}{%% Creator: Matplotlib, PGF backend
%%
%% To include the figure in your LaTeX document, write
%%   \input{<filename>.pgf}
%%
%% Make sure the required packages are loaded in your preamble
%%   \usepackage{pgf}
%%
%% Also ensure that all the required font packages are loaded; for instance,
%% the lmodern package is sometimes necessary when using math font.
%%   \usepackage{lmodern}
%%
%% Figures using additional raster images can only be included by \input if
%% they are in the same directory as the main LaTeX file. For loading figures
%% from other directories you can use the `import` package
%%   \usepackage{import}
%%
%% and then include the figures with
%%   \import{<path to file>}{<filename>.pgf}
%%
%% Matplotlib used the following preamble
%%   \def\mathdefault#1{#1}
%%   \everymath=\expandafter{\the\everymath\displaystyle}
%%   \IfFileExists{scrextend.sty}{
%%     \usepackage[fontsize=10.000000pt]{scrextend}
%%   }{
%%     \renewcommand{\normalsize}{\fontsize{10.000000}{12.000000}\selectfont}
%%     \normalsize
%%   }
%%   
%%   \ifdefined\pdftexversion\else  % non-pdftex case.
%%     \usepackage{fontspec}
%%     \setmainfont{DejaVuSans.ttf}[Path=\detokenize{/home/petr/Projects/PyRigi/.venv/lib/python3.12/site-packages/matplotlib/mpl-data/fonts/ttf/}]
%%     \setsansfont{DejaVuSans.ttf}[Path=\detokenize{/home/petr/Projects/PyRigi/.venv/lib/python3.12/site-packages/matplotlib/mpl-data/fonts/ttf/}]
%%     \setmonofont{DejaVuSansMono.ttf}[Path=\detokenize{/home/petr/Projects/PyRigi/.venv/lib/python3.12/site-packages/matplotlib/mpl-data/fonts/ttf/}]
%%   \fi
%%   \makeatletter\@ifpackageloaded{under\Score{}}{}{\usepackage[strings]{under\Score{}}}\makeatother
%%
\begingroup%
\makeatletter%
\begin{pgfpicture}%
\pgfpathrectangle{\pgfpointorigin}{\pgfqpoint{8.384376in}{2.841849in}}%
\pgfusepath{use as bounding box, clip}%
\begin{pgfscope}%
\pgfsetbuttcap%
\pgfsetmiterjoin%
\definecolor{currentfill}{rgb}{1.000000,1.000000,1.000000}%
\pgfsetfillcolor{currentfill}%
\pgfsetlinewidth{0.000000pt}%
\definecolor{currentstroke}{rgb}{1.000000,1.000000,1.000000}%
\pgfsetstrokecolor{currentstroke}%
\pgfsetdash{}{0pt}%
\pgfpathmoveto{\pgfqpoint{0.000000in}{0.000000in}}%
\pgfpathlineto{\pgfqpoint{8.384376in}{0.000000in}}%
\pgfpathlineto{\pgfqpoint{8.384376in}{2.841849in}}%
\pgfpathlineto{\pgfqpoint{0.000000in}{2.841849in}}%
\pgfpathlineto{\pgfqpoint{0.000000in}{0.000000in}}%
\pgfpathclose%
\pgfusepath{fill}%
\end{pgfscope}%
\begin{pgfscope}%
\pgfsetbuttcap%
\pgfsetmiterjoin%
\definecolor{currentfill}{rgb}{1.000000,1.000000,1.000000}%
\pgfsetfillcolor{currentfill}%
\pgfsetlinewidth{0.000000pt}%
\definecolor{currentstroke}{rgb}{0.000000,0.000000,0.000000}%
\pgfsetstrokecolor{currentstroke}%
\pgfsetstrokeopacity{0.000000}%
\pgfsetdash{}{0pt}%
\pgfpathmoveto{\pgfqpoint{0.588387in}{0.521603in}}%
\pgfpathlineto{\pgfqpoint{5.257411in}{0.521603in}}%
\pgfpathlineto{\pgfqpoint{5.257411in}{2.531888in}}%
\pgfpathlineto{\pgfqpoint{0.588387in}{2.531888in}}%
\pgfpathlineto{\pgfqpoint{0.588387in}{0.521603in}}%
\pgfpathclose%
\pgfusepath{fill}%
\end{pgfscope}%
\begin{pgfscope}%
\pgfsetbuttcap%
\pgfsetroundjoin%
\definecolor{currentfill}{rgb}{0.000000,0.000000,0.000000}%
\pgfsetfillcolor{currentfill}%
\pgfsetlinewidth{0.803000pt}%
\definecolor{currentstroke}{rgb}{0.000000,0.000000,0.000000}%
\pgfsetstrokecolor{currentstroke}%
\pgfsetdash{}{0pt}%
\pgfsys@defobject{currentmarker}{\pgfqpoint{0.000000in}{-0.048611in}}{\pgfqpoint{0.000000in}{0.000000in}}{%
\pgfpathmoveto{\pgfqpoint{0.000000in}{0.000000in}}%
\pgfpathlineto{\pgfqpoint{0.000000in}{-0.048611in}}%
\pgfusepath{stroke,fill}%
}%
\begin{pgfscope}%
\pgfsys@transformshift{0.985162in}{0.521603in}%
\pgfsys@useobject{currentmarker}{}%
\end{pgfscope}%
\end{pgfscope}%
\begin{pgfscope}%
\definecolor{textcolor}{rgb}{0.000000,0.000000,0.000000}%
\pgfsetstrokecolor{textcolor}%
\pgfsetfillcolor{textcolor}%
\pgftext[x=0.985162in,y=0.424381in,,top]{\color{textcolor}{\rmfamily\fontsize{10.000000}{12.000000}\selectfont\catcode`\^=\active\def^{\ifmmode\sp\else\^{}\fi}\catcode`\%=\active\def%{\%}$\mathdefault{12}$}}%
\end{pgfscope}%
\begin{pgfscope}%
\pgfsetbuttcap%
\pgfsetroundjoin%
\definecolor{currentfill}{rgb}{0.000000,0.000000,0.000000}%
\pgfsetfillcolor{currentfill}%
\pgfsetlinewidth{0.803000pt}%
\definecolor{currentstroke}{rgb}{0.000000,0.000000,0.000000}%
\pgfsetstrokecolor{currentstroke}%
\pgfsetdash{}{0pt}%
\pgfsys@defobject{currentmarker}{\pgfqpoint{0.000000in}{-0.048611in}}{\pgfqpoint{0.000000in}{0.000000in}}{%
\pgfpathmoveto{\pgfqpoint{0.000000in}{0.000000in}}%
\pgfpathlineto{\pgfqpoint{0.000000in}{-0.048611in}}%
\pgfusepath{stroke,fill}%
}%
\begin{pgfscope}%
\pgfsys@transformshift{1.538801in}{0.521603in}%
\pgfsys@useobject{currentmarker}{}%
\end{pgfscope}%
\end{pgfscope}%
\begin{pgfscope}%
\definecolor{textcolor}{rgb}{0.000000,0.000000,0.000000}%
\pgfsetstrokecolor{textcolor}%
\pgfsetfillcolor{textcolor}%
\pgftext[x=1.538801in,y=0.424381in,,top]{\color{textcolor}{\rmfamily\fontsize{10.000000}{12.000000}\selectfont\catcode`\^=\active\def^{\ifmmode\sp\else\^{}\fi}\catcode`\%=\active\def%{\%}$\mathdefault{18}$}}%
\end{pgfscope}%
\begin{pgfscope}%
\pgfsetbuttcap%
\pgfsetroundjoin%
\definecolor{currentfill}{rgb}{0.000000,0.000000,0.000000}%
\pgfsetfillcolor{currentfill}%
\pgfsetlinewidth{0.803000pt}%
\definecolor{currentstroke}{rgb}{0.000000,0.000000,0.000000}%
\pgfsetstrokecolor{currentstroke}%
\pgfsetdash{}{0pt}%
\pgfsys@defobject{currentmarker}{\pgfqpoint{0.000000in}{-0.048611in}}{\pgfqpoint{0.000000in}{0.000000in}}{%
\pgfpathmoveto{\pgfqpoint{0.000000in}{0.000000in}}%
\pgfpathlineto{\pgfqpoint{0.000000in}{-0.048611in}}%
\pgfusepath{stroke,fill}%
}%
\begin{pgfscope}%
\pgfsys@transformshift{2.092440in}{0.521603in}%
\pgfsys@useobject{currentmarker}{}%
\end{pgfscope}%
\end{pgfscope}%
\begin{pgfscope}%
\definecolor{textcolor}{rgb}{0.000000,0.000000,0.000000}%
\pgfsetstrokecolor{textcolor}%
\pgfsetfillcolor{textcolor}%
\pgftext[x=2.092440in,y=0.424381in,,top]{\color{textcolor}{\rmfamily\fontsize{10.000000}{12.000000}\selectfont\catcode`\^=\active\def^{\ifmmode\sp\else\^{}\fi}\catcode`\%=\active\def%{\%}$\mathdefault{24}$}}%
\end{pgfscope}%
\begin{pgfscope}%
\pgfsetbuttcap%
\pgfsetroundjoin%
\definecolor{currentfill}{rgb}{0.000000,0.000000,0.000000}%
\pgfsetfillcolor{currentfill}%
\pgfsetlinewidth{0.803000pt}%
\definecolor{currentstroke}{rgb}{0.000000,0.000000,0.000000}%
\pgfsetstrokecolor{currentstroke}%
\pgfsetdash{}{0pt}%
\pgfsys@defobject{currentmarker}{\pgfqpoint{0.000000in}{-0.048611in}}{\pgfqpoint{0.000000in}{0.000000in}}{%
\pgfpathmoveto{\pgfqpoint{0.000000in}{0.000000in}}%
\pgfpathlineto{\pgfqpoint{0.000000in}{-0.048611in}}%
\pgfusepath{stroke,fill}%
}%
\begin{pgfscope}%
\pgfsys@transformshift{2.646080in}{0.521603in}%
\pgfsys@useobject{currentmarker}{}%
\end{pgfscope}%
\end{pgfscope}%
\begin{pgfscope}%
\definecolor{textcolor}{rgb}{0.000000,0.000000,0.000000}%
\pgfsetstrokecolor{textcolor}%
\pgfsetfillcolor{textcolor}%
\pgftext[x=2.646080in,y=0.424381in,,top]{\color{textcolor}{\rmfamily\fontsize{10.000000}{12.000000}\selectfont\catcode`\^=\active\def^{\ifmmode\sp\else\^{}\fi}\catcode`\%=\active\def%{\%}$\mathdefault{30}$}}%
\end{pgfscope}%
\begin{pgfscope}%
\pgfsetbuttcap%
\pgfsetroundjoin%
\definecolor{currentfill}{rgb}{0.000000,0.000000,0.000000}%
\pgfsetfillcolor{currentfill}%
\pgfsetlinewidth{0.803000pt}%
\definecolor{currentstroke}{rgb}{0.000000,0.000000,0.000000}%
\pgfsetstrokecolor{currentstroke}%
\pgfsetdash{}{0pt}%
\pgfsys@defobject{currentmarker}{\pgfqpoint{0.000000in}{-0.048611in}}{\pgfqpoint{0.000000in}{0.000000in}}{%
\pgfpathmoveto{\pgfqpoint{0.000000in}{0.000000in}}%
\pgfpathlineto{\pgfqpoint{0.000000in}{-0.048611in}}%
\pgfusepath{stroke,fill}%
}%
\begin{pgfscope}%
\pgfsys@transformshift{3.199719in}{0.521603in}%
\pgfsys@useobject{currentmarker}{}%
\end{pgfscope}%
\end{pgfscope}%
\begin{pgfscope}%
\definecolor{textcolor}{rgb}{0.000000,0.000000,0.000000}%
\pgfsetstrokecolor{textcolor}%
\pgfsetfillcolor{textcolor}%
\pgftext[x=3.199719in,y=0.424381in,,top]{\color{textcolor}{\rmfamily\fontsize{10.000000}{12.000000}\selectfont\catcode`\^=\active\def^{\ifmmode\sp\else\^{}\fi}\catcode`\%=\active\def%{\%}$\mathdefault{36}$}}%
\end{pgfscope}%
\begin{pgfscope}%
\pgfsetbuttcap%
\pgfsetroundjoin%
\definecolor{currentfill}{rgb}{0.000000,0.000000,0.000000}%
\pgfsetfillcolor{currentfill}%
\pgfsetlinewidth{0.803000pt}%
\definecolor{currentstroke}{rgb}{0.000000,0.000000,0.000000}%
\pgfsetstrokecolor{currentstroke}%
\pgfsetdash{}{0pt}%
\pgfsys@defobject{currentmarker}{\pgfqpoint{0.000000in}{-0.048611in}}{\pgfqpoint{0.000000in}{0.000000in}}{%
\pgfpathmoveto{\pgfqpoint{0.000000in}{0.000000in}}%
\pgfpathlineto{\pgfqpoint{0.000000in}{-0.048611in}}%
\pgfusepath{stroke,fill}%
}%
\begin{pgfscope}%
\pgfsys@transformshift{3.753358in}{0.521603in}%
\pgfsys@useobject{currentmarker}{}%
\end{pgfscope}%
\end{pgfscope}%
\begin{pgfscope}%
\definecolor{textcolor}{rgb}{0.000000,0.000000,0.000000}%
\pgfsetstrokecolor{textcolor}%
\pgfsetfillcolor{textcolor}%
\pgftext[x=3.753358in,y=0.424381in,,top]{\color{textcolor}{\rmfamily\fontsize{10.000000}{12.000000}\selectfont\catcode`\^=\active\def^{\ifmmode\sp\else\^{}\fi}\catcode`\%=\active\def%{\%}$\mathdefault{42}$}}%
\end{pgfscope}%
\begin{pgfscope}%
\pgfsetbuttcap%
\pgfsetroundjoin%
\definecolor{currentfill}{rgb}{0.000000,0.000000,0.000000}%
\pgfsetfillcolor{currentfill}%
\pgfsetlinewidth{0.803000pt}%
\definecolor{currentstroke}{rgb}{0.000000,0.000000,0.000000}%
\pgfsetstrokecolor{currentstroke}%
\pgfsetdash{}{0pt}%
\pgfsys@defobject{currentmarker}{\pgfqpoint{0.000000in}{-0.048611in}}{\pgfqpoint{0.000000in}{0.000000in}}{%
\pgfpathmoveto{\pgfqpoint{0.000000in}{0.000000in}}%
\pgfpathlineto{\pgfqpoint{0.000000in}{-0.048611in}}%
\pgfusepath{stroke,fill}%
}%
\begin{pgfscope}%
\pgfsys@transformshift{4.306997in}{0.521603in}%
\pgfsys@useobject{currentmarker}{}%
\end{pgfscope}%
\end{pgfscope}%
\begin{pgfscope}%
\definecolor{textcolor}{rgb}{0.000000,0.000000,0.000000}%
\pgfsetstrokecolor{textcolor}%
\pgfsetfillcolor{textcolor}%
\pgftext[x=4.306997in,y=0.424381in,,top]{\color{textcolor}{\rmfamily\fontsize{10.000000}{12.000000}\selectfont\catcode`\^=\active\def^{\ifmmode\sp\else\^{}\fi}\catcode`\%=\active\def%{\%}$\mathdefault{48}$}}%
\end{pgfscope}%
\begin{pgfscope}%
\pgfsetbuttcap%
\pgfsetroundjoin%
\definecolor{currentfill}{rgb}{0.000000,0.000000,0.000000}%
\pgfsetfillcolor{currentfill}%
\pgfsetlinewidth{0.803000pt}%
\definecolor{currentstroke}{rgb}{0.000000,0.000000,0.000000}%
\pgfsetstrokecolor{currentstroke}%
\pgfsetdash{}{0pt}%
\pgfsys@defobject{currentmarker}{\pgfqpoint{0.000000in}{-0.048611in}}{\pgfqpoint{0.000000in}{0.000000in}}{%
\pgfpathmoveto{\pgfqpoint{0.000000in}{0.000000in}}%
\pgfpathlineto{\pgfqpoint{0.000000in}{-0.048611in}}%
\pgfusepath{stroke,fill}%
}%
\begin{pgfscope}%
\pgfsys@transformshift{4.860636in}{0.521603in}%
\pgfsys@useobject{currentmarker}{}%
\end{pgfscope}%
\end{pgfscope}%
\begin{pgfscope}%
\definecolor{textcolor}{rgb}{0.000000,0.000000,0.000000}%
\pgfsetstrokecolor{textcolor}%
\pgfsetfillcolor{textcolor}%
\pgftext[x=4.860636in,y=0.424381in,,top]{\color{textcolor}{\rmfamily\fontsize{10.000000}{12.000000}\selectfont\catcode`\^=\active\def^{\ifmmode\sp\else\^{}\fi}\catcode`\%=\active\def%{\%}$\mathdefault{54}$}}%
\end{pgfscope}%
\begin{pgfscope}%
\definecolor{textcolor}{rgb}{0.000000,0.000000,0.000000}%
\pgfsetstrokecolor{textcolor}%
\pgfsetfillcolor{textcolor}%
\pgftext[x=2.922899in,y=0.234413in,,top]{\color{textcolor}{\rmfamily\fontsize{10.000000}{12.000000}\selectfont\catcode`\^=\active\def^{\ifmmode\sp\else\^{}\fi}\catcode`\%=\active\def%{\%}Vertices}}%
\end{pgfscope}%
\begin{pgfscope}%
\pgfsetbuttcap%
\pgfsetroundjoin%
\definecolor{currentfill}{rgb}{0.000000,0.000000,0.000000}%
\pgfsetfillcolor{currentfill}%
\pgfsetlinewidth{0.803000pt}%
\definecolor{currentstroke}{rgb}{0.000000,0.000000,0.000000}%
\pgfsetstrokecolor{currentstroke}%
\pgfsetdash{}{0pt}%
\pgfsys@defobject{currentmarker}{\pgfqpoint{-0.048611in}{0.000000in}}{\pgfqpoint{-0.000000in}{0.000000in}}{%
\pgfpathmoveto{\pgfqpoint{-0.000000in}{0.000000in}}%
\pgfpathlineto{\pgfqpoint{-0.048611in}{0.000000in}}%
\pgfusepath{stroke,fill}%
}%
\begin{pgfscope}%
\pgfsys@transformshift{0.588387in}{1.028209in}%
\pgfsys@useobject{currentmarker}{}%
\end{pgfscope}%
\end{pgfscope}%
\begin{pgfscope}%
\definecolor{textcolor}{rgb}{0.000000,0.000000,0.000000}%
\pgfsetstrokecolor{textcolor}%
\pgfsetfillcolor{textcolor}%
\pgftext[x=0.289968in, y=0.975448in, left, base]{\color{textcolor}{\rmfamily\fontsize{10.000000}{12.000000}\selectfont\catcode`\^=\active\def^{\ifmmode\sp\else\^{}\fi}\catcode`\%=\active\def%{\%}$\mathdefault{10^{1}}$}}%
\end{pgfscope}%
\begin{pgfscope}%
\pgfsetbuttcap%
\pgfsetroundjoin%
\definecolor{currentfill}{rgb}{0.000000,0.000000,0.000000}%
\pgfsetfillcolor{currentfill}%
\pgfsetlinewidth{0.803000pt}%
\definecolor{currentstroke}{rgb}{0.000000,0.000000,0.000000}%
\pgfsetstrokecolor{currentstroke}%
\pgfsetdash{}{0pt}%
\pgfsys@defobject{currentmarker}{\pgfqpoint{-0.048611in}{0.000000in}}{\pgfqpoint{-0.000000in}{0.000000in}}{%
\pgfpathmoveto{\pgfqpoint{-0.000000in}{0.000000in}}%
\pgfpathlineto{\pgfqpoint{-0.048611in}{0.000000in}}%
\pgfusepath{stroke,fill}%
}%
\begin{pgfscope}%
\pgfsys@transformshift{0.588387in}{2.433286in}%
\pgfsys@useobject{currentmarker}{}%
\end{pgfscope}%
\end{pgfscope}%
\begin{pgfscope}%
\definecolor{textcolor}{rgb}{0.000000,0.000000,0.000000}%
\pgfsetstrokecolor{textcolor}%
\pgfsetfillcolor{textcolor}%
\pgftext[x=0.289968in, y=2.380524in, left, base]{\color{textcolor}{\rmfamily\fontsize{10.000000}{12.000000}\selectfont\catcode`\^=\active\def^{\ifmmode\sp\else\^{}\fi}\catcode`\%=\active\def%{\%}$\mathdefault{10^{2}}$}}%
\end{pgfscope}%
\begin{pgfscope}%
\pgfsetbuttcap%
\pgfsetroundjoin%
\definecolor{currentfill}{rgb}{0.000000,0.000000,0.000000}%
\pgfsetfillcolor{currentfill}%
\pgfsetlinewidth{0.602250pt}%
\definecolor{currentstroke}{rgb}{0.000000,0.000000,0.000000}%
\pgfsetstrokecolor{currentstroke}%
\pgfsetdash{}{0pt}%
\pgfsys@defobject{currentmarker}{\pgfqpoint{-0.027778in}{0.000000in}}{\pgfqpoint{-0.000000in}{0.000000in}}{%
\pgfpathmoveto{\pgfqpoint{-0.000000in}{0.000000in}}%
\pgfpathlineto{\pgfqpoint{-0.027778in}{0.000000in}}%
\pgfusepath{stroke,fill}%
}%
\begin{pgfscope}%
\pgfsys@transformshift{0.588387in}{0.605239in}%
\pgfsys@useobject{currentmarker}{}%
\end{pgfscope}%
\end{pgfscope}%
\begin{pgfscope}%
\pgfsetbuttcap%
\pgfsetroundjoin%
\definecolor{currentfill}{rgb}{0.000000,0.000000,0.000000}%
\pgfsetfillcolor{currentfill}%
\pgfsetlinewidth{0.602250pt}%
\definecolor{currentstroke}{rgb}{0.000000,0.000000,0.000000}%
\pgfsetstrokecolor{currentstroke}%
\pgfsetdash{}{0pt}%
\pgfsys@defobject{currentmarker}{\pgfqpoint{-0.027778in}{0.000000in}}{\pgfqpoint{-0.000000in}{0.000000in}}{%
\pgfpathmoveto{\pgfqpoint{-0.000000in}{0.000000in}}%
\pgfpathlineto{\pgfqpoint{-0.027778in}{0.000000in}}%
\pgfusepath{stroke,fill}%
}%
\begin{pgfscope}%
\pgfsys@transformshift{0.588387in}{0.716495in}%
\pgfsys@useobject{currentmarker}{}%
\end{pgfscope}%
\end{pgfscope}%
\begin{pgfscope}%
\pgfsetbuttcap%
\pgfsetroundjoin%
\definecolor{currentfill}{rgb}{0.000000,0.000000,0.000000}%
\pgfsetfillcolor{currentfill}%
\pgfsetlinewidth{0.602250pt}%
\definecolor{currentstroke}{rgb}{0.000000,0.000000,0.000000}%
\pgfsetstrokecolor{currentstroke}%
\pgfsetdash{}{0pt}%
\pgfsys@defobject{currentmarker}{\pgfqpoint{-0.027778in}{0.000000in}}{\pgfqpoint{-0.000000in}{0.000000in}}{%
\pgfpathmoveto{\pgfqpoint{-0.000000in}{0.000000in}}%
\pgfpathlineto{\pgfqpoint{-0.027778in}{0.000000in}}%
\pgfusepath{stroke,fill}%
}%
\begin{pgfscope}%
\pgfsys@transformshift{0.588387in}{0.810560in}%
\pgfsys@useobject{currentmarker}{}%
\end{pgfscope}%
\end{pgfscope}%
\begin{pgfscope}%
\pgfsetbuttcap%
\pgfsetroundjoin%
\definecolor{currentfill}{rgb}{0.000000,0.000000,0.000000}%
\pgfsetfillcolor{currentfill}%
\pgfsetlinewidth{0.602250pt}%
\definecolor{currentstroke}{rgb}{0.000000,0.000000,0.000000}%
\pgfsetstrokecolor{currentstroke}%
\pgfsetdash{}{0pt}%
\pgfsys@defobject{currentmarker}{\pgfqpoint{-0.027778in}{0.000000in}}{\pgfqpoint{-0.000000in}{0.000000in}}{%
\pgfpathmoveto{\pgfqpoint{-0.000000in}{0.000000in}}%
\pgfpathlineto{\pgfqpoint{-0.027778in}{0.000000in}}%
\pgfusepath{stroke,fill}%
}%
\begin{pgfscope}%
\pgfsys@transformshift{0.588387in}{0.892043in}%
\pgfsys@useobject{currentmarker}{}%
\end{pgfscope}%
\end{pgfscope}%
\begin{pgfscope}%
\pgfsetbuttcap%
\pgfsetroundjoin%
\definecolor{currentfill}{rgb}{0.000000,0.000000,0.000000}%
\pgfsetfillcolor{currentfill}%
\pgfsetlinewidth{0.602250pt}%
\definecolor{currentstroke}{rgb}{0.000000,0.000000,0.000000}%
\pgfsetstrokecolor{currentstroke}%
\pgfsetdash{}{0pt}%
\pgfsys@defobject{currentmarker}{\pgfqpoint{-0.027778in}{0.000000in}}{\pgfqpoint{-0.000000in}{0.000000in}}{%
\pgfpathmoveto{\pgfqpoint{-0.000000in}{0.000000in}}%
\pgfpathlineto{\pgfqpoint{-0.027778in}{0.000000in}}%
\pgfusepath{stroke,fill}%
}%
\begin{pgfscope}%
\pgfsys@transformshift{0.588387in}{0.963917in}%
\pgfsys@useobject{currentmarker}{}%
\end{pgfscope}%
\end{pgfscope}%
\begin{pgfscope}%
\pgfsetbuttcap%
\pgfsetroundjoin%
\definecolor{currentfill}{rgb}{0.000000,0.000000,0.000000}%
\pgfsetfillcolor{currentfill}%
\pgfsetlinewidth{0.602250pt}%
\definecolor{currentstroke}{rgb}{0.000000,0.000000,0.000000}%
\pgfsetstrokecolor{currentstroke}%
\pgfsetdash{}{0pt}%
\pgfsys@defobject{currentmarker}{\pgfqpoint{-0.027778in}{0.000000in}}{\pgfqpoint{-0.000000in}{0.000000in}}{%
\pgfpathmoveto{\pgfqpoint{-0.000000in}{0.000000in}}%
\pgfpathlineto{\pgfqpoint{-0.027778in}{0.000000in}}%
\pgfusepath{stroke,fill}%
}%
\begin{pgfscope}%
\pgfsys@transformshift{0.588387in}{1.451180in}%
\pgfsys@useobject{currentmarker}{}%
\end{pgfscope}%
\end{pgfscope}%
\begin{pgfscope}%
\pgfsetbuttcap%
\pgfsetroundjoin%
\definecolor{currentfill}{rgb}{0.000000,0.000000,0.000000}%
\pgfsetfillcolor{currentfill}%
\pgfsetlinewidth{0.602250pt}%
\definecolor{currentstroke}{rgb}{0.000000,0.000000,0.000000}%
\pgfsetstrokecolor{currentstroke}%
\pgfsetdash{}{0pt}%
\pgfsys@defobject{currentmarker}{\pgfqpoint{-0.027778in}{0.000000in}}{\pgfqpoint{-0.000000in}{0.000000in}}{%
\pgfpathmoveto{\pgfqpoint{-0.000000in}{0.000000in}}%
\pgfpathlineto{\pgfqpoint{-0.027778in}{0.000000in}}%
\pgfusepath{stroke,fill}%
}%
\begin{pgfscope}%
\pgfsys@transformshift{0.588387in}{1.698601in}%
\pgfsys@useobject{currentmarker}{}%
\end{pgfscope}%
\end{pgfscope}%
\begin{pgfscope}%
\pgfsetbuttcap%
\pgfsetroundjoin%
\definecolor{currentfill}{rgb}{0.000000,0.000000,0.000000}%
\pgfsetfillcolor{currentfill}%
\pgfsetlinewidth{0.602250pt}%
\definecolor{currentstroke}{rgb}{0.000000,0.000000,0.000000}%
\pgfsetstrokecolor{currentstroke}%
\pgfsetdash{}{0pt}%
\pgfsys@defobject{currentmarker}{\pgfqpoint{-0.027778in}{0.000000in}}{\pgfqpoint{-0.000000in}{0.000000in}}{%
\pgfpathmoveto{\pgfqpoint{-0.000000in}{0.000000in}}%
\pgfpathlineto{\pgfqpoint{-0.027778in}{0.000000in}}%
\pgfusepath{stroke,fill}%
}%
\begin{pgfscope}%
\pgfsys@transformshift{0.588387in}{1.874150in}%
\pgfsys@useobject{currentmarker}{}%
\end{pgfscope}%
\end{pgfscope}%
\begin{pgfscope}%
\pgfsetbuttcap%
\pgfsetroundjoin%
\definecolor{currentfill}{rgb}{0.000000,0.000000,0.000000}%
\pgfsetfillcolor{currentfill}%
\pgfsetlinewidth{0.602250pt}%
\definecolor{currentstroke}{rgb}{0.000000,0.000000,0.000000}%
\pgfsetstrokecolor{currentstroke}%
\pgfsetdash{}{0pt}%
\pgfsys@defobject{currentmarker}{\pgfqpoint{-0.027778in}{0.000000in}}{\pgfqpoint{-0.000000in}{0.000000in}}{%
\pgfpathmoveto{\pgfqpoint{-0.000000in}{0.000000in}}%
\pgfpathlineto{\pgfqpoint{-0.027778in}{0.000000in}}%
\pgfusepath{stroke,fill}%
}%
\begin{pgfscope}%
\pgfsys@transformshift{0.588387in}{2.010316in}%
\pgfsys@useobject{currentmarker}{}%
\end{pgfscope}%
\end{pgfscope}%
\begin{pgfscope}%
\pgfsetbuttcap%
\pgfsetroundjoin%
\definecolor{currentfill}{rgb}{0.000000,0.000000,0.000000}%
\pgfsetfillcolor{currentfill}%
\pgfsetlinewidth{0.602250pt}%
\definecolor{currentstroke}{rgb}{0.000000,0.000000,0.000000}%
\pgfsetstrokecolor{currentstroke}%
\pgfsetdash{}{0pt}%
\pgfsys@defobject{currentmarker}{\pgfqpoint{-0.027778in}{0.000000in}}{\pgfqpoint{-0.000000in}{0.000000in}}{%
\pgfpathmoveto{\pgfqpoint{-0.000000in}{0.000000in}}%
\pgfpathlineto{\pgfqpoint{-0.027778in}{0.000000in}}%
\pgfusepath{stroke,fill}%
}%
\begin{pgfscope}%
\pgfsys@transformshift{0.588387in}{2.121571in}%
\pgfsys@useobject{currentmarker}{}%
\end{pgfscope}%
\end{pgfscope}%
\begin{pgfscope}%
\pgfsetbuttcap%
\pgfsetroundjoin%
\definecolor{currentfill}{rgb}{0.000000,0.000000,0.000000}%
\pgfsetfillcolor{currentfill}%
\pgfsetlinewidth{0.602250pt}%
\definecolor{currentstroke}{rgb}{0.000000,0.000000,0.000000}%
\pgfsetstrokecolor{currentstroke}%
\pgfsetdash{}{0pt}%
\pgfsys@defobject{currentmarker}{\pgfqpoint{-0.027778in}{0.000000in}}{\pgfqpoint{-0.000000in}{0.000000in}}{%
\pgfpathmoveto{\pgfqpoint{-0.000000in}{0.000000in}}%
\pgfpathlineto{\pgfqpoint{-0.027778in}{0.000000in}}%
\pgfusepath{stroke,fill}%
}%
\begin{pgfscope}%
\pgfsys@transformshift{0.588387in}{2.215637in}%
\pgfsys@useobject{currentmarker}{}%
\end{pgfscope}%
\end{pgfscope}%
\begin{pgfscope}%
\pgfsetbuttcap%
\pgfsetroundjoin%
\definecolor{currentfill}{rgb}{0.000000,0.000000,0.000000}%
\pgfsetfillcolor{currentfill}%
\pgfsetlinewidth{0.602250pt}%
\definecolor{currentstroke}{rgb}{0.000000,0.000000,0.000000}%
\pgfsetstrokecolor{currentstroke}%
\pgfsetdash{}{0pt}%
\pgfsys@defobject{currentmarker}{\pgfqpoint{-0.027778in}{0.000000in}}{\pgfqpoint{-0.000000in}{0.000000in}}{%
\pgfpathmoveto{\pgfqpoint{-0.000000in}{0.000000in}}%
\pgfpathlineto{\pgfqpoint{-0.027778in}{0.000000in}}%
\pgfusepath{stroke,fill}%
}%
\begin{pgfscope}%
\pgfsys@transformshift{0.588387in}{2.297120in}%
\pgfsys@useobject{currentmarker}{}%
\end{pgfscope}%
\end{pgfscope}%
\begin{pgfscope}%
\pgfsetbuttcap%
\pgfsetroundjoin%
\definecolor{currentfill}{rgb}{0.000000,0.000000,0.000000}%
\pgfsetfillcolor{currentfill}%
\pgfsetlinewidth{0.602250pt}%
\definecolor{currentstroke}{rgb}{0.000000,0.000000,0.000000}%
\pgfsetstrokecolor{currentstroke}%
\pgfsetdash{}{0pt}%
\pgfsys@defobject{currentmarker}{\pgfqpoint{-0.027778in}{0.000000in}}{\pgfqpoint{-0.000000in}{0.000000in}}{%
\pgfpathmoveto{\pgfqpoint{-0.000000in}{0.000000in}}%
\pgfpathlineto{\pgfqpoint{-0.027778in}{0.000000in}}%
\pgfusepath{stroke,fill}%
}%
\begin{pgfscope}%
\pgfsys@transformshift{0.588387in}{2.368993in}%
\pgfsys@useobject{currentmarker}{}%
\end{pgfscope}%
\end{pgfscope}%
\begin{pgfscope}%
\definecolor{textcolor}{rgb}{0.000000,0.000000,0.000000}%
\pgfsetstrokecolor{textcolor}%
\pgfsetfillcolor{textcolor}%
\pgftext[x=0.234413in,y=1.526746in,,bottom,rotate=90.000000]{\color{textcolor}{\rmfamily\fontsize{10.000000}{12.000000}\selectfont\catcode`\^=\active\def^{\ifmmode\sp\else\^{}\fi}\catcode`\%=\active\def%{\%}Time [ms]}}%
\end{pgfscope}%
\begin{pgfscope}%
\pgfpathrectangle{\pgfqpoint{0.588387in}{0.521603in}}{\pgfqpoint{4.669024in}{2.010285in}}%
\pgfusepath{clip}%
\pgfsetrectcap%
\pgfsetroundjoin%
\pgfsetlinewidth{1.505625pt}%
\pgfsetstrokecolor{currentstroke1}%
\pgfsetdash{}{0pt}%
\pgfpathmoveto{\pgfqpoint{0.800616in}{0.673045in}}%
\pgfpathlineto{\pgfqpoint{0.892889in}{0.744736in}}%
\pgfpathlineto{\pgfqpoint{0.985162in}{0.805200in}}%
\pgfpathlineto{\pgfqpoint{1.077435in}{0.888954in}}%
\pgfpathlineto{\pgfqpoint{1.169708in}{0.957098in}}%
\pgfpathlineto{\pgfqpoint{1.261982in}{1.034281in}}%
\pgfpathlineto{\pgfqpoint{1.354255in}{1.114555in}}%
\pgfpathlineto{\pgfqpoint{1.446528in}{1.162420in}}%
\pgfpathlineto{\pgfqpoint{1.538801in}{1.238739in}}%
\pgfpathlineto{\pgfqpoint{1.631074in}{1.299564in}}%
\pgfpathlineto{\pgfqpoint{1.723348in}{1.315776in}}%
\pgfpathlineto{\pgfqpoint{1.815621in}{1.351649in}}%
\pgfpathlineto{\pgfqpoint{1.907894in}{1.453007in}}%
\pgfpathlineto{\pgfqpoint{2.000167in}{1.438540in}}%
\pgfpathlineto{\pgfqpoint{2.092440in}{1.473867in}}%
\pgfpathlineto{\pgfqpoint{2.184714in}{1.530064in}}%
\pgfpathlineto{\pgfqpoint{2.276987in}{1.561417in}}%
\pgfpathlineto{\pgfqpoint{2.369260in}{1.603959in}}%
\pgfpathlineto{\pgfqpoint{2.461533in}{1.666230in}}%
\pgfpathlineto{\pgfqpoint{2.553806in}{1.707486in}}%
\pgfpathlineto{\pgfqpoint{2.646080in}{1.749319in}}%
\pgfpathlineto{\pgfqpoint{2.738353in}{1.797183in}}%
\pgfpathlineto{\pgfqpoint{2.830626in}{1.816935in}}%
\pgfpathlineto{\pgfqpoint{2.922899in}{1.866084in}}%
\pgfpathlineto{\pgfqpoint{3.015172in}{1.834521in}}%
\pgfpathlineto{\pgfqpoint{3.107446in}{1.897643in}}%
\pgfpathlineto{\pgfqpoint{3.199719in}{1.939413in}}%
\pgfpathlineto{\pgfqpoint{3.291992in}{1.921395in}}%
\pgfpathlineto{\pgfqpoint{3.384265in}{1.998496in}}%
\pgfpathlineto{\pgfqpoint{3.476538in}{2.032930in}}%
\pgfpathlineto{\pgfqpoint{3.568812in}{2.031007in}}%
\pgfpathlineto{\pgfqpoint{3.661085in}{2.048310in}}%
\pgfpathlineto{\pgfqpoint{3.753358in}{2.098776in}}%
\pgfpathlineto{\pgfqpoint{3.845631in}{2.080241in}}%
\pgfpathlineto{\pgfqpoint{3.937904in}{2.098248in}}%
\pgfpathlineto{\pgfqpoint{4.030178in}{2.125667in}}%
\pgfpathlineto{\pgfqpoint{4.122451in}{2.118763in}}%
\pgfpathlineto{\pgfqpoint{4.214724in}{2.170794in}}%
\pgfpathlineto{\pgfqpoint{4.306997in}{2.219236in}}%
\pgfpathlineto{\pgfqpoint{4.399270in}{2.235704in}}%
\pgfpathlineto{\pgfqpoint{4.491544in}{2.264033in}}%
\pgfpathlineto{\pgfqpoint{4.583817in}{2.298063in}}%
\pgfpathlineto{\pgfqpoint{4.676090in}{2.312931in}}%
\pgfpathlineto{\pgfqpoint{4.768363in}{2.297040in}}%
\pgfpathlineto{\pgfqpoint{4.860636in}{2.256833in}}%
\pgfpathlineto{\pgfqpoint{4.952910in}{2.356749in}}%
\pgfpathlineto{\pgfqpoint{5.045183in}{2.406253in}}%
\pgfusepath{stroke}%
\end{pgfscope}%
\begin{pgfscope}%
\pgfpathrectangle{\pgfqpoint{0.588387in}{0.521603in}}{\pgfqpoint{4.669024in}{2.010285in}}%
\pgfusepath{clip}%
\pgfsetrectcap%
\pgfsetroundjoin%
\pgfsetlinewidth{1.505625pt}%
\pgfsetstrokecolor{currentstroke2}%
\pgfsetdash{}{0pt}%
\pgfpathmoveto{\pgfqpoint{0.800616in}{0.671273in}}%
\pgfpathlineto{\pgfqpoint{0.892889in}{0.734428in}}%
\pgfpathlineto{\pgfqpoint{0.985162in}{0.831096in}}%
\pgfpathlineto{\pgfqpoint{1.077435in}{0.954948in}}%
\pgfpathlineto{\pgfqpoint{1.169708in}{0.989152in}}%
\pgfpathlineto{\pgfqpoint{1.261982in}{1.059723in}}%
\pgfpathlineto{\pgfqpoint{1.354255in}{1.114833in}}%
\pgfpathlineto{\pgfqpoint{1.446528in}{1.179324in}}%
\pgfpathlineto{\pgfqpoint{1.538801in}{1.250297in}}%
\pgfpathlineto{\pgfqpoint{1.631074in}{1.334899in}}%
\pgfpathlineto{\pgfqpoint{1.723348in}{1.311924in}}%
\pgfpathlineto{\pgfqpoint{1.815621in}{1.355230in}}%
\pgfpathlineto{\pgfqpoint{1.907894in}{1.454830in}}%
\pgfpathlineto{\pgfqpoint{2.000167in}{1.458157in}}%
\pgfpathlineto{\pgfqpoint{2.092440in}{1.479498in}}%
\pgfpathlineto{\pgfqpoint{2.184714in}{1.524284in}}%
\pgfpathlineto{\pgfqpoint{2.276987in}{1.571019in}}%
\pgfpathlineto{\pgfqpoint{2.369260in}{1.588823in}}%
\pgfpathlineto{\pgfqpoint{2.461533in}{1.672418in}}%
\pgfpathlineto{\pgfqpoint{2.553806in}{1.725461in}}%
\pgfpathlineto{\pgfqpoint{2.646080in}{1.752679in}}%
\pgfpathlineto{\pgfqpoint{2.738353in}{1.782187in}}%
\pgfpathlineto{\pgfqpoint{2.830626in}{1.799946in}}%
\pgfpathlineto{\pgfqpoint{2.922899in}{1.829430in}}%
\pgfpathlineto{\pgfqpoint{3.015172in}{1.836392in}}%
\pgfpathlineto{\pgfqpoint{3.107446in}{1.881654in}}%
\pgfpathlineto{\pgfqpoint{3.199719in}{1.950798in}}%
\pgfpathlineto{\pgfqpoint{3.291992in}{1.921819in}}%
\pgfpathlineto{\pgfqpoint{3.384265in}{1.999270in}}%
\pgfpathlineto{\pgfqpoint{3.476538in}{2.032795in}}%
\pgfpathlineto{\pgfqpoint{3.568812in}{2.035804in}}%
\pgfpathlineto{\pgfqpoint{3.661085in}{2.045174in}}%
\pgfpathlineto{\pgfqpoint{3.753358in}{2.096404in}}%
\pgfpathlineto{\pgfqpoint{3.845631in}{2.085713in}}%
\pgfpathlineto{\pgfqpoint{3.937904in}{2.098459in}}%
\pgfpathlineto{\pgfqpoint{4.030178in}{2.128750in}}%
\pgfpathlineto{\pgfqpoint{4.122451in}{2.118378in}}%
\pgfpathlineto{\pgfqpoint{4.214724in}{2.160763in}}%
\pgfpathlineto{\pgfqpoint{4.306997in}{2.202864in}}%
\pgfpathlineto{\pgfqpoint{4.399270in}{2.231925in}}%
\pgfpathlineto{\pgfqpoint{4.491544in}{2.253035in}}%
\pgfpathlineto{\pgfqpoint{4.583817in}{2.284192in}}%
\pgfpathlineto{\pgfqpoint{4.676090in}{2.300960in}}%
\pgfpathlineto{\pgfqpoint{4.768363in}{2.297906in}}%
\pgfpathlineto{\pgfqpoint{4.860636in}{2.256009in}}%
\pgfpathlineto{\pgfqpoint{4.952910in}{2.354708in}}%
\pgfpathlineto{\pgfqpoint{5.045183in}{2.402215in}}%
\pgfusepath{stroke}%
\end{pgfscope}%
\begin{pgfscope}%
\pgfpathrectangle{\pgfqpoint{0.588387in}{0.521603in}}{\pgfqpoint{4.669024in}{2.010285in}}%
\pgfusepath{clip}%
\pgfsetrectcap%
\pgfsetroundjoin%
\pgfsetlinewidth{1.505625pt}%
\pgfsetstrokecolor{currentstroke3}%
\pgfsetdash{}{0pt}%
\pgfpathmoveto{\pgfqpoint{0.800616in}{0.612980in}}%
\pgfpathlineto{\pgfqpoint{0.892889in}{0.685066in}}%
\pgfpathlineto{\pgfqpoint{0.985162in}{0.730364in}}%
\pgfpathlineto{\pgfqpoint{1.077435in}{0.814259in}}%
\pgfpathlineto{\pgfqpoint{1.169708in}{0.938611in}}%
\pgfpathlineto{\pgfqpoint{1.261982in}{0.867984in}}%
\pgfpathlineto{\pgfqpoint{1.354255in}{0.938948in}}%
\pgfpathlineto{\pgfqpoint{1.446528in}{0.983926in}}%
\pgfpathlineto{\pgfqpoint{1.538801in}{0.993689in}}%
\pgfpathlineto{\pgfqpoint{1.631074in}{1.117198in}}%
\pgfpathlineto{\pgfqpoint{1.723348in}{1.140481in}}%
\pgfpathlineto{\pgfqpoint{1.815621in}{1.160457in}}%
\pgfpathlineto{\pgfqpoint{1.907894in}{1.225021in}}%
\pgfpathlineto{\pgfqpoint{2.000167in}{1.311955in}}%
\pgfpathlineto{\pgfqpoint{2.092440in}{1.356300in}}%
\pgfpathlineto{\pgfqpoint{2.184714in}{1.378696in}}%
\pgfpathlineto{\pgfqpoint{2.276987in}{1.388579in}}%
\pgfpathlineto{\pgfqpoint{2.369260in}{1.451789in}}%
\pgfpathlineto{\pgfqpoint{2.461533in}{1.492466in}}%
\pgfpathlineto{\pgfqpoint{2.553806in}{1.525263in}}%
\pgfpathlineto{\pgfqpoint{2.646080in}{1.523464in}}%
\pgfpathlineto{\pgfqpoint{2.738353in}{1.624512in}}%
\pgfpathlineto{\pgfqpoint{2.830626in}{1.635437in}}%
\pgfpathlineto{\pgfqpoint{2.922899in}{1.670644in}}%
\pgfpathlineto{\pgfqpoint{3.015172in}{1.706884in}}%
\pgfpathlineto{\pgfqpoint{3.107446in}{1.728955in}}%
\pgfpathlineto{\pgfqpoint{3.199719in}{1.767132in}}%
\pgfpathlineto{\pgfqpoint{3.291992in}{1.808669in}}%
\pgfpathlineto{\pgfqpoint{3.384265in}{1.827242in}}%
\pgfpathlineto{\pgfqpoint{3.476538in}{1.871834in}}%
\pgfpathlineto{\pgfqpoint{3.568812in}{1.978887in}}%
\pgfpathlineto{\pgfqpoint{3.661085in}{1.894700in}}%
\pgfpathlineto{\pgfqpoint{3.753358in}{1.934525in}}%
\pgfpathlineto{\pgfqpoint{3.845631in}{1.965507in}}%
\pgfpathlineto{\pgfqpoint{3.937904in}{2.039257in}}%
\pgfpathlineto{\pgfqpoint{4.030178in}{2.009827in}}%
\pgfpathlineto{\pgfqpoint{4.122451in}{1.994240in}}%
\pgfpathlineto{\pgfqpoint{4.214724in}{2.032015in}}%
\pgfpathlineto{\pgfqpoint{4.306997in}{2.098037in}}%
\pgfpathlineto{\pgfqpoint{4.399270in}{2.075864in}}%
\pgfpathlineto{\pgfqpoint{4.491544in}{2.111629in}}%
\pgfpathlineto{\pgfqpoint{4.583817in}{2.252180in}}%
\pgfpathlineto{\pgfqpoint{4.676090in}{2.168252in}}%
\pgfusepath{stroke}%
\end{pgfscope}%
\begin{pgfscope}%
\pgfpathrectangle{\pgfqpoint{0.588387in}{0.521603in}}{\pgfqpoint{4.669024in}{2.010285in}}%
\pgfusepath{clip}%
\pgfsetrectcap%
\pgfsetroundjoin%
\pgfsetlinewidth{1.505625pt}%
\pgfsetstrokecolor{currentstroke4}%
\pgfsetdash{}{0pt}%
\pgfpathmoveto{\pgfqpoint{0.800616in}{0.662545in}}%
\pgfpathlineto{\pgfqpoint{0.892889in}{0.713292in}}%
\pgfpathlineto{\pgfqpoint{0.985162in}{0.800027in}}%
\pgfpathlineto{\pgfqpoint{1.077435in}{0.876831in}}%
\pgfpathlineto{\pgfqpoint{1.169708in}{0.965626in}}%
\pgfpathlineto{\pgfqpoint{1.261982in}{0.987850in}}%
\pgfpathlineto{\pgfqpoint{1.354255in}{1.055069in}}%
\pgfpathlineto{\pgfqpoint{1.446528in}{1.100354in}}%
\pgfpathlineto{\pgfqpoint{1.538801in}{1.164619in}}%
\pgfpathlineto{\pgfqpoint{1.631074in}{1.235054in}}%
\pgfpathlineto{\pgfqpoint{1.723348in}{1.259138in}}%
\pgfpathlineto{\pgfqpoint{1.815621in}{1.295837in}}%
\pgfpathlineto{\pgfqpoint{1.907894in}{1.381951in}}%
\pgfpathlineto{\pgfqpoint{2.000167in}{1.399137in}}%
\pgfpathlineto{\pgfqpoint{2.092440in}{1.448581in}}%
\pgfpathlineto{\pgfqpoint{2.184714in}{1.488281in}}%
\pgfpathlineto{\pgfqpoint{2.276987in}{1.522646in}}%
\pgfpathlineto{\pgfqpoint{2.369260in}{1.562562in}}%
\pgfpathlineto{\pgfqpoint{2.461533in}{1.608692in}}%
\pgfpathlineto{\pgfqpoint{2.553806in}{1.658133in}}%
\pgfpathlineto{\pgfqpoint{2.646080in}{1.665479in}}%
\pgfpathlineto{\pgfqpoint{2.738353in}{1.728812in}}%
\pgfpathlineto{\pgfqpoint{2.830626in}{1.753052in}}%
\pgfpathlineto{\pgfqpoint{2.922899in}{1.784244in}}%
\pgfpathlineto{\pgfqpoint{3.015172in}{1.791008in}}%
\pgfpathlineto{\pgfqpoint{3.107446in}{1.836360in}}%
\pgfpathlineto{\pgfqpoint{3.199719in}{1.881233in}}%
\pgfpathlineto{\pgfqpoint{3.291992in}{1.881655in}}%
\pgfpathlineto{\pgfqpoint{3.384265in}{1.953791in}}%
\pgfpathlineto{\pgfqpoint{3.476538in}{1.987917in}}%
\pgfpathlineto{\pgfqpoint{3.568812in}{2.029259in}}%
\pgfpathlineto{\pgfqpoint{3.661085in}{2.001295in}}%
\pgfpathlineto{\pgfqpoint{3.753358in}{2.041518in}}%
\pgfpathlineto{\pgfqpoint{3.845631in}{2.044592in}}%
\pgfpathlineto{\pgfqpoint{3.937904in}{2.088556in}}%
\pgfpathlineto{\pgfqpoint{4.030178in}{2.091568in}}%
\pgfpathlineto{\pgfqpoint{4.122451in}{2.076989in}}%
\pgfpathlineto{\pgfqpoint{4.214724in}{2.116879in}}%
\pgfpathlineto{\pgfqpoint{4.306997in}{2.175929in}}%
\pgfpathlineto{\pgfqpoint{4.399270in}{2.181654in}}%
\pgfpathlineto{\pgfqpoint{4.491544in}{2.206218in}}%
\pgfpathlineto{\pgfqpoint{4.583817in}{2.291881in}}%
\pgfpathlineto{\pgfqpoint{4.676090in}{2.266024in}}%
\pgfpathlineto{\pgfqpoint{4.768363in}{2.302600in}}%
\pgfpathlineto{\pgfqpoint{4.860636in}{2.253449in}}%
\pgfpathlineto{\pgfqpoint{4.952910in}{2.345433in}}%
\pgfpathlineto{\pgfqpoint{5.045183in}{2.395528in}}%
\pgfusepath{stroke}%
\end{pgfscope}%
\begin{pgfscope}%
\pgfpathrectangle{\pgfqpoint{0.588387in}{0.521603in}}{\pgfqpoint{4.669024in}{2.010285in}}%
\pgfusepath{clip}%
\pgfsetrectcap%
\pgfsetroundjoin%
\pgfsetlinewidth{1.505625pt}%
\pgfsetstrokecolor{currentstroke5}%
\pgfsetdash{}{0pt}%
\pgfpathmoveto{\pgfqpoint{0.800616in}{0.657396in}}%
\pgfpathlineto{\pgfqpoint{0.892889in}{0.718124in}}%
\pgfpathlineto{\pgfqpoint{0.985162in}{0.797713in}}%
\pgfpathlineto{\pgfqpoint{1.077435in}{0.901360in}}%
\pgfpathlineto{\pgfqpoint{1.169708in}{0.970019in}}%
\pgfpathlineto{\pgfqpoint{1.261982in}{1.013698in}}%
\pgfpathlineto{\pgfqpoint{1.354255in}{1.077991in}}%
\pgfpathlineto{\pgfqpoint{1.446528in}{1.127137in}}%
\pgfpathlineto{\pgfqpoint{1.538801in}{1.194177in}}%
\pgfpathlineto{\pgfqpoint{1.631074in}{1.286118in}}%
\pgfpathlineto{\pgfqpoint{1.723348in}{1.260103in}}%
\pgfpathlineto{\pgfqpoint{1.815621in}{1.302036in}}%
\pgfpathlineto{\pgfqpoint{1.907894in}{1.420686in}}%
\pgfpathlineto{\pgfqpoint{2.000167in}{1.392483in}}%
\pgfpathlineto{\pgfqpoint{2.092440in}{1.445971in}}%
\pgfpathlineto{\pgfqpoint{2.184714in}{1.494317in}}%
\pgfpathlineto{\pgfqpoint{2.276987in}{1.521151in}}%
\pgfpathlineto{\pgfqpoint{2.369260in}{1.559862in}}%
\pgfpathlineto{\pgfqpoint{2.461533in}{1.617351in}}%
\pgfpathlineto{\pgfqpoint{2.553806in}{1.661324in}}%
\pgfpathlineto{\pgfqpoint{2.646080in}{1.673137in}}%
\pgfpathlineto{\pgfqpoint{2.738353in}{1.740184in}}%
\pgfpathlineto{\pgfqpoint{2.830626in}{1.759713in}}%
\pgfpathlineto{\pgfqpoint{2.922899in}{1.784868in}}%
\pgfpathlineto{\pgfqpoint{3.015172in}{1.788029in}}%
\pgfpathlineto{\pgfqpoint{3.107446in}{1.838256in}}%
\pgfpathlineto{\pgfqpoint{3.199719in}{1.887335in}}%
\pgfpathlineto{\pgfqpoint{3.291992in}{1.881314in}}%
\pgfpathlineto{\pgfqpoint{3.384265in}{1.950947in}}%
\pgfpathlineto{\pgfqpoint{3.476538in}{1.982923in}}%
\pgfpathlineto{\pgfqpoint{3.568812in}{2.031969in}}%
\pgfpathlineto{\pgfqpoint{3.661085in}{1.993524in}}%
\pgfpathlineto{\pgfqpoint{3.753358in}{2.046103in}}%
\pgfpathlineto{\pgfqpoint{3.845631in}{2.045697in}}%
\pgfpathlineto{\pgfqpoint{3.937904in}{2.092741in}}%
\pgfpathlineto{\pgfqpoint{4.030178in}{2.088234in}}%
\pgfpathlineto{\pgfqpoint{4.122451in}{2.078364in}}%
\pgfpathlineto{\pgfqpoint{4.214724in}{2.125778in}}%
\pgfpathlineto{\pgfqpoint{4.306997in}{2.177368in}}%
\pgfpathlineto{\pgfqpoint{4.399270in}{2.182075in}}%
\pgfpathlineto{\pgfqpoint{4.491544in}{2.224607in}}%
\pgfpathlineto{\pgfqpoint{4.583817in}{2.300922in}}%
\pgfpathlineto{\pgfqpoint{4.676090in}{2.272815in}}%
\pgfpathlineto{\pgfqpoint{4.768363in}{2.321623in}}%
\pgfpathlineto{\pgfqpoint{4.860636in}{2.252233in}}%
\pgfpathlineto{\pgfqpoint{4.952910in}{2.382981in}}%
\pgfpathlineto{\pgfqpoint{5.045183in}{2.440512in}}%
\pgfusepath{stroke}%
\end{pgfscope}%
\begin{pgfscope}%
\pgfpathrectangle{\pgfqpoint{0.588387in}{0.521603in}}{\pgfqpoint{4.669024in}{2.010285in}}%
\pgfusepath{clip}%
\pgfsetrectcap%
\pgfsetroundjoin%
\pgfsetlinewidth{1.505625pt}%
\pgfsetstrokecolor{currentstroke6}%
\pgfsetdash{}{0pt}%
\pgfpathmoveto{\pgfqpoint{0.800616in}{0.672575in}}%
\pgfpathlineto{\pgfqpoint{0.892889in}{0.713151in}}%
\pgfpathlineto{\pgfqpoint{0.985162in}{0.806003in}}%
\pgfpathlineto{\pgfqpoint{1.077435in}{0.876995in}}%
\pgfpathlineto{\pgfqpoint{1.169708in}{0.945752in}}%
\pgfpathlineto{\pgfqpoint{1.261982in}{0.981625in}}%
\pgfpathlineto{\pgfqpoint{1.354255in}{1.059143in}}%
\pgfpathlineto{\pgfqpoint{1.446528in}{1.112166in}}%
\pgfpathlineto{\pgfqpoint{1.538801in}{1.166325in}}%
\pgfpathlineto{\pgfqpoint{1.631074in}{1.232440in}}%
\pgfpathlineto{\pgfqpoint{1.723348in}{1.257044in}}%
\pgfpathlineto{\pgfqpoint{1.815621in}{1.295049in}}%
\pgfpathlineto{\pgfqpoint{1.907894in}{1.381951in}}%
\pgfpathlineto{\pgfqpoint{2.000167in}{1.404265in}}%
\pgfpathlineto{\pgfqpoint{2.092440in}{1.440096in}}%
\pgfpathlineto{\pgfqpoint{2.184714in}{1.510856in}}%
\pgfpathlineto{\pgfqpoint{2.276987in}{1.524137in}}%
\pgfpathlineto{\pgfqpoint{2.369260in}{1.560525in}}%
\pgfpathlineto{\pgfqpoint{2.461533in}{1.616304in}}%
\pgfpathlineto{\pgfqpoint{2.553806in}{1.647743in}}%
\pgfpathlineto{\pgfqpoint{2.646080in}{1.689172in}}%
\pgfpathlineto{\pgfqpoint{2.738353in}{1.733052in}}%
\pgfpathlineto{\pgfqpoint{2.830626in}{1.758700in}}%
\pgfpathlineto{\pgfqpoint{2.922899in}{1.770429in}}%
\pgfpathlineto{\pgfqpoint{3.015172in}{1.795970in}}%
\pgfpathlineto{\pgfqpoint{3.107446in}{1.840597in}}%
\pgfpathlineto{\pgfqpoint{3.199719in}{1.884487in}}%
\pgfpathlineto{\pgfqpoint{3.291992in}{1.880373in}}%
\pgfpathlineto{\pgfqpoint{3.384265in}{1.953208in}}%
\pgfpathlineto{\pgfqpoint{3.476538in}{1.983294in}}%
\pgfpathlineto{\pgfqpoint{3.568812in}{2.028444in}}%
\pgfpathlineto{\pgfqpoint{3.661085in}{1.998300in}}%
\pgfpathlineto{\pgfqpoint{3.753358in}{2.042973in}}%
\pgfpathlineto{\pgfqpoint{3.845631in}{2.046636in}}%
\pgfpathlineto{\pgfqpoint{3.937904in}{2.090592in}}%
\pgfpathlineto{\pgfqpoint{4.030178in}{2.089194in}}%
\pgfpathlineto{\pgfqpoint{4.122451in}{2.078204in}}%
\pgfpathlineto{\pgfqpoint{4.214724in}{2.128771in}}%
\pgfpathlineto{\pgfqpoint{4.306997in}{2.188306in}}%
\pgfpathlineto{\pgfqpoint{4.399270in}{2.181205in}}%
\pgfpathlineto{\pgfqpoint{4.491544in}{2.223011in}}%
\pgfpathlineto{\pgfqpoint{4.583817in}{2.312931in}}%
\pgfpathlineto{\pgfqpoint{4.676090in}{2.279231in}}%
\pgfpathlineto{\pgfqpoint{4.768363in}{2.323611in}}%
\pgfpathlineto{\pgfqpoint{4.860636in}{2.254276in}}%
\pgfpathlineto{\pgfqpoint{4.952910in}{2.368279in}}%
\pgfpathlineto{\pgfqpoint{5.045183in}{2.413470in}}%
\pgfusepath{stroke}%
\end{pgfscope}%
\begin{pgfscope}%
\pgfpathrectangle{\pgfqpoint{0.588387in}{0.521603in}}{\pgfqpoint{4.669024in}{2.010285in}}%
\pgfusepath{clip}%
\pgfsetrectcap%
\pgfsetroundjoin%
\pgfsetlinewidth{1.505625pt}%
\pgfsetstrokecolor{currentstroke7}%
\pgfsetdash{}{0pt}%
\pgfpathmoveto{\pgfqpoint{0.800616in}{0.642233in}}%
\pgfpathlineto{\pgfqpoint{0.892889in}{0.708008in}}%
\pgfpathlineto{\pgfqpoint{0.985162in}{0.803602in}}%
\pgfpathlineto{\pgfqpoint{1.077435in}{0.873071in}}%
\pgfpathlineto{\pgfqpoint{1.169708in}{0.941005in}}%
\pgfpathlineto{\pgfqpoint{1.261982in}{0.973336in}}%
\pgfpathlineto{\pgfqpoint{1.354255in}{1.082753in}}%
\pgfpathlineto{\pgfqpoint{1.446528in}{1.109079in}}%
\pgfpathlineto{\pgfqpoint{1.538801in}{1.182145in}}%
\pgfpathlineto{\pgfqpoint{1.631074in}{1.220693in}}%
\pgfpathlineto{\pgfqpoint{1.723348in}{1.233751in}}%
\pgfpathlineto{\pgfqpoint{1.815621in}{1.289905in}}%
\pgfpathlineto{\pgfqpoint{1.907894in}{1.387229in}}%
\pgfpathlineto{\pgfqpoint{2.000167in}{1.393965in}}%
\pgfpathlineto{\pgfqpoint{2.092440in}{1.423420in}}%
\pgfpathlineto{\pgfqpoint{2.184714in}{1.463142in}}%
\pgfpathlineto{\pgfqpoint{2.276987in}{1.509479in}}%
\pgfpathlineto{\pgfqpoint{2.369260in}{1.559875in}}%
\pgfpathlineto{\pgfqpoint{2.461533in}{1.624282in}}%
\pgfpathlineto{\pgfqpoint{2.553806in}{1.630938in}}%
\pgfpathlineto{\pgfqpoint{2.646080in}{1.679709in}}%
\pgfpathlineto{\pgfqpoint{2.738353in}{1.707507in}}%
\pgfpathlineto{\pgfqpoint{2.830626in}{1.751910in}}%
\pgfpathlineto{\pgfqpoint{2.922899in}{1.783171in}}%
\pgfpathlineto{\pgfqpoint{3.015172in}{1.785741in}}%
\pgfpathlineto{\pgfqpoint{3.107446in}{1.842042in}}%
\pgfpathlineto{\pgfqpoint{3.199719in}{1.877208in}}%
\pgfpathlineto{\pgfqpoint{3.291992in}{1.871076in}}%
\pgfpathlineto{\pgfqpoint{3.384265in}{1.939558in}}%
\pgfpathlineto{\pgfqpoint{3.476538in}{1.969050in}}%
\pgfpathlineto{\pgfqpoint{3.568812in}{2.020485in}}%
\pgfpathlineto{\pgfqpoint{3.661085in}{1.997362in}}%
\pgfpathlineto{\pgfqpoint{3.753358in}{2.034606in}}%
\pgfpathlineto{\pgfqpoint{3.845631in}{2.046457in}}%
\pgfpathlineto{\pgfqpoint{3.937904in}{2.091875in}}%
\pgfpathlineto{\pgfqpoint{4.030178in}{2.089200in}}%
\pgfpathlineto{\pgfqpoint{4.122451in}{2.077088in}}%
\pgfpathlineto{\pgfqpoint{4.214724in}{2.122384in}}%
\pgfpathlineto{\pgfqpoint{4.306997in}{2.182666in}}%
\pgfpathlineto{\pgfqpoint{4.399270in}{2.181474in}}%
\pgfpathlineto{\pgfqpoint{4.491544in}{2.215417in}}%
\pgfpathlineto{\pgfqpoint{4.583817in}{2.290280in}}%
\pgfpathlineto{\pgfqpoint{4.676090in}{2.261607in}}%
\pgfpathlineto{\pgfqpoint{4.768363in}{2.316469in}}%
\pgfpathlineto{\pgfqpoint{4.860636in}{2.255292in}}%
\pgfpathlineto{\pgfqpoint{4.952910in}{2.335608in}}%
\pgfpathlineto{\pgfqpoint{5.045183in}{2.401388in}}%
\pgfusepath{stroke}%
\end{pgfscope}%
\begin{pgfscope}%
\pgfsetrectcap%
\pgfsetmiterjoin%
\pgfsetlinewidth{0.803000pt}%
\definecolor{currentstroke}{rgb}{0.000000,0.000000,0.000000}%
\pgfsetstrokecolor{currentstroke}%
\pgfsetdash{}{0pt}%
\pgfpathmoveto{\pgfqpoint{0.588387in}{0.521603in}}%
\pgfpathlineto{\pgfqpoint{0.588387in}{2.531888in}}%
\pgfusepath{stroke}%
\end{pgfscope}%
\begin{pgfscope}%
\pgfsetrectcap%
\pgfsetmiterjoin%
\pgfsetlinewidth{0.803000pt}%
\definecolor{currentstroke}{rgb}{0.000000,0.000000,0.000000}%
\pgfsetstrokecolor{currentstroke}%
\pgfsetdash{}{0pt}%
\pgfpathmoveto{\pgfqpoint{5.257411in}{0.521603in}}%
\pgfpathlineto{\pgfqpoint{5.257411in}{2.531888in}}%
\pgfusepath{stroke}%
\end{pgfscope}%
\begin{pgfscope}%
\pgfsetrectcap%
\pgfsetmiterjoin%
\pgfsetlinewidth{0.803000pt}%
\definecolor{currentstroke}{rgb}{0.000000,0.000000,0.000000}%
\pgfsetstrokecolor{currentstroke}%
\pgfsetdash{}{0pt}%
\pgfpathmoveto{\pgfqpoint{0.588387in}{0.521603in}}%
\pgfpathlineto{\pgfqpoint{5.257411in}{0.521603in}}%
\pgfusepath{stroke}%
\end{pgfscope}%
\begin{pgfscope}%
\pgfsetrectcap%
\pgfsetmiterjoin%
\pgfsetlinewidth{0.803000pt}%
\definecolor{currentstroke}{rgb}{0.000000,0.000000,0.000000}%
\pgfsetstrokecolor{currentstroke}%
\pgfsetdash{}{0pt}%
\pgfpathmoveto{\pgfqpoint{0.588387in}{2.531888in}}%
\pgfpathlineto{\pgfqpoint{5.257411in}{2.531888in}}%
\pgfusepath{stroke}%
\end{pgfscope}%
\begin{pgfscope}%
\definecolor{textcolor}{rgb}{0.000000,0.000000,0.000000}%
\pgfsetstrokecolor{textcolor}%
\pgfsetfillcolor{textcolor}%
\pgftext[x=2.922899in,y=2.615222in,,base]{\color{textcolor}{\rmfamily\fontsize{12.000000}{14.400000}\selectfont\catcode`\^=\active\def^{\ifmmode\sp\else\^{}\fi}\catcode`\%=\active\def%{\%}Mean}}%
\end{pgfscope}%
\begin{pgfscope}%
\pgfsetbuttcap%
\pgfsetmiterjoin%
\definecolor{currentfill}{rgb}{1.000000,1.000000,1.000000}%
\pgfsetfillcolor{currentfill}%
\pgfsetfillopacity{0.800000}%
\pgfsetlinewidth{1.003750pt}%
\definecolor{currentstroke}{rgb}{0.800000,0.800000,0.800000}%
\pgfsetstrokecolor{currentstroke}%
\pgfsetstrokeopacity{0.800000}%
\pgfsetdash{}{0pt}%
\pgfpathmoveto{\pgfqpoint{5.344911in}{1.133672in}}%
\pgfpathlineto{\pgfqpoint{8.259376in}{1.133672in}}%
\pgfpathquadraticcurveto{\pgfqpoint{8.284376in}{1.133672in}}{\pgfqpoint{8.284376in}{1.158672in}}%
\pgfpathlineto{\pgfqpoint{8.284376in}{2.444388in}}%
\pgfpathquadraticcurveto{\pgfqpoint{8.284376in}{2.469388in}}{\pgfqpoint{8.259376in}{2.469388in}}%
\pgfpathlineto{\pgfqpoint{5.344911in}{2.469388in}}%
\pgfpathquadraticcurveto{\pgfqpoint{5.319911in}{2.469388in}}{\pgfqpoint{5.319911in}{2.444388in}}%
\pgfpathlineto{\pgfqpoint{5.319911in}{1.158672in}}%
\pgfpathquadraticcurveto{\pgfqpoint{5.319911in}{1.133672in}}{\pgfqpoint{5.344911in}{1.133672in}}%
\pgfpathlineto{\pgfqpoint{5.344911in}{1.133672in}}%
\pgfpathclose%
\pgfusepath{stroke,fill}%
\end{pgfscope}%
\begin{pgfscope}%
\pgfsetrectcap%
\pgfsetroundjoin%
\pgfsetlinewidth{1.505625pt}%
\pgfsetstrokecolor{currentstroke3}%
\pgfsetdash{}{0pt}%
\pgfpathmoveto{\pgfqpoint{5.369911in}{2.368168in}}%
\pgfpathlineto{\pgfqpoint{5.494911in}{2.368168in}}%
\pgfpathlineto{\pgfqpoint{5.619911in}{2.368168in}}%
\pgfusepath{stroke}%
\end{pgfscope}%
\begin{pgfscope}%
\definecolor{textcolor}{rgb}{0.000000,0.000000,0.000000}%
\pgfsetstrokecolor{textcolor}%
\pgfsetfillcolor{textcolor}%
\pgftext[x=5.719911in,y=2.324418in,left,base]{\color{textcolor}{\rmfamily\fontsize{9.000000}{10.800000}\selectfont\catcode`\^=\active\def^{\ifmmode\sp\else\^{}\fi}\catcode`\%=\active\def%{\%}\NaiveCycles{}}}%
\end{pgfscope}%
\begin{pgfscope}%
\pgfsetrectcap%
\pgfsetroundjoin%
\pgfsetlinewidth{1.505625pt}%
\pgfsetstrokecolor{currentstroke1}%
\pgfsetdash{}{0pt}%
\pgfpathmoveto{\pgfqpoint{5.369911in}{2.184696in}}%
\pgfpathlineto{\pgfqpoint{5.494911in}{2.184696in}}%
\pgfpathlineto{\pgfqpoint{5.619911in}{2.184696in}}%
\pgfusepath{stroke}%
\end{pgfscope}%
\begin{pgfscope}%
\definecolor{textcolor}{rgb}{0.000000,0.000000,0.000000}%
\pgfsetstrokecolor{textcolor}%
\pgfsetfillcolor{textcolor}%
\pgftext[x=5.719911in,y=2.140946in,left,base]{\color{textcolor}{\rmfamily\fontsize{9.000000}{10.800000}\selectfont\catcode`\^=\active\def^{\ifmmode\sp\else\^{}\fi}\catcode`\%=\active\def%{\%}\CyclesMatchChunks{} \& \MergeLinear{}}}%
\end{pgfscope}%
\begin{pgfscope}%
\pgfsetrectcap%
\pgfsetroundjoin%
\pgfsetlinewidth{1.505625pt}%
\pgfsetstrokecolor{currentstroke2}%
\pgfsetdash{}{0pt}%
\pgfpathmoveto{\pgfqpoint{5.369911in}{1.997746in}}%
\pgfpathlineto{\pgfqpoint{5.494911in}{1.997746in}}%
\pgfpathlineto{\pgfqpoint{5.619911in}{1.997746in}}%
\pgfusepath{stroke}%
\end{pgfscope}%
\begin{pgfscope}%
\definecolor{textcolor}{rgb}{0.000000,0.000000,0.000000}%
\pgfsetstrokecolor{textcolor}%
\pgfsetfillcolor{textcolor}%
\pgftext[x=5.719911in,y=1.953996in,left,base]{\color{textcolor}{\rmfamily\fontsize{9.000000}{10.800000}\selectfont\catcode`\^=\active\def^{\ifmmode\sp\else\^{}\fi}\catcode`\%=\active\def%{\%}\CyclesMatchChunks{} \& \SharedVertices{}}}%
\end{pgfscope}%
\begin{pgfscope}%
\pgfsetrectcap%
\pgfsetroundjoin%
\pgfsetlinewidth{1.505625pt}%
\pgfsetstrokecolor{currentstroke4}%
\pgfsetdash{}{0pt}%
\pgfpathmoveto{\pgfqpoint{5.369911in}{1.810795in}}%
\pgfpathlineto{\pgfqpoint{5.494911in}{1.810795in}}%
\pgfpathlineto{\pgfqpoint{5.619911in}{1.810795in}}%
\pgfusepath{stroke}%
\end{pgfscope}%
\begin{pgfscope}%
\definecolor{textcolor}{rgb}{0.000000,0.000000,0.000000}%
\pgfsetstrokecolor{textcolor}%
\pgfsetfillcolor{textcolor}%
\pgftext[x=5.719911in,y=1.767045in,left,base]{\color{textcolor}{\rmfamily\fontsize{9.000000}{10.800000}\selectfont\catcode`\^=\active\def^{\ifmmode\sp\else\^{}\fi}\catcode`\%=\active\def%{\%}\Neighbors{} \& \MergeLinear{}}}%
\end{pgfscope}%
\begin{pgfscope}%
\pgfsetrectcap%
\pgfsetroundjoin%
\pgfsetlinewidth{1.505625pt}%
\pgfsetstrokecolor{currentstroke5}%
\pgfsetdash{}{0pt}%
\pgfpathmoveto{\pgfqpoint{5.369911in}{1.627324in}}%
\pgfpathlineto{\pgfqpoint{5.494911in}{1.627324in}}%
\pgfpathlineto{\pgfqpoint{5.619911in}{1.627324in}}%
\pgfusepath{stroke}%
\end{pgfscope}%
\begin{pgfscope}%
\definecolor{textcolor}{rgb}{0.000000,0.000000,0.000000}%
\pgfsetstrokecolor{textcolor}%
\pgfsetfillcolor{textcolor}%
\pgftext[x=5.719911in,y=1.583574in,left,base]{\color{textcolor}{\rmfamily\fontsize{9.000000}{10.800000}\selectfont\catcode`\^=\active\def^{\ifmmode\sp\else\^{}\fi}\catcode`\%=\active\def%{\%}\Neighbors{} \& \SharedVertices{}}}%
\end{pgfscope}%
\begin{pgfscope}%
\pgfsetrectcap%
\pgfsetroundjoin%
\pgfsetlinewidth{1.505625pt}%
\pgfsetstrokecolor{currentstroke6}%
\pgfsetdash{}{0pt}%
\pgfpathmoveto{\pgfqpoint{5.369911in}{1.440373in}}%
\pgfpathlineto{\pgfqpoint{5.494911in}{1.440373in}}%
\pgfpathlineto{\pgfqpoint{5.619911in}{1.440373in}}%
\pgfusepath{stroke}%
\end{pgfscope}%
\begin{pgfscope}%
\definecolor{textcolor}{rgb}{0.000000,0.000000,0.000000}%
\pgfsetstrokecolor{textcolor}%
\pgfsetfillcolor{textcolor}%
\pgftext[x=5.719911in,y=1.396623in,left,base]{\color{textcolor}{\rmfamily\fontsize{9.000000}{10.800000}\selectfont\catcode`\^=\active\def^{\ifmmode\sp\else\^{}\fi}\catcode`\%=\active\def%{\%}\None{} \& \MergeLinear{}}}%
\end{pgfscope}%
\begin{pgfscope}%
\pgfsetrectcap%
\pgfsetroundjoin%
\pgfsetlinewidth{1.505625pt}%
\pgfsetstrokecolor{currentstroke7}%
\pgfsetdash{}{0pt}%
\pgfpathmoveto{\pgfqpoint{5.369911in}{1.256902in}}%
\pgfpathlineto{\pgfqpoint{5.494911in}{1.256902in}}%
\pgfpathlineto{\pgfqpoint{5.619911in}{1.256902in}}%
\pgfusepath{stroke}%
\end{pgfscope}%
\begin{pgfscope}%
\definecolor{textcolor}{rgb}{0.000000,0.000000,0.000000}%
\pgfsetstrokecolor{textcolor}%
\pgfsetfillcolor{textcolor}%
\pgftext[x=5.719911in,y=1.213152in,left,base]{\color{textcolor}{\rmfamily\fontsize{9.000000}{10.800000}\selectfont\catcode`\^=\active\def^{\ifmmode\sp\else\^{}\fi}\catcode`\%=\active\def%{\%}\None{} \& \SharedVertices{}}}%
\end{pgfscope}%
\end{pgfpicture}%
\makeatother%
\endgroup%
}
	\caption[Mean runtime for globally rigid graphs (some)]{
		Mean running time to find some NAC-coloring for globally rigid graphs.}%
	\label{fig:graph_globally_rigid_first_runtime}
\end{figure}%
% \begin{figure}[thbp]
% 	\centering
% 	\scalebox{\BenchFigureScale}{%% Creator: Matplotlib, PGF backend
%%
%% To include the figure in your LaTeX document, write
%%   \input{<filename>.pgf}
%%
%% Make sure the required packages are loaded in your preamble
%%   \usepackage{pgf}
%%
%% Also ensure that all the required font packages are loaded; for instance,
%% the lmodern package is sometimes necessary when using math font.
%%   \usepackage{lmodern}
%%
%% Figures using additional raster images can only be included by \input if
%% they are in the same directory as the main LaTeX file. For loading figures
%% from other directories you can use the `import` package
%%   \usepackage{import}
%%
%% and then include the figures with
%%   \import{<path to file>}{<filename>.pgf}
%%
%% Matplotlib used the following preamble
%%   \def\mathdefault#1{#1}
%%   \everymath=\expandafter{\the\everymath\displaystyle}
%%   \IfFileExists{scrextend.sty}{
%%     \usepackage[fontsize=10.000000pt]{scrextend}
%%   }{
%%     \renewcommand{\normalsize}{\fontsize{10.000000}{12.000000}\selectfont}
%%     \normalsize
%%   }
%%   
%%   \ifdefined\pdftexversion\else  % non-pdftex case.
%%     \usepackage{fontspec}
%%     \setmainfont{DejaVuSans.ttf}[Path=\detokenize{/home/petr/Projects/PyRigi/.venv/lib/python3.12/site-packages/matplotlib/mpl-data/fonts/ttf/}]
%%     \setsansfont{DejaVuSans.ttf}[Path=\detokenize{/home/petr/Projects/PyRigi/.venv/lib/python3.12/site-packages/matplotlib/mpl-data/fonts/ttf/}]
%%     \setmonofont{DejaVuSansMono.ttf}[Path=\detokenize{/home/petr/Projects/PyRigi/.venv/lib/python3.12/site-packages/matplotlib/mpl-data/fonts/ttf/}]
%%   \fi
%%   \makeatletter\@ifpackageloaded{underscore}{}{\usepackage[strings]{underscore}}\makeatother
%%
\begingroup%
\makeatletter%
\begin{pgfpicture}%
\pgfpathrectangle{\pgfpointorigin}{\pgfqpoint{8.384376in}{2.841849in}}%
\pgfusepath{use as bounding box, clip}%
\begin{pgfscope}%
\pgfsetbuttcap%
\pgfsetmiterjoin%
\definecolor{currentfill}{rgb}{1.000000,1.000000,1.000000}%
\pgfsetfillcolor{currentfill}%
\pgfsetlinewidth{0.000000pt}%
\definecolor{currentstroke}{rgb}{1.000000,1.000000,1.000000}%
\pgfsetstrokecolor{currentstroke}%
\pgfsetdash{}{0pt}%
\pgfpathmoveto{\pgfqpoint{0.000000in}{0.000000in}}%
\pgfpathlineto{\pgfqpoint{8.384376in}{0.000000in}}%
\pgfpathlineto{\pgfqpoint{8.384376in}{2.841849in}}%
\pgfpathlineto{\pgfqpoint{0.000000in}{2.841849in}}%
\pgfpathlineto{\pgfqpoint{0.000000in}{0.000000in}}%
\pgfpathclose%
\pgfusepath{fill}%
\end{pgfscope}%
\begin{pgfscope}%
\pgfsetbuttcap%
\pgfsetmiterjoin%
\definecolor{currentfill}{rgb}{1.000000,1.000000,1.000000}%
\pgfsetfillcolor{currentfill}%
\pgfsetlinewidth{0.000000pt}%
\definecolor{currentstroke}{rgb}{0.000000,0.000000,0.000000}%
\pgfsetstrokecolor{currentstroke}%
\pgfsetstrokeopacity{0.000000}%
\pgfsetdash{}{0pt}%
\pgfpathmoveto{\pgfqpoint{0.588387in}{0.521603in}}%
\pgfpathlineto{\pgfqpoint{5.257411in}{0.521603in}}%
\pgfpathlineto{\pgfqpoint{5.257411in}{2.531888in}}%
\pgfpathlineto{\pgfqpoint{0.588387in}{2.531888in}}%
\pgfpathlineto{\pgfqpoint{0.588387in}{0.521603in}}%
\pgfpathclose%
\pgfusepath{fill}%
\end{pgfscope}%
\begin{pgfscope}%
\pgfsetbuttcap%
\pgfsetroundjoin%
\definecolor{currentfill}{rgb}{0.000000,0.000000,0.000000}%
\pgfsetfillcolor{currentfill}%
\pgfsetlinewidth{0.803000pt}%
\definecolor{currentstroke}{rgb}{0.000000,0.000000,0.000000}%
\pgfsetstrokecolor{currentstroke}%
\pgfsetdash{}{0pt}%
\pgfsys@defobject{currentmarker}{\pgfqpoint{0.000000in}{-0.048611in}}{\pgfqpoint{0.000000in}{0.000000in}}{%
\pgfpathmoveto{\pgfqpoint{0.000000in}{0.000000in}}%
\pgfpathlineto{\pgfqpoint{0.000000in}{-0.048611in}}%
\pgfusepath{stroke,fill}%
}%
\begin{pgfscope}%
\pgfsys@transformshift{0.673912in}{0.521603in}%
\pgfsys@useobject{currentmarker}{}%
\end{pgfscope}%
\end{pgfscope}%
\begin{pgfscope}%
\definecolor{textcolor}{rgb}{0.000000,0.000000,0.000000}%
\pgfsetstrokecolor{textcolor}%
\pgfsetfillcolor{textcolor}%
\pgftext[x=0.673912in,y=0.424381in,,top]{\color{textcolor}{\rmfamily\fontsize{10.000000}{12.000000}\selectfont\catcode`\^=\active\def^{\ifmmode\sp\else\^{}\fi}\catcode`\%=\active\def%{\%}$\mathdefault{0}$}}%
\end{pgfscope}%
\begin{pgfscope}%
\pgfsetbuttcap%
\pgfsetroundjoin%
\definecolor{currentfill}{rgb}{0.000000,0.000000,0.000000}%
\pgfsetfillcolor{currentfill}%
\pgfsetlinewidth{0.803000pt}%
\definecolor{currentstroke}{rgb}{0.000000,0.000000,0.000000}%
\pgfsetstrokecolor{currentstroke}%
\pgfsetdash{}{0pt}%
\pgfsys@defobject{currentmarker}{\pgfqpoint{0.000000in}{-0.048611in}}{\pgfqpoint{0.000000in}{0.000000in}}{%
\pgfpathmoveto{\pgfqpoint{0.000000in}{0.000000in}}%
\pgfpathlineto{\pgfqpoint{0.000000in}{-0.048611in}}%
\pgfusepath{stroke,fill}%
}%
\begin{pgfscope}%
\pgfsys@transformshift{1.180726in}{0.521603in}%
\pgfsys@useobject{currentmarker}{}%
\end{pgfscope}%
\end{pgfscope}%
\begin{pgfscope}%
\definecolor{textcolor}{rgb}{0.000000,0.000000,0.000000}%
\pgfsetstrokecolor{textcolor}%
\pgfsetfillcolor{textcolor}%
\pgftext[x=1.180726in,y=0.424381in,,top]{\color{textcolor}{\rmfamily\fontsize{10.000000}{12.000000}\selectfont\catcode`\^=\active\def^{\ifmmode\sp\else\^{}\fi}\catcode`\%=\active\def%{\%}$\mathdefault{8}$}}%
\end{pgfscope}%
\begin{pgfscope}%
\pgfsetbuttcap%
\pgfsetroundjoin%
\definecolor{currentfill}{rgb}{0.000000,0.000000,0.000000}%
\pgfsetfillcolor{currentfill}%
\pgfsetlinewidth{0.803000pt}%
\definecolor{currentstroke}{rgb}{0.000000,0.000000,0.000000}%
\pgfsetstrokecolor{currentstroke}%
\pgfsetdash{}{0pt}%
\pgfsys@defobject{currentmarker}{\pgfqpoint{0.000000in}{-0.048611in}}{\pgfqpoint{0.000000in}{0.000000in}}{%
\pgfpathmoveto{\pgfqpoint{0.000000in}{0.000000in}}%
\pgfpathlineto{\pgfqpoint{0.000000in}{-0.048611in}}%
\pgfusepath{stroke,fill}%
}%
\begin{pgfscope}%
\pgfsys@transformshift{1.687540in}{0.521603in}%
\pgfsys@useobject{currentmarker}{}%
\end{pgfscope}%
\end{pgfscope}%
\begin{pgfscope}%
\definecolor{textcolor}{rgb}{0.000000,0.000000,0.000000}%
\pgfsetstrokecolor{textcolor}%
\pgfsetfillcolor{textcolor}%
\pgftext[x=1.687540in,y=0.424381in,,top]{\color{textcolor}{\rmfamily\fontsize{10.000000}{12.000000}\selectfont\catcode`\^=\active\def^{\ifmmode\sp\else\^{}\fi}\catcode`\%=\active\def%{\%}$\mathdefault{16}$}}%
\end{pgfscope}%
\begin{pgfscope}%
\pgfsetbuttcap%
\pgfsetroundjoin%
\definecolor{currentfill}{rgb}{0.000000,0.000000,0.000000}%
\pgfsetfillcolor{currentfill}%
\pgfsetlinewidth{0.803000pt}%
\definecolor{currentstroke}{rgb}{0.000000,0.000000,0.000000}%
\pgfsetstrokecolor{currentstroke}%
\pgfsetdash{}{0pt}%
\pgfsys@defobject{currentmarker}{\pgfqpoint{0.000000in}{-0.048611in}}{\pgfqpoint{0.000000in}{0.000000in}}{%
\pgfpathmoveto{\pgfqpoint{0.000000in}{0.000000in}}%
\pgfpathlineto{\pgfqpoint{0.000000in}{-0.048611in}}%
\pgfusepath{stroke,fill}%
}%
\begin{pgfscope}%
\pgfsys@transformshift{2.194354in}{0.521603in}%
\pgfsys@useobject{currentmarker}{}%
\end{pgfscope}%
\end{pgfscope}%
\begin{pgfscope}%
\definecolor{textcolor}{rgb}{0.000000,0.000000,0.000000}%
\pgfsetstrokecolor{textcolor}%
\pgfsetfillcolor{textcolor}%
\pgftext[x=2.194354in,y=0.424381in,,top]{\color{textcolor}{\rmfamily\fontsize{10.000000}{12.000000}\selectfont\catcode`\^=\active\def^{\ifmmode\sp\else\^{}\fi}\catcode`\%=\active\def%{\%}$\mathdefault{24}$}}%
\end{pgfscope}%
\begin{pgfscope}%
\pgfsetbuttcap%
\pgfsetroundjoin%
\definecolor{currentfill}{rgb}{0.000000,0.000000,0.000000}%
\pgfsetfillcolor{currentfill}%
\pgfsetlinewidth{0.803000pt}%
\definecolor{currentstroke}{rgb}{0.000000,0.000000,0.000000}%
\pgfsetstrokecolor{currentstroke}%
\pgfsetdash{}{0pt}%
\pgfsys@defobject{currentmarker}{\pgfqpoint{0.000000in}{-0.048611in}}{\pgfqpoint{0.000000in}{0.000000in}}{%
\pgfpathmoveto{\pgfqpoint{0.000000in}{0.000000in}}%
\pgfpathlineto{\pgfqpoint{0.000000in}{-0.048611in}}%
\pgfusepath{stroke,fill}%
}%
\begin{pgfscope}%
\pgfsys@transformshift{2.701168in}{0.521603in}%
\pgfsys@useobject{currentmarker}{}%
\end{pgfscope}%
\end{pgfscope}%
\begin{pgfscope}%
\definecolor{textcolor}{rgb}{0.000000,0.000000,0.000000}%
\pgfsetstrokecolor{textcolor}%
\pgfsetfillcolor{textcolor}%
\pgftext[x=2.701168in,y=0.424381in,,top]{\color{textcolor}{\rmfamily\fontsize{10.000000}{12.000000}\selectfont\catcode`\^=\active\def^{\ifmmode\sp\else\^{}\fi}\catcode`\%=\active\def%{\%}$\mathdefault{32}$}}%
\end{pgfscope}%
\begin{pgfscope}%
\pgfsetbuttcap%
\pgfsetroundjoin%
\definecolor{currentfill}{rgb}{0.000000,0.000000,0.000000}%
\pgfsetfillcolor{currentfill}%
\pgfsetlinewidth{0.803000pt}%
\definecolor{currentstroke}{rgb}{0.000000,0.000000,0.000000}%
\pgfsetstrokecolor{currentstroke}%
\pgfsetdash{}{0pt}%
\pgfsys@defobject{currentmarker}{\pgfqpoint{0.000000in}{-0.048611in}}{\pgfqpoint{0.000000in}{0.000000in}}{%
\pgfpathmoveto{\pgfqpoint{0.000000in}{0.000000in}}%
\pgfpathlineto{\pgfqpoint{0.000000in}{-0.048611in}}%
\pgfusepath{stroke,fill}%
}%
\begin{pgfscope}%
\pgfsys@transformshift{3.207982in}{0.521603in}%
\pgfsys@useobject{currentmarker}{}%
\end{pgfscope}%
\end{pgfscope}%
\begin{pgfscope}%
\definecolor{textcolor}{rgb}{0.000000,0.000000,0.000000}%
\pgfsetstrokecolor{textcolor}%
\pgfsetfillcolor{textcolor}%
\pgftext[x=3.207982in,y=0.424381in,,top]{\color{textcolor}{\rmfamily\fontsize{10.000000}{12.000000}\selectfont\catcode`\^=\active\def^{\ifmmode\sp\else\^{}\fi}\catcode`\%=\active\def%{\%}$\mathdefault{40}$}}%
\end{pgfscope}%
\begin{pgfscope}%
\pgfsetbuttcap%
\pgfsetroundjoin%
\definecolor{currentfill}{rgb}{0.000000,0.000000,0.000000}%
\pgfsetfillcolor{currentfill}%
\pgfsetlinewidth{0.803000pt}%
\definecolor{currentstroke}{rgb}{0.000000,0.000000,0.000000}%
\pgfsetstrokecolor{currentstroke}%
\pgfsetdash{}{0pt}%
\pgfsys@defobject{currentmarker}{\pgfqpoint{0.000000in}{-0.048611in}}{\pgfqpoint{0.000000in}{0.000000in}}{%
\pgfpathmoveto{\pgfqpoint{0.000000in}{0.000000in}}%
\pgfpathlineto{\pgfqpoint{0.000000in}{-0.048611in}}%
\pgfusepath{stroke,fill}%
}%
\begin{pgfscope}%
\pgfsys@transformshift{3.714796in}{0.521603in}%
\pgfsys@useobject{currentmarker}{}%
\end{pgfscope}%
\end{pgfscope}%
\begin{pgfscope}%
\definecolor{textcolor}{rgb}{0.000000,0.000000,0.000000}%
\pgfsetstrokecolor{textcolor}%
\pgfsetfillcolor{textcolor}%
\pgftext[x=3.714796in,y=0.424381in,,top]{\color{textcolor}{\rmfamily\fontsize{10.000000}{12.000000}\selectfont\catcode`\^=\active\def^{\ifmmode\sp\else\^{}\fi}\catcode`\%=\active\def%{\%}$\mathdefault{48}$}}%
\end{pgfscope}%
\begin{pgfscope}%
\pgfsetbuttcap%
\pgfsetroundjoin%
\definecolor{currentfill}{rgb}{0.000000,0.000000,0.000000}%
\pgfsetfillcolor{currentfill}%
\pgfsetlinewidth{0.803000pt}%
\definecolor{currentstroke}{rgb}{0.000000,0.000000,0.000000}%
\pgfsetstrokecolor{currentstroke}%
\pgfsetdash{}{0pt}%
\pgfsys@defobject{currentmarker}{\pgfqpoint{0.000000in}{-0.048611in}}{\pgfqpoint{0.000000in}{0.000000in}}{%
\pgfpathmoveto{\pgfqpoint{0.000000in}{0.000000in}}%
\pgfpathlineto{\pgfqpoint{0.000000in}{-0.048611in}}%
\pgfusepath{stroke,fill}%
}%
\begin{pgfscope}%
\pgfsys@transformshift{4.221610in}{0.521603in}%
\pgfsys@useobject{currentmarker}{}%
\end{pgfscope}%
\end{pgfscope}%
\begin{pgfscope}%
\definecolor{textcolor}{rgb}{0.000000,0.000000,0.000000}%
\pgfsetstrokecolor{textcolor}%
\pgfsetfillcolor{textcolor}%
\pgftext[x=4.221610in,y=0.424381in,,top]{\color{textcolor}{\rmfamily\fontsize{10.000000}{12.000000}\selectfont\catcode`\^=\active\def^{\ifmmode\sp\else\^{}\fi}\catcode`\%=\active\def%{\%}$\mathdefault{56}$}}%
\end{pgfscope}%
\begin{pgfscope}%
\pgfsetbuttcap%
\pgfsetroundjoin%
\definecolor{currentfill}{rgb}{0.000000,0.000000,0.000000}%
\pgfsetfillcolor{currentfill}%
\pgfsetlinewidth{0.803000pt}%
\definecolor{currentstroke}{rgb}{0.000000,0.000000,0.000000}%
\pgfsetstrokecolor{currentstroke}%
\pgfsetdash{}{0pt}%
\pgfsys@defobject{currentmarker}{\pgfqpoint{0.000000in}{-0.048611in}}{\pgfqpoint{0.000000in}{0.000000in}}{%
\pgfpathmoveto{\pgfqpoint{0.000000in}{0.000000in}}%
\pgfpathlineto{\pgfqpoint{0.000000in}{-0.048611in}}%
\pgfusepath{stroke,fill}%
}%
\begin{pgfscope}%
\pgfsys@transformshift{4.728424in}{0.521603in}%
\pgfsys@useobject{currentmarker}{}%
\end{pgfscope}%
\end{pgfscope}%
\begin{pgfscope}%
\definecolor{textcolor}{rgb}{0.000000,0.000000,0.000000}%
\pgfsetstrokecolor{textcolor}%
\pgfsetfillcolor{textcolor}%
\pgftext[x=4.728424in,y=0.424381in,,top]{\color{textcolor}{\rmfamily\fontsize{10.000000}{12.000000}\selectfont\catcode`\^=\active\def^{\ifmmode\sp\else\^{}\fi}\catcode`\%=\active\def%{\%}$\mathdefault{64}$}}%
\end{pgfscope}%
\begin{pgfscope}%
\pgfsetbuttcap%
\pgfsetroundjoin%
\definecolor{currentfill}{rgb}{0.000000,0.000000,0.000000}%
\pgfsetfillcolor{currentfill}%
\pgfsetlinewidth{0.803000pt}%
\definecolor{currentstroke}{rgb}{0.000000,0.000000,0.000000}%
\pgfsetstrokecolor{currentstroke}%
\pgfsetdash{}{0pt}%
\pgfsys@defobject{currentmarker}{\pgfqpoint{0.000000in}{-0.048611in}}{\pgfqpoint{0.000000in}{0.000000in}}{%
\pgfpathmoveto{\pgfqpoint{0.000000in}{0.000000in}}%
\pgfpathlineto{\pgfqpoint{0.000000in}{-0.048611in}}%
\pgfusepath{stroke,fill}%
}%
\begin{pgfscope}%
\pgfsys@transformshift{5.235238in}{0.521603in}%
\pgfsys@useobject{currentmarker}{}%
\end{pgfscope}%
\end{pgfscope}%
\begin{pgfscope}%
\definecolor{textcolor}{rgb}{0.000000,0.000000,0.000000}%
\pgfsetstrokecolor{textcolor}%
\pgfsetfillcolor{textcolor}%
\pgftext[x=5.235238in,y=0.424381in,,top]{\color{textcolor}{\rmfamily\fontsize{10.000000}{12.000000}\selectfont\catcode`\^=\active\def^{\ifmmode\sp\else\^{}\fi}\catcode`\%=\active\def%{\%}$\mathdefault{72}$}}%
\end{pgfscope}%
\begin{pgfscope}%
\definecolor{textcolor}{rgb}{0.000000,0.000000,0.000000}%
\pgfsetstrokecolor{textcolor}%
\pgfsetfillcolor{textcolor}%
\pgftext[x=2.922899in,y=0.234413in,,top]{\color{textcolor}{\rmfamily\fontsize{10.000000}{12.000000}\selectfont\catcode`\^=\active\def^{\ifmmode\sp\else\^{}\fi}\catcode`\%=\active\def%{\%}Monochromatic classes}}%
\end{pgfscope}%
\begin{pgfscope}%
\pgfsetbuttcap%
\pgfsetroundjoin%
\definecolor{currentfill}{rgb}{0.000000,0.000000,0.000000}%
\pgfsetfillcolor{currentfill}%
\pgfsetlinewidth{0.803000pt}%
\definecolor{currentstroke}{rgb}{0.000000,0.000000,0.000000}%
\pgfsetstrokecolor{currentstroke}%
\pgfsetdash{}{0pt}%
\pgfsys@defobject{currentmarker}{\pgfqpoint{-0.048611in}{0.000000in}}{\pgfqpoint{-0.000000in}{0.000000in}}{%
\pgfpathmoveto{\pgfqpoint{-0.000000in}{0.000000in}}%
\pgfpathlineto{\pgfqpoint{-0.048611in}{0.000000in}}%
\pgfusepath{stroke,fill}%
}%
\begin{pgfscope}%
\pgfsys@transformshift{0.588387in}{0.612980in}%
\pgfsys@useobject{currentmarker}{}%
\end{pgfscope}%
\end{pgfscope}%
\begin{pgfscope}%
\definecolor{textcolor}{rgb}{0.000000,0.000000,0.000000}%
\pgfsetstrokecolor{textcolor}%
\pgfsetfillcolor{textcolor}%
\pgftext[x=0.289968in, y=0.560218in, left, base]{\color{textcolor}{\rmfamily\fontsize{10.000000}{12.000000}\selectfont\catcode`\^=\active\def^{\ifmmode\sp\else\^{}\fi}\catcode`\%=\active\def%{\%}$\mathdefault{10^{0}}$}}%
\end{pgfscope}%
\begin{pgfscope}%
\pgfsetbuttcap%
\pgfsetroundjoin%
\definecolor{currentfill}{rgb}{0.000000,0.000000,0.000000}%
\pgfsetfillcolor{currentfill}%
\pgfsetlinewidth{0.803000pt}%
\definecolor{currentstroke}{rgb}{0.000000,0.000000,0.000000}%
\pgfsetstrokecolor{currentstroke}%
\pgfsetdash{}{0pt}%
\pgfsys@defobject{currentmarker}{\pgfqpoint{-0.048611in}{0.000000in}}{\pgfqpoint{-0.000000in}{0.000000in}}{%
\pgfpathmoveto{\pgfqpoint{-0.000000in}{0.000000in}}%
\pgfpathlineto{\pgfqpoint{-0.048611in}{0.000000in}}%
\pgfusepath{stroke,fill}%
}%
\begin{pgfscope}%
\pgfsys@transformshift{0.588387in}{1.036849in}%
\pgfsys@useobject{currentmarker}{}%
\end{pgfscope}%
\end{pgfscope}%
\begin{pgfscope}%
\definecolor{textcolor}{rgb}{0.000000,0.000000,0.000000}%
\pgfsetstrokecolor{textcolor}%
\pgfsetfillcolor{textcolor}%
\pgftext[x=0.289968in, y=0.984087in, left, base]{\color{textcolor}{\rmfamily\fontsize{10.000000}{12.000000}\selectfont\catcode`\^=\active\def^{\ifmmode\sp\else\^{}\fi}\catcode`\%=\active\def%{\%}$\mathdefault{10^{1}}$}}%
\end{pgfscope}%
\begin{pgfscope}%
\pgfsetbuttcap%
\pgfsetroundjoin%
\definecolor{currentfill}{rgb}{0.000000,0.000000,0.000000}%
\pgfsetfillcolor{currentfill}%
\pgfsetlinewidth{0.803000pt}%
\definecolor{currentstroke}{rgb}{0.000000,0.000000,0.000000}%
\pgfsetstrokecolor{currentstroke}%
\pgfsetdash{}{0pt}%
\pgfsys@defobject{currentmarker}{\pgfqpoint{-0.048611in}{0.000000in}}{\pgfqpoint{-0.000000in}{0.000000in}}{%
\pgfpathmoveto{\pgfqpoint{-0.000000in}{0.000000in}}%
\pgfpathlineto{\pgfqpoint{-0.048611in}{0.000000in}}%
\pgfusepath{stroke,fill}%
}%
\begin{pgfscope}%
\pgfsys@transformshift{0.588387in}{1.460718in}%
\pgfsys@useobject{currentmarker}{}%
\end{pgfscope}%
\end{pgfscope}%
\begin{pgfscope}%
\definecolor{textcolor}{rgb}{0.000000,0.000000,0.000000}%
\pgfsetstrokecolor{textcolor}%
\pgfsetfillcolor{textcolor}%
\pgftext[x=0.289968in, y=1.407956in, left, base]{\color{textcolor}{\rmfamily\fontsize{10.000000}{12.000000}\selectfont\catcode`\^=\active\def^{\ifmmode\sp\else\^{}\fi}\catcode`\%=\active\def%{\%}$\mathdefault{10^{2}}$}}%
\end{pgfscope}%
\begin{pgfscope}%
\pgfsetbuttcap%
\pgfsetroundjoin%
\definecolor{currentfill}{rgb}{0.000000,0.000000,0.000000}%
\pgfsetfillcolor{currentfill}%
\pgfsetlinewidth{0.803000pt}%
\definecolor{currentstroke}{rgb}{0.000000,0.000000,0.000000}%
\pgfsetstrokecolor{currentstroke}%
\pgfsetdash{}{0pt}%
\pgfsys@defobject{currentmarker}{\pgfqpoint{-0.048611in}{0.000000in}}{\pgfqpoint{-0.000000in}{0.000000in}}{%
\pgfpathmoveto{\pgfqpoint{-0.000000in}{0.000000in}}%
\pgfpathlineto{\pgfqpoint{-0.048611in}{0.000000in}}%
\pgfusepath{stroke,fill}%
}%
\begin{pgfscope}%
\pgfsys@transformshift{0.588387in}{1.884587in}%
\pgfsys@useobject{currentmarker}{}%
\end{pgfscope}%
\end{pgfscope}%
\begin{pgfscope}%
\definecolor{textcolor}{rgb}{0.000000,0.000000,0.000000}%
\pgfsetstrokecolor{textcolor}%
\pgfsetfillcolor{textcolor}%
\pgftext[x=0.289968in, y=1.831825in, left, base]{\color{textcolor}{\rmfamily\fontsize{10.000000}{12.000000}\selectfont\catcode`\^=\active\def^{\ifmmode\sp\else\^{}\fi}\catcode`\%=\active\def%{\%}$\mathdefault{10^{3}}$}}%
\end{pgfscope}%
\begin{pgfscope}%
\pgfsetbuttcap%
\pgfsetroundjoin%
\definecolor{currentfill}{rgb}{0.000000,0.000000,0.000000}%
\pgfsetfillcolor{currentfill}%
\pgfsetlinewidth{0.803000pt}%
\definecolor{currentstroke}{rgb}{0.000000,0.000000,0.000000}%
\pgfsetstrokecolor{currentstroke}%
\pgfsetdash{}{0pt}%
\pgfsys@defobject{currentmarker}{\pgfqpoint{-0.048611in}{0.000000in}}{\pgfqpoint{-0.000000in}{0.000000in}}{%
\pgfpathmoveto{\pgfqpoint{-0.000000in}{0.000000in}}%
\pgfpathlineto{\pgfqpoint{-0.048611in}{0.000000in}}%
\pgfusepath{stroke,fill}%
}%
\begin{pgfscope}%
\pgfsys@transformshift{0.588387in}{2.308456in}%
\pgfsys@useobject{currentmarker}{}%
\end{pgfscope}%
\end{pgfscope}%
\begin{pgfscope}%
\definecolor{textcolor}{rgb}{0.000000,0.000000,0.000000}%
\pgfsetstrokecolor{textcolor}%
\pgfsetfillcolor{textcolor}%
\pgftext[x=0.289968in, y=2.255694in, left, base]{\color{textcolor}{\rmfamily\fontsize{10.000000}{12.000000}\selectfont\catcode`\^=\active\def^{\ifmmode\sp\else\^{}\fi}\catcode`\%=\active\def%{\%}$\mathdefault{10^{4}}$}}%
\end{pgfscope}%
\begin{pgfscope}%
\pgfsetbuttcap%
\pgfsetroundjoin%
\definecolor{currentfill}{rgb}{0.000000,0.000000,0.000000}%
\pgfsetfillcolor{currentfill}%
\pgfsetlinewidth{0.602250pt}%
\definecolor{currentstroke}{rgb}{0.000000,0.000000,0.000000}%
\pgfsetstrokecolor{currentstroke}%
\pgfsetdash{}{0pt}%
\pgfsys@defobject{currentmarker}{\pgfqpoint{-0.027778in}{0.000000in}}{\pgfqpoint{-0.000000in}{0.000000in}}{%
\pgfpathmoveto{\pgfqpoint{-0.000000in}{0.000000in}}%
\pgfpathlineto{\pgfqpoint{-0.027778in}{0.000000in}}%
\pgfusepath{stroke,fill}%
}%
\begin{pgfscope}%
\pgfsys@transformshift{0.588387in}{0.547322in}%
\pgfsys@useobject{currentmarker}{}%
\end{pgfscope}%
\end{pgfscope}%
\begin{pgfscope}%
\pgfsetbuttcap%
\pgfsetroundjoin%
\definecolor{currentfill}{rgb}{0.000000,0.000000,0.000000}%
\pgfsetfillcolor{currentfill}%
\pgfsetlinewidth{0.602250pt}%
\definecolor{currentstroke}{rgb}{0.000000,0.000000,0.000000}%
\pgfsetstrokecolor{currentstroke}%
\pgfsetdash{}{0pt}%
\pgfsys@defobject{currentmarker}{\pgfqpoint{-0.027778in}{0.000000in}}{\pgfqpoint{-0.000000in}{0.000000in}}{%
\pgfpathmoveto{\pgfqpoint{-0.000000in}{0.000000in}}%
\pgfpathlineto{\pgfqpoint{-0.027778in}{0.000000in}}%
\pgfusepath{stroke,fill}%
}%
\begin{pgfscope}%
\pgfsys@transformshift{0.588387in}{0.571903in}%
\pgfsys@useobject{currentmarker}{}%
\end{pgfscope}%
\end{pgfscope}%
\begin{pgfscope}%
\pgfsetbuttcap%
\pgfsetroundjoin%
\definecolor{currentfill}{rgb}{0.000000,0.000000,0.000000}%
\pgfsetfillcolor{currentfill}%
\pgfsetlinewidth{0.602250pt}%
\definecolor{currentstroke}{rgb}{0.000000,0.000000,0.000000}%
\pgfsetstrokecolor{currentstroke}%
\pgfsetdash{}{0pt}%
\pgfsys@defobject{currentmarker}{\pgfqpoint{-0.027778in}{0.000000in}}{\pgfqpoint{-0.000000in}{0.000000in}}{%
\pgfpathmoveto{\pgfqpoint{-0.000000in}{0.000000in}}%
\pgfpathlineto{\pgfqpoint{-0.027778in}{0.000000in}}%
\pgfusepath{stroke,fill}%
}%
\begin{pgfscope}%
\pgfsys@transformshift{0.588387in}{0.593585in}%
\pgfsys@useobject{currentmarker}{}%
\end{pgfscope}%
\end{pgfscope}%
\begin{pgfscope}%
\pgfsetbuttcap%
\pgfsetroundjoin%
\definecolor{currentfill}{rgb}{0.000000,0.000000,0.000000}%
\pgfsetfillcolor{currentfill}%
\pgfsetlinewidth{0.602250pt}%
\definecolor{currentstroke}{rgb}{0.000000,0.000000,0.000000}%
\pgfsetstrokecolor{currentstroke}%
\pgfsetdash{}{0pt}%
\pgfsys@defobject{currentmarker}{\pgfqpoint{-0.027778in}{0.000000in}}{\pgfqpoint{-0.000000in}{0.000000in}}{%
\pgfpathmoveto{\pgfqpoint{-0.000000in}{0.000000in}}%
\pgfpathlineto{\pgfqpoint{-0.027778in}{0.000000in}}%
\pgfusepath{stroke,fill}%
}%
\begin{pgfscope}%
\pgfsys@transformshift{0.588387in}{0.740577in}%
\pgfsys@useobject{currentmarker}{}%
\end{pgfscope}%
\end{pgfscope}%
\begin{pgfscope}%
\pgfsetbuttcap%
\pgfsetroundjoin%
\definecolor{currentfill}{rgb}{0.000000,0.000000,0.000000}%
\pgfsetfillcolor{currentfill}%
\pgfsetlinewidth{0.602250pt}%
\definecolor{currentstroke}{rgb}{0.000000,0.000000,0.000000}%
\pgfsetstrokecolor{currentstroke}%
\pgfsetdash{}{0pt}%
\pgfsys@defobject{currentmarker}{\pgfqpoint{-0.027778in}{0.000000in}}{\pgfqpoint{-0.000000in}{0.000000in}}{%
\pgfpathmoveto{\pgfqpoint{-0.000000in}{0.000000in}}%
\pgfpathlineto{\pgfqpoint{-0.027778in}{0.000000in}}%
\pgfusepath{stroke,fill}%
}%
\begin{pgfscope}%
\pgfsys@transformshift{0.588387in}{0.815217in}%
\pgfsys@useobject{currentmarker}{}%
\end{pgfscope}%
\end{pgfscope}%
\begin{pgfscope}%
\pgfsetbuttcap%
\pgfsetroundjoin%
\definecolor{currentfill}{rgb}{0.000000,0.000000,0.000000}%
\pgfsetfillcolor{currentfill}%
\pgfsetlinewidth{0.602250pt}%
\definecolor{currentstroke}{rgb}{0.000000,0.000000,0.000000}%
\pgfsetstrokecolor{currentstroke}%
\pgfsetdash{}{0pt}%
\pgfsys@defobject{currentmarker}{\pgfqpoint{-0.027778in}{0.000000in}}{\pgfqpoint{-0.000000in}{0.000000in}}{%
\pgfpathmoveto{\pgfqpoint{-0.000000in}{0.000000in}}%
\pgfpathlineto{\pgfqpoint{-0.027778in}{0.000000in}}%
\pgfusepath{stroke,fill}%
}%
\begin{pgfscope}%
\pgfsys@transformshift{0.588387in}{0.868174in}%
\pgfsys@useobject{currentmarker}{}%
\end{pgfscope}%
\end{pgfscope}%
\begin{pgfscope}%
\pgfsetbuttcap%
\pgfsetroundjoin%
\definecolor{currentfill}{rgb}{0.000000,0.000000,0.000000}%
\pgfsetfillcolor{currentfill}%
\pgfsetlinewidth{0.602250pt}%
\definecolor{currentstroke}{rgb}{0.000000,0.000000,0.000000}%
\pgfsetstrokecolor{currentstroke}%
\pgfsetdash{}{0pt}%
\pgfsys@defobject{currentmarker}{\pgfqpoint{-0.027778in}{0.000000in}}{\pgfqpoint{-0.000000in}{0.000000in}}{%
\pgfpathmoveto{\pgfqpoint{-0.000000in}{0.000000in}}%
\pgfpathlineto{\pgfqpoint{-0.027778in}{0.000000in}}%
\pgfusepath{stroke,fill}%
}%
\begin{pgfscope}%
\pgfsys@transformshift{0.588387in}{0.909252in}%
\pgfsys@useobject{currentmarker}{}%
\end{pgfscope}%
\end{pgfscope}%
\begin{pgfscope}%
\pgfsetbuttcap%
\pgfsetroundjoin%
\definecolor{currentfill}{rgb}{0.000000,0.000000,0.000000}%
\pgfsetfillcolor{currentfill}%
\pgfsetlinewidth{0.602250pt}%
\definecolor{currentstroke}{rgb}{0.000000,0.000000,0.000000}%
\pgfsetstrokecolor{currentstroke}%
\pgfsetdash{}{0pt}%
\pgfsys@defobject{currentmarker}{\pgfqpoint{-0.027778in}{0.000000in}}{\pgfqpoint{-0.000000in}{0.000000in}}{%
\pgfpathmoveto{\pgfqpoint{-0.000000in}{0.000000in}}%
\pgfpathlineto{\pgfqpoint{-0.027778in}{0.000000in}}%
\pgfusepath{stroke,fill}%
}%
\begin{pgfscope}%
\pgfsys@transformshift{0.588387in}{0.942814in}%
\pgfsys@useobject{currentmarker}{}%
\end{pgfscope}%
\end{pgfscope}%
\begin{pgfscope}%
\pgfsetbuttcap%
\pgfsetroundjoin%
\definecolor{currentfill}{rgb}{0.000000,0.000000,0.000000}%
\pgfsetfillcolor{currentfill}%
\pgfsetlinewidth{0.602250pt}%
\definecolor{currentstroke}{rgb}{0.000000,0.000000,0.000000}%
\pgfsetstrokecolor{currentstroke}%
\pgfsetdash{}{0pt}%
\pgfsys@defobject{currentmarker}{\pgfqpoint{-0.027778in}{0.000000in}}{\pgfqpoint{-0.000000in}{0.000000in}}{%
\pgfpathmoveto{\pgfqpoint{-0.000000in}{0.000000in}}%
\pgfpathlineto{\pgfqpoint{-0.027778in}{0.000000in}}%
\pgfusepath{stroke,fill}%
}%
\begin{pgfscope}%
\pgfsys@transformshift{0.588387in}{0.971191in}%
\pgfsys@useobject{currentmarker}{}%
\end{pgfscope}%
\end{pgfscope}%
\begin{pgfscope}%
\pgfsetbuttcap%
\pgfsetroundjoin%
\definecolor{currentfill}{rgb}{0.000000,0.000000,0.000000}%
\pgfsetfillcolor{currentfill}%
\pgfsetlinewidth{0.602250pt}%
\definecolor{currentstroke}{rgb}{0.000000,0.000000,0.000000}%
\pgfsetstrokecolor{currentstroke}%
\pgfsetdash{}{0pt}%
\pgfsys@defobject{currentmarker}{\pgfqpoint{-0.027778in}{0.000000in}}{\pgfqpoint{-0.000000in}{0.000000in}}{%
\pgfpathmoveto{\pgfqpoint{-0.000000in}{0.000000in}}%
\pgfpathlineto{\pgfqpoint{-0.027778in}{0.000000in}}%
\pgfusepath{stroke,fill}%
}%
\begin{pgfscope}%
\pgfsys@transformshift{0.588387in}{0.995772in}%
\pgfsys@useobject{currentmarker}{}%
\end{pgfscope}%
\end{pgfscope}%
\begin{pgfscope}%
\pgfsetbuttcap%
\pgfsetroundjoin%
\definecolor{currentfill}{rgb}{0.000000,0.000000,0.000000}%
\pgfsetfillcolor{currentfill}%
\pgfsetlinewidth{0.602250pt}%
\definecolor{currentstroke}{rgb}{0.000000,0.000000,0.000000}%
\pgfsetstrokecolor{currentstroke}%
\pgfsetdash{}{0pt}%
\pgfsys@defobject{currentmarker}{\pgfqpoint{-0.027778in}{0.000000in}}{\pgfqpoint{-0.000000in}{0.000000in}}{%
\pgfpathmoveto{\pgfqpoint{-0.000000in}{0.000000in}}%
\pgfpathlineto{\pgfqpoint{-0.027778in}{0.000000in}}%
\pgfusepath{stroke,fill}%
}%
\begin{pgfscope}%
\pgfsys@transformshift{0.588387in}{1.017454in}%
\pgfsys@useobject{currentmarker}{}%
\end{pgfscope}%
\end{pgfscope}%
\begin{pgfscope}%
\pgfsetbuttcap%
\pgfsetroundjoin%
\definecolor{currentfill}{rgb}{0.000000,0.000000,0.000000}%
\pgfsetfillcolor{currentfill}%
\pgfsetlinewidth{0.602250pt}%
\definecolor{currentstroke}{rgb}{0.000000,0.000000,0.000000}%
\pgfsetstrokecolor{currentstroke}%
\pgfsetdash{}{0pt}%
\pgfsys@defobject{currentmarker}{\pgfqpoint{-0.027778in}{0.000000in}}{\pgfqpoint{-0.000000in}{0.000000in}}{%
\pgfpathmoveto{\pgfqpoint{-0.000000in}{0.000000in}}%
\pgfpathlineto{\pgfqpoint{-0.027778in}{0.000000in}}%
\pgfusepath{stroke,fill}%
}%
\begin{pgfscope}%
\pgfsys@transformshift{0.588387in}{1.164446in}%
\pgfsys@useobject{currentmarker}{}%
\end{pgfscope}%
\end{pgfscope}%
\begin{pgfscope}%
\pgfsetbuttcap%
\pgfsetroundjoin%
\definecolor{currentfill}{rgb}{0.000000,0.000000,0.000000}%
\pgfsetfillcolor{currentfill}%
\pgfsetlinewidth{0.602250pt}%
\definecolor{currentstroke}{rgb}{0.000000,0.000000,0.000000}%
\pgfsetstrokecolor{currentstroke}%
\pgfsetdash{}{0pt}%
\pgfsys@defobject{currentmarker}{\pgfqpoint{-0.027778in}{0.000000in}}{\pgfqpoint{-0.000000in}{0.000000in}}{%
\pgfpathmoveto{\pgfqpoint{-0.000000in}{0.000000in}}%
\pgfpathlineto{\pgfqpoint{-0.027778in}{0.000000in}}%
\pgfusepath{stroke,fill}%
}%
\begin{pgfscope}%
\pgfsys@transformshift{0.588387in}{1.239086in}%
\pgfsys@useobject{currentmarker}{}%
\end{pgfscope}%
\end{pgfscope}%
\begin{pgfscope}%
\pgfsetbuttcap%
\pgfsetroundjoin%
\definecolor{currentfill}{rgb}{0.000000,0.000000,0.000000}%
\pgfsetfillcolor{currentfill}%
\pgfsetlinewidth{0.602250pt}%
\definecolor{currentstroke}{rgb}{0.000000,0.000000,0.000000}%
\pgfsetstrokecolor{currentstroke}%
\pgfsetdash{}{0pt}%
\pgfsys@defobject{currentmarker}{\pgfqpoint{-0.027778in}{0.000000in}}{\pgfqpoint{-0.000000in}{0.000000in}}{%
\pgfpathmoveto{\pgfqpoint{-0.000000in}{0.000000in}}%
\pgfpathlineto{\pgfqpoint{-0.027778in}{0.000000in}}%
\pgfusepath{stroke,fill}%
}%
\begin{pgfscope}%
\pgfsys@transformshift{0.588387in}{1.292043in}%
\pgfsys@useobject{currentmarker}{}%
\end{pgfscope}%
\end{pgfscope}%
\begin{pgfscope}%
\pgfsetbuttcap%
\pgfsetroundjoin%
\definecolor{currentfill}{rgb}{0.000000,0.000000,0.000000}%
\pgfsetfillcolor{currentfill}%
\pgfsetlinewidth{0.602250pt}%
\definecolor{currentstroke}{rgb}{0.000000,0.000000,0.000000}%
\pgfsetstrokecolor{currentstroke}%
\pgfsetdash{}{0pt}%
\pgfsys@defobject{currentmarker}{\pgfqpoint{-0.027778in}{0.000000in}}{\pgfqpoint{-0.000000in}{0.000000in}}{%
\pgfpathmoveto{\pgfqpoint{-0.000000in}{0.000000in}}%
\pgfpathlineto{\pgfqpoint{-0.027778in}{0.000000in}}%
\pgfusepath{stroke,fill}%
}%
\begin{pgfscope}%
\pgfsys@transformshift{0.588387in}{1.333121in}%
\pgfsys@useobject{currentmarker}{}%
\end{pgfscope}%
\end{pgfscope}%
\begin{pgfscope}%
\pgfsetbuttcap%
\pgfsetroundjoin%
\definecolor{currentfill}{rgb}{0.000000,0.000000,0.000000}%
\pgfsetfillcolor{currentfill}%
\pgfsetlinewidth{0.602250pt}%
\definecolor{currentstroke}{rgb}{0.000000,0.000000,0.000000}%
\pgfsetstrokecolor{currentstroke}%
\pgfsetdash{}{0pt}%
\pgfsys@defobject{currentmarker}{\pgfqpoint{-0.027778in}{0.000000in}}{\pgfqpoint{-0.000000in}{0.000000in}}{%
\pgfpathmoveto{\pgfqpoint{-0.000000in}{0.000000in}}%
\pgfpathlineto{\pgfqpoint{-0.027778in}{0.000000in}}%
\pgfusepath{stroke,fill}%
}%
\begin{pgfscope}%
\pgfsys@transformshift{0.588387in}{1.366683in}%
\pgfsys@useobject{currentmarker}{}%
\end{pgfscope}%
\end{pgfscope}%
\begin{pgfscope}%
\pgfsetbuttcap%
\pgfsetroundjoin%
\definecolor{currentfill}{rgb}{0.000000,0.000000,0.000000}%
\pgfsetfillcolor{currentfill}%
\pgfsetlinewidth{0.602250pt}%
\definecolor{currentstroke}{rgb}{0.000000,0.000000,0.000000}%
\pgfsetstrokecolor{currentstroke}%
\pgfsetdash{}{0pt}%
\pgfsys@defobject{currentmarker}{\pgfqpoint{-0.027778in}{0.000000in}}{\pgfqpoint{-0.000000in}{0.000000in}}{%
\pgfpathmoveto{\pgfqpoint{-0.000000in}{0.000000in}}%
\pgfpathlineto{\pgfqpoint{-0.027778in}{0.000000in}}%
\pgfusepath{stroke,fill}%
}%
\begin{pgfscope}%
\pgfsys@transformshift{0.588387in}{1.395060in}%
\pgfsys@useobject{currentmarker}{}%
\end{pgfscope}%
\end{pgfscope}%
\begin{pgfscope}%
\pgfsetbuttcap%
\pgfsetroundjoin%
\definecolor{currentfill}{rgb}{0.000000,0.000000,0.000000}%
\pgfsetfillcolor{currentfill}%
\pgfsetlinewidth{0.602250pt}%
\definecolor{currentstroke}{rgb}{0.000000,0.000000,0.000000}%
\pgfsetstrokecolor{currentstroke}%
\pgfsetdash{}{0pt}%
\pgfsys@defobject{currentmarker}{\pgfqpoint{-0.027778in}{0.000000in}}{\pgfqpoint{-0.000000in}{0.000000in}}{%
\pgfpathmoveto{\pgfqpoint{-0.000000in}{0.000000in}}%
\pgfpathlineto{\pgfqpoint{-0.027778in}{0.000000in}}%
\pgfusepath{stroke,fill}%
}%
\begin{pgfscope}%
\pgfsys@transformshift{0.588387in}{1.419641in}%
\pgfsys@useobject{currentmarker}{}%
\end{pgfscope}%
\end{pgfscope}%
\begin{pgfscope}%
\pgfsetbuttcap%
\pgfsetroundjoin%
\definecolor{currentfill}{rgb}{0.000000,0.000000,0.000000}%
\pgfsetfillcolor{currentfill}%
\pgfsetlinewidth{0.602250pt}%
\definecolor{currentstroke}{rgb}{0.000000,0.000000,0.000000}%
\pgfsetstrokecolor{currentstroke}%
\pgfsetdash{}{0pt}%
\pgfsys@defobject{currentmarker}{\pgfqpoint{-0.027778in}{0.000000in}}{\pgfqpoint{-0.000000in}{0.000000in}}{%
\pgfpathmoveto{\pgfqpoint{-0.000000in}{0.000000in}}%
\pgfpathlineto{\pgfqpoint{-0.027778in}{0.000000in}}%
\pgfusepath{stroke,fill}%
}%
\begin{pgfscope}%
\pgfsys@transformshift{0.588387in}{1.441323in}%
\pgfsys@useobject{currentmarker}{}%
\end{pgfscope}%
\end{pgfscope}%
\begin{pgfscope}%
\pgfsetbuttcap%
\pgfsetroundjoin%
\definecolor{currentfill}{rgb}{0.000000,0.000000,0.000000}%
\pgfsetfillcolor{currentfill}%
\pgfsetlinewidth{0.602250pt}%
\definecolor{currentstroke}{rgb}{0.000000,0.000000,0.000000}%
\pgfsetstrokecolor{currentstroke}%
\pgfsetdash{}{0pt}%
\pgfsys@defobject{currentmarker}{\pgfqpoint{-0.027778in}{0.000000in}}{\pgfqpoint{-0.000000in}{0.000000in}}{%
\pgfpathmoveto{\pgfqpoint{-0.000000in}{0.000000in}}%
\pgfpathlineto{\pgfqpoint{-0.027778in}{0.000000in}}%
\pgfusepath{stroke,fill}%
}%
\begin{pgfscope}%
\pgfsys@transformshift{0.588387in}{1.588315in}%
\pgfsys@useobject{currentmarker}{}%
\end{pgfscope}%
\end{pgfscope}%
\begin{pgfscope}%
\pgfsetbuttcap%
\pgfsetroundjoin%
\definecolor{currentfill}{rgb}{0.000000,0.000000,0.000000}%
\pgfsetfillcolor{currentfill}%
\pgfsetlinewidth{0.602250pt}%
\definecolor{currentstroke}{rgb}{0.000000,0.000000,0.000000}%
\pgfsetstrokecolor{currentstroke}%
\pgfsetdash{}{0pt}%
\pgfsys@defobject{currentmarker}{\pgfqpoint{-0.027778in}{0.000000in}}{\pgfqpoint{-0.000000in}{0.000000in}}{%
\pgfpathmoveto{\pgfqpoint{-0.000000in}{0.000000in}}%
\pgfpathlineto{\pgfqpoint{-0.027778in}{0.000000in}}%
\pgfusepath{stroke,fill}%
}%
\begin{pgfscope}%
\pgfsys@transformshift{0.588387in}{1.662955in}%
\pgfsys@useobject{currentmarker}{}%
\end{pgfscope}%
\end{pgfscope}%
\begin{pgfscope}%
\pgfsetbuttcap%
\pgfsetroundjoin%
\definecolor{currentfill}{rgb}{0.000000,0.000000,0.000000}%
\pgfsetfillcolor{currentfill}%
\pgfsetlinewidth{0.602250pt}%
\definecolor{currentstroke}{rgb}{0.000000,0.000000,0.000000}%
\pgfsetstrokecolor{currentstroke}%
\pgfsetdash{}{0pt}%
\pgfsys@defobject{currentmarker}{\pgfqpoint{-0.027778in}{0.000000in}}{\pgfqpoint{-0.000000in}{0.000000in}}{%
\pgfpathmoveto{\pgfqpoint{-0.000000in}{0.000000in}}%
\pgfpathlineto{\pgfqpoint{-0.027778in}{0.000000in}}%
\pgfusepath{stroke,fill}%
}%
\begin{pgfscope}%
\pgfsys@transformshift{0.588387in}{1.715912in}%
\pgfsys@useobject{currentmarker}{}%
\end{pgfscope}%
\end{pgfscope}%
\begin{pgfscope}%
\pgfsetbuttcap%
\pgfsetroundjoin%
\definecolor{currentfill}{rgb}{0.000000,0.000000,0.000000}%
\pgfsetfillcolor{currentfill}%
\pgfsetlinewidth{0.602250pt}%
\definecolor{currentstroke}{rgb}{0.000000,0.000000,0.000000}%
\pgfsetstrokecolor{currentstroke}%
\pgfsetdash{}{0pt}%
\pgfsys@defobject{currentmarker}{\pgfqpoint{-0.027778in}{0.000000in}}{\pgfqpoint{-0.000000in}{0.000000in}}{%
\pgfpathmoveto{\pgfqpoint{-0.000000in}{0.000000in}}%
\pgfpathlineto{\pgfqpoint{-0.027778in}{0.000000in}}%
\pgfusepath{stroke,fill}%
}%
\begin{pgfscope}%
\pgfsys@transformshift{0.588387in}{1.756989in}%
\pgfsys@useobject{currentmarker}{}%
\end{pgfscope}%
\end{pgfscope}%
\begin{pgfscope}%
\pgfsetbuttcap%
\pgfsetroundjoin%
\definecolor{currentfill}{rgb}{0.000000,0.000000,0.000000}%
\pgfsetfillcolor{currentfill}%
\pgfsetlinewidth{0.602250pt}%
\definecolor{currentstroke}{rgb}{0.000000,0.000000,0.000000}%
\pgfsetstrokecolor{currentstroke}%
\pgfsetdash{}{0pt}%
\pgfsys@defobject{currentmarker}{\pgfqpoint{-0.027778in}{0.000000in}}{\pgfqpoint{-0.000000in}{0.000000in}}{%
\pgfpathmoveto{\pgfqpoint{-0.000000in}{0.000000in}}%
\pgfpathlineto{\pgfqpoint{-0.027778in}{0.000000in}}%
\pgfusepath{stroke,fill}%
}%
\begin{pgfscope}%
\pgfsys@transformshift{0.588387in}{1.790552in}%
\pgfsys@useobject{currentmarker}{}%
\end{pgfscope}%
\end{pgfscope}%
\begin{pgfscope}%
\pgfsetbuttcap%
\pgfsetroundjoin%
\definecolor{currentfill}{rgb}{0.000000,0.000000,0.000000}%
\pgfsetfillcolor{currentfill}%
\pgfsetlinewidth{0.602250pt}%
\definecolor{currentstroke}{rgb}{0.000000,0.000000,0.000000}%
\pgfsetstrokecolor{currentstroke}%
\pgfsetdash{}{0pt}%
\pgfsys@defobject{currentmarker}{\pgfqpoint{-0.027778in}{0.000000in}}{\pgfqpoint{-0.000000in}{0.000000in}}{%
\pgfpathmoveto{\pgfqpoint{-0.000000in}{0.000000in}}%
\pgfpathlineto{\pgfqpoint{-0.027778in}{0.000000in}}%
\pgfusepath{stroke,fill}%
}%
\begin{pgfscope}%
\pgfsys@transformshift{0.588387in}{1.818929in}%
\pgfsys@useobject{currentmarker}{}%
\end{pgfscope}%
\end{pgfscope}%
\begin{pgfscope}%
\pgfsetbuttcap%
\pgfsetroundjoin%
\definecolor{currentfill}{rgb}{0.000000,0.000000,0.000000}%
\pgfsetfillcolor{currentfill}%
\pgfsetlinewidth{0.602250pt}%
\definecolor{currentstroke}{rgb}{0.000000,0.000000,0.000000}%
\pgfsetstrokecolor{currentstroke}%
\pgfsetdash{}{0pt}%
\pgfsys@defobject{currentmarker}{\pgfqpoint{-0.027778in}{0.000000in}}{\pgfqpoint{-0.000000in}{0.000000in}}{%
\pgfpathmoveto{\pgfqpoint{-0.000000in}{0.000000in}}%
\pgfpathlineto{\pgfqpoint{-0.027778in}{0.000000in}}%
\pgfusepath{stroke,fill}%
}%
\begin{pgfscope}%
\pgfsys@transformshift{0.588387in}{1.843510in}%
\pgfsys@useobject{currentmarker}{}%
\end{pgfscope}%
\end{pgfscope}%
\begin{pgfscope}%
\pgfsetbuttcap%
\pgfsetroundjoin%
\definecolor{currentfill}{rgb}{0.000000,0.000000,0.000000}%
\pgfsetfillcolor{currentfill}%
\pgfsetlinewidth{0.602250pt}%
\definecolor{currentstroke}{rgb}{0.000000,0.000000,0.000000}%
\pgfsetstrokecolor{currentstroke}%
\pgfsetdash{}{0pt}%
\pgfsys@defobject{currentmarker}{\pgfqpoint{-0.027778in}{0.000000in}}{\pgfqpoint{-0.000000in}{0.000000in}}{%
\pgfpathmoveto{\pgfqpoint{-0.000000in}{0.000000in}}%
\pgfpathlineto{\pgfqpoint{-0.027778in}{0.000000in}}%
\pgfusepath{stroke,fill}%
}%
\begin{pgfscope}%
\pgfsys@transformshift{0.588387in}{1.865192in}%
\pgfsys@useobject{currentmarker}{}%
\end{pgfscope}%
\end{pgfscope}%
\begin{pgfscope}%
\pgfsetbuttcap%
\pgfsetroundjoin%
\definecolor{currentfill}{rgb}{0.000000,0.000000,0.000000}%
\pgfsetfillcolor{currentfill}%
\pgfsetlinewidth{0.602250pt}%
\definecolor{currentstroke}{rgb}{0.000000,0.000000,0.000000}%
\pgfsetstrokecolor{currentstroke}%
\pgfsetdash{}{0pt}%
\pgfsys@defobject{currentmarker}{\pgfqpoint{-0.027778in}{0.000000in}}{\pgfqpoint{-0.000000in}{0.000000in}}{%
\pgfpathmoveto{\pgfqpoint{-0.000000in}{0.000000in}}%
\pgfpathlineto{\pgfqpoint{-0.027778in}{0.000000in}}%
\pgfusepath{stroke,fill}%
}%
\begin{pgfscope}%
\pgfsys@transformshift{0.588387in}{2.012184in}%
\pgfsys@useobject{currentmarker}{}%
\end{pgfscope}%
\end{pgfscope}%
\begin{pgfscope}%
\pgfsetbuttcap%
\pgfsetroundjoin%
\definecolor{currentfill}{rgb}{0.000000,0.000000,0.000000}%
\pgfsetfillcolor{currentfill}%
\pgfsetlinewidth{0.602250pt}%
\definecolor{currentstroke}{rgb}{0.000000,0.000000,0.000000}%
\pgfsetstrokecolor{currentstroke}%
\pgfsetdash{}{0pt}%
\pgfsys@defobject{currentmarker}{\pgfqpoint{-0.027778in}{0.000000in}}{\pgfqpoint{-0.000000in}{0.000000in}}{%
\pgfpathmoveto{\pgfqpoint{-0.000000in}{0.000000in}}%
\pgfpathlineto{\pgfqpoint{-0.027778in}{0.000000in}}%
\pgfusepath{stroke,fill}%
}%
\begin{pgfscope}%
\pgfsys@transformshift{0.588387in}{2.086824in}%
\pgfsys@useobject{currentmarker}{}%
\end{pgfscope}%
\end{pgfscope}%
\begin{pgfscope}%
\pgfsetbuttcap%
\pgfsetroundjoin%
\definecolor{currentfill}{rgb}{0.000000,0.000000,0.000000}%
\pgfsetfillcolor{currentfill}%
\pgfsetlinewidth{0.602250pt}%
\definecolor{currentstroke}{rgb}{0.000000,0.000000,0.000000}%
\pgfsetstrokecolor{currentstroke}%
\pgfsetdash{}{0pt}%
\pgfsys@defobject{currentmarker}{\pgfqpoint{-0.027778in}{0.000000in}}{\pgfqpoint{-0.000000in}{0.000000in}}{%
\pgfpathmoveto{\pgfqpoint{-0.000000in}{0.000000in}}%
\pgfpathlineto{\pgfqpoint{-0.027778in}{0.000000in}}%
\pgfusepath{stroke,fill}%
}%
\begin{pgfscope}%
\pgfsys@transformshift{0.588387in}{2.139781in}%
\pgfsys@useobject{currentmarker}{}%
\end{pgfscope}%
\end{pgfscope}%
\begin{pgfscope}%
\pgfsetbuttcap%
\pgfsetroundjoin%
\definecolor{currentfill}{rgb}{0.000000,0.000000,0.000000}%
\pgfsetfillcolor{currentfill}%
\pgfsetlinewidth{0.602250pt}%
\definecolor{currentstroke}{rgb}{0.000000,0.000000,0.000000}%
\pgfsetstrokecolor{currentstroke}%
\pgfsetdash{}{0pt}%
\pgfsys@defobject{currentmarker}{\pgfqpoint{-0.027778in}{0.000000in}}{\pgfqpoint{-0.000000in}{0.000000in}}{%
\pgfpathmoveto{\pgfqpoint{-0.000000in}{0.000000in}}%
\pgfpathlineto{\pgfqpoint{-0.027778in}{0.000000in}}%
\pgfusepath{stroke,fill}%
}%
\begin{pgfscope}%
\pgfsys@transformshift{0.588387in}{2.180858in}%
\pgfsys@useobject{currentmarker}{}%
\end{pgfscope}%
\end{pgfscope}%
\begin{pgfscope}%
\pgfsetbuttcap%
\pgfsetroundjoin%
\definecolor{currentfill}{rgb}{0.000000,0.000000,0.000000}%
\pgfsetfillcolor{currentfill}%
\pgfsetlinewidth{0.602250pt}%
\definecolor{currentstroke}{rgb}{0.000000,0.000000,0.000000}%
\pgfsetstrokecolor{currentstroke}%
\pgfsetdash{}{0pt}%
\pgfsys@defobject{currentmarker}{\pgfqpoint{-0.027778in}{0.000000in}}{\pgfqpoint{-0.000000in}{0.000000in}}{%
\pgfpathmoveto{\pgfqpoint{-0.000000in}{0.000000in}}%
\pgfpathlineto{\pgfqpoint{-0.027778in}{0.000000in}}%
\pgfusepath{stroke,fill}%
}%
\begin{pgfscope}%
\pgfsys@transformshift{0.588387in}{2.214421in}%
\pgfsys@useobject{currentmarker}{}%
\end{pgfscope}%
\end{pgfscope}%
\begin{pgfscope}%
\pgfsetbuttcap%
\pgfsetroundjoin%
\definecolor{currentfill}{rgb}{0.000000,0.000000,0.000000}%
\pgfsetfillcolor{currentfill}%
\pgfsetlinewidth{0.602250pt}%
\definecolor{currentstroke}{rgb}{0.000000,0.000000,0.000000}%
\pgfsetstrokecolor{currentstroke}%
\pgfsetdash{}{0pt}%
\pgfsys@defobject{currentmarker}{\pgfqpoint{-0.027778in}{0.000000in}}{\pgfqpoint{-0.000000in}{0.000000in}}{%
\pgfpathmoveto{\pgfqpoint{-0.000000in}{0.000000in}}%
\pgfpathlineto{\pgfqpoint{-0.027778in}{0.000000in}}%
\pgfusepath{stroke,fill}%
}%
\begin{pgfscope}%
\pgfsys@transformshift{0.588387in}{2.242798in}%
\pgfsys@useobject{currentmarker}{}%
\end{pgfscope}%
\end{pgfscope}%
\begin{pgfscope}%
\pgfsetbuttcap%
\pgfsetroundjoin%
\definecolor{currentfill}{rgb}{0.000000,0.000000,0.000000}%
\pgfsetfillcolor{currentfill}%
\pgfsetlinewidth{0.602250pt}%
\definecolor{currentstroke}{rgb}{0.000000,0.000000,0.000000}%
\pgfsetstrokecolor{currentstroke}%
\pgfsetdash{}{0pt}%
\pgfsys@defobject{currentmarker}{\pgfqpoint{-0.027778in}{0.000000in}}{\pgfqpoint{-0.000000in}{0.000000in}}{%
\pgfpathmoveto{\pgfqpoint{-0.000000in}{0.000000in}}%
\pgfpathlineto{\pgfqpoint{-0.027778in}{0.000000in}}%
\pgfusepath{stroke,fill}%
}%
\begin{pgfscope}%
\pgfsys@transformshift{0.588387in}{2.267379in}%
\pgfsys@useobject{currentmarker}{}%
\end{pgfscope}%
\end{pgfscope}%
\begin{pgfscope}%
\pgfsetbuttcap%
\pgfsetroundjoin%
\definecolor{currentfill}{rgb}{0.000000,0.000000,0.000000}%
\pgfsetfillcolor{currentfill}%
\pgfsetlinewidth{0.602250pt}%
\definecolor{currentstroke}{rgb}{0.000000,0.000000,0.000000}%
\pgfsetstrokecolor{currentstroke}%
\pgfsetdash{}{0pt}%
\pgfsys@defobject{currentmarker}{\pgfqpoint{-0.027778in}{0.000000in}}{\pgfqpoint{-0.000000in}{0.000000in}}{%
\pgfpathmoveto{\pgfqpoint{-0.000000in}{0.000000in}}%
\pgfpathlineto{\pgfqpoint{-0.027778in}{0.000000in}}%
\pgfusepath{stroke,fill}%
}%
\begin{pgfscope}%
\pgfsys@transformshift{0.588387in}{2.289061in}%
\pgfsys@useobject{currentmarker}{}%
\end{pgfscope}%
\end{pgfscope}%
\begin{pgfscope}%
\pgfsetbuttcap%
\pgfsetroundjoin%
\definecolor{currentfill}{rgb}{0.000000,0.000000,0.000000}%
\pgfsetfillcolor{currentfill}%
\pgfsetlinewidth{0.602250pt}%
\definecolor{currentstroke}{rgb}{0.000000,0.000000,0.000000}%
\pgfsetstrokecolor{currentstroke}%
\pgfsetdash{}{0pt}%
\pgfsys@defobject{currentmarker}{\pgfqpoint{-0.027778in}{0.000000in}}{\pgfqpoint{-0.000000in}{0.000000in}}{%
\pgfpathmoveto{\pgfqpoint{-0.000000in}{0.000000in}}%
\pgfpathlineto{\pgfqpoint{-0.027778in}{0.000000in}}%
\pgfusepath{stroke,fill}%
}%
\begin{pgfscope}%
\pgfsys@transformshift{0.588387in}{2.436053in}%
\pgfsys@useobject{currentmarker}{}%
\end{pgfscope}%
\end{pgfscope}%
\begin{pgfscope}%
\pgfsetbuttcap%
\pgfsetroundjoin%
\definecolor{currentfill}{rgb}{0.000000,0.000000,0.000000}%
\pgfsetfillcolor{currentfill}%
\pgfsetlinewidth{0.602250pt}%
\definecolor{currentstroke}{rgb}{0.000000,0.000000,0.000000}%
\pgfsetstrokecolor{currentstroke}%
\pgfsetdash{}{0pt}%
\pgfsys@defobject{currentmarker}{\pgfqpoint{-0.027778in}{0.000000in}}{\pgfqpoint{-0.000000in}{0.000000in}}{%
\pgfpathmoveto{\pgfqpoint{-0.000000in}{0.000000in}}%
\pgfpathlineto{\pgfqpoint{-0.027778in}{0.000000in}}%
\pgfusepath{stroke,fill}%
}%
\begin{pgfscope}%
\pgfsys@transformshift{0.588387in}{2.510693in}%
\pgfsys@useobject{currentmarker}{}%
\end{pgfscope}%
\end{pgfscope}%
\begin{pgfscope}%
\definecolor{textcolor}{rgb}{0.000000,0.000000,0.000000}%
\pgfsetstrokecolor{textcolor}%
\pgfsetfillcolor{textcolor}%
\pgftext[x=0.234413in,y=1.526746in,,bottom,rotate=90.000000]{\color{textcolor}{\rmfamily\fontsize{10.000000}{12.000000}\selectfont\catcode`\^=\active\def^{\ifmmode\sp\else\^{}\fi}\catcode`\%=\active\def%{\%}Checks [call]}}%
\end{pgfscope}%
\begin{pgfscope}%
\pgfpathrectangle{\pgfqpoint{0.588387in}{0.521603in}}{\pgfqpoint{4.669024in}{2.010285in}}%
\pgfusepath{clip}%
\pgfsetrectcap%
\pgfsetroundjoin%
\pgfsetlinewidth{1.505625pt}%
\pgfsetstrokecolor{currentstroke1}%
\pgfsetdash{}{0pt}%
\pgfpathmoveto{\pgfqpoint{0.800616in}{0.740577in}}%
\pgfpathlineto{\pgfqpoint{0.863967in}{0.768728in}}%
\pgfpathlineto{\pgfqpoint{0.927319in}{0.827345in}}%
\pgfpathlineto{\pgfqpoint{0.990671in}{0.859901in}}%
\pgfpathlineto{\pgfqpoint{1.054023in}{0.916383in}}%
\pgfpathlineto{\pgfqpoint{1.117374in}{0.976538in}}%
\pgfpathlineto{\pgfqpoint{1.180726in}{1.102193in}}%
\pgfpathlineto{\pgfqpoint{1.244078in}{1.126293in}}%
\pgfpathlineto{\pgfqpoint{1.307430in}{1.201940in}}%
\pgfpathlineto{\pgfqpoint{1.370781in}{1.201924in}}%
\pgfpathlineto{\pgfqpoint{1.434133in}{1.240522in}}%
\pgfpathlineto{\pgfqpoint{1.497485in}{1.242072in}}%
\pgfpathlineto{\pgfqpoint{1.560837in}{1.261808in}}%
\pgfpathlineto{\pgfqpoint{1.624188in}{1.348557in}}%
\pgfpathlineto{\pgfqpoint{1.687540in}{1.302066in}}%
\pgfpathlineto{\pgfqpoint{1.750892in}{1.305020in}}%
\pgfpathlineto{\pgfqpoint{1.814244in}{1.321821in}}%
\pgfpathlineto{\pgfqpoint{1.877595in}{1.375079in}}%
\pgfpathlineto{\pgfqpoint{1.940947in}{1.362261in}}%
\pgfpathlineto{\pgfqpoint{2.004299in}{1.374001in}}%
\pgfpathlineto{\pgfqpoint{2.067651in}{1.455472in}}%
\pgfpathlineto{\pgfqpoint{2.131002in}{1.553274in}}%
\pgfpathlineto{\pgfqpoint{2.194354in}{1.412870in}}%
\pgfpathlineto{\pgfqpoint{2.257706in}{1.362022in}}%
\pgfpathlineto{\pgfqpoint{2.321058in}{1.493203in}}%
\pgfpathlineto{\pgfqpoint{2.384409in}{1.519340in}}%
\pgfpathlineto{\pgfqpoint{2.447761in}{1.440640in}}%
\pgfpathlineto{\pgfqpoint{2.511113in}{1.423436in}}%
\pgfpathlineto{\pgfqpoint{2.574465in}{1.470864in}}%
\pgfpathlineto{\pgfqpoint{2.637816in}{1.485040in}}%
\pgfpathlineto{\pgfqpoint{2.701168in}{1.472310in}}%
\pgfpathlineto{\pgfqpoint{2.764520in}{1.453841in}}%
\pgfpathlineto{\pgfqpoint{2.827872in}{1.428622in}}%
\pgfpathlineto{\pgfqpoint{2.891223in}{1.508542in}}%
\pgfpathlineto{\pgfqpoint{2.954575in}{1.470283in}}%
\pgfpathlineto{\pgfqpoint{3.017927in}{1.457936in}}%
\pgfpathlineto{\pgfqpoint{3.081279in}{1.484434in}}%
\pgfpathlineto{\pgfqpoint{3.144630in}{1.612435in}}%
\pgfpathlineto{\pgfqpoint{3.207982in}{1.528481in}}%
\pgfpathlineto{\pgfqpoint{3.271334in}{1.488831in}}%
\pgfpathlineto{\pgfqpoint{3.334686in}{1.476582in}}%
\pgfpathlineto{\pgfqpoint{3.524741in}{1.506161in}}%
\pgfpathlineto{\pgfqpoint{3.714796in}{1.546662in}}%
\pgfpathlineto{\pgfqpoint{3.841500in}{1.668396in}}%
\pgfpathlineto{\pgfqpoint{3.904851in}{1.580800in}}%
\pgfusepath{stroke}%
\end{pgfscope}%
\begin{pgfscope}%
\pgfpathrectangle{\pgfqpoint{0.588387in}{0.521603in}}{\pgfqpoint{4.669024in}{2.010285in}}%
\pgfusepath{clip}%
\pgfsetrectcap%
\pgfsetroundjoin%
\pgfsetlinewidth{1.505625pt}%
\pgfsetstrokecolor{currentstroke2}%
\pgfsetdash{}{0pt}%
\pgfpathmoveto{\pgfqpoint{0.800616in}{0.740577in}}%
\pgfpathlineto{\pgfqpoint{0.863967in}{0.768558in}}%
\pgfpathlineto{\pgfqpoint{0.927319in}{0.827345in}}%
\pgfpathlineto{\pgfqpoint{0.990671in}{0.859701in}}%
\pgfpathlineto{\pgfqpoint{1.054023in}{0.917639in}}%
\pgfpathlineto{\pgfqpoint{1.117374in}{0.976126in}}%
\pgfpathlineto{\pgfqpoint{1.180726in}{1.102193in}}%
\pgfpathlineto{\pgfqpoint{1.244078in}{1.125590in}}%
\pgfpathlineto{\pgfqpoint{1.307430in}{1.201940in}}%
\pgfpathlineto{\pgfqpoint{1.370781in}{1.201687in}}%
\pgfpathlineto{\pgfqpoint{1.434133in}{1.259270in}}%
\pgfpathlineto{\pgfqpoint{1.497485in}{1.255840in}}%
\pgfpathlineto{\pgfqpoint{1.560837in}{1.265805in}}%
\pgfpathlineto{\pgfqpoint{1.624188in}{1.359744in}}%
\pgfpathlineto{\pgfqpoint{1.687540in}{1.388387in}}%
\pgfpathlineto{\pgfqpoint{1.750892in}{1.314925in}}%
\pgfpathlineto{\pgfqpoint{1.814244in}{1.326608in}}%
\pgfpathlineto{\pgfqpoint{1.877595in}{1.490540in}}%
\pgfpathlineto{\pgfqpoint{1.940947in}{1.629348in}}%
\pgfpathlineto{\pgfqpoint{2.004299in}{1.409227in}}%
\pgfpathlineto{\pgfqpoint{2.067651in}{1.457446in}}%
\pgfpathlineto{\pgfqpoint{2.131002in}{1.556803in}}%
\pgfpathlineto{\pgfqpoint{2.194354in}{1.467273in}}%
\pgfpathlineto{\pgfqpoint{2.257706in}{1.369726in}}%
\pgfpathlineto{\pgfqpoint{2.321058in}{1.547468in}}%
\pgfpathlineto{\pgfqpoint{2.384409in}{1.770645in}}%
\pgfpathlineto{\pgfqpoint{2.447761in}{1.431281in}}%
\pgfpathlineto{\pgfqpoint{2.511113in}{1.423060in}}%
\pgfpathlineto{\pgfqpoint{2.574465in}{1.450628in}}%
\pgfpathlineto{\pgfqpoint{2.637816in}{1.612839in}}%
\pgfpathlineto{\pgfqpoint{2.701168in}{1.525592in}}%
\pgfpathlineto{\pgfqpoint{2.764520in}{1.476018in}}%
\pgfpathlineto{\pgfqpoint{2.827872in}{1.522657in}}%
\pgfpathlineto{\pgfqpoint{2.891223in}{1.499821in}}%
\pgfpathlineto{\pgfqpoint{2.954575in}{1.516869in}}%
\pgfpathlineto{\pgfqpoint{3.017927in}{1.515279in}}%
\pgfpathlineto{\pgfqpoint{3.081279in}{1.548766in}}%
\pgfpathlineto{\pgfqpoint{3.207982in}{1.488831in}}%
\pgfpathlineto{\pgfqpoint{3.271334in}{1.474031in}}%
\pgfpathlineto{\pgfqpoint{3.334686in}{1.478263in}}%
\pgfpathlineto{\pgfqpoint{3.524741in}{1.616429in}}%
\pgfpathlineto{\pgfqpoint{3.714796in}{1.661105in}}%
\pgfusepath{stroke}%
\end{pgfscope}%
\begin{pgfscope}%
\pgfpathrectangle{\pgfqpoint{0.588387in}{0.521603in}}{\pgfqpoint{4.669024in}{2.010285in}}%
\pgfusepath{clip}%
\pgfsetrectcap%
\pgfsetroundjoin%
\pgfsetlinewidth{1.505625pt}%
\pgfsetstrokecolor{currentstroke3}%
\pgfsetdash{}{0pt}%
\pgfpathmoveto{\pgfqpoint{0.800616in}{0.612980in}}%
\pgfpathlineto{\pgfqpoint{0.863967in}{0.665521in}}%
\pgfpathlineto{\pgfqpoint{0.927319in}{0.758539in}}%
\pgfpathlineto{\pgfqpoint{0.990671in}{0.798990in}}%
\pgfpathlineto{\pgfqpoint{1.054023in}{0.878537in}}%
\pgfpathlineto{\pgfqpoint{1.117374in}{0.936936in}}%
\pgfpathlineto{\pgfqpoint{1.180726in}{1.001917in}}%
\pgfpathlineto{\pgfqpoint{1.244078in}{1.063128in}}%
\pgfpathlineto{\pgfqpoint{1.307430in}{1.080899in}}%
\pgfpathlineto{\pgfqpoint{1.370781in}{1.086670in}}%
\pgfpathlineto{\pgfqpoint{1.434133in}{1.275480in}}%
\pgfpathlineto{\pgfqpoint{1.497485in}{1.260495in}}%
\pgfpathlineto{\pgfqpoint{1.560837in}{1.460469in}}%
\pgfpathlineto{\pgfqpoint{1.624188in}{1.448132in}}%
\pgfpathlineto{\pgfqpoint{1.687540in}{1.727310in}}%
\pgfpathlineto{\pgfqpoint{1.750892in}{1.686322in}}%
\pgfpathlineto{\pgfqpoint{1.814244in}{1.381556in}}%
\pgfpathlineto{\pgfqpoint{1.877595in}{2.107890in}}%
\pgfpathlineto{\pgfqpoint{1.940947in}{1.686176in}}%
\pgfpathlineto{\pgfqpoint{2.004299in}{1.956351in}}%
\pgfpathlineto{\pgfqpoint{2.067651in}{1.328837in}}%
\pgfpathlineto{\pgfqpoint{2.131002in}{2.096662in}}%
\pgfpathlineto{\pgfqpoint{2.194354in}{2.076352in}}%
\pgfpathlineto{\pgfqpoint{2.257706in}{1.268773in}}%
\pgfpathlineto{\pgfqpoint{2.321058in}{1.985413in}}%
\pgfpathlineto{\pgfqpoint{2.384409in}{2.101218in}}%
\pgfpathlineto{\pgfqpoint{2.447761in}{2.408021in}}%
\pgfpathlineto{\pgfqpoint{2.511113in}{1.946701in}}%
\pgfpathlineto{\pgfqpoint{2.574465in}{2.071899in}}%
\pgfpathlineto{\pgfqpoint{2.637816in}{1.859021in}}%
\pgfpathlineto{\pgfqpoint{2.701168in}{1.869837in}}%
\pgfpathlineto{\pgfqpoint{2.764520in}{2.284931in}}%
\pgfpathlineto{\pgfqpoint{2.827872in}{2.259864in}}%
\pgfpathlineto{\pgfqpoint{2.891223in}{1.583418in}}%
\pgfpathlineto{\pgfqpoint{2.954575in}{1.292043in}}%
\pgfpathlineto{\pgfqpoint{3.017927in}{2.247253in}}%
\pgfpathlineto{\pgfqpoint{3.081279in}{1.643218in}}%
\pgfpathlineto{\pgfqpoint{3.144630in}{2.155645in}}%
\pgfpathlineto{\pgfqpoint{3.207982in}{1.398963in}}%
\pgfpathlineto{\pgfqpoint{3.271334in}{1.737594in}}%
\pgfpathlineto{\pgfqpoint{3.334686in}{1.111488in}}%
\pgfpathlineto{\pgfqpoint{3.524741in}{2.017953in}}%
\pgfpathlineto{\pgfqpoint{3.714796in}{2.440512in}}%
\pgfpathlineto{\pgfqpoint{3.841500in}{1.377120in}}%
\pgfpathlineto{\pgfqpoint{3.904851in}{1.270074in}}%
\pgfpathlineto{\pgfqpoint{4.031555in}{2.004669in}}%
\pgfpathlineto{\pgfqpoint{4.284962in}{2.010705in}}%
\pgfpathlineto{\pgfqpoint{4.665072in}{1.759912in}}%
\pgfpathlineto{\pgfqpoint{5.045183in}{1.245122in}}%
\pgfusepath{stroke}%
\end{pgfscope}%
\begin{pgfscope}%
\pgfpathrectangle{\pgfqpoint{0.588387in}{0.521603in}}{\pgfqpoint{4.669024in}{2.010285in}}%
\pgfusepath{clip}%
\pgfsetrectcap%
\pgfsetroundjoin%
\pgfsetlinewidth{1.505625pt}%
\pgfsetstrokecolor{currentstroke4}%
\pgfsetdash{}{0pt}%
\pgfpathmoveto{\pgfqpoint{0.800616in}{0.740577in}}%
\pgfpathlineto{\pgfqpoint{0.863967in}{0.741792in}}%
\pgfpathlineto{\pgfqpoint{0.927319in}{0.809568in}}%
\pgfpathlineto{\pgfqpoint{0.990671in}{0.834391in}}%
\pgfpathlineto{\pgfqpoint{1.054023in}{0.902365in}}%
\pgfpathlineto{\pgfqpoint{1.117374in}{0.945726in}}%
\pgfpathlineto{\pgfqpoint{1.180726in}{1.057777in}}%
\pgfpathlineto{\pgfqpoint{1.244078in}{1.149628in}}%
\pgfpathlineto{\pgfqpoint{1.307430in}{1.165636in}}%
\pgfpathlineto{\pgfqpoint{1.370781in}{1.200019in}}%
\pgfpathlineto{\pgfqpoint{1.434133in}{1.291476in}}%
\pgfpathlineto{\pgfqpoint{1.497485in}{1.264623in}}%
\pgfpathlineto{\pgfqpoint{1.560837in}{1.358989in}}%
\pgfpathlineto{\pgfqpoint{1.624188in}{1.378621in}}%
\pgfpathlineto{\pgfqpoint{1.687540in}{1.443964in}}%
\pgfpathlineto{\pgfqpoint{1.750892in}{1.449161in}}%
\pgfpathlineto{\pgfqpoint{1.814244in}{1.369136in}}%
\pgfpathlineto{\pgfqpoint{1.877595in}{1.392022in}}%
\pgfpathlineto{\pgfqpoint{1.940947in}{1.426664in}}%
\pgfpathlineto{\pgfqpoint{2.004299in}{1.472358in}}%
\pgfpathlineto{\pgfqpoint{2.067651in}{1.468953in}}%
\pgfpathlineto{\pgfqpoint{2.131002in}{1.712963in}}%
\pgfpathlineto{\pgfqpoint{2.194354in}{1.433429in}}%
\pgfpathlineto{\pgfqpoint{2.257706in}{1.427401in}}%
\pgfpathlineto{\pgfqpoint{2.321058in}{1.488707in}}%
\pgfpathlineto{\pgfqpoint{2.384409in}{1.521337in}}%
\pgfpathlineto{\pgfqpoint{2.447761in}{1.523890in}}%
\pgfpathlineto{\pgfqpoint{2.511113in}{1.531534in}}%
\pgfpathlineto{\pgfqpoint{2.574465in}{1.487775in}}%
\pgfpathlineto{\pgfqpoint{2.637816in}{1.520008in}}%
\pgfpathlineto{\pgfqpoint{2.701168in}{1.565696in}}%
\pgfpathlineto{\pgfqpoint{2.764520in}{1.581120in}}%
\pgfpathlineto{\pgfqpoint{2.827872in}{1.638023in}}%
\pgfpathlineto{\pgfqpoint{2.891223in}{1.590277in}}%
\pgfpathlineto{\pgfqpoint{2.954575in}{1.546084in}}%
\pgfpathlineto{\pgfqpoint{3.017927in}{1.536276in}}%
\pgfpathlineto{\pgfqpoint{3.081279in}{1.549809in}}%
\pgfpathlineto{\pgfqpoint{3.144630in}{1.616033in}}%
\pgfpathlineto{\pgfqpoint{3.207982in}{1.592861in}}%
\pgfpathlineto{\pgfqpoint{3.271334in}{1.714874in}}%
\pgfpathlineto{\pgfqpoint{3.334686in}{1.541986in}}%
\pgfpathlineto{\pgfqpoint{3.524741in}{1.593629in}}%
\pgfpathlineto{\pgfqpoint{3.714796in}{1.567637in}}%
\pgfpathlineto{\pgfqpoint{3.841500in}{1.660069in}}%
\pgfpathlineto{\pgfqpoint{3.904851in}{1.674547in}}%
\pgfpathlineto{\pgfqpoint{4.221610in}{1.670175in}}%
\pgfusepath{stroke}%
\end{pgfscope}%
\begin{pgfscope}%
\pgfpathrectangle{\pgfqpoint{0.588387in}{0.521603in}}{\pgfqpoint{4.669024in}{2.010285in}}%
\pgfusepath{clip}%
\pgfsetrectcap%
\pgfsetroundjoin%
\pgfsetlinewidth{1.505625pt}%
\pgfsetstrokecolor{currentstroke5}%
\pgfsetdash{}{0pt}%
\pgfpathmoveto{\pgfqpoint{0.800616in}{0.740577in}}%
\pgfpathlineto{\pgfqpoint{0.863967in}{0.741791in}}%
\pgfpathlineto{\pgfqpoint{0.927319in}{0.809777in}}%
\pgfpathlineto{\pgfqpoint{0.990671in}{0.834634in}}%
\pgfpathlineto{\pgfqpoint{1.054023in}{0.901686in}}%
\pgfpathlineto{\pgfqpoint{1.117374in}{0.946025in}}%
\pgfpathlineto{\pgfqpoint{1.180726in}{1.058150in}}%
\pgfpathlineto{\pgfqpoint{1.244078in}{1.164495in}}%
\pgfpathlineto{\pgfqpoint{1.307430in}{1.163781in}}%
\pgfpathlineto{\pgfqpoint{1.370781in}{1.200292in}}%
\pgfpathlineto{\pgfqpoint{1.434133in}{1.310211in}}%
\pgfpathlineto{\pgfqpoint{1.497485in}{1.282427in}}%
\pgfpathlineto{\pgfqpoint{1.560837in}{1.363426in}}%
\pgfpathlineto{\pgfqpoint{1.624188in}{1.406562in}}%
\pgfpathlineto{\pgfqpoint{1.687540in}{1.454381in}}%
\pgfpathlineto{\pgfqpoint{1.750892in}{1.459401in}}%
\pgfpathlineto{\pgfqpoint{1.814244in}{1.502458in}}%
\pgfpathlineto{\pgfqpoint{1.877595in}{1.455686in}}%
\pgfpathlineto{\pgfqpoint{1.940947in}{1.450443in}}%
\pgfpathlineto{\pgfqpoint{2.004299in}{1.613804in}}%
\pgfpathlineto{\pgfqpoint{2.067651in}{1.621686in}}%
\pgfpathlineto{\pgfqpoint{2.131002in}{1.724622in}}%
\pgfpathlineto{\pgfqpoint{2.194354in}{2.039620in}}%
\pgfpathlineto{\pgfqpoint{2.257706in}{1.487156in}}%
\pgfpathlineto{\pgfqpoint{2.321058in}{1.476582in}}%
\pgfpathlineto{\pgfqpoint{2.384409in}{1.524489in}}%
\pgfpathlineto{\pgfqpoint{2.447761in}{1.704382in}}%
\pgfpathlineto{\pgfqpoint{2.511113in}{1.528481in}}%
\pgfpathlineto{\pgfqpoint{2.574465in}{1.586713in}}%
\pgfpathlineto{\pgfqpoint{2.637816in}{1.663285in}}%
\pgfpathlineto{\pgfqpoint{2.701168in}{1.596968in}}%
\pgfpathlineto{\pgfqpoint{2.764520in}{1.595977in}}%
\pgfpathlineto{\pgfqpoint{2.827872in}{1.607526in}}%
\pgfpathlineto{\pgfqpoint{2.891223in}{1.624669in}}%
\pgfpathlineto{\pgfqpoint{2.954575in}{1.553830in}}%
\pgfpathlineto{\pgfqpoint{3.017927in}{1.537493in}}%
\pgfpathlineto{\pgfqpoint{3.081279in}{1.516235in}}%
\pgfpathlineto{\pgfqpoint{3.144630in}{1.614842in}}%
\pgfpathlineto{\pgfqpoint{3.207982in}{1.624354in}}%
\pgfpathlineto{\pgfqpoint{3.271334in}{1.667949in}}%
\pgfpathlineto{\pgfqpoint{3.334686in}{1.478263in}}%
\pgfpathlineto{\pgfqpoint{3.524741in}{1.550658in}}%
\pgfpathlineto{\pgfqpoint{3.714796in}{1.614043in}}%
\pgfpathlineto{\pgfqpoint{3.841500in}{1.651129in}}%
\pgfpathlineto{\pgfqpoint{3.904851in}{1.727937in}}%
\pgfpathlineto{\pgfqpoint{4.221610in}{1.670175in}}%
\pgfusepath{stroke}%
\end{pgfscope}%
\begin{pgfscope}%
\pgfpathrectangle{\pgfqpoint{0.588387in}{0.521603in}}{\pgfqpoint{4.669024in}{2.010285in}}%
\pgfusepath{clip}%
\pgfsetrectcap%
\pgfsetroundjoin%
\pgfsetlinewidth{1.505625pt}%
\pgfsetstrokecolor{currentstroke6}%
\pgfsetdash{}{0pt}%
\pgfpathmoveto{\pgfqpoint{0.800616in}{0.740577in}}%
\pgfpathlineto{\pgfqpoint{0.863967in}{0.768809in}}%
\pgfpathlineto{\pgfqpoint{0.927319in}{0.826561in}}%
\pgfpathlineto{\pgfqpoint{0.990671in}{0.854846in}}%
\pgfpathlineto{\pgfqpoint{1.054023in}{0.917411in}}%
\pgfpathlineto{\pgfqpoint{1.117374in}{0.960637in}}%
\pgfpathlineto{\pgfqpoint{1.180726in}{1.084250in}}%
\pgfpathlineto{\pgfqpoint{1.244078in}{1.172060in}}%
\pgfpathlineto{\pgfqpoint{1.307430in}{1.177512in}}%
\pgfpathlineto{\pgfqpoint{1.370781in}{1.190809in}}%
\pgfpathlineto{\pgfqpoint{1.434133in}{1.299853in}}%
\pgfpathlineto{\pgfqpoint{1.497485in}{1.327303in}}%
\pgfpathlineto{\pgfqpoint{1.560837in}{1.413589in}}%
\pgfpathlineto{\pgfqpoint{1.624188in}{1.412030in}}%
\pgfpathlineto{\pgfqpoint{1.687540in}{1.473498in}}%
\pgfpathlineto{\pgfqpoint{1.750892in}{1.496228in}}%
\pgfpathlineto{\pgfqpoint{1.814244in}{1.436037in}}%
\pgfpathlineto{\pgfqpoint{1.877595in}{1.433109in}}%
\pgfpathlineto{\pgfqpoint{1.940947in}{1.493939in}}%
\pgfpathlineto{\pgfqpoint{2.004299in}{1.529324in}}%
\pgfpathlineto{\pgfqpoint{2.067651in}{1.524635in}}%
\pgfpathlineto{\pgfqpoint{2.131002in}{1.716578in}}%
\pgfpathlineto{\pgfqpoint{2.194354in}{1.706356in}}%
\pgfpathlineto{\pgfqpoint{2.257706in}{1.520408in}}%
\pgfpathlineto{\pgfqpoint{2.321058in}{1.500835in}}%
\pgfpathlineto{\pgfqpoint{2.384409in}{1.555854in}}%
\pgfpathlineto{\pgfqpoint{2.447761in}{1.617771in}}%
\pgfpathlineto{\pgfqpoint{2.511113in}{1.608972in}}%
\pgfpathlineto{\pgfqpoint{2.574465in}{1.571459in}}%
\pgfpathlineto{\pgfqpoint{2.637816in}{1.631443in}}%
\pgfpathlineto{\pgfqpoint{2.701168in}{1.606278in}}%
\pgfpathlineto{\pgfqpoint{2.764520in}{1.583024in}}%
\pgfpathlineto{\pgfqpoint{2.827872in}{1.676553in}}%
\pgfpathlineto{\pgfqpoint{2.891223in}{1.814986in}}%
\pgfpathlineto{\pgfqpoint{2.954575in}{1.579839in}}%
\pgfpathlineto{\pgfqpoint{3.017927in}{1.594648in}}%
\pgfpathlineto{\pgfqpoint{3.081279in}{1.622356in}}%
\pgfpathlineto{\pgfqpoint{3.144630in}{1.617610in}}%
\pgfpathlineto{\pgfqpoint{3.207982in}{1.676268in}}%
\pgfpathlineto{\pgfqpoint{3.271334in}{1.747353in}}%
\pgfpathlineto{\pgfqpoint{3.334686in}{1.564259in}}%
\pgfpathlineto{\pgfqpoint{3.524741in}{1.610542in}}%
\pgfpathlineto{\pgfqpoint{3.714796in}{1.727181in}}%
\pgfpathlineto{\pgfqpoint{3.841500in}{1.632676in}}%
\pgfpathlineto{\pgfqpoint{3.904851in}{1.786363in}}%
\pgfpathlineto{\pgfqpoint{4.221610in}{1.862513in}}%
\pgfpathlineto{\pgfqpoint{4.284962in}{1.748706in}}%
\pgfpathlineto{\pgfqpoint{4.411665in}{1.746770in}}%
\pgfpathlineto{\pgfqpoint{4.665072in}{1.852491in}}%
\pgfusepath{stroke}%
\end{pgfscope}%
\begin{pgfscope}%
\pgfpathrectangle{\pgfqpoint{0.588387in}{0.521603in}}{\pgfqpoint{4.669024in}{2.010285in}}%
\pgfusepath{clip}%
\pgfsetrectcap%
\pgfsetroundjoin%
\pgfsetlinewidth{1.505625pt}%
\pgfsetstrokecolor{currentstroke7}%
\pgfsetdash{}{0pt}%
\pgfpathmoveto{\pgfqpoint{0.800616in}{0.740577in}}%
\pgfpathlineto{\pgfqpoint{0.863967in}{0.768896in}}%
\pgfpathlineto{\pgfqpoint{0.927319in}{0.826404in}}%
\pgfpathlineto{\pgfqpoint{0.990671in}{0.854761in}}%
\pgfpathlineto{\pgfqpoint{1.054023in}{0.916704in}}%
\pgfpathlineto{\pgfqpoint{1.117374in}{0.960144in}}%
\pgfpathlineto{\pgfqpoint{1.180726in}{1.084548in}}%
\pgfpathlineto{\pgfqpoint{1.244078in}{1.171322in}}%
\pgfpathlineto{\pgfqpoint{1.307430in}{1.178364in}}%
\pgfpathlineto{\pgfqpoint{1.370781in}{1.190240in}}%
\pgfpathlineto{\pgfqpoint{1.434133in}{1.312091in}}%
\pgfpathlineto{\pgfqpoint{1.497485in}{1.330776in}}%
\pgfpathlineto{\pgfqpoint{1.560837in}{1.429653in}}%
\pgfpathlineto{\pgfqpoint{1.624188in}{1.445750in}}%
\pgfpathlineto{\pgfqpoint{1.687540in}{1.487983in}}%
\pgfpathlineto{\pgfqpoint{1.750892in}{1.520405in}}%
\pgfpathlineto{\pgfqpoint{1.814244in}{1.441833in}}%
\pgfpathlineto{\pgfqpoint{1.877595in}{1.461019in}}%
\pgfpathlineto{\pgfqpoint{1.940947in}{1.637632in}}%
\pgfpathlineto{\pgfqpoint{2.004299in}{1.555236in}}%
\pgfpathlineto{\pgfqpoint{2.067651in}{1.682993in}}%
\pgfpathlineto{\pgfqpoint{2.131002in}{1.744461in}}%
\pgfpathlineto{\pgfqpoint{2.194354in}{1.899721in}}%
\pgfpathlineto{\pgfqpoint{2.257706in}{1.682607in}}%
\pgfpathlineto{\pgfqpoint{2.321058in}{1.701430in}}%
\pgfpathlineto{\pgfqpoint{2.384409in}{1.867813in}}%
\pgfpathlineto{\pgfqpoint{2.447761in}{1.828816in}}%
\pgfpathlineto{\pgfqpoint{2.511113in}{1.592861in}}%
\pgfpathlineto{\pgfqpoint{2.574465in}{1.688516in}}%
\pgfpathlineto{\pgfqpoint{2.637816in}{1.780840in}}%
\pgfpathlineto{\pgfqpoint{2.701168in}{1.605022in}}%
\pgfpathlineto{\pgfqpoint{2.764520in}{1.777413in}}%
\pgfpathlineto{\pgfqpoint{2.827872in}{1.635903in}}%
\pgfpathlineto{\pgfqpoint{2.891223in}{1.619562in}}%
\pgfpathlineto{\pgfqpoint{2.954575in}{1.649067in}}%
\pgfpathlineto{\pgfqpoint{3.017927in}{1.568920in}}%
\pgfpathlineto{\pgfqpoint{3.081279in}{1.614043in}}%
\pgfpathlineto{\pgfqpoint{3.207982in}{1.635428in}}%
\pgfpathlineto{\pgfqpoint{3.271334in}{1.853747in}}%
\pgfpathlineto{\pgfqpoint{3.524741in}{1.734084in}}%
\pgfpathlineto{\pgfqpoint{3.714796in}{1.654157in}}%
\pgfpathlineto{\pgfqpoint{3.841500in}{1.731776in}}%
\pgfpathlineto{\pgfqpoint{3.904851in}{1.671936in}}%
\pgfpathlineto{\pgfqpoint{4.221610in}{1.702058in}}%
\pgfpathlineto{\pgfqpoint{4.665072in}{1.809486in}}%
\pgfusepath{stroke}%
\end{pgfscope}%
\begin{pgfscope}%
\pgfsetrectcap%
\pgfsetmiterjoin%
\pgfsetlinewidth{0.803000pt}%
\definecolor{currentstroke}{rgb}{0.000000,0.000000,0.000000}%
\pgfsetstrokecolor{currentstroke}%
\pgfsetdash{}{0pt}%
\pgfpathmoveto{\pgfqpoint{0.588387in}{0.521603in}}%
\pgfpathlineto{\pgfqpoint{0.588387in}{2.531888in}}%
\pgfusepath{stroke}%
\end{pgfscope}%
\begin{pgfscope}%
\pgfsetrectcap%
\pgfsetmiterjoin%
\pgfsetlinewidth{0.803000pt}%
\definecolor{currentstroke}{rgb}{0.000000,0.000000,0.000000}%
\pgfsetstrokecolor{currentstroke}%
\pgfsetdash{}{0pt}%
\pgfpathmoveto{\pgfqpoint{5.257411in}{0.521603in}}%
\pgfpathlineto{\pgfqpoint{5.257411in}{2.531888in}}%
\pgfusepath{stroke}%
\end{pgfscope}%
\begin{pgfscope}%
\pgfsetrectcap%
\pgfsetmiterjoin%
\pgfsetlinewidth{0.803000pt}%
\definecolor{currentstroke}{rgb}{0.000000,0.000000,0.000000}%
\pgfsetstrokecolor{currentstroke}%
\pgfsetdash{}{0pt}%
\pgfpathmoveto{\pgfqpoint{0.588387in}{0.521603in}}%
\pgfpathlineto{\pgfqpoint{5.257411in}{0.521603in}}%
\pgfusepath{stroke}%
\end{pgfscope}%
\begin{pgfscope}%
\pgfsetrectcap%
\pgfsetmiterjoin%
\pgfsetlinewidth{0.803000pt}%
\definecolor{currentstroke}{rgb}{0.000000,0.000000,0.000000}%
\pgfsetstrokecolor{currentstroke}%
\pgfsetdash{}{0pt}%
\pgfpathmoveto{\pgfqpoint{0.588387in}{2.531888in}}%
\pgfpathlineto{\pgfqpoint{5.257411in}{2.531888in}}%
\pgfusepath{stroke}%
\end{pgfscope}%
\begin{pgfscope}%
\definecolor{textcolor}{rgb}{0.000000,0.000000,0.000000}%
\pgfsetstrokecolor{textcolor}%
\pgfsetfillcolor{textcolor}%
\pgftext[x=2.922899in,y=2.615222in,,base]{\color{textcolor}{\rmfamily\fontsize{12.000000}{14.400000}\selectfont\catcode`\^=\active\def^{\ifmmode\sp\else\^{}\fi}\catcode`\%=\active\def%{\%}Mean}}%
\end{pgfscope}%
\begin{pgfscope}%
\pgfsetbuttcap%
\pgfsetmiterjoin%
\definecolor{currentfill}{rgb}{1.000000,1.000000,1.000000}%
\pgfsetfillcolor{currentfill}%
\pgfsetfillopacity{0.800000}%
\pgfsetlinewidth{1.003750pt}%
\definecolor{currentstroke}{rgb}{0.800000,0.800000,0.800000}%
\pgfsetstrokecolor{currentstroke}%
\pgfsetstrokeopacity{0.800000}%
\pgfsetdash{}{0pt}%
\pgfpathmoveto{\pgfqpoint{5.344911in}{1.133672in}}%
\pgfpathlineto{\pgfqpoint{8.259376in}{1.133672in}}%
\pgfpathquadraticcurveto{\pgfqpoint{8.284376in}{1.133672in}}{\pgfqpoint{8.284376in}{1.158672in}}%
\pgfpathlineto{\pgfqpoint{8.284376in}{2.444388in}}%
\pgfpathquadraticcurveto{\pgfqpoint{8.284376in}{2.469388in}}{\pgfqpoint{8.259376in}{2.469388in}}%
\pgfpathlineto{\pgfqpoint{5.344911in}{2.469388in}}%
\pgfpathquadraticcurveto{\pgfqpoint{5.319911in}{2.469388in}}{\pgfqpoint{5.319911in}{2.444388in}}%
\pgfpathlineto{\pgfqpoint{5.319911in}{1.158672in}}%
\pgfpathquadraticcurveto{\pgfqpoint{5.319911in}{1.133672in}}{\pgfqpoint{5.344911in}{1.133672in}}%
\pgfpathlineto{\pgfqpoint{5.344911in}{1.133672in}}%
\pgfpathclose%
\pgfusepath{stroke,fill}%
\end{pgfscope}%
\begin{pgfscope}%
\pgfsetrectcap%
\pgfsetroundjoin%
\pgfsetlinewidth{1.505625pt}%
\pgfsetstrokecolor{currentstroke3}%
\pgfsetdash{}{0pt}%
\pgfpathmoveto{\pgfqpoint{5.369911in}{2.368168in}}%
\pgfpathlineto{\pgfqpoint{5.494911in}{2.368168in}}%
\pgfpathlineto{\pgfqpoint{5.619911in}{2.368168in}}%
\pgfusepath{stroke}%
\end{pgfscope}%
\begin{pgfscope}%
\definecolor{textcolor}{rgb}{0.000000,0.000000,0.000000}%
\pgfsetstrokecolor{textcolor}%
\pgfsetfillcolor{textcolor}%
\pgftext[x=5.719911in,y=2.324418in,left,base]{\color{textcolor}{\rmfamily\fontsize{9.000000}{10.800000}\selectfont\catcode`\^=\active\def^{\ifmmode\sp\else\^{}\fi}\catcode`\%=\active\def%{\%}\NaiveCycles{}}}%
\end{pgfscope}%
\begin{pgfscope}%
\pgfsetrectcap%
\pgfsetroundjoin%
\pgfsetlinewidth{1.505625pt}%
\pgfsetstrokecolor{currentstroke1}%
\pgfsetdash{}{0pt}%
\pgfpathmoveto{\pgfqpoint{5.369911in}{2.184696in}}%
\pgfpathlineto{\pgfqpoint{5.494911in}{2.184696in}}%
\pgfpathlineto{\pgfqpoint{5.619911in}{2.184696in}}%
\pgfusepath{stroke}%
\end{pgfscope}%
\begin{pgfscope}%
\definecolor{textcolor}{rgb}{0.000000,0.000000,0.000000}%
\pgfsetstrokecolor{textcolor}%
\pgfsetfillcolor{textcolor}%
\pgftext[x=5.719911in,y=2.140946in,left,base]{\color{textcolor}{\rmfamily\fontsize{9.000000}{10.800000}\selectfont\catcode`\^=\active\def^{\ifmmode\sp\else\^{}\fi}\catcode`\%=\active\def%{\%}\CyclesMatchChunks{} \& \MergeLinear{}}}%
\end{pgfscope}%
\begin{pgfscope}%
\pgfsetrectcap%
\pgfsetroundjoin%
\pgfsetlinewidth{1.505625pt}%
\pgfsetstrokecolor{currentstroke2}%
\pgfsetdash{}{0pt}%
\pgfpathmoveto{\pgfqpoint{5.369911in}{1.997746in}}%
\pgfpathlineto{\pgfqpoint{5.494911in}{1.997746in}}%
\pgfpathlineto{\pgfqpoint{5.619911in}{1.997746in}}%
\pgfusepath{stroke}%
\end{pgfscope}%
\begin{pgfscope}%
\definecolor{textcolor}{rgb}{0.000000,0.000000,0.000000}%
\pgfsetstrokecolor{textcolor}%
\pgfsetfillcolor{textcolor}%
\pgftext[x=5.719911in,y=1.953996in,left,base]{\color{textcolor}{\rmfamily\fontsize{9.000000}{10.800000}\selectfont\catcode`\^=\active\def^{\ifmmode\sp\else\^{}\fi}\catcode`\%=\active\def%{\%}\CyclesMatchChunks{} \& \SharedVertices{}}}%
\end{pgfscope}%
\begin{pgfscope}%
\pgfsetrectcap%
\pgfsetroundjoin%
\pgfsetlinewidth{1.505625pt}%
\pgfsetstrokecolor{currentstroke4}%
\pgfsetdash{}{0pt}%
\pgfpathmoveto{\pgfqpoint{5.369911in}{1.810795in}}%
\pgfpathlineto{\pgfqpoint{5.494911in}{1.810795in}}%
\pgfpathlineto{\pgfqpoint{5.619911in}{1.810795in}}%
\pgfusepath{stroke}%
\end{pgfscope}%
\begin{pgfscope}%
\definecolor{textcolor}{rgb}{0.000000,0.000000,0.000000}%
\pgfsetstrokecolor{textcolor}%
\pgfsetfillcolor{textcolor}%
\pgftext[x=5.719911in,y=1.767045in,left,base]{\color{textcolor}{\rmfamily\fontsize{9.000000}{10.800000}\selectfont\catcode`\^=\active\def^{\ifmmode\sp\else\^{}\fi}\catcode`\%=\active\def%{\%}\Neighbors{} \& \MergeLinear{}}}%
\end{pgfscope}%
\begin{pgfscope}%
\pgfsetrectcap%
\pgfsetroundjoin%
\pgfsetlinewidth{1.505625pt}%
\pgfsetstrokecolor{currentstroke5}%
\pgfsetdash{}{0pt}%
\pgfpathmoveto{\pgfqpoint{5.369911in}{1.627324in}}%
\pgfpathlineto{\pgfqpoint{5.494911in}{1.627324in}}%
\pgfpathlineto{\pgfqpoint{5.619911in}{1.627324in}}%
\pgfusepath{stroke}%
\end{pgfscope}%
\begin{pgfscope}%
\definecolor{textcolor}{rgb}{0.000000,0.000000,0.000000}%
\pgfsetstrokecolor{textcolor}%
\pgfsetfillcolor{textcolor}%
\pgftext[x=5.719911in,y=1.583574in,left,base]{\color{textcolor}{\rmfamily\fontsize{9.000000}{10.800000}\selectfont\catcode`\^=\active\def^{\ifmmode\sp\else\^{}\fi}\catcode`\%=\active\def%{\%}\Neighbors{} \& \SharedVertices{}}}%
\end{pgfscope}%
\begin{pgfscope}%
\pgfsetrectcap%
\pgfsetroundjoin%
\pgfsetlinewidth{1.505625pt}%
\pgfsetstrokecolor{currentstroke6}%
\pgfsetdash{}{0pt}%
\pgfpathmoveto{\pgfqpoint{5.369911in}{1.440373in}}%
\pgfpathlineto{\pgfqpoint{5.494911in}{1.440373in}}%
\pgfpathlineto{\pgfqpoint{5.619911in}{1.440373in}}%
\pgfusepath{stroke}%
\end{pgfscope}%
\begin{pgfscope}%
\definecolor{textcolor}{rgb}{0.000000,0.000000,0.000000}%
\pgfsetstrokecolor{textcolor}%
\pgfsetfillcolor{textcolor}%
\pgftext[x=5.719911in,y=1.396623in,left,base]{\color{textcolor}{\rmfamily\fontsize{9.000000}{10.800000}\selectfont\catcode`\^=\active\def^{\ifmmode\sp\else\^{}\fi}\catcode`\%=\active\def%{\%}\None{} \& \MergeLinear{}}}%
\end{pgfscope}%
\begin{pgfscope}%
\pgfsetrectcap%
\pgfsetroundjoin%
\pgfsetlinewidth{1.505625pt}%
\pgfsetstrokecolor{currentstroke7}%
\pgfsetdash{}{0pt}%
\pgfpathmoveto{\pgfqpoint{5.369911in}{1.256902in}}%
\pgfpathlineto{\pgfqpoint{5.494911in}{1.256902in}}%
\pgfpathlineto{\pgfqpoint{5.619911in}{1.256902in}}%
\pgfusepath{stroke}%
\end{pgfscope}%
\begin{pgfscope}%
\definecolor{textcolor}{rgb}{0.000000,0.000000,0.000000}%
\pgfsetstrokecolor{textcolor}%
\pgfsetfillcolor{textcolor}%
\pgftext[x=5.719911in,y=1.213152in,left,base]{\color{textcolor}{\rmfamily\fontsize{9.000000}{10.800000}\selectfont\catcode`\^=\active\def^{\ifmmode\sp\else\^{}\fi}\catcode`\%=\active\def%{\%}\None{} \& \SharedVertices{}}}%
\end{pgfscope}%
\end{pgfpicture}%
\makeatother%
\endgroup%
}
% 	\caption[Checks performed for globally rigid graphs (some)]{
% 		The number of checks performed to find some NAC-coloring for globally rigid graphs.}%
% 	\label{fig:graph_globally_rigid_first_checks}
% \end{figure}%

\NaiveCycles{} algorithm is significantly slower when we list all NAC-colorings
as expected, see \Cref{fig:graph_globally_rigid_all_runtime}.
\None{} and \CycleMask{} strategies also lack behind \Neighbors{} and \NeighborsDegree{}.
We do not see an advantage of \MergeLinear{} over \SharedVertices{} anymore.
%
It can be also seen in \Cref{fig:graph_globally_rigid_all_checks}
that for the number of checks performed the same statements hold.
%
\begin{figure}[thbp]
	\centering
	\scalebox{\BenchFigureScale}{%% Creator: Matplotlib, PGF backend
%%
%% To include the figure in your LaTeX document, write
%%   \input{<filename>.pgf}
%%
%% Make sure the required packages are loaded in your preamble
%%   \usepackage{pgf}
%%
%% Also ensure that all the required font packages are loaded; for instance,
%% the lmodern package is sometimes necessary when using math font.
%%   \usepackage{lmodern}
%%
%% Figures using additional raster images can only be included by \input if
%% they are in the same directory as the main LaTeX file. For loading figures
%% from other directories you can use the `import` package
%%   \usepackage{import}
%%
%% and then include the figures with
%%   \import{<path to file>}{<filename>.pgf}
%%
%% Matplotlib used the following preamble
%%   \def\mathdefault#1{#1}
%%   \everymath=\expandafter{\the\everymath\displaystyle}
%%   \IfFileExists{scrextend.sty}{
%%     \usepackage[fontsize=10.000000pt]{scrextend}
%%   }{
%%     \renewcommand{\normalsize}{\fontsize{10.000000}{12.000000}\selectfont}
%%     \normalsize
%%   }
%%   
%%   \ifdefined\pdftexversion\else  % non-pdftex case.
%%     \usepackage{fontspec}
%%     \setmainfont{DejaVuSans.ttf}[Path=\detokenize{/home/petr/Projects/PyRigi/.venv/lib/python3.12/site-packages/matplotlib/mpl-data/fonts/ttf/}]
%%     \setsansfont{DejaVuSans.ttf}[Path=\detokenize{/home/petr/Projects/PyRigi/.venv/lib/python3.12/site-packages/matplotlib/mpl-data/fonts/ttf/}]
%%     \setmonofont{DejaVuSansMono.ttf}[Path=\detokenize{/home/petr/Projects/PyRigi/.venv/lib/python3.12/site-packages/matplotlib/mpl-data/fonts/ttf/}]
%%   \fi
%%   \makeatletter\@ifpackageloaded{under\Score{}}{}{\usepackage[strings]{under\Score{}}}\makeatother
%%
\begingroup%
\makeatletter%
\begin{pgfpicture}%
\pgfpathrectangle{\pgfpointorigin}{\pgfqpoint{8.384376in}{2.841853in}}%
\pgfusepath{use as bounding box, clip}%
\begin{pgfscope}%
\pgfsetbuttcap%
\pgfsetmiterjoin%
\definecolor{currentfill}{rgb}{1.000000,1.000000,1.000000}%
\pgfsetfillcolor{currentfill}%
\pgfsetlinewidth{0.000000pt}%
\definecolor{currentstroke}{rgb}{1.000000,1.000000,1.000000}%
\pgfsetstrokecolor{currentstroke}%
\pgfsetdash{}{0pt}%
\pgfpathmoveto{\pgfqpoint{0.000000in}{0.000000in}}%
\pgfpathlineto{\pgfqpoint{8.384376in}{0.000000in}}%
\pgfpathlineto{\pgfqpoint{8.384376in}{2.841853in}}%
\pgfpathlineto{\pgfqpoint{0.000000in}{2.841853in}}%
\pgfpathlineto{\pgfqpoint{0.000000in}{0.000000in}}%
\pgfpathclose%
\pgfusepath{fill}%
\end{pgfscope}%
\begin{pgfscope}%
\pgfsetbuttcap%
\pgfsetmiterjoin%
\definecolor{currentfill}{rgb}{1.000000,1.000000,1.000000}%
\pgfsetfillcolor{currentfill}%
\pgfsetlinewidth{0.000000pt}%
\definecolor{currentstroke}{rgb}{0.000000,0.000000,0.000000}%
\pgfsetstrokecolor{currentstroke}%
\pgfsetstrokeopacity{0.000000}%
\pgfsetdash{}{0pt}%
\pgfpathmoveto{\pgfqpoint{0.588387in}{0.521603in}}%
\pgfpathlineto{\pgfqpoint{4.248423in}{0.521603in}}%
\pgfpathlineto{\pgfqpoint{4.248423in}{2.713741in}}%
\pgfpathlineto{\pgfqpoint{0.588387in}{2.713741in}}%
\pgfpathlineto{\pgfqpoint{0.588387in}{0.521603in}}%
\pgfpathclose%
\pgfusepath{fill}%
\end{pgfscope}%
\begin{pgfscope}%
\pgfsetbuttcap%
\pgfsetroundjoin%
\definecolor{currentfill}{rgb}{0.000000,0.000000,0.000000}%
\pgfsetfillcolor{currentfill}%
\pgfsetlinewidth{0.803000pt}%
\definecolor{currentstroke}{rgb}{0.000000,0.000000,0.000000}%
\pgfsetstrokecolor{currentstroke}%
\pgfsetdash{}{0pt}%
\pgfsys@defobject{currentmarker}{\pgfqpoint{0.000000in}{-0.048611in}}{\pgfqpoint{0.000000in}{0.000000in}}{%
\pgfpathmoveto{\pgfqpoint{0.000000in}{0.000000in}}%
\pgfpathlineto{\pgfqpoint{0.000000in}{-0.048611in}}%
\pgfusepath{stroke,fill}%
}%
\begin{pgfscope}%
\pgfsys@transformshift{0.984222in}{0.521603in}%
\pgfsys@useobject{currentmarker}{}%
\end{pgfscope}%
\end{pgfscope}%
\begin{pgfscope}%
\definecolor{textcolor}{rgb}{0.000000,0.000000,0.000000}%
\pgfsetstrokecolor{textcolor}%
\pgfsetfillcolor{textcolor}%
\pgftext[x=0.984222in,y=0.424381in,,top]{\color{textcolor}{\rmfamily\fontsize{10.000000}{12.000000}\selectfont\catcode`\^=\active\def^{\ifmmode\sp\else\^{}\fi}\catcode`\%=\active\def%{\%}$\mathdefault{4}$}}%
\end{pgfscope}%
\begin{pgfscope}%
\pgfsetbuttcap%
\pgfsetroundjoin%
\definecolor{currentfill}{rgb}{0.000000,0.000000,0.000000}%
\pgfsetfillcolor{currentfill}%
\pgfsetlinewidth{0.803000pt}%
\definecolor{currentstroke}{rgb}{0.000000,0.000000,0.000000}%
\pgfsetstrokecolor{currentstroke}%
\pgfsetdash{}{0pt}%
\pgfsys@defobject{currentmarker}{\pgfqpoint{0.000000in}{-0.048611in}}{\pgfqpoint{0.000000in}{0.000000in}}{%
\pgfpathmoveto{\pgfqpoint{0.000000in}{0.000000in}}%
\pgfpathlineto{\pgfqpoint{0.000000in}{-0.048611in}}%
\pgfusepath{stroke,fill}%
}%
\begin{pgfscope}%
\pgfsys@transformshift{1.443160in}{0.521603in}%
\pgfsys@useobject{currentmarker}{}%
\end{pgfscope}%
\end{pgfscope}%
\begin{pgfscope}%
\definecolor{textcolor}{rgb}{0.000000,0.000000,0.000000}%
\pgfsetstrokecolor{textcolor}%
\pgfsetfillcolor{textcolor}%
\pgftext[x=1.443160in,y=0.424381in,,top]{\color{textcolor}{\rmfamily\fontsize{10.000000}{12.000000}\selectfont\catcode`\^=\active\def^{\ifmmode\sp\else\^{}\fi}\catcode`\%=\active\def%{\%}$\mathdefault{8}$}}%
\end{pgfscope}%
\begin{pgfscope}%
\pgfsetbuttcap%
\pgfsetroundjoin%
\definecolor{currentfill}{rgb}{0.000000,0.000000,0.000000}%
\pgfsetfillcolor{currentfill}%
\pgfsetlinewidth{0.803000pt}%
\definecolor{currentstroke}{rgb}{0.000000,0.000000,0.000000}%
\pgfsetstrokecolor{currentstroke}%
\pgfsetdash{}{0pt}%
\pgfsys@defobject{currentmarker}{\pgfqpoint{0.000000in}{-0.048611in}}{\pgfqpoint{0.000000in}{0.000000in}}{%
\pgfpathmoveto{\pgfqpoint{0.000000in}{0.000000in}}%
\pgfpathlineto{\pgfqpoint{0.000000in}{-0.048611in}}%
\pgfusepath{stroke,fill}%
}%
\begin{pgfscope}%
\pgfsys@transformshift{1.902099in}{0.521603in}%
\pgfsys@useobject{currentmarker}{}%
\end{pgfscope}%
\end{pgfscope}%
\begin{pgfscope}%
\definecolor{textcolor}{rgb}{0.000000,0.000000,0.000000}%
\pgfsetstrokecolor{textcolor}%
\pgfsetfillcolor{textcolor}%
\pgftext[x=1.902099in,y=0.424381in,,top]{\color{textcolor}{\rmfamily\fontsize{10.000000}{12.000000}\selectfont\catcode`\^=\active\def^{\ifmmode\sp\else\^{}\fi}\catcode`\%=\active\def%{\%}$\mathdefault{12}$}}%
\end{pgfscope}%
\begin{pgfscope}%
\pgfsetbuttcap%
\pgfsetroundjoin%
\definecolor{currentfill}{rgb}{0.000000,0.000000,0.000000}%
\pgfsetfillcolor{currentfill}%
\pgfsetlinewidth{0.803000pt}%
\definecolor{currentstroke}{rgb}{0.000000,0.000000,0.000000}%
\pgfsetstrokecolor{currentstroke}%
\pgfsetdash{}{0pt}%
\pgfsys@defobject{currentmarker}{\pgfqpoint{0.000000in}{-0.048611in}}{\pgfqpoint{0.000000in}{0.000000in}}{%
\pgfpathmoveto{\pgfqpoint{0.000000in}{0.000000in}}%
\pgfpathlineto{\pgfqpoint{0.000000in}{-0.048611in}}%
\pgfusepath{stroke,fill}%
}%
\begin{pgfscope}%
\pgfsys@transformshift{2.361038in}{0.521603in}%
\pgfsys@useobject{currentmarker}{}%
\end{pgfscope}%
\end{pgfscope}%
\begin{pgfscope}%
\definecolor{textcolor}{rgb}{0.000000,0.000000,0.000000}%
\pgfsetstrokecolor{textcolor}%
\pgfsetfillcolor{textcolor}%
\pgftext[x=2.361038in,y=0.424381in,,top]{\color{textcolor}{\rmfamily\fontsize{10.000000}{12.000000}\selectfont\catcode`\^=\active\def^{\ifmmode\sp\else\^{}\fi}\catcode`\%=\active\def%{\%}$\mathdefault{16}$}}%
\end{pgfscope}%
\begin{pgfscope}%
\pgfsetbuttcap%
\pgfsetroundjoin%
\definecolor{currentfill}{rgb}{0.000000,0.000000,0.000000}%
\pgfsetfillcolor{currentfill}%
\pgfsetlinewidth{0.803000pt}%
\definecolor{currentstroke}{rgb}{0.000000,0.000000,0.000000}%
\pgfsetstrokecolor{currentstroke}%
\pgfsetdash{}{0pt}%
\pgfsys@defobject{currentmarker}{\pgfqpoint{0.000000in}{-0.048611in}}{\pgfqpoint{0.000000in}{0.000000in}}{%
\pgfpathmoveto{\pgfqpoint{0.000000in}{0.000000in}}%
\pgfpathlineto{\pgfqpoint{0.000000in}{-0.048611in}}%
\pgfusepath{stroke,fill}%
}%
\begin{pgfscope}%
\pgfsys@transformshift{2.819976in}{0.521603in}%
\pgfsys@useobject{currentmarker}{}%
\end{pgfscope}%
\end{pgfscope}%
\begin{pgfscope}%
\definecolor{textcolor}{rgb}{0.000000,0.000000,0.000000}%
\pgfsetstrokecolor{textcolor}%
\pgfsetfillcolor{textcolor}%
\pgftext[x=2.819976in,y=0.424381in,,top]{\color{textcolor}{\rmfamily\fontsize{10.000000}{12.000000}\selectfont\catcode`\^=\active\def^{\ifmmode\sp\else\^{}\fi}\catcode`\%=\active\def%{\%}$\mathdefault{20}$}}%
\end{pgfscope}%
\begin{pgfscope}%
\pgfsetbuttcap%
\pgfsetroundjoin%
\definecolor{currentfill}{rgb}{0.000000,0.000000,0.000000}%
\pgfsetfillcolor{currentfill}%
\pgfsetlinewidth{0.803000pt}%
\definecolor{currentstroke}{rgb}{0.000000,0.000000,0.000000}%
\pgfsetstrokecolor{currentstroke}%
\pgfsetdash{}{0pt}%
\pgfsys@defobject{currentmarker}{\pgfqpoint{0.000000in}{-0.048611in}}{\pgfqpoint{0.000000in}{0.000000in}}{%
\pgfpathmoveto{\pgfqpoint{0.000000in}{0.000000in}}%
\pgfpathlineto{\pgfqpoint{0.000000in}{-0.048611in}}%
\pgfusepath{stroke,fill}%
}%
\begin{pgfscope}%
\pgfsys@transformshift{3.278915in}{0.521603in}%
\pgfsys@useobject{currentmarker}{}%
\end{pgfscope}%
\end{pgfscope}%
\begin{pgfscope}%
\definecolor{textcolor}{rgb}{0.000000,0.000000,0.000000}%
\pgfsetstrokecolor{textcolor}%
\pgfsetfillcolor{textcolor}%
\pgftext[x=3.278915in,y=0.424381in,,top]{\color{textcolor}{\rmfamily\fontsize{10.000000}{12.000000}\selectfont\catcode`\^=\active\def^{\ifmmode\sp\else\^{}\fi}\catcode`\%=\active\def%{\%}$\mathdefault{24}$}}%
\end{pgfscope}%
\begin{pgfscope}%
\pgfsetbuttcap%
\pgfsetroundjoin%
\definecolor{currentfill}{rgb}{0.000000,0.000000,0.000000}%
\pgfsetfillcolor{currentfill}%
\pgfsetlinewidth{0.803000pt}%
\definecolor{currentstroke}{rgb}{0.000000,0.000000,0.000000}%
\pgfsetstrokecolor{currentstroke}%
\pgfsetdash{}{0pt}%
\pgfsys@defobject{currentmarker}{\pgfqpoint{0.000000in}{-0.048611in}}{\pgfqpoint{0.000000in}{0.000000in}}{%
\pgfpathmoveto{\pgfqpoint{0.000000in}{0.000000in}}%
\pgfpathlineto{\pgfqpoint{0.000000in}{-0.048611in}}%
\pgfusepath{stroke,fill}%
}%
\begin{pgfscope}%
\pgfsys@transformshift{3.737854in}{0.521603in}%
\pgfsys@useobject{currentmarker}{}%
\end{pgfscope}%
\end{pgfscope}%
\begin{pgfscope}%
\definecolor{textcolor}{rgb}{0.000000,0.000000,0.000000}%
\pgfsetstrokecolor{textcolor}%
\pgfsetfillcolor{textcolor}%
\pgftext[x=3.737854in,y=0.424381in,,top]{\color{textcolor}{\rmfamily\fontsize{10.000000}{12.000000}\selectfont\catcode`\^=\active\def^{\ifmmode\sp\else\^{}\fi}\catcode`\%=\active\def%{\%}$\mathdefault{28}$}}%
\end{pgfscope}%
\begin{pgfscope}%
\pgfsetbuttcap%
\pgfsetroundjoin%
\definecolor{currentfill}{rgb}{0.000000,0.000000,0.000000}%
\pgfsetfillcolor{currentfill}%
\pgfsetlinewidth{0.803000pt}%
\definecolor{currentstroke}{rgb}{0.000000,0.000000,0.000000}%
\pgfsetstrokecolor{currentstroke}%
\pgfsetdash{}{0pt}%
\pgfsys@defobject{currentmarker}{\pgfqpoint{0.000000in}{-0.048611in}}{\pgfqpoint{0.000000in}{0.000000in}}{%
\pgfpathmoveto{\pgfqpoint{0.000000in}{0.000000in}}%
\pgfpathlineto{\pgfqpoint{0.000000in}{-0.048611in}}%
\pgfusepath{stroke,fill}%
}%
\begin{pgfscope}%
\pgfsys@transformshift{4.196792in}{0.521603in}%
\pgfsys@useobject{currentmarker}{}%
\end{pgfscope}%
\end{pgfscope}%
\begin{pgfscope}%
\definecolor{textcolor}{rgb}{0.000000,0.000000,0.000000}%
\pgfsetstrokecolor{textcolor}%
\pgfsetfillcolor{textcolor}%
\pgftext[x=4.196792in,y=0.424381in,,top]{\color{textcolor}{\rmfamily\fontsize{10.000000}{12.000000}\selectfont\catcode`\^=\active\def^{\ifmmode\sp\else\^{}\fi}\catcode`\%=\active\def%{\%}$\mathdefault{32}$}}%
\end{pgfscope}%
\begin{pgfscope}%
\definecolor{textcolor}{rgb}{0.000000,0.000000,0.000000}%
\pgfsetstrokecolor{textcolor}%
\pgfsetfillcolor{textcolor}%
\pgftext[x=2.418405in,y=0.234413in,,top]{\color{textcolor}{\rmfamily\fontsize{10.000000}{12.000000}\selectfont\catcode`\^=\active\def^{\ifmmode\sp\else\^{}\fi}\catcode`\%=\active\def%{\%}Monochromatic classes}}%
\end{pgfscope}%
\begin{pgfscope}%
\pgfsetbuttcap%
\pgfsetroundjoin%
\definecolor{currentfill}{rgb}{0.000000,0.000000,0.000000}%
\pgfsetfillcolor{currentfill}%
\pgfsetlinewidth{0.803000pt}%
\definecolor{currentstroke}{rgb}{0.000000,0.000000,0.000000}%
\pgfsetstrokecolor{currentstroke}%
\pgfsetdash{}{0pt}%
\pgfsys@defobject{currentmarker}{\pgfqpoint{-0.048611in}{0.000000in}}{\pgfqpoint{-0.000000in}{0.000000in}}{%
\pgfpathmoveto{\pgfqpoint{-0.000000in}{0.000000in}}%
\pgfpathlineto{\pgfqpoint{-0.048611in}{0.000000in}}%
\pgfusepath{stroke,fill}%
}%
\begin{pgfscope}%
\pgfsys@transformshift{0.588387in}{0.670551in}%
\pgfsys@useobject{currentmarker}{}%
\end{pgfscope}%
\end{pgfscope}%
\begin{pgfscope}%
\definecolor{textcolor}{rgb}{0.000000,0.000000,0.000000}%
\pgfsetstrokecolor{textcolor}%
\pgfsetfillcolor{textcolor}%
\pgftext[x=0.289968in, y=0.617790in, left, base]{\color{textcolor}{\rmfamily\fontsize{10.000000}{12.000000}\selectfont\catcode`\^=\active\def^{\ifmmode\sp\else\^{}\fi}\catcode`\%=\active\def%{\%}$\mathdefault{10^{1}}$}}%
\end{pgfscope}%
\begin{pgfscope}%
\pgfsetbuttcap%
\pgfsetroundjoin%
\definecolor{currentfill}{rgb}{0.000000,0.000000,0.000000}%
\pgfsetfillcolor{currentfill}%
\pgfsetlinewidth{0.803000pt}%
\definecolor{currentstroke}{rgb}{0.000000,0.000000,0.000000}%
\pgfsetstrokecolor{currentstroke}%
\pgfsetdash{}{0pt}%
\pgfsys@defobject{currentmarker}{\pgfqpoint{-0.048611in}{0.000000in}}{\pgfqpoint{-0.000000in}{0.000000in}}{%
\pgfpathmoveto{\pgfqpoint{-0.000000in}{0.000000in}}%
\pgfpathlineto{\pgfqpoint{-0.048611in}{0.000000in}}%
\pgfusepath{stroke,fill}%
}%
\begin{pgfscope}%
\pgfsys@transformshift{0.588387in}{1.343398in}%
\pgfsys@useobject{currentmarker}{}%
\end{pgfscope}%
\end{pgfscope}%
\begin{pgfscope}%
\definecolor{textcolor}{rgb}{0.000000,0.000000,0.000000}%
\pgfsetstrokecolor{textcolor}%
\pgfsetfillcolor{textcolor}%
\pgftext[x=0.289968in, y=1.290636in, left, base]{\color{textcolor}{\rmfamily\fontsize{10.000000}{12.000000}\selectfont\catcode`\^=\active\def^{\ifmmode\sp\else\^{}\fi}\catcode`\%=\active\def%{\%}$\mathdefault{10^{2}}$}}%
\end{pgfscope}%
\begin{pgfscope}%
\pgfsetbuttcap%
\pgfsetroundjoin%
\definecolor{currentfill}{rgb}{0.000000,0.000000,0.000000}%
\pgfsetfillcolor{currentfill}%
\pgfsetlinewidth{0.803000pt}%
\definecolor{currentstroke}{rgb}{0.000000,0.000000,0.000000}%
\pgfsetstrokecolor{currentstroke}%
\pgfsetdash{}{0pt}%
\pgfsys@defobject{currentmarker}{\pgfqpoint{-0.048611in}{0.000000in}}{\pgfqpoint{-0.000000in}{0.000000in}}{%
\pgfpathmoveto{\pgfqpoint{-0.000000in}{0.000000in}}%
\pgfpathlineto{\pgfqpoint{-0.048611in}{0.000000in}}%
\pgfusepath{stroke,fill}%
}%
\begin{pgfscope}%
\pgfsys@transformshift{0.588387in}{2.016245in}%
\pgfsys@useobject{currentmarker}{}%
\end{pgfscope}%
\end{pgfscope}%
\begin{pgfscope}%
\definecolor{textcolor}{rgb}{0.000000,0.000000,0.000000}%
\pgfsetstrokecolor{textcolor}%
\pgfsetfillcolor{textcolor}%
\pgftext[x=0.289968in, y=1.963483in, left, base]{\color{textcolor}{\rmfamily\fontsize{10.000000}{12.000000}\selectfont\catcode`\^=\active\def^{\ifmmode\sp\else\^{}\fi}\catcode`\%=\active\def%{\%}$\mathdefault{10^{3}}$}}%
\end{pgfscope}%
\begin{pgfscope}%
\pgfsetbuttcap%
\pgfsetroundjoin%
\definecolor{currentfill}{rgb}{0.000000,0.000000,0.000000}%
\pgfsetfillcolor{currentfill}%
\pgfsetlinewidth{0.803000pt}%
\definecolor{currentstroke}{rgb}{0.000000,0.000000,0.000000}%
\pgfsetstrokecolor{currentstroke}%
\pgfsetdash{}{0pt}%
\pgfsys@defobject{currentmarker}{\pgfqpoint{-0.048611in}{0.000000in}}{\pgfqpoint{-0.000000in}{0.000000in}}{%
\pgfpathmoveto{\pgfqpoint{-0.000000in}{0.000000in}}%
\pgfpathlineto{\pgfqpoint{-0.048611in}{0.000000in}}%
\pgfusepath{stroke,fill}%
}%
\begin{pgfscope}%
\pgfsys@transformshift{0.588387in}{2.689091in}%
\pgfsys@useobject{currentmarker}{}%
\end{pgfscope}%
\end{pgfscope}%
\begin{pgfscope}%
\definecolor{textcolor}{rgb}{0.000000,0.000000,0.000000}%
\pgfsetstrokecolor{textcolor}%
\pgfsetfillcolor{textcolor}%
\pgftext[x=0.289968in, y=2.636330in, left, base]{\color{textcolor}{\rmfamily\fontsize{10.000000}{12.000000}\selectfont\catcode`\^=\active\def^{\ifmmode\sp\else\^{}\fi}\catcode`\%=\active\def%{\%}$\mathdefault{10^{4}}$}}%
\end{pgfscope}%
\begin{pgfscope}%
\pgfsetbuttcap%
\pgfsetroundjoin%
\definecolor{currentfill}{rgb}{0.000000,0.000000,0.000000}%
\pgfsetfillcolor{currentfill}%
\pgfsetlinewidth{0.602250pt}%
\definecolor{currentstroke}{rgb}{0.000000,0.000000,0.000000}%
\pgfsetstrokecolor{currentstroke}%
\pgfsetdash{}{0pt}%
\pgfsys@defobject{currentmarker}{\pgfqpoint{-0.027778in}{0.000000in}}{\pgfqpoint{-0.000000in}{0.000000in}}{%
\pgfpathmoveto{\pgfqpoint{-0.000000in}{0.000000in}}%
\pgfpathlineto{\pgfqpoint{-0.027778in}{0.000000in}}%
\pgfusepath{stroke,fill}%
}%
\begin{pgfscope}%
\pgfsys@transformshift{0.588387in}{0.566326in}%
\pgfsys@useobject{currentmarker}{}%
\end{pgfscope}%
\end{pgfscope}%
\begin{pgfscope}%
\pgfsetbuttcap%
\pgfsetroundjoin%
\definecolor{currentfill}{rgb}{0.000000,0.000000,0.000000}%
\pgfsetfillcolor{currentfill}%
\pgfsetlinewidth{0.602250pt}%
\definecolor{currentstroke}{rgb}{0.000000,0.000000,0.000000}%
\pgfsetstrokecolor{currentstroke}%
\pgfsetdash{}{0pt}%
\pgfsys@defobject{currentmarker}{\pgfqpoint{-0.027778in}{0.000000in}}{\pgfqpoint{-0.000000in}{0.000000in}}{%
\pgfpathmoveto{\pgfqpoint{-0.000000in}{0.000000in}}%
\pgfpathlineto{\pgfqpoint{-0.027778in}{0.000000in}}%
\pgfusepath{stroke,fill}%
}%
\begin{pgfscope}%
\pgfsys@transformshift{0.588387in}{0.605346in}%
\pgfsys@useobject{currentmarker}{}%
\end{pgfscope}%
\end{pgfscope}%
\begin{pgfscope}%
\pgfsetbuttcap%
\pgfsetroundjoin%
\definecolor{currentfill}{rgb}{0.000000,0.000000,0.000000}%
\pgfsetfillcolor{currentfill}%
\pgfsetlinewidth{0.602250pt}%
\definecolor{currentstroke}{rgb}{0.000000,0.000000,0.000000}%
\pgfsetstrokecolor{currentstroke}%
\pgfsetdash{}{0pt}%
\pgfsys@defobject{currentmarker}{\pgfqpoint{-0.027778in}{0.000000in}}{\pgfqpoint{-0.000000in}{0.000000in}}{%
\pgfpathmoveto{\pgfqpoint{-0.000000in}{0.000000in}}%
\pgfpathlineto{\pgfqpoint{-0.027778in}{0.000000in}}%
\pgfusepath{stroke,fill}%
}%
\begin{pgfscope}%
\pgfsys@transformshift{0.588387in}{0.639763in}%
\pgfsys@useobject{currentmarker}{}%
\end{pgfscope}%
\end{pgfscope}%
\begin{pgfscope}%
\pgfsetbuttcap%
\pgfsetroundjoin%
\definecolor{currentfill}{rgb}{0.000000,0.000000,0.000000}%
\pgfsetfillcolor{currentfill}%
\pgfsetlinewidth{0.602250pt}%
\definecolor{currentstroke}{rgb}{0.000000,0.000000,0.000000}%
\pgfsetstrokecolor{currentstroke}%
\pgfsetdash{}{0pt}%
\pgfsys@defobject{currentmarker}{\pgfqpoint{-0.027778in}{0.000000in}}{\pgfqpoint{-0.000000in}{0.000000in}}{%
\pgfpathmoveto{\pgfqpoint{-0.000000in}{0.000000in}}%
\pgfpathlineto{\pgfqpoint{-0.027778in}{0.000000in}}%
\pgfusepath{stroke,fill}%
}%
\begin{pgfscope}%
\pgfsys@transformshift{0.588387in}{0.873098in}%
\pgfsys@useobject{currentmarker}{}%
\end{pgfscope}%
\end{pgfscope}%
\begin{pgfscope}%
\pgfsetbuttcap%
\pgfsetroundjoin%
\definecolor{currentfill}{rgb}{0.000000,0.000000,0.000000}%
\pgfsetfillcolor{currentfill}%
\pgfsetlinewidth{0.602250pt}%
\definecolor{currentstroke}{rgb}{0.000000,0.000000,0.000000}%
\pgfsetstrokecolor{currentstroke}%
\pgfsetdash{}{0pt}%
\pgfsys@defobject{currentmarker}{\pgfqpoint{-0.027778in}{0.000000in}}{\pgfqpoint{-0.000000in}{0.000000in}}{%
\pgfpathmoveto{\pgfqpoint{-0.000000in}{0.000000in}}%
\pgfpathlineto{\pgfqpoint{-0.027778in}{0.000000in}}%
\pgfusepath{stroke,fill}%
}%
\begin{pgfscope}%
\pgfsys@transformshift{0.588387in}{0.991581in}%
\pgfsys@useobject{currentmarker}{}%
\end{pgfscope}%
\end{pgfscope}%
\begin{pgfscope}%
\pgfsetbuttcap%
\pgfsetroundjoin%
\definecolor{currentfill}{rgb}{0.000000,0.000000,0.000000}%
\pgfsetfillcolor{currentfill}%
\pgfsetlinewidth{0.602250pt}%
\definecolor{currentstroke}{rgb}{0.000000,0.000000,0.000000}%
\pgfsetstrokecolor{currentstroke}%
\pgfsetdash{}{0pt}%
\pgfsys@defobject{currentmarker}{\pgfqpoint{-0.027778in}{0.000000in}}{\pgfqpoint{-0.000000in}{0.000000in}}{%
\pgfpathmoveto{\pgfqpoint{-0.000000in}{0.000000in}}%
\pgfpathlineto{\pgfqpoint{-0.027778in}{0.000000in}}%
\pgfusepath{stroke,fill}%
}%
\begin{pgfscope}%
\pgfsys@transformshift{0.588387in}{1.075645in}%
\pgfsys@useobject{currentmarker}{}%
\end{pgfscope}%
\end{pgfscope}%
\begin{pgfscope}%
\pgfsetbuttcap%
\pgfsetroundjoin%
\definecolor{currentfill}{rgb}{0.000000,0.000000,0.000000}%
\pgfsetfillcolor{currentfill}%
\pgfsetlinewidth{0.602250pt}%
\definecolor{currentstroke}{rgb}{0.000000,0.000000,0.000000}%
\pgfsetstrokecolor{currentstroke}%
\pgfsetdash{}{0pt}%
\pgfsys@defobject{currentmarker}{\pgfqpoint{-0.027778in}{0.000000in}}{\pgfqpoint{-0.000000in}{0.000000in}}{%
\pgfpathmoveto{\pgfqpoint{-0.000000in}{0.000000in}}%
\pgfpathlineto{\pgfqpoint{-0.027778in}{0.000000in}}%
\pgfusepath{stroke,fill}%
}%
\begin{pgfscope}%
\pgfsys@transformshift{0.588387in}{1.140851in}%
\pgfsys@useobject{currentmarker}{}%
\end{pgfscope}%
\end{pgfscope}%
\begin{pgfscope}%
\pgfsetbuttcap%
\pgfsetroundjoin%
\definecolor{currentfill}{rgb}{0.000000,0.000000,0.000000}%
\pgfsetfillcolor{currentfill}%
\pgfsetlinewidth{0.602250pt}%
\definecolor{currentstroke}{rgb}{0.000000,0.000000,0.000000}%
\pgfsetstrokecolor{currentstroke}%
\pgfsetdash{}{0pt}%
\pgfsys@defobject{currentmarker}{\pgfqpoint{-0.027778in}{0.000000in}}{\pgfqpoint{-0.000000in}{0.000000in}}{%
\pgfpathmoveto{\pgfqpoint{-0.000000in}{0.000000in}}%
\pgfpathlineto{\pgfqpoint{-0.027778in}{0.000000in}}%
\pgfusepath{stroke,fill}%
}%
\begin{pgfscope}%
\pgfsys@transformshift{0.588387in}{1.194128in}%
\pgfsys@useobject{currentmarker}{}%
\end{pgfscope}%
\end{pgfscope}%
\begin{pgfscope}%
\pgfsetbuttcap%
\pgfsetroundjoin%
\definecolor{currentfill}{rgb}{0.000000,0.000000,0.000000}%
\pgfsetfillcolor{currentfill}%
\pgfsetlinewidth{0.602250pt}%
\definecolor{currentstroke}{rgb}{0.000000,0.000000,0.000000}%
\pgfsetstrokecolor{currentstroke}%
\pgfsetdash{}{0pt}%
\pgfsys@defobject{currentmarker}{\pgfqpoint{-0.027778in}{0.000000in}}{\pgfqpoint{-0.000000in}{0.000000in}}{%
\pgfpathmoveto{\pgfqpoint{-0.000000in}{0.000000in}}%
\pgfpathlineto{\pgfqpoint{-0.027778in}{0.000000in}}%
\pgfusepath{stroke,fill}%
}%
\begin{pgfscope}%
\pgfsys@transformshift{0.588387in}{1.239173in}%
\pgfsys@useobject{currentmarker}{}%
\end{pgfscope}%
\end{pgfscope}%
\begin{pgfscope}%
\pgfsetbuttcap%
\pgfsetroundjoin%
\definecolor{currentfill}{rgb}{0.000000,0.000000,0.000000}%
\pgfsetfillcolor{currentfill}%
\pgfsetlinewidth{0.602250pt}%
\definecolor{currentstroke}{rgb}{0.000000,0.000000,0.000000}%
\pgfsetstrokecolor{currentstroke}%
\pgfsetdash{}{0pt}%
\pgfsys@defobject{currentmarker}{\pgfqpoint{-0.027778in}{0.000000in}}{\pgfqpoint{-0.000000in}{0.000000in}}{%
\pgfpathmoveto{\pgfqpoint{-0.000000in}{0.000000in}}%
\pgfpathlineto{\pgfqpoint{-0.027778in}{0.000000in}}%
\pgfusepath{stroke,fill}%
}%
\begin{pgfscope}%
\pgfsys@transformshift{0.588387in}{1.278192in}%
\pgfsys@useobject{currentmarker}{}%
\end{pgfscope}%
\end{pgfscope}%
\begin{pgfscope}%
\pgfsetbuttcap%
\pgfsetroundjoin%
\definecolor{currentfill}{rgb}{0.000000,0.000000,0.000000}%
\pgfsetfillcolor{currentfill}%
\pgfsetlinewidth{0.602250pt}%
\definecolor{currentstroke}{rgb}{0.000000,0.000000,0.000000}%
\pgfsetstrokecolor{currentstroke}%
\pgfsetdash{}{0pt}%
\pgfsys@defobject{currentmarker}{\pgfqpoint{-0.027778in}{0.000000in}}{\pgfqpoint{-0.000000in}{0.000000in}}{%
\pgfpathmoveto{\pgfqpoint{-0.000000in}{0.000000in}}%
\pgfpathlineto{\pgfqpoint{-0.027778in}{0.000000in}}%
\pgfusepath{stroke,fill}%
}%
\begin{pgfscope}%
\pgfsys@transformshift{0.588387in}{1.312610in}%
\pgfsys@useobject{currentmarker}{}%
\end{pgfscope}%
\end{pgfscope}%
\begin{pgfscope}%
\pgfsetbuttcap%
\pgfsetroundjoin%
\definecolor{currentfill}{rgb}{0.000000,0.000000,0.000000}%
\pgfsetfillcolor{currentfill}%
\pgfsetlinewidth{0.602250pt}%
\definecolor{currentstroke}{rgb}{0.000000,0.000000,0.000000}%
\pgfsetstrokecolor{currentstroke}%
\pgfsetdash{}{0pt}%
\pgfsys@defobject{currentmarker}{\pgfqpoint{-0.027778in}{0.000000in}}{\pgfqpoint{-0.000000in}{0.000000in}}{%
\pgfpathmoveto{\pgfqpoint{-0.000000in}{0.000000in}}%
\pgfpathlineto{\pgfqpoint{-0.027778in}{0.000000in}}%
\pgfusepath{stroke,fill}%
}%
\begin{pgfscope}%
\pgfsys@transformshift{0.588387in}{1.545945in}%
\pgfsys@useobject{currentmarker}{}%
\end{pgfscope}%
\end{pgfscope}%
\begin{pgfscope}%
\pgfsetbuttcap%
\pgfsetroundjoin%
\definecolor{currentfill}{rgb}{0.000000,0.000000,0.000000}%
\pgfsetfillcolor{currentfill}%
\pgfsetlinewidth{0.602250pt}%
\definecolor{currentstroke}{rgb}{0.000000,0.000000,0.000000}%
\pgfsetstrokecolor{currentstroke}%
\pgfsetdash{}{0pt}%
\pgfsys@defobject{currentmarker}{\pgfqpoint{-0.027778in}{0.000000in}}{\pgfqpoint{-0.000000in}{0.000000in}}{%
\pgfpathmoveto{\pgfqpoint{-0.000000in}{0.000000in}}%
\pgfpathlineto{\pgfqpoint{-0.027778in}{0.000000in}}%
\pgfusepath{stroke,fill}%
}%
\begin{pgfscope}%
\pgfsys@transformshift{0.588387in}{1.664427in}%
\pgfsys@useobject{currentmarker}{}%
\end{pgfscope}%
\end{pgfscope}%
\begin{pgfscope}%
\pgfsetbuttcap%
\pgfsetroundjoin%
\definecolor{currentfill}{rgb}{0.000000,0.000000,0.000000}%
\pgfsetfillcolor{currentfill}%
\pgfsetlinewidth{0.602250pt}%
\definecolor{currentstroke}{rgb}{0.000000,0.000000,0.000000}%
\pgfsetstrokecolor{currentstroke}%
\pgfsetdash{}{0pt}%
\pgfsys@defobject{currentmarker}{\pgfqpoint{-0.027778in}{0.000000in}}{\pgfqpoint{-0.000000in}{0.000000in}}{%
\pgfpathmoveto{\pgfqpoint{-0.000000in}{0.000000in}}%
\pgfpathlineto{\pgfqpoint{-0.027778in}{0.000000in}}%
\pgfusepath{stroke,fill}%
}%
\begin{pgfscope}%
\pgfsys@transformshift{0.588387in}{1.748492in}%
\pgfsys@useobject{currentmarker}{}%
\end{pgfscope}%
\end{pgfscope}%
\begin{pgfscope}%
\pgfsetbuttcap%
\pgfsetroundjoin%
\definecolor{currentfill}{rgb}{0.000000,0.000000,0.000000}%
\pgfsetfillcolor{currentfill}%
\pgfsetlinewidth{0.602250pt}%
\definecolor{currentstroke}{rgb}{0.000000,0.000000,0.000000}%
\pgfsetstrokecolor{currentstroke}%
\pgfsetdash{}{0pt}%
\pgfsys@defobject{currentmarker}{\pgfqpoint{-0.027778in}{0.000000in}}{\pgfqpoint{-0.000000in}{0.000000in}}{%
\pgfpathmoveto{\pgfqpoint{-0.000000in}{0.000000in}}%
\pgfpathlineto{\pgfqpoint{-0.027778in}{0.000000in}}%
\pgfusepath{stroke,fill}%
}%
\begin{pgfscope}%
\pgfsys@transformshift{0.588387in}{1.813697in}%
\pgfsys@useobject{currentmarker}{}%
\end{pgfscope}%
\end{pgfscope}%
\begin{pgfscope}%
\pgfsetbuttcap%
\pgfsetroundjoin%
\definecolor{currentfill}{rgb}{0.000000,0.000000,0.000000}%
\pgfsetfillcolor{currentfill}%
\pgfsetlinewidth{0.602250pt}%
\definecolor{currentstroke}{rgb}{0.000000,0.000000,0.000000}%
\pgfsetstrokecolor{currentstroke}%
\pgfsetdash{}{0pt}%
\pgfsys@defobject{currentmarker}{\pgfqpoint{-0.027778in}{0.000000in}}{\pgfqpoint{-0.000000in}{0.000000in}}{%
\pgfpathmoveto{\pgfqpoint{-0.000000in}{0.000000in}}%
\pgfpathlineto{\pgfqpoint{-0.027778in}{0.000000in}}%
\pgfusepath{stroke,fill}%
}%
\begin{pgfscope}%
\pgfsys@transformshift{0.588387in}{1.866974in}%
\pgfsys@useobject{currentmarker}{}%
\end{pgfscope}%
\end{pgfscope}%
\begin{pgfscope}%
\pgfsetbuttcap%
\pgfsetroundjoin%
\definecolor{currentfill}{rgb}{0.000000,0.000000,0.000000}%
\pgfsetfillcolor{currentfill}%
\pgfsetlinewidth{0.602250pt}%
\definecolor{currentstroke}{rgb}{0.000000,0.000000,0.000000}%
\pgfsetstrokecolor{currentstroke}%
\pgfsetdash{}{0pt}%
\pgfsys@defobject{currentmarker}{\pgfqpoint{-0.027778in}{0.000000in}}{\pgfqpoint{-0.000000in}{0.000000in}}{%
\pgfpathmoveto{\pgfqpoint{-0.000000in}{0.000000in}}%
\pgfpathlineto{\pgfqpoint{-0.027778in}{0.000000in}}%
\pgfusepath{stroke,fill}%
}%
\begin{pgfscope}%
\pgfsys@transformshift{0.588387in}{1.912019in}%
\pgfsys@useobject{currentmarker}{}%
\end{pgfscope}%
\end{pgfscope}%
\begin{pgfscope}%
\pgfsetbuttcap%
\pgfsetroundjoin%
\definecolor{currentfill}{rgb}{0.000000,0.000000,0.000000}%
\pgfsetfillcolor{currentfill}%
\pgfsetlinewidth{0.602250pt}%
\definecolor{currentstroke}{rgb}{0.000000,0.000000,0.000000}%
\pgfsetstrokecolor{currentstroke}%
\pgfsetdash{}{0pt}%
\pgfsys@defobject{currentmarker}{\pgfqpoint{-0.027778in}{0.000000in}}{\pgfqpoint{-0.000000in}{0.000000in}}{%
\pgfpathmoveto{\pgfqpoint{-0.000000in}{0.000000in}}%
\pgfpathlineto{\pgfqpoint{-0.027778in}{0.000000in}}%
\pgfusepath{stroke,fill}%
}%
\begin{pgfscope}%
\pgfsys@transformshift{0.588387in}{1.951039in}%
\pgfsys@useobject{currentmarker}{}%
\end{pgfscope}%
\end{pgfscope}%
\begin{pgfscope}%
\pgfsetbuttcap%
\pgfsetroundjoin%
\definecolor{currentfill}{rgb}{0.000000,0.000000,0.000000}%
\pgfsetfillcolor{currentfill}%
\pgfsetlinewidth{0.602250pt}%
\definecolor{currentstroke}{rgb}{0.000000,0.000000,0.000000}%
\pgfsetstrokecolor{currentstroke}%
\pgfsetdash{}{0pt}%
\pgfsys@defobject{currentmarker}{\pgfqpoint{-0.027778in}{0.000000in}}{\pgfqpoint{-0.000000in}{0.000000in}}{%
\pgfpathmoveto{\pgfqpoint{-0.000000in}{0.000000in}}%
\pgfpathlineto{\pgfqpoint{-0.027778in}{0.000000in}}%
\pgfusepath{stroke,fill}%
}%
\begin{pgfscope}%
\pgfsys@transformshift{0.588387in}{1.985457in}%
\pgfsys@useobject{currentmarker}{}%
\end{pgfscope}%
\end{pgfscope}%
\begin{pgfscope}%
\pgfsetbuttcap%
\pgfsetroundjoin%
\definecolor{currentfill}{rgb}{0.000000,0.000000,0.000000}%
\pgfsetfillcolor{currentfill}%
\pgfsetlinewidth{0.602250pt}%
\definecolor{currentstroke}{rgb}{0.000000,0.000000,0.000000}%
\pgfsetstrokecolor{currentstroke}%
\pgfsetdash{}{0pt}%
\pgfsys@defobject{currentmarker}{\pgfqpoint{-0.027778in}{0.000000in}}{\pgfqpoint{-0.000000in}{0.000000in}}{%
\pgfpathmoveto{\pgfqpoint{-0.000000in}{0.000000in}}%
\pgfpathlineto{\pgfqpoint{-0.027778in}{0.000000in}}%
\pgfusepath{stroke,fill}%
}%
\begin{pgfscope}%
\pgfsys@transformshift{0.588387in}{2.218792in}%
\pgfsys@useobject{currentmarker}{}%
\end{pgfscope}%
\end{pgfscope}%
\begin{pgfscope}%
\pgfsetbuttcap%
\pgfsetroundjoin%
\definecolor{currentfill}{rgb}{0.000000,0.000000,0.000000}%
\pgfsetfillcolor{currentfill}%
\pgfsetlinewidth{0.602250pt}%
\definecolor{currentstroke}{rgb}{0.000000,0.000000,0.000000}%
\pgfsetstrokecolor{currentstroke}%
\pgfsetdash{}{0pt}%
\pgfsys@defobject{currentmarker}{\pgfqpoint{-0.027778in}{0.000000in}}{\pgfqpoint{-0.000000in}{0.000000in}}{%
\pgfpathmoveto{\pgfqpoint{-0.000000in}{0.000000in}}%
\pgfpathlineto{\pgfqpoint{-0.027778in}{0.000000in}}%
\pgfusepath{stroke,fill}%
}%
\begin{pgfscope}%
\pgfsys@transformshift{0.588387in}{2.337274in}%
\pgfsys@useobject{currentmarker}{}%
\end{pgfscope}%
\end{pgfscope}%
\begin{pgfscope}%
\pgfsetbuttcap%
\pgfsetroundjoin%
\definecolor{currentfill}{rgb}{0.000000,0.000000,0.000000}%
\pgfsetfillcolor{currentfill}%
\pgfsetlinewidth{0.602250pt}%
\definecolor{currentstroke}{rgb}{0.000000,0.000000,0.000000}%
\pgfsetstrokecolor{currentstroke}%
\pgfsetdash{}{0pt}%
\pgfsys@defobject{currentmarker}{\pgfqpoint{-0.027778in}{0.000000in}}{\pgfqpoint{-0.000000in}{0.000000in}}{%
\pgfpathmoveto{\pgfqpoint{-0.000000in}{0.000000in}}%
\pgfpathlineto{\pgfqpoint{-0.027778in}{0.000000in}}%
\pgfusepath{stroke,fill}%
}%
\begin{pgfscope}%
\pgfsys@transformshift{0.588387in}{2.421339in}%
\pgfsys@useobject{currentmarker}{}%
\end{pgfscope}%
\end{pgfscope}%
\begin{pgfscope}%
\pgfsetbuttcap%
\pgfsetroundjoin%
\definecolor{currentfill}{rgb}{0.000000,0.000000,0.000000}%
\pgfsetfillcolor{currentfill}%
\pgfsetlinewidth{0.602250pt}%
\definecolor{currentstroke}{rgb}{0.000000,0.000000,0.000000}%
\pgfsetstrokecolor{currentstroke}%
\pgfsetdash{}{0pt}%
\pgfsys@defobject{currentmarker}{\pgfqpoint{-0.027778in}{0.000000in}}{\pgfqpoint{-0.000000in}{0.000000in}}{%
\pgfpathmoveto{\pgfqpoint{-0.000000in}{0.000000in}}%
\pgfpathlineto{\pgfqpoint{-0.027778in}{0.000000in}}%
\pgfusepath{stroke,fill}%
}%
\begin{pgfscope}%
\pgfsys@transformshift{0.588387in}{2.486544in}%
\pgfsys@useobject{currentmarker}{}%
\end{pgfscope}%
\end{pgfscope}%
\begin{pgfscope}%
\pgfsetbuttcap%
\pgfsetroundjoin%
\definecolor{currentfill}{rgb}{0.000000,0.000000,0.000000}%
\pgfsetfillcolor{currentfill}%
\pgfsetlinewidth{0.602250pt}%
\definecolor{currentstroke}{rgb}{0.000000,0.000000,0.000000}%
\pgfsetstrokecolor{currentstroke}%
\pgfsetdash{}{0pt}%
\pgfsys@defobject{currentmarker}{\pgfqpoint{-0.027778in}{0.000000in}}{\pgfqpoint{-0.000000in}{0.000000in}}{%
\pgfpathmoveto{\pgfqpoint{-0.000000in}{0.000000in}}%
\pgfpathlineto{\pgfqpoint{-0.027778in}{0.000000in}}%
\pgfusepath{stroke,fill}%
}%
\begin{pgfscope}%
\pgfsys@transformshift{0.588387in}{2.539821in}%
\pgfsys@useobject{currentmarker}{}%
\end{pgfscope}%
\end{pgfscope}%
\begin{pgfscope}%
\pgfsetbuttcap%
\pgfsetroundjoin%
\definecolor{currentfill}{rgb}{0.000000,0.000000,0.000000}%
\pgfsetfillcolor{currentfill}%
\pgfsetlinewidth{0.602250pt}%
\definecolor{currentstroke}{rgb}{0.000000,0.000000,0.000000}%
\pgfsetstrokecolor{currentstroke}%
\pgfsetdash{}{0pt}%
\pgfsys@defobject{currentmarker}{\pgfqpoint{-0.027778in}{0.000000in}}{\pgfqpoint{-0.000000in}{0.000000in}}{%
\pgfpathmoveto{\pgfqpoint{-0.000000in}{0.000000in}}%
\pgfpathlineto{\pgfqpoint{-0.027778in}{0.000000in}}%
\pgfusepath{stroke,fill}%
}%
\begin{pgfscope}%
\pgfsys@transformshift{0.588387in}{2.584866in}%
\pgfsys@useobject{currentmarker}{}%
\end{pgfscope}%
\end{pgfscope}%
\begin{pgfscope}%
\pgfsetbuttcap%
\pgfsetroundjoin%
\definecolor{currentfill}{rgb}{0.000000,0.000000,0.000000}%
\pgfsetfillcolor{currentfill}%
\pgfsetlinewidth{0.602250pt}%
\definecolor{currentstroke}{rgb}{0.000000,0.000000,0.000000}%
\pgfsetstrokecolor{currentstroke}%
\pgfsetdash{}{0pt}%
\pgfsys@defobject{currentmarker}{\pgfqpoint{-0.027778in}{0.000000in}}{\pgfqpoint{-0.000000in}{0.000000in}}{%
\pgfpathmoveto{\pgfqpoint{-0.000000in}{0.000000in}}%
\pgfpathlineto{\pgfqpoint{-0.027778in}{0.000000in}}%
\pgfusepath{stroke,fill}%
}%
\begin{pgfscope}%
\pgfsys@transformshift{0.588387in}{2.623886in}%
\pgfsys@useobject{currentmarker}{}%
\end{pgfscope}%
\end{pgfscope}%
\begin{pgfscope}%
\pgfsetbuttcap%
\pgfsetroundjoin%
\definecolor{currentfill}{rgb}{0.000000,0.000000,0.000000}%
\pgfsetfillcolor{currentfill}%
\pgfsetlinewidth{0.602250pt}%
\definecolor{currentstroke}{rgb}{0.000000,0.000000,0.000000}%
\pgfsetstrokecolor{currentstroke}%
\pgfsetdash{}{0pt}%
\pgfsys@defobject{currentmarker}{\pgfqpoint{-0.027778in}{0.000000in}}{\pgfqpoint{-0.000000in}{0.000000in}}{%
\pgfpathmoveto{\pgfqpoint{-0.000000in}{0.000000in}}%
\pgfpathlineto{\pgfqpoint{-0.027778in}{0.000000in}}%
\pgfusepath{stroke,fill}%
}%
\begin{pgfscope}%
\pgfsys@transformshift{0.588387in}{2.658303in}%
\pgfsys@useobject{currentmarker}{}%
\end{pgfscope}%
\end{pgfscope}%
\begin{pgfscope}%
\definecolor{textcolor}{rgb}{0.000000,0.000000,0.000000}%
\pgfsetstrokecolor{textcolor}%
\pgfsetfillcolor{textcolor}%
\pgftext[x=0.234413in,y=1.617672in,,bottom,rotate=90.000000]{\color{textcolor}{\rmfamily\fontsize{10.000000}{12.000000}\selectfont\catcode`\^=\active\def^{\ifmmode\sp\else\^{}\fi}\catcode`\%=\active\def%{\%}Time [ms]}}%
\end{pgfscope}%
\begin{pgfscope}%
\pgfpathrectangle{\pgfqpoint{0.588387in}{0.521603in}}{\pgfqpoint{3.660036in}{2.192138in}}%
\pgfusepath{clip}%
\pgfsetrectcap%
\pgfsetroundjoin%
\pgfsetlinewidth{1.505625pt}%
\pgfsetstrokecolor{currentstroke1}%
\pgfsetdash{}{0pt}%
\pgfpathmoveto{\pgfqpoint{0.754752in}{0.650495in}}%
\pgfpathlineto{\pgfqpoint{0.869487in}{0.708985in}}%
\pgfpathlineto{\pgfqpoint{0.984222in}{0.681895in}}%
\pgfpathlineto{\pgfqpoint{1.098956in}{0.712675in}}%
\pgfpathlineto{\pgfqpoint{1.213691in}{0.670369in}}%
\pgfpathlineto{\pgfqpoint{1.328426in}{0.669607in}}%
\pgfpathlineto{\pgfqpoint{1.443160in}{0.699589in}}%
\pgfpathlineto{\pgfqpoint{1.557895in}{0.690039in}}%
\pgfpathlineto{\pgfqpoint{1.672630in}{0.708009in}}%
\pgfpathlineto{\pgfqpoint{1.787364in}{0.752784in}}%
\pgfpathlineto{\pgfqpoint{1.902099in}{0.854173in}}%
\pgfpathlineto{\pgfqpoint{2.016834in}{0.868496in}}%
\pgfpathlineto{\pgfqpoint{2.131568in}{0.937271in}}%
\pgfpathlineto{\pgfqpoint{2.246303in}{0.995673in}}%
\pgfpathlineto{\pgfqpoint{2.361038in}{1.073200in}}%
\pgfpathlineto{\pgfqpoint{2.475772in}{1.206990in}}%
\pgfpathlineto{\pgfqpoint{2.590507in}{1.203120in}}%
\pgfpathlineto{\pgfqpoint{2.705242in}{1.221092in}}%
\pgfpathlineto{\pgfqpoint{2.819976in}{1.337196in}}%
\pgfpathlineto{\pgfqpoint{2.934711in}{1.379628in}}%
\pgfpathlineto{\pgfqpoint{3.049446in}{1.442241in}}%
\pgfpathlineto{\pgfqpoint{3.164180in}{1.372574in}}%
\pgfpathlineto{\pgfqpoint{3.278915in}{1.362485in}}%
\pgfpathlineto{\pgfqpoint{3.393650in}{1.427828in}}%
\pgfpathlineto{\pgfqpoint{3.508384in}{1.408603in}}%
\pgfpathlineto{\pgfqpoint{3.623119in}{1.570120in}}%
\pgfpathlineto{\pgfqpoint{3.967323in}{1.814573in}}%
\pgfpathlineto{\pgfqpoint{4.082057in}{2.117887in}}%
\pgfusepath{stroke}%
\end{pgfscope}%
\begin{pgfscope}%
\pgfpathrectangle{\pgfqpoint{0.588387in}{0.521603in}}{\pgfqpoint{3.660036in}{2.192138in}}%
\pgfusepath{clip}%
\pgfsetrectcap%
\pgfsetroundjoin%
\pgfsetlinewidth{1.505625pt}%
\pgfsetstrokecolor{currentstroke2}%
\pgfsetdash{}{0pt}%
\pgfpathmoveto{\pgfqpoint{0.754752in}{0.649984in}}%
\pgfpathlineto{\pgfqpoint{0.869487in}{0.706070in}}%
\pgfpathlineto{\pgfqpoint{0.984222in}{0.679740in}}%
\pgfpathlineto{\pgfqpoint{1.098956in}{0.713475in}}%
\pgfpathlineto{\pgfqpoint{1.213691in}{0.668892in}}%
\pgfpathlineto{\pgfqpoint{1.328426in}{0.671503in}}%
\pgfpathlineto{\pgfqpoint{1.443160in}{0.696049in}}%
\pgfpathlineto{\pgfqpoint{1.557895in}{0.689568in}}%
\pgfpathlineto{\pgfqpoint{1.672630in}{0.710268in}}%
\pgfpathlineto{\pgfqpoint{1.787364in}{0.748819in}}%
\pgfpathlineto{\pgfqpoint{1.902099in}{0.858878in}}%
\pgfpathlineto{\pgfqpoint{2.016834in}{0.862877in}}%
\pgfpathlineto{\pgfqpoint{2.131568in}{0.943753in}}%
\pgfpathlineto{\pgfqpoint{2.246303in}{0.994191in}}%
\pgfpathlineto{\pgfqpoint{2.361038in}{1.064160in}}%
\pgfpathlineto{\pgfqpoint{2.475772in}{1.181183in}}%
\pgfpathlineto{\pgfqpoint{2.590507in}{1.154411in}}%
\pgfpathlineto{\pgfqpoint{2.705242in}{1.230702in}}%
\pgfpathlineto{\pgfqpoint{2.819976in}{1.317121in}}%
\pgfpathlineto{\pgfqpoint{2.934711in}{1.367506in}}%
\pgfpathlineto{\pgfqpoint{3.049446in}{1.442241in}}%
\pgfpathlineto{\pgfqpoint{3.164180in}{1.304378in}}%
\pgfpathlineto{\pgfqpoint{3.278915in}{1.435390in}}%
\pgfpathlineto{\pgfqpoint{3.393650in}{1.409770in}}%
\pgfpathlineto{\pgfqpoint{3.508384in}{1.598612in}}%
\pgfpathlineto{\pgfqpoint{3.623119in}{1.598612in}}%
\pgfpathlineto{\pgfqpoint{3.967323in}{1.749222in}}%
\pgfpathlineto{\pgfqpoint{4.082057in}{1.965991in}}%
\pgfusepath{stroke}%
\end{pgfscope}%
\begin{pgfscope}%
\pgfpathrectangle{\pgfqpoint{0.588387in}{0.521603in}}{\pgfqpoint{3.660036in}{2.192138in}}%
\pgfusepath{clip}%
\pgfsetrectcap%
\pgfsetroundjoin%
\pgfsetlinewidth{1.505625pt}%
\pgfsetstrokecolor{currentstroke3}%
\pgfsetdash{}{0pt}%
\pgfpathmoveto{\pgfqpoint{0.754752in}{0.622189in}}%
\pgfpathlineto{\pgfqpoint{0.869487in}{0.680427in}}%
\pgfpathlineto{\pgfqpoint{0.984222in}{0.645249in}}%
\pgfpathlineto{\pgfqpoint{1.098956in}{0.676747in}}%
\pgfpathlineto{\pgfqpoint{1.213691in}{0.666666in}}%
\pgfpathlineto{\pgfqpoint{1.328426in}{0.621246in}}%
\pgfpathlineto{\pgfqpoint{1.443160in}{0.644913in}}%
\pgfpathlineto{\pgfqpoint{1.557895in}{0.646520in}}%
\pgfpathlineto{\pgfqpoint{1.672630in}{0.642073in}}%
\pgfpathlineto{\pgfqpoint{1.787364in}{0.711392in}}%
\pgfpathlineto{\pgfqpoint{1.902099in}{0.694916in}}%
\pgfpathlineto{\pgfqpoint{2.016834in}{0.737214in}}%
\pgfpathlineto{\pgfqpoint{2.131568in}{0.838540in}}%
\pgfpathlineto{\pgfqpoint{2.246303in}{0.947714in}}%
\pgfpathlineto{\pgfqpoint{2.361038in}{1.105482in}}%
\pgfpathlineto{\pgfqpoint{2.475772in}{1.295908in}}%
\pgfpathlineto{\pgfqpoint{2.590507in}{1.466710in}}%
\pgfpathlineto{\pgfqpoint{2.705242in}{1.643221in}}%
\pgfpathlineto{\pgfqpoint{2.819976in}{1.849411in}}%
\pgfpathlineto{\pgfqpoint{2.934711in}{2.014682in}}%
\pgfpathlineto{\pgfqpoint{3.049446in}{2.227500in}}%
\pgfpathlineto{\pgfqpoint{3.278915in}{2.614099in}}%
\pgfusepath{stroke}%
\end{pgfscope}%
\begin{pgfscope}%
\pgfpathrectangle{\pgfqpoint{0.588387in}{0.521603in}}{\pgfqpoint{3.660036in}{2.192138in}}%
\pgfusepath{clip}%
\pgfsetrectcap%
\pgfsetroundjoin%
\pgfsetlinewidth{1.505625pt}%
\pgfsetstrokecolor{currentstroke4}%
\pgfsetdash{}{0pt}%
\pgfpathmoveto{\pgfqpoint{0.754752in}{0.648839in}}%
\pgfpathlineto{\pgfqpoint{0.869487in}{0.703864in}}%
\pgfpathlineto{\pgfqpoint{0.984222in}{0.677371in}}%
\pgfpathlineto{\pgfqpoint{1.098956in}{0.708157in}}%
\pgfpathlineto{\pgfqpoint{1.213691in}{0.661085in}}%
\pgfpathlineto{\pgfqpoint{1.328426in}{0.668959in}}%
\pgfpathlineto{\pgfqpoint{1.443160in}{0.701206in}}%
\pgfpathlineto{\pgfqpoint{1.557895in}{0.682603in}}%
\pgfpathlineto{\pgfqpoint{1.672630in}{0.692393in}}%
\pgfpathlineto{\pgfqpoint{1.787364in}{0.735757in}}%
\pgfpathlineto{\pgfqpoint{1.902099in}{0.822440in}}%
\pgfpathlineto{\pgfqpoint{2.016834in}{0.842817in}}%
\pgfpathlineto{\pgfqpoint{2.131568in}{0.896223in}}%
\pgfpathlineto{\pgfqpoint{2.246303in}{0.944531in}}%
\pgfpathlineto{\pgfqpoint{2.361038in}{0.999587in}}%
\pgfpathlineto{\pgfqpoint{2.475772in}{1.074181in}}%
\pgfpathlineto{\pgfqpoint{2.590507in}{1.118069in}}%
\pgfpathlineto{\pgfqpoint{2.705242in}{1.153993in}}%
\pgfpathlineto{\pgfqpoint{2.819976in}{1.261275in}}%
\pgfpathlineto{\pgfqpoint{2.934711in}{1.290705in}}%
\pgfpathlineto{\pgfqpoint{3.049446in}{1.399401in}}%
\pgfpathlineto{\pgfqpoint{3.164180in}{1.208385in}}%
\pgfpathlineto{\pgfqpoint{3.278915in}{1.198958in}}%
\pgfpathlineto{\pgfqpoint{3.393650in}{1.546675in}}%
\pgfpathlineto{\pgfqpoint{3.508384in}{1.307699in}}%
\pgfpathlineto{\pgfqpoint{3.623119in}{1.625964in}}%
\pgfpathlineto{\pgfqpoint{3.967323in}{1.542269in}}%
\pgfpathlineto{\pgfqpoint{4.082057in}{1.483466in}}%
\pgfusepath{stroke}%
\end{pgfscope}%
\begin{pgfscope}%
\pgfpathrectangle{\pgfqpoint{0.588387in}{0.521603in}}{\pgfqpoint{3.660036in}{2.192138in}}%
\pgfusepath{clip}%
\pgfsetrectcap%
\pgfsetroundjoin%
\pgfsetlinewidth{1.505625pt}%
\pgfsetstrokecolor{currentstroke5}%
\pgfsetdash{}{0pt}%
\pgfpathmoveto{\pgfqpoint{0.754752in}{0.648732in}}%
\pgfpathlineto{\pgfqpoint{0.869487in}{0.702385in}}%
\pgfpathlineto{\pgfqpoint{0.984222in}{0.678261in}}%
\pgfpathlineto{\pgfqpoint{1.098956in}{0.706527in}}%
\pgfpathlineto{\pgfqpoint{1.213691in}{0.670364in}}%
\pgfpathlineto{\pgfqpoint{1.328426in}{0.694916in}}%
\pgfpathlineto{\pgfqpoint{1.443160in}{0.689547in}}%
\pgfpathlineto{\pgfqpoint{1.557895in}{0.685566in}}%
\pgfpathlineto{\pgfqpoint{1.672630in}{0.696788in}}%
\pgfpathlineto{\pgfqpoint{1.787364in}{0.736590in}}%
\pgfpathlineto{\pgfqpoint{1.902099in}{0.825608in}}%
\pgfpathlineto{\pgfqpoint{2.016834in}{0.840785in}}%
\pgfpathlineto{\pgfqpoint{2.131568in}{0.900558in}}%
\pgfpathlineto{\pgfqpoint{2.246303in}{0.948899in}}%
\pgfpathlineto{\pgfqpoint{2.361038in}{0.998796in}}%
\pgfpathlineto{\pgfqpoint{2.475772in}{1.072708in}}%
\pgfpathlineto{\pgfqpoint{2.590507in}{1.082861in}}%
\pgfpathlineto{\pgfqpoint{2.705242in}{1.111359in}}%
\pgfpathlineto{\pgfqpoint{2.819976in}{1.262819in}}%
\pgfpathlineto{\pgfqpoint{2.934711in}{1.314229in}}%
\pgfpathlineto{\pgfqpoint{3.049446in}{1.367237in}}%
\pgfpathlineto{\pgfqpoint{3.164180in}{1.191682in}}%
\pgfpathlineto{\pgfqpoint{3.278915in}{1.312610in}}%
\pgfpathlineto{\pgfqpoint{3.393650in}{1.473341in}}%
\pgfpathlineto{\pgfqpoint{3.508384in}{1.325317in}}%
\pgfpathlineto{\pgfqpoint{3.623119in}{1.547039in}}%
\pgfpathlineto{\pgfqpoint{3.967323in}{1.535534in}}%
\pgfpathlineto{\pgfqpoint{4.082057in}{1.527086in}}%
\pgfusepath{stroke}%
\end{pgfscope}%
\begin{pgfscope}%
\pgfpathrectangle{\pgfqpoint{0.588387in}{0.521603in}}{\pgfqpoint{3.660036in}{2.192138in}}%
\pgfusepath{clip}%
\pgfsetrectcap%
\pgfsetroundjoin%
\pgfsetlinewidth{1.505625pt}%
\pgfsetstrokecolor{currentstroke6}%
\pgfsetdash{}{0pt}%
\pgfpathmoveto{\pgfqpoint{0.754752in}{0.646306in}}%
\pgfpathlineto{\pgfqpoint{0.869487in}{0.706679in}}%
\pgfpathlineto{\pgfqpoint{0.984222in}{0.676179in}}%
\pgfpathlineto{\pgfqpoint{1.098956in}{0.708482in}}%
\pgfpathlineto{\pgfqpoint{1.213691in}{0.670174in}}%
\pgfpathlineto{\pgfqpoint{1.328426in}{0.660206in}}%
\pgfpathlineto{\pgfqpoint{1.443160in}{0.692798in}}%
\pgfpathlineto{\pgfqpoint{1.557895in}{0.684048in}}%
\pgfpathlineto{\pgfqpoint{1.672630in}{0.696812in}}%
\pgfpathlineto{\pgfqpoint{1.787364in}{0.739902in}}%
\pgfpathlineto{\pgfqpoint{1.902099in}{0.829026in}}%
\pgfpathlineto{\pgfqpoint{2.016834in}{0.848832in}}%
\pgfpathlineto{\pgfqpoint{2.131568in}{0.904444in}}%
\pgfpathlineto{\pgfqpoint{2.246303in}{0.941878in}}%
\pgfpathlineto{\pgfqpoint{2.361038in}{0.999192in}}%
\pgfpathlineto{\pgfqpoint{2.475772in}{1.080715in}}%
\pgfpathlineto{\pgfqpoint{2.590507in}{1.085521in}}%
\pgfpathlineto{\pgfqpoint{2.705242in}{1.163340in}}%
\pgfpathlineto{\pgfqpoint{2.819976in}{1.287184in}}%
\pgfpathlineto{\pgfqpoint{2.934711in}{1.341051in}}%
\pgfpathlineto{\pgfqpoint{3.049446in}{1.449698in}}%
\pgfpathlineto{\pgfqpoint{3.164180in}{1.276360in}}%
\pgfpathlineto{\pgfqpoint{3.278915in}{1.220870in}}%
\pgfpathlineto{\pgfqpoint{3.393650in}{1.674480in}}%
\pgfpathlineto{\pgfqpoint{3.508384in}{1.377816in}}%
\pgfpathlineto{\pgfqpoint{3.623119in}{1.632555in}}%
\pgfpathlineto{\pgfqpoint{3.967323in}{1.649439in}}%
\pgfpathlineto{\pgfqpoint{4.082057in}{1.593070in}}%
\pgfusepath{stroke}%
\end{pgfscope}%
\begin{pgfscope}%
\pgfpathrectangle{\pgfqpoint{0.588387in}{0.521603in}}{\pgfqpoint{3.660036in}{2.192138in}}%
\pgfusepath{clip}%
\pgfsetrectcap%
\pgfsetroundjoin%
\pgfsetlinewidth{1.505625pt}%
\pgfsetstrokecolor{currentstroke7}%
\pgfsetdash{}{0pt}%
\pgfpathmoveto{\pgfqpoint{0.754752in}{0.646776in}}%
\pgfpathlineto{\pgfqpoint{0.869487in}{0.706314in}}%
\pgfpathlineto{\pgfqpoint{0.984222in}{0.677569in}}%
\pgfpathlineto{\pgfqpoint{1.098956in}{0.707506in}}%
\pgfpathlineto{\pgfqpoint{1.213691in}{0.662298in}}%
\pgfpathlineto{\pgfqpoint{1.328426in}{0.663729in}}%
\pgfpathlineto{\pgfqpoint{1.443160in}{0.691890in}}%
\pgfpathlineto{\pgfqpoint{1.557895in}{0.680301in}}%
\pgfpathlineto{\pgfqpoint{1.672630in}{0.695213in}}%
\pgfpathlineto{\pgfqpoint{1.787364in}{0.737422in}}%
\pgfpathlineto{\pgfqpoint{1.902099in}{0.830720in}}%
\pgfpathlineto{\pgfqpoint{2.016834in}{0.841294in}}%
\pgfpathlineto{\pgfqpoint{2.131568in}{0.902896in}}%
\pgfpathlineto{\pgfqpoint{2.246303in}{0.946723in}}%
\pgfpathlineto{\pgfqpoint{2.361038in}{1.002338in}}%
\pgfpathlineto{\pgfqpoint{2.475772in}{1.082147in}}%
\pgfpathlineto{\pgfqpoint{2.590507in}{1.083750in}}%
\pgfpathlineto{\pgfqpoint{2.705242in}{1.167104in}}%
\pgfpathlineto{\pgfqpoint{2.819976in}{1.262819in}}%
\pgfpathlineto{\pgfqpoint{2.934711in}{1.320932in}}%
\pgfpathlineto{\pgfqpoint{3.049446in}{1.420626in}}%
\pgfpathlineto{\pgfqpoint{3.164180in}{1.226373in}}%
\pgfpathlineto{\pgfqpoint{3.278915in}{1.182959in}}%
\pgfpathlineto{\pgfqpoint{3.393650in}{1.555997in}}%
\pgfpathlineto{\pgfqpoint{3.508384in}{1.285408in}}%
\pgfpathlineto{\pgfqpoint{3.623119in}{1.581011in}}%
\pgfpathlineto{\pgfqpoint{3.967323in}{1.526306in}}%
\pgfpathlineto{\pgfqpoint{4.082057in}{1.562972in}}%
\pgfusepath{stroke}%
\end{pgfscope}%
\begin{pgfscope}%
\pgfpathrectangle{\pgfqpoint{0.588387in}{0.521603in}}{\pgfqpoint{3.660036in}{2.192138in}}%
\pgfusepath{clip}%
\pgfsetrectcap%
\pgfsetroundjoin%
\pgfsetlinewidth{1.505625pt}%
\definecolor{currentstroke}{rgb}{0.498039,0.498039,0.498039}%
\pgfsetstrokecolor{currentstroke}%
\pgfsetdash{}{0pt}%
\pgfpathmoveto{\pgfqpoint{0.754752in}{0.653001in}}%
\pgfpathlineto{\pgfqpoint{0.869487in}{0.707288in}}%
\pgfpathlineto{\pgfqpoint{0.984222in}{0.681504in}}%
\pgfpathlineto{\pgfqpoint{1.098956in}{0.710909in}}%
\pgfpathlineto{\pgfqpoint{1.213691in}{0.671131in}}%
\pgfpathlineto{\pgfqpoint{1.328426in}{0.668575in}}%
\pgfpathlineto{\pgfqpoint{1.443160in}{0.694349in}}%
\pgfpathlineto{\pgfqpoint{1.557895in}{0.687578in}}%
\pgfpathlineto{\pgfqpoint{1.672630in}{0.703122in}}%
\pgfpathlineto{\pgfqpoint{1.787364in}{0.754616in}}%
\pgfpathlineto{\pgfqpoint{1.902099in}{0.841498in}}%
\pgfpathlineto{\pgfqpoint{2.016834in}{0.864762in}}%
\pgfpathlineto{\pgfqpoint{2.131568in}{0.933101in}}%
\pgfpathlineto{\pgfqpoint{2.246303in}{1.006906in}}%
\pgfpathlineto{\pgfqpoint{2.361038in}{1.070424in}}%
\pgfpathlineto{\pgfqpoint{2.475772in}{1.204650in}}%
\pgfpathlineto{\pgfqpoint{2.590507in}{1.169365in}}%
\pgfpathlineto{\pgfqpoint{2.705242in}{1.206524in}}%
\pgfpathlineto{\pgfqpoint{2.819976in}{1.434749in}}%
\pgfpathlineto{\pgfqpoint{2.934711in}{1.373893in}}%
\pgfpathlineto{\pgfqpoint{3.049446in}{1.494125in}}%
\pgfpathlineto{\pgfqpoint{3.164180in}{1.186729in}}%
\pgfpathlineto{\pgfqpoint{3.278915in}{1.521580in}}%
\pgfpathlineto{\pgfqpoint{3.393650in}{1.651991in}}%
\pgfpathlineto{\pgfqpoint{3.508384in}{1.463822in}}%
\pgfpathlineto{\pgfqpoint{3.623119in}{1.867582in}}%
\pgfpathlineto{\pgfqpoint{3.967323in}{2.146188in}}%
\pgfpathlineto{\pgfqpoint{4.082057in}{1.946437in}}%
\pgfusepath{stroke}%
\end{pgfscope}%
\begin{pgfscope}%
\pgfpathrectangle{\pgfqpoint{0.588387in}{0.521603in}}{\pgfqpoint{3.660036in}{2.192138in}}%
\pgfusepath{clip}%
\pgfsetrectcap%
\pgfsetroundjoin%
\pgfsetlinewidth{1.505625pt}%
\definecolor{currentstroke}{rgb}{0.737255,0.741176,0.133333}%
\pgfsetstrokecolor{currentstroke}%
\pgfsetdash{}{0pt}%
\pgfpathmoveto{\pgfqpoint{0.754752in}{0.645339in}}%
\pgfpathlineto{\pgfqpoint{0.869487in}{0.701034in}}%
\pgfpathlineto{\pgfqpoint{0.984222in}{0.674081in}}%
\pgfpathlineto{\pgfqpoint{1.098956in}{0.703568in}}%
\pgfpathlineto{\pgfqpoint{1.213691in}{0.667811in}}%
\pgfpathlineto{\pgfqpoint{1.328426in}{0.662319in}}%
\pgfpathlineto{\pgfqpoint{1.443160in}{0.689410in}}%
\pgfpathlineto{\pgfqpoint{1.557895in}{0.689755in}}%
\pgfpathlineto{\pgfqpoint{1.672630in}{0.698402in}}%
\pgfpathlineto{\pgfqpoint{1.787364in}{0.748019in}}%
\pgfpathlineto{\pgfqpoint{1.902099in}{0.834357in}}%
\pgfpathlineto{\pgfqpoint{2.016834in}{0.856181in}}%
\pgfpathlineto{\pgfqpoint{2.131568in}{0.919857in}}%
\pgfpathlineto{\pgfqpoint{2.246303in}{0.990078in}}%
\pgfpathlineto{\pgfqpoint{2.361038in}{1.059050in}}%
\pgfpathlineto{\pgfqpoint{2.475772in}{1.192173in}}%
\pgfpathlineto{\pgfqpoint{2.590507in}{1.131196in}}%
\pgfpathlineto{\pgfqpoint{2.705242in}{1.207456in}}%
\pgfpathlineto{\pgfqpoint{2.819976in}{1.379370in}}%
\pgfpathlineto{\pgfqpoint{2.934711in}{1.343690in}}%
\pgfpathlineto{\pgfqpoint{3.049446in}{1.482560in}}%
\pgfpathlineto{\pgfqpoint{3.164180in}{1.184221in}}%
\pgfpathlineto{\pgfqpoint{3.278915in}{1.444317in}}%
\pgfpathlineto{\pgfqpoint{3.393650in}{1.781934in}}%
\pgfpathlineto{\pgfqpoint{3.508384in}{1.464788in}}%
\pgfpathlineto{\pgfqpoint{3.623119in}{1.799323in}}%
\pgfpathlineto{\pgfqpoint{3.967323in}{1.685107in}}%
\pgfusepath{stroke}%
\end{pgfscope}%
\begin{pgfscope}%
\pgfsetrectcap%
\pgfsetmiterjoin%
\pgfsetlinewidth{0.803000pt}%
\definecolor{currentstroke}{rgb}{0.000000,0.000000,0.000000}%
\pgfsetstrokecolor{currentstroke}%
\pgfsetdash{}{0pt}%
\pgfpathmoveto{\pgfqpoint{0.588387in}{0.521603in}}%
\pgfpathlineto{\pgfqpoint{0.588387in}{2.713741in}}%
\pgfusepath{stroke}%
\end{pgfscope}%
\begin{pgfscope}%
\pgfsetrectcap%
\pgfsetmiterjoin%
\pgfsetlinewidth{0.803000pt}%
\definecolor{currentstroke}{rgb}{0.000000,0.000000,0.000000}%
\pgfsetstrokecolor{currentstroke}%
\pgfsetdash{}{0pt}%
\pgfpathmoveto{\pgfqpoint{4.248423in}{0.521603in}}%
\pgfpathlineto{\pgfqpoint{4.248423in}{2.713741in}}%
\pgfusepath{stroke}%
\end{pgfscope}%
\begin{pgfscope}%
\pgfsetrectcap%
\pgfsetmiterjoin%
\pgfsetlinewidth{0.803000pt}%
\definecolor{currentstroke}{rgb}{0.000000,0.000000,0.000000}%
\pgfsetstrokecolor{currentstroke}%
\pgfsetdash{}{0pt}%
\pgfpathmoveto{\pgfqpoint{0.588387in}{0.521603in}}%
\pgfpathlineto{\pgfqpoint{4.248423in}{0.521603in}}%
\pgfusepath{stroke}%
\end{pgfscope}%
\begin{pgfscope}%
\pgfsetrectcap%
\pgfsetmiterjoin%
\pgfsetlinewidth{0.803000pt}%
\definecolor{currentstroke}{rgb}{0.000000,0.000000,0.000000}%
\pgfsetstrokecolor{currentstroke}%
\pgfsetdash{}{0pt}%
\pgfpathmoveto{\pgfqpoint{0.588387in}{2.713741in}}%
\pgfpathlineto{\pgfqpoint{4.248423in}{2.713741in}}%
\pgfusepath{stroke}%
\end{pgfscope}%
\begin{pgfscope}%
\pgfsetbuttcap%
\pgfsetmiterjoin%
\definecolor{currentfill}{rgb}{1.000000,1.000000,1.000000}%
\pgfsetfillcolor{currentfill}%
\pgfsetfillopacity{0.800000}%
\pgfsetlinewidth{1.003750pt}%
\definecolor{currentstroke}{rgb}{0.800000,0.800000,0.800000}%
\pgfsetstrokecolor{currentstroke}%
\pgfsetstrokeopacity{0.800000}%
\pgfsetdash{}{0pt}%
\pgfpathmoveto{\pgfqpoint{4.365089in}{0.350918in}}%
\pgfpathlineto{\pgfqpoint{8.251043in}{0.350918in}}%
\pgfpathquadraticcurveto{\pgfqpoint{8.284376in}{0.350918in}}{\pgfqpoint{8.284376in}{0.384251in}}%
\pgfpathlineto{\pgfqpoint{8.284376in}{2.597075in}}%
\pgfpathquadraticcurveto{\pgfqpoint{8.284376in}{2.630408in}}{\pgfqpoint{8.251043in}{2.630408in}}%
\pgfpathlineto{\pgfqpoint{4.365089in}{2.630408in}}%
\pgfpathquadraticcurveto{\pgfqpoint{4.331756in}{2.630408in}}{\pgfqpoint{4.331756in}{2.597075in}}%
\pgfpathlineto{\pgfqpoint{4.331756in}{0.384251in}}%
\pgfpathquadraticcurveto{\pgfqpoint{4.331756in}{0.350918in}}{\pgfqpoint{4.365089in}{0.350918in}}%
\pgfpathlineto{\pgfqpoint{4.365089in}{0.350918in}}%
\pgfpathclose%
\pgfusepath{stroke,fill}%
\end{pgfscope}%
\begin{pgfscope}%
\pgfsetrectcap%
\pgfsetroundjoin%
\pgfsetlinewidth{1.505625pt}%
\pgfsetstrokecolor{currentstroke3}%
\pgfsetdash{}{0pt}%
\pgfpathmoveto{\pgfqpoint{4.398423in}{2.495447in}}%
\pgfpathlineto{\pgfqpoint{4.565089in}{2.495447in}}%
\pgfpathlineto{\pgfqpoint{4.731756in}{2.495447in}}%
\pgfusepath{stroke}%
\end{pgfscope}%
\begin{pgfscope}%
\definecolor{textcolor}{rgb}{0.000000,0.000000,0.000000}%
\pgfsetstrokecolor{textcolor}%
\pgfsetfillcolor{textcolor}%
\pgftext[x=4.865089in,y=2.437114in,left,base]{\color{textcolor}{\rmfamily\fontsize{12.000000}{14.400000}\selectfont\catcode`\^=\active\def^{\ifmmode\sp\else\^{}\fi}\catcode`\%=\active\def%{\%}\NaiveCycles{}}}%
\end{pgfscope}%
\begin{pgfscope}%
\pgfsetrectcap%
\pgfsetroundjoin%
\pgfsetlinewidth{1.505625pt}%
\pgfsetstrokecolor{currentstroke1}%
\pgfsetdash{}{0pt}%
\pgfpathmoveto{\pgfqpoint{4.398423in}{2.250818in}}%
\pgfpathlineto{\pgfqpoint{4.565089in}{2.250818in}}%
\pgfpathlineto{\pgfqpoint{4.731756in}{2.250818in}}%
\pgfusepath{stroke}%
\end{pgfscope}%
\begin{pgfscope}%
\definecolor{textcolor}{rgb}{0.000000,0.000000,0.000000}%
\pgfsetstrokecolor{textcolor}%
\pgfsetfillcolor{textcolor}%
\pgftext[x=4.865089in,y=2.192485in,left,base]{\color{textcolor}{\rmfamily\fontsize{12.000000}{14.400000}\selectfont\catcode`\^=\active\def^{\ifmmode\sp\else\^{}\fi}\catcode`\%=\active\def%{\%}\CyclesMatchChunks{} \& \MergeLinear{}}}%
\end{pgfscope}%
\begin{pgfscope}%
\pgfsetrectcap%
\pgfsetroundjoin%
\pgfsetlinewidth{1.505625pt}%
\pgfsetstrokecolor{currentstroke2}%
\pgfsetdash{}{0pt}%
\pgfpathmoveto{\pgfqpoint{4.398423in}{2.001551in}}%
\pgfpathlineto{\pgfqpoint{4.565089in}{2.001551in}}%
\pgfpathlineto{\pgfqpoint{4.731756in}{2.001551in}}%
\pgfusepath{stroke}%
\end{pgfscope}%
\begin{pgfscope}%
\definecolor{textcolor}{rgb}{0.000000,0.000000,0.000000}%
\pgfsetstrokecolor{textcolor}%
\pgfsetfillcolor{textcolor}%
\pgftext[x=4.865089in,y=1.943218in,left,base]{\color{textcolor}{\rmfamily\fontsize{12.000000}{14.400000}\selectfont\catcode`\^=\active\def^{\ifmmode\sp\else\^{}\fi}\catcode`\%=\active\def%{\%}\CyclesMatchChunks{} \& \SharedVertices{}}}%
\end{pgfscope}%
\begin{pgfscope}%
\pgfsetrectcap%
\pgfsetroundjoin%
\pgfsetlinewidth{1.505625pt}%
\pgfsetstrokecolor{currentstroke4}%
\pgfsetdash{}{0pt}%
\pgfpathmoveto{\pgfqpoint{4.398423in}{1.752284in}}%
\pgfpathlineto{\pgfqpoint{4.565089in}{1.752284in}}%
\pgfpathlineto{\pgfqpoint{4.731756in}{1.752284in}}%
\pgfusepath{stroke}%
\end{pgfscope}%
\begin{pgfscope}%
\definecolor{textcolor}{rgb}{0.000000,0.000000,0.000000}%
\pgfsetstrokecolor{textcolor}%
\pgfsetfillcolor{textcolor}%
\pgftext[x=4.865089in,y=1.693950in,left,base]{\color{textcolor}{\rmfamily\fontsize{12.000000}{14.400000}\selectfont\catcode`\^=\active\def^{\ifmmode\sp\else\^{}\fi}\catcode`\%=\active\def%{\%}\Neighbors{} \& \MergeLinear{}}}%
\end{pgfscope}%
\begin{pgfscope}%
\pgfsetrectcap%
\pgfsetroundjoin%
\pgfsetlinewidth{1.505625pt}%
\pgfsetstrokecolor{currentstroke5}%
\pgfsetdash{}{0pt}%
\pgfpathmoveto{\pgfqpoint{4.398423in}{1.507655in}}%
\pgfpathlineto{\pgfqpoint{4.565089in}{1.507655in}}%
\pgfpathlineto{\pgfqpoint{4.731756in}{1.507655in}}%
\pgfusepath{stroke}%
\end{pgfscope}%
\begin{pgfscope}%
\definecolor{textcolor}{rgb}{0.000000,0.000000,0.000000}%
\pgfsetstrokecolor{textcolor}%
\pgfsetfillcolor{textcolor}%
\pgftext[x=4.865089in,y=1.449322in,left,base]{\color{textcolor}{\rmfamily\fontsize{12.000000}{14.400000}\selectfont\catcode`\^=\active\def^{\ifmmode\sp\else\^{}\fi}\catcode`\%=\active\def%{\%}\Neighbors{} \& \SharedVertices{}}}%
\end{pgfscope}%
\begin{pgfscope}%
\pgfsetrectcap%
\pgfsetroundjoin%
\pgfsetlinewidth{1.505625pt}%
\pgfsetstrokecolor{currentstroke6}%
\pgfsetdash{}{0pt}%
\pgfpathmoveto{\pgfqpoint{4.398423in}{1.258387in}}%
\pgfpathlineto{\pgfqpoint{4.565089in}{1.258387in}}%
\pgfpathlineto{\pgfqpoint{4.731756in}{1.258387in}}%
\pgfusepath{stroke}%
\end{pgfscope}%
\begin{pgfscope}%
\definecolor{textcolor}{rgb}{0.000000,0.000000,0.000000}%
\pgfsetstrokecolor{textcolor}%
\pgfsetfillcolor{textcolor}%
\pgftext[x=4.865089in,y=1.200054in,left,base]{\color{textcolor}{\rmfamily\fontsize{12.000000}{14.400000}\selectfont\catcode`\^=\active\def^{\ifmmode\sp\else\^{}\fi}\catcode`\%=\active\def%{\%}\NeighborsDegree{} \& \MergeLinear{}}}%
\end{pgfscope}%
\begin{pgfscope}%
\pgfsetrectcap%
\pgfsetroundjoin%
\pgfsetlinewidth{1.505625pt}%
\pgfsetstrokecolor{currentstroke7}%
\pgfsetdash{}{0pt}%
\pgfpathmoveto{\pgfqpoint{4.398423in}{1.009120in}}%
\pgfpathlineto{\pgfqpoint{4.565089in}{1.009120in}}%
\pgfpathlineto{\pgfqpoint{4.731756in}{1.009120in}}%
\pgfusepath{stroke}%
\end{pgfscope}%
\begin{pgfscope}%
\definecolor{textcolor}{rgb}{0.000000,0.000000,0.000000}%
\pgfsetstrokecolor{textcolor}%
\pgfsetfillcolor{textcolor}%
\pgftext[x=4.865089in,y=0.950787in,left,base]{\color{textcolor}{\rmfamily\fontsize{12.000000}{14.400000}\selectfont\catcode`\^=\active\def^{\ifmmode\sp\else\^{}\fi}\catcode`\%=\active\def%{\%}\NeighborsDegree{} \& \SharedVertices{}}}%
\end{pgfscope}%
\begin{pgfscope}%
\pgfsetrectcap%
\pgfsetroundjoin%
\pgfsetlinewidth{1.505625pt}%
\definecolor{currentstroke}{rgb}{0.498039,0.498039,0.498039}%
\pgfsetstrokecolor{currentstroke}%
\pgfsetdash{}{0pt}%
\pgfpathmoveto{\pgfqpoint{4.398423in}{0.759853in}}%
\pgfpathlineto{\pgfqpoint{4.565089in}{0.759853in}}%
\pgfpathlineto{\pgfqpoint{4.731756in}{0.759853in}}%
\pgfusepath{stroke}%
\end{pgfscope}%
\begin{pgfscope}%
\definecolor{textcolor}{rgb}{0.000000,0.000000,0.000000}%
\pgfsetstrokecolor{textcolor}%
\pgfsetfillcolor{textcolor}%
\pgftext[x=4.865089in,y=0.701519in,left,base]{\color{textcolor}{\rmfamily\fontsize{12.000000}{14.400000}\selectfont\catcode`\^=\active\def^{\ifmmode\sp\else\^{}\fi}\catcode`\%=\active\def%{\%}\None{} \& \MergeLinear{}}}%
\end{pgfscope}%
\begin{pgfscope}%
\pgfsetrectcap%
\pgfsetroundjoin%
\pgfsetlinewidth{1.505625pt}%
\definecolor{currentstroke}{rgb}{0.737255,0.741176,0.133333}%
\pgfsetstrokecolor{currentstroke}%
\pgfsetdash{}{0pt}%
\pgfpathmoveto{\pgfqpoint{4.398423in}{0.515224in}}%
\pgfpathlineto{\pgfqpoint{4.565089in}{0.515224in}}%
\pgfpathlineto{\pgfqpoint{4.731756in}{0.515224in}}%
\pgfusepath{stroke}%
\end{pgfscope}%
\begin{pgfscope}%
\definecolor{textcolor}{rgb}{0.000000,0.000000,0.000000}%
\pgfsetstrokecolor{textcolor}%
\pgfsetfillcolor{textcolor}%
\pgftext[x=4.865089in,y=0.456891in,left,base]{\color{textcolor}{\rmfamily\fontsize{12.000000}{14.400000}\selectfont\catcode`\^=\active\def^{\ifmmode\sp\else\^{}\fi}\catcode`\%=\active\def%{\%}\None{} \& \SharedVertices{}}}%
\end{pgfscope}%
\end{pgfpicture}%
\makeatother%
\endgroup%
}
	\caption[Mean runtime for globally rigid graphs (all)]{
		Mean running time to find all NAC-colorings for globally rigid graphs.}%
	\label{fig:graph_globally_rigid_all_runtime}
\end{figure}%
\begin{figure}[thbp]
	\centering
	\scalebox{\BenchFigureScale}{%% Creator: Matplotlib, PGF backend
%%
%% To include the figure in your LaTeX document, write
%%   \input{<filename>.pgf}
%%
%% Make sure the required packages are loaded in your preamble
%%   \usepackage{pgf}
%%
%% Also ensure that all the required font packages are loaded; for instance,
%% the lmodern package is sometimes necessary when using math font.
%%   \usepackage{lmodern}
%%
%% Figures using additional raster images can only be included by \input if
%% they are in the same directory as the main LaTeX file. For loading figures
%% from other directories you can use the `import` package
%%   \usepackage{import}
%%
%% and then include the figures with
%%   \import{<path to file>}{<filename>.pgf}
%%
%% Matplotlib used the following preamble
%%   \def\mathdefault#1{#1}
%%   \everymath=\expandafter{\the\everymath\displaystyle}
%%   \IfFileExists{scrextend.sty}{
%%     \usepackage[fontsize=10.000000pt]{scrextend}
%%   }{
%%     \renewcommand{\normalsize}{\fontsize{10.000000}{12.000000}\selectfont}
%%     \normalsize
%%   }
%%   
%%   \ifdefined\pdftexversion\else  % non-pdftex case.
%%     \usepackage{fontspec}
%%     \setmainfont{DejaVuSans.ttf}[Path=\detokenize{/home/petr/Projects/PyRigi/.venv/lib/python3.12/site-packages/matplotlib/mpl-data/fonts/ttf/}]
%%     \setsansfont{DejaVuSans.ttf}[Path=\detokenize{/home/petr/Projects/PyRigi/.venv/lib/python3.12/site-packages/matplotlib/mpl-data/fonts/ttf/}]
%%     \setmonofont{DejaVuSansMono.ttf}[Path=\detokenize{/home/petr/Projects/PyRigi/.venv/lib/python3.12/site-packages/matplotlib/mpl-data/fonts/ttf/}]
%%   \fi
%%   \makeatletter\@ifpackageloaded{under\Score{}}{}{\usepackage[strings]{under\Score{}}}\makeatother
%%
\begingroup%
\makeatletter%
\begin{pgfpicture}%
\pgfpathrectangle{\pgfpointorigin}{\pgfqpoint{8.384376in}{2.841849in}}%
\pgfusepath{use as bounding box, clip}%
\begin{pgfscope}%
\pgfsetbuttcap%
\pgfsetmiterjoin%
\definecolor{currentfill}{rgb}{1.000000,1.000000,1.000000}%
\pgfsetfillcolor{currentfill}%
\pgfsetlinewidth{0.000000pt}%
\definecolor{currentstroke}{rgb}{1.000000,1.000000,1.000000}%
\pgfsetstrokecolor{currentstroke}%
\pgfsetdash{}{0pt}%
\pgfpathmoveto{\pgfqpoint{0.000000in}{0.000000in}}%
\pgfpathlineto{\pgfqpoint{8.384376in}{0.000000in}}%
\pgfpathlineto{\pgfqpoint{8.384376in}{2.841849in}}%
\pgfpathlineto{\pgfqpoint{0.000000in}{2.841849in}}%
\pgfpathlineto{\pgfqpoint{0.000000in}{0.000000in}}%
\pgfpathclose%
\pgfusepath{fill}%
\end{pgfscope}%
\begin{pgfscope}%
\pgfsetbuttcap%
\pgfsetmiterjoin%
\definecolor{currentfill}{rgb}{1.000000,1.000000,1.000000}%
\pgfsetfillcolor{currentfill}%
\pgfsetlinewidth{0.000000pt}%
\definecolor{currentstroke}{rgb}{0.000000,0.000000,0.000000}%
\pgfsetstrokecolor{currentstroke}%
\pgfsetstrokeopacity{0.000000}%
\pgfsetdash{}{0pt}%
\pgfpathmoveto{\pgfqpoint{0.588387in}{0.521603in}}%
\pgfpathlineto{\pgfqpoint{5.257411in}{0.521603in}}%
\pgfpathlineto{\pgfqpoint{5.257411in}{2.741849in}}%
\pgfpathlineto{\pgfqpoint{0.588387in}{2.741849in}}%
\pgfpathlineto{\pgfqpoint{0.588387in}{0.521603in}}%
\pgfpathclose%
\pgfusepath{fill}%
\end{pgfscope}%
\begin{pgfscope}%
\pgfsetbuttcap%
\pgfsetroundjoin%
\definecolor{currentfill}{rgb}{0.000000,0.000000,0.000000}%
\pgfsetfillcolor{currentfill}%
\pgfsetlinewidth{0.803000pt}%
\definecolor{currentstroke}{rgb}{0.000000,0.000000,0.000000}%
\pgfsetstrokecolor{currentstroke}%
\pgfsetdash{}{0pt}%
\pgfsys@defobject{currentmarker}{\pgfqpoint{0.000000in}{-0.048611in}}{\pgfqpoint{0.000000in}{0.000000in}}{%
\pgfpathmoveto{\pgfqpoint{0.000000in}{0.000000in}}%
\pgfpathlineto{\pgfqpoint{0.000000in}{-0.048611in}}%
\pgfusepath{stroke,fill}%
}%
\begin{pgfscope}%
\pgfsys@transformshift{1.093344in}{0.521603in}%
\pgfsys@useobject{currentmarker}{}%
\end{pgfscope}%
\end{pgfscope}%
\begin{pgfscope}%
\definecolor{textcolor}{rgb}{0.000000,0.000000,0.000000}%
\pgfsetstrokecolor{textcolor}%
\pgfsetfillcolor{textcolor}%
\pgftext[x=1.093344in,y=0.424381in,,top]{\color{textcolor}{\rmfamily\fontsize{10.000000}{12.000000}\selectfont\catcode`\^=\active\def^{\ifmmode\sp\else\^{}\fi}\catcode`\%=\active\def%{\%}$\mathdefault{4}$}}%
\end{pgfscope}%
\begin{pgfscope}%
\pgfsetbuttcap%
\pgfsetroundjoin%
\definecolor{currentfill}{rgb}{0.000000,0.000000,0.000000}%
\pgfsetfillcolor{currentfill}%
\pgfsetlinewidth{0.803000pt}%
\definecolor{currentstroke}{rgb}{0.000000,0.000000,0.000000}%
\pgfsetstrokecolor{currentstroke}%
\pgfsetdash{}{0pt}%
\pgfsys@defobject{currentmarker}{\pgfqpoint{0.000000in}{-0.048611in}}{\pgfqpoint{0.000000in}{0.000000in}}{%
\pgfpathmoveto{\pgfqpoint{0.000000in}{0.000000in}}%
\pgfpathlineto{\pgfqpoint{0.000000in}{-0.048611in}}%
\pgfusepath{stroke,fill}%
}%
\begin{pgfscope}%
\pgfsys@transformshift{1.678802in}{0.521603in}%
\pgfsys@useobject{currentmarker}{}%
\end{pgfscope}%
\end{pgfscope}%
\begin{pgfscope}%
\definecolor{textcolor}{rgb}{0.000000,0.000000,0.000000}%
\pgfsetstrokecolor{textcolor}%
\pgfsetfillcolor{textcolor}%
\pgftext[x=1.678802in,y=0.424381in,,top]{\color{textcolor}{\rmfamily\fontsize{10.000000}{12.000000}\selectfont\catcode`\^=\active\def^{\ifmmode\sp\else\^{}\fi}\catcode`\%=\active\def%{\%}$\mathdefault{8}$}}%
\end{pgfscope}%
\begin{pgfscope}%
\pgfsetbuttcap%
\pgfsetroundjoin%
\definecolor{currentfill}{rgb}{0.000000,0.000000,0.000000}%
\pgfsetfillcolor{currentfill}%
\pgfsetlinewidth{0.803000pt}%
\definecolor{currentstroke}{rgb}{0.000000,0.000000,0.000000}%
\pgfsetstrokecolor{currentstroke}%
\pgfsetdash{}{0pt}%
\pgfsys@defobject{currentmarker}{\pgfqpoint{0.000000in}{-0.048611in}}{\pgfqpoint{0.000000in}{0.000000in}}{%
\pgfpathmoveto{\pgfqpoint{0.000000in}{0.000000in}}%
\pgfpathlineto{\pgfqpoint{0.000000in}{-0.048611in}}%
\pgfusepath{stroke,fill}%
}%
\begin{pgfscope}%
\pgfsys@transformshift{2.264259in}{0.521603in}%
\pgfsys@useobject{currentmarker}{}%
\end{pgfscope}%
\end{pgfscope}%
\begin{pgfscope}%
\definecolor{textcolor}{rgb}{0.000000,0.000000,0.000000}%
\pgfsetstrokecolor{textcolor}%
\pgfsetfillcolor{textcolor}%
\pgftext[x=2.264259in,y=0.424381in,,top]{\color{textcolor}{\rmfamily\fontsize{10.000000}{12.000000}\selectfont\catcode`\^=\active\def^{\ifmmode\sp\else\^{}\fi}\catcode`\%=\active\def%{\%}$\mathdefault{12}$}}%
\end{pgfscope}%
\begin{pgfscope}%
\pgfsetbuttcap%
\pgfsetroundjoin%
\definecolor{currentfill}{rgb}{0.000000,0.000000,0.000000}%
\pgfsetfillcolor{currentfill}%
\pgfsetlinewidth{0.803000pt}%
\definecolor{currentstroke}{rgb}{0.000000,0.000000,0.000000}%
\pgfsetstrokecolor{currentstroke}%
\pgfsetdash{}{0pt}%
\pgfsys@defobject{currentmarker}{\pgfqpoint{0.000000in}{-0.048611in}}{\pgfqpoint{0.000000in}{0.000000in}}{%
\pgfpathmoveto{\pgfqpoint{0.000000in}{0.000000in}}%
\pgfpathlineto{\pgfqpoint{0.000000in}{-0.048611in}}%
\pgfusepath{stroke,fill}%
}%
\begin{pgfscope}%
\pgfsys@transformshift{2.849717in}{0.521603in}%
\pgfsys@useobject{currentmarker}{}%
\end{pgfscope}%
\end{pgfscope}%
\begin{pgfscope}%
\definecolor{textcolor}{rgb}{0.000000,0.000000,0.000000}%
\pgfsetstrokecolor{textcolor}%
\pgfsetfillcolor{textcolor}%
\pgftext[x=2.849717in,y=0.424381in,,top]{\color{textcolor}{\rmfamily\fontsize{10.000000}{12.000000}\selectfont\catcode`\^=\active\def^{\ifmmode\sp\else\^{}\fi}\catcode`\%=\active\def%{\%}$\mathdefault{16}$}}%
\end{pgfscope}%
\begin{pgfscope}%
\pgfsetbuttcap%
\pgfsetroundjoin%
\definecolor{currentfill}{rgb}{0.000000,0.000000,0.000000}%
\pgfsetfillcolor{currentfill}%
\pgfsetlinewidth{0.803000pt}%
\definecolor{currentstroke}{rgb}{0.000000,0.000000,0.000000}%
\pgfsetstrokecolor{currentstroke}%
\pgfsetdash{}{0pt}%
\pgfsys@defobject{currentmarker}{\pgfqpoint{0.000000in}{-0.048611in}}{\pgfqpoint{0.000000in}{0.000000in}}{%
\pgfpathmoveto{\pgfqpoint{0.000000in}{0.000000in}}%
\pgfpathlineto{\pgfqpoint{0.000000in}{-0.048611in}}%
\pgfusepath{stroke,fill}%
}%
\begin{pgfscope}%
\pgfsys@transformshift{3.435175in}{0.521603in}%
\pgfsys@useobject{currentmarker}{}%
\end{pgfscope}%
\end{pgfscope}%
\begin{pgfscope}%
\definecolor{textcolor}{rgb}{0.000000,0.000000,0.000000}%
\pgfsetstrokecolor{textcolor}%
\pgfsetfillcolor{textcolor}%
\pgftext[x=3.435175in,y=0.424381in,,top]{\color{textcolor}{\rmfamily\fontsize{10.000000}{12.000000}\selectfont\catcode`\^=\active\def^{\ifmmode\sp\else\^{}\fi}\catcode`\%=\active\def%{\%}$\mathdefault{20}$}}%
\end{pgfscope}%
\begin{pgfscope}%
\pgfsetbuttcap%
\pgfsetroundjoin%
\definecolor{currentfill}{rgb}{0.000000,0.000000,0.000000}%
\pgfsetfillcolor{currentfill}%
\pgfsetlinewidth{0.803000pt}%
\definecolor{currentstroke}{rgb}{0.000000,0.000000,0.000000}%
\pgfsetstrokecolor{currentstroke}%
\pgfsetdash{}{0pt}%
\pgfsys@defobject{currentmarker}{\pgfqpoint{0.000000in}{-0.048611in}}{\pgfqpoint{0.000000in}{0.000000in}}{%
\pgfpathmoveto{\pgfqpoint{0.000000in}{0.000000in}}%
\pgfpathlineto{\pgfqpoint{0.000000in}{-0.048611in}}%
\pgfusepath{stroke,fill}%
}%
\begin{pgfscope}%
\pgfsys@transformshift{4.020632in}{0.521603in}%
\pgfsys@useobject{currentmarker}{}%
\end{pgfscope}%
\end{pgfscope}%
\begin{pgfscope}%
\definecolor{textcolor}{rgb}{0.000000,0.000000,0.000000}%
\pgfsetstrokecolor{textcolor}%
\pgfsetfillcolor{textcolor}%
\pgftext[x=4.020632in,y=0.424381in,,top]{\color{textcolor}{\rmfamily\fontsize{10.000000}{12.000000}\selectfont\catcode`\^=\active\def^{\ifmmode\sp\else\^{}\fi}\catcode`\%=\active\def%{\%}$\mathdefault{24}$}}%
\end{pgfscope}%
\begin{pgfscope}%
\pgfsetbuttcap%
\pgfsetroundjoin%
\definecolor{currentfill}{rgb}{0.000000,0.000000,0.000000}%
\pgfsetfillcolor{currentfill}%
\pgfsetlinewidth{0.803000pt}%
\definecolor{currentstroke}{rgb}{0.000000,0.000000,0.000000}%
\pgfsetstrokecolor{currentstroke}%
\pgfsetdash{}{0pt}%
\pgfsys@defobject{currentmarker}{\pgfqpoint{0.000000in}{-0.048611in}}{\pgfqpoint{0.000000in}{0.000000in}}{%
\pgfpathmoveto{\pgfqpoint{0.000000in}{0.000000in}}%
\pgfpathlineto{\pgfqpoint{0.000000in}{-0.048611in}}%
\pgfusepath{stroke,fill}%
}%
\begin{pgfscope}%
\pgfsys@transformshift{4.606090in}{0.521603in}%
\pgfsys@useobject{currentmarker}{}%
\end{pgfscope}%
\end{pgfscope}%
\begin{pgfscope}%
\definecolor{textcolor}{rgb}{0.000000,0.000000,0.000000}%
\pgfsetstrokecolor{textcolor}%
\pgfsetfillcolor{textcolor}%
\pgftext[x=4.606090in,y=0.424381in,,top]{\color{textcolor}{\rmfamily\fontsize{10.000000}{12.000000}\selectfont\catcode`\^=\active\def^{\ifmmode\sp\else\^{}\fi}\catcode`\%=\active\def%{\%}$\mathdefault{28}$}}%
\end{pgfscope}%
\begin{pgfscope}%
\pgfsetbuttcap%
\pgfsetroundjoin%
\definecolor{currentfill}{rgb}{0.000000,0.000000,0.000000}%
\pgfsetfillcolor{currentfill}%
\pgfsetlinewidth{0.803000pt}%
\definecolor{currentstroke}{rgb}{0.000000,0.000000,0.000000}%
\pgfsetstrokecolor{currentstroke}%
\pgfsetdash{}{0pt}%
\pgfsys@defobject{currentmarker}{\pgfqpoint{0.000000in}{-0.048611in}}{\pgfqpoint{0.000000in}{0.000000in}}{%
\pgfpathmoveto{\pgfqpoint{0.000000in}{0.000000in}}%
\pgfpathlineto{\pgfqpoint{0.000000in}{-0.048611in}}%
\pgfusepath{stroke,fill}%
}%
\begin{pgfscope}%
\pgfsys@transformshift{5.191547in}{0.521603in}%
\pgfsys@useobject{currentmarker}{}%
\end{pgfscope}%
\end{pgfscope}%
\begin{pgfscope}%
\definecolor{textcolor}{rgb}{0.000000,0.000000,0.000000}%
\pgfsetstrokecolor{textcolor}%
\pgfsetfillcolor{textcolor}%
\pgftext[x=5.191547in,y=0.424381in,,top]{\color{textcolor}{\rmfamily\fontsize{10.000000}{12.000000}\selectfont\catcode`\^=\active\def^{\ifmmode\sp\else\^{}\fi}\catcode`\%=\active\def%{\%}$\mathdefault{32}$}}%
\end{pgfscope}%
\begin{pgfscope}%
\definecolor{textcolor}{rgb}{0.000000,0.000000,0.000000}%
\pgfsetstrokecolor{textcolor}%
\pgfsetfillcolor{textcolor}%
\pgftext[x=2.922899in,y=0.234413in,,top]{\color{textcolor}{\rmfamily\fontsize{10.000000}{12.000000}\selectfont\catcode`\^=\active\def^{\ifmmode\sp\else\^{}\fi}\catcode`\%=\active\def%{\%}Monochromatic classes}}%
\end{pgfscope}%
\begin{pgfscope}%
\pgfsetbuttcap%
\pgfsetroundjoin%
\definecolor{currentfill}{rgb}{0.000000,0.000000,0.000000}%
\pgfsetfillcolor{currentfill}%
\pgfsetlinewidth{0.803000pt}%
\definecolor{currentstroke}{rgb}{0.000000,0.000000,0.000000}%
\pgfsetstrokecolor{currentstroke}%
\pgfsetdash{}{0pt}%
\pgfsys@defobject{currentmarker}{\pgfqpoint{-0.048611in}{0.000000in}}{\pgfqpoint{-0.000000in}{0.000000in}}{%
\pgfpathmoveto{\pgfqpoint{-0.000000in}{0.000000in}}%
\pgfpathlineto{\pgfqpoint{-0.048611in}{0.000000in}}%
\pgfusepath{stroke,fill}%
}%
\begin{pgfscope}%
\pgfsys@transformshift{0.588387in}{0.914045in}%
\pgfsys@useobject{currentmarker}{}%
\end{pgfscope}%
\end{pgfscope}%
\begin{pgfscope}%
\definecolor{textcolor}{rgb}{0.000000,0.000000,0.000000}%
\pgfsetstrokecolor{textcolor}%
\pgfsetfillcolor{textcolor}%
\pgftext[x=0.289968in, y=0.861284in, left, base]{\color{textcolor}{\rmfamily\fontsize{10.000000}{12.000000}\selectfont\catcode`\^=\active\def^{\ifmmode\sp\else\^{}\fi}\catcode`\%=\active\def%{\%}$\mathdefault{10^{1}}$}}%
\end{pgfscope}%
\begin{pgfscope}%
\pgfsetbuttcap%
\pgfsetroundjoin%
\definecolor{currentfill}{rgb}{0.000000,0.000000,0.000000}%
\pgfsetfillcolor{currentfill}%
\pgfsetlinewidth{0.803000pt}%
\definecolor{currentstroke}{rgb}{0.000000,0.000000,0.000000}%
\pgfsetstrokecolor{currentstroke}%
\pgfsetdash{}{0pt}%
\pgfsys@defobject{currentmarker}{\pgfqpoint{-0.048611in}{0.000000in}}{\pgfqpoint{-0.000000in}{0.000000in}}{%
\pgfpathmoveto{\pgfqpoint{-0.000000in}{0.000000in}}%
\pgfpathlineto{\pgfqpoint{-0.048611in}{0.000000in}}%
\pgfusepath{stroke,fill}%
}%
\begin{pgfscope}%
\pgfsys@transformshift{0.588387in}{1.497088in}%
\pgfsys@useobject{currentmarker}{}%
\end{pgfscope}%
\end{pgfscope}%
\begin{pgfscope}%
\definecolor{textcolor}{rgb}{0.000000,0.000000,0.000000}%
\pgfsetstrokecolor{textcolor}%
\pgfsetfillcolor{textcolor}%
\pgftext[x=0.289968in, y=1.444327in, left, base]{\color{textcolor}{\rmfamily\fontsize{10.000000}{12.000000}\selectfont\catcode`\^=\active\def^{\ifmmode\sp\else\^{}\fi}\catcode`\%=\active\def%{\%}$\mathdefault{10^{3}}$}}%
\end{pgfscope}%
\begin{pgfscope}%
\pgfsetbuttcap%
\pgfsetroundjoin%
\definecolor{currentfill}{rgb}{0.000000,0.000000,0.000000}%
\pgfsetfillcolor{currentfill}%
\pgfsetlinewidth{0.803000pt}%
\definecolor{currentstroke}{rgb}{0.000000,0.000000,0.000000}%
\pgfsetstrokecolor{currentstroke}%
\pgfsetdash{}{0pt}%
\pgfsys@defobject{currentmarker}{\pgfqpoint{-0.048611in}{0.000000in}}{\pgfqpoint{-0.000000in}{0.000000in}}{%
\pgfpathmoveto{\pgfqpoint{-0.000000in}{0.000000in}}%
\pgfpathlineto{\pgfqpoint{-0.048611in}{0.000000in}}%
\pgfusepath{stroke,fill}%
}%
\begin{pgfscope}%
\pgfsys@transformshift{0.588387in}{2.080132in}%
\pgfsys@useobject{currentmarker}{}%
\end{pgfscope}%
\end{pgfscope}%
\begin{pgfscope}%
\definecolor{textcolor}{rgb}{0.000000,0.000000,0.000000}%
\pgfsetstrokecolor{textcolor}%
\pgfsetfillcolor{textcolor}%
\pgftext[x=0.289968in, y=2.027370in, left, base]{\color{textcolor}{\rmfamily\fontsize{10.000000}{12.000000}\selectfont\catcode`\^=\active\def^{\ifmmode\sp\else\^{}\fi}\catcode`\%=\active\def%{\%}$\mathdefault{10^{5}}$}}%
\end{pgfscope}%
\begin{pgfscope}%
\pgfsetbuttcap%
\pgfsetroundjoin%
\definecolor{currentfill}{rgb}{0.000000,0.000000,0.000000}%
\pgfsetfillcolor{currentfill}%
\pgfsetlinewidth{0.803000pt}%
\definecolor{currentstroke}{rgb}{0.000000,0.000000,0.000000}%
\pgfsetstrokecolor{currentstroke}%
\pgfsetdash{}{0pt}%
\pgfsys@defobject{currentmarker}{\pgfqpoint{-0.048611in}{0.000000in}}{\pgfqpoint{-0.000000in}{0.000000in}}{%
\pgfpathmoveto{\pgfqpoint{-0.000000in}{0.000000in}}%
\pgfpathlineto{\pgfqpoint{-0.048611in}{0.000000in}}%
\pgfusepath{stroke,fill}%
}%
\begin{pgfscope}%
\pgfsys@transformshift{0.588387in}{2.663175in}%
\pgfsys@useobject{currentmarker}{}%
\end{pgfscope}%
\end{pgfscope}%
\begin{pgfscope}%
\definecolor{textcolor}{rgb}{0.000000,0.000000,0.000000}%
\pgfsetstrokecolor{textcolor}%
\pgfsetfillcolor{textcolor}%
\pgftext[x=0.289968in, y=2.610413in, left, base]{\color{textcolor}{\rmfamily\fontsize{10.000000}{12.000000}\selectfont\catcode`\^=\active\def^{\ifmmode\sp\else\^{}\fi}\catcode`\%=\active\def%{\%}$\mathdefault{10^{7}}$}}%
\end{pgfscope}%
\begin{pgfscope}%
\definecolor{textcolor}{rgb}{0.000000,0.000000,0.000000}%
\pgfsetstrokecolor{textcolor}%
\pgfsetfillcolor{textcolor}%
\pgftext[x=0.234413in,y=1.631726in,,bottom,rotate=90.000000]{\color{textcolor}{\rmfamily\fontsize{10.000000}{12.000000}\selectfont\catcode`\^=\active\def^{\ifmmode\sp\else\^{}\fi}\catcode`\%=\active\def%{\%}Checks [call]}}%
\end{pgfscope}%
\begin{pgfscope}%
\pgfpathrectangle{\pgfqpoint{0.588387in}{0.521603in}}{\pgfqpoint{4.669024in}{2.220246in}}%
\pgfusepath{clip}%
\pgfsetrectcap%
\pgfsetroundjoin%
\pgfsetlinewidth{1.505625pt}%
\pgfsetstrokecolor{currentstroke1}%
\pgfsetdash{}{0pt}%
\pgfpathmoveto{\pgfqpoint{0.800616in}{0.710280in}}%
\pgfpathlineto{\pgfqpoint{0.946980in}{0.798037in}}%
\pgfpathlineto{\pgfqpoint{1.093344in}{0.885794in}}%
\pgfpathlineto{\pgfqpoint{1.239709in}{0.973551in}}%
\pgfpathlineto{\pgfqpoint{1.386073in}{1.061307in}}%
\pgfpathlineto{\pgfqpoint{1.532438in}{1.149064in}}%
\pgfpathlineto{\pgfqpoint{1.678802in}{1.223639in}}%
\pgfpathlineto{\pgfqpoint{1.825166in}{1.295659in}}%
\pgfpathlineto{\pgfqpoint{1.971531in}{1.377301in}}%
\pgfpathlineto{\pgfqpoint{2.117895in}{1.457095in}}%
\pgfpathlineto{\pgfqpoint{2.264259in}{1.460454in}}%
\pgfpathlineto{\pgfqpoint{2.410624in}{1.506832in}}%
\pgfpathlineto{\pgfqpoint{2.556988in}{1.572446in}}%
\pgfpathlineto{\pgfqpoint{2.703353in}{1.647992in}}%
\pgfpathlineto{\pgfqpoint{2.849717in}{1.675006in}}%
\pgfpathlineto{\pgfqpoint{2.996081in}{1.759316in}}%
\pgfpathlineto{\pgfqpoint{3.142446in}{1.766920in}}%
\pgfpathlineto{\pgfqpoint{3.288810in}{1.758577in}}%
\pgfpathlineto{\pgfqpoint{3.435175in}{1.841024in}}%
\pgfpathlineto{\pgfqpoint{3.581539in}{1.894385in}}%
\pgfpathlineto{\pgfqpoint{3.727903in}{1.868614in}}%
\pgfpathlineto{\pgfqpoint{3.874268in}{1.836314in}}%
\pgfpathlineto{\pgfqpoint{4.020632in}{1.831607in}}%
\pgfpathlineto{\pgfqpoint{4.166997in}{1.735156in}}%
\pgfpathlineto{\pgfqpoint{4.313361in}{1.772541in}}%
\pgfpathlineto{\pgfqpoint{4.459725in}{1.909992in}}%
\pgfpathlineto{\pgfqpoint{4.898818in}{2.030715in}}%
\pgfpathlineto{\pgfqpoint{5.045183in}{2.182277in}}%
\pgfusepath{stroke}%
\end{pgfscope}%
\begin{pgfscope}%
\pgfpathrectangle{\pgfqpoint{0.588387in}{0.521603in}}{\pgfqpoint{4.669024in}{2.220246in}}%
\pgfusepath{clip}%
\pgfsetrectcap%
\pgfsetroundjoin%
\pgfsetlinewidth{1.505625pt}%
\pgfsetstrokecolor{currentstroke2}%
\pgfsetdash{}{0pt}%
\pgfpathmoveto{\pgfqpoint{0.800616in}{0.710280in}}%
\pgfpathlineto{\pgfqpoint{0.946980in}{0.798037in}}%
\pgfpathlineto{\pgfqpoint{1.093344in}{0.885794in}}%
\pgfpathlineto{\pgfqpoint{1.239709in}{0.973551in}}%
\pgfpathlineto{\pgfqpoint{1.386073in}{1.061307in}}%
\pgfpathlineto{\pgfqpoint{1.532438in}{1.149064in}}%
\pgfpathlineto{\pgfqpoint{1.678802in}{1.223843in}}%
\pgfpathlineto{\pgfqpoint{1.825166in}{1.295659in}}%
\pgfpathlineto{\pgfqpoint{1.971531in}{1.377301in}}%
\pgfpathlineto{\pgfqpoint{2.117895in}{1.457095in}}%
\pgfpathlineto{\pgfqpoint{2.264259in}{1.453537in}}%
\pgfpathlineto{\pgfqpoint{2.410624in}{1.488715in}}%
\pgfpathlineto{\pgfqpoint{2.556988in}{1.577533in}}%
\pgfpathlineto{\pgfqpoint{2.703353in}{1.635477in}}%
\pgfpathlineto{\pgfqpoint{2.849717in}{1.670399in}}%
\pgfpathlineto{\pgfqpoint{2.996081in}{1.751701in}}%
\pgfpathlineto{\pgfqpoint{3.142446in}{1.689274in}}%
\pgfpathlineto{\pgfqpoint{3.288810in}{1.732577in}}%
\pgfpathlineto{\pgfqpoint{3.435175in}{1.780139in}}%
\pgfpathlineto{\pgfqpoint{3.581539in}{1.807401in}}%
\pgfpathlineto{\pgfqpoint{3.727903in}{1.815087in}}%
\pgfpathlineto{\pgfqpoint{3.874268in}{1.773114in}}%
\pgfpathlineto{\pgfqpoint{4.020632in}{1.870846in}}%
\pgfpathlineto{\pgfqpoint{4.166997in}{1.700955in}}%
\pgfpathlineto{\pgfqpoint{4.313361in}{1.848495in}}%
\pgfpathlineto{\pgfqpoint{4.459725in}{1.943533in}}%
\pgfpathlineto{\pgfqpoint{4.898818in}{1.960411in}}%
\pgfpathlineto{\pgfqpoint{5.045183in}{2.153759in}}%
\pgfusepath{stroke}%
\end{pgfscope}%
\begin{pgfscope}%
\pgfpathrectangle{\pgfqpoint{0.588387in}{0.521603in}}{\pgfqpoint{4.669024in}{2.220246in}}%
\pgfusepath{clip}%
\pgfsetrectcap%
\pgfsetroundjoin%
\pgfsetlinewidth{1.505625pt}%
\pgfsetstrokecolor{currentstroke3}%
\pgfsetdash{}{0pt}%
\pgfpathmoveto{\pgfqpoint{0.800616in}{0.622524in}}%
\pgfpathlineto{\pgfqpoint{0.946980in}{0.761615in}}%
\pgfpathlineto{\pgfqpoint{1.093344in}{0.868888in}}%
\pgfpathlineto{\pgfqpoint{1.239709in}{0.965380in}}%
\pgfpathlineto{\pgfqpoint{1.386073in}{1.057288in}}%
\pgfpathlineto{\pgfqpoint{1.532438in}{1.147070in}}%
\pgfpathlineto{\pgfqpoint{1.678802in}{1.235828in}}%
\pgfpathlineto{\pgfqpoint{1.825166in}{1.324082in}}%
\pgfpathlineto{\pgfqpoint{1.971531in}{1.412087in}}%
\pgfpathlineto{\pgfqpoint{2.117895in}{1.499967in}}%
\pgfpathlineto{\pgfqpoint{2.264259in}{1.587786in}}%
\pgfpathlineto{\pgfqpoint{2.410624in}{1.675574in}}%
\pgfpathlineto{\pgfqpoint{2.556988in}{1.763346in}}%
\pgfpathlineto{\pgfqpoint{2.703353in}{1.851110in}}%
\pgfpathlineto{\pgfqpoint{2.849717in}{1.938871in}}%
\pgfpathlineto{\pgfqpoint{2.996081in}{2.026630in}}%
\pgfpathlineto{\pgfqpoint{3.142446in}{2.114387in}}%
\pgfpathlineto{\pgfqpoint{3.288810in}{2.202145in}}%
\pgfpathlineto{\pgfqpoint{3.435175in}{2.289902in}}%
\pgfpathlineto{\pgfqpoint{3.581539in}{2.377659in}}%
\pgfpathlineto{\pgfqpoint{3.727903in}{2.465415in}}%
\pgfpathlineto{\pgfqpoint{4.020632in}{2.640929in}}%
\pgfusepath{stroke}%
\end{pgfscope}%
\begin{pgfscope}%
\pgfpathrectangle{\pgfqpoint{0.588387in}{0.521603in}}{\pgfqpoint{4.669024in}{2.220246in}}%
\pgfusepath{clip}%
\pgfsetrectcap%
\pgfsetroundjoin%
\pgfsetlinewidth{1.505625pt}%
\pgfsetstrokecolor{currentstroke4}%
\pgfsetdash{}{0pt}%
\pgfpathmoveto{\pgfqpoint{0.800616in}{0.710280in}}%
\pgfpathlineto{\pgfqpoint{0.946980in}{0.798037in}}%
\pgfpathlineto{\pgfqpoint{1.093344in}{0.885794in}}%
\pgfpathlineto{\pgfqpoint{1.239709in}{0.973551in}}%
\pgfpathlineto{\pgfqpoint{1.386073in}{1.061307in}}%
\pgfpathlineto{\pgfqpoint{1.532438in}{1.149064in}}%
\pgfpathlineto{\pgfqpoint{1.678802in}{1.217703in}}%
\pgfpathlineto{\pgfqpoint{1.825166in}{1.296656in}}%
\pgfpathlineto{\pgfqpoint{1.971531in}{1.377166in}}%
\pgfpathlineto{\pgfqpoint{2.117895in}{1.457095in}}%
\pgfpathlineto{\pgfqpoint{2.264259in}{1.423687in}}%
\pgfpathlineto{\pgfqpoint{2.410624in}{1.498724in}}%
\pgfpathlineto{\pgfqpoint{2.556988in}{1.542279in}}%
\pgfpathlineto{\pgfqpoint{2.703353in}{1.613501in}}%
\pgfpathlineto{\pgfqpoint{2.849717in}{1.603600in}}%
\pgfpathlineto{\pgfqpoint{2.996081in}{1.706538in}}%
\pgfpathlineto{\pgfqpoint{3.142446in}{1.683280in}}%
\pgfpathlineto{\pgfqpoint{3.288810in}{1.749513in}}%
\pgfpathlineto{\pgfqpoint{3.435175in}{1.795134in}}%
\pgfpathlineto{\pgfqpoint{3.581539in}{1.792333in}}%
\pgfpathlineto{\pgfqpoint{3.727903in}{1.876304in}}%
\pgfpathlineto{\pgfqpoint{3.874268in}{1.696421in}}%
\pgfpathlineto{\pgfqpoint{4.020632in}{1.663142in}}%
\pgfpathlineto{\pgfqpoint{4.166997in}{1.774650in}}%
\pgfpathlineto{\pgfqpoint{4.313361in}{1.711718in}}%
\pgfpathlineto{\pgfqpoint{4.459725in}{1.943659in}}%
\pgfpathlineto{\pgfqpoint{4.898818in}{1.846298in}}%
\pgfpathlineto{\pgfqpoint{5.045183in}{1.798822in}}%
\pgfusepath{stroke}%
\end{pgfscope}%
\begin{pgfscope}%
\pgfpathrectangle{\pgfqpoint{0.588387in}{0.521603in}}{\pgfqpoint{4.669024in}{2.220246in}}%
\pgfusepath{clip}%
\pgfsetrectcap%
\pgfsetroundjoin%
\pgfsetlinewidth{1.505625pt}%
\pgfsetstrokecolor{currentstroke5}%
\pgfsetdash{}{0pt}%
\pgfpathmoveto{\pgfqpoint{0.800616in}{0.710280in}}%
\pgfpathlineto{\pgfqpoint{0.946980in}{0.798037in}}%
\pgfpathlineto{\pgfqpoint{1.093344in}{0.885794in}}%
\pgfpathlineto{\pgfqpoint{1.239709in}{0.973551in}}%
\pgfpathlineto{\pgfqpoint{1.386073in}{1.061307in}}%
\pgfpathlineto{\pgfqpoint{1.532438in}{1.149064in}}%
\pgfpathlineto{\pgfqpoint{1.678802in}{1.218011in}}%
\pgfpathlineto{\pgfqpoint{1.825166in}{1.295514in}}%
\pgfpathlineto{\pgfqpoint{1.971531in}{1.376550in}}%
\pgfpathlineto{\pgfqpoint{2.117895in}{1.457095in}}%
\pgfpathlineto{\pgfqpoint{2.264259in}{1.413000in}}%
\pgfpathlineto{\pgfqpoint{2.410624in}{1.494660in}}%
\pgfpathlineto{\pgfqpoint{2.556988in}{1.537693in}}%
\pgfpathlineto{\pgfqpoint{2.703353in}{1.610435in}}%
\pgfpathlineto{\pgfqpoint{2.849717in}{1.602172in}}%
\pgfpathlineto{\pgfqpoint{2.996081in}{1.704782in}}%
\pgfpathlineto{\pgfqpoint{3.142446in}{1.647554in}}%
\pgfpathlineto{\pgfqpoint{3.288810in}{1.683070in}}%
\pgfpathlineto{\pgfqpoint{3.435175in}{1.772368in}}%
\pgfpathlineto{\pgfqpoint{3.581539in}{1.783568in}}%
\pgfpathlineto{\pgfqpoint{3.727903in}{1.834027in}}%
\pgfpathlineto{\pgfqpoint{3.874268in}{1.697258in}}%
\pgfpathlineto{\pgfqpoint{4.020632in}{1.798453in}}%
\pgfpathlineto{\pgfqpoint{4.166997in}{1.822410in}}%
\pgfpathlineto{\pgfqpoint{4.313361in}{1.614507in}}%
\pgfpathlineto{\pgfqpoint{4.459725in}{1.878899in}}%
\pgfpathlineto{\pgfqpoint{4.898818in}{1.872223in}}%
\pgfpathlineto{\pgfqpoint{5.045183in}{1.821749in}}%
\pgfusepath{stroke}%
\end{pgfscope}%
\begin{pgfscope}%
\pgfpathrectangle{\pgfqpoint{0.588387in}{0.521603in}}{\pgfqpoint{4.669024in}{2.220246in}}%
\pgfusepath{clip}%
\pgfsetrectcap%
\pgfsetroundjoin%
\pgfsetlinewidth{1.505625pt}%
\pgfsetstrokecolor{currentstroke6}%
\pgfsetdash{}{0pt}%
\pgfpathmoveto{\pgfqpoint{0.800616in}{0.710280in}}%
\pgfpathlineto{\pgfqpoint{0.946980in}{0.798037in}}%
\pgfpathlineto{\pgfqpoint{1.093344in}{0.885794in}}%
\pgfpathlineto{\pgfqpoint{1.239709in}{0.973551in}}%
\pgfpathlineto{\pgfqpoint{1.386073in}{1.061307in}}%
\pgfpathlineto{\pgfqpoint{1.532438in}{1.149064in}}%
\pgfpathlineto{\pgfqpoint{1.678802in}{1.217462in}}%
\pgfpathlineto{\pgfqpoint{1.825166in}{1.294652in}}%
\pgfpathlineto{\pgfqpoint{1.971531in}{1.378194in}}%
\pgfpathlineto{\pgfqpoint{2.117895in}{1.458084in}}%
\pgfpathlineto{\pgfqpoint{2.264259in}{1.438431in}}%
\pgfpathlineto{\pgfqpoint{2.410624in}{1.467164in}}%
\pgfpathlineto{\pgfqpoint{2.556988in}{1.544878in}}%
\pgfpathlineto{\pgfqpoint{2.703353in}{1.618988in}}%
\pgfpathlineto{\pgfqpoint{2.849717in}{1.591664in}}%
\pgfpathlineto{\pgfqpoint{2.996081in}{1.693600in}}%
\pgfpathlineto{\pgfqpoint{3.142446in}{1.667763in}}%
\pgfpathlineto{\pgfqpoint{3.288810in}{1.719290in}}%
\pgfpathlineto{\pgfqpoint{3.435175in}{1.797507in}}%
\pgfpathlineto{\pgfqpoint{3.581539in}{1.845983in}}%
\pgfpathlineto{\pgfqpoint{3.727903in}{1.900584in}}%
\pgfpathlineto{\pgfqpoint{3.874268in}{1.747441in}}%
\pgfpathlineto{\pgfqpoint{4.020632in}{1.665641in}}%
\pgfpathlineto{\pgfqpoint{4.166997in}{1.814655in}}%
\pgfpathlineto{\pgfqpoint{4.313361in}{1.822410in}}%
\pgfpathlineto{\pgfqpoint{4.459725in}{1.880979in}}%
\pgfpathlineto{\pgfqpoint{4.898818in}{1.904436in}}%
\pgfpathlineto{\pgfqpoint{5.045183in}{1.917511in}}%
\pgfusepath{stroke}%
\end{pgfscope}%
\begin{pgfscope}%
\pgfpathrectangle{\pgfqpoint{0.588387in}{0.521603in}}{\pgfqpoint{4.669024in}{2.220246in}}%
\pgfusepath{clip}%
\pgfsetrectcap%
\pgfsetroundjoin%
\pgfsetlinewidth{1.505625pt}%
\pgfsetstrokecolor{currentstroke7}%
\pgfsetdash{}{0pt}%
\pgfpathmoveto{\pgfqpoint{0.800616in}{0.710280in}}%
\pgfpathlineto{\pgfqpoint{0.946980in}{0.798037in}}%
\pgfpathlineto{\pgfqpoint{1.093344in}{0.885794in}}%
\pgfpathlineto{\pgfqpoint{1.239709in}{0.973551in}}%
\pgfpathlineto{\pgfqpoint{1.386073in}{1.061307in}}%
\pgfpathlineto{\pgfqpoint{1.532438in}{1.149064in}}%
\pgfpathlineto{\pgfqpoint{1.678802in}{1.217144in}}%
\pgfpathlineto{\pgfqpoint{1.825166in}{1.295253in}}%
\pgfpathlineto{\pgfqpoint{1.971531in}{1.378194in}}%
\pgfpathlineto{\pgfqpoint{2.117895in}{1.458084in}}%
\pgfpathlineto{\pgfqpoint{2.264259in}{1.426644in}}%
\pgfpathlineto{\pgfqpoint{2.410624in}{1.464940in}}%
\pgfpathlineto{\pgfqpoint{2.556988in}{1.536869in}}%
\pgfpathlineto{\pgfqpoint{2.703353in}{1.607424in}}%
\pgfpathlineto{\pgfqpoint{2.849717in}{1.588506in}}%
\pgfpathlineto{\pgfqpoint{2.996081in}{1.695548in}}%
\pgfpathlineto{\pgfqpoint{3.142446in}{1.704547in}}%
\pgfpathlineto{\pgfqpoint{3.288810in}{1.740548in}}%
\pgfpathlineto{\pgfqpoint{3.435175in}{1.719014in}}%
\pgfpathlineto{\pgfqpoint{3.581539in}{1.809600in}}%
\pgfpathlineto{\pgfqpoint{3.727903in}{1.846186in}}%
\pgfpathlineto{\pgfqpoint{3.874268in}{1.741411in}}%
\pgfpathlineto{\pgfqpoint{4.020632in}{1.647013in}}%
\pgfpathlineto{\pgfqpoint{4.166997in}{1.824639in}}%
\pgfpathlineto{\pgfqpoint{4.313361in}{1.651802in}}%
\pgfpathlineto{\pgfqpoint{4.459725in}{1.869432in}}%
\pgfpathlineto{\pgfqpoint{4.898818in}{1.893244in}}%
\pgfpathlineto{\pgfqpoint{5.045183in}{1.911325in}}%
\pgfusepath{stroke}%
\end{pgfscope}%
\begin{pgfscope}%
\pgfpathrectangle{\pgfqpoint{0.588387in}{0.521603in}}{\pgfqpoint{4.669024in}{2.220246in}}%
\pgfusepath{clip}%
\pgfsetrectcap%
\pgfsetroundjoin%
\pgfsetlinewidth{1.505625pt}%
\definecolor{currentstroke}{rgb}{0.498039,0.498039,0.498039}%
\pgfsetstrokecolor{currentstroke}%
\pgfsetdash{}{0pt}%
\pgfpathmoveto{\pgfqpoint{0.800616in}{0.710280in}}%
\pgfpathlineto{\pgfqpoint{0.946980in}{0.798037in}}%
\pgfpathlineto{\pgfqpoint{1.093344in}{0.885794in}}%
\pgfpathlineto{\pgfqpoint{1.239709in}{0.973551in}}%
\pgfpathlineto{\pgfqpoint{1.386073in}{1.061307in}}%
\pgfpathlineto{\pgfqpoint{1.532438in}{1.149064in}}%
\pgfpathlineto{\pgfqpoint{1.678802in}{1.233172in}}%
\pgfpathlineto{\pgfqpoint{1.825166in}{1.308899in}}%
\pgfpathlineto{\pgfqpoint{1.971531in}{1.397238in}}%
\pgfpathlineto{\pgfqpoint{2.117895in}{1.476126in}}%
\pgfpathlineto{\pgfqpoint{2.264259in}{1.477206in}}%
\pgfpathlineto{\pgfqpoint{2.410624in}{1.538917in}}%
\pgfpathlineto{\pgfqpoint{2.556988in}{1.596043in}}%
\pgfpathlineto{\pgfqpoint{2.703353in}{1.689029in}}%
\pgfpathlineto{\pgfqpoint{2.849717in}{1.700261in}}%
\pgfpathlineto{\pgfqpoint{2.996081in}{1.806972in}}%
\pgfpathlineto{\pgfqpoint{3.142446in}{1.697881in}}%
\pgfpathlineto{\pgfqpoint{3.288810in}{1.768723in}}%
\pgfpathlineto{\pgfqpoint{3.435175in}{1.905220in}}%
\pgfpathlineto{\pgfqpoint{3.581539in}{1.857660in}}%
\pgfpathlineto{\pgfqpoint{3.727903in}{1.916952in}}%
\pgfpathlineto{\pgfqpoint{3.874268in}{1.655261in}}%
\pgfpathlineto{\pgfqpoint{4.020632in}{1.916519in}}%
\pgfpathlineto{\pgfqpoint{4.166997in}{1.954759in}}%
\pgfpathlineto{\pgfqpoint{4.313361in}{1.868587in}}%
\pgfpathlineto{\pgfqpoint{4.459725in}{2.082017in}}%
\pgfpathlineto{\pgfqpoint{4.898818in}{2.205231in}}%
\pgfpathlineto{\pgfqpoint{5.045183in}{2.113870in}}%
\pgfusepath{stroke}%
\end{pgfscope}%
\begin{pgfscope}%
\pgfpathrectangle{\pgfqpoint{0.588387in}{0.521603in}}{\pgfqpoint{4.669024in}{2.220246in}}%
\pgfusepath{clip}%
\pgfsetrectcap%
\pgfsetroundjoin%
\pgfsetlinewidth{1.505625pt}%
\definecolor{currentstroke}{rgb}{0.737255,0.741176,0.133333}%
\pgfsetstrokecolor{currentstroke}%
\pgfsetdash{}{0pt}%
\pgfpathmoveto{\pgfqpoint{0.800616in}{0.710280in}}%
\pgfpathlineto{\pgfqpoint{0.946980in}{0.798037in}}%
\pgfpathlineto{\pgfqpoint{1.093344in}{0.885794in}}%
\pgfpathlineto{\pgfqpoint{1.239709in}{0.973551in}}%
\pgfpathlineto{\pgfqpoint{1.386073in}{1.061307in}}%
\pgfpathlineto{\pgfqpoint{1.532438in}{1.149064in}}%
\pgfpathlineto{\pgfqpoint{1.678802in}{1.233172in}}%
\pgfpathlineto{\pgfqpoint{1.825166in}{1.308268in}}%
\pgfpathlineto{\pgfqpoint{1.971531in}{1.396994in}}%
\pgfpathlineto{\pgfqpoint{2.117895in}{1.477065in}}%
\pgfpathlineto{\pgfqpoint{2.264259in}{1.469248in}}%
\pgfpathlineto{\pgfqpoint{2.410624in}{1.535177in}}%
\pgfpathlineto{\pgfqpoint{2.556988in}{1.599273in}}%
\pgfpathlineto{\pgfqpoint{2.703353in}{1.683298in}}%
\pgfpathlineto{\pgfqpoint{2.849717in}{1.694962in}}%
\pgfpathlineto{\pgfqpoint{2.996081in}{1.803202in}}%
\pgfpathlineto{\pgfqpoint{3.142446in}{1.684726in}}%
\pgfpathlineto{\pgfqpoint{3.288810in}{1.768106in}}%
\pgfpathlineto{\pgfqpoint{3.435175in}{1.870325in}}%
\pgfpathlineto{\pgfqpoint{3.581539in}{1.847976in}}%
\pgfpathlineto{\pgfqpoint{3.727903in}{1.901905in}}%
\pgfpathlineto{\pgfqpoint{3.874268in}{1.655261in}}%
\pgfpathlineto{\pgfqpoint{4.020632in}{1.921969in}}%
\pgfpathlineto{\pgfqpoint{4.166997in}{2.022971in}}%
\pgfpathlineto{\pgfqpoint{4.313361in}{1.868587in}}%
\pgfpathlineto{\pgfqpoint{4.459725in}{2.093174in}}%
\pgfpathlineto{\pgfqpoint{4.898818in}{1.901833in}}%
\pgfusepath{stroke}%
\end{pgfscope}%
\begin{pgfscope}%
\pgfsetrectcap%
\pgfsetmiterjoin%
\pgfsetlinewidth{0.803000pt}%
\definecolor{currentstroke}{rgb}{0.000000,0.000000,0.000000}%
\pgfsetstrokecolor{currentstroke}%
\pgfsetdash{}{0pt}%
\pgfpathmoveto{\pgfqpoint{0.588387in}{0.521603in}}%
\pgfpathlineto{\pgfqpoint{0.588387in}{2.741849in}}%
\pgfusepath{stroke}%
\end{pgfscope}%
\begin{pgfscope}%
\pgfsetrectcap%
\pgfsetmiterjoin%
\pgfsetlinewidth{0.803000pt}%
\definecolor{currentstroke}{rgb}{0.000000,0.000000,0.000000}%
\pgfsetstrokecolor{currentstroke}%
\pgfsetdash{}{0pt}%
\pgfpathmoveto{\pgfqpoint{5.257411in}{0.521603in}}%
\pgfpathlineto{\pgfqpoint{5.257411in}{2.741849in}}%
\pgfusepath{stroke}%
\end{pgfscope}%
\begin{pgfscope}%
\pgfsetrectcap%
\pgfsetmiterjoin%
\pgfsetlinewidth{0.803000pt}%
\definecolor{currentstroke}{rgb}{0.000000,0.000000,0.000000}%
\pgfsetstrokecolor{currentstroke}%
\pgfsetdash{}{0pt}%
\pgfpathmoveto{\pgfqpoint{0.588387in}{0.521603in}}%
\pgfpathlineto{\pgfqpoint{5.257411in}{0.521603in}}%
\pgfusepath{stroke}%
\end{pgfscope}%
\begin{pgfscope}%
\pgfsetrectcap%
\pgfsetmiterjoin%
\pgfsetlinewidth{0.803000pt}%
\definecolor{currentstroke}{rgb}{0.000000,0.000000,0.000000}%
\pgfsetstrokecolor{currentstroke}%
\pgfsetdash{}{0pt}%
\pgfpathmoveto{\pgfqpoint{0.588387in}{2.741849in}}%
\pgfpathlineto{\pgfqpoint{5.257411in}{2.741849in}}%
\pgfusepath{stroke}%
\end{pgfscope}%
\begin{pgfscope}%
\pgfsetbuttcap%
\pgfsetmiterjoin%
\definecolor{currentfill}{rgb}{1.000000,1.000000,1.000000}%
\pgfsetfillcolor{currentfill}%
\pgfsetfillopacity{0.800000}%
\pgfsetlinewidth{1.003750pt}%
\definecolor{currentstroke}{rgb}{0.800000,0.800000,0.800000}%
\pgfsetstrokecolor{currentstroke}%
\pgfsetstrokeopacity{0.800000}%
\pgfsetdash{}{0pt}%
\pgfpathmoveto{\pgfqpoint{5.344911in}{0.969732in}}%
\pgfpathlineto{\pgfqpoint{8.259376in}{0.969732in}}%
\pgfpathquadraticcurveto{\pgfqpoint{8.284376in}{0.969732in}}{\pgfqpoint{8.284376in}{0.994732in}}%
\pgfpathlineto{\pgfqpoint{8.284376in}{2.654349in}}%
\pgfpathquadraticcurveto{\pgfqpoint{8.284376in}{2.679349in}}{\pgfqpoint{8.259376in}{2.679349in}}%
\pgfpathlineto{\pgfqpoint{5.344911in}{2.679349in}}%
\pgfpathquadraticcurveto{\pgfqpoint{5.319911in}{2.679349in}}{\pgfqpoint{5.319911in}{2.654349in}}%
\pgfpathlineto{\pgfqpoint{5.319911in}{0.994732in}}%
\pgfpathquadraticcurveto{\pgfqpoint{5.319911in}{0.969732in}}{\pgfqpoint{5.344911in}{0.969732in}}%
\pgfpathlineto{\pgfqpoint{5.344911in}{0.969732in}}%
\pgfpathclose%
\pgfusepath{stroke,fill}%
\end{pgfscope}%
\begin{pgfscope}%
\pgfsetrectcap%
\pgfsetroundjoin%
\pgfsetlinewidth{1.505625pt}%
\pgfsetstrokecolor{currentstroke3}%
\pgfsetdash{}{0pt}%
\pgfpathmoveto{\pgfqpoint{5.369911in}{2.578129in}}%
\pgfpathlineto{\pgfqpoint{5.494911in}{2.578129in}}%
\pgfpathlineto{\pgfqpoint{5.619911in}{2.578129in}}%
\pgfusepath{stroke}%
\end{pgfscope}%
\begin{pgfscope}%
\definecolor{textcolor}{rgb}{0.000000,0.000000,0.000000}%
\pgfsetstrokecolor{textcolor}%
\pgfsetfillcolor{textcolor}%
\pgftext[x=5.719911in,y=2.534379in,left,base]{\color{textcolor}{\rmfamily\fontsize{9.000000}{10.800000}\selectfont\catcode`\^=\active\def^{\ifmmode\sp\else\^{}\fi}\catcode`\%=\active\def%{\%}\NaiveCycles{}}}%
\end{pgfscope}%
\begin{pgfscope}%
\pgfsetrectcap%
\pgfsetroundjoin%
\pgfsetlinewidth{1.505625pt}%
\pgfsetstrokecolor{currentstroke1}%
\pgfsetdash{}{0pt}%
\pgfpathmoveto{\pgfqpoint{5.369911in}{2.394657in}}%
\pgfpathlineto{\pgfqpoint{5.494911in}{2.394657in}}%
\pgfpathlineto{\pgfqpoint{5.619911in}{2.394657in}}%
\pgfusepath{stroke}%
\end{pgfscope}%
\begin{pgfscope}%
\definecolor{textcolor}{rgb}{0.000000,0.000000,0.000000}%
\pgfsetstrokecolor{textcolor}%
\pgfsetfillcolor{textcolor}%
\pgftext[x=5.719911in,y=2.350907in,left,base]{\color{textcolor}{\rmfamily\fontsize{9.000000}{10.800000}\selectfont\catcode`\^=\active\def^{\ifmmode\sp\else\^{}\fi}\catcode`\%=\active\def%{\%}\CyclesMatchChunks{} \& \MergeLinear{}}}%
\end{pgfscope}%
\begin{pgfscope}%
\pgfsetrectcap%
\pgfsetroundjoin%
\pgfsetlinewidth{1.505625pt}%
\pgfsetstrokecolor{currentstroke2}%
\pgfsetdash{}{0pt}%
\pgfpathmoveto{\pgfqpoint{5.369911in}{2.207707in}}%
\pgfpathlineto{\pgfqpoint{5.494911in}{2.207707in}}%
\pgfpathlineto{\pgfqpoint{5.619911in}{2.207707in}}%
\pgfusepath{stroke}%
\end{pgfscope}%
\begin{pgfscope}%
\definecolor{textcolor}{rgb}{0.000000,0.000000,0.000000}%
\pgfsetstrokecolor{textcolor}%
\pgfsetfillcolor{textcolor}%
\pgftext[x=5.719911in,y=2.163957in,left,base]{\color{textcolor}{\rmfamily\fontsize{9.000000}{10.800000}\selectfont\catcode`\^=\active\def^{\ifmmode\sp\else\^{}\fi}\catcode`\%=\active\def%{\%}\CyclesMatchChunks{} \& \SharedVertices{}}}%
\end{pgfscope}%
\begin{pgfscope}%
\pgfsetrectcap%
\pgfsetroundjoin%
\pgfsetlinewidth{1.505625pt}%
\pgfsetstrokecolor{currentstroke4}%
\pgfsetdash{}{0pt}%
\pgfpathmoveto{\pgfqpoint{5.369911in}{2.020756in}}%
\pgfpathlineto{\pgfqpoint{5.494911in}{2.020756in}}%
\pgfpathlineto{\pgfqpoint{5.619911in}{2.020756in}}%
\pgfusepath{stroke}%
\end{pgfscope}%
\begin{pgfscope}%
\definecolor{textcolor}{rgb}{0.000000,0.000000,0.000000}%
\pgfsetstrokecolor{textcolor}%
\pgfsetfillcolor{textcolor}%
\pgftext[x=5.719911in,y=1.977006in,left,base]{\color{textcolor}{\rmfamily\fontsize{9.000000}{10.800000}\selectfont\catcode`\^=\active\def^{\ifmmode\sp\else\^{}\fi}\catcode`\%=\active\def%{\%}\Neighbors{} \& \MergeLinear{}}}%
\end{pgfscope}%
\begin{pgfscope}%
\pgfsetrectcap%
\pgfsetroundjoin%
\pgfsetlinewidth{1.505625pt}%
\pgfsetstrokecolor{currentstroke5}%
\pgfsetdash{}{0pt}%
\pgfpathmoveto{\pgfqpoint{5.369911in}{1.837285in}}%
\pgfpathlineto{\pgfqpoint{5.494911in}{1.837285in}}%
\pgfpathlineto{\pgfqpoint{5.619911in}{1.837285in}}%
\pgfusepath{stroke}%
\end{pgfscope}%
\begin{pgfscope}%
\definecolor{textcolor}{rgb}{0.000000,0.000000,0.000000}%
\pgfsetstrokecolor{textcolor}%
\pgfsetfillcolor{textcolor}%
\pgftext[x=5.719911in,y=1.793535in,left,base]{\color{textcolor}{\rmfamily\fontsize{9.000000}{10.800000}\selectfont\catcode`\^=\active\def^{\ifmmode\sp\else\^{}\fi}\catcode`\%=\active\def%{\%}\Neighbors{} \& \SharedVertices{}}}%
\end{pgfscope}%
\begin{pgfscope}%
\pgfsetrectcap%
\pgfsetroundjoin%
\pgfsetlinewidth{1.505625pt}%
\pgfsetstrokecolor{currentstroke6}%
\pgfsetdash{}{0pt}%
\pgfpathmoveto{\pgfqpoint{5.369911in}{1.650334in}}%
\pgfpathlineto{\pgfqpoint{5.494911in}{1.650334in}}%
\pgfpathlineto{\pgfqpoint{5.619911in}{1.650334in}}%
\pgfusepath{stroke}%
\end{pgfscope}%
\begin{pgfscope}%
\definecolor{textcolor}{rgb}{0.000000,0.000000,0.000000}%
\pgfsetstrokecolor{textcolor}%
\pgfsetfillcolor{textcolor}%
\pgftext[x=5.719911in,y=1.606584in,left,base]{\color{textcolor}{\rmfamily\fontsize{9.000000}{10.800000}\selectfont\catcode`\^=\active\def^{\ifmmode\sp\else\^{}\fi}\catcode`\%=\active\def%{\%}\NeighborsDegree{} \& \MergeLinear{}}}%
\end{pgfscope}%
\begin{pgfscope}%
\pgfsetrectcap%
\pgfsetroundjoin%
\pgfsetlinewidth{1.505625pt}%
\pgfsetstrokecolor{currentstroke7}%
\pgfsetdash{}{0pt}%
\pgfpathmoveto{\pgfqpoint{5.369911in}{1.463384in}}%
\pgfpathlineto{\pgfqpoint{5.494911in}{1.463384in}}%
\pgfpathlineto{\pgfqpoint{5.619911in}{1.463384in}}%
\pgfusepath{stroke}%
\end{pgfscope}%
\begin{pgfscope}%
\definecolor{textcolor}{rgb}{0.000000,0.000000,0.000000}%
\pgfsetstrokecolor{textcolor}%
\pgfsetfillcolor{textcolor}%
\pgftext[x=5.719911in,y=1.419634in,left,base]{\color{textcolor}{\rmfamily\fontsize{9.000000}{10.800000}\selectfont\catcode`\^=\active\def^{\ifmmode\sp\else\^{}\fi}\catcode`\%=\active\def%{\%}\NeighborsDegree{} \& \SharedVertices{}}}%
\end{pgfscope}%
\begin{pgfscope}%
\pgfsetrectcap%
\pgfsetroundjoin%
\pgfsetlinewidth{1.505625pt}%
\definecolor{currentstroke}{rgb}{0.498039,0.498039,0.498039}%
\pgfsetstrokecolor{currentstroke}%
\pgfsetdash{}{0pt}%
\pgfpathmoveto{\pgfqpoint{5.369911in}{1.276433in}}%
\pgfpathlineto{\pgfqpoint{5.494911in}{1.276433in}}%
\pgfpathlineto{\pgfqpoint{5.619911in}{1.276433in}}%
\pgfusepath{stroke}%
\end{pgfscope}%
\begin{pgfscope}%
\definecolor{textcolor}{rgb}{0.000000,0.000000,0.000000}%
\pgfsetstrokecolor{textcolor}%
\pgfsetfillcolor{textcolor}%
\pgftext[x=5.719911in,y=1.232683in,left,base]{\color{textcolor}{\rmfamily\fontsize{9.000000}{10.800000}\selectfont\catcode`\^=\active\def^{\ifmmode\sp\else\^{}\fi}\catcode`\%=\active\def%{\%}\None{} \& \MergeLinear{}}}%
\end{pgfscope}%
\begin{pgfscope}%
\pgfsetrectcap%
\pgfsetroundjoin%
\pgfsetlinewidth{1.505625pt}%
\definecolor{currentstroke}{rgb}{0.737255,0.741176,0.133333}%
\pgfsetstrokecolor{currentstroke}%
\pgfsetdash{}{0pt}%
\pgfpathmoveto{\pgfqpoint{5.369911in}{1.092962in}}%
\pgfpathlineto{\pgfqpoint{5.494911in}{1.092962in}}%
\pgfpathlineto{\pgfqpoint{5.619911in}{1.092962in}}%
\pgfusepath{stroke}%
\end{pgfscope}%
\begin{pgfscope}%
\definecolor{textcolor}{rgb}{0.000000,0.000000,0.000000}%
\pgfsetstrokecolor{textcolor}%
\pgfsetfillcolor{textcolor}%
\pgftext[x=5.719911in,y=1.049212in,left,base]{\color{textcolor}{\rmfamily\fontsize{9.000000}{10.800000}\selectfont\catcode`\^=\active\def^{\ifmmode\sp\else\^{}\fi}\catcode`\%=\active\def%{\%}\None{} \& \SharedVertices{}}}%
\end{pgfscope}%
\end{pgfpicture}%
\makeatother%
\endgroup%
}
	\caption[Checks performed for globally rigid graphs (all)]{
		The number of checks performed to find all NAC-colorings for globally rigid graphs.}%
	\label{fig:graph_globally_rigid_all_checks}
\end{figure}%



\subsection{Performance on graphs with no NAC-colorings}

In the previous section, the \Subgraphs{} algorithm
often performed worse considering runtime.
As explained, it is caused by the graphs being too simple
--- having plenty of NAC-colorings.
%
For many NP-complete problems, studied instances are often
those where there are only few or no solutions.
In this section, we focus on graphs with no NAC-colorings.

We searched for random graphs where \( |E| \ge 2|V(G)| - 2 \) that have
multiple monochromatic classes, but no NAC-coloring
\todo{Uncoment footnote}
% \footnote{
{We could have also used the same formula as for globally rigid graphs.}.
%
As this graph generation was slow, we searched only for
graphs with more than \( 2\sqrt{|V(G)|} \) \trcon{} components.
%
This once again shows how effective monochromatic classes are
in comparison with \trcon{} components.
We generated and tested five thousand of such graphs from 40 to 130 vertices in size.
Just one of them had more than one monochromatic class.
%
The following benchmarks are run with monochromatic classes disabled,
\trcon{} components are used instead.

For these graphs, \NaiveCycles{} algorithm needs to traverse all \( 2^{t-1} \) \(\red\)-\(\blue\)-colorings
where \( t \) is the number of \trcon{} components. It can be clearly seen that
this is not suitable for graphs as large as we use in this benchmark,
therefore, they are not present as they did not finish in reasonable time.
It can be seen from \Cref{fig:graph_no_nac_coloring_first_runtime}
that \SharedVertices{} is faster than \MergeLinear{},
for runtime and also for the number of checks performed
as shown in \Cref{fig:graph_no_nac_coloring_first_checks}.
%
It can be also seen that \NeighborsDegree{} strategy is
faster than the other strategies, and it holds for both merging strategies.
Also notice that,
runtime grows strictly exponentially.
This is in contrast with the previous section
as these graphs are not simple anymore.

\begin{figure}[thbp]
	\centering
	\scalebox{\BenchFigureScale}{%% Creator: Matplotlib, PGF backend
%%
%% To include the figure in your LaTeX document, write
%%   \input{<filename>.pgf}
%%
%% Make sure the required packages are loaded in your preamble
%%   \usepackage{pgf}
%%
%% Also ensure that all the required font packages are loaded; for instance,
%% the lmodern package is sometimes necessary when using math font.
%%   \usepackage{lmodern}
%%
%% Figures using additional raster images can only be included by \input if
%% they are in the same directory as the main LaTeX file. For loading figures
%% from other directories you can use the `import` package
%%   \usepackage{import}
%%
%% and then include the figures with
%%   \import{<path to file>}{<filename>.pgf}
%%
%% Matplotlib used the following preamble
%%   \def\mathdefault#1{#1}
%%   \everymath=\expandafter{\the\everymath\displaystyle}
%%   \IfFileExists{scrextend.sty}{
%%     \usepackage[fontsize=10.000000pt]{scrextend}
%%   }{
%%     \renewcommand{\normalsize}{\fontsize{10.000000}{12.000000}\selectfont}
%%     \normalsize
%%   }
%%   
%%   \ifdefined\pdftexversion\else  % non-pdftex case.
%%     \usepackage{fontspec}
%%     \setmainfont{DejaVuSans.ttf}[Path=\detokenize{/home/petr/Projects/PyRigi/.venv/lib/python3.12/site-packages/matplotlib/mpl-data/fonts/ttf/}]
%%     \setsansfont{DejaVuSans.ttf}[Path=\detokenize{/home/petr/Projects/PyRigi/.venv/lib/python3.12/site-packages/matplotlib/mpl-data/fonts/ttf/}]
%%     \setmonofont{DejaVuSansMono.ttf}[Path=\detokenize{/home/petr/Projects/PyRigi/.venv/lib/python3.12/site-packages/matplotlib/mpl-data/fonts/ttf/}]
%%   \fi
%%   \makeatletter\@ifpackageloaded{under\Score{}}{}{\usepackage[strings]{under\Score{}}}\makeatother
%%
\begingroup%
\makeatletter%
\begin{pgfpicture}%
\pgfpathrectangle{\pgfpointorigin}{\pgfqpoint{8.384376in}{2.841849in}}%
\pgfusepath{use as bounding box, clip}%
\begin{pgfscope}%
\pgfsetbuttcap%
\pgfsetmiterjoin%
\definecolor{currentfill}{rgb}{1.000000,1.000000,1.000000}%
\pgfsetfillcolor{currentfill}%
\pgfsetlinewidth{0.000000pt}%
\definecolor{currentstroke}{rgb}{1.000000,1.000000,1.000000}%
\pgfsetstrokecolor{currentstroke}%
\pgfsetdash{}{0pt}%
\pgfpathmoveto{\pgfqpoint{0.000000in}{0.000000in}}%
\pgfpathlineto{\pgfqpoint{8.384376in}{0.000000in}}%
\pgfpathlineto{\pgfqpoint{8.384376in}{2.841849in}}%
\pgfpathlineto{\pgfqpoint{0.000000in}{2.841849in}}%
\pgfpathlineto{\pgfqpoint{0.000000in}{0.000000in}}%
\pgfpathclose%
\pgfusepath{fill}%
\end{pgfscope}%
\begin{pgfscope}%
\pgfsetbuttcap%
\pgfsetmiterjoin%
\definecolor{currentfill}{rgb}{1.000000,1.000000,1.000000}%
\pgfsetfillcolor{currentfill}%
\pgfsetlinewidth{0.000000pt}%
\definecolor{currentstroke}{rgb}{0.000000,0.000000,0.000000}%
\pgfsetstrokecolor{currentstroke}%
\pgfsetstrokeopacity{0.000000}%
\pgfsetdash{}{0pt}%
\pgfpathmoveto{\pgfqpoint{0.588387in}{0.521603in}}%
\pgfpathlineto{\pgfqpoint{5.257411in}{0.521603in}}%
\pgfpathlineto{\pgfqpoint{5.257411in}{2.741849in}}%
\pgfpathlineto{\pgfqpoint{0.588387in}{2.741849in}}%
\pgfpathlineto{\pgfqpoint{0.588387in}{0.521603in}}%
\pgfpathclose%
\pgfusepath{fill}%
\end{pgfscope}%
\begin{pgfscope}%
\pgfsetbuttcap%
\pgfsetroundjoin%
\definecolor{currentfill}{rgb}{0.000000,0.000000,0.000000}%
\pgfsetfillcolor{currentfill}%
\pgfsetlinewidth{0.803000pt}%
\definecolor{currentstroke}{rgb}{0.000000,0.000000,0.000000}%
\pgfsetstrokecolor{currentstroke}%
\pgfsetdash{}{0pt}%
\pgfsys@defobject{currentmarker}{\pgfqpoint{0.000000in}{-0.048611in}}{\pgfqpoint{0.000000in}{0.000000in}}{%
\pgfpathmoveto{\pgfqpoint{0.000000in}{0.000000in}}%
\pgfpathlineto{\pgfqpoint{0.000000in}{-0.048611in}}%
\pgfusepath{stroke,fill}%
}%
\begin{pgfscope}%
\pgfsys@transformshift{1.009855in}{0.521603in}%
\pgfsys@useobject{currentmarker}{}%
\end{pgfscope}%
\end{pgfscope}%
\begin{pgfscope}%
\definecolor{textcolor}{rgb}{0.000000,0.000000,0.000000}%
\pgfsetstrokecolor{textcolor}%
\pgfsetfillcolor{textcolor}%
\pgftext[x=1.009855in,y=0.424381in,,top]{\color{textcolor}{\rmfamily\fontsize{10.000000}{12.000000}\selectfont\catcode`\^=\active\def^{\ifmmode\sp\else\^{}\fi}\catcode`\%=\active\def%{\%}$\mathdefault{20}$}}%
\end{pgfscope}%
\begin{pgfscope}%
\pgfsetbuttcap%
\pgfsetroundjoin%
\definecolor{currentfill}{rgb}{0.000000,0.000000,0.000000}%
\pgfsetfillcolor{currentfill}%
\pgfsetlinewidth{0.803000pt}%
\definecolor{currentstroke}{rgb}{0.000000,0.000000,0.000000}%
\pgfsetstrokecolor{currentstroke}%
\pgfsetdash{}{0pt}%
\pgfsys@defobject{currentmarker}{\pgfqpoint{0.000000in}{-0.048611in}}{\pgfqpoint{0.000000in}{0.000000in}}{%
\pgfpathmoveto{\pgfqpoint{0.000000in}{0.000000in}}%
\pgfpathlineto{\pgfqpoint{0.000000in}{-0.048611in}}%
\pgfusepath{stroke,fill}%
}%
\begin{pgfscope}%
\pgfsys@transformshift{1.607681in}{0.521603in}%
\pgfsys@useobject{currentmarker}{}%
\end{pgfscope}%
\end{pgfscope}%
\begin{pgfscope}%
\definecolor{textcolor}{rgb}{0.000000,0.000000,0.000000}%
\pgfsetstrokecolor{textcolor}%
\pgfsetfillcolor{textcolor}%
\pgftext[x=1.607681in,y=0.424381in,,top]{\color{textcolor}{\rmfamily\fontsize{10.000000}{12.000000}\selectfont\catcode`\^=\active\def^{\ifmmode\sp\else\^{}\fi}\catcode`\%=\active\def%{\%}$\mathdefault{40}$}}%
\end{pgfscope}%
\begin{pgfscope}%
\pgfsetbuttcap%
\pgfsetroundjoin%
\definecolor{currentfill}{rgb}{0.000000,0.000000,0.000000}%
\pgfsetfillcolor{currentfill}%
\pgfsetlinewidth{0.803000pt}%
\definecolor{currentstroke}{rgb}{0.000000,0.000000,0.000000}%
\pgfsetstrokecolor{currentstroke}%
\pgfsetdash{}{0pt}%
\pgfsys@defobject{currentmarker}{\pgfqpoint{0.000000in}{-0.048611in}}{\pgfqpoint{0.000000in}{0.000000in}}{%
\pgfpathmoveto{\pgfqpoint{0.000000in}{0.000000in}}%
\pgfpathlineto{\pgfqpoint{0.000000in}{-0.048611in}}%
\pgfusepath{stroke,fill}%
}%
\begin{pgfscope}%
\pgfsys@transformshift{2.205508in}{0.521603in}%
\pgfsys@useobject{currentmarker}{}%
\end{pgfscope}%
\end{pgfscope}%
\begin{pgfscope}%
\definecolor{textcolor}{rgb}{0.000000,0.000000,0.000000}%
\pgfsetstrokecolor{textcolor}%
\pgfsetfillcolor{textcolor}%
\pgftext[x=2.205508in,y=0.424381in,,top]{\color{textcolor}{\rmfamily\fontsize{10.000000}{12.000000}\selectfont\catcode`\^=\active\def^{\ifmmode\sp\else\^{}\fi}\catcode`\%=\active\def%{\%}$\mathdefault{60}$}}%
\end{pgfscope}%
\begin{pgfscope}%
\pgfsetbuttcap%
\pgfsetroundjoin%
\definecolor{currentfill}{rgb}{0.000000,0.000000,0.000000}%
\pgfsetfillcolor{currentfill}%
\pgfsetlinewidth{0.803000pt}%
\definecolor{currentstroke}{rgb}{0.000000,0.000000,0.000000}%
\pgfsetstrokecolor{currentstroke}%
\pgfsetdash{}{0pt}%
\pgfsys@defobject{currentmarker}{\pgfqpoint{0.000000in}{-0.048611in}}{\pgfqpoint{0.000000in}{0.000000in}}{%
\pgfpathmoveto{\pgfqpoint{0.000000in}{0.000000in}}%
\pgfpathlineto{\pgfqpoint{0.000000in}{-0.048611in}}%
\pgfusepath{stroke,fill}%
}%
\begin{pgfscope}%
\pgfsys@transformshift{2.803334in}{0.521603in}%
\pgfsys@useobject{currentmarker}{}%
\end{pgfscope}%
\end{pgfscope}%
\begin{pgfscope}%
\definecolor{textcolor}{rgb}{0.000000,0.000000,0.000000}%
\pgfsetstrokecolor{textcolor}%
\pgfsetfillcolor{textcolor}%
\pgftext[x=2.803334in,y=0.424381in,,top]{\color{textcolor}{\rmfamily\fontsize{10.000000}{12.000000}\selectfont\catcode`\^=\active\def^{\ifmmode\sp\else\^{}\fi}\catcode`\%=\active\def%{\%}$\mathdefault{80}$}}%
\end{pgfscope}%
\begin{pgfscope}%
\pgfsetbuttcap%
\pgfsetroundjoin%
\definecolor{currentfill}{rgb}{0.000000,0.000000,0.000000}%
\pgfsetfillcolor{currentfill}%
\pgfsetlinewidth{0.803000pt}%
\definecolor{currentstroke}{rgb}{0.000000,0.000000,0.000000}%
\pgfsetstrokecolor{currentstroke}%
\pgfsetdash{}{0pt}%
\pgfsys@defobject{currentmarker}{\pgfqpoint{0.000000in}{-0.048611in}}{\pgfqpoint{0.000000in}{0.000000in}}{%
\pgfpathmoveto{\pgfqpoint{0.000000in}{0.000000in}}%
\pgfpathlineto{\pgfqpoint{0.000000in}{-0.048611in}}%
\pgfusepath{stroke,fill}%
}%
\begin{pgfscope}%
\pgfsys@transformshift{3.401160in}{0.521603in}%
\pgfsys@useobject{currentmarker}{}%
\end{pgfscope}%
\end{pgfscope}%
\begin{pgfscope}%
\definecolor{textcolor}{rgb}{0.000000,0.000000,0.000000}%
\pgfsetstrokecolor{textcolor}%
\pgfsetfillcolor{textcolor}%
\pgftext[x=3.401160in,y=0.424381in,,top]{\color{textcolor}{\rmfamily\fontsize{10.000000}{12.000000}\selectfont\catcode`\^=\active\def^{\ifmmode\sp\else\^{}\fi}\catcode`\%=\active\def%{\%}$\mathdefault{100}$}}%
\end{pgfscope}%
\begin{pgfscope}%
\pgfsetbuttcap%
\pgfsetroundjoin%
\definecolor{currentfill}{rgb}{0.000000,0.000000,0.000000}%
\pgfsetfillcolor{currentfill}%
\pgfsetlinewidth{0.803000pt}%
\definecolor{currentstroke}{rgb}{0.000000,0.000000,0.000000}%
\pgfsetstrokecolor{currentstroke}%
\pgfsetdash{}{0pt}%
\pgfsys@defobject{currentmarker}{\pgfqpoint{0.000000in}{-0.048611in}}{\pgfqpoint{0.000000in}{0.000000in}}{%
\pgfpathmoveto{\pgfqpoint{0.000000in}{0.000000in}}%
\pgfpathlineto{\pgfqpoint{0.000000in}{-0.048611in}}%
\pgfusepath{stroke,fill}%
}%
\begin{pgfscope}%
\pgfsys@transformshift{3.998987in}{0.521603in}%
\pgfsys@useobject{currentmarker}{}%
\end{pgfscope}%
\end{pgfscope}%
\begin{pgfscope}%
\definecolor{textcolor}{rgb}{0.000000,0.000000,0.000000}%
\pgfsetstrokecolor{textcolor}%
\pgfsetfillcolor{textcolor}%
\pgftext[x=3.998987in,y=0.424381in,,top]{\color{textcolor}{\rmfamily\fontsize{10.000000}{12.000000}\selectfont\catcode`\^=\active\def^{\ifmmode\sp\else\^{}\fi}\catcode`\%=\active\def%{\%}$\mathdefault{120}$}}%
\end{pgfscope}%
\begin{pgfscope}%
\pgfsetbuttcap%
\pgfsetroundjoin%
\definecolor{currentfill}{rgb}{0.000000,0.000000,0.000000}%
\pgfsetfillcolor{currentfill}%
\pgfsetlinewidth{0.803000pt}%
\definecolor{currentstroke}{rgb}{0.000000,0.000000,0.000000}%
\pgfsetstrokecolor{currentstroke}%
\pgfsetdash{}{0pt}%
\pgfsys@defobject{currentmarker}{\pgfqpoint{0.000000in}{-0.048611in}}{\pgfqpoint{0.000000in}{0.000000in}}{%
\pgfpathmoveto{\pgfqpoint{0.000000in}{0.000000in}}%
\pgfpathlineto{\pgfqpoint{0.000000in}{-0.048611in}}%
\pgfusepath{stroke,fill}%
}%
\begin{pgfscope}%
\pgfsys@transformshift{4.596813in}{0.521603in}%
\pgfsys@useobject{currentmarker}{}%
\end{pgfscope}%
\end{pgfscope}%
\begin{pgfscope}%
\definecolor{textcolor}{rgb}{0.000000,0.000000,0.000000}%
\pgfsetstrokecolor{textcolor}%
\pgfsetfillcolor{textcolor}%
\pgftext[x=4.596813in,y=0.424381in,,top]{\color{textcolor}{\rmfamily\fontsize{10.000000}{12.000000}\selectfont\catcode`\^=\active\def^{\ifmmode\sp\else\^{}\fi}\catcode`\%=\active\def%{\%}$\mathdefault{140}$}}%
\end{pgfscope}%
\begin{pgfscope}%
\pgfsetbuttcap%
\pgfsetroundjoin%
\definecolor{currentfill}{rgb}{0.000000,0.000000,0.000000}%
\pgfsetfillcolor{currentfill}%
\pgfsetlinewidth{0.803000pt}%
\definecolor{currentstroke}{rgb}{0.000000,0.000000,0.000000}%
\pgfsetstrokecolor{currentstroke}%
\pgfsetdash{}{0pt}%
\pgfsys@defobject{currentmarker}{\pgfqpoint{0.000000in}{-0.048611in}}{\pgfqpoint{0.000000in}{0.000000in}}{%
\pgfpathmoveto{\pgfqpoint{0.000000in}{0.000000in}}%
\pgfpathlineto{\pgfqpoint{0.000000in}{-0.048611in}}%
\pgfusepath{stroke,fill}%
}%
\begin{pgfscope}%
\pgfsys@transformshift{5.194639in}{0.521603in}%
\pgfsys@useobject{currentmarker}{}%
\end{pgfscope}%
\end{pgfscope}%
\begin{pgfscope}%
\definecolor{textcolor}{rgb}{0.000000,0.000000,0.000000}%
\pgfsetstrokecolor{textcolor}%
\pgfsetfillcolor{textcolor}%
\pgftext[x=5.194639in,y=0.424381in,,top]{\color{textcolor}{\rmfamily\fontsize{10.000000}{12.000000}\selectfont\catcode`\^=\active\def^{\ifmmode\sp\else\^{}\fi}\catcode`\%=\active\def%{\%}$\mathdefault{160}$}}%
\end{pgfscope}%
\begin{pgfscope}%
\definecolor{textcolor}{rgb}{0.000000,0.000000,0.000000}%
\pgfsetstrokecolor{textcolor}%
\pgfsetfillcolor{textcolor}%
\pgftext[x=2.922899in,y=0.234413in,,top]{\color{textcolor}{\rmfamily\fontsize{10.000000}{12.000000}\selectfont\catcode`\^=\active\def^{\ifmmode\sp\else\^{}\fi}\catcode`\%=\active\def%{\%}Triangle components}}%
\end{pgfscope}%
\begin{pgfscope}%
\pgfsetbuttcap%
\pgfsetroundjoin%
\definecolor{currentfill}{rgb}{0.000000,0.000000,0.000000}%
\pgfsetfillcolor{currentfill}%
\pgfsetlinewidth{0.803000pt}%
\definecolor{currentstroke}{rgb}{0.000000,0.000000,0.000000}%
\pgfsetstrokecolor{currentstroke}%
\pgfsetdash{}{0pt}%
\pgfsys@defobject{currentmarker}{\pgfqpoint{-0.048611in}{0.000000in}}{\pgfqpoint{-0.000000in}{0.000000in}}{%
\pgfpathmoveto{\pgfqpoint{-0.000000in}{0.000000in}}%
\pgfpathlineto{\pgfqpoint{-0.048611in}{0.000000in}}%
\pgfusepath{stroke,fill}%
}%
\begin{pgfscope}%
\pgfsys@transformshift{0.588387in}{0.568951in}%
\pgfsys@useobject{currentmarker}{}%
\end{pgfscope}%
\end{pgfscope}%
\begin{pgfscope}%
\definecolor{textcolor}{rgb}{0.000000,0.000000,0.000000}%
\pgfsetstrokecolor{textcolor}%
\pgfsetfillcolor{textcolor}%
\pgftext[x=0.289968in, y=0.516190in, left, base]{\color{textcolor}{\rmfamily\fontsize{10.000000}{12.000000}\selectfont\catcode`\^=\active\def^{\ifmmode\sp\else\^{}\fi}\catcode`\%=\active\def%{\%}$\mathdefault{10^{2}}$}}%
\end{pgfscope}%
\begin{pgfscope}%
\pgfsetbuttcap%
\pgfsetroundjoin%
\definecolor{currentfill}{rgb}{0.000000,0.000000,0.000000}%
\pgfsetfillcolor{currentfill}%
\pgfsetlinewidth{0.803000pt}%
\definecolor{currentstroke}{rgb}{0.000000,0.000000,0.000000}%
\pgfsetstrokecolor{currentstroke}%
\pgfsetdash{}{0pt}%
\pgfsys@defobject{currentmarker}{\pgfqpoint{-0.048611in}{0.000000in}}{\pgfqpoint{-0.000000in}{0.000000in}}{%
\pgfpathmoveto{\pgfqpoint{-0.000000in}{0.000000in}}%
\pgfpathlineto{\pgfqpoint{-0.048611in}{0.000000in}}%
\pgfusepath{stroke,fill}%
}%
\begin{pgfscope}%
\pgfsys@transformshift{0.588387in}{1.835608in}%
\pgfsys@useobject{currentmarker}{}%
\end{pgfscope}%
\end{pgfscope}%
\begin{pgfscope}%
\definecolor{textcolor}{rgb}{0.000000,0.000000,0.000000}%
\pgfsetstrokecolor{textcolor}%
\pgfsetfillcolor{textcolor}%
\pgftext[x=0.289968in, y=1.782846in, left, base]{\color{textcolor}{\rmfamily\fontsize{10.000000}{12.000000}\selectfont\catcode`\^=\active\def^{\ifmmode\sp\else\^{}\fi}\catcode`\%=\active\def%{\%}$\mathdefault{10^{3}}$}}%
\end{pgfscope}%
\begin{pgfscope}%
\pgfsetbuttcap%
\pgfsetroundjoin%
\definecolor{currentfill}{rgb}{0.000000,0.000000,0.000000}%
\pgfsetfillcolor{currentfill}%
\pgfsetlinewidth{0.602250pt}%
\definecolor{currentstroke}{rgb}{0.000000,0.000000,0.000000}%
\pgfsetstrokecolor{currentstroke}%
\pgfsetdash{}{0pt}%
\pgfsys@defobject{currentmarker}{\pgfqpoint{-0.027778in}{0.000000in}}{\pgfqpoint{-0.000000in}{0.000000in}}{%
\pgfpathmoveto{\pgfqpoint{-0.000000in}{0.000000in}}%
\pgfpathlineto{\pgfqpoint{-0.027778in}{0.000000in}}%
\pgfusepath{stroke,fill}%
}%
\begin{pgfscope}%
\pgfsys@transformshift{0.588387in}{0.950253in}%
\pgfsys@useobject{currentmarker}{}%
\end{pgfscope}%
\end{pgfscope}%
\begin{pgfscope}%
\pgfsetbuttcap%
\pgfsetroundjoin%
\definecolor{currentfill}{rgb}{0.000000,0.000000,0.000000}%
\pgfsetfillcolor{currentfill}%
\pgfsetlinewidth{0.602250pt}%
\definecolor{currentstroke}{rgb}{0.000000,0.000000,0.000000}%
\pgfsetstrokecolor{currentstroke}%
\pgfsetdash{}{0pt}%
\pgfsys@defobject{currentmarker}{\pgfqpoint{-0.027778in}{0.000000in}}{\pgfqpoint{-0.000000in}{0.000000in}}{%
\pgfpathmoveto{\pgfqpoint{-0.000000in}{0.000000in}}%
\pgfpathlineto{\pgfqpoint{-0.027778in}{0.000000in}}%
\pgfusepath{stroke,fill}%
}%
\begin{pgfscope}%
\pgfsys@transformshift{0.588387in}{1.173300in}%
\pgfsys@useobject{currentmarker}{}%
\end{pgfscope}%
\end{pgfscope}%
\begin{pgfscope}%
\pgfsetbuttcap%
\pgfsetroundjoin%
\definecolor{currentfill}{rgb}{0.000000,0.000000,0.000000}%
\pgfsetfillcolor{currentfill}%
\pgfsetlinewidth{0.602250pt}%
\definecolor{currentstroke}{rgb}{0.000000,0.000000,0.000000}%
\pgfsetstrokecolor{currentstroke}%
\pgfsetdash{}{0pt}%
\pgfsys@defobject{currentmarker}{\pgfqpoint{-0.027778in}{0.000000in}}{\pgfqpoint{-0.000000in}{0.000000in}}{%
\pgfpathmoveto{\pgfqpoint{-0.000000in}{0.000000in}}%
\pgfpathlineto{\pgfqpoint{-0.027778in}{0.000000in}}%
\pgfusepath{stroke,fill}%
}%
\begin{pgfscope}%
\pgfsys@transformshift{0.588387in}{1.331554in}%
\pgfsys@useobject{currentmarker}{}%
\end{pgfscope}%
\end{pgfscope}%
\begin{pgfscope}%
\pgfsetbuttcap%
\pgfsetroundjoin%
\definecolor{currentfill}{rgb}{0.000000,0.000000,0.000000}%
\pgfsetfillcolor{currentfill}%
\pgfsetlinewidth{0.602250pt}%
\definecolor{currentstroke}{rgb}{0.000000,0.000000,0.000000}%
\pgfsetstrokecolor{currentstroke}%
\pgfsetdash{}{0pt}%
\pgfsys@defobject{currentmarker}{\pgfqpoint{-0.027778in}{0.000000in}}{\pgfqpoint{-0.000000in}{0.000000in}}{%
\pgfpathmoveto{\pgfqpoint{-0.000000in}{0.000000in}}%
\pgfpathlineto{\pgfqpoint{-0.027778in}{0.000000in}}%
\pgfusepath{stroke,fill}%
}%
\begin{pgfscope}%
\pgfsys@transformshift{0.588387in}{1.454306in}%
\pgfsys@useobject{currentmarker}{}%
\end{pgfscope}%
\end{pgfscope}%
\begin{pgfscope}%
\pgfsetbuttcap%
\pgfsetroundjoin%
\definecolor{currentfill}{rgb}{0.000000,0.000000,0.000000}%
\pgfsetfillcolor{currentfill}%
\pgfsetlinewidth{0.602250pt}%
\definecolor{currentstroke}{rgb}{0.000000,0.000000,0.000000}%
\pgfsetstrokecolor{currentstroke}%
\pgfsetdash{}{0pt}%
\pgfsys@defobject{currentmarker}{\pgfqpoint{-0.027778in}{0.000000in}}{\pgfqpoint{-0.000000in}{0.000000in}}{%
\pgfpathmoveto{\pgfqpoint{-0.000000in}{0.000000in}}%
\pgfpathlineto{\pgfqpoint{-0.027778in}{0.000000in}}%
\pgfusepath{stroke,fill}%
}%
\begin{pgfscope}%
\pgfsys@transformshift{0.588387in}{1.554602in}%
\pgfsys@useobject{currentmarker}{}%
\end{pgfscope}%
\end{pgfscope}%
\begin{pgfscope}%
\pgfsetbuttcap%
\pgfsetroundjoin%
\definecolor{currentfill}{rgb}{0.000000,0.000000,0.000000}%
\pgfsetfillcolor{currentfill}%
\pgfsetlinewidth{0.602250pt}%
\definecolor{currentstroke}{rgb}{0.000000,0.000000,0.000000}%
\pgfsetstrokecolor{currentstroke}%
\pgfsetdash{}{0pt}%
\pgfsys@defobject{currentmarker}{\pgfqpoint{-0.027778in}{0.000000in}}{\pgfqpoint{-0.000000in}{0.000000in}}{%
\pgfpathmoveto{\pgfqpoint{-0.000000in}{0.000000in}}%
\pgfpathlineto{\pgfqpoint{-0.027778in}{0.000000in}}%
\pgfusepath{stroke,fill}%
}%
\begin{pgfscope}%
\pgfsys@transformshift{0.588387in}{1.639400in}%
\pgfsys@useobject{currentmarker}{}%
\end{pgfscope}%
\end{pgfscope}%
\begin{pgfscope}%
\pgfsetbuttcap%
\pgfsetroundjoin%
\definecolor{currentfill}{rgb}{0.000000,0.000000,0.000000}%
\pgfsetfillcolor{currentfill}%
\pgfsetlinewidth{0.602250pt}%
\definecolor{currentstroke}{rgb}{0.000000,0.000000,0.000000}%
\pgfsetstrokecolor{currentstroke}%
\pgfsetdash{}{0pt}%
\pgfsys@defobject{currentmarker}{\pgfqpoint{-0.027778in}{0.000000in}}{\pgfqpoint{-0.000000in}{0.000000in}}{%
\pgfpathmoveto{\pgfqpoint{-0.000000in}{0.000000in}}%
\pgfpathlineto{\pgfqpoint{-0.027778in}{0.000000in}}%
\pgfusepath{stroke,fill}%
}%
\begin{pgfscope}%
\pgfsys@transformshift{0.588387in}{1.712856in}%
\pgfsys@useobject{currentmarker}{}%
\end{pgfscope}%
\end{pgfscope}%
\begin{pgfscope}%
\pgfsetbuttcap%
\pgfsetroundjoin%
\definecolor{currentfill}{rgb}{0.000000,0.000000,0.000000}%
\pgfsetfillcolor{currentfill}%
\pgfsetlinewidth{0.602250pt}%
\definecolor{currentstroke}{rgb}{0.000000,0.000000,0.000000}%
\pgfsetstrokecolor{currentstroke}%
\pgfsetdash{}{0pt}%
\pgfsys@defobject{currentmarker}{\pgfqpoint{-0.027778in}{0.000000in}}{\pgfqpoint{-0.000000in}{0.000000in}}{%
\pgfpathmoveto{\pgfqpoint{-0.000000in}{0.000000in}}%
\pgfpathlineto{\pgfqpoint{-0.027778in}{0.000000in}}%
\pgfusepath{stroke,fill}%
}%
\begin{pgfscope}%
\pgfsys@transformshift{0.588387in}{1.777649in}%
\pgfsys@useobject{currentmarker}{}%
\end{pgfscope}%
\end{pgfscope}%
\begin{pgfscope}%
\pgfsetbuttcap%
\pgfsetroundjoin%
\definecolor{currentfill}{rgb}{0.000000,0.000000,0.000000}%
\pgfsetfillcolor{currentfill}%
\pgfsetlinewidth{0.602250pt}%
\definecolor{currentstroke}{rgb}{0.000000,0.000000,0.000000}%
\pgfsetstrokecolor{currentstroke}%
\pgfsetdash{}{0pt}%
\pgfsys@defobject{currentmarker}{\pgfqpoint{-0.027778in}{0.000000in}}{\pgfqpoint{-0.000000in}{0.000000in}}{%
\pgfpathmoveto{\pgfqpoint{-0.000000in}{0.000000in}}%
\pgfpathlineto{\pgfqpoint{-0.027778in}{0.000000in}}%
\pgfusepath{stroke,fill}%
}%
\begin{pgfscope}%
\pgfsys@transformshift{0.588387in}{2.216909in}%
\pgfsys@useobject{currentmarker}{}%
\end{pgfscope}%
\end{pgfscope}%
\begin{pgfscope}%
\pgfsetbuttcap%
\pgfsetroundjoin%
\definecolor{currentfill}{rgb}{0.000000,0.000000,0.000000}%
\pgfsetfillcolor{currentfill}%
\pgfsetlinewidth{0.602250pt}%
\definecolor{currentstroke}{rgb}{0.000000,0.000000,0.000000}%
\pgfsetstrokecolor{currentstroke}%
\pgfsetdash{}{0pt}%
\pgfsys@defobject{currentmarker}{\pgfqpoint{-0.027778in}{0.000000in}}{\pgfqpoint{-0.000000in}{0.000000in}}{%
\pgfpathmoveto{\pgfqpoint{-0.000000in}{0.000000in}}%
\pgfpathlineto{\pgfqpoint{-0.027778in}{0.000000in}}%
\pgfusepath{stroke,fill}%
}%
\begin{pgfscope}%
\pgfsys@transformshift{0.588387in}{2.439956in}%
\pgfsys@useobject{currentmarker}{}%
\end{pgfscope}%
\end{pgfscope}%
\begin{pgfscope}%
\pgfsetbuttcap%
\pgfsetroundjoin%
\definecolor{currentfill}{rgb}{0.000000,0.000000,0.000000}%
\pgfsetfillcolor{currentfill}%
\pgfsetlinewidth{0.602250pt}%
\definecolor{currentstroke}{rgb}{0.000000,0.000000,0.000000}%
\pgfsetstrokecolor{currentstroke}%
\pgfsetdash{}{0pt}%
\pgfsys@defobject{currentmarker}{\pgfqpoint{-0.027778in}{0.000000in}}{\pgfqpoint{-0.000000in}{0.000000in}}{%
\pgfpathmoveto{\pgfqpoint{-0.000000in}{0.000000in}}%
\pgfpathlineto{\pgfqpoint{-0.027778in}{0.000000in}}%
\pgfusepath{stroke,fill}%
}%
\begin{pgfscope}%
\pgfsys@transformshift{0.588387in}{2.598211in}%
\pgfsys@useobject{currentmarker}{}%
\end{pgfscope}%
\end{pgfscope}%
\begin{pgfscope}%
\pgfsetbuttcap%
\pgfsetroundjoin%
\definecolor{currentfill}{rgb}{0.000000,0.000000,0.000000}%
\pgfsetfillcolor{currentfill}%
\pgfsetlinewidth{0.602250pt}%
\definecolor{currentstroke}{rgb}{0.000000,0.000000,0.000000}%
\pgfsetstrokecolor{currentstroke}%
\pgfsetdash{}{0pt}%
\pgfsys@defobject{currentmarker}{\pgfqpoint{-0.027778in}{0.000000in}}{\pgfqpoint{-0.000000in}{0.000000in}}{%
\pgfpathmoveto{\pgfqpoint{-0.000000in}{0.000000in}}%
\pgfpathlineto{\pgfqpoint{-0.027778in}{0.000000in}}%
\pgfusepath{stroke,fill}%
}%
\begin{pgfscope}%
\pgfsys@transformshift{0.588387in}{2.720962in}%
\pgfsys@useobject{currentmarker}{}%
\end{pgfscope}%
\end{pgfscope}%
\begin{pgfscope}%
\definecolor{textcolor}{rgb}{0.000000,0.000000,0.000000}%
\pgfsetstrokecolor{textcolor}%
\pgfsetfillcolor{textcolor}%
\pgftext[x=0.234413in,y=1.631726in,,bottom,rotate=90.000000]{\color{textcolor}{\rmfamily\fontsize{10.000000}{12.000000}\selectfont\catcode`\^=\active\def^{\ifmmode\sp\else\^{}\fi}\catcode`\%=\active\def%{\%}Time [ms]}}%
\end{pgfscope}%
\begin{pgfscope}%
\pgfpathrectangle{\pgfqpoint{0.588387in}{0.521603in}}{\pgfqpoint{4.669024in}{2.220246in}}%
\pgfusepath{clip}%
\pgfsetrectcap%
\pgfsetroundjoin%
\pgfsetlinewidth{1.505625pt}%
\pgfsetstrokecolor{currentstroke1}%
\pgfsetdash{}{0pt}%
\pgfpathmoveto{\pgfqpoint{0.800616in}{0.662737in}}%
\pgfpathlineto{\pgfqpoint{0.830507in}{0.741719in}}%
\pgfpathlineto{\pgfqpoint{0.860398in}{0.740483in}}%
\pgfpathlineto{\pgfqpoint{0.890290in}{0.828426in}}%
\pgfpathlineto{\pgfqpoint{0.920181in}{0.935959in}}%
\pgfpathlineto{\pgfqpoint{0.950072in}{1.077716in}}%
\pgfpathlineto{\pgfqpoint{0.979963in}{1.086115in}}%
\pgfpathlineto{\pgfqpoint{1.009855in}{1.163184in}}%
\pgfpathlineto{\pgfqpoint{1.039746in}{1.277070in}}%
\pgfpathlineto{\pgfqpoint{1.069637in}{1.152915in}}%
\pgfpathlineto{\pgfqpoint{1.099529in}{1.190484in}}%
\pgfpathlineto{\pgfqpoint{1.129420in}{1.185853in}}%
\pgfpathlineto{\pgfqpoint{1.159311in}{1.222042in}}%
\pgfpathlineto{\pgfqpoint{1.189203in}{1.131858in}}%
\pgfpathlineto{\pgfqpoint{1.219094in}{1.229507in}}%
\pgfpathlineto{\pgfqpoint{1.248985in}{1.261343in}}%
\pgfpathlineto{\pgfqpoint{1.278877in}{1.261339in}}%
\pgfpathlineto{\pgfqpoint{1.308768in}{1.245162in}}%
\pgfpathlineto{\pgfqpoint{1.338659in}{1.237934in}}%
\pgfpathlineto{\pgfqpoint{1.368551in}{1.232188in}}%
\pgfpathlineto{\pgfqpoint{1.398442in}{1.349545in}}%
\pgfpathlineto{\pgfqpoint{1.428333in}{1.352703in}}%
\pgfpathlineto{\pgfqpoint{1.458225in}{1.260209in}}%
\pgfpathlineto{\pgfqpoint{1.488116in}{1.339790in}}%
\pgfpathlineto{\pgfqpoint{1.518007in}{1.298705in}}%
\pgfpathlineto{\pgfqpoint{1.547899in}{1.418148in}}%
\pgfpathlineto{\pgfqpoint{1.577790in}{1.396110in}}%
\pgfpathlineto{\pgfqpoint{1.607681in}{1.397210in}}%
\pgfpathlineto{\pgfqpoint{1.637573in}{1.362743in}}%
\pgfpathlineto{\pgfqpoint{1.667464in}{1.498744in}}%
\pgfpathlineto{\pgfqpoint{1.697355in}{1.472102in}}%
\pgfpathlineto{\pgfqpoint{1.727246in}{1.479282in}}%
\pgfpathlineto{\pgfqpoint{1.757138in}{1.499448in}}%
\pgfpathlineto{\pgfqpoint{1.787029in}{1.484939in}}%
\pgfpathlineto{\pgfqpoint{1.816920in}{1.514796in}}%
\pgfpathlineto{\pgfqpoint{1.846812in}{1.493529in}}%
\pgfpathlineto{\pgfqpoint{1.876703in}{1.506085in}}%
\pgfpathlineto{\pgfqpoint{1.906594in}{1.540609in}}%
\pgfpathlineto{\pgfqpoint{1.936486in}{1.558644in}}%
\pgfpathlineto{\pgfqpoint{1.966377in}{1.583759in}}%
\pgfpathlineto{\pgfqpoint{1.996268in}{1.617152in}}%
\pgfpathlineto{\pgfqpoint{2.026160in}{1.578559in}}%
\pgfpathlineto{\pgfqpoint{2.056051in}{1.646336in}}%
\pgfpathlineto{\pgfqpoint{2.085942in}{1.616440in}}%
\pgfpathlineto{\pgfqpoint{2.115834in}{1.688941in}}%
\pgfpathlineto{\pgfqpoint{2.145725in}{1.667685in}}%
\pgfpathlineto{\pgfqpoint{2.175616in}{1.679755in}}%
\pgfpathlineto{\pgfqpoint{2.205508in}{1.703051in}}%
\pgfpathlineto{\pgfqpoint{2.235399in}{1.694923in}}%
\pgfpathlineto{\pgfqpoint{2.265290in}{1.792355in}}%
\pgfpathlineto{\pgfqpoint{2.295182in}{1.708021in}}%
\pgfpathlineto{\pgfqpoint{2.325073in}{1.786253in}}%
\pgfpathlineto{\pgfqpoint{2.354964in}{1.778412in}}%
\pgfpathlineto{\pgfqpoint{2.384855in}{1.767799in}}%
\pgfpathlineto{\pgfqpoint{2.414747in}{1.798888in}}%
\pgfpathlineto{\pgfqpoint{2.444638in}{1.810797in}}%
\pgfpathlineto{\pgfqpoint{2.474529in}{1.798989in}}%
\pgfpathlineto{\pgfqpoint{2.504421in}{1.801225in}}%
\pgfpathlineto{\pgfqpoint{2.534312in}{1.870143in}}%
\pgfpathlineto{\pgfqpoint{2.564203in}{1.858740in}}%
\pgfpathlineto{\pgfqpoint{2.594095in}{1.853385in}}%
\pgfpathlineto{\pgfqpoint{2.623986in}{1.870835in}}%
\pgfpathlineto{\pgfqpoint{2.653877in}{1.850336in}}%
\pgfpathlineto{\pgfqpoint{2.683769in}{1.918147in}}%
\pgfpathlineto{\pgfqpoint{2.713660in}{1.966057in}}%
\pgfpathlineto{\pgfqpoint{2.743551in}{1.971058in}}%
\pgfpathlineto{\pgfqpoint{2.773443in}{1.922868in}}%
\pgfpathlineto{\pgfqpoint{2.803334in}{1.983993in}}%
\pgfpathlineto{\pgfqpoint{2.833225in}{1.982784in}}%
\pgfpathlineto{\pgfqpoint{2.863117in}{1.985368in}}%
\pgfpathlineto{\pgfqpoint{2.893008in}{2.013562in}}%
\pgfpathlineto{\pgfqpoint{2.922899in}{2.025397in}}%
\pgfpathlineto{\pgfqpoint{2.952791in}{2.057480in}}%
\pgfpathlineto{\pgfqpoint{2.982682in}{2.038195in}}%
\pgfpathlineto{\pgfqpoint{3.012573in}{2.119973in}}%
\pgfpathlineto{\pgfqpoint{3.042464in}{2.090627in}}%
\pgfpathlineto{\pgfqpoint{3.072356in}{2.092980in}}%
\pgfpathlineto{\pgfqpoint{3.102247in}{2.126065in}}%
\pgfpathlineto{\pgfqpoint{3.132138in}{2.141893in}}%
\pgfpathlineto{\pgfqpoint{3.162030in}{2.226483in}}%
\pgfpathlineto{\pgfqpoint{3.191921in}{2.111314in}}%
\pgfpathlineto{\pgfqpoint{3.221812in}{2.176152in}}%
\pgfpathlineto{\pgfqpoint{3.251704in}{2.247241in}}%
\pgfpathlineto{\pgfqpoint{3.281595in}{2.179670in}}%
\pgfpathlineto{\pgfqpoint{3.311486in}{2.211274in}}%
\pgfpathlineto{\pgfqpoint{3.341378in}{2.236802in}}%
\pgfpathlineto{\pgfqpoint{3.371269in}{2.264936in}}%
\pgfpathlineto{\pgfqpoint{3.401160in}{2.288103in}}%
\pgfpathlineto{\pgfqpoint{3.431052in}{2.277475in}}%
\pgfpathlineto{\pgfqpoint{3.460943in}{2.365787in}}%
\pgfpathlineto{\pgfqpoint{3.490834in}{2.325395in}}%
\pgfpathlineto{\pgfqpoint{3.520726in}{2.322643in}}%
\pgfpathlineto{\pgfqpoint{3.550617in}{2.359263in}}%
\pgfpathlineto{\pgfqpoint{3.580508in}{2.376324in}}%
\pgfpathlineto{\pgfqpoint{3.610400in}{2.337257in}}%
\pgfpathlineto{\pgfqpoint{3.640291in}{2.412752in}}%
\pgfpathlineto{\pgfqpoint{3.670182in}{2.381059in}}%
\pgfpathlineto{\pgfqpoint{3.700073in}{2.374768in}}%
\pgfpathlineto{\pgfqpoint{3.729965in}{2.419145in}}%
\pgfpathlineto{\pgfqpoint{3.759856in}{2.422822in}}%
\pgfpathlineto{\pgfqpoint{3.789747in}{2.459209in}}%
\pgfpathlineto{\pgfqpoint{3.819639in}{2.422160in}}%
\pgfpathlineto{\pgfqpoint{3.849530in}{2.413474in}}%
\pgfpathlineto{\pgfqpoint{3.879421in}{2.422065in}}%
\pgfpathlineto{\pgfqpoint{3.909313in}{2.474025in}}%
\pgfpathlineto{\pgfqpoint{3.939204in}{2.456691in}}%
\pgfpathlineto{\pgfqpoint{3.969095in}{2.477518in}}%
\pgfpathlineto{\pgfqpoint{3.998987in}{2.470971in}}%
\pgfpathlineto{\pgfqpoint{4.028878in}{2.502626in}}%
\pgfpathlineto{\pgfqpoint{4.058769in}{2.531110in}}%
\pgfpathlineto{\pgfqpoint{4.088661in}{2.544362in}}%
\pgfpathlineto{\pgfqpoint{4.118552in}{2.570689in}}%
\pgfpathlineto{\pgfqpoint{4.148443in}{2.513641in}}%
\pgfpathlineto{\pgfqpoint{4.178335in}{2.613201in}}%
\pgfpathlineto{\pgfqpoint{4.208226in}{2.578517in}}%
\pgfpathlineto{\pgfqpoint{4.238117in}{2.493219in}}%
\pgfpathlineto{\pgfqpoint{4.268009in}{2.601194in}}%
\pgfpathlineto{\pgfqpoint{4.297900in}{2.599790in}}%
\pgfpathlineto{\pgfqpoint{4.327791in}{2.546632in}}%
\pgfpathlineto{\pgfqpoint{4.417465in}{2.593376in}}%
\pgfpathlineto{\pgfqpoint{4.566922in}{2.576899in}}%
\pgfusepath{stroke}%
\end{pgfscope}%
\begin{pgfscope}%
\pgfpathrectangle{\pgfqpoint{0.588387in}{0.521603in}}{\pgfqpoint{4.669024in}{2.220246in}}%
\pgfusepath{clip}%
\pgfsetrectcap%
\pgfsetroundjoin%
\pgfsetlinewidth{1.505625pt}%
\pgfsetstrokecolor{currentstroke2}%
\pgfsetdash{}{0pt}%
\pgfpathmoveto{\pgfqpoint{0.800616in}{0.659011in}}%
\pgfpathlineto{\pgfqpoint{0.830507in}{0.741033in}}%
\pgfpathlineto{\pgfqpoint{0.860398in}{0.741920in}}%
\pgfpathlineto{\pgfqpoint{0.890290in}{0.809879in}}%
\pgfpathlineto{\pgfqpoint{0.920181in}{0.934428in}}%
\pgfpathlineto{\pgfqpoint{0.950072in}{1.081082in}}%
\pgfpathlineto{\pgfqpoint{0.979963in}{1.087014in}}%
\pgfpathlineto{\pgfqpoint{1.009855in}{1.159019in}}%
\pgfpathlineto{\pgfqpoint{1.039746in}{1.276192in}}%
\pgfpathlineto{\pgfqpoint{1.069637in}{1.152905in}}%
\pgfpathlineto{\pgfqpoint{1.099529in}{1.191656in}}%
\pgfpathlineto{\pgfqpoint{1.129420in}{1.186320in}}%
\pgfpathlineto{\pgfqpoint{1.159311in}{1.211143in}}%
\pgfpathlineto{\pgfqpoint{1.189203in}{1.129471in}}%
\pgfpathlineto{\pgfqpoint{1.219094in}{1.230209in}}%
\pgfpathlineto{\pgfqpoint{1.248985in}{1.253656in}}%
\pgfpathlineto{\pgfqpoint{1.278877in}{1.259865in}}%
\pgfpathlineto{\pgfqpoint{1.308768in}{1.239637in}}%
\pgfpathlineto{\pgfqpoint{1.338659in}{1.226575in}}%
\pgfpathlineto{\pgfqpoint{1.368551in}{1.231460in}}%
\pgfpathlineto{\pgfqpoint{1.398442in}{1.340953in}}%
\pgfpathlineto{\pgfqpoint{1.428333in}{1.335114in}}%
\pgfpathlineto{\pgfqpoint{1.458225in}{1.258670in}}%
\pgfpathlineto{\pgfqpoint{1.488116in}{1.330456in}}%
\pgfpathlineto{\pgfqpoint{1.518007in}{1.280987in}}%
\pgfpathlineto{\pgfqpoint{1.547899in}{1.400661in}}%
\pgfpathlineto{\pgfqpoint{1.577790in}{1.389923in}}%
\pgfpathlineto{\pgfqpoint{1.607681in}{1.375522in}}%
\pgfpathlineto{\pgfqpoint{1.637573in}{1.354687in}}%
\pgfpathlineto{\pgfqpoint{1.667464in}{1.503298in}}%
\pgfpathlineto{\pgfqpoint{1.697355in}{1.454273in}}%
\pgfpathlineto{\pgfqpoint{1.727246in}{1.452653in}}%
\pgfpathlineto{\pgfqpoint{1.757138in}{1.481943in}}%
\pgfpathlineto{\pgfqpoint{1.787029in}{1.464803in}}%
\pgfpathlineto{\pgfqpoint{1.816920in}{1.490960in}}%
\pgfpathlineto{\pgfqpoint{1.846812in}{1.466055in}}%
\pgfpathlineto{\pgfqpoint{1.876703in}{1.476922in}}%
\pgfpathlineto{\pgfqpoint{1.906594in}{1.504923in}}%
\pgfpathlineto{\pgfqpoint{1.936486in}{1.537765in}}%
\pgfpathlineto{\pgfqpoint{1.966377in}{1.536793in}}%
\pgfpathlineto{\pgfqpoint{1.996268in}{1.585824in}}%
\pgfpathlineto{\pgfqpoint{2.026160in}{1.552726in}}%
\pgfpathlineto{\pgfqpoint{2.056051in}{1.607048in}}%
\pgfpathlineto{\pgfqpoint{2.085942in}{1.595761in}}%
\pgfpathlineto{\pgfqpoint{2.115834in}{1.652395in}}%
\pgfpathlineto{\pgfqpoint{2.145725in}{1.594783in}}%
\pgfpathlineto{\pgfqpoint{2.175616in}{1.654329in}}%
\pgfpathlineto{\pgfqpoint{2.205508in}{1.660606in}}%
\pgfpathlineto{\pgfqpoint{2.235399in}{1.636302in}}%
\pgfpathlineto{\pgfqpoint{2.265290in}{1.736893in}}%
\pgfpathlineto{\pgfqpoint{2.295182in}{1.660314in}}%
\pgfpathlineto{\pgfqpoint{2.325073in}{1.737214in}}%
\pgfpathlineto{\pgfqpoint{2.354964in}{1.734670in}}%
\pgfpathlineto{\pgfqpoint{2.384855in}{1.700599in}}%
\pgfpathlineto{\pgfqpoint{2.414747in}{1.749432in}}%
\pgfpathlineto{\pgfqpoint{2.444638in}{1.753726in}}%
\pgfpathlineto{\pgfqpoint{2.474529in}{1.770781in}}%
\pgfpathlineto{\pgfqpoint{2.504421in}{1.755894in}}%
\pgfpathlineto{\pgfqpoint{2.534312in}{1.827526in}}%
\pgfpathlineto{\pgfqpoint{2.564203in}{1.793753in}}%
\pgfpathlineto{\pgfqpoint{2.594095in}{1.790897in}}%
\pgfpathlineto{\pgfqpoint{2.623986in}{1.810797in}}%
\pgfpathlineto{\pgfqpoint{2.653877in}{1.785209in}}%
\pgfpathlineto{\pgfqpoint{2.683769in}{1.850840in}}%
\pgfpathlineto{\pgfqpoint{2.713660in}{1.882975in}}%
\pgfpathlineto{\pgfqpoint{2.743551in}{1.901889in}}%
\pgfpathlineto{\pgfqpoint{2.773443in}{1.850906in}}%
\pgfpathlineto{\pgfqpoint{2.803334in}{1.901530in}}%
\pgfpathlineto{\pgfqpoint{2.833225in}{1.947844in}}%
\pgfpathlineto{\pgfqpoint{2.863117in}{1.923652in}}%
\pgfpathlineto{\pgfqpoint{2.922899in}{1.957821in}}%
\pgfpathlineto{\pgfqpoint{2.952791in}{1.972136in}}%
\pgfpathlineto{\pgfqpoint{2.982682in}{1.955348in}}%
\pgfpathlineto{\pgfqpoint{3.012573in}{2.057737in}}%
\pgfpathlineto{\pgfqpoint{3.042464in}{2.016362in}}%
\pgfpathlineto{\pgfqpoint{3.072356in}{2.029727in}}%
\pgfpathlineto{\pgfqpoint{3.102247in}{2.040915in}}%
\pgfpathlineto{\pgfqpoint{3.132138in}{2.063101in}}%
\pgfpathlineto{\pgfqpoint{3.162030in}{2.141546in}}%
\pgfpathlineto{\pgfqpoint{3.191921in}{2.054952in}}%
\pgfpathlineto{\pgfqpoint{3.221812in}{2.087197in}}%
\pgfpathlineto{\pgfqpoint{3.251704in}{2.141248in}}%
\pgfpathlineto{\pgfqpoint{3.281595in}{2.108046in}}%
\pgfpathlineto{\pgfqpoint{3.311486in}{2.135808in}}%
\pgfpathlineto{\pgfqpoint{3.341378in}{2.133226in}}%
\pgfpathlineto{\pgfqpoint{3.371269in}{2.174777in}}%
\pgfpathlineto{\pgfqpoint{3.401160in}{2.180885in}}%
\pgfpathlineto{\pgfqpoint{3.431052in}{2.161660in}}%
\pgfpathlineto{\pgfqpoint{3.460943in}{2.232207in}}%
\pgfpathlineto{\pgfqpoint{3.490834in}{2.240718in}}%
\pgfpathlineto{\pgfqpoint{3.520726in}{2.235077in}}%
\pgfpathlineto{\pgfqpoint{3.550617in}{2.253733in}}%
\pgfpathlineto{\pgfqpoint{3.580508in}{2.282606in}}%
\pgfpathlineto{\pgfqpoint{3.610400in}{2.234077in}}%
\pgfpathlineto{\pgfqpoint{3.640291in}{2.329220in}}%
\pgfpathlineto{\pgfqpoint{3.670182in}{2.304733in}}%
\pgfpathlineto{\pgfqpoint{3.700073in}{2.356562in}}%
\pgfpathlineto{\pgfqpoint{3.759856in}{2.318235in}}%
\pgfpathlineto{\pgfqpoint{3.789747in}{2.405479in}}%
\pgfpathlineto{\pgfqpoint{3.819639in}{2.335104in}}%
\pgfpathlineto{\pgfqpoint{3.879421in}{2.371881in}}%
\pgfpathlineto{\pgfqpoint{3.909313in}{2.396574in}}%
\pgfpathlineto{\pgfqpoint{3.939204in}{2.430124in}}%
\pgfpathlineto{\pgfqpoint{3.969095in}{2.395282in}}%
\pgfpathlineto{\pgfqpoint{3.998987in}{2.372648in}}%
\pgfpathlineto{\pgfqpoint{4.028878in}{2.434057in}}%
\pgfpathlineto{\pgfqpoint{4.058769in}{2.478074in}}%
\pgfpathlineto{\pgfqpoint{4.088661in}{2.467341in}}%
\pgfpathlineto{\pgfqpoint{4.118552in}{2.504613in}}%
\pgfpathlineto{\pgfqpoint{4.148443in}{2.530423in}}%
\pgfpathlineto{\pgfqpoint{4.178335in}{2.456039in}}%
\pgfpathlineto{\pgfqpoint{4.208226in}{2.535687in}}%
\pgfpathlineto{\pgfqpoint{4.238117in}{2.532558in}}%
\pgfpathlineto{\pgfqpoint{4.268009in}{2.542520in}}%
\pgfpathlineto{\pgfqpoint{4.297900in}{2.582376in}}%
\pgfpathlineto{\pgfqpoint{4.327791in}{2.525278in}}%
\pgfpathlineto{\pgfqpoint{4.357683in}{2.530151in}}%
\pgfpathlineto{\pgfqpoint{4.387574in}{2.603752in}}%
\pgfpathlineto{\pgfqpoint{4.417465in}{2.575668in}}%
\pgfpathlineto{\pgfqpoint{4.447356in}{2.582730in}}%
\pgfpathlineto{\pgfqpoint{4.477248in}{2.587868in}}%
\pgfpathlineto{\pgfqpoint{4.507139in}{2.589431in}}%
\pgfpathlineto{\pgfqpoint{4.537030in}{2.598554in}}%
\pgfpathlineto{\pgfqpoint{4.566922in}{2.559840in}}%
\pgfpathlineto{\pgfqpoint{4.656596in}{2.583295in}}%
\pgfpathlineto{\pgfqpoint{4.716378in}{2.613937in}}%
\pgfpathlineto{\pgfqpoint{4.716378in}{2.613937in}}%
\pgfusepath{stroke}%
\end{pgfscope}%
\begin{pgfscope}%
\pgfpathrectangle{\pgfqpoint{0.588387in}{0.521603in}}{\pgfqpoint{4.669024in}{2.220246in}}%
\pgfusepath{clip}%
\pgfsetrectcap%
\pgfsetroundjoin%
\pgfsetlinewidth{1.505625pt}%
\pgfsetstrokecolor{currentstroke3}%
\pgfsetdash{}{0pt}%
\pgfpathmoveto{\pgfqpoint{0.800616in}{0.627020in}}%
\pgfpathlineto{\pgfqpoint{0.830507in}{0.720599in}}%
\pgfpathlineto{\pgfqpoint{0.860398in}{0.739966in}}%
\pgfpathlineto{\pgfqpoint{0.890290in}{0.821070in}}%
\pgfpathlineto{\pgfqpoint{0.920181in}{0.936462in}}%
\pgfpathlineto{\pgfqpoint{0.950072in}{1.069519in}}%
\pgfpathlineto{\pgfqpoint{0.979963in}{1.108580in}}%
\pgfpathlineto{\pgfqpoint{1.009855in}{1.210215in}}%
\pgfpathlineto{\pgfqpoint{1.039746in}{1.360391in}}%
\pgfpathlineto{\pgfqpoint{1.069637in}{1.224207in}}%
\pgfpathlineto{\pgfqpoint{1.099529in}{1.286297in}}%
\pgfpathlineto{\pgfqpoint{1.129420in}{1.331434in}}%
\pgfpathlineto{\pgfqpoint{1.159311in}{1.316520in}}%
\pgfpathlineto{\pgfqpoint{1.189203in}{1.243551in}}%
\pgfpathlineto{\pgfqpoint{1.219094in}{1.330232in}}%
\pgfpathlineto{\pgfqpoint{1.248985in}{1.367739in}}%
\pgfpathlineto{\pgfqpoint{1.278877in}{1.353423in}}%
\pgfpathlineto{\pgfqpoint{1.308768in}{1.350711in}}%
\pgfpathlineto{\pgfqpoint{1.338659in}{1.306006in}}%
\pgfpathlineto{\pgfqpoint{1.368551in}{1.336671in}}%
\pgfpathlineto{\pgfqpoint{1.398442in}{1.526905in}}%
\pgfpathlineto{\pgfqpoint{1.428333in}{1.452953in}}%
\pgfpathlineto{\pgfqpoint{1.458225in}{1.360987in}}%
\pgfpathlineto{\pgfqpoint{1.488116in}{1.440745in}}%
\pgfpathlineto{\pgfqpoint{1.518007in}{1.407958in}}%
\pgfpathlineto{\pgfqpoint{1.547899in}{1.570857in}}%
\pgfpathlineto{\pgfqpoint{1.577790in}{1.522009in}}%
\pgfpathlineto{\pgfqpoint{1.607681in}{1.522643in}}%
\pgfpathlineto{\pgfqpoint{1.637573in}{1.465689in}}%
\pgfpathlineto{\pgfqpoint{1.667464in}{1.603139in}}%
\pgfpathlineto{\pgfqpoint{1.697355in}{1.566430in}}%
\pgfpathlineto{\pgfqpoint{1.727246in}{1.589508in}}%
\pgfpathlineto{\pgfqpoint{1.757138in}{1.601488in}}%
\pgfpathlineto{\pgfqpoint{1.787029in}{1.604488in}}%
\pgfpathlineto{\pgfqpoint{1.816920in}{1.639204in}}%
\pgfpathlineto{\pgfqpoint{1.846812in}{1.572980in}}%
\pgfpathlineto{\pgfqpoint{1.876703in}{1.578890in}}%
\pgfpathlineto{\pgfqpoint{1.906594in}{1.626278in}}%
\pgfpathlineto{\pgfqpoint{1.936486in}{1.645602in}}%
\pgfpathlineto{\pgfqpoint{1.966377in}{1.692411in}}%
\pgfpathlineto{\pgfqpoint{1.996268in}{1.713640in}}%
\pgfpathlineto{\pgfqpoint{2.026160in}{1.682197in}}%
\pgfpathlineto{\pgfqpoint{2.056051in}{1.731040in}}%
\pgfpathlineto{\pgfqpoint{2.085942in}{1.714489in}}%
\pgfpathlineto{\pgfqpoint{2.115834in}{1.794402in}}%
\pgfpathlineto{\pgfqpoint{2.145725in}{1.705487in}}%
\pgfpathlineto{\pgfqpoint{2.175616in}{1.758937in}}%
\pgfpathlineto{\pgfqpoint{2.235399in}{1.737284in}}%
\pgfpathlineto{\pgfqpoint{2.265290in}{1.853205in}}%
\pgfpathlineto{\pgfqpoint{2.295182in}{1.780653in}}%
\pgfpathlineto{\pgfqpoint{2.325073in}{1.805484in}}%
\pgfpathlineto{\pgfqpoint{2.354964in}{1.820233in}}%
\pgfpathlineto{\pgfqpoint{2.384855in}{1.809586in}}%
\pgfpathlineto{\pgfqpoint{2.414747in}{1.858850in}}%
\pgfpathlineto{\pgfqpoint{2.444638in}{1.852246in}}%
\pgfpathlineto{\pgfqpoint{2.474529in}{1.818835in}}%
\pgfpathlineto{\pgfqpoint{2.504421in}{1.863245in}}%
\pgfpathlineto{\pgfqpoint{2.534312in}{1.921381in}}%
\pgfpathlineto{\pgfqpoint{2.564203in}{1.900015in}}%
\pgfpathlineto{\pgfqpoint{2.594095in}{1.899267in}}%
\pgfpathlineto{\pgfqpoint{2.623986in}{1.945070in}}%
\pgfpathlineto{\pgfqpoint{2.653877in}{1.855824in}}%
\pgfpathlineto{\pgfqpoint{2.683769in}{1.925593in}}%
\pgfpathlineto{\pgfqpoint{2.713660in}{2.015637in}}%
\pgfpathlineto{\pgfqpoint{2.743551in}{1.980816in}}%
\pgfpathlineto{\pgfqpoint{2.773443in}{1.924649in}}%
\pgfpathlineto{\pgfqpoint{2.803334in}{1.967816in}}%
\pgfpathlineto{\pgfqpoint{2.833225in}{2.040190in}}%
\pgfpathlineto{\pgfqpoint{2.863117in}{1.966162in}}%
\pgfpathlineto{\pgfqpoint{2.893008in}{2.029548in}}%
\pgfpathlineto{\pgfqpoint{2.922899in}{2.039861in}}%
\pgfpathlineto{\pgfqpoint{2.952791in}{2.069608in}}%
\pgfpathlineto{\pgfqpoint{2.982682in}{2.070888in}}%
\pgfpathlineto{\pgfqpoint{3.012573in}{2.070692in}}%
\pgfpathlineto{\pgfqpoint{3.042464in}{2.077937in}}%
\pgfpathlineto{\pgfqpoint{3.072356in}{2.070014in}}%
\pgfpathlineto{\pgfqpoint{3.102247in}{2.095048in}}%
\pgfpathlineto{\pgfqpoint{3.132138in}{2.103260in}}%
\pgfpathlineto{\pgfqpoint{3.162030in}{2.171169in}}%
\pgfpathlineto{\pgfqpoint{3.191921in}{2.107872in}}%
\pgfpathlineto{\pgfqpoint{3.221812in}{2.166666in}}%
\pgfpathlineto{\pgfqpoint{3.251704in}{2.219066in}}%
\pgfpathlineto{\pgfqpoint{3.281595in}{2.144100in}}%
\pgfpathlineto{\pgfqpoint{3.311486in}{2.183622in}}%
\pgfpathlineto{\pgfqpoint{3.341378in}{2.186755in}}%
\pgfpathlineto{\pgfqpoint{3.371269in}{2.207935in}}%
\pgfpathlineto{\pgfqpoint{3.401160in}{2.219039in}}%
\pgfpathlineto{\pgfqpoint{3.431052in}{2.254105in}}%
\pgfpathlineto{\pgfqpoint{3.460943in}{2.253547in}}%
\pgfpathlineto{\pgfqpoint{3.490834in}{2.270900in}}%
\pgfpathlineto{\pgfqpoint{3.520726in}{2.269485in}}%
\pgfpathlineto{\pgfqpoint{3.550617in}{2.250606in}}%
\pgfpathlineto{\pgfqpoint{3.580508in}{2.309751in}}%
\pgfpathlineto{\pgfqpoint{3.610400in}{2.289909in}}%
\pgfpathlineto{\pgfqpoint{3.640291in}{2.308891in}}%
\pgfpathlineto{\pgfqpoint{3.670182in}{2.298119in}}%
\pgfpathlineto{\pgfqpoint{3.700073in}{2.326351in}}%
\pgfpathlineto{\pgfqpoint{3.729965in}{2.349995in}}%
\pgfpathlineto{\pgfqpoint{3.759856in}{2.355532in}}%
\pgfpathlineto{\pgfqpoint{3.789747in}{2.386833in}}%
\pgfpathlineto{\pgfqpoint{3.819639in}{2.352116in}}%
\pgfpathlineto{\pgfqpoint{3.849530in}{2.355269in}}%
\pgfpathlineto{\pgfqpoint{3.879421in}{2.382850in}}%
\pgfpathlineto{\pgfqpoint{3.909313in}{2.422867in}}%
\pgfpathlineto{\pgfqpoint{3.939204in}{2.411747in}}%
\pgfpathlineto{\pgfqpoint{3.969095in}{2.394733in}}%
\pgfpathlineto{\pgfqpoint{3.998987in}{2.442287in}}%
\pgfpathlineto{\pgfqpoint{4.028878in}{2.428137in}}%
\pgfpathlineto{\pgfqpoint{4.058769in}{2.476169in}}%
\pgfpathlineto{\pgfqpoint{4.088661in}{2.466897in}}%
\pgfpathlineto{\pgfqpoint{4.118552in}{2.501862in}}%
\pgfpathlineto{\pgfqpoint{4.148443in}{2.498036in}}%
\pgfpathlineto{\pgfqpoint{4.178335in}{2.471289in}}%
\pgfpathlineto{\pgfqpoint{4.208226in}{2.528592in}}%
\pgfpathlineto{\pgfqpoint{4.238117in}{2.516271in}}%
\pgfpathlineto{\pgfqpoint{4.268009in}{2.540383in}}%
\pgfpathlineto{\pgfqpoint{4.297900in}{2.517827in}}%
\pgfpathlineto{\pgfqpoint{4.327791in}{2.524863in}}%
\pgfpathlineto{\pgfqpoint{4.357683in}{2.529831in}}%
\pgfpathlineto{\pgfqpoint{4.387574in}{2.571584in}}%
\pgfpathlineto{\pgfqpoint{4.417465in}{2.566829in}}%
\pgfpathlineto{\pgfqpoint{4.447356in}{2.558677in}}%
\pgfpathlineto{\pgfqpoint{4.477248in}{2.552028in}}%
\pgfpathlineto{\pgfqpoint{4.507139in}{2.592178in}}%
\pgfpathlineto{\pgfqpoint{4.537030in}{2.597075in}}%
\pgfpathlineto{\pgfqpoint{4.566922in}{2.573152in}}%
\pgfpathlineto{\pgfqpoint{4.596813in}{2.599790in}}%
\pgfpathlineto{\pgfqpoint{4.626704in}{2.611190in}}%
\pgfpathlineto{\pgfqpoint{4.656596in}{2.568109in}}%
\pgfpathlineto{\pgfqpoint{4.686487in}{2.604410in}}%
\pgfpathlineto{\pgfqpoint{4.716378in}{2.592404in}}%
\pgfpathlineto{\pgfqpoint{4.746270in}{2.582163in}}%
\pgfpathlineto{\pgfqpoint{4.776161in}{2.584988in}}%
\pgfpathlineto{\pgfqpoint{4.806052in}{2.609306in}}%
\pgfpathlineto{\pgfqpoint{4.835944in}{2.622424in}}%
\pgfpathlineto{\pgfqpoint{4.895726in}{2.634336in}}%
\pgfpathlineto{\pgfqpoint{4.955509in}{2.633240in}}%
\pgfpathlineto{\pgfqpoint{4.955509in}{2.633240in}}%
\pgfusepath{stroke}%
\end{pgfscope}%
\begin{pgfscope}%
\pgfpathrectangle{\pgfqpoint{0.588387in}{0.521603in}}{\pgfqpoint{4.669024in}{2.220246in}}%
\pgfusepath{clip}%
\pgfsetrectcap%
\pgfsetroundjoin%
\pgfsetlinewidth{1.505625pt}%
\pgfsetstrokecolor{currentstroke4}%
\pgfsetdash{}{0pt}%
\pgfpathmoveto{\pgfqpoint{0.800616in}{0.631740in}}%
\pgfpathlineto{\pgfqpoint{0.830507in}{0.728323in}}%
\pgfpathlineto{\pgfqpoint{0.860398in}{0.738461in}}%
\pgfpathlineto{\pgfqpoint{0.920181in}{0.938198in}}%
\pgfpathlineto{\pgfqpoint{0.950072in}{1.079547in}}%
\pgfpathlineto{\pgfqpoint{0.979963in}{1.085565in}}%
\pgfpathlineto{\pgfqpoint{1.009855in}{1.162623in}}%
\pgfpathlineto{\pgfqpoint{1.039746in}{1.280153in}}%
\pgfpathlineto{\pgfqpoint{1.069637in}{1.150978in}}%
\pgfpathlineto{\pgfqpoint{1.099529in}{1.196132in}}%
\pgfpathlineto{\pgfqpoint{1.129420in}{1.189221in}}%
\pgfpathlineto{\pgfqpoint{1.159311in}{1.226230in}}%
\pgfpathlineto{\pgfqpoint{1.189203in}{1.140158in}}%
\pgfpathlineto{\pgfqpoint{1.219094in}{1.240380in}}%
\pgfpathlineto{\pgfqpoint{1.248985in}{1.258739in}}%
\pgfpathlineto{\pgfqpoint{1.278877in}{1.273883in}}%
\pgfpathlineto{\pgfqpoint{1.308768in}{1.251847in}}%
\pgfpathlineto{\pgfqpoint{1.338659in}{1.227845in}}%
\pgfpathlineto{\pgfqpoint{1.368551in}{1.232574in}}%
\pgfpathlineto{\pgfqpoint{1.398442in}{1.342753in}}%
\pgfpathlineto{\pgfqpoint{1.428333in}{1.347944in}}%
\pgfpathlineto{\pgfqpoint{1.458225in}{1.261815in}}%
\pgfpathlineto{\pgfqpoint{1.488116in}{1.349403in}}%
\pgfpathlineto{\pgfqpoint{1.518007in}{1.295730in}}%
\pgfpathlineto{\pgfqpoint{1.547899in}{1.416871in}}%
\pgfpathlineto{\pgfqpoint{1.577790in}{1.405047in}}%
\pgfpathlineto{\pgfqpoint{1.637573in}{1.352994in}}%
\pgfpathlineto{\pgfqpoint{1.667464in}{1.505455in}}%
\pgfpathlineto{\pgfqpoint{1.697355in}{1.461609in}}%
\pgfpathlineto{\pgfqpoint{1.727246in}{1.460193in}}%
\pgfpathlineto{\pgfqpoint{1.757138in}{1.493670in}}%
\pgfpathlineto{\pgfqpoint{1.787029in}{1.472119in}}%
\pgfpathlineto{\pgfqpoint{1.816920in}{1.509104in}}%
\pgfpathlineto{\pgfqpoint{1.846812in}{1.474402in}}%
\pgfpathlineto{\pgfqpoint{1.876703in}{1.485453in}}%
\pgfpathlineto{\pgfqpoint{1.906594in}{1.500092in}}%
\pgfpathlineto{\pgfqpoint{1.936486in}{1.549329in}}%
\pgfpathlineto{\pgfqpoint{1.966377in}{1.540603in}}%
\pgfpathlineto{\pgfqpoint{1.996268in}{1.582365in}}%
\pgfpathlineto{\pgfqpoint{2.026160in}{1.552477in}}%
\pgfpathlineto{\pgfqpoint{2.056051in}{1.613062in}}%
\pgfpathlineto{\pgfqpoint{2.085942in}{1.590457in}}%
\pgfpathlineto{\pgfqpoint{2.115834in}{1.656524in}}%
\pgfpathlineto{\pgfqpoint{2.145725in}{1.594897in}}%
\pgfpathlineto{\pgfqpoint{2.175616in}{1.657009in}}%
\pgfpathlineto{\pgfqpoint{2.205508in}{1.661244in}}%
\pgfpathlineto{\pgfqpoint{2.235399in}{1.645068in}}%
\pgfpathlineto{\pgfqpoint{2.265290in}{1.760917in}}%
\pgfpathlineto{\pgfqpoint{2.295182in}{1.660787in}}%
\pgfpathlineto{\pgfqpoint{2.325073in}{1.739286in}}%
\pgfpathlineto{\pgfqpoint{2.354964in}{1.747014in}}%
\pgfpathlineto{\pgfqpoint{2.384855in}{1.718835in}}%
\pgfpathlineto{\pgfqpoint{2.414747in}{1.763617in}}%
\pgfpathlineto{\pgfqpoint{2.444638in}{1.752170in}}%
\pgfpathlineto{\pgfqpoint{2.474529in}{1.749260in}}%
\pgfpathlineto{\pgfqpoint{2.504421in}{1.748144in}}%
\pgfpathlineto{\pgfqpoint{2.534312in}{1.832666in}}%
\pgfpathlineto{\pgfqpoint{2.564203in}{1.793722in}}%
\pgfpathlineto{\pgfqpoint{2.594095in}{1.790486in}}%
\pgfpathlineto{\pgfqpoint{2.623986in}{1.823502in}}%
\pgfpathlineto{\pgfqpoint{2.653877in}{1.797244in}}%
\pgfpathlineto{\pgfqpoint{2.683769in}{1.851765in}}%
\pgfpathlineto{\pgfqpoint{2.713660in}{1.869733in}}%
\pgfpathlineto{\pgfqpoint{2.743551in}{1.925034in}}%
\pgfpathlineto{\pgfqpoint{2.773443in}{1.848439in}}%
\pgfpathlineto{\pgfqpoint{2.803334in}{1.906328in}}%
\pgfpathlineto{\pgfqpoint{2.833225in}{1.928945in}}%
\pgfpathlineto{\pgfqpoint{2.863117in}{1.931190in}}%
\pgfpathlineto{\pgfqpoint{2.893008in}{1.957478in}}%
\pgfpathlineto{\pgfqpoint{2.922899in}{1.932018in}}%
\pgfpathlineto{\pgfqpoint{2.952791in}{1.968173in}}%
\pgfpathlineto{\pgfqpoint{2.982682in}{1.945155in}}%
\pgfpathlineto{\pgfqpoint{3.012573in}{2.055575in}}%
\pgfpathlineto{\pgfqpoint{3.042464in}{2.018594in}}%
\pgfpathlineto{\pgfqpoint{3.072356in}{2.016550in}}%
\pgfpathlineto{\pgfqpoint{3.102247in}{2.038410in}}%
\pgfpathlineto{\pgfqpoint{3.132138in}{2.065784in}}%
\pgfpathlineto{\pgfqpoint{3.162030in}{2.146623in}}%
\pgfpathlineto{\pgfqpoint{3.191921in}{2.041946in}}%
\pgfpathlineto{\pgfqpoint{3.221812in}{2.091359in}}%
\pgfpathlineto{\pgfqpoint{3.251704in}{2.143925in}}%
\pgfpathlineto{\pgfqpoint{3.281595in}{2.094657in}}%
\pgfpathlineto{\pgfqpoint{3.311486in}{2.137912in}}%
\pgfpathlineto{\pgfqpoint{3.341378in}{2.142621in}}%
\pgfpathlineto{\pgfqpoint{3.371269in}{2.180390in}}%
\pgfpathlineto{\pgfqpoint{3.431052in}{2.183391in}}%
\pgfpathlineto{\pgfqpoint{3.460943in}{2.283313in}}%
\pgfpathlineto{\pgfqpoint{3.490834in}{2.224265in}}%
\pgfpathlineto{\pgfqpoint{3.520726in}{2.233108in}}%
\pgfpathlineto{\pgfqpoint{3.580508in}{2.287684in}}%
\pgfpathlineto{\pgfqpoint{3.610400in}{2.235275in}}%
\pgfpathlineto{\pgfqpoint{3.640291in}{2.272083in}}%
\pgfpathlineto{\pgfqpoint{3.670182in}{2.306675in}}%
\pgfpathlineto{\pgfqpoint{3.700073in}{2.411032in}}%
\pgfpathlineto{\pgfqpoint{3.729965in}{2.333521in}}%
\pgfpathlineto{\pgfqpoint{3.759856in}{2.334465in}}%
\pgfpathlineto{\pgfqpoint{3.789747in}{2.393913in}}%
\pgfpathlineto{\pgfqpoint{3.819639in}{2.347092in}}%
\pgfpathlineto{\pgfqpoint{3.849530in}{2.327018in}}%
\pgfpathlineto{\pgfqpoint{3.879421in}{2.365703in}}%
\pgfpathlineto{\pgfqpoint{3.939204in}{2.435564in}}%
\pgfpathlineto{\pgfqpoint{3.969095in}{2.391330in}}%
\pgfpathlineto{\pgfqpoint{3.998987in}{2.364506in}}%
\pgfpathlineto{\pgfqpoint{4.028878in}{2.406633in}}%
\pgfpathlineto{\pgfqpoint{4.058769in}{2.474556in}}%
\pgfpathlineto{\pgfqpoint{4.088661in}{2.449307in}}%
\pgfpathlineto{\pgfqpoint{4.118552in}{2.512150in}}%
\pgfpathlineto{\pgfqpoint{4.148443in}{2.527616in}}%
\pgfpathlineto{\pgfqpoint{4.178335in}{2.519861in}}%
\pgfpathlineto{\pgfqpoint{4.208226in}{2.508053in}}%
\pgfpathlineto{\pgfqpoint{4.238117in}{2.498108in}}%
\pgfpathlineto{\pgfqpoint{4.268009in}{2.554481in}}%
\pgfpathlineto{\pgfqpoint{4.297900in}{2.552492in}}%
\pgfpathlineto{\pgfqpoint{4.327791in}{2.538721in}}%
\pgfpathlineto{\pgfqpoint{4.357683in}{2.507351in}}%
\pgfpathlineto{\pgfqpoint{4.387574in}{2.608699in}}%
\pgfpathlineto{\pgfqpoint{4.417465in}{2.563404in}}%
\pgfpathlineto{\pgfqpoint{4.447356in}{2.561729in}}%
\pgfpathlineto{\pgfqpoint{4.477248in}{2.626620in}}%
\pgfpathlineto{\pgfqpoint{4.507139in}{2.594946in}}%
\pgfpathlineto{\pgfqpoint{4.566922in}{2.573314in}}%
\pgfpathlineto{\pgfqpoint{4.596813in}{2.630999in}}%
\pgfpathlineto{\pgfqpoint{4.656596in}{2.612331in}}%
\pgfpathlineto{\pgfqpoint{4.716378in}{2.636970in}}%
\pgfpathlineto{\pgfqpoint{4.835944in}{2.640929in}}%
\pgfpathlineto{\pgfqpoint{4.835944in}{2.640929in}}%
\pgfusepath{stroke}%
\end{pgfscope}%
\begin{pgfscope}%
\pgfpathrectangle{\pgfqpoint{0.588387in}{0.521603in}}{\pgfqpoint{4.669024in}{2.220246in}}%
\pgfusepath{clip}%
\pgfsetrectcap%
\pgfsetroundjoin%
\pgfsetlinewidth{1.505625pt}%
\pgfsetstrokecolor{currentstroke5}%
\pgfsetdash{}{0pt}%
\pgfpathmoveto{\pgfqpoint{0.800616in}{0.636327in}}%
\pgfpathlineto{\pgfqpoint{0.830507in}{0.721367in}}%
\pgfpathlineto{\pgfqpoint{0.860398in}{0.729949in}}%
\pgfpathlineto{\pgfqpoint{0.890290in}{0.815942in}}%
\pgfpathlineto{\pgfqpoint{0.920181in}{0.931623in}}%
\pgfpathlineto{\pgfqpoint{0.950072in}{1.070624in}}%
\pgfpathlineto{\pgfqpoint{0.979963in}{1.055670in}}%
\pgfpathlineto{\pgfqpoint{1.009855in}{1.105182in}}%
\pgfpathlineto{\pgfqpoint{1.039746in}{1.155649in}}%
\pgfpathlineto{\pgfqpoint{1.069637in}{1.071988in}}%
\pgfpathlineto{\pgfqpoint{1.099529in}{1.074788in}}%
\pgfpathlineto{\pgfqpoint{1.129420in}{0.996189in}}%
\pgfpathlineto{\pgfqpoint{1.159311in}{1.105632in}}%
\pgfpathlineto{\pgfqpoint{1.189203in}{1.032548in}}%
\pgfpathlineto{\pgfqpoint{1.219094in}{1.119784in}}%
\pgfpathlineto{\pgfqpoint{1.248985in}{1.136868in}}%
\pgfpathlineto{\pgfqpoint{1.278877in}{1.168670in}}%
\pgfpathlineto{\pgfqpoint{1.308768in}{1.133479in}}%
\pgfpathlineto{\pgfqpoint{1.338659in}{1.133499in}}%
\pgfpathlineto{\pgfqpoint{1.368551in}{1.110853in}}%
\pgfpathlineto{\pgfqpoint{1.398442in}{1.162489in}}%
\pgfpathlineto{\pgfqpoint{1.428333in}{1.232625in}}%
\pgfpathlineto{\pgfqpoint{1.458225in}{1.134333in}}%
\pgfpathlineto{\pgfqpoint{1.488116in}{1.219269in}}%
\pgfpathlineto{\pgfqpoint{1.518007in}{1.209406in}}%
\pgfpathlineto{\pgfqpoint{1.547899in}{1.267737in}}%
\pgfpathlineto{\pgfqpoint{1.577790in}{1.238780in}}%
\pgfpathlineto{\pgfqpoint{1.607681in}{1.230823in}}%
\pgfpathlineto{\pgfqpoint{1.637573in}{1.249398in}}%
\pgfpathlineto{\pgfqpoint{1.667464in}{1.373458in}}%
\pgfpathlineto{\pgfqpoint{1.697355in}{1.335611in}}%
\pgfpathlineto{\pgfqpoint{1.727246in}{1.363746in}}%
\pgfpathlineto{\pgfqpoint{1.757138in}{1.403823in}}%
\pgfpathlineto{\pgfqpoint{1.787029in}{1.322238in}}%
\pgfpathlineto{\pgfqpoint{1.816920in}{1.394746in}}%
\pgfpathlineto{\pgfqpoint{1.846812in}{1.408349in}}%
\pgfpathlineto{\pgfqpoint{1.876703in}{1.394191in}}%
\pgfpathlineto{\pgfqpoint{1.906594in}{1.331387in}}%
\pgfpathlineto{\pgfqpoint{1.936486in}{1.425647in}}%
\pgfpathlineto{\pgfqpoint{1.966377in}{1.428487in}}%
\pgfpathlineto{\pgfqpoint{1.996268in}{1.492661in}}%
\pgfpathlineto{\pgfqpoint{2.026160in}{1.461661in}}%
\pgfpathlineto{\pgfqpoint{2.056051in}{1.485321in}}%
\pgfpathlineto{\pgfqpoint{2.085942in}{1.450186in}}%
\pgfpathlineto{\pgfqpoint{2.115834in}{1.493293in}}%
\pgfpathlineto{\pgfqpoint{2.145725in}{1.531696in}}%
\pgfpathlineto{\pgfqpoint{2.175616in}{1.512169in}}%
\pgfpathlineto{\pgfqpoint{2.205508in}{1.581528in}}%
\pgfpathlineto{\pgfqpoint{2.235399in}{1.577962in}}%
\pgfpathlineto{\pgfqpoint{2.265290in}{1.592178in}}%
\pgfpathlineto{\pgfqpoint{2.295182in}{1.564933in}}%
\pgfpathlineto{\pgfqpoint{2.325073in}{1.678574in}}%
\pgfpathlineto{\pgfqpoint{2.354964in}{1.651063in}}%
\pgfpathlineto{\pgfqpoint{2.384855in}{1.637005in}}%
\pgfpathlineto{\pgfqpoint{2.414747in}{1.727253in}}%
\pgfpathlineto{\pgfqpoint{2.444638in}{1.705651in}}%
\pgfpathlineto{\pgfqpoint{2.474529in}{1.712718in}}%
\pgfpathlineto{\pgfqpoint{2.504421in}{1.694467in}}%
\pgfpathlineto{\pgfqpoint{2.534312in}{1.729383in}}%
\pgfpathlineto{\pgfqpoint{2.564203in}{1.772164in}}%
\pgfpathlineto{\pgfqpoint{2.594095in}{1.732079in}}%
\pgfpathlineto{\pgfqpoint{2.623986in}{1.774615in}}%
\pgfpathlineto{\pgfqpoint{2.653877in}{1.747216in}}%
\pgfpathlineto{\pgfqpoint{2.683769in}{1.821645in}}%
\pgfpathlineto{\pgfqpoint{2.713660in}{1.857976in}}%
\pgfpathlineto{\pgfqpoint{2.743551in}{1.845846in}}%
\pgfpathlineto{\pgfqpoint{2.773443in}{1.778717in}}%
\pgfpathlineto{\pgfqpoint{2.803334in}{1.810135in}}%
\pgfpathlineto{\pgfqpoint{2.833225in}{1.880062in}}%
\pgfpathlineto{\pgfqpoint{2.863117in}{1.901308in}}%
\pgfpathlineto{\pgfqpoint{2.893008in}{1.877880in}}%
\pgfpathlineto{\pgfqpoint{2.922899in}{1.880738in}}%
\pgfpathlineto{\pgfqpoint{2.952791in}{1.846501in}}%
\pgfpathlineto{\pgfqpoint{2.982682in}{1.908265in}}%
\pgfpathlineto{\pgfqpoint{3.012573in}{2.071522in}}%
\pgfpathlineto{\pgfqpoint{3.042464in}{2.034668in}}%
\pgfpathlineto{\pgfqpoint{3.072356in}{2.020000in}}%
\pgfpathlineto{\pgfqpoint{3.102247in}{2.018457in}}%
\pgfpathlineto{\pgfqpoint{3.132138in}{2.056481in}}%
\pgfpathlineto{\pgfqpoint{3.162030in}{2.135857in}}%
\pgfpathlineto{\pgfqpoint{3.191921in}{2.073980in}}%
\pgfpathlineto{\pgfqpoint{3.221812in}{2.063838in}}%
\pgfpathlineto{\pgfqpoint{3.251704in}{2.160247in}}%
\pgfpathlineto{\pgfqpoint{3.281595in}{2.080054in}}%
\pgfpathlineto{\pgfqpoint{3.311486in}{2.123121in}}%
\pgfpathlineto{\pgfqpoint{3.341378in}{2.105949in}}%
\pgfpathlineto{\pgfqpoint{3.371269in}{2.188355in}}%
\pgfpathlineto{\pgfqpoint{3.401160in}{2.236962in}}%
\pgfpathlineto{\pgfqpoint{3.431052in}{2.159103in}}%
\pgfpathlineto{\pgfqpoint{3.460943in}{2.311444in}}%
\pgfpathlineto{\pgfqpoint{3.490834in}{2.167140in}}%
\pgfpathlineto{\pgfqpoint{3.520726in}{2.251616in}}%
\pgfpathlineto{\pgfqpoint{3.550617in}{2.320632in}}%
\pgfpathlineto{\pgfqpoint{3.580508in}{2.222383in}}%
\pgfpathlineto{\pgfqpoint{3.610400in}{2.291676in}}%
\pgfpathlineto{\pgfqpoint{3.670182in}{2.326973in}}%
\pgfpathlineto{\pgfqpoint{3.700073in}{2.547839in}}%
\pgfpathlineto{\pgfqpoint{3.729965in}{2.348934in}}%
\pgfpathlineto{\pgfqpoint{3.759856in}{2.364243in}}%
\pgfpathlineto{\pgfqpoint{3.789747in}{2.421686in}}%
\pgfpathlineto{\pgfqpoint{3.939204in}{2.423075in}}%
\pgfpathlineto{\pgfqpoint{4.058769in}{2.536803in}}%
\pgfpathlineto{\pgfqpoint{4.268009in}{2.595868in}}%
\pgfpathlineto{\pgfqpoint{4.387574in}{2.617666in}}%
\pgfusepath{stroke}%
\end{pgfscope}%
\begin{pgfscope}%
\pgfpathrectangle{\pgfqpoint{0.588387in}{0.521603in}}{\pgfqpoint{4.669024in}{2.220246in}}%
\pgfusepath{clip}%
\pgfsetrectcap%
\pgfsetroundjoin%
\pgfsetlinewidth{1.505625pt}%
\pgfsetstrokecolor{currentstroke6}%
\pgfsetdash{}{0pt}%
\pgfpathmoveto{\pgfqpoint{0.800616in}{0.622524in}}%
\pgfpathlineto{\pgfqpoint{0.830507in}{0.722492in}}%
\pgfpathlineto{\pgfqpoint{0.860398in}{0.733329in}}%
\pgfpathlineto{\pgfqpoint{0.890290in}{0.807244in}}%
\pgfpathlineto{\pgfqpoint{0.920181in}{0.930796in}}%
\pgfpathlineto{\pgfqpoint{0.950072in}{1.067908in}}%
\pgfpathlineto{\pgfqpoint{0.979963in}{1.056498in}}%
\pgfpathlineto{\pgfqpoint{1.009855in}{1.096113in}}%
\pgfpathlineto{\pgfqpoint{1.039746in}{1.159395in}}%
\pgfpathlineto{\pgfqpoint{1.069637in}{1.078085in}}%
\pgfpathlineto{\pgfqpoint{1.099529in}{1.063045in}}%
\pgfpathlineto{\pgfqpoint{1.129420in}{1.007804in}}%
\pgfpathlineto{\pgfqpoint{1.159311in}{1.092655in}}%
\pgfpathlineto{\pgfqpoint{1.189203in}{1.036718in}}%
\pgfpathlineto{\pgfqpoint{1.219094in}{1.119467in}}%
\pgfpathlineto{\pgfqpoint{1.248985in}{1.139827in}}%
\pgfpathlineto{\pgfqpoint{1.278877in}{1.186993in}}%
\pgfpathlineto{\pgfqpoint{1.308768in}{1.159345in}}%
\pgfpathlineto{\pgfqpoint{1.338659in}{1.142751in}}%
\pgfpathlineto{\pgfqpoint{1.368551in}{1.117595in}}%
\pgfpathlineto{\pgfqpoint{1.398442in}{1.174243in}}%
\pgfpathlineto{\pgfqpoint{1.428333in}{1.232390in}}%
\pgfpathlineto{\pgfqpoint{1.458225in}{1.128848in}}%
\pgfpathlineto{\pgfqpoint{1.488116in}{1.226619in}}%
\pgfpathlineto{\pgfqpoint{1.518007in}{1.212092in}}%
\pgfpathlineto{\pgfqpoint{1.547899in}{1.263064in}}%
\pgfpathlineto{\pgfqpoint{1.577790in}{1.235068in}}%
\pgfpathlineto{\pgfqpoint{1.607681in}{1.229192in}}%
\pgfpathlineto{\pgfqpoint{1.637573in}{1.249938in}}%
\pgfpathlineto{\pgfqpoint{1.667464in}{1.368846in}}%
\pgfpathlineto{\pgfqpoint{1.697355in}{1.330745in}}%
\pgfpathlineto{\pgfqpoint{1.727246in}{1.357781in}}%
\pgfpathlineto{\pgfqpoint{1.757138in}{1.392426in}}%
\pgfpathlineto{\pgfqpoint{1.787029in}{1.296992in}}%
\pgfpathlineto{\pgfqpoint{1.816920in}{1.394086in}}%
\pgfpathlineto{\pgfqpoint{1.846812in}{1.417204in}}%
\pgfpathlineto{\pgfqpoint{1.876703in}{1.388418in}}%
\pgfpathlineto{\pgfqpoint{1.906594in}{1.325314in}}%
\pgfpathlineto{\pgfqpoint{1.936486in}{1.405617in}}%
\pgfpathlineto{\pgfqpoint{1.966377in}{1.431592in}}%
\pgfpathlineto{\pgfqpoint{1.996268in}{1.492580in}}%
\pgfpathlineto{\pgfqpoint{2.026160in}{1.464908in}}%
\pgfpathlineto{\pgfqpoint{2.056051in}{1.462688in}}%
\pgfpathlineto{\pgfqpoint{2.085942in}{1.459361in}}%
\pgfpathlineto{\pgfqpoint{2.115834in}{1.500309in}}%
\pgfpathlineto{\pgfqpoint{2.145725in}{1.513606in}}%
\pgfpathlineto{\pgfqpoint{2.175616in}{1.506695in}}%
\pgfpathlineto{\pgfqpoint{2.205508in}{1.581121in}}%
\pgfpathlineto{\pgfqpoint{2.235399in}{1.567059in}}%
\pgfpathlineto{\pgfqpoint{2.265290in}{1.559773in}}%
\pgfpathlineto{\pgfqpoint{2.295182in}{1.553856in}}%
\pgfpathlineto{\pgfqpoint{2.325073in}{1.674493in}}%
\pgfpathlineto{\pgfqpoint{2.354964in}{1.636730in}}%
\pgfpathlineto{\pgfqpoint{2.384855in}{1.610459in}}%
\pgfpathlineto{\pgfqpoint{2.414747in}{1.708814in}}%
\pgfpathlineto{\pgfqpoint{2.444638in}{1.705398in}}%
\pgfpathlineto{\pgfqpoint{2.474529in}{1.696029in}}%
\pgfpathlineto{\pgfqpoint{2.504421in}{1.680644in}}%
\pgfpathlineto{\pgfqpoint{2.534312in}{1.696525in}}%
\pgfpathlineto{\pgfqpoint{2.564203in}{1.752137in}}%
\pgfpathlineto{\pgfqpoint{2.594095in}{1.708992in}}%
\pgfpathlineto{\pgfqpoint{2.623986in}{1.781941in}}%
\pgfpathlineto{\pgfqpoint{2.653877in}{1.720071in}}%
\pgfpathlineto{\pgfqpoint{2.683769in}{1.795132in}}%
\pgfpathlineto{\pgfqpoint{2.713660in}{1.822174in}}%
\pgfpathlineto{\pgfqpoint{2.743551in}{1.822848in}}%
\pgfpathlineto{\pgfqpoint{2.773443in}{1.763617in}}%
\pgfpathlineto{\pgfqpoint{2.803334in}{1.805361in}}%
\pgfpathlineto{\pgfqpoint{2.833225in}{1.852757in}}%
\pgfpathlineto{\pgfqpoint{2.863117in}{1.909167in}}%
\pgfpathlineto{\pgfqpoint{2.893008in}{1.864408in}}%
\pgfpathlineto{\pgfqpoint{2.922899in}{1.868828in}}%
\pgfpathlineto{\pgfqpoint{2.952791in}{1.821116in}}%
\pgfpathlineto{\pgfqpoint{2.982682in}{1.921882in}}%
\pgfpathlineto{\pgfqpoint{3.012573in}{2.003946in}}%
\pgfpathlineto{\pgfqpoint{3.042464in}{2.018339in}}%
\pgfpathlineto{\pgfqpoint{3.072356in}{1.995990in}}%
\pgfpathlineto{\pgfqpoint{3.102247in}{1.977687in}}%
\pgfpathlineto{\pgfqpoint{3.132138in}{2.043629in}}%
\pgfpathlineto{\pgfqpoint{3.162030in}{2.134860in}}%
\pgfpathlineto{\pgfqpoint{3.191921in}{2.064446in}}%
\pgfpathlineto{\pgfqpoint{3.221812in}{2.042277in}}%
\pgfpathlineto{\pgfqpoint{3.251704in}{2.139193in}}%
\pgfpathlineto{\pgfqpoint{3.281595in}{2.075004in}}%
\pgfpathlineto{\pgfqpoint{3.311486in}{2.095788in}}%
\pgfpathlineto{\pgfqpoint{3.341378in}{2.088629in}}%
\pgfpathlineto{\pgfqpoint{3.371269in}{2.170143in}}%
\pgfpathlineto{\pgfqpoint{3.401160in}{2.198876in}}%
\pgfpathlineto{\pgfqpoint{3.431052in}{2.185596in}}%
\pgfpathlineto{\pgfqpoint{3.460943in}{2.273450in}}%
\pgfpathlineto{\pgfqpoint{3.490834in}{2.161237in}}%
\pgfpathlineto{\pgfqpoint{3.520726in}{2.233670in}}%
\pgfpathlineto{\pgfqpoint{3.550617in}{2.331235in}}%
\pgfpathlineto{\pgfqpoint{3.580508in}{2.197453in}}%
\pgfpathlineto{\pgfqpoint{3.610400in}{2.252412in}}%
\pgfpathlineto{\pgfqpoint{3.670182in}{2.308308in}}%
\pgfpathlineto{\pgfqpoint{3.700073in}{2.572666in}}%
\pgfpathlineto{\pgfqpoint{3.729965in}{2.285114in}}%
\pgfpathlineto{\pgfqpoint{3.759856in}{2.345461in}}%
\pgfpathlineto{\pgfqpoint{3.789747in}{2.446790in}}%
\pgfpathlineto{\pgfqpoint{3.939204in}{2.462501in}}%
\pgfpathlineto{\pgfqpoint{4.058769in}{2.507675in}}%
\pgfpathlineto{\pgfqpoint{4.268009in}{2.582942in}}%
\pgfusepath{stroke}%
\end{pgfscope}%
\begin{pgfscope}%
\pgfpathrectangle{\pgfqpoint{0.588387in}{0.521603in}}{\pgfqpoint{4.669024in}{2.220246in}}%
\pgfusepath{clip}%
\pgfsetrectcap%
\pgfsetroundjoin%
\pgfsetlinewidth{1.505625pt}%
\pgfsetstrokecolor{currentstroke7}%
\pgfsetdash{}{0pt}%
\pgfpathmoveto{\pgfqpoint{0.800616in}{0.659091in}}%
\pgfpathlineto{\pgfqpoint{0.830507in}{0.725312in}}%
\pgfpathlineto{\pgfqpoint{0.860398in}{0.743011in}}%
\pgfpathlineto{\pgfqpoint{0.890290in}{0.807455in}}%
\pgfpathlineto{\pgfqpoint{0.920181in}{0.924613in}}%
\pgfpathlineto{\pgfqpoint{0.950072in}{1.053122in}}%
\pgfpathlineto{\pgfqpoint{0.979963in}{1.101361in}}%
\pgfpathlineto{\pgfqpoint{1.009855in}{1.199388in}}%
\pgfpathlineto{\pgfqpoint{1.039746in}{1.351863in}}%
\pgfpathlineto{\pgfqpoint{1.069637in}{1.222909in}}%
\pgfpathlineto{\pgfqpoint{1.099529in}{1.277026in}}%
\pgfpathlineto{\pgfqpoint{1.129420in}{1.319036in}}%
\pgfpathlineto{\pgfqpoint{1.159311in}{1.301410in}}%
\pgfpathlineto{\pgfqpoint{1.189203in}{1.232072in}}%
\pgfpathlineto{\pgfqpoint{1.219094in}{1.316749in}}%
\pgfpathlineto{\pgfqpoint{1.248985in}{1.350284in}}%
\pgfpathlineto{\pgfqpoint{1.308768in}{1.332164in}}%
\pgfpathlineto{\pgfqpoint{1.338659in}{1.294730in}}%
\pgfpathlineto{\pgfqpoint{1.368551in}{1.327967in}}%
\pgfpathlineto{\pgfqpoint{1.398442in}{1.520191in}}%
\pgfpathlineto{\pgfqpoint{1.428333in}{1.441755in}}%
\pgfpathlineto{\pgfqpoint{1.458225in}{1.349304in}}%
\pgfpathlineto{\pgfqpoint{1.488116in}{1.421255in}}%
\pgfpathlineto{\pgfqpoint{1.518007in}{1.393428in}}%
\pgfpathlineto{\pgfqpoint{1.547899in}{1.553190in}}%
\pgfpathlineto{\pgfqpoint{1.577790in}{1.513916in}}%
\pgfpathlineto{\pgfqpoint{1.607681in}{1.502317in}}%
\pgfpathlineto{\pgfqpoint{1.637573in}{1.451662in}}%
\pgfpathlineto{\pgfqpoint{1.667464in}{1.594474in}}%
\pgfpathlineto{\pgfqpoint{1.697355in}{1.550952in}}%
\pgfpathlineto{\pgfqpoint{1.727246in}{1.575231in}}%
\pgfpathlineto{\pgfqpoint{1.757138in}{1.572678in}}%
\pgfpathlineto{\pgfqpoint{1.816920in}{1.611820in}}%
\pgfpathlineto{\pgfqpoint{1.846812in}{1.561880in}}%
\pgfpathlineto{\pgfqpoint{1.876703in}{1.557160in}}%
\pgfpathlineto{\pgfqpoint{1.906594in}{1.604672in}}%
\pgfpathlineto{\pgfqpoint{1.936486in}{1.615304in}}%
\pgfpathlineto{\pgfqpoint{1.966377in}{1.675978in}}%
\pgfpathlineto{\pgfqpoint{1.996268in}{1.690778in}}%
\pgfpathlineto{\pgfqpoint{2.026160in}{1.661492in}}%
\pgfpathlineto{\pgfqpoint{2.056051in}{1.710371in}}%
\pgfpathlineto{\pgfqpoint{2.085942in}{1.686511in}}%
\pgfpathlineto{\pgfqpoint{2.115834in}{1.776735in}}%
\pgfpathlineto{\pgfqpoint{2.145725in}{1.672705in}}%
\pgfpathlineto{\pgfqpoint{2.175616in}{1.740698in}}%
\pgfpathlineto{\pgfqpoint{2.205508in}{1.731089in}}%
\pgfpathlineto{\pgfqpoint{2.235399in}{1.718108in}}%
\pgfpathlineto{\pgfqpoint{2.265290in}{1.836270in}}%
\pgfpathlineto{\pgfqpoint{2.295182in}{1.769833in}}%
\pgfpathlineto{\pgfqpoint{2.325073in}{1.784356in}}%
\pgfpathlineto{\pgfqpoint{2.354964in}{1.797120in}}%
\pgfpathlineto{\pgfqpoint{2.384855in}{1.796793in}}%
\pgfpathlineto{\pgfqpoint{2.414747in}{1.817427in}}%
\pgfpathlineto{\pgfqpoint{2.444638in}{1.823967in}}%
\pgfpathlineto{\pgfqpoint{2.474529in}{1.797672in}}%
\pgfpathlineto{\pgfqpoint{2.504421in}{1.837122in}}%
\pgfpathlineto{\pgfqpoint{2.534312in}{1.906418in}}%
\pgfpathlineto{\pgfqpoint{2.564203in}{1.874760in}}%
\pgfpathlineto{\pgfqpoint{2.594095in}{1.886951in}}%
\pgfpathlineto{\pgfqpoint{2.623986in}{1.922291in}}%
\pgfpathlineto{\pgfqpoint{2.653877in}{1.828894in}}%
\pgfpathlineto{\pgfqpoint{2.683769in}{1.893310in}}%
\pgfpathlineto{\pgfqpoint{2.713660in}{1.992916in}}%
\pgfpathlineto{\pgfqpoint{2.743551in}{1.962151in}}%
\pgfpathlineto{\pgfqpoint{2.773443in}{1.904308in}}%
\pgfpathlineto{\pgfqpoint{2.803334in}{1.932627in}}%
\pgfpathlineto{\pgfqpoint{2.833225in}{2.012105in}}%
\pgfpathlineto{\pgfqpoint{2.863117in}{1.938527in}}%
\pgfpathlineto{\pgfqpoint{2.893008in}{2.011074in}}%
\pgfpathlineto{\pgfqpoint{2.922899in}{1.991514in}}%
\pgfpathlineto{\pgfqpoint{2.952791in}{2.042655in}}%
\pgfpathlineto{\pgfqpoint{2.982682in}{2.030846in}}%
\pgfpathlineto{\pgfqpoint{3.012573in}{2.043982in}}%
\pgfpathlineto{\pgfqpoint{3.042464in}{2.036625in}}%
\pgfpathlineto{\pgfqpoint{3.072356in}{2.038451in}}%
\pgfpathlineto{\pgfqpoint{3.102247in}{2.049341in}}%
\pgfpathlineto{\pgfqpoint{3.132138in}{2.076368in}}%
\pgfpathlineto{\pgfqpoint{3.162030in}{2.131960in}}%
\pgfpathlineto{\pgfqpoint{3.191921in}{2.084917in}}%
\pgfpathlineto{\pgfqpoint{3.221812in}{2.135076in}}%
\pgfpathlineto{\pgfqpoint{3.251704in}{2.182756in}}%
\pgfpathlineto{\pgfqpoint{3.281595in}{2.112541in}}%
\pgfpathlineto{\pgfqpoint{3.311486in}{2.140524in}}%
\pgfpathlineto{\pgfqpoint{3.341378in}{2.153193in}}%
\pgfpathlineto{\pgfqpoint{3.371269in}{2.185955in}}%
\pgfpathlineto{\pgfqpoint{3.401160in}{2.177701in}}%
\pgfpathlineto{\pgfqpoint{3.431052in}{2.204791in}}%
\pgfpathlineto{\pgfqpoint{3.460943in}{2.211987in}}%
\pgfpathlineto{\pgfqpoint{3.490834in}{2.230029in}}%
\pgfpathlineto{\pgfqpoint{3.520726in}{2.214223in}}%
\pgfpathlineto{\pgfqpoint{3.550617in}{2.226082in}}%
\pgfpathlineto{\pgfqpoint{3.580508in}{2.284579in}}%
\pgfpathlineto{\pgfqpoint{3.610400in}{2.249741in}}%
\pgfpathlineto{\pgfqpoint{3.640291in}{2.295240in}}%
\pgfpathlineto{\pgfqpoint{3.670182in}{2.276814in}}%
\pgfpathlineto{\pgfqpoint{3.700073in}{2.314140in}}%
\pgfpathlineto{\pgfqpoint{3.729965in}{2.310680in}}%
\pgfpathlineto{\pgfqpoint{3.759856in}{2.310931in}}%
\pgfpathlineto{\pgfqpoint{3.789747in}{2.366041in}}%
\pgfpathlineto{\pgfqpoint{3.819639in}{2.342013in}}%
\pgfpathlineto{\pgfqpoint{3.849530in}{2.341776in}}%
\pgfpathlineto{\pgfqpoint{3.879421in}{2.357171in}}%
\pgfpathlineto{\pgfqpoint{3.909313in}{2.378967in}}%
\pgfpathlineto{\pgfqpoint{3.939204in}{2.394027in}}%
\pgfpathlineto{\pgfqpoint{3.969095in}{2.369218in}}%
\pgfpathlineto{\pgfqpoint{3.998987in}{2.423519in}}%
\pgfpathlineto{\pgfqpoint{4.028878in}{2.410446in}}%
\pgfpathlineto{\pgfqpoint{4.058769in}{2.430457in}}%
\pgfpathlineto{\pgfqpoint{4.088661in}{2.428581in}}%
\pgfpathlineto{\pgfqpoint{4.118552in}{2.483562in}}%
\pgfpathlineto{\pgfqpoint{4.148443in}{2.466588in}}%
\pgfpathlineto{\pgfqpoint{4.178335in}{2.413955in}}%
\pgfpathlineto{\pgfqpoint{4.208226in}{2.510278in}}%
\pgfpathlineto{\pgfqpoint{4.238117in}{2.491751in}}%
\pgfpathlineto{\pgfqpoint{4.268009in}{2.505334in}}%
\pgfpathlineto{\pgfqpoint{4.297900in}{2.527342in}}%
\pgfpathlineto{\pgfqpoint{4.327791in}{2.489471in}}%
\pgfpathlineto{\pgfqpoint{4.357683in}{2.509544in}}%
\pgfpathlineto{\pgfqpoint{4.387574in}{2.534646in}}%
\pgfpathlineto{\pgfqpoint{4.417465in}{2.553642in}}%
\pgfpathlineto{\pgfqpoint{4.447356in}{2.540364in}}%
\pgfpathlineto{\pgfqpoint{4.477248in}{2.538220in}}%
\pgfpathlineto{\pgfqpoint{4.507139in}{2.573523in}}%
\pgfpathlineto{\pgfqpoint{4.537030in}{2.573943in}}%
\pgfpathlineto{\pgfqpoint{4.566922in}{2.555695in}}%
\pgfpathlineto{\pgfqpoint{4.596813in}{2.579651in}}%
\pgfpathlineto{\pgfqpoint{4.626704in}{2.586184in}}%
\pgfpathlineto{\pgfqpoint{4.656596in}{2.582564in}}%
\pgfpathlineto{\pgfqpoint{4.686487in}{2.591662in}}%
\pgfpathlineto{\pgfqpoint{4.716378in}{2.586606in}}%
\pgfpathlineto{\pgfqpoint{4.746270in}{2.579324in}}%
\pgfpathlineto{\pgfqpoint{4.776161in}{2.527967in}}%
\pgfpathlineto{\pgfqpoint{4.806052in}{2.601729in}}%
\pgfpathlineto{\pgfqpoint{4.835944in}{2.609194in}}%
\pgfpathlineto{\pgfqpoint{4.895726in}{2.607788in}}%
\pgfpathlineto{\pgfqpoint{4.925618in}{2.636842in}}%
\pgfpathlineto{\pgfqpoint{4.955509in}{2.621942in}}%
\pgfpathlineto{\pgfqpoint{4.985400in}{2.632466in}}%
\pgfpathlineto{\pgfqpoint{5.045183in}{2.626489in}}%
\pgfpathlineto{\pgfqpoint{5.045183in}{2.626489in}}%
\pgfusepath{stroke}%
\end{pgfscope}%
\begin{pgfscope}%
\pgfpathrectangle{\pgfqpoint{0.588387in}{0.521603in}}{\pgfqpoint{4.669024in}{2.220246in}}%
\pgfusepath{clip}%
\pgfsetrectcap%
\pgfsetroundjoin%
\pgfsetlinewidth{1.505625pt}%
\definecolor{currentstroke}{rgb}{0.498039,0.498039,0.498039}%
\pgfsetstrokecolor{currentstroke}%
\pgfsetdash{}{0pt}%
\pgfpathmoveto{\pgfqpoint{0.800616in}{0.666374in}}%
\pgfpathlineto{\pgfqpoint{0.830507in}{0.714301in}}%
\pgfpathlineto{\pgfqpoint{0.860398in}{0.724663in}}%
\pgfpathlineto{\pgfqpoint{0.890290in}{0.815585in}}%
\pgfpathlineto{\pgfqpoint{0.920181in}{0.936424in}}%
\pgfpathlineto{\pgfqpoint{0.950072in}{1.072040in}}%
\pgfpathlineto{\pgfqpoint{0.979963in}{1.078573in}}%
\pgfpathlineto{\pgfqpoint{1.009855in}{1.150535in}}%
\pgfpathlineto{\pgfqpoint{1.039746in}{1.277643in}}%
\pgfpathlineto{\pgfqpoint{1.069637in}{1.159802in}}%
\pgfpathlineto{\pgfqpoint{1.099529in}{1.197366in}}%
\pgfpathlineto{\pgfqpoint{1.129420in}{1.191424in}}%
\pgfpathlineto{\pgfqpoint{1.159311in}{1.206073in}}%
\pgfpathlineto{\pgfqpoint{1.189203in}{1.134382in}}%
\pgfpathlineto{\pgfqpoint{1.219094in}{1.241060in}}%
\pgfpathlineto{\pgfqpoint{1.248985in}{1.263319in}}%
\pgfpathlineto{\pgfqpoint{1.278877in}{1.266753in}}%
\pgfpathlineto{\pgfqpoint{1.308768in}{1.248914in}}%
\pgfpathlineto{\pgfqpoint{1.338659in}{1.218201in}}%
\pgfpathlineto{\pgfqpoint{1.368551in}{1.227865in}}%
\pgfpathlineto{\pgfqpoint{1.398442in}{1.333940in}}%
\pgfpathlineto{\pgfqpoint{1.428333in}{1.344163in}}%
\pgfpathlineto{\pgfqpoint{1.458225in}{1.246110in}}%
\pgfpathlineto{\pgfqpoint{1.488116in}{1.325838in}}%
\pgfpathlineto{\pgfqpoint{1.518007in}{1.286415in}}%
\pgfpathlineto{\pgfqpoint{1.547899in}{1.400932in}}%
\pgfpathlineto{\pgfqpoint{1.577790in}{1.392429in}}%
\pgfpathlineto{\pgfqpoint{1.607681in}{1.360781in}}%
\pgfpathlineto{\pgfqpoint{1.637573in}{1.342413in}}%
\pgfpathlineto{\pgfqpoint{1.667464in}{1.491827in}}%
\pgfpathlineto{\pgfqpoint{1.697355in}{1.456818in}}%
\pgfpathlineto{\pgfqpoint{1.727246in}{1.451301in}}%
\pgfpathlineto{\pgfqpoint{1.757138in}{1.474838in}}%
\pgfpathlineto{\pgfqpoint{1.787029in}{1.454727in}}%
\pgfpathlineto{\pgfqpoint{1.816920in}{1.486116in}}%
\pgfpathlineto{\pgfqpoint{1.846812in}{1.449518in}}%
\pgfpathlineto{\pgfqpoint{1.876703in}{1.476486in}}%
\pgfpathlineto{\pgfqpoint{1.906594in}{1.492411in}}%
\pgfpathlineto{\pgfqpoint{1.936486in}{1.532491in}}%
\pgfpathlineto{\pgfqpoint{1.966377in}{1.528638in}}%
\pgfpathlineto{\pgfqpoint{1.996268in}{1.565248in}}%
\pgfpathlineto{\pgfqpoint{2.026160in}{1.533655in}}%
\pgfpathlineto{\pgfqpoint{2.056051in}{1.607799in}}%
\pgfpathlineto{\pgfqpoint{2.085942in}{1.552836in}}%
\pgfpathlineto{\pgfqpoint{2.115834in}{1.639059in}}%
\pgfpathlineto{\pgfqpoint{2.145725in}{1.584826in}}%
\pgfpathlineto{\pgfqpoint{2.175616in}{1.633097in}}%
\pgfpathlineto{\pgfqpoint{2.205508in}{1.655474in}}%
\pgfpathlineto{\pgfqpoint{2.235399in}{1.604107in}}%
\pgfpathlineto{\pgfqpoint{2.265290in}{1.732112in}}%
\pgfpathlineto{\pgfqpoint{2.295182in}{1.663457in}}%
\pgfpathlineto{\pgfqpoint{2.325073in}{1.727863in}}%
\pgfpathlineto{\pgfqpoint{2.354964in}{1.714649in}}%
\pgfpathlineto{\pgfqpoint{2.384855in}{1.689949in}}%
\pgfpathlineto{\pgfqpoint{2.414747in}{1.717096in}}%
\pgfpathlineto{\pgfqpoint{2.444638in}{1.735708in}}%
\pgfpathlineto{\pgfqpoint{2.474529in}{1.736499in}}%
\pgfpathlineto{\pgfqpoint{2.504421in}{1.707007in}}%
\pgfpathlineto{\pgfqpoint{2.534312in}{1.802325in}}%
\pgfpathlineto{\pgfqpoint{2.564203in}{1.766601in}}%
\pgfpathlineto{\pgfqpoint{2.594095in}{1.758604in}}%
\pgfpathlineto{\pgfqpoint{2.623986in}{1.790417in}}%
\pgfpathlineto{\pgfqpoint{2.653877in}{1.756002in}}%
\pgfpathlineto{\pgfqpoint{2.683769in}{1.813262in}}%
\pgfpathlineto{\pgfqpoint{2.713660in}{1.840578in}}%
\pgfpathlineto{\pgfqpoint{2.743551in}{1.884382in}}%
\pgfpathlineto{\pgfqpoint{2.773443in}{1.833762in}}%
\pgfpathlineto{\pgfqpoint{2.803334in}{1.881814in}}%
\pgfpathlineto{\pgfqpoint{2.833225in}{1.903448in}}%
\pgfpathlineto{\pgfqpoint{2.863117in}{1.895888in}}%
\pgfpathlineto{\pgfqpoint{2.893008in}{1.942713in}}%
\pgfpathlineto{\pgfqpoint{2.922899in}{1.921714in}}%
\pgfpathlineto{\pgfqpoint{2.952791in}{1.949844in}}%
\pgfpathlineto{\pgfqpoint{2.982682in}{1.919509in}}%
\pgfpathlineto{\pgfqpoint{3.012573in}{2.006069in}}%
\pgfpathlineto{\pgfqpoint{3.042464in}{1.966250in}}%
\pgfpathlineto{\pgfqpoint{3.072356in}{1.984261in}}%
\pgfpathlineto{\pgfqpoint{3.102247in}{2.016778in}}%
\pgfpathlineto{\pgfqpoint{3.132138in}{2.045944in}}%
\pgfpathlineto{\pgfqpoint{3.162030in}{2.097395in}}%
\pgfpathlineto{\pgfqpoint{3.191921in}{2.012462in}}%
\pgfpathlineto{\pgfqpoint{3.221812in}{2.063445in}}%
\pgfpathlineto{\pgfqpoint{3.251704in}{2.120383in}}%
\pgfpathlineto{\pgfqpoint{3.281595in}{2.084668in}}%
\pgfpathlineto{\pgfqpoint{3.311486in}{2.093390in}}%
\pgfpathlineto{\pgfqpoint{3.341378in}{2.138326in}}%
\pgfpathlineto{\pgfqpoint{3.371269in}{2.140009in}}%
\pgfpathlineto{\pgfqpoint{3.401160in}{2.171273in}}%
\pgfpathlineto{\pgfqpoint{3.431052in}{2.121070in}}%
\pgfpathlineto{\pgfqpoint{3.460943in}{2.212241in}}%
\pgfpathlineto{\pgfqpoint{3.490834in}{2.191492in}}%
\pgfpathlineto{\pgfqpoint{3.520726in}{2.195246in}}%
\pgfpathlineto{\pgfqpoint{3.550617in}{2.228445in}}%
\pgfpathlineto{\pgfqpoint{3.580508in}{2.243853in}}%
\pgfpathlineto{\pgfqpoint{3.610400in}{2.210157in}}%
\pgfpathlineto{\pgfqpoint{3.640291in}{2.255476in}}%
\pgfpathlineto{\pgfqpoint{3.670182in}{2.272854in}}%
\pgfpathlineto{\pgfqpoint{3.700073in}{2.339807in}}%
\pgfpathlineto{\pgfqpoint{3.729965in}{2.307172in}}%
\pgfpathlineto{\pgfqpoint{3.759856in}{2.289350in}}%
\pgfpathlineto{\pgfqpoint{3.789747in}{2.349015in}}%
\pgfpathlineto{\pgfqpoint{3.819639in}{2.310459in}}%
\pgfpathlineto{\pgfqpoint{3.849530in}{2.296655in}}%
\pgfpathlineto{\pgfqpoint{3.879421in}{2.304592in}}%
\pgfpathlineto{\pgfqpoint{3.909313in}{2.364243in}}%
\pgfpathlineto{\pgfqpoint{3.939204in}{2.394202in}}%
\pgfpathlineto{\pgfqpoint{3.969095in}{2.359668in}}%
\pgfpathlineto{\pgfqpoint{3.998987in}{2.429092in}}%
\pgfpathlineto{\pgfqpoint{4.028878in}{2.388814in}}%
\pgfpathlineto{\pgfqpoint{4.058769in}{2.422302in}}%
\pgfpathlineto{\pgfqpoint{4.088661in}{2.426052in}}%
\pgfpathlineto{\pgfqpoint{4.118552in}{2.460414in}}%
\pgfpathlineto{\pgfqpoint{4.148443in}{2.475674in}}%
\pgfpathlineto{\pgfqpoint{4.178335in}{2.411643in}}%
\pgfpathlineto{\pgfqpoint{4.208226in}{2.514002in}}%
\pgfpathlineto{\pgfqpoint{4.238117in}{2.472788in}}%
\pgfpathlineto{\pgfqpoint{4.268009in}{2.513401in}}%
\pgfpathlineto{\pgfqpoint{4.297900in}{2.483268in}}%
\pgfpathlineto{\pgfqpoint{4.327791in}{2.479313in}}%
\pgfpathlineto{\pgfqpoint{4.357683in}{2.524912in}}%
\pgfpathlineto{\pgfqpoint{4.387574in}{2.578018in}}%
\pgfpathlineto{\pgfqpoint{4.417465in}{2.545914in}}%
\pgfpathlineto{\pgfqpoint{4.447356in}{2.522787in}}%
\pgfpathlineto{\pgfqpoint{4.477248in}{2.550020in}}%
\pgfpathlineto{\pgfqpoint{4.507139in}{2.526690in}}%
\pgfpathlineto{\pgfqpoint{4.537030in}{2.532713in}}%
\pgfpathlineto{\pgfqpoint{4.566922in}{2.542235in}}%
\pgfpathlineto{\pgfqpoint{4.596813in}{2.612967in}}%
\pgfpathlineto{\pgfqpoint{4.656596in}{2.567382in}}%
\pgfpathlineto{\pgfqpoint{4.716378in}{2.548291in}}%
\pgfpathlineto{\pgfqpoint{4.806052in}{2.596972in}}%
\pgfpathlineto{\pgfqpoint{4.835944in}{2.598623in}}%
\pgfpathlineto{\pgfqpoint{4.835944in}{2.598623in}}%
\pgfusepath{stroke}%
\end{pgfscope}%
\begin{pgfscope}%
\pgfsetrectcap%
\pgfsetmiterjoin%
\pgfsetlinewidth{0.803000pt}%
\definecolor{currentstroke}{rgb}{0.000000,0.000000,0.000000}%
\pgfsetstrokecolor{currentstroke}%
\pgfsetdash{}{0pt}%
\pgfpathmoveto{\pgfqpoint{0.588387in}{0.521603in}}%
\pgfpathlineto{\pgfqpoint{0.588387in}{2.741849in}}%
\pgfusepath{stroke}%
\end{pgfscope}%
\begin{pgfscope}%
\pgfsetrectcap%
\pgfsetmiterjoin%
\pgfsetlinewidth{0.803000pt}%
\definecolor{currentstroke}{rgb}{0.000000,0.000000,0.000000}%
\pgfsetstrokecolor{currentstroke}%
\pgfsetdash{}{0pt}%
\pgfpathmoveto{\pgfqpoint{5.257411in}{0.521603in}}%
\pgfpathlineto{\pgfqpoint{5.257411in}{2.741849in}}%
\pgfusepath{stroke}%
\end{pgfscope}%
\begin{pgfscope}%
\pgfsetrectcap%
\pgfsetmiterjoin%
\pgfsetlinewidth{0.803000pt}%
\definecolor{currentstroke}{rgb}{0.000000,0.000000,0.000000}%
\pgfsetstrokecolor{currentstroke}%
\pgfsetdash{}{0pt}%
\pgfpathmoveto{\pgfqpoint{0.588387in}{0.521603in}}%
\pgfpathlineto{\pgfqpoint{5.257411in}{0.521603in}}%
\pgfusepath{stroke}%
\end{pgfscope}%
\begin{pgfscope}%
\pgfsetrectcap%
\pgfsetmiterjoin%
\pgfsetlinewidth{0.803000pt}%
\definecolor{currentstroke}{rgb}{0.000000,0.000000,0.000000}%
\pgfsetstrokecolor{currentstroke}%
\pgfsetdash{}{0pt}%
\pgfpathmoveto{\pgfqpoint{0.588387in}{2.741849in}}%
\pgfpathlineto{\pgfqpoint{5.257411in}{2.741849in}}%
\pgfusepath{stroke}%
\end{pgfscope}%
\begin{pgfscope}%
\pgfsetbuttcap%
\pgfsetmiterjoin%
\definecolor{currentfill}{rgb}{1.000000,1.000000,1.000000}%
\pgfsetfillcolor{currentfill}%
\pgfsetfillopacity{0.800000}%
\pgfsetlinewidth{1.003750pt}%
\definecolor{currentstroke}{rgb}{0.800000,0.800000,0.800000}%
\pgfsetstrokecolor{currentstroke}%
\pgfsetstrokeopacity{0.800000}%
\pgfsetdash{}{0pt}%
\pgfpathmoveto{\pgfqpoint{5.344911in}{1.153204in}}%
\pgfpathlineto{\pgfqpoint{8.259376in}{1.153204in}}%
\pgfpathquadraticcurveto{\pgfqpoint{8.284376in}{1.153204in}}{\pgfqpoint{8.284376in}{1.178204in}}%
\pgfpathlineto{\pgfqpoint{8.284376in}{2.654349in}}%
\pgfpathquadraticcurveto{\pgfqpoint{8.284376in}{2.679349in}}{\pgfqpoint{8.259376in}{2.679349in}}%
\pgfpathlineto{\pgfqpoint{5.344911in}{2.679349in}}%
\pgfpathquadraticcurveto{\pgfqpoint{5.319911in}{2.679349in}}{\pgfqpoint{5.319911in}{2.654349in}}%
\pgfpathlineto{\pgfqpoint{5.319911in}{1.178204in}}%
\pgfpathquadraticcurveto{\pgfqpoint{5.319911in}{1.153204in}}{\pgfqpoint{5.344911in}{1.153204in}}%
\pgfpathlineto{\pgfqpoint{5.344911in}{1.153204in}}%
\pgfpathclose%
\pgfusepath{stroke,fill}%
\end{pgfscope}%
\begin{pgfscope}%
\pgfsetrectcap%
\pgfsetroundjoin%
\pgfsetlinewidth{1.505625pt}%
\pgfsetstrokecolor{currentstroke1}%
\pgfsetdash{}{0pt}%
\pgfpathmoveto{\pgfqpoint{5.369911in}{2.578129in}}%
\pgfpathlineto{\pgfqpoint{5.494911in}{2.578129in}}%
\pgfpathlineto{\pgfqpoint{5.619911in}{2.578129in}}%
\pgfusepath{stroke}%
\end{pgfscope}%
\begin{pgfscope}%
\definecolor{textcolor}{rgb}{0.000000,0.000000,0.000000}%
\pgfsetstrokecolor{textcolor}%
\pgfsetfillcolor{textcolor}%
\pgftext[x=5.719911in,y=2.534379in,left,base]{\color{textcolor}{\rmfamily\fontsize{9.000000}{10.800000}\selectfont\catcode`\^=\active\def^{\ifmmode\sp\else\^{}\fi}\catcode`\%=\active\def%{\%}\CyclesMatchChunks{} \& \MergeLinear{}}}%
\end{pgfscope}%
\begin{pgfscope}%
\pgfsetrectcap%
\pgfsetroundjoin%
\pgfsetlinewidth{1.505625pt}%
\pgfsetstrokecolor{currentstroke2}%
\pgfsetdash{}{0pt}%
\pgfpathmoveto{\pgfqpoint{5.369911in}{2.391178in}}%
\pgfpathlineto{\pgfqpoint{5.494911in}{2.391178in}}%
\pgfpathlineto{\pgfqpoint{5.619911in}{2.391178in}}%
\pgfusepath{stroke}%
\end{pgfscope}%
\begin{pgfscope}%
\definecolor{textcolor}{rgb}{0.000000,0.000000,0.000000}%
\pgfsetstrokecolor{textcolor}%
\pgfsetfillcolor{textcolor}%
\pgftext[x=5.719911in,y=2.347428in,left,base]{\color{textcolor}{\rmfamily\fontsize{9.000000}{10.800000}\selectfont\catcode`\^=\active\def^{\ifmmode\sp\else\^{}\fi}\catcode`\%=\active\def%{\%}\CyclesMatchChunks{} \& \SharedVertices{}}}%
\end{pgfscope}%
\begin{pgfscope}%
\pgfsetrectcap%
\pgfsetroundjoin%
\pgfsetlinewidth{1.505625pt}%
\pgfsetstrokecolor{currentstroke3}%
\pgfsetdash{}{0pt}%
\pgfpathmoveto{\pgfqpoint{5.369911in}{2.204228in}}%
\pgfpathlineto{\pgfqpoint{5.494911in}{2.204228in}}%
\pgfpathlineto{\pgfqpoint{5.619911in}{2.204228in}}%
\pgfusepath{stroke}%
\end{pgfscope}%
\begin{pgfscope}%
\definecolor{textcolor}{rgb}{0.000000,0.000000,0.000000}%
\pgfsetstrokecolor{textcolor}%
\pgfsetfillcolor{textcolor}%
\pgftext[x=5.719911in,y=2.160478in,left,base]{\color{textcolor}{\rmfamily\fontsize{9.000000}{10.800000}\selectfont\catcode`\^=\active\def^{\ifmmode\sp\else\^{}\fi}\catcode`\%=\active\def%{\%}\Neighbors{} \& \MergeLinear{}}}%
\end{pgfscope}%
\begin{pgfscope}%
\pgfsetrectcap%
\pgfsetroundjoin%
\pgfsetlinewidth{1.505625pt}%
\pgfsetstrokecolor{currentstroke4}%
\pgfsetdash{}{0pt}%
\pgfpathmoveto{\pgfqpoint{5.369911in}{2.020756in}}%
\pgfpathlineto{\pgfqpoint{5.494911in}{2.020756in}}%
\pgfpathlineto{\pgfqpoint{5.619911in}{2.020756in}}%
\pgfusepath{stroke}%
\end{pgfscope}%
\begin{pgfscope}%
\definecolor{textcolor}{rgb}{0.000000,0.000000,0.000000}%
\pgfsetstrokecolor{textcolor}%
\pgfsetfillcolor{textcolor}%
\pgftext[x=5.719911in,y=1.977006in,left,base]{\color{textcolor}{\rmfamily\fontsize{9.000000}{10.800000}\selectfont\catcode`\^=\active\def^{\ifmmode\sp\else\^{}\fi}\catcode`\%=\active\def%{\%}\Neighbors{} \& \SharedVertices{}}}%
\end{pgfscope}%
\begin{pgfscope}%
\pgfsetrectcap%
\pgfsetroundjoin%
\pgfsetlinewidth{1.505625pt}%
\pgfsetstrokecolor{currentstroke5}%
\pgfsetdash{}{0pt}%
\pgfpathmoveto{\pgfqpoint{5.369911in}{1.833806in}}%
\pgfpathlineto{\pgfqpoint{5.494911in}{1.833806in}}%
\pgfpathlineto{\pgfqpoint{5.619911in}{1.833806in}}%
\pgfusepath{stroke}%
\end{pgfscope}%
\begin{pgfscope}%
\definecolor{textcolor}{rgb}{0.000000,0.000000,0.000000}%
\pgfsetstrokecolor{textcolor}%
\pgfsetfillcolor{textcolor}%
\pgftext[x=5.719911in,y=1.790056in,left,base]{\color{textcolor}{\rmfamily\fontsize{9.000000}{10.800000}\selectfont\catcode`\^=\active\def^{\ifmmode\sp\else\^{}\fi}\catcode`\%=\active\def%{\%}\NeighborsDegree{} \& \MergeLinear{}}}%
\end{pgfscope}%
\begin{pgfscope}%
\pgfsetrectcap%
\pgfsetroundjoin%
\pgfsetlinewidth{1.505625pt}%
\pgfsetstrokecolor{currentstroke6}%
\pgfsetdash{}{0pt}%
\pgfpathmoveto{\pgfqpoint{5.369911in}{1.646855in}}%
\pgfpathlineto{\pgfqpoint{5.494911in}{1.646855in}}%
\pgfpathlineto{\pgfqpoint{5.619911in}{1.646855in}}%
\pgfusepath{stroke}%
\end{pgfscope}%
\begin{pgfscope}%
\definecolor{textcolor}{rgb}{0.000000,0.000000,0.000000}%
\pgfsetstrokecolor{textcolor}%
\pgfsetfillcolor{textcolor}%
\pgftext[x=5.719911in,y=1.603105in,left,base]{\color{textcolor}{\rmfamily\fontsize{9.000000}{10.800000}\selectfont\catcode`\^=\active\def^{\ifmmode\sp\else\^{}\fi}\catcode`\%=\active\def%{\%}\NeighborsDegree{} \& \SharedVertices{}}}%
\end{pgfscope}%
\begin{pgfscope}%
\pgfsetrectcap%
\pgfsetroundjoin%
\pgfsetlinewidth{1.505625pt}%
\pgfsetstrokecolor{currentstroke7}%
\pgfsetdash{}{0pt}%
\pgfpathmoveto{\pgfqpoint{5.369911in}{1.459905in}}%
\pgfpathlineto{\pgfqpoint{5.494911in}{1.459905in}}%
\pgfpathlineto{\pgfqpoint{5.619911in}{1.459905in}}%
\pgfusepath{stroke}%
\end{pgfscope}%
\begin{pgfscope}%
\definecolor{textcolor}{rgb}{0.000000,0.000000,0.000000}%
\pgfsetstrokecolor{textcolor}%
\pgfsetfillcolor{textcolor}%
\pgftext[x=5.719911in,y=1.416155in,left,base]{\color{textcolor}{\rmfamily\fontsize{9.000000}{10.800000}\selectfont\catcode`\^=\active\def^{\ifmmode\sp\else\^{}\fi}\catcode`\%=\active\def%{\%}\None{} \& \MergeLinear{}}}%
\end{pgfscope}%
\begin{pgfscope}%
\pgfsetrectcap%
\pgfsetroundjoin%
\pgfsetlinewidth{1.505625pt}%
\definecolor{currentstroke}{rgb}{0.498039,0.498039,0.498039}%
\pgfsetstrokecolor{currentstroke}%
\pgfsetdash{}{0pt}%
\pgfpathmoveto{\pgfqpoint{5.369911in}{1.276433in}}%
\pgfpathlineto{\pgfqpoint{5.494911in}{1.276433in}}%
\pgfpathlineto{\pgfqpoint{5.619911in}{1.276433in}}%
\pgfusepath{stroke}%
\end{pgfscope}%
\begin{pgfscope}%
\definecolor{textcolor}{rgb}{0.000000,0.000000,0.000000}%
\pgfsetstrokecolor{textcolor}%
\pgfsetfillcolor{textcolor}%
\pgftext[x=5.719911in,y=1.232683in,left,base]{\color{textcolor}{\rmfamily\fontsize{9.000000}{10.800000}\selectfont\catcode`\^=\active\def^{\ifmmode\sp\else\^{}\fi}\catcode`\%=\active\def%{\%}\None{} \& \SharedVertices{}}}%
\end{pgfscope}%
\end{pgfpicture}%
\makeatother%
\endgroup%
}
	\caption[Mean runtime for graphs with no NAC-coloring]{
		Mean running time to finish search for graphs with no NAC-coloring.}%
	\label{fig:graph_no_nac_coloring_first_runtime}
\end{figure}%
\begin{figure}[thbp]
	\centering
	\scalebox{\BenchFigureScale}{%% Creator: Matplotlib, PGF backend
%%
%% To include the figure in your LaTeX document, write
%%   \input{<filename>.pgf}
%%
%% Make sure the required packages are loaded in your preamble
%%   \usepackage{pgf}
%%
%% Also ensure that all the required font packages are loaded; for instance,
%% the lmodern package is sometimes necessary when using math font.
%%   \usepackage{lmodern}
%%
%% Figures using additional raster images can only be included by \input if
%% they are in the same directory as the main LaTeX file. For loading figures
%% from other directories you can use the `import` package
%%   \usepackage{import}
%%
%% and then include the figures with
%%   \import{<path to file>}{<filename>.pgf}
%%
%% Matplotlib used the following preamble
%%   \def\mathdefault#1{#1}
%%   \everymath=\expandafter{\the\everymath\displaystyle}
%%   \IfFileExists{scrextend.sty}{
%%     \usepackage[fontsize=10.000000pt]{scrextend}
%%   }{
%%     \renewcommand{\normalsize}{\fontsize{10.000000}{12.000000}\selectfont}
%%     \normalsize
%%   }
%%   
%%   \ifdefined\pdftexversion\else  % non-pdftex case.
%%     \usepackage{fontspec}
%%     \setmainfont{DejaVuSans.ttf}[Path=\detokenize{/home/petr/Projects/PyRigi/.venv/lib/python3.12/site-packages/matplotlib/mpl-data/fonts/ttf/}]
%%     \setsansfont{DejaVuSans.ttf}[Path=\detokenize{/home/petr/Projects/PyRigi/.venv/lib/python3.12/site-packages/matplotlib/mpl-data/fonts/ttf/}]
%%     \setmonofont{DejaVuSansMono.ttf}[Path=\detokenize{/home/petr/Projects/PyRigi/.venv/lib/python3.12/site-packages/matplotlib/mpl-data/fonts/ttf/}]
%%   \fi
%%   \makeatletter\@ifpackageloaded{under\Score{}}{}{\usepackage[strings]{under\Score{}}}\makeatother
%%
\begingroup%
\makeatletter%
\begin{pgfpicture}%
\pgfpathrectangle{\pgfpointorigin}{\pgfqpoint{8.384376in}{2.841849in}}%
\pgfusepath{use as bounding box, clip}%
\begin{pgfscope}%
\pgfsetbuttcap%
\pgfsetmiterjoin%
\definecolor{currentfill}{rgb}{1.000000,1.000000,1.000000}%
\pgfsetfillcolor{currentfill}%
\pgfsetlinewidth{0.000000pt}%
\definecolor{currentstroke}{rgb}{1.000000,1.000000,1.000000}%
\pgfsetstrokecolor{currentstroke}%
\pgfsetdash{}{0pt}%
\pgfpathmoveto{\pgfqpoint{0.000000in}{0.000000in}}%
\pgfpathlineto{\pgfqpoint{8.384376in}{0.000000in}}%
\pgfpathlineto{\pgfqpoint{8.384376in}{2.841849in}}%
\pgfpathlineto{\pgfqpoint{0.000000in}{2.841849in}}%
\pgfpathlineto{\pgfqpoint{0.000000in}{0.000000in}}%
\pgfpathclose%
\pgfusepath{fill}%
\end{pgfscope}%
\begin{pgfscope}%
\pgfsetbuttcap%
\pgfsetmiterjoin%
\definecolor{currentfill}{rgb}{1.000000,1.000000,1.000000}%
\pgfsetfillcolor{currentfill}%
\pgfsetlinewidth{0.000000pt}%
\definecolor{currentstroke}{rgb}{0.000000,0.000000,0.000000}%
\pgfsetstrokecolor{currentstroke}%
\pgfsetstrokeopacity{0.000000}%
\pgfsetdash{}{0pt}%
\pgfpathmoveto{\pgfqpoint{0.588387in}{0.521603in}}%
\pgfpathlineto{\pgfqpoint{5.257411in}{0.521603in}}%
\pgfpathlineto{\pgfqpoint{5.257411in}{2.741849in}}%
\pgfpathlineto{\pgfqpoint{0.588387in}{2.741849in}}%
\pgfpathlineto{\pgfqpoint{0.588387in}{0.521603in}}%
\pgfpathclose%
\pgfusepath{fill}%
\end{pgfscope}%
\begin{pgfscope}%
\pgfsetbuttcap%
\pgfsetroundjoin%
\definecolor{currentfill}{rgb}{0.000000,0.000000,0.000000}%
\pgfsetfillcolor{currentfill}%
\pgfsetlinewidth{0.803000pt}%
\definecolor{currentstroke}{rgb}{0.000000,0.000000,0.000000}%
\pgfsetstrokecolor{currentstroke}%
\pgfsetdash{}{0pt}%
\pgfsys@defobject{currentmarker}{\pgfqpoint{0.000000in}{-0.048611in}}{\pgfqpoint{0.000000in}{0.000000in}}{%
\pgfpathmoveto{\pgfqpoint{0.000000in}{0.000000in}}%
\pgfpathlineto{\pgfqpoint{0.000000in}{-0.048611in}}%
\pgfusepath{stroke,fill}%
}%
\begin{pgfscope}%
\pgfsys@transformshift{1.009855in}{0.521603in}%
\pgfsys@useobject{currentmarker}{}%
\end{pgfscope}%
\end{pgfscope}%
\begin{pgfscope}%
\definecolor{textcolor}{rgb}{0.000000,0.000000,0.000000}%
\pgfsetstrokecolor{textcolor}%
\pgfsetfillcolor{textcolor}%
\pgftext[x=1.009855in,y=0.424381in,,top]{\color{textcolor}{\rmfamily\fontsize{10.000000}{12.000000}\selectfont\catcode`\^=\active\def^{\ifmmode\sp\else\^{}\fi}\catcode`\%=\active\def%{\%}$\mathdefault{20}$}}%
\end{pgfscope}%
\begin{pgfscope}%
\pgfsetbuttcap%
\pgfsetroundjoin%
\definecolor{currentfill}{rgb}{0.000000,0.000000,0.000000}%
\pgfsetfillcolor{currentfill}%
\pgfsetlinewidth{0.803000pt}%
\definecolor{currentstroke}{rgb}{0.000000,0.000000,0.000000}%
\pgfsetstrokecolor{currentstroke}%
\pgfsetdash{}{0pt}%
\pgfsys@defobject{currentmarker}{\pgfqpoint{0.000000in}{-0.048611in}}{\pgfqpoint{0.000000in}{0.000000in}}{%
\pgfpathmoveto{\pgfqpoint{0.000000in}{0.000000in}}%
\pgfpathlineto{\pgfqpoint{0.000000in}{-0.048611in}}%
\pgfusepath{stroke,fill}%
}%
\begin{pgfscope}%
\pgfsys@transformshift{1.607681in}{0.521603in}%
\pgfsys@useobject{currentmarker}{}%
\end{pgfscope}%
\end{pgfscope}%
\begin{pgfscope}%
\definecolor{textcolor}{rgb}{0.000000,0.000000,0.000000}%
\pgfsetstrokecolor{textcolor}%
\pgfsetfillcolor{textcolor}%
\pgftext[x=1.607681in,y=0.424381in,,top]{\color{textcolor}{\rmfamily\fontsize{10.000000}{12.000000}\selectfont\catcode`\^=\active\def^{\ifmmode\sp\else\^{}\fi}\catcode`\%=\active\def%{\%}$\mathdefault{40}$}}%
\end{pgfscope}%
\begin{pgfscope}%
\pgfsetbuttcap%
\pgfsetroundjoin%
\definecolor{currentfill}{rgb}{0.000000,0.000000,0.000000}%
\pgfsetfillcolor{currentfill}%
\pgfsetlinewidth{0.803000pt}%
\definecolor{currentstroke}{rgb}{0.000000,0.000000,0.000000}%
\pgfsetstrokecolor{currentstroke}%
\pgfsetdash{}{0pt}%
\pgfsys@defobject{currentmarker}{\pgfqpoint{0.000000in}{-0.048611in}}{\pgfqpoint{0.000000in}{0.000000in}}{%
\pgfpathmoveto{\pgfqpoint{0.000000in}{0.000000in}}%
\pgfpathlineto{\pgfqpoint{0.000000in}{-0.048611in}}%
\pgfusepath{stroke,fill}%
}%
\begin{pgfscope}%
\pgfsys@transformshift{2.205508in}{0.521603in}%
\pgfsys@useobject{currentmarker}{}%
\end{pgfscope}%
\end{pgfscope}%
\begin{pgfscope}%
\definecolor{textcolor}{rgb}{0.000000,0.000000,0.000000}%
\pgfsetstrokecolor{textcolor}%
\pgfsetfillcolor{textcolor}%
\pgftext[x=2.205508in,y=0.424381in,,top]{\color{textcolor}{\rmfamily\fontsize{10.000000}{12.000000}\selectfont\catcode`\^=\active\def^{\ifmmode\sp\else\^{}\fi}\catcode`\%=\active\def%{\%}$\mathdefault{60}$}}%
\end{pgfscope}%
\begin{pgfscope}%
\pgfsetbuttcap%
\pgfsetroundjoin%
\definecolor{currentfill}{rgb}{0.000000,0.000000,0.000000}%
\pgfsetfillcolor{currentfill}%
\pgfsetlinewidth{0.803000pt}%
\definecolor{currentstroke}{rgb}{0.000000,0.000000,0.000000}%
\pgfsetstrokecolor{currentstroke}%
\pgfsetdash{}{0pt}%
\pgfsys@defobject{currentmarker}{\pgfqpoint{0.000000in}{-0.048611in}}{\pgfqpoint{0.000000in}{0.000000in}}{%
\pgfpathmoveto{\pgfqpoint{0.000000in}{0.000000in}}%
\pgfpathlineto{\pgfqpoint{0.000000in}{-0.048611in}}%
\pgfusepath{stroke,fill}%
}%
\begin{pgfscope}%
\pgfsys@transformshift{2.803334in}{0.521603in}%
\pgfsys@useobject{currentmarker}{}%
\end{pgfscope}%
\end{pgfscope}%
\begin{pgfscope}%
\definecolor{textcolor}{rgb}{0.000000,0.000000,0.000000}%
\pgfsetstrokecolor{textcolor}%
\pgfsetfillcolor{textcolor}%
\pgftext[x=2.803334in,y=0.424381in,,top]{\color{textcolor}{\rmfamily\fontsize{10.000000}{12.000000}\selectfont\catcode`\^=\active\def^{\ifmmode\sp\else\^{}\fi}\catcode`\%=\active\def%{\%}$\mathdefault{80}$}}%
\end{pgfscope}%
\begin{pgfscope}%
\pgfsetbuttcap%
\pgfsetroundjoin%
\definecolor{currentfill}{rgb}{0.000000,0.000000,0.000000}%
\pgfsetfillcolor{currentfill}%
\pgfsetlinewidth{0.803000pt}%
\definecolor{currentstroke}{rgb}{0.000000,0.000000,0.000000}%
\pgfsetstrokecolor{currentstroke}%
\pgfsetdash{}{0pt}%
\pgfsys@defobject{currentmarker}{\pgfqpoint{0.000000in}{-0.048611in}}{\pgfqpoint{0.000000in}{0.000000in}}{%
\pgfpathmoveto{\pgfqpoint{0.000000in}{0.000000in}}%
\pgfpathlineto{\pgfqpoint{0.000000in}{-0.048611in}}%
\pgfusepath{stroke,fill}%
}%
\begin{pgfscope}%
\pgfsys@transformshift{3.401160in}{0.521603in}%
\pgfsys@useobject{currentmarker}{}%
\end{pgfscope}%
\end{pgfscope}%
\begin{pgfscope}%
\definecolor{textcolor}{rgb}{0.000000,0.000000,0.000000}%
\pgfsetstrokecolor{textcolor}%
\pgfsetfillcolor{textcolor}%
\pgftext[x=3.401160in,y=0.424381in,,top]{\color{textcolor}{\rmfamily\fontsize{10.000000}{12.000000}\selectfont\catcode`\^=\active\def^{\ifmmode\sp\else\^{}\fi}\catcode`\%=\active\def%{\%}$\mathdefault{100}$}}%
\end{pgfscope}%
\begin{pgfscope}%
\pgfsetbuttcap%
\pgfsetroundjoin%
\definecolor{currentfill}{rgb}{0.000000,0.000000,0.000000}%
\pgfsetfillcolor{currentfill}%
\pgfsetlinewidth{0.803000pt}%
\definecolor{currentstroke}{rgb}{0.000000,0.000000,0.000000}%
\pgfsetstrokecolor{currentstroke}%
\pgfsetdash{}{0pt}%
\pgfsys@defobject{currentmarker}{\pgfqpoint{0.000000in}{-0.048611in}}{\pgfqpoint{0.000000in}{0.000000in}}{%
\pgfpathmoveto{\pgfqpoint{0.000000in}{0.000000in}}%
\pgfpathlineto{\pgfqpoint{0.000000in}{-0.048611in}}%
\pgfusepath{stroke,fill}%
}%
\begin{pgfscope}%
\pgfsys@transformshift{3.998987in}{0.521603in}%
\pgfsys@useobject{currentmarker}{}%
\end{pgfscope}%
\end{pgfscope}%
\begin{pgfscope}%
\definecolor{textcolor}{rgb}{0.000000,0.000000,0.000000}%
\pgfsetstrokecolor{textcolor}%
\pgfsetfillcolor{textcolor}%
\pgftext[x=3.998987in,y=0.424381in,,top]{\color{textcolor}{\rmfamily\fontsize{10.000000}{12.000000}\selectfont\catcode`\^=\active\def^{\ifmmode\sp\else\^{}\fi}\catcode`\%=\active\def%{\%}$\mathdefault{120}$}}%
\end{pgfscope}%
\begin{pgfscope}%
\pgfsetbuttcap%
\pgfsetroundjoin%
\definecolor{currentfill}{rgb}{0.000000,0.000000,0.000000}%
\pgfsetfillcolor{currentfill}%
\pgfsetlinewidth{0.803000pt}%
\definecolor{currentstroke}{rgb}{0.000000,0.000000,0.000000}%
\pgfsetstrokecolor{currentstroke}%
\pgfsetdash{}{0pt}%
\pgfsys@defobject{currentmarker}{\pgfqpoint{0.000000in}{-0.048611in}}{\pgfqpoint{0.000000in}{0.000000in}}{%
\pgfpathmoveto{\pgfqpoint{0.000000in}{0.000000in}}%
\pgfpathlineto{\pgfqpoint{0.000000in}{-0.048611in}}%
\pgfusepath{stroke,fill}%
}%
\begin{pgfscope}%
\pgfsys@transformshift{4.596813in}{0.521603in}%
\pgfsys@useobject{currentmarker}{}%
\end{pgfscope}%
\end{pgfscope}%
\begin{pgfscope}%
\definecolor{textcolor}{rgb}{0.000000,0.000000,0.000000}%
\pgfsetstrokecolor{textcolor}%
\pgfsetfillcolor{textcolor}%
\pgftext[x=4.596813in,y=0.424381in,,top]{\color{textcolor}{\rmfamily\fontsize{10.000000}{12.000000}\selectfont\catcode`\^=\active\def^{\ifmmode\sp\else\^{}\fi}\catcode`\%=\active\def%{\%}$\mathdefault{140}$}}%
\end{pgfscope}%
\begin{pgfscope}%
\pgfsetbuttcap%
\pgfsetroundjoin%
\definecolor{currentfill}{rgb}{0.000000,0.000000,0.000000}%
\pgfsetfillcolor{currentfill}%
\pgfsetlinewidth{0.803000pt}%
\definecolor{currentstroke}{rgb}{0.000000,0.000000,0.000000}%
\pgfsetstrokecolor{currentstroke}%
\pgfsetdash{}{0pt}%
\pgfsys@defobject{currentmarker}{\pgfqpoint{0.000000in}{-0.048611in}}{\pgfqpoint{0.000000in}{0.000000in}}{%
\pgfpathmoveto{\pgfqpoint{0.000000in}{0.000000in}}%
\pgfpathlineto{\pgfqpoint{0.000000in}{-0.048611in}}%
\pgfusepath{stroke,fill}%
}%
\begin{pgfscope}%
\pgfsys@transformshift{5.194639in}{0.521603in}%
\pgfsys@useobject{currentmarker}{}%
\end{pgfscope}%
\end{pgfscope}%
\begin{pgfscope}%
\definecolor{textcolor}{rgb}{0.000000,0.000000,0.000000}%
\pgfsetstrokecolor{textcolor}%
\pgfsetfillcolor{textcolor}%
\pgftext[x=5.194639in,y=0.424381in,,top]{\color{textcolor}{\rmfamily\fontsize{10.000000}{12.000000}\selectfont\catcode`\^=\active\def^{\ifmmode\sp\else\^{}\fi}\catcode`\%=\active\def%{\%}$\mathdefault{160}$}}%
\end{pgfscope}%
\begin{pgfscope}%
\definecolor{textcolor}{rgb}{0.000000,0.000000,0.000000}%
\pgfsetstrokecolor{textcolor}%
\pgfsetfillcolor{textcolor}%
\pgftext[x=2.922899in,y=0.234413in,,top]{\color{textcolor}{\rmfamily\fontsize{10.000000}{12.000000}\selectfont\catcode`\^=\active\def^{\ifmmode\sp\else\^{}\fi}\catcode`\%=\active\def%{\%}Triangle components}}%
\end{pgfscope}%
\begin{pgfscope}%
\pgfsetbuttcap%
\pgfsetroundjoin%
\definecolor{currentfill}{rgb}{0.000000,0.000000,0.000000}%
\pgfsetfillcolor{currentfill}%
\pgfsetlinewidth{0.803000pt}%
\definecolor{currentstroke}{rgb}{0.000000,0.000000,0.000000}%
\pgfsetstrokecolor{currentstroke}%
\pgfsetdash{}{0pt}%
\pgfsys@defobject{currentmarker}{\pgfqpoint{-0.048611in}{0.000000in}}{\pgfqpoint{-0.000000in}{0.000000in}}{%
\pgfpathmoveto{\pgfqpoint{-0.000000in}{0.000000in}}%
\pgfpathlineto{\pgfqpoint{-0.048611in}{0.000000in}}%
\pgfusepath{stroke,fill}%
}%
\begin{pgfscope}%
\pgfsys@transformshift{0.588387in}{0.617054in}%
\pgfsys@useobject{currentmarker}{}%
\end{pgfscope}%
\end{pgfscope}%
\begin{pgfscope}%
\definecolor{textcolor}{rgb}{0.000000,0.000000,0.000000}%
\pgfsetstrokecolor{textcolor}%
\pgfsetfillcolor{textcolor}%
\pgftext[x=0.289968in, y=0.564293in, left, base]{\color{textcolor}{\rmfamily\fontsize{10.000000}{12.000000}\selectfont\catcode`\^=\active\def^{\ifmmode\sp\else\^{}\fi}\catcode`\%=\active\def%{\%}$\mathdefault{10^{2}}$}}%
\end{pgfscope}%
\begin{pgfscope}%
\pgfsetbuttcap%
\pgfsetroundjoin%
\definecolor{currentfill}{rgb}{0.000000,0.000000,0.000000}%
\pgfsetfillcolor{currentfill}%
\pgfsetlinewidth{0.803000pt}%
\definecolor{currentstroke}{rgb}{0.000000,0.000000,0.000000}%
\pgfsetstrokecolor{currentstroke}%
\pgfsetdash{}{0pt}%
\pgfsys@defobject{currentmarker}{\pgfqpoint{-0.048611in}{0.000000in}}{\pgfqpoint{-0.000000in}{0.000000in}}{%
\pgfpathmoveto{\pgfqpoint{-0.000000in}{0.000000in}}%
\pgfpathlineto{\pgfqpoint{-0.048611in}{0.000000in}}%
\pgfusepath{stroke,fill}%
}%
\begin{pgfscope}%
\pgfsys@transformshift{0.588387in}{2.274401in}%
\pgfsys@useobject{currentmarker}{}%
\end{pgfscope}%
\end{pgfscope}%
\begin{pgfscope}%
\definecolor{textcolor}{rgb}{0.000000,0.000000,0.000000}%
\pgfsetstrokecolor{textcolor}%
\pgfsetfillcolor{textcolor}%
\pgftext[x=0.289968in, y=2.221640in, left, base]{\color{textcolor}{\rmfamily\fontsize{10.000000}{12.000000}\selectfont\catcode`\^=\active\def^{\ifmmode\sp\else\^{}\fi}\catcode`\%=\active\def%{\%}$\mathdefault{10^{3}}$}}%
\end{pgfscope}%
\begin{pgfscope}%
\pgfsetbuttcap%
\pgfsetroundjoin%
\definecolor{currentfill}{rgb}{0.000000,0.000000,0.000000}%
\pgfsetfillcolor{currentfill}%
\pgfsetlinewidth{0.602250pt}%
\definecolor{currentstroke}{rgb}{0.000000,0.000000,0.000000}%
\pgfsetstrokecolor{currentstroke}%
\pgfsetdash{}{0pt}%
\pgfsys@defobject{currentmarker}{\pgfqpoint{-0.027778in}{0.000000in}}{\pgfqpoint{-0.000000in}{0.000000in}}{%
\pgfpathmoveto{\pgfqpoint{-0.000000in}{0.000000in}}%
\pgfpathlineto{\pgfqpoint{-0.027778in}{0.000000in}}%
\pgfusepath{stroke,fill}%
}%
\begin{pgfscope}%
\pgfsys@transformshift{0.588387in}{0.541218in}%
\pgfsys@useobject{currentmarker}{}%
\end{pgfscope}%
\end{pgfscope}%
\begin{pgfscope}%
\pgfsetbuttcap%
\pgfsetroundjoin%
\definecolor{currentfill}{rgb}{0.000000,0.000000,0.000000}%
\pgfsetfillcolor{currentfill}%
\pgfsetlinewidth{0.602250pt}%
\definecolor{currentstroke}{rgb}{0.000000,0.000000,0.000000}%
\pgfsetstrokecolor{currentstroke}%
\pgfsetdash{}{0pt}%
\pgfsys@defobject{currentmarker}{\pgfqpoint{-0.027778in}{0.000000in}}{\pgfqpoint{-0.000000in}{0.000000in}}{%
\pgfpathmoveto{\pgfqpoint{-0.000000in}{0.000000in}}%
\pgfpathlineto{\pgfqpoint{-0.027778in}{0.000000in}}%
\pgfusepath{stroke,fill}%
}%
\begin{pgfscope}%
\pgfsys@transformshift{0.588387in}{1.115965in}%
\pgfsys@useobject{currentmarker}{}%
\end{pgfscope}%
\end{pgfscope}%
\begin{pgfscope}%
\pgfsetbuttcap%
\pgfsetroundjoin%
\definecolor{currentfill}{rgb}{0.000000,0.000000,0.000000}%
\pgfsetfillcolor{currentfill}%
\pgfsetlinewidth{0.602250pt}%
\definecolor{currentstroke}{rgb}{0.000000,0.000000,0.000000}%
\pgfsetstrokecolor{currentstroke}%
\pgfsetdash{}{0pt}%
\pgfsys@defobject{currentmarker}{\pgfqpoint{-0.027778in}{0.000000in}}{\pgfqpoint{-0.000000in}{0.000000in}}{%
\pgfpathmoveto{\pgfqpoint{-0.000000in}{0.000000in}}%
\pgfpathlineto{\pgfqpoint{-0.027778in}{0.000000in}}%
\pgfusepath{stroke,fill}%
}%
\begin{pgfscope}%
\pgfsys@transformshift{0.588387in}{1.407810in}%
\pgfsys@useobject{currentmarker}{}%
\end{pgfscope}%
\end{pgfscope}%
\begin{pgfscope}%
\pgfsetbuttcap%
\pgfsetroundjoin%
\definecolor{currentfill}{rgb}{0.000000,0.000000,0.000000}%
\pgfsetfillcolor{currentfill}%
\pgfsetlinewidth{0.602250pt}%
\definecolor{currentstroke}{rgb}{0.000000,0.000000,0.000000}%
\pgfsetstrokecolor{currentstroke}%
\pgfsetdash{}{0pt}%
\pgfsys@defobject{currentmarker}{\pgfqpoint{-0.027778in}{0.000000in}}{\pgfqpoint{-0.000000in}{0.000000in}}{%
\pgfpathmoveto{\pgfqpoint{-0.000000in}{0.000000in}}%
\pgfpathlineto{\pgfqpoint{-0.027778in}{0.000000in}}%
\pgfusepath{stroke,fill}%
}%
\begin{pgfscope}%
\pgfsys@transformshift{0.588387in}{1.614876in}%
\pgfsys@useobject{currentmarker}{}%
\end{pgfscope}%
\end{pgfscope}%
\begin{pgfscope}%
\pgfsetbuttcap%
\pgfsetroundjoin%
\definecolor{currentfill}{rgb}{0.000000,0.000000,0.000000}%
\pgfsetfillcolor{currentfill}%
\pgfsetlinewidth{0.602250pt}%
\definecolor{currentstroke}{rgb}{0.000000,0.000000,0.000000}%
\pgfsetstrokecolor{currentstroke}%
\pgfsetdash{}{0pt}%
\pgfsys@defobject{currentmarker}{\pgfqpoint{-0.027778in}{0.000000in}}{\pgfqpoint{-0.000000in}{0.000000in}}{%
\pgfpathmoveto{\pgfqpoint{-0.000000in}{0.000000in}}%
\pgfpathlineto{\pgfqpoint{-0.027778in}{0.000000in}}%
\pgfusepath{stroke,fill}%
}%
\begin{pgfscope}%
\pgfsys@transformshift{0.588387in}{1.775490in}%
\pgfsys@useobject{currentmarker}{}%
\end{pgfscope}%
\end{pgfscope}%
\begin{pgfscope}%
\pgfsetbuttcap%
\pgfsetroundjoin%
\definecolor{currentfill}{rgb}{0.000000,0.000000,0.000000}%
\pgfsetfillcolor{currentfill}%
\pgfsetlinewidth{0.602250pt}%
\definecolor{currentstroke}{rgb}{0.000000,0.000000,0.000000}%
\pgfsetstrokecolor{currentstroke}%
\pgfsetdash{}{0pt}%
\pgfsys@defobject{currentmarker}{\pgfqpoint{-0.027778in}{0.000000in}}{\pgfqpoint{-0.000000in}{0.000000in}}{%
\pgfpathmoveto{\pgfqpoint{-0.000000in}{0.000000in}}%
\pgfpathlineto{\pgfqpoint{-0.027778in}{0.000000in}}%
\pgfusepath{stroke,fill}%
}%
\begin{pgfscope}%
\pgfsys@transformshift{0.588387in}{1.906721in}%
\pgfsys@useobject{currentmarker}{}%
\end{pgfscope}%
\end{pgfscope}%
\begin{pgfscope}%
\pgfsetbuttcap%
\pgfsetroundjoin%
\definecolor{currentfill}{rgb}{0.000000,0.000000,0.000000}%
\pgfsetfillcolor{currentfill}%
\pgfsetlinewidth{0.602250pt}%
\definecolor{currentstroke}{rgb}{0.000000,0.000000,0.000000}%
\pgfsetstrokecolor{currentstroke}%
\pgfsetdash{}{0pt}%
\pgfsys@defobject{currentmarker}{\pgfqpoint{-0.027778in}{0.000000in}}{\pgfqpoint{-0.000000in}{0.000000in}}{%
\pgfpathmoveto{\pgfqpoint{-0.000000in}{0.000000in}}%
\pgfpathlineto{\pgfqpoint{-0.027778in}{0.000000in}}%
\pgfusepath{stroke,fill}%
}%
\begin{pgfscope}%
\pgfsys@transformshift{0.588387in}{2.017675in}%
\pgfsys@useobject{currentmarker}{}%
\end{pgfscope}%
\end{pgfscope}%
\begin{pgfscope}%
\pgfsetbuttcap%
\pgfsetroundjoin%
\definecolor{currentfill}{rgb}{0.000000,0.000000,0.000000}%
\pgfsetfillcolor{currentfill}%
\pgfsetlinewidth{0.602250pt}%
\definecolor{currentstroke}{rgb}{0.000000,0.000000,0.000000}%
\pgfsetstrokecolor{currentstroke}%
\pgfsetdash{}{0pt}%
\pgfsys@defobject{currentmarker}{\pgfqpoint{-0.027778in}{0.000000in}}{\pgfqpoint{-0.000000in}{0.000000in}}{%
\pgfpathmoveto{\pgfqpoint{-0.000000in}{0.000000in}}%
\pgfpathlineto{\pgfqpoint{-0.027778in}{0.000000in}}%
\pgfusepath{stroke,fill}%
}%
\begin{pgfscope}%
\pgfsys@transformshift{0.588387in}{2.113788in}%
\pgfsys@useobject{currentmarker}{}%
\end{pgfscope}%
\end{pgfscope}%
\begin{pgfscope}%
\pgfsetbuttcap%
\pgfsetroundjoin%
\definecolor{currentfill}{rgb}{0.000000,0.000000,0.000000}%
\pgfsetfillcolor{currentfill}%
\pgfsetlinewidth{0.602250pt}%
\definecolor{currentstroke}{rgb}{0.000000,0.000000,0.000000}%
\pgfsetstrokecolor{currentstroke}%
\pgfsetdash{}{0pt}%
\pgfsys@defobject{currentmarker}{\pgfqpoint{-0.027778in}{0.000000in}}{\pgfqpoint{-0.000000in}{0.000000in}}{%
\pgfpathmoveto{\pgfqpoint{-0.000000in}{0.000000in}}%
\pgfpathlineto{\pgfqpoint{-0.027778in}{0.000000in}}%
\pgfusepath{stroke,fill}%
}%
\begin{pgfscope}%
\pgfsys@transformshift{0.588387in}{2.198565in}%
\pgfsys@useobject{currentmarker}{}%
\end{pgfscope}%
\end{pgfscope}%
\begin{pgfscope}%
\definecolor{textcolor}{rgb}{0.000000,0.000000,0.000000}%
\pgfsetstrokecolor{textcolor}%
\pgfsetfillcolor{textcolor}%
\pgftext[x=0.234413in,y=1.631726in,,bottom,rotate=90.000000]{\color{textcolor}{\rmfamily\fontsize{10.000000}{12.000000}\selectfont\catcode`\^=\active\def^{\ifmmode\sp\else\^{}\fi}\catcode`\%=\active\def%{\%}Checks [call]}}%
\end{pgfscope}%
\begin{pgfscope}%
\pgfpathrectangle{\pgfqpoint{0.588387in}{0.521603in}}{\pgfqpoint{4.669024in}{2.220246in}}%
\pgfusepath{clip}%
\pgfsetrectcap%
\pgfsetroundjoin%
\pgfsetlinewidth{1.505625pt}%
\pgfsetstrokecolor{currentstroke1}%
\pgfsetdash{}{0pt}%
\pgfpathmoveto{\pgfqpoint{0.800616in}{0.830864in}}%
\pgfpathlineto{\pgfqpoint{0.830507in}{1.131821in}}%
\pgfpathlineto{\pgfqpoint{0.860398in}{0.696192in}}%
\pgfpathlineto{\pgfqpoint{0.890290in}{0.680605in}}%
\pgfpathlineto{\pgfqpoint{0.920181in}{0.847420in}}%
\pgfpathlineto{\pgfqpoint{0.950072in}{1.015654in}}%
\pgfpathlineto{\pgfqpoint{0.979963in}{1.123583in}}%
\pgfpathlineto{\pgfqpoint{1.009855in}{0.826297in}}%
\pgfpathlineto{\pgfqpoint{1.039746in}{0.864990in}}%
\pgfpathlineto{\pgfqpoint{1.069637in}{1.044881in}}%
\pgfpathlineto{\pgfqpoint{1.099529in}{1.171189in}}%
\pgfpathlineto{\pgfqpoint{1.129420in}{1.191170in}}%
\pgfpathlineto{\pgfqpoint{1.159311in}{1.009794in}}%
\pgfpathlineto{\pgfqpoint{1.189203in}{1.091039in}}%
\pgfpathlineto{\pgfqpoint{1.219094in}{1.208475in}}%
\pgfpathlineto{\pgfqpoint{1.248985in}{1.225004in}}%
\pgfpathlineto{\pgfqpoint{1.278877in}{1.305678in}}%
\pgfpathlineto{\pgfqpoint{1.308768in}{1.182593in}}%
\pgfpathlineto{\pgfqpoint{1.338659in}{1.249649in}}%
\pgfpathlineto{\pgfqpoint{1.368551in}{1.264848in}}%
\pgfpathlineto{\pgfqpoint{1.398442in}{1.335519in}}%
\pgfpathlineto{\pgfqpoint{1.428333in}{1.421759in}}%
\pgfpathlineto{\pgfqpoint{1.458225in}{1.323882in}}%
\pgfpathlineto{\pgfqpoint{1.488116in}{1.293649in}}%
\pgfpathlineto{\pgfqpoint{1.518007in}{1.381994in}}%
\pgfpathlineto{\pgfqpoint{1.547899in}{1.430670in}}%
\pgfpathlineto{\pgfqpoint{1.577790in}{1.517966in}}%
\pgfpathlineto{\pgfqpoint{1.607681in}{1.370780in}}%
\pgfpathlineto{\pgfqpoint{1.637573in}{1.404603in}}%
\pgfpathlineto{\pgfqpoint{1.667464in}{1.438987in}}%
\pgfpathlineto{\pgfqpoint{1.697355in}{1.535883in}}%
\pgfpathlineto{\pgfqpoint{1.727246in}{1.548900in}}%
\pgfpathlineto{\pgfqpoint{1.757138in}{1.451033in}}%
\pgfpathlineto{\pgfqpoint{1.787029in}{1.521119in}}%
\pgfpathlineto{\pgfqpoint{1.816920in}{1.591205in}}%
\pgfpathlineto{\pgfqpoint{1.846812in}{1.592010in}}%
\pgfpathlineto{\pgfqpoint{1.876703in}{1.598000in}}%
\pgfpathlineto{\pgfqpoint{1.906594in}{1.543439in}}%
\pgfpathlineto{\pgfqpoint{1.936486in}{1.574563in}}%
\pgfpathlineto{\pgfqpoint{1.966377in}{1.567735in}}%
\pgfpathlineto{\pgfqpoint{1.996268in}{1.608924in}}%
\pgfpathlineto{\pgfqpoint{2.026160in}{1.737117in}}%
\pgfpathlineto{\pgfqpoint{2.056051in}{1.618256in}}%
\pgfpathlineto{\pgfqpoint{2.085942in}{1.586262in}}%
\pgfpathlineto{\pgfqpoint{2.115834in}{1.620612in}}%
\pgfpathlineto{\pgfqpoint{2.145725in}{1.704966in}}%
\pgfpathlineto{\pgfqpoint{2.175616in}{1.719475in}}%
\pgfpathlineto{\pgfqpoint{2.205508in}{1.638168in}}%
\pgfpathlineto{\pgfqpoint{2.235399in}{1.697834in}}%
\pgfpathlineto{\pgfqpoint{2.265290in}{1.673851in}}%
\pgfpathlineto{\pgfqpoint{2.295182in}{1.750095in}}%
\pgfpathlineto{\pgfqpoint{2.325073in}{1.715572in}}%
\pgfpathlineto{\pgfqpoint{2.354964in}{1.663576in}}%
\pgfpathlineto{\pgfqpoint{2.384855in}{1.729289in}}%
\pgfpathlineto{\pgfqpoint{2.414747in}{1.803720in}}%
\pgfpathlineto{\pgfqpoint{2.444638in}{1.783127in}}%
\pgfpathlineto{\pgfqpoint{2.474529in}{1.794433in}}%
\pgfpathlineto{\pgfqpoint{2.504421in}{1.794001in}}%
\pgfpathlineto{\pgfqpoint{2.534312in}{1.761927in}}%
\pgfpathlineto{\pgfqpoint{2.564203in}{1.816143in}}%
\pgfpathlineto{\pgfqpoint{2.594095in}{1.816241in}}%
\pgfpathlineto{\pgfqpoint{2.623986in}{1.864563in}}%
\pgfpathlineto{\pgfqpoint{2.653877in}{1.861720in}}%
\pgfpathlineto{\pgfqpoint{2.683769in}{1.825946in}}%
\pgfpathlineto{\pgfqpoint{2.713660in}{1.832932in}}%
\pgfpathlineto{\pgfqpoint{2.743551in}{1.898101in}}%
\pgfpathlineto{\pgfqpoint{2.773443in}{1.905760in}}%
\pgfpathlineto{\pgfqpoint{2.803334in}{1.871693in}}%
\pgfpathlineto{\pgfqpoint{2.833225in}{1.833546in}}%
\pgfpathlineto{\pgfqpoint{2.863117in}{1.953281in}}%
\pgfpathlineto{\pgfqpoint{2.893008in}{1.895522in}}%
\pgfpathlineto{\pgfqpoint{2.922899in}{1.878448in}}%
\pgfpathlineto{\pgfqpoint{2.952791in}{1.844615in}}%
\pgfpathlineto{\pgfqpoint{2.982682in}{1.922357in}}%
\pgfpathlineto{\pgfqpoint{3.012573in}{1.882009in}}%
\pgfpathlineto{\pgfqpoint{3.042464in}{1.912624in}}%
\pgfpathlineto{\pgfqpoint{3.072356in}{2.030973in}}%
\pgfpathlineto{\pgfqpoint{3.102247in}{1.990858in}}%
\pgfpathlineto{\pgfqpoint{3.132138in}{2.011389in}}%
\pgfpathlineto{\pgfqpoint{3.162030in}{1.947151in}}%
\pgfpathlineto{\pgfqpoint{3.191921in}{1.953736in}}%
\pgfpathlineto{\pgfqpoint{3.221812in}{2.024657in}}%
\pgfpathlineto{\pgfqpoint{3.251704in}{1.973138in}}%
\pgfpathlineto{\pgfqpoint{3.281595in}{1.944120in}}%
\pgfpathlineto{\pgfqpoint{3.311486in}{1.939197in}}%
\pgfpathlineto{\pgfqpoint{3.341378in}{2.009646in}}%
\pgfpathlineto{\pgfqpoint{3.371269in}{2.048335in}}%
\pgfpathlineto{\pgfqpoint{3.401160in}{1.978708in}}%
\pgfpathlineto{\pgfqpoint{3.431052in}{2.094333in}}%
\pgfpathlineto{\pgfqpoint{3.460943in}{1.966545in}}%
\pgfpathlineto{\pgfqpoint{3.490834in}{1.986147in}}%
\pgfpathlineto{\pgfqpoint{3.520726in}{2.073655in}}%
\pgfpathlineto{\pgfqpoint{3.550617in}{1.987633in}}%
\pgfpathlineto{\pgfqpoint{3.580508in}{2.050439in}}%
\pgfpathlineto{\pgfqpoint{3.610400in}{2.085154in}}%
\pgfpathlineto{\pgfqpoint{3.640291in}{1.944120in}}%
\pgfpathlineto{\pgfqpoint{3.670182in}{2.085154in}}%
\pgfpathlineto{\pgfqpoint{3.700073in}{1.979672in}}%
\pgfpathlineto{\pgfqpoint{3.729965in}{2.047880in}}%
\pgfpathlineto{\pgfqpoint{3.759856in}{2.117378in}}%
\pgfpathlineto{\pgfqpoint{3.789747in}{2.125010in}}%
\pgfpathlineto{\pgfqpoint{3.819639in}{2.007318in}}%
\pgfpathlineto{\pgfqpoint{3.849530in}{2.023410in}}%
\pgfpathlineto{\pgfqpoint{3.879421in}{1.991711in}}%
\pgfpathlineto{\pgfqpoint{3.909313in}{2.008013in}}%
\pgfpathlineto{\pgfqpoint{3.939204in}{2.309976in}}%
\pgfpathlineto{\pgfqpoint{3.969095in}{2.050635in}}%
\pgfpathlineto{\pgfqpoint{3.998987in}{2.021776in}}%
\pgfpathlineto{\pgfqpoint{4.028878in}{2.029909in}}%
\pgfpathlineto{\pgfqpoint{4.058769in}{2.173612in}}%
\pgfpathlineto{\pgfqpoint{4.088661in}{2.074157in}}%
\pgfpathlineto{\pgfqpoint{4.118552in}{2.049062in}}%
\pgfpathlineto{\pgfqpoint{4.148443in}{2.053771in}}%
\pgfpathlineto{\pgfqpoint{4.178335in}{2.084405in}}%
\pgfpathlineto{\pgfqpoint{4.208226in}{2.096794in}}%
\pgfpathlineto{\pgfqpoint{4.238117in}{2.069251in}}%
\pgfpathlineto{\pgfqpoint{4.268009in}{2.073070in}}%
\pgfpathlineto{\pgfqpoint{4.297900in}{2.099246in}}%
\pgfpathlineto{\pgfqpoint{4.327791in}{2.117378in}}%
\pgfpathlineto{\pgfqpoint{4.417465in}{2.238995in}}%
\pgfpathlineto{\pgfqpoint{4.566922in}{2.162487in}}%
\pgfusepath{stroke}%
\end{pgfscope}%
\begin{pgfscope}%
\pgfpathrectangle{\pgfqpoint{0.588387in}{0.521603in}}{\pgfqpoint{4.669024in}{2.220246in}}%
\pgfusepath{clip}%
\pgfsetrectcap%
\pgfsetroundjoin%
\pgfsetlinewidth{1.505625pt}%
\pgfsetstrokecolor{currentstroke2}%
\pgfsetdash{}{0pt}%
\pgfpathmoveto{\pgfqpoint{0.800616in}{0.842213in}}%
\pgfpathlineto{\pgfqpoint{0.830507in}{1.131821in}}%
\pgfpathlineto{\pgfqpoint{0.860398in}{0.696320in}}%
\pgfpathlineto{\pgfqpoint{0.890290in}{0.685490in}}%
\pgfpathlineto{\pgfqpoint{0.950072in}{1.015911in}}%
\pgfpathlineto{\pgfqpoint{0.979963in}{1.121475in}}%
\pgfpathlineto{\pgfqpoint{1.009855in}{0.826393in}}%
\pgfpathlineto{\pgfqpoint{1.039746in}{0.866935in}}%
\pgfpathlineto{\pgfqpoint{1.069637in}{1.047363in}}%
\pgfpathlineto{\pgfqpoint{1.099529in}{1.175280in}}%
\pgfpathlineto{\pgfqpoint{1.129420in}{1.198592in}}%
\pgfpathlineto{\pgfqpoint{1.159311in}{1.013580in}}%
\pgfpathlineto{\pgfqpoint{1.189203in}{1.092519in}}%
\pgfpathlineto{\pgfqpoint{1.219094in}{1.208475in}}%
\pgfpathlineto{\pgfqpoint{1.248985in}{1.222794in}}%
\pgfpathlineto{\pgfqpoint{1.278877in}{1.308221in}}%
\pgfpathlineto{\pgfqpoint{1.308768in}{1.184274in}}%
\pgfpathlineto{\pgfqpoint{1.338659in}{1.250645in}}%
\pgfpathlineto{\pgfqpoint{1.368551in}{1.268391in}}%
\pgfpathlineto{\pgfqpoint{1.398442in}{1.337033in}}%
\pgfpathlineto{\pgfqpoint{1.428333in}{1.419707in}}%
\pgfpathlineto{\pgfqpoint{1.458225in}{1.339016in}}%
\pgfpathlineto{\pgfqpoint{1.488116in}{1.289086in}}%
\pgfpathlineto{\pgfqpoint{1.518007in}{1.394183in}}%
\pgfpathlineto{\pgfqpoint{1.547899in}{1.436776in}}%
\pgfpathlineto{\pgfqpoint{1.577790in}{1.515946in}}%
\pgfpathlineto{\pgfqpoint{1.607681in}{1.373849in}}%
\pgfpathlineto{\pgfqpoint{1.667464in}{1.438987in}}%
\pgfpathlineto{\pgfqpoint{1.697355in}{1.545194in}}%
\pgfpathlineto{\pgfqpoint{1.727246in}{1.557822in}}%
\pgfpathlineto{\pgfqpoint{1.757138in}{1.451033in}}%
\pgfpathlineto{\pgfqpoint{1.816920in}{1.598172in}}%
\pgfpathlineto{\pgfqpoint{1.846812in}{1.586326in}}%
\pgfpathlineto{\pgfqpoint{1.876703in}{1.603128in}}%
\pgfpathlineto{\pgfqpoint{1.906594in}{1.560771in}}%
\pgfpathlineto{\pgfqpoint{1.936486in}{1.584855in}}%
\pgfpathlineto{\pgfqpoint{1.966377in}{1.581237in}}%
\pgfpathlineto{\pgfqpoint{1.996268in}{1.608355in}}%
\pgfpathlineto{\pgfqpoint{2.026160in}{1.727435in}}%
\pgfpathlineto{\pgfqpoint{2.056051in}{1.614876in}}%
\pgfpathlineto{\pgfqpoint{2.085942in}{1.595801in}}%
\pgfpathlineto{\pgfqpoint{2.115834in}{1.611027in}}%
\pgfpathlineto{\pgfqpoint{2.145725in}{1.704401in}}%
\pgfpathlineto{\pgfqpoint{2.175616in}{1.724722in}}%
\pgfpathlineto{\pgfqpoint{2.205508in}{1.650527in}}%
\pgfpathlineto{\pgfqpoint{2.235399in}{1.695717in}}%
\pgfpathlineto{\pgfqpoint{2.265290in}{1.686492in}}%
\pgfpathlineto{\pgfqpoint{2.295182in}{1.750095in}}%
\pgfpathlineto{\pgfqpoint{2.325073in}{1.721706in}}%
\pgfpathlineto{\pgfqpoint{2.354964in}{1.663576in}}%
\pgfpathlineto{\pgfqpoint{2.384855in}{1.729990in}}%
\pgfpathlineto{\pgfqpoint{2.414747in}{1.780681in}}%
\pgfpathlineto{\pgfqpoint{2.444638in}{1.797418in}}%
\pgfpathlineto{\pgfqpoint{2.474529in}{1.818697in}}%
\pgfpathlineto{\pgfqpoint{2.504421in}{1.820558in}}%
\pgfpathlineto{\pgfqpoint{2.534312in}{1.763880in}}%
\pgfpathlineto{\pgfqpoint{2.564203in}{1.802262in}}%
\pgfpathlineto{\pgfqpoint{2.594095in}{1.834209in}}%
\pgfpathlineto{\pgfqpoint{2.623986in}{1.869969in}}%
\pgfpathlineto{\pgfqpoint{2.653877in}{1.857061in}}%
\pgfpathlineto{\pgfqpoint{2.683769in}{1.815548in}}%
\pgfpathlineto{\pgfqpoint{2.713660in}{1.850706in}}%
\pgfpathlineto{\pgfqpoint{2.743551in}{1.898216in}}%
\pgfpathlineto{\pgfqpoint{2.773443in}{1.905760in}}%
\pgfpathlineto{\pgfqpoint{2.803334in}{1.881698in}}%
\pgfpathlineto{\pgfqpoint{2.833225in}{1.849308in}}%
\pgfpathlineto{\pgfqpoint{2.863117in}{1.961798in}}%
\pgfpathlineto{\pgfqpoint{2.893008in}{1.889339in}}%
\pgfpathlineto{\pgfqpoint{2.922899in}{1.895978in}}%
\pgfpathlineto{\pgfqpoint{2.952791in}{1.857061in}}%
\pgfpathlineto{\pgfqpoint{2.982682in}{1.921251in}}%
\pgfpathlineto{\pgfqpoint{3.012573in}{1.882009in}}%
\pgfpathlineto{\pgfqpoint{3.042464in}{1.915976in}}%
\pgfpathlineto{\pgfqpoint{3.072356in}{2.014638in}}%
\pgfpathlineto{\pgfqpoint{3.102247in}{1.940466in}}%
\pgfpathlineto{\pgfqpoint{3.132138in}{1.996074in}}%
\pgfpathlineto{\pgfqpoint{3.162030in}{1.931611in}}%
\pgfpathlineto{\pgfqpoint{3.191921in}{1.979672in}}%
\pgfpathlineto{\pgfqpoint{3.221812in}{2.025615in}}%
\pgfpathlineto{\pgfqpoint{3.251704in}{2.011479in}}%
\pgfpathlineto{\pgfqpoint{3.281595in}{1.942462in}}%
\pgfpathlineto{\pgfqpoint{3.311486in}{1.953866in}}%
\pgfpathlineto{\pgfqpoint{3.341378in}{2.019366in}}%
\pgfpathlineto{\pgfqpoint{3.371269in}{2.041021in}}%
\pgfpathlineto{\pgfqpoint{3.401160in}{2.034388in}}%
\pgfpathlineto{\pgfqpoint{3.431052in}{2.091039in}}%
\pgfpathlineto{\pgfqpoint{3.460943in}{1.968309in}}%
\pgfpathlineto{\pgfqpoint{3.490834in}{1.986147in}}%
\pgfpathlineto{\pgfqpoint{3.520726in}{2.094996in}}%
\pgfpathlineto{\pgfqpoint{3.550617in}{1.992891in}}%
\pgfpathlineto{\pgfqpoint{3.580508in}{2.062906in}}%
\pgfpathlineto{\pgfqpoint{3.610400in}{2.085154in}}%
\pgfpathlineto{\pgfqpoint{3.640291in}{1.944120in}}%
\pgfpathlineto{\pgfqpoint{3.670182in}{2.024226in}}%
\pgfpathlineto{\pgfqpoint{3.700073in}{2.177136in}}%
\pgfpathlineto{\pgfqpoint{3.729965in}{2.045905in}}%
\pgfpathlineto{\pgfqpoint{3.759856in}{2.113788in}}%
\pgfpathlineto{\pgfqpoint{3.789747in}{2.135936in}}%
\pgfpathlineto{\pgfqpoint{3.819639in}{2.047880in}}%
\pgfpathlineto{\pgfqpoint{3.849530in}{2.023410in}}%
\pgfpathlineto{\pgfqpoint{3.879421in}{1.991711in}}%
\pgfpathlineto{\pgfqpoint{3.909313in}{2.011479in}}%
\pgfpathlineto{\pgfqpoint{3.939204in}{2.443682in}}%
\pgfpathlineto{\pgfqpoint{3.969095in}{2.050635in}}%
\pgfpathlineto{\pgfqpoint{3.998987in}{2.021776in}}%
\pgfpathlineto{\pgfqpoint{4.028878in}{2.029909in}}%
\pgfpathlineto{\pgfqpoint{4.058769in}{2.278707in}}%
\pgfpathlineto{\pgfqpoint{4.088661in}{2.078287in}}%
\pgfpathlineto{\pgfqpoint{4.118552in}{2.041939in}}%
\pgfpathlineto{\pgfqpoint{4.148443in}{2.095564in}}%
\pgfpathlineto{\pgfqpoint{4.178335in}{2.099246in}}%
\pgfpathlineto{\pgfqpoint{4.208226in}{2.106554in}}%
\pgfpathlineto{\pgfqpoint{4.238117in}{2.076868in}}%
\pgfpathlineto{\pgfqpoint{4.268009in}{2.130682in}}%
\pgfpathlineto{\pgfqpoint{4.297900in}{2.094641in}}%
\pgfpathlineto{\pgfqpoint{4.327791in}{2.118570in}}%
\pgfpathlineto{\pgfqpoint{4.357683in}{2.128041in}}%
\pgfpathlineto{\pgfqpoint{4.387574in}{2.477498in}}%
\pgfpathlineto{\pgfqpoint{4.417465in}{2.135063in}}%
\pgfpathlineto{\pgfqpoint{4.447356in}{2.142018in}}%
\pgfpathlineto{\pgfqpoint{4.477248in}{2.106554in}}%
\pgfpathlineto{\pgfqpoint{4.507139in}{2.120950in}}%
\pgfpathlineto{\pgfqpoint{4.537030in}{2.169182in}}%
\pgfpathlineto{\pgfqpoint{4.566922in}{2.162487in}}%
\pgfpathlineto{\pgfqpoint{4.656596in}{2.148906in}}%
\pgfpathlineto{\pgfqpoint{4.716378in}{2.155728in}}%
\pgfpathlineto{\pgfqpoint{4.716378in}{2.155728in}}%
\pgfusepath{stroke}%
\end{pgfscope}%
\begin{pgfscope}%
\pgfpathrectangle{\pgfqpoint{0.588387in}{0.521603in}}{\pgfqpoint{4.669024in}{2.220246in}}%
\pgfusepath{clip}%
\pgfsetrectcap%
\pgfsetroundjoin%
\pgfsetlinewidth{1.505625pt}%
\pgfsetstrokecolor{currentstroke3}%
\pgfsetdash{}{0pt}%
\pgfpathmoveto{\pgfqpoint{0.800616in}{0.722227in}}%
\pgfpathlineto{\pgfqpoint{0.830507in}{1.031008in}}%
\pgfpathlineto{\pgfqpoint{0.860398in}{0.705776in}}%
\pgfpathlineto{\pgfqpoint{0.890290in}{0.622893in}}%
\pgfpathlineto{\pgfqpoint{0.920181in}{0.759775in}}%
\pgfpathlineto{\pgfqpoint{0.950072in}{0.929663in}}%
\pgfpathlineto{\pgfqpoint{0.979963in}{1.055849in}}%
\pgfpathlineto{\pgfqpoint{1.009855in}{0.776936in}}%
\pgfpathlineto{\pgfqpoint{1.039746in}{0.833505in}}%
\pgfpathlineto{\pgfqpoint{1.069637in}{0.993930in}}%
\pgfpathlineto{\pgfqpoint{1.099529in}{1.131226in}}%
\pgfpathlineto{\pgfqpoint{1.129420in}{1.124064in}}%
\pgfpathlineto{\pgfqpoint{1.159311in}{0.971227in}}%
\pgfpathlineto{\pgfqpoint{1.189203in}{1.056521in}}%
\pgfpathlineto{\pgfqpoint{1.219094in}{1.173921in}}%
\pgfpathlineto{\pgfqpoint{1.248985in}{1.160461in}}%
\pgfpathlineto{\pgfqpoint{1.278877in}{1.233253in}}%
\pgfpathlineto{\pgfqpoint{1.308768in}{1.153001in}}%
\pgfpathlineto{\pgfqpoint{1.338659in}{1.218926in}}%
\pgfpathlineto{\pgfqpoint{1.368551in}{1.213112in}}%
\pgfpathlineto{\pgfqpoint{1.398442in}{1.301445in}}%
\pgfpathlineto{\pgfqpoint{1.428333in}{1.362667in}}%
\pgfpathlineto{\pgfqpoint{1.458225in}{1.300674in}}%
\pgfpathlineto{\pgfqpoint{1.488116in}{1.257469in}}%
\pgfpathlineto{\pgfqpoint{1.547899in}{1.411374in}}%
\pgfpathlineto{\pgfqpoint{1.577790in}{1.479998in}}%
\pgfpathlineto{\pgfqpoint{1.607681in}{1.338690in}}%
\pgfpathlineto{\pgfqpoint{1.637573in}{1.364828in}}%
\pgfpathlineto{\pgfqpoint{1.667464in}{1.431987in}}%
\pgfpathlineto{\pgfqpoint{1.697355in}{1.517894in}}%
\pgfpathlineto{\pgfqpoint{1.727246in}{1.508426in}}%
\pgfpathlineto{\pgfqpoint{1.757138in}{1.436955in}}%
\pgfpathlineto{\pgfqpoint{1.787029in}{1.478509in}}%
\pgfpathlineto{\pgfqpoint{1.816920in}{1.543487in}}%
\pgfpathlineto{\pgfqpoint{1.846812in}{1.529303in}}%
\pgfpathlineto{\pgfqpoint{1.876703in}{1.561953in}}%
\pgfpathlineto{\pgfqpoint{1.906594in}{1.537673in}}%
\pgfpathlineto{\pgfqpoint{1.936486in}{1.567810in}}%
\pgfpathlineto{\pgfqpoint{1.966377in}{1.564476in}}%
\pgfpathlineto{\pgfqpoint{1.996268in}{1.596827in}}%
\pgfpathlineto{\pgfqpoint{2.026160in}{1.664433in}}%
\pgfpathlineto{\pgfqpoint{2.056051in}{1.622476in}}%
\pgfpathlineto{\pgfqpoint{2.085942in}{1.594474in}}%
\pgfpathlineto{\pgfqpoint{2.115834in}{1.631737in}}%
\pgfpathlineto{\pgfqpoint{2.145725in}{1.694839in}}%
\pgfpathlineto{\pgfqpoint{2.175616in}{1.745553in}}%
\pgfpathlineto{\pgfqpoint{2.205508in}{1.643107in}}%
\pgfpathlineto{\pgfqpoint{2.235399in}{1.670196in}}%
\pgfpathlineto{\pgfqpoint{2.265290in}{1.712897in}}%
\pgfpathlineto{\pgfqpoint{2.295182in}{1.757689in}}%
\pgfpathlineto{\pgfqpoint{2.325073in}{1.750490in}}%
\pgfpathlineto{\pgfqpoint{2.354964in}{1.698287in}}%
\pgfpathlineto{\pgfqpoint{2.384855in}{1.751629in}}%
\pgfpathlineto{\pgfqpoint{2.414747in}{1.810850in}}%
\pgfpathlineto{\pgfqpoint{2.444638in}{1.779025in}}%
\pgfpathlineto{\pgfqpoint{2.474529in}{1.804977in}}%
\pgfpathlineto{\pgfqpoint{2.504421in}{1.782221in}}%
\pgfpathlineto{\pgfqpoint{2.534312in}{1.802373in}}%
\pgfpathlineto{\pgfqpoint{2.564203in}{1.799941in}}%
\pgfpathlineto{\pgfqpoint{2.594095in}{1.835012in}}%
\pgfpathlineto{\pgfqpoint{2.623986in}{1.887952in}}%
\pgfpathlineto{\pgfqpoint{2.653877in}{1.845456in}}%
\pgfpathlineto{\pgfqpoint{2.683769in}{1.825029in}}%
\pgfpathlineto{\pgfqpoint{2.713660in}{1.867595in}}%
\pgfpathlineto{\pgfqpoint{2.743551in}{1.900294in}}%
\pgfpathlineto{\pgfqpoint{2.773443in}{1.924447in}}%
\pgfpathlineto{\pgfqpoint{2.803334in}{1.881761in}}%
\pgfpathlineto{\pgfqpoint{2.833225in}{1.881229in}}%
\pgfpathlineto{\pgfqpoint{2.863117in}{1.915013in}}%
\pgfpathlineto{\pgfqpoint{2.893008in}{1.951265in}}%
\pgfpathlineto{\pgfqpoint{2.922899in}{1.963646in}}%
\pgfpathlineto{\pgfqpoint{2.952791in}{1.916517in}}%
\pgfpathlineto{\pgfqpoint{2.982682in}{1.933841in}}%
\pgfpathlineto{\pgfqpoint{3.012573in}{1.978788in}}%
\pgfpathlineto{\pgfqpoint{3.042464in}{1.976742in}}%
\pgfpathlineto{\pgfqpoint{3.072356in}{1.996281in}}%
\pgfpathlineto{\pgfqpoint{3.102247in}{1.975234in}}%
\pgfpathlineto{\pgfqpoint{3.132138in}{2.024647in}}%
\pgfpathlineto{\pgfqpoint{3.162030in}{1.983379in}}%
\pgfpathlineto{\pgfqpoint{3.191921in}{2.007454in}}%
\pgfpathlineto{\pgfqpoint{3.221812in}{2.070391in}}%
\pgfpathlineto{\pgfqpoint{3.251704in}{2.060654in}}%
\pgfpathlineto{\pgfqpoint{3.281595in}{1.995175in}}%
\pgfpathlineto{\pgfqpoint{3.311486in}{2.026362in}}%
\pgfpathlineto{\pgfqpoint{3.341378in}{2.062249in}}%
\pgfpathlineto{\pgfqpoint{3.371269in}{2.079704in}}%
\pgfpathlineto{\pgfqpoint{3.401160in}{2.080256in}}%
\pgfpathlineto{\pgfqpoint{3.431052in}{2.087352in}}%
\pgfpathlineto{\pgfqpoint{3.460943in}{2.060979in}}%
\pgfpathlineto{\pgfqpoint{3.490834in}{2.107256in}}%
\pgfpathlineto{\pgfqpoint{3.520726in}{2.127643in}}%
\pgfpathlineto{\pgfqpoint{3.550617in}{2.057339in}}%
\pgfpathlineto{\pgfqpoint{3.580508in}{2.083842in}}%
\pgfpathlineto{\pgfqpoint{3.610400in}{2.098853in}}%
\pgfpathlineto{\pgfqpoint{3.640291in}{2.099678in}}%
\pgfpathlineto{\pgfqpoint{3.670182in}{2.122844in}}%
\pgfpathlineto{\pgfqpoint{3.700073in}{2.129048in}}%
\pgfpathlineto{\pgfqpoint{3.729965in}{2.148398in}}%
\pgfpathlineto{\pgfqpoint{3.759856in}{2.157787in}}%
\pgfpathlineto{\pgfqpoint{3.789747in}{2.153688in}}%
\pgfpathlineto{\pgfqpoint{3.819639in}{2.206857in}}%
\pgfpathlineto{\pgfqpoint{3.849530in}{2.145832in}}%
\pgfpathlineto{\pgfqpoint{3.879421in}{2.137057in}}%
\pgfpathlineto{\pgfqpoint{3.939204in}{2.290909in}}%
\pgfpathlineto{\pgfqpoint{3.969095in}{2.208925in}}%
\pgfpathlineto{\pgfqpoint{3.998987in}{2.175556in}}%
\pgfpathlineto{\pgfqpoint{4.028878in}{2.168670in}}%
\pgfpathlineto{\pgfqpoint{4.058769in}{2.203030in}}%
\pgfpathlineto{\pgfqpoint{4.088661in}{2.222915in}}%
\pgfpathlineto{\pgfqpoint{4.118552in}{2.245129in}}%
\pgfpathlineto{\pgfqpoint{4.178335in}{2.219970in}}%
\pgfpathlineto{\pgfqpoint{4.208226in}{2.268837in}}%
\pgfpathlineto{\pgfqpoint{4.238117in}{2.268438in}}%
\pgfpathlineto{\pgfqpoint{4.268009in}{2.247157in}}%
\pgfpathlineto{\pgfqpoint{4.297900in}{2.234749in}}%
\pgfpathlineto{\pgfqpoint{4.327791in}{2.261327in}}%
\pgfpathlineto{\pgfqpoint{4.357683in}{2.298776in}}%
\pgfpathlineto{\pgfqpoint{4.387574in}{2.351244in}}%
\pgfpathlineto{\pgfqpoint{4.417465in}{2.400656in}}%
\pgfpathlineto{\pgfqpoint{4.447356in}{2.280672in}}%
\pgfpathlineto{\pgfqpoint{4.477248in}{2.262792in}}%
\pgfpathlineto{\pgfqpoint{4.507139in}{2.310718in}}%
\pgfpathlineto{\pgfqpoint{4.537030in}{2.402476in}}%
\pgfpathlineto{\pgfqpoint{4.566922in}{2.396882in}}%
\pgfpathlineto{\pgfqpoint{4.596813in}{2.353397in}}%
\pgfpathlineto{\pgfqpoint{4.626704in}{2.465456in}}%
\pgfpathlineto{\pgfqpoint{4.656596in}{2.428071in}}%
\pgfpathlineto{\pgfqpoint{4.686487in}{2.511426in}}%
\pgfpathlineto{\pgfqpoint{4.716378in}{2.477136in}}%
\pgfpathlineto{\pgfqpoint{4.746270in}{2.480207in}}%
\pgfpathlineto{\pgfqpoint{4.776161in}{2.345616in}}%
\pgfpathlineto{\pgfqpoint{4.806052in}{2.524765in}}%
\pgfpathlineto{\pgfqpoint{4.835944in}{2.504140in}}%
\pgfpathlineto{\pgfqpoint{4.895726in}{2.497307in}}%
\pgfpathlineto{\pgfqpoint{4.955509in}{2.532853in}}%
\pgfpathlineto{\pgfqpoint{4.955509in}{2.532853in}}%
\pgfusepath{stroke}%
\end{pgfscope}%
\begin{pgfscope}%
\pgfpathrectangle{\pgfqpoint{0.588387in}{0.521603in}}{\pgfqpoint{4.669024in}{2.220246in}}%
\pgfusepath{clip}%
\pgfsetrectcap%
\pgfsetroundjoin%
\pgfsetlinewidth{1.505625pt}%
\pgfsetstrokecolor{currentstroke4}%
\pgfsetdash{}{0pt}%
\pgfpathmoveto{\pgfqpoint{0.800616in}{0.830635in}}%
\pgfpathlineto{\pgfqpoint{0.830507in}{1.139005in}}%
\pgfpathlineto{\pgfqpoint{0.860398in}{0.701688in}}%
\pgfpathlineto{\pgfqpoint{0.890290in}{0.677761in}}%
\pgfpathlineto{\pgfqpoint{0.920181in}{0.841081in}}%
\pgfpathlineto{\pgfqpoint{0.950072in}{1.013748in}}%
\pgfpathlineto{\pgfqpoint{0.979963in}{1.127363in}}%
\pgfpathlineto{\pgfqpoint{1.009855in}{0.821442in}}%
\pgfpathlineto{\pgfqpoint{1.039746in}{0.862287in}}%
\pgfpathlineto{\pgfqpoint{1.069637in}{1.043426in}}%
\pgfpathlineto{\pgfqpoint{1.099529in}{1.177447in}}%
\pgfpathlineto{\pgfqpoint{1.129420in}{1.187144in}}%
\pgfpathlineto{\pgfqpoint{1.159311in}{1.005888in}}%
\pgfpathlineto{\pgfqpoint{1.189203in}{1.085790in}}%
\pgfpathlineto{\pgfqpoint{1.219094in}{1.205720in}}%
\pgfpathlineto{\pgfqpoint{1.248985in}{1.220340in}}%
\pgfpathlineto{\pgfqpoint{1.278877in}{1.304635in}}%
\pgfpathlineto{\pgfqpoint{1.308768in}{1.180740in}}%
\pgfpathlineto{\pgfqpoint{1.338659in}{1.245570in}}%
\pgfpathlineto{\pgfqpoint{1.368551in}{1.262503in}}%
\pgfpathlineto{\pgfqpoint{1.398442in}{1.341682in}}%
\pgfpathlineto{\pgfqpoint{1.428333in}{1.404435in}}%
\pgfpathlineto{\pgfqpoint{1.458225in}{1.324470in}}%
\pgfpathlineto{\pgfqpoint{1.488116in}{1.286793in}}%
\pgfpathlineto{\pgfqpoint{1.518007in}{1.388275in}}%
\pgfpathlineto{\pgfqpoint{1.547899in}{1.438000in}}%
\pgfpathlineto{\pgfqpoint{1.577790in}{1.505035in}}%
\pgfpathlineto{\pgfqpoint{1.607681in}{1.359434in}}%
\pgfpathlineto{\pgfqpoint{1.637573in}{1.392137in}}%
\pgfpathlineto{\pgfqpoint{1.667464in}{1.436163in}}%
\pgfpathlineto{\pgfqpoint{1.697355in}{1.544653in}}%
\pgfpathlineto{\pgfqpoint{1.727246in}{1.541655in}}%
\pgfpathlineto{\pgfqpoint{1.757138in}{1.448533in}}%
\pgfpathlineto{\pgfqpoint{1.787029in}{1.490891in}}%
\pgfpathlineto{\pgfqpoint{1.816920in}{1.558227in}}%
\pgfpathlineto{\pgfqpoint{1.846812in}{1.574279in}}%
\pgfpathlineto{\pgfqpoint{1.876703in}{1.588048in}}%
\pgfpathlineto{\pgfqpoint{1.906594in}{1.540349in}}%
\pgfpathlineto{\pgfqpoint{1.936486in}{1.568872in}}%
\pgfpathlineto{\pgfqpoint{1.966377in}{1.580383in}}%
\pgfpathlineto{\pgfqpoint{1.996268in}{1.602636in}}%
\pgfpathlineto{\pgfqpoint{2.026160in}{1.686336in}}%
\pgfpathlineto{\pgfqpoint{2.056051in}{1.610204in}}%
\pgfpathlineto{\pgfqpoint{2.085942in}{1.582121in}}%
\pgfpathlineto{\pgfqpoint{2.115834in}{1.604242in}}%
\pgfpathlineto{\pgfqpoint{2.145725in}{1.712477in}}%
\pgfpathlineto{\pgfqpoint{2.175616in}{1.731738in}}%
\pgfpathlineto{\pgfqpoint{2.205508in}{1.643107in}}%
\pgfpathlineto{\pgfqpoint{2.235399in}{1.680647in}}%
\pgfpathlineto{\pgfqpoint{2.265290in}{1.686242in}}%
\pgfpathlineto{\pgfqpoint{2.295182in}{1.747106in}}%
\pgfpathlineto{\pgfqpoint{2.325073in}{1.732770in}}%
\pgfpathlineto{\pgfqpoint{2.354964in}{1.661705in}}%
\pgfpathlineto{\pgfqpoint{2.384855in}{1.748332in}}%
\pgfpathlineto{\pgfqpoint{2.414747in}{1.793631in}}%
\pgfpathlineto{\pgfqpoint{2.444638in}{1.779891in}}%
\pgfpathlineto{\pgfqpoint{2.474529in}{1.790684in}}%
\pgfpathlineto{\pgfqpoint{2.504421in}{1.793857in}}%
\pgfpathlineto{\pgfqpoint{2.534312in}{1.754308in}}%
\pgfpathlineto{\pgfqpoint{2.564203in}{1.799629in}}%
\pgfpathlineto{\pgfqpoint{2.594095in}{1.815561in}}%
\pgfpathlineto{\pgfqpoint{2.623986in}{1.866764in}}%
\pgfpathlineto{\pgfqpoint{2.653877in}{1.847060in}}%
\pgfpathlineto{\pgfqpoint{2.683769in}{1.805846in}}%
\pgfpathlineto{\pgfqpoint{2.713660in}{1.828831in}}%
\pgfpathlineto{\pgfqpoint{2.743551in}{1.892412in}}%
\pgfpathlineto{\pgfqpoint{2.773443in}{1.897059in}}%
\pgfpathlineto{\pgfqpoint{2.803334in}{1.840155in}}%
\pgfpathlineto{\pgfqpoint{2.833225in}{1.829105in}}%
\pgfpathlineto{\pgfqpoint{2.863117in}{1.906264in}}%
\pgfpathlineto{\pgfqpoint{2.893008in}{1.889339in}}%
\pgfpathlineto{\pgfqpoint{2.922899in}{1.881767in}}%
\pgfpathlineto{\pgfqpoint{2.952791in}{1.850865in}}%
\pgfpathlineto{\pgfqpoint{2.982682in}{1.919589in}}%
\pgfpathlineto{\pgfqpoint{3.012573in}{1.898881in}}%
\pgfpathlineto{\pgfqpoint{3.042464in}{1.937930in}}%
\pgfpathlineto{\pgfqpoint{3.072356in}{1.983995in}}%
\pgfpathlineto{\pgfqpoint{3.102247in}{1.934028in}}%
\pgfpathlineto{\pgfqpoint{3.132138in}{1.987174in}}%
\pgfpathlineto{\pgfqpoint{3.162030in}{1.926960in}}%
\pgfpathlineto{\pgfqpoint{3.191921in}{1.955420in}}%
\pgfpathlineto{\pgfqpoint{3.221812in}{2.066657in}}%
\pgfpathlineto{\pgfqpoint{3.251704in}{2.091864in}}%
\pgfpathlineto{\pgfqpoint{3.281595in}{1.915393in}}%
\pgfpathlineto{\pgfqpoint{3.311486in}{1.928355in}}%
\pgfpathlineto{\pgfqpoint{3.341378in}{2.059273in}}%
\pgfpathlineto{\pgfqpoint{3.371269in}{2.038566in}}%
\pgfpathlineto{\pgfqpoint{3.401160in}{2.000564in}}%
\pgfpathlineto{\pgfqpoint{3.431052in}{2.109778in}}%
\pgfpathlineto{\pgfqpoint{3.460943in}{1.961226in}}%
\pgfpathlineto{\pgfqpoint{3.490834in}{1.990267in}}%
\pgfpathlineto{\pgfqpoint{3.520726in}{2.072190in}}%
\pgfpathlineto{\pgfqpoint{3.550617in}{1.979005in}}%
\pgfpathlineto{\pgfqpoint{3.580508in}{2.022594in}}%
\pgfpathlineto{\pgfqpoint{3.610400in}{2.068293in}}%
\pgfpathlineto{\pgfqpoint{3.640291in}{1.927996in}}%
\pgfpathlineto{\pgfqpoint{3.670182in}{2.001032in}}%
\pgfpathlineto{\pgfqpoint{3.729965in}{2.055725in}}%
\pgfpathlineto{\pgfqpoint{3.759856in}{2.099017in}}%
\pgfpathlineto{\pgfqpoint{3.789747in}{2.106554in}}%
\pgfpathlineto{\pgfqpoint{3.819639in}{2.020753in}}%
\pgfpathlineto{\pgfqpoint{3.849530in}{2.023410in}}%
\pgfpathlineto{\pgfqpoint{3.879421in}{1.990004in}}%
\pgfpathlineto{\pgfqpoint{3.909313in}{2.015615in}}%
\pgfpathlineto{\pgfqpoint{3.939204in}{2.341131in}}%
\pgfpathlineto{\pgfqpoint{3.969095in}{2.049062in}}%
\pgfpathlineto{\pgfqpoint{3.998987in}{2.009401in}}%
\pgfpathlineto{\pgfqpoint{4.028878in}{2.016303in}}%
\pgfpathlineto{\pgfqpoint{4.058769in}{2.148906in}}%
\pgfpathlineto{\pgfqpoint{4.088661in}{2.079029in}}%
\pgfpathlineto{\pgfqpoint{4.118552in}{2.054051in}}%
\pgfpathlineto{\pgfqpoint{4.148443in}{2.049849in}}%
\pgfpathlineto{\pgfqpoint{4.178335in}{2.069251in}}%
\pgfpathlineto{\pgfqpoint{4.208226in}{2.086900in}}%
\pgfpathlineto{\pgfqpoint{4.238117in}{2.057673in}}%
\pgfpathlineto{\pgfqpoint{4.268009in}{2.150047in}}%
\pgfpathlineto{\pgfqpoint{4.297900in}{2.091864in}}%
\pgfpathlineto{\pgfqpoint{4.327791in}{2.108369in}}%
\pgfpathlineto{\pgfqpoint{4.357683in}{2.106554in}}%
\pgfpathlineto{\pgfqpoint{4.387574in}{2.102909in}}%
\pgfpathlineto{\pgfqpoint{4.417465in}{2.184024in}}%
\pgfpathlineto{\pgfqpoint{4.447356in}{2.135063in}}%
\pgfpathlineto{\pgfqpoint{4.477248in}{2.110180in}}%
\pgfpathlineto{\pgfqpoint{4.507139in}{2.116183in}}%
\pgfpathlineto{\pgfqpoint{4.566922in}{2.152325in}}%
\pgfpathlineto{\pgfqpoint{4.596813in}{2.125685in}}%
\pgfpathlineto{\pgfqpoint{4.656596in}{2.162487in}}%
\pgfpathlineto{\pgfqpoint{4.716378in}{2.155728in}}%
\pgfpathlineto{\pgfqpoint{4.835944in}{2.162487in}}%
\pgfpathlineto{\pgfqpoint{4.835944in}{2.162487in}}%
\pgfusepath{stroke}%
\end{pgfscope}%
\begin{pgfscope}%
\pgfpathrectangle{\pgfqpoint{0.588387in}{0.521603in}}{\pgfqpoint{4.669024in}{2.220246in}}%
\pgfusepath{clip}%
\pgfsetrectcap%
\pgfsetroundjoin%
\pgfsetlinewidth{1.505625pt}%
\pgfsetstrokecolor{currentstroke5}%
\pgfsetdash{}{0pt}%
\pgfpathmoveto{\pgfqpoint{0.800616in}{0.819448in}}%
\pgfpathlineto{\pgfqpoint{0.830507in}{1.122976in}}%
\pgfpathlineto{\pgfqpoint{0.860398in}{0.697089in}}%
\pgfpathlineto{\pgfqpoint{0.890290in}{0.682630in}}%
\pgfpathlineto{\pgfqpoint{0.920181in}{0.851381in}}%
\pgfpathlineto{\pgfqpoint{0.950072in}{1.014335in}}%
\pgfpathlineto{\pgfqpoint{0.979963in}{1.135282in}}%
\pgfpathlineto{\pgfqpoint{1.009855in}{0.830392in}}%
\pgfpathlineto{\pgfqpoint{1.039746in}{0.898188in}}%
\pgfpathlineto{\pgfqpoint{1.069637in}{1.059232in}}%
\pgfpathlineto{\pgfqpoint{1.099529in}{1.199583in}}%
\pgfpathlineto{\pgfqpoint{1.129420in}{1.243153in}}%
\pgfpathlineto{\pgfqpoint{1.159311in}{1.023141in}}%
\pgfpathlineto{\pgfqpoint{1.189203in}{1.100215in}}%
\pgfpathlineto{\pgfqpoint{1.219094in}{1.221780in}}%
\pgfpathlineto{\pgfqpoint{1.248985in}{1.251357in}}%
\pgfpathlineto{\pgfqpoint{1.278877in}{1.347915in}}%
\pgfpathlineto{\pgfqpoint{1.308768in}{1.194363in}}%
\pgfpathlineto{\pgfqpoint{1.338659in}{1.252942in}}%
\pgfpathlineto{\pgfqpoint{1.368551in}{1.283686in}}%
\pgfpathlineto{\pgfqpoint{1.398442in}{1.378794in}}%
\pgfpathlineto{\pgfqpoint{1.428333in}{1.451559in}}%
\pgfpathlineto{\pgfqpoint{1.458225in}{1.331204in}}%
\pgfpathlineto{\pgfqpoint{1.488116in}{1.315216in}}%
\pgfpathlineto{\pgfqpoint{1.518007in}{1.402516in}}%
\pgfpathlineto{\pgfqpoint{1.547899in}{1.482262in}}%
\pgfpathlineto{\pgfqpoint{1.577790in}{1.558559in}}%
\pgfpathlineto{\pgfqpoint{1.607681in}{1.403934in}}%
\pgfpathlineto{\pgfqpoint{1.637573in}{1.421556in}}%
\pgfpathlineto{\pgfqpoint{1.667464in}{1.491641in}}%
\pgfpathlineto{\pgfqpoint{1.697355in}{1.582400in}}%
\pgfpathlineto{\pgfqpoint{1.727246in}{1.612422in}}%
\pgfpathlineto{\pgfqpoint{1.757138in}{1.487362in}}%
\pgfpathlineto{\pgfqpoint{1.787029in}{1.524253in}}%
\pgfpathlineto{\pgfqpoint{1.816920in}{1.587797in}}%
\pgfpathlineto{\pgfqpoint{1.846812in}{1.603186in}}%
\pgfpathlineto{\pgfqpoint{1.876703in}{1.650680in}}%
\pgfpathlineto{\pgfqpoint{1.906594in}{1.581091in}}%
\pgfpathlineto{\pgfqpoint{1.936486in}{1.644327in}}%
\pgfpathlineto{\pgfqpoint{1.966377in}{1.624351in}}%
\pgfpathlineto{\pgfqpoint{1.996268in}{1.670972in}}%
\pgfpathlineto{\pgfqpoint{2.026160in}{1.759120in}}%
\pgfpathlineto{\pgfqpoint{2.056051in}{1.664564in}}%
\pgfpathlineto{\pgfqpoint{2.085942in}{1.637495in}}%
\pgfpathlineto{\pgfqpoint{2.115834in}{1.690712in}}%
\pgfpathlineto{\pgfqpoint{2.145725in}{1.766372in}}%
\pgfpathlineto{\pgfqpoint{2.175616in}{1.827321in}}%
\pgfpathlineto{\pgfqpoint{2.205508in}{1.706658in}}%
\pgfpathlineto{\pgfqpoint{2.235399in}{1.730341in}}%
\pgfpathlineto{\pgfqpoint{2.265290in}{1.791623in}}%
\pgfpathlineto{\pgfqpoint{2.295182in}{1.807344in}}%
\pgfpathlineto{\pgfqpoint{2.325073in}{1.842199in}}%
\pgfpathlineto{\pgfqpoint{2.354964in}{1.744606in}}%
\pgfpathlineto{\pgfqpoint{2.384855in}{1.794316in}}%
\pgfpathlineto{\pgfqpoint{2.414747in}{1.932640in}}%
\pgfpathlineto{\pgfqpoint{2.444638in}{1.849072in}}%
\pgfpathlineto{\pgfqpoint{2.474529in}{1.858987in}}%
\pgfpathlineto{\pgfqpoint{2.504421in}{1.841733in}}%
\pgfpathlineto{\pgfqpoint{2.534312in}{1.840418in}}%
\pgfpathlineto{\pgfqpoint{2.564203in}{1.840531in}}%
\pgfpathlineto{\pgfqpoint{2.594095in}{1.914714in}}%
\pgfpathlineto{\pgfqpoint{2.623986in}{1.918146in}}%
\pgfpathlineto{\pgfqpoint{2.653877in}{1.866954in}}%
\pgfpathlineto{\pgfqpoint{2.683769in}{1.896451in}}%
\pgfpathlineto{\pgfqpoint{2.713660in}{1.912694in}}%
\pgfpathlineto{\pgfqpoint{2.743551in}{1.984303in}}%
\pgfpathlineto{\pgfqpoint{2.773443in}{1.979491in}}%
\pgfpathlineto{\pgfqpoint{2.803334in}{1.944689in}}%
\pgfpathlineto{\pgfqpoint{2.833225in}{1.914674in}}%
\pgfpathlineto{\pgfqpoint{2.863117in}{2.054330in}}%
\pgfpathlineto{\pgfqpoint{2.893008in}{2.016903in}}%
\pgfpathlineto{\pgfqpoint{2.922899in}{1.986147in}}%
\pgfpathlineto{\pgfqpoint{2.952791in}{1.932061in}}%
\pgfpathlineto{\pgfqpoint{2.982682in}{2.016028in}}%
\pgfpathlineto{\pgfqpoint{3.012573in}{1.979672in}}%
\pgfpathlineto{\pgfqpoint{3.042464in}{2.112887in}}%
\pgfpathlineto{\pgfqpoint{3.072356in}{2.037523in}}%
\pgfpathlineto{\pgfqpoint{3.102247in}{2.006274in}}%
\pgfpathlineto{\pgfqpoint{3.132138in}{2.140863in}}%
\pgfpathlineto{\pgfqpoint{3.162030in}{2.062520in}}%
\pgfpathlineto{\pgfqpoint{3.191921in}{2.102909in}}%
\pgfpathlineto{\pgfqpoint{3.221812in}{2.202075in}}%
\pgfpathlineto{\pgfqpoint{3.251704in}{2.185654in}}%
\pgfpathlineto{\pgfqpoint{3.281595in}{2.049357in}}%
\pgfpathlineto{\pgfqpoint{3.311486in}{2.063485in}}%
\pgfpathlineto{\pgfqpoint{3.341378in}{2.252972in}}%
\pgfpathlineto{\pgfqpoint{3.371269in}{2.200960in}}%
\pgfpathlineto{\pgfqpoint{3.401160in}{2.065412in}}%
\pgfpathlineto{\pgfqpoint{3.431052in}{2.186739in}}%
\pgfpathlineto{\pgfqpoint{3.460943in}{2.073070in}}%
\pgfpathlineto{\pgfqpoint{3.490834in}{2.131561in}}%
\pgfpathlineto{\pgfqpoint{3.520726in}{2.200362in}}%
\pgfpathlineto{\pgfqpoint{3.550617in}{2.104278in}}%
\pgfpathlineto{\pgfqpoint{3.580508in}{2.113788in}}%
\pgfpathlineto{\pgfqpoint{3.610400in}{2.196162in}}%
\pgfpathlineto{\pgfqpoint{3.670182in}{2.388638in}}%
\pgfpathlineto{\pgfqpoint{3.700073in}{2.598157in}}%
\pgfpathlineto{\pgfqpoint{3.729965in}{2.210463in}}%
\pgfpathlineto{\pgfqpoint{3.759856in}{2.329796in}}%
\pgfpathlineto{\pgfqpoint{3.789747in}{2.309519in}}%
\pgfpathlineto{\pgfqpoint{3.939204in}{2.571981in}}%
\pgfpathlineto{\pgfqpoint{4.058769in}{2.242013in}}%
\pgfpathlineto{\pgfqpoint{4.268009in}{2.424575in}}%
\pgfpathlineto{\pgfqpoint{4.387574in}{2.452085in}}%
\pgfusepath{stroke}%
\end{pgfscope}%
\begin{pgfscope}%
\pgfpathrectangle{\pgfqpoint{0.588387in}{0.521603in}}{\pgfqpoint{4.669024in}{2.220246in}}%
\pgfusepath{clip}%
\pgfsetrectcap%
\pgfsetroundjoin%
\pgfsetlinewidth{1.505625pt}%
\pgfsetstrokecolor{currentstroke6}%
\pgfsetdash{}{0pt}%
\pgfpathmoveto{\pgfqpoint{0.800616in}{0.808740in}}%
\pgfpathlineto{\pgfqpoint{0.830507in}{1.131751in}}%
\pgfpathlineto{\pgfqpoint{0.860398in}{0.697066in}}%
\pgfpathlineto{\pgfqpoint{0.890290in}{0.682834in}}%
\pgfpathlineto{\pgfqpoint{0.920181in}{0.851938in}}%
\pgfpathlineto{\pgfqpoint{0.950072in}{1.014335in}}%
\pgfpathlineto{\pgfqpoint{0.979963in}{1.134721in}}%
\pgfpathlineto{\pgfqpoint{1.009855in}{0.832950in}}%
\pgfpathlineto{\pgfqpoint{1.039746in}{0.898542in}}%
\pgfpathlineto{\pgfqpoint{1.069637in}{1.062293in}}%
\pgfpathlineto{\pgfqpoint{1.099529in}{1.202842in}}%
\pgfpathlineto{\pgfqpoint{1.129420in}{1.243549in}}%
\pgfpathlineto{\pgfqpoint{1.159311in}{1.024069in}}%
\pgfpathlineto{\pgfqpoint{1.189203in}{1.100215in}}%
\pgfpathlineto{\pgfqpoint{1.219094in}{1.220955in}}%
\pgfpathlineto{\pgfqpoint{1.248985in}{1.253083in}}%
\pgfpathlineto{\pgfqpoint{1.278877in}{1.345961in}}%
\pgfpathlineto{\pgfqpoint{1.308768in}{1.194363in}}%
\pgfpathlineto{\pgfqpoint{1.338659in}{1.254757in}}%
\pgfpathlineto{\pgfqpoint{1.368551in}{1.285980in}}%
\pgfpathlineto{\pgfqpoint{1.398442in}{1.375259in}}%
\pgfpathlineto{\pgfqpoint{1.428333in}{1.451903in}}%
\pgfpathlineto{\pgfqpoint{1.458225in}{1.336467in}}%
\pgfpathlineto{\pgfqpoint{1.488116in}{1.316379in}}%
\pgfpathlineto{\pgfqpoint{1.518007in}{1.402168in}}%
\pgfpathlineto{\pgfqpoint{1.547899in}{1.480182in}}%
\pgfpathlineto{\pgfqpoint{1.577790in}{1.548379in}}%
\pgfpathlineto{\pgfqpoint{1.607681in}{1.398823in}}%
\pgfpathlineto{\pgfqpoint{1.637573in}{1.421339in}}%
\pgfpathlineto{\pgfqpoint{1.667464in}{1.491641in}}%
\pgfpathlineto{\pgfqpoint{1.697355in}{1.579441in}}%
\pgfpathlineto{\pgfqpoint{1.727246in}{1.597673in}}%
\pgfpathlineto{\pgfqpoint{1.757138in}{1.493068in}}%
\pgfpathlineto{\pgfqpoint{1.787029in}{1.523019in}}%
\pgfpathlineto{\pgfqpoint{1.816920in}{1.586262in}}%
\pgfpathlineto{\pgfqpoint{1.846812in}{1.608046in}}%
\pgfpathlineto{\pgfqpoint{1.876703in}{1.646283in}}%
\pgfpathlineto{\pgfqpoint{1.906594in}{1.574808in}}%
\pgfpathlineto{\pgfqpoint{1.936486in}{1.612756in}}%
\pgfpathlineto{\pgfqpoint{1.966377in}{1.623996in}}%
\pgfpathlineto{\pgfqpoint{1.996268in}{1.669569in}}%
\pgfpathlineto{\pgfqpoint{2.026160in}{1.739757in}}%
\pgfpathlineto{\pgfqpoint{2.056051in}{1.651002in}}%
\pgfpathlineto{\pgfqpoint{2.085942in}{1.638567in}}%
\pgfpathlineto{\pgfqpoint{2.115834in}{1.690712in}}%
\pgfpathlineto{\pgfqpoint{2.145725in}{1.771839in}}%
\pgfpathlineto{\pgfqpoint{2.175616in}{1.813117in}}%
\pgfpathlineto{\pgfqpoint{2.205508in}{1.704542in}}%
\pgfpathlineto{\pgfqpoint{2.235399in}{1.718686in}}%
\pgfpathlineto{\pgfqpoint{2.265290in}{1.785024in}}%
\pgfpathlineto{\pgfqpoint{2.295182in}{1.807344in}}%
\pgfpathlineto{\pgfqpoint{2.325073in}{1.828065in}}%
\pgfpathlineto{\pgfqpoint{2.354964in}{1.744606in}}%
\pgfpathlineto{\pgfqpoint{2.384855in}{1.785852in}}%
\pgfpathlineto{\pgfqpoint{2.414747in}{1.887090in}}%
\pgfpathlineto{\pgfqpoint{2.444638in}{1.842901in}}%
\pgfpathlineto{\pgfqpoint{2.474529in}{1.858987in}}%
\pgfpathlineto{\pgfqpoint{2.504421in}{1.836109in}}%
\pgfpathlineto{\pgfqpoint{2.534312in}{1.840418in}}%
\pgfpathlineto{\pgfqpoint{2.564203in}{1.840531in}}%
\pgfpathlineto{\pgfqpoint{2.594095in}{1.900577in}}%
\pgfpathlineto{\pgfqpoint{2.623986in}{1.920033in}}%
\pgfpathlineto{\pgfqpoint{2.653877in}{1.864414in}}%
\pgfpathlineto{\pgfqpoint{2.683769in}{1.863459in}}%
\pgfpathlineto{\pgfqpoint{2.713660in}{1.893402in}}%
\pgfpathlineto{\pgfqpoint{2.743551in}{1.968907in}}%
\pgfpathlineto{\pgfqpoint{2.773443in}{1.975141in}}%
\pgfpathlineto{\pgfqpoint{2.803334in}{1.915662in}}%
\pgfpathlineto{\pgfqpoint{2.833225in}{1.914674in}}%
\pgfpathlineto{\pgfqpoint{2.863117in}{2.002834in}}%
\pgfpathlineto{\pgfqpoint{2.893008in}{2.008621in}}%
\pgfpathlineto{\pgfqpoint{2.922899in}{1.986147in}}%
\pgfpathlineto{\pgfqpoint{2.952791in}{1.932061in}}%
\pgfpathlineto{\pgfqpoint{2.982682in}{1.995963in}}%
\pgfpathlineto{\pgfqpoint{3.012573in}{1.979672in}}%
\pgfpathlineto{\pgfqpoint{3.042464in}{2.100164in}}%
\pgfpathlineto{\pgfqpoint{3.072356in}{2.030631in}}%
\pgfpathlineto{\pgfqpoint{3.102247in}{1.991072in}}%
\pgfpathlineto{\pgfqpoint{3.132138in}{2.142018in}}%
\pgfpathlineto{\pgfqpoint{3.162030in}{2.024837in}}%
\pgfpathlineto{\pgfqpoint{3.191921in}{2.078760in}}%
\pgfpathlineto{\pgfqpoint{3.221812in}{2.200322in}}%
\pgfpathlineto{\pgfqpoint{3.251704in}{2.214385in}}%
\pgfpathlineto{\pgfqpoint{3.281595in}{2.031424in}}%
\pgfpathlineto{\pgfqpoint{3.311486in}{2.063485in}}%
\pgfpathlineto{\pgfqpoint{3.341378in}{2.172783in}}%
\pgfpathlineto{\pgfqpoint{3.371269in}{2.179384in}}%
\pgfpathlineto{\pgfqpoint{3.401160in}{2.088144in}}%
\pgfpathlineto{\pgfqpoint{3.431052in}{2.191062in}}%
\pgfpathlineto{\pgfqpoint{3.460943in}{2.073070in}}%
\pgfpathlineto{\pgfqpoint{3.490834in}{2.131561in}}%
\pgfpathlineto{\pgfqpoint{3.520726in}{2.212034in}}%
\pgfpathlineto{\pgfqpoint{3.550617in}{2.132876in}}%
\pgfpathlineto{\pgfqpoint{3.580508in}{2.113788in}}%
\pgfpathlineto{\pgfqpoint{3.610400in}{2.158834in}}%
\pgfpathlineto{\pgfqpoint{3.670182in}{2.175816in}}%
\pgfpathlineto{\pgfqpoint{3.700073in}{2.271516in}}%
\pgfpathlineto{\pgfqpoint{3.729965in}{2.165005in}}%
\pgfpathlineto{\pgfqpoint{3.759856in}{2.264253in}}%
\pgfpathlineto{\pgfqpoint{3.789747in}{2.248011in}}%
\pgfpathlineto{\pgfqpoint{3.939204in}{2.346919in}}%
\pgfpathlineto{\pgfqpoint{4.058769in}{2.242013in}}%
\pgfpathlineto{\pgfqpoint{4.268009in}{2.271516in}}%
\pgfusepath{stroke}%
\end{pgfscope}%
\begin{pgfscope}%
\pgfpathrectangle{\pgfqpoint{0.588387in}{0.521603in}}{\pgfqpoint{4.669024in}{2.220246in}}%
\pgfusepath{clip}%
\pgfsetrectcap%
\pgfsetroundjoin%
\pgfsetlinewidth{1.505625pt}%
\pgfsetstrokecolor{currentstroke7}%
\pgfsetdash{}{0pt}%
\pgfpathmoveto{\pgfqpoint{0.800616in}{0.740045in}}%
\pgfpathlineto{\pgfqpoint{0.830507in}{1.036119in}}%
\pgfpathlineto{\pgfqpoint{0.860398in}{0.707609in}}%
\pgfpathlineto{\pgfqpoint{0.890290in}{0.622524in}}%
\pgfpathlineto{\pgfqpoint{0.920181in}{0.769049in}}%
\pgfpathlineto{\pgfqpoint{0.950072in}{0.931633in}}%
\pgfpathlineto{\pgfqpoint{0.979963in}{1.055643in}}%
\pgfpathlineto{\pgfqpoint{1.009855in}{0.780636in}}%
\pgfpathlineto{\pgfqpoint{1.039746in}{0.834470in}}%
\pgfpathlineto{\pgfqpoint{1.069637in}{0.998423in}}%
\pgfpathlineto{\pgfqpoint{1.099529in}{1.137967in}}%
\pgfpathlineto{\pgfqpoint{1.129420in}{1.140643in}}%
\pgfpathlineto{\pgfqpoint{1.159311in}{0.978029in}}%
\pgfpathlineto{\pgfqpoint{1.189203in}{1.063999in}}%
\pgfpathlineto{\pgfqpoint{1.219094in}{1.179370in}}%
\pgfpathlineto{\pgfqpoint{1.248985in}{1.164664in}}%
\pgfpathlineto{\pgfqpoint{1.278877in}{1.237096in}}%
\pgfpathlineto{\pgfqpoint{1.308768in}{1.158592in}}%
\pgfpathlineto{\pgfqpoint{1.338659in}{1.223873in}}%
\pgfpathlineto{\pgfqpoint{1.368551in}{1.228768in}}%
\pgfpathlineto{\pgfqpoint{1.398442in}{1.291602in}}%
\pgfpathlineto{\pgfqpoint{1.428333in}{1.387499in}}%
\pgfpathlineto{\pgfqpoint{1.458225in}{1.320770in}}%
\pgfpathlineto{\pgfqpoint{1.488116in}{1.265554in}}%
\pgfpathlineto{\pgfqpoint{1.518007in}{1.370072in}}%
\pgfpathlineto{\pgfqpoint{1.547899in}{1.411763in}}%
\pgfpathlineto{\pgfqpoint{1.577790in}{1.488439in}}%
\pgfpathlineto{\pgfqpoint{1.607681in}{1.386514in}}%
\pgfpathlineto{\pgfqpoint{1.637573in}{1.376012in}}%
\pgfpathlineto{\pgfqpoint{1.667464in}{1.434712in}}%
\pgfpathlineto{\pgfqpoint{1.697355in}{1.534736in}}%
\pgfpathlineto{\pgfqpoint{1.727246in}{1.529028in}}%
\pgfpathlineto{\pgfqpoint{1.757138in}{1.480250in}}%
\pgfpathlineto{\pgfqpoint{1.787029in}{1.533211in}}%
\pgfpathlineto{\pgfqpoint{1.816920in}{1.541497in}}%
\pgfpathlineto{\pgfqpoint{1.846812in}{1.563369in}}%
\pgfpathlineto{\pgfqpoint{1.876703in}{1.568260in}}%
\pgfpathlineto{\pgfqpoint{1.906594in}{1.564188in}}%
\pgfpathlineto{\pgfqpoint{1.936486in}{1.595317in}}%
\pgfpathlineto{\pgfqpoint{1.966377in}{1.582948in}}%
\pgfpathlineto{\pgfqpoint{1.996268in}{1.604205in}}%
\pgfpathlineto{\pgfqpoint{2.026160in}{1.693688in}}%
\pgfpathlineto{\pgfqpoint{2.056051in}{1.625904in}}%
\pgfpathlineto{\pgfqpoint{2.085942in}{1.614573in}}%
\pgfpathlineto{\pgfqpoint{2.115834in}{1.636787in}}%
\pgfpathlineto{\pgfqpoint{2.145725in}{1.713942in}}%
\pgfpathlineto{\pgfqpoint{2.175616in}{1.751304in}}%
\pgfpathlineto{\pgfqpoint{2.205508in}{1.671736in}}%
\pgfpathlineto{\pgfqpoint{2.235399in}{1.680089in}}%
\pgfpathlineto{\pgfqpoint{2.265290in}{1.750204in}}%
\pgfpathlineto{\pgfqpoint{2.295182in}{1.768048in}}%
\pgfpathlineto{\pgfqpoint{2.325073in}{1.755900in}}%
\pgfpathlineto{\pgfqpoint{2.354964in}{1.713008in}}%
\pgfpathlineto{\pgfqpoint{2.384855in}{1.759049in}}%
\pgfpathlineto{\pgfqpoint{2.414747in}{1.855244in}}%
\pgfpathlineto{\pgfqpoint{2.444638in}{1.806950in}}%
\pgfpathlineto{\pgfqpoint{2.474529in}{1.820633in}}%
\pgfpathlineto{\pgfqpoint{2.504421in}{1.803591in}}%
\pgfpathlineto{\pgfqpoint{2.534312in}{1.808014in}}%
\pgfpathlineto{\pgfqpoint{2.564203in}{1.824389in}}%
\pgfpathlineto{\pgfqpoint{2.594095in}{1.859627in}}%
\pgfpathlineto{\pgfqpoint{2.623986in}{1.906599in}}%
\pgfpathlineto{\pgfqpoint{2.653877in}{1.877773in}}%
\pgfpathlineto{\pgfqpoint{2.683769in}{1.881854in}}%
\pgfpathlineto{\pgfqpoint{2.713660in}{1.872643in}}%
\pgfpathlineto{\pgfqpoint{2.743551in}{1.921170in}}%
\pgfpathlineto{\pgfqpoint{2.773443in}{1.943756in}}%
\pgfpathlineto{\pgfqpoint{2.803334in}{1.896086in}}%
\pgfpathlineto{\pgfqpoint{2.833225in}{1.895397in}}%
\pgfpathlineto{\pgfqpoint{2.863117in}{2.157666in}}%
\pgfpathlineto{\pgfqpoint{2.893008in}{1.963019in}}%
\pgfpathlineto{\pgfqpoint{2.922899in}{1.981762in}}%
\pgfpathlineto{\pgfqpoint{2.952791in}{1.948158in}}%
\pgfpathlineto{\pgfqpoint{2.982682in}{1.946045in}}%
\pgfpathlineto{\pgfqpoint{3.012573in}{1.990194in}}%
\pgfpathlineto{\pgfqpoint{3.042464in}{2.007173in}}%
\pgfpathlineto{\pgfqpoint{3.072356in}{2.026224in}}%
\pgfpathlineto{\pgfqpoint{3.102247in}{2.010335in}}%
\pgfpathlineto{\pgfqpoint{3.132138in}{2.022884in}}%
\pgfpathlineto{\pgfqpoint{3.162030in}{1.981263in}}%
\pgfpathlineto{\pgfqpoint{3.191921in}{2.020708in}}%
\pgfpathlineto{\pgfqpoint{3.221812in}{2.086635in}}%
\pgfpathlineto{\pgfqpoint{3.251704in}{2.102387in}}%
\pgfpathlineto{\pgfqpoint{3.281595in}{2.051533in}}%
\pgfpathlineto{\pgfqpoint{3.311486in}{2.077556in}}%
\pgfpathlineto{\pgfqpoint{3.341378in}{2.083805in}}%
\pgfpathlineto{\pgfqpoint{3.371269in}{2.087805in}}%
\pgfpathlineto{\pgfqpoint{3.401160in}{2.106303in}}%
\pgfpathlineto{\pgfqpoint{3.431052in}{2.117703in}}%
\pgfpathlineto{\pgfqpoint{3.460943in}{2.058968in}}%
\pgfpathlineto{\pgfqpoint{3.490834in}{2.112478in}}%
\pgfpathlineto{\pgfqpoint{3.520726in}{2.129406in}}%
\pgfpathlineto{\pgfqpoint{3.550617in}{2.117684in}}%
\pgfpathlineto{\pgfqpoint{3.580508in}{2.108006in}}%
\pgfpathlineto{\pgfqpoint{3.610400in}{2.133064in}}%
\pgfpathlineto{\pgfqpoint{3.640291in}{2.107408in}}%
\pgfpathlineto{\pgfqpoint{3.670182in}{2.196602in}}%
\pgfpathlineto{\pgfqpoint{3.700073in}{2.194134in}}%
\pgfpathlineto{\pgfqpoint{3.729965in}{2.206226in}}%
\pgfpathlineto{\pgfqpoint{3.759856in}{2.198108in}}%
\pgfpathlineto{\pgfqpoint{3.789747in}{2.184400in}}%
\pgfpathlineto{\pgfqpoint{3.819639in}{2.174400in}}%
\pgfpathlineto{\pgfqpoint{3.849530in}{2.168480in}}%
\pgfpathlineto{\pgfqpoint{3.879421in}{2.144978in}}%
\pgfpathlineto{\pgfqpoint{3.909313in}{2.239667in}}%
\pgfpathlineto{\pgfqpoint{3.939204in}{2.497511in}}%
\pgfpathlineto{\pgfqpoint{3.969095in}{2.218249in}}%
\pgfpathlineto{\pgfqpoint{3.998987in}{2.207434in}}%
\pgfpathlineto{\pgfqpoint{4.028878in}{2.178352in}}%
\pgfpathlineto{\pgfqpoint{4.058769in}{2.238131in}}%
\pgfpathlineto{\pgfqpoint{4.088661in}{2.238188in}}%
\pgfpathlineto{\pgfqpoint{4.118552in}{2.353486in}}%
\pgfpathlineto{\pgfqpoint{4.148443in}{2.196964in}}%
\pgfpathlineto{\pgfqpoint{4.178335in}{2.227308in}}%
\pgfpathlineto{\pgfqpoint{4.208226in}{2.254229in}}%
\pgfpathlineto{\pgfqpoint{4.238117in}{2.262060in}}%
\pgfpathlineto{\pgfqpoint{4.268009in}{2.267564in}}%
\pgfpathlineto{\pgfqpoint{4.297900in}{2.263766in}}%
\pgfpathlineto{\pgfqpoint{4.327791in}{2.277911in}}%
\pgfpathlineto{\pgfqpoint{4.357683in}{2.298593in}}%
\pgfpathlineto{\pgfqpoint{4.387574in}{2.424575in}}%
\pgfpathlineto{\pgfqpoint{4.417465in}{2.370937in}}%
\pgfpathlineto{\pgfqpoint{4.447356in}{2.277848in}}%
\pgfpathlineto{\pgfqpoint{4.477248in}{2.287370in}}%
\pgfpathlineto{\pgfqpoint{4.507139in}{2.280656in}}%
\pgfpathlineto{\pgfqpoint{4.537030in}{2.410415in}}%
\pgfpathlineto{\pgfqpoint{4.566922in}{2.412200in}}%
\pgfpathlineto{\pgfqpoint{4.596813in}{2.392558in}}%
\pgfpathlineto{\pgfqpoint{4.626704in}{2.408027in}}%
\pgfpathlineto{\pgfqpoint{4.656596in}{2.353397in}}%
\pgfpathlineto{\pgfqpoint{4.686487in}{2.484341in}}%
\pgfpathlineto{\pgfqpoint{4.716378in}{2.436165in}}%
\pgfpathlineto{\pgfqpoint{4.746270in}{2.400817in}}%
\pgfpathlineto{\pgfqpoint{4.776161in}{2.361095in}}%
\pgfpathlineto{\pgfqpoint{4.806052in}{2.435398in}}%
\pgfpathlineto{\pgfqpoint{4.835944in}{2.552683in}}%
\pgfpathlineto{\pgfqpoint{4.865835in}{2.575779in}}%
\pgfpathlineto{\pgfqpoint{4.895726in}{2.532853in}}%
\pgfpathlineto{\pgfqpoint{4.925618in}{2.528820in}}%
\pgfpathlineto{\pgfqpoint{4.955509in}{2.640929in}}%
\pgfpathlineto{\pgfqpoint{4.985400in}{2.575779in}}%
\pgfpathlineto{\pgfqpoint{5.045183in}{2.544816in}}%
\pgfpathlineto{\pgfqpoint{5.045183in}{2.544816in}}%
\pgfusepath{stroke}%
\end{pgfscope}%
\begin{pgfscope}%
\pgfpathrectangle{\pgfqpoint{0.588387in}{0.521603in}}{\pgfqpoint{4.669024in}{2.220246in}}%
\pgfusepath{clip}%
\pgfsetrectcap%
\pgfsetroundjoin%
\pgfsetlinewidth{1.505625pt}%
\definecolor{currentstroke}{rgb}{0.498039,0.498039,0.498039}%
\pgfsetstrokecolor{currentstroke}%
\pgfsetdash{}{0pt}%
\pgfpathmoveto{\pgfqpoint{0.800616in}{0.827711in}}%
\pgfpathlineto{\pgfqpoint{0.830507in}{1.131304in}}%
\pgfpathlineto{\pgfqpoint{0.860398in}{0.707708in}}%
\pgfpathlineto{\pgfqpoint{0.890290in}{0.681736in}}%
\pgfpathlineto{\pgfqpoint{0.920181in}{0.859341in}}%
\pgfpathlineto{\pgfqpoint{0.950072in}{1.015508in}}%
\pgfpathlineto{\pgfqpoint{0.979963in}{1.129132in}}%
\pgfpathlineto{\pgfqpoint{1.009855in}{0.824744in}}%
\pgfpathlineto{\pgfqpoint{1.039746in}{0.864311in}}%
\pgfpathlineto{\pgfqpoint{1.069637in}{1.046801in}}%
\pgfpathlineto{\pgfqpoint{1.099529in}{1.188128in}}%
\pgfpathlineto{\pgfqpoint{1.129420in}{1.205349in}}%
\pgfpathlineto{\pgfqpoint{1.159311in}{1.009379in}}%
\pgfpathlineto{\pgfqpoint{1.189203in}{1.095469in}}%
\pgfpathlineto{\pgfqpoint{1.219094in}{1.210423in}}%
\pgfpathlineto{\pgfqpoint{1.248985in}{1.226421in}}%
\pgfpathlineto{\pgfqpoint{1.278877in}{1.311560in}}%
\pgfpathlineto{\pgfqpoint{1.308768in}{1.187748in}}%
\pgfpathlineto{\pgfqpoint{1.338659in}{1.249857in}}%
\pgfpathlineto{\pgfqpoint{1.368551in}{1.276857in}}%
\pgfpathlineto{\pgfqpoint{1.398442in}{1.327389in}}%
\pgfpathlineto{\pgfqpoint{1.428333in}{1.430061in}}%
\pgfpathlineto{\pgfqpoint{1.458225in}{1.341316in}}%
\pgfpathlineto{\pgfqpoint{1.488116in}{1.303937in}}%
\pgfpathlineto{\pgfqpoint{1.518007in}{1.408979in}}%
\pgfpathlineto{\pgfqpoint{1.547899in}{1.441711in}}%
\pgfpathlineto{\pgfqpoint{1.577790in}{1.512779in}}%
\pgfpathlineto{\pgfqpoint{1.607681in}{1.379730in}}%
\pgfpathlineto{\pgfqpoint{1.637573in}{1.406825in}}%
\pgfpathlineto{\pgfqpoint{1.667464in}{1.439721in}}%
\pgfpathlineto{\pgfqpoint{1.697355in}{1.554623in}}%
\pgfpathlineto{\pgfqpoint{1.727246in}{1.539867in}}%
\pgfpathlineto{\pgfqpoint{1.757138in}{1.453726in}}%
\pgfpathlineto{\pgfqpoint{1.787029in}{1.529093in}}%
\pgfpathlineto{\pgfqpoint{1.816920in}{1.556699in}}%
\pgfpathlineto{\pgfqpoint{1.846812in}{1.597297in}}%
\pgfpathlineto{\pgfqpoint{1.876703in}{1.599168in}}%
\pgfpathlineto{\pgfqpoint{1.906594in}{1.554321in}}%
\pgfpathlineto{\pgfqpoint{1.936486in}{1.586769in}}%
\pgfpathlineto{\pgfqpoint{1.966377in}{1.613517in}}%
\pgfpathlineto{\pgfqpoint{1.996268in}{1.616286in}}%
\pgfpathlineto{\pgfqpoint{2.026160in}{1.716777in}}%
\pgfpathlineto{\pgfqpoint{2.056051in}{1.616568in}}%
\pgfpathlineto{\pgfqpoint{2.085942in}{1.599204in}}%
\pgfpathlineto{\pgfqpoint{2.115834in}{1.610545in}}%
\pgfpathlineto{\pgfqpoint{2.145725in}{1.707783in}}%
\pgfpathlineto{\pgfqpoint{2.175616in}{1.751317in}}%
\pgfpathlineto{\pgfqpoint{2.205508in}{1.685511in}}%
\pgfpathlineto{\pgfqpoint{2.235399in}{1.694107in}}%
\pgfpathlineto{\pgfqpoint{2.265290in}{1.693476in}}%
\pgfpathlineto{\pgfqpoint{2.295182in}{1.759969in}}%
\pgfpathlineto{\pgfqpoint{2.325073in}{1.745357in}}%
\pgfpathlineto{\pgfqpoint{2.354964in}{1.672120in}}%
\pgfpathlineto{\pgfqpoint{2.384855in}{1.756317in}}%
\pgfpathlineto{\pgfqpoint{2.414747in}{1.847201in}}%
\pgfpathlineto{\pgfqpoint{2.444638in}{1.803720in}}%
\pgfpathlineto{\pgfqpoint{2.474529in}{1.809602in}}%
\pgfpathlineto{\pgfqpoint{2.504421in}{1.802868in}}%
\pgfpathlineto{\pgfqpoint{2.534312in}{1.762904in}}%
\pgfpathlineto{\pgfqpoint{2.564203in}{1.807788in}}%
\pgfpathlineto{\pgfqpoint{2.594095in}{1.822842in}}%
\pgfpathlineto{\pgfqpoint{2.623986in}{1.908638in}}%
\pgfpathlineto{\pgfqpoint{2.653877in}{1.860326in}}%
\pgfpathlineto{\pgfqpoint{2.683769in}{1.823050in}}%
\pgfpathlineto{\pgfqpoint{2.713660in}{1.836197in}}%
\pgfpathlineto{\pgfqpoint{2.743551in}{1.929658in}}%
\pgfpathlineto{\pgfqpoint{2.773443in}{1.918146in}}%
\pgfpathlineto{\pgfqpoint{2.803334in}{1.862184in}}%
\pgfpathlineto{\pgfqpoint{2.833225in}{1.835314in}}%
\pgfpathlineto{\pgfqpoint{2.863117in}{1.944554in}}%
\pgfpathlineto{\pgfqpoint{2.922899in}{1.891635in}}%
\pgfpathlineto{\pgfqpoint{2.952791in}{1.850865in}}%
\pgfpathlineto{\pgfqpoint{2.982682in}{1.931141in}}%
\pgfpathlineto{\pgfqpoint{3.012573in}{1.894623in}}%
\pgfpathlineto{\pgfqpoint{3.042464in}{1.951054in}}%
\pgfpathlineto{\pgfqpoint{3.072356in}{1.999035in}}%
\pgfpathlineto{\pgfqpoint{3.102247in}{1.958552in}}%
\pgfpathlineto{\pgfqpoint{3.132138in}{1.988665in}}%
\pgfpathlineto{\pgfqpoint{3.162030in}{1.926701in}}%
\pgfpathlineto{\pgfqpoint{3.191921in}{1.968749in}}%
\pgfpathlineto{\pgfqpoint{3.221812in}{2.032284in}}%
\pgfpathlineto{\pgfqpoint{3.251704in}{2.078760in}}%
\pgfpathlineto{\pgfqpoint{3.281595in}{1.974132in}}%
\pgfpathlineto{\pgfqpoint{3.311486in}{1.946918in}}%
\pgfpathlineto{\pgfqpoint{3.341378in}{2.032284in}}%
\pgfpathlineto{\pgfqpoint{3.371269in}{2.036720in}}%
\pgfpathlineto{\pgfqpoint{3.401160in}{2.022684in}}%
\pgfpathlineto{\pgfqpoint{3.431052in}{2.120157in}}%
\pgfpathlineto{\pgfqpoint{3.460943in}{1.968309in}}%
\pgfpathlineto{\pgfqpoint{3.490834in}{1.997461in}}%
\pgfpathlineto{\pgfqpoint{3.520726in}{2.100375in}}%
\pgfpathlineto{\pgfqpoint{3.550617in}{2.034869in}}%
\pgfpathlineto{\pgfqpoint{3.580508in}{2.052205in}}%
\pgfpathlineto{\pgfqpoint{3.610400in}{2.104369in}}%
\pgfpathlineto{\pgfqpoint{3.640291in}{1.944120in}}%
\pgfpathlineto{\pgfqpoint{3.670182in}{2.018497in}}%
\pgfpathlineto{\pgfqpoint{3.700073in}{2.062842in}}%
\pgfpathlineto{\pgfqpoint{3.759856in}{2.102909in}}%
\pgfpathlineto{\pgfqpoint{3.789747in}{2.124504in}}%
\pgfpathlineto{\pgfqpoint{3.819639in}{2.007318in}}%
\pgfpathlineto{\pgfqpoint{3.849530in}{2.029909in}}%
\pgfpathlineto{\pgfqpoint{3.879421in}{1.996810in}}%
\pgfpathlineto{\pgfqpoint{3.909313in}{2.029909in}}%
\pgfpathlineto{\pgfqpoint{3.939204in}{2.430559in}}%
\pgfpathlineto{\pgfqpoint{3.969095in}{2.053771in}}%
\pgfpathlineto{\pgfqpoint{3.998987in}{2.021776in}}%
\pgfpathlineto{\pgfqpoint{4.028878in}{2.029909in}}%
\pgfpathlineto{\pgfqpoint{4.058769in}{2.162487in}}%
\pgfpathlineto{\pgfqpoint{4.088661in}{2.080646in}}%
\pgfpathlineto{\pgfqpoint{4.118552in}{2.108973in}}%
\pgfpathlineto{\pgfqpoint{4.148443in}{2.053771in}}%
\pgfpathlineto{\pgfqpoint{4.208226in}{2.099246in}}%
\pgfpathlineto{\pgfqpoint{4.238117in}{2.069251in}}%
\pgfpathlineto{\pgfqpoint{4.268009in}{2.151472in}}%
\pgfpathlineto{\pgfqpoint{4.297900in}{2.102909in}}%
\pgfpathlineto{\pgfqpoint{4.327791in}{2.120950in}}%
\pgfpathlineto{\pgfqpoint{4.357683in}{2.099246in}}%
\pgfpathlineto{\pgfqpoint{4.387574in}{2.331571in}}%
\pgfpathlineto{\pgfqpoint{4.417465in}{2.165842in}}%
\pgfpathlineto{\pgfqpoint{4.447356in}{2.142018in}}%
\pgfpathlineto{\pgfqpoint{4.477248in}{2.113788in}}%
\pgfpathlineto{\pgfqpoint{4.507139in}{2.120950in}}%
\pgfpathlineto{\pgfqpoint{4.566922in}{2.162487in}}%
\pgfpathlineto{\pgfqpoint{4.596813in}{2.135063in}}%
\pgfpathlineto{\pgfqpoint{4.656596in}{2.162487in}}%
\pgfpathlineto{\pgfqpoint{4.716378in}{2.155728in}}%
\pgfpathlineto{\pgfqpoint{4.806052in}{2.201757in}}%
\pgfpathlineto{\pgfqpoint{4.835944in}{2.175816in}}%
\pgfpathlineto{\pgfqpoint{4.835944in}{2.175816in}}%
\pgfusepath{stroke}%
\end{pgfscope}%
\begin{pgfscope}%
\pgfsetrectcap%
\pgfsetmiterjoin%
\pgfsetlinewidth{0.803000pt}%
\definecolor{currentstroke}{rgb}{0.000000,0.000000,0.000000}%
\pgfsetstrokecolor{currentstroke}%
\pgfsetdash{}{0pt}%
\pgfpathmoveto{\pgfqpoint{0.588387in}{0.521603in}}%
\pgfpathlineto{\pgfqpoint{0.588387in}{2.741849in}}%
\pgfusepath{stroke}%
\end{pgfscope}%
\begin{pgfscope}%
\pgfsetrectcap%
\pgfsetmiterjoin%
\pgfsetlinewidth{0.803000pt}%
\definecolor{currentstroke}{rgb}{0.000000,0.000000,0.000000}%
\pgfsetstrokecolor{currentstroke}%
\pgfsetdash{}{0pt}%
\pgfpathmoveto{\pgfqpoint{5.257411in}{0.521603in}}%
\pgfpathlineto{\pgfqpoint{5.257411in}{2.741849in}}%
\pgfusepath{stroke}%
\end{pgfscope}%
\begin{pgfscope}%
\pgfsetrectcap%
\pgfsetmiterjoin%
\pgfsetlinewidth{0.803000pt}%
\definecolor{currentstroke}{rgb}{0.000000,0.000000,0.000000}%
\pgfsetstrokecolor{currentstroke}%
\pgfsetdash{}{0pt}%
\pgfpathmoveto{\pgfqpoint{0.588387in}{0.521603in}}%
\pgfpathlineto{\pgfqpoint{5.257411in}{0.521603in}}%
\pgfusepath{stroke}%
\end{pgfscope}%
\begin{pgfscope}%
\pgfsetrectcap%
\pgfsetmiterjoin%
\pgfsetlinewidth{0.803000pt}%
\definecolor{currentstroke}{rgb}{0.000000,0.000000,0.000000}%
\pgfsetstrokecolor{currentstroke}%
\pgfsetdash{}{0pt}%
\pgfpathmoveto{\pgfqpoint{0.588387in}{2.741849in}}%
\pgfpathlineto{\pgfqpoint{5.257411in}{2.741849in}}%
\pgfusepath{stroke}%
\end{pgfscope}%
\begin{pgfscope}%
\pgfsetbuttcap%
\pgfsetmiterjoin%
\definecolor{currentfill}{rgb}{1.000000,1.000000,1.000000}%
\pgfsetfillcolor{currentfill}%
\pgfsetfillopacity{0.800000}%
\pgfsetlinewidth{1.003750pt}%
\definecolor{currentstroke}{rgb}{0.800000,0.800000,0.800000}%
\pgfsetstrokecolor{currentstroke}%
\pgfsetstrokeopacity{0.800000}%
\pgfsetdash{}{0pt}%
\pgfpathmoveto{\pgfqpoint{5.344911in}{1.153204in}}%
\pgfpathlineto{\pgfqpoint{8.259376in}{1.153204in}}%
\pgfpathquadraticcurveto{\pgfqpoint{8.284376in}{1.153204in}}{\pgfqpoint{8.284376in}{1.178204in}}%
\pgfpathlineto{\pgfqpoint{8.284376in}{2.654349in}}%
\pgfpathquadraticcurveto{\pgfqpoint{8.284376in}{2.679349in}}{\pgfqpoint{8.259376in}{2.679349in}}%
\pgfpathlineto{\pgfqpoint{5.344911in}{2.679349in}}%
\pgfpathquadraticcurveto{\pgfqpoint{5.319911in}{2.679349in}}{\pgfqpoint{5.319911in}{2.654349in}}%
\pgfpathlineto{\pgfqpoint{5.319911in}{1.178204in}}%
\pgfpathquadraticcurveto{\pgfqpoint{5.319911in}{1.153204in}}{\pgfqpoint{5.344911in}{1.153204in}}%
\pgfpathlineto{\pgfqpoint{5.344911in}{1.153204in}}%
\pgfpathclose%
\pgfusepath{stroke,fill}%
\end{pgfscope}%
\begin{pgfscope}%
\pgfsetrectcap%
\pgfsetroundjoin%
\pgfsetlinewidth{1.505625pt}%
\pgfsetstrokecolor{currentstroke1}%
\pgfsetdash{}{0pt}%
\pgfpathmoveto{\pgfqpoint{5.369911in}{2.578129in}}%
\pgfpathlineto{\pgfqpoint{5.494911in}{2.578129in}}%
\pgfpathlineto{\pgfqpoint{5.619911in}{2.578129in}}%
\pgfusepath{stroke}%
\end{pgfscope}%
\begin{pgfscope}%
\definecolor{textcolor}{rgb}{0.000000,0.000000,0.000000}%
\pgfsetstrokecolor{textcolor}%
\pgfsetfillcolor{textcolor}%
\pgftext[x=5.719911in,y=2.534379in,left,base]{\color{textcolor}{\rmfamily\fontsize{9.000000}{10.800000}\selectfont\catcode`\^=\active\def^{\ifmmode\sp\else\^{}\fi}\catcode`\%=\active\def%{\%}\CyclesMatchChunks{} \& \MergeLinear{}}}%
\end{pgfscope}%
\begin{pgfscope}%
\pgfsetrectcap%
\pgfsetroundjoin%
\pgfsetlinewidth{1.505625pt}%
\pgfsetstrokecolor{currentstroke2}%
\pgfsetdash{}{0pt}%
\pgfpathmoveto{\pgfqpoint{5.369911in}{2.391178in}}%
\pgfpathlineto{\pgfqpoint{5.494911in}{2.391178in}}%
\pgfpathlineto{\pgfqpoint{5.619911in}{2.391178in}}%
\pgfusepath{stroke}%
\end{pgfscope}%
\begin{pgfscope}%
\definecolor{textcolor}{rgb}{0.000000,0.000000,0.000000}%
\pgfsetstrokecolor{textcolor}%
\pgfsetfillcolor{textcolor}%
\pgftext[x=5.719911in,y=2.347428in,left,base]{\color{textcolor}{\rmfamily\fontsize{9.000000}{10.800000}\selectfont\catcode`\^=\active\def^{\ifmmode\sp\else\^{}\fi}\catcode`\%=\active\def%{\%}\CyclesMatchChunks{} \& \SharedVertices{}}}%
\end{pgfscope}%
\begin{pgfscope}%
\pgfsetrectcap%
\pgfsetroundjoin%
\pgfsetlinewidth{1.505625pt}%
\pgfsetstrokecolor{currentstroke3}%
\pgfsetdash{}{0pt}%
\pgfpathmoveto{\pgfqpoint{5.369911in}{2.204228in}}%
\pgfpathlineto{\pgfqpoint{5.494911in}{2.204228in}}%
\pgfpathlineto{\pgfqpoint{5.619911in}{2.204228in}}%
\pgfusepath{stroke}%
\end{pgfscope}%
\begin{pgfscope}%
\definecolor{textcolor}{rgb}{0.000000,0.000000,0.000000}%
\pgfsetstrokecolor{textcolor}%
\pgfsetfillcolor{textcolor}%
\pgftext[x=5.719911in,y=2.160478in,left,base]{\color{textcolor}{\rmfamily\fontsize{9.000000}{10.800000}\selectfont\catcode`\^=\active\def^{\ifmmode\sp\else\^{}\fi}\catcode`\%=\active\def%{\%}\Neighbors{} \& \MergeLinear{}}}%
\end{pgfscope}%
\begin{pgfscope}%
\pgfsetrectcap%
\pgfsetroundjoin%
\pgfsetlinewidth{1.505625pt}%
\pgfsetstrokecolor{currentstroke4}%
\pgfsetdash{}{0pt}%
\pgfpathmoveto{\pgfqpoint{5.369911in}{2.020756in}}%
\pgfpathlineto{\pgfqpoint{5.494911in}{2.020756in}}%
\pgfpathlineto{\pgfqpoint{5.619911in}{2.020756in}}%
\pgfusepath{stroke}%
\end{pgfscope}%
\begin{pgfscope}%
\definecolor{textcolor}{rgb}{0.000000,0.000000,0.000000}%
\pgfsetstrokecolor{textcolor}%
\pgfsetfillcolor{textcolor}%
\pgftext[x=5.719911in,y=1.977006in,left,base]{\color{textcolor}{\rmfamily\fontsize{9.000000}{10.800000}\selectfont\catcode`\^=\active\def^{\ifmmode\sp\else\^{}\fi}\catcode`\%=\active\def%{\%}\Neighbors{} \& \SharedVertices{}}}%
\end{pgfscope}%
\begin{pgfscope}%
\pgfsetrectcap%
\pgfsetroundjoin%
\pgfsetlinewidth{1.505625pt}%
\pgfsetstrokecolor{currentstroke5}%
\pgfsetdash{}{0pt}%
\pgfpathmoveto{\pgfqpoint{5.369911in}{1.833806in}}%
\pgfpathlineto{\pgfqpoint{5.494911in}{1.833806in}}%
\pgfpathlineto{\pgfqpoint{5.619911in}{1.833806in}}%
\pgfusepath{stroke}%
\end{pgfscope}%
\begin{pgfscope}%
\definecolor{textcolor}{rgb}{0.000000,0.000000,0.000000}%
\pgfsetstrokecolor{textcolor}%
\pgfsetfillcolor{textcolor}%
\pgftext[x=5.719911in,y=1.790056in,left,base]{\color{textcolor}{\rmfamily\fontsize{9.000000}{10.800000}\selectfont\catcode`\^=\active\def^{\ifmmode\sp\else\^{}\fi}\catcode`\%=\active\def%{\%}\NeighborsDegree{} \& \MergeLinear{}}}%
\end{pgfscope}%
\begin{pgfscope}%
\pgfsetrectcap%
\pgfsetroundjoin%
\pgfsetlinewidth{1.505625pt}%
\pgfsetstrokecolor{currentstroke6}%
\pgfsetdash{}{0pt}%
\pgfpathmoveto{\pgfqpoint{5.369911in}{1.646855in}}%
\pgfpathlineto{\pgfqpoint{5.494911in}{1.646855in}}%
\pgfpathlineto{\pgfqpoint{5.619911in}{1.646855in}}%
\pgfusepath{stroke}%
\end{pgfscope}%
\begin{pgfscope}%
\definecolor{textcolor}{rgb}{0.000000,0.000000,0.000000}%
\pgfsetstrokecolor{textcolor}%
\pgfsetfillcolor{textcolor}%
\pgftext[x=5.719911in,y=1.603105in,left,base]{\color{textcolor}{\rmfamily\fontsize{9.000000}{10.800000}\selectfont\catcode`\^=\active\def^{\ifmmode\sp\else\^{}\fi}\catcode`\%=\active\def%{\%}\NeighborsDegree{} \& \SharedVertices{}}}%
\end{pgfscope}%
\begin{pgfscope}%
\pgfsetrectcap%
\pgfsetroundjoin%
\pgfsetlinewidth{1.505625pt}%
\pgfsetstrokecolor{currentstroke7}%
\pgfsetdash{}{0pt}%
\pgfpathmoveto{\pgfqpoint{5.369911in}{1.459905in}}%
\pgfpathlineto{\pgfqpoint{5.494911in}{1.459905in}}%
\pgfpathlineto{\pgfqpoint{5.619911in}{1.459905in}}%
\pgfusepath{stroke}%
\end{pgfscope}%
\begin{pgfscope}%
\definecolor{textcolor}{rgb}{0.000000,0.000000,0.000000}%
\pgfsetstrokecolor{textcolor}%
\pgfsetfillcolor{textcolor}%
\pgftext[x=5.719911in,y=1.416155in,left,base]{\color{textcolor}{\rmfamily\fontsize{9.000000}{10.800000}\selectfont\catcode`\^=\active\def^{\ifmmode\sp\else\^{}\fi}\catcode`\%=\active\def%{\%}\None{} \& \MergeLinear{}}}%
\end{pgfscope}%
\begin{pgfscope}%
\pgfsetrectcap%
\pgfsetroundjoin%
\pgfsetlinewidth{1.505625pt}%
\definecolor{currentstroke}{rgb}{0.498039,0.498039,0.498039}%
\pgfsetstrokecolor{currentstroke}%
\pgfsetdash{}{0pt}%
\pgfpathmoveto{\pgfqpoint{5.369911in}{1.276433in}}%
\pgfpathlineto{\pgfqpoint{5.494911in}{1.276433in}}%
\pgfpathlineto{\pgfqpoint{5.619911in}{1.276433in}}%
\pgfusepath{stroke}%
\end{pgfscope}%
\begin{pgfscope}%
\definecolor{textcolor}{rgb}{0.000000,0.000000,0.000000}%
\pgfsetstrokecolor{textcolor}%
\pgfsetfillcolor{textcolor}%
\pgftext[x=5.719911in,y=1.232683in,left,base]{\color{textcolor}{\rmfamily\fontsize{9.000000}{10.800000}\selectfont\catcode`\^=\active\def^{\ifmmode\sp\else\^{}\fi}\catcode`\%=\active\def%{\%}\None{} \& \SharedVertices{}}}%
\end{pgfscope}%
\end{pgfpicture}%
\makeatother%
\endgroup%
}
	\caption[Checks performed for graphs with no NAC-coloring]{
		The number of checks performed to finish search for graphs with no NAC-coloring.}%
	\label{fig:graph_no_nac_coloring_first_checks}
\end{figure}%

In \Subgraphs{} algorithm description, an important parameter was the size of subgraphs \( k \).
Majority of the benchmarks in the previous section were run with \( k = 4 \).
%
We show the impact of	\( k \) on runtime and number of checks.
Note that you see averages over all the strategies used for benchmarking
graphs with no NAC-colorings.
%
From graphs in \Cref{fig:graph_no_nac_coloring_first_runtime_subgraph_size,fig:graph_no_nac_coloring_first_checks_subgraph_size},
it can be seen that the number of checks is reduced for smaller \( k \).
On the other hand, the runtime improves slightly for larger \( k \)
and the difference between strategies becomes negligible for larger graphs.

\begin{figure}[thbp]
	\centering
	\scalebox{\BenchFigureScale}{%% Creator: Matplotlib, PGF backend
%%
%% To include the figure in your LaTeX document, write
%%   \input{<filename>.pgf}
%%
%% Make sure the required packages are loaded in your preamble
%%   \usepackage{pgf}
%%
%% Also ensure that all the required font packages are loaded; for instance,
%% the lmodern package is sometimes necessary when using math font.
%%   \usepackage{lmodern}
%%
%% Figures using additional raster images can only be included by \input if
%% they are in the same directory as the main LaTeX file. For loading figures
%% from other directories you can use the `import` package
%%   \usepackage{import}
%%
%% and then include the figures with
%%   \import{<path to file>}{<filename>.pgf}
%%
%% Matplotlib used the following preamble
%%   \def\mathdefault#1{#1}
%%   \everymath=\expandafter{\the\everymath\displaystyle}
%%   \IfFileExists{scrextend.sty}{
%%     \usepackage[fontsize=10.000000pt]{scrextend}
%%   }{
%%     \renewcommand{\normalsize}{\fontsize{10.000000}{12.000000}\selectfont}
%%     \normalsize
%%   }
%%   
%%   \ifdefined\pdftexversion\else  % non-pdftex case.
%%     \usepackage{fontspec}
%%     \setmainfont{DejaVuSans.ttf}[Path=\detokenize{/home/petr/Projects/PyRigi/.venv/lib/python3.12/site-packages/matplotlib/mpl-data/fonts/ttf/}]
%%     \setsansfont{DejaVuSans.ttf}[Path=\detokenize{/home/petr/Projects/PyRigi/.venv/lib/python3.12/site-packages/matplotlib/mpl-data/fonts/ttf/}]
%%     \setmonofont{DejaVuSansMono.ttf}[Path=\detokenize{/home/petr/Projects/PyRigi/.venv/lib/python3.12/site-packages/matplotlib/mpl-data/fonts/ttf/}]
%%   \fi
%%   \makeatletter\@ifpackageloaded{underscore}{}{\usepackage[strings]{underscore}}\makeatother
%%
\begingroup%
\makeatletter%
\begin{pgfpicture}%
\pgfpathrectangle{\pgfpointorigin}{\pgfqpoint{8.384376in}{2.841849in}}%
\pgfusepath{use as bounding box, clip}%
\begin{pgfscope}%
\pgfsetbuttcap%
\pgfsetmiterjoin%
\definecolor{currentfill}{rgb}{1.000000,1.000000,1.000000}%
\pgfsetfillcolor{currentfill}%
\pgfsetlinewidth{0.000000pt}%
\definecolor{currentstroke}{rgb}{1.000000,1.000000,1.000000}%
\pgfsetstrokecolor{currentstroke}%
\pgfsetdash{}{0pt}%
\pgfpathmoveto{\pgfqpoint{0.000000in}{0.000000in}}%
\pgfpathlineto{\pgfqpoint{8.384376in}{0.000000in}}%
\pgfpathlineto{\pgfqpoint{8.384376in}{2.841849in}}%
\pgfpathlineto{\pgfqpoint{0.000000in}{2.841849in}}%
\pgfpathlineto{\pgfqpoint{0.000000in}{0.000000in}}%
\pgfpathclose%
\pgfusepath{fill}%
\end{pgfscope}%
\begin{pgfscope}%
\pgfsetbuttcap%
\pgfsetmiterjoin%
\definecolor{currentfill}{rgb}{1.000000,1.000000,1.000000}%
\pgfsetfillcolor{currentfill}%
\pgfsetlinewidth{0.000000pt}%
\definecolor{currentstroke}{rgb}{0.000000,0.000000,0.000000}%
\pgfsetstrokecolor{currentstroke}%
\pgfsetstrokeopacity{0.000000}%
\pgfsetdash{}{0pt}%
\pgfpathmoveto{\pgfqpoint{0.588387in}{0.521603in}}%
\pgfpathlineto{\pgfqpoint{7.692348in}{0.521603in}}%
\pgfpathlineto{\pgfqpoint{7.692348in}{2.531888in}}%
\pgfpathlineto{\pgfqpoint{0.588387in}{2.531888in}}%
\pgfpathlineto{\pgfqpoint{0.588387in}{0.521603in}}%
\pgfpathclose%
\pgfusepath{fill}%
\end{pgfscope}%
\begin{pgfscope}%
\pgfsetbuttcap%
\pgfsetroundjoin%
\definecolor{currentfill}{rgb}{0.000000,0.000000,0.000000}%
\pgfsetfillcolor{currentfill}%
\pgfsetlinewidth{0.803000pt}%
\definecolor{currentstroke}{rgb}{0.000000,0.000000,0.000000}%
\pgfsetstrokecolor{currentstroke}%
\pgfsetdash{}{0pt}%
\pgfsys@defobject{currentmarker}{\pgfqpoint{0.000000in}{-0.048611in}}{\pgfqpoint{0.000000in}{0.000000in}}{%
\pgfpathmoveto{\pgfqpoint{0.000000in}{0.000000in}}%
\pgfpathlineto{\pgfqpoint{0.000000in}{-0.048611in}}%
\pgfusepath{stroke,fill}%
}%
\begin{pgfscope}%
\pgfsys@transformshift{1.241273in}{0.521603in}%
\pgfsys@useobject{currentmarker}{}%
\end{pgfscope}%
\end{pgfscope}%
\begin{pgfscope}%
\definecolor{textcolor}{rgb}{0.000000,0.000000,0.000000}%
\pgfsetstrokecolor{textcolor}%
\pgfsetfillcolor{textcolor}%
\pgftext[x=1.241273in,y=0.424381in,,top]{\color{textcolor}{\rmfamily\fontsize{10.000000}{12.000000}\selectfont\catcode`\^=\active\def^{\ifmmode\sp\else\^{}\fi}\catcode`\%=\active\def%{\%}$\mathdefault{20}$}}%
\end{pgfscope}%
\begin{pgfscope}%
\pgfsetbuttcap%
\pgfsetroundjoin%
\definecolor{currentfill}{rgb}{0.000000,0.000000,0.000000}%
\pgfsetfillcolor{currentfill}%
\pgfsetlinewidth{0.803000pt}%
\definecolor{currentstroke}{rgb}{0.000000,0.000000,0.000000}%
\pgfsetstrokecolor{currentstroke}%
\pgfsetdash{}{0pt}%
\pgfsys@defobject{currentmarker}{\pgfqpoint{0.000000in}{-0.048611in}}{\pgfqpoint{0.000000in}{0.000000in}}{%
\pgfpathmoveto{\pgfqpoint{0.000000in}{0.000000in}}%
\pgfpathlineto{\pgfqpoint{0.000000in}{-0.048611in}}%
\pgfusepath{stroke,fill}%
}%
\begin{pgfscope}%
\pgfsys@transformshift{2.184068in}{0.521603in}%
\pgfsys@useobject{currentmarker}{}%
\end{pgfscope}%
\end{pgfscope}%
\begin{pgfscope}%
\definecolor{textcolor}{rgb}{0.000000,0.000000,0.000000}%
\pgfsetstrokecolor{textcolor}%
\pgfsetfillcolor{textcolor}%
\pgftext[x=2.184068in,y=0.424381in,,top]{\color{textcolor}{\rmfamily\fontsize{10.000000}{12.000000}\selectfont\catcode`\^=\active\def^{\ifmmode\sp\else\^{}\fi}\catcode`\%=\active\def%{\%}$\mathdefault{40}$}}%
\end{pgfscope}%
\begin{pgfscope}%
\pgfsetbuttcap%
\pgfsetroundjoin%
\definecolor{currentfill}{rgb}{0.000000,0.000000,0.000000}%
\pgfsetfillcolor{currentfill}%
\pgfsetlinewidth{0.803000pt}%
\definecolor{currentstroke}{rgb}{0.000000,0.000000,0.000000}%
\pgfsetstrokecolor{currentstroke}%
\pgfsetdash{}{0pt}%
\pgfsys@defobject{currentmarker}{\pgfqpoint{0.000000in}{-0.048611in}}{\pgfqpoint{0.000000in}{0.000000in}}{%
\pgfpathmoveto{\pgfqpoint{0.000000in}{0.000000in}}%
\pgfpathlineto{\pgfqpoint{0.000000in}{-0.048611in}}%
\pgfusepath{stroke,fill}%
}%
\begin{pgfscope}%
\pgfsys@transformshift{3.126863in}{0.521603in}%
\pgfsys@useobject{currentmarker}{}%
\end{pgfscope}%
\end{pgfscope}%
\begin{pgfscope}%
\definecolor{textcolor}{rgb}{0.000000,0.000000,0.000000}%
\pgfsetstrokecolor{textcolor}%
\pgfsetfillcolor{textcolor}%
\pgftext[x=3.126863in,y=0.424381in,,top]{\color{textcolor}{\rmfamily\fontsize{10.000000}{12.000000}\selectfont\catcode`\^=\active\def^{\ifmmode\sp\else\^{}\fi}\catcode`\%=\active\def%{\%}$\mathdefault{60}$}}%
\end{pgfscope}%
\begin{pgfscope}%
\pgfsetbuttcap%
\pgfsetroundjoin%
\definecolor{currentfill}{rgb}{0.000000,0.000000,0.000000}%
\pgfsetfillcolor{currentfill}%
\pgfsetlinewidth{0.803000pt}%
\definecolor{currentstroke}{rgb}{0.000000,0.000000,0.000000}%
\pgfsetstrokecolor{currentstroke}%
\pgfsetdash{}{0pt}%
\pgfsys@defobject{currentmarker}{\pgfqpoint{0.000000in}{-0.048611in}}{\pgfqpoint{0.000000in}{0.000000in}}{%
\pgfpathmoveto{\pgfqpoint{0.000000in}{0.000000in}}%
\pgfpathlineto{\pgfqpoint{0.000000in}{-0.048611in}}%
\pgfusepath{stroke,fill}%
}%
\begin{pgfscope}%
\pgfsys@transformshift{4.069658in}{0.521603in}%
\pgfsys@useobject{currentmarker}{}%
\end{pgfscope}%
\end{pgfscope}%
\begin{pgfscope}%
\definecolor{textcolor}{rgb}{0.000000,0.000000,0.000000}%
\pgfsetstrokecolor{textcolor}%
\pgfsetfillcolor{textcolor}%
\pgftext[x=4.069658in,y=0.424381in,,top]{\color{textcolor}{\rmfamily\fontsize{10.000000}{12.000000}\selectfont\catcode`\^=\active\def^{\ifmmode\sp\else\^{}\fi}\catcode`\%=\active\def%{\%}$\mathdefault{80}$}}%
\end{pgfscope}%
\begin{pgfscope}%
\pgfsetbuttcap%
\pgfsetroundjoin%
\definecolor{currentfill}{rgb}{0.000000,0.000000,0.000000}%
\pgfsetfillcolor{currentfill}%
\pgfsetlinewidth{0.803000pt}%
\definecolor{currentstroke}{rgb}{0.000000,0.000000,0.000000}%
\pgfsetstrokecolor{currentstroke}%
\pgfsetdash{}{0pt}%
\pgfsys@defobject{currentmarker}{\pgfqpoint{0.000000in}{-0.048611in}}{\pgfqpoint{0.000000in}{0.000000in}}{%
\pgfpathmoveto{\pgfqpoint{0.000000in}{0.000000in}}%
\pgfpathlineto{\pgfqpoint{0.000000in}{-0.048611in}}%
\pgfusepath{stroke,fill}%
}%
\begin{pgfscope}%
\pgfsys@transformshift{5.012453in}{0.521603in}%
\pgfsys@useobject{currentmarker}{}%
\end{pgfscope}%
\end{pgfscope}%
\begin{pgfscope}%
\definecolor{textcolor}{rgb}{0.000000,0.000000,0.000000}%
\pgfsetstrokecolor{textcolor}%
\pgfsetfillcolor{textcolor}%
\pgftext[x=5.012453in,y=0.424381in,,top]{\color{textcolor}{\rmfamily\fontsize{10.000000}{12.000000}\selectfont\catcode`\^=\active\def^{\ifmmode\sp\else\^{}\fi}\catcode`\%=\active\def%{\%}$\mathdefault{100}$}}%
\end{pgfscope}%
\begin{pgfscope}%
\pgfsetbuttcap%
\pgfsetroundjoin%
\definecolor{currentfill}{rgb}{0.000000,0.000000,0.000000}%
\pgfsetfillcolor{currentfill}%
\pgfsetlinewidth{0.803000pt}%
\definecolor{currentstroke}{rgb}{0.000000,0.000000,0.000000}%
\pgfsetstrokecolor{currentstroke}%
\pgfsetdash{}{0pt}%
\pgfsys@defobject{currentmarker}{\pgfqpoint{0.000000in}{-0.048611in}}{\pgfqpoint{0.000000in}{0.000000in}}{%
\pgfpathmoveto{\pgfqpoint{0.000000in}{0.000000in}}%
\pgfpathlineto{\pgfqpoint{0.000000in}{-0.048611in}}%
\pgfusepath{stroke,fill}%
}%
\begin{pgfscope}%
\pgfsys@transformshift{5.955248in}{0.521603in}%
\pgfsys@useobject{currentmarker}{}%
\end{pgfscope}%
\end{pgfscope}%
\begin{pgfscope}%
\definecolor{textcolor}{rgb}{0.000000,0.000000,0.000000}%
\pgfsetstrokecolor{textcolor}%
\pgfsetfillcolor{textcolor}%
\pgftext[x=5.955248in,y=0.424381in,,top]{\color{textcolor}{\rmfamily\fontsize{10.000000}{12.000000}\selectfont\catcode`\^=\active\def^{\ifmmode\sp\else\^{}\fi}\catcode`\%=\active\def%{\%}$\mathdefault{120}$}}%
\end{pgfscope}%
\begin{pgfscope}%
\pgfsetbuttcap%
\pgfsetroundjoin%
\definecolor{currentfill}{rgb}{0.000000,0.000000,0.000000}%
\pgfsetfillcolor{currentfill}%
\pgfsetlinewidth{0.803000pt}%
\definecolor{currentstroke}{rgb}{0.000000,0.000000,0.000000}%
\pgfsetstrokecolor{currentstroke}%
\pgfsetdash{}{0pt}%
\pgfsys@defobject{currentmarker}{\pgfqpoint{0.000000in}{-0.048611in}}{\pgfqpoint{0.000000in}{0.000000in}}{%
\pgfpathmoveto{\pgfqpoint{0.000000in}{0.000000in}}%
\pgfpathlineto{\pgfqpoint{0.000000in}{-0.048611in}}%
\pgfusepath{stroke,fill}%
}%
\begin{pgfscope}%
\pgfsys@transformshift{6.898043in}{0.521603in}%
\pgfsys@useobject{currentmarker}{}%
\end{pgfscope}%
\end{pgfscope}%
\begin{pgfscope}%
\definecolor{textcolor}{rgb}{0.000000,0.000000,0.000000}%
\pgfsetstrokecolor{textcolor}%
\pgfsetfillcolor{textcolor}%
\pgftext[x=6.898043in,y=0.424381in,,top]{\color{textcolor}{\rmfamily\fontsize{10.000000}{12.000000}\selectfont\catcode`\^=\active\def^{\ifmmode\sp\else\^{}\fi}\catcode`\%=\active\def%{\%}$\mathdefault{140}$}}%
\end{pgfscope}%
\begin{pgfscope}%
\definecolor{textcolor}{rgb}{0.000000,0.000000,0.000000}%
\pgfsetstrokecolor{textcolor}%
\pgfsetfillcolor{textcolor}%
\pgftext[x=4.140367in,y=0.234413in,,top]{\color{textcolor}{\rmfamily\fontsize{10.000000}{12.000000}\selectfont\catcode`\^=\active\def^{\ifmmode\sp\else\^{}\fi}\catcode`\%=\active\def%{\%}Triangle components}}%
\end{pgfscope}%
\begin{pgfscope}%
\pgfsetbuttcap%
\pgfsetroundjoin%
\definecolor{currentfill}{rgb}{0.000000,0.000000,0.000000}%
\pgfsetfillcolor{currentfill}%
\pgfsetlinewidth{0.803000pt}%
\definecolor{currentstroke}{rgb}{0.000000,0.000000,0.000000}%
\pgfsetstrokecolor{currentstroke}%
\pgfsetdash{}{0pt}%
\pgfsys@defobject{currentmarker}{\pgfqpoint{-0.048611in}{0.000000in}}{\pgfqpoint{-0.000000in}{0.000000in}}{%
\pgfpathmoveto{\pgfqpoint{-0.000000in}{0.000000in}}%
\pgfpathlineto{\pgfqpoint{-0.048611in}{0.000000in}}%
\pgfusepath{stroke,fill}%
}%
\begin{pgfscope}%
\pgfsys@transformshift{0.588387in}{0.581013in}%
\pgfsys@useobject{currentmarker}{}%
\end{pgfscope}%
\end{pgfscope}%
\begin{pgfscope}%
\definecolor{textcolor}{rgb}{0.000000,0.000000,0.000000}%
\pgfsetstrokecolor{textcolor}%
\pgfsetfillcolor{textcolor}%
\pgftext[x=0.289968in, y=0.528251in, left, base]{\color{textcolor}{\rmfamily\fontsize{10.000000}{12.000000}\selectfont\catcode`\^=\active\def^{\ifmmode\sp\else\^{}\fi}\catcode`\%=\active\def%{\%}$\mathdefault{10^{2}}$}}%
\end{pgfscope}%
\begin{pgfscope}%
\pgfsetbuttcap%
\pgfsetroundjoin%
\definecolor{currentfill}{rgb}{0.000000,0.000000,0.000000}%
\pgfsetfillcolor{currentfill}%
\pgfsetlinewidth{0.803000pt}%
\definecolor{currentstroke}{rgb}{0.000000,0.000000,0.000000}%
\pgfsetstrokecolor{currentstroke}%
\pgfsetdash{}{0pt}%
\pgfsys@defobject{currentmarker}{\pgfqpoint{-0.048611in}{0.000000in}}{\pgfqpoint{-0.000000in}{0.000000in}}{%
\pgfpathmoveto{\pgfqpoint{-0.000000in}{0.000000in}}%
\pgfpathlineto{\pgfqpoint{-0.048611in}{0.000000in}}%
\pgfusepath{stroke,fill}%
}%
\begin{pgfscope}%
\pgfsys@transformshift{0.588387in}{1.741469in}%
\pgfsys@useobject{currentmarker}{}%
\end{pgfscope}%
\end{pgfscope}%
\begin{pgfscope}%
\definecolor{textcolor}{rgb}{0.000000,0.000000,0.000000}%
\pgfsetstrokecolor{textcolor}%
\pgfsetfillcolor{textcolor}%
\pgftext[x=0.289968in, y=1.688708in, left, base]{\color{textcolor}{\rmfamily\fontsize{10.000000}{12.000000}\selectfont\catcode`\^=\active\def^{\ifmmode\sp\else\^{}\fi}\catcode`\%=\active\def%{\%}$\mathdefault{10^{3}}$}}%
\end{pgfscope}%
\begin{pgfscope}%
\pgfsetbuttcap%
\pgfsetroundjoin%
\definecolor{currentfill}{rgb}{0.000000,0.000000,0.000000}%
\pgfsetfillcolor{currentfill}%
\pgfsetlinewidth{0.602250pt}%
\definecolor{currentstroke}{rgb}{0.000000,0.000000,0.000000}%
\pgfsetstrokecolor{currentstroke}%
\pgfsetdash{}{0pt}%
\pgfsys@defobject{currentmarker}{\pgfqpoint{-0.027778in}{0.000000in}}{\pgfqpoint{-0.000000in}{0.000000in}}{%
\pgfpathmoveto{\pgfqpoint{-0.000000in}{0.000000in}}%
\pgfpathlineto{\pgfqpoint{-0.027778in}{0.000000in}}%
\pgfusepath{stroke,fill}%
}%
\begin{pgfscope}%
\pgfsys@transformshift{0.588387in}{0.527913in}%
\pgfsys@useobject{currentmarker}{}%
\end{pgfscope}%
\end{pgfscope}%
\begin{pgfscope}%
\pgfsetbuttcap%
\pgfsetroundjoin%
\definecolor{currentfill}{rgb}{0.000000,0.000000,0.000000}%
\pgfsetfillcolor{currentfill}%
\pgfsetlinewidth{0.602250pt}%
\definecolor{currentstroke}{rgb}{0.000000,0.000000,0.000000}%
\pgfsetstrokecolor{currentstroke}%
\pgfsetdash{}{0pt}%
\pgfsys@defobject{currentmarker}{\pgfqpoint{-0.027778in}{0.000000in}}{\pgfqpoint{-0.000000in}{0.000000in}}{%
\pgfpathmoveto{\pgfqpoint{-0.000000in}{0.000000in}}%
\pgfpathlineto{\pgfqpoint{-0.027778in}{0.000000in}}%
\pgfusepath{stroke,fill}%
}%
\begin{pgfscope}%
\pgfsys@transformshift{0.588387in}{0.930345in}%
\pgfsys@useobject{currentmarker}{}%
\end{pgfscope}%
\end{pgfscope}%
\begin{pgfscope}%
\pgfsetbuttcap%
\pgfsetroundjoin%
\definecolor{currentfill}{rgb}{0.000000,0.000000,0.000000}%
\pgfsetfillcolor{currentfill}%
\pgfsetlinewidth{0.602250pt}%
\definecolor{currentstroke}{rgb}{0.000000,0.000000,0.000000}%
\pgfsetstrokecolor{currentstroke}%
\pgfsetdash{}{0pt}%
\pgfsys@defobject{currentmarker}{\pgfqpoint{-0.027778in}{0.000000in}}{\pgfqpoint{-0.000000in}{0.000000in}}{%
\pgfpathmoveto{\pgfqpoint{-0.000000in}{0.000000in}}%
\pgfpathlineto{\pgfqpoint{-0.027778in}{0.000000in}}%
\pgfusepath{stroke,fill}%
}%
\begin{pgfscope}%
\pgfsys@transformshift{0.588387in}{1.134691in}%
\pgfsys@useobject{currentmarker}{}%
\end{pgfscope}%
\end{pgfscope}%
\begin{pgfscope}%
\pgfsetbuttcap%
\pgfsetroundjoin%
\definecolor{currentfill}{rgb}{0.000000,0.000000,0.000000}%
\pgfsetfillcolor{currentfill}%
\pgfsetlinewidth{0.602250pt}%
\definecolor{currentstroke}{rgb}{0.000000,0.000000,0.000000}%
\pgfsetstrokecolor{currentstroke}%
\pgfsetdash{}{0pt}%
\pgfsys@defobject{currentmarker}{\pgfqpoint{-0.027778in}{0.000000in}}{\pgfqpoint{-0.000000in}{0.000000in}}{%
\pgfpathmoveto{\pgfqpoint{-0.000000in}{0.000000in}}%
\pgfpathlineto{\pgfqpoint{-0.027778in}{0.000000in}}%
\pgfusepath{stroke,fill}%
}%
\begin{pgfscope}%
\pgfsys@transformshift{0.588387in}{1.279677in}%
\pgfsys@useobject{currentmarker}{}%
\end{pgfscope}%
\end{pgfscope}%
\begin{pgfscope}%
\pgfsetbuttcap%
\pgfsetroundjoin%
\definecolor{currentfill}{rgb}{0.000000,0.000000,0.000000}%
\pgfsetfillcolor{currentfill}%
\pgfsetlinewidth{0.602250pt}%
\definecolor{currentstroke}{rgb}{0.000000,0.000000,0.000000}%
\pgfsetstrokecolor{currentstroke}%
\pgfsetdash{}{0pt}%
\pgfsys@defobject{currentmarker}{\pgfqpoint{-0.027778in}{0.000000in}}{\pgfqpoint{-0.000000in}{0.000000in}}{%
\pgfpathmoveto{\pgfqpoint{-0.000000in}{0.000000in}}%
\pgfpathlineto{\pgfqpoint{-0.027778in}{0.000000in}}%
\pgfusepath{stroke,fill}%
}%
\begin{pgfscope}%
\pgfsys@transformshift{0.588387in}{1.392137in}%
\pgfsys@useobject{currentmarker}{}%
\end{pgfscope}%
\end{pgfscope}%
\begin{pgfscope}%
\pgfsetbuttcap%
\pgfsetroundjoin%
\definecolor{currentfill}{rgb}{0.000000,0.000000,0.000000}%
\pgfsetfillcolor{currentfill}%
\pgfsetlinewidth{0.602250pt}%
\definecolor{currentstroke}{rgb}{0.000000,0.000000,0.000000}%
\pgfsetstrokecolor{currentstroke}%
\pgfsetdash{}{0pt}%
\pgfsys@defobject{currentmarker}{\pgfqpoint{-0.027778in}{0.000000in}}{\pgfqpoint{-0.000000in}{0.000000in}}{%
\pgfpathmoveto{\pgfqpoint{-0.000000in}{0.000000in}}%
\pgfpathlineto{\pgfqpoint{-0.027778in}{0.000000in}}%
\pgfusepath{stroke,fill}%
}%
\begin{pgfscope}%
\pgfsys@transformshift{0.588387in}{1.484024in}%
\pgfsys@useobject{currentmarker}{}%
\end{pgfscope}%
\end{pgfscope}%
\begin{pgfscope}%
\pgfsetbuttcap%
\pgfsetroundjoin%
\definecolor{currentfill}{rgb}{0.000000,0.000000,0.000000}%
\pgfsetfillcolor{currentfill}%
\pgfsetlinewidth{0.602250pt}%
\definecolor{currentstroke}{rgb}{0.000000,0.000000,0.000000}%
\pgfsetstrokecolor{currentstroke}%
\pgfsetdash{}{0pt}%
\pgfsys@defobject{currentmarker}{\pgfqpoint{-0.027778in}{0.000000in}}{\pgfqpoint{-0.000000in}{0.000000in}}{%
\pgfpathmoveto{\pgfqpoint{-0.000000in}{0.000000in}}%
\pgfpathlineto{\pgfqpoint{-0.027778in}{0.000000in}}%
\pgfusepath{stroke,fill}%
}%
\begin{pgfscope}%
\pgfsys@transformshift{0.588387in}{1.561712in}%
\pgfsys@useobject{currentmarker}{}%
\end{pgfscope}%
\end{pgfscope}%
\begin{pgfscope}%
\pgfsetbuttcap%
\pgfsetroundjoin%
\definecolor{currentfill}{rgb}{0.000000,0.000000,0.000000}%
\pgfsetfillcolor{currentfill}%
\pgfsetlinewidth{0.602250pt}%
\definecolor{currentstroke}{rgb}{0.000000,0.000000,0.000000}%
\pgfsetstrokecolor{currentstroke}%
\pgfsetdash{}{0pt}%
\pgfsys@defobject{currentmarker}{\pgfqpoint{-0.027778in}{0.000000in}}{\pgfqpoint{-0.000000in}{0.000000in}}{%
\pgfpathmoveto{\pgfqpoint{-0.000000in}{0.000000in}}%
\pgfpathlineto{\pgfqpoint{-0.027778in}{0.000000in}}%
\pgfusepath{stroke,fill}%
}%
\begin{pgfscope}%
\pgfsys@transformshift{0.588387in}{1.629010in}%
\pgfsys@useobject{currentmarker}{}%
\end{pgfscope}%
\end{pgfscope}%
\begin{pgfscope}%
\pgfsetbuttcap%
\pgfsetroundjoin%
\definecolor{currentfill}{rgb}{0.000000,0.000000,0.000000}%
\pgfsetfillcolor{currentfill}%
\pgfsetlinewidth{0.602250pt}%
\definecolor{currentstroke}{rgb}{0.000000,0.000000,0.000000}%
\pgfsetstrokecolor{currentstroke}%
\pgfsetdash{}{0pt}%
\pgfsys@defobject{currentmarker}{\pgfqpoint{-0.027778in}{0.000000in}}{\pgfqpoint{-0.000000in}{0.000000in}}{%
\pgfpathmoveto{\pgfqpoint{-0.000000in}{0.000000in}}%
\pgfpathlineto{\pgfqpoint{-0.027778in}{0.000000in}}%
\pgfusepath{stroke,fill}%
}%
\begin{pgfscope}%
\pgfsys@transformshift{0.588387in}{1.688370in}%
\pgfsys@useobject{currentmarker}{}%
\end{pgfscope}%
\end{pgfscope}%
\begin{pgfscope}%
\pgfsetbuttcap%
\pgfsetroundjoin%
\definecolor{currentfill}{rgb}{0.000000,0.000000,0.000000}%
\pgfsetfillcolor{currentfill}%
\pgfsetlinewidth{0.602250pt}%
\definecolor{currentstroke}{rgb}{0.000000,0.000000,0.000000}%
\pgfsetstrokecolor{currentstroke}%
\pgfsetdash{}{0pt}%
\pgfsys@defobject{currentmarker}{\pgfqpoint{-0.027778in}{0.000000in}}{\pgfqpoint{-0.000000in}{0.000000in}}{%
\pgfpathmoveto{\pgfqpoint{-0.000000in}{0.000000in}}%
\pgfpathlineto{\pgfqpoint{-0.027778in}{0.000000in}}%
\pgfusepath{stroke,fill}%
}%
\begin{pgfscope}%
\pgfsys@transformshift{0.588387in}{2.090802in}%
\pgfsys@useobject{currentmarker}{}%
\end{pgfscope}%
\end{pgfscope}%
\begin{pgfscope}%
\pgfsetbuttcap%
\pgfsetroundjoin%
\definecolor{currentfill}{rgb}{0.000000,0.000000,0.000000}%
\pgfsetfillcolor{currentfill}%
\pgfsetlinewidth{0.602250pt}%
\definecolor{currentstroke}{rgb}{0.000000,0.000000,0.000000}%
\pgfsetstrokecolor{currentstroke}%
\pgfsetdash{}{0pt}%
\pgfsys@defobject{currentmarker}{\pgfqpoint{-0.027778in}{0.000000in}}{\pgfqpoint{-0.000000in}{0.000000in}}{%
\pgfpathmoveto{\pgfqpoint{-0.000000in}{0.000000in}}%
\pgfpathlineto{\pgfqpoint{-0.027778in}{0.000000in}}%
\pgfusepath{stroke,fill}%
}%
\begin{pgfscope}%
\pgfsys@transformshift{0.588387in}{2.295148in}%
\pgfsys@useobject{currentmarker}{}%
\end{pgfscope}%
\end{pgfscope}%
\begin{pgfscope}%
\pgfsetbuttcap%
\pgfsetroundjoin%
\definecolor{currentfill}{rgb}{0.000000,0.000000,0.000000}%
\pgfsetfillcolor{currentfill}%
\pgfsetlinewidth{0.602250pt}%
\definecolor{currentstroke}{rgb}{0.000000,0.000000,0.000000}%
\pgfsetstrokecolor{currentstroke}%
\pgfsetdash{}{0pt}%
\pgfsys@defobject{currentmarker}{\pgfqpoint{-0.027778in}{0.000000in}}{\pgfqpoint{-0.000000in}{0.000000in}}{%
\pgfpathmoveto{\pgfqpoint{-0.000000in}{0.000000in}}%
\pgfpathlineto{\pgfqpoint{-0.027778in}{0.000000in}}%
\pgfusepath{stroke,fill}%
}%
\begin{pgfscope}%
\pgfsys@transformshift{0.588387in}{2.440134in}%
\pgfsys@useobject{currentmarker}{}%
\end{pgfscope}%
\end{pgfscope}%
\begin{pgfscope}%
\definecolor{textcolor}{rgb}{0.000000,0.000000,0.000000}%
\pgfsetstrokecolor{textcolor}%
\pgfsetfillcolor{textcolor}%
\pgftext[x=0.234413in,y=1.526746in,,bottom,rotate=90.000000]{\color{textcolor}{\rmfamily\fontsize{10.000000}{12.000000}\selectfont\catcode`\^=\active\def^{\ifmmode\sp\else\^{}\fi}\catcode`\%=\active\def%{\%}Time [ms]}}%
\end{pgfscope}%
\begin{pgfscope}%
\pgfpathrectangle{\pgfqpoint{0.588387in}{0.521603in}}{\pgfqpoint{7.103961in}{2.010285in}}%
\pgfusepath{clip}%
\pgfsetrectcap%
\pgfsetroundjoin%
\pgfsetlinewidth{1.505625pt}%
\pgfsetstrokecolor{currentstroke1}%
\pgfsetdash{}{0pt}%
\pgfpathmoveto{\pgfqpoint{0.911295in}{0.612980in}}%
\pgfpathlineto{\pgfqpoint{0.958434in}{0.679300in}}%
\pgfpathlineto{\pgfqpoint{1.005574in}{0.721834in}}%
\pgfpathlineto{\pgfqpoint{1.052714in}{0.812634in}}%
\pgfpathlineto{\pgfqpoint{1.099854in}{0.919076in}}%
\pgfpathlineto{\pgfqpoint{1.146993in}{1.021444in}}%
\pgfpathlineto{\pgfqpoint{1.194133in}{1.005146in}}%
\pgfpathlineto{\pgfqpoint{1.241273in}{1.083468in}}%
\pgfpathlineto{\pgfqpoint{1.288413in}{1.121728in}}%
\pgfpathlineto{\pgfqpoint{1.335552in}{1.045540in}}%
\pgfpathlineto{\pgfqpoint{1.382692in}{1.046385in}}%
\pgfpathlineto{\pgfqpoint{1.429832in}{0.975757in}}%
\pgfpathlineto{\pgfqpoint{1.476972in}{1.084315in}}%
\pgfpathlineto{\pgfqpoint{1.524111in}{1.037404in}}%
\pgfpathlineto{\pgfqpoint{1.571251in}{1.096538in}}%
\pgfpathlineto{\pgfqpoint{1.618391in}{1.111748in}}%
\pgfpathlineto{\pgfqpoint{1.665531in}{1.134441in}}%
\pgfpathlineto{\pgfqpoint{1.712670in}{1.131715in}}%
\pgfpathlineto{\pgfqpoint{1.759810in}{1.116065in}}%
\pgfpathlineto{\pgfqpoint{1.806950in}{1.098909in}}%
\pgfpathlineto{\pgfqpoint{1.854090in}{1.160254in}}%
\pgfpathlineto{\pgfqpoint{1.901229in}{1.197824in}}%
\pgfpathlineto{\pgfqpoint{1.948369in}{1.128925in}}%
\pgfpathlineto{\pgfqpoint{1.995509in}{1.208576in}}%
\pgfpathlineto{\pgfqpoint{2.042649in}{1.211111in}}%
\pgfpathlineto{\pgfqpoint{2.089788in}{1.234009in}}%
\pgfpathlineto{\pgfqpoint{2.136928in}{1.214909in}}%
\pgfpathlineto{\pgfqpoint{2.184068in}{1.240355in}}%
\pgfpathlineto{\pgfqpoint{2.231208in}{1.239771in}}%
\pgfpathlineto{\pgfqpoint{2.278347in}{1.318969in}}%
\pgfpathlineto{\pgfqpoint{2.325487in}{1.323945in}}%
\pgfpathlineto{\pgfqpoint{2.372627in}{1.333210in}}%
\pgfpathlineto{\pgfqpoint{2.419767in}{1.371291in}}%
\pgfpathlineto{\pgfqpoint{2.466906in}{1.311865in}}%
\pgfpathlineto{\pgfqpoint{2.514046in}{1.362721in}}%
\pgfpathlineto{\pgfqpoint{2.561186in}{1.397435in}}%
\pgfpathlineto{\pgfqpoint{2.608326in}{1.368023in}}%
\pgfpathlineto{\pgfqpoint{2.655465in}{1.344101in}}%
\pgfpathlineto{\pgfqpoint{2.702605in}{1.403721in}}%
\pgfpathlineto{\pgfqpoint{2.749745in}{1.424898in}}%
\pgfpathlineto{\pgfqpoint{2.796885in}{1.447443in}}%
\pgfpathlineto{\pgfqpoint{2.844024in}{1.455359in}}%
\pgfpathlineto{\pgfqpoint{2.891164in}{1.478815in}}%
\pgfpathlineto{\pgfqpoint{2.938304in}{1.457365in}}%
\pgfpathlineto{\pgfqpoint{2.985444in}{1.505503in}}%
\pgfpathlineto{\pgfqpoint{3.032583in}{1.523719in}}%
\pgfpathlineto{\pgfqpoint{3.079723in}{1.515614in}}%
\pgfpathlineto{\pgfqpoint{3.126863in}{1.573630in}}%
\pgfpathlineto{\pgfqpoint{3.174003in}{1.581728in}}%
\pgfpathlineto{\pgfqpoint{3.221142in}{1.581565in}}%
\pgfpathlineto{\pgfqpoint{3.268282in}{1.567951in}}%
\pgfpathlineto{\pgfqpoint{3.315422in}{1.690504in}}%
\pgfpathlineto{\pgfqpoint{3.362562in}{1.649154in}}%
\pgfpathlineto{\pgfqpoint{3.409701in}{1.646468in}}%
\pgfpathlineto{\pgfqpoint{3.456841in}{1.730663in}}%
\pgfpathlineto{\pgfqpoint{3.503981in}{1.709328in}}%
\pgfpathlineto{\pgfqpoint{3.551121in}{1.715327in}}%
\pgfpathlineto{\pgfqpoint{3.598260in}{1.704005in}}%
\pgfpathlineto{\pgfqpoint{3.645400in}{1.707476in}}%
\pgfpathlineto{\pgfqpoint{3.692540in}{1.794996in}}%
\pgfpathlineto{\pgfqpoint{3.739680in}{1.750293in}}%
\pgfpathlineto{\pgfqpoint{3.786819in}{1.780991in}}%
\pgfpathlineto{\pgfqpoint{3.833959in}{1.768445in}}%
\pgfpathlineto{\pgfqpoint{3.881099in}{1.835138in}}%
\pgfpathlineto{\pgfqpoint{3.928239in}{1.885691in}}%
\pgfpathlineto{\pgfqpoint{3.975378in}{1.855137in}}%
\pgfpathlineto{\pgfqpoint{4.022518in}{1.784458in}}%
\pgfpathlineto{\pgfqpoint{4.069658in}{1.854471in}}%
\pgfpathlineto{\pgfqpoint{4.116798in}{1.881788in}}%
\pgfpathlineto{\pgfqpoint{4.163937in}{1.917089in}}%
\pgfpathlineto{\pgfqpoint{4.211077in}{1.883201in}}%
\pgfpathlineto{\pgfqpoint{4.258217in}{1.914319in}}%
\pgfpathlineto{\pgfqpoint{4.305357in}{1.857520in}}%
\pgfpathlineto{\pgfqpoint{4.352496in}{1.917741in}}%
\pgfpathlineto{\pgfqpoint{4.399636in}{2.048539in}}%
\pgfpathlineto{\pgfqpoint{4.446776in}{2.017994in}}%
\pgfpathlineto{\pgfqpoint{4.493916in}{2.028010in}}%
\pgfpathlineto{\pgfqpoint{4.541055in}{2.023281in}}%
\pgfpathlineto{\pgfqpoint{4.588195in}{2.064862in}}%
\pgfpathlineto{\pgfqpoint{4.635335in}{2.152397in}}%
\pgfpathlineto{\pgfqpoint{4.682475in}{2.081174in}}%
\pgfpathlineto{\pgfqpoint{4.729614in}{2.066163in}}%
\pgfpathlineto{\pgfqpoint{4.776754in}{2.157665in}}%
\pgfpathlineto{\pgfqpoint{4.823894in}{2.090723in}}%
\pgfpathlineto{\pgfqpoint{4.871034in}{2.135604in}}%
\pgfpathlineto{\pgfqpoint{4.918173in}{2.147213in}}%
\pgfpathlineto{\pgfqpoint{4.965313in}{2.203908in}}%
\pgfpathlineto{\pgfqpoint{5.012453in}{2.232342in}}%
\pgfpathlineto{\pgfqpoint{5.059593in}{2.167310in}}%
\pgfpathlineto{\pgfqpoint{5.106732in}{2.231295in}}%
\pgfpathlineto{\pgfqpoint{5.153872in}{2.184391in}}%
\pgfpathlineto{\pgfqpoint{5.201012in}{2.244871in}}%
\pgfpathlineto{\pgfqpoint{5.248152in}{2.340354in}}%
\pgfpathlineto{\pgfqpoint{5.295291in}{2.249166in}}%
\pgfpathlineto{\pgfqpoint{5.342431in}{2.284015in}}%
\pgfpathlineto{\pgfqpoint{5.436711in}{2.337102in}}%
\pgfpathlineto{\pgfqpoint{5.530990in}{2.320331in}}%
\pgfpathlineto{\pgfqpoint{5.578130in}{2.356063in}}%
\pgfpathlineto{\pgfqpoint{5.625270in}{2.379247in}}%
\pgfpathlineto{\pgfqpoint{5.860968in}{2.365311in}}%
\pgfusepath{stroke}%
\end{pgfscope}%
\begin{pgfscope}%
\pgfpathrectangle{\pgfqpoint{0.588387in}{0.521603in}}{\pgfqpoint{7.103961in}{2.010285in}}%
\pgfusepath{clip}%
\pgfsetrectcap%
\pgfsetroundjoin%
\pgfsetlinewidth{1.505625pt}%
\pgfsetstrokecolor{currentstroke2}%
\pgfsetdash{}{0pt}%
\pgfpathmoveto{\pgfqpoint{0.911295in}{0.614811in}}%
\pgfpathlineto{\pgfqpoint{0.958434in}{0.700568in}}%
\pgfpathlineto{\pgfqpoint{1.005574in}{0.749382in}}%
\pgfpathlineto{\pgfqpoint{1.052714in}{0.819428in}}%
\pgfpathlineto{\pgfqpoint{1.099854in}{0.922103in}}%
\pgfpathlineto{\pgfqpoint{1.146993in}{1.043091in}}%
\pgfpathlineto{\pgfqpoint{1.194133in}{1.078006in}}%
\pgfpathlineto{\pgfqpoint{1.241273in}{1.167035in}}%
\pgfpathlineto{\pgfqpoint{1.288413in}{1.290579in}}%
\pgfpathlineto{\pgfqpoint{1.335552in}{1.173735in}}%
\pgfpathlineto{\pgfqpoint{1.382692in}{1.223492in}}%
\pgfpathlineto{\pgfqpoint{1.429832in}{1.236793in}}%
\pgfpathlineto{\pgfqpoint{1.476972in}{1.256253in}}%
\pgfpathlineto{\pgfqpoint{1.524111in}{1.178345in}}%
\pgfpathlineto{\pgfqpoint{1.571251in}{1.263073in}}%
\pgfpathlineto{\pgfqpoint{1.618391in}{1.276579in}}%
\pgfpathlineto{\pgfqpoint{1.665531in}{1.274372in}}%
\pgfpathlineto{\pgfqpoint{1.712670in}{1.278687in}}%
\pgfpathlineto{\pgfqpoint{1.759810in}{1.246045in}}%
\pgfpathlineto{\pgfqpoint{1.806950in}{1.251438in}}%
\pgfpathlineto{\pgfqpoint{1.854090in}{1.383766in}}%
\pgfpathlineto{\pgfqpoint{1.901229in}{1.350101in}}%
\pgfpathlineto{\pgfqpoint{1.948369in}{1.273264in}}%
\pgfpathlineto{\pgfqpoint{1.995509in}{1.341209in}}%
\pgfpathlineto{\pgfqpoint{2.042649in}{1.308174in}}%
\pgfpathlineto{\pgfqpoint{2.089788in}{1.428461in}}%
\pgfpathlineto{\pgfqpoint{2.136928in}{1.408738in}}%
\pgfpathlineto{\pgfqpoint{2.184068in}{1.405285in}}%
\pgfpathlineto{\pgfqpoint{2.231208in}{1.359957in}}%
\pgfpathlineto{\pgfqpoint{2.278347in}{1.487132in}}%
\pgfpathlineto{\pgfqpoint{2.325487in}{1.460770in}}%
\pgfpathlineto{\pgfqpoint{2.372627in}{1.453015in}}%
\pgfpathlineto{\pgfqpoint{2.419767in}{1.472264in}}%
\pgfpathlineto{\pgfqpoint{2.466906in}{1.482654in}}%
\pgfpathlineto{\pgfqpoint{2.514046in}{1.496485in}}%
\pgfpathlineto{\pgfqpoint{2.561186in}{1.453683in}}%
\pgfpathlineto{\pgfqpoint{2.608326in}{1.463768in}}%
\pgfpathlineto{\pgfqpoint{2.655465in}{1.508117in}}%
\pgfpathlineto{\pgfqpoint{2.702605in}{1.525773in}}%
\pgfpathlineto{\pgfqpoint{2.749745in}{1.545798in}}%
\pgfpathlineto{\pgfqpoint{2.796885in}{1.573325in}}%
\pgfpathlineto{\pgfqpoint{2.844024in}{1.539815in}}%
\pgfpathlineto{\pgfqpoint{2.891164in}{1.601421in}}%
\pgfpathlineto{\pgfqpoint{2.938304in}{1.578102in}}%
\pgfpathlineto{\pgfqpoint{2.985444in}{1.645595in}}%
\pgfpathlineto{\pgfqpoint{3.032583in}{1.580106in}}%
\pgfpathlineto{\pgfqpoint{3.079723in}{1.631063in}}%
\pgfpathlineto{\pgfqpoint{3.126863in}{1.629853in}}%
\pgfpathlineto{\pgfqpoint{3.174003in}{1.616622in}}%
\pgfpathlineto{\pgfqpoint{3.221142in}{1.723003in}}%
\pgfpathlineto{\pgfqpoint{3.268282in}{1.646908in}}%
\pgfpathlineto{\pgfqpoint{3.315422in}{1.690635in}}%
\pgfpathlineto{\pgfqpoint{3.362562in}{1.699352in}}%
\pgfpathlineto{\pgfqpoint{3.409701in}{1.687202in}}%
\pgfpathlineto{\pgfqpoint{3.456841in}{1.715681in}}%
\pgfpathlineto{\pgfqpoint{3.503981in}{1.719796in}}%
\pgfpathlineto{\pgfqpoint{3.551121in}{1.703898in}}%
\pgfpathlineto{\pgfqpoint{3.598260in}{1.726869in}}%
\pgfpathlineto{\pgfqpoint{3.645400in}{1.789272in}}%
\pgfpathlineto{\pgfqpoint{3.692540in}{1.759605in}}%
\pgfpathlineto{\pgfqpoint{3.739680in}{1.763009in}}%
\pgfpathlineto{\pgfqpoint{3.786819in}{1.793067in}}%
\pgfpathlineto{\pgfqpoint{3.833959in}{1.745216in}}%
\pgfpathlineto{\pgfqpoint{3.881099in}{1.798188in}}%
\pgfpathlineto{\pgfqpoint{3.928239in}{1.858277in}}%
\pgfpathlineto{\pgfqpoint{3.975378in}{1.857053in}}%
\pgfpathlineto{\pgfqpoint{4.022518in}{1.805209in}}%
\pgfpathlineto{\pgfqpoint{4.069658in}{1.848283in}}%
\pgfpathlineto{\pgfqpoint{4.116798in}{1.889655in}}%
\pgfpathlineto{\pgfqpoint{4.163937in}{1.860911in}}%
\pgfpathlineto{\pgfqpoint{4.211077in}{1.897762in}}%
\pgfpathlineto{\pgfqpoint{4.258217in}{1.890587in}}%
\pgfpathlineto{\pgfqpoint{4.305357in}{1.928747in}}%
\pgfpathlineto{\pgfqpoint{4.352496in}{1.921955in}}%
\pgfpathlineto{\pgfqpoint{4.399636in}{1.956423in}}%
\pgfpathlineto{\pgfqpoint{4.446776in}{1.944743in}}%
\pgfpathlineto{\pgfqpoint{4.493916in}{1.950870in}}%
\pgfpathlineto{\pgfqpoint{4.588195in}{1.991080in}}%
\pgfpathlineto{\pgfqpoint{4.635335in}{2.044837in}}%
\pgfpathlineto{\pgfqpoint{4.682475in}{1.975477in}}%
\pgfpathlineto{\pgfqpoint{4.729614in}{2.027816in}}%
\pgfpathlineto{\pgfqpoint{4.776754in}{2.082586in}}%
\pgfpathlineto{\pgfqpoint{4.823894in}{2.019650in}}%
\pgfpathlineto{\pgfqpoint{4.871034in}{2.047009in}}%
\pgfpathlineto{\pgfqpoint{4.918173in}{2.059556in}}%
\pgfpathlineto{\pgfqpoint{4.965313in}{2.086058in}}%
\pgfpathlineto{\pgfqpoint{5.012453in}{2.093400in}}%
\pgfpathlineto{\pgfqpoint{5.059593in}{2.109303in}}%
\pgfpathlineto{\pgfqpoint{5.106732in}{2.140477in}}%
\pgfpathlineto{\pgfqpoint{5.153872in}{2.131680in}}%
\pgfpathlineto{\pgfqpoint{5.248152in}{2.142590in}}%
\pgfpathlineto{\pgfqpoint{5.295291in}{2.179983in}}%
\pgfpathlineto{\pgfqpoint{5.342431in}{2.154254in}}%
\pgfpathlineto{\pgfqpoint{5.389571in}{2.184740in}}%
\pgfpathlineto{\pgfqpoint{5.436711in}{2.182001in}}%
\pgfpathlineto{\pgfqpoint{5.483850in}{2.190262in}}%
\pgfpathlineto{\pgfqpoint{5.530990in}{2.220882in}}%
\pgfpathlineto{\pgfqpoint{5.578130in}{2.216632in}}%
\pgfpathlineto{\pgfqpoint{5.625270in}{2.261917in}}%
\pgfpathlineto{\pgfqpoint{5.672409in}{2.225476in}}%
\pgfpathlineto{\pgfqpoint{5.719549in}{2.223477in}}%
\pgfpathlineto{\pgfqpoint{5.766689in}{2.243661in}}%
\pgfpathlineto{\pgfqpoint{5.813829in}{2.273173in}}%
\pgfpathlineto{\pgfqpoint{5.860968in}{2.270000in}}%
\pgfpathlineto{\pgfqpoint{5.908108in}{2.262879in}}%
\pgfpathlineto{\pgfqpoint{5.955248in}{2.298018in}}%
\pgfpathlineto{\pgfqpoint{6.002388in}{2.295831in}}%
\pgfpathlineto{\pgfqpoint{6.049527in}{2.325066in}}%
\pgfpathlineto{\pgfqpoint{6.096667in}{2.318887in}}%
\pgfpathlineto{\pgfqpoint{6.143807in}{2.341583in}}%
\pgfpathlineto{\pgfqpoint{6.190947in}{2.356808in}}%
\pgfpathlineto{\pgfqpoint{6.238086in}{2.317949in}}%
\pgfpathlineto{\pgfqpoint{6.285226in}{2.378786in}}%
\pgfpathlineto{\pgfqpoint{6.332366in}{2.356344in}}%
\pgfpathlineto{\pgfqpoint{6.379506in}{2.386948in}}%
\pgfpathlineto{\pgfqpoint{6.426645in}{2.374138in}}%
\pgfpathlineto{\pgfqpoint{6.473785in}{2.373408in}}%
\pgfpathlineto{\pgfqpoint{6.520925in}{2.363995in}}%
\pgfpathlineto{\pgfqpoint{6.568065in}{2.403876in}}%
\pgfpathlineto{\pgfqpoint{6.615204in}{2.407500in}}%
\pgfpathlineto{\pgfqpoint{6.662344in}{2.401863in}}%
\pgfpathlineto{\pgfqpoint{6.709484in}{2.404954in}}%
\pgfpathlineto{\pgfqpoint{6.756624in}{2.406458in}}%
\pgfpathlineto{\pgfqpoint{6.803763in}{2.406565in}}%
\pgfpathlineto{\pgfqpoint{6.850903in}{2.391286in}}%
\pgfpathlineto{\pgfqpoint{6.898043in}{2.437025in}}%
\pgfpathlineto{\pgfqpoint{6.992322in}{2.424965in}}%
\pgfpathlineto{\pgfqpoint{7.086602in}{2.394400in}}%
\pgfpathlineto{\pgfqpoint{7.133742in}{2.426339in}}%
\pgfpathlineto{\pgfqpoint{7.180881in}{2.411624in}}%
\pgfpathlineto{\pgfqpoint{7.228021in}{2.438999in}}%
\pgfpathlineto{\pgfqpoint{7.275161in}{2.440512in}}%
\pgfpathlineto{\pgfqpoint{7.275161in}{2.440512in}}%
\pgfusepath{stroke}%
\end{pgfscope}%
\begin{pgfscope}%
\pgfpathrectangle{\pgfqpoint{0.588387in}{0.521603in}}{\pgfqpoint{7.103961in}{2.010285in}}%
\pgfusepath{clip}%
\pgfsetrectcap%
\pgfsetroundjoin%
\pgfsetlinewidth{1.505625pt}%
\pgfsetstrokecolor{currentstroke3}%
\pgfsetdash{}{0pt}%
\pgfpathmoveto{\pgfqpoint{0.911295in}{0.660581in}}%
\pgfpathlineto{\pgfqpoint{0.958434in}{0.733771in}}%
\pgfpathlineto{\pgfqpoint{1.005574in}{0.715712in}}%
\pgfpathlineto{\pgfqpoint{1.052714in}{0.794670in}}%
\pgfpathlineto{\pgfqpoint{1.099854in}{0.903216in}}%
\pgfpathlineto{\pgfqpoint{1.146993in}{1.036254in}}%
\pgfpathlineto{\pgfqpoint{1.194133in}{1.025846in}}%
\pgfpathlineto{\pgfqpoint{1.241273in}{1.072128in}}%
\pgfpathlineto{\pgfqpoint{1.288413in}{1.165896in}}%
\pgfpathlineto{\pgfqpoint{1.335552in}{1.063131in}}%
\pgfpathlineto{\pgfqpoint{1.382692in}{1.081720in}}%
\pgfpathlineto{\pgfqpoint{1.476972in}{1.098030in}}%
\pgfpathlineto{\pgfqpoint{1.524111in}{1.045135in}}%
\pgfpathlineto{\pgfqpoint{1.571251in}{1.124369in}}%
\pgfpathlineto{\pgfqpoint{1.618391in}{1.167633in}}%
\pgfpathlineto{\pgfqpoint{1.665531in}{1.179562in}}%
\pgfpathlineto{\pgfqpoint{1.712670in}{1.141387in}}%
\pgfpathlineto{\pgfqpoint{1.759810in}{1.125378in}}%
\pgfpathlineto{\pgfqpoint{1.806950in}{1.143579in}}%
\pgfpathlineto{\pgfqpoint{1.854090in}{1.286416in}}%
\pgfpathlineto{\pgfqpoint{1.901229in}{1.261881in}}%
\pgfpathlineto{\pgfqpoint{1.948369in}{1.162369in}}%
\pgfpathlineto{\pgfqpoint{1.995509in}{1.248648in}}%
\pgfpathlineto{\pgfqpoint{2.042649in}{1.219322in}}%
\pgfpathlineto{\pgfqpoint{2.089788in}{1.344531in}}%
\pgfpathlineto{\pgfqpoint{2.136928in}{1.299766in}}%
\pgfpathlineto{\pgfqpoint{2.231208in}{1.271976in}}%
\pgfpathlineto{\pgfqpoint{2.278347in}{1.412175in}}%
\pgfpathlineto{\pgfqpoint{2.325487in}{1.353110in}}%
\pgfpathlineto{\pgfqpoint{2.372627in}{1.402229in}}%
\pgfpathlineto{\pgfqpoint{2.419767in}{1.419325in}}%
\pgfpathlineto{\pgfqpoint{2.466906in}{1.370275in}}%
\pgfpathlineto{\pgfqpoint{2.514046in}{1.436165in}}%
\pgfpathlineto{\pgfqpoint{2.561186in}{1.409788in}}%
\pgfpathlineto{\pgfqpoint{2.608326in}{1.400568in}}%
\pgfpathlineto{\pgfqpoint{2.655465in}{1.393585in}}%
\pgfpathlineto{\pgfqpoint{2.702605in}{1.437376in}}%
\pgfpathlineto{\pgfqpoint{2.749745in}{1.488173in}}%
\pgfpathlineto{\pgfqpoint{2.796885in}{1.512766in}}%
\pgfpathlineto{\pgfqpoint{2.844024in}{1.491460in}}%
\pgfpathlineto{\pgfqpoint{2.891164in}{1.518816in}}%
\pgfpathlineto{\pgfqpoint{2.938304in}{1.509636in}}%
\pgfpathlineto{\pgfqpoint{2.985444in}{1.592515in}}%
\pgfpathlineto{\pgfqpoint{3.032583in}{1.514784in}}%
\pgfpathlineto{\pgfqpoint{3.079723in}{1.556203in}}%
\pgfpathlineto{\pgfqpoint{3.126863in}{1.573986in}}%
\pgfpathlineto{\pgfqpoint{3.174003in}{1.550308in}}%
\pgfpathlineto{\pgfqpoint{3.221142in}{1.655049in}}%
\pgfpathlineto{\pgfqpoint{3.268282in}{1.590631in}}%
\pgfpathlineto{\pgfqpoint{3.315422in}{1.642882in}}%
\pgfpathlineto{\pgfqpoint{3.362562in}{1.637053in}}%
\pgfpathlineto{\pgfqpoint{3.409701in}{1.613902in}}%
\pgfpathlineto{\pgfqpoint{3.456841in}{1.678913in}}%
\pgfpathlineto{\pgfqpoint{3.503981in}{1.665636in}}%
\pgfpathlineto{\pgfqpoint{3.551121in}{1.658583in}}%
\pgfpathlineto{\pgfqpoint{3.598260in}{1.653938in}}%
\pgfpathlineto{\pgfqpoint{3.645400in}{1.741519in}}%
\pgfpathlineto{\pgfqpoint{3.692540in}{1.722069in}}%
\pgfpathlineto{\pgfqpoint{3.739680in}{1.705152in}}%
\pgfpathlineto{\pgfqpoint{3.786819in}{1.754118in}}%
\pgfpathlineto{\pgfqpoint{3.833959in}{1.666906in}}%
\pgfpathlineto{\pgfqpoint{3.881099in}{1.748630in}}%
\pgfpathlineto{\pgfqpoint{3.928239in}{1.840151in}}%
\pgfpathlineto{\pgfqpoint{3.975378in}{1.786098in}}%
\pgfpathlineto{\pgfqpoint{4.022518in}{1.737224in}}%
\pgfpathlineto{\pgfqpoint{4.069658in}{1.768673in}}%
\pgfpathlineto{\pgfqpoint{4.116798in}{1.861278in}}%
\pgfpathlineto{\pgfqpoint{4.163937in}{1.786918in}}%
\pgfpathlineto{\pgfqpoint{4.211077in}{1.851588in}}%
\pgfpathlineto{\pgfqpoint{4.258217in}{1.858827in}}%
\pgfpathlineto{\pgfqpoint{4.305357in}{1.876865in}}%
\pgfpathlineto{\pgfqpoint{4.352496in}{1.859061in}}%
\pgfpathlineto{\pgfqpoint{4.399636in}{1.916676in}}%
\pgfpathlineto{\pgfqpoint{4.446776in}{1.898107in}}%
\pgfpathlineto{\pgfqpoint{4.493916in}{1.889157in}}%
\pgfpathlineto{\pgfqpoint{4.541055in}{1.892103in}}%
\pgfpathlineto{\pgfqpoint{4.588195in}{1.916571in}}%
\pgfpathlineto{\pgfqpoint{4.635335in}{1.999292in}}%
\pgfpathlineto{\pgfqpoint{4.682475in}{1.943091in}}%
\pgfpathlineto{\pgfqpoint{4.729614in}{1.963584in}}%
\pgfpathlineto{\pgfqpoint{4.776754in}{2.021850in}}%
\pgfpathlineto{\pgfqpoint{4.823894in}{1.965659in}}%
\pgfpathlineto{\pgfqpoint{4.871034in}{1.999126in}}%
\pgfpathlineto{\pgfqpoint{4.918173in}{1.997923in}}%
\pgfpathlineto{\pgfqpoint{4.965313in}{2.034782in}}%
\pgfpathlineto{\pgfqpoint{5.012453in}{2.047303in}}%
\pgfpathlineto{\pgfqpoint{5.059593in}{2.046007in}}%
\pgfpathlineto{\pgfqpoint{5.106732in}{2.072269in}}%
\pgfpathlineto{\pgfqpoint{5.153872in}{2.081093in}}%
\pgfpathlineto{\pgfqpoint{5.201012in}{2.077559in}}%
\pgfpathlineto{\pgfqpoint{5.248152in}{2.097076in}}%
\pgfpathlineto{\pgfqpoint{5.295291in}{2.098911in}}%
\pgfpathlineto{\pgfqpoint{5.342431in}{2.098135in}}%
\pgfpathlineto{\pgfqpoint{5.389571in}{2.140304in}}%
\pgfpathlineto{\pgfqpoint{5.436711in}{2.123570in}}%
\pgfpathlineto{\pgfqpoint{5.483850in}{2.209572in}}%
\pgfpathlineto{\pgfqpoint{5.530990in}{2.149282in}}%
\pgfpathlineto{\pgfqpoint{5.578130in}{2.166094in}}%
\pgfpathlineto{\pgfqpoint{5.625270in}{2.226986in}}%
\pgfpathlineto{\pgfqpoint{5.672409in}{2.167240in}}%
\pgfpathlineto{\pgfqpoint{5.719549in}{2.172476in}}%
\pgfpathlineto{\pgfqpoint{5.766689in}{2.190450in}}%
\pgfpathlineto{\pgfqpoint{5.813829in}{2.226725in}}%
\pgfpathlineto{\pgfqpoint{5.860968in}{2.255066in}}%
\pgfpathlineto{\pgfqpoint{5.908108in}{2.210380in}}%
\pgfpathlineto{\pgfqpoint{5.955248in}{2.247894in}}%
\pgfpathlineto{\pgfqpoint{6.002388in}{2.236077in}}%
\pgfpathlineto{\pgfqpoint{6.049527in}{2.301796in}}%
\pgfpathlineto{\pgfqpoint{6.096667in}{2.270820in}}%
\pgfpathlineto{\pgfqpoint{6.143807in}{2.293594in}}%
\pgfpathlineto{\pgfqpoint{6.190947in}{2.287716in}}%
\pgfpathlineto{\pgfqpoint{6.238086in}{2.248705in}}%
\pgfpathlineto{\pgfqpoint{6.285226in}{2.329700in}}%
\pgfpathlineto{\pgfqpoint{6.332366in}{2.319897in}}%
\pgfpathlineto{\pgfqpoint{6.379506in}{2.349290in}}%
\pgfpathlineto{\pgfqpoint{6.426645in}{2.335987in}}%
\pgfpathlineto{\pgfqpoint{6.473785in}{2.317125in}}%
\pgfpathlineto{\pgfqpoint{6.520925in}{2.336514in}}%
\pgfpathlineto{\pgfqpoint{6.568065in}{2.360852in}}%
\pgfpathlineto{\pgfqpoint{6.615204in}{2.365428in}}%
\pgfpathlineto{\pgfqpoint{6.662344in}{2.341895in}}%
\pgfpathlineto{\pgfqpoint{6.709484in}{2.354392in}}%
\pgfpathlineto{\pgfqpoint{6.756624in}{2.388153in}}%
\pgfpathlineto{\pgfqpoint{6.803763in}{2.400239in}}%
\pgfpathlineto{\pgfqpoint{6.850903in}{2.401161in}}%
\pgfpathlineto{\pgfqpoint{6.898043in}{2.404567in}}%
\pgfpathlineto{\pgfqpoint{6.945183in}{2.416445in}}%
\pgfpathlineto{\pgfqpoint{6.992322in}{2.408816in}}%
\pgfpathlineto{\pgfqpoint{7.039462in}{2.408849in}}%
\pgfpathlineto{\pgfqpoint{7.086602in}{2.402712in}}%
\pgfpathlineto{\pgfqpoint{7.133742in}{2.384789in}}%
\pgfpathlineto{\pgfqpoint{7.180881in}{2.354732in}}%
\pgfpathlineto{\pgfqpoint{7.228021in}{2.424046in}}%
\pgfpathlineto{\pgfqpoint{7.275161in}{2.440260in}}%
\pgfpathlineto{\pgfqpoint{7.322301in}{2.437987in}}%
\pgfpathlineto{\pgfqpoint{7.369440in}{2.433603in}}%
\pgfpathlineto{\pgfqpoint{7.369440in}{2.433603in}}%
\pgfusepath{stroke}%
\end{pgfscope}%
\begin{pgfscope}%
\pgfpathrectangle{\pgfqpoint{0.588387in}{0.521603in}}{\pgfqpoint{7.103961in}{2.010285in}}%
\pgfusepath{clip}%
\pgfsetrectcap%
\pgfsetroundjoin%
\pgfsetlinewidth{1.505625pt}%
\pgfsetstrokecolor{currentstroke4}%
\pgfsetdash{}{0pt}%
\pgfpathmoveto{\pgfqpoint{0.911295in}{0.642650in}}%
\pgfpathlineto{\pgfqpoint{0.958434in}{0.739852in}}%
\pgfpathlineto{\pgfqpoint{1.005574in}{0.874090in}}%
\pgfpathlineto{\pgfqpoint{1.052714in}{1.015033in}}%
\pgfpathlineto{\pgfqpoint{1.099854in}{1.131319in}}%
\pgfpathlineto{\pgfqpoint{1.146993in}{1.045667in}}%
\pgfpathlineto{\pgfqpoint{1.194133in}{1.013453in}}%
\pgfpathlineto{\pgfqpoint{1.241273in}{1.121682in}}%
\pgfpathlineto{\pgfqpoint{1.288413in}{1.158936in}}%
\pgfpathlineto{\pgfqpoint{1.335552in}{1.129157in}}%
\pgfpathlineto{\pgfqpoint{1.382692in}{1.157716in}}%
\pgfpathlineto{\pgfqpoint{1.429832in}{0.981525in}}%
\pgfpathlineto{\pgfqpoint{1.476972in}{1.078176in}}%
\pgfpathlineto{\pgfqpoint{1.524111in}{1.067731in}}%
\pgfpathlineto{\pgfqpoint{1.571251in}{1.143275in}}%
\pgfpathlineto{\pgfqpoint{1.618391in}{1.163889in}}%
\pgfpathlineto{\pgfqpoint{1.665531in}{1.205993in}}%
\pgfpathlineto{\pgfqpoint{1.712670in}{1.118762in}}%
\pgfpathlineto{\pgfqpoint{1.759810in}{1.118741in}}%
\pgfpathlineto{\pgfqpoint{1.806950in}{1.133665in}}%
\pgfpathlineto{\pgfqpoint{1.854090in}{1.172143in}}%
\pgfpathlineto{\pgfqpoint{1.901229in}{1.219209in}}%
\pgfpathlineto{\pgfqpoint{1.948369in}{1.160226in}}%
\pgfpathlineto{\pgfqpoint{1.995509in}{1.180905in}}%
\pgfpathlineto{\pgfqpoint{2.042649in}{1.183627in}}%
\pgfpathlineto{\pgfqpoint{2.089788in}{1.210462in}}%
\pgfpathlineto{\pgfqpoint{2.136928in}{1.208573in}}%
\pgfpathlineto{\pgfqpoint{2.184068in}{1.232740in}}%
\pgfpathlineto{\pgfqpoint{2.231208in}{1.257180in}}%
\pgfpathlineto{\pgfqpoint{2.278347in}{1.295828in}}%
\pgfpathlineto{\pgfqpoint{2.325487in}{1.271953in}}%
\pgfpathlineto{\pgfqpoint{2.372627in}{1.307632in}}%
\pgfpathlineto{\pgfqpoint{2.419767in}{1.344206in}}%
\pgfpathlineto{\pgfqpoint{2.466906in}{1.279817in}}%
\pgfpathlineto{\pgfqpoint{2.514046in}{1.391051in}}%
\pgfpathlineto{\pgfqpoint{2.561186in}{1.322121in}}%
\pgfpathlineto{\pgfqpoint{2.608326in}{1.346079in}}%
\pgfpathlineto{\pgfqpoint{2.655465in}{1.289086in}}%
\pgfpathlineto{\pgfqpoint{2.702605in}{1.370071in}}%
\pgfpathlineto{\pgfqpoint{2.749745in}{1.383243in}}%
\pgfpathlineto{\pgfqpoint{2.796885in}{1.439753in}}%
\pgfpathlineto{\pgfqpoint{2.844024in}{1.373837in}}%
\pgfpathlineto{\pgfqpoint{2.891164in}{1.394935in}}%
\pgfpathlineto{\pgfqpoint{2.938304in}{1.394867in}}%
\pgfpathlineto{\pgfqpoint{2.985444in}{1.448425in}}%
\pgfpathlineto{\pgfqpoint{3.032583in}{1.455983in}}%
\pgfpathlineto{\pgfqpoint{3.079723in}{1.449312in}}%
\pgfpathlineto{\pgfqpoint{3.126863in}{1.507083in}}%
\pgfpathlineto{\pgfqpoint{3.174003in}{1.495948in}}%
\pgfpathlineto{\pgfqpoint{3.221142in}{1.495420in}}%
\pgfpathlineto{\pgfqpoint{3.268282in}{1.486354in}}%
\pgfpathlineto{\pgfqpoint{3.315422in}{1.574754in}}%
\pgfpathlineto{\pgfqpoint{3.362562in}{1.569898in}}%
\pgfpathlineto{\pgfqpoint{3.409701in}{1.521837in}}%
\pgfpathlineto{\pgfqpoint{3.456841in}{1.603871in}}%
\pgfpathlineto{\pgfqpoint{3.503981in}{1.598571in}}%
\pgfpathlineto{\pgfqpoint{3.551121in}{1.608387in}}%
\pgfpathlineto{\pgfqpoint{3.598260in}{1.593346in}}%
\pgfpathlineto{\pgfqpoint{3.645400in}{1.618360in}}%
\pgfpathlineto{\pgfqpoint{3.692540in}{1.654726in}}%
\pgfpathlineto{\pgfqpoint{3.739680in}{1.602678in}}%
\pgfpathlineto{\pgfqpoint{3.786819in}{1.650147in}}%
\pgfpathlineto{\pgfqpoint{3.833959in}{1.623425in}}%
\pgfpathlineto{\pgfqpoint{3.881099in}{1.674586in}}%
\pgfpathlineto{\pgfqpoint{3.928239in}{1.761748in}}%
\pgfpathlineto{\pgfqpoint{3.975378in}{1.733533in}}%
\pgfpathlineto{\pgfqpoint{4.022518in}{1.641944in}}%
\pgfpathlineto{\pgfqpoint{4.069658in}{1.692321in}}%
\pgfpathlineto{\pgfqpoint{4.116798in}{1.747783in}}%
\pgfpathlineto{\pgfqpoint{4.163937in}{1.782983in}}%
\pgfpathlineto{\pgfqpoint{4.211077in}{1.758564in}}%
\pgfpathlineto{\pgfqpoint{4.258217in}{1.739374in}}%
\pgfpathlineto{\pgfqpoint{4.305357in}{1.705505in}}%
\pgfpathlineto{\pgfqpoint{4.352496in}{1.776198in}}%
\pgfpathlineto{\pgfqpoint{4.399636in}{1.836252in}}%
\pgfpathlineto{\pgfqpoint{4.446776in}{1.868881in}}%
\pgfpathlineto{\pgfqpoint{4.493916in}{1.853473in}}%
\pgfpathlineto{\pgfqpoint{4.541055in}{1.840519in}}%
\pgfpathlineto{\pgfqpoint{4.588195in}{1.877233in}}%
\pgfpathlineto{\pgfqpoint{4.635335in}{1.982179in}}%
\pgfpathlineto{\pgfqpoint{4.682475in}{1.882753in}}%
\pgfpathlineto{\pgfqpoint{4.729614in}{1.885202in}}%
\pgfpathlineto{\pgfqpoint{4.776754in}{1.962585in}}%
\pgfpathlineto{\pgfqpoint{4.823894in}{1.918785in}}%
\pgfpathlineto{\pgfqpoint{4.871034in}{1.923138in}}%
\pgfpathlineto{\pgfqpoint{4.918173in}{1.944196in}}%
\pgfpathlineto{\pgfqpoint{4.965313in}{2.028479in}}%
\pgfpathlineto{\pgfqpoint{5.012453in}{2.017857in}}%
\pgfpathlineto{\pgfqpoint{5.059593in}{1.983376in}}%
\pgfpathlineto{\pgfqpoint{5.106732in}{2.040912in}}%
\pgfpathlineto{\pgfqpoint{5.153872in}{1.977356in}}%
\pgfpathlineto{\pgfqpoint{5.201012in}{2.049487in}}%
\pgfpathlineto{\pgfqpoint{5.248152in}{2.104719in}}%
\pgfpathlineto{\pgfqpoint{5.295291in}{2.018250in}}%
\pgfpathlineto{\pgfqpoint{5.342431in}{2.073126in}}%
\pgfpathlineto{\pgfqpoint{5.436711in}{2.110467in}}%
\pgfpathlineto{\pgfqpoint{5.483850in}{2.348892in}}%
\pgfpathlineto{\pgfqpoint{5.530990in}{2.117785in}}%
\pgfpathlineto{\pgfqpoint{5.578130in}{2.154489in}}%
\pgfpathlineto{\pgfqpoint{5.625270in}{2.210529in}}%
\pgfpathlineto{\pgfqpoint{5.860968in}{2.244562in}}%
\pgfpathlineto{\pgfqpoint{6.049527in}{2.318135in}}%
\pgfpathlineto{\pgfqpoint{6.379506in}{2.368226in}}%
\pgfpathlineto{\pgfqpoint{6.568065in}{2.416679in}}%
\pgfusepath{stroke}%
\end{pgfscope}%
\begin{pgfscope}%
\pgfpathrectangle{\pgfqpoint{0.588387in}{0.521603in}}{\pgfqpoint{7.103961in}{2.010285in}}%
\pgfusepath{clip}%
\pgfsetrectcap%
\pgfsetroundjoin%
\pgfsetlinewidth{1.505625pt}%
\pgfsetstrokecolor{currentstroke5}%
\pgfsetdash{}{0pt}%
\pgfpathmoveto{\pgfqpoint{0.911295in}{1.575132in}}%
\pgfpathlineto{\pgfqpoint{0.958434in}{0.735836in}}%
\pgfpathlineto{\pgfqpoint{1.005574in}{0.821425in}}%
\pgfpathlineto{\pgfqpoint{1.052714in}{0.896106in}}%
\pgfpathlineto{\pgfqpoint{1.099854in}{0.955742in}}%
\pgfpathlineto{\pgfqpoint{1.146993in}{1.075527in}}%
\pgfpathlineto{\pgfqpoint{1.194133in}{1.123069in}}%
\pgfpathlineto{\pgfqpoint{1.241273in}{1.227993in}}%
\pgfpathlineto{\pgfqpoint{1.288413in}{0.856847in}}%
\pgfpathlineto{\pgfqpoint{1.335552in}{0.835515in}}%
\pgfpathlineto{\pgfqpoint{1.382692in}{0.882874in}}%
\pgfpathlineto{\pgfqpoint{1.429832in}{0.958353in}}%
\pgfpathlineto{\pgfqpoint{1.476972in}{0.965514in}}%
\pgfpathlineto{\pgfqpoint{1.524111in}{1.008521in}}%
\pgfpathlineto{\pgfqpoint{1.571251in}{1.077148in}}%
\pgfpathlineto{\pgfqpoint{1.618391in}{0.889251in}}%
\pgfpathlineto{\pgfqpoint{1.665531in}{0.916689in}}%
\pgfpathlineto{\pgfqpoint{1.712670in}{0.949778in}}%
\pgfpathlineto{\pgfqpoint{1.759810in}{0.983971in}}%
\pgfpathlineto{\pgfqpoint{1.806950in}{1.031583in}}%
\pgfpathlineto{\pgfqpoint{1.854090in}{0.966753in}}%
\pgfpathlineto{\pgfqpoint{1.901229in}{1.093460in}}%
\pgfpathlineto{\pgfqpoint{1.948369in}{0.998443in}}%
\pgfpathlineto{\pgfqpoint{1.995509in}{0.995410in}}%
\pgfpathlineto{\pgfqpoint{2.042649in}{1.007840in}}%
\pgfpathlineto{\pgfqpoint{2.089788in}{1.129121in}}%
\pgfpathlineto{\pgfqpoint{2.136928in}{1.095690in}}%
\pgfpathlineto{\pgfqpoint{2.184068in}{1.092530in}}%
\pgfpathlineto{\pgfqpoint{2.231208in}{1.128905in}}%
\pgfpathlineto{\pgfqpoint{2.278347in}{1.091783in}}%
\pgfpathlineto{\pgfqpoint{2.325487in}{1.099042in}}%
\pgfpathlineto{\pgfqpoint{2.372627in}{1.116213in}}%
\pgfpathlineto{\pgfqpoint{2.419767in}{1.168981in}}%
\pgfpathlineto{\pgfqpoint{2.466906in}{1.139606in}}%
\pgfpathlineto{\pgfqpoint{2.514046in}{1.301710in}}%
\pgfpathlineto{\pgfqpoint{2.561186in}{1.229910in}}%
\pgfpathlineto{\pgfqpoint{2.608326in}{1.181005in}}%
\pgfpathlineto{\pgfqpoint{2.655465in}{1.210466in}}%
\pgfpathlineto{\pgfqpoint{2.702605in}{1.254342in}}%
\pgfpathlineto{\pgfqpoint{2.749745in}{1.359880in}}%
\pgfpathlineto{\pgfqpoint{2.796885in}{1.335667in}}%
\pgfpathlineto{\pgfqpoint{2.844024in}{1.382598in}}%
\pgfpathlineto{\pgfqpoint{2.891164in}{1.352133in}}%
\pgfpathlineto{\pgfqpoint{2.938304in}{1.329855in}}%
\pgfpathlineto{\pgfqpoint{2.985444in}{1.407768in}}%
\pgfpathlineto{\pgfqpoint{3.032583in}{1.337243in}}%
\pgfpathlineto{\pgfqpoint{3.126863in}{1.546918in}}%
\pgfpathlineto{\pgfqpoint{3.174003in}{1.300956in}}%
\pgfpathlineto{\pgfqpoint{3.221142in}{1.467372in}}%
\pgfpathlineto{\pgfqpoint{3.315422in}{1.559367in}}%
\pgfpathlineto{\pgfqpoint{3.362562in}{1.589122in}}%
\pgfpathlineto{\pgfqpoint{3.409701in}{1.446545in}}%
\pgfpathlineto{\pgfqpoint{3.456841in}{1.621233in}}%
\pgfpathlineto{\pgfqpoint{3.503981in}{1.527166in}}%
\pgfpathlineto{\pgfqpoint{3.598260in}{1.519429in}}%
\pgfpathlineto{\pgfqpoint{3.692540in}{1.696631in}}%
\pgfpathlineto{\pgfqpoint{3.833959in}{1.536619in}}%
\pgfpathlineto{\pgfqpoint{4.069658in}{1.765518in}}%
\pgfusepath{stroke}%
\end{pgfscope}%
\begin{pgfscope}%
\pgfsetrectcap%
\pgfsetmiterjoin%
\pgfsetlinewidth{0.803000pt}%
\definecolor{currentstroke}{rgb}{0.000000,0.000000,0.000000}%
\pgfsetstrokecolor{currentstroke}%
\pgfsetdash{}{0pt}%
\pgfpathmoveto{\pgfqpoint{0.588387in}{0.521603in}}%
\pgfpathlineto{\pgfqpoint{0.588387in}{2.531888in}}%
\pgfusepath{stroke}%
\end{pgfscope}%
\begin{pgfscope}%
\pgfsetrectcap%
\pgfsetmiterjoin%
\pgfsetlinewidth{0.803000pt}%
\definecolor{currentstroke}{rgb}{0.000000,0.000000,0.000000}%
\pgfsetstrokecolor{currentstroke}%
\pgfsetdash{}{0pt}%
\pgfpathmoveto{\pgfqpoint{7.692348in}{0.521603in}}%
\pgfpathlineto{\pgfqpoint{7.692348in}{2.531888in}}%
\pgfusepath{stroke}%
\end{pgfscope}%
\begin{pgfscope}%
\pgfsetrectcap%
\pgfsetmiterjoin%
\pgfsetlinewidth{0.803000pt}%
\definecolor{currentstroke}{rgb}{0.000000,0.000000,0.000000}%
\pgfsetstrokecolor{currentstroke}%
\pgfsetdash{}{0pt}%
\pgfpathmoveto{\pgfqpoint{0.588387in}{0.521603in}}%
\pgfpathlineto{\pgfqpoint{7.692348in}{0.521603in}}%
\pgfusepath{stroke}%
\end{pgfscope}%
\begin{pgfscope}%
\pgfsetrectcap%
\pgfsetmiterjoin%
\pgfsetlinewidth{0.803000pt}%
\definecolor{currentstroke}{rgb}{0.000000,0.000000,0.000000}%
\pgfsetstrokecolor{currentstroke}%
\pgfsetdash{}{0pt}%
\pgfpathmoveto{\pgfqpoint{0.588387in}{2.531888in}}%
\pgfpathlineto{\pgfqpoint{7.692348in}{2.531888in}}%
\pgfusepath{stroke}%
\end{pgfscope}%
\begin{pgfscope}%
\definecolor{textcolor}{rgb}{0.000000,0.000000,0.000000}%
\pgfsetstrokecolor{textcolor}%
\pgfsetfillcolor{textcolor}%
\pgftext[x=4.140367in,y=2.615222in,,base]{\color{textcolor}{\rmfamily\fontsize{12.000000}{14.400000}\selectfont\catcode`\^=\active\def^{\ifmmode\sp\else\^{}\fi}\catcode`\%=\active\def%{\%}Mean}}%
\end{pgfscope}%
\begin{pgfscope}%
\pgfsetbuttcap%
\pgfsetmiterjoin%
\definecolor{currentfill}{rgb}{1.000000,1.000000,1.000000}%
\pgfsetfillcolor{currentfill}%
\pgfsetfillopacity{0.800000}%
\pgfsetlinewidth{1.003750pt}%
\definecolor{currentstroke}{rgb}{0.800000,0.800000,0.800000}%
\pgfsetstrokecolor{currentstroke}%
\pgfsetstrokeopacity{0.800000}%
\pgfsetdash{}{0pt}%
\pgfpathmoveto{\pgfqpoint{7.779848in}{1.514531in}}%
\pgfpathlineto{\pgfqpoint{8.259376in}{1.514531in}}%
\pgfpathquadraticcurveto{\pgfqpoint{8.284376in}{1.514531in}}{\pgfqpoint{8.284376in}{1.539531in}}%
\pgfpathlineto{\pgfqpoint{8.284376in}{2.444388in}}%
\pgfpathquadraticcurveto{\pgfqpoint{8.284376in}{2.469388in}}{\pgfqpoint{8.259376in}{2.469388in}}%
\pgfpathlineto{\pgfqpoint{7.779848in}{2.469388in}}%
\pgfpathquadraticcurveto{\pgfqpoint{7.754848in}{2.469388in}}{\pgfqpoint{7.754848in}{2.444388in}}%
\pgfpathlineto{\pgfqpoint{7.754848in}{1.539531in}}%
\pgfpathquadraticcurveto{\pgfqpoint{7.754848in}{1.514531in}}{\pgfqpoint{7.779848in}{1.514531in}}%
\pgfpathlineto{\pgfqpoint{7.779848in}{1.514531in}}%
\pgfpathclose%
\pgfusepath{stroke,fill}%
\end{pgfscope}%
\begin{pgfscope}%
\pgfsetrectcap%
\pgfsetroundjoin%
\pgfsetlinewidth{1.505625pt}%
\pgfsetstrokecolor{currentstroke1}%
\pgfsetdash{}{0pt}%
\pgfpathmoveto{\pgfqpoint{7.804848in}{2.368168in}}%
\pgfpathlineto{\pgfqpoint{7.929848in}{2.368168in}}%
\pgfpathlineto{\pgfqpoint{8.054848in}{2.368168in}}%
\pgfusepath{stroke}%
\end{pgfscope}%
\begin{pgfscope}%
\definecolor{textcolor}{rgb}{0.000000,0.000000,0.000000}%
\pgfsetstrokecolor{textcolor}%
\pgfsetfillcolor{textcolor}%
\pgftext[x=8.154848in,y=2.324418in,left,base]{\color{textcolor}{\rmfamily\fontsize{9.000000}{10.800000}\selectfont\catcode`\^=\active\def^{\ifmmode\sp\else\^{}\fi}\catcode`\%=\active\def%{\%}3}}%
\end{pgfscope}%
\begin{pgfscope}%
\pgfsetrectcap%
\pgfsetroundjoin%
\pgfsetlinewidth{1.505625pt}%
\pgfsetstrokecolor{currentstroke2}%
\pgfsetdash{}{0pt}%
\pgfpathmoveto{\pgfqpoint{7.804848in}{2.184696in}}%
\pgfpathlineto{\pgfqpoint{7.929848in}{2.184696in}}%
\pgfpathlineto{\pgfqpoint{8.054848in}{2.184696in}}%
\pgfusepath{stroke}%
\end{pgfscope}%
\begin{pgfscope}%
\definecolor{textcolor}{rgb}{0.000000,0.000000,0.000000}%
\pgfsetstrokecolor{textcolor}%
\pgfsetfillcolor{textcolor}%
\pgftext[x=8.154848in,y=2.140946in,left,base]{\color{textcolor}{\rmfamily\fontsize{9.000000}{10.800000}\selectfont\catcode`\^=\active\def^{\ifmmode\sp\else\^{}\fi}\catcode`\%=\active\def%{\%}4}}%
\end{pgfscope}%
\begin{pgfscope}%
\pgfsetrectcap%
\pgfsetroundjoin%
\pgfsetlinewidth{1.505625pt}%
\pgfsetstrokecolor{currentstroke3}%
\pgfsetdash{}{0pt}%
\pgfpathmoveto{\pgfqpoint{7.804848in}{2.001225in}}%
\pgfpathlineto{\pgfqpoint{7.929848in}{2.001225in}}%
\pgfpathlineto{\pgfqpoint{8.054848in}{2.001225in}}%
\pgfusepath{stroke}%
\end{pgfscope}%
\begin{pgfscope}%
\definecolor{textcolor}{rgb}{0.000000,0.000000,0.000000}%
\pgfsetstrokecolor{textcolor}%
\pgfsetfillcolor{textcolor}%
\pgftext[x=8.154848in,y=1.957475in,left,base]{\color{textcolor}{\rmfamily\fontsize{9.000000}{10.800000}\selectfont\catcode`\^=\active\def^{\ifmmode\sp\else\^{}\fi}\catcode`\%=\active\def%{\%}5}}%
\end{pgfscope}%
\begin{pgfscope}%
\pgfsetrectcap%
\pgfsetroundjoin%
\pgfsetlinewidth{1.505625pt}%
\pgfsetstrokecolor{currentstroke4}%
\pgfsetdash{}{0pt}%
\pgfpathmoveto{\pgfqpoint{7.804848in}{1.817753in}}%
\pgfpathlineto{\pgfqpoint{7.929848in}{1.817753in}}%
\pgfpathlineto{\pgfqpoint{8.054848in}{1.817753in}}%
\pgfusepath{stroke}%
\end{pgfscope}%
\begin{pgfscope}%
\definecolor{textcolor}{rgb}{0.000000,0.000000,0.000000}%
\pgfsetstrokecolor{textcolor}%
\pgfsetfillcolor{textcolor}%
\pgftext[x=8.154848in,y=1.774003in,left,base]{\color{textcolor}{\rmfamily\fontsize{9.000000}{10.800000}\selectfont\catcode`\^=\active\def^{\ifmmode\sp\else\^{}\fi}\catcode`\%=\active\def%{\%}6}}%
\end{pgfscope}%
\begin{pgfscope}%
\pgfsetrectcap%
\pgfsetroundjoin%
\pgfsetlinewidth{1.505625pt}%
\pgfsetstrokecolor{currentstroke5}%
\pgfsetdash{}{0pt}%
\pgfpathmoveto{\pgfqpoint{7.804848in}{1.634281in}}%
\pgfpathlineto{\pgfqpoint{7.929848in}{1.634281in}}%
\pgfpathlineto{\pgfqpoint{8.054848in}{1.634281in}}%
\pgfusepath{stroke}%
\end{pgfscope}%
\begin{pgfscope}%
\definecolor{textcolor}{rgb}{0.000000,0.000000,0.000000}%
\pgfsetstrokecolor{textcolor}%
\pgfsetfillcolor{textcolor}%
\pgftext[x=8.154848in,y=1.590531in,left,base]{\color{textcolor}{\rmfamily\fontsize{9.000000}{10.800000}\selectfont\catcode`\^=\active\def^{\ifmmode\sp\else\^{}\fi}\catcode`\%=\active\def%{\%}7}}%
\end{pgfscope}%
\end{pgfpicture}%
\makeatother%
\endgroup%
}
	\caption[Mean runtime for graphs with no NAC-coloring]{
		Mean running time to finish search for graphs with no NAC-coloring for different subgraph sizes \( k \).}%
	\label{fig:graph_no_nac_coloring_first_runtime_subgraph_size}
\end{figure}%
\begin{figure}[thbp]
	\centering
	\scalebox{\BenchFigureScale}{%% Creator: Matplotlib, PGF backend
%%
%% To include the figure in your LaTeX document, write
%%   \input{<filename>.pgf}
%%
%% Make sure the required packages are loaded in your preamble
%%   \usepackage{pgf}
%%
%% Also ensure that all the required font packages are loaded; for instance,
%% the lmodern package is sometimes necessary when using math font.
%%   \usepackage{lmodern}
%%
%% Figures using additional raster images can only be included by \input if
%% they are in the same directory as the main LaTeX file. For loading figures
%% from other directories you can use the `import` package
%%   \usepackage{import}
%%
%% and then include the figures with
%%   \import{<path to file>}{<filename>.pgf}
%%
%% Matplotlib used the following preamble
%%   \def\mathdefault#1{#1}
%%   \everymath=\expandafter{\the\everymath\displaystyle}
%%   \IfFileExists{scrextend.sty}{
%%     \usepackage[fontsize=10.000000pt]{scrextend}
%%   }{
%%     \renewcommand{\normalsize}{\fontsize{10.000000}{12.000000}\selectfont}
%%     \normalsize
%%   }
%%   
%%   \ifdefined\pdftexversion\else  % non-pdftex case.
%%     \usepackage{fontspec}
%%     \setmainfont{DejaVuSans.ttf}[Path=\detokenize{/home/petr/Projects/PyRigi/.venv/lib/python3.12/site-packages/matplotlib/mpl-data/fonts/ttf/}]
%%     \setsansfont{DejaVuSans.ttf}[Path=\detokenize{/home/petr/Projects/PyRigi/.venv/lib/python3.12/site-packages/matplotlib/mpl-data/fonts/ttf/}]
%%     \setmonofont{DejaVuSansMono.ttf}[Path=\detokenize{/home/petr/Projects/PyRigi/.venv/lib/python3.12/site-packages/matplotlib/mpl-data/fonts/ttf/}]
%%   \fi
%%   \makeatletter\@ifpackageloaded{under\Score{}}{}{\usepackage[strings]{under\Score{}}}\makeatother
%%
\begingroup%
\makeatletter%
\begin{pgfpicture}%
\pgfpathrectangle{\pgfpointorigin}{\pgfqpoint{8.384376in}{2.841849in}}%
\pgfusepath{use as bounding box, clip}%
\begin{pgfscope}%
\pgfsetbuttcap%
\pgfsetmiterjoin%
\definecolor{currentfill}{rgb}{1.000000,1.000000,1.000000}%
\pgfsetfillcolor{currentfill}%
\pgfsetlinewidth{0.000000pt}%
\definecolor{currentstroke}{rgb}{1.000000,1.000000,1.000000}%
\pgfsetstrokecolor{currentstroke}%
\pgfsetdash{}{0pt}%
\pgfpathmoveto{\pgfqpoint{0.000000in}{0.000000in}}%
\pgfpathlineto{\pgfqpoint{8.384376in}{0.000000in}}%
\pgfpathlineto{\pgfqpoint{8.384376in}{2.841849in}}%
\pgfpathlineto{\pgfqpoint{0.000000in}{2.841849in}}%
\pgfpathlineto{\pgfqpoint{0.000000in}{0.000000in}}%
\pgfpathclose%
\pgfusepath{fill}%
\end{pgfscope}%
\begin{pgfscope}%
\pgfsetbuttcap%
\pgfsetmiterjoin%
\definecolor{currentfill}{rgb}{1.000000,1.000000,1.000000}%
\pgfsetfillcolor{currentfill}%
\pgfsetlinewidth{0.000000pt}%
\definecolor{currentstroke}{rgb}{0.000000,0.000000,0.000000}%
\pgfsetstrokecolor{currentstroke}%
\pgfsetstrokeopacity{0.000000}%
\pgfsetdash{}{0pt}%
\pgfpathmoveto{\pgfqpoint{0.588387in}{0.521603in}}%
\pgfpathlineto{\pgfqpoint{7.495005in}{0.521603in}}%
\pgfpathlineto{\pgfqpoint{7.495005in}{2.741849in}}%
\pgfpathlineto{\pgfqpoint{0.588387in}{2.741849in}}%
\pgfpathlineto{\pgfqpoint{0.588387in}{0.521603in}}%
\pgfpathclose%
\pgfusepath{fill}%
\end{pgfscope}%
\begin{pgfscope}%
\pgfsetbuttcap%
\pgfsetroundjoin%
\definecolor{currentfill}{rgb}{0.000000,0.000000,0.000000}%
\pgfsetfillcolor{currentfill}%
\pgfsetlinewidth{0.803000pt}%
\definecolor{currentstroke}{rgb}{0.000000,0.000000,0.000000}%
\pgfsetstrokecolor{currentstroke}%
\pgfsetdash{}{0pt}%
\pgfsys@defobject{currentmarker}{\pgfqpoint{0.000000in}{-0.048611in}}{\pgfqpoint{0.000000in}{0.000000in}}{%
\pgfpathmoveto{\pgfqpoint{0.000000in}{0.000000in}}%
\pgfpathlineto{\pgfqpoint{0.000000in}{-0.048611in}}%
\pgfusepath{stroke,fill}%
}%
\begin{pgfscope}%
\pgfsys@transformshift{1.183462in}{0.521603in}%
\pgfsys@useobject{currentmarker}{}%
\end{pgfscope}%
\end{pgfscope}%
\begin{pgfscope}%
\definecolor{textcolor}{rgb}{0.000000,0.000000,0.000000}%
\pgfsetstrokecolor{textcolor}%
\pgfsetfillcolor{textcolor}%
\pgftext[x=1.183462in,y=0.424381in,,top]{\color{textcolor}{\rmfamily\fontsize{10.000000}{12.000000}\selectfont\catcode`\^=\active\def^{\ifmmode\sp\else\^{}\fi}\catcode`\%=\active\def%{\%}$\mathdefault{16}$}}%
\end{pgfscope}%
\begin{pgfscope}%
\pgfsetbuttcap%
\pgfsetroundjoin%
\definecolor{currentfill}{rgb}{0.000000,0.000000,0.000000}%
\pgfsetfillcolor{currentfill}%
\pgfsetlinewidth{0.803000pt}%
\definecolor{currentstroke}{rgb}{0.000000,0.000000,0.000000}%
\pgfsetstrokecolor{currentstroke}%
\pgfsetdash{}{0pt}%
\pgfsys@defobject{currentmarker}{\pgfqpoint{0.000000in}{-0.048611in}}{\pgfqpoint{0.000000in}{0.000000in}}{%
\pgfpathmoveto{\pgfqpoint{0.000000in}{0.000000in}}%
\pgfpathlineto{\pgfqpoint{0.000000in}{-0.048611in}}%
\pgfusepath{stroke,fill}%
}%
\begin{pgfscope}%
\pgfsys@transformshift{1.933163in}{0.521603in}%
\pgfsys@useobject{currentmarker}{}%
\end{pgfscope}%
\end{pgfscope}%
\begin{pgfscope}%
\definecolor{textcolor}{rgb}{0.000000,0.000000,0.000000}%
\pgfsetstrokecolor{textcolor}%
\pgfsetfillcolor{textcolor}%
\pgftext[x=1.933163in,y=0.424381in,,top]{\color{textcolor}{\rmfamily\fontsize{10.000000}{12.000000}\selectfont\catcode`\^=\active\def^{\ifmmode\sp\else\^{}\fi}\catcode`\%=\active\def%{\%}$\mathdefault{24}$}}%
\end{pgfscope}%
\begin{pgfscope}%
\pgfsetbuttcap%
\pgfsetroundjoin%
\definecolor{currentfill}{rgb}{0.000000,0.000000,0.000000}%
\pgfsetfillcolor{currentfill}%
\pgfsetlinewidth{0.803000pt}%
\definecolor{currentstroke}{rgb}{0.000000,0.000000,0.000000}%
\pgfsetstrokecolor{currentstroke}%
\pgfsetdash{}{0pt}%
\pgfsys@defobject{currentmarker}{\pgfqpoint{0.000000in}{-0.048611in}}{\pgfqpoint{0.000000in}{0.000000in}}{%
\pgfpathmoveto{\pgfqpoint{0.000000in}{0.000000in}}%
\pgfpathlineto{\pgfqpoint{0.000000in}{-0.048611in}}%
\pgfusepath{stroke,fill}%
}%
\begin{pgfscope}%
\pgfsys@transformshift{2.682863in}{0.521603in}%
\pgfsys@useobject{currentmarker}{}%
\end{pgfscope}%
\end{pgfscope}%
\begin{pgfscope}%
\definecolor{textcolor}{rgb}{0.000000,0.000000,0.000000}%
\pgfsetstrokecolor{textcolor}%
\pgfsetfillcolor{textcolor}%
\pgftext[x=2.682863in,y=0.424381in,,top]{\color{textcolor}{\rmfamily\fontsize{10.000000}{12.000000}\selectfont\catcode`\^=\active\def^{\ifmmode\sp\else\^{}\fi}\catcode`\%=\active\def%{\%}$\mathdefault{32}$}}%
\end{pgfscope}%
\begin{pgfscope}%
\pgfsetbuttcap%
\pgfsetroundjoin%
\definecolor{currentfill}{rgb}{0.000000,0.000000,0.000000}%
\pgfsetfillcolor{currentfill}%
\pgfsetlinewidth{0.803000pt}%
\definecolor{currentstroke}{rgb}{0.000000,0.000000,0.000000}%
\pgfsetstrokecolor{currentstroke}%
\pgfsetdash{}{0pt}%
\pgfsys@defobject{currentmarker}{\pgfqpoint{0.000000in}{-0.048611in}}{\pgfqpoint{0.000000in}{0.000000in}}{%
\pgfpathmoveto{\pgfqpoint{0.000000in}{0.000000in}}%
\pgfpathlineto{\pgfqpoint{0.000000in}{-0.048611in}}%
\pgfusepath{stroke,fill}%
}%
\begin{pgfscope}%
\pgfsys@transformshift{3.432564in}{0.521603in}%
\pgfsys@useobject{currentmarker}{}%
\end{pgfscope}%
\end{pgfscope}%
\begin{pgfscope}%
\definecolor{textcolor}{rgb}{0.000000,0.000000,0.000000}%
\pgfsetstrokecolor{textcolor}%
\pgfsetfillcolor{textcolor}%
\pgftext[x=3.432564in,y=0.424381in,,top]{\color{textcolor}{\rmfamily\fontsize{10.000000}{12.000000}\selectfont\catcode`\^=\active\def^{\ifmmode\sp\else\^{}\fi}\catcode`\%=\active\def%{\%}$\mathdefault{40}$}}%
\end{pgfscope}%
\begin{pgfscope}%
\pgfsetbuttcap%
\pgfsetroundjoin%
\definecolor{currentfill}{rgb}{0.000000,0.000000,0.000000}%
\pgfsetfillcolor{currentfill}%
\pgfsetlinewidth{0.803000pt}%
\definecolor{currentstroke}{rgb}{0.000000,0.000000,0.000000}%
\pgfsetstrokecolor{currentstroke}%
\pgfsetdash{}{0pt}%
\pgfsys@defobject{currentmarker}{\pgfqpoint{0.000000in}{-0.048611in}}{\pgfqpoint{0.000000in}{0.000000in}}{%
\pgfpathmoveto{\pgfqpoint{0.000000in}{0.000000in}}%
\pgfpathlineto{\pgfqpoint{0.000000in}{-0.048611in}}%
\pgfusepath{stroke,fill}%
}%
\begin{pgfscope}%
\pgfsys@transformshift{4.182265in}{0.521603in}%
\pgfsys@useobject{currentmarker}{}%
\end{pgfscope}%
\end{pgfscope}%
\begin{pgfscope}%
\definecolor{textcolor}{rgb}{0.000000,0.000000,0.000000}%
\pgfsetstrokecolor{textcolor}%
\pgfsetfillcolor{textcolor}%
\pgftext[x=4.182265in,y=0.424381in,,top]{\color{textcolor}{\rmfamily\fontsize{10.000000}{12.000000}\selectfont\catcode`\^=\active\def^{\ifmmode\sp\else\^{}\fi}\catcode`\%=\active\def%{\%}$\mathdefault{48}$}}%
\end{pgfscope}%
\begin{pgfscope}%
\pgfsetbuttcap%
\pgfsetroundjoin%
\definecolor{currentfill}{rgb}{0.000000,0.000000,0.000000}%
\pgfsetfillcolor{currentfill}%
\pgfsetlinewidth{0.803000pt}%
\definecolor{currentstroke}{rgb}{0.000000,0.000000,0.000000}%
\pgfsetstrokecolor{currentstroke}%
\pgfsetdash{}{0pt}%
\pgfsys@defobject{currentmarker}{\pgfqpoint{0.000000in}{-0.048611in}}{\pgfqpoint{0.000000in}{0.000000in}}{%
\pgfpathmoveto{\pgfqpoint{0.000000in}{0.000000in}}%
\pgfpathlineto{\pgfqpoint{0.000000in}{-0.048611in}}%
\pgfusepath{stroke,fill}%
}%
\begin{pgfscope}%
\pgfsys@transformshift{4.931965in}{0.521603in}%
\pgfsys@useobject{currentmarker}{}%
\end{pgfscope}%
\end{pgfscope}%
\begin{pgfscope}%
\definecolor{textcolor}{rgb}{0.000000,0.000000,0.000000}%
\pgfsetstrokecolor{textcolor}%
\pgfsetfillcolor{textcolor}%
\pgftext[x=4.931965in,y=0.424381in,,top]{\color{textcolor}{\rmfamily\fontsize{10.000000}{12.000000}\selectfont\catcode`\^=\active\def^{\ifmmode\sp\else\^{}\fi}\catcode`\%=\active\def%{\%}$\mathdefault{56}$}}%
\end{pgfscope}%
\begin{pgfscope}%
\pgfsetbuttcap%
\pgfsetroundjoin%
\definecolor{currentfill}{rgb}{0.000000,0.000000,0.000000}%
\pgfsetfillcolor{currentfill}%
\pgfsetlinewidth{0.803000pt}%
\definecolor{currentstroke}{rgb}{0.000000,0.000000,0.000000}%
\pgfsetstrokecolor{currentstroke}%
\pgfsetdash{}{0pt}%
\pgfsys@defobject{currentmarker}{\pgfqpoint{0.000000in}{-0.048611in}}{\pgfqpoint{0.000000in}{0.000000in}}{%
\pgfpathmoveto{\pgfqpoint{0.000000in}{0.000000in}}%
\pgfpathlineto{\pgfqpoint{0.000000in}{-0.048611in}}%
\pgfusepath{stroke,fill}%
}%
\begin{pgfscope}%
\pgfsys@transformshift{5.681666in}{0.521603in}%
\pgfsys@useobject{currentmarker}{}%
\end{pgfscope}%
\end{pgfscope}%
\begin{pgfscope}%
\definecolor{textcolor}{rgb}{0.000000,0.000000,0.000000}%
\pgfsetstrokecolor{textcolor}%
\pgfsetfillcolor{textcolor}%
\pgftext[x=5.681666in,y=0.424381in,,top]{\color{textcolor}{\rmfamily\fontsize{10.000000}{12.000000}\selectfont\catcode`\^=\active\def^{\ifmmode\sp\else\^{}\fi}\catcode`\%=\active\def%{\%}$\mathdefault{64}$}}%
\end{pgfscope}%
\begin{pgfscope}%
\pgfsetbuttcap%
\pgfsetroundjoin%
\definecolor{currentfill}{rgb}{0.000000,0.000000,0.000000}%
\pgfsetfillcolor{currentfill}%
\pgfsetlinewidth{0.803000pt}%
\definecolor{currentstroke}{rgb}{0.000000,0.000000,0.000000}%
\pgfsetstrokecolor{currentstroke}%
\pgfsetdash{}{0pt}%
\pgfsys@defobject{currentmarker}{\pgfqpoint{0.000000in}{-0.048611in}}{\pgfqpoint{0.000000in}{0.000000in}}{%
\pgfpathmoveto{\pgfqpoint{0.000000in}{0.000000in}}%
\pgfpathlineto{\pgfqpoint{0.000000in}{-0.048611in}}%
\pgfusepath{stroke,fill}%
}%
\begin{pgfscope}%
\pgfsys@transformshift{6.431367in}{0.521603in}%
\pgfsys@useobject{currentmarker}{}%
\end{pgfscope}%
\end{pgfscope}%
\begin{pgfscope}%
\definecolor{textcolor}{rgb}{0.000000,0.000000,0.000000}%
\pgfsetstrokecolor{textcolor}%
\pgfsetfillcolor{textcolor}%
\pgftext[x=6.431367in,y=0.424381in,,top]{\color{textcolor}{\rmfamily\fontsize{10.000000}{12.000000}\selectfont\catcode`\^=\active\def^{\ifmmode\sp\else\^{}\fi}\catcode`\%=\active\def%{\%}$\mathdefault{72}$}}%
\end{pgfscope}%
\begin{pgfscope}%
\pgfsetbuttcap%
\pgfsetroundjoin%
\definecolor{currentfill}{rgb}{0.000000,0.000000,0.000000}%
\pgfsetfillcolor{currentfill}%
\pgfsetlinewidth{0.803000pt}%
\definecolor{currentstroke}{rgb}{0.000000,0.000000,0.000000}%
\pgfsetstrokecolor{currentstroke}%
\pgfsetdash{}{0pt}%
\pgfsys@defobject{currentmarker}{\pgfqpoint{0.000000in}{-0.048611in}}{\pgfqpoint{0.000000in}{0.000000in}}{%
\pgfpathmoveto{\pgfqpoint{0.000000in}{0.000000in}}%
\pgfpathlineto{\pgfqpoint{0.000000in}{-0.048611in}}%
\pgfusepath{stroke,fill}%
}%
\begin{pgfscope}%
\pgfsys@transformshift{7.181068in}{0.521603in}%
\pgfsys@useobject{currentmarker}{}%
\end{pgfscope}%
\end{pgfscope}%
\begin{pgfscope}%
\definecolor{textcolor}{rgb}{0.000000,0.000000,0.000000}%
\pgfsetstrokecolor{textcolor}%
\pgfsetfillcolor{textcolor}%
\pgftext[x=7.181068in,y=0.424381in,,top]{\color{textcolor}{\rmfamily\fontsize{10.000000}{12.000000}\selectfont\catcode`\^=\active\def^{\ifmmode\sp\else\^{}\fi}\catcode`\%=\active\def%{\%}$\mathdefault{80}$}}%
\end{pgfscope}%
\begin{pgfscope}%
\definecolor{textcolor}{rgb}{0.000000,0.000000,0.000000}%
\pgfsetstrokecolor{textcolor}%
\pgfsetfillcolor{textcolor}%
\pgftext[x=4.041696in,y=0.234413in,,top]{\color{textcolor}{\rmfamily\fontsize{10.000000}{12.000000}\selectfont\catcode`\^=\active\def^{\ifmmode\sp\else\^{}\fi}\catcode`\%=\active\def%{\%}$\triangle$-connected components}}%
\end{pgfscope}%
\begin{pgfscope}%
\pgfsetbuttcap%
\pgfsetroundjoin%
\definecolor{currentfill}{rgb}{0.000000,0.000000,0.000000}%
\pgfsetfillcolor{currentfill}%
\pgfsetlinewidth{0.803000pt}%
\definecolor{currentstroke}{rgb}{0.000000,0.000000,0.000000}%
\pgfsetstrokecolor{currentstroke}%
\pgfsetdash{}{0pt}%
\pgfsys@defobject{currentmarker}{\pgfqpoint{-0.048611in}{0.000000in}}{\pgfqpoint{-0.000000in}{0.000000in}}{%
\pgfpathmoveto{\pgfqpoint{-0.000000in}{0.000000in}}%
\pgfpathlineto{\pgfqpoint{-0.048611in}{0.000000in}}%
\pgfusepath{stroke,fill}%
}%
\begin{pgfscope}%
\pgfsys@transformshift{0.588387in}{0.902440in}%
\pgfsys@useobject{currentmarker}{}%
\end{pgfscope}%
\end{pgfscope}%
\begin{pgfscope}%
\definecolor{textcolor}{rgb}{0.000000,0.000000,0.000000}%
\pgfsetstrokecolor{textcolor}%
\pgfsetfillcolor{textcolor}%
\pgftext[x=0.289968in, y=0.849679in, left, base]{\color{textcolor}{\rmfamily\fontsize{10.000000}{12.000000}\selectfont\catcode`\^=\active\def^{\ifmmode\sp\else\^{}\fi}\catcode`\%=\active\def%{\%}$\mathdefault{10^{2}}$}}%
\end{pgfscope}%
\begin{pgfscope}%
\pgfsetbuttcap%
\pgfsetroundjoin%
\definecolor{currentfill}{rgb}{0.000000,0.000000,0.000000}%
\pgfsetfillcolor{currentfill}%
\pgfsetlinewidth{0.803000pt}%
\definecolor{currentstroke}{rgb}{0.000000,0.000000,0.000000}%
\pgfsetstrokecolor{currentstroke}%
\pgfsetdash{}{0pt}%
\pgfsys@defobject{currentmarker}{\pgfqpoint{-0.048611in}{0.000000in}}{\pgfqpoint{-0.000000in}{0.000000in}}{%
\pgfpathmoveto{\pgfqpoint{-0.000000in}{0.000000in}}%
\pgfpathlineto{\pgfqpoint{-0.048611in}{0.000000in}}%
\pgfusepath{stroke,fill}%
}%
\begin{pgfscope}%
\pgfsys@transformshift{0.588387in}{1.687517in}%
\pgfsys@useobject{currentmarker}{}%
\end{pgfscope}%
\end{pgfscope}%
\begin{pgfscope}%
\definecolor{textcolor}{rgb}{0.000000,0.000000,0.000000}%
\pgfsetstrokecolor{textcolor}%
\pgfsetfillcolor{textcolor}%
\pgftext[x=0.289968in, y=1.634755in, left, base]{\color{textcolor}{\rmfamily\fontsize{10.000000}{12.000000}\selectfont\catcode`\^=\active\def^{\ifmmode\sp\else\^{}\fi}\catcode`\%=\active\def%{\%}$\mathdefault{10^{3}}$}}%
\end{pgfscope}%
\begin{pgfscope}%
\pgfsetbuttcap%
\pgfsetroundjoin%
\definecolor{currentfill}{rgb}{0.000000,0.000000,0.000000}%
\pgfsetfillcolor{currentfill}%
\pgfsetlinewidth{0.803000pt}%
\definecolor{currentstroke}{rgb}{0.000000,0.000000,0.000000}%
\pgfsetstrokecolor{currentstroke}%
\pgfsetdash{}{0pt}%
\pgfsys@defobject{currentmarker}{\pgfqpoint{-0.048611in}{0.000000in}}{\pgfqpoint{-0.000000in}{0.000000in}}{%
\pgfpathmoveto{\pgfqpoint{-0.000000in}{0.000000in}}%
\pgfpathlineto{\pgfqpoint{-0.048611in}{0.000000in}}%
\pgfusepath{stroke,fill}%
}%
\begin{pgfscope}%
\pgfsys@transformshift{0.588387in}{2.472593in}%
\pgfsys@useobject{currentmarker}{}%
\end{pgfscope}%
\end{pgfscope}%
\begin{pgfscope}%
\definecolor{textcolor}{rgb}{0.000000,0.000000,0.000000}%
\pgfsetstrokecolor{textcolor}%
\pgfsetfillcolor{textcolor}%
\pgftext[x=0.289968in, y=2.419831in, left, base]{\color{textcolor}{\rmfamily\fontsize{10.000000}{12.000000}\selectfont\catcode`\^=\active\def^{\ifmmode\sp\else\^{}\fi}\catcode`\%=\active\def%{\%}$\mathdefault{10^{4}}$}}%
\end{pgfscope}%
\begin{pgfscope}%
\pgfsetbuttcap%
\pgfsetroundjoin%
\definecolor{currentfill}{rgb}{0.000000,0.000000,0.000000}%
\pgfsetfillcolor{currentfill}%
\pgfsetlinewidth{0.602250pt}%
\definecolor{currentstroke}{rgb}{0.000000,0.000000,0.000000}%
\pgfsetstrokecolor{currentstroke}%
\pgfsetdash{}{0pt}%
\pgfsys@defobject{currentmarker}{\pgfqpoint{-0.027778in}{0.000000in}}{\pgfqpoint{-0.000000in}{0.000000in}}{%
\pgfpathmoveto{\pgfqpoint{-0.000000in}{0.000000in}}%
\pgfpathlineto{\pgfqpoint{-0.027778in}{0.000000in}}%
\pgfusepath{stroke,fill}%
}%
\begin{pgfscope}%
\pgfsys@transformshift{0.588387in}{0.590027in}%
\pgfsys@useobject{currentmarker}{}%
\end{pgfscope}%
\end{pgfscope}%
\begin{pgfscope}%
\pgfsetbuttcap%
\pgfsetroundjoin%
\definecolor{currentfill}{rgb}{0.000000,0.000000,0.000000}%
\pgfsetfillcolor{currentfill}%
\pgfsetlinewidth{0.602250pt}%
\definecolor{currentstroke}{rgb}{0.000000,0.000000,0.000000}%
\pgfsetstrokecolor{currentstroke}%
\pgfsetdash{}{0pt}%
\pgfsys@defobject{currentmarker}{\pgfqpoint{-0.027778in}{0.000000in}}{\pgfqpoint{-0.000000in}{0.000000in}}{%
\pgfpathmoveto{\pgfqpoint{-0.000000in}{0.000000in}}%
\pgfpathlineto{\pgfqpoint{-0.027778in}{0.000000in}}%
\pgfusepath{stroke,fill}%
}%
\begin{pgfscope}%
\pgfsys@transformshift{0.588387in}{0.666109in}%
\pgfsys@useobject{currentmarker}{}%
\end{pgfscope}%
\end{pgfscope}%
\begin{pgfscope}%
\pgfsetbuttcap%
\pgfsetroundjoin%
\definecolor{currentfill}{rgb}{0.000000,0.000000,0.000000}%
\pgfsetfillcolor{currentfill}%
\pgfsetlinewidth{0.602250pt}%
\definecolor{currentstroke}{rgb}{0.000000,0.000000,0.000000}%
\pgfsetstrokecolor{currentstroke}%
\pgfsetdash{}{0pt}%
\pgfsys@defobject{currentmarker}{\pgfqpoint{-0.027778in}{0.000000in}}{\pgfqpoint{-0.000000in}{0.000000in}}{%
\pgfpathmoveto{\pgfqpoint{-0.000000in}{0.000000in}}%
\pgfpathlineto{\pgfqpoint{-0.027778in}{0.000000in}}%
\pgfusepath{stroke,fill}%
}%
\begin{pgfscope}%
\pgfsys@transformshift{0.588387in}{0.728272in}%
\pgfsys@useobject{currentmarker}{}%
\end{pgfscope}%
\end{pgfscope}%
\begin{pgfscope}%
\pgfsetbuttcap%
\pgfsetroundjoin%
\definecolor{currentfill}{rgb}{0.000000,0.000000,0.000000}%
\pgfsetfillcolor{currentfill}%
\pgfsetlinewidth{0.602250pt}%
\definecolor{currentstroke}{rgb}{0.000000,0.000000,0.000000}%
\pgfsetstrokecolor{currentstroke}%
\pgfsetdash{}{0pt}%
\pgfsys@defobject{currentmarker}{\pgfqpoint{-0.027778in}{0.000000in}}{\pgfqpoint{-0.000000in}{0.000000in}}{%
\pgfpathmoveto{\pgfqpoint{-0.000000in}{0.000000in}}%
\pgfpathlineto{\pgfqpoint{-0.027778in}{0.000000in}}%
\pgfusepath{stroke,fill}%
}%
\begin{pgfscope}%
\pgfsys@transformshift{0.588387in}{0.780831in}%
\pgfsys@useobject{currentmarker}{}%
\end{pgfscope}%
\end{pgfscope}%
\begin{pgfscope}%
\pgfsetbuttcap%
\pgfsetroundjoin%
\definecolor{currentfill}{rgb}{0.000000,0.000000,0.000000}%
\pgfsetfillcolor{currentfill}%
\pgfsetlinewidth{0.602250pt}%
\definecolor{currentstroke}{rgb}{0.000000,0.000000,0.000000}%
\pgfsetstrokecolor{currentstroke}%
\pgfsetdash{}{0pt}%
\pgfsys@defobject{currentmarker}{\pgfqpoint{-0.027778in}{0.000000in}}{\pgfqpoint{-0.000000in}{0.000000in}}{%
\pgfpathmoveto{\pgfqpoint{-0.000000in}{0.000000in}}%
\pgfpathlineto{\pgfqpoint{-0.027778in}{0.000000in}}%
\pgfusepath{stroke,fill}%
}%
\begin{pgfscope}%
\pgfsys@transformshift{0.588387in}{0.826359in}%
\pgfsys@useobject{currentmarker}{}%
\end{pgfscope}%
\end{pgfscope}%
\begin{pgfscope}%
\pgfsetbuttcap%
\pgfsetroundjoin%
\definecolor{currentfill}{rgb}{0.000000,0.000000,0.000000}%
\pgfsetfillcolor{currentfill}%
\pgfsetlinewidth{0.602250pt}%
\definecolor{currentstroke}{rgb}{0.000000,0.000000,0.000000}%
\pgfsetstrokecolor{currentstroke}%
\pgfsetdash{}{0pt}%
\pgfsys@defobject{currentmarker}{\pgfqpoint{-0.027778in}{0.000000in}}{\pgfqpoint{-0.000000in}{0.000000in}}{%
\pgfpathmoveto{\pgfqpoint{-0.000000in}{0.000000in}}%
\pgfpathlineto{\pgfqpoint{-0.027778in}{0.000000in}}%
\pgfusepath{stroke,fill}%
}%
\begin{pgfscope}%
\pgfsys@transformshift{0.588387in}{0.866517in}%
\pgfsys@useobject{currentmarker}{}%
\end{pgfscope}%
\end{pgfscope}%
\begin{pgfscope}%
\pgfsetbuttcap%
\pgfsetroundjoin%
\definecolor{currentfill}{rgb}{0.000000,0.000000,0.000000}%
\pgfsetfillcolor{currentfill}%
\pgfsetlinewidth{0.602250pt}%
\definecolor{currentstroke}{rgb}{0.000000,0.000000,0.000000}%
\pgfsetstrokecolor{currentstroke}%
\pgfsetdash{}{0pt}%
\pgfsys@defobject{currentmarker}{\pgfqpoint{-0.027778in}{0.000000in}}{\pgfqpoint{-0.000000in}{0.000000in}}{%
\pgfpathmoveto{\pgfqpoint{-0.000000in}{0.000000in}}%
\pgfpathlineto{\pgfqpoint{-0.027778in}{0.000000in}}%
\pgfusepath{stroke,fill}%
}%
\begin{pgfscope}%
\pgfsys@transformshift{0.588387in}{1.138772in}%
\pgfsys@useobject{currentmarker}{}%
\end{pgfscope}%
\end{pgfscope}%
\begin{pgfscope}%
\pgfsetbuttcap%
\pgfsetroundjoin%
\definecolor{currentfill}{rgb}{0.000000,0.000000,0.000000}%
\pgfsetfillcolor{currentfill}%
\pgfsetlinewidth{0.602250pt}%
\definecolor{currentstroke}{rgb}{0.000000,0.000000,0.000000}%
\pgfsetstrokecolor{currentstroke}%
\pgfsetdash{}{0pt}%
\pgfsys@defobject{currentmarker}{\pgfqpoint{-0.027778in}{0.000000in}}{\pgfqpoint{-0.000000in}{0.000000in}}{%
\pgfpathmoveto{\pgfqpoint{-0.000000in}{0.000000in}}%
\pgfpathlineto{\pgfqpoint{-0.027778in}{0.000000in}}%
\pgfusepath{stroke,fill}%
}%
\begin{pgfscope}%
\pgfsys@transformshift{0.588387in}{1.277017in}%
\pgfsys@useobject{currentmarker}{}%
\end{pgfscope}%
\end{pgfscope}%
\begin{pgfscope}%
\pgfsetbuttcap%
\pgfsetroundjoin%
\definecolor{currentfill}{rgb}{0.000000,0.000000,0.000000}%
\pgfsetfillcolor{currentfill}%
\pgfsetlinewidth{0.602250pt}%
\definecolor{currentstroke}{rgb}{0.000000,0.000000,0.000000}%
\pgfsetstrokecolor{currentstroke}%
\pgfsetdash{}{0pt}%
\pgfsys@defobject{currentmarker}{\pgfqpoint{-0.027778in}{0.000000in}}{\pgfqpoint{-0.000000in}{0.000000in}}{%
\pgfpathmoveto{\pgfqpoint{-0.000000in}{0.000000in}}%
\pgfpathlineto{\pgfqpoint{-0.027778in}{0.000000in}}%
\pgfusepath{stroke,fill}%
}%
\begin{pgfscope}%
\pgfsys@transformshift{0.588387in}{1.375103in}%
\pgfsys@useobject{currentmarker}{}%
\end{pgfscope}%
\end{pgfscope}%
\begin{pgfscope}%
\pgfsetbuttcap%
\pgfsetroundjoin%
\definecolor{currentfill}{rgb}{0.000000,0.000000,0.000000}%
\pgfsetfillcolor{currentfill}%
\pgfsetlinewidth{0.602250pt}%
\definecolor{currentstroke}{rgb}{0.000000,0.000000,0.000000}%
\pgfsetstrokecolor{currentstroke}%
\pgfsetdash{}{0pt}%
\pgfsys@defobject{currentmarker}{\pgfqpoint{-0.027778in}{0.000000in}}{\pgfqpoint{-0.000000in}{0.000000in}}{%
\pgfpathmoveto{\pgfqpoint{-0.000000in}{0.000000in}}%
\pgfpathlineto{\pgfqpoint{-0.027778in}{0.000000in}}%
\pgfusepath{stroke,fill}%
}%
\begin{pgfscope}%
\pgfsys@transformshift{0.588387in}{1.451185in}%
\pgfsys@useobject{currentmarker}{}%
\end{pgfscope}%
\end{pgfscope}%
\begin{pgfscope}%
\pgfsetbuttcap%
\pgfsetroundjoin%
\definecolor{currentfill}{rgb}{0.000000,0.000000,0.000000}%
\pgfsetfillcolor{currentfill}%
\pgfsetlinewidth{0.602250pt}%
\definecolor{currentstroke}{rgb}{0.000000,0.000000,0.000000}%
\pgfsetstrokecolor{currentstroke}%
\pgfsetdash{}{0pt}%
\pgfsys@defobject{currentmarker}{\pgfqpoint{-0.027778in}{0.000000in}}{\pgfqpoint{-0.000000in}{0.000000in}}{%
\pgfpathmoveto{\pgfqpoint{-0.000000in}{0.000000in}}%
\pgfpathlineto{\pgfqpoint{-0.027778in}{0.000000in}}%
\pgfusepath{stroke,fill}%
}%
\begin{pgfscope}%
\pgfsys@transformshift{0.588387in}{1.513349in}%
\pgfsys@useobject{currentmarker}{}%
\end{pgfscope}%
\end{pgfscope}%
\begin{pgfscope}%
\pgfsetbuttcap%
\pgfsetroundjoin%
\definecolor{currentfill}{rgb}{0.000000,0.000000,0.000000}%
\pgfsetfillcolor{currentfill}%
\pgfsetlinewidth{0.602250pt}%
\definecolor{currentstroke}{rgb}{0.000000,0.000000,0.000000}%
\pgfsetstrokecolor{currentstroke}%
\pgfsetdash{}{0pt}%
\pgfsys@defobject{currentmarker}{\pgfqpoint{-0.027778in}{0.000000in}}{\pgfqpoint{-0.000000in}{0.000000in}}{%
\pgfpathmoveto{\pgfqpoint{-0.000000in}{0.000000in}}%
\pgfpathlineto{\pgfqpoint{-0.027778in}{0.000000in}}%
\pgfusepath{stroke,fill}%
}%
\begin{pgfscope}%
\pgfsys@transformshift{0.588387in}{1.565907in}%
\pgfsys@useobject{currentmarker}{}%
\end{pgfscope}%
\end{pgfscope}%
\begin{pgfscope}%
\pgfsetbuttcap%
\pgfsetroundjoin%
\definecolor{currentfill}{rgb}{0.000000,0.000000,0.000000}%
\pgfsetfillcolor{currentfill}%
\pgfsetlinewidth{0.602250pt}%
\definecolor{currentstroke}{rgb}{0.000000,0.000000,0.000000}%
\pgfsetstrokecolor{currentstroke}%
\pgfsetdash{}{0pt}%
\pgfsys@defobject{currentmarker}{\pgfqpoint{-0.027778in}{0.000000in}}{\pgfqpoint{-0.000000in}{0.000000in}}{%
\pgfpathmoveto{\pgfqpoint{-0.000000in}{0.000000in}}%
\pgfpathlineto{\pgfqpoint{-0.027778in}{0.000000in}}%
\pgfusepath{stroke,fill}%
}%
\begin{pgfscope}%
\pgfsys@transformshift{0.588387in}{1.611435in}%
\pgfsys@useobject{currentmarker}{}%
\end{pgfscope}%
\end{pgfscope}%
\begin{pgfscope}%
\pgfsetbuttcap%
\pgfsetroundjoin%
\definecolor{currentfill}{rgb}{0.000000,0.000000,0.000000}%
\pgfsetfillcolor{currentfill}%
\pgfsetlinewidth{0.602250pt}%
\definecolor{currentstroke}{rgb}{0.000000,0.000000,0.000000}%
\pgfsetstrokecolor{currentstroke}%
\pgfsetdash{}{0pt}%
\pgfsys@defobject{currentmarker}{\pgfqpoint{-0.027778in}{0.000000in}}{\pgfqpoint{-0.000000in}{0.000000in}}{%
\pgfpathmoveto{\pgfqpoint{-0.000000in}{0.000000in}}%
\pgfpathlineto{\pgfqpoint{-0.027778in}{0.000000in}}%
\pgfusepath{stroke,fill}%
}%
\begin{pgfscope}%
\pgfsys@transformshift{0.588387in}{1.651594in}%
\pgfsys@useobject{currentmarker}{}%
\end{pgfscope}%
\end{pgfscope}%
\begin{pgfscope}%
\pgfsetbuttcap%
\pgfsetroundjoin%
\definecolor{currentfill}{rgb}{0.000000,0.000000,0.000000}%
\pgfsetfillcolor{currentfill}%
\pgfsetlinewidth{0.602250pt}%
\definecolor{currentstroke}{rgb}{0.000000,0.000000,0.000000}%
\pgfsetstrokecolor{currentstroke}%
\pgfsetdash{}{0pt}%
\pgfsys@defobject{currentmarker}{\pgfqpoint{-0.027778in}{0.000000in}}{\pgfqpoint{-0.000000in}{0.000000in}}{%
\pgfpathmoveto{\pgfqpoint{-0.000000in}{0.000000in}}%
\pgfpathlineto{\pgfqpoint{-0.027778in}{0.000000in}}%
\pgfusepath{stroke,fill}%
}%
\begin{pgfscope}%
\pgfsys@transformshift{0.588387in}{1.923848in}%
\pgfsys@useobject{currentmarker}{}%
\end{pgfscope}%
\end{pgfscope}%
\begin{pgfscope}%
\pgfsetbuttcap%
\pgfsetroundjoin%
\definecolor{currentfill}{rgb}{0.000000,0.000000,0.000000}%
\pgfsetfillcolor{currentfill}%
\pgfsetlinewidth{0.602250pt}%
\definecolor{currentstroke}{rgb}{0.000000,0.000000,0.000000}%
\pgfsetstrokecolor{currentstroke}%
\pgfsetdash{}{0pt}%
\pgfsys@defobject{currentmarker}{\pgfqpoint{-0.027778in}{0.000000in}}{\pgfqpoint{-0.000000in}{0.000000in}}{%
\pgfpathmoveto{\pgfqpoint{-0.000000in}{0.000000in}}%
\pgfpathlineto{\pgfqpoint{-0.027778in}{0.000000in}}%
\pgfusepath{stroke,fill}%
}%
\begin{pgfscope}%
\pgfsys@transformshift{0.588387in}{2.062093in}%
\pgfsys@useobject{currentmarker}{}%
\end{pgfscope}%
\end{pgfscope}%
\begin{pgfscope}%
\pgfsetbuttcap%
\pgfsetroundjoin%
\definecolor{currentfill}{rgb}{0.000000,0.000000,0.000000}%
\pgfsetfillcolor{currentfill}%
\pgfsetlinewidth{0.602250pt}%
\definecolor{currentstroke}{rgb}{0.000000,0.000000,0.000000}%
\pgfsetstrokecolor{currentstroke}%
\pgfsetdash{}{0pt}%
\pgfsys@defobject{currentmarker}{\pgfqpoint{-0.027778in}{0.000000in}}{\pgfqpoint{-0.000000in}{0.000000in}}{%
\pgfpathmoveto{\pgfqpoint{-0.000000in}{0.000000in}}%
\pgfpathlineto{\pgfqpoint{-0.027778in}{0.000000in}}%
\pgfusepath{stroke,fill}%
}%
\begin{pgfscope}%
\pgfsys@transformshift{0.588387in}{2.160180in}%
\pgfsys@useobject{currentmarker}{}%
\end{pgfscope}%
\end{pgfscope}%
\begin{pgfscope}%
\pgfsetbuttcap%
\pgfsetroundjoin%
\definecolor{currentfill}{rgb}{0.000000,0.000000,0.000000}%
\pgfsetfillcolor{currentfill}%
\pgfsetlinewidth{0.602250pt}%
\definecolor{currentstroke}{rgb}{0.000000,0.000000,0.000000}%
\pgfsetstrokecolor{currentstroke}%
\pgfsetdash{}{0pt}%
\pgfsys@defobject{currentmarker}{\pgfqpoint{-0.027778in}{0.000000in}}{\pgfqpoint{-0.000000in}{0.000000in}}{%
\pgfpathmoveto{\pgfqpoint{-0.000000in}{0.000000in}}%
\pgfpathlineto{\pgfqpoint{-0.027778in}{0.000000in}}%
\pgfusepath{stroke,fill}%
}%
\begin{pgfscope}%
\pgfsys@transformshift{0.588387in}{2.236261in}%
\pgfsys@useobject{currentmarker}{}%
\end{pgfscope}%
\end{pgfscope}%
\begin{pgfscope}%
\pgfsetbuttcap%
\pgfsetroundjoin%
\definecolor{currentfill}{rgb}{0.000000,0.000000,0.000000}%
\pgfsetfillcolor{currentfill}%
\pgfsetlinewidth{0.602250pt}%
\definecolor{currentstroke}{rgb}{0.000000,0.000000,0.000000}%
\pgfsetstrokecolor{currentstroke}%
\pgfsetdash{}{0pt}%
\pgfsys@defobject{currentmarker}{\pgfqpoint{-0.027778in}{0.000000in}}{\pgfqpoint{-0.000000in}{0.000000in}}{%
\pgfpathmoveto{\pgfqpoint{-0.000000in}{0.000000in}}%
\pgfpathlineto{\pgfqpoint{-0.027778in}{0.000000in}}%
\pgfusepath{stroke,fill}%
}%
\begin{pgfscope}%
\pgfsys@transformshift{0.588387in}{2.298425in}%
\pgfsys@useobject{currentmarker}{}%
\end{pgfscope}%
\end{pgfscope}%
\begin{pgfscope}%
\pgfsetbuttcap%
\pgfsetroundjoin%
\definecolor{currentfill}{rgb}{0.000000,0.000000,0.000000}%
\pgfsetfillcolor{currentfill}%
\pgfsetlinewidth{0.602250pt}%
\definecolor{currentstroke}{rgb}{0.000000,0.000000,0.000000}%
\pgfsetstrokecolor{currentstroke}%
\pgfsetdash{}{0pt}%
\pgfsys@defobject{currentmarker}{\pgfqpoint{-0.027778in}{0.000000in}}{\pgfqpoint{-0.000000in}{0.000000in}}{%
\pgfpathmoveto{\pgfqpoint{-0.000000in}{0.000000in}}%
\pgfpathlineto{\pgfqpoint{-0.027778in}{0.000000in}}%
\pgfusepath{stroke,fill}%
}%
\begin{pgfscope}%
\pgfsys@transformshift{0.588387in}{2.350983in}%
\pgfsys@useobject{currentmarker}{}%
\end{pgfscope}%
\end{pgfscope}%
\begin{pgfscope}%
\pgfsetbuttcap%
\pgfsetroundjoin%
\definecolor{currentfill}{rgb}{0.000000,0.000000,0.000000}%
\pgfsetfillcolor{currentfill}%
\pgfsetlinewidth{0.602250pt}%
\definecolor{currentstroke}{rgb}{0.000000,0.000000,0.000000}%
\pgfsetstrokecolor{currentstroke}%
\pgfsetdash{}{0pt}%
\pgfsys@defobject{currentmarker}{\pgfqpoint{-0.027778in}{0.000000in}}{\pgfqpoint{-0.000000in}{0.000000in}}{%
\pgfpathmoveto{\pgfqpoint{-0.000000in}{0.000000in}}%
\pgfpathlineto{\pgfqpoint{-0.027778in}{0.000000in}}%
\pgfusepath{stroke,fill}%
}%
\begin{pgfscope}%
\pgfsys@transformshift{0.588387in}{2.396511in}%
\pgfsys@useobject{currentmarker}{}%
\end{pgfscope}%
\end{pgfscope}%
\begin{pgfscope}%
\pgfsetbuttcap%
\pgfsetroundjoin%
\definecolor{currentfill}{rgb}{0.000000,0.000000,0.000000}%
\pgfsetfillcolor{currentfill}%
\pgfsetlinewidth{0.602250pt}%
\definecolor{currentstroke}{rgb}{0.000000,0.000000,0.000000}%
\pgfsetstrokecolor{currentstroke}%
\pgfsetdash{}{0pt}%
\pgfsys@defobject{currentmarker}{\pgfqpoint{-0.027778in}{0.000000in}}{\pgfqpoint{-0.000000in}{0.000000in}}{%
\pgfpathmoveto{\pgfqpoint{-0.000000in}{0.000000in}}%
\pgfpathlineto{\pgfqpoint{-0.027778in}{0.000000in}}%
\pgfusepath{stroke,fill}%
}%
\begin{pgfscope}%
\pgfsys@transformshift{0.588387in}{2.436670in}%
\pgfsys@useobject{currentmarker}{}%
\end{pgfscope}%
\end{pgfscope}%
\begin{pgfscope}%
\pgfsetbuttcap%
\pgfsetroundjoin%
\definecolor{currentfill}{rgb}{0.000000,0.000000,0.000000}%
\pgfsetfillcolor{currentfill}%
\pgfsetlinewidth{0.602250pt}%
\definecolor{currentstroke}{rgb}{0.000000,0.000000,0.000000}%
\pgfsetstrokecolor{currentstroke}%
\pgfsetdash{}{0pt}%
\pgfsys@defobject{currentmarker}{\pgfqpoint{-0.027778in}{0.000000in}}{\pgfqpoint{-0.000000in}{0.000000in}}{%
\pgfpathmoveto{\pgfqpoint{-0.000000in}{0.000000in}}%
\pgfpathlineto{\pgfqpoint{-0.027778in}{0.000000in}}%
\pgfusepath{stroke,fill}%
}%
\begin{pgfscope}%
\pgfsys@transformshift{0.588387in}{2.708924in}%
\pgfsys@useobject{currentmarker}{}%
\end{pgfscope}%
\end{pgfscope}%
\begin{pgfscope}%
\definecolor{textcolor}{rgb}{0.000000,0.000000,0.000000}%
\pgfsetstrokecolor{textcolor}%
\pgfsetfillcolor{textcolor}%
\pgftext[x=0.234413in,y=1.631726in,,bottom,rotate=90.000000]{\color{textcolor}{\rmfamily\fontsize{10.000000}{12.000000}\selectfont\catcode`\^=\active\def^{\ifmmode\sp\else\^{}\fi}\catcode`\%=\active\def%{\%}Checks [call]}}%
\end{pgfscope}%
\begin{pgfscope}%
\pgfpathrectangle{\pgfqpoint{0.588387in}{0.521603in}}{\pgfqpoint{6.906617in}{2.220246in}}%
\pgfusepath{clip}%
\pgfsetrectcap%
\pgfsetroundjoin%
\pgfsetlinewidth{1.505625pt}%
\pgfsetstrokecolor{currentstroke1}%
\pgfsetdash{}{0pt}%
\pgfpathmoveto{\pgfqpoint{0.902324in}{0.622524in}}%
\pgfpathlineto{\pgfqpoint{0.996037in}{0.704749in}}%
\pgfpathlineto{\pgfqpoint{1.089750in}{0.681808in}}%
\pgfpathlineto{\pgfqpoint{1.183462in}{0.706353in}}%
\pgfpathlineto{\pgfqpoint{1.277175in}{0.771380in}}%
\pgfpathlineto{\pgfqpoint{1.370887in}{0.750277in}}%
\pgfpathlineto{\pgfqpoint{1.464600in}{0.771746in}}%
\pgfpathlineto{\pgfqpoint{1.558312in}{0.827894in}}%
\pgfpathlineto{\pgfqpoint{1.652025in}{0.810613in}}%
\pgfpathlineto{\pgfqpoint{1.745738in}{0.827679in}}%
\pgfpathlineto{\pgfqpoint{1.839450in}{0.879801in}}%
\pgfpathlineto{\pgfqpoint{1.933163in}{0.864169in}}%
\pgfpathlineto{\pgfqpoint{2.026875in}{0.876811in}}%
\pgfpathlineto{\pgfqpoint{2.120588in}{0.915813in}}%
\pgfpathlineto{\pgfqpoint{2.214301in}{0.908714in}}%
\pgfpathlineto{\pgfqpoint{2.308013in}{0.918835in}}%
\pgfpathlineto{\pgfqpoint{2.401726in}{0.958703in}}%
\pgfpathlineto{\pgfqpoint{2.495438in}{0.942094in}}%
\pgfpathlineto{\pgfqpoint{2.589151in}{0.955486in}}%
\pgfpathlineto{\pgfqpoint{2.682863in}{0.990675in}}%
\pgfpathlineto{\pgfqpoint{2.776576in}{0.998974in}}%
\pgfpathlineto{\pgfqpoint{2.870289in}{0.994383in}}%
\pgfpathlineto{\pgfqpoint{2.964001in}{1.026494in}}%
\pgfpathlineto{\pgfqpoint{3.057714in}{1.018267in}}%
\pgfpathlineto{\pgfqpoint{3.151426in}{1.036109in}}%
\pgfpathlineto{\pgfqpoint{3.245139in}{1.055421in}}%
\pgfpathlineto{\pgfqpoint{3.338852in}{1.040552in}}%
\pgfpathlineto{\pgfqpoint{3.432564in}{1.069667in}}%
\pgfpathlineto{\pgfqpoint{3.526277in}{1.087843in}}%
\pgfpathlineto{\pgfqpoint{3.619989in}{1.062690in}}%
\pgfpathlineto{\pgfqpoint{3.713702in}{1.084790in}}%
\pgfpathlineto{\pgfqpoint{3.807414in}{1.136205in}}%
\pgfpathlineto{\pgfqpoint{3.901127in}{1.138772in}}%
\pgfpathlineto{\pgfqpoint{3.994840in}{1.119664in}}%
\pgfpathlineto{\pgfqpoint{4.088552in}{1.144408in}}%
\pgfpathlineto{\pgfqpoint{4.182265in}{1.127605in}}%
\pgfpathlineto{\pgfqpoint{4.275977in}{1.117675in}}%
\pgfpathlineto{\pgfqpoint{4.369690in}{1.182448in}}%
\pgfpathlineto{\pgfqpoint{4.463403in}{1.172429in}}%
\pgfpathlineto{\pgfqpoint{4.557115in}{1.193757in}}%
\pgfpathlineto{\pgfqpoint{4.650828in}{1.158639in}}%
\pgfpathlineto{\pgfqpoint{4.744540in}{1.354007in}}%
\pgfpathlineto{\pgfqpoint{4.838253in}{1.176192in}}%
\pgfpathlineto{\pgfqpoint{4.931965in}{1.207965in}}%
\pgfpathlineto{\pgfqpoint{5.025678in}{1.171268in}}%
\pgfpathlineto{\pgfqpoint{5.119391in}{1.184939in}}%
\pgfpathlineto{\pgfqpoint{5.306816in}{1.303257in}}%
\pgfpathlineto{\pgfqpoint{5.400528in}{1.200935in}}%
\pgfpathlineto{\pgfqpoint{5.494241in}{1.274736in}}%
\pgfpathlineto{\pgfqpoint{5.681666in}{1.331518in}}%
\pgfpathlineto{\pgfqpoint{5.775379in}{1.238559in}}%
\pgfpathlineto{\pgfqpoint{5.869091in}{1.236645in}}%
\pgfpathlineto{\pgfqpoint{5.962804in}{1.356716in}}%
\pgfpathlineto{\pgfqpoint{6.056517in}{1.241725in}}%
\pgfpathlineto{\pgfqpoint{6.243942in}{1.288746in}}%
\pgfpathlineto{\pgfqpoint{6.431367in}{1.342010in}}%
\pgfpathlineto{\pgfqpoint{6.712505in}{1.320693in}}%
\pgfpathlineto{\pgfqpoint{7.181068in}{1.346674in}}%
\pgfusepath{stroke}%
\end{pgfscope}%
\begin{pgfscope}%
\pgfpathrectangle{\pgfqpoint{0.588387in}{0.521603in}}{\pgfqpoint{6.906617in}{2.220246in}}%
\pgfusepath{clip}%
\pgfsetrectcap%
\pgfsetroundjoin%
\pgfsetlinewidth{1.505625pt}%
\pgfsetstrokecolor{currentstroke2}%
\pgfsetdash{}{0pt}%
\pgfpathmoveto{\pgfqpoint{0.902324in}{0.748356in}}%
\pgfpathlineto{\pgfqpoint{0.996037in}{0.858402in}}%
\pgfpathlineto{\pgfqpoint{1.089750in}{0.942538in}}%
\pgfpathlineto{\pgfqpoint{1.183462in}{0.828300in}}%
\pgfpathlineto{\pgfqpoint{1.277175in}{0.859279in}}%
\pgfpathlineto{\pgfqpoint{1.370887in}{0.941163in}}%
\pgfpathlineto{\pgfqpoint{1.464600in}{1.008463in}}%
\pgfpathlineto{\pgfqpoint{1.558312in}{0.917843in}}%
\pgfpathlineto{\pgfqpoint{1.652025in}{0.942038in}}%
\pgfpathlineto{\pgfqpoint{1.745738in}{1.006702in}}%
\pgfpathlineto{\pgfqpoint{1.839450in}{1.068999in}}%
\pgfpathlineto{\pgfqpoint{1.933163in}{0.993368in}}%
\pgfpathlineto{\pgfqpoint{2.026875in}{1.009361in}}%
\pgfpathlineto{\pgfqpoint{2.120588in}{1.061937in}}%
\pgfpathlineto{\pgfqpoint{2.214301in}{1.112322in}}%
\pgfpathlineto{\pgfqpoint{2.308013in}{1.047190in}}%
\pgfpathlineto{\pgfqpoint{2.401726in}{1.064640in}}%
\pgfpathlineto{\pgfqpoint{2.495438in}{1.111914in}}%
\pgfpathlineto{\pgfqpoint{2.589151in}{1.153399in}}%
\pgfpathlineto{\pgfqpoint{2.682863in}{1.100840in}}%
\pgfpathlineto{\pgfqpoint{2.776576in}{1.121870in}}%
\pgfpathlineto{\pgfqpoint{2.870289in}{1.160655in}}%
\pgfpathlineto{\pgfqpoint{2.964001in}{1.196115in}}%
\pgfpathlineto{\pgfqpoint{3.057714in}{1.148067in}}%
\pgfpathlineto{\pgfqpoint{3.151426in}{1.170710in}}%
\pgfpathlineto{\pgfqpoint{3.245139in}{1.195060in}}%
\pgfpathlineto{\pgfqpoint{3.338852in}{1.227330in}}%
\pgfpathlineto{\pgfqpoint{3.432564in}{1.208201in}}%
\pgfpathlineto{\pgfqpoint{3.526277in}{1.205732in}}%
\pgfpathlineto{\pgfqpoint{3.619989in}{1.222407in}}%
\pgfpathlineto{\pgfqpoint{3.713702in}{1.268601in}}%
\pgfpathlineto{\pgfqpoint{3.807414in}{1.258810in}}%
\pgfpathlineto{\pgfqpoint{3.901127in}{1.273361in}}%
\pgfpathlineto{\pgfqpoint{3.994840in}{1.284036in}}%
\pgfpathlineto{\pgfqpoint{4.088552in}{1.307441in}}%
\pgfpathlineto{\pgfqpoint{4.182265in}{1.258970in}}%
\pgfpathlineto{\pgfqpoint{4.275977in}{1.251540in}}%
\pgfpathlineto{\pgfqpoint{4.369690in}{1.326837in}}%
\pgfpathlineto{\pgfqpoint{4.463403in}{1.345188in}}%
\pgfpathlineto{\pgfqpoint{4.557115in}{1.320493in}}%
\pgfpathlineto{\pgfqpoint{4.650828in}{1.278831in}}%
\pgfpathlineto{\pgfqpoint{4.744540in}{1.406358in}}%
\pgfpathlineto{\pgfqpoint{4.838253in}{1.344219in}}%
\pgfpathlineto{\pgfqpoint{4.931965in}{1.337915in}}%
\pgfpathlineto{\pgfqpoint{5.025678in}{1.305773in}}%
\pgfpathlineto{\pgfqpoint{5.119391in}{1.351823in}}%
\pgfpathlineto{\pgfqpoint{5.306816in}{1.493218in}}%
\pgfpathlineto{\pgfqpoint{5.400528in}{1.329965in}}%
\pgfpathlineto{\pgfqpoint{5.494241in}{1.451730in}}%
\pgfpathlineto{\pgfqpoint{5.681666in}{1.437976in}}%
\pgfpathlineto{\pgfqpoint{5.775379in}{1.351093in}}%
\pgfpathlineto{\pgfqpoint{5.869091in}{1.389131in}}%
\pgfpathlineto{\pgfqpoint{5.962804in}{1.573612in}}%
\pgfpathlineto{\pgfqpoint{6.056517in}{1.362956in}}%
\pgfpathlineto{\pgfqpoint{6.243942in}{1.422459in}}%
\pgfpathlineto{\pgfqpoint{6.431367in}{1.424530in}}%
\pgfpathlineto{\pgfqpoint{6.712505in}{1.444019in}}%
\pgfpathlineto{\pgfqpoint{7.181068in}{1.492252in}}%
\pgfusepath{stroke}%
\end{pgfscope}%
\begin{pgfscope}%
\pgfpathrectangle{\pgfqpoint{0.588387in}{0.521603in}}{\pgfqpoint{6.906617in}{2.220246in}}%
\pgfusepath{clip}%
\pgfsetrectcap%
\pgfsetroundjoin%
\pgfsetlinewidth{1.505625pt}%
\pgfsetstrokecolor{currentstroke3}%
\pgfsetdash{}{0pt}%
\pgfpathmoveto{\pgfqpoint{0.902324in}{1.058532in}}%
\pgfpathlineto{\pgfqpoint{0.996037in}{1.221180in}}%
\pgfpathlineto{\pgfqpoint{1.089750in}{0.942544in}}%
\pgfpathlineto{\pgfqpoint{1.183462in}{0.988096in}}%
\pgfpathlineto{\pgfqpoint{1.277175in}{1.094919in}}%
\pgfpathlineto{\pgfqpoint{1.370887in}{1.176824in}}%
\pgfpathlineto{\pgfqpoint{1.464600in}{1.224549in}}%
\pgfpathlineto{\pgfqpoint{1.558312in}{1.064346in}}%
\pgfpathlineto{\pgfqpoint{1.652025in}{1.096785in}}%
\pgfpathlineto{\pgfqpoint{1.745738in}{1.176812in}}%
\pgfpathlineto{\pgfqpoint{1.839450in}{1.248504in}}%
\pgfpathlineto{\pgfqpoint{1.933163in}{1.302778in}}%
\pgfpathlineto{\pgfqpoint{2.026875in}{1.153919in}}%
\pgfpathlineto{\pgfqpoint{2.120588in}{1.176096in}}%
\pgfpathlineto{\pgfqpoint{2.214301in}{1.244701in}}%
\pgfpathlineto{\pgfqpoint{2.308013in}{1.298666in}}%
\pgfpathlineto{\pgfqpoint{2.401726in}{1.347260in}}%
\pgfpathlineto{\pgfqpoint{2.495438in}{1.222490in}}%
\pgfpathlineto{\pgfqpoint{2.589151in}{1.245770in}}%
\pgfpathlineto{\pgfqpoint{2.682863in}{1.300562in}}%
\pgfpathlineto{\pgfqpoint{2.776576in}{1.367780in}}%
\pgfpathlineto{\pgfqpoint{2.870289in}{1.397560in}}%
\pgfpathlineto{\pgfqpoint{2.964001in}{1.294698in}}%
\pgfpathlineto{\pgfqpoint{3.057714in}{1.304308in}}%
\pgfpathlineto{\pgfqpoint{3.151426in}{1.367372in}}%
\pgfpathlineto{\pgfqpoint{3.245139in}{1.396764in}}%
\pgfpathlineto{\pgfqpoint{3.338852in}{1.432716in}}%
\pgfpathlineto{\pgfqpoint{3.432564in}{1.359358in}}%
\pgfpathlineto{\pgfqpoint{3.526277in}{1.366719in}}%
\pgfpathlineto{\pgfqpoint{3.619989in}{1.385843in}}%
\pgfpathlineto{\pgfqpoint{3.713702in}{1.454612in}}%
\pgfpathlineto{\pgfqpoint{3.807414in}{1.501260in}}%
\pgfpathlineto{\pgfqpoint{3.901127in}{1.452773in}}%
\pgfpathlineto{\pgfqpoint{3.994840in}{1.424916in}}%
\pgfpathlineto{\pgfqpoint{4.088552in}{1.479314in}}%
\pgfpathlineto{\pgfqpoint{4.182265in}{1.461324in}}%
\pgfpathlineto{\pgfqpoint{4.275977in}{1.484919in}}%
\pgfpathlineto{\pgfqpoint{4.369690in}{1.457172in}}%
\pgfpathlineto{\pgfqpoint{4.463403in}{1.464230in}}%
\pgfpathlineto{\pgfqpoint{4.557115in}{1.502963in}}%
\pgfpathlineto{\pgfqpoint{4.650828in}{1.484919in}}%
\pgfpathlineto{\pgfqpoint{4.744540in}{1.690233in}}%
\pgfpathlineto{\pgfqpoint{4.838253in}{1.459737in}}%
\pgfpathlineto{\pgfqpoint{4.931965in}{1.477846in}}%
\pgfpathlineto{\pgfqpoint{5.025678in}{1.477425in}}%
\pgfpathlineto{\pgfqpoint{5.119391in}{1.527403in}}%
\pgfpathlineto{\pgfqpoint{5.306816in}{1.662682in}}%
\pgfpathlineto{\pgfqpoint{5.400528in}{1.513987in}}%
\pgfpathlineto{\pgfqpoint{5.494241in}{1.610688in}}%
\pgfpathlineto{\pgfqpoint{5.681666in}{1.640818in}}%
\pgfpathlineto{\pgfqpoint{5.775379in}{1.499430in}}%
\pgfpathlineto{\pgfqpoint{5.869091in}{1.518145in}}%
\pgfpathlineto{\pgfqpoint{5.962804in}{1.708026in}}%
\pgfpathlineto{\pgfqpoint{6.056517in}{1.561001in}}%
\pgfpathlineto{\pgfqpoint{6.243942in}{1.561495in}}%
\pgfpathlineto{\pgfqpoint{6.431367in}{1.566394in}}%
\pgfpathlineto{\pgfqpoint{6.712505in}{1.576458in}}%
\pgfpathlineto{\pgfqpoint{7.181068in}{1.627054in}}%
\pgfusepath{stroke}%
\end{pgfscope}%
\begin{pgfscope}%
\pgfpathrectangle{\pgfqpoint{0.588387in}{0.521603in}}{\pgfqpoint{6.906617in}{2.220246in}}%
\pgfusepath{clip}%
\pgfsetrectcap%
\pgfsetroundjoin%
\pgfsetlinewidth{1.505625pt}%
\pgfsetstrokecolor{currentstroke4}%
\pgfsetdash{}{0pt}%
\pgfpathmoveto{\pgfqpoint{0.902324in}{1.058771in}}%
\pgfpathlineto{\pgfqpoint{0.996037in}{1.221462in}}%
\pgfpathlineto{\pgfqpoint{1.089750in}{1.299325in}}%
\pgfpathlineto{\pgfqpoint{1.183462in}{1.459866in}}%
\pgfpathlineto{\pgfqpoint{1.277175in}{1.533223in}}%
\pgfpathlineto{\pgfqpoint{1.370887in}{1.177189in}}%
\pgfpathlineto{\pgfqpoint{1.464600in}{1.223652in}}%
\pgfpathlineto{\pgfqpoint{1.558312in}{1.331333in}}%
\pgfpathlineto{\pgfqpoint{1.652025in}{1.412794in}}%
\pgfpathlineto{\pgfqpoint{1.745738in}{1.457470in}}%
\pgfpathlineto{\pgfqpoint{1.839450in}{1.570540in}}%
\pgfpathlineto{\pgfqpoint{1.933163in}{1.300935in}}%
\pgfpathlineto{\pgfqpoint{2.026875in}{1.331476in}}%
\pgfpathlineto{\pgfqpoint{2.120588in}{1.412045in}}%
\pgfpathlineto{\pgfqpoint{2.214301in}{1.481665in}}%
\pgfpathlineto{\pgfqpoint{2.308013in}{1.536440in}}%
\pgfpathlineto{\pgfqpoint{2.401726in}{1.569076in}}%
\pgfpathlineto{\pgfqpoint{2.495438in}{1.388813in}}%
\pgfpathlineto{\pgfqpoint{2.589151in}{1.415262in}}%
\pgfpathlineto{\pgfqpoint{2.682863in}{1.481252in}}%
\pgfpathlineto{\pgfqpoint{2.776576in}{1.546619in}}%
\pgfpathlineto{\pgfqpoint{2.870289in}{1.586711in}}%
\pgfpathlineto{\pgfqpoint{2.964001in}{1.630896in}}%
\pgfpathlineto{\pgfqpoint{3.057714in}{1.460902in}}%
\pgfpathlineto{\pgfqpoint{3.151426in}{1.494759in}}%
\pgfpathlineto{\pgfqpoint{3.245139in}{1.543037in}}%
\pgfpathlineto{\pgfqpoint{3.338852in}{1.593529in}}%
\pgfpathlineto{\pgfqpoint{3.432564in}{1.645471in}}%
\pgfpathlineto{\pgfqpoint{3.526277in}{1.672213in}}%
\pgfpathlineto{\pgfqpoint{3.619989in}{1.517407in}}%
\pgfpathlineto{\pgfqpoint{3.713702in}{1.557813in}}%
\pgfpathlineto{\pgfqpoint{3.807414in}{1.624611in}}%
\pgfpathlineto{\pgfqpoint{3.901127in}{1.660006in}}%
\pgfpathlineto{\pgfqpoint{3.994840in}{1.683536in}}%
\pgfpathlineto{\pgfqpoint{4.088552in}{1.716398in}}%
\pgfpathlineto{\pgfqpoint{4.182265in}{1.578467in}}%
\pgfpathlineto{\pgfqpoint{4.275977in}{1.603049in}}%
\pgfpathlineto{\pgfqpoint{4.369690in}{1.664331in}}%
\pgfpathlineto{\pgfqpoint{4.463403in}{1.690217in}}%
\pgfpathlineto{\pgfqpoint{4.557115in}{1.726461in}}%
\pgfpathlineto{\pgfqpoint{4.650828in}{1.726461in}}%
\pgfpathlineto{\pgfqpoint{4.744540in}{1.702883in}}%
\pgfpathlineto{\pgfqpoint{4.838253in}{1.645257in}}%
\pgfpathlineto{\pgfqpoint{4.931965in}{1.684572in}}%
\pgfpathlineto{\pgfqpoint{5.025678in}{1.712311in}}%
\pgfpathlineto{\pgfqpoint{5.119391in}{1.746046in}}%
\pgfpathlineto{\pgfqpoint{5.213103in}{1.764144in}}%
\pgfpathlineto{\pgfqpoint{5.306816in}{1.840940in}}%
\pgfpathlineto{\pgfqpoint{5.400528in}{1.684916in}}%
\pgfpathlineto{\pgfqpoint{5.494241in}{1.785518in}}%
\pgfpathlineto{\pgfqpoint{5.587954in}{1.726157in}}%
\pgfpathlineto{\pgfqpoint{5.681666in}{1.811938in}}%
\pgfpathlineto{\pgfqpoint{5.775379in}{1.773147in}}%
\pgfpathlineto{\pgfqpoint{5.869091in}{1.700014in}}%
\pgfpathlineto{\pgfqpoint{5.962804in}{1.855436in}}%
\pgfpathlineto{\pgfqpoint{6.056517in}{1.720323in}}%
\pgfpathlineto{\pgfqpoint{6.150229in}{1.771685in}}%
\pgfpathlineto{\pgfqpoint{6.243942in}{1.801193in}}%
\pgfpathlineto{\pgfqpoint{6.431367in}{1.740815in}}%
\pgfpathlineto{\pgfqpoint{6.618792in}{1.809944in}}%
\pgfpathlineto{\pgfqpoint{6.712505in}{1.828629in}}%
\pgfpathlineto{\pgfqpoint{6.899930in}{2.307840in}}%
\pgfpathlineto{\pgfqpoint{7.181068in}{1.852370in}}%
\pgfusepath{stroke}%
\end{pgfscope}%
\begin{pgfscope}%
\pgfpathrectangle{\pgfqpoint{0.588387in}{0.521603in}}{\pgfqpoint{6.906617in}{2.220246in}}%
\pgfusepath{clip}%
\pgfsetrectcap%
\pgfsetroundjoin%
\pgfsetlinewidth{1.505625pt}%
\pgfsetstrokecolor{currentstroke5}%
\pgfsetdash{}{0pt}%
\pgfpathmoveto{\pgfqpoint{0.902324in}{2.168266in}}%
\pgfpathlineto{\pgfqpoint{0.996037in}{1.221167in}}%
\pgfpathlineto{\pgfqpoint{1.089750in}{1.300138in}}%
\pgfpathlineto{\pgfqpoint{1.183462in}{1.460530in}}%
\pgfpathlineto{\pgfqpoint{1.277175in}{1.532900in}}%
\pgfpathlineto{\pgfqpoint{1.370887in}{1.693223in}}%
\pgfpathlineto{\pgfqpoint{1.464600in}{1.771847in}}%
\pgfpathlineto{\pgfqpoint{1.558312in}{1.929236in}}%
\pgfpathlineto{\pgfqpoint{1.652025in}{1.413944in}}%
\pgfpathlineto{\pgfqpoint{1.745738in}{1.456706in}}%
\pgfpathlineto{\pgfqpoint{1.839450in}{1.570894in}}%
\pgfpathlineto{\pgfqpoint{1.933163in}{1.647162in}}%
\pgfpathlineto{\pgfqpoint{2.026875in}{1.692118in}}%
\pgfpathlineto{\pgfqpoint{2.120588in}{1.795921in}}%
\pgfpathlineto{\pgfqpoint{2.214301in}{1.885870in}}%
\pgfpathlineto{\pgfqpoint{2.308013in}{1.534953in}}%
\pgfpathlineto{\pgfqpoint{2.401726in}{1.569424in}}%
\pgfpathlineto{\pgfqpoint{2.495438in}{1.645805in}}%
\pgfpathlineto{\pgfqpoint{2.589151in}{1.712695in}}%
\pgfpathlineto{\pgfqpoint{2.682863in}{1.769245in}}%
\pgfpathlineto{\pgfqpoint{2.776576in}{1.813650in}}%
\pgfpathlineto{\pgfqpoint{2.870289in}{1.891347in}}%
\pgfpathlineto{\pgfqpoint{2.964001in}{1.640990in}}%
\pgfpathlineto{\pgfqpoint{3.057714in}{1.650822in}}%
\pgfpathlineto{\pgfqpoint{3.151426in}{1.740405in}}%
\pgfpathlineto{\pgfqpoint{3.245139in}{1.778752in}}%
\pgfpathlineto{\pgfqpoint{3.338852in}{1.825471in}}%
\pgfpathlineto{\pgfqpoint{3.432564in}{1.891904in}}%
\pgfpathlineto{\pgfqpoint{3.526277in}{1.888054in}}%
\pgfpathlineto{\pgfqpoint{3.619989in}{1.689577in}}%
\pgfpathlineto{\pgfqpoint{3.713702in}{1.737312in}}%
\pgfpathlineto{\pgfqpoint{3.807414in}{1.815845in}}%
\pgfpathlineto{\pgfqpoint{3.901127in}{1.950888in}}%
\pgfpathlineto{\pgfqpoint{3.994840in}{1.896435in}}%
\pgfpathlineto{\pgfqpoint{4.088552in}{1.925124in}}%
\pgfpathlineto{\pgfqpoint{4.182265in}{1.922073in}}%
\pgfpathlineto{\pgfqpoint{4.275977in}{1.740659in}}%
\pgfpathlineto{\pgfqpoint{4.369690in}{1.796059in}}%
\pgfpathlineto{\pgfqpoint{4.463403in}{1.852469in}}%
\pgfpathlineto{\pgfqpoint{4.557115in}{1.897193in}}%
\pgfpathlineto{\pgfqpoint{4.650828in}{1.892067in}}%
\pgfpathlineto{\pgfqpoint{4.744540in}{2.056016in}}%
\pgfpathlineto{\pgfqpoint{4.838253in}{1.970679in}}%
\pgfpathlineto{\pgfqpoint{4.931965in}{1.819183in}}%
\pgfpathlineto{\pgfqpoint{5.025678in}{1.803211in}}%
\pgfpathlineto{\pgfqpoint{5.119391in}{1.869686in}}%
\pgfpathlineto{\pgfqpoint{5.306816in}{2.057747in}}%
\pgfpathlineto{\pgfqpoint{5.400528in}{1.962184in}}%
\pgfpathlineto{\pgfqpoint{5.494241in}{2.050534in}}%
\pgfpathlineto{\pgfqpoint{5.681666in}{1.931101in}}%
\pgfpathlineto{\pgfqpoint{5.775379in}{1.892441in}}%
\pgfpathlineto{\pgfqpoint{5.869091in}{1.914282in}}%
\pgfpathlineto{\pgfqpoint{5.962804in}{2.094176in}}%
\pgfpathlineto{\pgfqpoint{6.056517in}{1.976212in}}%
\pgfpathlineto{\pgfqpoint{6.243942in}{1.872248in}}%
\pgfpathlineto{\pgfqpoint{6.431367in}{1.919387in}}%
\pgfpathlineto{\pgfqpoint{6.712505in}{1.969717in}}%
\pgfpathlineto{\pgfqpoint{7.181068in}{2.025538in}}%
\pgfusepath{stroke}%
\end{pgfscope}%
\begin{pgfscope}%
\pgfpathrectangle{\pgfqpoint{0.588387in}{0.521603in}}{\pgfqpoint{6.906617in}{2.220246in}}%
\pgfusepath{clip}%
\pgfsetrectcap%
\pgfsetroundjoin%
\pgfsetlinewidth{1.505625pt}%
\pgfsetstrokecolor{currentstroke6}%
\pgfsetdash{}{0pt}%
\pgfpathmoveto{\pgfqpoint{0.902324in}{2.168266in}}%
\pgfpathlineto{\pgfqpoint{0.996037in}{2.404597in}}%
\pgfpathlineto{\pgfqpoint{1.089750in}{2.640929in}}%
\pgfpathlineto{\pgfqpoint{1.183462in}{1.460435in}}%
\pgfpathlineto{\pgfqpoint{1.277175in}{1.532176in}}%
\pgfpathlineto{\pgfqpoint{1.370887in}{1.692929in}}%
\pgfpathlineto{\pgfqpoint{1.464600in}{1.770711in}}%
\pgfpathlineto{\pgfqpoint{1.558312in}{1.928723in}}%
\pgfpathlineto{\pgfqpoint{1.652025in}{2.005327in}}%
\pgfpathlineto{\pgfqpoint{1.745738in}{2.159391in}}%
\pgfpathlineto{\pgfqpoint{1.839450in}{2.249634in}}%
\pgfpathlineto{\pgfqpoint{1.933163in}{1.646936in}}%
\pgfpathlineto{\pgfqpoint{2.026875in}{1.690572in}}%
\pgfpathlineto{\pgfqpoint{2.120588in}{1.797453in}}%
\pgfpathlineto{\pgfqpoint{2.214301in}{1.886616in}}%
\pgfpathlineto{\pgfqpoint{2.308013in}{1.930409in}}%
\pgfpathlineto{\pgfqpoint{2.401726in}{2.037882in}}%
\pgfpathlineto{\pgfqpoint{2.495438in}{2.118261in}}%
\pgfpathlineto{\pgfqpoint{2.589151in}{2.160834in}}%
\pgfpathlineto{\pgfqpoint{2.682863in}{1.769761in}}%
\pgfpathlineto{\pgfqpoint{2.776576in}{1.807511in}}%
\pgfpathlineto{\pgfqpoint{2.870289in}{1.888991in}}%
\pgfpathlineto{\pgfqpoint{2.964001in}{1.959147in}}%
\pgfpathlineto{\pgfqpoint{3.057714in}{2.005387in}}%
\pgfpathlineto{\pgfqpoint{3.151426in}{2.055715in}}%
\pgfpathlineto{\pgfqpoint{3.245139in}{2.120495in}}%
\pgfpathlineto{\pgfqpoint{3.338852in}{2.185049in}}%
\pgfpathlineto{\pgfqpoint{3.432564in}{1.891981in}}%
\pgfpathlineto{\pgfqpoint{3.526277in}{1.892441in}}%
\pgfpathlineto{\pgfqpoint{3.619989in}{1.950088in}}%
\pgfpathlineto{\pgfqpoint{3.713702in}{2.025809in}}%
\pgfpathlineto{\pgfqpoint{3.807414in}{2.110042in}}%
\pgfpathlineto{\pgfqpoint{3.901127in}{2.161088in}}%
\pgfpathlineto{\pgfqpoint{3.994840in}{2.174371in}}%
\pgfpathlineto{\pgfqpoint{4.088552in}{2.203255in}}%
\pgfpathlineto{\pgfqpoint{4.182265in}{1.949017in}}%
\pgfpathlineto{\pgfqpoint{4.275977in}{1.940700in}}%
\pgfpathlineto{\pgfqpoint{4.369690in}{2.056132in}}%
\pgfpathlineto{\pgfqpoint{4.463403in}{2.088754in}}%
\pgfpathlineto{\pgfqpoint{4.557115in}{2.187680in}}%
\pgfpathlineto{\pgfqpoint{4.650828in}{2.116594in}}%
\pgfpathlineto{\pgfqpoint{4.744540in}{2.315276in}}%
\pgfpathlineto{\pgfqpoint{4.838253in}{2.199063in}}%
\pgfpathlineto{\pgfqpoint{4.931965in}{2.012953in}}%
\pgfpathlineto{\pgfqpoint{5.025678in}{1.980281in}}%
\pgfpathlineto{\pgfqpoint{5.119391in}{2.074974in}}%
\pgfpathlineto{\pgfqpoint{5.306816in}{2.399228in}}%
\pgfpathlineto{\pgfqpoint{5.400528in}{2.146261in}}%
\pgfpathlineto{\pgfqpoint{5.494241in}{2.328849in}}%
\pgfpathlineto{\pgfqpoint{5.681666in}{2.090432in}}%
\pgfpathlineto{\pgfqpoint{5.775379in}{2.048175in}}%
\pgfpathlineto{\pgfqpoint{5.869091in}{2.086646in}}%
\pgfpathlineto{\pgfqpoint{5.962804in}{2.396213in}}%
\pgfpathlineto{\pgfqpoint{6.056517in}{2.157441in}}%
\pgfpathlineto{\pgfqpoint{6.243942in}{2.235511in}}%
\pgfpathlineto{\pgfqpoint{6.431367in}{2.111913in}}%
\pgfpathlineto{\pgfqpoint{6.712505in}{2.168266in}}%
\pgfpathlineto{\pgfqpoint{7.181068in}{2.156753in}}%
\pgfusepath{stroke}%
\end{pgfscope}%
\begin{pgfscope}%
\pgfsetrectcap%
\pgfsetmiterjoin%
\pgfsetlinewidth{0.803000pt}%
\definecolor{currentstroke}{rgb}{0.000000,0.000000,0.000000}%
\pgfsetstrokecolor{currentstroke}%
\pgfsetdash{}{0pt}%
\pgfpathmoveto{\pgfqpoint{0.588387in}{0.521603in}}%
\pgfpathlineto{\pgfqpoint{0.588387in}{2.741849in}}%
\pgfusepath{stroke}%
\end{pgfscope}%
\begin{pgfscope}%
\pgfsetrectcap%
\pgfsetmiterjoin%
\pgfsetlinewidth{0.803000pt}%
\definecolor{currentstroke}{rgb}{0.000000,0.000000,0.000000}%
\pgfsetstrokecolor{currentstroke}%
\pgfsetdash{}{0pt}%
\pgfpathmoveto{\pgfqpoint{7.495005in}{0.521603in}}%
\pgfpathlineto{\pgfqpoint{7.495005in}{2.741849in}}%
\pgfusepath{stroke}%
\end{pgfscope}%
\begin{pgfscope}%
\pgfsetrectcap%
\pgfsetmiterjoin%
\pgfsetlinewidth{0.803000pt}%
\definecolor{currentstroke}{rgb}{0.000000,0.000000,0.000000}%
\pgfsetstrokecolor{currentstroke}%
\pgfsetdash{}{0pt}%
\pgfpathmoveto{\pgfqpoint{0.588387in}{0.521603in}}%
\pgfpathlineto{\pgfqpoint{7.495005in}{0.521603in}}%
\pgfusepath{stroke}%
\end{pgfscope}%
\begin{pgfscope}%
\pgfsetrectcap%
\pgfsetmiterjoin%
\pgfsetlinewidth{0.803000pt}%
\definecolor{currentstroke}{rgb}{0.000000,0.000000,0.000000}%
\pgfsetstrokecolor{currentstroke}%
\pgfsetdash{}{0pt}%
\pgfpathmoveto{\pgfqpoint{0.588387in}{2.741849in}}%
\pgfpathlineto{\pgfqpoint{7.495005in}{2.741849in}}%
\pgfusepath{stroke}%
\end{pgfscope}%
\begin{pgfscope}%
\pgfsetbuttcap%
\pgfsetmiterjoin%
\definecolor{currentfill}{rgb}{1.000000,1.000000,1.000000}%
\pgfsetfillcolor{currentfill}%
\pgfsetfillopacity{0.800000}%
\pgfsetlinewidth{1.003750pt}%
\definecolor{currentstroke}{rgb}{0.800000,0.800000,0.800000}%
\pgfsetstrokecolor{currentstroke}%
\pgfsetstrokeopacity{0.800000}%
\pgfsetdash{}{0pt}%
\pgfpathmoveto{\pgfqpoint{7.611671in}{1.140743in}}%
\pgfpathlineto{\pgfqpoint{8.251043in}{1.140743in}}%
\pgfpathquadraticcurveto{\pgfqpoint{8.284376in}{1.140743in}}{\pgfqpoint{8.284376in}{1.174077in}}%
\pgfpathlineto{\pgfqpoint{8.284376in}{2.625183in}}%
\pgfpathquadraticcurveto{\pgfqpoint{8.284376in}{2.658516in}}{\pgfqpoint{8.251043in}{2.658516in}}%
\pgfpathlineto{\pgfqpoint{7.611671in}{2.658516in}}%
\pgfpathquadraticcurveto{\pgfqpoint{7.578338in}{2.658516in}}{\pgfqpoint{7.578338in}{2.625183in}}%
\pgfpathlineto{\pgfqpoint{7.578338in}{1.174077in}}%
\pgfpathquadraticcurveto{\pgfqpoint{7.578338in}{1.140743in}}{\pgfqpoint{7.611671in}{1.140743in}}%
\pgfpathlineto{\pgfqpoint{7.611671in}{1.140743in}}%
\pgfpathclose%
\pgfusepath{stroke,fill}%
\end{pgfscope}%
\begin{pgfscope}%
\pgfsetrectcap%
\pgfsetroundjoin%
\pgfsetlinewidth{1.505625pt}%
\pgfsetstrokecolor{currentstroke1}%
\pgfsetdash{}{0pt}%
\pgfpathmoveto{\pgfqpoint{7.645005in}{2.523555in}}%
\pgfpathlineto{\pgfqpoint{7.811671in}{2.523555in}}%
\pgfpathlineto{\pgfqpoint{7.978338in}{2.523555in}}%
\pgfusepath{stroke}%
\end{pgfscope}%
\begin{pgfscope}%
\definecolor{textcolor}{rgb}{0.000000,0.000000,0.000000}%
\pgfsetstrokecolor{textcolor}%
\pgfsetfillcolor{textcolor}%
\pgftext[x=8.111671in,y=2.465222in,left,base]{\color{textcolor}{\rmfamily\fontsize{12.000000}{14.400000}\selectfont\catcode`\^=\active\def^{\ifmmode\sp\else\^{}\fi}\catcode`\%=\active\def%{\%}3}}%
\end{pgfscope}%
\begin{pgfscope}%
\pgfsetrectcap%
\pgfsetroundjoin%
\pgfsetlinewidth{1.505625pt}%
\pgfsetstrokecolor{currentstroke2}%
\pgfsetdash{}{0pt}%
\pgfpathmoveto{\pgfqpoint{7.645005in}{2.278926in}}%
\pgfpathlineto{\pgfqpoint{7.811671in}{2.278926in}}%
\pgfpathlineto{\pgfqpoint{7.978338in}{2.278926in}}%
\pgfusepath{stroke}%
\end{pgfscope}%
\begin{pgfscope}%
\definecolor{textcolor}{rgb}{0.000000,0.000000,0.000000}%
\pgfsetstrokecolor{textcolor}%
\pgfsetfillcolor{textcolor}%
\pgftext[x=8.111671in,y=2.220593in,left,base]{\color{textcolor}{\rmfamily\fontsize{12.000000}{14.400000}\selectfont\catcode`\^=\active\def^{\ifmmode\sp\else\^{}\fi}\catcode`\%=\active\def%{\%}4}}%
\end{pgfscope}%
\begin{pgfscope}%
\pgfsetrectcap%
\pgfsetroundjoin%
\pgfsetlinewidth{1.505625pt}%
\pgfsetstrokecolor{currentstroke3}%
\pgfsetdash{}{0pt}%
\pgfpathmoveto{\pgfqpoint{7.645005in}{2.034297in}}%
\pgfpathlineto{\pgfqpoint{7.811671in}{2.034297in}}%
\pgfpathlineto{\pgfqpoint{7.978338in}{2.034297in}}%
\pgfusepath{stroke}%
\end{pgfscope}%
\begin{pgfscope}%
\definecolor{textcolor}{rgb}{0.000000,0.000000,0.000000}%
\pgfsetstrokecolor{textcolor}%
\pgfsetfillcolor{textcolor}%
\pgftext[x=8.111671in,y=1.975964in,left,base]{\color{textcolor}{\rmfamily\fontsize{12.000000}{14.400000}\selectfont\catcode`\^=\active\def^{\ifmmode\sp\else\^{}\fi}\catcode`\%=\active\def%{\%}5}}%
\end{pgfscope}%
\begin{pgfscope}%
\pgfsetrectcap%
\pgfsetroundjoin%
\pgfsetlinewidth{1.505625pt}%
\pgfsetstrokecolor{currentstroke4}%
\pgfsetdash{}{0pt}%
\pgfpathmoveto{\pgfqpoint{7.645005in}{1.789669in}}%
\pgfpathlineto{\pgfqpoint{7.811671in}{1.789669in}}%
\pgfpathlineto{\pgfqpoint{7.978338in}{1.789669in}}%
\pgfusepath{stroke}%
\end{pgfscope}%
\begin{pgfscope}%
\definecolor{textcolor}{rgb}{0.000000,0.000000,0.000000}%
\pgfsetstrokecolor{textcolor}%
\pgfsetfillcolor{textcolor}%
\pgftext[x=8.111671in,y=1.731335in,left,base]{\color{textcolor}{\rmfamily\fontsize{12.000000}{14.400000}\selectfont\catcode`\^=\active\def^{\ifmmode\sp\else\^{}\fi}\catcode`\%=\active\def%{\%}6}}%
\end{pgfscope}%
\begin{pgfscope}%
\pgfsetrectcap%
\pgfsetroundjoin%
\pgfsetlinewidth{1.505625pt}%
\pgfsetstrokecolor{currentstroke5}%
\pgfsetdash{}{0pt}%
\pgfpathmoveto{\pgfqpoint{7.645005in}{1.545040in}}%
\pgfpathlineto{\pgfqpoint{7.811671in}{1.545040in}}%
\pgfpathlineto{\pgfqpoint{7.978338in}{1.545040in}}%
\pgfusepath{stroke}%
\end{pgfscope}%
\begin{pgfscope}%
\definecolor{textcolor}{rgb}{0.000000,0.000000,0.000000}%
\pgfsetstrokecolor{textcolor}%
\pgfsetfillcolor{textcolor}%
\pgftext[x=8.111671in,y=1.486707in,left,base]{\color{textcolor}{\rmfamily\fontsize{12.000000}{14.400000}\selectfont\catcode`\^=\active\def^{\ifmmode\sp\else\^{}\fi}\catcode`\%=\active\def%{\%}7}}%
\end{pgfscope}%
\begin{pgfscope}%
\pgfsetrectcap%
\pgfsetroundjoin%
\pgfsetlinewidth{1.505625pt}%
\pgfsetstrokecolor{currentstroke6}%
\pgfsetdash{}{0pt}%
\pgfpathmoveto{\pgfqpoint{7.645005in}{1.300411in}}%
\pgfpathlineto{\pgfqpoint{7.811671in}{1.300411in}}%
\pgfpathlineto{\pgfqpoint{7.978338in}{1.300411in}}%
\pgfusepath{stroke}%
\end{pgfscope}%
\begin{pgfscope}%
\definecolor{textcolor}{rgb}{0.000000,0.000000,0.000000}%
\pgfsetstrokecolor{textcolor}%
\pgfsetfillcolor{textcolor}%
\pgftext[x=8.111671in,y=1.242078in,left,base]{\color{textcolor}{\rmfamily\fontsize{12.000000}{14.400000}\selectfont\catcode`\^=\active\def^{\ifmmode\sp\else\^{}\fi}\catcode`\%=\active\def%{\%}8}}%
\end{pgfscope}%
\end{pgfpicture}%
\makeatother%
\endgroup%
}
	\caption[Checks performed for graphs with no NAC-coloring]{
		The number of checks performed to finish search for graphs with no NAC-coloring for different subgraph sizes \( k \).}%
	\label{fig:graph_no_nac_coloring_first_checks_subgraph_size}
\end{figure}%


\subsection{Other strategies}%
\label{sec:other_strategies}

In this section, we show the performance of other strategies described in \Cref{chapter:alg}.
We do not show these strategies in previous figures as they would influence
the scale and would make figures and legends unreadable and unclear.

Some of these strategies perform as well as
than our preferred strategies for some graph classes,
but fail for others and therefore are not universal enough to use in a library.
%
In the following figures, we fixed
split strategy to \Neighbors{} or merging strategy to \MergeLinear{}
as they perform the best as shown in \Cref{sec:bench_graph_classes}.

First, we show in \Cref{fig:graph_mimimally_rigid_failing_merging_first_runtime,fig:graph_no_nac_coloring_generated_rigid_failing_merging_first_runtime}
how strategies like \Log{} and \PromisingCycles{} fail on minimally rigid graphs.
We also shown in \Cref{fig:graph_mimimally_rigid_failing_split_first_runtime},
\KernighanLin{} and \Cuts{} perform worse.
%
Graphs with no three nor four cycles behave similarly for these strategies.
%
\begin{figure}[thbp]
	\centering
	\scalebox{\BenchFigureScale}{%% Creator: Matplotlib, PGF backend
%%
%% To include the figure in your LaTeX document, write
%%   \input{<filename>.pgf}
%%
%% Make sure the required packages are loaded in your preamble
%%   \usepackage{pgf}
%%
%% Also ensure that all the required font packages are loaded; for instance,
%% the lmodern package is sometimes necessary when using math font.
%%   \usepackage{lmodern}
%%
%% Figures using additional raster images can only be included by \input if
%% they are in the same directory as the main LaTeX file. For loading figures
%% from other directories you can use the `import` package
%%   \usepackage{import}
%%
%% and then include the figures with
%%   \import{<path to file>}{<filename>.pgf}
%%
%% Matplotlib used the following preamble
%%   \def\mathdefault#1{#1}
%%   \everymath=\expandafter{\the\everymath\displaystyle}
%%   \IfFileExists{scrextend.sty}{
%%     \usepackage[fontsize=10.000000pt]{scrextend}
%%   }{
%%     \renewcommand{\normalsize}{\fontsize{10.000000}{12.000000}\selectfont}
%%     \normalsize
%%   }
%%   
%%   \ifdefined\pdftexversion\else  % non-pdftex case.
%%     \usepackage{fontspec}
%%     \setmainfont{DejaVuSans.ttf}[Path=\detokenize{/home/petr/Projects/PyRigi/.venv/lib/python3.12/site-packages/matplotlib/mpl-data/fonts/ttf/}]
%%     \setsansfont{DejaVuSans.ttf}[Path=\detokenize{/home/petr/Projects/PyRigi/.venv/lib/python3.12/site-packages/matplotlib/mpl-data/fonts/ttf/}]
%%     \setmonofont{DejaVuSansMono.ttf}[Path=\detokenize{/home/petr/Projects/PyRigi/.venv/lib/python3.12/site-packages/matplotlib/mpl-data/fonts/ttf/}]
%%   \fi
%%   \makeatletter\@ifpackageloaded{under\Score{}}{}{\usepackage[strings]{under\Score{}}}\makeatother
%%
\begingroup%
\makeatletter%
\begin{pgfpicture}%
\pgfpathrectangle{\pgfpointorigin}{\pgfqpoint{8.384376in}{2.841849in}}%
\pgfusepath{use as bounding box, clip}%
\begin{pgfscope}%
\pgfsetbuttcap%
\pgfsetmiterjoin%
\definecolor{currentfill}{rgb}{1.000000,1.000000,1.000000}%
\pgfsetfillcolor{currentfill}%
\pgfsetlinewidth{0.000000pt}%
\definecolor{currentstroke}{rgb}{1.000000,1.000000,1.000000}%
\pgfsetstrokecolor{currentstroke}%
\pgfsetdash{}{0pt}%
\pgfpathmoveto{\pgfqpoint{0.000000in}{0.000000in}}%
\pgfpathlineto{\pgfqpoint{8.384376in}{0.000000in}}%
\pgfpathlineto{\pgfqpoint{8.384376in}{2.841849in}}%
\pgfpathlineto{\pgfqpoint{0.000000in}{2.841849in}}%
\pgfpathlineto{\pgfqpoint{0.000000in}{0.000000in}}%
\pgfpathclose%
\pgfusepath{fill}%
\end{pgfscope}%
\begin{pgfscope}%
\pgfsetbuttcap%
\pgfsetmiterjoin%
\definecolor{currentfill}{rgb}{1.000000,1.000000,1.000000}%
\pgfsetfillcolor{currentfill}%
\pgfsetlinewidth{0.000000pt}%
\definecolor{currentstroke}{rgb}{0.000000,0.000000,0.000000}%
\pgfsetstrokecolor{currentstroke}%
\pgfsetstrokeopacity{0.000000}%
\pgfsetdash{}{0pt}%
\pgfpathmoveto{\pgfqpoint{0.588387in}{0.521603in}}%
\pgfpathlineto{\pgfqpoint{5.903102in}{0.521603in}}%
\pgfpathlineto{\pgfqpoint{5.903102in}{2.741849in}}%
\pgfpathlineto{\pgfqpoint{0.588387in}{2.741849in}}%
\pgfpathlineto{\pgfqpoint{0.588387in}{0.521603in}}%
\pgfpathclose%
\pgfusepath{fill}%
\end{pgfscope}%
\begin{pgfscope}%
\pgfsetbuttcap%
\pgfsetroundjoin%
\definecolor{currentfill}{rgb}{0.000000,0.000000,0.000000}%
\pgfsetfillcolor{currentfill}%
\pgfsetlinewidth{0.803000pt}%
\definecolor{currentstroke}{rgb}{0.000000,0.000000,0.000000}%
\pgfsetstrokecolor{currentstroke}%
\pgfsetdash{}{0pt}%
\pgfsys@defobject{currentmarker}{\pgfqpoint{0.000000in}{-0.048611in}}{\pgfqpoint{0.000000in}{0.000000in}}{%
\pgfpathmoveto{\pgfqpoint{0.000000in}{0.000000in}}%
\pgfpathlineto{\pgfqpoint{0.000000in}{-0.048611in}}%
\pgfusepath{stroke,fill}%
}%
\begin{pgfscope}%
\pgfsys@transformshift{1.027172in}{0.521603in}%
\pgfsys@useobject{currentmarker}{}%
\end{pgfscope}%
\end{pgfscope}%
\begin{pgfscope}%
\definecolor{textcolor}{rgb}{0.000000,0.000000,0.000000}%
\pgfsetstrokecolor{textcolor}%
\pgfsetfillcolor{textcolor}%
\pgftext[x=1.027172in,y=0.424381in,,top]{\color{textcolor}{\rmfamily\fontsize{10.000000}{12.000000}\selectfont\catcode`\^=\active\def^{\ifmmode\sp\else\^{}\fi}\catcode`\%=\active\def%{\%}$\mathdefault{12}$}}%
\end{pgfscope}%
\begin{pgfscope}%
\pgfsetbuttcap%
\pgfsetroundjoin%
\definecolor{currentfill}{rgb}{0.000000,0.000000,0.000000}%
\pgfsetfillcolor{currentfill}%
\pgfsetlinewidth{0.803000pt}%
\definecolor{currentstroke}{rgb}{0.000000,0.000000,0.000000}%
\pgfsetstrokecolor{currentstroke}%
\pgfsetdash{}{0pt}%
\pgfsys@defobject{currentmarker}{\pgfqpoint{0.000000in}{-0.048611in}}{\pgfqpoint{0.000000in}{0.000000in}}{%
\pgfpathmoveto{\pgfqpoint{0.000000in}{0.000000in}}%
\pgfpathlineto{\pgfqpoint{0.000000in}{-0.048611in}}%
\pgfusepath{stroke,fill}%
}%
\begin{pgfscope}%
\pgfsys@transformshift{1.618791in}{0.521603in}%
\pgfsys@useobject{currentmarker}{}%
\end{pgfscope}%
\end{pgfscope}%
\begin{pgfscope}%
\definecolor{textcolor}{rgb}{0.000000,0.000000,0.000000}%
\pgfsetstrokecolor{textcolor}%
\pgfsetfillcolor{textcolor}%
\pgftext[x=1.618791in,y=0.424381in,,top]{\color{textcolor}{\rmfamily\fontsize{10.000000}{12.000000}\selectfont\catcode`\^=\active\def^{\ifmmode\sp\else\^{}\fi}\catcode`\%=\active\def%{\%}$\mathdefault{18}$}}%
\end{pgfscope}%
\begin{pgfscope}%
\pgfsetbuttcap%
\pgfsetroundjoin%
\definecolor{currentfill}{rgb}{0.000000,0.000000,0.000000}%
\pgfsetfillcolor{currentfill}%
\pgfsetlinewidth{0.803000pt}%
\definecolor{currentstroke}{rgb}{0.000000,0.000000,0.000000}%
\pgfsetstrokecolor{currentstroke}%
\pgfsetdash{}{0pt}%
\pgfsys@defobject{currentmarker}{\pgfqpoint{0.000000in}{-0.048611in}}{\pgfqpoint{0.000000in}{0.000000in}}{%
\pgfpathmoveto{\pgfqpoint{0.000000in}{0.000000in}}%
\pgfpathlineto{\pgfqpoint{0.000000in}{-0.048611in}}%
\pgfusepath{stroke,fill}%
}%
\begin{pgfscope}%
\pgfsys@transformshift{2.210411in}{0.521603in}%
\pgfsys@useobject{currentmarker}{}%
\end{pgfscope}%
\end{pgfscope}%
\begin{pgfscope}%
\definecolor{textcolor}{rgb}{0.000000,0.000000,0.000000}%
\pgfsetstrokecolor{textcolor}%
\pgfsetfillcolor{textcolor}%
\pgftext[x=2.210411in,y=0.424381in,,top]{\color{textcolor}{\rmfamily\fontsize{10.000000}{12.000000}\selectfont\catcode`\^=\active\def^{\ifmmode\sp\else\^{}\fi}\catcode`\%=\active\def%{\%}$\mathdefault{24}$}}%
\end{pgfscope}%
\begin{pgfscope}%
\pgfsetbuttcap%
\pgfsetroundjoin%
\definecolor{currentfill}{rgb}{0.000000,0.000000,0.000000}%
\pgfsetfillcolor{currentfill}%
\pgfsetlinewidth{0.803000pt}%
\definecolor{currentstroke}{rgb}{0.000000,0.000000,0.000000}%
\pgfsetstrokecolor{currentstroke}%
\pgfsetdash{}{0pt}%
\pgfsys@defobject{currentmarker}{\pgfqpoint{0.000000in}{-0.048611in}}{\pgfqpoint{0.000000in}{0.000000in}}{%
\pgfpathmoveto{\pgfqpoint{0.000000in}{0.000000in}}%
\pgfpathlineto{\pgfqpoint{0.000000in}{-0.048611in}}%
\pgfusepath{stroke,fill}%
}%
\begin{pgfscope}%
\pgfsys@transformshift{2.802030in}{0.521603in}%
\pgfsys@useobject{currentmarker}{}%
\end{pgfscope}%
\end{pgfscope}%
\begin{pgfscope}%
\definecolor{textcolor}{rgb}{0.000000,0.000000,0.000000}%
\pgfsetstrokecolor{textcolor}%
\pgfsetfillcolor{textcolor}%
\pgftext[x=2.802030in,y=0.424381in,,top]{\color{textcolor}{\rmfamily\fontsize{10.000000}{12.000000}\selectfont\catcode`\^=\active\def^{\ifmmode\sp\else\^{}\fi}\catcode`\%=\active\def%{\%}$\mathdefault{30}$}}%
\end{pgfscope}%
\begin{pgfscope}%
\pgfsetbuttcap%
\pgfsetroundjoin%
\definecolor{currentfill}{rgb}{0.000000,0.000000,0.000000}%
\pgfsetfillcolor{currentfill}%
\pgfsetlinewidth{0.803000pt}%
\definecolor{currentstroke}{rgb}{0.000000,0.000000,0.000000}%
\pgfsetstrokecolor{currentstroke}%
\pgfsetdash{}{0pt}%
\pgfsys@defobject{currentmarker}{\pgfqpoint{0.000000in}{-0.048611in}}{\pgfqpoint{0.000000in}{0.000000in}}{%
\pgfpathmoveto{\pgfqpoint{0.000000in}{0.000000in}}%
\pgfpathlineto{\pgfqpoint{0.000000in}{-0.048611in}}%
\pgfusepath{stroke,fill}%
}%
\begin{pgfscope}%
\pgfsys@transformshift{3.393649in}{0.521603in}%
\pgfsys@useobject{currentmarker}{}%
\end{pgfscope}%
\end{pgfscope}%
\begin{pgfscope}%
\definecolor{textcolor}{rgb}{0.000000,0.000000,0.000000}%
\pgfsetstrokecolor{textcolor}%
\pgfsetfillcolor{textcolor}%
\pgftext[x=3.393649in,y=0.424381in,,top]{\color{textcolor}{\rmfamily\fontsize{10.000000}{12.000000}\selectfont\catcode`\^=\active\def^{\ifmmode\sp\else\^{}\fi}\catcode`\%=\active\def%{\%}$\mathdefault{36}$}}%
\end{pgfscope}%
\begin{pgfscope}%
\pgfsetbuttcap%
\pgfsetroundjoin%
\definecolor{currentfill}{rgb}{0.000000,0.000000,0.000000}%
\pgfsetfillcolor{currentfill}%
\pgfsetlinewidth{0.803000pt}%
\definecolor{currentstroke}{rgb}{0.000000,0.000000,0.000000}%
\pgfsetstrokecolor{currentstroke}%
\pgfsetdash{}{0pt}%
\pgfsys@defobject{currentmarker}{\pgfqpoint{0.000000in}{-0.048611in}}{\pgfqpoint{0.000000in}{0.000000in}}{%
\pgfpathmoveto{\pgfqpoint{0.000000in}{0.000000in}}%
\pgfpathlineto{\pgfqpoint{0.000000in}{-0.048611in}}%
\pgfusepath{stroke,fill}%
}%
\begin{pgfscope}%
\pgfsys@transformshift{3.985269in}{0.521603in}%
\pgfsys@useobject{currentmarker}{}%
\end{pgfscope}%
\end{pgfscope}%
\begin{pgfscope}%
\definecolor{textcolor}{rgb}{0.000000,0.000000,0.000000}%
\pgfsetstrokecolor{textcolor}%
\pgfsetfillcolor{textcolor}%
\pgftext[x=3.985269in,y=0.424381in,,top]{\color{textcolor}{\rmfamily\fontsize{10.000000}{12.000000}\selectfont\catcode`\^=\active\def^{\ifmmode\sp\else\^{}\fi}\catcode`\%=\active\def%{\%}$\mathdefault{42}$}}%
\end{pgfscope}%
\begin{pgfscope}%
\pgfsetbuttcap%
\pgfsetroundjoin%
\definecolor{currentfill}{rgb}{0.000000,0.000000,0.000000}%
\pgfsetfillcolor{currentfill}%
\pgfsetlinewidth{0.803000pt}%
\definecolor{currentstroke}{rgb}{0.000000,0.000000,0.000000}%
\pgfsetstrokecolor{currentstroke}%
\pgfsetdash{}{0pt}%
\pgfsys@defobject{currentmarker}{\pgfqpoint{0.000000in}{-0.048611in}}{\pgfqpoint{0.000000in}{0.000000in}}{%
\pgfpathmoveto{\pgfqpoint{0.000000in}{0.000000in}}%
\pgfpathlineto{\pgfqpoint{0.000000in}{-0.048611in}}%
\pgfusepath{stroke,fill}%
}%
\begin{pgfscope}%
\pgfsys@transformshift{4.576888in}{0.521603in}%
\pgfsys@useobject{currentmarker}{}%
\end{pgfscope}%
\end{pgfscope}%
\begin{pgfscope}%
\definecolor{textcolor}{rgb}{0.000000,0.000000,0.000000}%
\pgfsetstrokecolor{textcolor}%
\pgfsetfillcolor{textcolor}%
\pgftext[x=4.576888in,y=0.424381in,,top]{\color{textcolor}{\rmfamily\fontsize{10.000000}{12.000000}\selectfont\catcode`\^=\active\def^{\ifmmode\sp\else\^{}\fi}\catcode`\%=\active\def%{\%}$\mathdefault{48}$}}%
\end{pgfscope}%
\begin{pgfscope}%
\pgfsetbuttcap%
\pgfsetroundjoin%
\definecolor{currentfill}{rgb}{0.000000,0.000000,0.000000}%
\pgfsetfillcolor{currentfill}%
\pgfsetlinewidth{0.803000pt}%
\definecolor{currentstroke}{rgb}{0.000000,0.000000,0.000000}%
\pgfsetstrokecolor{currentstroke}%
\pgfsetdash{}{0pt}%
\pgfsys@defobject{currentmarker}{\pgfqpoint{0.000000in}{-0.048611in}}{\pgfqpoint{0.000000in}{0.000000in}}{%
\pgfpathmoveto{\pgfqpoint{0.000000in}{0.000000in}}%
\pgfpathlineto{\pgfqpoint{0.000000in}{-0.048611in}}%
\pgfusepath{stroke,fill}%
}%
\begin{pgfscope}%
\pgfsys@transformshift{5.168508in}{0.521603in}%
\pgfsys@useobject{currentmarker}{}%
\end{pgfscope}%
\end{pgfscope}%
\begin{pgfscope}%
\definecolor{textcolor}{rgb}{0.000000,0.000000,0.000000}%
\pgfsetstrokecolor{textcolor}%
\pgfsetfillcolor{textcolor}%
\pgftext[x=5.168508in,y=0.424381in,,top]{\color{textcolor}{\rmfamily\fontsize{10.000000}{12.000000}\selectfont\catcode`\^=\active\def^{\ifmmode\sp\else\^{}\fi}\catcode`\%=\active\def%{\%}$\mathdefault{54}$}}%
\end{pgfscope}%
\begin{pgfscope}%
\pgfsetbuttcap%
\pgfsetroundjoin%
\definecolor{currentfill}{rgb}{0.000000,0.000000,0.000000}%
\pgfsetfillcolor{currentfill}%
\pgfsetlinewidth{0.803000pt}%
\definecolor{currentstroke}{rgb}{0.000000,0.000000,0.000000}%
\pgfsetstrokecolor{currentstroke}%
\pgfsetdash{}{0pt}%
\pgfsys@defobject{currentmarker}{\pgfqpoint{0.000000in}{-0.048611in}}{\pgfqpoint{0.000000in}{0.000000in}}{%
\pgfpathmoveto{\pgfqpoint{0.000000in}{0.000000in}}%
\pgfpathlineto{\pgfqpoint{0.000000in}{-0.048611in}}%
\pgfusepath{stroke,fill}%
}%
\begin{pgfscope}%
\pgfsys@transformshift{5.760127in}{0.521603in}%
\pgfsys@useobject{currentmarker}{}%
\end{pgfscope}%
\end{pgfscope}%
\begin{pgfscope}%
\definecolor{textcolor}{rgb}{0.000000,0.000000,0.000000}%
\pgfsetstrokecolor{textcolor}%
\pgfsetfillcolor{textcolor}%
\pgftext[x=5.760127in,y=0.424381in,,top]{\color{textcolor}{\rmfamily\fontsize{10.000000}{12.000000}\selectfont\catcode`\^=\active\def^{\ifmmode\sp\else\^{}\fi}\catcode`\%=\active\def%{\%}$\mathdefault{60}$}}%
\end{pgfscope}%
\begin{pgfscope}%
\definecolor{textcolor}{rgb}{0.000000,0.000000,0.000000}%
\pgfsetstrokecolor{textcolor}%
\pgfsetfillcolor{textcolor}%
\pgftext[x=3.245745in,y=0.234413in,,top]{\color{textcolor}{\rmfamily\fontsize{10.000000}{12.000000}\selectfont\catcode`\^=\active\def^{\ifmmode\sp\else\^{}\fi}\catcode`\%=\active\def%{\%}Vertices}}%
\end{pgfscope}%
\begin{pgfscope}%
\pgfsetbuttcap%
\pgfsetroundjoin%
\definecolor{currentfill}{rgb}{0.000000,0.000000,0.000000}%
\pgfsetfillcolor{currentfill}%
\pgfsetlinewidth{0.803000pt}%
\definecolor{currentstroke}{rgb}{0.000000,0.000000,0.000000}%
\pgfsetstrokecolor{currentstroke}%
\pgfsetdash{}{0pt}%
\pgfsys@defobject{currentmarker}{\pgfqpoint{-0.048611in}{0.000000in}}{\pgfqpoint{-0.000000in}{0.000000in}}{%
\pgfpathmoveto{\pgfqpoint{-0.000000in}{0.000000in}}%
\pgfpathlineto{\pgfqpoint{-0.048611in}{0.000000in}}%
\pgfusepath{stroke,fill}%
}%
\begin{pgfscope}%
\pgfsys@transformshift{0.588387in}{0.962462in}%
\pgfsys@useobject{currentmarker}{}%
\end{pgfscope}%
\end{pgfscope}%
\begin{pgfscope}%
\definecolor{textcolor}{rgb}{0.000000,0.000000,0.000000}%
\pgfsetstrokecolor{textcolor}%
\pgfsetfillcolor{textcolor}%
\pgftext[x=0.289968in, y=0.909700in, left, base]{\color{textcolor}{\rmfamily\fontsize{10.000000}{12.000000}\selectfont\catcode`\^=\active\def^{\ifmmode\sp\else\^{}\fi}\catcode`\%=\active\def%{\%}$\mathdefault{10^{1}}$}}%
\end{pgfscope}%
\begin{pgfscope}%
\pgfsetbuttcap%
\pgfsetroundjoin%
\definecolor{currentfill}{rgb}{0.000000,0.000000,0.000000}%
\pgfsetfillcolor{currentfill}%
\pgfsetlinewidth{0.803000pt}%
\definecolor{currentstroke}{rgb}{0.000000,0.000000,0.000000}%
\pgfsetstrokecolor{currentstroke}%
\pgfsetdash{}{0pt}%
\pgfsys@defobject{currentmarker}{\pgfqpoint{-0.048611in}{0.000000in}}{\pgfqpoint{-0.000000in}{0.000000in}}{%
\pgfpathmoveto{\pgfqpoint{-0.000000in}{0.000000in}}%
\pgfpathlineto{\pgfqpoint{-0.048611in}{0.000000in}}%
\pgfusepath{stroke,fill}%
}%
\begin{pgfscope}%
\pgfsys@transformshift{0.588387in}{1.657727in}%
\pgfsys@useobject{currentmarker}{}%
\end{pgfscope}%
\end{pgfscope}%
\begin{pgfscope}%
\definecolor{textcolor}{rgb}{0.000000,0.000000,0.000000}%
\pgfsetstrokecolor{textcolor}%
\pgfsetfillcolor{textcolor}%
\pgftext[x=0.289968in, y=1.604966in, left, base]{\color{textcolor}{\rmfamily\fontsize{10.000000}{12.000000}\selectfont\catcode`\^=\active\def^{\ifmmode\sp\else\^{}\fi}\catcode`\%=\active\def%{\%}$\mathdefault{10^{2}}$}}%
\end{pgfscope}%
\begin{pgfscope}%
\pgfsetbuttcap%
\pgfsetroundjoin%
\definecolor{currentfill}{rgb}{0.000000,0.000000,0.000000}%
\pgfsetfillcolor{currentfill}%
\pgfsetlinewidth{0.803000pt}%
\definecolor{currentstroke}{rgb}{0.000000,0.000000,0.000000}%
\pgfsetstrokecolor{currentstroke}%
\pgfsetdash{}{0pt}%
\pgfsys@defobject{currentmarker}{\pgfqpoint{-0.048611in}{0.000000in}}{\pgfqpoint{-0.000000in}{0.000000in}}{%
\pgfpathmoveto{\pgfqpoint{-0.000000in}{0.000000in}}%
\pgfpathlineto{\pgfqpoint{-0.048611in}{0.000000in}}%
\pgfusepath{stroke,fill}%
}%
\begin{pgfscope}%
\pgfsys@transformshift{0.588387in}{2.352993in}%
\pgfsys@useobject{currentmarker}{}%
\end{pgfscope}%
\end{pgfscope}%
\begin{pgfscope}%
\definecolor{textcolor}{rgb}{0.000000,0.000000,0.000000}%
\pgfsetstrokecolor{textcolor}%
\pgfsetfillcolor{textcolor}%
\pgftext[x=0.289968in, y=2.300232in, left, base]{\color{textcolor}{\rmfamily\fontsize{10.000000}{12.000000}\selectfont\catcode`\^=\active\def^{\ifmmode\sp\else\^{}\fi}\catcode`\%=\active\def%{\%}$\mathdefault{10^{3}}$}}%
\end{pgfscope}%
\begin{pgfscope}%
\pgfsetbuttcap%
\pgfsetroundjoin%
\definecolor{currentfill}{rgb}{0.000000,0.000000,0.000000}%
\pgfsetfillcolor{currentfill}%
\pgfsetlinewidth{0.602250pt}%
\definecolor{currentstroke}{rgb}{0.000000,0.000000,0.000000}%
\pgfsetstrokecolor{currentstroke}%
\pgfsetdash{}{0pt}%
\pgfsys@defobject{currentmarker}{\pgfqpoint{-0.027778in}{0.000000in}}{\pgfqpoint{-0.000000in}{0.000000in}}{%
\pgfpathmoveto{\pgfqpoint{-0.000000in}{0.000000in}}%
\pgfpathlineto{\pgfqpoint{-0.027778in}{0.000000in}}%
\pgfusepath{stroke,fill}%
}%
\begin{pgfscope}%
\pgfsys@transformshift{0.588387in}{0.598922in}%
\pgfsys@useobject{currentmarker}{}%
\end{pgfscope}%
\end{pgfscope}%
\begin{pgfscope}%
\pgfsetbuttcap%
\pgfsetroundjoin%
\definecolor{currentfill}{rgb}{0.000000,0.000000,0.000000}%
\pgfsetfillcolor{currentfill}%
\pgfsetlinewidth{0.602250pt}%
\definecolor{currentstroke}{rgb}{0.000000,0.000000,0.000000}%
\pgfsetstrokecolor{currentstroke}%
\pgfsetdash{}{0pt}%
\pgfsys@defobject{currentmarker}{\pgfqpoint{-0.027778in}{0.000000in}}{\pgfqpoint{-0.000000in}{0.000000in}}{%
\pgfpathmoveto{\pgfqpoint{-0.000000in}{0.000000in}}%
\pgfpathlineto{\pgfqpoint{-0.027778in}{0.000000in}}%
\pgfusepath{stroke,fill}%
}%
\begin{pgfscope}%
\pgfsys@transformshift{0.588387in}{0.685787in}%
\pgfsys@useobject{currentmarker}{}%
\end{pgfscope}%
\end{pgfscope}%
\begin{pgfscope}%
\pgfsetbuttcap%
\pgfsetroundjoin%
\definecolor{currentfill}{rgb}{0.000000,0.000000,0.000000}%
\pgfsetfillcolor{currentfill}%
\pgfsetlinewidth{0.602250pt}%
\definecolor{currentstroke}{rgb}{0.000000,0.000000,0.000000}%
\pgfsetstrokecolor{currentstroke}%
\pgfsetdash{}{0pt}%
\pgfsys@defobject{currentmarker}{\pgfqpoint{-0.027778in}{0.000000in}}{\pgfqpoint{-0.000000in}{0.000000in}}{%
\pgfpathmoveto{\pgfqpoint{-0.000000in}{0.000000in}}%
\pgfpathlineto{\pgfqpoint{-0.027778in}{0.000000in}}%
\pgfusepath{stroke,fill}%
}%
\begin{pgfscope}%
\pgfsys@transformshift{0.588387in}{0.753166in}%
\pgfsys@useobject{currentmarker}{}%
\end{pgfscope}%
\end{pgfscope}%
\begin{pgfscope}%
\pgfsetbuttcap%
\pgfsetroundjoin%
\definecolor{currentfill}{rgb}{0.000000,0.000000,0.000000}%
\pgfsetfillcolor{currentfill}%
\pgfsetlinewidth{0.602250pt}%
\definecolor{currentstroke}{rgb}{0.000000,0.000000,0.000000}%
\pgfsetstrokecolor{currentstroke}%
\pgfsetdash{}{0pt}%
\pgfsys@defobject{currentmarker}{\pgfqpoint{-0.027778in}{0.000000in}}{\pgfqpoint{-0.000000in}{0.000000in}}{%
\pgfpathmoveto{\pgfqpoint{-0.000000in}{0.000000in}}%
\pgfpathlineto{\pgfqpoint{-0.027778in}{0.000000in}}%
\pgfusepath{stroke,fill}%
}%
\begin{pgfscope}%
\pgfsys@transformshift{0.588387in}{0.808218in}%
\pgfsys@useobject{currentmarker}{}%
\end{pgfscope}%
\end{pgfscope}%
\begin{pgfscope}%
\pgfsetbuttcap%
\pgfsetroundjoin%
\definecolor{currentfill}{rgb}{0.000000,0.000000,0.000000}%
\pgfsetfillcolor{currentfill}%
\pgfsetlinewidth{0.602250pt}%
\definecolor{currentstroke}{rgb}{0.000000,0.000000,0.000000}%
\pgfsetstrokecolor{currentstroke}%
\pgfsetdash{}{0pt}%
\pgfsys@defobject{currentmarker}{\pgfqpoint{-0.027778in}{0.000000in}}{\pgfqpoint{-0.000000in}{0.000000in}}{%
\pgfpathmoveto{\pgfqpoint{-0.000000in}{0.000000in}}%
\pgfpathlineto{\pgfqpoint{-0.027778in}{0.000000in}}%
\pgfusepath{stroke,fill}%
}%
\begin{pgfscope}%
\pgfsys@transformshift{0.588387in}{0.854763in}%
\pgfsys@useobject{currentmarker}{}%
\end{pgfscope}%
\end{pgfscope}%
\begin{pgfscope}%
\pgfsetbuttcap%
\pgfsetroundjoin%
\definecolor{currentfill}{rgb}{0.000000,0.000000,0.000000}%
\pgfsetfillcolor{currentfill}%
\pgfsetlinewidth{0.602250pt}%
\definecolor{currentstroke}{rgb}{0.000000,0.000000,0.000000}%
\pgfsetstrokecolor{currentstroke}%
\pgfsetdash{}{0pt}%
\pgfsys@defobject{currentmarker}{\pgfqpoint{-0.027778in}{0.000000in}}{\pgfqpoint{-0.000000in}{0.000000in}}{%
\pgfpathmoveto{\pgfqpoint{-0.000000in}{0.000000in}}%
\pgfpathlineto{\pgfqpoint{-0.027778in}{0.000000in}}%
\pgfusepath{stroke,fill}%
}%
\begin{pgfscope}%
\pgfsys@transformshift{0.588387in}{0.895083in}%
\pgfsys@useobject{currentmarker}{}%
\end{pgfscope}%
\end{pgfscope}%
\begin{pgfscope}%
\pgfsetbuttcap%
\pgfsetroundjoin%
\definecolor{currentfill}{rgb}{0.000000,0.000000,0.000000}%
\pgfsetfillcolor{currentfill}%
\pgfsetlinewidth{0.602250pt}%
\definecolor{currentstroke}{rgb}{0.000000,0.000000,0.000000}%
\pgfsetstrokecolor{currentstroke}%
\pgfsetdash{}{0pt}%
\pgfsys@defobject{currentmarker}{\pgfqpoint{-0.027778in}{0.000000in}}{\pgfqpoint{-0.000000in}{0.000000in}}{%
\pgfpathmoveto{\pgfqpoint{-0.000000in}{0.000000in}}%
\pgfpathlineto{\pgfqpoint{-0.027778in}{0.000000in}}%
\pgfusepath{stroke,fill}%
}%
\begin{pgfscope}%
\pgfsys@transformshift{0.588387in}{0.930648in}%
\pgfsys@useobject{currentmarker}{}%
\end{pgfscope}%
\end{pgfscope}%
\begin{pgfscope}%
\pgfsetbuttcap%
\pgfsetroundjoin%
\definecolor{currentfill}{rgb}{0.000000,0.000000,0.000000}%
\pgfsetfillcolor{currentfill}%
\pgfsetlinewidth{0.602250pt}%
\definecolor{currentstroke}{rgb}{0.000000,0.000000,0.000000}%
\pgfsetstrokecolor{currentstroke}%
\pgfsetdash{}{0pt}%
\pgfsys@defobject{currentmarker}{\pgfqpoint{-0.027778in}{0.000000in}}{\pgfqpoint{-0.000000in}{0.000000in}}{%
\pgfpathmoveto{\pgfqpoint{-0.000000in}{0.000000in}}%
\pgfpathlineto{\pgfqpoint{-0.027778in}{0.000000in}}%
\pgfusepath{stroke,fill}%
}%
\begin{pgfscope}%
\pgfsys@transformshift{0.588387in}{1.171757in}%
\pgfsys@useobject{currentmarker}{}%
\end{pgfscope}%
\end{pgfscope}%
\begin{pgfscope}%
\pgfsetbuttcap%
\pgfsetroundjoin%
\definecolor{currentfill}{rgb}{0.000000,0.000000,0.000000}%
\pgfsetfillcolor{currentfill}%
\pgfsetlinewidth{0.602250pt}%
\definecolor{currentstroke}{rgb}{0.000000,0.000000,0.000000}%
\pgfsetstrokecolor{currentstroke}%
\pgfsetdash{}{0pt}%
\pgfsys@defobject{currentmarker}{\pgfqpoint{-0.027778in}{0.000000in}}{\pgfqpoint{-0.000000in}{0.000000in}}{%
\pgfpathmoveto{\pgfqpoint{-0.000000in}{0.000000in}}%
\pgfpathlineto{\pgfqpoint{-0.027778in}{0.000000in}}%
\pgfusepath{stroke,fill}%
}%
\begin{pgfscope}%
\pgfsys@transformshift{0.588387in}{1.294188in}%
\pgfsys@useobject{currentmarker}{}%
\end{pgfscope}%
\end{pgfscope}%
\begin{pgfscope}%
\pgfsetbuttcap%
\pgfsetroundjoin%
\definecolor{currentfill}{rgb}{0.000000,0.000000,0.000000}%
\pgfsetfillcolor{currentfill}%
\pgfsetlinewidth{0.602250pt}%
\definecolor{currentstroke}{rgb}{0.000000,0.000000,0.000000}%
\pgfsetstrokecolor{currentstroke}%
\pgfsetdash{}{0pt}%
\pgfsys@defobject{currentmarker}{\pgfqpoint{-0.027778in}{0.000000in}}{\pgfqpoint{-0.000000in}{0.000000in}}{%
\pgfpathmoveto{\pgfqpoint{-0.000000in}{0.000000in}}%
\pgfpathlineto{\pgfqpoint{-0.027778in}{0.000000in}}%
\pgfusepath{stroke,fill}%
}%
\begin{pgfscope}%
\pgfsys@transformshift{0.588387in}{1.381053in}%
\pgfsys@useobject{currentmarker}{}%
\end{pgfscope}%
\end{pgfscope}%
\begin{pgfscope}%
\pgfsetbuttcap%
\pgfsetroundjoin%
\definecolor{currentfill}{rgb}{0.000000,0.000000,0.000000}%
\pgfsetfillcolor{currentfill}%
\pgfsetlinewidth{0.602250pt}%
\definecolor{currentstroke}{rgb}{0.000000,0.000000,0.000000}%
\pgfsetstrokecolor{currentstroke}%
\pgfsetdash{}{0pt}%
\pgfsys@defobject{currentmarker}{\pgfqpoint{-0.027778in}{0.000000in}}{\pgfqpoint{-0.000000in}{0.000000in}}{%
\pgfpathmoveto{\pgfqpoint{-0.000000in}{0.000000in}}%
\pgfpathlineto{\pgfqpoint{-0.027778in}{0.000000in}}%
\pgfusepath{stroke,fill}%
}%
\begin{pgfscope}%
\pgfsys@transformshift{0.588387in}{1.448432in}%
\pgfsys@useobject{currentmarker}{}%
\end{pgfscope}%
\end{pgfscope}%
\begin{pgfscope}%
\pgfsetbuttcap%
\pgfsetroundjoin%
\definecolor{currentfill}{rgb}{0.000000,0.000000,0.000000}%
\pgfsetfillcolor{currentfill}%
\pgfsetlinewidth{0.602250pt}%
\definecolor{currentstroke}{rgb}{0.000000,0.000000,0.000000}%
\pgfsetstrokecolor{currentstroke}%
\pgfsetdash{}{0pt}%
\pgfsys@defobject{currentmarker}{\pgfqpoint{-0.027778in}{0.000000in}}{\pgfqpoint{-0.000000in}{0.000000in}}{%
\pgfpathmoveto{\pgfqpoint{-0.000000in}{0.000000in}}%
\pgfpathlineto{\pgfqpoint{-0.027778in}{0.000000in}}%
\pgfusepath{stroke,fill}%
}%
\begin{pgfscope}%
\pgfsys@transformshift{0.588387in}{1.503484in}%
\pgfsys@useobject{currentmarker}{}%
\end{pgfscope}%
\end{pgfscope}%
\begin{pgfscope}%
\pgfsetbuttcap%
\pgfsetroundjoin%
\definecolor{currentfill}{rgb}{0.000000,0.000000,0.000000}%
\pgfsetfillcolor{currentfill}%
\pgfsetlinewidth{0.602250pt}%
\definecolor{currentstroke}{rgb}{0.000000,0.000000,0.000000}%
\pgfsetstrokecolor{currentstroke}%
\pgfsetdash{}{0pt}%
\pgfsys@defobject{currentmarker}{\pgfqpoint{-0.027778in}{0.000000in}}{\pgfqpoint{-0.000000in}{0.000000in}}{%
\pgfpathmoveto{\pgfqpoint{-0.000000in}{0.000000in}}%
\pgfpathlineto{\pgfqpoint{-0.027778in}{0.000000in}}%
\pgfusepath{stroke,fill}%
}%
\begin{pgfscope}%
\pgfsys@transformshift{0.588387in}{1.550029in}%
\pgfsys@useobject{currentmarker}{}%
\end{pgfscope}%
\end{pgfscope}%
\begin{pgfscope}%
\pgfsetbuttcap%
\pgfsetroundjoin%
\definecolor{currentfill}{rgb}{0.000000,0.000000,0.000000}%
\pgfsetfillcolor{currentfill}%
\pgfsetlinewidth{0.602250pt}%
\definecolor{currentstroke}{rgb}{0.000000,0.000000,0.000000}%
\pgfsetstrokecolor{currentstroke}%
\pgfsetdash{}{0pt}%
\pgfsys@defobject{currentmarker}{\pgfqpoint{-0.027778in}{0.000000in}}{\pgfqpoint{-0.000000in}{0.000000in}}{%
\pgfpathmoveto{\pgfqpoint{-0.000000in}{0.000000in}}%
\pgfpathlineto{\pgfqpoint{-0.027778in}{0.000000in}}%
\pgfusepath{stroke,fill}%
}%
\begin{pgfscope}%
\pgfsys@transformshift{0.588387in}{1.590349in}%
\pgfsys@useobject{currentmarker}{}%
\end{pgfscope}%
\end{pgfscope}%
\begin{pgfscope}%
\pgfsetbuttcap%
\pgfsetroundjoin%
\definecolor{currentfill}{rgb}{0.000000,0.000000,0.000000}%
\pgfsetfillcolor{currentfill}%
\pgfsetlinewidth{0.602250pt}%
\definecolor{currentstroke}{rgb}{0.000000,0.000000,0.000000}%
\pgfsetstrokecolor{currentstroke}%
\pgfsetdash{}{0pt}%
\pgfsys@defobject{currentmarker}{\pgfqpoint{-0.027778in}{0.000000in}}{\pgfqpoint{-0.000000in}{0.000000in}}{%
\pgfpathmoveto{\pgfqpoint{-0.000000in}{0.000000in}}%
\pgfpathlineto{\pgfqpoint{-0.027778in}{0.000000in}}%
\pgfusepath{stroke,fill}%
}%
\begin{pgfscope}%
\pgfsys@transformshift{0.588387in}{1.625914in}%
\pgfsys@useobject{currentmarker}{}%
\end{pgfscope}%
\end{pgfscope}%
\begin{pgfscope}%
\pgfsetbuttcap%
\pgfsetroundjoin%
\definecolor{currentfill}{rgb}{0.000000,0.000000,0.000000}%
\pgfsetfillcolor{currentfill}%
\pgfsetlinewidth{0.602250pt}%
\definecolor{currentstroke}{rgb}{0.000000,0.000000,0.000000}%
\pgfsetstrokecolor{currentstroke}%
\pgfsetdash{}{0pt}%
\pgfsys@defobject{currentmarker}{\pgfqpoint{-0.027778in}{0.000000in}}{\pgfqpoint{-0.000000in}{0.000000in}}{%
\pgfpathmoveto{\pgfqpoint{-0.000000in}{0.000000in}}%
\pgfpathlineto{\pgfqpoint{-0.027778in}{0.000000in}}%
\pgfusepath{stroke,fill}%
}%
\begin{pgfscope}%
\pgfsys@transformshift{0.588387in}{1.867023in}%
\pgfsys@useobject{currentmarker}{}%
\end{pgfscope}%
\end{pgfscope}%
\begin{pgfscope}%
\pgfsetbuttcap%
\pgfsetroundjoin%
\definecolor{currentfill}{rgb}{0.000000,0.000000,0.000000}%
\pgfsetfillcolor{currentfill}%
\pgfsetlinewidth{0.602250pt}%
\definecolor{currentstroke}{rgb}{0.000000,0.000000,0.000000}%
\pgfsetstrokecolor{currentstroke}%
\pgfsetdash{}{0pt}%
\pgfsys@defobject{currentmarker}{\pgfqpoint{-0.027778in}{0.000000in}}{\pgfqpoint{-0.000000in}{0.000000in}}{%
\pgfpathmoveto{\pgfqpoint{-0.000000in}{0.000000in}}%
\pgfpathlineto{\pgfqpoint{-0.027778in}{0.000000in}}%
\pgfusepath{stroke,fill}%
}%
\begin{pgfscope}%
\pgfsys@transformshift{0.588387in}{1.989454in}%
\pgfsys@useobject{currentmarker}{}%
\end{pgfscope}%
\end{pgfscope}%
\begin{pgfscope}%
\pgfsetbuttcap%
\pgfsetroundjoin%
\definecolor{currentfill}{rgb}{0.000000,0.000000,0.000000}%
\pgfsetfillcolor{currentfill}%
\pgfsetlinewidth{0.602250pt}%
\definecolor{currentstroke}{rgb}{0.000000,0.000000,0.000000}%
\pgfsetstrokecolor{currentstroke}%
\pgfsetdash{}{0pt}%
\pgfsys@defobject{currentmarker}{\pgfqpoint{-0.027778in}{0.000000in}}{\pgfqpoint{-0.000000in}{0.000000in}}{%
\pgfpathmoveto{\pgfqpoint{-0.000000in}{0.000000in}}%
\pgfpathlineto{\pgfqpoint{-0.027778in}{0.000000in}}%
\pgfusepath{stroke,fill}%
}%
\begin{pgfscope}%
\pgfsys@transformshift{0.588387in}{2.076319in}%
\pgfsys@useobject{currentmarker}{}%
\end{pgfscope}%
\end{pgfscope}%
\begin{pgfscope}%
\pgfsetbuttcap%
\pgfsetroundjoin%
\definecolor{currentfill}{rgb}{0.000000,0.000000,0.000000}%
\pgfsetfillcolor{currentfill}%
\pgfsetlinewidth{0.602250pt}%
\definecolor{currentstroke}{rgb}{0.000000,0.000000,0.000000}%
\pgfsetstrokecolor{currentstroke}%
\pgfsetdash{}{0pt}%
\pgfsys@defobject{currentmarker}{\pgfqpoint{-0.027778in}{0.000000in}}{\pgfqpoint{-0.000000in}{0.000000in}}{%
\pgfpathmoveto{\pgfqpoint{-0.000000in}{0.000000in}}%
\pgfpathlineto{\pgfqpoint{-0.027778in}{0.000000in}}%
\pgfusepath{stroke,fill}%
}%
\begin{pgfscope}%
\pgfsys@transformshift{0.588387in}{2.143697in}%
\pgfsys@useobject{currentmarker}{}%
\end{pgfscope}%
\end{pgfscope}%
\begin{pgfscope}%
\pgfsetbuttcap%
\pgfsetroundjoin%
\definecolor{currentfill}{rgb}{0.000000,0.000000,0.000000}%
\pgfsetfillcolor{currentfill}%
\pgfsetlinewidth{0.602250pt}%
\definecolor{currentstroke}{rgb}{0.000000,0.000000,0.000000}%
\pgfsetstrokecolor{currentstroke}%
\pgfsetdash{}{0pt}%
\pgfsys@defobject{currentmarker}{\pgfqpoint{-0.027778in}{0.000000in}}{\pgfqpoint{-0.000000in}{0.000000in}}{%
\pgfpathmoveto{\pgfqpoint{-0.000000in}{0.000000in}}%
\pgfpathlineto{\pgfqpoint{-0.027778in}{0.000000in}}%
\pgfusepath{stroke,fill}%
}%
\begin{pgfscope}%
\pgfsys@transformshift{0.588387in}{2.198749in}%
\pgfsys@useobject{currentmarker}{}%
\end{pgfscope}%
\end{pgfscope}%
\begin{pgfscope}%
\pgfsetbuttcap%
\pgfsetroundjoin%
\definecolor{currentfill}{rgb}{0.000000,0.000000,0.000000}%
\pgfsetfillcolor{currentfill}%
\pgfsetlinewidth{0.602250pt}%
\definecolor{currentstroke}{rgb}{0.000000,0.000000,0.000000}%
\pgfsetstrokecolor{currentstroke}%
\pgfsetdash{}{0pt}%
\pgfsys@defobject{currentmarker}{\pgfqpoint{-0.027778in}{0.000000in}}{\pgfqpoint{-0.000000in}{0.000000in}}{%
\pgfpathmoveto{\pgfqpoint{-0.000000in}{0.000000in}}%
\pgfpathlineto{\pgfqpoint{-0.027778in}{0.000000in}}%
\pgfusepath{stroke,fill}%
}%
\begin{pgfscope}%
\pgfsys@transformshift{0.588387in}{2.245295in}%
\pgfsys@useobject{currentmarker}{}%
\end{pgfscope}%
\end{pgfscope}%
\begin{pgfscope}%
\pgfsetbuttcap%
\pgfsetroundjoin%
\definecolor{currentfill}{rgb}{0.000000,0.000000,0.000000}%
\pgfsetfillcolor{currentfill}%
\pgfsetlinewidth{0.602250pt}%
\definecolor{currentstroke}{rgb}{0.000000,0.000000,0.000000}%
\pgfsetstrokecolor{currentstroke}%
\pgfsetdash{}{0pt}%
\pgfsys@defobject{currentmarker}{\pgfqpoint{-0.027778in}{0.000000in}}{\pgfqpoint{-0.000000in}{0.000000in}}{%
\pgfpathmoveto{\pgfqpoint{-0.000000in}{0.000000in}}%
\pgfpathlineto{\pgfqpoint{-0.027778in}{0.000000in}}%
\pgfusepath{stroke,fill}%
}%
\begin{pgfscope}%
\pgfsys@transformshift{0.588387in}{2.285615in}%
\pgfsys@useobject{currentmarker}{}%
\end{pgfscope}%
\end{pgfscope}%
\begin{pgfscope}%
\pgfsetbuttcap%
\pgfsetroundjoin%
\definecolor{currentfill}{rgb}{0.000000,0.000000,0.000000}%
\pgfsetfillcolor{currentfill}%
\pgfsetlinewidth{0.602250pt}%
\definecolor{currentstroke}{rgb}{0.000000,0.000000,0.000000}%
\pgfsetstrokecolor{currentstroke}%
\pgfsetdash{}{0pt}%
\pgfsys@defobject{currentmarker}{\pgfqpoint{-0.027778in}{0.000000in}}{\pgfqpoint{-0.000000in}{0.000000in}}{%
\pgfpathmoveto{\pgfqpoint{-0.000000in}{0.000000in}}%
\pgfpathlineto{\pgfqpoint{-0.027778in}{0.000000in}}%
\pgfusepath{stroke,fill}%
}%
\begin{pgfscope}%
\pgfsys@transformshift{0.588387in}{2.321180in}%
\pgfsys@useobject{currentmarker}{}%
\end{pgfscope}%
\end{pgfscope}%
\begin{pgfscope}%
\pgfsetbuttcap%
\pgfsetroundjoin%
\definecolor{currentfill}{rgb}{0.000000,0.000000,0.000000}%
\pgfsetfillcolor{currentfill}%
\pgfsetlinewidth{0.602250pt}%
\definecolor{currentstroke}{rgb}{0.000000,0.000000,0.000000}%
\pgfsetstrokecolor{currentstroke}%
\pgfsetdash{}{0pt}%
\pgfsys@defobject{currentmarker}{\pgfqpoint{-0.027778in}{0.000000in}}{\pgfqpoint{-0.000000in}{0.000000in}}{%
\pgfpathmoveto{\pgfqpoint{-0.000000in}{0.000000in}}%
\pgfpathlineto{\pgfqpoint{-0.027778in}{0.000000in}}%
\pgfusepath{stroke,fill}%
}%
\begin{pgfscope}%
\pgfsys@transformshift{0.588387in}{2.562289in}%
\pgfsys@useobject{currentmarker}{}%
\end{pgfscope}%
\end{pgfscope}%
\begin{pgfscope}%
\pgfsetbuttcap%
\pgfsetroundjoin%
\definecolor{currentfill}{rgb}{0.000000,0.000000,0.000000}%
\pgfsetfillcolor{currentfill}%
\pgfsetlinewidth{0.602250pt}%
\definecolor{currentstroke}{rgb}{0.000000,0.000000,0.000000}%
\pgfsetstrokecolor{currentstroke}%
\pgfsetdash{}{0pt}%
\pgfsys@defobject{currentmarker}{\pgfqpoint{-0.027778in}{0.000000in}}{\pgfqpoint{-0.000000in}{0.000000in}}{%
\pgfpathmoveto{\pgfqpoint{-0.000000in}{0.000000in}}%
\pgfpathlineto{\pgfqpoint{-0.027778in}{0.000000in}}%
\pgfusepath{stroke,fill}%
}%
\begin{pgfscope}%
\pgfsys@transformshift{0.588387in}{2.684720in}%
\pgfsys@useobject{currentmarker}{}%
\end{pgfscope}%
\end{pgfscope}%
\begin{pgfscope}%
\definecolor{textcolor}{rgb}{0.000000,0.000000,0.000000}%
\pgfsetstrokecolor{textcolor}%
\pgfsetfillcolor{textcolor}%
\pgftext[x=0.234413in,y=1.631726in,,bottom,rotate=90.000000]{\color{textcolor}{\rmfamily\fontsize{10.000000}{12.000000}\selectfont\catcode`\^=\active\def^{\ifmmode\sp\else\^{}\fi}\catcode`\%=\active\def%{\%}Time [ms]}}%
\end{pgfscope}%
\begin{pgfscope}%
\pgfpathrectangle{\pgfqpoint{0.588387in}{0.521603in}}{\pgfqpoint{5.314715in}{2.220246in}}%
\pgfusepath{clip}%
\pgfsetrectcap%
\pgfsetroundjoin%
\pgfsetlinewidth{1.505625pt}%
\pgfsetstrokecolor{currentstroke1}%
\pgfsetdash{}{0pt}%
\pgfpathmoveto{\pgfqpoint{0.829965in}{0.633787in}}%
\pgfpathlineto{\pgfqpoint{0.928568in}{0.667555in}}%
\pgfpathlineto{\pgfqpoint{1.027172in}{0.737181in}}%
\pgfpathlineto{\pgfqpoint{1.125775in}{0.808218in}}%
\pgfpathlineto{\pgfqpoint{1.224378in}{0.886269in}}%
\pgfpathlineto{\pgfqpoint{1.322981in}{0.923970in}}%
\pgfpathlineto{\pgfqpoint{1.421585in}{0.991240in}}%
\pgfpathlineto{\pgfqpoint{1.520188in}{1.027757in}}%
\pgfpathlineto{\pgfqpoint{1.618791in}{1.046185in}}%
\pgfpathlineto{\pgfqpoint{1.717394in}{1.130493in}}%
\pgfpathlineto{\pgfqpoint{1.815998in}{1.185590in}}%
\pgfpathlineto{\pgfqpoint{1.914601in}{1.204713in}}%
\pgfpathlineto{\pgfqpoint{2.013204in}{1.224243in}}%
\pgfpathlineto{\pgfqpoint{2.111807in}{1.272173in}}%
\pgfpathlineto{\pgfqpoint{2.210411in}{1.291344in}}%
\pgfpathlineto{\pgfqpoint{2.309014in}{1.333641in}}%
\pgfpathlineto{\pgfqpoint{2.407617in}{1.336935in}}%
\pgfpathlineto{\pgfqpoint{2.506220in}{1.380227in}}%
\pgfpathlineto{\pgfqpoint{2.604824in}{1.405222in}}%
\pgfpathlineto{\pgfqpoint{2.703427in}{1.424176in}}%
\pgfpathlineto{\pgfqpoint{2.802030in}{1.468155in}}%
\pgfpathlineto{\pgfqpoint{2.900633in}{1.478238in}}%
\pgfpathlineto{\pgfqpoint{2.999237in}{1.497928in}}%
\pgfpathlineto{\pgfqpoint{3.097840in}{1.529389in}}%
\pgfpathlineto{\pgfqpoint{3.196443in}{1.550703in}}%
\pgfpathlineto{\pgfqpoint{3.295046in}{1.562053in}}%
\pgfpathlineto{\pgfqpoint{3.393649in}{1.585234in}}%
\pgfpathlineto{\pgfqpoint{3.492253in}{1.604744in}}%
\pgfpathlineto{\pgfqpoint{3.590856in}{1.616435in}}%
\pgfpathlineto{\pgfqpoint{3.689459in}{1.634391in}}%
\pgfpathlineto{\pgfqpoint{3.788062in}{1.651627in}}%
\pgfpathlineto{\pgfqpoint{3.886666in}{1.674965in}}%
\pgfpathlineto{\pgfqpoint{3.985269in}{1.661851in}}%
\pgfpathlineto{\pgfqpoint{4.083872in}{1.728110in}}%
\pgfpathlineto{\pgfqpoint{4.182475in}{1.741987in}}%
\pgfpathlineto{\pgfqpoint{4.281079in}{1.751819in}}%
\pgfpathlineto{\pgfqpoint{4.379682in}{1.747224in}}%
\pgfpathlineto{\pgfqpoint{4.478285in}{1.763874in}}%
\pgfpathlineto{\pgfqpoint{4.576888in}{1.781538in}}%
\pgfpathlineto{\pgfqpoint{4.675492in}{1.780912in}}%
\pgfpathlineto{\pgfqpoint{4.774095in}{1.805717in}}%
\pgfpathlineto{\pgfqpoint{4.872698in}{1.789815in}}%
\pgfpathlineto{\pgfqpoint{4.971301in}{1.828638in}}%
\pgfpathlineto{\pgfqpoint{5.069905in}{1.879229in}}%
\pgfpathlineto{\pgfqpoint{5.168508in}{1.863034in}}%
\pgfpathlineto{\pgfqpoint{5.267111in}{1.882832in}}%
\pgfpathlineto{\pgfqpoint{5.365714in}{1.884440in}}%
\pgfpathlineto{\pgfqpoint{5.464318in}{1.896830in}}%
\pgfpathlineto{\pgfqpoint{5.562921in}{1.912489in}}%
\pgfpathlineto{\pgfqpoint{5.661524in}{1.903426in}}%
\pgfusepath{stroke}%
\end{pgfscope}%
\begin{pgfscope}%
\pgfpathrectangle{\pgfqpoint{0.588387in}{0.521603in}}{\pgfqpoint{5.314715in}{2.220246in}}%
\pgfusepath{clip}%
\pgfsetrectcap%
\pgfsetroundjoin%
\pgfsetlinewidth{1.505625pt}%
\pgfsetstrokecolor{currentstroke2}%
\pgfsetdash{}{0pt}%
\pgfpathmoveto{\pgfqpoint{0.829965in}{0.630032in}}%
\pgfpathlineto{\pgfqpoint{0.928568in}{0.657395in}}%
\pgfpathlineto{\pgfqpoint{1.027172in}{0.773192in}}%
\pgfpathlineto{\pgfqpoint{1.125775in}{0.806415in}}%
\pgfpathlineto{\pgfqpoint{1.224378in}{0.983830in}}%
\pgfpathlineto{\pgfqpoint{1.322981in}{1.008020in}}%
\pgfpathlineto{\pgfqpoint{1.421585in}{1.066263in}}%
\pgfpathlineto{\pgfqpoint{1.520188in}{1.214427in}}%
\pgfpathlineto{\pgfqpoint{1.618791in}{1.175615in}}%
\pgfpathlineto{\pgfqpoint{1.717394in}{1.509857in}}%
\pgfpathlineto{\pgfqpoint{1.815998in}{1.618347in}}%
\pgfpathlineto{\pgfqpoint{1.914601in}{2.080655in}}%
\pgfpathlineto{\pgfqpoint{2.013204in}{2.066985in}}%
\pgfpathlineto{\pgfqpoint{2.111807in}{2.198638in}}%
\pgfpathlineto{\pgfqpoint{2.210411in}{2.166376in}}%
\pgfpathlineto{\pgfqpoint{2.309014in}{2.407693in}}%
\pgfpathlineto{\pgfqpoint{2.407617in}{2.386169in}}%
\pgfpathlineto{\pgfqpoint{2.506220in}{2.467788in}}%
\pgfpathlineto{\pgfqpoint{2.604824in}{2.476133in}}%
\pgfpathlineto{\pgfqpoint{2.703427in}{2.483449in}}%
\pgfpathlineto{\pgfqpoint{2.802030in}{2.541511in}}%
\pgfpathlineto{\pgfqpoint{2.900633in}{2.517537in}}%
\pgfpathlineto{\pgfqpoint{2.999237in}{2.524490in}}%
\pgfpathlineto{\pgfqpoint{3.097840in}{2.578742in}}%
\pgfpathlineto{\pgfqpoint{3.196443in}{2.567528in}}%
\pgfpathlineto{\pgfqpoint{3.393649in}{2.456312in}}%
\pgfpathlineto{\pgfqpoint{3.590856in}{2.640929in}}%
\pgfusepath{stroke}%
\end{pgfscope}%
\begin{pgfscope}%
\pgfpathrectangle{\pgfqpoint{0.588387in}{0.521603in}}{\pgfqpoint{5.314715in}{2.220246in}}%
\pgfusepath{clip}%
\pgfsetrectcap%
\pgfsetroundjoin%
\pgfsetlinewidth{1.505625pt}%
\pgfsetstrokecolor{currentstroke3}%
\pgfsetdash{}{0pt}%
\pgfpathmoveto{\pgfqpoint{0.829965in}{0.731807in}}%
\pgfpathlineto{\pgfqpoint{0.928568in}{0.786750in}}%
\pgfpathlineto{\pgfqpoint{1.027172in}{1.055861in}}%
\pgfpathlineto{\pgfqpoint{1.125775in}{1.209892in}}%
\pgfpathlineto{\pgfqpoint{1.224378in}{1.517149in}}%
\pgfpathlineto{\pgfqpoint{1.322981in}{1.687420in}}%
\pgfpathlineto{\pgfqpoint{1.421585in}{1.859151in}}%
\pgfpathlineto{\pgfqpoint{1.520188in}{1.957233in}}%
\pgfpathlineto{\pgfqpoint{1.618791in}{2.149317in}}%
\pgfpathlineto{\pgfqpoint{1.717394in}{2.307306in}}%
\pgfpathlineto{\pgfqpoint{1.815998in}{2.375952in}}%
\pgfpathlineto{\pgfqpoint{1.914601in}{2.466088in}}%
\pgfpathlineto{\pgfqpoint{2.013204in}{2.552677in}}%
\pgfpathlineto{\pgfqpoint{2.111807in}{2.392160in}}%
\pgfpathlineto{\pgfqpoint{2.210411in}{2.600659in}}%
\pgfusepath{stroke}%
\end{pgfscope}%
\begin{pgfscope}%
\pgfpathrectangle{\pgfqpoint{0.588387in}{0.521603in}}{\pgfqpoint{5.314715in}{2.220246in}}%
\pgfusepath{clip}%
\pgfsetrectcap%
\pgfsetroundjoin%
\pgfsetlinewidth{1.505625pt}%
\pgfsetstrokecolor{currentstroke4}%
\pgfsetdash{}{0pt}%
\pgfpathmoveto{\pgfqpoint{0.829965in}{0.674369in}}%
\pgfpathlineto{\pgfqpoint{0.928568in}{0.747450in}}%
\pgfpathlineto{\pgfqpoint{1.027172in}{0.889728in}}%
\pgfpathlineto{\pgfqpoint{1.125775in}{1.023354in}}%
\pgfpathlineto{\pgfqpoint{1.224378in}{1.206148in}}%
\pgfpathlineto{\pgfqpoint{1.322981in}{1.358026in}}%
\pgfpathlineto{\pgfqpoint{1.421585in}{1.530420in}}%
\pgfpathlineto{\pgfqpoint{1.520188in}{1.716317in}}%
\pgfpathlineto{\pgfqpoint{1.618791in}{1.756260in}}%
\pgfpathlineto{\pgfqpoint{1.717394in}{2.032486in}}%
\pgfpathlineto{\pgfqpoint{1.815998in}{2.282067in}}%
\pgfpathlineto{\pgfqpoint{1.914601in}{2.317242in}}%
\pgfpathlineto{\pgfqpoint{2.013204in}{2.365471in}}%
\pgfpathlineto{\pgfqpoint{2.111807in}{2.424393in}}%
\pgfpathlineto{\pgfqpoint{2.210411in}{2.438389in}}%
\pgfpathlineto{\pgfqpoint{2.309014in}{2.494122in}}%
\pgfpathlineto{\pgfqpoint{2.407617in}{2.372632in}}%
\pgfpathlineto{\pgfqpoint{2.506220in}{2.388236in}}%
\pgfpathlineto{\pgfqpoint{2.604824in}{2.399315in}}%
\pgfpathlineto{\pgfqpoint{2.703427in}{2.579551in}}%
\pgfpathlineto{\pgfqpoint{2.999237in}{2.461204in}}%
\pgfpathlineto{\pgfqpoint{3.097840in}{2.638358in}}%
\pgfusepath{stroke}%
\end{pgfscope}%
\begin{pgfscope}%
\pgfpathrectangle{\pgfqpoint{0.588387in}{0.521603in}}{\pgfqpoint{5.314715in}{2.220246in}}%
\pgfusepath{clip}%
\pgfsetrectcap%
\pgfsetroundjoin%
\pgfsetlinewidth{1.505625pt}%
\pgfsetstrokecolor{currentstroke5}%
\pgfsetdash{}{0pt}%
\pgfpathmoveto{\pgfqpoint{0.829965in}{0.630263in}}%
\pgfpathlineto{\pgfqpoint{0.928568in}{0.670051in}}%
\pgfpathlineto{\pgfqpoint{1.027172in}{0.776753in}}%
\pgfpathlineto{\pgfqpoint{1.125775in}{0.828000in}}%
\pgfpathlineto{\pgfqpoint{1.224378in}{0.933299in}}%
\pgfpathlineto{\pgfqpoint{1.322981in}{0.959114in}}%
\pgfpathlineto{\pgfqpoint{1.421585in}{1.036332in}}%
\pgfpathlineto{\pgfqpoint{1.520188in}{1.092808in}}%
\pgfpathlineto{\pgfqpoint{1.618791in}{1.183101in}}%
\pgfpathlineto{\pgfqpoint{1.717394in}{1.352369in}}%
\pgfpathlineto{\pgfqpoint{1.815998in}{1.273860in}}%
\pgfpathlineto{\pgfqpoint{1.914601in}{1.462235in}}%
\pgfpathlineto{\pgfqpoint{2.013204in}{1.566809in}}%
\pgfpathlineto{\pgfqpoint{2.111807in}{1.624495in}}%
\pgfpathlineto{\pgfqpoint{2.210411in}{1.705811in}}%
\pgfpathlineto{\pgfqpoint{2.309014in}{1.638592in}}%
\pgfpathlineto{\pgfqpoint{2.407617in}{1.674341in}}%
\pgfpathlineto{\pgfqpoint{2.506220in}{1.505457in}}%
\pgfpathlineto{\pgfqpoint{2.604824in}{1.559346in}}%
\pgfpathlineto{\pgfqpoint{2.703427in}{1.565291in}}%
\pgfpathlineto{\pgfqpoint{2.802030in}{1.824756in}}%
\pgfpathlineto{\pgfqpoint{2.900633in}{1.660828in}}%
\pgfpathlineto{\pgfqpoint{2.999237in}{1.822650in}}%
\pgfpathlineto{\pgfqpoint{3.097840in}{1.600289in}}%
\pgfpathlineto{\pgfqpoint{3.196443in}{1.914181in}}%
\pgfpathlineto{\pgfqpoint{3.295046in}{1.863938in}}%
\pgfpathlineto{\pgfqpoint{3.393649in}{1.886039in}}%
\pgfpathlineto{\pgfqpoint{3.492253in}{1.933759in}}%
\pgfpathlineto{\pgfqpoint{3.590856in}{1.992329in}}%
\pgfpathlineto{\pgfqpoint{3.689459in}{1.833471in}}%
\pgfpathlineto{\pgfqpoint{3.788062in}{1.954417in}}%
\pgfpathlineto{\pgfqpoint{3.886666in}{1.780467in}}%
\pgfpathlineto{\pgfqpoint{3.985269in}{1.975515in}}%
\pgfpathlineto{\pgfqpoint{4.083872in}{1.900230in}}%
\pgfpathlineto{\pgfqpoint{4.182475in}{1.985082in}}%
\pgfpathlineto{\pgfqpoint{4.281079in}{2.031012in}}%
\pgfpathlineto{\pgfqpoint{4.379682in}{2.176545in}}%
\pgfpathlineto{\pgfqpoint{4.478285in}{1.972039in}}%
\pgfpathlineto{\pgfqpoint{4.576888in}{1.937924in}}%
\pgfpathlineto{\pgfqpoint{4.675492in}{1.962562in}}%
\pgfpathlineto{\pgfqpoint{4.774095in}{1.933191in}}%
\pgfpathlineto{\pgfqpoint{4.872698in}{1.862076in}}%
\pgfpathlineto{\pgfqpoint{4.971301in}{1.887453in}}%
\pgfpathlineto{\pgfqpoint{5.069905in}{1.949019in}}%
\pgfpathlineto{\pgfqpoint{5.168508in}{1.892649in}}%
\pgfpathlineto{\pgfqpoint{5.267111in}{2.021871in}}%
\pgfpathlineto{\pgfqpoint{5.365714in}{2.032529in}}%
\pgfpathlineto{\pgfqpoint{5.464318in}{2.135211in}}%
\pgfpathlineto{\pgfqpoint{5.562921in}{2.036861in}}%
\pgfpathlineto{\pgfqpoint{5.661524in}{2.080443in}}%
\pgfusepath{stroke}%
\end{pgfscope}%
\begin{pgfscope}%
\pgfpathrectangle{\pgfqpoint{0.588387in}{0.521603in}}{\pgfqpoint{5.314715in}{2.220246in}}%
\pgfusepath{clip}%
\pgfsetrectcap%
\pgfsetroundjoin%
\pgfsetlinewidth{1.505625pt}%
\pgfsetstrokecolor{currentstroke6}%
\pgfsetdash{}{0pt}%
\pgfpathmoveto{\pgfqpoint{0.829965in}{0.622524in}}%
\pgfpathlineto{\pgfqpoint{0.928568in}{0.632506in}}%
\pgfpathlineto{\pgfqpoint{1.027172in}{0.739658in}}%
\pgfpathlineto{\pgfqpoint{1.125775in}{0.760055in}}%
\pgfpathlineto{\pgfqpoint{1.224378in}{0.886577in}}%
\pgfpathlineto{\pgfqpoint{1.322981in}{0.910686in}}%
\pgfpathlineto{\pgfqpoint{1.421585in}{0.983885in}}%
\pgfpathlineto{\pgfqpoint{1.520188in}{1.005755in}}%
\pgfpathlineto{\pgfqpoint{1.618791in}{1.052717in}}%
\pgfpathlineto{\pgfqpoint{1.717394in}{1.122685in}}%
\pgfpathlineto{\pgfqpoint{1.815998in}{1.127093in}}%
\pgfpathlineto{\pgfqpoint{1.914601in}{1.164113in}}%
\pgfpathlineto{\pgfqpoint{2.013204in}{1.205342in}}%
\pgfpathlineto{\pgfqpoint{2.111807in}{1.236440in}}%
\pgfpathlineto{\pgfqpoint{2.210411in}{1.241713in}}%
\pgfpathlineto{\pgfqpoint{2.309014in}{1.277302in}}%
\pgfpathlineto{\pgfqpoint{2.407617in}{1.302365in}}%
\pgfpathlineto{\pgfqpoint{2.506220in}{1.346431in}}%
\pgfpathlineto{\pgfqpoint{2.604824in}{1.394886in}}%
\pgfpathlineto{\pgfqpoint{2.703427in}{1.367827in}}%
\pgfpathlineto{\pgfqpoint{2.802030in}{1.408850in}}%
\pgfpathlineto{\pgfqpoint{2.900633in}{1.449938in}}%
\pgfpathlineto{\pgfqpoint{2.999237in}{1.454411in}}%
\pgfpathlineto{\pgfqpoint{3.097840in}{1.478580in}}%
\pgfpathlineto{\pgfqpoint{3.196443in}{1.494402in}}%
\pgfpathlineto{\pgfqpoint{3.295046in}{1.535900in}}%
\pgfpathlineto{\pgfqpoint{3.393649in}{1.538438in}}%
\pgfpathlineto{\pgfqpoint{3.492253in}{1.541910in}}%
\pgfpathlineto{\pgfqpoint{3.590856in}{1.587245in}}%
\pgfpathlineto{\pgfqpoint{3.689459in}{1.558536in}}%
\pgfpathlineto{\pgfqpoint{3.788062in}{1.611747in}}%
\pgfpathlineto{\pgfqpoint{3.886666in}{1.608655in}}%
\pgfpathlineto{\pgfqpoint{3.985269in}{1.621083in}}%
\pgfpathlineto{\pgfqpoint{4.083872in}{1.670150in}}%
\pgfpathlineto{\pgfqpoint{4.182475in}{1.700977in}}%
\pgfpathlineto{\pgfqpoint{4.281079in}{1.701239in}}%
\pgfpathlineto{\pgfqpoint{4.379682in}{1.709829in}}%
\pgfpathlineto{\pgfqpoint{4.478285in}{1.716668in}}%
\pgfpathlineto{\pgfqpoint{4.576888in}{1.778138in}}%
\pgfpathlineto{\pgfqpoint{4.675492in}{1.724299in}}%
\pgfpathlineto{\pgfqpoint{4.774095in}{1.767831in}}%
\pgfpathlineto{\pgfqpoint{4.872698in}{1.760760in}}%
\pgfpathlineto{\pgfqpoint{4.971301in}{1.784157in}}%
\pgfpathlineto{\pgfqpoint{5.069905in}{1.819722in}}%
\pgfpathlineto{\pgfqpoint{5.168508in}{1.746099in}}%
\pgfpathlineto{\pgfqpoint{5.365714in}{2.018233in}}%
\pgfpathlineto{\pgfqpoint{5.464318in}{1.849942in}}%
\pgfpathlineto{\pgfqpoint{5.562921in}{1.829281in}}%
\pgfpathlineto{\pgfqpoint{5.661524in}{1.835210in}}%
\pgfusepath{stroke}%
\end{pgfscope}%
\begin{pgfscope}%
\pgfsetrectcap%
\pgfsetmiterjoin%
\pgfsetlinewidth{0.803000pt}%
\definecolor{currentstroke}{rgb}{0.000000,0.000000,0.000000}%
\pgfsetstrokecolor{currentstroke}%
\pgfsetdash{}{0pt}%
\pgfpathmoveto{\pgfqpoint{0.588387in}{0.521603in}}%
\pgfpathlineto{\pgfqpoint{0.588387in}{2.741849in}}%
\pgfusepath{stroke}%
\end{pgfscope}%
\begin{pgfscope}%
\pgfsetrectcap%
\pgfsetmiterjoin%
\pgfsetlinewidth{0.803000pt}%
\definecolor{currentstroke}{rgb}{0.000000,0.000000,0.000000}%
\pgfsetstrokecolor{currentstroke}%
\pgfsetdash{}{0pt}%
\pgfpathmoveto{\pgfqpoint{5.903102in}{0.521603in}}%
\pgfpathlineto{\pgfqpoint{5.903102in}{2.741849in}}%
\pgfusepath{stroke}%
\end{pgfscope}%
\begin{pgfscope}%
\pgfsetrectcap%
\pgfsetmiterjoin%
\pgfsetlinewidth{0.803000pt}%
\definecolor{currentstroke}{rgb}{0.000000,0.000000,0.000000}%
\pgfsetstrokecolor{currentstroke}%
\pgfsetdash{}{0pt}%
\pgfpathmoveto{\pgfqpoint{0.588387in}{0.521603in}}%
\pgfpathlineto{\pgfqpoint{5.903102in}{0.521603in}}%
\pgfusepath{stroke}%
\end{pgfscope}%
\begin{pgfscope}%
\pgfsetrectcap%
\pgfsetmiterjoin%
\pgfsetlinewidth{0.803000pt}%
\definecolor{currentstroke}{rgb}{0.000000,0.000000,0.000000}%
\pgfsetstrokecolor{currentstroke}%
\pgfsetdash{}{0pt}%
\pgfpathmoveto{\pgfqpoint{0.588387in}{2.741849in}}%
\pgfpathlineto{\pgfqpoint{5.903102in}{2.741849in}}%
\pgfusepath{stroke}%
\end{pgfscope}%
\begin{pgfscope}%
\pgfsetbuttcap%
\pgfsetmiterjoin%
\definecolor{currentfill}{rgb}{1.000000,1.000000,1.000000}%
\pgfsetfillcolor{currentfill}%
\pgfsetfillopacity{0.800000}%
\pgfsetlinewidth{1.003750pt}%
\definecolor{currentstroke}{rgb}{0.800000,0.800000,0.800000}%
\pgfsetstrokecolor{currentstroke}%
\pgfsetstrokeopacity{0.800000}%
\pgfsetdash{}{0pt}%
\pgfpathmoveto{\pgfqpoint{5.990602in}{1.527105in}}%
\pgfpathlineto{\pgfqpoint{8.259376in}{1.527105in}}%
\pgfpathquadraticcurveto{\pgfqpoint{8.284376in}{1.527105in}}{\pgfqpoint{8.284376in}{1.552105in}}%
\pgfpathlineto{\pgfqpoint{8.284376in}{2.654349in}}%
\pgfpathquadraticcurveto{\pgfqpoint{8.284376in}{2.679349in}}{\pgfqpoint{8.259376in}{2.679349in}}%
\pgfpathlineto{\pgfqpoint{5.990602in}{2.679349in}}%
\pgfpathquadraticcurveto{\pgfqpoint{5.965602in}{2.679349in}}{\pgfqpoint{5.965602in}{2.654349in}}%
\pgfpathlineto{\pgfqpoint{5.965602in}{1.552105in}}%
\pgfpathquadraticcurveto{\pgfqpoint{5.965602in}{1.527105in}}{\pgfqpoint{5.990602in}{1.527105in}}%
\pgfpathlineto{\pgfqpoint{5.990602in}{1.527105in}}%
\pgfpathclose%
\pgfusepath{stroke,fill}%
\end{pgfscope}%
\begin{pgfscope}%
\pgfsetrectcap%
\pgfsetroundjoin%
\pgfsetlinewidth{1.505625pt}%
\pgfsetstrokecolor{currentstroke1}%
\pgfsetdash{}{0pt}%
\pgfpathmoveto{\pgfqpoint{6.015602in}{2.578129in}}%
\pgfpathlineto{\pgfqpoint{6.140602in}{2.578129in}}%
\pgfpathlineto{\pgfqpoint{6.265602in}{2.578129in}}%
\pgfusepath{stroke}%
\end{pgfscope}%
\begin{pgfscope}%
\definecolor{textcolor}{rgb}{0.000000,0.000000,0.000000}%
\pgfsetstrokecolor{textcolor}%
\pgfsetfillcolor{textcolor}%
\pgftext[x=6.365602in,y=2.534379in,left,base]{\color{textcolor}{\rmfamily\fontsize{9.000000}{10.800000}\selectfont\catcode`\^=\active\def^{\ifmmode\sp\else\^{}\fi}\catcode`\%=\active\def%{\%}\Neighbors{} \& \MergeLinear{}}}%
\end{pgfscope}%
\begin{pgfscope}%
\pgfsetrectcap%
\pgfsetroundjoin%
\pgfsetlinewidth{1.505625pt}%
\pgfsetstrokecolor{currentstroke2}%
\pgfsetdash{}{0pt}%
\pgfpathmoveto{\pgfqpoint{6.015602in}{2.394657in}}%
\pgfpathlineto{\pgfqpoint{6.140602in}{2.394657in}}%
\pgfpathlineto{\pgfqpoint{6.265602in}{2.394657in}}%
\pgfusepath{stroke}%
\end{pgfscope}%
\begin{pgfscope}%
\definecolor{textcolor}{rgb}{0.000000,0.000000,0.000000}%
\pgfsetstrokecolor{textcolor}%
\pgfsetfillcolor{textcolor}%
\pgftext[x=6.365602in,y=2.350907in,left,base]{\color{textcolor}{\rmfamily\fontsize{9.000000}{10.800000}\selectfont\catcode`\^=\active\def^{\ifmmode\sp\else\^{}\fi}\catcode`\%=\active\def%{\%}\Neighbors{} \& \Log{}}}%
\end{pgfscope}%
\begin{pgfscope}%
\pgfsetrectcap%
\pgfsetroundjoin%
\pgfsetlinewidth{1.505625pt}%
\pgfsetstrokecolor{currentstroke3}%
\pgfsetdash{}{0pt}%
\pgfpathmoveto{\pgfqpoint{6.015602in}{2.211185in}}%
\pgfpathlineto{\pgfqpoint{6.140602in}{2.211185in}}%
\pgfpathlineto{\pgfqpoint{6.265602in}{2.211185in}}%
\pgfusepath{stroke}%
\end{pgfscope}%
\begin{pgfscope}%
\definecolor{textcolor}{rgb}{0.000000,0.000000,0.000000}%
\pgfsetstrokecolor{textcolor}%
\pgfsetfillcolor{textcolor}%
\pgftext[x=6.365602in,y=2.167435in,left,base]{\color{textcolor}{\rmfamily\fontsize{9.000000}{10.800000}\selectfont\catcode`\^=\active\def^{\ifmmode\sp\else\^{}\fi}\catcode`\%=\active\def%{\%}\Neighbors{} \& \MinMax{}}}%
\end{pgfscope}%
\begin{pgfscope}%
\pgfsetrectcap%
\pgfsetroundjoin%
\pgfsetlinewidth{1.505625pt}%
\pgfsetstrokecolor{currentstroke4}%
\pgfsetdash{}{0pt}%
\pgfpathmoveto{\pgfqpoint{6.015602in}{2.024235in}}%
\pgfpathlineto{\pgfqpoint{6.140602in}{2.024235in}}%
\pgfpathlineto{\pgfqpoint{6.265602in}{2.024235in}}%
\pgfusepath{stroke}%
\end{pgfscope}%
\begin{pgfscope}%
\definecolor{textcolor}{rgb}{0.000000,0.000000,0.000000}%
\pgfsetstrokecolor{textcolor}%
\pgfsetfillcolor{textcolor}%
\pgftext[x=6.365602in,y=1.980485in,left,base]{\color{textcolor}{\rmfamily\fontsize{9.000000}{10.800000}\selectfont\catcode`\^=\active\def^{\ifmmode\sp\else\^{}\fi}\catcode`\%=\active\def%{\%}\Neighbors{} \& \PromisingCycles{}}}%
\end{pgfscope}%
\begin{pgfscope}%
\pgfsetrectcap%
\pgfsetroundjoin%
\pgfsetlinewidth{1.505625pt}%
\pgfsetstrokecolor{currentstroke5}%
\pgfsetdash{}{0pt}%
\pgfpathmoveto{\pgfqpoint{6.015602in}{1.837285in}}%
\pgfpathlineto{\pgfqpoint{6.140602in}{1.837285in}}%
\pgfpathlineto{\pgfqpoint{6.265602in}{1.837285in}}%
\pgfusepath{stroke}%
\end{pgfscope}%
\begin{pgfscope}%
\definecolor{textcolor}{rgb}{0.000000,0.000000,0.000000}%
\pgfsetstrokecolor{textcolor}%
\pgfsetfillcolor{textcolor}%
\pgftext[x=6.365602in,y=1.793535in,left,base]{\color{textcolor}{\rmfamily\fontsize{9.000000}{10.800000}\selectfont\catcode`\^=\active\def^{\ifmmode\sp\else\^{}\fi}\catcode`\%=\active\def%{\%}\Neighbors{} \& \SharedVertices{}}}%
\end{pgfscope}%
\begin{pgfscope}%
\pgfsetrectcap%
\pgfsetroundjoin%
\pgfsetlinewidth{1.505625pt}%
\pgfsetstrokecolor{currentstroke6}%
\pgfsetdash{}{0pt}%
\pgfpathmoveto{\pgfqpoint{6.015602in}{1.650334in}}%
\pgfpathlineto{\pgfqpoint{6.140602in}{1.650334in}}%
\pgfpathlineto{\pgfqpoint{6.265602in}{1.650334in}}%
\pgfusepath{stroke}%
\end{pgfscope}%
\begin{pgfscope}%
\definecolor{textcolor}{rgb}{0.000000,0.000000,0.000000}%
\pgfsetstrokecolor{textcolor}%
\pgfsetfillcolor{textcolor}%
\pgftext[x=6.365602in,y=1.606584in,left,base]{\color{textcolor}{\rmfamily\fontsize{9.000000}{10.800000}\selectfont\catcode`\^=\active\def^{\ifmmode\sp\else\^{}\fi}\catcode`\%=\active\def%{\%}\Neighbors{} \& \SortedBits{}}}%
\end{pgfscope}%
\end{pgfpicture}%
\makeatother%
\endgroup%
}
	\caption[Other merging strategies for minimally rigid graphs]{
		Mean running time to find some NAC-coloring for minimally rigid graphs with other merging strategies.}%
	\label{fig:graph_mimimally_rigid_failing_merging_first_runtime}
\end{figure}%
\begin{figure}[thbp]
	\centering
	\scalebox{\BenchFigureScale}{%% Creator: Matplotlib, PGF backend
%%
%% To include the figure in your LaTeX document, write
%%   \input{<filename>.pgf}
%%
%% Make sure the required packages are loaded in your preamble
%%   \usepackage{pgf}
%%
%% Also ensure that all the required font packages are loaded; for instance,
%% the lmodern package is sometimes necessary when using math font.
%%   \usepackage{lmodern}
%%
%% Figures using additional raster images can only be included by \input if
%% they are in the same directory as the main LaTeX file. For loading figures
%% from other directories you can use the `import` package
%%   \usepackage{import}
%%
%% and then include the figures with
%%   \import{<path to file>}{<filename>.pgf}
%%
%% Matplotlib used the following preamble
%%   \def\mathdefault#1{#1}
%%   \everymath=\expandafter{\the\everymath\displaystyle}
%%   \IfFileExists{scrextend.sty}{
%%     \usepackage[fontsize=10.000000pt]{scrextend}
%%   }{
%%     \renewcommand{\normalsize}{\fontsize{10.000000}{12.000000}\selectfont}
%%     \normalsize
%%   }
%%   
%%   \ifdefined\pdftexversion\else  % non-pdftex case.
%%     \usepackage{fontspec}
%%     \setmainfont{DejaVuSans.ttf}[Path=\detokenize{/home/petr/Projects/PyRigi/.venv/lib/python3.12/site-packages/matplotlib/mpl-data/fonts/ttf/}]
%%     \setsansfont{DejaVuSans.ttf}[Path=\detokenize{/home/petr/Projects/PyRigi/.venv/lib/python3.12/site-packages/matplotlib/mpl-data/fonts/ttf/}]
%%     \setmonofont{DejaVuSansMono.ttf}[Path=\detokenize{/home/petr/Projects/PyRigi/.venv/lib/python3.12/site-packages/matplotlib/mpl-data/fonts/ttf/}]
%%   \fi
%%   \makeatletter\@ifpackageloaded{under\Score{}}{}{\usepackage[strings]{under\Score{}}}\makeatother
%%
\begingroup%
\makeatletter%
\begin{pgfpicture}%
\pgfpathrectangle{\pgfpointorigin}{\pgfqpoint{8.384376in}{2.841849in}}%
\pgfusepath{use as bounding box, clip}%
\begin{pgfscope}%
\pgfsetbuttcap%
\pgfsetmiterjoin%
\definecolor{currentfill}{rgb}{1.000000,1.000000,1.000000}%
\pgfsetfillcolor{currentfill}%
\pgfsetlinewidth{0.000000pt}%
\definecolor{currentstroke}{rgb}{1.000000,1.000000,1.000000}%
\pgfsetstrokecolor{currentstroke}%
\pgfsetdash{}{0pt}%
\pgfpathmoveto{\pgfqpoint{0.000000in}{0.000000in}}%
\pgfpathlineto{\pgfqpoint{8.384376in}{0.000000in}}%
\pgfpathlineto{\pgfqpoint{8.384376in}{2.841849in}}%
\pgfpathlineto{\pgfqpoint{0.000000in}{2.841849in}}%
\pgfpathlineto{\pgfqpoint{0.000000in}{0.000000in}}%
\pgfpathclose%
\pgfusepath{fill}%
\end{pgfscope}%
\begin{pgfscope}%
\pgfsetbuttcap%
\pgfsetmiterjoin%
\definecolor{currentfill}{rgb}{1.000000,1.000000,1.000000}%
\pgfsetfillcolor{currentfill}%
\pgfsetlinewidth{0.000000pt}%
\definecolor{currentstroke}{rgb}{0.000000,0.000000,0.000000}%
\pgfsetstrokecolor{currentstroke}%
\pgfsetstrokeopacity{0.000000}%
\pgfsetdash{}{0pt}%
\pgfpathmoveto{\pgfqpoint{0.588387in}{0.521603in}}%
\pgfpathlineto{\pgfqpoint{5.888942in}{0.521603in}}%
\pgfpathlineto{\pgfqpoint{5.888942in}{2.741849in}}%
\pgfpathlineto{\pgfqpoint{0.588387in}{2.741849in}}%
\pgfpathlineto{\pgfqpoint{0.588387in}{0.521603in}}%
\pgfpathclose%
\pgfusepath{fill}%
\end{pgfscope}%
\begin{pgfscope}%
\pgfsetbuttcap%
\pgfsetroundjoin%
\definecolor{currentfill}{rgb}{0.000000,0.000000,0.000000}%
\pgfsetfillcolor{currentfill}%
\pgfsetlinewidth{0.803000pt}%
\definecolor{currentstroke}{rgb}{0.000000,0.000000,0.000000}%
\pgfsetstrokecolor{currentstroke}%
\pgfsetdash{}{0pt}%
\pgfsys@defobject{currentmarker}{\pgfqpoint{0.000000in}{-0.048611in}}{\pgfqpoint{0.000000in}{0.000000in}}{%
\pgfpathmoveto{\pgfqpoint{0.000000in}{0.000000in}}%
\pgfpathlineto{\pgfqpoint{0.000000in}{-0.048611in}}%
\pgfusepath{stroke,fill}%
}%
\begin{pgfscope}%
\pgfsys@transformshift{1.026003in}{0.521603in}%
\pgfsys@useobject{currentmarker}{}%
\end{pgfscope}%
\end{pgfscope}%
\begin{pgfscope}%
\definecolor{textcolor}{rgb}{0.000000,0.000000,0.000000}%
\pgfsetstrokecolor{textcolor}%
\pgfsetfillcolor{textcolor}%
\pgftext[x=1.026003in,y=0.424381in,,top]{\color{textcolor}{\rmfamily\fontsize{10.000000}{12.000000}\selectfont\catcode`\^=\active\def^{\ifmmode\sp\else\^{}\fi}\catcode`\%=\active\def%{\%}$\mathdefault{12}$}}%
\end{pgfscope}%
\begin{pgfscope}%
\pgfsetbuttcap%
\pgfsetroundjoin%
\definecolor{currentfill}{rgb}{0.000000,0.000000,0.000000}%
\pgfsetfillcolor{currentfill}%
\pgfsetlinewidth{0.803000pt}%
\definecolor{currentstroke}{rgb}{0.000000,0.000000,0.000000}%
\pgfsetstrokecolor{currentstroke}%
\pgfsetdash{}{0pt}%
\pgfsys@defobject{currentmarker}{\pgfqpoint{0.000000in}{-0.048611in}}{\pgfqpoint{0.000000in}{0.000000in}}{%
\pgfpathmoveto{\pgfqpoint{0.000000in}{0.000000in}}%
\pgfpathlineto{\pgfqpoint{0.000000in}{-0.048611in}}%
\pgfusepath{stroke,fill}%
}%
\begin{pgfscope}%
\pgfsys@transformshift{1.616046in}{0.521603in}%
\pgfsys@useobject{currentmarker}{}%
\end{pgfscope}%
\end{pgfscope}%
\begin{pgfscope}%
\definecolor{textcolor}{rgb}{0.000000,0.000000,0.000000}%
\pgfsetstrokecolor{textcolor}%
\pgfsetfillcolor{textcolor}%
\pgftext[x=1.616046in,y=0.424381in,,top]{\color{textcolor}{\rmfamily\fontsize{10.000000}{12.000000}\selectfont\catcode`\^=\active\def^{\ifmmode\sp\else\^{}\fi}\catcode`\%=\active\def%{\%}$\mathdefault{18}$}}%
\end{pgfscope}%
\begin{pgfscope}%
\pgfsetbuttcap%
\pgfsetroundjoin%
\definecolor{currentfill}{rgb}{0.000000,0.000000,0.000000}%
\pgfsetfillcolor{currentfill}%
\pgfsetlinewidth{0.803000pt}%
\definecolor{currentstroke}{rgb}{0.000000,0.000000,0.000000}%
\pgfsetstrokecolor{currentstroke}%
\pgfsetdash{}{0pt}%
\pgfsys@defobject{currentmarker}{\pgfqpoint{0.000000in}{-0.048611in}}{\pgfqpoint{0.000000in}{0.000000in}}{%
\pgfpathmoveto{\pgfqpoint{0.000000in}{0.000000in}}%
\pgfpathlineto{\pgfqpoint{0.000000in}{-0.048611in}}%
\pgfusepath{stroke,fill}%
}%
\begin{pgfscope}%
\pgfsys@transformshift{2.206089in}{0.521603in}%
\pgfsys@useobject{currentmarker}{}%
\end{pgfscope}%
\end{pgfscope}%
\begin{pgfscope}%
\definecolor{textcolor}{rgb}{0.000000,0.000000,0.000000}%
\pgfsetstrokecolor{textcolor}%
\pgfsetfillcolor{textcolor}%
\pgftext[x=2.206089in,y=0.424381in,,top]{\color{textcolor}{\rmfamily\fontsize{10.000000}{12.000000}\selectfont\catcode`\^=\active\def^{\ifmmode\sp\else\^{}\fi}\catcode`\%=\active\def%{\%}$\mathdefault{24}$}}%
\end{pgfscope}%
\begin{pgfscope}%
\pgfsetbuttcap%
\pgfsetroundjoin%
\definecolor{currentfill}{rgb}{0.000000,0.000000,0.000000}%
\pgfsetfillcolor{currentfill}%
\pgfsetlinewidth{0.803000pt}%
\definecolor{currentstroke}{rgb}{0.000000,0.000000,0.000000}%
\pgfsetstrokecolor{currentstroke}%
\pgfsetdash{}{0pt}%
\pgfsys@defobject{currentmarker}{\pgfqpoint{0.000000in}{-0.048611in}}{\pgfqpoint{0.000000in}{0.000000in}}{%
\pgfpathmoveto{\pgfqpoint{0.000000in}{0.000000in}}%
\pgfpathlineto{\pgfqpoint{0.000000in}{-0.048611in}}%
\pgfusepath{stroke,fill}%
}%
\begin{pgfscope}%
\pgfsys@transformshift{2.796132in}{0.521603in}%
\pgfsys@useobject{currentmarker}{}%
\end{pgfscope}%
\end{pgfscope}%
\begin{pgfscope}%
\definecolor{textcolor}{rgb}{0.000000,0.000000,0.000000}%
\pgfsetstrokecolor{textcolor}%
\pgfsetfillcolor{textcolor}%
\pgftext[x=2.796132in,y=0.424381in,,top]{\color{textcolor}{\rmfamily\fontsize{10.000000}{12.000000}\selectfont\catcode`\^=\active\def^{\ifmmode\sp\else\^{}\fi}\catcode`\%=\active\def%{\%}$\mathdefault{30}$}}%
\end{pgfscope}%
\begin{pgfscope}%
\pgfsetbuttcap%
\pgfsetroundjoin%
\definecolor{currentfill}{rgb}{0.000000,0.000000,0.000000}%
\pgfsetfillcolor{currentfill}%
\pgfsetlinewidth{0.803000pt}%
\definecolor{currentstroke}{rgb}{0.000000,0.000000,0.000000}%
\pgfsetstrokecolor{currentstroke}%
\pgfsetdash{}{0pt}%
\pgfsys@defobject{currentmarker}{\pgfqpoint{0.000000in}{-0.048611in}}{\pgfqpoint{0.000000in}{0.000000in}}{%
\pgfpathmoveto{\pgfqpoint{0.000000in}{0.000000in}}%
\pgfpathlineto{\pgfqpoint{0.000000in}{-0.048611in}}%
\pgfusepath{stroke,fill}%
}%
\begin{pgfscope}%
\pgfsys@transformshift{3.386175in}{0.521603in}%
\pgfsys@useobject{currentmarker}{}%
\end{pgfscope}%
\end{pgfscope}%
\begin{pgfscope}%
\definecolor{textcolor}{rgb}{0.000000,0.000000,0.000000}%
\pgfsetstrokecolor{textcolor}%
\pgfsetfillcolor{textcolor}%
\pgftext[x=3.386175in,y=0.424381in,,top]{\color{textcolor}{\rmfamily\fontsize{10.000000}{12.000000}\selectfont\catcode`\^=\active\def^{\ifmmode\sp\else\^{}\fi}\catcode`\%=\active\def%{\%}$\mathdefault{36}$}}%
\end{pgfscope}%
\begin{pgfscope}%
\pgfsetbuttcap%
\pgfsetroundjoin%
\definecolor{currentfill}{rgb}{0.000000,0.000000,0.000000}%
\pgfsetfillcolor{currentfill}%
\pgfsetlinewidth{0.803000pt}%
\definecolor{currentstroke}{rgb}{0.000000,0.000000,0.000000}%
\pgfsetstrokecolor{currentstroke}%
\pgfsetdash{}{0pt}%
\pgfsys@defobject{currentmarker}{\pgfqpoint{0.000000in}{-0.048611in}}{\pgfqpoint{0.000000in}{0.000000in}}{%
\pgfpathmoveto{\pgfqpoint{0.000000in}{0.000000in}}%
\pgfpathlineto{\pgfqpoint{0.000000in}{-0.048611in}}%
\pgfusepath{stroke,fill}%
}%
\begin{pgfscope}%
\pgfsys@transformshift{3.976219in}{0.521603in}%
\pgfsys@useobject{currentmarker}{}%
\end{pgfscope}%
\end{pgfscope}%
\begin{pgfscope}%
\definecolor{textcolor}{rgb}{0.000000,0.000000,0.000000}%
\pgfsetstrokecolor{textcolor}%
\pgfsetfillcolor{textcolor}%
\pgftext[x=3.976219in,y=0.424381in,,top]{\color{textcolor}{\rmfamily\fontsize{10.000000}{12.000000}\selectfont\catcode`\^=\active\def^{\ifmmode\sp\else\^{}\fi}\catcode`\%=\active\def%{\%}$\mathdefault{42}$}}%
\end{pgfscope}%
\begin{pgfscope}%
\pgfsetbuttcap%
\pgfsetroundjoin%
\definecolor{currentfill}{rgb}{0.000000,0.000000,0.000000}%
\pgfsetfillcolor{currentfill}%
\pgfsetlinewidth{0.803000pt}%
\definecolor{currentstroke}{rgb}{0.000000,0.000000,0.000000}%
\pgfsetstrokecolor{currentstroke}%
\pgfsetdash{}{0pt}%
\pgfsys@defobject{currentmarker}{\pgfqpoint{0.000000in}{-0.048611in}}{\pgfqpoint{0.000000in}{0.000000in}}{%
\pgfpathmoveto{\pgfqpoint{0.000000in}{0.000000in}}%
\pgfpathlineto{\pgfqpoint{0.000000in}{-0.048611in}}%
\pgfusepath{stroke,fill}%
}%
\begin{pgfscope}%
\pgfsys@transformshift{4.566262in}{0.521603in}%
\pgfsys@useobject{currentmarker}{}%
\end{pgfscope}%
\end{pgfscope}%
\begin{pgfscope}%
\definecolor{textcolor}{rgb}{0.000000,0.000000,0.000000}%
\pgfsetstrokecolor{textcolor}%
\pgfsetfillcolor{textcolor}%
\pgftext[x=4.566262in,y=0.424381in,,top]{\color{textcolor}{\rmfamily\fontsize{10.000000}{12.000000}\selectfont\catcode`\^=\active\def^{\ifmmode\sp\else\^{}\fi}\catcode`\%=\active\def%{\%}$\mathdefault{48}$}}%
\end{pgfscope}%
\begin{pgfscope}%
\pgfsetbuttcap%
\pgfsetroundjoin%
\definecolor{currentfill}{rgb}{0.000000,0.000000,0.000000}%
\pgfsetfillcolor{currentfill}%
\pgfsetlinewidth{0.803000pt}%
\definecolor{currentstroke}{rgb}{0.000000,0.000000,0.000000}%
\pgfsetstrokecolor{currentstroke}%
\pgfsetdash{}{0pt}%
\pgfsys@defobject{currentmarker}{\pgfqpoint{0.000000in}{-0.048611in}}{\pgfqpoint{0.000000in}{0.000000in}}{%
\pgfpathmoveto{\pgfqpoint{0.000000in}{0.000000in}}%
\pgfpathlineto{\pgfqpoint{0.000000in}{-0.048611in}}%
\pgfusepath{stroke,fill}%
}%
\begin{pgfscope}%
\pgfsys@transformshift{5.156305in}{0.521603in}%
\pgfsys@useobject{currentmarker}{}%
\end{pgfscope}%
\end{pgfscope}%
\begin{pgfscope}%
\definecolor{textcolor}{rgb}{0.000000,0.000000,0.000000}%
\pgfsetstrokecolor{textcolor}%
\pgfsetfillcolor{textcolor}%
\pgftext[x=5.156305in,y=0.424381in,,top]{\color{textcolor}{\rmfamily\fontsize{10.000000}{12.000000}\selectfont\catcode`\^=\active\def^{\ifmmode\sp\else\^{}\fi}\catcode`\%=\active\def%{\%}$\mathdefault{54}$}}%
\end{pgfscope}%
\begin{pgfscope}%
\pgfsetbuttcap%
\pgfsetroundjoin%
\definecolor{currentfill}{rgb}{0.000000,0.000000,0.000000}%
\pgfsetfillcolor{currentfill}%
\pgfsetlinewidth{0.803000pt}%
\definecolor{currentstroke}{rgb}{0.000000,0.000000,0.000000}%
\pgfsetstrokecolor{currentstroke}%
\pgfsetdash{}{0pt}%
\pgfsys@defobject{currentmarker}{\pgfqpoint{0.000000in}{-0.048611in}}{\pgfqpoint{0.000000in}{0.000000in}}{%
\pgfpathmoveto{\pgfqpoint{0.000000in}{0.000000in}}%
\pgfpathlineto{\pgfqpoint{0.000000in}{-0.048611in}}%
\pgfusepath{stroke,fill}%
}%
\begin{pgfscope}%
\pgfsys@transformshift{5.746348in}{0.521603in}%
\pgfsys@useobject{currentmarker}{}%
\end{pgfscope}%
\end{pgfscope}%
\begin{pgfscope}%
\definecolor{textcolor}{rgb}{0.000000,0.000000,0.000000}%
\pgfsetstrokecolor{textcolor}%
\pgfsetfillcolor{textcolor}%
\pgftext[x=5.746348in,y=0.424381in,,top]{\color{textcolor}{\rmfamily\fontsize{10.000000}{12.000000}\selectfont\catcode`\^=\active\def^{\ifmmode\sp\else\^{}\fi}\catcode`\%=\active\def%{\%}$\mathdefault{60}$}}%
\end{pgfscope}%
\begin{pgfscope}%
\definecolor{textcolor}{rgb}{0.000000,0.000000,0.000000}%
\pgfsetstrokecolor{textcolor}%
\pgfsetfillcolor{textcolor}%
\pgftext[x=3.238665in,y=0.234413in,,top]{\color{textcolor}{\rmfamily\fontsize{10.000000}{12.000000}\selectfont\catcode`\^=\active\def^{\ifmmode\sp\else\^{}\fi}\catcode`\%=\active\def%{\%}Vertices}}%
\end{pgfscope}%
\begin{pgfscope}%
\pgfsetbuttcap%
\pgfsetroundjoin%
\definecolor{currentfill}{rgb}{0.000000,0.000000,0.000000}%
\pgfsetfillcolor{currentfill}%
\pgfsetlinewidth{0.803000pt}%
\definecolor{currentstroke}{rgb}{0.000000,0.000000,0.000000}%
\pgfsetstrokecolor{currentstroke}%
\pgfsetdash{}{0pt}%
\pgfsys@defobject{currentmarker}{\pgfqpoint{-0.048611in}{0.000000in}}{\pgfqpoint{-0.000000in}{0.000000in}}{%
\pgfpathmoveto{\pgfqpoint{-0.000000in}{0.000000in}}%
\pgfpathlineto{\pgfqpoint{-0.048611in}{0.000000in}}%
\pgfusepath{stroke,fill}%
}%
\begin{pgfscope}%
\pgfsys@transformshift{0.588387in}{1.145119in}%
\pgfsys@useobject{currentmarker}{}%
\end{pgfscope}%
\end{pgfscope}%
\begin{pgfscope}%
\definecolor{textcolor}{rgb}{0.000000,0.000000,0.000000}%
\pgfsetstrokecolor{textcolor}%
\pgfsetfillcolor{textcolor}%
\pgftext[x=0.289968in, y=1.092357in, left, base]{\color{textcolor}{\rmfamily\fontsize{10.000000}{12.000000}\selectfont\catcode`\^=\active\def^{\ifmmode\sp\else\^{}\fi}\catcode`\%=\active\def%{\%}$\mathdefault{10^{1}}$}}%
\end{pgfscope}%
\begin{pgfscope}%
\pgfsetbuttcap%
\pgfsetroundjoin%
\definecolor{currentfill}{rgb}{0.000000,0.000000,0.000000}%
\pgfsetfillcolor{currentfill}%
\pgfsetlinewidth{0.803000pt}%
\definecolor{currentstroke}{rgb}{0.000000,0.000000,0.000000}%
\pgfsetstrokecolor{currentstroke}%
\pgfsetdash{}{0pt}%
\pgfsys@defobject{currentmarker}{\pgfqpoint{-0.048611in}{0.000000in}}{\pgfqpoint{-0.000000in}{0.000000in}}{%
\pgfpathmoveto{\pgfqpoint{-0.000000in}{0.000000in}}%
\pgfpathlineto{\pgfqpoint{-0.048611in}{0.000000in}}%
\pgfusepath{stroke,fill}%
}%
\begin{pgfscope}%
\pgfsys@transformshift{0.588387in}{2.279973in}%
\pgfsys@useobject{currentmarker}{}%
\end{pgfscope}%
\end{pgfscope}%
\begin{pgfscope}%
\definecolor{textcolor}{rgb}{0.000000,0.000000,0.000000}%
\pgfsetstrokecolor{textcolor}%
\pgfsetfillcolor{textcolor}%
\pgftext[x=0.289968in, y=2.227212in, left, base]{\color{textcolor}{\rmfamily\fontsize{10.000000}{12.000000}\selectfont\catcode`\^=\active\def^{\ifmmode\sp\else\^{}\fi}\catcode`\%=\active\def%{\%}$\mathdefault{10^{2}}$}}%
\end{pgfscope}%
\begin{pgfscope}%
\pgfsetbuttcap%
\pgfsetroundjoin%
\definecolor{currentfill}{rgb}{0.000000,0.000000,0.000000}%
\pgfsetfillcolor{currentfill}%
\pgfsetlinewidth{0.602250pt}%
\definecolor{currentstroke}{rgb}{0.000000,0.000000,0.000000}%
\pgfsetstrokecolor{currentstroke}%
\pgfsetdash{}{0pt}%
\pgfsys@defobject{currentmarker}{\pgfqpoint{-0.027778in}{0.000000in}}{\pgfqpoint{-0.000000in}{0.000000in}}{%
\pgfpathmoveto{\pgfqpoint{-0.000000in}{0.000000in}}%
\pgfpathlineto{\pgfqpoint{-0.027778in}{0.000000in}}%
\pgfusepath{stroke,fill}%
}%
\begin{pgfscope}%
\pgfsys@transformshift{0.588387in}{0.551728in}%
\pgfsys@useobject{currentmarker}{}%
\end{pgfscope}%
\end{pgfscope}%
\begin{pgfscope}%
\pgfsetbuttcap%
\pgfsetroundjoin%
\definecolor{currentfill}{rgb}{0.000000,0.000000,0.000000}%
\pgfsetfillcolor{currentfill}%
\pgfsetlinewidth{0.602250pt}%
\definecolor{currentstroke}{rgb}{0.000000,0.000000,0.000000}%
\pgfsetstrokecolor{currentstroke}%
\pgfsetdash{}{0pt}%
\pgfsys@defobject{currentmarker}{\pgfqpoint{-0.027778in}{0.000000in}}{\pgfqpoint{-0.000000in}{0.000000in}}{%
\pgfpathmoveto{\pgfqpoint{-0.000000in}{0.000000in}}%
\pgfpathlineto{\pgfqpoint{-0.027778in}{0.000000in}}%
\pgfusepath{stroke,fill}%
}%
\begin{pgfscope}%
\pgfsys@transformshift{0.588387in}{0.693515in}%
\pgfsys@useobject{currentmarker}{}%
\end{pgfscope}%
\end{pgfscope}%
\begin{pgfscope}%
\pgfsetbuttcap%
\pgfsetroundjoin%
\definecolor{currentfill}{rgb}{0.000000,0.000000,0.000000}%
\pgfsetfillcolor{currentfill}%
\pgfsetlinewidth{0.602250pt}%
\definecolor{currentstroke}{rgb}{0.000000,0.000000,0.000000}%
\pgfsetstrokecolor{currentstroke}%
\pgfsetdash{}{0pt}%
\pgfsys@defobject{currentmarker}{\pgfqpoint{-0.027778in}{0.000000in}}{\pgfqpoint{-0.000000in}{0.000000in}}{%
\pgfpathmoveto{\pgfqpoint{-0.000000in}{0.000000in}}%
\pgfpathlineto{\pgfqpoint{-0.027778in}{0.000000in}}%
\pgfusepath{stroke,fill}%
}%
\begin{pgfscope}%
\pgfsys@transformshift{0.588387in}{0.803494in}%
\pgfsys@useobject{currentmarker}{}%
\end{pgfscope}%
\end{pgfscope}%
\begin{pgfscope}%
\pgfsetbuttcap%
\pgfsetroundjoin%
\definecolor{currentfill}{rgb}{0.000000,0.000000,0.000000}%
\pgfsetfillcolor{currentfill}%
\pgfsetlinewidth{0.602250pt}%
\definecolor{currentstroke}{rgb}{0.000000,0.000000,0.000000}%
\pgfsetstrokecolor{currentstroke}%
\pgfsetdash{}{0pt}%
\pgfsys@defobject{currentmarker}{\pgfqpoint{-0.027778in}{0.000000in}}{\pgfqpoint{-0.000000in}{0.000000in}}{%
\pgfpathmoveto{\pgfqpoint{-0.000000in}{0.000000in}}%
\pgfpathlineto{\pgfqpoint{-0.027778in}{0.000000in}}%
\pgfusepath{stroke,fill}%
}%
\begin{pgfscope}%
\pgfsys@transformshift{0.588387in}{0.893353in}%
\pgfsys@useobject{currentmarker}{}%
\end{pgfscope}%
\end{pgfscope}%
\begin{pgfscope}%
\pgfsetbuttcap%
\pgfsetroundjoin%
\definecolor{currentfill}{rgb}{0.000000,0.000000,0.000000}%
\pgfsetfillcolor{currentfill}%
\pgfsetlinewidth{0.602250pt}%
\definecolor{currentstroke}{rgb}{0.000000,0.000000,0.000000}%
\pgfsetstrokecolor{currentstroke}%
\pgfsetdash{}{0pt}%
\pgfsys@defobject{currentmarker}{\pgfqpoint{-0.027778in}{0.000000in}}{\pgfqpoint{-0.000000in}{0.000000in}}{%
\pgfpathmoveto{\pgfqpoint{-0.000000in}{0.000000in}}%
\pgfpathlineto{\pgfqpoint{-0.027778in}{0.000000in}}%
\pgfusepath{stroke,fill}%
}%
\begin{pgfscope}%
\pgfsys@transformshift{0.588387in}{0.969328in}%
\pgfsys@useobject{currentmarker}{}%
\end{pgfscope}%
\end{pgfscope}%
\begin{pgfscope}%
\pgfsetbuttcap%
\pgfsetroundjoin%
\definecolor{currentfill}{rgb}{0.000000,0.000000,0.000000}%
\pgfsetfillcolor{currentfill}%
\pgfsetlinewidth{0.602250pt}%
\definecolor{currentstroke}{rgb}{0.000000,0.000000,0.000000}%
\pgfsetstrokecolor{currentstroke}%
\pgfsetdash{}{0pt}%
\pgfsys@defobject{currentmarker}{\pgfqpoint{-0.027778in}{0.000000in}}{\pgfqpoint{-0.000000in}{0.000000in}}{%
\pgfpathmoveto{\pgfqpoint{-0.000000in}{0.000000in}}%
\pgfpathlineto{\pgfqpoint{-0.027778in}{0.000000in}}%
\pgfusepath{stroke,fill}%
}%
\begin{pgfscope}%
\pgfsys@transformshift{0.588387in}{1.035140in}%
\pgfsys@useobject{currentmarker}{}%
\end{pgfscope}%
\end{pgfscope}%
\begin{pgfscope}%
\pgfsetbuttcap%
\pgfsetroundjoin%
\definecolor{currentfill}{rgb}{0.000000,0.000000,0.000000}%
\pgfsetfillcolor{currentfill}%
\pgfsetlinewidth{0.602250pt}%
\definecolor{currentstroke}{rgb}{0.000000,0.000000,0.000000}%
\pgfsetstrokecolor{currentstroke}%
\pgfsetdash{}{0pt}%
\pgfsys@defobject{currentmarker}{\pgfqpoint{-0.027778in}{0.000000in}}{\pgfqpoint{-0.000000in}{0.000000in}}{%
\pgfpathmoveto{\pgfqpoint{-0.000000in}{0.000000in}}%
\pgfpathlineto{\pgfqpoint{-0.027778in}{0.000000in}}%
\pgfusepath{stroke,fill}%
}%
\begin{pgfscope}%
\pgfsys@transformshift{0.588387in}{1.093191in}%
\pgfsys@useobject{currentmarker}{}%
\end{pgfscope}%
\end{pgfscope}%
\begin{pgfscope}%
\pgfsetbuttcap%
\pgfsetroundjoin%
\definecolor{currentfill}{rgb}{0.000000,0.000000,0.000000}%
\pgfsetfillcolor{currentfill}%
\pgfsetlinewidth{0.602250pt}%
\definecolor{currentstroke}{rgb}{0.000000,0.000000,0.000000}%
\pgfsetstrokecolor{currentstroke}%
\pgfsetdash{}{0pt}%
\pgfsys@defobject{currentmarker}{\pgfqpoint{-0.027778in}{0.000000in}}{\pgfqpoint{-0.000000in}{0.000000in}}{%
\pgfpathmoveto{\pgfqpoint{-0.000000in}{0.000000in}}%
\pgfpathlineto{\pgfqpoint{-0.027778in}{0.000000in}}%
\pgfusepath{stroke,fill}%
}%
\begin{pgfscope}%
\pgfsys@transformshift{0.588387in}{1.486744in}%
\pgfsys@useobject{currentmarker}{}%
\end{pgfscope}%
\end{pgfscope}%
\begin{pgfscope}%
\pgfsetbuttcap%
\pgfsetroundjoin%
\definecolor{currentfill}{rgb}{0.000000,0.000000,0.000000}%
\pgfsetfillcolor{currentfill}%
\pgfsetlinewidth{0.602250pt}%
\definecolor{currentstroke}{rgb}{0.000000,0.000000,0.000000}%
\pgfsetstrokecolor{currentstroke}%
\pgfsetdash{}{0pt}%
\pgfsys@defobject{currentmarker}{\pgfqpoint{-0.027778in}{0.000000in}}{\pgfqpoint{-0.000000in}{0.000000in}}{%
\pgfpathmoveto{\pgfqpoint{-0.000000in}{0.000000in}}%
\pgfpathlineto{\pgfqpoint{-0.027778in}{0.000000in}}%
\pgfusepath{stroke,fill}%
}%
\begin{pgfscope}%
\pgfsys@transformshift{0.588387in}{1.686582in}%
\pgfsys@useobject{currentmarker}{}%
\end{pgfscope}%
\end{pgfscope}%
\begin{pgfscope}%
\pgfsetbuttcap%
\pgfsetroundjoin%
\definecolor{currentfill}{rgb}{0.000000,0.000000,0.000000}%
\pgfsetfillcolor{currentfill}%
\pgfsetlinewidth{0.602250pt}%
\definecolor{currentstroke}{rgb}{0.000000,0.000000,0.000000}%
\pgfsetstrokecolor{currentstroke}%
\pgfsetdash{}{0pt}%
\pgfsys@defobject{currentmarker}{\pgfqpoint{-0.027778in}{0.000000in}}{\pgfqpoint{-0.000000in}{0.000000in}}{%
\pgfpathmoveto{\pgfqpoint{-0.000000in}{0.000000in}}%
\pgfpathlineto{\pgfqpoint{-0.027778in}{0.000000in}}%
\pgfusepath{stroke,fill}%
}%
\begin{pgfscope}%
\pgfsys@transformshift{0.588387in}{1.828369in}%
\pgfsys@useobject{currentmarker}{}%
\end{pgfscope}%
\end{pgfscope}%
\begin{pgfscope}%
\pgfsetbuttcap%
\pgfsetroundjoin%
\definecolor{currentfill}{rgb}{0.000000,0.000000,0.000000}%
\pgfsetfillcolor{currentfill}%
\pgfsetlinewidth{0.602250pt}%
\definecolor{currentstroke}{rgb}{0.000000,0.000000,0.000000}%
\pgfsetstrokecolor{currentstroke}%
\pgfsetdash{}{0pt}%
\pgfsys@defobject{currentmarker}{\pgfqpoint{-0.027778in}{0.000000in}}{\pgfqpoint{-0.000000in}{0.000000in}}{%
\pgfpathmoveto{\pgfqpoint{-0.000000in}{0.000000in}}%
\pgfpathlineto{\pgfqpoint{-0.027778in}{0.000000in}}%
\pgfusepath{stroke,fill}%
}%
\begin{pgfscope}%
\pgfsys@transformshift{0.588387in}{1.938348in}%
\pgfsys@useobject{currentmarker}{}%
\end{pgfscope}%
\end{pgfscope}%
\begin{pgfscope}%
\pgfsetbuttcap%
\pgfsetroundjoin%
\definecolor{currentfill}{rgb}{0.000000,0.000000,0.000000}%
\pgfsetfillcolor{currentfill}%
\pgfsetlinewidth{0.602250pt}%
\definecolor{currentstroke}{rgb}{0.000000,0.000000,0.000000}%
\pgfsetstrokecolor{currentstroke}%
\pgfsetdash{}{0pt}%
\pgfsys@defobject{currentmarker}{\pgfqpoint{-0.027778in}{0.000000in}}{\pgfqpoint{-0.000000in}{0.000000in}}{%
\pgfpathmoveto{\pgfqpoint{-0.000000in}{0.000000in}}%
\pgfpathlineto{\pgfqpoint{-0.027778in}{0.000000in}}%
\pgfusepath{stroke,fill}%
}%
\begin{pgfscope}%
\pgfsys@transformshift{0.588387in}{2.028207in}%
\pgfsys@useobject{currentmarker}{}%
\end{pgfscope}%
\end{pgfscope}%
\begin{pgfscope}%
\pgfsetbuttcap%
\pgfsetroundjoin%
\definecolor{currentfill}{rgb}{0.000000,0.000000,0.000000}%
\pgfsetfillcolor{currentfill}%
\pgfsetlinewidth{0.602250pt}%
\definecolor{currentstroke}{rgb}{0.000000,0.000000,0.000000}%
\pgfsetstrokecolor{currentstroke}%
\pgfsetdash{}{0pt}%
\pgfsys@defobject{currentmarker}{\pgfqpoint{-0.027778in}{0.000000in}}{\pgfqpoint{-0.000000in}{0.000000in}}{%
\pgfpathmoveto{\pgfqpoint{-0.000000in}{0.000000in}}%
\pgfpathlineto{\pgfqpoint{-0.027778in}{0.000000in}}%
\pgfusepath{stroke,fill}%
}%
\begin{pgfscope}%
\pgfsys@transformshift{0.588387in}{2.104182in}%
\pgfsys@useobject{currentmarker}{}%
\end{pgfscope}%
\end{pgfscope}%
\begin{pgfscope}%
\pgfsetbuttcap%
\pgfsetroundjoin%
\definecolor{currentfill}{rgb}{0.000000,0.000000,0.000000}%
\pgfsetfillcolor{currentfill}%
\pgfsetlinewidth{0.602250pt}%
\definecolor{currentstroke}{rgb}{0.000000,0.000000,0.000000}%
\pgfsetstrokecolor{currentstroke}%
\pgfsetdash{}{0pt}%
\pgfsys@defobject{currentmarker}{\pgfqpoint{-0.027778in}{0.000000in}}{\pgfqpoint{-0.000000in}{0.000000in}}{%
\pgfpathmoveto{\pgfqpoint{-0.000000in}{0.000000in}}%
\pgfpathlineto{\pgfqpoint{-0.027778in}{0.000000in}}%
\pgfusepath{stroke,fill}%
}%
\begin{pgfscope}%
\pgfsys@transformshift{0.588387in}{2.169995in}%
\pgfsys@useobject{currentmarker}{}%
\end{pgfscope}%
\end{pgfscope}%
\begin{pgfscope}%
\pgfsetbuttcap%
\pgfsetroundjoin%
\definecolor{currentfill}{rgb}{0.000000,0.000000,0.000000}%
\pgfsetfillcolor{currentfill}%
\pgfsetlinewidth{0.602250pt}%
\definecolor{currentstroke}{rgb}{0.000000,0.000000,0.000000}%
\pgfsetstrokecolor{currentstroke}%
\pgfsetdash{}{0pt}%
\pgfsys@defobject{currentmarker}{\pgfqpoint{-0.027778in}{0.000000in}}{\pgfqpoint{-0.000000in}{0.000000in}}{%
\pgfpathmoveto{\pgfqpoint{-0.000000in}{0.000000in}}%
\pgfpathlineto{\pgfqpoint{-0.027778in}{0.000000in}}%
\pgfusepath{stroke,fill}%
}%
\begin{pgfscope}%
\pgfsys@transformshift{0.588387in}{2.228045in}%
\pgfsys@useobject{currentmarker}{}%
\end{pgfscope}%
\end{pgfscope}%
\begin{pgfscope}%
\pgfsetbuttcap%
\pgfsetroundjoin%
\definecolor{currentfill}{rgb}{0.000000,0.000000,0.000000}%
\pgfsetfillcolor{currentfill}%
\pgfsetlinewidth{0.602250pt}%
\definecolor{currentstroke}{rgb}{0.000000,0.000000,0.000000}%
\pgfsetstrokecolor{currentstroke}%
\pgfsetdash{}{0pt}%
\pgfsys@defobject{currentmarker}{\pgfqpoint{-0.027778in}{0.000000in}}{\pgfqpoint{-0.000000in}{0.000000in}}{%
\pgfpathmoveto{\pgfqpoint{-0.000000in}{0.000000in}}%
\pgfpathlineto{\pgfqpoint{-0.027778in}{0.000000in}}%
\pgfusepath{stroke,fill}%
}%
\begin{pgfscope}%
\pgfsys@transformshift{0.588387in}{2.621599in}%
\pgfsys@useobject{currentmarker}{}%
\end{pgfscope}%
\end{pgfscope}%
\begin{pgfscope}%
\definecolor{textcolor}{rgb}{0.000000,0.000000,0.000000}%
\pgfsetstrokecolor{textcolor}%
\pgfsetfillcolor{textcolor}%
\pgftext[x=0.234413in,y=1.631726in,,bottom,rotate=90.000000]{\color{textcolor}{\rmfamily\fontsize{10.000000}{12.000000}\selectfont\catcode`\^=\active\def^{\ifmmode\sp\else\^{}\fi}\catcode`\%=\active\def%{\%}Time [ms]}}%
\end{pgfscope}%
\begin{pgfscope}%
\pgfpathrectangle{\pgfqpoint{0.588387in}{0.521603in}}{\pgfqpoint{5.300555in}{2.220246in}}%
\pgfusepath{clip}%
\pgfsetrectcap%
\pgfsetroundjoin%
\pgfsetlinewidth{1.505625pt}%
\pgfsetstrokecolor{currentstroke1}%
\pgfsetdash{}{0pt}%
\pgfpathmoveto{\pgfqpoint{0.829322in}{0.629445in}}%
\pgfpathlineto{\pgfqpoint{0.927662in}{0.688678in}}%
\pgfpathlineto{\pgfqpoint{1.026003in}{0.814912in}}%
\pgfpathlineto{\pgfqpoint{1.124343in}{0.932210in}}%
\pgfpathlineto{\pgfqpoint{1.222684in}{1.049595in}}%
\pgfpathlineto{\pgfqpoint{1.321024in}{1.110976in}}%
\pgfpathlineto{\pgfqpoint{1.419365in}{1.214845in}}%
\pgfpathlineto{\pgfqpoint{1.517705in}{1.279144in}}%
\pgfpathlineto{\pgfqpoint{1.616046in}{1.305699in}}%
\pgfpathlineto{\pgfqpoint{1.714386in}{1.417620in}}%
\pgfpathlineto{\pgfqpoint{1.812727in}{1.503029in}}%
\pgfpathlineto{\pgfqpoint{1.911067in}{1.558728in}}%
\pgfpathlineto{\pgfqpoint{2.009408in}{1.599488in}}%
\pgfpathlineto{\pgfqpoint{2.107748in}{1.666142in}}%
\pgfpathlineto{\pgfqpoint{2.206089in}{1.713068in}}%
\pgfpathlineto{\pgfqpoint{2.304430in}{1.752779in}}%
\pgfpathlineto{\pgfqpoint{2.402770in}{1.782817in}}%
\pgfpathlineto{\pgfqpoint{2.501111in}{1.830483in}}%
\pgfpathlineto{\pgfqpoint{2.599451in}{1.884329in}}%
\pgfpathlineto{\pgfqpoint{2.697792in}{1.920310in}}%
\pgfpathlineto{\pgfqpoint{2.796132in}{1.987558in}}%
\pgfpathlineto{\pgfqpoint{2.894473in}{1.987279in}}%
\pgfpathlineto{\pgfqpoint{2.992813in}{2.035343in}}%
\pgfpathlineto{\pgfqpoint{3.091154in}{2.089442in}}%
\pgfpathlineto{\pgfqpoint{3.189494in}{2.130737in}}%
\pgfpathlineto{\pgfqpoint{3.287835in}{2.143294in}}%
\pgfpathlineto{\pgfqpoint{3.386175in}{2.177489in}}%
\pgfpathlineto{\pgfqpoint{3.484516in}{2.215918in}}%
\pgfpathlineto{\pgfqpoint{3.582856in}{2.241882in}}%
\pgfpathlineto{\pgfqpoint{3.681197in}{2.240549in}}%
\pgfpathlineto{\pgfqpoint{3.779537in}{2.280897in}}%
\pgfpathlineto{\pgfqpoint{3.877878in}{2.313320in}}%
\pgfpathlineto{\pgfqpoint{3.976219in}{2.315904in}}%
\pgfpathlineto{\pgfqpoint{4.074559in}{2.415561in}}%
\pgfpathlineto{\pgfqpoint{4.172900in}{2.443381in}}%
\pgfpathlineto{\pgfqpoint{4.271240in}{2.464164in}}%
\pgfpathlineto{\pgfqpoint{4.369581in}{2.456473in}}%
\pgfpathlineto{\pgfqpoint{4.467921in}{2.482678in}}%
\pgfpathlineto{\pgfqpoint{4.566262in}{2.499536in}}%
\pgfpathlineto{\pgfqpoint{4.664602in}{2.511620in}}%
\pgfpathlineto{\pgfqpoint{4.762943in}{2.540670in}}%
\pgfpathlineto{\pgfqpoint{4.861283in}{2.534565in}}%
\pgfpathlineto{\pgfqpoint{4.959624in}{2.571379in}}%
\pgfpathlineto{\pgfqpoint{5.057964in}{2.622214in}}%
\pgfpathlineto{\pgfqpoint{5.156305in}{2.583174in}}%
\pgfpathlineto{\pgfqpoint{5.254645in}{2.584062in}}%
\pgfpathlineto{\pgfqpoint{5.352986in}{2.612144in}}%
\pgfpathlineto{\pgfqpoint{5.451327in}{2.590228in}}%
\pgfpathlineto{\pgfqpoint{5.549667in}{2.634969in}}%
\pgfpathlineto{\pgfqpoint{5.648008in}{2.614150in}}%
\pgfusepath{stroke}%
\end{pgfscope}%
\begin{pgfscope}%
\pgfpathrectangle{\pgfqpoint{0.588387in}{0.521603in}}{\pgfqpoint{5.300555in}{2.220246in}}%
\pgfusepath{clip}%
\pgfsetrectcap%
\pgfsetroundjoin%
\pgfsetlinewidth{1.505625pt}%
\pgfsetstrokecolor{currentstroke2}%
\pgfsetdash{}{0pt}%
\pgfpathmoveto{\pgfqpoint{0.829322in}{0.646210in}}%
\pgfpathlineto{\pgfqpoint{0.927662in}{0.683558in}}%
\pgfpathlineto{\pgfqpoint{1.026003in}{0.795732in}}%
\pgfpathlineto{\pgfqpoint{1.124343in}{0.904144in}}%
\pgfpathlineto{\pgfqpoint{1.222684in}{1.040490in}}%
\pgfpathlineto{\pgfqpoint{1.321024in}{1.081409in}}%
\pgfpathlineto{\pgfqpoint{1.419365in}{1.183194in}}%
\pgfpathlineto{\pgfqpoint{1.517705in}{1.257249in}}%
\pgfpathlineto{\pgfqpoint{1.616046in}{1.284397in}}%
\pgfpathlineto{\pgfqpoint{1.714386in}{1.400262in}}%
\pgfpathlineto{\pgfqpoint{1.812727in}{1.474266in}}%
\pgfpathlineto{\pgfqpoint{1.911067in}{1.522388in}}%
\pgfpathlineto{\pgfqpoint{2.009408in}{1.561948in}}%
\pgfpathlineto{\pgfqpoint{2.107748in}{1.625441in}}%
\pgfpathlineto{\pgfqpoint{2.206089in}{1.662381in}}%
\pgfpathlineto{\pgfqpoint{2.304430in}{1.716945in}}%
\pgfpathlineto{\pgfqpoint{2.402770in}{1.739818in}}%
\pgfpathlineto{\pgfqpoint{2.501111in}{1.809732in}}%
\pgfpathlineto{\pgfqpoint{2.599451in}{1.846042in}}%
\pgfpathlineto{\pgfqpoint{2.697792in}{1.888810in}}%
\pgfpathlineto{\pgfqpoint{2.796132in}{1.965611in}}%
\pgfpathlineto{\pgfqpoint{2.894473in}{1.969097in}}%
\pgfpathlineto{\pgfqpoint{2.992813in}{2.025376in}}%
\pgfpathlineto{\pgfqpoint{3.091154in}{2.084200in}}%
\pgfpathlineto{\pgfqpoint{3.189494in}{2.097760in}}%
\pgfpathlineto{\pgfqpoint{3.287835in}{2.136334in}}%
\pgfpathlineto{\pgfqpoint{3.386175in}{2.152156in}}%
\pgfpathlineto{\pgfqpoint{3.484516in}{2.187210in}}%
\pgfpathlineto{\pgfqpoint{3.582856in}{2.200779in}}%
\pgfpathlineto{\pgfqpoint{3.681197in}{2.213104in}}%
\pgfpathlineto{\pgfqpoint{3.779537in}{2.241549in}}%
\pgfpathlineto{\pgfqpoint{3.877878in}{2.293173in}}%
\pgfpathlineto{\pgfqpoint{3.976219in}{2.280897in}}%
\pgfpathlineto{\pgfqpoint{4.074559in}{2.357091in}}%
\pgfpathlineto{\pgfqpoint{4.172900in}{2.377727in}}%
\pgfpathlineto{\pgfqpoint{4.271240in}{2.385993in}}%
\pgfpathlineto{\pgfqpoint{4.369581in}{2.397047in}}%
\pgfpathlineto{\pgfqpoint{4.467921in}{2.408808in}}%
\pgfpathlineto{\pgfqpoint{4.566262in}{2.439608in}}%
\pgfpathlineto{\pgfqpoint{4.664602in}{2.440276in}}%
\pgfpathlineto{\pgfqpoint{4.762943in}{2.501112in}}%
\pgfpathlineto{\pgfqpoint{4.861283in}{2.447314in}}%
\pgfpathlineto{\pgfqpoint{4.959624in}{2.506199in}}%
\pgfpathlineto{\pgfqpoint{5.057964in}{2.601479in}}%
\pgfpathlineto{\pgfqpoint{5.156305in}{2.539563in}}%
\pgfpathlineto{\pgfqpoint{5.254645in}{2.584770in}}%
\pgfpathlineto{\pgfqpoint{5.352986in}{2.587595in}}%
\pgfpathlineto{\pgfqpoint{5.451327in}{2.570582in}}%
\pgfpathlineto{\pgfqpoint{5.549667in}{2.640929in}}%
\pgfpathlineto{\pgfqpoint{5.648008in}{2.597613in}}%
\pgfusepath{stroke}%
\end{pgfscope}%
\begin{pgfscope}%
\pgfpathrectangle{\pgfqpoint{0.588387in}{0.521603in}}{\pgfqpoint{5.300555in}{2.220246in}}%
\pgfusepath{clip}%
\pgfsetrectcap%
\pgfsetroundjoin%
\pgfsetlinewidth{1.505625pt}%
\pgfsetstrokecolor{currentstroke3}%
\pgfsetdash{}{0pt}%
\pgfpathmoveto{\pgfqpoint{0.829322in}{0.622524in}}%
\pgfpathlineto{\pgfqpoint{0.927662in}{0.663756in}}%
\pgfpathlineto{\pgfqpoint{1.026003in}{0.777402in}}%
\pgfpathlineto{\pgfqpoint{1.124343in}{0.893353in}}%
\pgfpathlineto{\pgfqpoint{1.222684in}{1.020753in}}%
\pgfpathlineto{\pgfqpoint{1.321024in}{1.082291in}}%
\pgfpathlineto{\pgfqpoint{1.419365in}{1.192094in}}%
\pgfpathlineto{\pgfqpoint{1.517705in}{1.251698in}}%
\pgfpathlineto{\pgfqpoint{1.616046in}{1.281778in}}%
\pgfpathlineto{\pgfqpoint{1.714386in}{1.420051in}}%
\pgfpathlineto{\pgfqpoint{1.812727in}{1.509322in}}%
\pgfpathlineto{\pgfqpoint{1.911067in}{1.540537in}}%
\pgfpathlineto{\pgfqpoint{2.009408in}{1.572414in}}%
\pgfpathlineto{\pgfqpoint{2.107748in}{1.650649in}}%
\pgfpathlineto{\pgfqpoint{2.206089in}{1.681940in}}%
\pgfpathlineto{\pgfqpoint{2.304430in}{1.750981in}}%
\pgfpathlineto{\pgfqpoint{2.402770in}{1.756357in}}%
\pgfpathlineto{\pgfqpoint{2.501111in}{1.827020in}}%
\pgfpathlineto{\pgfqpoint{2.599451in}{1.867819in}}%
\pgfpathlineto{\pgfqpoint{2.697792in}{1.898757in}}%
\pgfpathlineto{\pgfqpoint{2.796132in}{1.970542in}}%
\pgfpathlineto{\pgfqpoint{2.894473in}{1.987000in}}%
\pgfpathlineto{\pgfqpoint{2.992813in}{2.019140in}}%
\pgfpathlineto{\pgfqpoint{3.091154in}{2.070493in}}%
\pgfpathlineto{\pgfqpoint{3.189494in}{2.105281in}}%
\pgfpathlineto{\pgfqpoint{3.287835in}{2.123809in}}%
\pgfpathlineto{\pgfqpoint{3.386175in}{2.161646in}}%
\pgfpathlineto{\pgfqpoint{3.484516in}{2.193491in}}%
\pgfpathlineto{\pgfqpoint{3.582856in}{2.212574in}}%
\pgfpathlineto{\pgfqpoint{3.681197in}{2.241882in}}%
\pgfpathlineto{\pgfqpoint{3.779537in}{2.270016in}}%
\pgfpathlineto{\pgfqpoint{3.877878in}{2.308110in}}%
\pgfpathlineto{\pgfqpoint{3.976219in}{2.286704in}}%
\pgfpathlineto{\pgfqpoint{4.074559in}{2.394856in}}%
\pgfpathlineto{\pgfqpoint{4.172900in}{2.417507in}}%
\pgfpathlineto{\pgfqpoint{4.271240in}{2.433555in}}%
\pgfpathlineto{\pgfqpoint{4.369581in}{2.426054in}}%
\pgfpathlineto{\pgfqpoint{4.467921in}{2.453232in}}%
\pgfpathlineto{\pgfqpoint{4.566262in}{2.482065in}}%
\pgfpathlineto{\pgfqpoint{4.664602in}{2.481042in}}%
\pgfpathlineto{\pgfqpoint{4.762943in}{2.521531in}}%
\pgfpathlineto{\pgfqpoint{4.861283in}{2.495575in}}%
\pgfpathlineto{\pgfqpoint{4.959624in}{2.558944in}}%
\pgfpathlineto{\pgfqpoint{5.057964in}{2.605315in}}%
\pgfpathlineto{\pgfqpoint{5.156305in}{2.534933in}}%
\pgfpathlineto{\pgfqpoint{5.254645in}{2.579833in}}%
\pgfpathlineto{\pgfqpoint{5.352986in}{2.604550in}}%
\pgfpathlineto{\pgfqpoint{5.451327in}{2.593717in}}%
\pgfpathlineto{\pgfqpoint{5.549667in}{2.636167in}}%
\pgfpathlineto{\pgfqpoint{5.648008in}{2.601479in}}%
\pgfusepath{stroke}%
\end{pgfscope}%
\begin{pgfscope}%
\pgfpathrectangle{\pgfqpoint{0.588387in}{0.521603in}}{\pgfqpoint{5.300555in}{2.220246in}}%
\pgfusepath{clip}%
\pgfsetrectcap%
\pgfsetroundjoin%
\pgfsetlinewidth{1.505625pt}%
\pgfsetstrokecolor{currentstroke4}%
\pgfsetdash{}{0pt}%
\pgfpathmoveto{\pgfqpoint{0.829322in}{0.627130in}}%
\pgfpathlineto{\pgfqpoint{0.927662in}{0.660679in}}%
\pgfpathlineto{\pgfqpoint{1.026003in}{0.791016in}}%
\pgfpathlineto{\pgfqpoint{1.124343in}{0.891424in}}%
\pgfpathlineto{\pgfqpoint{1.222684in}{1.025761in}}%
\pgfpathlineto{\pgfqpoint{1.321024in}{1.087553in}}%
\pgfpathlineto{\pgfqpoint{1.419365in}{1.184629in}}%
\pgfpathlineto{\pgfqpoint{1.517705in}{1.251387in}}%
\pgfpathlineto{\pgfqpoint{1.616046in}{1.284107in}}%
\pgfpathlineto{\pgfqpoint{1.714386in}{1.393330in}}%
\pgfpathlineto{\pgfqpoint{1.812727in}{1.481324in}}%
\pgfpathlineto{\pgfqpoint{1.911067in}{1.523865in}}%
\pgfpathlineto{\pgfqpoint{2.009408in}{1.572738in}}%
\pgfpathlineto{\pgfqpoint{2.107748in}{1.640608in}}%
\pgfpathlineto{\pgfqpoint{2.206089in}{1.689653in}}%
\pgfpathlineto{\pgfqpoint{2.304430in}{1.722465in}}%
\pgfpathlineto{\pgfqpoint{2.402770in}{1.769548in}}%
\pgfpathlineto{\pgfqpoint{2.501111in}{1.809131in}}%
\pgfpathlineto{\pgfqpoint{2.599451in}{1.876055in}}%
\pgfpathlineto{\pgfqpoint{2.697792in}{1.912906in}}%
\pgfpathlineto{\pgfqpoint{2.796132in}{1.958566in}}%
\pgfpathlineto{\pgfqpoint{2.894473in}{1.973132in}}%
\pgfpathlineto{\pgfqpoint{2.992813in}{2.006427in}}%
\pgfpathlineto{\pgfqpoint{3.091154in}{2.065758in}}%
\pgfpathlineto{\pgfqpoint{3.189494in}{2.080058in}}%
\pgfpathlineto{\pgfqpoint{3.287835in}{2.114201in}}%
\pgfpathlineto{\pgfqpoint{3.386175in}{2.144106in}}%
\pgfpathlineto{\pgfqpoint{3.484516in}{2.200598in}}%
\pgfpathlineto{\pgfqpoint{3.582856in}{2.219588in}}%
\pgfpathlineto{\pgfqpoint{3.681197in}{2.209030in}}%
\pgfpathlineto{\pgfqpoint{3.779537in}{2.260815in}}%
\pgfpathlineto{\pgfqpoint{3.877878in}{2.301078in}}%
\pgfpathlineto{\pgfqpoint{3.976219in}{2.273150in}}%
\pgfpathlineto{\pgfqpoint{4.074559in}{2.380246in}}%
\pgfpathlineto{\pgfqpoint{4.172900in}{2.411174in}}%
\pgfpathlineto{\pgfqpoint{4.271240in}{2.416807in}}%
\pgfpathlineto{\pgfqpoint{4.369581in}{2.408334in}}%
\pgfpathlineto{\pgfqpoint{4.467921in}{2.446247in}}%
\pgfpathlineto{\pgfqpoint{4.566262in}{2.470273in}}%
\pgfpathlineto{\pgfqpoint{4.664602in}{2.463952in}}%
\pgfpathlineto{\pgfqpoint{4.762943in}{2.522662in}}%
\pgfpathlineto{\pgfqpoint{4.861283in}{2.474858in}}%
\pgfpathlineto{\pgfqpoint{4.959624in}{2.550477in}}%
\pgfpathlineto{\pgfqpoint{5.057964in}{2.612897in}}%
\pgfpathlineto{\pgfqpoint{5.156305in}{2.575568in}}%
\pgfpathlineto{\pgfqpoint{5.254645in}{2.592598in}}%
\pgfpathlineto{\pgfqpoint{5.352986in}{2.603017in}}%
\pgfpathlineto{\pgfqpoint{5.451327in}{2.599421in}}%
\pgfpathlineto{\pgfqpoint{5.549667in}{2.631359in}}%
\pgfpathlineto{\pgfqpoint{5.648008in}{2.608277in}}%
\pgfusepath{stroke}%
\end{pgfscope}%
\begin{pgfscope}%
\pgfpathrectangle{\pgfqpoint{0.588387in}{0.521603in}}{\pgfqpoint{5.300555in}{2.220246in}}%
\pgfusepath{clip}%
\pgfsetrectcap%
\pgfsetroundjoin%
\pgfsetlinewidth{1.505625pt}%
\pgfsetstrokecolor{currentstroke5}%
\pgfsetdash{}{0pt}%
\pgfpathmoveto{\pgfqpoint{0.829322in}{0.623322in}}%
\pgfpathlineto{\pgfqpoint{0.927662in}{0.660917in}}%
\pgfpathlineto{\pgfqpoint{1.026003in}{0.761728in}}%
\pgfpathlineto{\pgfqpoint{1.124343in}{0.856038in}}%
\pgfpathlineto{\pgfqpoint{1.222684in}{0.971558in}}%
\pgfpathlineto{\pgfqpoint{1.321024in}{1.030755in}}%
\pgfpathlineto{\pgfqpoint{1.419365in}{1.117997in}}%
\pgfpathlineto{\pgfqpoint{1.517705in}{1.186461in}}%
\pgfpathlineto{\pgfqpoint{1.616046in}{1.208954in}}%
\pgfpathlineto{\pgfqpoint{1.714386in}{1.314242in}}%
\pgfpathlineto{\pgfqpoint{1.812727in}{1.400262in}}%
\pgfpathlineto{\pgfqpoint{1.911067in}{1.457684in}}%
\pgfpathlineto{\pgfqpoint{2.009408in}{1.493627in}}%
\pgfpathlineto{\pgfqpoint{2.107748in}{1.563928in}}%
\pgfpathlineto{\pgfqpoint{2.206089in}{1.585822in}}%
\pgfpathlineto{\pgfqpoint{2.304430in}{1.616645in}}%
\pgfpathlineto{\pgfqpoint{2.402770in}{1.650649in}}%
\pgfpathlineto{\pgfqpoint{2.501111in}{1.705221in}}%
\pgfpathlineto{\pgfqpoint{2.599451in}{1.749419in}}%
\pgfpathlineto{\pgfqpoint{2.697792in}{1.799632in}}%
\pgfpathlineto{\pgfqpoint{2.796132in}{1.843536in}}%
\pgfpathlineto{\pgfqpoint{2.894473in}{1.887446in}}%
\pgfpathlineto{\pgfqpoint{2.992813in}{1.900590in}}%
\pgfpathlineto{\pgfqpoint{3.091154in}{1.946596in}}%
\pgfpathlineto{\pgfqpoint{3.189494in}{1.962395in}}%
\pgfpathlineto{\pgfqpoint{3.287835in}{2.005353in}}%
\pgfpathlineto{\pgfqpoint{3.386175in}{2.052987in}}%
\pgfpathlineto{\pgfqpoint{3.484516in}{2.066946in}}%
\pgfpathlineto{\pgfqpoint{3.582856in}{2.097537in}}%
\pgfpathlineto{\pgfqpoint{3.681197in}{2.111607in}}%
\pgfpathlineto{\pgfqpoint{3.779537in}{2.134889in}}%
\pgfpathlineto{\pgfqpoint{3.877878in}{2.169224in}}%
\pgfpathlineto{\pgfqpoint{3.976219in}{2.187024in}}%
\pgfpathlineto{\pgfqpoint{4.074559in}{2.268127in}}%
\pgfpathlineto{\pgfqpoint{4.172900in}{2.272837in}}%
\pgfpathlineto{\pgfqpoint{4.271240in}{2.288221in}}%
\pgfpathlineto{\pgfqpoint{4.369581in}{2.342386in}}%
\pgfpathlineto{\pgfqpoint{4.467921in}{2.312744in}}%
\pgfpathlineto{\pgfqpoint{4.566262in}{2.346171in}}%
\pgfpathlineto{\pgfqpoint{4.664602in}{2.353655in}}%
\pgfpathlineto{\pgfqpoint{4.762943in}{2.399229in}}%
\pgfpathlineto{\pgfqpoint{4.861283in}{2.408808in}}%
\pgfpathlineto{\pgfqpoint{4.959624in}{2.409282in}}%
\pgfpathlineto{\pgfqpoint{5.057964in}{2.494608in}}%
\pgfpathlineto{\pgfqpoint{5.156305in}{2.469854in}}%
\pgfpathlineto{\pgfqpoint{5.254645in}{2.494152in}}%
\pgfpathlineto{\pgfqpoint{5.352986in}{2.518882in}}%
\pgfpathlineto{\pgfqpoint{5.451327in}{2.529021in}}%
\pgfpathlineto{\pgfqpoint{5.549667in}{2.502291in}}%
\pgfpathlineto{\pgfqpoint{5.648008in}{2.540774in}}%
\pgfusepath{stroke}%
\end{pgfscope}%
\begin{pgfscope}%
\pgfsetrectcap%
\pgfsetmiterjoin%
\pgfsetlinewidth{0.803000pt}%
\definecolor{currentstroke}{rgb}{0.000000,0.000000,0.000000}%
\pgfsetstrokecolor{currentstroke}%
\pgfsetdash{}{0pt}%
\pgfpathmoveto{\pgfqpoint{0.588387in}{0.521603in}}%
\pgfpathlineto{\pgfqpoint{0.588387in}{2.741849in}}%
\pgfusepath{stroke}%
\end{pgfscope}%
\begin{pgfscope}%
\pgfsetrectcap%
\pgfsetmiterjoin%
\pgfsetlinewidth{0.803000pt}%
\definecolor{currentstroke}{rgb}{0.000000,0.000000,0.000000}%
\pgfsetstrokecolor{currentstroke}%
\pgfsetdash{}{0pt}%
\pgfpathmoveto{\pgfqpoint{5.888942in}{0.521603in}}%
\pgfpathlineto{\pgfqpoint{5.888942in}{2.741849in}}%
\pgfusepath{stroke}%
\end{pgfscope}%
\begin{pgfscope}%
\pgfsetrectcap%
\pgfsetmiterjoin%
\pgfsetlinewidth{0.803000pt}%
\definecolor{currentstroke}{rgb}{0.000000,0.000000,0.000000}%
\pgfsetstrokecolor{currentstroke}%
\pgfsetdash{}{0pt}%
\pgfpathmoveto{\pgfqpoint{0.588387in}{0.521603in}}%
\pgfpathlineto{\pgfqpoint{5.888942in}{0.521603in}}%
\pgfusepath{stroke}%
\end{pgfscope}%
\begin{pgfscope}%
\pgfsetrectcap%
\pgfsetmiterjoin%
\pgfsetlinewidth{0.803000pt}%
\definecolor{currentstroke}{rgb}{0.000000,0.000000,0.000000}%
\pgfsetstrokecolor{currentstroke}%
\pgfsetdash{}{0pt}%
\pgfpathmoveto{\pgfqpoint{0.588387in}{2.741849in}}%
\pgfpathlineto{\pgfqpoint{5.888942in}{2.741849in}}%
\pgfusepath{stroke}%
\end{pgfscope}%
\begin{pgfscope}%
\pgfsetbuttcap%
\pgfsetmiterjoin%
\definecolor{currentfill}{rgb}{1.000000,1.000000,1.000000}%
\pgfsetfillcolor{currentfill}%
\pgfsetfillopacity{0.800000}%
\pgfsetlinewidth{1.003750pt}%
\definecolor{currentstroke}{rgb}{0.800000,0.800000,0.800000}%
\pgfsetstrokecolor{currentstroke}%
\pgfsetstrokeopacity{0.800000}%
\pgfsetdash{}{0pt}%
\pgfpathmoveto{\pgfqpoint{5.976442in}{1.714055in}}%
\pgfpathlineto{\pgfqpoint{8.259376in}{1.714055in}}%
\pgfpathquadraticcurveto{\pgfqpoint{8.284376in}{1.714055in}}{\pgfqpoint{8.284376in}{1.739055in}}%
\pgfpathlineto{\pgfqpoint{8.284376in}{2.654349in}}%
\pgfpathquadraticcurveto{\pgfqpoint{8.284376in}{2.679349in}}{\pgfqpoint{8.259376in}{2.679349in}}%
\pgfpathlineto{\pgfqpoint{5.976442in}{2.679349in}}%
\pgfpathquadraticcurveto{\pgfqpoint{5.951442in}{2.679349in}}{\pgfqpoint{5.951442in}{2.654349in}}%
\pgfpathlineto{\pgfqpoint{5.951442in}{1.739055in}}%
\pgfpathquadraticcurveto{\pgfqpoint{5.951442in}{1.714055in}}{\pgfqpoint{5.976442in}{1.714055in}}%
\pgfpathlineto{\pgfqpoint{5.976442in}{1.714055in}}%
\pgfpathclose%
\pgfusepath{stroke,fill}%
\end{pgfscope}%
\begin{pgfscope}%
\pgfsetrectcap%
\pgfsetroundjoin%
\pgfsetlinewidth{1.505625pt}%
\pgfsetstrokecolor{currentstroke1}%
\pgfsetdash{}{0pt}%
\pgfpathmoveto{\pgfqpoint{6.001442in}{2.578129in}}%
\pgfpathlineto{\pgfqpoint{6.126442in}{2.578129in}}%
\pgfpathlineto{\pgfqpoint{6.251442in}{2.578129in}}%
\pgfusepath{stroke}%
\end{pgfscope}%
\begin{pgfscope}%
\definecolor{textcolor}{rgb}{0.000000,0.000000,0.000000}%
\pgfsetstrokecolor{textcolor}%
\pgfsetfillcolor{textcolor}%
\pgftext[x=6.351442in,y=2.534379in,left,base]{\color{textcolor}{\rmfamily\fontsize{9.000000}{10.800000}\selectfont\catcode`\^=\active\def^{\ifmmode\sp\else\^{}\fi}\catcode`\%=\active\def%{\%}\CyclesMatchChunks{} \& \MergeLinear{}}}%
\end{pgfscope}%
\begin{pgfscope}%
\pgfsetrectcap%
\pgfsetroundjoin%
\pgfsetlinewidth{1.505625pt}%
\pgfsetstrokecolor{currentstroke3}%
\pgfsetdash{}{0pt}%
\pgfpathmoveto{\pgfqpoint{6.001442in}{2.391178in}}%
\pgfpathlineto{\pgfqpoint{6.126442in}{2.391178in}}%
\pgfpathlineto{\pgfqpoint{6.251442in}{2.391178in}}%
\pgfusepath{stroke}%
\end{pgfscope}%
\begin{pgfscope}%
\definecolor{textcolor}{rgb}{0.000000,0.000000,0.000000}%
\pgfsetstrokecolor{textcolor}%
\pgfsetfillcolor{textcolor}%
\pgftext[x=6.351442in,y=2.347428in,left,base]{\color{textcolor}{\rmfamily\fontsize{9.000000}{10.800000}\selectfont\catcode`\^=\active\def^{\ifmmode\sp\else\^{}\fi}\catcode`\%=\active\def%{\%}\Neighbors{} \& \MergeLinear{}}}%
\end{pgfscope}%
\begin{pgfscope}%
\pgfsetrectcap%
\pgfsetroundjoin%
\pgfsetlinewidth{1.505625pt}%
\pgfsetstrokecolor{currentstroke4}%
\pgfsetdash{}{0pt}%
\pgfpathmoveto{\pgfqpoint{6.001442in}{2.207707in}}%
\pgfpathlineto{\pgfqpoint{6.126442in}{2.207707in}}%
\pgfpathlineto{\pgfqpoint{6.251442in}{2.207707in}}%
\pgfusepath{stroke}%
\end{pgfscope}%
\begin{pgfscope}%
\definecolor{textcolor}{rgb}{0.000000,0.000000,0.000000}%
\pgfsetstrokecolor{textcolor}%
\pgfsetfillcolor{textcolor}%
\pgftext[x=6.351442in,y=2.163957in,left,base]{\color{textcolor}{\rmfamily\fontsize{9.000000}{10.800000}\selectfont\catcode`\^=\active\def^{\ifmmode\sp\else\^{}\fi}\catcode`\%=\active\def%{\%}\NeighborsDegree{} \& \MergeLinear{}}}%
\end{pgfscope}%
\begin{pgfscope}%
\pgfsetrectcap%
\pgfsetroundjoin%
\pgfsetlinewidth{1.505625pt}%
\pgfsetstrokecolor{currentstroke5}%
\pgfsetdash{}{0pt}%
\pgfpathmoveto{\pgfqpoint{6.001442in}{2.020756in}}%
\pgfpathlineto{\pgfqpoint{6.126442in}{2.020756in}}%
\pgfpathlineto{\pgfqpoint{6.251442in}{2.020756in}}%
\pgfusepath{stroke}%
\end{pgfscope}%
\begin{pgfscope}%
\definecolor{textcolor}{rgb}{0.000000,0.000000,0.000000}%
\pgfsetstrokecolor{textcolor}%
\pgfsetfillcolor{textcolor}%
\pgftext[x=6.351442in,y=1.977006in,left,base]{\color{textcolor}{\rmfamily\fontsize{9.000000}{10.800000}\selectfont\catcode`\^=\active\def^{\ifmmode\sp\else\^{}\fi}\catcode`\%=\active\def%{\%}\None{} \& \MergeLinear{}}}%
\end{pgfscope}%
\begin{pgfscope}%
\pgfsetrectcap%
\pgfsetroundjoin%
\pgfsetlinewidth{1.505625pt}%
\pgfsetstrokecolor{currentstroke2}%
\pgfsetdash{}{0pt}%
\pgfpathmoveto{\pgfqpoint{6.001442in}{1.837285in}}%
\pgfpathlineto{\pgfqpoint{6.126442in}{1.837285in}}%
\pgfpathlineto{\pgfqpoint{6.251442in}{1.837285in}}%
\pgfusepath{stroke}%
\end{pgfscope}%
\begin{pgfscope}%
\definecolor{textcolor}{rgb}{0.000000,0.000000,0.000000}%
\pgfsetstrokecolor{textcolor}%
\pgfsetfillcolor{textcolor}%
\pgftext[x=6.351442in,y=1.793535in,left,base]{\color{textcolor}{\rmfamily\fontsize{9.000000}{10.800000}\selectfont\catcode`\^=\active\def^{\ifmmode\sp\else\^{}\fi}\catcode`\%=\active\def%{\%}\KernighanLin{} \& \MergeLinear{}}}%
\end{pgfscope}%
\end{pgfpicture}%
\makeatother%
\endgroup%
}
	\caption[Other splitting strategies for minimally rigid graphs]{
		Mean running time to find some NAC-coloring for minimally rigid graphs with other splitting strategies.}%
	\label{fig:graph_mimimally_rigid_failing_split_first_runtime}
\end{figure}%


As shown in \Cref{fig:graph_no_nac_coloring_generated_rigid_failing_merging_first_runtime},
strategies \Log{}, \SortedBits{} and \MinMax{} fail,
\PromisingCycles{} perform well on the other hand.
%
It can be seen that strategies \SortedSize{} and \Score{} perform
as well as our preferred strategies on graphs with no NAC-coloring.
But as they must list all the NAC-colorings on each subgraph,
they are not suitable for cases where only one NAC-coloring is requested.
%
In \Cref{fig:graph_no_nac_coloring_generated_rigid_failing_split_first_runtime}
it can be seen that performance of \KernighanLin{} and \Cuts{} is worse.
%
\begin{figure}[thbp]
	\centering
	\scalebox{\BenchFigureScale}{%% Creator: Matplotlib, PGF backend
%%
%% To include the figure in your LaTeX document, write
%%   \input{<filename>.pgf}
%%
%% Make sure the required packages are loaded in your preamble
%%   \usepackage{pgf}
%%
%% Also ensure that all the required font packages are loaded; for instance,
%% the lmodern package is sometimes necessary when using math font.
%%   \usepackage{lmodern}
%%
%% Figures using additional raster images can only be included by \input if
%% they are in the same directory as the main LaTeX file. For loading figures
%% from other directories you can use the `import` package
%%   \usepackage{import}
%%
%% and then include the figures with
%%   \import{<path to file>}{<filename>.pgf}
%%
%% Matplotlib used the following preamble
%%   \def\mathdefault#1{#1}
%%   \everymath=\expandafter{\the\everymath\displaystyle}
%%   \IfFileExists{scrextend.sty}{
%%     \usepackage[fontsize=10.000000pt]{scrextend}
%%   }{
%%     \renewcommand{\normalsize}{\fontsize{10.000000}{12.000000}\selectfont}
%%     \normalsize
%%   }
%%   
%%   \ifdefined\pdftexversion\else  % non-pdftex case.
%%     \usepackage{fontspec}
%%     \setmainfont{DejaVuSans.ttf}[Path=\detokenize{/home/petr/Projects/PyRigi/.venv/lib/python3.12/site-packages/matplotlib/mpl-data/fonts/ttf/}]
%%     \setsansfont{DejaVuSans.ttf}[Path=\detokenize{/home/petr/Projects/PyRigi/.venv/lib/python3.12/site-packages/matplotlib/mpl-data/fonts/ttf/}]
%%     \setmonofont{DejaVuSansMono.ttf}[Path=\detokenize{/home/petr/Projects/PyRigi/.venv/lib/python3.12/site-packages/matplotlib/mpl-data/fonts/ttf/}]
%%   \fi
%%   \makeatletter\@ifpackageloaded{underscore}{}{\usepackage[strings]{underscore}}\makeatother
%%
\begingroup%
\makeatletter%
\begin{pgfpicture}%
\pgfpathrectangle{\pgfpointorigin}{\pgfqpoint{8.384376in}{2.841849in}}%
\pgfusepath{use as bounding box, clip}%
\begin{pgfscope}%
\pgfsetbuttcap%
\pgfsetmiterjoin%
\definecolor{currentfill}{rgb}{1.000000,1.000000,1.000000}%
\pgfsetfillcolor{currentfill}%
\pgfsetlinewidth{0.000000pt}%
\definecolor{currentstroke}{rgb}{1.000000,1.000000,1.000000}%
\pgfsetstrokecolor{currentstroke}%
\pgfsetdash{}{0pt}%
\pgfpathmoveto{\pgfqpoint{0.000000in}{0.000000in}}%
\pgfpathlineto{\pgfqpoint{8.384376in}{0.000000in}}%
\pgfpathlineto{\pgfqpoint{8.384376in}{2.841849in}}%
\pgfpathlineto{\pgfqpoint{0.000000in}{2.841849in}}%
\pgfpathlineto{\pgfqpoint{0.000000in}{0.000000in}}%
\pgfpathclose%
\pgfusepath{fill}%
\end{pgfscope}%
\begin{pgfscope}%
\pgfsetbuttcap%
\pgfsetmiterjoin%
\definecolor{currentfill}{rgb}{1.000000,1.000000,1.000000}%
\pgfsetfillcolor{currentfill}%
\pgfsetlinewidth{0.000000pt}%
\definecolor{currentstroke}{rgb}{0.000000,0.000000,0.000000}%
\pgfsetstrokecolor{currentstroke}%
\pgfsetstrokeopacity{0.000000}%
\pgfsetdash{}{0pt}%
\pgfpathmoveto{\pgfqpoint{0.588387in}{0.521603in}}%
\pgfpathlineto{\pgfqpoint{5.903102in}{0.521603in}}%
\pgfpathlineto{\pgfqpoint{5.903102in}{2.531888in}}%
\pgfpathlineto{\pgfqpoint{0.588387in}{2.531888in}}%
\pgfpathlineto{\pgfqpoint{0.588387in}{0.521603in}}%
\pgfpathclose%
\pgfusepath{fill}%
\end{pgfscope}%
\begin{pgfscope}%
\pgfsetbuttcap%
\pgfsetroundjoin%
\definecolor{currentfill}{rgb}{0.000000,0.000000,0.000000}%
\pgfsetfillcolor{currentfill}%
\pgfsetlinewidth{0.803000pt}%
\definecolor{currentstroke}{rgb}{0.000000,0.000000,0.000000}%
\pgfsetstrokecolor{currentstroke}%
\pgfsetdash{}{0pt}%
\pgfsys@defobject{currentmarker}{\pgfqpoint{0.000000in}{-0.048611in}}{\pgfqpoint{0.000000in}{0.000000in}}{%
\pgfpathmoveto{\pgfqpoint{0.000000in}{0.000000in}}%
\pgfpathlineto{\pgfqpoint{0.000000in}{-0.048611in}}%
\pgfusepath{stroke,fill}%
}%
\begin{pgfscope}%
\pgfsys@transformshift{1.046304in}{0.521603in}%
\pgfsys@useobject{currentmarker}{}%
\end{pgfscope}%
\end{pgfscope}%
\begin{pgfscope}%
\definecolor{textcolor}{rgb}{0.000000,0.000000,0.000000}%
\pgfsetstrokecolor{textcolor}%
\pgfsetfillcolor{textcolor}%
\pgftext[x=1.046304in,y=0.424381in,,top]{\color{textcolor}{\rmfamily\fontsize{10.000000}{12.000000}\selectfont\catcode`\^=\active\def^{\ifmmode\sp\else\^{}\fi}\catcode`\%=\active\def%{\%}$\mathdefault{16}$}}%
\end{pgfscope}%
\begin{pgfscope}%
\pgfsetbuttcap%
\pgfsetroundjoin%
\definecolor{currentfill}{rgb}{0.000000,0.000000,0.000000}%
\pgfsetfillcolor{currentfill}%
\pgfsetlinewidth{0.803000pt}%
\definecolor{currentstroke}{rgb}{0.000000,0.000000,0.000000}%
\pgfsetstrokecolor{currentstroke}%
\pgfsetdash{}{0pt}%
\pgfsys@defobject{currentmarker}{\pgfqpoint{0.000000in}{-0.048611in}}{\pgfqpoint{0.000000in}{0.000000in}}{%
\pgfpathmoveto{\pgfqpoint{0.000000in}{0.000000in}}%
\pgfpathlineto{\pgfqpoint{0.000000in}{-0.048611in}}%
\pgfusepath{stroke,fill}%
}%
\begin{pgfscope}%
\pgfsys@transformshift{1.623206in}{0.521603in}%
\pgfsys@useobject{currentmarker}{}%
\end{pgfscope}%
\end{pgfscope}%
\begin{pgfscope}%
\definecolor{textcolor}{rgb}{0.000000,0.000000,0.000000}%
\pgfsetstrokecolor{textcolor}%
\pgfsetfillcolor{textcolor}%
\pgftext[x=1.623206in,y=0.424381in,,top]{\color{textcolor}{\rmfamily\fontsize{10.000000}{12.000000}\selectfont\catcode`\^=\active\def^{\ifmmode\sp\else\^{}\fi}\catcode`\%=\active\def%{\%}$\mathdefault{24}$}}%
\end{pgfscope}%
\begin{pgfscope}%
\pgfsetbuttcap%
\pgfsetroundjoin%
\definecolor{currentfill}{rgb}{0.000000,0.000000,0.000000}%
\pgfsetfillcolor{currentfill}%
\pgfsetlinewidth{0.803000pt}%
\definecolor{currentstroke}{rgb}{0.000000,0.000000,0.000000}%
\pgfsetstrokecolor{currentstroke}%
\pgfsetdash{}{0pt}%
\pgfsys@defobject{currentmarker}{\pgfqpoint{0.000000in}{-0.048611in}}{\pgfqpoint{0.000000in}{0.000000in}}{%
\pgfpathmoveto{\pgfqpoint{0.000000in}{0.000000in}}%
\pgfpathlineto{\pgfqpoint{0.000000in}{-0.048611in}}%
\pgfusepath{stroke,fill}%
}%
\begin{pgfscope}%
\pgfsys@transformshift{2.200109in}{0.521603in}%
\pgfsys@useobject{currentmarker}{}%
\end{pgfscope}%
\end{pgfscope}%
\begin{pgfscope}%
\definecolor{textcolor}{rgb}{0.000000,0.000000,0.000000}%
\pgfsetstrokecolor{textcolor}%
\pgfsetfillcolor{textcolor}%
\pgftext[x=2.200109in,y=0.424381in,,top]{\color{textcolor}{\rmfamily\fontsize{10.000000}{12.000000}\selectfont\catcode`\^=\active\def^{\ifmmode\sp\else\^{}\fi}\catcode`\%=\active\def%{\%}$\mathdefault{32}$}}%
\end{pgfscope}%
\begin{pgfscope}%
\pgfsetbuttcap%
\pgfsetroundjoin%
\definecolor{currentfill}{rgb}{0.000000,0.000000,0.000000}%
\pgfsetfillcolor{currentfill}%
\pgfsetlinewidth{0.803000pt}%
\definecolor{currentstroke}{rgb}{0.000000,0.000000,0.000000}%
\pgfsetstrokecolor{currentstroke}%
\pgfsetdash{}{0pt}%
\pgfsys@defobject{currentmarker}{\pgfqpoint{0.000000in}{-0.048611in}}{\pgfqpoint{0.000000in}{0.000000in}}{%
\pgfpathmoveto{\pgfqpoint{0.000000in}{0.000000in}}%
\pgfpathlineto{\pgfqpoint{0.000000in}{-0.048611in}}%
\pgfusepath{stroke,fill}%
}%
\begin{pgfscope}%
\pgfsys@transformshift{2.777011in}{0.521603in}%
\pgfsys@useobject{currentmarker}{}%
\end{pgfscope}%
\end{pgfscope}%
\begin{pgfscope}%
\definecolor{textcolor}{rgb}{0.000000,0.000000,0.000000}%
\pgfsetstrokecolor{textcolor}%
\pgfsetfillcolor{textcolor}%
\pgftext[x=2.777011in,y=0.424381in,,top]{\color{textcolor}{\rmfamily\fontsize{10.000000}{12.000000}\selectfont\catcode`\^=\active\def^{\ifmmode\sp\else\^{}\fi}\catcode`\%=\active\def%{\%}$\mathdefault{40}$}}%
\end{pgfscope}%
\begin{pgfscope}%
\pgfsetbuttcap%
\pgfsetroundjoin%
\definecolor{currentfill}{rgb}{0.000000,0.000000,0.000000}%
\pgfsetfillcolor{currentfill}%
\pgfsetlinewidth{0.803000pt}%
\definecolor{currentstroke}{rgb}{0.000000,0.000000,0.000000}%
\pgfsetstrokecolor{currentstroke}%
\pgfsetdash{}{0pt}%
\pgfsys@defobject{currentmarker}{\pgfqpoint{0.000000in}{-0.048611in}}{\pgfqpoint{0.000000in}{0.000000in}}{%
\pgfpathmoveto{\pgfqpoint{0.000000in}{0.000000in}}%
\pgfpathlineto{\pgfqpoint{0.000000in}{-0.048611in}}%
\pgfusepath{stroke,fill}%
}%
\begin{pgfscope}%
\pgfsys@transformshift{3.353914in}{0.521603in}%
\pgfsys@useobject{currentmarker}{}%
\end{pgfscope}%
\end{pgfscope}%
\begin{pgfscope}%
\definecolor{textcolor}{rgb}{0.000000,0.000000,0.000000}%
\pgfsetstrokecolor{textcolor}%
\pgfsetfillcolor{textcolor}%
\pgftext[x=3.353914in,y=0.424381in,,top]{\color{textcolor}{\rmfamily\fontsize{10.000000}{12.000000}\selectfont\catcode`\^=\active\def^{\ifmmode\sp\else\^{}\fi}\catcode`\%=\active\def%{\%}$\mathdefault{48}$}}%
\end{pgfscope}%
\begin{pgfscope}%
\pgfsetbuttcap%
\pgfsetroundjoin%
\definecolor{currentfill}{rgb}{0.000000,0.000000,0.000000}%
\pgfsetfillcolor{currentfill}%
\pgfsetlinewidth{0.803000pt}%
\definecolor{currentstroke}{rgb}{0.000000,0.000000,0.000000}%
\pgfsetstrokecolor{currentstroke}%
\pgfsetdash{}{0pt}%
\pgfsys@defobject{currentmarker}{\pgfqpoint{0.000000in}{-0.048611in}}{\pgfqpoint{0.000000in}{0.000000in}}{%
\pgfpathmoveto{\pgfqpoint{0.000000in}{0.000000in}}%
\pgfpathlineto{\pgfqpoint{0.000000in}{-0.048611in}}%
\pgfusepath{stroke,fill}%
}%
\begin{pgfscope}%
\pgfsys@transformshift{3.930816in}{0.521603in}%
\pgfsys@useobject{currentmarker}{}%
\end{pgfscope}%
\end{pgfscope}%
\begin{pgfscope}%
\definecolor{textcolor}{rgb}{0.000000,0.000000,0.000000}%
\pgfsetstrokecolor{textcolor}%
\pgfsetfillcolor{textcolor}%
\pgftext[x=3.930816in,y=0.424381in,,top]{\color{textcolor}{\rmfamily\fontsize{10.000000}{12.000000}\selectfont\catcode`\^=\active\def^{\ifmmode\sp\else\^{}\fi}\catcode`\%=\active\def%{\%}$\mathdefault{56}$}}%
\end{pgfscope}%
\begin{pgfscope}%
\pgfsetbuttcap%
\pgfsetroundjoin%
\definecolor{currentfill}{rgb}{0.000000,0.000000,0.000000}%
\pgfsetfillcolor{currentfill}%
\pgfsetlinewidth{0.803000pt}%
\definecolor{currentstroke}{rgb}{0.000000,0.000000,0.000000}%
\pgfsetstrokecolor{currentstroke}%
\pgfsetdash{}{0pt}%
\pgfsys@defobject{currentmarker}{\pgfqpoint{0.000000in}{-0.048611in}}{\pgfqpoint{0.000000in}{0.000000in}}{%
\pgfpathmoveto{\pgfqpoint{0.000000in}{0.000000in}}%
\pgfpathlineto{\pgfqpoint{0.000000in}{-0.048611in}}%
\pgfusepath{stroke,fill}%
}%
\begin{pgfscope}%
\pgfsys@transformshift{4.507719in}{0.521603in}%
\pgfsys@useobject{currentmarker}{}%
\end{pgfscope}%
\end{pgfscope}%
\begin{pgfscope}%
\definecolor{textcolor}{rgb}{0.000000,0.000000,0.000000}%
\pgfsetstrokecolor{textcolor}%
\pgfsetfillcolor{textcolor}%
\pgftext[x=4.507719in,y=0.424381in,,top]{\color{textcolor}{\rmfamily\fontsize{10.000000}{12.000000}\selectfont\catcode`\^=\active\def^{\ifmmode\sp\else\^{}\fi}\catcode`\%=\active\def%{\%}$\mathdefault{64}$}}%
\end{pgfscope}%
\begin{pgfscope}%
\pgfsetbuttcap%
\pgfsetroundjoin%
\definecolor{currentfill}{rgb}{0.000000,0.000000,0.000000}%
\pgfsetfillcolor{currentfill}%
\pgfsetlinewidth{0.803000pt}%
\definecolor{currentstroke}{rgb}{0.000000,0.000000,0.000000}%
\pgfsetstrokecolor{currentstroke}%
\pgfsetdash{}{0pt}%
\pgfsys@defobject{currentmarker}{\pgfqpoint{0.000000in}{-0.048611in}}{\pgfqpoint{0.000000in}{0.000000in}}{%
\pgfpathmoveto{\pgfqpoint{0.000000in}{0.000000in}}%
\pgfpathlineto{\pgfqpoint{0.000000in}{-0.048611in}}%
\pgfusepath{stroke,fill}%
}%
\begin{pgfscope}%
\pgfsys@transformshift{5.084622in}{0.521603in}%
\pgfsys@useobject{currentmarker}{}%
\end{pgfscope}%
\end{pgfscope}%
\begin{pgfscope}%
\definecolor{textcolor}{rgb}{0.000000,0.000000,0.000000}%
\pgfsetstrokecolor{textcolor}%
\pgfsetfillcolor{textcolor}%
\pgftext[x=5.084622in,y=0.424381in,,top]{\color{textcolor}{\rmfamily\fontsize{10.000000}{12.000000}\selectfont\catcode`\^=\active\def^{\ifmmode\sp\else\^{}\fi}\catcode`\%=\active\def%{\%}$\mathdefault{72}$}}%
\end{pgfscope}%
\begin{pgfscope}%
\pgfsetbuttcap%
\pgfsetroundjoin%
\definecolor{currentfill}{rgb}{0.000000,0.000000,0.000000}%
\pgfsetfillcolor{currentfill}%
\pgfsetlinewidth{0.803000pt}%
\definecolor{currentstroke}{rgb}{0.000000,0.000000,0.000000}%
\pgfsetstrokecolor{currentstroke}%
\pgfsetdash{}{0pt}%
\pgfsys@defobject{currentmarker}{\pgfqpoint{0.000000in}{-0.048611in}}{\pgfqpoint{0.000000in}{0.000000in}}{%
\pgfpathmoveto{\pgfqpoint{0.000000in}{0.000000in}}%
\pgfpathlineto{\pgfqpoint{0.000000in}{-0.048611in}}%
\pgfusepath{stroke,fill}%
}%
\begin{pgfscope}%
\pgfsys@transformshift{5.661524in}{0.521603in}%
\pgfsys@useobject{currentmarker}{}%
\end{pgfscope}%
\end{pgfscope}%
\begin{pgfscope}%
\definecolor{textcolor}{rgb}{0.000000,0.000000,0.000000}%
\pgfsetstrokecolor{textcolor}%
\pgfsetfillcolor{textcolor}%
\pgftext[x=5.661524in,y=0.424381in,,top]{\color{textcolor}{\rmfamily\fontsize{10.000000}{12.000000}\selectfont\catcode`\^=\active\def^{\ifmmode\sp\else\^{}\fi}\catcode`\%=\active\def%{\%}$\mathdefault{80}$}}%
\end{pgfscope}%
\begin{pgfscope}%
\definecolor{textcolor}{rgb}{0.000000,0.000000,0.000000}%
\pgfsetstrokecolor{textcolor}%
\pgfsetfillcolor{textcolor}%
\pgftext[x=3.245745in,y=0.234413in,,top]{\color{textcolor}{\rmfamily\fontsize{10.000000}{12.000000}\selectfont\catcode`\^=\active\def^{\ifmmode\sp\else\^{}\fi}\catcode`\%=\active\def%{\%}Triangle components}}%
\end{pgfscope}%
\begin{pgfscope}%
\pgfsetbuttcap%
\pgfsetroundjoin%
\definecolor{currentfill}{rgb}{0.000000,0.000000,0.000000}%
\pgfsetfillcolor{currentfill}%
\pgfsetlinewidth{0.803000pt}%
\definecolor{currentstroke}{rgb}{0.000000,0.000000,0.000000}%
\pgfsetstrokecolor{currentstroke}%
\pgfsetdash{}{0pt}%
\pgfsys@defobject{currentmarker}{\pgfqpoint{-0.048611in}{0.000000in}}{\pgfqpoint{-0.000000in}{0.000000in}}{%
\pgfpathmoveto{\pgfqpoint{-0.000000in}{0.000000in}}%
\pgfpathlineto{\pgfqpoint{-0.048611in}{0.000000in}}%
\pgfusepath{stroke,fill}%
}%
\begin{pgfscope}%
\pgfsys@transformshift{0.588387in}{2.271204in}%
\pgfsys@useobject{currentmarker}{}%
\end{pgfscope}%
\end{pgfscope}%
\begin{pgfscope}%
\definecolor{textcolor}{rgb}{0.000000,0.000000,0.000000}%
\pgfsetstrokecolor{textcolor}%
\pgfsetfillcolor{textcolor}%
\pgftext[x=0.289968in, y=2.218442in, left, base]{\color{textcolor}{\rmfamily\fontsize{10.000000}{12.000000}\selectfont\catcode`\^=\active\def^{\ifmmode\sp\else\^{}\fi}\catcode`\%=\active\def%{\%}$\mathdefault{10^{3}}$}}%
\end{pgfscope}%
\begin{pgfscope}%
\pgfsetbuttcap%
\pgfsetroundjoin%
\definecolor{currentfill}{rgb}{0.000000,0.000000,0.000000}%
\pgfsetfillcolor{currentfill}%
\pgfsetlinewidth{0.602250pt}%
\definecolor{currentstroke}{rgb}{0.000000,0.000000,0.000000}%
\pgfsetstrokecolor{currentstroke}%
\pgfsetdash{}{0pt}%
\pgfsys@defobject{currentmarker}{\pgfqpoint{-0.027778in}{0.000000in}}{\pgfqpoint{-0.000000in}{0.000000in}}{%
\pgfpathmoveto{\pgfqpoint{-0.000000in}{0.000000in}}%
\pgfpathlineto{\pgfqpoint{-0.027778in}{0.000000in}}%
\pgfusepath{stroke,fill}%
}%
\begin{pgfscope}%
\pgfsys@transformshift{0.588387in}{1.009191in}%
\pgfsys@useobject{currentmarker}{}%
\end{pgfscope}%
\end{pgfscope}%
\begin{pgfscope}%
\pgfsetbuttcap%
\pgfsetroundjoin%
\definecolor{currentfill}{rgb}{0.000000,0.000000,0.000000}%
\pgfsetfillcolor{currentfill}%
\pgfsetlinewidth{0.602250pt}%
\definecolor{currentstroke}{rgb}{0.000000,0.000000,0.000000}%
\pgfsetstrokecolor{currentstroke}%
\pgfsetdash{}{0pt}%
\pgfsys@defobject{currentmarker}{\pgfqpoint{-0.027778in}{0.000000in}}{\pgfqpoint{-0.000000in}{0.000000in}}{%
\pgfpathmoveto{\pgfqpoint{-0.000000in}{0.000000in}}%
\pgfpathlineto{\pgfqpoint{-0.027778in}{0.000000in}}%
\pgfusepath{stroke,fill}%
}%
\begin{pgfscope}%
\pgfsys@transformshift{0.588387in}{1.327129in}%
\pgfsys@useobject{currentmarker}{}%
\end{pgfscope}%
\end{pgfscope}%
\begin{pgfscope}%
\pgfsetbuttcap%
\pgfsetroundjoin%
\definecolor{currentfill}{rgb}{0.000000,0.000000,0.000000}%
\pgfsetfillcolor{currentfill}%
\pgfsetlinewidth{0.602250pt}%
\definecolor{currentstroke}{rgb}{0.000000,0.000000,0.000000}%
\pgfsetstrokecolor{currentstroke}%
\pgfsetdash{}{0pt}%
\pgfsys@defobject{currentmarker}{\pgfqpoint{-0.027778in}{0.000000in}}{\pgfqpoint{-0.000000in}{0.000000in}}{%
\pgfpathmoveto{\pgfqpoint{-0.000000in}{0.000000in}}%
\pgfpathlineto{\pgfqpoint{-0.027778in}{0.000000in}}%
\pgfusepath{stroke,fill}%
}%
\begin{pgfscope}%
\pgfsys@transformshift{0.588387in}{1.552710in}%
\pgfsys@useobject{currentmarker}{}%
\end{pgfscope}%
\end{pgfscope}%
\begin{pgfscope}%
\pgfsetbuttcap%
\pgfsetroundjoin%
\definecolor{currentfill}{rgb}{0.000000,0.000000,0.000000}%
\pgfsetfillcolor{currentfill}%
\pgfsetlinewidth{0.602250pt}%
\definecolor{currentstroke}{rgb}{0.000000,0.000000,0.000000}%
\pgfsetstrokecolor{currentstroke}%
\pgfsetdash{}{0pt}%
\pgfsys@defobject{currentmarker}{\pgfqpoint{-0.027778in}{0.000000in}}{\pgfqpoint{-0.000000in}{0.000000in}}{%
\pgfpathmoveto{\pgfqpoint{-0.000000in}{0.000000in}}%
\pgfpathlineto{\pgfqpoint{-0.027778in}{0.000000in}}%
\pgfusepath{stroke,fill}%
}%
\begin{pgfscope}%
\pgfsys@transformshift{0.588387in}{1.727684in}%
\pgfsys@useobject{currentmarker}{}%
\end{pgfscope}%
\end{pgfscope}%
\begin{pgfscope}%
\pgfsetbuttcap%
\pgfsetroundjoin%
\definecolor{currentfill}{rgb}{0.000000,0.000000,0.000000}%
\pgfsetfillcolor{currentfill}%
\pgfsetlinewidth{0.602250pt}%
\definecolor{currentstroke}{rgb}{0.000000,0.000000,0.000000}%
\pgfsetstrokecolor{currentstroke}%
\pgfsetdash{}{0pt}%
\pgfsys@defobject{currentmarker}{\pgfqpoint{-0.027778in}{0.000000in}}{\pgfqpoint{-0.000000in}{0.000000in}}{%
\pgfpathmoveto{\pgfqpoint{-0.000000in}{0.000000in}}%
\pgfpathlineto{\pgfqpoint{-0.027778in}{0.000000in}}%
\pgfusepath{stroke,fill}%
}%
\begin{pgfscope}%
\pgfsys@transformshift{0.588387in}{1.870649in}%
\pgfsys@useobject{currentmarker}{}%
\end{pgfscope}%
\end{pgfscope}%
\begin{pgfscope}%
\pgfsetbuttcap%
\pgfsetroundjoin%
\definecolor{currentfill}{rgb}{0.000000,0.000000,0.000000}%
\pgfsetfillcolor{currentfill}%
\pgfsetlinewidth{0.602250pt}%
\definecolor{currentstroke}{rgb}{0.000000,0.000000,0.000000}%
\pgfsetstrokecolor{currentstroke}%
\pgfsetdash{}{0pt}%
\pgfsys@defobject{currentmarker}{\pgfqpoint{-0.027778in}{0.000000in}}{\pgfqpoint{-0.000000in}{0.000000in}}{%
\pgfpathmoveto{\pgfqpoint{-0.000000in}{0.000000in}}%
\pgfpathlineto{\pgfqpoint{-0.027778in}{0.000000in}}%
\pgfusepath{stroke,fill}%
}%
\begin{pgfscope}%
\pgfsys@transformshift{0.588387in}{1.991523in}%
\pgfsys@useobject{currentmarker}{}%
\end{pgfscope}%
\end{pgfscope}%
\begin{pgfscope}%
\pgfsetbuttcap%
\pgfsetroundjoin%
\definecolor{currentfill}{rgb}{0.000000,0.000000,0.000000}%
\pgfsetfillcolor{currentfill}%
\pgfsetlinewidth{0.602250pt}%
\definecolor{currentstroke}{rgb}{0.000000,0.000000,0.000000}%
\pgfsetstrokecolor{currentstroke}%
\pgfsetdash{}{0pt}%
\pgfsys@defobject{currentmarker}{\pgfqpoint{-0.027778in}{0.000000in}}{\pgfqpoint{-0.000000in}{0.000000in}}{%
\pgfpathmoveto{\pgfqpoint{-0.000000in}{0.000000in}}%
\pgfpathlineto{\pgfqpoint{-0.027778in}{0.000000in}}%
\pgfusepath{stroke,fill}%
}%
\begin{pgfscope}%
\pgfsys@transformshift{0.588387in}{2.096230in}%
\pgfsys@useobject{currentmarker}{}%
\end{pgfscope}%
\end{pgfscope}%
\begin{pgfscope}%
\pgfsetbuttcap%
\pgfsetroundjoin%
\definecolor{currentfill}{rgb}{0.000000,0.000000,0.000000}%
\pgfsetfillcolor{currentfill}%
\pgfsetlinewidth{0.602250pt}%
\definecolor{currentstroke}{rgb}{0.000000,0.000000,0.000000}%
\pgfsetstrokecolor{currentstroke}%
\pgfsetdash{}{0pt}%
\pgfsys@defobject{currentmarker}{\pgfqpoint{-0.027778in}{0.000000in}}{\pgfqpoint{-0.000000in}{0.000000in}}{%
\pgfpathmoveto{\pgfqpoint{-0.000000in}{0.000000in}}%
\pgfpathlineto{\pgfqpoint{-0.027778in}{0.000000in}}%
\pgfusepath{stroke,fill}%
}%
\begin{pgfscope}%
\pgfsys@transformshift{0.588387in}{2.188587in}%
\pgfsys@useobject{currentmarker}{}%
\end{pgfscope}%
\end{pgfscope}%
\begin{pgfscope}%
\definecolor{textcolor}{rgb}{0.000000,0.000000,0.000000}%
\pgfsetstrokecolor{textcolor}%
\pgfsetfillcolor{textcolor}%
\pgftext[x=0.234413in,y=1.526746in,,bottom,rotate=90.000000]{\color{textcolor}{\rmfamily\fontsize{10.000000}{12.000000}\selectfont\catcode`\^=\active\def^{\ifmmode\sp\else\^{}\fi}\catcode`\%=\active\def%{\%}Time [ms]}}%
\end{pgfscope}%
\begin{pgfscope}%
\pgfpathrectangle{\pgfqpoint{0.588387in}{0.521603in}}{\pgfqpoint{5.314715in}{2.010285in}}%
\pgfusepath{clip}%
\pgfsetrectcap%
\pgfsetroundjoin%
\pgfsetlinewidth{1.505625pt}%
\pgfsetstrokecolor{currentstroke1}%
\pgfsetdash{}{0pt}%
\pgfpathmoveto{\pgfqpoint{0.829965in}{1.082525in}}%
\pgfpathlineto{\pgfqpoint{0.902078in}{0.676001in}}%
\pgfpathlineto{\pgfqpoint{0.974191in}{0.817390in}}%
\pgfpathlineto{\pgfqpoint{1.046304in}{0.968169in}}%
\pgfpathlineto{\pgfqpoint{1.118416in}{1.129763in}}%
\pgfpathlineto{\pgfqpoint{1.190529in}{1.189793in}}%
\pgfpathlineto{\pgfqpoint{1.262642in}{1.205914in}}%
\pgfpathlineto{\pgfqpoint{1.334755in}{1.359376in}}%
\pgfpathlineto{\pgfqpoint{1.406868in}{1.451888in}}%
\pgfpathlineto{\pgfqpoint{1.478981in}{1.315659in}}%
\pgfpathlineto{\pgfqpoint{1.551093in}{1.379365in}}%
\pgfpathlineto{\pgfqpoint{1.623206in}{1.354145in}}%
\pgfpathlineto{\pgfqpoint{1.695319in}{1.386955in}}%
\pgfpathlineto{\pgfqpoint{1.767432in}{1.326328in}}%
\pgfpathlineto{\pgfqpoint{1.839545in}{1.445732in}}%
\pgfpathlineto{\pgfqpoint{1.911657in}{1.470441in}}%
\pgfpathlineto{\pgfqpoint{1.983770in}{1.482046in}}%
\pgfpathlineto{\pgfqpoint{2.055883in}{1.446525in}}%
\pgfpathlineto{\pgfqpoint{2.127996in}{1.405018in}}%
\pgfpathlineto{\pgfqpoint{2.200109in}{1.444808in}}%
\pgfpathlineto{\pgfqpoint{2.272222in}{1.664201in}}%
\pgfpathlineto{\pgfqpoint{2.344334in}{1.598951in}}%
\pgfpathlineto{\pgfqpoint{2.416447in}{1.471262in}}%
\pgfpathlineto{\pgfqpoint{2.488560in}{1.581459in}}%
\pgfpathlineto{\pgfqpoint{2.560673in}{1.547697in}}%
\pgfpathlineto{\pgfqpoint{2.632786in}{1.724869in}}%
\pgfpathlineto{\pgfqpoint{2.704898in}{1.667440in}}%
\pgfpathlineto{\pgfqpoint{2.777011in}{1.694556in}}%
\pgfpathlineto{\pgfqpoint{2.849124in}{1.640133in}}%
\pgfpathlineto{\pgfqpoint{2.921237in}{1.820623in}}%
\pgfpathlineto{\pgfqpoint{2.993350in}{1.753664in}}%
\pgfpathlineto{\pgfqpoint{3.065463in}{1.802270in}}%
\pgfpathlineto{\pgfqpoint{3.137575in}{1.836723in}}%
\pgfpathlineto{\pgfqpoint{3.209688in}{1.804941in}}%
\pgfpathlineto{\pgfqpoint{3.281801in}{1.879672in}}%
\pgfpathlineto{\pgfqpoint{3.353914in}{1.800209in}}%
\pgfpathlineto{\pgfqpoint{3.426027in}{1.813864in}}%
\pgfpathlineto{\pgfqpoint{3.498139in}{1.844773in}}%
\pgfpathlineto{\pgfqpoint{3.570252in}{1.883421in}}%
\pgfpathlineto{\pgfqpoint{3.642365in}{1.946904in}}%
\pgfpathlineto{\pgfqpoint{3.714478in}{1.989497in}}%
\pgfpathlineto{\pgfqpoint{3.786591in}{1.934634in}}%
\pgfpathlineto{\pgfqpoint{3.858704in}{2.014108in}}%
\pgfpathlineto{\pgfqpoint{3.930816in}{2.005556in}}%
\pgfpathlineto{\pgfqpoint{4.002929in}{2.108633in}}%
\pgfpathlineto{\pgfqpoint{4.075042in}{2.009314in}}%
\pgfpathlineto{\pgfqpoint{4.147155in}{2.069443in}}%
\pgfpathlineto{\pgfqpoint{4.219268in}{2.068177in}}%
\pgfpathlineto{\pgfqpoint{4.291381in}{2.037354in}}%
\pgfpathlineto{\pgfqpoint{4.363493in}{2.184740in}}%
\pgfpathlineto{\pgfqpoint{4.435606in}{2.087937in}}%
\pgfpathlineto{\pgfqpoint{4.507719in}{2.198473in}}%
\pgfpathlineto{\pgfqpoint{4.579832in}{2.192599in}}%
\pgfpathlineto{\pgfqpoint{4.651945in}{2.161156in}}%
\pgfpathlineto{\pgfqpoint{4.724057in}{2.217410in}}%
\pgfpathlineto{\pgfqpoint{4.796170in}{2.204722in}}%
\pgfpathlineto{\pgfqpoint{4.868283in}{2.196835in}}%
\pgfpathlineto{\pgfqpoint{4.940396in}{2.181623in}}%
\pgfpathlineto{\pgfqpoint{5.012509in}{2.326895in}}%
\pgfpathlineto{\pgfqpoint{5.084622in}{2.263518in}}%
\pgfpathlineto{\pgfqpoint{5.156734in}{2.262093in}}%
\pgfpathlineto{\pgfqpoint{5.228847in}{2.312625in}}%
\pgfpathlineto{\pgfqpoint{5.300960in}{2.259932in}}%
\pgfpathlineto{\pgfqpoint{5.373073in}{2.321420in}}%
\pgfpathlineto{\pgfqpoint{5.445186in}{2.439957in}}%
\pgfpathlineto{\pgfqpoint{5.517298in}{2.373042in}}%
\pgfpathlineto{\pgfqpoint{5.589411in}{2.312998in}}%
\pgfpathlineto{\pgfqpoint{5.661524in}{2.378049in}}%
\pgfusepath{stroke}%
\end{pgfscope}%
\begin{pgfscope}%
\pgfpathrectangle{\pgfqpoint{0.588387in}{0.521603in}}{\pgfqpoint{5.314715in}{2.010285in}}%
\pgfusepath{clip}%
\pgfsetrectcap%
\pgfsetroundjoin%
\pgfsetlinewidth{1.505625pt}%
\pgfsetstrokecolor{currentstroke2}%
\pgfsetdash{}{0pt}%
\pgfpathmoveto{\pgfqpoint{0.829965in}{0.651666in}}%
\pgfpathlineto{\pgfqpoint{0.902078in}{0.734273in}}%
\pgfpathlineto{\pgfqpoint{0.974191in}{0.925669in}}%
\pgfpathlineto{\pgfqpoint{1.046304in}{1.080605in}}%
\pgfpathlineto{\pgfqpoint{1.118416in}{1.178130in}}%
\pgfpathlineto{\pgfqpoint{1.190529in}{0.906240in}}%
\pgfpathlineto{\pgfqpoint{1.262642in}{0.902241in}}%
\pgfpathlineto{\pgfqpoint{1.334755in}{0.962838in}}%
\pgfpathlineto{\pgfqpoint{1.406868in}{0.976476in}}%
\pgfpathlineto{\pgfqpoint{1.478981in}{0.973507in}}%
\pgfpathlineto{\pgfqpoint{1.551093in}{1.070105in}}%
\pgfpathlineto{\pgfqpoint{1.623206in}{1.531943in}}%
\pgfpathlineto{\pgfqpoint{1.695319in}{1.490408in}}%
\pgfpathlineto{\pgfqpoint{1.767432in}{1.519714in}}%
\pgfpathlineto{\pgfqpoint{1.839545in}{1.910840in}}%
\pgfpathlineto{\pgfqpoint{1.911657in}{2.059047in}}%
\pgfpathlineto{\pgfqpoint{1.983770in}{2.183658in}}%
\pgfpathlineto{\pgfqpoint{2.055883in}{1.303357in}}%
\pgfpathlineto{\pgfqpoint{2.127996in}{1.392803in}}%
\pgfpathlineto{\pgfqpoint{2.200109in}{1.360778in}}%
\pgfpathlineto{\pgfqpoint{2.272222in}{1.648892in}}%
\pgfpathlineto{\pgfqpoint{2.344334in}{1.855473in}}%
\pgfpathlineto{\pgfqpoint{2.416447in}{2.006191in}}%
\pgfpathlineto{\pgfqpoint{2.488560in}{1.728781in}}%
\pgfpathlineto{\pgfqpoint{2.560673in}{1.555799in}}%
\pgfpathlineto{\pgfqpoint{2.632786in}{1.779564in}}%
\pgfpathlineto{\pgfqpoint{2.704898in}{1.836139in}}%
\pgfpathlineto{\pgfqpoint{2.777011in}{1.954832in}}%
\pgfpathlineto{\pgfqpoint{2.849124in}{2.070119in}}%
\pgfpathlineto{\pgfqpoint{2.993350in}{2.440512in}}%
\pgfusepath{stroke}%
\end{pgfscope}%
\begin{pgfscope}%
\pgfpathrectangle{\pgfqpoint{0.588387in}{0.521603in}}{\pgfqpoint{5.314715in}{2.010285in}}%
\pgfusepath{clip}%
\pgfsetrectcap%
\pgfsetroundjoin%
\pgfsetlinewidth{1.505625pt}%
\pgfsetstrokecolor{currentstroke3}%
\pgfsetdash{}{0pt}%
\pgfpathmoveto{\pgfqpoint{0.829965in}{1.205819in}}%
\pgfpathlineto{\pgfqpoint{0.902078in}{0.699496in}}%
\pgfpathlineto{\pgfqpoint{0.974191in}{0.751420in}}%
\pgfpathlineto{\pgfqpoint{1.046304in}{0.807140in}}%
\pgfpathlineto{\pgfqpoint{1.118416in}{0.958410in}}%
\pgfpathlineto{\pgfqpoint{1.190529in}{0.864304in}}%
\pgfpathlineto{\pgfqpoint{1.262642in}{0.918954in}}%
\pgfpathlineto{\pgfqpoint{1.334755in}{1.334144in}}%
\pgfpathlineto{\pgfqpoint{1.406868in}{0.824107in}}%
\pgfpathlineto{\pgfqpoint{1.478981in}{0.799316in}}%
\pgfpathlineto{\pgfqpoint{1.551093in}{1.610998in}}%
\pgfpathlineto{\pgfqpoint{1.623206in}{1.115096in}}%
\pgfpathlineto{\pgfqpoint{1.695319in}{1.486591in}}%
\pgfpathlineto{\pgfqpoint{1.767432in}{1.271700in}}%
\pgfpathlineto{\pgfqpoint{1.839545in}{1.219445in}}%
\pgfpathlineto{\pgfqpoint{1.911657in}{1.543894in}}%
\pgfpathlineto{\pgfqpoint{1.983770in}{1.664279in}}%
\pgfpathlineto{\pgfqpoint{2.055883in}{1.488241in}}%
\pgfpathlineto{\pgfqpoint{2.127996in}{1.390238in}}%
\pgfpathlineto{\pgfqpoint{2.200109in}{1.276723in}}%
\pgfpathlineto{\pgfqpoint{2.272222in}{1.425103in}}%
\pgfpathlineto{\pgfqpoint{2.344334in}{1.442554in}}%
\pgfpathlineto{\pgfqpoint{2.416447in}{2.277711in}}%
\pgfpathlineto{\pgfqpoint{2.488560in}{1.802895in}}%
\pgfpathlineto{\pgfqpoint{2.560673in}{1.624266in}}%
\pgfpathlineto{\pgfqpoint{2.632786in}{1.740903in}}%
\pgfpathlineto{\pgfqpoint{2.704898in}{1.704493in}}%
\pgfpathlineto{\pgfqpoint{2.777011in}{1.971958in}}%
\pgfpathlineto{\pgfqpoint{2.849124in}{2.130979in}}%
\pgfpathlineto{\pgfqpoint{2.921237in}{2.042480in}}%
\pgfpathlineto{\pgfqpoint{2.993350in}{2.271831in}}%
\pgfpathlineto{\pgfqpoint{3.065463in}{2.133553in}}%
\pgfpathlineto{\pgfqpoint{3.137575in}{1.956396in}}%
\pgfpathlineto{\pgfqpoint{3.209688in}{2.375322in}}%
\pgfpathlineto{\pgfqpoint{3.353914in}{1.512489in}}%
\pgfusepath{stroke}%
\end{pgfscope}%
\begin{pgfscope}%
\pgfpathrectangle{\pgfqpoint{0.588387in}{0.521603in}}{\pgfqpoint{5.314715in}{2.010285in}}%
\pgfusepath{clip}%
\pgfsetrectcap%
\pgfsetroundjoin%
\pgfsetlinewidth{1.505625pt}%
\pgfsetstrokecolor{currentstroke4}%
\pgfsetdash{}{0pt}%
\pgfpathmoveto{\pgfqpoint{0.829965in}{0.623079in}}%
\pgfpathlineto{\pgfqpoint{0.902078in}{0.739983in}}%
\pgfpathlineto{\pgfqpoint{0.974191in}{0.938807in}}%
\pgfpathlineto{\pgfqpoint{1.046304in}{1.091271in}}%
\pgfpathlineto{\pgfqpoint{1.118416in}{1.179919in}}%
\pgfpathlineto{\pgfqpoint{1.190529in}{0.923236in}}%
\pgfpathlineto{\pgfqpoint{1.262642in}{0.905201in}}%
\pgfpathlineto{\pgfqpoint{1.334755in}{0.985792in}}%
\pgfpathlineto{\pgfqpoint{1.406868in}{0.992245in}}%
\pgfpathlineto{\pgfqpoint{1.478981in}{0.972770in}}%
\pgfpathlineto{\pgfqpoint{1.551093in}{1.063160in}}%
\pgfpathlineto{\pgfqpoint{1.623206in}{0.907736in}}%
\pgfpathlineto{\pgfqpoint{1.695319in}{0.904639in}}%
\pgfpathlineto{\pgfqpoint{1.767432in}{0.934049in}}%
\pgfpathlineto{\pgfqpoint{1.839545in}{1.042767in}}%
\pgfpathlineto{\pgfqpoint{1.911657in}{1.064586in}}%
\pgfpathlineto{\pgfqpoint{1.983770in}{1.061865in}}%
\pgfpathlineto{\pgfqpoint{2.055883in}{0.971717in}}%
\pgfpathlineto{\pgfqpoint{2.127996in}{1.060869in}}%
\pgfpathlineto{\pgfqpoint{2.200109in}{1.046874in}}%
\pgfpathlineto{\pgfqpoint{2.272222in}{1.063343in}}%
\pgfpathlineto{\pgfqpoint{2.344334in}{1.113753in}}%
\pgfpathlineto{\pgfqpoint{2.416447in}{1.176628in}}%
\pgfpathlineto{\pgfqpoint{2.488560in}{1.104679in}}%
\pgfpathlineto{\pgfqpoint{2.560673in}{1.147671in}}%
\pgfpathlineto{\pgfqpoint{2.632786in}{1.257188in}}%
\pgfpathlineto{\pgfqpoint{2.704898in}{1.254373in}}%
\pgfpathlineto{\pgfqpoint{2.777011in}{1.229400in}}%
\pgfpathlineto{\pgfqpoint{2.849124in}{1.237884in}}%
\pgfpathlineto{\pgfqpoint{2.921237in}{1.280693in}}%
\pgfpathlineto{\pgfqpoint{2.993350in}{1.261747in}}%
\pgfpathlineto{\pgfqpoint{3.065463in}{1.331040in}}%
\pgfpathlineto{\pgfqpoint{3.137575in}{1.341118in}}%
\pgfpathlineto{\pgfqpoint{3.209688in}{1.368424in}}%
\pgfpathlineto{\pgfqpoint{3.281801in}{1.529499in}}%
\pgfpathlineto{\pgfqpoint{3.353914in}{1.395902in}}%
\pgfpathlineto{\pgfqpoint{3.426027in}{1.376919in}}%
\pgfpathlineto{\pgfqpoint{3.498139in}{1.456176in}}%
\pgfpathlineto{\pgfqpoint{3.570252in}{1.491578in}}%
\pgfpathlineto{\pgfqpoint{3.642365in}{1.629819in}}%
\pgfpathlineto{\pgfqpoint{3.714478in}{1.544829in}}%
\pgfpathlineto{\pgfqpoint{3.786591in}{1.716629in}}%
\pgfpathlineto{\pgfqpoint{3.858704in}{1.656141in}}%
\pgfpathlineto{\pgfqpoint{3.930816in}{1.639238in}}%
\pgfpathlineto{\pgfqpoint{4.002929in}{1.755417in}}%
\pgfpathlineto{\pgfqpoint{4.075042in}{1.674564in}}%
\pgfpathlineto{\pgfqpoint{4.219268in}{1.957176in}}%
\pgfpathlineto{\pgfqpoint{4.291381in}{1.600248in}}%
\pgfpathlineto{\pgfqpoint{4.363493in}{1.835911in}}%
\pgfpathlineto{\pgfqpoint{4.507719in}{2.052907in}}%
\pgfpathlineto{\pgfqpoint{4.579832in}{2.036156in}}%
\pgfpathlineto{\pgfqpoint{4.651945in}{1.868030in}}%
\pgfpathlineto{\pgfqpoint{4.724057in}{2.214299in}}%
\pgfpathlineto{\pgfqpoint{4.796170in}{1.929785in}}%
\pgfpathlineto{\pgfqpoint{4.940396in}{2.002646in}}%
\pgfpathlineto{\pgfqpoint{5.084622in}{2.332277in}}%
\pgfpathlineto{\pgfqpoint{5.661524in}{2.376694in}}%
\pgfusepath{stroke}%
\end{pgfscope}%
\begin{pgfscope}%
\pgfpathrectangle{\pgfqpoint{0.588387in}{0.521603in}}{\pgfqpoint{5.314715in}{2.010285in}}%
\pgfusepath{clip}%
\pgfsetrectcap%
\pgfsetroundjoin%
\pgfsetlinewidth{1.505625pt}%
\pgfsetstrokecolor{currentstroke5}%
\pgfsetdash{}{0pt}%
\pgfpathmoveto{\pgfqpoint{0.829965in}{1.153039in}}%
\pgfpathlineto{\pgfqpoint{0.902078in}{0.683763in}}%
\pgfpathlineto{\pgfqpoint{0.974191in}{0.793383in}}%
\pgfpathlineto{\pgfqpoint{1.046304in}{0.947862in}}%
\pgfpathlineto{\pgfqpoint{1.118416in}{1.096323in}}%
\pgfpathlineto{\pgfqpoint{1.190529in}{1.181997in}}%
\pgfpathlineto{\pgfqpoint{1.262642in}{1.189085in}}%
\pgfpathlineto{\pgfqpoint{1.334755in}{1.311716in}}%
\pgfpathlineto{\pgfqpoint{1.406868in}{1.387209in}}%
\pgfpathlineto{\pgfqpoint{1.478981in}{1.245574in}}%
\pgfpathlineto{\pgfqpoint{1.551093in}{1.291651in}}%
\pgfpathlineto{\pgfqpoint{1.623206in}{1.215154in}}%
\pgfpathlineto{\pgfqpoint{1.695319in}{1.315072in}}%
\pgfpathlineto{\pgfqpoint{1.767432in}{1.226287in}}%
\pgfpathlineto{\pgfqpoint{1.839545in}{1.357750in}}%
\pgfpathlineto{\pgfqpoint{1.911657in}{1.369214in}}%
\pgfpathlineto{\pgfqpoint{1.983770in}{1.404403in}}%
\pgfpathlineto{\pgfqpoint{2.055883in}{1.349929in}}%
\pgfpathlineto{\pgfqpoint{2.127996in}{1.338625in}}%
\pgfpathlineto{\pgfqpoint{2.200109in}{1.335051in}}%
\pgfpathlineto{\pgfqpoint{2.272222in}{1.468888in}}%
\pgfpathlineto{\pgfqpoint{2.344334in}{1.497629in}}%
\pgfpathlineto{\pgfqpoint{2.416447in}{1.377943in}}%
\pgfpathlineto{\pgfqpoint{2.488560in}{1.488910in}}%
\pgfpathlineto{\pgfqpoint{2.560673in}{1.428789in}}%
\pgfpathlineto{\pgfqpoint{2.632786in}{1.568414in}}%
\pgfpathlineto{\pgfqpoint{2.704898in}{1.548502in}}%
\pgfpathlineto{\pgfqpoint{2.777011in}{1.538993in}}%
\pgfpathlineto{\pgfqpoint{2.849124in}{1.527457in}}%
\pgfpathlineto{\pgfqpoint{2.921237in}{1.726010in}}%
\pgfpathlineto{\pgfqpoint{2.993350in}{1.646025in}}%
\pgfpathlineto{\pgfqpoint{3.065463in}{1.669076in}}%
\pgfpathlineto{\pgfqpoint{3.137575in}{1.731650in}}%
\pgfpathlineto{\pgfqpoint{3.209688in}{1.653808in}}%
\pgfpathlineto{\pgfqpoint{3.281801in}{1.753412in}}%
\pgfpathlineto{\pgfqpoint{3.353914in}{1.708427in}}%
\pgfpathlineto{\pgfqpoint{3.426027in}{1.723098in}}%
\pgfpathlineto{\pgfqpoint{3.498139in}{1.703359in}}%
\pgfpathlineto{\pgfqpoint{3.570252in}{1.795665in}}%
\pgfpathlineto{\pgfqpoint{3.642365in}{1.802523in}}%
\pgfpathlineto{\pgfqpoint{3.714478in}{1.866025in}}%
\pgfpathlineto{\pgfqpoint{3.786591in}{1.830570in}}%
\pgfpathlineto{\pgfqpoint{3.858704in}{1.881326in}}%
\pgfpathlineto{\pgfqpoint{3.930816in}{1.829563in}}%
\pgfpathlineto{\pgfqpoint{4.002929in}{1.950854in}}%
\pgfpathlineto{\pgfqpoint{4.075042in}{1.884088in}}%
\pgfpathlineto{\pgfqpoint{4.147155in}{1.942928in}}%
\pgfpathlineto{\pgfqpoint{4.219268in}{1.998562in}}%
\pgfpathlineto{\pgfqpoint{4.291381in}{1.953510in}}%
\pgfpathlineto{\pgfqpoint{4.363493in}{2.061886in}}%
\pgfpathlineto{\pgfqpoint{4.435606in}{1.984717in}}%
\pgfpathlineto{\pgfqpoint{4.507719in}{2.115868in}}%
\pgfpathlineto{\pgfqpoint{4.579832in}{2.113523in}}%
\pgfpathlineto{\pgfqpoint{4.651945in}{2.065059in}}%
\pgfpathlineto{\pgfqpoint{4.724057in}{2.163658in}}%
\pgfpathlineto{\pgfqpoint{4.796170in}{2.137593in}}%
\pgfpathlineto{\pgfqpoint{4.868283in}{2.144487in}}%
\pgfpathlineto{\pgfqpoint{4.940396in}{2.111991in}}%
\pgfpathlineto{\pgfqpoint{5.012509in}{2.233141in}}%
\pgfpathlineto{\pgfqpoint{5.084622in}{2.211267in}}%
\pgfpathlineto{\pgfqpoint{5.156734in}{2.194596in}}%
\pgfpathlineto{\pgfqpoint{5.228847in}{2.237869in}}%
\pgfpathlineto{\pgfqpoint{5.300960in}{2.216205in}}%
\pgfpathlineto{\pgfqpoint{5.373073in}{2.272369in}}%
\pgfpathlineto{\pgfqpoint{5.445186in}{2.327079in}}%
\pgfpathlineto{\pgfqpoint{5.517298in}{2.329120in}}%
\pgfpathlineto{\pgfqpoint{5.589411in}{2.255078in}}%
\pgfpathlineto{\pgfqpoint{5.661524in}{2.335924in}}%
\pgfusepath{stroke}%
\end{pgfscope}%
\begin{pgfscope}%
\pgfpathrectangle{\pgfqpoint{0.588387in}{0.521603in}}{\pgfqpoint{5.314715in}{2.010285in}}%
\pgfusepath{clip}%
\pgfsetrectcap%
\pgfsetroundjoin%
\pgfsetlinewidth{1.505625pt}%
\pgfsetstrokecolor{currentstroke6}%
\pgfsetdash{}{0pt}%
\pgfpathmoveto{\pgfqpoint{0.829965in}{1.189805in}}%
\pgfpathlineto{\pgfqpoint{0.902078in}{0.698275in}}%
\pgfpathlineto{\pgfqpoint{0.974191in}{0.770457in}}%
\pgfpathlineto{\pgfqpoint{1.046304in}{0.803850in}}%
\pgfpathlineto{\pgfqpoint{1.118416in}{0.854754in}}%
\pgfpathlineto{\pgfqpoint{1.190529in}{0.945183in}}%
\pgfpathlineto{\pgfqpoint{1.262642in}{0.940804in}}%
\pgfpathlineto{\pgfqpoint{1.334755in}{1.011916in}}%
\pgfpathlineto{\pgfqpoint{1.406868in}{0.791831in}}%
\pgfpathlineto{\pgfqpoint{1.478981in}{0.765931in}}%
\pgfpathlineto{\pgfqpoint{1.551093in}{0.847856in}}%
\pgfpathlineto{\pgfqpoint{1.623206in}{0.828969in}}%
\pgfpathlineto{\pgfqpoint{1.695319in}{0.874310in}}%
\pgfpathlineto{\pgfqpoint{1.767432in}{0.869987in}}%
\pgfpathlineto{\pgfqpoint{1.839545in}{0.938023in}}%
\pgfpathlineto{\pgfqpoint{1.911657in}{0.836663in}}%
\pgfpathlineto{\pgfqpoint{1.983770in}{0.886282in}}%
\pgfpathlineto{\pgfqpoint{2.055883in}{0.890106in}}%
\pgfpathlineto{\pgfqpoint{2.127996in}{0.917386in}}%
\pgfpathlineto{\pgfqpoint{2.200109in}{0.942542in}}%
\pgfpathlineto{\pgfqpoint{2.272222in}{0.904192in}}%
\pgfpathlineto{\pgfqpoint{2.344334in}{1.002348in}}%
\pgfpathlineto{\pgfqpoint{2.416447in}{1.037522in}}%
\pgfpathlineto{\pgfqpoint{2.488560in}{1.055824in}}%
\pgfpathlineto{\pgfqpoint{2.560673in}{1.023984in}}%
\pgfpathlineto{\pgfqpoint{2.632786in}{1.195224in}}%
\pgfpathlineto{\pgfqpoint{2.704898in}{1.095444in}}%
\pgfpathlineto{\pgfqpoint{2.777011in}{1.125071in}}%
\pgfpathlineto{\pgfqpoint{2.849124in}{1.185859in}}%
\pgfpathlineto{\pgfqpoint{2.921237in}{1.121258in}}%
\pgfpathlineto{\pgfqpoint{2.993350in}{1.158403in}}%
\pgfpathlineto{\pgfqpoint{3.065463in}{1.277312in}}%
\pgfpathlineto{\pgfqpoint{3.137575in}{1.301987in}}%
\pgfpathlineto{\pgfqpoint{3.209688in}{1.216004in}}%
\pgfpathlineto{\pgfqpoint{3.281801in}{1.462333in}}%
\pgfpathlineto{\pgfqpoint{3.353914in}{1.329015in}}%
\pgfpathlineto{\pgfqpoint{3.426027in}{1.352208in}}%
\pgfpathlineto{\pgfqpoint{3.498139in}{1.415216in}}%
\pgfpathlineto{\pgfqpoint{3.786591in}{1.630749in}}%
\pgfpathlineto{\pgfqpoint{3.858704in}{1.666976in}}%
\pgfpathlineto{\pgfqpoint{3.930816in}{1.654593in}}%
\pgfpathlineto{\pgfqpoint{4.075042in}{1.582285in}}%
\pgfpathlineto{\pgfqpoint{4.291381in}{1.541143in}}%
\pgfpathlineto{\pgfqpoint{4.363493in}{1.899830in}}%
\pgfpathlineto{\pgfqpoint{4.507719in}{2.079888in}}%
\pgfpathlineto{\pgfqpoint{4.579832in}{2.059612in}}%
\pgfpathlineto{\pgfqpoint{4.724057in}{2.165601in}}%
\pgfusepath{stroke}%
\end{pgfscope}%
\begin{pgfscope}%
\pgfpathrectangle{\pgfqpoint{0.588387in}{0.521603in}}{\pgfqpoint{5.314715in}{2.010285in}}%
\pgfusepath{clip}%
\pgfsetrectcap%
\pgfsetroundjoin%
\pgfsetlinewidth{1.505625pt}%
\pgfsetstrokecolor{currentstroke7}%
\pgfsetdash{}{0pt}%
\pgfpathmoveto{\pgfqpoint{0.902078in}{0.612980in}}%
\pgfpathlineto{\pgfqpoint{0.974191in}{1.330607in}}%
\pgfpathlineto{\pgfqpoint{1.046304in}{1.100386in}}%
\pgfpathlineto{\pgfqpoint{1.190529in}{1.548123in}}%
\pgfpathlineto{\pgfqpoint{1.262642in}{1.230348in}}%
\pgfpathlineto{\pgfqpoint{1.334755in}{0.979220in}}%
\pgfpathlineto{\pgfqpoint{1.478981in}{1.895728in}}%
\pgfpathlineto{\pgfqpoint{1.623206in}{1.392302in}}%
\pgfpathlineto{\pgfqpoint{1.767432in}{2.066768in}}%
\pgfusepath{stroke}%
\end{pgfscope}%
\begin{pgfscope}%
\pgfsetrectcap%
\pgfsetmiterjoin%
\pgfsetlinewidth{0.803000pt}%
\definecolor{currentstroke}{rgb}{0.000000,0.000000,0.000000}%
\pgfsetstrokecolor{currentstroke}%
\pgfsetdash{}{0pt}%
\pgfpathmoveto{\pgfqpoint{0.588387in}{0.521603in}}%
\pgfpathlineto{\pgfqpoint{0.588387in}{2.531888in}}%
\pgfusepath{stroke}%
\end{pgfscope}%
\begin{pgfscope}%
\pgfsetrectcap%
\pgfsetmiterjoin%
\pgfsetlinewidth{0.803000pt}%
\definecolor{currentstroke}{rgb}{0.000000,0.000000,0.000000}%
\pgfsetstrokecolor{currentstroke}%
\pgfsetdash{}{0pt}%
\pgfpathmoveto{\pgfqpoint{5.903102in}{0.521603in}}%
\pgfpathlineto{\pgfqpoint{5.903102in}{2.531888in}}%
\pgfusepath{stroke}%
\end{pgfscope}%
\begin{pgfscope}%
\pgfsetrectcap%
\pgfsetmiterjoin%
\pgfsetlinewidth{0.803000pt}%
\definecolor{currentstroke}{rgb}{0.000000,0.000000,0.000000}%
\pgfsetstrokecolor{currentstroke}%
\pgfsetdash{}{0pt}%
\pgfpathmoveto{\pgfqpoint{0.588387in}{0.521603in}}%
\pgfpathlineto{\pgfqpoint{5.903102in}{0.521603in}}%
\pgfusepath{stroke}%
\end{pgfscope}%
\begin{pgfscope}%
\pgfsetrectcap%
\pgfsetmiterjoin%
\pgfsetlinewidth{0.803000pt}%
\definecolor{currentstroke}{rgb}{0.000000,0.000000,0.000000}%
\pgfsetstrokecolor{currentstroke}%
\pgfsetdash{}{0pt}%
\pgfpathmoveto{\pgfqpoint{0.588387in}{2.531888in}}%
\pgfpathlineto{\pgfqpoint{5.903102in}{2.531888in}}%
\pgfusepath{stroke}%
\end{pgfscope}%
\begin{pgfscope}%
\definecolor{textcolor}{rgb}{0.000000,0.000000,0.000000}%
\pgfsetstrokecolor{textcolor}%
\pgfsetfillcolor{textcolor}%
\pgftext[x=3.245745in,y=2.615222in,,base]{\color{textcolor}{\rmfamily\fontsize{12.000000}{14.400000}\selectfont\catcode`\^=\active\def^{\ifmmode\sp\else\^{}\fi}\catcode`\%=\active\def%{\%}Mean}}%
\end{pgfscope}%
\begin{pgfscope}%
\pgfsetbuttcap%
\pgfsetmiterjoin%
\definecolor{currentfill}{rgb}{1.000000,1.000000,1.000000}%
\pgfsetfillcolor{currentfill}%
\pgfsetfillopacity{0.800000}%
\pgfsetlinewidth{1.003750pt}%
\definecolor{currentstroke}{rgb}{0.800000,0.800000,0.800000}%
\pgfsetstrokecolor{currentstroke}%
\pgfsetstrokeopacity{0.800000}%
\pgfsetdash{}{0pt}%
\pgfpathmoveto{\pgfqpoint{5.990602in}{1.130193in}}%
\pgfpathlineto{\pgfqpoint{8.259376in}{1.130193in}}%
\pgfpathquadraticcurveto{\pgfqpoint{8.284376in}{1.130193in}}{\pgfqpoint{8.284376in}{1.155193in}}%
\pgfpathlineto{\pgfqpoint{8.284376in}{2.444388in}}%
\pgfpathquadraticcurveto{\pgfqpoint{8.284376in}{2.469388in}}{\pgfqpoint{8.259376in}{2.469388in}}%
\pgfpathlineto{\pgfqpoint{5.990602in}{2.469388in}}%
\pgfpathquadraticcurveto{\pgfqpoint{5.965602in}{2.469388in}}{\pgfqpoint{5.965602in}{2.444388in}}%
\pgfpathlineto{\pgfqpoint{5.965602in}{1.155193in}}%
\pgfpathquadraticcurveto{\pgfqpoint{5.965602in}{1.130193in}}{\pgfqpoint{5.990602in}{1.130193in}}%
\pgfpathlineto{\pgfqpoint{5.990602in}{1.130193in}}%
\pgfpathclose%
\pgfusepath{stroke,fill}%
\end{pgfscope}%
\begin{pgfscope}%
\pgfsetrectcap%
\pgfsetroundjoin%
\pgfsetlinewidth{1.505625pt}%
\pgfsetstrokecolor{currentstroke1}%
\pgfsetdash{}{0pt}%
\pgfpathmoveto{\pgfqpoint{6.015602in}{2.368168in}}%
\pgfpathlineto{\pgfqpoint{6.140602in}{2.368168in}}%
\pgfpathlineto{\pgfqpoint{6.265602in}{2.368168in}}%
\pgfusepath{stroke}%
\end{pgfscope}%
\begin{pgfscope}%
\definecolor{textcolor}{rgb}{0.000000,0.000000,0.000000}%
\pgfsetstrokecolor{textcolor}%
\pgfsetfillcolor{textcolor}%
\pgftext[x=6.365602in,y=2.324418in,left,base]{\color{textcolor}{\rmfamily\fontsize{9.000000}{10.800000}\selectfont\catcode`\^=\active\def^{\ifmmode\sp\else\^{}\fi}\catcode`\%=\active\def%{\%}\Neighbors{} \& \MergeLinear{}}}%
\end{pgfscope}%
\begin{pgfscope}%
\pgfsetrectcap%
\pgfsetroundjoin%
\pgfsetlinewidth{1.505625pt}%
\pgfsetstrokecolor{currentstroke2}%
\pgfsetdash{}{0pt}%
\pgfpathmoveto{\pgfqpoint{6.015602in}{2.184696in}}%
\pgfpathlineto{\pgfqpoint{6.140602in}{2.184696in}}%
\pgfpathlineto{\pgfqpoint{6.265602in}{2.184696in}}%
\pgfusepath{stroke}%
\end{pgfscope}%
\begin{pgfscope}%
\definecolor{textcolor}{rgb}{0.000000,0.000000,0.000000}%
\pgfsetstrokecolor{textcolor}%
\pgfsetfillcolor{textcolor}%
\pgftext[x=6.365602in,y=2.140946in,left,base]{\color{textcolor}{\rmfamily\fontsize{9.000000}{10.800000}\selectfont\catcode`\^=\active\def^{\ifmmode\sp\else\^{}\fi}\catcode`\%=\active\def%{\%}\Neighbors{} \& \Log{}}}%
\end{pgfscope}%
\begin{pgfscope}%
\pgfsetrectcap%
\pgfsetroundjoin%
\pgfsetlinewidth{1.505625pt}%
\pgfsetstrokecolor{currentstroke3}%
\pgfsetdash{}{0pt}%
\pgfpathmoveto{\pgfqpoint{6.015602in}{2.001225in}}%
\pgfpathlineto{\pgfqpoint{6.140602in}{2.001225in}}%
\pgfpathlineto{\pgfqpoint{6.265602in}{2.001225in}}%
\pgfusepath{stroke}%
\end{pgfscope}%
\begin{pgfscope}%
\definecolor{textcolor}{rgb}{0.000000,0.000000,0.000000}%
\pgfsetstrokecolor{textcolor}%
\pgfsetfillcolor{textcolor}%
\pgftext[x=6.365602in,y=1.957475in,left,base]{\color{textcolor}{\rmfamily\fontsize{9.000000}{10.800000}\selectfont\catcode`\^=\active\def^{\ifmmode\sp\else\^{}\fi}\catcode`\%=\active\def%{\%}\Neighbors{} \& min\_max}}%
\end{pgfscope}%
\begin{pgfscope}%
\pgfsetrectcap%
\pgfsetroundjoin%
\pgfsetlinewidth{1.505625pt}%
\pgfsetstrokecolor{currentstroke4}%
\pgfsetdash{}{0pt}%
\pgfpathmoveto{\pgfqpoint{6.015602in}{1.814274in}}%
\pgfpathlineto{\pgfqpoint{6.140602in}{1.814274in}}%
\pgfpathlineto{\pgfqpoint{6.265602in}{1.814274in}}%
\pgfusepath{stroke}%
\end{pgfscope}%
\begin{pgfscope}%
\definecolor{textcolor}{rgb}{0.000000,0.000000,0.000000}%
\pgfsetstrokecolor{textcolor}%
\pgfsetfillcolor{textcolor}%
\pgftext[x=6.365602in,y=1.770524in,left,base]{\color{textcolor}{\rmfamily\fontsize{9.000000}{10.800000}\selectfont\catcode`\^=\active\def^{\ifmmode\sp\else\^{}\fi}\catcode`\%=\active\def%{\%}\Neighbors{} \& \PromisingCycles{}}}%
\end{pgfscope}%
\begin{pgfscope}%
\pgfsetrectcap%
\pgfsetroundjoin%
\pgfsetlinewidth{1.505625pt}%
\pgfsetstrokecolor{currentstroke5}%
\pgfsetdash{}{0pt}%
\pgfpathmoveto{\pgfqpoint{6.015602in}{1.627324in}}%
\pgfpathlineto{\pgfqpoint{6.140602in}{1.627324in}}%
\pgfpathlineto{\pgfqpoint{6.265602in}{1.627324in}}%
\pgfusepath{stroke}%
\end{pgfscope}%
\begin{pgfscope}%
\definecolor{textcolor}{rgb}{0.000000,0.000000,0.000000}%
\pgfsetstrokecolor{textcolor}%
\pgfsetfillcolor{textcolor}%
\pgftext[x=6.365602in,y=1.583574in,left,base]{\color{textcolor}{\rmfamily\fontsize{9.000000}{10.800000}\selectfont\catcode`\^=\active\def^{\ifmmode\sp\else\^{}\fi}\catcode`\%=\active\def%{\%}\Neighbors{} \& \SharedVertices{}}}%
\end{pgfscope}%
\begin{pgfscope}%
\pgfsetrectcap%
\pgfsetroundjoin%
\pgfsetlinewidth{1.505625pt}%
\pgfsetstrokecolor{currentstroke6}%
\pgfsetdash{}{0pt}%
\pgfpathmoveto{\pgfqpoint{6.015602in}{1.440373in}}%
\pgfpathlineto{\pgfqpoint{6.140602in}{1.440373in}}%
\pgfpathlineto{\pgfqpoint{6.265602in}{1.440373in}}%
\pgfusepath{stroke}%
\end{pgfscope}%
\begin{pgfscope}%
\definecolor{textcolor}{rgb}{0.000000,0.000000,0.000000}%
\pgfsetstrokecolor{textcolor}%
\pgfsetfillcolor{textcolor}%
\pgftext[x=6.365602in,y=1.396623in,left,base]{\color{textcolor}{\rmfamily\fontsize{9.000000}{10.800000}\selectfont\catcode`\^=\active\def^{\ifmmode\sp\else\^{}\fi}\catcode`\%=\active\def%{\%}\Neighbors{} \& sorted\_size}}%
\end{pgfscope}%
\begin{pgfscope}%
\pgfsetrectcap%
\pgfsetroundjoin%
\pgfsetlinewidth{1.505625pt}%
\pgfsetstrokecolor{currentstroke7}%
\pgfsetdash{}{0pt}%
\pgfpathmoveto{\pgfqpoint{6.015602in}{1.253423in}}%
\pgfpathlineto{\pgfqpoint{6.140602in}{1.253423in}}%
\pgfpathlineto{\pgfqpoint{6.265602in}{1.253423in}}%
\pgfusepath{stroke}%
\end{pgfscope}%
\begin{pgfscope}%
\definecolor{textcolor}{rgb}{0.000000,0.000000,0.000000}%
\pgfsetstrokecolor{textcolor}%
\pgfsetfillcolor{textcolor}%
\pgftext[x=6.365602in,y=1.209673in,left,base]{\color{textcolor}{\rmfamily\fontsize{9.000000}{10.800000}\selectfont\catcode`\^=\active\def^{\ifmmode\sp\else\^{}\fi}\catcode`\%=\active\def%{\%}\Neighbors{} \& sorted\_bits}}%
\end{pgfscope}%
\end{pgfpicture}%
\makeatother%
\endgroup%
}
	\caption[Other merging strategies for graphs with no NAC-coloring]{
		Mean running time to finish search for graphs with no NAC-coloring with other merging strategies.}%
	\label{fig:graph_no_nac_coloring_generated_rigid_failing_merging_first_runtime}
\end{figure}%
\begin{figure}[thbp]
	\centering
	\scalebox{\BenchFigureScale}{%% Creator: Matplotlib, PGF backend
%%
%% To include the figure in your LaTeX document, write
%%   \input{<filename>.pgf}
%%
%% Make sure the required packages are loaded in your preamble
%%   \usepackage{pgf}
%%
%% Also ensure that all the required font packages are loaded; for instance,
%% the lmodern package is sometimes necessary when using math font.
%%   \usepackage{lmodern}
%%
%% Figures using additional raster images can only be included by \input if
%% they are in the same directory as the main LaTeX file. For loading figures
%% from other directories you can use the `import` package
%%   \usepackage{import}
%%
%% and then include the figures with
%%   \import{<path to file>}{<filename>.pgf}
%%
%% Matplotlib used the following preamble
%%   \def\mathdefault#1{#1}
%%   \everymath=\expandafter{\the\everymath\displaystyle}
%%   \IfFileExists{scrextend.sty}{
%%     \usepackage[fontsize=10.000000pt]{scrextend}
%%   }{
%%     \renewcommand{\normalsize}{\fontsize{10.000000}{12.000000}\selectfont}
%%     \normalsize
%%   }
%%   
%%   \ifdefined\pdftexversion\else  % non-pdftex case.
%%     \usepackage{fontspec}
%%     \setmainfont{DejaVuSans.ttf}[Path=\detokenize{/home/petr/Projects/PyRigi/.venv/lib/python3.12/site-packages/matplotlib/mpl-data/fonts/ttf/}]
%%     \setsansfont{DejaVuSans.ttf}[Path=\detokenize{/home/petr/Projects/PyRigi/.venv/lib/python3.12/site-packages/matplotlib/mpl-data/fonts/ttf/}]
%%     \setmonofont{DejaVuSansMono.ttf}[Path=\detokenize{/home/petr/Projects/PyRigi/.venv/lib/python3.12/site-packages/matplotlib/mpl-data/fonts/ttf/}]
%%   \fi
%%   \makeatletter\@ifpackageloaded{under\Score{}}{}{\usepackage[strings]{under\Score{}}}\makeatother
%%
\begingroup%
\makeatletter%
\begin{pgfpicture}%
\pgfpathrectangle{\pgfpointorigin}{\pgfqpoint{8.384376in}{2.841849in}}%
\pgfusepath{use as bounding box, clip}%
\begin{pgfscope}%
\pgfsetbuttcap%
\pgfsetmiterjoin%
\definecolor{currentfill}{rgb}{1.000000,1.000000,1.000000}%
\pgfsetfillcolor{currentfill}%
\pgfsetlinewidth{0.000000pt}%
\definecolor{currentstroke}{rgb}{1.000000,1.000000,1.000000}%
\pgfsetstrokecolor{currentstroke}%
\pgfsetdash{}{0pt}%
\pgfpathmoveto{\pgfqpoint{0.000000in}{0.000000in}}%
\pgfpathlineto{\pgfqpoint{8.384376in}{0.000000in}}%
\pgfpathlineto{\pgfqpoint{8.384376in}{2.841849in}}%
\pgfpathlineto{\pgfqpoint{0.000000in}{2.841849in}}%
\pgfpathlineto{\pgfqpoint{0.000000in}{0.000000in}}%
\pgfpathclose%
\pgfusepath{fill}%
\end{pgfscope}%
\begin{pgfscope}%
\pgfsetbuttcap%
\pgfsetmiterjoin%
\definecolor{currentfill}{rgb}{1.000000,1.000000,1.000000}%
\pgfsetfillcolor{currentfill}%
\pgfsetlinewidth{0.000000pt}%
\definecolor{currentstroke}{rgb}{0.000000,0.000000,0.000000}%
\pgfsetstrokecolor{currentstroke}%
\pgfsetstrokeopacity{0.000000}%
\pgfsetdash{}{0pt}%
\pgfpathmoveto{\pgfqpoint{0.588387in}{0.521603in}}%
\pgfpathlineto{\pgfqpoint{5.888942in}{0.521603in}}%
\pgfpathlineto{\pgfqpoint{5.888942in}{2.741849in}}%
\pgfpathlineto{\pgfqpoint{0.588387in}{2.741849in}}%
\pgfpathlineto{\pgfqpoint{0.588387in}{0.521603in}}%
\pgfpathclose%
\pgfusepath{fill}%
\end{pgfscope}%
\begin{pgfscope}%
\pgfsetbuttcap%
\pgfsetroundjoin%
\definecolor{currentfill}{rgb}{0.000000,0.000000,0.000000}%
\pgfsetfillcolor{currentfill}%
\pgfsetlinewidth{0.803000pt}%
\definecolor{currentstroke}{rgb}{0.000000,0.000000,0.000000}%
\pgfsetstrokecolor{currentstroke}%
\pgfsetdash{}{0pt}%
\pgfsys@defobject{currentmarker}{\pgfqpoint{0.000000in}{-0.048611in}}{\pgfqpoint{0.000000in}{0.000000in}}{%
\pgfpathmoveto{\pgfqpoint{0.000000in}{0.000000in}}%
\pgfpathlineto{\pgfqpoint{0.000000in}{-0.048611in}}%
\pgfusepath{stroke,fill}%
}%
\begin{pgfscope}%
\pgfsys@transformshift{0.665048in}{0.521603in}%
\pgfsys@useobject{currentmarker}{}%
\end{pgfscope}%
\end{pgfscope}%
\begin{pgfscope}%
\definecolor{textcolor}{rgb}{0.000000,0.000000,0.000000}%
\pgfsetstrokecolor{textcolor}%
\pgfsetfillcolor{textcolor}%
\pgftext[x=0.665048in,y=0.424381in,,top]{\color{textcolor}{\rmfamily\fontsize{10.000000}{12.000000}\selectfont\catcode`\^=\active\def^{\ifmmode\sp\else\^{}\fi}\catcode`\%=\active\def%{\%}$\mathdefault{10}$}}%
\end{pgfscope}%
\begin{pgfscope}%
\pgfsetbuttcap%
\pgfsetroundjoin%
\definecolor{currentfill}{rgb}{0.000000,0.000000,0.000000}%
\pgfsetfillcolor{currentfill}%
\pgfsetlinewidth{0.803000pt}%
\definecolor{currentstroke}{rgb}{0.000000,0.000000,0.000000}%
\pgfsetstrokecolor{currentstroke}%
\pgfsetdash{}{0pt}%
\pgfsys@defobject{currentmarker}{\pgfqpoint{0.000000in}{-0.048611in}}{\pgfqpoint{0.000000in}{0.000000in}}{%
\pgfpathmoveto{\pgfqpoint{0.000000in}{0.000000in}}%
\pgfpathlineto{\pgfqpoint{0.000000in}{-0.048611in}}%
\pgfusepath{stroke,fill}%
}%
\begin{pgfscope}%
\pgfsys@transformshift{1.212626in}{0.521603in}%
\pgfsys@useobject{currentmarker}{}%
\end{pgfscope}%
\end{pgfscope}%
\begin{pgfscope}%
\definecolor{textcolor}{rgb}{0.000000,0.000000,0.000000}%
\pgfsetstrokecolor{textcolor}%
\pgfsetfillcolor{textcolor}%
\pgftext[x=1.212626in,y=0.424381in,,top]{\color{textcolor}{\rmfamily\fontsize{10.000000}{12.000000}\selectfont\catcode`\^=\active\def^{\ifmmode\sp\else\^{}\fi}\catcode`\%=\active\def%{\%}$\mathdefault{20}$}}%
\end{pgfscope}%
\begin{pgfscope}%
\pgfsetbuttcap%
\pgfsetroundjoin%
\definecolor{currentfill}{rgb}{0.000000,0.000000,0.000000}%
\pgfsetfillcolor{currentfill}%
\pgfsetlinewidth{0.803000pt}%
\definecolor{currentstroke}{rgb}{0.000000,0.000000,0.000000}%
\pgfsetstrokecolor{currentstroke}%
\pgfsetdash{}{0pt}%
\pgfsys@defobject{currentmarker}{\pgfqpoint{0.000000in}{-0.048611in}}{\pgfqpoint{0.000000in}{0.000000in}}{%
\pgfpathmoveto{\pgfqpoint{0.000000in}{0.000000in}}%
\pgfpathlineto{\pgfqpoint{0.000000in}{-0.048611in}}%
\pgfusepath{stroke,fill}%
}%
\begin{pgfscope}%
\pgfsys@transformshift{1.760204in}{0.521603in}%
\pgfsys@useobject{currentmarker}{}%
\end{pgfscope}%
\end{pgfscope}%
\begin{pgfscope}%
\definecolor{textcolor}{rgb}{0.000000,0.000000,0.000000}%
\pgfsetstrokecolor{textcolor}%
\pgfsetfillcolor{textcolor}%
\pgftext[x=1.760204in,y=0.424381in,,top]{\color{textcolor}{\rmfamily\fontsize{10.000000}{12.000000}\selectfont\catcode`\^=\active\def^{\ifmmode\sp\else\^{}\fi}\catcode`\%=\active\def%{\%}$\mathdefault{30}$}}%
\end{pgfscope}%
\begin{pgfscope}%
\pgfsetbuttcap%
\pgfsetroundjoin%
\definecolor{currentfill}{rgb}{0.000000,0.000000,0.000000}%
\pgfsetfillcolor{currentfill}%
\pgfsetlinewidth{0.803000pt}%
\definecolor{currentstroke}{rgb}{0.000000,0.000000,0.000000}%
\pgfsetstrokecolor{currentstroke}%
\pgfsetdash{}{0pt}%
\pgfsys@defobject{currentmarker}{\pgfqpoint{0.000000in}{-0.048611in}}{\pgfqpoint{0.000000in}{0.000000in}}{%
\pgfpathmoveto{\pgfqpoint{0.000000in}{0.000000in}}%
\pgfpathlineto{\pgfqpoint{0.000000in}{-0.048611in}}%
\pgfusepath{stroke,fill}%
}%
\begin{pgfscope}%
\pgfsys@transformshift{2.307782in}{0.521603in}%
\pgfsys@useobject{currentmarker}{}%
\end{pgfscope}%
\end{pgfscope}%
\begin{pgfscope}%
\definecolor{textcolor}{rgb}{0.000000,0.000000,0.000000}%
\pgfsetstrokecolor{textcolor}%
\pgfsetfillcolor{textcolor}%
\pgftext[x=2.307782in,y=0.424381in,,top]{\color{textcolor}{\rmfamily\fontsize{10.000000}{12.000000}\selectfont\catcode`\^=\active\def^{\ifmmode\sp\else\^{}\fi}\catcode`\%=\active\def%{\%}$\mathdefault{40}$}}%
\end{pgfscope}%
\begin{pgfscope}%
\pgfsetbuttcap%
\pgfsetroundjoin%
\definecolor{currentfill}{rgb}{0.000000,0.000000,0.000000}%
\pgfsetfillcolor{currentfill}%
\pgfsetlinewidth{0.803000pt}%
\definecolor{currentstroke}{rgb}{0.000000,0.000000,0.000000}%
\pgfsetstrokecolor{currentstroke}%
\pgfsetdash{}{0pt}%
\pgfsys@defobject{currentmarker}{\pgfqpoint{0.000000in}{-0.048611in}}{\pgfqpoint{0.000000in}{0.000000in}}{%
\pgfpathmoveto{\pgfqpoint{0.000000in}{0.000000in}}%
\pgfpathlineto{\pgfqpoint{0.000000in}{-0.048611in}}%
\pgfusepath{stroke,fill}%
}%
\begin{pgfscope}%
\pgfsys@transformshift{2.855360in}{0.521603in}%
\pgfsys@useobject{currentmarker}{}%
\end{pgfscope}%
\end{pgfscope}%
\begin{pgfscope}%
\definecolor{textcolor}{rgb}{0.000000,0.000000,0.000000}%
\pgfsetstrokecolor{textcolor}%
\pgfsetfillcolor{textcolor}%
\pgftext[x=2.855360in,y=0.424381in,,top]{\color{textcolor}{\rmfamily\fontsize{10.000000}{12.000000}\selectfont\catcode`\^=\active\def^{\ifmmode\sp\else\^{}\fi}\catcode`\%=\active\def%{\%}$\mathdefault{50}$}}%
\end{pgfscope}%
\begin{pgfscope}%
\pgfsetbuttcap%
\pgfsetroundjoin%
\definecolor{currentfill}{rgb}{0.000000,0.000000,0.000000}%
\pgfsetfillcolor{currentfill}%
\pgfsetlinewidth{0.803000pt}%
\definecolor{currentstroke}{rgb}{0.000000,0.000000,0.000000}%
\pgfsetstrokecolor{currentstroke}%
\pgfsetdash{}{0pt}%
\pgfsys@defobject{currentmarker}{\pgfqpoint{0.000000in}{-0.048611in}}{\pgfqpoint{0.000000in}{0.000000in}}{%
\pgfpathmoveto{\pgfqpoint{0.000000in}{0.000000in}}%
\pgfpathlineto{\pgfqpoint{0.000000in}{-0.048611in}}%
\pgfusepath{stroke,fill}%
}%
\begin{pgfscope}%
\pgfsys@transformshift{3.402938in}{0.521603in}%
\pgfsys@useobject{currentmarker}{}%
\end{pgfscope}%
\end{pgfscope}%
\begin{pgfscope}%
\definecolor{textcolor}{rgb}{0.000000,0.000000,0.000000}%
\pgfsetstrokecolor{textcolor}%
\pgfsetfillcolor{textcolor}%
\pgftext[x=3.402938in,y=0.424381in,,top]{\color{textcolor}{\rmfamily\fontsize{10.000000}{12.000000}\selectfont\catcode`\^=\active\def^{\ifmmode\sp\else\^{}\fi}\catcode`\%=\active\def%{\%}$\mathdefault{60}$}}%
\end{pgfscope}%
\begin{pgfscope}%
\pgfsetbuttcap%
\pgfsetroundjoin%
\definecolor{currentfill}{rgb}{0.000000,0.000000,0.000000}%
\pgfsetfillcolor{currentfill}%
\pgfsetlinewidth{0.803000pt}%
\definecolor{currentstroke}{rgb}{0.000000,0.000000,0.000000}%
\pgfsetstrokecolor{currentstroke}%
\pgfsetdash{}{0pt}%
\pgfsys@defobject{currentmarker}{\pgfqpoint{0.000000in}{-0.048611in}}{\pgfqpoint{0.000000in}{0.000000in}}{%
\pgfpathmoveto{\pgfqpoint{0.000000in}{0.000000in}}%
\pgfpathlineto{\pgfqpoint{0.000000in}{-0.048611in}}%
\pgfusepath{stroke,fill}%
}%
\begin{pgfscope}%
\pgfsys@transformshift{3.950516in}{0.521603in}%
\pgfsys@useobject{currentmarker}{}%
\end{pgfscope}%
\end{pgfscope}%
\begin{pgfscope}%
\definecolor{textcolor}{rgb}{0.000000,0.000000,0.000000}%
\pgfsetstrokecolor{textcolor}%
\pgfsetfillcolor{textcolor}%
\pgftext[x=3.950516in,y=0.424381in,,top]{\color{textcolor}{\rmfamily\fontsize{10.000000}{12.000000}\selectfont\catcode`\^=\active\def^{\ifmmode\sp\else\^{}\fi}\catcode`\%=\active\def%{\%}$\mathdefault{70}$}}%
\end{pgfscope}%
\begin{pgfscope}%
\pgfsetbuttcap%
\pgfsetroundjoin%
\definecolor{currentfill}{rgb}{0.000000,0.000000,0.000000}%
\pgfsetfillcolor{currentfill}%
\pgfsetlinewidth{0.803000pt}%
\definecolor{currentstroke}{rgb}{0.000000,0.000000,0.000000}%
\pgfsetstrokecolor{currentstroke}%
\pgfsetdash{}{0pt}%
\pgfsys@defobject{currentmarker}{\pgfqpoint{0.000000in}{-0.048611in}}{\pgfqpoint{0.000000in}{0.000000in}}{%
\pgfpathmoveto{\pgfqpoint{0.000000in}{0.000000in}}%
\pgfpathlineto{\pgfqpoint{0.000000in}{-0.048611in}}%
\pgfusepath{stroke,fill}%
}%
\begin{pgfscope}%
\pgfsys@transformshift{4.498094in}{0.521603in}%
\pgfsys@useobject{currentmarker}{}%
\end{pgfscope}%
\end{pgfscope}%
\begin{pgfscope}%
\definecolor{textcolor}{rgb}{0.000000,0.000000,0.000000}%
\pgfsetstrokecolor{textcolor}%
\pgfsetfillcolor{textcolor}%
\pgftext[x=4.498094in,y=0.424381in,,top]{\color{textcolor}{\rmfamily\fontsize{10.000000}{12.000000}\selectfont\catcode`\^=\active\def^{\ifmmode\sp\else\^{}\fi}\catcode`\%=\active\def%{\%}$\mathdefault{80}$}}%
\end{pgfscope}%
\begin{pgfscope}%
\pgfsetbuttcap%
\pgfsetroundjoin%
\definecolor{currentfill}{rgb}{0.000000,0.000000,0.000000}%
\pgfsetfillcolor{currentfill}%
\pgfsetlinewidth{0.803000pt}%
\definecolor{currentstroke}{rgb}{0.000000,0.000000,0.000000}%
\pgfsetstrokecolor{currentstroke}%
\pgfsetdash{}{0pt}%
\pgfsys@defobject{currentmarker}{\pgfqpoint{0.000000in}{-0.048611in}}{\pgfqpoint{0.000000in}{0.000000in}}{%
\pgfpathmoveto{\pgfqpoint{0.000000in}{0.000000in}}%
\pgfpathlineto{\pgfqpoint{0.000000in}{-0.048611in}}%
\pgfusepath{stroke,fill}%
}%
\begin{pgfscope}%
\pgfsys@transformshift{5.045672in}{0.521603in}%
\pgfsys@useobject{currentmarker}{}%
\end{pgfscope}%
\end{pgfscope}%
\begin{pgfscope}%
\definecolor{textcolor}{rgb}{0.000000,0.000000,0.000000}%
\pgfsetstrokecolor{textcolor}%
\pgfsetfillcolor{textcolor}%
\pgftext[x=5.045672in,y=0.424381in,,top]{\color{textcolor}{\rmfamily\fontsize{10.000000}{12.000000}\selectfont\catcode`\^=\active\def^{\ifmmode\sp\else\^{}\fi}\catcode`\%=\active\def%{\%}$\mathdefault{90}$}}%
\end{pgfscope}%
\begin{pgfscope}%
\pgfsetbuttcap%
\pgfsetroundjoin%
\definecolor{currentfill}{rgb}{0.000000,0.000000,0.000000}%
\pgfsetfillcolor{currentfill}%
\pgfsetlinewidth{0.803000pt}%
\definecolor{currentstroke}{rgb}{0.000000,0.000000,0.000000}%
\pgfsetstrokecolor{currentstroke}%
\pgfsetdash{}{0pt}%
\pgfsys@defobject{currentmarker}{\pgfqpoint{0.000000in}{-0.048611in}}{\pgfqpoint{0.000000in}{0.000000in}}{%
\pgfpathmoveto{\pgfqpoint{0.000000in}{0.000000in}}%
\pgfpathlineto{\pgfqpoint{0.000000in}{-0.048611in}}%
\pgfusepath{stroke,fill}%
}%
\begin{pgfscope}%
\pgfsys@transformshift{5.593250in}{0.521603in}%
\pgfsys@useobject{currentmarker}{}%
\end{pgfscope}%
\end{pgfscope}%
\begin{pgfscope}%
\definecolor{textcolor}{rgb}{0.000000,0.000000,0.000000}%
\pgfsetstrokecolor{textcolor}%
\pgfsetfillcolor{textcolor}%
\pgftext[x=5.593250in,y=0.424381in,,top]{\color{textcolor}{\rmfamily\fontsize{10.000000}{12.000000}\selectfont\catcode`\^=\active\def^{\ifmmode\sp\else\^{}\fi}\catcode`\%=\active\def%{\%}$\mathdefault{100}$}}%
\end{pgfscope}%
\begin{pgfscope}%
\definecolor{textcolor}{rgb}{0.000000,0.000000,0.000000}%
\pgfsetstrokecolor{textcolor}%
\pgfsetfillcolor{textcolor}%
\pgftext[x=3.238665in,y=0.234413in,,top]{\color{textcolor}{\rmfamily\fontsize{10.000000}{12.000000}\selectfont\catcode`\^=\active\def^{\ifmmode\sp\else\^{}\fi}\catcode`\%=\active\def%{\%}Triangle components}}%
\end{pgfscope}%
\begin{pgfscope}%
\pgfsetbuttcap%
\pgfsetroundjoin%
\definecolor{currentfill}{rgb}{0.000000,0.000000,0.000000}%
\pgfsetfillcolor{currentfill}%
\pgfsetlinewidth{0.803000pt}%
\definecolor{currentstroke}{rgb}{0.000000,0.000000,0.000000}%
\pgfsetstrokecolor{currentstroke}%
\pgfsetdash{}{0pt}%
\pgfsys@defobject{currentmarker}{\pgfqpoint{-0.048611in}{0.000000in}}{\pgfqpoint{-0.000000in}{0.000000in}}{%
\pgfpathmoveto{\pgfqpoint{-0.000000in}{0.000000in}}%
\pgfpathlineto{\pgfqpoint{-0.048611in}{0.000000in}}%
\pgfusepath{stroke,fill}%
}%
\begin{pgfscope}%
\pgfsys@transformshift{0.588387in}{0.585015in}%
\pgfsys@useobject{currentmarker}{}%
\end{pgfscope}%
\end{pgfscope}%
\begin{pgfscope}%
\definecolor{textcolor}{rgb}{0.000000,0.000000,0.000000}%
\pgfsetstrokecolor{textcolor}%
\pgfsetfillcolor{textcolor}%
\pgftext[x=0.289968in, y=0.532254in, left, base]{\color{textcolor}{\rmfamily\fontsize{10.000000}{12.000000}\selectfont\catcode`\^=\active\def^{\ifmmode\sp\else\^{}\fi}\catcode`\%=\active\def%{\%}$\mathdefault{10^{2}}$}}%
\end{pgfscope}%
\begin{pgfscope}%
\pgfsetbuttcap%
\pgfsetroundjoin%
\definecolor{currentfill}{rgb}{0.000000,0.000000,0.000000}%
\pgfsetfillcolor{currentfill}%
\pgfsetlinewidth{0.803000pt}%
\definecolor{currentstroke}{rgb}{0.000000,0.000000,0.000000}%
\pgfsetstrokecolor{currentstroke}%
\pgfsetdash{}{0pt}%
\pgfsys@defobject{currentmarker}{\pgfqpoint{-0.048611in}{0.000000in}}{\pgfqpoint{-0.000000in}{0.000000in}}{%
\pgfpathmoveto{\pgfqpoint{-0.000000in}{0.000000in}}%
\pgfpathlineto{\pgfqpoint{-0.048611in}{0.000000in}}%
\pgfusepath{stroke,fill}%
}%
\begin{pgfscope}%
\pgfsys@transformshift{0.588387in}{1.779228in}%
\pgfsys@useobject{currentmarker}{}%
\end{pgfscope}%
\end{pgfscope}%
\begin{pgfscope}%
\definecolor{textcolor}{rgb}{0.000000,0.000000,0.000000}%
\pgfsetstrokecolor{textcolor}%
\pgfsetfillcolor{textcolor}%
\pgftext[x=0.289968in, y=1.726467in, left, base]{\color{textcolor}{\rmfamily\fontsize{10.000000}{12.000000}\selectfont\catcode`\^=\active\def^{\ifmmode\sp\else\^{}\fi}\catcode`\%=\active\def%{\%}$\mathdefault{10^{3}}$}}%
\end{pgfscope}%
\begin{pgfscope}%
\pgfsetbuttcap%
\pgfsetroundjoin%
\definecolor{currentfill}{rgb}{0.000000,0.000000,0.000000}%
\pgfsetfillcolor{currentfill}%
\pgfsetlinewidth{0.602250pt}%
\definecolor{currentstroke}{rgb}{0.000000,0.000000,0.000000}%
\pgfsetstrokecolor{currentstroke}%
\pgfsetdash{}{0pt}%
\pgfsys@defobject{currentmarker}{\pgfqpoint{-0.027778in}{0.000000in}}{\pgfqpoint{-0.000000in}{0.000000in}}{%
\pgfpathmoveto{\pgfqpoint{-0.000000in}{0.000000in}}%
\pgfpathlineto{\pgfqpoint{-0.027778in}{0.000000in}}%
\pgfusepath{stroke,fill}%
}%
\begin{pgfscope}%
\pgfsys@transformshift{0.588387in}{0.530371in}%
\pgfsys@useobject{currentmarker}{}%
\end{pgfscope}%
\end{pgfscope}%
\begin{pgfscope}%
\pgfsetbuttcap%
\pgfsetroundjoin%
\definecolor{currentfill}{rgb}{0.000000,0.000000,0.000000}%
\pgfsetfillcolor{currentfill}%
\pgfsetlinewidth{0.602250pt}%
\definecolor{currentstroke}{rgb}{0.000000,0.000000,0.000000}%
\pgfsetstrokecolor{currentstroke}%
\pgfsetdash{}{0pt}%
\pgfsys@defobject{currentmarker}{\pgfqpoint{-0.027778in}{0.000000in}}{\pgfqpoint{-0.000000in}{0.000000in}}{%
\pgfpathmoveto{\pgfqpoint{-0.000000in}{0.000000in}}%
\pgfpathlineto{\pgfqpoint{-0.027778in}{0.000000in}}%
\pgfusepath{stroke,fill}%
}%
\begin{pgfscope}%
\pgfsys@transformshift{0.588387in}{0.944509in}%
\pgfsys@useobject{currentmarker}{}%
\end{pgfscope}%
\end{pgfscope}%
\begin{pgfscope}%
\pgfsetbuttcap%
\pgfsetroundjoin%
\definecolor{currentfill}{rgb}{0.000000,0.000000,0.000000}%
\pgfsetfillcolor{currentfill}%
\pgfsetlinewidth{0.602250pt}%
\definecolor{currentstroke}{rgb}{0.000000,0.000000,0.000000}%
\pgfsetstrokecolor{currentstroke}%
\pgfsetdash{}{0pt}%
\pgfsys@defobject{currentmarker}{\pgfqpoint{-0.027778in}{0.000000in}}{\pgfqpoint{-0.000000in}{0.000000in}}{%
\pgfpathmoveto{\pgfqpoint{-0.000000in}{0.000000in}}%
\pgfpathlineto{\pgfqpoint{-0.027778in}{0.000000in}}%
\pgfusepath{stroke,fill}%
}%
\begin{pgfscope}%
\pgfsys@transformshift{0.588387in}{1.154800in}%
\pgfsys@useobject{currentmarker}{}%
\end{pgfscope}%
\end{pgfscope}%
\begin{pgfscope}%
\pgfsetbuttcap%
\pgfsetroundjoin%
\definecolor{currentfill}{rgb}{0.000000,0.000000,0.000000}%
\pgfsetfillcolor{currentfill}%
\pgfsetlinewidth{0.602250pt}%
\definecolor{currentstroke}{rgb}{0.000000,0.000000,0.000000}%
\pgfsetstrokecolor{currentstroke}%
\pgfsetdash{}{0pt}%
\pgfsys@defobject{currentmarker}{\pgfqpoint{-0.027778in}{0.000000in}}{\pgfqpoint{-0.000000in}{0.000000in}}{%
\pgfpathmoveto{\pgfqpoint{-0.000000in}{0.000000in}}%
\pgfpathlineto{\pgfqpoint{-0.027778in}{0.000000in}}%
\pgfusepath{stroke,fill}%
}%
\begin{pgfscope}%
\pgfsys@transformshift{0.588387in}{1.304003in}%
\pgfsys@useobject{currentmarker}{}%
\end{pgfscope}%
\end{pgfscope}%
\begin{pgfscope}%
\pgfsetbuttcap%
\pgfsetroundjoin%
\definecolor{currentfill}{rgb}{0.000000,0.000000,0.000000}%
\pgfsetfillcolor{currentfill}%
\pgfsetlinewidth{0.602250pt}%
\definecolor{currentstroke}{rgb}{0.000000,0.000000,0.000000}%
\pgfsetstrokecolor{currentstroke}%
\pgfsetdash{}{0pt}%
\pgfsys@defobject{currentmarker}{\pgfqpoint{-0.027778in}{0.000000in}}{\pgfqpoint{-0.000000in}{0.000000in}}{%
\pgfpathmoveto{\pgfqpoint{-0.000000in}{0.000000in}}%
\pgfpathlineto{\pgfqpoint{-0.027778in}{0.000000in}}%
\pgfusepath{stroke,fill}%
}%
\begin{pgfscope}%
\pgfsys@transformshift{0.588387in}{1.419734in}%
\pgfsys@useobject{currentmarker}{}%
\end{pgfscope}%
\end{pgfscope}%
\begin{pgfscope}%
\pgfsetbuttcap%
\pgfsetroundjoin%
\definecolor{currentfill}{rgb}{0.000000,0.000000,0.000000}%
\pgfsetfillcolor{currentfill}%
\pgfsetlinewidth{0.602250pt}%
\definecolor{currentstroke}{rgb}{0.000000,0.000000,0.000000}%
\pgfsetstrokecolor{currentstroke}%
\pgfsetdash{}{0pt}%
\pgfsys@defobject{currentmarker}{\pgfqpoint{-0.027778in}{0.000000in}}{\pgfqpoint{-0.000000in}{0.000000in}}{%
\pgfpathmoveto{\pgfqpoint{-0.000000in}{0.000000in}}%
\pgfpathlineto{\pgfqpoint{-0.027778in}{0.000000in}}%
\pgfusepath{stroke,fill}%
}%
\begin{pgfscope}%
\pgfsys@transformshift{0.588387in}{1.514294in}%
\pgfsys@useobject{currentmarker}{}%
\end{pgfscope}%
\end{pgfscope}%
\begin{pgfscope}%
\pgfsetbuttcap%
\pgfsetroundjoin%
\definecolor{currentfill}{rgb}{0.000000,0.000000,0.000000}%
\pgfsetfillcolor{currentfill}%
\pgfsetlinewidth{0.602250pt}%
\definecolor{currentstroke}{rgb}{0.000000,0.000000,0.000000}%
\pgfsetstrokecolor{currentstroke}%
\pgfsetdash{}{0pt}%
\pgfsys@defobject{currentmarker}{\pgfqpoint{-0.027778in}{0.000000in}}{\pgfqpoint{-0.000000in}{0.000000in}}{%
\pgfpathmoveto{\pgfqpoint{-0.000000in}{0.000000in}}%
\pgfpathlineto{\pgfqpoint{-0.027778in}{0.000000in}}%
\pgfusepath{stroke,fill}%
}%
\begin{pgfscope}%
\pgfsys@transformshift{0.588387in}{1.594242in}%
\pgfsys@useobject{currentmarker}{}%
\end{pgfscope}%
\end{pgfscope}%
\begin{pgfscope}%
\pgfsetbuttcap%
\pgfsetroundjoin%
\definecolor{currentfill}{rgb}{0.000000,0.000000,0.000000}%
\pgfsetfillcolor{currentfill}%
\pgfsetlinewidth{0.602250pt}%
\definecolor{currentstroke}{rgb}{0.000000,0.000000,0.000000}%
\pgfsetstrokecolor{currentstroke}%
\pgfsetdash{}{0pt}%
\pgfsys@defobject{currentmarker}{\pgfqpoint{-0.027778in}{0.000000in}}{\pgfqpoint{-0.000000in}{0.000000in}}{%
\pgfpathmoveto{\pgfqpoint{-0.000000in}{0.000000in}}%
\pgfpathlineto{\pgfqpoint{-0.027778in}{0.000000in}}%
\pgfusepath{stroke,fill}%
}%
\begin{pgfscope}%
\pgfsys@transformshift{0.588387in}{1.663497in}%
\pgfsys@useobject{currentmarker}{}%
\end{pgfscope}%
\end{pgfscope}%
\begin{pgfscope}%
\pgfsetbuttcap%
\pgfsetroundjoin%
\definecolor{currentfill}{rgb}{0.000000,0.000000,0.000000}%
\pgfsetfillcolor{currentfill}%
\pgfsetlinewidth{0.602250pt}%
\definecolor{currentstroke}{rgb}{0.000000,0.000000,0.000000}%
\pgfsetstrokecolor{currentstroke}%
\pgfsetdash{}{0pt}%
\pgfsys@defobject{currentmarker}{\pgfqpoint{-0.027778in}{0.000000in}}{\pgfqpoint{-0.000000in}{0.000000in}}{%
\pgfpathmoveto{\pgfqpoint{-0.000000in}{0.000000in}}%
\pgfpathlineto{\pgfqpoint{-0.027778in}{0.000000in}}%
\pgfusepath{stroke,fill}%
}%
\begin{pgfscope}%
\pgfsys@transformshift{0.588387in}{1.724584in}%
\pgfsys@useobject{currentmarker}{}%
\end{pgfscope}%
\end{pgfscope}%
\begin{pgfscope}%
\pgfsetbuttcap%
\pgfsetroundjoin%
\definecolor{currentfill}{rgb}{0.000000,0.000000,0.000000}%
\pgfsetfillcolor{currentfill}%
\pgfsetlinewidth{0.602250pt}%
\definecolor{currentstroke}{rgb}{0.000000,0.000000,0.000000}%
\pgfsetstrokecolor{currentstroke}%
\pgfsetdash{}{0pt}%
\pgfsys@defobject{currentmarker}{\pgfqpoint{-0.027778in}{0.000000in}}{\pgfqpoint{-0.000000in}{0.000000in}}{%
\pgfpathmoveto{\pgfqpoint{-0.000000in}{0.000000in}}%
\pgfpathlineto{\pgfqpoint{-0.027778in}{0.000000in}}%
\pgfusepath{stroke,fill}%
}%
\begin{pgfscope}%
\pgfsys@transformshift{0.588387in}{2.138722in}%
\pgfsys@useobject{currentmarker}{}%
\end{pgfscope}%
\end{pgfscope}%
\begin{pgfscope}%
\pgfsetbuttcap%
\pgfsetroundjoin%
\definecolor{currentfill}{rgb}{0.000000,0.000000,0.000000}%
\pgfsetfillcolor{currentfill}%
\pgfsetlinewidth{0.602250pt}%
\definecolor{currentstroke}{rgb}{0.000000,0.000000,0.000000}%
\pgfsetstrokecolor{currentstroke}%
\pgfsetdash{}{0pt}%
\pgfsys@defobject{currentmarker}{\pgfqpoint{-0.027778in}{0.000000in}}{\pgfqpoint{-0.000000in}{0.000000in}}{%
\pgfpathmoveto{\pgfqpoint{-0.000000in}{0.000000in}}%
\pgfpathlineto{\pgfqpoint{-0.027778in}{0.000000in}}%
\pgfusepath{stroke,fill}%
}%
\begin{pgfscope}%
\pgfsys@transformshift{0.588387in}{2.349013in}%
\pgfsys@useobject{currentmarker}{}%
\end{pgfscope}%
\end{pgfscope}%
\begin{pgfscope}%
\pgfsetbuttcap%
\pgfsetroundjoin%
\definecolor{currentfill}{rgb}{0.000000,0.000000,0.000000}%
\pgfsetfillcolor{currentfill}%
\pgfsetlinewidth{0.602250pt}%
\definecolor{currentstroke}{rgb}{0.000000,0.000000,0.000000}%
\pgfsetstrokecolor{currentstroke}%
\pgfsetdash{}{0pt}%
\pgfsys@defobject{currentmarker}{\pgfqpoint{-0.027778in}{0.000000in}}{\pgfqpoint{-0.000000in}{0.000000in}}{%
\pgfpathmoveto{\pgfqpoint{-0.000000in}{0.000000in}}%
\pgfpathlineto{\pgfqpoint{-0.027778in}{0.000000in}}%
\pgfusepath{stroke,fill}%
}%
\begin{pgfscope}%
\pgfsys@transformshift{0.588387in}{2.498216in}%
\pgfsys@useobject{currentmarker}{}%
\end{pgfscope}%
\end{pgfscope}%
\begin{pgfscope}%
\pgfsetbuttcap%
\pgfsetroundjoin%
\definecolor{currentfill}{rgb}{0.000000,0.000000,0.000000}%
\pgfsetfillcolor{currentfill}%
\pgfsetlinewidth{0.602250pt}%
\definecolor{currentstroke}{rgb}{0.000000,0.000000,0.000000}%
\pgfsetstrokecolor{currentstroke}%
\pgfsetdash{}{0pt}%
\pgfsys@defobject{currentmarker}{\pgfqpoint{-0.027778in}{0.000000in}}{\pgfqpoint{-0.000000in}{0.000000in}}{%
\pgfpathmoveto{\pgfqpoint{-0.000000in}{0.000000in}}%
\pgfpathlineto{\pgfqpoint{-0.027778in}{0.000000in}}%
\pgfusepath{stroke,fill}%
}%
\begin{pgfscope}%
\pgfsys@transformshift{0.588387in}{2.613948in}%
\pgfsys@useobject{currentmarker}{}%
\end{pgfscope}%
\end{pgfscope}%
\begin{pgfscope}%
\pgfsetbuttcap%
\pgfsetroundjoin%
\definecolor{currentfill}{rgb}{0.000000,0.000000,0.000000}%
\pgfsetfillcolor{currentfill}%
\pgfsetlinewidth{0.602250pt}%
\definecolor{currentstroke}{rgb}{0.000000,0.000000,0.000000}%
\pgfsetstrokecolor{currentstroke}%
\pgfsetdash{}{0pt}%
\pgfsys@defobject{currentmarker}{\pgfqpoint{-0.027778in}{0.000000in}}{\pgfqpoint{-0.000000in}{0.000000in}}{%
\pgfpathmoveto{\pgfqpoint{-0.000000in}{0.000000in}}%
\pgfpathlineto{\pgfqpoint{-0.027778in}{0.000000in}}%
\pgfusepath{stroke,fill}%
}%
\begin{pgfscope}%
\pgfsys@transformshift{0.588387in}{2.708507in}%
\pgfsys@useobject{currentmarker}{}%
\end{pgfscope}%
\end{pgfscope}%
\begin{pgfscope}%
\definecolor{textcolor}{rgb}{0.000000,0.000000,0.000000}%
\pgfsetstrokecolor{textcolor}%
\pgfsetfillcolor{textcolor}%
\pgftext[x=0.234413in,y=1.631726in,,bottom,rotate=90.000000]{\color{textcolor}{\rmfamily\fontsize{10.000000}{12.000000}\selectfont\catcode`\^=\active\def^{\ifmmode\sp\else\^{}\fi}\catcode`\%=\active\def%{\%}Time [ms]}}%
\end{pgfscope}%
\begin{pgfscope}%
\pgfpathrectangle{\pgfqpoint{0.588387in}{0.521603in}}{\pgfqpoint{5.300555in}{2.220246in}}%
\pgfusepath{clip}%
\pgfsetrectcap%
\pgfsetroundjoin%
\pgfsetlinewidth{1.505625pt}%
\pgfsetstrokecolor{currentstroke1}%
\pgfsetdash{}{0pt}%
\pgfpathmoveto{\pgfqpoint{0.829322in}{0.643970in}}%
\pgfpathlineto{\pgfqpoint{0.884079in}{0.709519in}}%
\pgfpathlineto{\pgfqpoint{0.938837in}{0.800567in}}%
\pgfpathlineto{\pgfqpoint{0.993595in}{0.806106in}}%
\pgfpathlineto{\pgfqpoint{1.048353in}{0.886143in}}%
\pgfpathlineto{\pgfqpoint{1.103111in}{0.931237in}}%
\pgfpathlineto{\pgfqpoint{1.157868in}{0.913201in}}%
\pgfpathlineto{\pgfqpoint{1.212626in}{0.937149in}}%
\pgfpathlineto{\pgfqpoint{1.267384in}{0.977475in}}%
\pgfpathlineto{\pgfqpoint{1.322142in}{0.980098in}}%
\pgfpathlineto{\pgfqpoint{1.376899in}{1.002687in}}%
\pgfpathlineto{\pgfqpoint{1.431657in}{1.104162in}}%
\pgfpathlineto{\pgfqpoint{1.486415in}{1.056810in}}%
\pgfpathlineto{\pgfqpoint{1.541173in}{1.143786in}}%
\pgfpathlineto{\pgfqpoint{1.595931in}{1.183489in}}%
\pgfpathlineto{\pgfqpoint{1.650688in}{1.212804in}}%
\pgfpathlineto{\pgfqpoint{1.705446in}{1.243424in}}%
\pgfpathlineto{\pgfqpoint{1.760204in}{1.340329in}}%
\pgfpathlineto{\pgfqpoint{1.814962in}{1.375831in}}%
\pgfpathlineto{\pgfqpoint{1.869720in}{1.374360in}}%
\pgfpathlineto{\pgfqpoint{1.924477in}{1.538980in}}%
\pgfpathlineto{\pgfqpoint{1.979235in}{1.529519in}}%
\pgfpathlineto{\pgfqpoint{2.033993in}{1.562821in}}%
\pgfpathlineto{\pgfqpoint{2.088751in}{1.536479in}}%
\pgfpathlineto{\pgfqpoint{2.143509in}{1.666371in}}%
\pgfpathlineto{\pgfqpoint{2.198266in}{1.656434in}}%
\pgfpathlineto{\pgfqpoint{2.253024in}{1.773623in}}%
\pgfpathlineto{\pgfqpoint{2.307782in}{1.796611in}}%
\pgfpathlineto{\pgfqpoint{2.362540in}{1.881297in}}%
\pgfpathlineto{\pgfqpoint{2.417298in}{1.815368in}}%
\pgfpathlineto{\pgfqpoint{2.472055in}{1.914189in}}%
\pgfpathlineto{\pgfqpoint{2.526813in}{2.004388in}}%
\pgfpathlineto{\pgfqpoint{2.581571in}{1.870427in}}%
\pgfpathlineto{\pgfqpoint{2.636329in}{2.087259in}}%
\pgfpathlineto{\pgfqpoint{2.691087in}{2.025094in}}%
\pgfpathlineto{\pgfqpoint{2.745844in}{2.222867in}}%
\pgfpathlineto{\pgfqpoint{2.800602in}{2.347830in}}%
\pgfpathlineto{\pgfqpoint{2.855360in}{2.264724in}}%
\pgfpathlineto{\pgfqpoint{2.910118in}{2.351772in}}%
\pgfpathlineto{\pgfqpoint{2.964876in}{2.329549in}}%
\pgfpathlineto{\pgfqpoint{3.019633in}{2.099549in}}%
\pgfpathlineto{\pgfqpoint{3.074391in}{2.420822in}}%
\pgfpathlineto{\pgfqpoint{3.129149in}{2.352717in}}%
\pgfpathlineto{\pgfqpoint{3.183907in}{2.421975in}}%
\pgfpathlineto{\pgfqpoint{3.238665in}{2.309606in}}%
\pgfpathlineto{\pgfqpoint{3.293422in}{2.402697in}}%
\pgfpathlineto{\pgfqpoint{3.402938in}{2.559476in}}%
\pgfpathlineto{\pgfqpoint{3.621969in}{2.640929in}}%
\pgfusepath{stroke}%
\end{pgfscope}%
\begin{pgfscope}%
\pgfpathrectangle{\pgfqpoint{0.588387in}{0.521603in}}{\pgfqpoint{5.300555in}{2.220246in}}%
\pgfusepath{clip}%
\pgfsetrectcap%
\pgfsetroundjoin%
\pgfsetlinewidth{1.505625pt}%
\pgfsetstrokecolor{currentstroke2}%
\pgfsetdash{}{0pt}%
\pgfpathmoveto{\pgfqpoint{0.829322in}{0.654370in}}%
\pgfpathlineto{\pgfqpoint{0.884079in}{0.719560in}}%
\pgfpathlineto{\pgfqpoint{0.938837in}{0.718026in}}%
\pgfpathlineto{\pgfqpoint{0.993595in}{0.805773in}}%
\pgfpathlineto{\pgfqpoint{1.048353in}{0.891346in}}%
\pgfpathlineto{\pgfqpoint{1.103111in}{0.933898in}}%
\pgfpathlineto{\pgfqpoint{1.157868in}{0.906626in}}%
\pgfpathlineto{\pgfqpoint{1.212626in}{0.910678in}}%
\pgfpathlineto{\pgfqpoint{1.267384in}{0.950581in}}%
\pgfpathlineto{\pgfqpoint{1.322142in}{0.889884in}}%
\pgfpathlineto{\pgfqpoint{1.376899in}{0.916242in}}%
\pgfpathlineto{\pgfqpoint{1.431657in}{0.903755in}}%
\pgfpathlineto{\pgfqpoint{1.486415in}{0.868963in}}%
\pgfpathlineto{\pgfqpoint{1.541173in}{0.887839in}}%
\pgfpathlineto{\pgfqpoint{1.595931in}{0.954518in}}%
\pgfpathlineto{\pgfqpoint{1.650688in}{0.960660in}}%
\pgfpathlineto{\pgfqpoint{1.705446in}{0.978958in}}%
\pgfpathlineto{\pgfqpoint{1.760204in}{0.954563in}}%
\pgfpathlineto{\pgfqpoint{1.814962in}{1.035370in}}%
\pgfpathlineto{\pgfqpoint{1.869720in}{0.987476in}}%
\pgfpathlineto{\pgfqpoint{1.924477in}{1.048195in}}%
\pgfpathlineto{\pgfqpoint{1.979235in}{1.060962in}}%
\pgfpathlineto{\pgfqpoint{2.033993in}{1.029094in}}%
\pgfpathlineto{\pgfqpoint{2.088751in}{1.090883in}}%
\pgfpathlineto{\pgfqpoint{2.143509in}{1.059438in}}%
\pgfpathlineto{\pgfqpoint{2.198266in}{1.114714in}}%
\pgfpathlineto{\pgfqpoint{2.253024in}{1.113137in}}%
\pgfpathlineto{\pgfqpoint{2.307782in}{1.145643in}}%
\pgfpathlineto{\pgfqpoint{2.362540in}{1.116064in}}%
\pgfpathlineto{\pgfqpoint{2.417298in}{1.186584in}}%
\pgfpathlineto{\pgfqpoint{2.472055in}{1.207042in}}%
\pgfpathlineto{\pgfqpoint{2.526813in}{1.256115in}}%
\pgfpathlineto{\pgfqpoint{2.581571in}{1.237576in}}%
\pgfpathlineto{\pgfqpoint{2.636329in}{1.216236in}}%
\pgfpathlineto{\pgfqpoint{2.691087in}{1.257265in}}%
\pgfpathlineto{\pgfqpoint{2.745844in}{1.281237in}}%
\pgfpathlineto{\pgfqpoint{2.800602in}{1.274378in}}%
\pgfpathlineto{\pgfqpoint{2.855360in}{1.309826in}}%
\pgfpathlineto{\pgfqpoint{2.910118in}{1.315702in}}%
\pgfpathlineto{\pgfqpoint{2.964876in}{1.399858in}}%
\pgfpathlineto{\pgfqpoint{3.019633in}{1.370441in}}%
\pgfpathlineto{\pgfqpoint{3.074391in}{1.405498in}}%
\pgfpathlineto{\pgfqpoint{3.129149in}{1.394767in}}%
\pgfpathlineto{\pgfqpoint{3.183907in}{1.399719in}}%
\pgfpathlineto{\pgfqpoint{3.238665in}{1.500986in}}%
\pgfpathlineto{\pgfqpoint{3.293422in}{1.509767in}}%
\pgfpathlineto{\pgfqpoint{3.348180in}{1.495144in}}%
\pgfpathlineto{\pgfqpoint{3.402938in}{1.487805in}}%
\pgfpathlineto{\pgfqpoint{3.457696in}{1.547316in}}%
\pgfpathlineto{\pgfqpoint{3.512454in}{1.690651in}}%
\pgfpathlineto{\pgfqpoint{3.567211in}{1.569683in}}%
\pgfpathlineto{\pgfqpoint{3.621969in}{1.685395in}}%
\pgfpathlineto{\pgfqpoint{3.676727in}{1.698588in}}%
\pgfpathlineto{\pgfqpoint{3.731485in}{1.667887in}}%
\pgfpathlineto{\pgfqpoint{3.786243in}{1.751533in}}%
\pgfpathlineto{\pgfqpoint{3.841000in}{1.666793in}}%
\pgfpathlineto{\pgfqpoint{3.895758in}{1.560177in}}%
\pgfpathlineto{\pgfqpoint{3.950516in}{1.690138in}}%
\pgfpathlineto{\pgfqpoint{4.005274in}{1.572298in}}%
\pgfpathlineto{\pgfqpoint{4.060032in}{1.814077in}}%
\pgfpathlineto{\pgfqpoint{4.114789in}{1.725448in}}%
\pgfpathlineto{\pgfqpoint{4.169547in}{1.825586in}}%
\pgfpathlineto{\pgfqpoint{4.224305in}{1.735137in}}%
\pgfpathlineto{\pgfqpoint{4.279063in}{1.852165in}}%
\pgfpathlineto{\pgfqpoint{4.388578in}{1.800317in}}%
\pgfpathlineto{\pgfqpoint{4.443336in}{1.689727in}}%
\pgfpathlineto{\pgfqpoint{4.498094in}{1.867045in}}%
\pgfpathlineto{\pgfqpoint{4.607609in}{2.049991in}}%
\pgfpathlineto{\pgfqpoint{4.990914in}{2.068281in}}%
\pgfpathlineto{\pgfqpoint{5.100430in}{2.170165in}}%
\pgfpathlineto{\pgfqpoint{5.155187in}{2.420295in}}%
\pgfpathlineto{\pgfqpoint{5.209945in}{2.214580in}}%
\pgfpathlineto{\pgfqpoint{5.483734in}{2.295137in}}%
\pgfpathlineto{\pgfqpoint{5.538492in}{2.291286in}}%
\pgfpathlineto{\pgfqpoint{5.648008in}{2.164521in}}%
\pgfusepath{stroke}%
\end{pgfscope}%
\begin{pgfscope}%
\pgfpathrectangle{\pgfqpoint{0.588387in}{0.521603in}}{\pgfqpoint{5.300555in}{2.220246in}}%
\pgfusepath{clip}%
\pgfsetrectcap%
\pgfsetroundjoin%
\pgfsetlinewidth{1.505625pt}%
\pgfsetstrokecolor{currentstroke3}%
\pgfsetdash{}{0pt}%
\pgfpathmoveto{\pgfqpoint{0.829322in}{0.785370in}}%
\pgfpathlineto{\pgfqpoint{0.884079in}{0.821481in}}%
\pgfpathlineto{\pgfqpoint{0.938837in}{1.187190in}}%
\pgfpathlineto{\pgfqpoint{0.993595in}{1.418741in}}%
\pgfpathlineto{\pgfqpoint{1.048353in}{1.504184in}}%
\pgfpathlineto{\pgfqpoint{1.103111in}{1.673162in}}%
\pgfpathlineto{\pgfqpoint{1.157868in}{1.732291in}}%
\pgfpathlineto{\pgfqpoint{1.212626in}{1.702152in}}%
\pgfpathlineto{\pgfqpoint{1.267384in}{1.926123in}}%
\pgfpathlineto{\pgfqpoint{1.322142in}{1.855267in}}%
\pgfpathlineto{\pgfqpoint{1.376899in}{1.781513in}}%
\pgfpathlineto{\pgfqpoint{1.431657in}{1.850108in}}%
\pgfpathlineto{\pgfqpoint{1.486415in}{2.071202in}}%
\pgfpathlineto{\pgfqpoint{1.541173in}{1.897256in}}%
\pgfpathlineto{\pgfqpoint{1.595931in}{2.115955in}}%
\pgfpathlineto{\pgfqpoint{1.650688in}{1.923071in}}%
\pgfpathlineto{\pgfqpoint{1.705446in}{2.265588in}}%
\pgfpathlineto{\pgfqpoint{1.760204in}{1.868357in}}%
\pgfpathlineto{\pgfqpoint{1.814962in}{2.403242in}}%
\pgfpathlineto{\pgfqpoint{1.869720in}{1.992510in}}%
\pgfpathlineto{\pgfqpoint{1.924477in}{1.862869in}}%
\pgfpathlineto{\pgfqpoint{1.979235in}{1.218493in}}%
\pgfpathlineto{\pgfqpoint{2.033993in}{2.056717in}}%
\pgfpathlineto{\pgfqpoint{2.088751in}{1.800865in}}%
\pgfpathlineto{\pgfqpoint{2.143509in}{0.864777in}}%
\pgfpathlineto{\pgfqpoint{2.198266in}{1.494321in}}%
\pgfpathlineto{\pgfqpoint{2.307782in}{1.359994in}}%
\pgfpathlineto{\pgfqpoint{2.417298in}{1.703512in}}%
\pgfpathlineto{\pgfqpoint{2.526813in}{1.776525in}}%
\pgfpathlineto{\pgfqpoint{2.636329in}{2.396923in}}%
\pgfpathlineto{\pgfqpoint{2.745844in}{1.419216in}}%
\pgfpathlineto{\pgfqpoint{2.855360in}{2.015399in}}%
\pgfpathlineto{\pgfqpoint{2.964876in}{2.149374in}}%
\pgfpathlineto{\pgfqpoint{3.074391in}{2.542198in}}%
\pgfpathlineto{\pgfqpoint{3.183907in}{2.277481in}}%
\pgfpathlineto{\pgfqpoint{3.348180in}{2.622792in}}%
\pgfpathlineto{\pgfqpoint{3.402938in}{1.588281in}}%
\pgfpathlineto{\pgfqpoint{3.621969in}{2.354516in}}%
\pgfpathlineto{\pgfqpoint{3.731485in}{2.429998in}}%
\pgfpathlineto{\pgfqpoint{3.841000in}{1.690651in}}%
\pgfpathlineto{\pgfqpoint{4.279063in}{1.837310in}}%
\pgfpathlineto{\pgfqpoint{4.498094in}{1.875657in}}%
\pgfusepath{stroke}%
\end{pgfscope}%
\begin{pgfscope}%
\pgfpathrectangle{\pgfqpoint{0.588387in}{0.521603in}}{\pgfqpoint{5.300555in}{2.220246in}}%
\pgfusepath{clip}%
\pgfsetrectcap%
\pgfsetroundjoin%
\pgfsetlinewidth{1.505625pt}%
\pgfsetstrokecolor{currentstroke4}%
\pgfsetdash{}{0pt}%
\pgfpathmoveto{\pgfqpoint{0.829322in}{0.622524in}}%
\pgfpathlineto{\pgfqpoint{0.884079in}{0.705696in}}%
\pgfpathlineto{\pgfqpoint{0.938837in}{0.729954in}}%
\pgfpathlineto{\pgfqpoint{0.993595in}{0.802729in}}%
\pgfpathlineto{\pgfqpoint{1.048353in}{0.898260in}}%
\pgfpathlineto{\pgfqpoint{1.103111in}{0.925843in}}%
\pgfpathlineto{\pgfqpoint{1.157868in}{0.902307in}}%
\pgfpathlineto{\pgfqpoint{1.212626in}{0.913535in}}%
\pgfpathlineto{\pgfqpoint{1.267384in}{0.937932in}}%
\pgfpathlineto{\pgfqpoint{1.322142in}{0.890865in}}%
\pgfpathlineto{\pgfqpoint{1.376899in}{0.927003in}}%
\pgfpathlineto{\pgfqpoint{1.431657in}{0.898438in}}%
\pgfpathlineto{\pgfqpoint{1.486415in}{0.864020in}}%
\pgfpathlineto{\pgfqpoint{1.541173in}{0.886327in}}%
\pgfpathlineto{\pgfqpoint{1.595931in}{0.959492in}}%
\pgfpathlineto{\pgfqpoint{1.650688in}{0.948384in}}%
\pgfpathlineto{\pgfqpoint{1.705446in}{0.963143in}}%
\pgfpathlineto{\pgfqpoint{1.760204in}{0.954435in}}%
\pgfpathlineto{\pgfqpoint{1.814962in}{1.028037in}}%
\pgfpathlineto{\pgfqpoint{1.869720in}{0.976240in}}%
\pgfpathlineto{\pgfqpoint{1.924477in}{1.027781in}}%
\pgfpathlineto{\pgfqpoint{1.979235in}{1.040363in}}%
\pgfpathlineto{\pgfqpoint{2.033993in}{1.010853in}}%
\pgfpathlineto{\pgfqpoint{2.088751in}{1.070511in}}%
\pgfpathlineto{\pgfqpoint{2.143509in}{1.049030in}}%
\pgfpathlineto{\pgfqpoint{2.198266in}{1.140237in}}%
\pgfpathlineto{\pgfqpoint{2.253024in}{1.089746in}}%
\pgfpathlineto{\pgfqpoint{2.307782in}{1.127854in}}%
\pgfpathlineto{\pgfqpoint{2.362540in}{1.098231in}}%
\pgfpathlineto{\pgfqpoint{2.417298in}{1.191286in}}%
\pgfpathlineto{\pgfqpoint{2.472055in}{1.181318in}}%
\pgfpathlineto{\pgfqpoint{2.526813in}{1.230203in}}%
\pgfpathlineto{\pgfqpoint{2.581571in}{1.221059in}}%
\pgfpathlineto{\pgfqpoint{2.636329in}{1.186683in}}%
\pgfpathlineto{\pgfqpoint{2.691087in}{1.242631in}}%
\pgfpathlineto{\pgfqpoint{2.745844in}{1.232684in}}%
\pgfpathlineto{\pgfqpoint{2.800602in}{1.244252in}}%
\pgfpathlineto{\pgfqpoint{2.855360in}{1.255245in}}%
\pgfpathlineto{\pgfqpoint{2.910118in}{1.274606in}}%
\pgfpathlineto{\pgfqpoint{2.964876in}{1.341029in}}%
\pgfpathlineto{\pgfqpoint{3.019633in}{1.329719in}}%
\pgfpathlineto{\pgfqpoint{3.074391in}{1.348949in}}%
\pgfpathlineto{\pgfqpoint{3.129149in}{1.367688in}}%
\pgfpathlineto{\pgfqpoint{3.183907in}{1.352199in}}%
\pgfpathlineto{\pgfqpoint{3.238665in}{1.469292in}}%
\pgfpathlineto{\pgfqpoint{3.293422in}{1.430924in}}%
\pgfpathlineto{\pgfqpoint{3.348180in}{1.469428in}}%
\pgfpathlineto{\pgfqpoint{3.402938in}{1.461886in}}%
\pgfpathlineto{\pgfqpoint{3.457696in}{1.497055in}}%
\pgfpathlineto{\pgfqpoint{3.512454in}{1.559253in}}%
\pgfpathlineto{\pgfqpoint{3.567211in}{1.520381in}}%
\pgfpathlineto{\pgfqpoint{3.621969in}{1.714008in}}%
\pgfpathlineto{\pgfqpoint{3.676727in}{1.639277in}}%
\pgfpathlineto{\pgfqpoint{3.731485in}{1.569249in}}%
\pgfpathlineto{\pgfqpoint{3.786243in}{1.730220in}}%
\pgfpathlineto{\pgfqpoint{3.841000in}{1.593551in}}%
\pgfpathlineto{\pgfqpoint{3.895758in}{1.544514in}}%
\pgfpathlineto{\pgfqpoint{3.950516in}{1.629371in}}%
\pgfpathlineto{\pgfqpoint{4.005274in}{1.559253in}}%
\pgfpathlineto{\pgfqpoint{4.060032in}{1.806259in}}%
\pgfpathlineto{\pgfqpoint{4.114789in}{1.672920in}}%
\pgfpathlineto{\pgfqpoint{4.169547in}{1.765388in}}%
\pgfpathlineto{\pgfqpoint{4.224305in}{1.642100in}}%
\pgfpathlineto{\pgfqpoint{4.279063in}{1.768971in}}%
\pgfpathlineto{\pgfqpoint{4.388578in}{1.760930in}}%
\pgfpathlineto{\pgfqpoint{4.443336in}{1.677357in}}%
\pgfpathlineto{\pgfqpoint{4.498094in}{1.786695in}}%
\pgfpathlineto{\pgfqpoint{4.607609in}{2.022451in}}%
\pgfpathlineto{\pgfqpoint{4.990914in}{2.023261in}}%
\pgfpathlineto{\pgfqpoint{5.100430in}{2.116469in}}%
\pgfpathlineto{\pgfqpoint{5.155187in}{2.347339in}}%
\pgfpathlineto{\pgfqpoint{5.209945in}{2.165425in}}%
\pgfpathlineto{\pgfqpoint{5.483734in}{2.242633in}}%
\pgfpathlineto{\pgfqpoint{5.538492in}{2.152203in}}%
\pgfpathlineto{\pgfqpoint{5.648008in}{2.072718in}}%
\pgfusepath{stroke}%
\end{pgfscope}%
\begin{pgfscope}%
\pgfpathrectangle{\pgfqpoint{0.588387in}{0.521603in}}{\pgfqpoint{5.300555in}{2.220246in}}%
\pgfusepath{clip}%
\pgfsetrectcap%
\pgfsetroundjoin%
\pgfsetlinewidth{1.505625pt}%
\pgfsetstrokecolor{currentstroke5}%
\pgfsetdash{}{0pt}%
\pgfpathmoveto{\pgfqpoint{0.829322in}{0.632266in}}%
\pgfpathlineto{\pgfqpoint{0.884079in}{0.713285in}}%
\pgfpathlineto{\pgfqpoint{0.938837in}{0.718107in}}%
\pgfpathlineto{\pgfqpoint{0.993595in}{0.794642in}}%
\pgfpathlineto{\pgfqpoint{1.048353in}{0.887162in}}%
\pgfpathlineto{\pgfqpoint{1.103111in}{0.925160in}}%
\pgfpathlineto{\pgfqpoint{1.157868in}{0.909263in}}%
\pgfpathlineto{\pgfqpoint{1.212626in}{0.908148in}}%
\pgfpathlineto{\pgfqpoint{1.267384in}{0.942491in}}%
\pgfpathlineto{\pgfqpoint{1.322142in}{0.888232in}}%
\pgfpathlineto{\pgfqpoint{1.376899in}{0.911313in}}%
\pgfpathlineto{\pgfqpoint{1.431657in}{0.896274in}}%
\pgfpathlineto{\pgfqpoint{1.486415in}{0.872550in}}%
\pgfpathlineto{\pgfqpoint{1.541173in}{0.874942in}}%
\pgfpathlineto{\pgfqpoint{1.595931in}{0.944897in}}%
\pgfpathlineto{\pgfqpoint{1.650688in}{0.950479in}}%
\pgfpathlineto{\pgfqpoint{1.705446in}{0.967133in}}%
\pgfpathlineto{\pgfqpoint{1.760204in}{0.943405in}}%
\pgfpathlineto{\pgfqpoint{1.814962in}{1.011123in}}%
\pgfpathlineto{\pgfqpoint{1.869720in}{0.972979in}}%
\pgfpathlineto{\pgfqpoint{1.924477in}{1.018122in}}%
\pgfpathlineto{\pgfqpoint{1.979235in}{1.030925in}}%
\pgfpathlineto{\pgfqpoint{2.033993in}{1.009563in}}%
\pgfpathlineto{\pgfqpoint{2.088751in}{1.066226in}}%
\pgfpathlineto{\pgfqpoint{2.143509in}{1.036063in}}%
\pgfpathlineto{\pgfqpoint{2.198266in}{1.099289in}}%
\pgfpathlineto{\pgfqpoint{2.253024in}{1.080582in}}%
\pgfpathlineto{\pgfqpoint{2.307782in}{1.115377in}}%
\pgfpathlineto{\pgfqpoint{2.362540in}{1.092364in}}%
\pgfpathlineto{\pgfqpoint{2.417298in}{1.195802in}}%
\pgfpathlineto{\pgfqpoint{2.472055in}{1.173821in}}%
\pgfpathlineto{\pgfqpoint{2.526813in}{1.247338in}}%
\pgfpathlineto{\pgfqpoint{2.581571in}{1.200666in}}%
\pgfpathlineto{\pgfqpoint{2.636329in}{1.192576in}}%
\pgfpathlineto{\pgfqpoint{2.691087in}{1.240815in}}%
\pgfpathlineto{\pgfqpoint{2.745844in}{1.212694in}}%
\pgfpathlineto{\pgfqpoint{2.800602in}{1.252411in}}%
\pgfpathlineto{\pgfqpoint{2.855360in}{1.260303in}}%
\pgfpathlineto{\pgfqpoint{2.910118in}{1.275006in}}%
\pgfpathlineto{\pgfqpoint{2.964876in}{1.341451in}}%
\pgfpathlineto{\pgfqpoint{3.019633in}{1.377428in}}%
\pgfpathlineto{\pgfqpoint{3.074391in}{1.336571in}}%
\pgfpathlineto{\pgfqpoint{3.129149in}{1.372591in}}%
\pgfpathlineto{\pgfqpoint{3.183907in}{1.351326in}}%
\pgfpathlineto{\pgfqpoint{3.238665in}{1.430513in}}%
\pgfpathlineto{\pgfqpoint{3.293422in}{1.427164in}}%
\pgfpathlineto{\pgfqpoint{3.348180in}{1.465776in}}%
\pgfpathlineto{\pgfqpoint{3.402938in}{1.438202in}}%
\pgfpathlineto{\pgfqpoint{3.457696in}{1.493722in}}%
\pgfpathlineto{\pgfqpoint{3.512454in}{1.538774in}}%
\pgfpathlineto{\pgfqpoint{3.567211in}{1.532136in}}%
\pgfpathlineto{\pgfqpoint{3.621969in}{1.649701in}}%
\pgfpathlineto{\pgfqpoint{3.676727in}{1.630889in}}%
\pgfpathlineto{\pgfqpoint{3.731485in}{1.573379in}}%
\pgfpathlineto{\pgfqpoint{3.786243in}{1.702210in}}%
\pgfpathlineto{\pgfqpoint{3.841000in}{1.597860in}}%
\pgfpathlineto{\pgfqpoint{3.895758in}{1.558989in}}%
\pgfpathlineto{\pgfqpoint{3.950516in}{1.642325in}}%
\pgfpathlineto{\pgfqpoint{4.005274in}{1.555807in}}%
\pgfpathlineto{\pgfqpoint{4.060032in}{1.735701in}}%
\pgfpathlineto{\pgfqpoint{4.114789in}{1.659810in}}%
\pgfpathlineto{\pgfqpoint{4.169547in}{1.770336in}}%
\pgfpathlineto{\pgfqpoint{4.224305in}{1.645523in}}%
\pgfpathlineto{\pgfqpoint{4.279063in}{1.775455in}}%
\pgfpathlineto{\pgfqpoint{4.388578in}{1.752626in}}%
\pgfpathlineto{\pgfqpoint{4.443336in}{1.646696in}}%
\pgfpathlineto{\pgfqpoint{4.498094in}{1.763565in}}%
\pgfpathlineto{\pgfqpoint{4.607609in}{1.939655in}}%
\pgfpathlineto{\pgfqpoint{4.990914in}{2.015153in}}%
\pgfpathlineto{\pgfqpoint{5.100430in}{2.098849in}}%
\pgfpathlineto{\pgfqpoint{5.155187in}{2.260640in}}%
\pgfpathlineto{\pgfqpoint{5.209945in}{2.173692in}}%
\pgfpathlineto{\pgfqpoint{5.483734in}{2.142083in}}%
\pgfpathlineto{\pgfqpoint{5.538492in}{2.140922in}}%
\pgfpathlineto{\pgfqpoint{5.648008in}{2.073012in}}%
\pgfusepath{stroke}%
\end{pgfscope}%
\begin{pgfscope}%
\pgfpathrectangle{\pgfqpoint{0.588387in}{0.521603in}}{\pgfqpoint{5.300555in}{2.220246in}}%
\pgfusepath{clip}%
\pgfsetrectcap%
\pgfsetroundjoin%
\pgfsetlinewidth{1.505625pt}%
\pgfsetstrokecolor{currentstroke6}%
\pgfsetdash{}{0pt}%
\pgfpathmoveto{\pgfqpoint{0.829322in}{0.637971in}}%
\pgfpathlineto{\pgfqpoint{0.884079in}{0.709339in}}%
\pgfpathlineto{\pgfqpoint{0.938837in}{0.728483in}}%
\pgfpathlineto{\pgfqpoint{0.993595in}{0.784401in}}%
\pgfpathlineto{\pgfqpoint{1.048353in}{0.884537in}}%
\pgfpathlineto{\pgfqpoint{1.103111in}{0.908644in}}%
\pgfpathlineto{\pgfqpoint{1.157868in}{0.888167in}}%
\pgfpathlineto{\pgfqpoint{1.212626in}{0.897781in}}%
\pgfpathlineto{\pgfqpoint{1.267384in}{0.927381in}}%
\pgfpathlineto{\pgfqpoint{1.322142in}{0.891769in}}%
\pgfpathlineto{\pgfqpoint{1.376899in}{0.904557in}}%
\pgfpathlineto{\pgfqpoint{1.431657in}{0.887954in}}%
\pgfpathlineto{\pgfqpoint{1.486415in}{0.846665in}}%
\pgfpathlineto{\pgfqpoint{1.541173in}{0.871022in}}%
\pgfpathlineto{\pgfqpoint{1.595931in}{0.943228in}}%
\pgfpathlineto{\pgfqpoint{1.650688in}{0.932103in}}%
\pgfpathlineto{\pgfqpoint{1.705446in}{0.960288in}}%
\pgfpathlineto{\pgfqpoint{1.760204in}{0.934946in}}%
\pgfpathlineto{\pgfqpoint{1.814962in}{1.023456in}}%
\pgfpathlineto{\pgfqpoint{1.869720in}{0.962925in}}%
\pgfpathlineto{\pgfqpoint{1.924477in}{1.033240in}}%
\pgfpathlineto{\pgfqpoint{1.979235in}{1.032873in}}%
\pgfpathlineto{\pgfqpoint{2.033993in}{0.995482in}}%
\pgfpathlineto{\pgfqpoint{2.088751in}{1.042010in}}%
\pgfpathlineto{\pgfqpoint{2.143509in}{1.038503in}}%
\pgfpathlineto{\pgfqpoint{2.198266in}{1.074105in}}%
\pgfpathlineto{\pgfqpoint{2.253024in}{1.072760in}}%
\pgfpathlineto{\pgfqpoint{2.307782in}{1.109371in}}%
\pgfpathlineto{\pgfqpoint{2.362540in}{1.082154in}}%
\pgfpathlineto{\pgfqpoint{2.417298in}{1.209135in}}%
\pgfpathlineto{\pgfqpoint{2.472055in}{1.174151in}}%
\pgfpathlineto{\pgfqpoint{2.526813in}{1.193892in}}%
\pgfpathlineto{\pgfqpoint{2.581571in}{1.179668in}}%
\pgfpathlineto{\pgfqpoint{2.636329in}{1.171736in}}%
\pgfpathlineto{\pgfqpoint{2.691087in}{1.217084in}}%
\pgfpathlineto{\pgfqpoint{2.745844in}{1.218597in}}%
\pgfpathlineto{\pgfqpoint{2.800602in}{1.226492in}}%
\pgfpathlineto{\pgfqpoint{2.855360in}{1.239239in}}%
\pgfpathlineto{\pgfqpoint{2.910118in}{1.236105in}}%
\pgfpathlineto{\pgfqpoint{2.964876in}{1.316050in}}%
\pgfpathlineto{\pgfqpoint{3.019633in}{1.328209in}}%
\pgfpathlineto{\pgfqpoint{3.074391in}{1.330541in}}%
\pgfpathlineto{\pgfqpoint{3.129149in}{1.346624in}}%
\pgfpathlineto{\pgfqpoint{3.183907in}{1.335155in}}%
\pgfpathlineto{\pgfqpoint{3.238665in}{1.425306in}}%
\pgfpathlineto{\pgfqpoint{3.293422in}{1.371418in}}%
\pgfpathlineto{\pgfqpoint{3.348180in}{1.427683in}}%
\pgfpathlineto{\pgfqpoint{3.402938in}{1.474716in}}%
\pgfpathlineto{\pgfqpoint{3.457696in}{1.450632in}}%
\pgfpathlineto{\pgfqpoint{3.512454in}{1.548980in}}%
\pgfpathlineto{\pgfqpoint{3.567211in}{1.506236in}}%
\pgfpathlineto{\pgfqpoint{3.621969in}{1.604876in}}%
\pgfpathlineto{\pgfqpoint{3.676727in}{1.590400in}}%
\pgfpathlineto{\pgfqpoint{3.731485in}{1.577068in}}%
\pgfpathlineto{\pgfqpoint{3.786243in}{1.646808in}}%
\pgfpathlineto{\pgfqpoint{3.841000in}{1.576813in}}%
\pgfpathlineto{\pgfqpoint{3.895758in}{1.543970in}}%
\pgfpathlineto{\pgfqpoint{3.950516in}{1.620252in}}%
\pgfpathlineto{\pgfqpoint{4.005274in}{1.524847in}}%
\pgfpathlineto{\pgfqpoint{4.060032in}{1.746861in}}%
\pgfpathlineto{\pgfqpoint{4.114789in}{1.642025in}}%
\pgfpathlineto{\pgfqpoint{4.169547in}{1.761825in}}%
\pgfpathlineto{\pgfqpoint{4.224305in}{1.631176in}}%
\pgfpathlineto{\pgfqpoint{4.279063in}{1.727696in}}%
\pgfpathlineto{\pgfqpoint{4.388578in}{1.772617in}}%
\pgfpathlineto{\pgfqpoint{4.443336in}{1.601599in}}%
\pgfpathlineto{\pgfqpoint{4.498094in}{1.773492in}}%
\pgfpathlineto{\pgfqpoint{4.607609in}{2.007973in}}%
\pgfpathlineto{\pgfqpoint{4.990914in}{1.951633in}}%
\pgfpathlineto{\pgfqpoint{5.100430in}{2.141653in}}%
\pgfpathlineto{\pgfqpoint{5.155187in}{2.299466in}}%
\pgfpathlineto{\pgfqpoint{5.209945in}{2.141825in}}%
\pgfpathlineto{\pgfqpoint{5.483734in}{2.169921in}}%
\pgfpathlineto{\pgfqpoint{5.538492in}{2.177917in}}%
\pgfpathlineto{\pgfqpoint{5.648008in}{2.067885in}}%
\pgfusepath{stroke}%
\end{pgfscope}%
\begin{pgfscope}%
\pgfsetrectcap%
\pgfsetmiterjoin%
\pgfsetlinewidth{0.803000pt}%
\definecolor{currentstroke}{rgb}{0.000000,0.000000,0.000000}%
\pgfsetstrokecolor{currentstroke}%
\pgfsetdash{}{0pt}%
\pgfpathmoveto{\pgfqpoint{0.588387in}{0.521603in}}%
\pgfpathlineto{\pgfqpoint{0.588387in}{2.741849in}}%
\pgfusepath{stroke}%
\end{pgfscope}%
\begin{pgfscope}%
\pgfsetrectcap%
\pgfsetmiterjoin%
\pgfsetlinewidth{0.803000pt}%
\definecolor{currentstroke}{rgb}{0.000000,0.000000,0.000000}%
\pgfsetstrokecolor{currentstroke}%
\pgfsetdash{}{0pt}%
\pgfpathmoveto{\pgfqpoint{5.888942in}{0.521603in}}%
\pgfpathlineto{\pgfqpoint{5.888942in}{2.741849in}}%
\pgfusepath{stroke}%
\end{pgfscope}%
\begin{pgfscope}%
\pgfsetrectcap%
\pgfsetmiterjoin%
\pgfsetlinewidth{0.803000pt}%
\definecolor{currentstroke}{rgb}{0.000000,0.000000,0.000000}%
\pgfsetstrokecolor{currentstroke}%
\pgfsetdash{}{0pt}%
\pgfpathmoveto{\pgfqpoint{0.588387in}{0.521603in}}%
\pgfpathlineto{\pgfqpoint{5.888942in}{0.521603in}}%
\pgfusepath{stroke}%
\end{pgfscope}%
\begin{pgfscope}%
\pgfsetrectcap%
\pgfsetmiterjoin%
\pgfsetlinewidth{0.803000pt}%
\definecolor{currentstroke}{rgb}{0.000000,0.000000,0.000000}%
\pgfsetstrokecolor{currentstroke}%
\pgfsetdash{}{0pt}%
\pgfpathmoveto{\pgfqpoint{0.588387in}{2.741849in}}%
\pgfpathlineto{\pgfqpoint{5.888942in}{2.741849in}}%
\pgfusepath{stroke}%
\end{pgfscope}%
\begin{pgfscope}%
\pgfsetbuttcap%
\pgfsetmiterjoin%
\definecolor{currentfill}{rgb}{1.000000,1.000000,1.000000}%
\pgfsetfillcolor{currentfill}%
\pgfsetfillopacity{0.800000}%
\pgfsetlinewidth{1.003750pt}%
\definecolor{currentstroke}{rgb}{0.800000,0.800000,0.800000}%
\pgfsetstrokecolor{currentstroke}%
\pgfsetstrokeopacity{0.800000}%
\pgfsetdash{}{0pt}%
\pgfpathmoveto{\pgfqpoint{5.976442in}{1.530583in}}%
\pgfpathlineto{\pgfqpoint{8.259376in}{1.530583in}}%
\pgfpathquadraticcurveto{\pgfqpoint{8.284376in}{1.530583in}}{\pgfqpoint{8.284376in}{1.555583in}}%
\pgfpathlineto{\pgfqpoint{8.284376in}{2.654349in}}%
\pgfpathquadraticcurveto{\pgfqpoint{8.284376in}{2.679349in}}{\pgfqpoint{8.259376in}{2.679349in}}%
\pgfpathlineto{\pgfqpoint{5.976442in}{2.679349in}}%
\pgfpathquadraticcurveto{\pgfqpoint{5.951442in}{2.679349in}}{\pgfqpoint{5.951442in}{2.654349in}}%
\pgfpathlineto{\pgfqpoint{5.951442in}{1.555583in}}%
\pgfpathquadraticcurveto{\pgfqpoint{5.951442in}{1.530583in}}{\pgfqpoint{5.976442in}{1.530583in}}%
\pgfpathlineto{\pgfqpoint{5.976442in}{1.530583in}}%
\pgfpathclose%
\pgfusepath{stroke,fill}%
\end{pgfscope}%
\begin{pgfscope}%
\pgfsetrectcap%
\pgfsetroundjoin%
\pgfsetlinewidth{1.505625pt}%
\pgfsetstrokecolor{currentstroke2}%
\pgfsetdash{}{0pt}%
\pgfpathmoveto{\pgfqpoint{6.001442in}{2.578129in}}%
\pgfpathlineto{\pgfqpoint{6.126442in}{2.578129in}}%
\pgfpathlineto{\pgfqpoint{6.251442in}{2.578129in}}%
\pgfusepath{stroke}%
\end{pgfscope}%
\begin{pgfscope}%
\definecolor{textcolor}{rgb}{0.000000,0.000000,0.000000}%
\pgfsetstrokecolor{textcolor}%
\pgfsetfillcolor{textcolor}%
\pgftext[x=6.351442in,y=2.534379in,left,base]{\color{textcolor}{\rmfamily\fontsize{9.000000}{10.800000}\selectfont\catcode`\^=\active\def^{\ifmmode\sp\else\^{}\fi}\catcode`\%=\active\def%{\%}\CyclesMatchChunks{} \& \MergeLinear{}}}%
\end{pgfscope}%
\begin{pgfscope}%
\pgfsetrectcap%
\pgfsetroundjoin%
\pgfsetlinewidth{1.505625pt}%
\pgfsetstrokecolor{currentstroke4}%
\pgfsetdash{}{0pt}%
\pgfpathmoveto{\pgfqpoint{6.001442in}{2.391178in}}%
\pgfpathlineto{\pgfqpoint{6.126442in}{2.391178in}}%
\pgfpathlineto{\pgfqpoint{6.251442in}{2.391178in}}%
\pgfusepath{stroke}%
\end{pgfscope}%
\begin{pgfscope}%
\definecolor{textcolor}{rgb}{0.000000,0.000000,0.000000}%
\pgfsetstrokecolor{textcolor}%
\pgfsetfillcolor{textcolor}%
\pgftext[x=6.351442in,y=2.347428in,left,base]{\color{textcolor}{\rmfamily\fontsize{9.000000}{10.800000}\selectfont\catcode`\^=\active\def^{\ifmmode\sp\else\^{}\fi}\catcode`\%=\active\def%{\%}\Neighbors{} \& \MergeLinear{}}}%
\end{pgfscope}%
\begin{pgfscope}%
\pgfsetrectcap%
\pgfsetroundjoin%
\pgfsetlinewidth{1.505625pt}%
\pgfsetstrokecolor{currentstroke5}%
\pgfsetdash{}{0pt}%
\pgfpathmoveto{\pgfqpoint{6.001442in}{2.207707in}}%
\pgfpathlineto{\pgfqpoint{6.126442in}{2.207707in}}%
\pgfpathlineto{\pgfqpoint{6.251442in}{2.207707in}}%
\pgfusepath{stroke}%
\end{pgfscope}%
\begin{pgfscope}%
\definecolor{textcolor}{rgb}{0.000000,0.000000,0.000000}%
\pgfsetstrokecolor{textcolor}%
\pgfsetfillcolor{textcolor}%
\pgftext[x=6.351442in,y=2.163957in,left,base]{\color{textcolor}{\rmfamily\fontsize{9.000000}{10.800000}\selectfont\catcode`\^=\active\def^{\ifmmode\sp\else\^{}\fi}\catcode`\%=\active\def%{\%}\NeighborsDegree{} \& \MergeLinear{}}}%
\end{pgfscope}%
\begin{pgfscope}%
\pgfsetrectcap%
\pgfsetroundjoin%
\pgfsetlinewidth{1.505625pt}%
\pgfsetstrokecolor{currentstroke6}%
\pgfsetdash{}{0pt}%
\pgfpathmoveto{\pgfqpoint{6.001442in}{2.020756in}}%
\pgfpathlineto{\pgfqpoint{6.126442in}{2.020756in}}%
\pgfpathlineto{\pgfqpoint{6.251442in}{2.020756in}}%
\pgfusepath{stroke}%
\end{pgfscope}%
\begin{pgfscope}%
\definecolor{textcolor}{rgb}{0.000000,0.000000,0.000000}%
\pgfsetstrokecolor{textcolor}%
\pgfsetfillcolor{textcolor}%
\pgftext[x=6.351442in,y=1.977006in,left,base]{\color{textcolor}{\rmfamily\fontsize{9.000000}{10.800000}\selectfont\catcode`\^=\active\def^{\ifmmode\sp\else\^{}\fi}\catcode`\%=\active\def%{\%}\None{} \& \MergeLinear{}}}%
\end{pgfscope}%
\begin{pgfscope}%
\pgfsetrectcap%
\pgfsetroundjoin%
\pgfsetlinewidth{1.505625pt}%
\pgfsetstrokecolor{currentstroke1}%
\pgfsetdash{}{0pt}%
\pgfpathmoveto{\pgfqpoint{6.001442in}{1.837285in}}%
\pgfpathlineto{\pgfqpoint{6.126442in}{1.837285in}}%
\pgfpathlineto{\pgfqpoint{6.251442in}{1.837285in}}%
\pgfusepath{stroke}%
\end{pgfscope}%
\begin{pgfscope}%
\definecolor{textcolor}{rgb}{0.000000,0.000000,0.000000}%
\pgfsetstrokecolor{textcolor}%
\pgfsetfillcolor{textcolor}%
\pgftext[x=6.351442in,y=1.793535in,left,base]{\color{textcolor}{\rmfamily\fontsize{9.000000}{10.800000}\selectfont\catcode`\^=\active\def^{\ifmmode\sp\else\^{}\fi}\catcode`\%=\active\def%{\%}\Cuts{} \& \MergeLinear{}}}%
\end{pgfscope}%
\begin{pgfscope}%
\pgfsetrectcap%
\pgfsetroundjoin%
\pgfsetlinewidth{1.505625pt}%
\pgfsetstrokecolor{currentstroke3}%
\pgfsetdash{}{0pt}%
\pgfpathmoveto{\pgfqpoint{6.001442in}{1.653813in}}%
\pgfpathlineto{\pgfqpoint{6.126442in}{1.653813in}}%
\pgfpathlineto{\pgfqpoint{6.251442in}{1.653813in}}%
\pgfusepath{stroke}%
\end{pgfscope}%
\begin{pgfscope}%
\definecolor{textcolor}{rgb}{0.000000,0.000000,0.000000}%
\pgfsetstrokecolor{textcolor}%
\pgfsetfillcolor{textcolor}%
\pgftext[x=6.351442in,y=1.610063in,left,base]{\color{textcolor}{\rmfamily\fontsize{9.000000}{10.800000}\selectfont\catcode`\^=\active\def^{\ifmmode\sp\else\^{}\fi}\catcode`\%=\active\def%{\%}\KernighanLin{} \& \MergeLinear{}}}%
\end{pgfscope}%
\end{pgfpicture}%
\makeatother%
\endgroup%
}
	\caption[Other splitting strategies for graphs with no NAC-coloring]{
		Mean running time to finish search for graphs with no NAC-coloring with other splitting strategies.}%
	\label{fig:graph_no_nac_coloring_generated_rigid_failing_split_first_runtime}
\end{figure}%


Smart split described in \Cref{sec:smart_split}
did not improve the runtime.
We expected minor performance hit for smaller graphs because heuristic is run
multiple times, but gains for larger graphs where subgraphs merging order
should join subgraphs near to each other together. This is not the case.

\subsection{Final comparison}

Based on our benchmarks, most of which we presented in the previous sections,
we evaluate respective strategies and choose ones
that should we recommend are kept and merged into PyRigi.
%
For graph classes with a lot of NAC-colorings,
\NaiveCycles{} is usually the best choice
when we search for a single NAC-coloring.
%
The user of the library has to pay attention while using this strategy
as if there is just a few or no NAC-coloring and the graph does not trivially collapse
into few monochromatic classes, the runtime is huge.
Therefore, \NaiveCycles{} should be an available option in PyRigi,
but the default algorithm should be based on \Subgraphs{}.

First we evaluate strategies for splitting:
Strategy \None{} works well in most of the cases and for simple
instances it outperforms other strategies as it requires little to no overhead.
%
\CyclesMatchChunks{} perform well in the majority cases similar to \None{},
but it is generally outperformed by \Neighbors{}.
%
\Neighbors{} and \NeighborsDegree{} generally managed to reduce the number
of \IsNACColoring{} calls and for complex cases they often
perform slightly better then \None{}, usually for more complex instances.
Note that the implementation of \Neighbors{} is slightly simpler than
the one of \NeighborsDegree{}.
%
\KernighanLin{} results are not consistent.
\Cuts{} perform even worse for even small instance.
This is probably because they operate on a graph of monochromatic
components that does not preserve all the properties
of the original graph well.
They were never noticeably faster than \Neighbors{}.
Therefore, these algorithms cannot be used in general case.
%
Based on this evaluation, we recommend strategies \None{} and \Neighbors{}
as they perform the best across all graph classes and
as \Neighbors{} strategy has simpler implementation than \NeighborsDegree{}.

\MergeLinear{} strategy performed the best across the board.
It seems that the idea of merging one growing subgraph
with subgraphs of initial small size works the best
in a general case.
%
For listing all NAC-colorings, \SharedVertices{} performs
slightly better, but for a search for a single NAC-coloring for simple instances,
its runtime becomes less predictable.
%
From the general idea of the \Subgraphs{} algorithm,
it makes a lot of sense \Log{} strategy would work great.
This is not the case as \Log{} is slow for all the cases we tested.
%
\MinMax{} and \SortedBits{} strategies often perform badly
and if they do not, they do not outperform other strategies.
%
\PromisingCycles{} performs similarly well as \Neighbors{}
for graphs with no NAC-coloring,
but fails for simple instances which makes it not universal enough.
%
\SortedSize{} and \Score{} also perform well for graphs with no NAC-coloring,
but as explained in \Cref{sec:merging},
they are unsuitable for instances where only a single NAC-coloring
should be found as these strategies always list all NAC-colorings
on all subgraphs.
%
Based on this evaluation, we recommend strategy \MergeLinear{} as a general good choice.
We also recommend \SharedVertices{} for graphs that are not simple instances.

