
\chapter{Stable cuts}%
\label{chapter:stable_cuts}

\begin{chapterabstract}
	%
	We introduce relation between stable cuts and NAC-colorings.
	We summarize the progress done on stable cuts existence, search and complexity.
	Lastly, we briefly describe an algorithm that can find a stable cut
	for any flexible graph.
	%
\end{chapterabstract}


In this chapter, we focus on connected graphs only as for us graph cuts
lose their importance on disconnected graphs.
We recall the definition of
stable cut in~\Cref{def:stable_cut}.

\section{Properties}

Notice how stable cuts are related to NAC-colorings.
%
\begin{lemma}[\cite{legersky_original}]
	\label[lemma]{lemma:stable_cut_implies_nac_coloring}
	Let \( S \) be a stable cut in a connected graph \( G \).
	Then \( G \) has a NAC-coloring.
\end{lemma}
%
\begin{proof}
	For a vertex cut \( S \subseteq V(G) \), let us have \( G^\prime = G \setminus S \).
	Nonempty sets \( A, B \subsetneq V(G) \) are called \emph{partitions} of the cut \( S \)
	if \( A \cup B = V(G), A \cap B = S \) and for each \( v \in A \setminus S, u \in B \setminus S \)
	there exists no \( u \)--\( v \)--path in \( G^\prime \).
	%
	As \( S \) is a cut in \( G \), let us have some partitions \( A, B \),
	We know that \( |S| < |A|, |S| < |B| \)
	as disconnected partitions are nonempty and \( G \) is connected.
	There is at least one edge incident to vertices in both \( A \) and \( B \).
	We color \( E(G[A]) \) \( \red \) and \( E(G[B]) \) \( \blue \).
	Both colors are used, so the coloring is surjective.
	We also need to show that no almost cycle was created.
	As~\Cref{lemma:is_nac_coloring}
	states, there is no almost cycle in a graph if and only
	if components \( G_\red^\delta, G_\blue^\delta \)
	are induced subgraphs of \( G \).
	That is the case as there is no edge with both the endpoints in \( S \)
	as \( S \) is a stable set.
	So there are no almost cycles and a NAC-coloring is created.
\end{proof}
%
The above lemma can be also proven from the fact that every cycle that
uses a vertex in \( S \) must pass
through at least one vertex in \( A \setminus S \) and \( B \setminus S \).
Such a cycle has at least four edges,
at least two are \( \red \) and at least two are \( \blue \).

The following simple observation can be deduced
from~\Cref{lemma:stable_cut_implies_nac_coloring}.
%
\begin{corollary}[\cite{legersky_original}]%
	\label[corollary]{cor:vertex_no_triangle_stable_cut}
	%
	Let \( G \) be a connected graph with \( |E(G)| \ge 2 \).
	If there is a vertex \( v \in V(G) \) such that it is
	not contained in any triangle \( C_3 \) in \( G \),
	then the graph \( G \) has a NAC-coloring.
\end{corollary}
%
\begin{proof}
	If \( v \) is in no triangle, then there is no edge interconnecting
	vertices in \( N(v) \). Therefore, \( N(v) \) is a stable cut and \( G \)
	has a NAC-coloring.
\end{proof}

In the following paragraphs, we construct graph classes by joining graphs
that are already in the class.
Graph \( G \) is created from graphs \( G_1, G_2 \) by an \emph{edge identification}
on an edge \( \{u, v\} \)
if \( V(G_1) \cap V(G_2) = \{u, v\}, E(G_1) \cap E(G_2) = \{\{u, v\}\} \)
and \( G = (V(G_1) \cup V(G_2), E(G_1) \cup E(G_2)) \).
Graph \( G \) is created from graphs \( G_1, G_2 \) by a \emph{triangle identification}
if \( V(G_1) \cap V(G_2) = \{u, v, w\}, E(G_1) \cap E(G_2) = \{\{u, v\}, \{v, w\}, \{w, u\}\} \)
and \( G = (V(G_1) \cup V(G_2), E(G_1) \cup E(G_2)) \).

It is known for some classes that a stable cut must exist.
Flexibility of such graphs can be seen from~\Cref{lemma:laman_characterization}.
%
\begin{theorem}[\cite{stable_cuts_2v_4}]
	Let \( G \) be a graph with \( |E(G)| \le 2|V(G)|-4 \).
	Then \( G \) contains a stable cut.
\end{theorem}
%
This result can be extended to graphs \( |E(G)| \le 2|V(G)|-3 \) with exception
for graphs from graph class \( \GSC \):
%
\begin{itemize}
	\item Triangle and prism are in \( \GSC \).
	\item If \( H, K \in \GSC \), and \( G \) is a graph
	      formed from \( H, K \) by an edge identification,
	      then also \( G \in \GSC \).
	\item If \( H, K \in \GSC \), and \( G \) is a graph
	      formed from \( H, K \) by a triangle identification,
	      then also \( G \in \GSC \).
\end{itemize}
%
%
\begin{theorem}[\cite{stable_cuts_2v_3,stable_cuts_2v_3_revisit}]
	Let \( G \) be a graph with \( |E(G)| \le 2|V(G)|-3 \). Then \( G \) contains
	a stable cut or \( G \in \GSC \).
\end{theorem}
%
Note that 2-trees (\Cref{def:2-tree})
are in \( \GSC \) by definition.
%
\begin{lemma}[\cite{stable_cuts_legersky}]%
	\label[lemma]{lemma:stable_cut_or_2_tree}
	%
	For every \( G \in \GSC \), either \( G \) has a NAC-coloring
	or \( G \) is a 2-tree.
\end{lemma}
%
This shows that the properties of having a NAC-coloring and having a stable cut
are not equivalent.

There are other notable similarities between NAC-colorings and stable cuts.
Namely, as proved in~\cite{stable_cuts_complexity_base} it is NP-complete
to decide whether a graph has a stable cut.
After it was shown in~\cite{stable_cuts_complexity_deg_five} that
it is NP-complete to decide whether a \emph{line} graph with maximum degree five
admits a stable cut.
%
Also, it is NP-complete to decide whether a graph with average degree of
$4+\varepsilon$ for any $\varepsilon > 0$ admits a stable cut~\cite{stable_cuts_complexity_deg_five}.
Both results are similar to what we show for NAC-colorings in \Cref{chapter:np}.
%

\section{Algorithms}

The papers~\cite{stable_cuts_2v_3,stable_cuts_2v_3_revisit} do not provide
an algorithm to find a stable cut (if present)
for graphs where \(|E(G)| \le 2|V(G)|-3 \),
they only provide a list of claims
from which a stable cut set can be found.
All the checks can be run in polynomial time.
Unfortunately, the list is long,
and it is beyond the scope of this thesis
to describe and implement such an algorithm.

As shown in~\cite[Algorithm 1]{stable_cuts_legersky}, a stable cut can be found
in a polynomial time for any flexible graph.
As mentioned before, that also includes graphs
where \( |E(G)| \le 2|V(G)| - 4 \) (not for \( |E(G)| \le 2|V(G)| - 3 \)).
A rigid component of a graph is maximal induced subgraph that is rigid.
The main idea of the algorithm works as follows:
%
\begin{itemize}
	\item Rigid components of the graphs are found in a polynomial time.
	\item Vertices \( u, v \) from different rigid components are chosen.
	\item If neighborhood of \( u \) is a stable cut, return it.
	\item Otherwise, contract one edge of a triangle with \( u \) and start again while preserving \( u \).
\end{itemize}

Here we show pseudocode of the algorithm as described
in the original paper~\cite{stable_cuts_legersky}.
%
\begin{algorithm}[ht]
	\caption{\textsc{Stable cut of a connected flexible graph}}%
	\label{alg:stable_cut_flexible}%
	\begin{algorithmic}[1]
		\Require{} a connected flexible graph $G$, vertices $u$ and $v$ not in the same rigid component of $G$
		\Ensure{} a stable cut $S$ of $G$ such that $u$ and $v$ are separated by $S$
		\If{the neighborhood of $u$ is stable}
		\State\Return{} the neighborhood of $u$
		\Else{}
		\State{} $x_1,x_2 :={}$ neighbors of $u$ such that $(u,x_1,x_2)$  is a $3$-cycle
		\For{$i\in\{1,2\}$}
		\State{} $G'_i :={}$ the graph obtained from $G$ by contracting the edge $ux_i$
		\State{} $u'_i :={}$ the vertex of $G'_i$ corresponding to the contracted edge $ux_i$
		\EndFor{}
		\If{$u'_1$ and $v$ are in different rigid components of $G'_1$}
		\State\Return{} a stable cut of $G'_1$ separating $u'_1$ and $v$
		\Else{}
		\State\Return{} a stable cut of $G'_2$ separating $u'_2$ and $v$
		\EndIf{}
		\EndIf{}
	\end{algorithmic}
\end{algorithm}
%

\Cref{alg:stable_cut_flexible}
was presented as a practical result of the following theorem.
%
\begin{theorem}[\cite{stable_cuts_legersky}]
	Let \( G \) be a flexible graph and \( u, v \in V (G) \) be such that no rigid component of \( G \)
	contains both \( u \) and \( v \). Then there is a stable cut \( S \) of \( G \) that separates \( u \) and \( v \), and such that
	every rigid component of \( G \) contains at most one vertex of \( S \). Moreover, if \( G \) is 2-connected,
	then for every vertex \( v \in V(G) \), \( G \) has a stable cut avoiding \( v \).
\end{theorem}
%

\subsection{Implementation}%
\label{sec:stable_cuts_implementation}

Notice that~\Cref{alg:stable_cut_flexible}
accepts vertices \( u, v \) arbitrary as long as they do not share a rigid component.
In an implementation, a check ensuring that \( u \) and \( v \)
are in different rigid components has to be added.

In the implementation, it is also beneficial to offer an option to choose
vertices \( u \) and/or \(  v \) for a user automatically.
First in the user specifies only vertex \( u \), the goal is to find a stable cut
that avoids it. Our algorithm finds \( v \)
such that it is not in the same rigid component with \( u \).
This is always possible unless \( u \) is the only articulation vertex
and blocks are rigid components. In this case, an error is reported back to the user.

If neither of \( u, v \) is given by the user,
we need to choose vertices in such a way
that a stable cut is always found even if it is only the single articulation vertex.
This can be done by choosing a vertex that is not in all the rigid components of the graph.

%
\Cref{alg:stable_cut_flexible}
is implemented and contributed as part of this thesis
to PyRigi~\cite{pyrigi} in pull request~\cite{pyrigi_pr_stable_cuts}
as method \texttt{stable\_separating\_set}.
Other helper and cuts related function
were also implemented in the module \texttt{separating\_set.py}.

