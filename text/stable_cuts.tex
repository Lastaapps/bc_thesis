
\chapter{Stable cuts}

\begin{chapterabstract}
	%
	We introduce relation between stable cuts and NAC-colorings.
	We summarize progress done on stable cuts existence and search.
	Lastly we briefly describe an algorithm that can find a stable cut
	for any flexible graph.
	%
\end{chapterabstract}

\section{Stable cuts}

In this chapter we focus on connected graphs only as graph cuts lose their
meaning on disconnected graphs.
%
\begin{definition}[Stable cut]
	Independent set of a graph \( G \) is set \( I \subseteq V(G) \)
	such that \( \forall u, v \in I : \{u, v\} \not\in E(G) \).
	%
	Vertex cut of a graph \( G \) is set \( C \subseteq V(G) \)
	such that \( G \setminus C \) (\( G \) with \( C \) removed)
	is a disconnected graph.
	%
	Stable cut of a graph \( G \) is a set \( S \subseteq V(G) \) such that
	\( S \) is independent set in \( G \) and also \( S \) is vertex cut in \( G \).
\end{definition}
%

The relation to NAC-coloring is quite simple.
%
\begin{lemma}
	Let \( S \) be a stable cut in \( G \).
	Then \( G \) has a NAC-coloring \( \delta \).
\end{lemma}
\todo[inline]{Cite}
%
\begin{proof}
	As \( S \) is a cut in \( G \), let us have corresponding partitions \( A, B \).
	We know that \( |S| < |A|, |S| < |B| \)
	as disconnected partitions are nonempty and \( G \) was connected.
	There is at least one edge incident to vertices in both \( A \) and \( B \).
	We color \( E(G[A]) \) \( \red \) and \( E(G[B]) \) \( \blue \).
	Both colors were used, so the coloring is surjective.
	We also need to show that no almost cycle was created.
	As~\cite[Lemma 2.4]{legersky_original} states, there is no almost cycle
	in a graph if and only if components \( G_\red^\delta, G_\blue^\delta \)
	are induced subgraphs of \( G \).
	That is the case as there is no edge incident to vertices in \( S \).
	It holds that \( G_\red^\delta = G[A], G_\blue^\delta = G[B] \).
	So there are no almost cycles and \( \delta \) is a NAC-coloring.
\end{proof}
%
You can also prove the above lemma from the fact that every cycle that
uses a vertex in \( S \) must pass
through at least one vertex in \( A \setminus S \) and \( B \setminus S \).
Such cycle has at least four edges,
at least two are \( \red \) and at least two are \( \blue \).

It is known for some classes that stable cut must exists.
\todo[inline]{Show and cite that \( 2V-4 \) are flexible}
%
\begin{theorem}[\cite{stable_cuts_2v_4}]
	Let \( G \) be a graph with \( |E(G)| \le 2|V(G)|-4 \).
	Then \( G \) contains a stable cut.
\end{theorem}
%
This result can be extended to graphs \( |E(G)| \le 2|V(G)|-3 \) with exception
for graphs from graph class \( \GSC \).
%
\begin{itemize}
	\item Triangle and prism are in \( \GSC \).
	\item If \( H, K \in \GSC \), and \( G \) is a graph
	      formed from \( H, K \) by an edge identification,
	      then also \( G \in \GSC \).
	\item If \( H, K \in \GSC \), and \( G \) is a graph
	      formed from \( H, K \) by a triangle identification,
	      then also \( G \in \GSC \).
\end{itemize}
%
%
\begin{theorem}[\cite{stable_cuts_2v_3,stable_cuts_2v_3_revisit}]
	Let \( G \) be a graph with \( |E(G)| \le 2|V(G)|-3 \). Then \( G \) contains
	a stable cut or \( G \in \GSC \).
\end{theorem}
%
A 2-tree is a graph formed by merging triangles by edge identification.
Note that such graphs are in \( \GSC \).
%
\begin{lemma}[\cite{stable_cuts_legersky}]
	For every \( G \in \GSC \), either \( G \) has a NAC-coloring
	or \( G \) is a 2-tree.
\end{lemma}
%
This shows that the property of having a NAC-coloring and having a stable cut
are not equivalent.

\section{Algorithms}

As proved in~\cite{stable_cuts_complexity} it is NP-complete
to decide whether a line graph with maximum degree five admits a stable cut
-- similar result as we got for NAC-coloring.

The papers~\cite{stable_cuts_2v_3,stable_cuts_2v_3_revisit} do not provide
an algorithm to find the stable cut (if present)
for graphs where \(|E(G)| \le 2|V(G)|-3 \),
they only provide list of claims
from whose a stable cut set can be found.
All the checks can be run in polynomial time.
Unfortunately, the list is long and beyond the scope of this thesis.

As shown in~\cite[Algorithm 1]{stable_cuts_legersky} a stable cut can be found
in a polynomial time for any flexible graph.
As mentioned before, that also includes graphs where \( |E(G)| \le 2|V(G)| - 4 \).
Rigid component of a graph is maximal induced subgraph such that
The main idea of the algorithm works as follows:
%
\begin{itemize}
	\item Rigid components of the graphs are found.
	\item Vertices \( u, v \) from different rigid components are chosen.
	\item If neighborhood of \( u \) is a stable cut, return it.
	\item Otherwise, contract one edge of a triangle with \( u \) and start again while preserving \( u \).
\end{itemize}
%
This algorithm is implemented and
contributed to PyRigi~\cite{pyrigi} as part of this thesis.

